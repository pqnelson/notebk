\section{Newton's Laws of Motion}

\M
Newton laid down three ``laws'' of motion. The first Law is necessary
for the second Law, but the third Law may be viewed as a consequence of
the second Law.

\subsection{Newton's First Law}

\N{First Law of Motion}
This is the famous, ``A body at rest tends to stay at rest, a body in
motion [in a constant rectlinear velocity] tends to stay in motion,
unless acted upon by a force.''

Implicitly there are two claims being made:
\begin{enumerate}
\item For any trajectory $\gamma\colon I\to\RR^{3}$, we have
  $\D\gamma(t)/\D t$ be constant unless there is a force acting upon the body;
\item When we apply this to observers, there is a special class of
  observers whose trajectories are described by $\D\gamma(t)/\D t$ being
  constant --- these are \define{Inertial Observers}. These are special
  because the laws of physics ``look the same'' to all inertial observers.
\end{enumerate}

\M
This second claim strains credulity, it's not immediately obvious
\emph{this} is the result of Newton's First Law of Motion.
But this is because if an obsever is non-inertial, there are forces
acting on the observer which must be ``accounted for'' when setting up
the equations of motion for other bodies. Inertial observers lack such
``bonus parts'' in equations of motion relative to inertial coordinate
systems.

\N{Galilean Relativity}
If we have two different inertial observers $K_{1}$ and $K_{2}$ with coordinates $\vec{r}_{1}$
and $\vec{r}_{2}$, respectively, and if $K_{2}$ moves relative to
$K_{1}$ with a constant velocity $\vec{v}_{2|1}$, then we have:
\begin{equation}
\vec{r}_{1} = \vec{r}_{2} + \vec{v}_{2|1}t.
\end{equation}
The time is the same in the two frames, in the sense that when we agree
on the units the clocks can differ by a constant displacement $t_{2}=t_{1}+\mbox{(constant)}$.
This is known as \define{Galilean Relativity} and relates inertial
reference frames to each other.

\N{Non-inertial reference frames}
There are two ways to have a non-inertial reference frame (either one
disqualifies a reference frame from being inertial):
\begin{enumerate}[label=(\arabic*)]
\item Rotating reference frames are not inertial,
\item Accelerating reference frames are not inertial.
\end{enumerate}

\N{Objection: Status of Inertial Observers}
The clever reader may object that inertial observers cannot possibly
exist. There are two separate arguments:
\begin{enumerate}[label=(\arabic*)]
\item We are on a rotating planet, Earth, which also revolves around the
  Sun --- that is, we experience rotation. So we're a rotating reference
  frame, which is not inertial.
\item We are told rumour that the expansion of the universe is
  accelerating. \emph{We} are in the universe. Therefore \emph{we} are
  an accelerating reference frame.
\end{enumerate}
This is all true, and in
Appendix~\ref{section:eom-in-non-inertial-frame} we modify Newton's
laws of motion to facilitate these claims (and quantify the degree to
which they affect experiments).

\N{Puzzle}
Consider the trajectory of a body moving in uniform circular motion,
something like:
\begin{equation}
\gamma(t) = r_{0}\cos(\omega t)\vec{e}_{1} + r_{0}\sin(\omega t)\vec{e}_{2},
\end{equation}
where $\vec{e}_{1}$, $\vec{e}_{2}$ are unit vectors for the $x$, $y$
axes in Cartesian coordinates (relative to some inertial observer);
$r_{0}>0$ is the radius of the circular motion; $\omega\neq0$ is the
constant ``angular velocity'' ($\omega=2\pi/T$ where it takes a time
interval $T$ seconds long to orbit the circle once).

But this is not a trajectory with a ``uniform rectilinear velocity''. We
can see this by taking its time derivative:
\begin{equation}
\frac{\gamma(t)}{\D t} = -r_{0}\omega\sin(\omega t)\vec{e}_{1} + r_{0}\omega\cos(\omega t)\vec{e}_{2}.
\end{equation}
We see the \emph{speed} is constant $\|\D\gamma/\D t\|=r_{0}\omega$, so
the direction of the velocity changes.
But by Newton's first Law, this means there is some force acting on the
body. What ``force'' could this be?

\subsection{Newton's Second Law}

\N{Mathematical Statement}
This is the big contribution, the main tool, of Newtonian mechanics. It
is often presented as: The equations of motion for a body looks like
(for every coordinate of the body):
\begin{equation}
\mbox{Force} = (\mbox{mass})(\mbox{acceleration}).
\end{equation}
This is a special case of Newton's second Law, it's true when a body's
mass is constant. When a body's mass can change over time $m=m(t)$, we
instead need the notion of momentum
\begin{equation}
  \begin{split}
    \vec{p}(t) &:= m(t)\vec{v}(t)\\
    \mbox{(momentum)}(t) &:= \mbox{(mass)}(t)\mbox{(velocity)}(t)
  \end{split}
\end{equation}
where ``$:=$'' means ``is defined as''. Then Newton's second Law is
\begin{equation}
\boxed{\vec{F} = \frac{\D\vec{p}(t)}{\D t}}
\end{equation}
where $\vec{F}$ is the sum of all forces acting on the body.
Usually mass $m$ is constant, which is how we obtained the
$\vec{F}=m\vec{a}$ form of Newton's second Law.

\begin{remark}[Physical bodies with changing mass]
  Rockets have variable mass because fuel is expended continuously. We
  would predict incorrcet motion with $\vec{F}=m\vec{a}$: we would be neglecting the
  entire source of the rocket's motion.

  Other systems (like automobiles or airplanes) experience a similar
  phenomenon, but it is much less pronounced. For an automobile, $10$
  gallons of gas weighs $61$ pounds (or $27.67~\mathrm{kg}$) compared to
  about $1500~\mathrm{kg}$ for the rest of the car, i.e., the gas is
  less than $1.85\%$ the mass of the total car, less than the weight of one
  adult passenger.

  Airplanes, like the Boeing 737, have around $26\,000~\mathrm{L}$ fuel
  capacity, and airplane fuel has a density of
  $0.8~\mathrm{kg}/\mathrm{L}$ for a total mass of
  $20\,828~\mathrm{kg}$. The maximum weight of a 737 is around
  $55\times10^{3}~\mathrm{kg}$, the fuel's consumption is non-negligible
  in this case (since the fuel is greater than $36\%$ of the initial mass).
\end{remark}

\begin{remark}[Inertial mass and equivalence principle]
The mass which appears in Newton's second law is sometimes referred to
as a body's \emph{inertial mass}. This is contrasted with a body's
\emph{gravitational mass} --- analogous to a body's \emph{electric charge}
which generates an electric field, the gravitational mass generates a
gravitational field. Galileo determined experimentally these two masses
are the same, and this is the celebrated \emph{equivalence principle}.
\end{remark}

\N{Equations of Motion}
Newton's second Law gives us the equations of motion relative to an
inertial reference frame. All we need to do is write down the forces
acting on the body, sum them together, then plug the net force vector
into the left-hand side of Newton's second Law.

\N{How to Study This}
The way we should study this is by introducing new forces, one at a
time, and studying how they act on physical bodies. For a rough roadmap
of how most texts study forces:
\begin{enumerate}[label=(\arabic*)]
\item Uniform gravitational force --- on Earth, the gravitational force
  from the Earth is approximately constant $\vec{F}_{\text{const.~gravity}}=m_{\text{body}}g\vec{e}_{\text{downwards}}$
  where $g\approx 9.8~\mathrm{m}/\mathrm{s}^{2}$ and $m_{\text{body}}$
  is the mass of the body experiencing the gravitational force. Newton's
  second Law becomes $\D^{2}\gamma(t)/\D t^{2}=g\vec{e}_{\text{downwards}}$.
  This is studied in the variety of problems:
  \begin{itemize}
  \item Projectile motion
  \item Block on inclined plane, which experiences on uniform
    gravitational force.
  \item Simple pendulum whose ``bob'' experiences uniform gravitational
    force, and is connected to the ceiling by a massless rigid rod.
  \end{itemize}
\item Tension (in a rope or chain), which ``pulls'' on a body
  \begin{itemize}
  \item Atwoods machines attaches point masses to a [massless] rope
    which passes through a [massless, frictionless] pulley. The masses
    experience uniform gravitational force, possibly placed on an
    inclined plane. This can be ``iterated'' --- we could replace one of
    the point masses with another Atwood's machine, the rope is holding
    up the [massless, frictionless] pulley.
  \end{itemize}
\item Various contact forces --- static friction (the
  ``hesitation'' of a body from accelerating due to the floor's
  microscopic ``jaggedness''), kinetic friction (the ``ongoing'' friction once the
  static friction has been overcome), normal force (the floor ``pushing
  against'' the object placed against it, to prevent the object falling
  through the floor), and fluid resistance (recall, a fluid is either a
  liquid or a gas, and air is a gas, so a body needs to ``push through''
  the stagnant air which is precisely air resistance).

  These are introduced one at a time, which is used to re-examing the
  previous examples of tension and uniform gravitational force.
\item Hooke's law for simple harmonic motion. This is usually presented
  as a way to describe how [massless] springs work, with a force
  proportional to the distance the spring is dislocated from
  equilibrium.

  More generally, any force proportional to displacement
  $\vec{F}=k\vec{x}$ follows Hooke's law. This is important because
  every [non-constant] force can be Taylor expanded, and Hooke's law
  describes the first nontrivial contribution.
\end{enumerate}
This actually guides the ``next steps'' in a physics textbook, because
the other forces of interest are electromagnetism and gravity.

\N{``Composite'' Problems}
We can compose these forces together to form more complicated problems.
For example, we could attach a massless pulley to the ceiling by a thin
massless rod (or, respectively, we could replace the point-mass ``bob''
of a simple pendulum by an Atwood's machine); we could consider a
``double pendulum'' (attach a thin massless rod to the ceiling, and
attach bob $A$ to the other end; we attach a thin massless rod to bob
$A$ and then attach a point-mass $B$ to the other end of this second
massless rod); we could replace ropes (or massless rods) with springs;
so on and so forth.

As we invent more complicated problems, we should think about how we
could ``freeze out'' complications to recover simpler problems. The
solutions to the complicated problems should have some ``limiting behaviour''
to recover solutions to the simpler problems. This guides one part of
the ``evaluate'' step to writing solutions to these problems.

\subsection{Newton's Third Law}

\M This is the famous, ``To every action there is always opposed an
equal reaction; or, the mutual actions of two bodies upon each other are
always equal, and directed to contrary parts'', law. Sadly, this is
misunderstood. We should interpret this as ``If body $A$ acts on body
$B$ by a force $\vec{F}_{A~\text{on}~B}$, then $B$ acts on body $A$ by
an equal but opposite force [i.e., a force equal in magnitude]
$\vec{F}_{B~\text{on}~A}=-\vec{F}_{A~\text{on}~B}$''.

This is assuming there are no external forces, the only force which body
$B$ experiences is $\vec{F}_{A~\text{on}~B}$ and the only force which
body $A$ experiences is $\vec{F}_{B~\text{on}~A}$.

In light of Newton's second Law, we see this is precisely the
conservation of momentum (i.e., the momentum for the total system
$\vec{p}=\vec{p}_{A} + \vec{p}_{B}$ [being the sum of momentum for body
  $A$ and the momentum for body $B$] is a constant in time):
\begin{equation}
\frac{\D\vec{p}}{\D t} = \frac{\D\vec{p}_{A}}{\D t} +
\frac{\D\vec{p}_{B}}{\D t} = \vec{0}.
\end{equation}
Why believe this? Well, this is because the sum of forces, which
Newton's third Law asserts are equal in magnitude but opposite in sign
(hence sums to zero).

\N{Conservation of Momentum}
Thus, we identify Newton's third Law with the conservation of momentum
for a composite system.
Phrased differently: an isolated system experiences no force.

\begin{remark}[Rockets move by conservation of momentum]
The conservation of momentum may be used to \emph{propel} bodies
forward. This is how a rocket works.

The same principle applies if you were in a boat with a load of bricks.
By throwing one brick out at a time, with most of the initial momentum
of the brick being in the direction parallel to the water, you could
move your boat foward. Conservation laws are powerful.
\end{remark}

\subsection{Concluding Remarks about Newton's Laws}

\N{Free Body Diagrams}
We have a nice way to obtain the equations of motion by invoking
physical principles to obtain the force vector acting on bodies.
It is useful to draw a diagram for each body in the system, where each
force acting on the body is drawn. This is known as a ``free body
diagram''. [It should be drawn in the ``setup'' step of writing a solution.]

\N{Rigid body motion}
Advanced Newtonian mechanics just works with more complicated
situations, like modeling a coin as a rigid body. Euler pioneered this
study of rigid body motion --- the analogous propositions to Newton's
Laws of motion for rigid body motion is called ``Euler's Laws of motion''.
It turns out the problem may be decomposed as:
\begin{enumerate}[label=(\arabic*)]
\item Treat the center-of-mass of the system as a point particle subject
  to Newton's laws of motion, and
\item Treat the rigid body as rotating about its center of mass; by
  choosing the center-of-mass as the origin, this problem simplifies considerably.
\end{enumerate}
Euler proved this all works fine, but it's tedious to do the
calculations here.

\N{Differential Equations are Hard}
Newton's laws of motion gives us a system of second-order [ordinary] differential
equations. In general, second-order ordinary differential equations are
hard to solve. This motivates us to find ways to extract information
about physical systems which avoids explicitly solving these equations
of motion, like the ``work--energy theorem'' relating the force applied
to a body over some distance is equal to the change in the body's
kinetic energy.

\N{Setting up problems is hard}
It is surprisingly nontrivial to setup a problem correctly. We could
easily forget or overlook certain forces, or translate it as an
incorrect vector. For this reason, physicists invented different
formalisms for classical mechanics like Lagrangian mechanics and
Hamiltonian mechanics.
