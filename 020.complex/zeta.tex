%%
%% zeta.tex
%% 
%% Made by Alex Nelson
%% Login   <alex@tomato>
%% 
%% Started on  Wed Jun 10 11:57:19 2009 Alex Nelson
%% Last update Wed Jun 10 11:57:19 2009 Alex Nelson
%%
The definition for the Riemann zeta function is
\begin{equation}%\label{eq:}
\zeta(s)=\sum_{n=1}^{\infty}\frac{1}{n^s}=\frac{1}{1^s} + \frac{1}{2^s} + \frac{1}{3^s} + \cdots
\end{equation}
It is holomorphic everwhere except for a simple pole at $s=1$
with residue 1.

For any positive even integer $2n$, we have
\begin{equation}%\label{eq:}
\zeta(2n) = (-1)^{n+1}\frac{B_{2n}(2\pi)^{2n}}{2(2n)!}
\end{equation}
where $B_{2n}$ is a Bernoulli number, and for negative integers
we have
\begin{equation}%\label{eq:}
\zeta(-n)=\frac{-B_{n+1}}{n+1}
\end{equation}
for $n\geq1$.

Let
\begin{equation}%\label{eq:}
f(x) = \frac{x}{e^x-1}
\end{equation}
then the Bernoulli numbers may be found from
\begin{equation}%\label{eq:}
 B_n=\lim_{x\to0}\frac{d^n}{dx^n}\frac{x}{(e^x-1)}. 
\end{equation}
Observe that for $n=1$
\begin{equation}%\label{eq:}
f'(x) = \left({{1}\over{e^{x}-1}}\right)\left(1-\frac{f(x)}{e^x-1}\right)
\end{equation}
and now observe that
\begin{equation}%\label{eq:}
\frac{d}{dx}\left(\frac{1}{e^x-1}\right) = -e^x\left({{1}\over{e^{x}-1}}\right)^2
\end{equation}
and we can use the product rule to find all of our favorite
Bernoulli numbers.

We have a table of the first few Bernoulli numbers:
\begin{center}
\begin{tabular}{|c|c|}
\hline
$n$ & $B_n$\\\hline
0 & 1\\\hline
1 & -1/2\\\hline
2 & 1/6\\\hline
4 & -1/30\\\hline
6 & 1/42\\\hline
8 & -1/30\\\hline
10& 5/66$\approx$ 0.07575757576\\\hline
12& -691/2730$\approx$-0.25311355311\\\hline
14& 7/6\\\hline
16& -3617/510$\approx$ -7.09125686275\\\hline
18& 43867/798$\approx$ 54.9711779448\\\hline
\end{tabular}
\end{center}

The zeta function satisfies the functional equation
\begin{equation}%\label{eq:}
\zeta(s) = 2^s\pi^{s-1}\ \sin\left(\frac{\pi s}{2}\right)\ \Gamma(1-s)\ \zeta(1-s) \!,
\end{equation}
valid for all $s\in\mathbb{C}$. An equivalent relationship may be
expressed as a sum
\begin{equation}%\label{eq:}
\zeta(s)(1-{2^{1-s}})= \sum_{n=1}^\infty \frac{(-1)^{n+1}}{n^s}.\!
\end{equation}


\subsection{Mellin Transform}

The Mellin transform of a function $f(x)$ is defined as
\begin{equation}%\label{eq:}
 \int_0^\infty f(x)x^{s-1}\, dx,\!
\end{equation}
when defined. We can relate the zeta function to one million and
one things this way, we have
\begin{equation}%\label{eq:}
\Gamma(s)\zeta(s) =\int_0^\infty\frac{x^{s-1}}{\exp(x)-1}dx,\!
\end{equation}
where $\Gamma$ is our favorite gamma function, and
\begin{equation}%\label{eq:}
2\sin(\pi s)\Gamma(s)\zeta(s) =i\oint_{C}\frac{(-x)^{s-1}}{\exp(x)-1}dx\!
\end{equation}
for all $s$ where the contour $C$ begins and ends at $+\infty$
and circles the origin once.

\subsection{Laurent Series}

Since the zeta function has a single simple pole at $s=1$ we can
expand it around the singular point. The series is
\begin{equation}%\label{eq:}
\zeta(s)=\frac{1}{s-1}+\sum_{n=0}^\infty \frac{(-1)^n}{n!} \gamma_n \; (s-1)^n.
\end{equation}
where $\gamma_n$ are the Stieltjes constants, defined by the
limit
\begin{equation}%\label{eq:}
 \gamma_n = \lim_{m \rightarrow \infty} {\left(\left(\sum_{k = 1}^m \frac{(\log k)^n}{k}\right) - \frac{(\log m)^{n+1}}{n+1}\right)}.
\end{equation}
where the constant $n=0$ term in the Laurent series is just
$\gamma_0$ the Euler-Mascheroni constant.
