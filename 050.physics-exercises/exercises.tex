\documentclass{amsart}% article}

\usepackage[en-US,useregional=text]{datetime2}
\DTMlangsetup[en-US]{mapzone=true,zone=pacific}
\usepackage{macros}

\def\D{\mathrm{d}}

\title{Physics Exercises}
\def\homeurl{\url{https://pqnelson.github.io/notebk/}}
\author{Alex Nelson}
%\thanks{This is a page from \homeurl{}\hfil\break\indent\hspace{0.8em}Compiled:\enspace\today\ at \DTMcurrenttime\ (\DTMcurrentzone)}
\thanks{This is a page from \homeurl{}\hfil\break\indent{}Compiled:\enspace\today\ at \DTMcurrenttime\ (\DTMcurrentzone)}
\date{May 25, 2023} %9:11:13 am (PDT)
\begin{document}

\maketitle
\tableofcontents

\section{Introduction}

These are assorted exercises I've been thinking about. They should be
solvable, and related to further exercises. For the most part, the
inspiration for the exercises stem from particle physics, quantum
gravity, and climate science.

I've been thinking about Wheeler's geometrodynamics, which led to
wondering if quantum field theory ``recycles'' familiar physics
terminology (``mass'', ``spin'', etc.) in a way which is completely
disconnected from their intuitive meanings. This also led to me the
paper by Anderson and Brill~\cite{Anderson:1996pu} about geons, from
which I am certain exercises may be extracted.

Exercises are numbered sequentially. I may include remarks about an
exercise, each remark is ``subnumbered'' --- for example they look like
$\langle x\rangle.\langle r\rangle$ for remark number $r$ on exercise
numbered $x$.

\section{Classical Scattering}

\subsection{The ``Classical Electron''}
Before continuing on, I will use the phrase ``classical electron'' to
refer to a sphere of radius
\begin{equation}
r_{e} = \frac{1}{4\pi\varepsilon_{0}}\frac{e^{2}}{m_{e}c}\approx2.81794\times10^{-15}~\mathrm{m}.
\end{equation}
It has uniform electric charge density, summing to a total charge equalling
\begin{equation}
q_{e} = \left(\frac{4}{3}\pi r_{e}^{3}\right)\rho_{q} = e\approx-1.602176634\times10^{-19}~\mathrm{C},
\end{equation}
and similarly has a uniform mass density, summing to a total mass of:
\begin{equation}
m_{e} = \left(\frac{4}{3}\pi r_{e}^{3}\right)\rho_{m} \approx 9.1093837015(28)\times10^{-31}~\mathrm{kg}.
\end{equation}
The total angular momentum from the rotation is $S=\hbar/2$, and the
axis of rotation is determined by the velocity vector.


\begin{exercise}[Classical ``electron'' scattering]
\textit{Fluff:} the spin of the electron contributes meaningfully to
scattering experiments. We will see if this is true classically.

\textit{Parameters:}
Consider a fixed point-particle of charge $q\neq0$.
Suppose we have a solid sphere of radius $r_{e}$ initially a distance
$\ell\gg r_{e}$ away from the fixed point-particle; the sphere has
uniform charge density $\rho_{c}$, uniform mass density $\rho_{m}$, and
rotates about its ``forward direction'' (the unit velocity
vector). Suppose its total angular momentum comes from this spin is $S=\hbar/2$.

%% For concrete values, the total charge of the sphere is equal to that of
%% the electron,
%% \begin{subequations}
%% \begin{equation}
%% q_{e} = \left(\frac{4}{3}\pi r_{e}^{3}\right)\rho_{q} = e\approx-1.602176634\times10^{-19}~\mathrm{C},
%% \end{equation}
%% and the total mass of the sphere is that of the electron,
%% \begin{equation}
%% m_{e} = \left(\frac{4}{3}\pi r_{e}^{3}\right)\rho_{m} \approx 9.1093837015(28)\times10^{-31}~\mathrm{kg}.
%% \end{equation}
%% Suppose the radius of the sphere is the classical electron radius,
%% \begin{equation}
%% r_{e} = \frac{1}{4\pi\varepsilon_{0}}\frac{e^{2}}{m_{e}c}\approx2.81794\times10^{-15}~\mathrm{m}.
%% \end{equation}
%% \end{subequations}
We could suppose $\ell=1.00~\mathrm{m}$, for example, which would
intuitively fit inside any laboratory.

\textit{Problem statement:}
If the solid sphere has impact parameter $b$, then what is its
scattering angle $\Theta$? What is its differential cross-sectional area $\D\sigma$?
How do these quantities depend on the spin $S$?
\end{exercise}

\begin{remark}
Recall, the moment of inertia for a sphere would be $I=\frac{2}{5}mr^{2}$
so its angular momentum would be $S=I\omega$.
\end{remark}

\begin{remark}
Since the ``electron'' is spinning and electrically charged, it will
generate a magnetic field. Does our solution reflect this? If we take
$r_{e}\to0$ limit, will our solution recover the familiar scattering of
a charged point-particle? If not, why not?
\end{remark}

\begin{exercise}[Stern--Gerlach, classically]
Suppose we have an inhomogeneous magnetic field $\vec{B}$.
Suppose we have a ``classical electron'' --- that is, a sphere of radius
$r_{e}$, mass $m_{e}$, uniform charge density which sums to a total
charge of $e$, uniform mass density, etc. Suppose further the
``classical electron'' is rotating about its direction of motion (i.e.,
the unit velocity vector for the electron).

If we send the classical electron through the inhomogeneous magnetic
field, then where will it end up? This is precisely the experimental
setup Stern and Gerlach devised, which was later interpreted as
measuring the spin of an electron.
\end{exercise}

\begin{remark}
Hint: use Larmor precession.
\end{remark}

\section{General Relativity}

\begin{exercise}
Write an exercise about experimentally measuring the torsion of a
connection on spacetime.
\end{exercise}

\subsection{Regge Calculus}

\begin{exercise}[Disphenoids for Minkowski vacuum]
If we want to describe Minkowski spacetime using Regge calculus,
then we need to use disphenoids (regular tetrahedra do not tessalate
[``fill''] space in 3 dimensions) with vertices located at $(1,0,0)$,
$(-1,0,0)$, $(0,1,1)$, $(0,1,-1)$.

\begin{enumerate}
\item What are the surface areas of each face of the disphenoid?
\item What is the volume of the disphenoid?
\item What is the centroid for the disphenoid? [Hint: we can compute it
  as the average of the vertices]
\item Find the unit normal vectors for each surface, which extends to
  the centroid (in the sense that if we had a line passing through its
  basepoint, moving in the direction of the unit normal, then it passes
  through the centroid).
\item Prove the unit normal vectors from the previous step all sum to
  the zero vector.
\end{enumerate}
\end{exercise}

\begin{exercise}
If we consider a disphenoid (as described in the previous problem),
there are four normal vectors $\vec{n}_{1}$, \dots, $\vec{n}_{4}$ on
each face. Suppose the magnitude of each normal is equal to the area of
its face, and suppose the base-point of the normal is such that it
connects to the centroid of the disphenoid if extended by a scalar
multiple.

Penrose and Rindler show how to obtain a spinor from the future and past
lightcones of an event. Can we use this to obtain 4 spinors --- one for
each normal vector --- associated to a disphenoid? Do they obey a
condition reflecting/encoding the
constraint $\sum_{i}\vec{n}_{i} = \vec{0}$? What is the ``total spin''
for the disphenoid if we describe it as a ``composite'' of these four spinors?
Could we measure the ``entanglement entropy'' for this spinorial
disphenoid --- if so, how does it relate to the surface area of a disphenoid?
\end{exercise}


%%
%% bib.tex
%% 
%% Made by alex
%% Login   <alex@tomato>
%% 
%% Started on  Thu Nov  3 14:28:43 2011 alex
%% Last update Wed May 30 10:14:48 2012 Alex Nelson
%%


\vfill\eject
\phantomsection\addcontentsline{toc}{section}{References} 
\begin{thebibliography}{99}
\bibitem{knuth} Donald Knuth,\newblock
``Teach Calculus with Big O.''\newblock
Eprint: \url{http://www-cs-staff.stanford.edu/~uno/ocalc.tex}\newblock
\emph{Not.\ of the AMS} {\bf45} (6): 687
\bibitem{livshits} Michael Livshits, \newblock
``You could simplify calculus.''\newblock
Eprint: \arXiv{0905.3611} \texttt{[math.HO]}
\bibitem{shchepin}
E.\ V.\ Shechepin,\newblock
``Gateway to Calculus: On Euler's Footsteps.''\newblock
Eprint: \url{http://www.mi.ras.ru/~scepin/uppsala.ps}
\end{thebibliography}


\end{document}