\begin{exercise}[Disphenoids for Minkowski vacuum]
If we want to describe Minkowski spacetime using Regge calculus,
then we need to use disphenoids (regular tetrahedra do not tessalate
[``fill''] space in 3 dimensions) with vertices located at $(1,0,0)$,
$(-1,0,0)$, $(0,1,1)$, $(0,1,-1)$.

\begin{enumerate}
\item What are the surface areas of each face of the disphenoid?
\item What is the volume of the disphenoid?
\item What is the centroid for the disphenoid? [Hint: we can compute it
  as the average of the vertices]
\item Find the unit normal vectors for each surface, which extends to
  the centroid (in the sense that if we had a line passing through its
  basepoint, moving in the direction of the unit normal, then it passes
  through the centroid).
\item Prove the unit normal vectors from the previous step all sum to
  the zero vector.
\end{enumerate}
\end{exercise}

\begin{exercise}
If we consider a disphenoid (as described in the previous problem),
there are four normal vectors $\vec{n}_{1}$, \dots, $\vec{n}_{4}$ on
each face. Suppose the magnitude of each normal is equal to the area of
its face, and suppose the base-point of the normal is such that it
connects to the centroid of the disphenoid if extended by a scalar
multiple.

Penrose and Rindler show how to obtain a spinor from the future and past
lightcones of an event. Can we use this to obtain 4 spinors --- one for
each normal vector --- associated to a disphenoid? Do they obey a
condition reflecting/encoding the
constraint $\sum_{i}\vec{n}_{i} = \vec{0}$? What is the ``total spin''
for the disphenoid if we describe it as a ``composite'' of these four spinors?
Could we measure the ``entanglement entropy'' for this spinorial
disphenoid --- if so, how does it relate to the surface area of a disphenoid?
\end{exercise}
