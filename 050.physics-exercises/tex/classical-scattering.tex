\section{Classical Scattering}

\subsection{The ``Classical Electron''}
Before continuing on, I will use the phrase ``classical electron'' to
refer to a sphere of radius
\begin{equation}
r_{e} = \frac{1}{4\pi\varepsilon_{0}}\frac{e^{2}}{m_{e}c}\approx2.81794\times10^{-15}~\mathrm{m}.
\end{equation}
It has uniform electric charge density, summing to a total charge equalling
\begin{equation}
q_{e} = \left(\frac{4}{3}\pi r_{e}^{3}\right)\rho_{q} = e\approx-1.602176634\times10^{-19}~\mathrm{C},
\end{equation}
and similarly has a uniform mass density, summing to a total mass of:
\begin{equation}
m_{e} = \left(\frac{4}{3}\pi r_{e}^{3}\right)\rho_{m} \approx 9.1093837015(28)\times10^{-31}~\mathrm{kg}.
\end{equation}
The total angular momentum from the rotation is $S=\hbar/2$, and the
axis of rotation is determined by the velocity vector.


\begin{exercise}[Classical ``electron'' scattering]
\textit{Fluff:} the spin of the electron contributes meaningfully to
scattering experiments. We will see if this is true classically.

\textit{Parameters:}
Consider a fixed point-particle of charge $q\neq0$.
Suppose we have a solid sphere of radius $r_{e}$ initially a distance
$\ell\gg r_{e}$ away from the fixed point-particle; the sphere has
uniform charge density $\rho_{c}$, uniform mass density $\rho_{m}$, and
rotates about its ``forward direction'' (the unit velocity
vector). Suppose its total angular momentum comes from this spin is $S=\hbar/2$.

%% For concrete values, the total charge of the sphere is equal to that of
%% the electron,
%% \begin{subequations}
%% \begin{equation}
%% q_{e} = \left(\frac{4}{3}\pi r_{e}^{3}\right)\rho_{q} = e\approx-1.602176634\times10^{-19}~\mathrm{C},
%% \end{equation}
%% and the total mass of the sphere is that of the electron,
%% \begin{equation}
%% m_{e} = \left(\frac{4}{3}\pi r_{e}^{3}\right)\rho_{m} \approx 9.1093837015(28)\times10^{-31}~\mathrm{kg}.
%% \end{equation}
%% Suppose the radius of the sphere is the classical electron radius,
%% \begin{equation}
%% r_{e} = \frac{1}{4\pi\varepsilon_{0}}\frac{e^{2}}{m_{e}c}\approx2.81794\times10^{-15}~\mathrm{m}.
%% \end{equation}
%% \end{subequations}
We could suppose $\ell=1.00~\mathrm{m}$, for example, which would
intuitively fit inside any laboratory.

\textit{Problem statement:}
If the solid sphere has impact parameter $b$, then what is its
scattering angle $\Theta$? What is its differential cross-sectional area $\D\sigma$?
How do these quantities depend on the spin $S$?
\end{exercise}

\begin{remark}
Recall, the moment of inertia for a sphere would be $I=\frac{2}{5}mr^{2}$
so its angular momentum would be $S=I\omega$.
\end{remark}

\begin{remark}
Since the ``electron'' is spinning and electrically charged, it will
generate a magnetic field. Does our solution reflect this? If we take
$r_{e}\to0$ limit, will our solution recover the familiar scattering of
a charged point-particle? If not, why not?
\end{remark}

\begin{exercise}[Stern--Gerlach, classically]
Suppose we have an inhomogeneous magnetic field $\vec{B}$.
Suppose we have a ``classical electron'' --- that is, a sphere of radius
$r_{e}$, mass $m_{e}$, uniform charge density which sums to a total
charge of $e$, uniform mass density, etc. Suppose further the
``classical electron'' is rotating about its direction of motion (i.e.,
the unit velocity vector for the electron).

If we send the classical electron through the inhomogeneous magnetic
field, then where will it end up? This is precisely the experimental
setup Stern and Gerlach devised, which was later interpreted as
measuring the spin of an electron.
\end{exercise}

\begin{remark}
Hint: use Larmor precession.
\end{remark}