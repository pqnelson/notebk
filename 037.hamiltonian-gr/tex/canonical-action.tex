\section{Canonical Action}

The first step, having rewritten the Lagrangian using 3-geometric
quantities, is to write down the action:
\begin{subequations}
  \begin{align}
  I &= \frac{1}{2\kappa^{2}}\int\mathcal{L}_{EH}\,\D^{3}x\,\D t\\
  &= \frac{1}{2\kappa^{2}}\int[{}^{(3)}R - K^{2} + K^{ij}K_{ij}]\lapse\sqrt{q}\,\D^{3}x\,\D t
  \end{align}
\end{subequations}
where $\kappa^{2}=8\pi G$. We now find the conjugate momentum density
\begin{equation}\label{eq:canonical-action:momentum-density-defn}
\pi^{ij} = \frac{\partial\mathcal{L}}{\partial(\partial_{0}q_{ij})}.
\end{equation}
But the action depends on $\partial_{0}q_{ij}$ only through the
extrinsic curvature, as we have seen by
Eq~\eqref{eq:extrinsic-curvature:as-time-derivative-of-three-metric}, so
we find:
\begin{equation}\label{eq:canonical-action:momentum-density}
\boxed{\pi^{ij} = \frac{1}{2\kappa^{2}}\sqrt{q}(K^{ij} - q^{ij}K)}.
\end{equation}
Thus concludes the first step in canonical analysis of general
relativity using metric variables.

The first trick is to write the Lagrangian using canonical
variables. Towards that end, we need to express the extrinsic curvature
tensor (and its trace) in terms of the momentum density, 
\begin{subequations}
\begin{equation}
K^{ij} = \frac{2\kappa^{2}}{2\sqrt{q}}(\pi q^{ij} - 2\pi^{ij})
\end{equation}
where $\pi=\pi^{ij}q_{ij}$, specifically
\begin{equation}
K = \frac{2\kappa^{2}\pi}{2\sqrt{q}}.
\end{equation}
Simple algebraic manipulation of 
Eq~\eqref{eq:extrinsic-curvature:as-time-derivative-of-three-metric}
and using the previous equations, we can readily express
$\partial_{t}q_{ij}$ in terms of the canonical momenta,
\begin{equation}\label{eq:canonical-action:time-derivative-of-metric}
\partial_{t}q_{ij} = {}^{(3)}\nabla_{i}\shift_{j} + {}^{(3)}\nabla_{j}\shift_{i}
- \lapse\frac{2\kappa^{2}}{\sqrt{q}}(\pi q_{ij} - 2\pi_{ij}).
\end{equation}
\end{subequations}
We can rewrite the Lagrangian using canonical variables
\begin{equation}
\boxed{\mathcal{L} = \frac{1}{2\kappa^{2}}{{}^{(3)}\!}R\; \lapse\sqrt{q} + \frac{2\kappa^{2}\lapse}{\sqrt{q}}(\pi^{ij}\pi_{ij}-\frac{1}{2}\pi^{2}).}
\end{equation}

%% \begin{danger}
%% We are a little sloppy here, we dropped the factor of $2\kappa^{2}$ from
%% the equation relating $K^{ij}$ in terms of the momenta density. Since
%% $\pi^{ij}\sim(2\kappa^{2})^{-1}$, we need to multiply by a factor of
%% $2\kappa^{2}$ in the ``kinetic term'' while dividing by $2\kappa^{2}$ in
%% the ``potential term'', to get:
%% \begin{equation}
%% \boxed{\mathcal{L} = \frac{1}{2\kappa^{2}}{{}^{(3)}\!}R\; \lapse\sqrt{q} + \frac{\lapse2\kappa^{2}}{\sqrt{q}}(\pi^{ij}\pi_{ij}-\frac{1}{2}\pi^{2}).}
%% \end{equation}
%% \end{danger}

We can then find the Hamiltonian density using our trusty Legendre transform,
\begin{equation}
\mathcal{H} = \pi^{ij}\partial_{0}q_{ij} - \mathcal{L}.
\end{equation}
We find, using Eq~\eqref{eq:canonical-action:time-derivative-of-metric},
\begin{equation}
  \begin{split}
  \pi^{ij}\partial_{0}q_{ij}
  &= \pi^{ij}({}^{(3)}\nabla_{i}\shift_{j} + {}^{(3)}\nabla_{j}\shift_{i}) - \frac{2\kappa^{2}\lapse}{\sqrt{q}}(\pi^{2} - 2\pi^{ij}\pi_{ij})\\
&= \pi^{ij}({}^{(3)}\nabla_{i}\shift_{j} + {}^{(3)}\nabla_{j}\shift_{i}) + 2\frac{2\kappa^{2}\lapse}{\sqrt{q}}(\pi^{ij}\pi_{ij}-\frac{1}{2}\pi^{2})
  \end{split}
\end{equation}
Hence
\begin{equation}
  \mathcal{H} = \pi^{ij}({}^{(3)}\nabla_{i}\shift_{j} + {}^{(3)}\nabla_{j}\shift_{i})
          - \frac{1}{2\kappa^{2}}{{}^{(3)}\!}R\; \lapse\sqrt{q} + \frac{2\kappa^{2}\lapse}{\sqrt{q}}(\pi^{ij}\pi_{ij}-\frac{1}{2}\pi^{2}).
\end{equation}
Since $\pi^{ij}$ is a symmetric tensor, the first term can be simplified
to $\pi^{ij}\nabla_{i}\shift_{j}$, giving us:
\begin{equation}
  \mathcal{H} = 2\pi^{ij}\;{{}^{(3)}\nabla_{i}}\shift_{j}
          - \frac{1}{2\kappa^{2}}{{}^{(3)}\!}R\; \lapse\sqrt{q} + \frac{2\kappa^{2}\lapse}{\sqrt{q}}(\pi^{ij}\pi_{ij}-\frac{1}{2}\pi^{2}).
\end{equation}
We are almost done.

If we care about the Hamiltonian (not just the Hamiltonian
\emph{density}), then we should instead work with:
\begin{equation}
H = \int_{\Sigma}\mathcal{H}\,\D^{3}x.
\end{equation}
We can perform integration by parts on the $\pi^{ij}\nabla_{i} \shift_{j}$
term, to get:
\begin{equation}
\boxed{H = \int_{\Sigma}\left(\lapse\frac{2\kappa^{2}}{\sqrt{q}}(\pi^{ij}\pi_{ij}-\frac{1}{2}\pi^{2})
-  \lapse\frac{1}{2\kappa^{2}}{{}^{(3)}\!}R\;\sqrt{q} - 2\shift_{j}\;{}^{(3)}\nabla_{i}\pi^{ij}
\right)\,\D^{3}x.}
\end{equation}
This form is preferable, because it fleshes out the role of the lapse
and shift functions as Lagrange multipliers. That is to say, we get
\begin{equation}
  H = \int_{\Sigma}\left(\lapse\mathcal{H}_{\perp} + \shift_{i}\mathcal{H}^{i}\right)\,\D^{3}x
\end{equation}
where
\begin{subequations}
  \begin{align}
    \mathcal{H}^{i} &= -2\;{}^{(3)}\nabla_{j}\pi^{ij}\\
    \mathcal{H}_{\perp} &= \frac{2\kappa^{2}}{\sqrt{q}}(\pi^{ij}\pi_{ij}-\frac{1}{2}\pi^{2})
- \frac{1}{2\kappa^{2}}{{}^{(3)}\!}R\; \sqrt{q}.
  \end{align}
\end{subequations}
These are the momentum and Hamiltonian constraints, respectively. We
should expect there to be constraints, since General Relativity is a
gauge system.

We may take a moment to note, we may write the action for general
relativity using the familiar canonical formalism for constrained
systems as
\begin{equation}
I_{ADM} = \int^{t_{1}}_{t_{0}}\int_{\Sigma}(\pi^{ij}\partial_{t}q_{ij} - \lapse\mathcal{H}_{\perp}
- \shift_{i}\mathcal{H}^{i})\,\D^{3}x\,\D t.
\end{equation}
Here it is made manifestly obvious the lapse function $\lapse$ acts as
Lagrange multiplier to the diffeomorphism constraint $\mathcal{H}_{\perp}$
and the shift vector $\shift_{i}$ acts as Lagrange multipliers to the
momentum constraints $\mathcal{H}^{i}$.

Now, as usual, we may define the Poisson bracket on a fixed time-slice
as
\begin{equation}
\{q_{ij}(\mathbf{x}), \pi^{k\ell}(\mathbf{y})\} = \frac{1}{2}(\delta^{k}_{i}\delta^{\ell}_{j}+\delta^{k}_{j}\delta^{\ell}_{i})\delta^{(3)}(\mathbf{x}-\mathbf{y}).
\end{equation}
Then Hamilton's equations are:
\begin{subequations}
  \begin{equation}
\partial_{t}q_{ij} = \{q_{ij},H\}
  \end{equation}
  and
  \begin{equation}
\partial_{t}\pi^{ij} = \{\pi^{ij},H\}.
  \end{equation}
\end{subequations}
These are equivalent to Einstein's field equations.

If we were being cautious, we should check that the constraints are
``consistent'' (in the sense that their Poisson bracket with the
Hamiltonian weakly vanishes).