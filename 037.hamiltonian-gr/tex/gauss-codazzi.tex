\section{Gauss--Codazzi Equation}

The slick way to relate ${}^{(4)}R$ to 3-geometric quantities, we begin
with the familiar relation of the Riemann curvature tensor to the
commutator of covariant derivatives:
\begin{equation}
{}^{(3)}{R^{\rho}}_{\sigma\mu\nu}v^{\sigma}
  = [{}^{(3)}\nabla_{\mu}, {}^{(3)}\nabla_{\nu}]v^{\rho}
\end{equation}
where $v^{\mu}\in T_{p}M$ is tangent to $\Sigma$ with base point
$p\in\Sigma$.
We arrive at the relation
\begin{equation}
{q^{\rho}}_{\alpha}{q^{\beta}}_{\sigma}{q^{\gamma}}_{\mu}{q^{\delta}}_{\nu}
{{}^{(4)}\!}{R^{\alpha}}_{\beta\gamma\delta}
={{}^{(3)}\!}{R^{\rho}}_{\sigma\mu\nu} + {K^{\rho}}_{\mu}K_{\sigma\nu}-{K^{\rho}}_{\nu}K_{\sigma\mu}.
\end{equation}
Contracting indices gives us Gauss's \textit{Theorema Egregium}:
\begin{equation}
{{}^{(4)}\!}{R} + 2{{}^{(4)}\!}{R}_{\mu\nu}n^{\mu}n^{\nu}
={{}^{(3)}\!}{R} + K^{2} - K_{\mu\nu}K^{\mu\nu}.
\end{equation}
Note, here $K=K_{ij}q^{ij}$ is the trace of the extrinsic curvature, so
the second term $K^{2}$ is the square of the trace.
This is half the battle.

We may cleverly pick the normal vector field in our considerations
\begin{equation}
{}^{(4)}{R^{\rho}}_{\sigma\mu\nu}n^{\sigma}
  = [{}^{(4)}\nabla_{\mu}, {}^{(4)}\nabla_{\nu}]n^{\rho}.
\end{equation}
\textbf{TODO:} finish the calculations here.

These slick considerations may be found in Danieli's thesis~\cite[see \S2.3]{danieli}.

\subsection{Rewriting the Lagrangian}

We may then rewrite the Einstein--Hilbert Lagrangian density
\begin{equation}
\mathcal{L}_{EH} = {}^{(4)}R\sqrt{-g}.
\end{equation}
We find
\begin{equation}
  \begin{split}
\mathcal{L}_{EH} &= [{}^{(3)}R - K^{2} + K^{ij}K_{ij}-2\nabla_{\alpha}(n^{\mu}\nabla_{\mu}n^{\alpha}-n^{\alpha}\nabla_{\mu}n^{\mu})]\lapse\sqrt{q}\\
&= [{}^{(3)}R - K^{2} + K^{ij}K_{ij}]\lapse\sqrt{q}+(\mbox{total deriative term}).
\end{split}
\end{equation}
Since total derivatives do not contribute to the action, we may discard
it, to get:
\begin{equation}
\boxed{\mathcal{L}_{EH}= [{}^{(3)}R - K^{2} + K^{ij}K_{ij}]\lapse\sqrt{q}.}
\end{equation}
Intuitively, we can think of the $-{}^{(3)}R$ term as the potential
energy, and the $K^{ij}K_{ij}-K^{2}$ as the kinetic energy (one way to
see this is that $\partial_{t}q_{ij}$ appears only in the extrinsic
curvature $K_{ij}$).

\begin{danger}
The Lagrangian we have written down produces the correct equations of
motion, but we often find the boundary terms are non-negligible. For
this reason, we typically need to add the Gibbons--York boundary term to
the Lagrangian. It doesn't change the canonical analysis of general
relativity substantially, but it impacts numerical modeling. Since we
are interested in the dynamics of general relativity, we will be
cavalier and stoic in light of these problems.
\end{danger}