\section{Schwarzschild Solution in ADM Formalism}

\textbf{TODO:} This turns out to be a very lengthy series of
calculations. It is probably a good idea to transform this into a series
of exercises, or isolate this into its own article.

\bigbreak

We will study the Schwarzschild metric
\begin{equation}
\D s^{2} = -\left(1 - \frac{2GM}{c^{2}}\frac{1}{r}\right)c^{2}\,\D t^{2}
+\left(1 - \frac{2GM}{c^{2}}\frac{1}{r}\right)^{-1}\D r^{2}
+r^{2}\,\D\Omega^{2}
\end{equation}
where $M$ is the mass of the gravitating body, $G$ is Newton's
gravitational constant, and
\begin{equation}
\D\Omega^{2} = \D\theta^{2} + \sin^{2}(\theta)\,\D\phi^{2}
\end{equation}
is the metric for the 2-sphere. Let us consider spatial hypersurfaces in
this choice of coordinates. We may frequently rely upon the parameter,
\begin{equation}
r_{s} = \frac{2GM}{c^{2}},
\end{equation}
known as the Schwarzschild radius.

We observe there is no term of the form $\D t\,\D x^{i}$ in the
Schwarzschild line element, which means the shift function is zero,
\begin{subequations}
\begin{equation}
\shift_{i}=\shift^{i}=0.
\end{equation}
But then examining the ADM line element in
Eq~\eqref{eq:cauchy-surface:adm-line-element},
we see that
\begin{equation*}
-\lapse^{2} = -\left(1 - \frac{2GM}{c^{2}}\frac{1}{r}\right),
\end{equation*}
or explicitly,
\begin{equation}
\lapse = \sqrt{1 - \frac{2GM}{c^{2}}\frac{1}{r}} = \sqrt{1 - \frac{r_{s}}{r}}.
\end{equation}
We can also determine the nonzero components of the 3-metric:
\begin{align}
  q_{rr} &= \left(1 - \frac{2GM}{c^{2}}\frac{1}{r}\right)^{-1}\\
  q_{\theta\theta} &= r^{2}\\
  q_{\phi\phi} &= r^{2}\sin^{2}(\theta).
\end{align}
\end{subequations}
All other components of the 3-metric vanish. The normal vector for this
particular choice of time-slicing is, using Eq~\eqref{eq;cauchy-surface:normal-vector-in-obvious-time-slicing},
\begin{subequations}
\begin{equation}
n^{\mu} = (-1/\lapse,\vec{0}),
\end{equation}
and
\begin{equation}
n_{\mu} = (\lapse,\vec{0}).
\end{equation}
\end{subequations}
So far, so good.

\begin{exercise}
Verify the non-vanishing spatial Christoffel symbols ${}^{(3)}{\Gamma^{i}}_{jk}$
associated with the 3-metric are:
\begin{subequations}
\begin{align}
{}^{(3)}{\Gamma^{r}}_{rr} &= \frac{-r_{s}}{2r(r-r_{s})}\\
{}^{(3)}{\Gamma^{r}}_{\theta\theta} &= -(r-r_{s})\\
{}^{(3)}{\Gamma^{r}}_{\phi\phi} &= -(r-r_{s})\sin^{2}(\theta)\\
{}^{(3)}{\Gamma^{\theta}}_{r\theta} &= 1/r\\
{}^{(3)}{\Gamma^{\phi}}_{r\phi} &= 1/r\\
{}^{(3)}{\Gamma^{\phi}}_{\theta\phi} &= \cot(\theta).
\end{align}
\end{subequations}
\end{exercise}

The first step in the canonical analysis is to write down the conjugate
momenta. We see from Eq~\eqref{eq:canonical-action:momentum-density-defn}
we need the extrinsic curvature $K^{ij}$ and its trace $K$.
We use Eq~\eqref{eq:extrinsic-curvature:as-time-derivative-of-metric}
to compute (neglecting terms involving $\shift_{i}$, since they're zero),
\begin{equation}
K_{ij} = \frac{1}{2N}\partial_{t}q_{ij} = 0,
\end{equation}
since $\partial_{0}q_{ij}=0$.

\subsection{Isotropic Radial Coordinates}

We have encountered a difficulty due to our choice of coordinates. Let
us see what happens with isotropic radial coordinates, where we use a
new radial coordinate $\bar{r}$ related to our old $r$ by:
\begin{equation}
r = \bar{r}\left(1 + \frac{r_{s}}{2\bar{r}}\right)^{2}.
\end{equation}
The line element becomes,
\begin{equation}
\D s^{2} = -\left(\frac{\displaystyle 1 - \frac{r_{s}}{4\bar{r}}}{\displaystyle 1 + \frac{r_{s}}{4\bar{r}}}\right)^{2}\,c^{2}\,\D t^{2} +
\left(1 + \frac{r_{s}}{4\bar{r}}\right)^{4}(\D\bar{r}^{2} +\bar{r}^{2}\,\D\theta^{2}+\bar{r}^{2}\sin^{2}(\theta)\,\D\varphi^{2}).
\end{equation}
Note: this is valid only for $\bar{r}>r_{s}/4$.

We find, in these new coordinates, as there is no term like $\D x^{i}\,\D t$
in the line element, the shift function is zero:
\begin{equation}
\shift^{i}=\shift_{i}=0.
\end{equation}
The lapse function is then
\begin{equation}
\lapse = \frac{\displaystyle 1 - \frac{r_{s}}{4\bar{r}}}{\displaystyle 1 + \frac{r_{s}}{4\bar{r}}}.
\end{equation}
The spatial metric has nonzero components
\begin{subequations}
\begin{align}
q_{rr} &= \left(1 + \frac{r_{s}}{4\bar{r}}\right)^{4},\\
q_{\theta\theta} &= \left(1 + \frac{r_{s}}{4\bar{r}}\right)^{4}\bar{r}^{2},\\
q_{\varphi\varphi} &= \left(1 + \frac{r_{s}}{4\bar{r}}\right)^{4}\bar{r}^{2}\sin^{2}(\theta).
\end{align}
\end{subequations}
This experiences the same problem as before, the extrinsic curvature
vanishes.

\subsection{Painlev\'{e}--Gullstrand Coordinates}
The really interesting situation is when we use the
Painlev\'{e}--Gullstrand coordinates,
\begin{equation}
\D s^{2} = -c^{2}\,\D t^{2} + \left(\D r + \sqrt{\frac{r_{s}}{r}}c\,\D t\right)^{2}
+r^{2}\,\D\Omega^{2}.
\end{equation}
We find
\begin{subequations}
\begin{align}
\shift &= 1\\
\lapse^{i} &= (\sqrt{r_{s}/r}, 0, 0)\\
q_{ij} &= \mathrm{diag}(1,r^{2},r^{2}\sin^{2}(\theta)).
\end{align}
\end{subequations}
We can observe that $q_{ij}$ describes a flat metric.

We find
\begin{subequations}
\begin{align}
K_{rr} &= -\sqrt{\frac{r_{s}}{4r^{3}}}\\
K_{\theta\theta} &= \sqrt{r_{s}r}\\
K_{\varphi\varphi} &= \sqrt{r_{s}r}\sin^{2}(\theta).
\end{align}
\end{subequations}
This implies the trace of the extrinsic curvature is
\begin{equation}
  \begin{split}
K &= K_{ij}q^{ij} = -\sqrt{\frac{r_{s}}{4r^{3}}} + \sqrt{\frac{r_{s}}{r}} + \sqrt{\frac{r_{s}}{r}}\\
&= \left(2 - \frac{1}{2r}\right)\sqrt{\frac{r_{s}}{r}}.
  \end{split}
\end{equation}
The inverse of the extrinsic curvature is given by
$K^{ij}=q^{ik}q^{j\ell}K_{k\ell}$, so we find
\begin{align*}
K^{rr} = q^{rk}q^{r\ell}K_{k\ell} = q^{rr}q^{rr}K_{rr}\\
K^{\theta\theta}=q^{\theta k}q^{\theta\ell}K_{k\ell} = q^{\theta\theta}q^{\theta\theta}K_{\theta\theta}\\
K^{\varphi\varphi}=q^{\varphi k}q^{\varphi\ell}K_{k\ell} = q^{\varphi\varphi}q^{\varphi\varphi}K_{\varphi\varphi}.
\end{align*}
These have components
\begin{subequations}
\begin{align}
K^{rr} &= K_{rr} = -\sqrt{\frac{r_{s}}{4r^{3}}}\\
K^{\theta\theta} &= r^{-4}K_{\theta\theta} = \frac{1}{r^{3}}\sqrt{\frac{r_{s}}{r}}\\
K^{\varphi\varphi} &= \frac{1}{r^{4}\sin^{4}(\theta)}K_{\varphi\varphi}
= \frac{1}{r^{2}\sin^{2}(\theta)}\sqrt{r_{s}r}.
\end{align}
\end{subequations}
Observe $\sqrt{q}=r^{2}\sin(\theta)$.
We then find the conjugate momenta using
Eq~\eqref{eq:canonical-action:momentum-density}, which is diagonal with components:
\begin{subequations}
\begin{align}
\pi^{rr} &= \frac{\sqrt{q}}{8\pi G}\left(-\sqrt{\frac{r_{s}}{4r^{3}}}-\left(2 - \frac{1}{2r}\right)\sqrt{\frac{r_{s}}{r}}\right)\\
\pi^{\theta\theta} &= \frac{\sqrt{q}}{8\pi G}\left(\frac{\sqrt{r_{s}r}}{r^{4}}-\left(2 - \frac{1}{2r}\right)\sqrt{\frac{r_{s}}{r}}\right)\\
\pi^{\varphi\varphi} &= \frac{\sqrt{q}}{8\pi G}\left(
\frac{\sqrt{r_{s}r}}{r^{2}\sin^{2}(\theta)} - \left(2 - \frac{1}{2r}\right)\sqrt{\frac{r_{s}}{r}}.
\right)
\end{align}
\end{subequations}
We see that $\pi^{ii}\sim\sqrt{q}\sqrt{r_{s}r}/r^{3\pm1}$.