\section{Cauchy Surfaces}

We will work with the East coast metric signature $(-+++)$. The proper
distance between two infinitesimally close points in spacetime $x^{\mu}$
and $x^{\mu} + \D x^{\mu}$ is given by
\begin{equation}
\D s^{2} = g_{\mu\nu}\,\D x^{\mu}\D x^{\nu}.
\end{equation}
But we will foliate spacetime into a family of spatial
hypersurfaces. For now, suppose we have such a ``time-slicing'', a
foliation of spacetime into spatial hypersurfaces $\Sigma_{t}$ indexed
by the time parameter.

The two points have local coordinates $(t,x^{i})$ and $(t+\D t, x^{i} + \D x^{i})$.
The basic steps are as follows:
\begin{enumerate}
\item Choose two nearby spatial hypersurfaces $\Sigma_{t}$ and
  $\Sigma_{t+\D t}$ with positive-definite spatial metrics $q_{ij}(t)$
  and $q_{ij}(t + \D t)$, respectively.
\item Pick an initial point $(t, x^{i})$ on $\Sigma_{t}$ and move
  orthogonally (i.e., according to the normal unit vector) to
  $\Sigma_{t+\D t}$. This would take proper time $\D\tau=\lapse\,\D t$ where
  $\lapse$ is the \define{Lapse Function} which measures the rate of flow of
  proper time $\tau$ with respect to coordinate time $t$ as one moves
  normally (``orthogonally'').
\item The motion will end at some point on $\Sigma_{t + \D t}$, but its
  spatial coordinates may not necessarily be $x^{i}$ (these are
  arbitrary coordinates) but may be shifted to nearby values
  $x^{i}+\shift^{i}\,\D t$; this ``drift'' is because we move along the unit
  normal rather than the gradient of the time function, so we need to
  consult this \define{Shift Vector} $\shift^{i}$ which measures how much the
  local spatial coordinate system shifts tangential to $\Sigma_{t}$ when
  moving to $\Sigma_{t + \D t}$.
\item On $\Sigma_{t+\D t}$ we can now perform the additional
  displacement $x^{i} + \D x^{i}$.
\end{enumerate}
Combining everything together, we find obtain the ADM line element
\begin{equation}
\D s^{2} = -\lapse^{2}\,\D t^{2}
  + q_{ij}(\D x^{i} + \shift^{i}\,\D t)(\D x^{j} + \shift^{j}\,\D t).
\end{equation}
We use Latin indices starting from $i$ to track spatial components and
3-tensors. The convention will be to raise and lower spatial indices by
$q^{ij}$ (the inverse of the 3-metric $q_{ij}$) and $q_{ij}$, respectively; in particular, $\shift_{i}=q_{ij}\shift^{j}$.
We can rewrite the ADM line element as
\begin{equation}
  \D s^{2} = (-\lapse^{2} + \shift_{i}\shift^{i})\,\D t^{2}
  + 2 \shift_{k}\,\D x^{k}\D t
  + q_{ij}\,\D x^{i}\D x^{j}.
\end{equation}
Comparing to the ``covariant'' version, we find the 4-metric can be
written in block form as
\begin{equation}
  g_{\mu\nu} = \begin{pmatrix}
    -\lapse^{2} + \shift_{k}\shift^{k} & \shift_{j}\\
    \shift_{i} & q_{ij}
\end{pmatrix}.
\end{equation}
What, then, is $g^{\mu\nu}$ the inverse 4-metric in terms of the
inverse 3-metric $q^{ij}$, lapse function, and shift vector?

Recall we can compute the matrix inverse for a block matrix as:
\begin{equation}
  \begin{bmatrix}
    \mathbf{A} & \mathbf{B} \\
    \mathbf{C} & \mathbf{D}
   \end{bmatrix}^{-1} 
    = \begin{bmatrix}
     \left(\mathbf{A} - \mathbf{BD}^{-1}\mathbf{C}\right)^{-1} &
    -\left(\mathbf{A} - \mathbf{BD}^{-1}\mathbf{C}\right)^{-1}\mathbf{BD}^{-1} \\
    -\mathbf{D}^{-1}\mathbf{C}\left(\mathbf{A} - \mathbf{BD}^{-1}\mathbf{C}\right)^{-1} &
    %\quad
    \mathbf{D}^{-1} + \mathbf{D}^{-1}\mathbf{C}\left(\mathbf{A} - \mathbf{BD}^{-1}\mathbf{C}\right)^{-1}\mathbf{BD}^{-1}
  \end{bmatrix}.
\end{equation}
Using this, we find
\begin{equation}
  g^{\mu\nu}
  = \begin{pmatrix} -1/\lapse^{2} & \shift^{j}/\lapse^{2}\\
    \shift^{i}/\lapse^{2} & q^{ij} - (\shift^{i}\shift^{j}/\lapse^{2})
  \end{pmatrix}.
\end{equation}
\textbf{Important:} the spatial part of the inverse spacetime metric
$g^{ij}$ \emph{is not equal to} the inverse of the spatial metric $q^{ij}$.

Similarly, using the property that the determinant for a matrix in block
form
\begin{equation}
\det\begin{pmatrix}A & B\\ C & D
\end{pmatrix} = \det(D)\det(A - CD^{-1}B),
\end{equation}
we find
\begin{equation}
\det(g_{\mu\nu}) = -N^{2}\det(q_{ij}).
\end{equation}

\subsection{Choice of Time-Slicing}

We only required foliating space-time by spatial hypersurfaces
$\Sigma_{t}$, a ``one-parameter family'' of such surfaces. The parameter
$t$ is commonly identified with $x^{0}$ the time coordinate. But we
could pick \emph{arbitrary} coordinates. This corresponds to choosing
different families of spatial hypersurfaces.

If we pick our family of spatial hypersurfaces such that a [particular,
fixed] scalar function $T(x^{\mu})$ is constant, then $T$ is a perfectly
valid alternative choice of time coordinate. For example, later we will
introduce a notion of ``extrinsic curvature'' $K_{ij}$ and its trace
$K=q^{ij}K_{ij}$ is a perfectly good time coordinate for a large class
of manifolds called \define{York Time Slicing}.

We should recall, in this particular case, we can find the normal vector
$n^{\mu}$ to the hypersurface $\Sigma_{T}$ by taking the gradient
\begin{equation}
n^{\mu} = g^{\mu\nu}\partial_{\nu}T(x).
\end{equation}
This is a well-defined 4-vector when restricted to the spatial
hypersurface $\Sigma_{T}$. For space-like hypersurfaces, we see
\begin{equation}
n^{\mu}n_{\mu} = -1.
\end{equation}
For time-like hypersurfaces, its norm would be $+1$; and null
hypersurfaces have null normal vectors.

We may construct the three-metric from the space-time metric
$g_{\mu\nu}$ using the normal vectors:
\begin{equation}
q_{\mu\nu} = g_{\mu\nu} + n_{\mu}n_{\nu}.
\end{equation}
The quantity $q^{\mu\rho}q_{\rho\nu} = {q^{\mu}}_{\nu}$ acts like a
projection operator onto the spatial part of 4-quantities. We observe
this intuition is warranted since
\begin{subequations}
\begin{align}
q^{\mu\rho}q_{\rho\nu}
&= (g^{\mu\rho} + n^{\mu}n^{\rho})(g_{\rho\nu} + n_{\rho}n_{\nu})\\
&= {\delta^{\mu}}_{\rho} + n^{\mu}n_{\nu} + n^{\mu}n_{\nu} + n^{\mu}n^{\rho}n_{\rho}n_{\nu}\\
&= {\delta^{\mu}}_{\rho} + n^{\mu}n_{\nu}.
\end{align}
\end{subequations}

We are free to describe the ``flow of time''
using a time-like vector field $t^{\mu}$ such that
\begin{equation}
t^{\mu}\nabla_{\mu}T=1.
\end{equation}
This may or may not coincide with the normal vector. It is important to
note that the ``evolution'' vector $t^{\mu}$ satisfies
\begin{equation}
t^{\alpha}\partial_{\alpha} = \partial_{T}.
\end{equation}

Using these two vectors (the unit normal vector and the ``flow of time''
vector), however, we may construct the lapse function:
\begin{equation}
\lapse = -g_{\mu\nu}t^{\mu}n^{\nu}.
\end{equation}
We may also construct the shift vector:
\begin{equation}
\shift^{\mu}  = {q^{\mu}}_{\nu}t^{\nu}.
\end{equation}
Alternatively, we may define the lapse $\lapse$ and shift vector
$\shift^{\mu}$ by
\begin{equation}\label{eq:lapse-and-shift-in-terms-of-t-and-normal}
t^{\mu} = \lapse n^{\mu} + \shift^{\mu} \implies n^{\mu} = \frac{t^{\mu}-\shift^{\mu}}{\lapse};
\end{equation}
that is, the lapse equals the projection of $t^{\mu}$ onto the unit
normal $n^{\mu}$ to $\Sigma_{T}$, while the shift is the projection of
$t^{\mu}$ onto the Cauchy surface $\Sigma_{T}$.

We may interpret, as before, $\lapse$ as the ratio of proper time (given
by $t^{\mu}\nabla_{\mu} T=1$) to coordinate time $n^{\mu}\nabla_{\mu}T$.

We claim, and may be verified by the reader, that $\shift^{\mu}$ is a
three-dimensional object, in the sense that
\begin{equation}
h_{\mu\nu}\shift^{\mu} = 0.
\end{equation}
Then we claim that
\begin{equation}
-n_{\mu}t^{\mu} = -\underbrace{n_{\mu}n^{\mu}}_{=-1}\lapse - \underbrace{n_{\mu}\shift^{\mu}}_{=0}
= \lapse.
\end{equation}
The components of the normal vector can be found from the one-form
$n_{\mu}\,\D x^{\mu}=-\lapse\,\D t$ (which matches the motivation for the
lapse function in the first place) as
\begin{equation}
n^{\mu} = g^{\mu\nu}n_{\nu} = \left(\frac{-1}{\lapse},\frac{\shift^{i}}{\lapse}\right).
\end{equation}
When compared to Eq~\eqref{eq:lapse-and-shift-in-terms-of-t-and-normal},
we find
\begin{equation}
t^{\mu} = (-1, 0, 0, 0)
\end{equation}
as we would expect.