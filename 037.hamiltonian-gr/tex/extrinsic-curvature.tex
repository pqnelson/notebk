\section{Extrinsic Curvature}

\begin{ddanger}
The conventions on the \textsc{sign} of extrinsic curvature varies wildly. I
believe I have a consistent choice, but it is possible I have erred.
\end{ddanger}

From the Hamiltonian perspective, the first step amounts to finding the
conjugate momentum. Ordinarily, these are just the time derivatives of
the generalized positions. For general relativity, with $q_{ij}$ as the
generalized positions, the time derivatives $\partial_{t}q_{ij}$ do not
suffice: they do not form a tensor. In general, the time derivatives of
the components of a tensor will not form a tensor.\footnote{This
motivation comes from Carlip~\cite{carlip2019}, which I find
compelling.} Thus we need to take a geometric detour.

If we have foliated space-time into spatial hypersurfaces $\Sigma_{t}$,
then we may consider unit normal 4-vectors $n^{\mu}$. The extrinsic
curvature may be computed using the Lie derivative of the 3-metric with
respect to $n^{\mu}$ as
\begin{equation}\label{eq:extrinsic-curvature:as-time-derivative-of-metric}
  \begin{split}
  K_{ij} &= \frac{-1}{2}(\mathcal{L}_{\mathbf{n}}q)_{ij}\\
  &= \frac{-1}{2}n^{\mu}\partial_{\mu}q_{ij} - \frac{1}{2} q_{\mu j}(\partial_{i}n^{\mu}) - \frac{1}{2} q_{i\mu}(\partial_{j}n^{\mu}),
  \end{split}
\end{equation}
where $\nabla_{\mu}$ is the covariant derivative compatible with the
4-metric $g_{\mu\nu}$, and we recall
\begin{equation}
q_{\mu\nu} = g_{\mu\nu} + n_{\mu}n_{\nu}.
\end{equation}
We may also consider $K_{ij}$ as the spatial  part of
\begin{equation}
K_{\mu\nu} = n_{(\mu;\nu)} = \frac{1}{2}(\nabla_{\mu}n_{\nu} + \nabla_{\nu}n_{\mu}).
\end{equation}
Both formulations make clear the extrinsic curvature tensor is
symmetric:
\begin{equation}
K_{ij} = K_{ji}.
\end{equation}

A third route to explain the extrinsic curvature tensor is simply by
considering the spatial projection of the covariant derivative of the unit
normal vector:
\begin{equation}
K_{\mu\nu} = {q_{\mu}}^{\rho}\nabla_{\rho}n_{\nu}.
\end{equation}
It requires invoking Frobenius's theorem (from differential topology) to
demonstrate this is a symmetric tensor field. We recall
\begin{equation}
n^{\mu}{q_{\mu}}^{\nu} = 0,
\end{equation}
which implies
\begin{equation}
K_{\mu\nu}n^{\mu} = 0.
\end{equation}
Thus this tensor is purely spatial.

Specifically, the useful \emph{fourth} version of extrinsic curvature,
\begin{equation}\label{eq:extrinsic-curvature:as-time-derivative-of-three-metric}
K_{ij} = \frac{1}{2N}(\partial_{t}q_{ij} - {{}^{(3)}}\nabla_{i}N_{j} - {{}^{(3)}}\nabla_{j}N_{i}).
\end{equation}
This requires a length calculation. The basic idea is to define the
extrinsic curvature in a coordinate-free manner
\begin{equation}
K(u,v) = -g(\nabla_{u}v,\vec{n}).
\end{equation}

\begin{lemma}
In local coordinates, $K_{ij} = N\Gamma^{0}_{ij}$.
\end{lemma}
\begin{proof}[Proof (length calculation)]
Using local coordinates, we find the components of the extrinsic curvature as
\begin{subequations}
  \begin{align}
    K_{ij} &= K(\partial_{i}, \partial_{j})\\
    &= -g(\nabla_{\partial_{i}}\partial_{j}, \vec{n})\\
    &= -g(\Gamma^{\mu}_{ij}\partial_{\mu}, n^{\nu}\partial_{\nu})\\
    &= -\Gamma^{\mu}_{ij}n^{\nu}g_{\mu\nu}\\
    &= -\Gamma^{0}_{ij}n_{0}\\
    &= N\Gamma^{0}_{ij}
  \end{align}
\end{subequations}
where we have implicitly used the fact that
$n_{\mu}=g_{\mu\nu}n^{\nu}=(-N,0,0,0)$.
\end{proof}

\begin{lemma}
  The Christoffel symbol is
  \begin{equation}
\Gamma^{0}_{ij} = \frac{1}{2N^{2}}(\partial_{t}q_{ij} - {{}^{(3)}}\nabla_{i}N_{j} - {{}^{(3)}}\nabla_{j}N_{i}).
  \end{equation}
\end{lemma}
\begin{proof}[Proof (length calculation)]
  We can compute this directly,
  \begin{subequations}
    \begin{align}
\Gamma^{0}_{ij} &= \frac{1}{2} g^{0\mu}(\partial_{i}g_{\mu j} + \partial_{j}g_{i\mu} - \partial_{\mu} g_{ij}) \\
        &= \frac{1}{2}\left[ g^{00}(\partial_{i}g_{0j} + \partial_{j}g_{i0} - \partial_{0} g_{ij}) + g^{0k}(\partial_{i}g_{kj} + \partial_{j}g_{ik} - \partial_{k} g_{ij})  \right] \\
        &= \frac{1}{2}\left[ -\frac{1}{N^{2}} (\partial_{i}N_{j}+ \partial_{j}N_{i}- \partial_{t} q_{ij}) + \frac{N^{k}}{N^{2}} (\partial_{i}q_{kj} + \partial_{j}q_{ik} - \partial_{k} q_{ij}) \right] \\
        &= \frac{1}{2N^{2}}\left[ (\partial_{t} q_{ij} - \partial_{i}N_{j}- \partial_{j}N_{i}) + N_{\ell} q^{\ell k} (\partial_{i}q_{kj} + \partial_{j}q_{ik} - \partial_{k} q_{ij}) \right] \\
        &= \frac{1}{2N^{2}}\left[ (\partial_{t} q_{ij} - \partial_{i} N_{j} - \partial_{j}N_{i}) + 2 N_{\ell} \,{{}^{(3)}}\Gamma^{\ell}_{ij} \right] \\
        &= \frac{1}{2N^{2}} \left(\partial_{t} q_{ij} - {{}^{(3)}\nabla}_{i}N_{j}- {{}^{(3)}\nabla}_{j}N_{i}\right),
    \end{align}
  \end{subequations}
  where we have explicitly used the Christoffel symbol for the spatial hypersurface,
  \begin{equation}
    {{}^{(3)}}\Gamma^{\ell}_{ij} = \frac{1}{2}q^{\ell k}(\partial_{i}q_{kj} + \partial_{j}q_{ik} - \partial_{k}q_{ij}).
  \end{equation}
  Thus concludes our calculation.
\end{proof}

Now combining our two lemmas together, we obtain the desired form of the
extrinsic curvature in Eq~\eqref{eq:extrinsic-curvature:as-time-derivative-of-three-metric}.

However, the intuition of the extrinsic curvature is that it is the
``velocity'' corresponding to the 3-metric $q_{ij}$:
\begin{equation}
\boxed{K_{ij}\mathrel{\mbox{``=''}}\partial_{t}q_{ij}.}
\end{equation}
The strategy
\emph{now} is to rewrite the Ricci scalar for spacetime ${}^{(4)}R$ in
terms of the extrinsic curvature $K_{ij}$, spatial 3-curvature tensor
${}^{(3)}R_{ijk\ell}$, and spatial Ricci 3-tensor ${}^{(3)}R_{ij}$. This
will be used to write the Einstein--Hilbert Lagrangian in terms of
spatial tensors, then we will find the canonically conjugate momentum
$\pi^{ij}$ to the 3-metric $q_{ij}$, and continue on our merry way with
the Hamiltonian formalism.
