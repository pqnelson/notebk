\documentclass{article}

\usepackage[12hr,us]{datetime}
\usepackage[top=6pc,textheight=8.9in,textwidth=33pc,inner=6pc,marginparwidth=9pc,marginparsep=1.5pc]{geometry}
\usepackage{amsmath,amsfonts,amscd,amssymb}
\usepackage{extpfeil,manfnt}
%\usepackage{epic,eepic}
\usepackage{arydshln} % for the dashed lines in the ADM metric
\usepackage{float}
\usepackage{framed}
\usepackage{ifpdf}
\usepackage[final]{graphicx}
\ifpdf
\DeclareGraphicsRule{*}{mps}{*}{}
\usepackage[kerning]{microtype}
\fi
\usepackage{mathrsfs}
\usepackage{marginnote}
\usepackage{wrapfig} % used for wrapping figures!
\usepackage{amsthm}
\usepackage{fancyhdr}
\usepackage[final,colorlinks=true, 
            hyperindex=true,
            citecolor=black,
            filecolor=black,
            menucolor=black,
            linkcolor=black,
            urlcolor=black,
            bookmarksopen=true,
            pdfauthor={Alex Nelson}]{hyperref}
\usepackage[all]{hypcap}
\usepackage{notation}
\usepackage{danger}
\numberwithin{equation}{section}
\def\homeurl{\url{https://pqnelson.github.io/notebk/}}
\newcommand\arXiv[1]{\href{https://arxiv.org/abs/#1}{\texttt{arXiv:#1}}}

\newtheorem{lemma}{Lemma}
\newtheorem{exercise}{Exercise}

\title{Hamiltonian Formalism of General Relativity}
\author{Alex Nelson\thanks{This is a page from \homeurl{}\hfil\break\indent\;\, Compiled:\enspace\today\ at \currenttime\ (PST)}}
\date{October 3, 2022}
\begin{document}
\maketitle

% http://www.tapir.caltech.edu/~chirata/ph236/2011-12/
% https://arxiv.org/abs/gr-qc/0006091
% https://arxiv.org/abs/gr-qc/0405109
\section{Cauchy Surfaces}

We will work with the East coast metric signature $(-+++)$. The proper
distance between two infinitesimally close points in spacetime $x^{\mu}$
and $x^{\mu} + \D x^{\mu}$ is given by
\begin{equation}
\D s^{2} = g_{\mu\nu}\,\D x^{\mu}\D x^{\nu}.
\end{equation}
But we will foliate spacetime into a family of spatial
hypersurfaces. For now, suppose we have such a ``time-slicing'', a
foliation of spacetime into spatial hypersurfaces $\Sigma_{t}$ indexed
by the time parameter.

The two points have local coordinates $(t,x^{i})$ and $(t+\D t, x^{i} + \D x^{i})$.
The basic steps are as follows:
\begin{enumerate}
\item Choose two nearby spatial hypersurfaces $\Sigma_{t}$ and
  $\Sigma_{t+\D t}$ with positive-definite spatial metrics $q_{ij}(t)$
  and $q_{ij}(t + \D t)$, respectively.
\item Pick an initial point $(t, x^{i})$ on $\Sigma_{t}$ and move
  orthogonally (i.e., according to the normal unit vector) to
  $\Sigma_{t+\D t}$. This would take proper time $\D\tau=\lapse\,\D t$ where
  $\lapse$ is the \define{Lapse Function} which measures the rate of flow of
  proper time $\tau$ with respect to coordinate time $t$ as one moves
  normally (``orthogonally'').
\item The motion will end at some point on $\Sigma_{t + \D t}$, but its
  spatial coordinates may not necessarily be $x^{i}$ (these are
  arbitrary coordinates) but may be shifted to nearby values
  $x^{i}+\shift^{i}\,\D t$; this ``drift'' is because we move along the unit
  normal rather than the gradient of the time function, so we need to
  consult this \define{Shift Vector} $\shift^{i}$ which measures how much the
  local spatial coordinate system shifts tangential to $\Sigma_{t}$ when
  moving to $\Sigma_{t + \D t}$.
\item On $\Sigma_{t+\D t}$ we can now perform the additional
  displacement $x^{i} + \D x^{i}$.
\end{enumerate}
Combining everything together, we find obtain the ADM line element
\begin{equation}
\D s^{2} = -\lapse^{2}\,\D t^{2}
  + q_{ij}(\D x^{i} + \shift^{i}\,\D t)(\D x^{j} + \shift^{j}\,\D t).
\end{equation}
We use Latin indices starting from $i$ to track spatial components and
3-tensors. The convention will be to raise and lower spatial indices by
$q^{ij}$ (the inverse of the 3-metric $q_{ij}$) and $q_{ij}$, respectively; in particular, $\shift_{i}=q_{ij}\shift^{j}$.
We can rewrite the ADM line element as
\begin{equation}
  \D s^{2} = (-\lapse^{2} + \shift_{i}\shift^{i})\,\D t^{2}
  + 2 \shift_{k}\,\D x^{k}\D t
  + q_{ij}\,\D x^{i}\D x^{j}.
\end{equation}
Comparing to the ``covariant'' version, we find the 4-metric can be
written in block form as
\begin{equation}
  g_{\mu\nu} = \begin{pmatrix}
    -\lapse^{2} + \shift_{k}\shift^{k} & \shift_{j}\\
    \shift_{i} & q_{ij}
\end{pmatrix}.
\end{equation}
What, then, is $g^{\mu\nu}$ the inverse 4-metric in terms of the
inverse 3-metric $q^{ij}$, lapse function, and shift vector?

Recall we can compute the matrix inverse for a block matrix as:
\begin{equation}
  \begin{bmatrix}
    \mathbf{A} & \mathbf{B} \\
    \mathbf{C} & \mathbf{D}
   \end{bmatrix}^{-1} 
    = \begin{bmatrix}
     \left(\mathbf{A} - \mathbf{BD}^{-1}\mathbf{C}\right)^{-1} &
    -\left(\mathbf{A} - \mathbf{BD}^{-1}\mathbf{C}\right)^{-1}\mathbf{BD}^{-1} \\
    -\mathbf{D}^{-1}\mathbf{C}\left(\mathbf{A} - \mathbf{BD}^{-1}\mathbf{C}\right)^{-1} &
    %\quad
    \mathbf{D}^{-1} + \mathbf{D}^{-1}\mathbf{C}\left(\mathbf{A} - \mathbf{BD}^{-1}\mathbf{C}\right)^{-1}\mathbf{BD}^{-1}
  \end{bmatrix}.
\end{equation}
Using this, we find
\begin{equation}
  g^{\mu\nu}
  = \begin{pmatrix} -1/\lapse^{2} & \shift^{j}/\lapse^{2}\\
    \shift^{i}/\lapse^{2} & q^{ij} - (\shift^{i}\shift^{j}/\lapse^{2})
  \end{pmatrix}.
\end{equation}
\textbf{Important:} the spatial part of the inverse spacetime metric
$g^{ij}$ \emph{is not equal to} the inverse of the spatial metric $q^{ij}$.

Similarly, using the property that the determinant for a matrix in block
form
\begin{equation}
\det\begin{pmatrix}A & B\\ C & D
\end{pmatrix} = \det(D)\det(A - CD^{-1}B),
\end{equation}
we find
\begin{equation}
\det(g_{\mu\nu}) = -N^{2}\det(q_{ij}).
\end{equation}

\subsection{Choice of Time-Slicing}

We only required foliating space-time by spatial hypersurfaces
$\Sigma_{t}$, a ``one-parameter family'' of such surfaces. The parameter
$t$ is commonly identified with $x^{0}$ the time coordinate. But we
could pick \emph{arbitrary} coordinates. This corresponds to choosing
different families of spatial hypersurfaces.

If we pick our family of spatial hypersurfaces such that a [particular,
fixed] scalar function $T(x^{\mu})$ is constant, then $T$ is a perfectly
valid alternative choice of time coordinate. For example, later we will
introduce a notion of ``extrinsic curvature'' $K_{ij}$ and its trace
$K=q^{ij}K_{ij}$ is a perfectly good time coordinate for a large class
of manifolds called \define{York Time Slicing}.

We should recall, in this particular case, we can find the normal vector
$n^{\mu}$ to the hypersurface $\Sigma_{T}$ by taking the gradient
\begin{equation}
n^{\mu} = g^{\mu\nu}\partial_{\nu}T(x).
\end{equation}
This is a well-defined 4-vector when restricted to the spatial
hypersurface $\Sigma_{T}$. For space-like hypersurfaces, we see
\begin{equation}
n^{\mu}n_{\mu} = -1.
\end{equation}
For time-like hypersurfaces, its norm would be $+1$; and null
hypersurfaces have null normal vectors.

We may construct the three-metric from the space-time metric
$g_{\mu\nu}$ using the normal vectors:
\begin{equation}
q_{\mu\nu} = g_{\mu\nu} + n_{\mu}n_{\nu}.
\end{equation}
The quantity $q^{\mu\rho}q_{\rho\nu} = {q^{\mu}}_{\nu}$ acts like a
projection operator onto the spatial part of 4-quantities. We observe
this intuition is warranted since
\begin{subequations}
\begin{align}
q^{\mu\rho}q_{\rho\nu}
&= (g^{\mu\rho} + n^{\mu}n^{\rho})(g_{\rho\nu} + n_{\rho}n_{\nu})\\
&= {\delta^{\mu}}_{\rho} + n^{\mu}n_{\nu} + n^{\mu}n_{\nu} + n^{\mu}n^{\rho}n_{\rho}n_{\nu}\\
&= {\delta^{\mu}}_{\rho} + n^{\mu}n_{\nu}.
\end{align}
\end{subequations}

We are free to describe the ``flow of time''
using a time-like vector field $t^{\mu}$ such that
\begin{equation}
t^{\mu}\nabla_{\mu}T=1.
\end{equation}
This may or may not coincide with the normal vector. It is important to
note that the ``evolution'' vector $t^{\mu}$ satisfies
\begin{equation}
t^{\alpha}\partial_{\alpha} = \partial_{T}.
\end{equation}

Using these two vectors (the unit normal vector and the ``flow of time''
vector), however, we may construct the lapse function:
\begin{equation}
\lapse = -g_{\mu\nu}t^{\mu}n^{\nu}.
\end{equation}
We may also construct the shift vector:
\begin{equation}
\shift^{\mu}  = {q^{\mu}}_{\nu}t^{\nu}.
\end{equation}
Alternatively, we may define the lapse $\lapse$ and shift vector
$\shift^{\mu}$ by
\begin{equation}\label{eq:lapse-and-shift-in-terms-of-t-and-normal}
t^{\mu} = \lapse n^{\mu} + \shift^{\mu} \implies n^{\mu} = \frac{t^{\mu}-\shift^{\mu}}{\lapse};
\end{equation}
that is, the lapse equals the projection of $t^{\mu}$ onto the unit
normal $n^{\mu}$ to $\Sigma_{T}$, while the shift is the projection of
$t^{\mu}$ onto the Cauchy surface $\Sigma_{T}$.

We may interpret, as before, $\lapse$ as the ratio of proper time (given
by $t^{\mu}\nabla_{\mu} T=1$) to coordinate time $n^{\mu}\nabla_{\mu}T$.

We claim, and may be verified by the reader, that $\shift^{\mu}$ is a
three-dimensional object, in the sense that
\begin{equation}
h_{\mu\nu}\shift^{\mu} = 0.
\end{equation}
Then we claim that
\begin{equation}
-n_{\mu}t^{\mu} = -\underbrace{n_{\mu}n^{\mu}}_{=-1}\lapse - \underbrace{n_{\mu}\shift^{\mu}}_{=0}
= \lapse.
\end{equation}
The components of the normal vector can be found from the one-form
$n_{\mu}\,\D x^{\mu}=-\lapse\,\D t$ (which matches the motivation for the
lapse function in the first place) as
\begin{equation}
n^{\mu} = g^{\mu\nu}n_{\nu} = \left(\frac{-1}{\lapse},\frac{\shift^{i}}{\lapse}\right).
\end{equation}
When compared to Eq~\eqref{eq:lapse-and-shift-in-terms-of-t-and-normal},
we find
\begin{equation}
t^{\mu} = (-1, 0, 0, 0)
\end{equation}
as we would expect.
\section{Extrinsic Curvature}

\begin{ddanger}
The conventions on the \textsc{sign} of extrinsic curvature varies wildly. I
believe I have a consistent choice, but it is possible I have erred.
\end{ddanger}

From the Hamiltonian perspective, the first step amounts to finding the
conjugate momentum. Ordinarily, these are just the time derivatives of
the generalized positions. For general relativity, with $q_{ij}$ as the
generalized positions, the time derivatives $\partial_{t}q_{ij}$ do not
suffice: they do not form a tensor. In general, the time derivatives of
the components of a tensor will not form a tensor.\footnote{This
motivation comes from Carlip~\cite{carlip2019}, which I find
compelling.} Thus we need to take a geometric detour.

If we have foliated space-time into spatial hypersurfaces $\Sigma_{t}$,
then we may consider unit normal 4-vectors $n^{\mu}$. The extrinsic
curvature may be computed using the Lie derivative of the 3-metric with
respect to $n^{\mu}$ as
\begin{equation}\label{eq:extrinsic-curvature:as-time-derivative-of-metric}
  \begin{split}
  K_{ij} &= \frac{-1}{2}(\mathcal{L}_{\mathbf{n}}q)_{ij}\\
  &= \frac{-1}{2}n^{\mu}\partial_{\mu}q_{ij} - \frac{1}{2} q_{\mu j}(\partial_{i}n^{\mu}) - \frac{1}{2} q_{i\mu}(\partial_{j}n^{\mu}),
  \end{split}
\end{equation}
where $\nabla_{\mu}$ is the covariant derivative compatible with the
4-metric $g_{\mu\nu}$, and we recall
\begin{equation}
q_{\mu\nu} = g_{\mu\nu} + n_{\mu}n_{\nu}.
\end{equation}
We may also consider $K_{ij}$ as the spatial  part of
\begin{equation}
K_{\mu\nu} = n_{(\mu;\nu)} = \frac{1}{2}(\nabla_{\mu}n_{\nu} + \nabla_{\nu}n_{\mu}).
\end{equation}
Both formulations make clear the extrinsic curvature tensor is
symmetric:
\begin{equation}
K_{ij} = K_{ji}.
\end{equation}

A third route to explain the extrinsic curvature tensor is simply by
considering the spatial projection of the covariant derivative of the unit
normal vector:
\begin{equation}
K_{\mu\nu} = {q_{\mu}}^{\rho}\nabla_{\rho}n_{\nu}.
\end{equation}
It requires invoking Frobenius's theorem (from differential topology) to
demonstrate this is a symmetric tensor field. We recall
\begin{equation}
n^{\mu}{q_{\mu}}^{\nu} = 0,
\end{equation}
which implies
\begin{equation}
K_{\mu\nu}n^{\mu} = 0.
\end{equation}
Thus this tensor is purely spatial.

Specifically, the useful \emph{fourth} version of extrinsic curvature,
\begin{equation}\label{eq:extrinsic-curvature:as-time-derivative-of-three-metric}
K_{ij} = \frac{1}{2N}(\partial_{t}q_{ij} - {{}^{(3)}}\nabla_{i}N_{j} - {{}^{(3)}}\nabla_{j}N_{i}).
\end{equation}
This requires a length calculation. The basic idea is to define the
extrinsic curvature in a coordinate-free manner
\begin{equation}
K(u,v) = -g(\nabla_{u}v,\vec{n}).
\end{equation}

\begin{lemma}
In local coordinates, $K_{ij} = N\Gamma^{0}_{ij}$.
\end{lemma}
\begin{proof}[Proof (length calculation)]
Using local coordinates, we find the components of the extrinsic curvature as
\begin{subequations}
  \begin{align}
    K_{ij} &= K(\partial_{i}, \partial_{j})\\
    &= -g(\nabla_{\partial_{i}}\partial_{j}, \vec{n})\\
    &= -g(\Gamma^{\mu}_{ij}\partial_{\mu}, n^{\nu}\partial_{\nu})\\
    &= -\Gamma^{\mu}_{ij}n^{\nu}g_{\mu\nu}\\
    &= -\Gamma^{0}_{ij}n_{0}\\
    &= N\Gamma^{0}_{ij}
  \end{align}
\end{subequations}
where we have implicitly used the fact that
$n_{\mu}=g_{\mu\nu}n^{\nu}=(-N,0,0,0)$.
\end{proof}

\begin{lemma}
  The Christoffel symbol is
  \begin{equation}
\Gamma^{0}_{ij} = \frac{1}{2N^{2}}(\partial_{t}q_{ij} - {{}^{(3)}}\nabla_{i}N_{j} - {{}^{(3)}}\nabla_{j}N_{i}).
  \end{equation}
\end{lemma}
\begin{proof}[Proof (length calculation)]
  We can compute this directly,
  \begin{subequations}
    \begin{align}
\Gamma^{0}_{ij} &= \frac{1}{2} g^{0\mu}(\partial_{i}g_{\mu j} + \partial_{j}g_{i\mu} - \partial_{\mu} g_{ij}) \\
        &= \frac{1}{2}\left[ g^{00}(\partial_{i}g_{0j} + \partial_{j}g_{i0} - \partial_{0} g_{ij}) + g^{0k}(\partial_{i}g_{kj} + \partial_{j}g_{ik} - \partial_{k} g_{ij})  \right] \\
        &= \frac{1}{2}\left[ -\frac{1}{N^{2}} (\partial_{i}N_{j}+ \partial_{j}N_{i}- \partial_{t} q_{ij}) + \frac{N^{k}}{N^{2}} (\partial_{i}q_{kj} + \partial_{j}q_{ik} - \partial_{k} q_{ij}) \right] \\
        &= \frac{1}{2N^{2}}\left[ (\partial_{t} q_{ij} - \partial_{i}N_{j}- \partial_{j}N_{i}) + N_{\ell} q^{\ell k} (\partial_{i}q_{kj} + \partial_{j}q_{ik} - \partial_{k} q_{ij}) \right] \\
        &= \frac{1}{2N^{2}}\left[ (\partial_{t} q_{ij} - \partial_{i} N_{j} - \partial_{j}N_{i}) + 2 N_{\ell} \,{{}^{(3)}}\Gamma^{\ell}_{ij} \right] \\
        &= \frac{1}{2N^{2}} \left(\partial_{t} q_{ij} - {{}^{(3)}\nabla}_{i}N_{j}- {{}^{(3)}\nabla}_{j}N_{i}\right),
    \end{align}
  \end{subequations}
  where we have explicitly used the Christoffel symbol for the spatial hypersurface,
  \begin{equation}
    {{}^{(3)}}\Gamma^{\ell}_{ij} = \frac{1}{2}q^{\ell k}(\partial_{i}q_{kj} + \partial_{j}q_{ik} - \partial_{k}q_{ij}).
  \end{equation}
  Thus concludes our calculation.
\end{proof}

Now combining our two lemmas together, we obtain the desired form of the
extrinsic curvature in Eq~\eqref{eq:extrinsic-curvature:as-time-derivative-of-three-metric}.

However, the intuition of the extrinsic curvature is that it is the
``velocity'' corresponding to the 3-metric $q_{ij}$:
\begin{equation}
\boxed{K_{ij}\mathrel{\mbox{``=''}}\partial_{t}q_{ij}.}
\end{equation}
The strategy
\emph{now} is to rewrite the Ricci scalar for spacetime ${}^{(4)}R$ in
terms of the extrinsic curvature $K_{ij}$, spatial 3-curvature tensor
${}^{(3)}R_{ijk\ell}$, and spatial Ricci 3-tensor ${}^{(3)}R_{ij}$. This
will be used to write the Einstein--Hilbert Lagrangian in terms of
spatial tensors, then we will find the canonically conjugate momentum
$\pi^{ij}$ to the 3-metric $q_{ij}$, and continue on our merry way with
the Hamiltonian formalism.

\section{Gauss--Codazzi Equation}

The slick way to relate ${}^{(4)}R$ to 3-geometric quantities, we begin
with the familiar relation of the Riemann curvature tensor to the
commutator of covariant derivatives:
\begin{equation}
{}^{(3)}{R^{\rho}}_{\sigma\mu\nu}v^{\sigma}
  = [{}^{(3)}\nabla_{\mu}, {}^{(3)}\nabla_{\nu}]v^{\rho}
\end{equation}
where $v^{\mu}\in T_{p}M$ is tangent to $\Sigma$ with base point
$p\in\Sigma$.
We arrive at the relation
\begin{equation}
{q^{\rho}}_{\alpha}{q^{\beta}}_{\sigma}{q^{\gamma}}_{\mu}{q^{\delta}}_{\nu}
{{}^{(4)}\!}{R^{\alpha}}_{\beta\gamma\delta}
={{}^{(3)}\!}{R^{\rho}}_{\sigma\mu\nu} + {K^{\rho}}_{\mu}K_{\sigma\nu}-{K^{\rho}}_{\nu}K_{\sigma\mu}.
\end{equation}
Contracting indices gives us Gauss's \textit{Theorema Egregium}:
\begin{equation}
{{}^{(4)}\!}{R} + 2{{}^{(4)}\!}{R}_{\mu\nu}n^{\mu}n^{\nu}
={{}^{(3)}\!}{R} + K^{2} - K_{\mu\nu}K^{\mu\nu}.
\end{equation}
Note, here $K=K_{ij}q^{ij}$ is the trace of the extrinsic curvature, so
the second term $K^{2}$ is the square of the trace.
This is half the battle.

We may cleverly pick the normal vector field in our considerations
\begin{equation}
{}^{(4)}{R^{\rho}}_{\sigma\mu\nu}n^{\sigma}
  = [{}^{(4)}\nabla_{\mu}, {}^{(4)}\nabla_{\nu}]n^{\rho}.
\end{equation}
\textbf{TODO:} finish the calculations here.

These slick considerations may be found in Danieli's thesis~\cite[see \S2.3]{danieli}.

\subsection{Rewriting the Lagrangian}

We may then rewrite the Einstein--Hilbert Lagrangian density
\begin{equation}
\mathcal{L}_{EH} = {}^{(4)}R\sqrt{-g}.
\end{equation}
We find
\begin{equation}
  \begin{split}
\mathcal{L}_{EH} &= [{}^{(3)}R - K^{2} + K^{ij}K_{ij}-2\nabla_{\alpha}(n^{\mu}\nabla_{\mu}n^{\alpha}-n^{\alpha}\nabla_{\mu}n^{\mu})]\lapse\sqrt{q}\\
&= [{}^{(3)}R - K^{2} + K^{ij}K_{ij}]\lapse\sqrt{q}+(\mbox{total deriative term}).
\end{split}
\end{equation}
Since total derivatives do not contribute to the action, we may discard
it, to get:
\begin{equation}
\boxed{\mathcal{L}_{EH}= [{}^{(3)}R - K^{2} + K^{ij}K_{ij}]\lapse\sqrt{q}.}
\end{equation}
Intuitively, we can think of the $-{}^{(3)}R$ term as the potential
energy, and the $K^{ij}K_{ij}-K^{2}$ as the kinetic energy (one way to
see this is that $\partial_{t}q_{ij}$ appears only in the extrinsic
curvature $K_{ij}$).

\begin{danger}
The Lagrangian we have written down produces the correct equations of
motion, but we often find the boundary terms are non-negligible. For
this reason, we typically need to add the Gibbons--York boundary term to
the Lagrangian. It doesn't change the canonical analysis of general
relativity substantially, but it impacts numerical modeling. Since we
are interested in the dynamics of general relativity, we will be
cavalier and stoic in light of these problems.
\end{danger}
\section{Canonical Action}

The first step, having rewritten the Lagrangian using 3-geometric
quantities, is to write down the action:
\begin{subequations}
  \begin{align}
  I &= \frac{1}{2\kappa^{2}}\int\mathcal{L}_{EH}\,\D^{3}x\,\D t\\
  &= \frac{1}{2\kappa^{2}}\int[{}^{(3)}R - K^{2} + K^{ij}K_{ij}]\lapse\sqrt{q}\,\D^{3}x\,\D t
  \end{align}
\end{subequations}
where $\kappa^{2}=8\pi G$. We now find the conjugate momentum density
\begin{equation}
\pi^{ij} = \frac{\partial\mathcal{L}}{\partial(\partial_{0}q_{ij})}.
\end{equation}
But the action depends on $\partial_{0}q_{ij}$ only through the
extrinsic curvature, so we find\footnote{\textbf{TODO:} work through
this calculation more explicitly and carefully.}
\begin{equation}
\boxed{\pi^{ij} = \frac{1}{2\kappa^{2}}\sqrt{q}(K^{ij} - q^{ij}K)}.
\end{equation}
Thus concludes the first step in canonical analysis of general
relativity using metric variables.

The first trick is to write the Lagrangian using canonical
variables. Towards that end, we need to express the extrinsic curvature
tensor (and its trace) in terms of the momentum density, 
\begin{subequations}
\begin{equation}
K^{ij} = \frac{2\kappa^{2}}{2\sqrt{q}}(\pi q^{ij} - 2\pi^{ij})
\end{equation}
where $\pi=\pi^{ij}q_{ij}$, specifically\footnote{\textbf{TODO:} the
equation for $\partial_{t}q_{ij}$ follows from various forms of the
extrinsic curvature, but I do not believe I wrote them down
explicitly. Double check I did.}
\begin{equation}
K = \frac{2\kappa^{2}\pi}{2\sqrt{q}}
\end{equation}
\begin{equation}\label{eq:canonical-action:time-derivative-of-metric}
\partial_{t}q_{ij} = {}^{(3)}\nabla_{i}\shift_{j} + {}^{(3)}\nabla_{j}\shift_{i}
- \lapse\frac{2\kappa^{2}}{\sqrt{q}}(\pi q_{ij} - 2\pi_{ij}).
\end{equation}
\end{subequations}
We can rewrite the Lagrangian using canonical variables
\begin{equation}
\boxed{\mathcal{L} = \frac{1}{2\kappa^{2}}{{}^{(3)}\!}R\; \lapse\sqrt{q} + \frac{2\kappa^{2}\lapse}{\sqrt{q}}(\pi^{ij}\pi_{ij}-\frac{1}{2}\pi^{2}).}
\end{equation}

%% \begin{danger}
%% We are a little sloppy here, we dropped the factor of $2\kappa^{2}$ from
%% the equation relating $K^{ij}$ in terms of the momenta density. Since
%% $\pi^{ij}\sim(2\kappa^{2})^{-1}$, we need to multiply by a factor of
%% $2\kappa^{2}$ in the ``kinetic term'' while dividing by $2\kappa^{2}$ in
%% the ``potential term'', to get:
%% \begin{equation}
%% \boxed{\mathcal{L} = \frac{1}{2\kappa^{2}}{{}^{(3)}\!}R\; \lapse\sqrt{q} + \frac{\lapse2\kappa^{2}}{\sqrt{q}}(\pi^{ij}\pi_{ij}-\frac{1}{2}\pi^{2}).}
%% \end{equation}
%% \end{danger}

We can then find the Hamiltonian density using our trusty Legendre transform,
\begin{equation}
\mathcal{H} = \pi^{ij}\partial_{0}q_{ij} - \mathcal{L}.
\end{equation}
We find, using Eq~\eqref{eq:canonical-action:time-derivative-of-metric},
\begin{equation}
  \begin{split}
  \pi^{ij}\partial_{0}q_{ij}
  &= \pi^{ij}({}^{(3)}\nabla_{i}\shift_{j} + {}^{(3)}\nabla_{j}\shift_{i}) - \frac{2\kappa^{2}\lapse}{\sqrt{q}}(\pi^{2} - 2\pi^{ij}\pi_{ij})\\
&= \pi^{ij}({}^{(3)}\nabla_{i}\shift_{j} + {}^{(3)}\nabla_{j}\shift_{i}) + 2\frac{2\kappa^{2}\lapse}{\sqrt{q}}(\pi^{ij}\pi_{ij}-\frac{1}{2}\pi^{2})
  \end{split}
\end{equation}
Hence
\begin{equation}
  \mathcal{H} = \pi^{ij}({}^{(3)}\nabla_{i}\shift_{j} + {}^{(3)}\nabla_{j}\shift_{i})
          - \frac{1}{2\kappa^{2}}{{}^{(3)}\!}R\; \lapse\sqrt{q} + \frac{2\kappa^{2}\lapse}{\sqrt{q}}(\pi^{ij}\pi_{ij}-\frac{1}{2}\pi^{2}).
\end{equation}
Since $\pi^{ij}$ is a symmetric tensor, the first term can be simplified
to $\pi^{ij}\nabla_{i}\shift_{j}$, giving us:
\begin{equation}
  \mathcal{H} = 2\pi^{ij}\;{{}^{(3)}\nabla_{i}}\shift_{j}
          - \frac{1}{2\kappa^{2}}{{}^{(3)}\!}R\; \lapse\sqrt{q} + \frac{2\kappa^{2}\lapse}{\sqrt{q}}(\pi^{ij}\pi_{ij}-\frac{1}{2}\pi^{2}).
\end{equation}
We are almost done.

If we care about the Hamiltonian (not just the Hamiltonian
\emph{density}), then we should instead work with:
\begin{equation}
H = \int_{\Sigma}\mathcal{H}\,\D^{3}x.
\end{equation}
We can perform integration by parts on the $\pi^{ij}\nabla_{i} \shift_{j}$
term, to get:
\begin{equation}
\boxed{H = \int_{\Sigma}\left(\lapse\frac{2\kappa^{2}}{\sqrt{q}}(\pi^{ij}\pi_{ij}-\frac{1}{2}\pi^{2})
-  \lapse\frac{1}{2\kappa^{2}}{{}^{(3)}\!}R\;\sqrt{q} - 2\shift_{j}\;{}^{(3)}\nabla_{i}\pi^{ij}
\right)\,\D^{3}x.}
\end{equation}
This form is preferable, because it fleshes out the role of the lapse
and shift functions as Lagrange multipliers. That is to say, we get
\begin{equation}
  H = \int_{\Sigma}\left(\lapse\mathcal{H}_{\perp} + \shift_{i}\mathcal{H}^{i}\right)\,\D^{3}x
\end{equation}
where
\begin{subequations}
  \begin{align}
    \mathcal{H}^{i} &= -2\;{}^{(3)}\nabla_{j}\pi^{ij}\\
    \mathcal{H}_{\perp} &= \frac{2\kappa^{2}}{\sqrt{q}}(\pi^{ij}\pi_{ij}-\frac{1}{2}\pi^{2})
- \frac{1}{2\kappa^{2}}{{}^{(3)}\!}R\; \sqrt{q}.
  \end{align}
\end{subequations}
These are the momentum and Hamiltonian constraints, respectively. We
should expect there to be constraints, since General Relativity is a
gauge system.

We may take a moment to note, we may write the action for general
relativity using the familiar canonical formalism for constrained
systems as
\begin{equation}
I_{ADM} = \int^{t_{1}}_{t_{0}}\int_{\Sigma}(\pi^{ij}\partial_{t}q_{ij} - \lapse\mathcal{H}_{\perp}
- \shift_{i}\mathcal{H}^{i})\,\D^{3}x\,\D t.
\end{equation}
Here it is made manifestly obvious the lapse function $\lapse$ acts as
Lagrange multiplier to the diffeomorphism constraint $\mathcal{H}_{\perp}$
and the shift vector $\shift_{i}$ acts as Lagrange multipliers to the
momentum constraints $\mathcal{H}^{i}$.

Now, as usual, we may define the Poisson bracket on a fixed time-slice
as
\begin{equation}
\{q_{ij}(\mathbf{x}), \pi^{k\ell}(\mathbf{y})\} = \frac{1}{2}(\delta^{k}_{i}\delta^{\ell}_{j}+\delta^{k}_{j}\delta^{\ell}_{i})\delta^{(3)}(\mathbf{x}-\mathbf{y}).
\end{equation}
Then Hamilton's equations are:
\begin{subequations}
  \begin{equation}
\partial_{t}q_{ij} = \{q_{ij},H\}
  \end{equation}
  and
  \begin{equation}
\partial_{t}\pi^{ij} = \{\pi^{ij},H\}.
  \end{equation}
\end{subequations}
These are equivalent to Einstein's field equations.

If we were being cautious, we should check that the constraints are
``consistent'' (in the sense that their Poisson bracket with the
Hamiltonian weakly vanishes).
\section*{To Do Items}

Things I should do:
\begin{enumerate}
\item Discuss the ADM Mass.
\item Compute the gauge transformations generated by
  \begin{enumerate}
  \item the momentum constraint;
  \item the Hamiltonian/diffeomorphism constraint.
  \end{enumerate}
\item Perform the ADM decomposition for the Schwarzschild solution.
\item Introduce the York time-slicing. This is carefully discussed in
  chapter VII of 
  Yvonne Choquet-Bruhat,
  \textit{General Relativity and the Einstein Equations}.
\end{enumerate}

Things I'm considering:
\begin{enumerate}
\item Introduce the Gibbons-York boundary term.
\item Discuss the ``superspace'' [i.e., the moduli space of Riemann
  3-metrics] and the DeWitt metric.
\end{enumerate}
\begin{thebibliography}{99}
\footnotesize%  \small

\bibitem{adams1996:ex}
J.F.\ Adams,
\textit{Lectures on Exceptional Lie Groups}.
University of Chicago Press, 1996.

%\cite{Baez:2001dm}
\bibitem{Baez:2001dm}
John C.~Baez,
``The Octonions''.
\journal{Bull.Am.Math.Soc.} \textbf{39} (2002) 145--205
[erratum: \journal{Bull.Am.Math.Soc.} \textbf{42} (2005) 213]
{\tt\doi{10.1090/S0273-0979-01-00934-X}}
[\arXiv{math/0105155} [math.RA]].
%396 citations counted in INSPIRE as of 30 May 2023

%\cite{Ekins:1975yu}
\bibitem{Ekins:1975yu}
J.~M.~Ekins and J.~F.~Cornwell,
``Semisimple Real Subalgebras of Noncompact Semisimple Real Lie Algebras. 5.,''
\journal{Rept.Math.Phys.} \textbf{7} (1975) 167--203
{\tt\doi{10.1016/0034-4877(75)90026-9}}
%4 citations counted in INSPIRE as of 02 Jun 2023


%\cite{Figueroa-OFarrill:2007jcv}
\bibitem{Figueroa-OFarrill:2007jcv}
Jos\'e Figueroa-O'Farrill,
``A Geometric construction of the exceptional Lie algebras $F_{4}$ and $E_{8}$''.
\journal{Commun.Math.Phys.} \textbf{283} (2008) 663--674
{\tt\doi{10.1007/s00220-008-0581-7}}
[\arXiv{0706.2829} [math.DG]].
%14 citations counted in INSPIRE as of 02 Jun 2023

%\cite{Garling:2011zz}
\bibitem{Garling:2011zz}
D.~J.~H.~Garling,
\textit{Clifford Algebras: An Introduction}.
Cambridge University Press, 2011.
%4 citations counted in INSPIRE as of 02 Jun 2023

%\cite{Gogberashvili:2019ojg}
\bibitem{Gogberashvili:2019ojg}
Merab Gogberashvili and Alexandre Gurchumelia,
``Geometry of the Non-Compact $G(2)$''.
\journal{J.Geom.Phys.} \textbf{144} (2019) 308--313
{\tt\doi{10.1016/j.geomphys.2019.06.015}}
[\arXiv{1903.04888} [physics.gen-ph]].
%3 citations counted in INSPIRE as of 02 Jun 2023

%\cite{Gunaydin:2001bt}
\bibitem{Gunaydin:2001bt}
M.~Gunaydin, K.~Koepsell and H.~Nicolai,
``The Minimal unitary representation of $\mathtt{E}_{8(8)}$''.
\journal{Adv.Theor.Math.Phys.} \textbf{5} (2002) 923--946
{\tt\doi{10.4310/ATMP.2001.v5.n5.a3}}
[\arXiv{hep-th/0109005} [hep-th]].
%59 citations counted in INSPIRE as of 02 Jun 2023

\bibitem{1212.3182}
Aaron Wangberg, Tevian Dray,
``$E_{6}$, the Group: The structure of $\SL(3,\OO)$''.
\arXiv{1212.3182}

%\cite{Zhang:2011ym}
\bibitem{Zhang:2011ym}
R.B.~Zhang,
``Serre presentations of Lie superalgebras''.
[\arXiv{1101.3114} [math.RT]].
%4 citations counted in INSPIRE as of 30 May 2023

\end{thebibliography}

% Tits construction of the exceptional simple Lie algebras
% https://arxiv.org/abs/0907.3789


\end{document}