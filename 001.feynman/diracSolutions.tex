\subsection{Solutions to the Dirac Equation}

We are going to perform second quantization in the next section, so we will need to find solutions to the Dirac equation!

As is usual in quantum theory, we will put the system in a cube box of volume
$V$ with periodic boundary conditions\footnote{Periodic Boundary Conditions are
used so we may use Fourier Series and thus find eigenvalues of our differential
operators.}. A complete set of plane wave states can be defined as follows. For
each momentum $\bold{p}$ allowed by the periodic boundary conditions and the
positive energy
\begin{equation}
cp_0 = E_{\bold{p}} = + (m^2c^4 + c^2\bold{p}\cdot\bold{p})^{1/2},
\end{equation}
the Dirac equation posses four independent solutions. These can be written as
\begin{equation}\label{solutions}
u_{r}(\bold{p})\frac{\exp(-ip^\mu x_\mu/\hbar)}{\sqrt{V}},\quad v_{r}(\bold{p})\frac{\exp(ip^\mu x_\mu/\hbar)}{\sqrt{V}},\quad r=1,2.
\end{equation}
We require some knowledge of $u_{r}(\bold{p})$ and $v_{r}(\bold{p})$, so we have
them be constant spinors that satisfy the equations
\begin{equation}
(\slashchar{p}-mc)u_r(\bold{p}) = 0,\quad (\slashchar{p}+mc)v_r(\bold{p}) = 0, \quad r=1,2.
\end{equation}
Here (again) we use the convenience of the Feynman slash:
\begin{equation}
\slashchar{A} \equiv \gamma^\mu A_\mu.
\end{equation}
Due to their time dependence, the solutions in Eq (\ref{solutions}) involving $u_r$ and
$v_r$ are  referred to as the positive and negative energy solutions respectively.
(Sometimes the phrases ``positive frequency'' and ``negative frequency'' are used
instead of ``positive energy'' and ``negative energy''.)

Now the question arises: ``What use is the index $r$ in these solutions?'' Ah
good question! The so-called ``two-fold degeneracies'' of the two positive and
two negative energy solutions for a given momentum $\bold{p}$ result from the
possible spin orientations. With us and our Dirac equation, only longitudinal
spin components (i.e. parallel to $\pm\bold{p}$) are constants of motion, and
we shall choose these spin eigenstates for the solutions in Eq (\ref{solutions}).
We denote
\begin{equation}
\sigma_{\bold{p}} = \frac{\bold{\sigma}\cdot\bold{p}}{|\bold{p}|}
\end{equation}
(where $\bold{\sigma}$ are the Pauli matrices), we can then choose the spinors
in Eq (\ref{solutions}) such that
\begin{equation}
\sigma_{\bold{p}}u_r(\bold{p}) = (-1)^{r+1}u_{r}(\bold{p}),
\quad\sigma_{\bold{p}}v_{r}(\bold{p}) = (-1)^{r}v_{r}(\bold{p}),
\quad r=1,2.
\end{equation}
The asymmetry in labelling $u$ and $v$ spinors will be convenient for labelling
the spin properties of particles and antiparticles.

We can normalize our spinors $u_r$ and $v_r$ such that
\begin{equation}
u_{r}^{\dag}(\bold{p})u_{r}(\bold{p}) = v_{r}^{\dag}(\bold{p})v_{r}(\bold{p})=\frac{E_{\bold{p}}}{mc^2}.
\end{equation}
It follows that they satisfy the orthonormality relations
\begin{equation}
\begin{array}{rcl}
u_{r}^{\dag}(\bold{p})u_{s}(\bold{p}) = v_{r}^{\dag}(\bold{p})v_{s}(\bold{p}) &=& (E_{\bold{p}}/mc^2)\delta_{rs} \\
u^{\dag}_{r}(\bold{p})v_{s}(-\bold{p}) &=& 0
\end{array}
\end{equation}
and the states form a complete orthonormal set of solutions of the free particle
Dirac equation, normalized to $E_{\bold{p}}/mc^2$ in a volume $V$.
