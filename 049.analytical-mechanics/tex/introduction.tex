\section{Introduction}

\N{Review: Arclength of a curve}
Recall Pythagoras's theorem: if we have a right triangle with sides
$\Delta x$ and $\Delta y$, then its hypoteneuse has length $\Delta s$ satisfying:
\begin{equation}
(\Delta s)^{2} = (\Delta x)^{2} + (\Delta y)^{2}.
\end{equation}
We will need this in a moment.

Now, recall a curve on the [real] plane may be described using a
``suitably smooth'' function
\begin{equation}
\gamma\colon [a,b]\to\RR^{2}
\end{equation}
where $a < b$ are real numbers and $[a,b] = \{x\in\RR\mid a\leq x\leq b\}$
is the closed interval of real numbers between $a$ and $b$. For our
purposes, ``suitably smooth'' means ``has at least one derivative which
is continuous''.\footnote{We could weaken this by allowing countably many
jump discontinuities. But we will not need this.}
We can look at any point of the curve with our ``infinitesimal
microscope'', as doodled below:
\begin{center}
  \includegraphics{img/img.0}
\end{center}
We invoke Pythagoras's theorem to write
\begin{equation}
(\D s)^{2} =  (\D x)^{2} + (\D y)^{2}.
\end{equation}
How do we actually compute this quantity? And how do we use it?

We intend to use $\D s$ to compute the arclength:
\begin{equation}
s = \int^{\gamma(b)}_{\gamma(a)}\D s.
\end{equation}
Now, as far as how we compute it, that's a bit trickier. The intuition
is to invoke Taylor's theorem to first order for the curve:
\begin{equation}
\gamma(t+\D t) = \gamma(t) + \frac{\D \gamma}{\D t}\D t,
\end{equation}
then we argue the quantity 
\begin{equation}
\vec{v}(t)\,\D t := \frac{\D \gamma}{\D t}\D t
\end{equation}
is precisely the vector $\D\vec{s}$ with the base point $\gamma(t)$,
which is the hypoteneuse in the infinitesimal microscope pointing to the
next point along the curve. We then find the scalar quantity
\begin{equation}
(\D s)^{2} = (\vec{v}(t)\,\D t)\cdot(\vec{v}(t)\,\D t)
\end{equation}
which we will integrate from the initial ``time'' to the final
``time''. Taking the square root and factoring out the quantities $\D t$,
we obtain
\begin{equation}
\D s = (\sqrt{\vec{v}(t)\cdot\vec{v}(t)})\,\D t.
\end{equation}
We can plug this change-of-variables back into our integral, and obtain
the desired result for the arclength of the curve:
\begin{equation}
s = \int^{b}_{a}(\sqrt{\vec{v}(t)\cdot\vec{v}(t)})\,\D t,
\end{equation}
where
\begin{equation}
\vec{v}(t) = \frac{\D\gamma(t)}{\D t}
\end{equation}
is the ``velocity'' vector for the curve at time $t$.

\N{Puzzle}
Given two points $(x_{0},y_{0})\in\RR^{2}$ and
$(x_{1},y_{1})\in\RR^{2}$, what is the shortest curve connecting them?

\N{Strategizing a Solution}
This sounds a lot like ``Find the minima of a function''. We know how to
find the minima (and maxima) of a function: they are the critical
points, i.e., the points where the derivative vanishes. Intuition
suggests a similar approach should work for us with our puzzle.

Our ``function'' is then an integral
\begin{equation}\label{eq:introduction:arclength-functional}
I[\gamma] = \int^{b}_{a}\sqrt{\frac{\D\gamma(t)}{\D t}\cdot\frac{\D\gamma(t)}{\D t}}\,\D t,
\end{equation}
which ``eats in'' some curve $\gamma$ and produces a positive
number. The only problem is: what are we differentiating with respect
to?

We can avoid answering this question by cheating. Assume we already have
a solution $\gamma_{\text{cr}}$ which is the shortest path (we will
derive an equation describing it at the end of our cheating). Now we
consider a formal infinitesimal $\varepsilon\neq0$ but
$\varepsilon^{2}=0$ and any arbitrary deformation $\eta(t)$ to the
solution. For $\eta$ to be a deformation, we require the endpoints of
$\gamma+\varepsilon\eta$ be fixed at the endpoints --- this means
$\eta(a)=0$ and $\eta(b)=0$. Then we Taylor expand, which will give us:
\begin{equation}
I[\gamma_{\text{cr}}+\varepsilon\eta] =
I[\gamma_{\text{cr}}] + \varepsilon\int(\mbox{something})\eta(t)\,\D t.
\end{equation}
This is analogous to
\begin{equation}
f(x_{\text{cr}} + h) = f(x_{\text{cr}}) + hf'(x_{\text{cr}}).
\end{equation}
Now, the analogous condition to ``the first derivative vanishes at the
critical points'' would be for the first-order correction to vanish,
i.e.,
\begin{equation}
\varepsilon\int(\mbox{something})\eta(t)\,\D t = 0.
\end{equation}
Since $\eta$ is completely arbitrary, this works if the ``something'' in
the integrand is identically zero --- which is precisely the equation we
will use to find $\gamma_{\text{cr}}$.

\N{Executing the strategy}
Now, we return to Eq~\eqref{eq:introduction:arclength-functional} and
plug in the deformed path
\begin{equation}
  I[\gamma + \varepsilon\eta]
  = \int^{b}_{a}\sqrt{\frac{\D[\gamma(t)+ \varepsilon\eta(t)]}{\D t}\cdot\frac{\D[\gamma(t)+ \varepsilon\eta(t)]}{\D t}}\,\D t.
\end{equation}
We can massage the integrand, Taylor expanding it:
\begin{subequations}
  \begin{align}
\sqrt{\frac{\D[\gamma(t)+ \varepsilon\eta(t)]}{\D t}\cdot\frac{\D[\gamma(t)+ \varepsilon\eta(t)]}{\D t}}
&= \sqrt{\frac{\D\gamma(t)}{\D t}\cdot\frac{\D\gamma(t)}{\D t} + 2\varepsilon \frac{\D\eta(t)}{\D t}\cdot\frac{\D\gamma(t)}{\D t}}\\
&= \sqrt{\frac{\D\gamma(t)}{\D t}\cdot\frac{\D\gamma(t)}{\D t}}\sqrt{1 + 2\varepsilon \frac{\frac{\D\eta(t)}{\D t}\cdot\frac{\D\gamma(t)}{\D t}}{\frac{\D\gamma(t)}{\D t}\cdot\frac{\D\gamma(t)}{\D t}}}\\
&= \sqrt{\frac{\D\gamma(t)}{\D t}\cdot\frac{\D\gamma(t)}{\D
    t}}\left[1 + \varepsilon \frac{\frac{\D\eta(t)}{\D t}\cdot\frac{\D\gamma(t)}{\D t}}{\frac{\D\gamma(t)}{\D t}\cdot\frac{\D\gamma(t)}{\D t}}\right]
+\mathcal{O}(\varepsilon^{2})\\
&= \sqrt{\frac{\D\gamma(t)}{\D t}\cdot\frac{\D\gamma(t)}{\D t}} + \varepsilon \frac{\D\eta(t)}{\D t}\cdot\frac{\D\gamma(t)}{\D t} +\mathcal{O}(\varepsilon^{2}).
  \end{align}
\end{subequations}
We discard terms quadratic in $\varepsilon$, and we find
\begin{subequations}
  \begin{align}
I[\gamma + \varepsilon\eta] &= \int^{b}_{a} \left[\sqrt{\frac{\D\gamma(t)}{\D t}\cdot\frac{\D\gamma(t)}{\D t}} + \varepsilon \frac{\D\eta(t)}{\D t}\cdot\frac{\D\gamma(t)}{\D t}\right]\,
\D t\\
&= \int^{b}_{a} \sqrt{\frac{\D\gamma(t)}{\D t}\cdot\frac{\D\gamma(t)}{\D t}}\D t +
\varepsilon \int^{b}_{a} \frac{\D\eta(t)}{\D t}\cdot\frac{\D\gamma(t)}{\D t}\D t\\
&= I[\gamma] + \varepsilon \int^{b}_{a} \frac{\D\eta(t)}{\D t}\cdot\frac{\D\gamma(t)}{\D t}\D t.
  \end{align}
\end{subequations}
Well, we haven't quite worked things out to make the integrand look like
``$(\mbox{something})\eta(t)$'', but we are close! We can use
integration by parts:
\begin{equation}
  \int^{b}_{a} \frac{\D\eta(t)}{\D t}\cdot\frac{\D\gamma(t)}{\D t}\D t
  = \left.\eta(t)\cdot\frac{\D\gamma(t)}{\D t}\right|^{t=b}_{t=a}
  -\int^{b}_{a} \eta(t)\cdot\frac{\D^{2}\gamma(t)}{\D t^{2}}\D t.
\end{equation}
Since $\eta$ is a deformation, we have $\eta(b)=0$ and $\eta(a)=0$. This
kills the boundary term, giving us:
\begin{equation}
  \int^{b}_{a} \frac{\D\eta(t)}{\D t}\cdot\frac{\D\gamma(t)}{\D t}\D t
  = 0 - \int^{b}_{a} \eta(t)\cdot\frac{\D^{2}\gamma(t)}{\D t^{2}}\D t.
\end{equation}
We have our condition to find the ``critical paths'' of the functional,
namely all paths $\gamma\colon[a,b]\to\RR^{2}$ with boundary conditions
$\gamma(a)$ and $\gamma(b)$ given, and
\begin{equation}
\frac{\D^{2}\gamma(t)}{\D t^{2}} = \vec{0}.
\end{equation}
This is precisely a curve with constant velocity
\begin{equation}
\vec{v} = \frac{\gamma(b) - \gamma(a)}{b - a},
\end{equation}
and the general solution looks like:
\begin{equation}
\gamma(t) = \gamma(a) + (t - a)\vec{v}.
\end{equation}
Thus the critical path connecting two points is a straight line. Is it
the shortest or longest path?

It's not hard to find a longer path: simply take any point $\vec{x}$ not on the
curve, then construct a curve by taking two lines from $\gamma(a)$ to
$\vec{x}$, then from $\vec{x}$ to $\gamma(b)$. We recall from basic
geometry this is longer than $\gamma$. This suggests (but does not
prove) $\gamma$ is the shortest path connecting the endpoints.

\N{Review: Re-Phrasing What We Just Did}
We started with some ``functional'' [a monster which eats functions and
  produces numbers]
$I[\gamma]$. We pretended we had a critical path $\gamma_{\text{cr}}$
which was such that, for any other path $\eta$ satisfying
$\eta(a)=\eta(b)=0$, we have
\begin{equation}
I[\gamma_{\text{cr}} + \varepsilon\eta] = I[\gamma_{\text{cr}}] + \mathcal{O}(\varepsilon^{2}).
\end{equation}
This is equivalent to saying
\begin{equation}
\left.\frac{\D}{\D\varepsilon}I[\gamma_{\text{cr}} + \varepsilon\eta]\right|_{\varepsilon=0}=0.
\end{equation}
Frequently, in the literature, people write
$\delta\gamma=\varepsilon\eta$ for the ``variation'' of the trajectory
$\gamma$ and the ``first variation'' of $I$ is:
\begin{equation}
\delta I[\gamma,\eta] = 
\left.\frac{\D}{\D\varepsilon}I[\gamma + \varepsilon\eta]\right|_{\varepsilon=0}.
\end{equation}
We are looking for $\gamma$ such that $\delta I[\gamma,\eta]=0$ for any
arbitrary $\eta$.

Now, we had expressed the functional as an integral
\begin{equation}
I[\gamma] = \int^{b}_{a}L(\gamma)\,\D t.
\end{equation}
We then worked to express $\delta I[\gamma,\eta]=0$ as a differential
equation involving derivatives of $L(\gamma)$.

\N{Applications to Physics}
We can apply this general idea to physics to find the trajectories of
point-particles using the variations of some functional. The
trajectories physical bodies ``take'' then are the extrema of this
functional.

One critical difference is that the functional will treat velocities and
positions as ``independent'' of each other. Traditionally positions are
denoted $q^{\mu}$ and confusingly velocities are $\dot{q}^{\mu}$, where
$\mu=1,2,\dots,3N$ for $N$ point-particles in three-dimensional
space. (In $D$-dimensional space, $\mu$ would range up to $ND$.) We
treat these as functions of time. The
functional is usually written as $S$ and called the \define{Action}:
\begin{equation}
S[q^{\mu},\dot{q}^{\mu}] =
\int^{t_{1}}_{t_{0}}L(q^{\mu},\dot{q}^{\mu})\,\D t.
\end{equation}
Here we use the usual abbreviation $S[q^{\mu},\dot{q}^{\mu}] = S[q^{1},\dots,q^{3N},\dot{q}^{1},\dots,\dot{q}^{3N}]$
and similarly $L(q^{\mu},\dot{q}^{\mu}) = L(q^{1},\dots,q^{3N},\dot{q}^{1},\dots,\dot{q}^{3N})$.
The function $L$ is called the \define{Lagrangian}. The equations of
motion are obtained from the \define{Principle of Stationary Action}
\begin{equation}
\delta S = 0.
\end{equation}
Its solutions are the physically realized trajectories of the point-particles.

We can derive this from Newtonian mechanics, or we can apply various
heuristics and recover Newtonian mechanics from this ``principle of
stationary action''.

\M
Dually, we could take $\delta S=0$ as the starting point for physics and
derive everything from it. For example, insisting physics is independent
of choice of inertial reference frame forces our hand, we realize the
``free point-particle'' Lagrangian is precisely:
\begin{equation}
L(q^{\mu},\dot{q}^{\mu}) = \sum_{\mu}\frac{1}{2}m\dot{q}^{\mu}\dot{q}^{\mu}.
\end{equation}
That is, the free point-particle's Lagrangian is the kinetic energy of
the point-particle. If we demand other basic symmetries hold (like
``There are no preferred directions''), then we get additional
restrictions like the Lagrangian can only depend on the magnitude of
relative distances between different bodies. This is worked out in
Arnold's \textit{Mathematical Methods of Classical Mechanics} as well as
in Landau and Lifschitz's \textit{Mechanics}.