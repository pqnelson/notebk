\section{Lagrangian Mechanics}

\M
Lagrangian mechanics refers to working with the Lagrangian as a function
of ``generalized positions'' $q^{\mu}$ and ``generalized velocities'' $\dot{q}^{\mu}$.
We want to first derive the equations of motion for Lagrangian
mechanics, then try to recover Newtonian mechanics. Again, before
continuing, we stress that $q^{\mu}$ are functions completely
independent of $\dot{q}^{\mu}$, so in particular
\begin{equation}
\frac{\partial q^{\mu}}{\partial\dot{q}^{\nu}}=0,\quad\mbox{and}\quad
\frac{\partial\dot{q}^{\nu}}{\partial q^{\mu}}=0
\end{equation}
for every $\mu,\nu=1,\dots,3N$.

\N{Deriving the equations of motion}
We can consider the first variation of the action, which we write as
\begin{equation}
  S[q^{\mu} + \delta q^{\mu}, \dot{q}^{\mu} + \delta\dot{q}^{\mu}]
  =
  S[q^{\mu}, \dot{q}^{\mu}] + \int\sum_{\nu}\frac{\delta L(q^{\mu},
    \dot{q}^{\mu})}{\delta q^{\nu}}\delta q^{\nu}\,\D t +
  \mathcal{O}((\delta q)^{2},(\delta\dot{q})^{2}).
\end{equation}
We just use more formal notation than writing ``(something)'' in our
first variation of the action.

Now, we find by plugging in the deformed paths into the action integral,
\begin{equation}
  S[q^{\mu} + \delta q^{\mu}, \dot{q}^{\mu} + \delta\dot{q}^{\mu}]
  = \int L(q^{\mu} + \delta q^{\mu}, \dot{q}^{\mu} + \delta\dot{q}^{\mu})\,\D t.
\end{equation}
Taylor expanding to first order in the variations
\begin{equation}
\int L(q^{\mu} + \delta q^{\mu}, \dot{q}^{\mu} +
\delta\dot{q}^{\mu})\,\D t
= \int L(q^{\mu}, \dot{q}^{\mu})\,\D t
+ \int \left(\sum_{\nu} \delta q^{\nu}\frac{\partial L(q^{\mu}, \dot{q}^{\mu})}{\partial q^{\nu}}
+\delta \dot{q}^{\nu}\frac{\partial L(q^{\mu}, \dot{q}^{\mu})}{\partial \dot{q}^{\nu}}\right)\D t.
\end{equation}
Now we ``parametrize'' the variations of velocities as
\begin{equation}
\dot{q}^{\nu} = \frac{\D}{\D t}\delta q^{\nu},
\end{equation}
then rewrite the first-term in the Taylor expanded variation as
\begin{equation}
\int \left(\sum_{\nu} \delta q^{\nu}\frac{\partial L(q^{\mu}, \dot{q}^{\mu})}{\partial q^{\nu}}
+\delta \dot{q}^{\nu}\frac{\partial L(q^{\mu}, \dot{q}^{\mu})}{\partial \dot{q}^{\nu}}\right)\D t
= \int \left(\sum_{\nu} \delta q^{\nu}\frac{\partial L(q^{\mu}, \dot{q}^{\mu})}{\partial q^{\nu}}
+\frac{\D}{\D t}\delta q^{\nu}\frac{\partial L(q^{\mu}, \dot{q}^{\mu})}{\partial \dot{q}^{\nu}}\right)\D t.
\end{equation}
Remember our strategy is to rearrange terms so the integrand looks like
\begin{equation}
\int \left(\sum_{\nu} \delta q^{\nu}\frac{\partial L(q^{\mu}, \dot{q}^{\mu})}{\partial q^{\nu}}
+\frac{\D}{\D t}\delta q^{\nu}\frac{\partial L(q^{\mu}, \dot{q}^{\mu})}{\partial \dot{q}^{\nu}}\right)\D t
= \int\sum_{\nu}\frac{\delta L(q^{\mu},
    \dot{q}^{\mu})}{\delta q^{\nu}}\delta q^{\nu}\,\D t.
\end{equation}
We use integration by parts on the time derivative of the variation to get
\begin{equation}
\int \left(\sum_{\nu} \delta q^{\nu}\frac{\partial L(q^{\mu}, \dot{q}^{\mu})}{\partial q^{\nu}}
+\frac{\D}{\D t}\delta q^{\nu}\frac{\partial L(q^{\mu}, \dot{q}^{\mu})}{\partial \dot{q}^{\nu}}\right)\D t
=\left.\delta q^{\nu}\frac{\partial L(q^{\mu}, \dot{q}^{\mu})}{\partial \dot{q}^{\nu}}\right|_{\text{bdry}}
+\int \left(\sum_{\nu} \delta q^{\nu}\frac{\partial L(q^{\mu}, \dot{q}^{\mu})}{\partial q^{\nu}}
-\delta q^{\nu}\frac{\D}{\D t}\frac{\partial L(q^{\mu}, \dot{q}^{\mu})}{\partial \dot{q}^{\nu}}\right)\D t.
\end{equation}
The boundary contribution vanishes, so we are left with
\begin{equation}
\int \left(\sum_{\nu} \delta q^{\nu}\frac{\partial L(q^{\mu}, \dot{q}^{\mu})}{\partial q^{\nu}}
+\frac{\D}{\D t}\delta q^{\nu}\frac{\partial L(q^{\mu}, \dot{q}^{\mu})}{\partial \dot{q}^{\nu}}\right)\D t
=\int \left(\sum_{\nu} \delta q^{\nu}\frac{\partial L(q^{\mu}, \dot{q}^{\mu})}{\partial q^{\nu}}
-\delta q^{\nu}\frac{\D}{\D t}\frac{\partial L(q^{\mu}, \dot{q}^{\mu})}{\partial \dot{q}^{\nu}}\right)\D t.
\end{equation}
We can now use distributivity to write this as
\begin{equation}
\int \left(\sum_{\nu} \delta q^{\nu}\frac{\partial L(q^{\mu}, \dot{q}^{\mu})}{\partial q^{\nu}}
+\frac{\D}{\D t}\delta q^{\nu}\frac{\partial L(q^{\mu}, \dot{q}^{\mu})}{\partial \dot{q}^{\nu}}\right)\D t
=\sum_{\nu} \int \underbrace{\left(\frac{\partial L(q^{\mu}, \dot{q}^{\mu})}{\partial q^{\nu}}
-\frac{\D}{\D t}\frac{\partial L(q^{\mu}, \dot{q}^{\mu})}{\partial
  \dot{q}^{\nu}}\right)}_{=\delta L/\delta q^{\nu}}\delta q^{\nu}\,\D t.
\end{equation}
The first variation of the action vanishes if $\delta L/\delta q^{\nu}=0$ for each $\nu$.
This requires, for arbitrary variation $\delta q^{\nu}$, that
\begin{equation}
\boxed{\frac{\partial L(q^{\mu}, \dot{q}^{\mu})}{\partial q^{\nu}}
-\frac{\D}{\D t}\frac{\partial L(q^{\mu}, \dot{q}^{\mu})}{\partial \dot{q}^{\nu}}
= 0}
\end{equation}
for each $\nu=1,\dots,3N$. This is precisely the \define{Euler--Lagrange Equations},
and they are the equations of motion for Lagrangian mechanics.

\M
Observe the only place where acceleration can enter the Euler--Lagrange
equation is from the total time derivative. If we want to recover
Newton's second Law, we need
\begin{equation}
\frac{\D}{\D t}\frac{\partial L(q^{\mu}, \dot{q}^{\mu})}{\partial \dot{q}^{\nu}}
=m_{\nu}\ddot{q}^{\nu} + (\mbox{other stuff}),
\end{equation}
where $m_{\nu}$ is the mass for the body with acceleration component $\ddot{q}^{\nu}$.
Hence we see, for Newtonian systems described by Lagrangian mechanics,
we need
\begin{equation}
L(q^{\mu}, \dot{q}^{\mu}) =
\sum_{\nu}\frac{1}{2}m_{\nu}\dot{q}^{\nu}\dot{q}^{\nu} + (\mbox{other stuff}).
\end{equation}

\N{Free bodies}
For us to describe free bodies experiencing no forces, we need to
recover the equations of motion
\begin{equation}
\ddot{q}^{\nu} = 0.
\end{equation}
This is precisely described by
\begin{equation}
L_{\text{free}}(q^{\mu}, \dot{q}^{\mu}) = \sum_{\nu}\frac{1}{2}m_{\nu}\dot{q}^{\nu}\dot{q}^{\nu}.
\end{equation}

\N{Forces}
We now want to recover forces in Lagrangian mechanics. If we recall that
the potential energy $U(q)$ describes a force $\vec{F}$ by taking the
gradient of the negative potential energy $\vec{F}=-\nabla U$, then this
suggests we should consider the Lagrangian of the form:
\begin{equation}
L(q^{\mu}, \dot{q}^{\mu}) =
\left(\sum_{\nu}\frac{1}{2}m_{\nu}\dot{q}^{\nu}\dot{q}^{\nu}\right) \pm U(q^{\mu}).
\end{equation}
We just need to determine the sign of the potential energy contribution.

(\textbf{Exercise:} stop! Are we making sense? Check the dimensions of
the terms we are adding together, they should be the same. What does
that make the dimensions of the action?)

We find, swapping dummy variables we sum over, that the equations of
motion are:
\begin{equation}
\begin{split}
0 = &\frac{\partial}{\partial q^{\nu}}\left[\left(\sum_{\mu}\frac{1}{2}m_{\mu}\dot{q}^{\mu}\dot{q}^{\mu}\right) \pm U(q^{\mu})\right]
-\frac{\D}{\D t}\frac{\partial }{\partial \dot{q}^{\nu}}\left[\left(\sum_{\mu}\frac{1}{2}m_{\mu}\dot{q}^{\mu}\dot{q}^{\mu}\right) \pm U(q^{\mu})\right]\\
&= 
\frac{\partial}{\partial q^{\nu}}\left[\pm U(q^{\mu})\right]
-\frac{\D}{\D t}\frac{\partial }{\partial \dot{q}^{\nu}}\left(\sum_{\mu}\frac{1}{2}m_{\mu}\dot{q}^{\mu}\dot{q}^{\nu}\right).
\end{split}
\end{equation}
Hence
\begin{equation}
\mp\frac{\partial U(q^{\mu})}{\partial q^{\nu}} = m_{\nu}\ddot{q}^{\nu}.
\end{equation}
We can recover the familiar ``negative gradient of potential energy is
force'' physical principle by taking the Lagrangian for a generic
Newtonian system as:
\begin{equation}
\boxed{L(q^{\mu}, \dot{q}^{\mu}) =
\left(\sum_{\nu}\frac{1}{2}m_{\nu}\dot{q}^{\nu}\dot{q}^{\nu}\right) - U(q^{\mu}).}
\end{equation}
More generally, this suggests we should take the Lagrangian to be of the form:
\begin{equation}
L = \begin{pmatrix}\mbox{kinetic}\\\mbox{energy}\end{pmatrix}
-\begin{pmatrix}\mbox{potential}\\\mbox{energy}\end{pmatrix}.
\end{equation}
This recovers Newtonian mechanics correctly, and if we wish to
generalize to other settings (like relativistic mechanics or field theory) then we just
need to find the analogous quantities for kinetic and potential
energies.

\N{Observation 1: Arclength as Lagrangian problem}
We see that our motivating problem with arclength worked with a
different ``Lagrangian'', namely
\begin{equation}
L(q^{\mu},\dot{q}^{\mu}) = \sqrt{\sum_{\nu}\dot{q}^{\nu}\dot{q}^{\nu}}.
\end{equation}
Determine the equations of motion for this Lagrangian, and confirm we
recover the $\ddot\gamma=0$ condition.

\N{Observation 2: Generalized Momentum}
We see that Newton's second Law, in full generality, is
\begin{equation}
\frac{\D\vec{p}}{\D t} =\vec{F}
\end{equation}
where $\vec{p}$ is the momentum for the body. This suggests that there
is an analogous quantity for momentum in the Lagrangian setting, aptly
called the \define{Generalized Momentum}
\begin{equation}
p_{\nu}(q^{\mu},\dot{q}^{\mu}) := \frac{\partial L(q^{\mu},\dot{q}^{\mu})}{\partial\dot{q}^{\nu}}.
\end{equation}
Then the Euler--Lagrange equations is
\begin{equation}
\frac{\D p_{\nu}(q^{\mu},\dot{q}^{\mu})}{\D t} = \frac{\partial L(q^{\mu},\dot{q}^{\mu})}{\partial q^{\nu}}.
\end{equation}
Some people use the term \define{Generalized Force} to refer to the
quantity
\begin{equation}
F_{\nu} := \frac{\partial L(q^{\mu},\dot{q}^{\mu})}{\partial q^{\nu}}.
\end{equation}

\N{Exercise}
Jacobi considered a slightly different Lagrangian, one of the form:
\begin{equation}
L_{\text{Jacobi}} = \sqrt{\begin{pmatrix}\mbox{kinetic}\\\mbox{energy}\end{pmatrix}\begin{pmatrix}\mbox{potential}\\\mbox{energy}\end{pmatrix}}.
\end{equation}
This makes sense when we have nonzero potential energy. For a
1-dimensional point-particle, what does the equations of motion look
like for:
\begin{enumerate}
\item Spring potential: $U(q)=kq^{2}$
\item Gravitational potential: $U(q) = U_{0}/q$
\item Constant potential: $U(q)=U_{0}\neq0$.
\end{enumerate}
Are they the same equations of motion for Jacobi's Lagrangian as the
usual Lagrangian?

\N{Kinetic Energies in Different Coordinate Systems}
How can we find the kinetic energy in different coordinate systems? For
example, in 2-dimensions, should we expect kinetic energy for polar
coordinates to be
\begin{equation}
K = \frac{\dot{r}^{2}}{2m} + \frac{\dot{\theta}^{2}}{2m}?
\end{equation}
The problem is that $\theta$ is dimensionless, and so this cannot
possibly function by dimensional analysis. What we should do is write
\begin{equation}
x = r\cos(\theta),\quad\mbox{and}\quad y=r\sin(\theta).
\end{equation}
Taking the time derivative
\begin{equation}
  \dot{x} = \dot{r}\cos\theta -r\dot\theta\sin\theta,\quad\mbox{and}\quad
  \dot{y} = \dot{r}\sin\theta +r\dot\theta\cos\theta.
\end{equation}
Then
\begin{subequations}
\begin{align}
\dot{x}^{2}+\dot{y}^{2} &= \left(\dot{r}\cos\theta -r\dot\theta\sin\theta\right)^{2}
\left(\dot{r}\sin\theta +r\dot\theta\cos\theta\right)^2\\
&= \left(\dot{r}^{2}\cos^{2}\theta
-2r\dot{r}\dot\theta\cos\theta\sin\theta
+r^{2}\dot\theta^{2}\sin^{2}\theta\right)
+ \left(\dot{r}^2\sin^2\theta + 2\dot{r}r\dot\theta\cos\theta\sin\theta +r^2\dot\theta^2\cos^2\theta\right)\\
%% &=\dot{r}^{2}\cos^{2}\theta
%% +r^{2}\dot\theta^{2}\sin^{2}\theta
%% \left(\dot{r}^2\sin^2\theta +r^2\dot\theta^2\cos^2\theta\right)\\
&=\dot{r}^{2}(\cos^{2}\theta + \sin^2\theta) + r^2\dot\theta^2(\sin^2\theta + \cos^2\theta)\\
&=\dot{r}^{2} + r^2\dot\theta^2.
\end{align}
\end{subequations}
This is generically how we determine kinetic energies in different
coordinate systems: find the velocities in the old coordinate system
expressed using the velocities in the new coordinate system, then plug
them into the expression for the kinetic energy.

\N{Exercise}
The free particle in 2-dimensions polar coordinates has its Lagrangian
be
\begin{equation}
L = \frac{m}{2}\dot{r}^{2} + \frac{m}{2}r^2\dot\theta^2.
\end{equation}
This gives us the generalized force
\begin{equation}
\frac{\partial L}{\partial r} = mr\dot\theta^{2}.
\end{equation}
But by definition, a free body experiences no force. What's going on?
[Hint: think about pseudoforces like the coriolis effect and centrifugal forces.]