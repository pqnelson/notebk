\section{Lagrangian Mechanics}

\M
Lagrangian mechanics refers to working with the Lagrangian as a function
of ``generalized positions'' $q^{\mu}$ and ``generalized velocities'' $\dot{q}^{\mu}$.
We want to first derive the equations of motion for Lagrangian
mechanics, then try to recover Newtonian mechanics. Again, before
continuing, we stress that $q^{\mu}$ are functions completely
independent of $\dot{q}^{\mu}$, so in particular
\begin{equation}
\frac{\partial q^{\mu}}{\partial\dot{q}^{\nu}}=0,\quad\mbox{and}\quad
\frac{\partial\dot{q}^{\nu}}{\partial q^{\mu}}=0
\end{equation}
for every $\mu,\nu=1,\dots,3N$. However,
\begin{equation}
\frac{\partial\dot{q}^{\mu}}{\partial\dot{q}^{\nu}}=
\frac{\partial q^{\mu}}{\partial q^{\nu}}={\delta^{\mu}}_{\nu}
\end{equation}
where ${\delta^{\mu}}_{\nu}$ is 1 when $\mu=\nu$ and 0 otherwise.

\N{Deriving the equations of motion}
We can consider the first variation of the action, which we write as
\begin{equation}\label{eq:lagrange:first-variation-of-action}
  S[q^{\mu} + \delta q^{\mu}, \dot{q}^{\mu} + \delta\dot{q}^{\mu}]
  =
  S[q^{\mu}, \dot{q}^{\mu}] + \int\sum_{\nu}\frac{\delta L(q^{\mu},
    \dot{q}^{\mu})}{\delta q^{\nu}}\delta q^{\nu}\,\D t +
  \mathcal{O}((\delta q)^{2},(\delta\dot{q})^{2}).
\end{equation}
We just use more formal notation than writing ``(something)'' in our
first variation of the action.

\begin{ddanger}
There is sleight-of-hand here, since the chain-rule would suggest we
would have a term $\int (\delta L/\delta\dot{q}^{\nu})\delta\dot{q}^{\nu}\,\D t$
in the right-hand side of Eq~\eqref{eq:lagrange:first-variation-of-action}.
We side-step this by insisting $\delta\dot{q}^{\mu}=\D(\delta q^{\mu})/\D t$.
Physicists are fast-and-loose with their mathematics, sometimes. This is
one of those times.
\end{ddanger}

\M
Now, we find by plugging in the deformed paths into the action integral,
\begin{equation}
  S[q^{\mu} + \delta q^{\mu}, \dot{q}^{\mu} + \delta\dot{q}^{\mu}]
  = \int L(q^{\mu} + \delta q^{\mu}, \dot{q}^{\mu} + \delta\dot{q}^{\mu})\,\D t.
\end{equation}
Taylor expanding to first order in the variations
\begin{equation}
\int L(q^{\mu} + \delta q^{\mu}, \dot{q}^{\mu} +
\delta\dot{q}^{\mu})\,\D t
= \int L(q^{\mu}, \dot{q}^{\mu})\,\D t
+ \int \left(\sum_{\nu} \delta q^{\nu}\frac{\partial L(q^{\mu}, \dot{q}^{\mu})}{\partial q^{\nu}}
+\delta \dot{q}^{\nu}\frac{\partial L(q^{\mu}, \dot{q}^{\mu})}{\partial \dot{q}^{\nu}}\right)\D t.
\end{equation}
Now we ``parametrize'' the variations of velocities as
\begin{equation}
\dot{q}^{\nu} = \frac{\D}{\D t}\delta q^{\nu},
\end{equation}
then rewrite the first-term in the Taylor expanded variation as
\begin{equation}
\int \left(\sum_{\nu} \delta q^{\nu}\frac{\partial L(q^{\mu}, \dot{q}^{\mu})}{\partial q^{\nu}}
+\delta \dot{q}^{\nu}\frac{\partial L(q^{\mu}, \dot{q}^{\mu})}{\partial \dot{q}^{\nu}}\right)\D t
= \int \left(\sum_{\nu} \delta q^{\nu}\frac{\partial L(q^{\mu}, \dot{q}^{\mu})}{\partial q^{\nu}}
+\frac{\D}{\D t}\delta q^{\nu}\frac{\partial L(q^{\mu}, \dot{q}^{\mu})}{\partial \dot{q}^{\nu}}\right)\D t.
\end{equation}
Remember our strategy is to rearrange terms so the integrand looks like
\begin{equation}
\int \left(\sum_{\nu} \delta q^{\nu}\frac{\partial L(q^{\mu}, \dot{q}^{\mu})}{\partial q^{\nu}}
+\frac{\D}{\D t}\delta q^{\nu}\frac{\partial L(q^{\mu}, \dot{q}^{\mu})}{\partial \dot{q}^{\nu}}\right)\D t
= \int\sum_{\nu}\frac{\delta L(q^{\mu},
    \dot{q}^{\mu})}{\delta q^{\nu}}\delta q^{\nu}\,\D t.
\end{equation}
We use integration by parts on the time derivative of the variation to get
\begin{equation}
  \begin{split}
\int &\left(\sum_{\nu} \delta q^{\nu}\frac{\partial L(q^{\mu}, \dot{q}^{\mu})}{\partial q^{\nu}}
+\frac{\D}{\D t}\delta q^{\nu}\frac{\partial L(q^{\mu}, \dot{q}^{\mu})}{\partial \dot{q}^{\nu}}\right)\D t\\
&=\left.\delta q^{\nu}\frac{\partial L(q^{\mu}, \dot{q}^{\mu})}{\partial \dot{q}^{\nu}}\right|_{\text{bdry}}
+\int \left(\sum_{\nu} \delta q^{\nu}\frac{\partial L(q^{\mu}, \dot{q}^{\mu})}{\partial q^{\nu}}
-\delta q^{\nu}\frac{\D}{\D t}\frac{\partial L(q^{\mu}, \dot{q}^{\mu})}{\partial \dot{q}^{\nu}}\right)\D t.
  \end{split}
\end{equation}
The boundary contribution vanishes, so we are left with
\begin{equation}
\int \left(\sum_{\nu} \delta q^{\nu}\frac{\partial L(q^{\mu}, \dot{q}^{\mu})}{\partial q^{\nu}}
+\frac{\D}{\D t}\delta q^{\nu}\frac{\partial L(q^{\mu}, \dot{q}^{\mu})}{\partial \dot{q}^{\nu}}\right)\D t
=\int \left(\sum_{\nu} \delta q^{\nu}\frac{\partial L(q^{\mu}, \dot{q}^{\mu})}{\partial q^{\nu}}
-\delta q^{\nu}\frac{\D}{\D t}\frac{\partial L(q^{\mu}, \dot{q}^{\mu})}{\partial \dot{q}^{\nu}}\right)\D t.
\end{equation}
We can now use distributivity to write this as
\begin{equation}
\int \left(\sum_{\nu} \delta q^{\nu}\frac{\partial L(q^{\mu}, \dot{q}^{\mu})}{\partial q^{\nu}}
+\frac{\D}{\D t}\delta q^{\nu}\frac{\partial L(q^{\mu}, \dot{q}^{\mu})}{\partial \dot{q}^{\nu}}\right)\D t
=\sum_{\nu} \int \underbrace{\left(\frac{\partial L(q^{\mu}, \dot{q}^{\mu})}{\partial q^{\nu}}
-\frac{\D}{\D t}\frac{\partial L(q^{\mu}, \dot{q}^{\mu})}{\partial
  \dot{q}^{\nu}}\right)}_{=\delta L/\delta q^{\nu}}\delta q^{\nu}\,\D t.
\end{equation}
The first variation of the action vanishes if $\delta L/\delta q^{\nu}=0$ for each $\nu$.
This requires, for arbitrary variation $\delta q^{\nu}$, that
\begin{equation}
\boxed{\frac{\partial L(q^{\mu}, \dot{q}^{\mu})}{\partial q^{\nu}}
-\frac{\D}{\D t}\frac{\partial L(q^{\mu}, \dot{q}^{\mu})}{\partial \dot{q}^{\nu}}
= 0}
\end{equation}
for each $\nu=1,\dots,3N$. This is precisely the \define{Euler--Lagrange Equations},
and they are the equations of motion for Lagrangian mechanics.

\N{Principle: Generalizations Generalize}
In any field of study, whenver we introduce a generalization of an
existing concept, we should check that the ``existing concept'' can be
recovered by the generalization. For us, we have introduced this new way
to obtain the equations of motion (the Euler--Lagrange equations). We
should therefore check that we can recover Newton's equations of motion.
If we cannot, then something has gone horribly, horribly wrong.

\subsection{Recovering Newton}

\M
Observe the only place where acceleration can enter the Euler--Lagrange
equation is from the total time derivative. If we want to recover
Newton's second Law, we need
\begin{equation}
\frac{\D}{\D t}\frac{\partial L(q^{\mu}, \dot{q}^{\mu})}{\partial \dot{q}^{\nu}}
=m_{\nu}\ddot{q}^{\nu} + (\mbox{other stuff}),
\end{equation}
where $m_{\nu}$ is the mass for the body with acceleration component $\ddot{q}^{\nu}$.
Hence we see, for Newtonian systems described by Lagrangian mechanics,
we need
\begin{equation}
L(q^{\mu}, \dot{q}^{\mu}) =
\sum_{\nu}\frac{1}{2}m_{\nu}\dot{q}^{\nu}\dot{q}^{\nu} + (\mbox{other stuff}).
\end{equation}

\N{Free bodies}
For us to describe free bodies experiencing no forces, we need to
recover the equations of motion
\begin{equation}
\ddot{q}^{\nu} = 0.
\end{equation}
This is precisely described by
\begin{equation}
L_{\text{free}}(q^{\mu}, \dot{q}^{\mu}) = \sum_{\nu}\frac{1}{2}m_{\nu}\dot{q}^{\nu}\dot{q}^{\nu}.
\end{equation}

\N{Forces}
We now want to recover forces in Lagrangian mechanics. If we recall that
the potential energy $U(q)$ describes a force $\vec{F}$ by taking the
gradient of the negative potential energy $\vec{F}=-\nabla U$, then this
suggests we should consider the Lagrangian of the form:
\begin{equation}
L(q^{\mu}, \dot{q}^{\mu}) =
\left(\sum_{\nu}\frac{1}{2}m_{\nu}\dot{q}^{\nu}\dot{q}^{\nu}\right) \pm U(q^{\mu}).
\end{equation}
We just need to determine the sign of the potential energy contribution.

(\textbf{Exercise:} stop! Are we making sense? Check the dimensions of
the terms we are adding together, they should be the same. What does
that make the dimensions of the action?)

We find, swapping dummy variables we sum over, that the equations of
motion are:
\begin{equation}
\begin{split}
0 = &\frac{\partial}{\partial q^{\nu}}\left[\left(\sum_{\mu}\frac{1}{2}m_{\mu}\dot{q}^{\mu}\dot{q}^{\mu}\right) \pm U(q^{\mu})\right]
-\frac{\D}{\D t}\frac{\partial }{\partial \dot{q}^{\nu}}\left[\left(\sum_{\mu}\frac{1}{2}m_{\mu}\dot{q}^{\mu}\dot{q}^{\mu}\right) \pm U(q^{\mu})\right]\\
&= 
\frac{\partial}{\partial q^{\nu}}\left[\pm U(q^{\mu})\right]
-\frac{\D}{\D t}\frac{\partial }{\partial \dot{q}^{\nu}}\left(\sum_{\mu}\frac{1}{2}m_{\mu}\dot{q}^{\mu}\dot{q}^{\nu}\right).
\end{split}
\end{equation}
Hence
\begin{equation}
\mp\frac{\partial U(q^{\mu})}{\partial q^{\nu}} = m_{\nu}\ddot{q}^{\nu}.
\end{equation}
We can recover the familiar ``negative gradient of potential energy is
force'' physical principle by taking the Lagrangian for a generic
Newtonian system as:
\begin{equation}
\boxed{L(q^{\mu}, \dot{q}^{\mu}) =
\left(\sum_{\nu}\frac{1}{2}m_{\nu}\dot{q}^{\nu}\dot{q}^{\nu}\right) - U(q^{\mu}).}
\end{equation}
More generally, this suggests we should take the Lagrangian to be of the form:
\begin{equation}
L = \begin{pmatrix}\mbox{kinetic}\\\mbox{energy}\end{pmatrix}
-\begin{pmatrix}\mbox{potential}\\\mbox{energy}\end{pmatrix}.
\end{equation}
This recovers Newtonian mechanics correctly, and if we wish to
generalize to other settings (like relativistic mechanics or field theory) then we just
need to find the analogous quantities for kinetic and potential
energies.

\N{Observation 1: Arclength as Lagrangian problem}
We see that our motivating problem with arclength worked with a
different ``Lagrangian'', namely
\begin{equation}
L(q^{\mu},\dot{q}^{\mu}) = \sqrt{\sum_{\nu}\dot{q}^{\nu}\dot{q}^{\nu}}.
\end{equation}
Determine the equations of motion for this Lagrangian, and confirm we
recover the $\ddot\gamma=0$ condition.

\N{Observation 2: Generalized Momentum}
We see that Newton's second Law, in full generality, is
\begin{equation}
\frac{\D\vec{p}}{\D t} =\vec{F}
\end{equation}
where $\vec{p}$ is the momentum for the body. This suggests that there
is an analogous quantity for momentum in the Lagrangian setting, aptly
called the \define{Generalized Momentum}
\begin{equation}
p_{\nu}(q^{\mu},\dot{q}^{\mu}) := \frac{\partial L(q^{\mu},\dot{q}^{\mu})}{\partial\dot{q}^{\nu}}.
\end{equation}
Then the Euler--Lagrange equations is
\begin{equation}
\frac{\D p_{\nu}(q^{\mu},\dot{q}^{\mu})}{\D t} = \frac{\partial L(q^{\mu},\dot{q}^{\mu})}{\partial q^{\nu}}.
\end{equation}
Some people use the term \define{Generalized Force} to refer to the
quantity
\begin{equation}
F_{\nu} := \frac{\partial L(q^{\mu},\dot{q}^{\mu})}{\partial q^{\nu}}.
\end{equation}

\N{Exercise}
Jacobi considered a slightly different Lagrangian, one of the form:
\begin{equation}
L_{\text{Jacobi}} = \sqrt{\begin{pmatrix}\mbox{kinetic}\\\mbox{energy}\end{pmatrix}\begin{pmatrix}\mbox{potential}\\\mbox{energy}\end{pmatrix}}.
\end{equation}
This makes sense when we have nonzero potential energy. For a
1-dimensional point-particle, what does the equations of motion look
like for:
\begin{enumerate}
\item Spring potential: $U(q)=kq^{2}$
\item Gravitational potential: $U(q) = U_{0}/q$
\item Constant potential: $U(q)=U_{0}\neq0$.
\end{enumerate}
Are they the same equations of motion for Jacobi's Lagrangian as the
usual Lagrangian?

\subsection{Examples}

\N{How to solve mechanics problems}
Solving mechanics problems is a ``one-two punch'': set up the equations
of motion, then solve them. More explicitly,
\begin{enumerate}
\item Choose a coordinate system, i.e., a set of generalized coordinates
  $q^{1}$, \dots, $q^{n}$
\item Find the kinetic energy $K(q^{\mu},\dot{q}^{\mu})$ and potential
  energy $U(q^{\mu})$, then write the Lagrangian down as $L=K-U$.
\item Find the generalized momenta $p_{\nu} = \partial L/\partial\dot{q}^{\nu}$
  and generalized forces $F_{\nu} = \partial L/\partial q^{\nu}$.
\item Evaluate the time derivatives $\dot{p}_{\nu}$ and write down the
  equations of motion $\dot{p}_{\nu}=F_{\nu}$.
\end{enumerate}
It may be helpful to find conserved quantities.

\subsubsection{One-Dimensional Motion}

\N{Conservative Forces}
Consider one-dimensional mechanical system with potential energy
$U(x)$. Our Lagrangian is then
\begin{equation}
L = K - U = \frac{1}{2}m\dot{x}^{2} - U(x).
\end{equation}
We find its generalized momentum
\begin{equation}
p = \frac{\partial L}{\partial\dot{x}} = m\dot{x}.
\end{equation}
The equations of motion, from the Euler--Lagrange equations, is then:
\begin{equation}
\frac{\D}{\D t}\frac{\partial L}{\partial\dot{x}} - \frac{\partial L}{\partial x} = m\ddot{x} + U'(x) = 0.
\end{equation}
This is a second-order differential equation, which is generally hard to solve.
A lot of physics boils down to trying to avoid differential equations,
leveraging conserved quantities to extract information.

\N{Finding Conserved Quantities}
Now, we can multiply through by $\dot{x}$ to get:
\begin{equation}
0 = \dot{x}\left(m\ddot{x} + U'(x)\right) = \frac{\D}{\D t}\left(\frac{1}{2}m\dot{x}^{2}+U(x)\right)
= \frac{\D E}{\D t},
\end{equation}
where $E=K+U$ is the total energy of the system. Hence the total energy
is a constant with respect to time, i.e., it is a conserved
quantity. We can convert this to a first-order differential equation,
which may or may not be solvable, but it is more amenable to numerical
methods (i.e., the computer can brute force its way to an approximate solution).

\subsubsection{Two-Dimensional Motion}

\N{Kinetic Energies in Different Coordinate Systems}
How can we find the kinetic energy in different coordinate systems? For
example, in 2-dimensions, should we expect kinetic energy for polar
coordinates to be
\begin{equation}
K = \frac{\dot{r}^{2}}{2m} + \frac{\dot{\theta}^{2}}{2m}?
\end{equation}
The problem is that $\theta$ is dimensionless, and so this cannot
possibly function by dimensional analysis. What we should do is write
\begin{equation}
x = r\cos(\theta),\quad\mbox{and}\quad y=r\sin(\theta).
\end{equation}
Taking the time derivative
\begin{equation}
  \dot{x} = \dot{r}\cos\theta -r\dot\theta\sin\theta,\quad\mbox{and}\quad
  \dot{y} = \dot{r}\sin\theta +r\dot\theta\cos\theta.
\end{equation}
Then
\begin{subequations}
\begin{align}
\dot{x}^{2}+\dot{y}^{2} &= \left(\dot{r}\cos\theta -r\dot\theta\sin\theta\right)^{2}
+ \left(\dot{r}\sin\theta +r\dot\theta\cos\theta\right)^2\\
&= \left(\dot{r}^{2}\cos^{2}\theta
-2r\dot{r}\dot\theta\cos\theta\sin\theta
+r^{2}\dot\theta^{2}\sin^{2}\theta\right)
+ \left(\dot{r}^2\sin^2\theta + 2\dot{r}r\dot\theta\cos\theta\sin\theta +r^2\dot\theta^2\cos^2\theta\right)\\
%% &=\dot{r}^{2}\cos^{2}\theta
%% +r^{2}\dot\theta^{2}\sin^{2}\theta
%% \left(\dot{r}^2\sin^2\theta +r^2\dot\theta^2\cos^2\theta\right)\\
&=\dot{r}^{2}(\cos^{2}\theta + \sin^2\theta) + r^2\dot\theta^2(\sin^2\theta + \cos^2\theta)\\
&=\dot{r}^{2} + r^2\dot\theta^2.
\end{align}
\end{subequations}
This is generically how we determine kinetic energies in different
coordinate systems: find the velocities in the old coordinate system
expressed using the velocities in the new coordinate system, then plug
them into the expression for the kinetic energy.


\N{Exercise}
The free particle in 2-dimensions polar coordinates has its Lagrangian
be
\begin{equation}
L = \frac{m}{2}\dot{r}^{2} + \frac{m}{2}r^2\dot\theta^2.
\end{equation}
This gives us the generalized force
\begin{equation}
\frac{\partial L}{\partial r} = mr\dot\theta^{2}.
\end{equation}
But by definition, a free body experiences no force. What's going on?
[Hint: think about pseudoforces like the coriolis effect and centrifugal forces.]

\subsubsection{Spring-loaded pendulum}
\N{Problem Statement}
Consider a pendulum attached to a spring:
\begin{center}
  \includegraphics{img/img.1}
\end{center}
The bob of the pendulum has coordinate $(x_{1}, y_{1})$, and we suppose
the coordinates of the mass attached to the spring lies along the $y=0$
axis with its position being $(x + a, 0)$ where $a$ is the equilibrium
position of the mass. \emph{Assume there is no gravitational force
acting on the spring's block}. The
length of the pendulum's massless rod is $\ell$, and forms an angle
$\theta$ with the vertical. We can find the coordinates for the bob
explicitly as
\begin{subequations}
  \begin{align}
    x_{1} &= (a + x) + \ell\sin(\theta),\\
    \intertext{and}
    y_{1} &= - \ell\cos(\theta).
  \end{align}
\end{subequations}
Now we need to find the kinetic and potential energies (so we can
construct the Lagrangian).

\M
We can find the kinetic energy for the bob and block separately, then
add them together. The bob
\begin{subequations}
  \begin{align}
K_{\text{bob}} &= \frac{1}{2}m(\dot{x}_{1}^{2} + \dot{y}_{1}^{2})\\
&= \frac{1}{2}m([\dot{x} + \ell\dot\theta\cos(\theta)]^{2} + [\ell\dot\theta\sin(\theta)]^{2})\\
&= \frac{1}{2}m(\dot{x}^{2} + 2\ell\dot{x}\dot{\theta}\cos(\theta) + \ell^{2}\dot\theta^{2}\cos^{2}(\theta) + \ell^{2}\dot\theta^{2}\sin^{2}(\theta))\\
&= \frac{1}{2}m(\dot{x}^{2} + 2\ell\dot{x}\dot{\theta}\cos(\theta) + \ell^{2}\dot\theta^{2}).
  \end{align}
\end{subequations}
The block has kinetic energy:
\begin{equation}
K_{\text{block}} = \frac{1}{2}M\dot{x}^{2}.
\end{equation}
Together, we have the total kinetic energy for the system:
\begin{equation}
K = K_{\text{bob}} + K_{\text{block}} = \frac{1}{2}(M + m)\dot{x}^{2} + \frac{1}{2}m(2\ell\dot{x}\dot{\theta}\cos(\theta) + \ell^{2}\dot\theta^{2}).
\end{equation}
The potential energy is far simpler, it's the potential for the spring
$kx^{2}$ plus the gravitational force acting on the pendulum's bob
$gy_{1}$, hence
\begin{equation}
U = \frac{1}{2}kx^{2} - mg\ell\cos(\theta).
\end{equation}
Thus we obtain the Lagrangian,
\begin{equation}
L = \frac{1}{2}(M + m)\dot{x}^{2} + \frac{1}{2}m(2\ell\dot{x}\dot{\theta}\cos(\theta) + \ell^{2}\dot\theta^{2})
- \frac{1}{2}kx^{2} + mg\ell\cos(\theta).
\end{equation}
Observe there are two degrees of freedom, i.e., there are two
coordinates $x$ and $\theta$ from which we can determine the state of
the system. Therefore, we expect two equations of motion (one for $x$,
another for $\theta$).

\N{Setting up the Equations of Motion}
Now, the rest of the solution is just ``marching through the
calculations systematically''. We find the momenta:
\begin{subequations}
\begin{align}
p_{x} &= \frac{\partial L}{\partial\dot{x}} = (M + m)\dot{x} + m\ell\dot{\theta}\cos(\theta)\\
p_{\theta} &= \frac{\partial L}{\partial\dot{\theta}}
= m\ell\dot{x}\cos(\theta) + m\ell^{2}\dot\theta.
\end{align}
\end{subequations}
The generalized forces,
\begin{subequations}
\begin{align}
F_{x} &= \frac{\partial L}{\partial x} = -kx\\
F_{\theta} &= \frac{\partial L}{\partial \theta} = -m\ell\dot{x}\dot{\theta}\sin(\theta)
  -mg\ell\sin(\theta).
\end{align}
\end{subequations}
The equations of motion are therefore:
\begin{subequations}
\begin{align}
  (M + m)\ddot{x} + m\ell\ddot{\theta}\cos(\theta)
  - m\ell\dot{\theta}^{2}\sin(\theta) &= -kx\\
  m\ell\ddot{x}\cos(\theta)
  - m\ell\dot{x}\dot{\theta}\sin(\theta)
  + m\ell^{2}\ddot\theta &= -m\ell\dot{x}\dot{\theta}\sin(\theta)
  -mg\ell\sin(\theta).
\end{align}
\end{subequations}
Observe the second equation of motion simplifies to:
\begin{equation}
  m\ell\ddot{x}\cos(\theta) + m\ell^{2}\ddot\theta =  -mg\ell\sin(\theta).
\end{equation}

\N{Evaluate}
Does this make sense? Well, we recover the familiar pendulum by taking
$x=0$ and $k=0$, and the $\theta$ equation of motion survives as
\begin{equation}
m\ell^{2}\ddot\theta =  -mg\ell\sin(\theta).
\end{equation}
This is the familiar pendulum's equation of motion, so we're on the
right track.

\subsection{Properties}

\N{Non-Uniqueness of Lagrangian}
The equations of motion are obtained by demanding the variation of the
action be stationary
\begin{equation}
\delta S = 0.
\end{equation}
But if we shift the action by a constant, then we obtain the same
equations of motion. Equivalently, if we add a total time derivative to
the Lagrangian, then we add a constant term to the action. Therefore,
two Lagrangians $L_{1}$ and $L_{2}$ are equivalent if they differ by a
total time derivative. But this means Lagrangians are not unique: we
have many possible distinct Lagrangians describing the same system.

\N{Coordinate Independence}
The action is independent of choice of coordinates, and thus the
Euler--Lagrange equations do not depend on choice of coordinates.
Amazingly enough, pseudoforces (like the Coriolis effect) are taken into
account quite easily in Lagrangian mechanics.

\N{Problem: Conservation of Misery}
One problem is that we have not simplified Newtonian mechanics, but
provided a basis for generalization. We still end up with second-order
differential equations, which is a nightmare to solve. Sometimes that's
okay, but it motivates Hamiltonian mechanics which works with a system
of first-order differential equations.