\section{Hamiltonian Mechanics}

\N{Big Idea}
The big idea of Hamiltonian mechanics is to do a change of coordinates
from the Lagrangian ``generalized'' positions and velocities
$(q^{\mu},\dot{q}^{\mu})$ to positions and momenta $(q^{\mu}, p_{\mu})$.
If $\mu=1,2,\dots,N$, then the Lagrangian equations of motion would
consist of $N$ second-order differential equations. The Hamiltonian
approach will transform this into $2N$ first-order differential
equations
\begin{subequations}
\begin{align}
\frac{\D}{\D t}p_{\nu} &= f_{\nu}(q^{\mu},p_{\mu})\\
\intertext{and}
\frac{\D}{\D t}q^{\nu} &= v^{\nu}(q^{\mu},p_{\mu}).
\end{align}
\end{subequations}
The beautiful thing about Hamiltonian mechanics is that we can determine
the right-hand side of these equations \emph{algebraically} using a
fancy gadget called ``the Poisson bracket''.

\subsection{Basic Algorithm}

\N{Step 0: Write Down the Lagrangian}
The first step in Hamiltonian mechanics is to write down the
Lagrangian. This might sound weird, but it's the best way to approach
Hamiltonian mechanics. Once you've had some practice, this step becomes
``subconscious'' and you don't realize you're doing it. We will assume
you have obtained $L(q^{\mu}, \dot{q}^{\mu})$.

\N{Step 1: Find the Momentum}
Recall (\S\ref{chunk:lagrange:generalized-momentum}) we introduced the
idea of the generalized momentum. In Hamiltonian mechanics, we call it
the \define{Canonical Momentum} (or ``conjugate momentum''). The
conjugate momentum is then
\begin{equation}
p_{\nu}(q^{\mu},\dot{q}^{\mu}) := \frac{\partial L(q^{\mu},\dot{q}^{\mu})}{\partial\dot{q}^{\nu}}.
\end{equation}
Usually in Newtonian mechanics, $p_{\nu}\sim m_{\nu} f(q)\dot{q}^{\nu}$ where
$m_{\nu}$ is the mass of the body and $f(q)$ is some function of the
positions (possible just $f(q)=1$).

\N{Step 2: Change Coordinates}
Now we want to invert the relationship between momentum and
velocities. That is, we want to write down some expression relating
generalized velocities in terms of positions and momenta,
\begin{equation}
\dot{q}^{\nu} = \dot{q}^{\nu}(q^{\mu}, p_{\mu}).
\end{equation}
In Newtonian mechanics, this is usually easy, since
\begin{equation}
p_{\nu} = m_{\nu} f(q)\dot{q}^{\nu},
\end{equation}
which means
\begin{equation}
\dot{q}^{\nu} = \frac{1}{f(q)}\frac{p_{\nu}}{m_{\nu}},
\end{equation}
as we'd expect.

\begin{ddanger}
There are some caveats. For example, if we were measuring the position
of a body relative to some moving base, then the kinetic term in the
Lagrangian would look like:
\begin{equation}
L = \frac{m}{2}(\dot{q}_{\text{body}}-\dot{q}_{\text{base}})^{2} - U(q).
\end{equation}
We see that the momentum conjugate to $\dot{q}_{\text{body}}$ is just
$-1$ times the momentum conjugate to $\dot{q}_{\text{base}}$. So what?
Well, when we do change of coordinates, we need the Jacobian to be
invertible --- this is the technical condition for saying our new
coordinates is ``just as good'' as our old coordinates (in the sense
that we're able to label \emph{all} the same points using either
coordinate system). But if the momentum variables are not independent,
then the Jacobian matrix for the $\dot{q}^{\mu}\to p_{\mu}$ change of
coordinates is not invertible.

This might sound like a pathological case, and for the most part you're
right in this intuition: there are some redundancies in our description
of the physical system. This is precisely what ``gauge theory'' studies.
[That's ``gauge'' as in ``measure'' or ``appraise'', not as in ``gouge''
  or swindle.]
The ``redundancies'' are a sort of symmetry (a ``gauge symmetry''). This
is an extremely beautiful subject I wish I could pursue in greater
detail, but it is a huge tangent.
\end{ddanger}

\N{Step 3: Construct the Hamiltonian}
Most texts use the big, scary phrase ``Legendre transform the Lagrangian''.
Instead, consider writing down the total energy. We know that the
kinetic energy would be
\begin{equation}
K = \sum_{\mu}\frac{1}{2}p_{\mu}\dot{q}^{\mu},
\end{equation}
so we could be clever and obtain the total energy using the Lagrangian,
\begin{equation}
E = 2K - L.
\end{equation}
This is precisely how we construct the Hamiltonian,
\begin{equation}
\boxed{H(q^{\mu}, p_{\mu}) := \left(\sum_{\nu}p_{\nu}\dot{q}^{\nu}(q,p)\right) - L\bigl(q^{\mu}, \dot{q}^{\mu}(q,p)\bigr).}
\end{equation}
We do this for \emph{every} Lagrangian thrown at us, even ones as weird
as Jacobi's $L=\sqrt{KU}$.

\N{Step 4: Write the Action Integral}
We now can observe that the Lagrangian is related to the Hamiltonian by,
\begin{equation}
L\bigl(q^{\mu}, \dot{q}^{\mu}(q,p)\bigr) = \left(\sum_{\nu}p_{\nu}\dot{q}^{\nu}(q,p)\right) - 
H(q^{\mu}, p_{\mu}).
\end{equation}
Just substitute this into the action integral, and we get the action
``in Hamiltonian form'':
\begin{equation}\label{eq:hamiltonian:action-integral}
S[q^{\mu}, p_{\mu}] = \int \left[\left(\sum_{\nu}p_{\nu}\dot{q}^{\nu}(q,p)\right) - 
H(q^{\mu}, p_{\mu})\right]\D t
\end{equation}
We obtain the equations of motion by considering arbitrary variations of
the action with respect to the position $\delta q^{\mu}$ and momenta
$\delta p_{\mu}$ which vanish at the endpoints. That's it, that's how we
obtain the Hamiltonian equations of motion, which resemble the
Euler--Lagrange equations of motion. I'll work this out in detail in the
next section (which you could skip), but that's what we're looking for
with any system of analytical mechanics: equations of motion.

\N{Hamilton's Equations}
The equations of motion for Hamiltonian mechanics are
\begin{equation}
\frac{\D q^{\mu}}{\D t} = \frac{\partial H}{\partial p_{\mu}}
\end{equation}
and
\begin{equation}
\frac{\D p_{\mu}}{\D t} = -\frac{\partial H}{\partial q^{\mu}}.
\end{equation}

\M
As a consistency check, we can look at the ``stock example'' of a
1-dimensional Newtonian system described by the Lagrangian
\begin{equation}
L = \frac{1}{2}m\dot{q}^{2} - U(q).
\end{equation}
We see the momentum variable would be,
\begin{equation}
p = m\dot{q},
\end{equation}
then the Hamiltonian would be,
\begin{equation}
H = \frac{1}{2m}p^{2} + U(q).
\end{equation}
The equations of motion are then
\begin{subequations}
\begin{align}
\frac{\D q}{\D t} &= \frac{\partial H}{\partial p}\\
&= \frac{1}{m}p
\end{align}
\end{subequations}
and
\begin{subequations}
\begin{align}
\frac{\D p}{\D t} &= -\frac{\partial H}{\partial q}\\
&= -\frac{\D U(q)}{\D q}.
\end{align}
\end{subequations}
These match our intuition that the time derivative of position is
precisely velocity (i.e., directly proportional to momentum) and the
time derivative of momentum is force. This is good news: Hamiltonian
mechanics reproduces the equations of motion we know and love (in
1-dimension, at least).

\subsection{Derivation of Equations of Motion}

\M
The basic idea is to insert into the action $q^{\mu} + \delta q^{\mu}$
and $p_{\mu} + \delta p_{\mu}$, then collect terms about the variations.
As usual, we ignore boundary terms, then insist the coefficient of
$\delta q^{\mu}$ and $\delta p_{\mu}$ must vanish.

This is messy if we start with the action integral in Eq~\eqref{eq:hamiltonian:action-integral},
let us try. We have
\begin{equation}
\delta S = \int^{t_{2}}_{t_{1}}\left(\sum_{\rho}\dot{q}^{\rho}\,\delta
p_{\rho}
+p_{\rho}\,\delta\dot{q}^{\rho}
-\frac{\partial H}{\partial q^{\rho}}\delta q^{\rho}
-\frac{\partial H}{\partial p_{\rho}}\delta p_{\rho}
\right)\D t
\end{equation}
We cheat and write
\begin{equation}
\delta\dot{q}^{\rho} = \frac{\D}{\D t}\delta q^{\rho},
\end{equation}
then integrate-by-parts on that term (throwing away boundary terms) to
obtain
\begin{equation}
\delta S = \int^{t_{2}}_{t_{1}}\left(\sum_{\rho}\dot{q}^{\rho}\,\delta
p_{\rho}
-\frac{\D p_{\rho}}{\D t}\,\delta q^{\rho}
-\frac{\partial H}{\partial q^{\rho}}\delta q^{\rho}
-\frac{\partial H}{\partial p_{\rho}}\delta p_{\rho}
\right)\D t.
\end{equation}
Now we gather terms
\begin{equation}
  \delta S = \sum_{\rho}\int^{t_{2}}_{t_{1}}\left[\left(\dot{q}^{\rho}
-\frac{\partial H}{\partial p_{\rho}}\right)\delta
p_{\rho}
+\left(-\frac{\D p_{\rho}}{\D t}-\frac{\partial H}{\partial q^{\rho}}\right)\delta q^{\rho}
\right]\D t.
\end{equation}
Each coefficient must separately vanish for each $\delta p_{\rho}$,
and similarly each coefficient of $\delta q^{\rho}$ must separately
vanish.
This gives us, from varying $\delta q$,
\begin{equation}
-\frac{\D p_{\rho}}{\D t}-\frac{\partial H}{\partial q^{\rho}} = 0,
\end{equation}
and from varying momentum $\delta p$,
\begin{equation}
\frac{\D q^{\rho}}{\D t} - \frac{\partial H}{\partial p_{\rho}} = 0.
\end{equation}
These are Hamilton's equations of motion. Huzzah.

\subsection{Applications and Things to Ponder}

\M
The big application where Hamiltonian mechanics is useful occurs when
the total energy is conserved for a system. Then $H$ is ``just'' a
constant.

\N{Differential Geometry}
Another thing worth pondering is, well, what are we doing
\emph{geometrically}? Mechanics concerns itself with the trajectory of
bodies, which are mathematically described by curves. It turns out there
is a beautiful relationship between the tangent bundle $\mathrm{T}\RR^{3N}$
and the space of positions and velocities, namely they are identical.
If we have a different ``configuration space'' of possible positions,
call it $\mathcal{C}$, then the Lagrangian description uses
$\mathrm{T}\mathcal{C}$. 

The space of positions and momenta have a surprisingly dual geometric
description, they are described as the cotangent bundle
$\mathrm{T}^{*}\RR^{3N}$, or more generally for a configuration space
$\mathcal{C}$ we would use $\mathrm{T}^{*}\mathcal{C}$ the cotangent bundle.

\N{Source for Numerical Methods}
The last application worth mentioning, just in passing, is that the
Hamiltonian formalism can serve as the basis for coming up with
numerical methods to solve the equations of motion. These are so-called
``Symplectic Integrators''. One minor problem some symplectic
integrators face is, when run over a very long period of time, the
conservation of energy is violated slightly. This can be
detrimental. Energy drift resembles a parametrized resonance. I'm not
well versed enough to say whether we could do something clever and add
some ``fictitious'' physical phenomena which cancels out the energy
drift contribution, or if we are helpless to watch in horror.

\subsection{Symmetries and Poisson Brackets}

\N{Poisson Brackets}
We can write \emph{any} physical quantity as a function of the phase
space variables [position and momenta] and time, $f(q^{\mu}, p_{\mu}, t)$.
Its change over time is given by the chain rule,
\begin{equation}
\frac{\D f}{\D t} = \frac{\partial f}{\partial t} +
\sum_{\rho}\frac{\partial f}{\partial q^{\rho}}\dot{q}^{\rho} +  
\frac{\partial f}{\partial p_{\rho}}\dot{p}_{\rho}.
\end{equation}
We can use Hamilton's equations to write the time derivatives of
positions and momentas in terms of partial derivatives of the Hamiltonian,
\begin{equation}
\frac{\D f}{\D t} = \frac{\partial f}{\partial t} +
\sum_{\rho}\frac{\partial f}{\partial q^{\rho}}\frac{\partial H}{\partial p_{\rho}} -
\frac{\partial f}{\partial p_{\rho}}\frac{\partial H}{\partial q^{\rho}}.
\end{equation}
But this summation contribution is useful, and we abstract it away using
a new gadget called the \define{Poisson Bracket}, writing:
\begin{equation}
\frac{\D f}{\D t} = \frac{\partial f}{\partial t} + \{f, H\},
\end{equation}
where
\begin{equation}
\{f, H\} = \sum_{\rho}\frac{\partial f}{\partial q^{\rho}}\frac{\partial H}{\partial p_{\rho}}
- \frac{\partial H}{\partial q^{\rho}}\frac{\partial f}{\partial p_{\rho}}
\end{equation}
is the Poisson bracket. We see we can recover Hamilton's equations if we
insert $f=p_{\rho}$ or $f=q^{\rho}$.

Our intuition of the Poisson bracket should be that, when we fix the
second slot to be $H$, it acts like a time derivative operator (when a
function does not explicitly depend on time),
\begin{equation}
\{-,H\}\approx\frac{\D}{\D t}.
\end{equation}
There are a bunch of neat properties the Poisson bracket satisfies,
which I'll turn into exercises.

\begin{exercise}
Prove, for any function $h(q^{\mu}, p_{\mu})$, that $\{-,h\}$ is a
derivation. That is, $\{f + g, h\} = \{f, h\} + \{g, h\}$ and $\{fg,h\} = \{f,h\}g + f\{g,h\}$.
\end{exercise}

\begin{exercise}
Prove the Poisson bracket is anticommutative, that is, $\{f,g\}=-\{g,f\}$.
In particular, this implies $\{f,f\}=0.$
\end{exercise}

\N{Constants of Motion}
It's important to note we can find constants of motion $k(q^{\mu},p_{\mu})$
using the Poisson bracket: they commute with the Hamiltonian! That is,
\begin{equation}
\{k, H\} = 0.
\end{equation}
These are first-order differential equations, but knowing the constants
of motion can greatly simplify solving the equations of motion.

\N{Definition}
When the positions and momenta satisfy
\begin{align}
  \{q^{\mu},q^{\nu}\} &= 0\\
  \{p_{\mu},p_{\nu}\} &= 0\\
  \{q^{\mu},p_{\nu}\} &= {\delta^{\mu}}_{\nu}
\end{align}
where ${\delta^{\mu}}_{\nu} = 1$ if $\mu=\nu$ and $0$ otherwise, then we
call these particular coordinates \define{Canonical}.

\N{Symmetries}
We can encode symmetries using the Poisson bracket. This is what Sophus
Lie studied, which led him to invent Lie groups and Lie
algebras. The basic idea is, just as $\{-,H\}$ acts like a time
derivative operator, then $\{-,H\,\D t\}$ acts like the first term in a
Taylor expansion. We replace $H\,\D t$ with a ``generator'' of a
symmetry, something like $\varepsilon(t)\,C(q^{\mu},p_{\mu})$ where
$\varepsilon(t)$ is an ``infinitesimal'' parametrized smoothly by time. This
would describe an ``infinitesimal symmetry'' of the system if it leaves
the equations of motion invariant. When would this happen? Well, if we
apply it to the Lagrangian, then we should get a total time derivative
(since they contribute nothing to the action):
\begin{equation}
\{\sum_{\rho}p_{\rho}\dot{q}^{\rho} - H, \varepsilon(t)\,C(q^{\mu},p_{\mu})\}
 = \frac{\D f}{\D t}
\end{equation}
for some function $f$. Then variations generated by
\begin{equation}
\delta q^{\rho} = \{q^{\rho}, \varepsilon(t)\,C(q^{\mu},p_{\mu})\}
\end{equation}
and
\begin{equation}
\delta p_{\rho} = \{p_{\rho}, \varepsilon(t)\,C(q^{\mu},p_{\mu})\}
\end{equation}
contribute nothing --- these imposters describe pure symmetries.

\N{Canonical Transformations}
One way to find the generators for symmetries is to consider change of
coordinates
\begin{equation}
Q^{\rho}=Q^{\rho}(q^{\mu},p_{\mu}),\quad\mbox{and}\quad
P_{\rho}=P_{\rho}(q^{\mu},p_{\mu}),
\end{equation}
such that they preserve the Poisson bracket. This is precisely the
demand that $Q^{\rho}$ and $P_{\rho}$ are canonical coordinates
themselves. That is, if we adopt the notation writing the coordinates
for the Poisson bracket as a suffixed indexed,
\begin{equation}
\{f, g\}_{qp} := \sum_{\rho}\frac{\partial f}{\partial q^{\rho}}\frac{\partial g}{\partial p_{\rho}}
- \frac{\partial g}{\partial q^{\rho}}\frac{\partial f}{\partial p_{\rho}},
\end{equation}
then
\begin{equation}
\{Q^{\rho}, Q^{\sigma}\}_{qp} = \{P_{\rho}, P_{\sigma}\}_{qp} = 0
\end{equation}
and
\begin{equation}
\{Q^{\rho}, P_{\sigma}\}_{qp} = {\delta^{\rho}}_{\sigma}.
\end{equation}

\begin{exercise}
Find the equations of motion for $Q^{\rho}$, $P_{\rho}$ ``as if'' they
were generic functions of $q^{\mu}$, $p_{\mu}$. You should find for
$Q^{\rho}$ that
\begin{equation}
\dot{Q}^{\rho} = \sum_{\sigma}\{Q^{\rho}, Q^{\sigma}\}\frac{\partial H}{\partial Q^{\sigma}}
+\{Q^{\rho}, P_{\sigma}\}\frac{\partial H}{\partial P_{\sigma}}.
\end{equation}
Observe if $Q^{\rho}$, $P_{\rho}$ were canonical coordinates, then $\{Q^{\rho},Q^{\sigma}\}=0$
and we recover Hamilton's equations.
\end{exercise}

\begin{exercise}
If we invert the change of coordinates, writing
$q^{\mu}=q^{\mu}(Q^{\rho}, P_{\rho})$ and $p_{\mu}=p_{\mu}(Q^{\rho}, P_{\rho})$
as functions, and
\begin{equation}
\{f, g\}_{QP} = \sum_{\rho}\frac{\partial f}{\partial Q^{\rho}}\frac{\partial g}{\partial P_{\rho}}
- \frac{\partial g}{\partial Q^{\rho}}\frac{\partial f}{\partial P_{\rho}},
\end{equation}
then is it true that $\{q^{\mu},q^{\nu}\}_{QP}=\{p_{\mu},p_{\nu}\}_{QP}=0$
and $\{q^{\mu},p_{\nu}\}_{QP}={\delta^{\mu}}_{\nu}$?
\end{exercise}

\M
This gives us some sense of a ``symmetry'' of a Hamiltonian system as a
change of coordinates which leaves the system ``invariant''. A natural
line of questioning when we are handed a Hamiltonian system is to ask
ourselves, ``What are the symmetries of this thing?''

We have discovered, thanks to the heroic efforts of Paul Dirac and Peter
Bergmann, analyzing the symmetries of a Hamiltonian system is described
by constraints on the phase space. This is precisely what gauge theory
studies. The \emph{really} exciting symmetries occur when the change of
coordinates from Lagrangian velocities to Hamiltonian momenta is
degenerate --- this is when there are redundancies in our Lagrangian
description of the system, which requires careful analysis. Sadly, the
only textbook on this subject is contained in the first 5 chapters (or so) of
Henneaux and Teitelboim's \textit{Quantization of Gauge Systems}.

\begin{exercise}
  For the Lagrangian given by
  \begin{equation}
L(q^{\mu}, \dot{q}^{\mu}) = \sqrt{\sum_{\rho}\dot{q}^{\rho}\dot{q}^{\rho}},
  \end{equation}
  find the canonically conjugate momenta, then find the Hamiltonian.
  This Lagrangian enjoys reparametrization invariance, i.e., we can
  ``change variables'' for time rather freely. The associated constraint
  for such systems should be that $H$ ``vanishes'', which should be
  interpreted as describing a surface given by the zeros of $H(q,p)$.
\end{exercise}

\begin{exercise}
For the Jacobi Lagrangian, $L =\sqrt{KU}$, what is the canonical
momentum? What is the Hamiltonian? Are there any interesting symmetries
when we have the simple harmonic oscillator $U(q)=kq^{2}$? What about
for gravity $U(q)=g/q$?
\end{exercise}

\N{``Canonical'' Mechanics}
Jacobi referred to Hamiltonian mechanics as ``canonical'', introducing
the term in ``Note sur l'int\'egration des \'equations diff\'erentielles de la Dynamique''
(1835):
\begin{quote}
Given any system of elements between which and
time we have, in the perturbed motion, a system of differential equations
of canonical form, find all the other
systems of elements which enjoy the same property.\footnote{Translated
from the French: ``\textit{\'Etant donn\'e un syst\`eme quelconque d'\'el\'emente entre lesquels et
le temps on a, dans le mouvement troubl\'e, un syst\`em d'\'equations
diff\'erentielles de la form canonique, trouver tous les autres
syst\`emes d'\'el\'ements qui jouissent de la m\^eme propri\'et\'e.}''}
\end{quote}
And in Jacobi's ``Note sur l’int\'egration des \'equations g\'en\'erales de la dynamique'' (1837):

\begin{quote}
  We find by means of this theorem, by calculation itself,
elements whose differential values, in the disturbed movement,
take the simple form that they have in the theorem, a form that I
is referred to in my text as \emph{canonical}.\footnote{Translation from
the French original, ``\textit{On trouve au moyen de ce th\'eor\`eme, par le
calcul m\^eme, des \'el\'ements dont les valeur differentialles, dans le
mouvement troubl\'e, prennent la form simple qu'elles ont dans le
th\'eor\`eme, forme que je d\'esigne dans mon m\'emoire sous le nom de
\emph{canonique}.}''}
\end{quote}

It is rather magical that \emph{all} physical systems in Hamiltonian
mechanics can have canonical coordinates. Jacobi believed that this made
Hamiltonian mechanics \emph{canonical} or ``obviously preferred'' [in
some sense].

Craig Fraser and Michiyo Nakane have written a bit about the history of
canonical transformations in Hamiltonian mechanics, which may be
interesting to peruse.\footnote{See
\url{https://notes.math.ca/en/article/a-collaborative-research-project-in-the-history-of-mathematics-the-history-of-canonical-transformations-in-hamilton-jacobi-theory/}}

\subsection{Dirac's Homogeneous Lagrangian Theorem}

\M This is a series of exercises. They're based on
Paul Dirac's article ``Homogeneous variables in classical dynamics''
\textit{Mathematical Proceedings of the Cambridge Philosophical Society}
\textbf{29} no.3 (1933) pp.389--400 {\tt\doi{10.1017/S0305004100016431}}.

\N{Physical Setting}
Suppose we want to make time a coordinate. We do this by distinguishing
two different notions of time:
\begin{enumerate}
\item a formal parameter ``label time'' $\tau$
  (which intuitively is analogous to proper time), and
\item ``coordinate time'' $t$ (Newton's absolute time) now elevated to
  be a position variable.
\end{enumerate}
Working with one spatial dimension, so we don't have to worry about
indices. Our initial action is,
\begin{equation}
S[q, \D q/\D t] = \int L\left(q,\frac{\D q}{\D t}\right)\,\D t.
\end{equation}
Now, we want to parametrize ``coordinate time'' by some label, and
update the action to reflect this.
We parametrize position $q$ using $\tau$, and time $t$ is parametrized
by $\tau$, so the action now becomes
\begin{equation}
S[q(\tau), \frac{\D q(\tau)}{\D \tau}, t(\tau)] = \int L\left(q(\tau), \frac{\D q(\tau)/\D \tau}{\D t(\tau)/\D \tau}\right) \underbrace{\frac{\D t(\tau)}{\D \tau}\,\D \tau}_{=\,\D t}.
\end{equation}
Here we used the chain rule to write
\begin{equation}
\frac{\D q}{\D t} \mbox{``=''} 
\frac{\D q}{\D \tau}
\frac{\D \tau}{\D t(\tau)}
=\frac{\D q}{\D \tau}
\frac{1}{\D t(\tau)/\D \tau}
\end{equation}
If we denote differentiation with respect to $\tau$ by dots, and define
\begin{equation}
\widetilde{L}(q,\dot{q},\dot{t}) := \dot{t} L\left(q, \frac{\dot{q}}{\dot{t}}\right),
\end{equation}
we can now take as exercises proving the following claims.

\begin{exercise}
The action is invariant under the symmetry for the Lagrangian, using any
function of label time $f(\tau)$,
\begin{equation}\label{eq:hamiltonian:dirac-xca:homogeneous-condition}
\widetilde{L}(q,f(\tau)\dot{q},f(\tau)\dot{t})
=f(\tau)\widetilde{L}(q,\dot{q},\dot{t}).
\end{equation}
\end{exercise}

\begin{exercise}
It's also not hard to prove the momentum from the original Lagrangian $L[q(t),q'(t)]$ (where prime indicates differentiation with respect to $t$) 
\begin{equation}
p = \frac{\partial L}{\partial q'(t)}
\end{equation}
when compared to $\widetilde{p}(\tau)$ the momentum with respect to $q$ in $\widetilde{L}$
\begin{equation}
\widetilde{p} = \frac{\partial \widetilde{L}}{\partial \dot{q}(\tau)}
\end{equation}
are the same, i.e., satisfy
\begin{equation}
p = \dot{t}\frac{\partial L}{\partial (\dot{q}/\dot{t})}\frac{1}{\dot{t}} = \widetilde{p}.
\end{equation}
\end{exercise}

\begin{exercise}
Prove the conjugate momentum for $\dot{t}(\tau)$ in $\widetilde{L}$ is
the negative Hamiltonian
\begin{subequations}\label{eq:hamiltonian:dirac-xca:conjugate-momentum-is-negative-hamiltonian}
\begin{align}
p_{t} &= \frac{\partial\widetilde{L}}{\partial\dot{t}} \\
&= L\left(q,\frac{\D q}{\D t}\right) - \frac{\D }{\D t}\frac{\partial L(q,\D q/\D t)}{\partial(\D q/\D t))}\\
&= -H(q,\D q/\D t)
\end{align}
\end{subequations}
\end{exercise}

\begin{exercise}
Verify the Hamiltonian associated with $\widetilde{L}$ may be written, using the Legendre transform, as
\begin{equation}
\widetilde{H} = p_{t}\dot{t} + \widetilde{p}\dot{q} - \widetilde{L}.
\end{equation}
Prove
\begin{equation}\label{eq:hamiltonian:dirac-xca:new-hamiltonian}
\widetilde{H} = \dot{t}(H + p_{t}).
\end{equation}
And now we should begin to worry. Why? See, in Equation~\eqref{eq:hamiltonian:dirac-xca:conjugate-momentum-is-negative-hamiltonian} we proved $p_{t}=-H$, and in Equation~\eqref{eq:hamiltonian:dirac-xca:new-hamiltonian} we proved the Hamiltonian associated with $\widetilde{L}$ is directly proportional to $p_{t}+H$. But Equation~\eqref{eq:hamiltonian:dirac-xca:conjugate-momentum-is-negative-hamiltonian} implies $p_{t}+H=0$, which means we expect $\widetilde{H}=0$. That's odd.

This is because $\widetilde{H}=0$ is a \textbf{constraint} (in the sense that, not every initial value of $t(\tau_{0})$ and $q(\tau_{0})$ and $\dot{q}(\tau_{0})$ satisfy this --- we are constrained in our choice of initial data by this constraint). And, looking at Equation~\eqref{eq:hamiltonian:dirac-xca:new-hamiltonian}, we see that $\dot{t}$ acts like the Lagrange multiplier enforcing this constraint.
\end{exercise}

\begin{exercise}
Prove every Lagrangian which is Homogeneous in velocities --- i.e., satisfies Eq~\eqref{eq:hamiltonian:dirac-xca:homogeneous-condition} --- will have a Hamiltonian constraint $H=0$.
\end{exercise}