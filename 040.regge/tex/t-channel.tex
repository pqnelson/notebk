\chapter{Physics of the \texorpdfstring{$t$}{t}-Channel and Complex Angular Momenta}

\M We have discussed analytic properties of the invariant amplitude
$\amplitude(s,t,u)$ describing the scattering of neutral spinless
particles (i.e., $\pi^{0}$ mesons).

We stressed how it is convenient to depict the kinematics of the
reactions and locations of singularities for $\amplitude(s,t)$ on the
Mandelstam plane. We also saw crossing symmetry lets the same amplitude
$\amplitude(s,t,u)$ describe the three processes:
\begin{equation}
\left.\begin{array}{ccc}
a + b\to c+d          \quad& s>4\mu^{2},~ t,u<0 & \mbox{$s$-channel}\\
a+\bar{c}\to\bar{b}+d & t>4\mu^{2},~ s,u<0 & \mbox{$t$-channel}\\
a+\bar{d}\to\bar{b}+c & u>4\mu^{2},~ t,s<0 & \mbox{$u$-channel}\\
\end{array}\right\}
\end{equation}
We also saw the invariant amplitude has singularities at the thresholds
of the corresponding reactions $s=4\mu^{2},16\mu^{2},\dots$;
 $t=4\mu^{2},16\mu^{2},\dots$;  $u=4\mu^{2},16\mu^{2},\dots$.

\M
In the physical region of the $s$-channel, the singularities of the
amplitude $\amplitude(s,t)$ in $s$ are related to the possibility of the
transition of the initial state in the process of scattering into the
intermediate states with two, four, etc., particles. Its imaginary part
(which Gribov denotes by $\amplitude_{1}(s,t)$) is determined by the
unitarity condition in the $s$-channel, which is equivalent to the
physical requirement that the sum of probabilities for transitions into
all possible states $n$ should be equal to unity
\begin{equation}
\Im_{s}\amplitude := \amplitude_{1} = \frac{1}{2}\sum_{n}\amplitude_{n}\amplitude_{n}^{*}.
\end{equation}
Gribov tells us this can be represented graphically using Feynman diagrams:
\begin{equation}
  2\I\Im_{s}\left(
  \eqgraph{0pc}{0pc}{\fmfframe(12,12)(12,12){\begin{fmfgraph*}(60,48)%(48,48)
    \fmfpen{0.5bp}
    \fmfleft{i1,i2}
    \fmfright{o1,o2}
    \fmf{plain}{i1,v}
    \fmf{plain}{i2,v}
    \fmf{plain}{v,o1}
    \fmf{plain}{v,o2}
    \fmfv{l.a=180,label=$p_{1}$}{i1}
    \fmfv{l.a=180,label=$p_{2}$}{i2}
    \fmfv{l.a=0,label=$p_{3}$}{o1}
    \fmfv{l.a=0,label=$p_{4}$}{o2}
    \fmfblob{0.25w}{v}
  \end{fmfgraph*}}}
\right)
  \quad=
  \eqgraph{0pc}{0pc}{
  \begin{fmfgraph*}(120,48)
    \fmfpen{0.5bp}
    \fmfcmd{%
      vardef cross_bar (expr p, len, ang) =
        ((-len/2,0)--(len/2,0))
          rotated (ang + angle direction length(p)/2 of p)
          shifted point length(p)/2 of p
      enddef;
      style_def crossed expr p =
        cdraw p; label(btex $\times$ etex, point length(p)/2 of p);
%        ccutdraw cross_bar (p, 6pt, 45);
%        ccutdraw cross_bar (p, 6pt, -45)
      enddef;}
    \fmfleft{i1,i2}
    \fmfright{o1,o2}
    \fmf{plain}{i1,v1}\fmf{plain}{v2,o1}
    \fmf{plain}{i2,v1}\fmf{plain}{v2,o2}
    \fmf{crossed,left=0.25}{v1,v2}
    \fmfblob{0.1w}{v1}
    \fmfblob{0.1w}{v2}
%%     \fmfv{l.a=45,label=$A(s+{\rm i}\varepsilon)$}{i2}
%%     \fmfv{l.a=135,label=$A(s-{\rm i}\varepsilon)$}{o2}
    \fmfv{l.a=90,label=$A(s+{\rm i}\varepsilon)$}{i2}
    \fmfv{l.a=90,label=$A(s-{\rm i}\varepsilon)$}{o2}
    \fmffreeze
    \fmf{crossed,left}{v1,v2}
    \fmf{crossed,left=0.75}{v1,v2}
    \fmf{crossed,right=0.25}{v1,v2}
    \fmf{crossed,right}{v1,v2}
    \fmf{crossed,right=0.75}{v1,v2}
  \end{fmfgraph*}}
\end{equation}

\N{Imaginary part from cutting virtual lines}
To find the discontinuity (imaginary part) of the scattering amplitude
correspnding to a given Feynman diagram, it suffices to cut the diagram
in all possible ways into two connected parts, having incoming and
outgoing particles on opposite sides of the cut. The cut must be
identified as real particles in the intermediate state, and we can
associate with each of them the factor $2\pi\theta(\pi_{i,0})\delta(\pi_{i}^{2}-m^{2})$.
These lines are marked by crosses $\times$.

Further, we have to replace $\I\varepsilon$ by $-\I\varepsilon$ in all
propagators of the block lying to the right of the cut, which
corresponds to the conjugate amplitude $\amplitude^{*}(s+\I\varepsilon)=\amplitude(s-\I\varepsilon)$.

Apart from these modifications, the Feynman rules remain unchanged.

\section{Analytical Continuation of the \texorpdfstring{$t$}{t}-Channel Unitarity Condition}

\subsection{The Mandelstam Representation}

\subsection{Inconsistency of the ``Black Disk'' Model of Diffraction}

\section{Complex Angular Momenta}


\section{Partial Wave Expansion and Sommerfeld--Watson Representation}

\M We can consider the partial-wave expansion for the amplitude
$\amplitude(s,t)$ in the physical region of the $t$-channel, where
$t>4\mu^{2}$ and $s<0$:
\begin{subequations}
  \begin{align}
    \amplitude(s,t) &= \sum^{\infty}_{n=0}(2n+1)f_{n}(t)P_{n}(z_{t})\\
    \intertext{where}
    z_{t} &= 1 + \frac{2s}{t-4\mu^{2}}.
  \end{align}
\end{subequations}
The series is convergent up to the first $s$ plane singularity of
$\amplitude(s,t)$, that is, for $s<4\mu^{2}$. Note: we use $n$
instead of $\ell$ to emphasize the partial waves in this sum are defined
only for integer values of angular momentum.

\M It is natural to expect the partial-wave amplitudes are large up to
$n=kr_{0}$ where $r_{0}$ is the interaction radius.

\N{Question: asymptotic behaviour?}
What is the asymptotic behaviour for this series at large $z$? We know
the series diverges.

Perhaps there is an appropriate analytical continuation for which the
partial-wave amplitudes with $n\sim kr_{0}$ are dominant at $n\leq n_{0}=kr_{0}$.
At large $s$ this would behave like $\amplitude(s,t)\sim z^{n_{0}}$.
This implies the asymptotics of $\amplitude(s,t)$ are related to the
effective orbital angular momentum that is possible at a given $t$.

(Sommerfeld, Fock, and Schwinger developed the mathematical machinery we
seek. They were interested in radio waves bouncing around the Earth's
atmosphere. Regge applied it to quantum mechanics.)

\N{Sommerfeld--Watson Transformed Amplitude}
Suppose we can find an analytic function $f_{\ell}(t)$ that does not
increase exponentially in any direction in the right-half of the complex
$\ell$ plane, and whose values coincide with the partial wave amplitudes
at all integer $\ell$:
\begin{equation}
\left.f_{\ell}\right|_{\ell=n}=f_{n}.
\end{equation}
Then we can write the partial-wave expansion as a contour integral:
\begin{equation}\label{eq:t-channel:sommerfeld-watson:deformed-integral}
\boxed{\amplitude = \frac{\I}{2}\int_{L}f_{\ell}P_{\ell}(-z)(2\ell+1)\frac{\D\ell}{\sin(\pi\ell)},}
\end{equation}
where the contour $L$ is drawn below:
\begin{center}
  \includegraphics{img/ch2sec3.0}
\end{center}

\M
We recall the large $\ell$ asymptotics of the Legendre function
\begin{subequations}
\begin{equation}\label{eq:t-channel:sommerfeld-watson:asymptotics-of-legendre1}
P_{\ell}(z)\sim\exp(\I\ell\theta) + \exp(-\I\ell\theta)
\end{equation}
as $\ell\to\infty$ (where $z=\cos(\theta)$), and therefore
\begin{equation}\label{eq:t-channel:sommerfeld-watson:asymptotics-of-legendre2}
P_{\ell}(-z)\sim\exp(\I\ell(\pi-\theta)) + \exp(-\I\ell(\pi-\theta)).
\end{equation}
\end{subequations}
These tell us that $P_{\ell}(-z)/\sin(\pi\ell)$ falls exponentially as
$\Im(\ell)\to\infty$ in the physical region $0<\theta<\pi$.

Therefore we now deform the contour $L$ in the $\ell$-plane, avoiding
any possible singularities of $f_{\ell}(t)$ (specifically passing them
``on the right''), as doodled below:
\begin{center}
  \includegraphics{img/ch2sec3.1}
\end{center}
After the contour has been deformed, the function defined by Eq~\eqref{eq:t-channel:sommerfeld-watson:deformed-integral}
is defined over the entire complex $z$-plane.

\N[1]{Check the region \texorpdfstring{$z<-1$}{z < -1}}
Let us check that Eq~\eqref{eq:t-channel:sommerfeld-watson:deformed-integral}
is defined in the region $z<-1$. We know from Eq~\eqref{eq:t-channel:sommerfeld-watson:asymptotics-of-legendre1} $P_{\ell}(z)\sim\cosh(l\chi)$
where $\chi$ is a positive number related to $z$ in a simple way.

Then as $\ell$ moves away from the imaginary axis, $P_{\ell}(-z)$
oscillates. This means the integral in Eq~\eqref{eq:t-channel:sommerfeld-watson:deformed-integral}
converges and defines an analytic function $\amplitude(t,z)$ free of
singularities in the left-half of the $z$-plane.

\N{Check the region \texorpdfstring{$z>1$}{z > 1}}
In the region $z > 1$, we know from Eq~\eqref{eq:t-channel:sommerfeld-watson:asymptotics-of-legendre2}
that as $\Im(\ell)\to\infty$ we have $P_{\ell}(-z)\sim\exp(-\I\ell\pi)\exp(\ell\chi)$.
Then $\exp(-\I\ell\pi)$ grows as $\Im(\ell)\to\infty$ and this is
``cancelled'' by $\sin(\pi\ell)$ under this limit. Therefore,
convergence of the integral depends entirely on the behaviour of
$f_{\ell}$ as $\Im(\ell)\to\infty$. 

Generally speaking, the amplitude $\amplitude(t,z)$ may have
singularities on the right-half of the $z$-plane, which may only occur
at real positive $z>1$. Since there are no singularities at complex
values of $z$, taking $\Im(z)\neq0$ guarantees a convergent integral.

\N[-1]{Asymptotics}
Now we have established the amplitude in Eq~\eqref{eq:t-channel:sommerfeld-watson:deformed-integral}
is defined for all $z$, its asymptotics can be computed. We use the
asymptotic behaviour of Legendre functions:
\begin{equation}
P_{\ell}(-z)\sim z^{\ell}\exp(-\I\ell\pi)\quad\mbox{as}\quad z\to\infty.
\end{equation}
Gribov asserts the asymptotics of Eq~\eqref{eq:t-channel:sommerfeld-watson:deformed-integral}
is then governed by the rightmost singularity of the partial-wave
amplitude in the $\ell$-plane. If the rightmost singularity of
$f_{\ell}(t)$ is a simple pole at $\ell=\alpha(t)$, then
\begin{equation}
\amplitude(s,t)\to\pi\frac{2\alpha+1}{\sin(\pi\alpha)}P_{\alpha}(-z)\Res(f_{\alpha}(t))+\int_{\Re(\ell)<\Re(\alpha(t))}\mbox{(stuff)}\,\D\ell
\end{equation}
where $\Res(f_{\alpha}(t))$ is the residue of the partial-wave amplitude
at this pole, and the integral in the right-hand side is called the ``background
integral'' (we can see it is asymptotically subdominant with respect to
the first term).

Thus the $\ell$-plane singularities for $f_{\ell}(t)$ define the
asymptotics of the amplitude $\amplitude(s,t)$. That is to say, the
asymptotics is defined by some $\ell_{\text{eff}}$, which matches the
intuitive picture of Mandelstam.

\N{References}
See also chapter 2 section 2 and appendix A of Donnachie \textit{et al}.~\cite{Donnachie:2002en}
for another derivation of the Sommerfeld--Watson transform of the amplitudes.
White~\cite{White:2000zs} provides a historic review of $S$-matrix
theory, and in the process reviews Sommerfeld--Watson transform quite
well. Iurato~\cite{Iurato:2015} reviews this calculation with an eye
towards the dual resonance model. Section 3.1 of Caron-Huot and Sandor~\cite{Caron-Huot:2020nem}
reviews the Sommerfeld--Watson transform very graphically in $S$-matrix theory.

\section{Continuation of the Partial Wave Amplitudes to Complex \texorpdfstring{$\ell$}{l}}

\M Gribov studies the issue of analytically continuing the partial-wave
amplitudes $f_{n}(t)$ into the complex $\ell$-plane. This amounts to
interpreting Eq~\eqref{eq:t-channel:sommerfeld-watson:deformed-integral}
physically in nonrelativistic and relativistic settings, separately.

\subsection{Nonrelativistic Quantum Mechanics}

\M
The ``problem'' of analytic continuation is pretty easy in
nonrelativistic quantum mechanics since, for the analytical continuation
of the radial wave function $\Psi_{\ell}(r)$, it suffices to consider
the Schr\"{o}dinger equation for the quantity $\ell(\ell+1)$ to be
complex-valued and look for the solution which, at small $r$, satisfies
the condition
\begin{equation}
\Psi_{\ell}(r)\to cr^{\ell}\quad\mbox{as }r\to 0.
\end{equation}

\M
The partial-wave amplitudes
\begin{equation}
f_{\ell}(E) = \frac{1}{k}\sin(\delta_{\ell}(E))\exp(\I\delta_{\ell}(E))
\end{equation}
may be found by studying the asymptotics of $\Psi_{\ell}(r)$ at large
$r$
\begin{equation}
\Psi_{\ell}(r)\sim\frac{1}{r}\sin\left(kr - \frac{\ell\pi}{2}+\delta_{\ell}(E)\right)\quad\mbox{as}~r\to\infty,
\end{equation}
where $\delta_{\ell}(E)$ is the scattering phase shift.

The partial-wave amplitudes $f_{\ell}(E)$ defined in this way will
decrease at large $\ell$. From the physical picture, it is clear the
scattering at large impact parameters should be small, as long as we
restrict ourselves to potentials with finite interaction radius.

Therefore, for $f_{\ell}$ analytically continued in this way, the
representation in Eq~\eqref{eq:t-channel:sommerfeld-watson:deformed-integral}
will be valid.

\subsection{Relativistic Quantum Mechanics}

\N{Remark}
Generally speaking, there exists no method of constructing an analytic
function $f_{\ell}$ that coincides with an arbitrary set of numbers
$f_{n}$ at integer values $\ell=n$. In mathematics, this problem is
solved using an infinite series which is not convergent for all sets of
$f_{n}$.

\N{Uniqueness: Carlson's Theorem}
We can consider the uniqueness question for constructing an analytic
function $f_{\ell}$ out of its values $f_{n}$ at integer points. There
is an obscure result called Carlson's theorem\footnote{It seems this
part of Regge theory is responsible for keeping Carlson's theorem alive,
I haven't heard of it outside the Regge theory literature.} which states
if we have found a function which is analytic at $\Re(\ell)>\ell_{0}$
and which grows in any direction in the right half-plane slower than
$\exp(\pm\I\ell\pi)$, then this function is unique.

If we have two such functions $f_{1}(\ell)$ and $f_{2}(\ell)$, then the
function
\begin{equation}
\phi(\ell) :=\frac{f_{1}(\ell)-f_{2}(\ell)}{\sin(\pi\ell)}
\end{equation}
would be analytic and exponentially falling along any ray in the
right-half plane. ``It is next to obvious that such a function is
identically equal to zero.''

\M
In the relativistic theory, a function $f_{\ell}(E)$ which satisfies all
the above mentioned properties cannot exist. Otherwise $\amplitude(s,t)$
defined by Eq~\eqref{eq:t-channel:sommerfeld-watson:deformed-integral}
would have no singularities for $z<-1$, and this contradicts the fact
that $\amplitude(s,t)$ does have a singularity at
\begin{equation}
u = -2p^{2}(1+z) > 4\mu^{2}.
\end{equation}

\section{Gribov--Froissart Projection}

\M
We start by writing down the formula for the partial-wave amplitudes
\begin{equation}\label{eq:t-channel:gribov-froissart:defn-partial-wave-amplitude}
f_{n}(t) = \frac{1}{2}\int^{1}_{-1}P_{n}(z)\amplitude(t,z)\,\D z.
\end{equation}

\M
We rewrite the partial-wave amplitudes by using the Legendre function of
the second kind
\begin{equation}
Q_{n}(z) = \frac{1}{2}\int^{1}_{-1}\frac{P_{n}(z')\,\D z'}{z-z'}.
\end{equation}
At large $z$, this function decreases like
\begin{equation}
Q_{n}(z)\sim\frac{c}{z^{n+1}}\quad\mbox{as }|z|\to\infty,
\end{equation}
where for $-1<z<1$ it is complex valued and satisfies
\begin{equation}
Q_{n}(z+\I\varepsilon)-Q_{n}(z-\I\varepsilon)=-\I\pi P_{n}(z).
\end{equation}
Using this we rewrite the partial-wave amplitude Eq~\eqref{eq:t-channel:gribov-froissart:defn-partial-wave-amplitude}
as
\begin{equation}
f_{n}(t) = \frac{1}{2\pi\I}\oint_{\gamma} Q_{n}(z)\amplitude(t,z)\,\D z.
\end{equation}
The contour is doodled below:
\begin{center}
  \includegraphics{img/ch2sec5.0}
\end{center}

\M
The thresholds are at $s=4\mu^{2}$ and $u=4\mu^{2}$, so that
\begin{equation}
z_{1} = z_{2} = 1 + \frac{4\mu^{2}}{2k_{t}^{2}} > 1.
\end{equation}
Note, Gribov Eq (2.5a) states:
\begin{equation}
k_{t} := \frac{\sqrt{t - 4\mu^{2}}}{2},\quad s = -2k_{t}^{2}(1 - z_{t}).
\end{equation}
The discontinuities across these cuts are, respectively, $2\I\amplitude_{1}(z,t)=D_{t}(s,t(s,z_{t}))$
and $2\I\amplitude_{2}(z,t) = D_{u}(s,t(s,z_{t}))$.

We will be working with the cosine of scattering angle,
\begin{equation}
z_{t} = \cos(\theta_{t}) = 1 + \frac{2s}{t - 4\mu^{2}} = \frac{u-s}{u+s}.
\end{equation}
Observe
\begin{equation}
2s = (t - 4\mu^{2})(z_{t}-1)\implies s\propto z_{t}-1,
\end{equation}
so
\begin{equation}
s\ln(s) \propto (z_{t}-1)\ln(z_{t}-1).
\end{equation}
For large $z_{t}\gg1$, $s[\ln(s)]^{2}\sim z_{t}[\ln(z_{t})]^{2}$.

\M
In the $t$-channel partial-wave expansion,
we can rewrite the partial-wave amplitudes for large enough $n$ as
\begin{equation}
f_{n} = \frac{1}{\I2\pi}\int^{\infty}_{z_{1}}Q_{n}(z_{t}')D_{t}(s,t(s,z_{t}'))\,\D z_{t}'
+ \frac{1}{\I2\pi}\int^{-\infty}_{-z_{2}}Q_{n}(z_{t}')D_{u}(s,u(s,z_{t}'))\,\D z_{t}'.
\end{equation}
%% \textbf{NOTE:} I think I may be muddled here, I think I should be
%% working with $D_{s}$ and $D_{u}$, which means I would be integrating
%% over $z_{s}'$, wouldn't I? See Chapter 2 of
%% Collins~\cite{Collins:1977jy}, especially Eq~(2.3.4).

%% \textsc{Answer:} No, this is in the $t$-channel, Collins works in the
%% $s$-channel. 

\M
Using the relation
\begin{equation}
Q_{n}(-z)=(-1)^{n}Q_{n}(z),
\end{equation}
we get
\begin{equation}
f_{n} = \frac{1}{\I2\pi}\int^{\infty}_{z_{1}}Q_{n}(z_{t}')D_{t}(s,t(s,z_{t}'))\,\D z_{t}'
+ \frac{(-1)^{n}}{\I2\pi}\int^{\infty}_{z_{2}}Q_{n}(z_{t}')D_{u}(s,u(s,-z_{t}'))\,\D z_{t}'.
\end{equation}
The factor of $(-1)^{n}$ causes problems when we want to study the
asymptotic behaviour of $f_{n}$, or if we want to analytically continue
this to complex angular momenta.

\begin{theorem}
We introduce two analytic functions $f^{+}_{\ell}$ and
$f^{-}_{\ell}$ such that
\begin{equation}
f^{+}_{\ell}|_{\ell=n=2k}=f_{n},\quad\mbox{and}\quad
f^{-}_{\ell}|_{\ell=n=2k+1}=f_{n}.
\end{equation}
Then we have the expansion become
\begin{equation}\label{eq:t-channel:froissart-gribov-transformed}
f^{\pm}_{n} = \frac{1}{\I2\pi}\int^{\infty}_{z_{1}}Q_{n}(z_{t}')D_{t}(s,t(s,z_{t}'))\,\D z_{t}'
\pm \frac{1}{\I2\pi}\int^{\infty}_{z_{2}}Q_{n}(z_{t}')D_{u}(s,u(s,-z_{t}'))\,\D z_{t}'.
\end{equation}
\end{theorem}

\begin{corollary}
We can represent the scattering amplitudes in odd and even parts in $z$,
$\amplitude = \amplitude^{+} + \amplitude^{-}$ where
\begin{subequations}\label{eq:expansion-of-even-and-odd-parts-of-amplitude}
\begin{equation}
\amplitude^{+} = \sum_{n=2r}P_{n}(z)(2n+1)f_{n}^{+}
\end{equation}
and
\begin{equation}
\amplitude^{-} = \sum_{n=2r+1}P_{n}(z)(2n+1)f_{n}^{-}.
\end{equation}
\end{subequations}
\end{corollary}

\M
Note Theorem~\ref{thm:froissart-bound}, the Froissart bound combined
with the optical theorem to tell us
\begin{subequations}\label{eq:froissart-plus-optical-forces-upper-bound}
\begin{equation}
D_{u}(u,t) < \mbox{(const.)}s[\ln(s)]^{2}
\end{equation}
and
\begin{equation}
D_{t}(s,t) < \mbox{(const.)}s[\ln(s)]^{2}
\end{equation}
\end{subequations}
Since $z_{t}\propto s$, this means to obtain Eq~\eqref{eq:t-channel:froissart-gribov-transformed},
we must have $|D_{s}(z_{t},t)|<z^{\ell_{0}}$ as $z\to\infty$ to ensure
convergence, and so this works only for $n>n_{0}\geq\ell_{0}$. 
But Eq~\eqref{eq:froissart-plus-optical-forces-upper-bound} tells us
that $n_{0}<2$.

This forces us to add a few extra terms from the series in Eq~\eqref{eq:expansion-of-even-and-odd-parts-of-amplitude} to the 
Eq~\eqref{eq:t-channel:froissart-gribov-transformed},
permitting us to deform the contour of integration to neglect the
integral over the great circle of infinite radius in the $z$ plane,
giving us:
\begin{subequations}\label{eq:expansion-of-even-and-odd-parts-of-amplitude2}
\begin{equation}
\amplitude^{+} = \sum^{n_{0}}_{n=2r}P_{n}(z)(2n+1)f_{n}^{+} + \frac{\I}{4}\int^{\ell_{0}+\I\infty}_{\ell_{0}-\I\infty}\frac{(2\ell+1)}{\sin(\pi\ell)}f^{+}_{\ell}[P_{\ell}(-z)+P_{\ell}(z)]\,\D\ell
\end{equation}
and
\begin{equation}
\amplitude^{-} = \sum^{n_{0}}_{n=2r+1}P_{n}(z)(2n+1)f_{n}^{-} + \frac{\I}{4}\int^{\ell_{0}+\I\infty}_{\ell_{0}-\I\infty}\frac{(2\ell+1)}{\sin(\pi\ell)}f^{-}_{\ell}[P_{\ell}(-z)-P_{\ell}(z)]\,\D\ell.
\end{equation}
\end{subequations}
These sums contribute at most one term.

\N{References}
I have found \S2.3 of \arXiv{2006.08221} useful, and section 1.6 [for
the $s$-channel version] as well
as appendix A [for the $t$-channel version] in Donnachie \textit{et al.}~\cite{Donnachie:2002en}
useful supplements. The history surrounding the development of this may
be found in Dokshitzer (\arXiv{hep-ph/0510200}).
Collins~\cite{Collins:1977jy} has been very useful for double checking
calculations. 