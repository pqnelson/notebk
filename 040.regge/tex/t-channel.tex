\chapter{Physics of the \texorpdfstring{$t$}{t}-Channel and Complex Angular Momenta}

\section{Analytical Continuation of the \texorpdfstring{$t$}{t}-Channel Unitarity Condition}
\section{Complex Angular Momenta}
\section{Partial Wave Expansion and Sommerfeld--Watson Representation}
\section{Continuation of the Partial Wave Amplitudes to Complex \texorpdfstring{$\ell$}{l}}
\section{Gribov--Froissart Projection}

\M
The thresholds are at $s=4\mu^{2}$ and $u=4\mu^{2}$, so that
\begin{equation}
z_{1} = z_{2} = 1 + \frac{4\mu^{2}}{2k_{t}^{2}} > 1.
\end{equation}
The discontinuities across these cuts are, respectively, $2\I\amplitude_{1}(z,t)=D_{t}(s,t(s,z_{t}))$
and $2\I\amplitude_{2}(z,t) = =D_{u}(s,t(s,z_{t}))$.

\M
In the $t$-channel partial-wave expansion,
we can rewrite the partial-wave amplitudes for large enough $n$ as
\begin{equation}
f_{n} = \frac{1}{\I2\pi}\int^{\infty}_{z_{1}}Q_{n}(z_{t}')D_{t}(s,t(s,z_{t}'))\,\D z_{t}'
+ \frac{1}{\I2\pi}\int^{-\infty}_{-z_{2}}Q_{n}(z_{t}')D_{u}(s,u(s,z_{t}'))\,\D z_{t}'.
\end{equation}

\M
Using the relation
\begin{equation}
Q_{n}(-z)=(-1)^{n}Q_{n}(z),
\end{equation}
we get
\begin{equation}
f_{n} = \frac{1}{\I2\pi}\int^{\infty}_{z_{1}}Q_{n}(z_{t}')D_{t}(s,t(s,z_{t}'))\,\D z_{t}'
+ \frac{(-1)^{n}}{\I2\pi}\int^{\infty}_{z_{2}}Q_{n}(z_{t}')D_{u}(s,u(s,-z_{t}'))\,\D z_{t}'.
\end{equation}
The factor of $(-1)^{n}$ causes problems when we want to study the
asymptotic behaviour of $f_{n}$, or if we want to analytically continue
this to complex angular momenta.

\begin{theorem}
We introduce two analytic functions $f^{+}_{\ell}$ and
$f^{-}_{\ell}$ such that
\begin{equation}
f^{+}_{\ell}|_{\ell=n=2k}=f_{n},\quad\mbox{and}\quad
f^{-}_{\ell}|_{\ell=n=2k+1}=f_{n}.
\end{equation}
Then we have the expansion become
\begin{equation}\label{eq:t-channel:froissart-gribov-transformed}
f^{\pm}_{n} = \frac{1}{\I2\pi}\int^{\infty}_{z_{1}}Q_{n}(z_{t}')D_{t}(s,t(s,z_{t}'))\,\D z_{t}'
\pm \frac{1}{\I2\pi}\int^{\infty}_{z_{2}}Q_{n}(z_{t}')D_{u}(s,u(s,-z_{t}'))\,\D z_{t}'.
\end{equation}
\end{theorem}

\begin{corollary}
We can represent the scattering amplitudes in odd and even parts in $z$,
$\amplitude = \amplitude^{+} + \amplitude^{-}$ where
\begin{subequations}\label{eq:expansion-of-even-and-odd-parts-of-amplitude}
\begin{equation}
\amplitude^{+} = \sum_{n=2r}P_{n}(z)(2n+1)f_{n}^{+}
\end{equation}
and
\begin{equation}
\amplitude^{-} = \sum_{n=2r+1}P_{n}(z)(2n+1)f_{n}^{-}.
\end{equation}
\end{subequations}
\end{corollary}

\M
Note Theorem~\ref{thm:froissart-bound}, the Froissart bound combined
with the optical theorem to tell us
\begin{subequations}\label{eq:froissart-plus-optical-forces-upper-bound}
\begin{equation}
D_{u}(u,t) < \mbox{(const.)}s[\ln(s)]^{2}
\end{equation}
and
\begin{equation}
D_{s}(s,t) < \mbox{(const.)}s[\ln(s)]^{2}
\end{equation}
\end{subequations}
This means to obtain Eq~\eqref{eq:t-channel:froissart-gribov-transformed},
we must have $|D_{s}(z,t)|<z^{\ell_{0}}$ as $z\to\infty$ to ensure
convergence, and so this works only for $n>n_{0}\geq\ell_{0}$. 
But Eq~\eqref{eq:froissart-plus-optical-forces-upper-bound} tells us
that $n_{0}<2$.

This forces us to add a few extra terms from the series in Eq~\eqref{eq:expansion-of-even-and-odd-parts-of-amplitude} to the 
Eq~\eqref{eq:t-channel:froissart-gribov-transformed},
permitting us to deform the contour of integration to neglect the
integral over the great circle of infinite radius in the $z$ plane,
giving us:
\begin{subequations}\label{eq:expansion-of-even-and-odd-parts-of-amplitude2}
\begin{equation}
\amplitude^{+} = \sum^{n_{0}}_{n=2r}P_{n}(z)(2n+1)f_{n}^{+} + \frac{\I}{4}\int^{\ell_{0}+\I\infty}_{\ell_{0}-\I\infty}\frac{(2\ell+1)}{\sin(\pi\ell)}f^{+}_{\ell}[P_{\ell}(-z)+P_{\ell}(z)]\,\D\ell
\end{equation}
and
\begin{equation}
\amplitude^{-} = \sum^{n_{0}}_{n=2r+1}P_{n}(z)(2n+1)f_{n}^{-} + \frac{\I}{4}\int^{\ell_{0}+\I\infty}_{\ell_{0}-\I\infty}\frac{(2\ell+1)}{\sin(\pi\ell)}f^{-}_{\ell}[P_{\ell}(-z)-P_{\ell}(z)]\,\D\ell.
\end{equation}
\end{subequations}
These sums contribute at most one term.

\N{References}
I have found \S2.3 of \arXiv{2006.08221} useful, and section 1.6 [for
the $s$-channel version] as well
as appendix A [for the $t$-channel version] in Donnachie \textit{et al.}~\cite{Donnachie:2002en}
useful supplements. The history surrounding the development of this may
be found in Dokshitzer (\arXiv{hep-ph/0510200}).
Collins~\cite{Collins:1977jy} has been very useful for double checking
calculations. 