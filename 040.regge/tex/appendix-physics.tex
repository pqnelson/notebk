\chapter{Physics Results}

\M We assume the following results from physics.
References will be given for further perusal.


\section{High Energy Physics}

\begin{theorem}[Optical theorem]
  For $1+2\to1+2$ scattering, when $m_{3}=m_{1}$ and $m_{4}=m_{2}$,
  the forward scattering direction
  $\theta_{s}=0$ has elastic scattering amplitude $\amplitude(s,t)$
  related to the total cross section by
  \begin{equation}
\sigma_{\text{tot}} = \frac{1}{2|\vec{p}_{1}|\sqrt{s}}\Im\amplitude(s,t=0).
  \end{equation}
  Here $\vec{p}_{1}$ is the magnitude of the initial center-of-mass
  3-momentum.
\end{theorem}

Following the argument and presentation of Donnachie \textit{et al.}~\cite{Donnachie:2002en},
section 1.3.

\begin{proof}[Proof sketch]
  This follows from unitarity
  \begin{equation}
    \delta_{ji} = \langle j\vert SS^{\dagger}\vert i\rangle
    = \sum_{f}\langle j\vert S\vert f\rangle\langle f\vert S^{\dagger}\vert i\rangle
  \end{equation}
  Using the definition of the $T$-matrix, this is
  \begin{equation}
\langle j\vert T\vert i\rangle - \langle j\vert T^{\dagger}\vert i\rangle
= (2\pi)^{4}\I\sum_{f}\delta^{4}(P^{f}-P^{i})\langle j\vert T\vert f\rangle\langle f\vert T^{\dagger}\vert i\rangle.
  \end{equation}
  Setting $j=i$ gives us
  \begin{equation}
2\Im\langle i\vert T\vert i\rangle
=\sum_{f}(2\pi)^{4}\delta^{4}(P^{f}-P^{i})|\langle f\vert T\vert i\rangle|^{2}.
  \end{equation}
  We see the right-hand side is the total cross sectional area for
  $2\to2$ scattering,
  \begin{equation}
\sigma_{\text{tot}} = \frac{1}{2|\vec{p}_{1}|\sqrt{s}}\Im\langle i\vert T\vert i\rangle.
  \end{equation}
  We see this is precisely the desired result.
\end{proof}

\begin{remark}
The ``forward direction'' in the $s$-channel scattering would be
precisely when $\theta_{s}=0$ (or equivalently, its cosine $z_{s}=1$).
\end{remark}

\subsection{Propagators}

\N{Spin-0 Propagator}
The spin-0 propagator looks like
\begin{equation}
\propagator^{(0)}(p) = \frac{1}{p^{2} - m^{2} + \I\varepsilon}.
\end{equation}


\N[-1]{Spin-$1/2$ Propagator}
The massive spin-$1/2$ propagator in QED looks like:
\begin{equation}
\propagator^{(1/2)}(p) = \frac{(\gamma^{\mu}p_{\mu}+m)}{p^{2}-m^{2}+\I\varepsilon},
\end{equation}
where $\gamma^{\mu}$ are the Dirac gamma matrices. The massless
spin-$1/2$ propagators may be obtained taking the $m\to0$ limit.

\N{Spin-1 Propagator}
The general form of the spin-1 propagator for a massive particle
requires a choice of gauge. The most general expression for the spin-1
propagator is parametrized by $\lambda$ (the gauge parameter, usually
$\lambda=0$ or $\lambda=\infty$):
\begin{equation}
\propagator_{\mu\nu}(p) = -\I\frac{g_{\mu\nu} + \left(1 - \frac{1}{\lambda}\right)p_{\mu}p_{\nu}/p^{2}}{p^{2}+\I\varepsilon}.
\end{equation}
The particular case for a massive spin-1 particle:
\begin{equation}
\propagator^{\text{(massive)}}_{\mu\nu}(p) = \frac{g_{\mu\nu} - k_{\mu}k_{\nu}/m^{2}}{k^{2}-m^{2}+\I\varepsilon}
+\frac{g_{\mu\nu} - k_{\mu}k_{\nu}/m^{2}}{k^{2}-(m^{2}/\lambda)+\I\varepsilon}.
\end{equation}
%% A few examples of the various different gauge choices:
%% \begin{itemize}
%% \item Unitary Gauge $
%% The Landau gauge

\N{Graviton Propagator}
The spin-2 propagator is a nightmare, but simplifies in some settings.

\N[1]{Graviton Propagator in Minkowski Spacetime}
Veltman~\cite[see Eq~(11)]{Veltman:1975vx} computes for us,
in $D$-dimensional Minkowski spacetime with background metric
$\eta_{\mu\nu}$, the spin-2 (``graviton'') propagator in harmonic
gauge\footnote{Claus Kiefer's \emph{Quantum Gravity}, third ed., pg 53
discusses this a bit, too.}:
\begin{subequations}
\begin{align}
G_{\alpha\beta~\mu\nu} &= \frac{\mathcal{P}^{2}_{\alpha\beta~\mu\nu}}{k^{2}} - \frac{\mathcal{P}^{0}_{s}{}_{\alpha\beta~\mu\nu}}{2k^{2}}\\
&= \frac{\eta_{\alpha\mu}\eta_{\beta\nu}+\eta_{\beta\mu}\eta_{\alpha\nu}-\frac{2}{D-2}\eta_{\mu\nu}\eta_{\alpha\beta}}{k^{2}}
\end{align}
\end{subequations}
where $\mathcal{P}^{2}$ is the transverse and traceless spin-2
projection operator, $\mathcal{P}^{0}$ is the spin-0 scalar multiplet.
Donoghue~\cite[see Eq (39)]{Donoghue:1995cz}, also working with the harmonic gauge, confirms this calculation.

For the graviton in the Prenkti gauge, see 't Hooft and
Veltman~\cite[see Eq~(2.9)]{tHooft:1974toh}. The main difference is in
the expression for scalar multiplet.

%% \N{Graviton Propagator in Anti-de Sitter Spacetime}
%% D'Hoker and friends~\cite[see \S5 \emph{et seq}.]{DHoker:1999bve} have
%% computed the graviton propagator in $\mathrm{AdS}(d+1)$ spacetime as

\N[-1]{Propagator for Spin-$J$ Particles}
The canonical reference\footnote{A
particularly enlightening reference is \url{https://physics.stackexchange.com/a/15164/9290}} for this result is
Weinberg~\cite{Weinberg:1964cn,Weinberg:1964ev}. Specifically, table II
of Weinberg~\cite{Weinberg:1964cn} establishes the propagator for a
massive particle of spin $J$ and 4-momentum $p_{\mu}$ is
\begin{subequations}
\begin{equation}
\propagator_{\text{massive}}(p) =\frac{1}{\I m^{2J}}\frac{\mathcal{P}(p) + m^{2J}}{p^{2} - m^{2}-\I\varepsilon}
\end{equation}
where $\mathcal{P}(p)\sim(p^{2})^{J}$ plus lower-order terms. (Weinberg
uses $S(q)$ to denote his propagators.) Massless
particles of spin $J$ look like (according to \S8 of
Weinberg~\cite{Weinberg:1964ev})
\begin{equation}
\propagator_{\text{massless}}(p) =\frac{1}{\I}\frac{\mathcal{P}(p)}{p^{2}-\I\varepsilon}.
\end{equation}
\end{subequations}
In either case, the asymptotic approximation holds:
\begin{equation}
\propagator(p)\sim\frac{(p^{2})^{J}}{p^{2} - m^{2} - \I\varepsilon}.
\end{equation}
Note: we use the opposite metric signature than Weinberg, hence our
denominators look like $p^{2}-m^{2}$ whereas Weinberg's published
formulas look like $p^{2}+m^{2}$.


\subsection{Scattering Amplitudes}


\N{Yukawa interaction}
If we suppose there is a Yukawa interaction between a spin-$1/2$ field
$\psi$ and its antipartner $\bar{\psi}$ governed by
$V= g\bar{\psi}\phi\psi$, then the nonrelativistic approximation of the tree-level amplitude for the Yukawa-mediated scattering $\psi+\bar{\psi}\to\psi+\bar{\psi}$
is (in center-of-mass frame):
\begin{equation}
\amplitude(k) \sim g^{2}\frac{4\pi}{k^{2}+m^{2}}.
\end{equation}

\N{Graviton Scattering}
When all four particles are gravitons, we have $s+t+u=0$ (since
gravitons are massless) and the amplitude is
\begin{equation}
\amplitude_{\text{$s$-channel}}\sim\frac{\kappa^{2}}{4}\frac{s^{4}}{stu},
\end{equation}
where $\kappa^{2}=32\pi G_{N}$ is just Newton's constant times some
numerical factor.
For the $t$-channel, the only difference is the numerator: replace the
$s^{4}$ by $t^{4}$. Similarly, the $u$-channel replaces $s^{4}$ by $u^{4}$.

See \S2 of Diego Blas, Jorge Martin Camalich, Jose Antonio Oller, ``Unitarization of infinite-range forces: graviton-graviton scattering'' (\arXiv{2010.12459})
who, in turn, cite the calculation done by Grisaru, van Nieuwenhuizen, and Wu~\cite{Grisaru:1975bx}
in effective field theory.

Here we would have Eq~\eqref{eq:introduction:z-of-s} give us, with $\mu=0$,
\begin{subequations}
\begin{equation}
\cos(\theta_{s}) = \frac{s+2t}{s} = \frac{t-u}{s} = z_{s}
\end{equation}
In particular, $sz_{s} = t-u$. If we note the mass-shell condition for
the graviton Mandelstam variables $s+t+u=0$ implies $s+t=-u$, then
$sz_{s} = s+2t$. Rearranging terms gives us
\begin{equation}
t = \frac{z_{s}-1}{2}s,
\end{equation}
$-u = s+t = s + (z_{s}-1)s/2 = (z_{s} + 1)s/2 = (4z_{s}+1)s/2$. That is,
\begin{equation}
t = \frac{z_{s}-1}{2}s,\quad\mbox{and}\quad u = \frac{-(z_{s}+1)}{2}s.
\end{equation}
\end{subequations}
Hence the
amplitude would look like (suppressing the constant coefficient):
\begin{equation}
  \begin{split}
\amplitude(s,z_{s})\sim\frac{s^{4}}{s[(z_{s}-1)s/2][-(z_{s} + 1)s/2]} &= \frac{-4s^{4}}{(z_{s}^{2}-1)s^{3}}
=\frac{4s}{1-z_{s}^{2}}\\
&=\left(\frac{1}{1-z_{s}}+\frac{1}{1+z_{s}}\right)2s.
  \end{split}
  \end{equation}
The $t$-channel amplitude enjoys similar manipulations, $\amplitude_{t\text{-channel}}\sim4t/(1-z_{t}^{2})$.
We observe this does not look like
$f_{\ell}P_{\ell}(z_{s})\delta_{\ell,n_{0}}$, which would corroborate
Gribov's claims.

\M
In the $t$-channel, we would have
\begin{subequations}
\begin{equation}
z_{t} = 1 + \frac{2s}{t} = \frac{(t+s)+s}{t}=\frac{-u+s}{t}=\frac{s-u}{t}.
\end{equation}
We also find $tz_{t} = t+2s$ simplifies to
\begin{equation}
\frac{(z_{t}-1)}{2}t=s.
\end{equation}
The remaining Mandelstam variable for graviton scattering in the
$t$-channel would be $u=-s-t$, or,
\begin{equation}
u = \frac{1-z_{t}}{2}t - t = \frac{-(1+z_{t})}{2}t.
\end{equation}
Hence we work out the amplitude
\begin{equation}
\amplitude(t,z_{t})\sim\frac{t^{4}}{stu} =
\frac{t^{4}}{-(1+z_{t})(z_{t}-1)t^{3}/4} = \frac{4t}{1-z_{t}^{2}}.
\end{equation}
\end{subequations}
This is consistent with our previous result, which is good.
The problem is, the partial wave expansion would have coefficients like:
\begin{equation}
f_{\ell} = \int^{1}_{-1}\amplitude(t,z_{t}) P_{\ell}(z_{t})\,\D z_{t}\sim\int^{1}_{-1}\frac{P_{\ell}(z)}{1-z^{2}}\D z.
\end{equation}
This diverges, it has a logarithmic divergence.

\N{Scattering Amplitude for Exchange of Spin $J$ Particle}
It is a fact the scattering amplitude for a spin $J$ particle at high
energies is
\begin{equation}
\amplitude_{J}(s,t) = -\frac{g^{2}(-s)^{J}}{t-M^{2}}.
\end{equation}
This may be found in Williams \emph{Introduction to Elementary Particles},
More precisely, we expect
\begin{equation}
\amplitude_{J}(s,t) = -\frac{g^{2}(p^{2})^{J}}{p^{2}-M^{2}}.
\end{equation}
The pole is determined by unitarity.
