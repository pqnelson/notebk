

\section{Froissart's Theorem}

\begin{lemma}[Asymptotics of Legendre Polynomials]
We have as $\ell\to\infty$,
\begin{subequations}
  \begin{align}
  P_\ell (\cos \theta) &= \sqrt{\frac{\theta}{\sin \theta}} \, J_0((\ell+1/2)\theta) + \mathcal{O}\left(\ell^{-1}\right) \\
&= \frac{2}{\sqrt{2\pi \ell\sin\theta}}\cos\left(\left(\ell + \tfrac12\right)\theta - \frac{\pi}{4}\right) + \mathcal{O}\left(\ell^{-3/2}\right), \quad \theta \in (0,\pi),
\end{align}
\end{subequations}
where $J_{0}$ is the Bessel function of the first-kind.
\end{lemma}

For a proof, see:
G\'abor Szeg\~o~\cite{szego},
\emph{Orthogonal polynomials}, (4th ed.) AMS Publishers, pp.194
\emph{et seq.} (see especially, Theorem 8.21.2).

\begin{theorem}[Froissart]
  Asymptotically as $s\to\infty$, we have
  \begin{equation}
\left.\Im\amplitude(s,t)\right|_{t=0}\leq(\mbox{const})\cdot s[\ln(s/s_{0})]^{2}.
  \end{equation}
\end{theorem}

This is due to the analyticity and unitarity properties. The proof is a
multi-step procedure, but each step is fairly straightforward.

\N[1]{Proof Step 1: estimate $f_{\ell}$ at large $s$}
We estimage $f_{\ell}$ at large $s$ using the fact that the singularity
of $\Im_{s}(\amplitude(s,t))=D_{s}(s,t,u)$ closest to the physical
region of the $s$-channel is located at $t=4\mu^{2}$.

Now, at large $\ell$, the partial wave amplitude falls exponentially.
Gribov argues this follows from $k_{s}^{2}\propto s\gg t$, Eq~\eqref{eq:introduction:asymptotic-t-behaviour-karplus-curve-example}
gives us $\alpha\simeq\sqrt{t}/k_{s}$, which gives us 
\begin{equation}\label{eq:intro:froissart-thm:step-1}
f_{\ell}(s)\simeq c(s,\ell)\exp\left(\frac{-\ell}{k_{s}}\sqrt{4\mu^{2}}\right).
\end{equation}
where $c(s,\ell)$ is slowly [non-exponentially] varying with $\ell$.

\M[1]
Just to clarify, this is because $\cosh(\alpha) \approx 1 + \alpha^{2}/2! + \alpha^{4}/4! + \dots$,
and we would have small $\alpha$ giving us
\begin{equation}
1 + \frac{\alpha^{2}}{2} = 1 + \frac{4\mu^{2}}{2k_{s}^{2}},
\end{equation}
which gives the relation
\begin{equation}
\alpha^{2} = \frac{4\mu^{2}}{k_{s}^{2}}\implies\alpha=\pm\frac{\sqrt{4\mu^{2}}}{k_{s}}.
\end{equation}
Since $k_{s}\gg\sqrt{t}$, the corrections to this approximation are
negligible, of order $(\sqrt{t}/k_{s})^{3}$.

\M
Also note that
\begin{equation}
z_{s} = 1 + \frac{4\mu^{2}}{2k_{s}^{2}} \approx 1 + \frac{t}{2k_{s}^{2}}.
\end{equation}

\N[-1]{Step 2.}
Assume for $t$ arbitrarily close to $4\mu^{2}$, the amplitude grows with
$s$ not faster than some power.

Then the same is valid for $\Im c(s,\ell)$.

Arguably, $\Im f_{\ell}$ is positive due to the unitarity condition, and
so is $P_{\ell}(1 + t/2k_{s}^{2})$ for $t\geq 0$.

Therefore each partial wave we have an estimate,
\begin{subequations}
  \begin{align}
    \left(\frac{s}{s_{0}}\right)^{N}
    &>\Im\amplitude(s,t) =
    \sum^{\infty}_{\ell=0}\Im(f_{\ell}(s))(2\ell+1)P_{\ell}\left(1+\frac{t}{2k_{s}^{2}}\right)\\
    \intertext{then using Eq~\eqref{eq:intro:froissart-thm:step-1},
      observing $\sinh\alpha\approx\alpha$, using
      the asymptotic approximation
      $P_{\ell}(\cosh\alpha)\sim\sqrt{\alpha/\sinh\alpha}I_{0}((\ell + 1/2)\alpha)$,
      and Hankel's asymptotic approximation of the Bessel function ${I_{0}(z)\sim\E^{z}/\sqrt{2\pi z}}$,
      we rewrite the right-hand side as holding for each term (since
      each term is a product of positive factors):}
    \left(\frac{s}{s_{0}}\right)^{N}
    &>\Im c(s,\ell)\left(2\pi\ell\frac{\sqrt{t}}{k_{s}}\right)^{-1/2}\exp\left[\frac{\ell}{k_{s}}(\sqrt{t}-\sqrt{4\mu^{2}})\right].
  \end{align}
\end{subequations}
Since this holds for arbitrary positive $t<4\mu^{2}$, we conclude
\begin{equation}
\Im c(s,\ell) < (s/s_{0})^{N},
\end{equation}
and, modulo an ``irrelevant pre-exponential factor'', that
\begin{equation}\label{eq:froissart-thm:imaginary-part-of-partial-wave}
\Im(f_{\ell}(s))\leq\left(\frac{s}{s_{0}}\right)^{N}\exp(-2\mu\ell/k).
\end{equation}

\N{Step 3: Estimate Imaginary Part of Forward Scattering Amplitude}
We now can estimate
\begin{equation}
\Im\amplitude(s,t=0)\leq8\pi\sum^{L}_{\ell=0}(2\ell+1) + \sum^{\infty}_{\ell=L+1}\Im(f_{\ell}(s))(2\ell+1),
\end{equation}
where when $\ell<L$ in which the partial waves are large,
$\Im(f_{\ell})\simeq|f_{\ell}|=\bigOh(1)$, and we can estimate the first
term as
\begin{equation}
\sum^{L}_{\ell=0}(2\ell+1)\simeq L^{2}.
\end{equation}

\N[1]{Determining $L$}
We can determine $L$ as the bound above which partial wave amplitudes
become small
\begin{equation}
\Im(f_{\ell>L})\ll 1,
\end{equation}
and fall exponentially with $\ell$. According to
$\Im(c(s,\ell))<(s/s_{0})^{N}$ from step 2, this means
\begin{equation}
(s/s_{0})^{N}\exp(-2\mu L/k_{s})\simeq1\implies \boxed{L\simeq\frac{k_{s}}{2\mu}\ln(s/s_{0}).}
\end{equation}

\N[-1]{Step 4: Estimate Infinite Sum in Tail}
We can now estimate the infinite sum using $f_{L+n}\sim f_{K}\exp(-2\mu n/k_{s})$,
and this turns out to be subdominant:
\begin{equation}
\sum^{\infty}_{n=0}2(L+n)\exp\left(\frac{-2\mu}{k_{s}}n\right)\simeq\frac{k_{s}}{\mu}L
+\frac{k_{s}^{2}}{2\mu^{2}},
\end{equation}
and as $s\to\infty$ we find
\begin{equation}
\lim_{s\to\infty}\frac{k_{s}}{\mu}L
+\frac{k_{s}^{2}}{2\mu^{2}}\ll L^{2}.
\end{equation}
That is to say, the infinite sum is dominated by the finite sum's
contribution, which is approximately $L^{2}$.

\N{Step 5: Combine Results}
We have then step 4 simplifies the sum from step 3, giving us that
$\Im\amplitude(s,t=0)\sim L^{2}$ (plus lower order terms)
and step 3 also gave us an estimate $L^{2}\propto s[\ln(s/s_{0})]^{2}$
(since $L^{2}\simeq (k_{s}/2\mu)^{2}[\ln(s/s_{0})]^{2}$ and
$k_{s}^{2}\simeq s$)
which then together implies Froissart's theorem.\qquad\qedsymbol

\adjustSection{-1}

\begin{corollary}
Analogous consideration, together with the unitarity condition, leads to
the similar inequality for the real part of the forward scattering
amplitude
\begin{equation}
|\Re(\amplitude(s,t=0))|<\mbox{\normalfont(const.)}s[\ln(s/s_{0})]^{2}.
\end{equation}
\end{corollary}

\begin{corollary}
Since, according to the optical theorem, $\Im\amplitude(s,t=0)=s\sigma_{\text{tot}}(s)$,
Froissart's theorem implies the total cross section cann grow with the
center-of-mass energy $\sqrt{s}$ faster than the squared logarithm of
$s$,
\begin{equation}
\sigma_{\text{tot}}(s)\leq\sigma_{0}[\ln(s/s_{0})]^{2},
\end{equation}
and the interaction radius cannot grow faster than the logarithm of $s$.
\end{corollary}

\begin{corollary}
Let $\rho$ be the impact parameter, $k_{s}\rho = \ell+1/2$.
Then using
$t\simeq-(k_{s}\theta)^{2}$ we have
\begin{equation}
\amplitude(s,t)\simeq k_{s}^{2}\int f(\rho,s)J_{0}(\rho\sqrt{-t})2\rho\,\D\rho.
\end{equation}
\end{corollary}

\begin{proof}
This is because the cross section cannot decrease with increasing
energy, which requires the amplitude $\amplitude(s,t=0)$ must grow.
Consequently, the number of partial waves contributing to the
partial-wave expansion sum must be large. This allows us to replace the
sum by an integral over $\ell$, using the approximation
\begin{equation}
P_{\ell}(\cos(\theta))\simeq J_{0}\left[(2\ell+1)\frac{\theta}{2}\right],\quad\ell\gg1,\quad\mbox{and}\quad\theta\ll1.
\end{equation}
We obtain
\begin{equation}
\amplitude(s,t)\simeq\int f_{\ell}(s)J_{0}\left[(2\ell+1)\frac{\theta}{2}\right](2\ell+1)\,\D\ell.
\end{equation}
When we replace $\ell$ by the impact parameter, and using $t\simeq-(k_{s}\theta)^{2}$,
we obtain the desired result.
\end{proof}

\M[1]
If the values of $\rho$ giving the dominant contribution to this
integral do not depend on $s$ (which is the case for the usual picture
of diffractive scattering off a finite-sized object, Gribov tells us),
then it is natural to expect the amplitude takes the factorized form
\begin{equation}
\amplitude(s,t)\simeq a(s)F(t).
\end{equation}
If we further assume the partial wave amplitudes dominant in
Eq~\eqref{eq:froissart-thm:imaginary-part-of-partial-wave}
approach constant values as $s\to\infty$, then $A(s,t)\sim sF(t)$ and
the total cross section tends to a constant.
