\chapter{Introduction}


\M
Regge theory starts with quantum mechanics, so we will review the basic
ideas underlying quantum theory. We want to compute the transition
amplitude for a scattering process, which acts like the ``squareroot of
probability''. The scattering amplitude $\amplitude$ is a function of
kinematical invariants (like $p_{i}^{2}$ or $p_{j}p_{k}$ or whatever).

\N{Scattering Channels}
The $s$-channel scattering describes particles 1 and 2 annihilate each
other, create a virtual particle, which then decays into the scattering
of particles 3 and 4.

The $t$-channel scattering describes the process where particle 1 emits
an intermediate particle (and particle 1 then becomes particle 3);
particle 2 absorbs the intermediate particle, thereby becoming particle
4.

The $u$-channel is the $t$-channel with the roles of particles 3 and 4
swapped.
\section{Basic Principles}

\M
The cross-section of any process $2\to n$ (where 2 particles scatter,
and produces a final state consisting of $n$ particles) is related to
the ``invariant amplitude'' $\amplitude$ as
\begin{equation}
  \begin{split}
\D\sigma_{n} &= (2\pi)^{4}\delta\left(p_{1}+p_{2}-\sum_{i}p_{i}'\right)|\amplitude|^{2}
\prod^{n}_{i=1}\frac{\D^{3}p'_{i}}{2p'_{i0}(2\pi)^{3}}\frac{1}{I}\\
&=\begin{pmatrix}\mbox{Energy-Momentum}\\
\mbox{Conservation}
\end{pmatrix}|\amplitude|^{2}\prod^{n}_{i=1}\begin{pmatrix}\mbox{phase space}\\
\mbox{volume}
\end{pmatrix}\frac{1}{I},
  \end{split}
  \end{equation}
where $I$ is the Moller factor,
\begin{equation}
I = 4p_{10}p_{20}J = 4\sqrt{(p_{1}p_{2})^{2} - m_{1}^{2}m_{2}^{2}},
\end{equation}
which combines the flux density $J$ of the initial particles and
$(2p_{10}\cdot2p_{20})^{-1}$ coming from thei wave functions.

I believe this comes from Fermi's golden rule. Doubtless, there are a
million different derivations of it.

\N[1]{``Invariant'' Amplitude?} This is the first time I've seen the
phase ``invariant amplitude'' used, in Gribov \S1.1.1; perhaps I just
haven't been paying attention.

\N[-1]{Analyticity and Causality}
We assume the scattering amplitude $\amplitude$ is an analytic function
of its arguments.

\M[1]
Gribov asserts this is ``a manifestation of the causality principle''.
Without analyticity, Gribov asserts scattered waves could appear at
their source before being emitted. I do not see how, not at present.
I am told this is somehow related to the Kramers--Kronig relations, but
again I need to sit down and work through this explicitly.

\N{Conjecture: Amplitude is Bounded}
We conjecture, as one of the kinematical invariants tends to infinity
[with the remaining invariants fixed],
the growth of scattering amplitudes is polynomially bounded,
\begin{equation}
|\amplitude(p_{1},\dots,p_{n})|<(p_{i}p_{j})^{N}.
\end{equation}
This is ``closely related to causality and the locality of the interaction''.

Gribov concedes, ``One needs it in order to write the dispersion representation
for the amplitudes (to be able to close the integration contour over an
infinitely large circle).''

\N[-1]{Singularities}
We assume all singularities of the amplitude on the physical sheet have
the meaning of reaction thresholds (in the sense that, they are
determined by the physical masses of the intermediate state particles).

\N[1]{Landau Singularities}\index{Landau Singularity}
In terms of Feynman diagrams, these are the Landau singularities. This
appears to be how Landau~\cite{Landau:1959fi} originally envisioned them.

\N[-1]{Crossing Symmetry}
Gribov takes a diversion to explain crossing symmetries of amplitudes.
The basic idea being, since the amplitude is a function of kinematical
invariants, we can change the sign of $p_{i0}$ (the time component of
the 4-momentum for particle $i$) without altering the amplitude. For
example, the same analytic function describes the reaction:
\begin{subequations}\label{eq:introduction:crossing-symmetry:assumptions}
\begin{equation}
a(p_{1}) + b(p_{2})\to c(p_{3}) + d(p_{4})
\end{equation}
for $p_{10}>0$, $p_{20}>0$, $p_{30}>0$, and $p_{40}>0$; as well as,
\begin{equation}
a(p_{1}) + \bar{c}(-p_{3})\to\bar{b}(-p_{2})  + d(p_{4})
\end{equation}
(where $\bar{c}$ is the antiparticle of $c$, $\bar{b}$ the antiparticle
for $b$) for $p_{10}>0$ and $p_{40}>0$, but $p_{20}<0$ and $p_{30}<0$; and
\begin{equation}
a(p_{1}) + \bar{d}(-p_{4}) \to\bar{b}(-p_{2}) + c(p_{3})
\end{equation}
\end{subequations}
for $p_{10}>0$ and $p_{30}>0$, but $p_{20}<0$ and $p_{40}<0$.

\M[1]
What's going on here is a little subtle. Basically, we're considering
the physical $s$-channel region ($s>0$, $t<0$, $u<0$) for the process
$a+b\to c+d$. The amplitude is then $\amplitude_{a+b\to c+d}(s,t,u)$.

We can analytically continue this to the $t>0$, $s<0$, $u<0$ region for
the amplitude to describe the $a+\bar{c}\to\bar{b}+d$ process. Crossing
symmetry then forces the relation
\begin{equation}
\amplitude_{a+b\to c+d}(s,t,u)=\amplitude_{a+\bar{c}\to\bar{b}+d}(t,s,u).
\end{equation}
Similarly, we could analytically continue to the region $u>0$, $s<0$, $t<0$
describing the $a+\bar{d}\to\bar{b}+c$ process, giving us the relation
\begin{equation}
\amplitude_{a+b\to c+d}(s,t,u)=\amplitude_{a+\bar{d}\to\bar{b}+c}(u,t,s).
\end{equation}
Donnachie \textit{et al}.~\cite{Donnachie:2002en} is our source for this
clarification.

\N[1]{CPT Theorem}
The crossing symmetry implies the $CPT$-theorem: invariance of the
amplitude $\amplitude$ with respect to the combination of charge
conjugation $C$, space reflection $P$, and time reversal $T$.

\N{Derivation from Three Assumptions}
We can derive the crossing symmetry from three assumptions stated as Eq~\eqref{eq:introduction:crossing-symmetry:assumptions}.

\N{The Pauli Theorem}
We can use Eq~\eqref{eq:introduction:crossing-symmetry:assumptions}
to derive the spin-statistics relation theorem.

\N[-1]{Unitarity Condition for the Scattering Matrix}. The physical
meaning of unitarity is quite simple: the sum of probabilities of all
processes which are possible (at a given energy) is equal to 1,
$SS^{\dagger}=1$.

\N[1]{Amplitude}
The amplitude $\amplitude$ may be related to the $S$-matrix by writing
\begin{equation}
S = 1 + \I A.
\end{equation}
We then have
\begin{equation}
SS^{\dagger} = (1 + \I A)(1 - \I A^{\dagger}) = 1 + AA^{\dagger} + \I(A-A^{\dagger}),
\end{equation}
which implies
\begin{equation}
AA^{\dagger} + \I(A-A^{\dagger})=0.
\end{equation}
If we write the amplitude explicitly in terms of its real and imaginary components,
\begin{equation}
A = \Re(A) + \I\Im(A),
\end{equation}
then the unitarity condition becomes
\begin{equation}\label{eq:unitarity:in-terms-of-amplitude}
2\Im(A) = AA^{\dagger}.
\end{equation}

\M
We should be careful, because it's a bit misleading. If we have some set
of orthonormal states $\langle j\mid$ and $\mid i\rangle$, unitarity
means
\begin{equation}
  \delta_{ji} = \langle j| SS^{\dagger}| i\rangle
  =\sum_{f}\langle j| S| f\rangle\langle f| S^{\dagger}| i\rangle.
\end{equation}
Using our definition of $\amplitude$, we would have
\begin{equation}
\langle j|\amplitude|i\rangle - \langle j|\amplitude^{\dagger}|i\rangle
=(2\pi)^{4}\I\sum_{f}\delta^{(4)}(P^{f}-P^{i})
\langle j|\amplitude^{\dagger}|f\rangle\langle f|\amplitude|i\rangle.
\end{equation}
For the particular case when $j=i$, we have (note the omitted factor of $\I$):
\begin{equation}
2\Im\langle i|\amplitude|i\rangle
=(2\pi)^{4}\sum_{f}\delta^{(4)}(P^{f}-P^{i}) |\langle f|\amplitude|i\rangle|^{2}.
\end{equation}
We can use this to derive the optical theorem, see Donnachie \textit{et al.}~\cite[esp.~\S\S1.2--1.3]{Donnachie:2002en}
for details.

\section{Mandelstam Variables for Two-Particle Scattering}

\begin{definition}
  The three \define{Mandelstam variables} for $2\to2$ scattering
  processes are
  \begin{subequations}
    \begin{align}
      s &= (p_{1} + p_{2})^{2},\\
      t &= (p_{1} - p_{3})^{2},\\
      u &= (p_{1} - p_{4})^{2}.
    \end{align}
  \end{subequations}
  Further, they satisfy the identity,
  \begin{equation}
s + t + u = \sum^{4}_{j=1}m_{j}^{2},
  \end{equation}
  where we stress we sum over all particles involved in the scattering
  process (i.e., incoming \emph{and outgoing} particles).
\end{definition}

\N{Assumption} We restrict ourselves to the case of equal masses
$m_{j}=\mu$.

\N{Physical Meaning} In the center-of-mass frame of the reaction $a+b\to c+d$
(the so-called $s$-channel), we find $s$ is the square of the total
energy of the colliding particles and $t=-(\vec{p}_{1}-\vec{p}_{3})^{2}$
is the square of the momentum transfer from $a$ to $c$. Note that
\begin{subequations}
  \begin{align}
    t &= (p_{1} + p_{3})^{2}\\
    &= [(E,\vec{p}_{1}) - (E,\vec{p}_{3})]^{2}\\
    &= -(\vec{p}_{1}-\vec{p}_{3})^{2},
  \end{align}
\end{subequations}
where we used the fact the signature is $(+---)$. This is \emph{not quite}
true, since there may be an angular contribution (a factor like
$1\pm\cos\theta$) to the scattering, see Eq~\eqref{eq:introduction:mandelstam-in-cm-frame-with-identical-masses}
below.

In the center-of-mass frame for the reaction $a+\bar{c}\to\bar{b}+d$
($t$-channel), we find $t$ plays the role of total energy squared, and
$s$ is the square of the momentum transfer.

Since the $u$-channel is analogous to the $t$-channel, we find the
variable $u$ in the $u$-channel describes the square of the total energy
of the collision.

\subsection{Mandelstam Plane}

\M
Gribov cites Landau as first to represent the kinematics of the three
reactions graphically on the Mandelstam plane. We will draw three lines
based on the $s+t+u=4\mu^{2}$ relation.

\N{Fact about triangles}
The sum of the distances of a point on the plane to the sides of an
equilateral triangle does not depend on the position of the point.

\M[1]
Since $s+t+u=4\mu^{2}$, we interpret $s$, $t$, $u$ as distances to the
side of an equilateral triangle. Then $s=0$ describes one line, $t=0$
describes a second line, and $u=0$ describes a third line.

\begin{center}
  \includegraphics{img/img.0}
\end{center}


We see the top vertex of the triangle is where $t=u=0$, which forces
$s=4\mu^{2}$.

\M
The ``physical region'' for the reaction $a+\bar{c}\to\bar{b}+d$
corresponds to $t\geq4\mu^{2}$, $s\leq0$, $u\leq0$. We can draw this
region shaded in light red below:

\begin{center}
  \includegraphics{img/img.1}
\end{center}

Similar reasoning suggests $a+b\to c+d$ has $s\geq4\mu^{2}$, $t\leq0$,
$u\leq0$, which corresponds to the light blue shaded region.

The remaining region, shaded in light green, corresponds to
$a+\bar{d}\to\bar{b}+c$.

\begin{center}
  \includegraphics{img/img.2}
\end{center}

\M
We have $\widehat{e}_{s}=\widehat{y}$ be the unit vector from the
midpoint of the $s=0$ edge of our equilateral triangle. Using some
trigonometry, we can find
\begin{equation}
\widehat{e}_{u} = -\frac{\sqrt{3}}{2}\widehat{x} - \frac{1}{2}\widehat{y}
\end{equation}
and
\begin{equation}
\widehat{e}_{t} = \frac{\sqrt{3}}{2}\widehat{x} - \frac{1}{2}\widehat{y}.
\end{equation}
Hence these three sum to zero.\footnote{See also notes from
\url{http://physicspages.com/pdf/Field theory/Mandelstam variables.pdf}}

\M[-1]
Gribov makes a remark about moving from the $s$-channel region in the
Mandelstam plane to the $u$-channel region, the energy dependence of the
scattering amplitude turns into the angular dependence.

\subsection{Threshold Singularities on the Mandelstam Plane}

\M
Gribov begins exploring the importance of singularities by way of
example.

\M[1]
Consider elastic scattering of neutral pions
$\pi^{0}+\pi^{0}\to\pi^{0}+\pi^{0}$.

\N{Assumptions} We assume (in accordance with experiment):
\begin{enumerate}
\item pions are the lightest stable hadrons, and
\item there is no bound state of two neutral pions.
\end{enumerate}

\M Then the amplitude has no singularities at $s<4\mu^{2}$.

\M
The first threshold lies at $s=(2\mu)^{2}$. This corresponds to the
two-particle intermediate state.

\M The next, three-particle threshold could have appeared at
$s=(3\mu)^{2}$.
However, in reality, the second threshold in the pion scattering
amplitude is situated at $s=(4\mu)^{2}$, the four-particle state, since
the transition of two pions into three is forbidden by $G$-parity
conservation.

\N[-1]{Interpretation of Momentum Transfer}
Such singularities in \emph{energy} are known to appear in quantum
mechanics (e.g., the threshold singularity at $s\to 4\mu^{2}$).
But there is a principal difference between relativistic and
nonrelativistic theories in the interpretation of the singularities of
\emph{momentum transfer}.

\N[1]{In Quantum Mechanics} Such singularities in quantum mechanics are
determined by the potential.

\N{Example: Yukawa} For example, the Yukawa potential
\begin{equation}
V(r)=-g^{2}\frac{\exp(- mr)}{r}.
\end{equation}
This corresponds to a pole of the scattering amplitude in the plane of
the squared momentum transfer,
\begin{equation}
\amplitude(k^{2})=-g^{2}\frac{4\pi}{k^{2}-m^{2}}.
\end{equation}
(Gribov has $\amplitude\propto 1/(k^{2}+m^{2})$ which I believe to be an error.)
The singularity occurs at the mass of the actual particle.

\N{In Relativistic Theory} The role of the potential is played by energy
singularities in the $t$-channel, thresholds at $t=4\mu^{2}$,
$16\mu^{2}$, and so on.

\M[-1]
Thus, Gribov concludes, the assumption that all singularities of the
scattering amplitude are determined by the masses of real particles
implies that there are no potentials with an infinite radius (since all
hadrons have nonzero masses).

\section{Partial Wave Expanion and Unitarity}

\N{Partial Waves}
From the conservation of angular momentum, the unitarity condition for
scattering amplitudes with given angular momentum $\ell$ becomes
diagonal. Thus it's convenient to expand the $s$-channel amplitude into
partial waves
\begin{equation}
\amplitude(s,t) = \sum^{\infty}_{\ell=0}f_{\ell}(s)(2\ell+1)P_{\ell}(z),
\end{equation}
where $P_{\ell}(z)$ are Legendre polynomials, and $z$ is defined by the
cosine of the scattering angle $\theta_{s}$,
\begin{equation}\label{eq:introduction:z-of-s}
z = z_{s} := \cos(\theta_{s}) = 1 + \frac{2t}{s-4\mu^{2}}=\frac{u-t}{u+t}.
\end{equation}
This implies, in the physical region of the $s$-channel (where
$u,t\leq0$) we have $-1\leq z\leq 1$ as expected.

\N[1]{Caution: Normalization Constant}
The convention people normally take is to use a different normalization,
so the nonrelativistic limit preserves the partial wave expansion. You
will find in most modern texts $\amplitude(s,t)=16\pi\sum(\dots)$ to be
the convention. (I discovered this conferring with Donnachie \textit{et al.}~\cite{Donnachie:2002en}.)

\M
Also note that in the $s$-channel, the momentum transfer $t$ varies
linearly with $z_{s} = \cos\theta_{s}$. We can therefore write $t=t(s,z_{s})$.
We can see this because
\begin{subequations}
  \begin{align}
\vec{p}_{1}^{2} &= \frac{1}{4s}[s - (m_{1}+m_{2})^{2}][s - (m_{1}-m_{2})^{2}]\\
\vec{p}_{3}^{2} &= \frac{1}{4s}[s - (m_{3}+m_{4})^{2}][s - (m_{3}-m_{4})^{2}],
  \end{align}
\end{subequations}
and
\begin{subequations}
  \begin{align}
t &= m_{1}^{2} + m_{3}^{2} - 2(E_{1}E_{3} - |\vec{p}_{1}|\,|\vec{p}|_{3}\cos\theta_{s})\\
u &= m_{1}^{2} + m_{4}^{2} - 2(E_{1}E_{4} - |\vec{p}_{1}|\,|\vec{p}|_{4}\cos\theta_{s}).
  \end{align}
\end{subequations}
For arbitrary masses, this is a muddle, but for equal masses $m_{i}=\mu$
we have in the $s$-channel, $\vec{p}_{1}=\vec{p}_{3}=\vec{p}$, and
\begin{subequations}\label{eq:introduction:mandelstam-in-cm-frame-with-identical-masses}
  \begin{align}
    s &= 4(\vec{p}^{2}+\mu^{2})\\
    t &= -2\vec{p}^{2}(1 - \cos\theta_{s})\\
    u &= -2\vec{p}^{2}(1 + \cos\theta_{s}).
\end{align}
\end{subequations}
Note: $s=4\vec{p}^{2}+4\mu^{2}$ implies $-4\vec{p}^{2}=4\mu^{2}-s$ and
thus
\begin{equation}
-2\vec{p}^{2}=2\mu^{2}-(s/2).
\end{equation}
So we could write:
\begin{subequations}
  \begin{align}
    s &= 4(\vec{p}^{2}+\mu^{2})\\
    t &= \left(2\mu^{2} - \frac{s}{2}\right)(1 - \cos\theta_{s})\\
    u &= \left(2\mu^{2} - \frac{s}{2}\right)(1 + \cos\theta_{s}).
\end{align}
\end{subequations}

\M[-1] From the relation Eq~\eqref{eq:unitarity:in-terms-of-amplitude},
we find
\begin{equation}\label{eq:ch1:unitarity:partial-wave}
\Im(f_{\ell}(s)) = \frac{k_{s}}{16\pi\omega_{s}} f_{\ell}(s)f^{*}_{\ell}(s) + \Delta,
\end{equation}
where $\Delta$ represents contributions from inelastic channels (where
$\Delta>0$) and vanish for the elastic case, $p$ and $\omega$ stand for
center-of-mass frame momentum and energy, respectively,
\begin{subequations}
  \begin{align}
    k_{s} &= \frac{\sqrt{s-4\mu^{2}}}{2}\\
    \omega_{s} &= \frac{\sqrt{s}}{2}.
  \end{align}
\end{subequations}
We can explicitly solve in the elastic case
\begin{subequations}
  \begin{align}
    f_{\ell}(s)&=\I\frac{8\pi}{v}\left(1 - \E^{2\I\delta_{\ell}(s)}\right),\\
    v &= k_{s}/\omega_{s},
  \end{align}
\end{subequations}
with $\delta_{\ell}$ the scattering phase.

\N[1]{Relativistic Elastic Unitarity Condition}
The solution of the elastic unitarity condition has the same form as in
nonrelativistic quantum mechanics, except for the velocity factor
$v=k/\omega$ which arises due to relativistic normalization of the
amplitude $\amplitude$.

\N[-1]{General Solution}
The general solution to the unitarity condition
Eq~\eqref{eq:ch1:unitarity:partial-wave} can be parametrized by an
``elasticity parameter'' $\eta_{\ell}(s)\leq 1$, satisfying,
\begin{equation}
\eta_{\ell}^{2} = 1 - \frac{v}{4\pi}\Delta.
\end{equation}
Then the general solution looks like,
\begin{equation}\label{eq:ch1:general-form-of-unitary-partial-wave}
f_{\ell}(s) = \I\frac{8\pi}{v}\left(1 - \eta_{\ell}\E^{2\I\delta_{\ell}}\right).
\end{equation}

\N[1]{Bounds}
It follows that the partial wave amplitudes are bounded from above,
\begin{equation}
\Im(f_{\ell})\leq|f_{\ell}|\leq 16\pi/v\qquad(\eta_{\ell}=1).
\end{equation}
Maximal inelasticity of the scattering in a given partial wave
corresponds to $\eta_{\ell}=0$. In the high energy limit ($s\to\infty$),
this leads to
\begin{equation}
\Im(f_{\ell})\leq|f_{\ell}|\leq 8\pi\qquad(\eta_{\ell}=0).
\end{equation}
Observe $v\to1$ as $s\to\infty$. Then the $\eta_{\ell}=0$ forces Eq~\eqref{eq:ch1:general-form-of-unitary-partial-wave}
to look like $f_{\ell} = \I 8\pi/v$ and then the $s\to\infty$ limit
produces the result.

\M
In the $\eta_{\ell}=0$ case, the amplitude $f_{\ell}$ in
Eq~\eqref{eq:ch1:general-form-of-unitary-partial-wave} is purely
imaginary. Gribov explains this means the elastic scattering is but a
``shadow'' of inelastic channels, though I do not adequately understand.

\N{Black Disk Model}
The model
\begin{equation}
  f_{\ell} = \begin{cases}
    \I 8\pi/v,\quad \eta_{\ell=0}, & \mbox{for }\ell<\ell_{0}=k_{s}R,\\
    0,\quad \eta_{\ell}=1,\delta_{\ell}=0, & \mbox{for} \ell>\ell_{0}
  \end{cases}
\end{equation}
is known as the ``black disk'' model for diffractive scattering.

\M
At high energies $s\simeq 4k_{s}^{2}\gg\mu^{2}$ ($v\simeq 1$) when
$\ell_{0}\gg1$, the black disk leads to the forward scattering amplitude
\begin{equation}
\amplitude(s,0) = \sum_{\ell}(2\ell+1)f_{\ell}\approx
\ell_{0}^{2}\cdot8\pi\I\approx\I s\cdot 2\pi R^{2}.
\end{equation}
According to the optical theorem, this results in,
\begin{equation}
\sigma_{\text{tot}} = \frac{\Im(\amplitude(s,0))}{vs}\approx2\pi R^{2}=\pi R^{2}|_{\text{inelastic}}+\pi
R^{2}|_{\text{diffraction}}.
\end{equation}
This is the pattern of diffraction off an absorbing disk of radius $R$.

\N[-1]{Remark}
I do not believe anything, thus far stated concerning partial wave
expansions, has been unique to Gribov. We may find similar remarks in
other quantum mechanics texts.

\subsection{Threshold Behaviour of Partial Wave Amplitudes}

\M We know from quantum mechanics, for potentials of finite range
$r_{0}$, the partial waves behave like $(kr_{0})^{\ell}$ for $k\to 0$.
Similar results hold in $S$-matrix theory.

\M[1] The singularity in $t$ of the amplitude $\amplitude(s,t)$---the
closest to the physical region in the $s$-channel---is located at
$t=4\mu^{2}$. Therefore, the partial wave expansion should be convergent
for $z$ up to $z_{0}=1 + (4\mu^{2}/2k_{s}^{2})$.

\M
For $t>0$ and $s\to4\mu^{2}$, we get $z\to\infty$ and $P_{\ell}$ grows
like $P_{\ell}(z)\sim z^{\ell}$. For the partial wave expansion to
converge, we require $f_{\ell}$ falls with $\ell$ like
$(2k_{s}^{2}/4\mu^{2})^{\ell}$, but not faster since at $t=4\mu^{2}$ the
series \emph{must} be divergent.

\subsection{Singularities of $\Im(A)$ on the Mandelstam Plane}


\section{Froissart's Theorem}

\begin{lemma}[Asymptotics of Legendre Polynomials]
We have as $\ell\to\infty$,
\begin{subequations}
  \begin{align}
  P_\ell (\cos \theta) &= \sqrt{\frac{\theta}{\sin \theta}} \, J_0((\ell+1/2)\theta) + \mathcal{O}\left(\ell^{-1}\right) \\
&= \frac{2}{\sqrt{2\pi \ell\sin\theta}}\cos\left(\left(\ell + \tfrac12\right)\theta - \frac{\pi}{4}\right) + \mathcal{O}\left(\ell^{-3/2}\right), \quad \theta \in (0,\pi),
\end{align}
\end{subequations}
where $J_{0}$ is the Bessel function of the first-kind.
\end{lemma}

For a proof, see:
G\'abor Szeg\~o~\cite{szego},
\emph{Orthogonal polynomials}, (4th ed.) AMS Publishers, pp.194
\emph{et seq.} (see especially, Theorem 8.21.2).

\begin{theorem}[Froissart]
  Asymptotically as $s\to\infty$, we have
  \begin{equation}
\left.\Im\amplitude(s,t)\right|_{t=0}\leq(\mbox{const})\cdot s[\ln(s/s_{0})]^{2}.
  \end{equation}
\end{theorem}

This is due to the analyticity and unitarity properties. The proof is a
multi-step procedure, but each step is fairly straightforward.

\N[1]{Proof Step 1: estimate $f_{\ell}$ at large $s$}
We will show, as $\ell,s\to\infty$,
\begin{equation}
f_{\ell}(s)\simeq c(s,\ell)\exp\left(\frac{-\ell}{k_{s}}\sqrt{4\mu^{2}}\right).
\end{equation}
where $c(s,\ell)$ is slowly [non-exponentially] varying with $\ell$.

\N{Step 2.}
Assume for $t$ arbitrarily close to $4\mu^{2}$, the amplitude grows with
$s$ not faster than some power.

Then the same is valid for $\Im c(s,\ell)$.

Arguably, $\Im f_{\ell}$ is positive due to the unitarity condition, and
so is $P_{\ell}(1 + t/2k_{s}^{2})$ for $t\geq 0$.

Therefore each partial wave we have an estimate,
\begin{subequations}
  \begin{align}
    \left(\frac{s}{s_{0}}\right)^{N}
    &>\Im\amplitude(s,t) = \sum^{\infty}_{\ell=0}\Im(f_{\ell}(s))(2\ell+1)P_{\ell}\left(1+\frac{t}{2k_{s}^{2}}\right)\\
    &>\Im c(s,\ell)\left(2\pi\ell\frac{\sqrt{t}}{k_{s}}\right)^{-1/2}\exp\left[\frac{\ell}{k_{s}}(\sqrt{t}-\sqrt{4\mu^{2}})\right].
  \end{align}
\end{subequations}
Since this holds for arbitrary positive $t<4\mu^{2}$, we conclude
\begin{equation}
\Im c(s,\ell) < (s/s_{0})^{N},
\end{equation}
and, modulo an ``irrelevant pre-exponential factor'', that
\begin{equation}
\Im(f_{\ell}(s))\leq\left(\frac{s}{s_{0}}\right)^{N}\exp(-2\mu\ell/k).
\end{equation}

\N{Step 3: Estimate Imaginary Part of Forward Scattering Amplitude}
We now can estimate
\begin{equation}
\Im\amplitude(s,t=0)\leq8\pi\sum^{L}_{\ell=0}(2\ell+1) + \sum^{\infty}_{\ell=L+1}\Im(f_{\ell}(s))(2\ell+1),
\end{equation}
where when $\ell<L$ in which the partial waves are large,
$\Im(f_{\ell})\simeq|f_{\ell}|=\bigOh(1)$, and we can estimate the first
term as
\begin{equation}
\sum^{L}_{\ell=0}(2\ell+1)\simeq L^{2}.
\end{equation}

\N[1]{Determining $L$}
We can determine $L$ as the bound above which partial wave amplitudes
become small
\begin{equation}
\Im(f_{\ell>L})\ll 1,
\end{equation}
and fall exponentially with $\ell$. According to
$\Im(c(s,\ell))<(s/s_{0})^{N}$ from step 2, this means
\begin{equation}
(s/s_{0})^{N}\exp(-2\mu L/k_{s})\simeq1\implies \boxed{L\simeq\frac{k_{s}}{2\mu}\ln(s/s_{0}).}
\end{equation}

\N[-1]{Step 4: Estimate Infinite Sum in Tail}
We can now estimate the infinite sum using $f_{L+n}\sim f_{K}\exp(-2\mu n/k_{s})$,
and this turns out to be subdominant:
\begin{equation}
\sum^{\infty}_{n=0}2(L+n)\exp\left(\frac{-2\mu}{k_{s}}n\right)\simeq\frac{k_{s}}{\mu}L
+\frac{k_{s}^{2}}{2\mu^{2}},
\end{equation}
and as $s\to\infty$ we find
\begin{equation}
\lim_{s\to\infty}\frac{k_{s}}{\mu}L
+\frac{k_{s}^{2}}{2\mu^{2}}\ll L^{2}.
\end{equation}
That is to say, the infinite sum is dominated by the finite sum's
contribution, which is approximately $L^{2}$.

\N{Step 5: Combine Results}
We have then step 4 simplifies the sum from step 3, giving us that
$\Im\amplitude(s,t=0)\sim L^{2}$ (plus lower order terms)
and step 3 also gave us an estimate $L^{2}\propto s[\ln(s/s_{0})]^{2}$
(since $L^{2}\simeq (k_{s}/2\mu)^{2}[\ln(s/s_{0})]^{2}$ and
$k_{s}^{2}\simeq s$)
which then together implies Froissart's theorem.\qquad\qedsymbol


\section{The Pomeranchuk theorem}

\begin{lemma}[{Schwarz Reflection Principle~\cite[Theorem 6.1.4]{marsden1998}}]
Let $\mathcal{U}$ be a region in the upper half plane whose boundary
$\partial\mathcal{U}$ intersects the real axis in an interval $[a,b]$
(or finite union of disjoint intervals). Let $f\colon\mathcal{U}\to\CC$
be analytic, and continuous on $\mathcal{U}\cup\partial\mathcal{U}$.
Let $\mathcal{U}^{*}=\{z\mid\overline{z}\in\mathcal{U}\}$ be the
reflection of $\mathcal{U}$ and define $g$ on $\mathcal{U}^{*}$ by
\begin{equation}
g(z) = \overline{f(\overline{z})}.
\end{equation}
Assume $f$ is real on the open interval $(a,b)$. Then $g$ is analytic
and is the unique analytic continuation of $f$ to
$\mathcal{U}\cup(a,b)\cup\mathcal{U}^{*}$.
\end{lemma}

\N{Imaginary Part of $s$-Channel Amplitude}
Gribov writes $\Im_{s}\amplitude(s,t)$ for the imaginary part of the
$s$-channel amplitude, which is frustratingly vague. What he means is
the following: recall the Schwarz reflection principle in complex
analysis (see the preceding lemma for a formal statement), where if
$f\colon U\subset\CC\to\CC$ is defined on a
neighborhood $U\cap\RR\neq\emptyset$, then we can extend it by
$f(z^{*})=(f(z))^{*}$.
We do the same, finding
\begin{equation}\label{eq:defn:discontinuity-in-s-channel}
2\I\Im_{s}(\amplitude(s+\I\varepsilon,t)) = \amplitude(s+\I\varepsilon,t)-\amplitude(s-\I\varepsilon,t),
\end{equation}
taking the $\varepsilon\to0$ limit on both sides.
Other authors call the right-hand side of Eq~\eqref{eq:defn:discontinuity-in-s-channel} the discontinuity in the $s$-channel and denote
it $D_{s}(s,t,u)$.
