\renewcommand\paragraphprefix{\thechapter.}

\chapter*{Preface}

\M
This is my collection of notes on Regge theory and S-matrix theory.
It's largely based on Gribov's lectures on complex angular momentum~\cite{Gribov:2003nw}.
My interests were piqued by his claim there is no elementary particle of
spin $J>1$, which would imply the graviton is not elementary.

\N{Writing style idiosyncracies}
The only way I believe I can comprehend Gribov's argument is to
reconstruct it. Towards that end, I am using numbered paragraph
``chunks''. The numbering indicates depth, and when I need to take care
to discuss ``sub-points'' of a chunk, I append a period followed by a
sub-chunk counter.

\M[1]\label{preface:example-of-subchunk}
This is an example of discussing a ``sub-point'' of a paragraph
chunk: this paragraph is preceded by the number
``\ref{preface:example-of-subchunk}'', which appends a period followed
by ``1'' to the parent paragraph chunk.
Usually ``sub-chunks'' are used to deconstruct an argument Gribov gives, which
I have difficulty wrestling with.

\M
Not every paragraph will receive a number, despite appearances in this
preface.

\N{Numbering scheme} Generically, the chunk number will look like
``$\langle$chapter$\rangle$.$\langle$section$\rangle$.$\langle$chunk number$\rangle$''
where ``$\langle$chunk number$\rangle$'' follows the ISO-2145 standard
of decimal numbering. The only exception is this preface, where there
are no sections, so it's omitted from the ``paragraph chunk numbering
scheme''.
