\chapter*{Preface}

\M{}
This is my collection of notes on Regge theory and S-matrix theory.
It's largely based on Gribov's lectures on complex angular momentum~\cite{Gribov:2003nw}.
My interests were piqued by his claim there is no elementary particle of
spin $J>1$, which would imply the graviton is not elementary.

\N{Writing style idiosyncracies}
The only way I believe I can comprehend Gribov's argument is to
reconstruct it. Towards that end, I am using numbered paragraph
``chunks''. The numbering indicates depth, and when I need to take care
to discuss ``sub-points'' of a chunk, I append a period followed by a
sub-chunk counter.

\M[1]\label{preface:example-of-subchunk}
This is an example of discussing a ``sub-point'' of a paragraph
chunk: this paragraph is preceded by the number
``\ref{preface:example-of-subchunk}'', which appends a period followed
by ``1'' to the parent paragraph chunk.
Usually ``sub-chunks'' are used to deconstruct an argument Gribov gives, which
I have difficulty wrestling with.

\M
Not every paragraph will receive a number, despite appearances in this
preface.

\makeatletter
\N{Numbering scheme} Generically, the chunk number will look like
\begin{center}
``$\langle$chapter$\rangle$\chunk@separator$\langle$section$\rangle$\chunk@separator$\langle$subsection$\rangle$\chunk@divider$\langle$chunk number$\rangle$''
\end{center}
where ``$\langle$chunk number$\rangle$'' follows the ISO-2145 standard
of decimal numbering. The only exception is this preface, where there
are no sections (or subsections), so it's omitted from the ``paragraph
chunk numbering scheme''.

To distinguish chunk ``3.1'' in chapter 1 from chunk ``1'' in section 3
of chapter 1, I have opted to use a slash to demarcate section numbers
from chunk numbers. The alternative was to use upper dots for the
section number and lower dots for the chunk number and a colon instead
of a slash (like ``1\textudot3:1'' which has section part
``1\textudot3'' [chapter 1, section 3] and chunk number ``1''; as
distinct from ``1:3.1'' for chapter 1, chunk number ``3.1'').
\makeatother

\N{Exercises} I have added some exercises I thought would be
useful. They interrupt the flow, and end with the symbol ``\exercisesymbol''.