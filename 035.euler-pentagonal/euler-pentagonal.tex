%%% euler-pentagonal.tex --- 
%% 
%% Filename: euler-pentagonal.tex
%% Description: 
%% Author: alex
%% Maintainer: 
%% Created: Sat Feb 13 11:04:44 2016 (-0800)
\documentclass{article}
\usepackage{chunk}
\usepackage{notebk}

\title{Euler's Pentagonal Theorem}
\date{February 13, 2015}

\begin{document}
\maketitle
\begin{abstract}
  We introduce Euler's pentagonal function. Then we relate it to the
  partition function and divisor function. Last, we examine the Jacobi
  triple product formula with it. We work entirely via puzzle-driven
  discovery. 
\end{abstract}
\tableofcontents

\section{Problem Statement}
\N{Puzzle}
Consider the sequence of functions defined as finite products
\begin{equation}
  f_{n}(x) = \prod^{n}_{k=0}(1-x^{k}).
\end{equation}
What is this equal to as $n\to\infty$? Perhaps we can gleam some insight
if we generate a table of such functions.

\begin{align}
  f_{3}(x) &= - x^{6} + x^{5} + x^{4} - x^{2} - x + 1\\
  f_{4}(x) &= x^{10} - x^{9} - x^{8} + 2x^{5} - x^{2} - x + 1\\
  f_{5}(x) &= -x^{15} + x^{14} + x^{13} - x^{10} - x^{9} - x^{8} + x^{7} + x^{6} + x^{5} - x^{2} - x + 1\\
  f_{6}(x) &= x^{21} - x^{20} - x^{19} + x^{16} + 2x^{14} - x^{12} - x^{11} - x^{10} - x^{9}\nonumber\\
  &\qquad + 2x^{7} + x^{5} - x^{2} - x + 1\\
  f_{7}(x) &= -x^{28} + x^{27} + x^{26} - x^{23} - x^{21} - x^{20} +
  x^{18} + x^{17} + 2x^{16}\nonumber\\
&\qquad- 2x^{12} - x^{11} - x^{10} + x^{8} + x^{7} + x^{5} - x^{2} - x + 1
\end{align}


\M We have the following table of terms (higher order terms are swept
into a big-O prison):
\begin{align}
f_{4}(x) &= 1- x - x^{2} + \mathcal{O}(x^{5})\\
f_{5}(x) &= 1 -x - x^{2} + x^{5} + \mathcal{O}(x^{6}) \\
f_{6}(x) &= 1 -x - x^{2} + x^{5} + \mathcal{O}(x^{7}) \\
f_{7}(x) &= 1 -x - x^{2} + x^{5} + x^{7} + \mathcal{O}(x^{8}) \\
\dots & \dots \\
f_{20}(x) &= 1 - x - x^{2} + x^{5} + x^{7} - x^{12} - x^{15} + \mathcal{O}(x^{21})
\end{align}
We have several observations.

First, as $n$ increases, the lower order terms ``stabilize'' in the sense:
the coefficient of $x^{n}$ remains the same in $f_{n+1}(x)$,
$f_{n+2}(x)$, and so on. Hence multiplying by $(1-x^{m})$ for $m>n$ does
not affect the $x^{n}$ term. Hence it makes sense to talk about the
``stable product''
\begin{equation}
f(x) = f_{\infty}(x) = \prod^{\infty}_{n=0}(1-x^{n})
\end{equation}
which is not a polynomial but a formal series.

\M
The second observation is that many terms cancel. Although we should
expect \emph{some} terms cancel, based on the above table, lets consider a
concrete example. The polynomial $f_{10}(x)$ has 36 terms, but only 5
are stable. If we consider the stable terms less than degree 100, we
find
\begin{equation}
  \begin{split}
f(x) =& 1 - x - x^{2} + x^{5} + x^{7} - x^{12} - x^{15} \\
& + x^{22} + x^{26} - x^{35} - x^{40} +x^{51} + x^{57} \\
&- x^{70} - x^{77} + x^{92} + x^{100} + \dots
  \end{split}
\end{equation}

\M
We also can observe the coefficients are always either $+1$ or $-1$. And
they always come in pairs (two negative, followed by two positive),
alternating in sign. We write down a little table summarizing the
patterns: 
\begin{center}
\begin{tabular}{c | c}
  Exponent & Coefficients\\\hline
  0        &    1\\
  1, 2     &   $-1$\\
  5, 7     &    1\\
  12, 15   &   $-1$\\
  22, 26   &    1\\
  35, 40   &   $-1$\\
  51, 57   &    1\\
  70, 77   &   $-1$\\
  92, 100  &    1\\
\end{tabular}
\end{center}
\noindent\ignorespaces%
What is the pattern behind the exponents? Well, we can fumble around,
and after a little experimentation we find the pairs obey $(3n^{2}\pm
n)/2$ for $n\geq1$. 

Great! We can now explicitly write down the series representation:
\begin{equation}
f(x) = \sum^{\infty}_{n=0} -1^{n}(x^{(3n^{2}-n)/2} + x^{(3n^{2}+n)/2})
\end{equation}
Or more succinctly (if not, more cavalier):
\begin{equation}
\prod^{\infty}_{n=0}(1-x^{n}) = \sum^{\infty}_{k=-\infty}(-1)^{k}x^{(3k^{2}+k)/2}
\end{equation}
This equation is called the Euler identity. (Not to be confused with the
other 1238123 equations called ``Euler identities'')

\M
Euler, who wrestled with far harder problems, had immense difficulty
with this particular series. Daniel Bernoulli asked Euler about a
related problem in a letter dated January 28, 1741. Bernoulli was
interested in a product of the form $\prod_{n}(1-x^{-n})$. Euler
wrestled with the problem since then, and wrote an entire treatise
dedicated to this very subject (delivered on April 6, 1741 but published
a decade later). The history of Euler's work with this problem may be
perused in Jordan Bell's ``Euler and the pentagonal number theorem''
(\arXiv{math/0510054}), a fascinating read and an authoritative
article on the matter. 

We will quote Polya's translation of Euler's remarks on the identity:

\begin{quote}
In considering the partitions of numbers, I examined, a long time ago,
the expression
\begin{equation*}
 (1 - x)(1 - x^2 )(1 - x^3 )(1 - x^4 )(1 - x^5 )(1 - x^6 )(1 - x^7 )(1 - x^8 ) \dots , 
\end{equation*}
in which the product is assumed to be infinite. In order to see what
kind of series will result, I multiplied actually a great number of
factors and found
\begin{equation*}
 1 - x - x^2 + x^5 + x^7 - x^{12} - x^{15} + x^{22} + x^{26} - x^{35} - x^{40} + \dots 
\end{equation*}
The exponents of $x$ are the same which enter into the above formula
[referring to a preceding part of Euler's article containing an
  explanation of the sequences 1, 5, 12, 22, 35, \dots and 2, 7, 15, 26,
  40, \dots]; also the signs $+$ and $-$ arise twice in succession. It
suffices to undertake this multiplication and to continue it as far as
it is deemed proper to become convinced of the truth of these
series. Yet I have no other evidence for this, except a long induction
which I have carried out so far that I cannot in any way doubt the law
governing the formation of these terms and their exponents. I have long
searched in vain for a rigorous demonstration of the equation between
the series and the above infinite product $(1 - x)(1 - x^2 )(1 - x^3
)\dots$, and I proposed the same question to some of my friends with
whose ability in these matters I am familiar, but all have agreed with
me on the truth of this transformation of the product into a series,
without being able to unearth any clue of a demonstration.
\end{quote}

\section{Proof}

The basic proof will be to observe in the product, what we do for polynomials (in the naive way) we just multiply out all the terms, then sum them together. 

\N*{Step 1} A generic term will look like
\begin{equation}
(-1)^{k}x^{n_{1}+\dots+n_{k}}
\end{equation}
for $k\geq0$ and $n_{1}<\dots<n_{k}$.

\N*{Step 2} We argue that the coefficient of $x^{n}$ will be equal to:
\begin{equation}
\begin{pmatrix}
\mbox{number of partitions}\\
n = n_{1} + \dots + n_{k}\\
(0<n_{1}<\dots<n_{k})\\
\mbox{with even }k\\
\end{pmatrix}
-
\begin{pmatrix}
\mbox{number of partitions}\\
n = n_{1} + \dots + n_{k}\\
(0<n_{1}<\dots<n_{k})\\
\mbox{with odd }k\\
\end{pmatrix}
\end{equation}

\N*{Step 3} We introduce, for a given partition $n=n_{1}+\dots+n_{k}$,
there is a number $s$ of the largest consecutive number of $n_{i}$'s
which terminate at $n_{k}$. What's an example? For $1+1+2+3$, we have
$s=3$.

For $1+2+3+4+5+7+8+8+8+9$, we have $s=3$ because $(7,8,9)$ is the
consecutive sequence. Note that we always start at the largest value,
and work towards getting smaller values (we started with $9$, then found a
sequence of consecutive numbers ending in $9$). There is the other, more
obvious, consecutive sequence $(1,2,3,4,5)$ which is larger\dots but has
the obvious flaw it does not end in $9$.

\N*{Step 4} We claim that any partition of $n=n_{1}+\dots+n_{k}$ for
$0<n_{1}<n_{2}<n_{k}$ must fall into one of three categories: 
\begin{enumerate}
\item $n_{1}\leq s$ excluding when $n_{1}=s=k$
\item $n_{1}>s$ excluding when $n_{1}+1=s+1=k+1$
\item The two excluded cases, when $n_{1}=s=k$ or $n_{1}+1=s+1=k+1$.
\end{enumerate}

\N*{Step 5} We argue that there is a one-to-one correspondence between
partitions in class 1 and partitions in class 2. (If you accept this, we
can see immediately that almost everything must cancel since partitions
in these classes have opposite signs from each other.)

\N*{Step 6} The only remaining partitions are in the third class, and
are of the form $k+(k+1)+(k+2)+\dots+(2k-1)$ or $(k+1)+(\dots)+(2k)$
which correspond to the terms
\begin{equation}
(-1)^{k}x^{k+(k+1)+\dots+(2k-1)}=(-1)^{k}x^{k(3k-1)/2}
\end{equation}
and
\begin{equation}
(-1)^{k}x^{(k+1)+\dots+(2k)}=(-1)^{k}x^{k(3k+1)/2}
\end{equation}
respectively.

\section{Partition Function}

\N{Definition}
Let $n$ be a positive integer. A \define{Partition} of $n$ is a finite
sequence of strictly positive integers $0<n_{1}\leq n_{2}\leq\dots\leq n_{k}$
such that
\begin{equation}
  n = n_{1} + \dots + n_{k}
\end{equation}
The \define{Partition Function} is a function $p\colon\NN\to\NN$ defined
such that
\begin{equation}
  p(0)=p(1)=1
\end{equation}
and
\begin{equation}
  p(n) = |\{(n_{1},\dots,n_{k})\mbox{ is a partition of }n\}|.
\end{equation}
We adopt the convention $p(-n)=0$ for $n>0$, hence extending $p$ to be
well-defined over the integers.

\N*{Warning:} (1) The ``partitions'' we considered in the proof of
Euler's pentagonal theorem differ from the ``partition'' we now define,
in that we required \emph{before} to have \emph{distinct} numbers in the
partition $n_{1}<n_{2}<\dots<n_{k}$. But now we relax the condition and
allow duplicates $n_{1}\leq n_{2}\leq\dots\leq n_{k}$.

(2) \emph{Do not} confuse this ``partition function'' from number theory
with the ``partition function'' from physics! They are completely
different functions!

\N{Puzzle} For the sake of gaining familiarity with the partition
function, what is the value of $p(n)$ for $n=1$, \dots, $7$?

\N*{Solution}
We provide some explicit reasoning for various numbers, to show how to
reason with such a function. Observe 2 has two partitions:
\begin{equation}
  2 = 2 = 1+1 \implies p(2)=2.
\end{equation}
Similarly 3 has 3 partitions, since
\begin{equation}
  3 = 3 = 2+1 = (1+1)+1 \implies p(3)=3.
\end{equation}
We must resist the tempting pattern $p(n)=n$, because $4$ provides us a
counter-example:
\begin{equation}
  \begin{split}
    4 &= 4 = 2 + 2 = (1 + 1) + 2\\
    &= 3 + 1 = (2 + 1) + 1 = 1+1+1
  \end{split}
\end{equation}
but $2+1+1=1+1+2$, hence avoiding this duplicate we find
\begin{equation}
  p(4) = 5\neq 4.
\end{equation}
Similarly, we find
\begin{equation}
  5 = 5 = 4 + 1 = 3 + 2
\end{equation}
and we know there are $5$ different partitions for 4, which would
incorporate the various partitions stemming from $3+2$, so we get
\begin{equation}
  p(5) = 2 + 5 = 7.
\end{equation}
This is a bit cavalier, but it demonstrates a recursive aspect to the
partition function.

\N{Generating Function}
A common trick we can try: consider the function $p(x)$ (we abuse
notation, when $x$ is used it's a different function) defined by
\begin{equation}
  p(x) \eqdef
  1 + x + 2x^{2} + 3x^{3} + 5x^{4} + \dots
  = \sum^{\infty}_{r=0}p(r)x^{r}
\end{equation}
We use the partition function to define the coefficients in this formal
power series.

\N{Puzzle}
What is $f(x)p(x)$?

\N*{Guess}
The first thing we could try doing is multiplying the two series
together, just to see what happens. We write down:
\begin{equation}
  f(x)p(x) = (1 - x - x^{2} + x^{5} + x^{7} - x^{12} - x^{15} + \dots)(1
  + p(1)x + p(2)x^{2} + \dots)
\end{equation}
Now we use a trick from mathematical physics, and examine the
coefficients of $x^{n}$. We find
\begin{equation}
  \begin{split}
    f(x)p(x) =& 1\\
    &+ (p(1) - 1)x\\
    &+ (p(2) - p(1) - 1)x^{2}\\
    &+ (p(3) - p(2)-p(1))x^{3}\\
    &+ (p(4) - p(3) - p(2))x^{4}\\
    &+ (p(5) - p(4) - p(3) + 1)x^{5} + \dots
  \end{split}
\end{equation}
But we know, truncating the expansion here, this is precisely equal to
$1$.

\N{Claim:} $f(x)p(x)=1$.

\begin{proof}
It suffices to show $1/f(x)=p(x)$. We see
\begin{equation}
  \frac{1}{f(x)} = \frac{1}{\prod^{\infty}_{n=1}(1-x^{n})} = \prod^{\infty}_{n=1}(1 - x^{n})^{-1}
\end{equation}
then using the geometric series expansion for
\begin{equation}
  (1 - x^{n})^{-1} = \sum^{\infty}_{k=0}x^{nk}
\end{equation}
we get
\begin{equation}
  \begin{split}
  \frac{1}{f(x)} &= \prod^{\infty}_{n=1}\left(\sum^{\infty}_{k=0}x^{nk}\right)\\
  &= \prod^{\infty}_{n=1}\left(1 + x^{n} + x^{2n} + x^{3n} + \dots\right).
  \end{split}
\end{equation}
Now we do the same trick as we did in our guessing: examine the
coefficient of $x^{r}$ in the product.

We take only one summand from each factor, and multiply them
together. We then get
\begin{equation}
  x^{1\cdot k_{1}} + x^{2\cdot k_{2}} + \dots + x^{m k_{m}} = x^{k_{1} + 2k_{2} + \dots + m k_{m}} = x^{r}
\end{equation}
How many such products exist? (That is to say, what is the coefficient?)
We find that
\begin{equation}
  r = k_{1} + \dots + mk_{m} = \underbrace{1+\dots+1}_{k_{1}} + \dots + \underbrace{m+\dots+m}_{k_{m}}
\end{equation}
is precisely a partition of $r$, so the total number of factors which
multiply out to $x^{r}$ is the total number of partitions of $r$. That
is to say, the coefficient of $x^{r}$ in $1/f(x)$ is $p(r)$.
\end{proof}

\section{Divisor Function}

\N{Definition}
Let $n\in\NN$ be an arbitrary positive integer. A \define{Divisor} of
$n$ is a positive integer $k$ such that $n = mk$ for some $m\in\NN$. We
write $k\divides n$ to indicate $k$ is a divisor of $n$.

The \define{Divisor Function} $d\colon\NN\to\NN$ is such that $d(n)$ is
the sum of the divisors of $n$, i.e., 
\begin{equation}
  d(n) = \sum_{k\divides n}k.
\end{equation}

\N{Puzzle}
Is there a closed-form expression for $d(n)$?

\N*{Simpler Puzzle}
Let $p$ be a prime number, and $k\in\NN$ be an arbitrary natural
number. What is $d(p^{k})$? This is quite a bit simpler, since the only
possible divisors of $p^{k}$ are smaller powers of $p$. How many? Well,
we have $p^{0}$, $p^{1}$, \dots, $p^{k-1}$, and $p^{k}$. So
\begin{equation}
  d(p^{k}) = \sum^{k}_{j=0}p^{j}.
\end{equation}
We can then recall the geometric formula for sums of powers, giving us
the closed form expression
\begin{equation}
  d(p^{k}) = \frac{p^{k+1}-1}{p-1}.
\end{equation}

\N*{Solution}
We know from the fundamental theorem of arithmetic any natural number
can be written as the product of primes $n = p_{1}^{k_{1}}(\dots)p_{m}^{k_{m}}$
for prime numbers $p_{1}$, \dots, $p_{m}$ and natural numbers $k_{1}$,
\dots, $k_{m}$. We also know that prime numbers are only divisible by
themselves, so $d(p_{1}p_{2})=d(p_{1})d(p_{2})$ for distinct primes
$p_{1}\neq p_{2}$.

Hence we deduce
\begin{equation}
  d(n) = d(p_{1}^{k_{1}}(\dots)p_{m}^{k_{m}}) = d(p_{1}^{k_{1}})(\dots)d(p_{m}^{k_{m}}).
\end{equation}
We now may use our simpler puzzle to substitute the values of
$d(p_{j}^{k_{j}})$, giving us
\begin{equation}
  d(n) = \prod^{m}_{j=1}\frac{p_{j}^{k_{j}+1}-1}{p_{j}-1}.
\end{equation}

\N{Exercise} Again, just to get a feel of doing the calculations, write
down $d(n)$ for $n=1$, $2$, \dots, $10$. Also compute $d(1000)$ and
$d(1000000)$.

\N{Generating Function}
Doing the same trick as for the partition function, lets consider the
generating function associated to the divisor function:
\begin{equation}
d(x) = \sum^{\infty}_{k=1}d(k)x^{k} = x + 3x^{2} + 4x^{3} + 7x^{4} + 6x^{5} + \dots
\end{equation}

\N{Puzzle}
What is a closed-form expression for $d(x)$?

\M
This is a good puzzle, and one the reader should attempt very hard
before examining our solution (which follows the footsteps of Euler). 

\N*{Step 1:} We claim
\begin{equation}
  d(x) = \sum^{\infty}_{n=1}\frac{nx^{n}}{1-x^{n}}.
\end{equation}
(Note: This is not the end of the story, since the right hand side is
another power series!) We use the geometric series for $(1 -
x^{n})^{-1}$ to write
\begin{equation}
\sum^{\infty}_{n=1}\frac{nx^{n}}{1-x^{n}} = \sum^{\infty}_{n=1}n(x^{n} + x^{2n} + x^{3n} + \dots).
\end{equation}
Consider the coefficient of $x^{r}$. If $d_{j}\divides r$, then we have
\begin{equation}
 c_{r}x^{r} = (\mbox{something} + d_{j})x^{r}
\end{equation}
The other terms in the coefficient must also be a divisor of $r$, i.e.,
come from some term that looks like $d_{j}x^{d_{j}\cdot k_{j}}$ such
that $d_{j}k_{j} = r$. But this necessarily means $c_{r}=d(r)$. Hence we
get the result immediately.

\N*{Step 2:} We claim
\begin{equation}
  \frac{nx^{n}}{1-x^{n}}=-x\frac{\D}{\D x}\ln(1 - x^{n}).
\end{equation}
This can be verified by direct computation
\begin{equation}
    \frac{\D}{\D x}\ln(1 - x^{n}) = \frac{1}{1-x^{n}}\frac{\D}{\D x}(1-x^{n})
\end{equation}
by chain rule, and product rules gives us
\begin{equation}
\frac{1}{1-x^{n}}\frac{\D}{\D x}(1-x^{n}) = \frac{-nx^{n-1}}{1 - x^{n}},
\end{equation}
hence the result.

\N*{Step 3:} We claim
\begin{equation}
  \frac{\D}{\D x}\sum^{\infty}_{n=1}\ln(1 - x^{n}) = \frac{f'(x)}{f(x)}.
\end{equation}
Using the law of logarithms\footnote{Strictly speaking, this is dodgy
  when applied to series, but Euler was adventurous!} we find
\begin{equation}
\sum^{\infty}_{n=1}\ln(1 - x^{n}) = \ln\left(\prod^{\infty}_{n=1}(1 - x^{n})\right)
\end{equation}
but the right hand side is precisely
\begin{equation}
 \ln\left(\prod^{\infty}_{n=1}(1 - x^{n})\right) = \ln(f(x)).
\end{equation}
Now taking the derivative gives the result.

\N*{Step 4:} We conclude
\begin{equation}
  d(x) = \frac{-xf'(x)}{f(x)}
\end{equation}
This comes from
\begin{align*}
d(x) &= \sum^{\infty}_{n=1}\frac{nx^{n}}{1 - x^{n}}&\mbox{(step 1)}\\
&= \sum^{\infty}_{n=1}-x\frac{\D}{\D x}\ln(1 - x^{n})&\mbox{(step 2)}\\
&= -x\frac{\D}{\D x}\sum^{\infty}_{n=1}\ln(1 - x^{n})&\mbox{(distributivity)}\\
&= -x\left(\frac{\D}{\D x}\sum^{\infty}_{n=1}\ln(1 - x^{n})\right)&\mbox{(associativity)}\\
&= -x\frac{f'(x)}{f(x)}&\mbox{(step 3)}.
\end{align*}

\section{Gauss and Jacobi Identities}

\N{Gauss Identity}
Some 70 years after Euler's investigations, Gauss discovered the most
startling identity involving the cube of Euler's function:
\begin{equation}
  \begin{split}
f(x)^{3} &= 1 - 3x + 5x^{3} - 7x^{6} + 9x^{10} - 11x^{15} + \dots\\
    &= 1 + \sum^{\infty}_{n=1} (-1)^{n}(2n+1)x^{(n+1)(n+2)/2}.
  \end{split}
\end{equation}
This is doubly surprising since its square does not reveal any regular
pattern or interesting properties
\begin{equation}
  f(x)^{2} = 1 - 2x - x^{2} + 2x^{3} + x^{4} + 2x^{5} - 2x^{6} - 2x^{8}
  - 2x^{9} + x^{10} + \dots.
\end{equation}
(Truthfully, this work was never published by Gauss, but found after his
death; it may be perused in Gauss' \emph{Werke: Bd. Analysis},
vol.III. The work was done in 1809.)
This result may be derived from the Jacobi triple product identity.

\N{Jacobi Triple Product}
The triple product formula appears in many different fields of
mathematics, in very different contexts (indicating this is a profound
result indeed). Jacobi published his triple product in his work on
elliptic functions~\cite{jacobi}:
\begin{equation}
  \prod^{\infty}_{m=1}
  (1 - z^{2m})
  (1 + z^{2m-1}y)
  (1 + y^{-1}z^{2m-1}) = \sum_{r=-\infty}^{\infty} y^{r}z^{r^{2}}.
\end{equation}
Proving this relation is quite involved, so lets first make a detour to
show Gauss' identity derives from the triple product.

\N{From Jacobi to Gauss}



\begin{thebibliography}{99}
\bibitem{fuchs}
  Dmitry Fuchs and Serge Tabachnikov,
  \emph{Mathematical Omnibus: Thirty Lectures on Classic Mathematics}.
  AMS Publisher, 2007.
\bibitem{jacobi}
  Carl Jacobi,
  \emph{Fundamenta Nova Theoriae Functionum Ellipticarum}.
  1829.
\bibitem{leibenzon}
  Z. Leibenzon,
  ``A simple combinatorial method for proving the Jacobi identity and its generalizations''.
  \emph{Funct. Anal. Appl.} \textbf{20} no.1 (1986), pp.~66--68.
\end{thebibliography}

\end{document}

