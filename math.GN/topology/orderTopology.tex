%%
%% orderTopology.tex
%% 
%% Made by Alex Nelson
%% Login   <alex@tomato>
%% 
%% Started on  Tue Jun  2 19:01:59 2009 Alex Nelson
%% Last update Tue Jun  2 19:01:59 2009 Alex Nelson
%%
\begin{prob}
Topologies seem more or less abstract. Isn't there some way to
simplify specifying open sets? We are going to introduce the
notion of specifying open sets using open intervals. How do we
specify open intervals? By using an order relation $a < b$, we
can specify open intervals.
\end{prob}

If $X$ is a simply ordered set, there is a standard topology for
it defined using the order relation. It's called the
\textbf{order topology}.

We should probably first generalize the notion of intervals
familiar from real analysis. Since $X$ is a simply ordered set,
there is a (simple) order relation $<$. So suppose we have
$a,b\in X$ such that $a<b$, then we have 4 possible subsets of
$X$:
\begin{subequations}
\begin{align}
(a,b) &= \{x|a<x<b\}\label{eq:openInterval}\\
(a,b] &= \{x|a<x\leq b\}\label{eq:clopenInterval}\\
[a,b) &= \{x|a\leq x<b\}\label{eq:oposedInterval}\\
[a,b] &= \{x|a\leq x\leq b\}\label{eq:closedInterval}.
\end{align}
\end{subequations}
We call the eq \eqref{eq:openInterval} an ``open interval'', eqs
\eqref{eq:clopenInterval} \eqref{eq:oposedInterval} ``clopen
intervals'', and lastly eq \eqref{eq:closedInterval} a ``closed
interval''.

\begin{defn}\label{defn:orderTopology}
Let $X$ be a set with a simple order relation. Suppose $X$ has
more than one element. Let $\mathscr{B}$ be the collection of all
sets of the following types:
\begin{enumerate}
\item All open intervals $(a,b)$ in $X$.
\item All intervals of the form $[a_{0},b)$, where $a_0$ is the
  smallest element (if any) of $X$.
\item All intervals of the form $(a,b_{0}]$, where $b_{0}$ is the
  largest element (if any) of $X$.
\end{enumerate}
The collection $\mathscr{B}$ is a basis for a topology on $X$,
which is called the \textbf{order topology}.
\end{defn}
\begin{rmk}\label{rmk:onOrderTopology}
If $X$ has no smallest element, there are no sets of type (2),
and if $X$ has no largest element there are no sets of type (3).
\end{rmk}

\begin{defn}\label{defn:rays}
If $X$ is an ordered set, and $a$ is an element of $X$, there are
four subsets of $X$ that are called the \textbf{rays} determined
by $a$. They are the following
\begin{subequations}
\begin{align}
(a,+\infty) &= \{x|x>a\}\\
(-\infty,a) &= \{x|x<a\}\\
[a,+\infty) &= \{x|x\geq a\}\\
(-\infty,a] &= \{x|x\leq a\}
\end{align}
\end{subequations}
The first two are \textbf{open rays}, the last two are
\textbf{closed rays}.
\end{defn}
