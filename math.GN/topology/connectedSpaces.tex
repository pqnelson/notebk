%%
%% connectedSpaces.tex
%% 
%% Made by Alex Nelson
%% Login   <alex@tomato>
%% 
%% Started on  Sun Jun  7 17:58:50 2009 Alex Nelson
%% Last update Sun Jun  7 17:58:50 2009 Alex Nelson
%%
\begin{prob}
How can we tell if a space is ``connected'' or not? That is, if
it's ``disconnected'', then we should be able to ``break it up''
into independent pieces. How do we make this notion of
``disconnectedness'' rigorous?
\end{prob}
\begin{defn}\label{defn:seperation}
Let $X$ be a topological space. A \define{Seperation} of $X$ is a
pair $U,V$ of disjoint, nonempty, open sets whose union is
$X$. The space $X$ is said to be ``\define{Connected}'' if there
is no seperation in it.
\end{defn}
\begin{rmk}
Observe that we didn't really define connectedness. Instead we
defined what it means for it to have a seperation, and then
proceeded to define ``connected'' as ``not seperated''. So, to
prove a topological space is connected is equivalent to
disproving the existence of a seperation. Consequently, we will
use proof by contradiction a lot when dealing with connectedness.
\end{rmk}
\begin{rmk}
Observe that if $X$ is connected and $Y$ is homeomorphic to $X$,
then $Y$ is necessarily connected.
\end{rmk}
\begin{rmk}
A space $X$ is connected if and only if the only closed subsets
of $X$ that are both open and closed are $X$ and
$\emptyset$. Otherwise if we had some additional set $U$ that is
closed and open, then $X-U$ is also closed and open. But $U$ and
$X-U$ are disjoint, nonempty, open subsets of $X$ whose union is
$X$. This couldn't happen if $X$ were connected, so $U$ needs to
be empty or equal to all of $X$.
\end{rmk}
\begin{lem}\label{lemma:subspaceConnectedness}
If $Y$ is a subspace of $X$, a seperation of $Y$ is a pair of
disjoint nonempty sets $A$ and $B$ whose union is $Y$, neither of
which contains a limit point of the other. The space $Y$ is
connected if there exists no seperation of $Y$.
\end{lem}
\begin{proof}
Let $A$, $B$ form a seperation of $Y$. We see that
\begin{equation}%\label{eq:}
A=\overline{A}\cap Y\subset Y
\end{equation}
and
\begin{equation}%\label{eq:}
A^{C}=(\overline{A}\cap Y)^{C}=B,
\end{equation}
which implies
\begin{equation}%\label{eq:}
(\overline{A}\cap Y)\cap B=\emptyset.
\end{equation}
We can argue similarly for $(\overline{B}\cap Y)\cap A=\emptyset$.
\end{proof}
\begin{prop}%\label{prop:}
Let $\mathcal{T}$, $\mathcal{T}'$ be two topologies on $X$. If
$\mathcal{T}\subset\mathcal{T}'$, then
\begin{enumerate}
\item the existence of a seperation
in $\mathcal{T}$ implies existence of a seperation in
$\mathcal{T}'$;
\item connectedness in $\mathcal{T}'$ implies connectedness in
  $\mathcal{T}$.
\end{enumerate}
\end{prop}
\begin{proof}
\noindent\begin{enumerate}
\item Observe that if $A$, $B$ forms a seperation in the
  $\mathcal{T}$ topology, then
  $A,B\in\mathcal{T}\subset\mathcal{T}'$. So the seperation is in
  the $\mathcal{T}'$ topology too.
\item Observe that if $X$ is connected in the $\mathcal{T}'$
  topology, then there is no seperation (no pair $A,B$ that form
  a seperation of $X$ that live in $\mathcal{T}'$). This means
  that no such $A,B$ exist in $\mathcal{T}$, which implies there
  is no seperation of $X$ in the $\mathcal{T}$ topology. So $X$
  is connected in the $\mathcal{T}$ topology.
\end{enumerate}
\end{proof}
\begin{prob}
How can we \emph{construct} connected topological spaces? This
seems especially daunting since we only can really know if a
topological space is \emph{disconnected}.
\end{prob}
\begin{thm}%\label{thm:}
If the sets $C$ and $D$ form a seperation of $X$, and if $Y$ is a
connected subspace of $X$, then $Y$ lies entirely in $C$ or
entirely in $D$.
\end{thm}
\begin{proof}
Assume for contradiction that $Y$ can lie in both $C$ and
$D$. Then we can observe that $C\cap Y$ and $D\cap Y$ are subsets
of $Y$ that are both open and closed, as well as disjoint and
their union would be $Y$. How can we see this last point? Well,
we know since $C$ and $D$ form a seperation of $X$ that
\begin{equation}%\label{eq:}
C\cup D=X
\end{equation}
so it follows that
\begin{equation}%\label{eq:}
(C\cap Y)\cup(D\cap Y) = (C\cup D)\cap Y = X\cap Y = Y.
\end{equation}
We have our contradiction, it implies $Y$ has a seperation. We
reject our assumption and say that $Y$ has to lie entirely in $C$
or entirely in $D$.
\end{proof}
\begin{thm}%\label{thm:}
The union of connected subspaces of $X$ with a common shared
point is connected.
\end{thm}
\begin{proof}
Let $\{A_{\alpha}\}$ be our collection of connected subspaces,
\begin{equation}%\label{eq:}
Y=\bigcup_{\alpha} A_{\alpha}
\end{equation}
be the subspace we are trying to show is connected,
\begin{equation}%\label{eq:}
\{y_0\} = \bigcap_{\alpha}A_{\alpha} 
\end{equation}
be the point common to all connected subspaces.

Assume for contradiction $Y$ is disconnected. Then there exists a
pair $C$, $D$ of disjoint, nonempty, open subsets of $Y$ whose
union is $Y$. Since $A_\alpha$ is connected, by our previous
theorem, it must lie entirely in $C$ or $D$. 

Since all $A_\alpha$ share a common point, it implies that all
$A_\alpha$ lie entirely in $C$ or they all lie entirely in
$D$. If some lived entirely in $C$ while the rest entirely in
$D$, there would be no shared point $y_0$ since $C$ and $D$ are
disjoint. 

So it follows that their union lies entirely in one. But this
means that either $C=\emptyset$ and $D=Y$, or $C=Y$ and
$D=\emptyset$, which is a contradiction of our assumption that
$C\neq\emptyset$ and $D\neq\emptyset$.
\end{proof}
\begin{thm}%\label{thm:}
Let $A$ be a connected subspace of $X$. If $A\subset
B\subset\overline{A}$, then $B$ is connected.
\end{thm}
\begin{proof}
Assume for contradiction that $B$ has a seperation. More
precisely, there is a pair of nonempty disjoint open sets $C$,
$D$ whose union is $B$. Since $A\subset B$, and $A$ is connected,
then $A$ is contained entirely in $C$ or it is contained entirely
in $D$. Suppose it is contained entirely in $C$, then
\begin{equation}%\label{eq:}
C\cup D\subset \overline{A}\;\Rightarrow\; D \subset A'
\end{equation}
where $A'$ is the set of limit points of $A$. Recall a limit
point $x$ is such that 
\begin{equation}%\label{eq:}
x\in\overline{A-\{x\}}.
\end{equation}
But observe that
\begin{equation}%\label{eq:}
x\in\overline{A-\{x\}}\subset \overline{C-\{x\}}\subset\overline{C}
\end{equation}
but this implies that
\begin{equation}%\label{eq:}
A'\subset\overline{C}\; \Rightarrow\; B\subset\overline{A}\subset\overline{C}
\end{equation}
or equivalently, $D=\emptyset$ since we previously proven
$D\cap\overline{C}=\emptyset$, which is a contradiction.
\end{proof}
\begin{thm}%\label{thm:}
The image of a connected space under a continuous mapping is connected.
\end{thm}
\begin{proof}
Let $f:X\to Y$ be continuous, $X$ be connected. Assume for
contradiction there is a seperation (i.e. a pair $A,B$ of
nonempty, open, disjoint subsets whose union is $Y$) in
$Y$. Their preimage would be a seperation in $X$, which
contradicts our hypothesis that $X$ is connected. We reject our
assumption that $Y$ has a seperation and conclude it is connected.
\end{proof}
\begin{thm}%\label{thm:}
A finite Cartesian product of connected spaces is connected.
\end{thm}
\begin{proof}
We will do a sort of inductive proof by contradiction.

\noindent\textbf{Base Case $(n=2)$:} Lets consider two connected
spaces $X_1$ and $X_2$. Let
\begin{equation}%\label{eq:}
Y=X_1\times X_2.
\end{equation}
For contradiction, assume $Y$ is disconnected. Then there is a
pair $A$, $B$ of disjoint, nonempty subsets of $Y$ whose union is
$Y$. This means that either $\pi_{1}(A)$, $\pi_{1}(B)$ forms a
seperation of $X_1$ or $\pi_{2}(A)$, $\pi_{2}(B)$ forms a
seperation of $X_2$. But we assumed that $X_i$ ($i=1,2$) was
connected. So we have a contradiction, reject our assumption that
$Y$ is disconnected, and conclude $Y$ is connected.

\noindent\textbf{Inductive Hypothesis (arbitrary $n$):} Suppose
this works for
\begin{equation}%\label{eq:}
Y = X_1\times\cdots\times X_n.
\end{equation}

\noindent\textbf{Inductive Cast $(n+1)$:} Lets consider $X_1$,
$\ldots$, $X_{n+1}$ connected spaces, let
\begin{equation}%\label{eq:}
X = X_1\times\cdots\times X_n
\end{equation} 
and
\begin{equation}%\label{eq:}
Y = X\times X_{n+1}.
\end{equation}
Then this is just the base case, and $Y$ was shown to be connected.
\end{proof}
\begin{defn}%\label{defn:}
A space is ``\define{Totally Disconnected}'' if its only
connected subspaces are one-point sets.
\end{defn}
\begin{thm}%\label{thm:}
If $X$ is a topological space equipped with the discrete
topology, then it is totally disconnected.
\end{thm}
\begin{proof}
Since $X$ has the discrete topology, we see that every subset of
$X$ is open and consequently its compliment (being a subset of
$X$) is open, so every subset is also closed. The only subspaces
of $X$ which has the only subsets be both open and closed would
necessarily be the one-point sets. 

(Observe that an arbitrary set $Y\subset X$ with more than one
element could be partitioned into two subsets, which are both
open and closed. This implies that $Y$ is not connected.)

It then follows by definition that $X$ is totally disconnected.
\end{proof}
\begin{prop}%\label{prop:}
Let $Y\subset X$, let $X$ and $Y$ be connected. If $A$ and $B$
form a seperation of $X-Y$, then $A\cup Y$ and $B\cup Y$ are
connected.
\end{prop}
\begin{proof} We will prove that if $A\cup Y$ has a seperation, it
  means that $X$ has a seperation in the sense of a seperation in
  a subspace (it is more general this way -- if $X$ isn't a
  subspace, it's still a kosher seperation).
\noindent\begin{enumerate}
\item Let $A\cup B=X-Y$, assume for contradiction that
  $Z:=A\cup Y$ has a seperation $C\cup D=Z$.
\item We will show that $B$, $C\cup D$ form a seperation of
  $X$. Observe first that $C\cup D=A\cup Y$.
\item We see that $\overline{B}\cap A=B\cap\overline{A}=\emptyset$ since $A,B$ form
  a seperation of $X-Y$.
\item We see that $Y$ is closed since $X-Y=A\cup B$ is the union
  of open sets (thus open).
\item We see that $\overline{Y}\cap B=Y\cap B=\emptyset$.
\item We see that $\overline{B}\subset X-Y$ so $\overline{B}\cap
  Y=\emptyset$.
\item Observe that 
\begin{equation}%\label{eq:}
\overline{Z}\cap B=(\overline{A\cup Y})\cap B = (\overline{A}\cup\overline{Y})\cap B=\emptyset
\end{equation}
and similarly
\begin{equation}%\label{eq:}
Z\cap\overline{B}=(A\cup Y)\cap\overline{B}=\emptyset.
\end{equation}
\item We see that $B$ and $Z=A\cup Y$ form a seperation of
  $X$, implying $X$ is disconnected. This is a contradiction. We
  have to reject the assumption that $Z$ has a seperation.
\end{enumerate}
A similar argument holds for $B\cup Y$ being connected.
\end{proof}
