%%
%% main.tex
%% 
%% Made by Alex Nelson
%% Login   <alex@tomato>
%% 
%% Started on  Sun May 31 10:38:38 2009 Alex Nelson
%% Last update Sun May 31 10:38:38 2009 Alex Nelson
%%
\documentclass{amsart}
%\usepackage{amsmidx}
%\makeindex{main}
\usepackage{manfnt}
% \usepackage{hyperref}
\usepackage{url}
\usepackage{amsthm}
\usepackage{amsmath}
\usepackage{amsthm}
\usepackage{amssymb}
\usepackage{amsfonts}
\usepackage{amscd}
\usepackage{graphicx}
\DeclareGraphicsRule{*}{mps}{*}{}
\usepackage{mathrsfs} % for mathscr used in defining a basis
\usepackage{xmpincl}
\includexmp{license}
\numberwithin{equation}{section}

\theoremstyle{definition}
\newtheorem{defn}{Definition}[section]
\newtheorem{thm}[defn]{Theorem}
\newtheorem{rmk}[defn]{Remark}
\newtheorem{lem}[defn]{Lemma}
\newtheorem{cor}[defn]{Corollary}
\newtheorem{ex}[defn]{Example}
\newtheorem{prop}[defn]{Proposition}
\newtheorem{sch}[defn]{Scholium}
\newtheorem{axm}[defn]{Axiom}
\newtheorem*{prob}{Problem}

\def\re{\operatorname{Re}}
\def\tr{\operatorname{Tr}}
\def\<{\langle}
\def\>{\rangle}

%%
% This macro header is what controls the ``dangerous bend''
% paragraph
%%
\def\rd{\noindent\begingroup\hangindent=3pc\hangafter=-2\def\par{\endgraf\endgroup}\hbox
  to0pt{\hskip-\hangindent\dbend\hfill}\ignorespaces}
%%
% This command allows you to write stuff in small font size and
% use the
% bourbaki ``dangerous bend'' so it's great when you want to
% ramble on 
% about some extra stuff!
%%
\newcommand{\danger}[1] {\rd{\small {#1}}}

%%
% This macro header is what controls the ``dangerous bend''
% paragraph
%%
\def\ddbend{\dbend\kern1pt\dbend}

\def\rdd{\noindent\begingroup\hangindent=4pc\hangafter=-2\def\par{\endgraf\endgroup}\hbox
  to0pt{\hskip-\hangindent\ddbend\hfill}\ignorespaces}

\newcommand{\ddanger}[1] {\rdd{\small {#1}}}

%\newcommand{\define}[1]{\textbf{#1}\index{main}{\capitalize{#1}}}
%\newcommand{\define}[1] {\textbf{#1}\index{main}{\MakeUppercase #1}}
\newcommand{\define}[1] {\textbf{#1}\index{#1}}

\title{Notes on Topology}
\date{May 31, 2009}
\email{pqnelson@gmail.com}
\author{Alex Nelson}
\begin{document}
\begin{abstract}
Topology is the study of boundaries. We usually think of an ``open
set'' as not containing its boundary, and a ``closed set'' as one that
contains its boundary. We generalize these notions in topology
suitably. 
\end{abstract}
\maketitle
\tableofcontents
%\chapter{Topology}
\section{Introduction}
%%
%% intro.tex
%% 
%% Made by Alex Nelson
%% Login   <alex@tomato>
%% 
%% Started on  Wed Jun  3 16:30:09 2009 Alex Nelson
%% Last update Wed Jun  3 16:30:09 2009 Alex Nelson
%%
Recall for classical mechanics, the Harmonic oscillator potential
is
\begin{equation}%\label{eq:}
V(x) = \frac{1}{2}kx^2 = \frac{1}{2}m\omega^{2}x^{2}
\end{equation}
where $\omega$ is the angular velocity. We plug this into
Schrodinger's equation
\begin{equation}%\label{eq:}
\left[\frac{-\hbar^2}{2m}\frac{\partial^{2}}{\partial x^{2}}+\frac{1}{2}kx^{2}\right]|\psi\>=E|\psi\>.
\end{equation}
How to solve this? Well, there are two ways: the smart way and
the stupid way. We'll do it the smart way.

We introduce a change of variable\footnote{Note that this is done
  \emph{classically}, that is \emph{before} quantization. After
  quantization changing coordinates is always a fuzzy
  subject. These two steps are done tacitly in most derivations,
  but it should be known in the back of one's mind what's going on.}:
\begin{equation}%\label{eq:}
Q = \sqrt{\frac{m\omega}{2\hbar}},\qquad
\widehat{P}=\frac{\partial}{\partial Q}=\sqrt{\frac{-1}{2m\omega\hbar}}\widehat{p}.
\end{equation}
Note these are dimensionless and simplify computations
significantly. We can factor the Hamiltonian, since in these new
variables we have
\begin{equation}%\label{eq:}
\hbar\omega\left[Q^{2}-\frac{\partial^2}{\partial Q^2}\right] = \widehat{H}.
\end{equation}
We want to use the coordinates
\begin{equation}%\label{eq:}
a=\left[Q+\frac{\partial}{\partial Q}\right],\qquad a^{\dag}=\left[Q-\frac{\partial}{\partial Q}\right]
\end{equation}
which we call ``\define{annihilation and creation operators}''
respectively.

Observe that
\begin{subequations}
\begin{align}
2a^{\dag}a &= \left[Q+\frac{\partial}{\partial
    Q}\right]\left[Q-\frac{\partial}{\partial Q}\right]\\
&= Q^2 + \frac{\partial}{\partial Q} Q -
Q\frac{\partial}{\partial Q} - \frac{\partial^{2}}{\partial
  Q^{2}}\\
&= Q^{2}-\frac{\partial^2}{\partial Q^2} + [\widehat{P},Q].
\end{align}
\end{subequations}
All computation is by definition and substitution. Nothing too
fancy so far. We can now write the Hamiltonian operator as
\begin{equation}%\label{eq:}
\widehat{H} = \hbar\omega\left(a^{\dag}a+\frac{1}{2}[Q,\widehat{P}]\right)
\end{equation}
Observe that
\begin{equation}%\label{eq:}
[Q,\widehat{P}] = 1
\end{equation}
since $Q$ and $\widehat{P}$ are the dimensionless counterparts to
$x$ and $\widehat{p}$ which implies we set $\hbar\to1$. We
end up with the form of the Hamiltonian operator
\begin{equation}%\label{eq:}
\widehat{H} = \hbar\omega\left(a^{\dag}a+\frac{1}{2}\right).
\end{equation}
It would then be logical to investigate how these creation and
annihilation operators behave.

\part{Act I: Topological Spaces}
\section{Properties of Sets}
%%
%% sets.tex
%% 
%% Made by Alex Nelson
%% Login   <alex@tomato>
%% 
%% Started on  Sun May 31 13:51:53 2009 Alex Nelson
%% Last update Sun May 31 13:51:53 2009 Alex Nelson
%%

Just a review of a few properties of sets, and some basic
approaches to proofs.

\begin{prop}\label{prop:emptySets}
  For any set $A$
\begin{enumerate}
\item The empty set is a subset of $A$:
\begin{equation*}
  \emptyset\subseteq A,\qquad\forall A.
\end{equation*}
\item The union of $A$ with the empty set is $A$
\begin{equation*}
  A\cup\emptyset=A,\qquad\forall A.
\end{equation*}
\item The intersection of $A$ with the empty set is the empty et
\begin{equation*}%\label{eq:}
  A\cap\emptyset=\emptyset,\qquad\forall A.
\end{equation*}
\item The Cartesian product of $A$ and the empty set is empty
\begin{equation*}%\label{eq:}
  A\times\emptyset=\emptyset,\qquad\forall A.
\end{equation*}
\item The only subset of the empty set is itself
\begin{equation*}%\label{eq:}
  A\subseteq\emptyset\Rightarrow A=\emptyset,\qquad\forall A
\end{equation*}
\item The power set of the empty set is a set containing only the
  empty set
\begin{equation*}%\label{eq:}
  2^{\emptyset}=\{\emptyset\}
\end{equation*}
\item The empty set has cardinality zero (it has zero
  elements). Moreover, the empty set is finite
\begin{equation*}%\label{eq:}
  |\emptyset|=0.
\end{equation*}
\end{enumerate}
\end{prop}

Given some statement
\begin{equation}%\label{eq:}
\text{If }P,\text{ then }Q.
\end{equation}
its \textbf{contrapositive} is 
\begin{equation}%\label{eq:}
\text{If }Q\text{ is not true, then }P\text{ is not true.} 
\end{equation}
Its \textbf{converse} is 
\begin{equation}%\label{eq:}
\text{If }P\text{, then }Q.
\end{equation}
Sometimes it's easier to prove the converse than to prove a given
proposition.

To show that two sets $A, B$ are equal, we need to show $A\subset
B$ and $B\subset A$.

\section{Topological Spaces}
%%% topology.tex --- 
%% 
%% Filename: topology.tex
%% Description: 
%% Author: alex
%% Maintainer: 
%% Created: Sat Jan 30 10:05:43 2016 (-0800)
%% Version: 
%% Package-Requires: ()
%% Last-Updated: 
%%           By: 
%%     Update #: 0
%% URL: 
%% Doc URL: 

\documentclass{article}
\usepackage{chunk}
\usepackage[swapnumbers]{notebk}
\makeatletter
\newcommand*{\dupcntr}[2]{%
    \expandafter\let\csname c@#1\expandafter\endcsname\csname c@#2\endcsname
}
\dupcntr{chunk@ctr}{thm}
\makeatother

\usepackage{notation}

\usepackage{sproof}
\pdfinfo{/CreationDate (D:20160130100543)}

\title{Notes on Topology}
\date{January 30, 2016}
\begin{document}
\maketitle
\tableofcontents

%%
%% intro.tex
%% 
%% Made by Alex Nelson
%% Login   <alex@tomato>
%% 
%% Started on  Wed Jun  3 16:30:09 2009 Alex Nelson
%% Last update Wed Jun  3 16:30:09 2009 Alex Nelson
%%
Recall for classical mechanics, the Harmonic oscillator potential
is
\begin{equation}%\label{eq:}
V(x) = \frac{1}{2}kx^2 = \frac{1}{2}m\omega^{2}x^{2}
\end{equation}
where $\omega$ is the angular velocity. We plug this into
Schrodinger's equation
\begin{equation}%\label{eq:}
\left[\frac{-\hbar^2}{2m}\frac{\partial^{2}}{\partial x^{2}}+\frac{1}{2}kx^{2}\right]|\psi\>=E|\psi\>.
\end{equation}
How to solve this? Well, there are two ways: the smart way and
the stupid way. We'll do it the smart way.

We introduce a change of variable\footnote{Note that this is done
  \emph{classically}, that is \emph{before} quantization. After
  quantization changing coordinates is always a fuzzy
  subject. These two steps are done tacitly in most derivations,
  but it should be known in the back of one's mind what's going on.}:
\begin{equation}%\label{eq:}
Q = \sqrt{\frac{m\omega}{2\hbar}},\qquad
\widehat{P}=\frac{\partial}{\partial Q}=\sqrt{\frac{-1}{2m\omega\hbar}}\widehat{p}.
\end{equation}
Note these are dimensionless and simplify computations
significantly. We can factor the Hamiltonian, since in these new
variables we have
\begin{equation}%\label{eq:}
\hbar\omega\left[Q^{2}-\frac{\partial^2}{\partial Q^2}\right] = \widehat{H}.
\end{equation}
We want to use the coordinates
\begin{equation}%\label{eq:}
a=\left[Q+\frac{\partial}{\partial Q}\right],\qquad a^{\dag}=\left[Q-\frac{\partial}{\partial Q}\right]
\end{equation}
which we call ``\define{annihilation and creation operators}''
respectively.

Observe that
\begin{subequations}
\begin{align}
2a^{\dag}a &= \left[Q+\frac{\partial}{\partial
    Q}\right]\left[Q-\frac{\partial}{\partial Q}\right]\\
&= Q^2 + \frac{\partial}{\partial Q} Q -
Q\frac{\partial}{\partial Q} - \frac{\partial^{2}}{\partial
  Q^{2}}\\
&= Q^{2}-\frac{\partial^2}{\partial Q^2} + [\widehat{P},Q].
\end{align}
\end{subequations}
All computation is by definition and substitution. Nothing too
fancy so far. We can now write the Hamiltonian operator as
\begin{equation}%\label{eq:}
\widehat{H} = \hbar\omega\left(a^{\dag}a+\frac{1}{2}[Q,\widehat{P}]\right)
\end{equation}
Observe that
\begin{equation}%\label{eq:}
[Q,\widehat{P}] = 1
\end{equation}
since $Q$ and $\widehat{P}$ are the dimensionless counterparts to
$x$ and $\widehat{p}$ which implies we set $\hbar\to1$. We
end up with the form of the Hamiltonian operator
\begin{equation}%\label{eq:}
\widehat{H} = \hbar\omega\left(a^{\dag}a+\frac{1}{2}\right).
\end{equation}
It would then be logical to investigate how these creation and
annihilation operators behave.


\end{document}

\section{Basis For Topological Spaces}
\section{Linear Dependence and Bases}\label{section:basis}

\M
We want to address the question of whether this is a ``best'' spanning
set for a subspace, and we saw in some sense ``redundant elements''
should be avoided. If $\vec{s}_{1}\in S$ and $\vec{s}_{2}\in S$, then it
would be redundant to have $\vec{s}_{1}+\vec{s}_{2}\in S$. Let us try to
formalize this intuition of ``redundant combinations''.

\begin{definition}\label{defn:basis:linearly-dependent}
Let $V$ be a vector space, let $\vec{v}_{1}$, \dots, $v_{n}\in V$ be
nonzero vectors $\vec{v}_{j}\neq\vec{0}$ for $j=1,\dots,n$.
We call them \define{Linearly Dependent} if there are coefficients (not
all zero) $c_{1},\dots,c_{n}\in\RR$ such that
\begin{equation}
c_{1}\vec{v}_{1} + c_{2}\vec{v}_{2}+\cdots+c_{n}\vec{v}_{n}=\vec{0}.
\end{equation}
If the only solution for this is $c_{1}=c_{2}=\cdots=c_{n}=0$ for all
coefficients to be zero, then we call the vectors \define{Linearly Independent}.
\end{definition}

\begin{example}
  In $\RR^{2}$, consider the vectors
  \begin{equation}
\vec{v}_{1} = \begin{pmatrix} 1\\0 \end{pmatrix},
\vec{v}_{2} = \begin{pmatrix} 0\\1 \end{pmatrix},
\vec{v}_{3} = \begin{pmatrix} 1\\1 \end{pmatrix},
\vec{v}_{4} = \begin{pmatrix} 1\\-1 \end{pmatrix}.
  \end{equation}
  Any three or more vectors from this list are linearly dependent since
  $\vec{v}_{3}=\vec{v}_{1}+\vec{v}_{2}$ and
  $\vec{v}_{4}=\vec{v}_{1}-\vec{v}_{2}$. But
  any two vectors from this list are linearly independent.
\end{example}

\begin{theorem}[Criterion for Linear Dependence]
A set of nonzero vectors $\{\vec{v}_{1},\dots,\vec{v}_{n}\}$ is linearly
dependent if and only if at least one of the vectors $\vec{v}_{k}$ is
expressible as a linear combination of the others
\begin{equation}
\vec{v}_{k} = \sum^{n}_{\substack{j=1\\j\neq k}}c_{j}\vec{v}_{j} =c_{1}\vec{v}_{1} + \cdots + c_{k-1}\vec{v}_{k-1} + c_{k+1}\vec{v}_{k+1} +
  \cdots + c_{n}\vec{v}_{n},
\end{equation}
where not all coefficients $c_{j}\in\RR$ are zero.
\end{theorem}

\begin{proof}
  $(\implies)$ Assume the vectors $\vec{v}_{1}$, \dots, $\vec{v}_{n}$
  are linearly dependent. Then by Definition~\ref{defn:basis:linearly-dependent},
  there are coefficients $c_{1}$, \dots, $c_{n}$ (not all zero) such that
\begin{equation}
c_{1}\vec{v}_{1} + c_{2}\vec{v}_{2}+\cdots+c_{n}\vec{v}_{n}=\vec{0}.
\end{equation}
Let $k$ be the last index for which $c_{k}\neq0$ (so for indices $\ell$
such that  $k<\ell\leq n$, then $c_{\ell}=0$). Then we can subtract
$c_{k}\vec{v}_{k}$ from both sides to get
\begin{equation}
c_{1}\vec{v}_{1} + c_{2}\vec{v}_{2}+\cdots+c_{k}\vec{v}_{k}-c_{k}\vec{v}_{k}=\vec{0}-c_{k}\vec{v}_{k},
\end{equation}
and dividing both sides by $-c_{k}$ gives us $\vec{v}_{k}$ as a linear
combination of $\vec{v}_{1}$, \dots, $\vec{v}_{k-1}$. This concludes the
forward direction of the proof.

  $(\impliedby)$ Assume there exists a vector $\vec{v}_{k}$ such that we
  can write it as a linear combination of the remaining vectors
\begin{equation}
\begin{split}
  \vec{v}_{k} &= \sum^{n}_{\substack{j=1\\j\neq k}}c_{j}\vec{v}_{j}\\
  &=c_{1}\vec{v}_{1} + \cdots + c_{k-1}\vec{v}_{k-1} + c_{k+1}\vec{v}_{k+1} +
  \cdots + c_{n}\vec{v}_{n},
\end{split}
\end{equation}
where not all $c_{j}\in\RR$ are zero. Then subtracting $\vec{v}_{k}$
from both sides gives us
\begin{equation}
\vec{0} = c_{1}\vec{v}_{1} + \cdots + c_{k-1}\vec{v}_{k-1} - \vec{v}_{k} + c_{k+1}\vec{v}_{k+1} +
  \cdots + c_{n}\vec{v}_{n}.
\end{equation}
Then by Definition~\ref{defn:basis:linearly-dependent}, since not all
coefficients $c_{j}$ are zero, we have the vectors are linearly dependent.
\end{proof}

\begin{theorem}[Nonzero determinant iff columns are linearly independent]
Let $\{\vec{v}_{1},\dots,\vec{v}_{n}\}\subset\RR^{n}$ be a list of $n$
distinct $n$-vectors, and
\begin{equation}
\mat{M} = (\vec{v}_{1}|\dots|\vec{v}_{n})
\end{equation}
be a matrix whose columns are the given $n$ column vectors.
Then $\det(\mat{M})\neq0$ if and only if $\{\vec{v}_{1},\dots,\vec{v}_{n}\}$
are linearly independent.
\end{theorem}

\begin{proof}
$(\implies)$ Assume $\det(\mat{M})\neq0$. Then $\mat{M}$ is invertible
  (by Theorem~\ref{thm:determinant:singular-matrices-have-zero-det}).
Then $\mat{M}\vec{x}=\vec{0}$ has a unique solution (\S\ref{par:matrix-algebra:solving-systems-of-equations}), namely $\vec{x}=\vec{0}$.
This is equivalent to saying
\begin{equation}
x_{1}\vec{v}_{1} + x_{2}\vec{v}_{2} + \cdots + x_{n}\vec{v}_{n} = 0
\end{equation}
implies $x_{1}=x_{2}=\cdots=x_{n}=0$. But by Definition~\ref{defn:basis:linearly-dependent},
this is precisely the condition for $\vec{v}_{1}$, \dots, $\vec{v}_{n}$
being linearly independent.

$(\impliedby)$ Assume $\vec{v}_{1}$, \dots, $\vec{v}_{n}$ are linearly
independent. Then by Definition~\ref{defn:basis:linearly-dependent}, the
only solution to
\begin{equation}
c_{1}\vec{v}_{1} + \cdots + c_{n}\vec{v}_{n} = \vec{0}
\end{equation}
is $c_{1}=\cdots=c_{n}=0$. In matrix form, if $\vec{x}=(c_{1},\dots,c_{n})$
is a column $n$-vector, then
\begin{equation}
\mat{M}\vec{x} = \vec{0}
\end{equation}
has $\vec{x}=\vec{0}$ be its only solution. This is true if and only if
$\mat{M}$ is invertible. But $\mat{M}$ is invertible if and only if
$\det(\mat{M})\neq0$. And by our assumption, $\vec{x}=\vec{0}$ is the
only solution, hence the result.
\end{proof}

\begin{corollary}\label{cor:basis:invertible-matrix-iff-columns-are-linearly-independent}
An $n\times n$ matrix $\mat{M}$ is invertible if and only if its columns are
linearly independent vectors.
\end{corollary}
\begin{proof}
We know from the previous theorem $\mat{M}$ has nonzero determinant if
and only if its columns are linearly independent vectors. We know from Theorem~\ref{thm:determinant:singular-matrices-have-zero-det}
$\mat{M}$ has a nonzero determinant if and only if $\mat{M}$ is
invertible.
Therefore, we know $\mat{M}$ is invertible if and only if its columns
are linearly independent vectors.
\end{proof}

\begin{definition}
Let $V$ be a real vector space and $B$ a set of vectors from $V$ such
that
\begin{enumerate}
\item it spans $V$: $\Span(B)=V$
\item there is no $A\subset B$ such that $\Span(A)=V$.
\end{enumerate}
Then we call $B$ a \define{Basis} of $V$.
\end{definition}

\begin{remark}[Need to prove existence of basis]
We have just defined a word, ``basis'', but we have no guarantee that a
basis will exist. This must be proven. The proof is not enlightening,
and requires the axiom of choice (pick some nonzero vector, now pick
another which is linearly independent of the first, keep picking
linearly independent vectors --- how? By the axiom of choice, it's
always possible \emph{somehow}; then show an arbitrary vector may be
written as a linear combination of our collection of chosen vectors).
\end{remark}

\begin{example}
  In $\RR^{2}$, the vectors
  \begin{equation}
\vec{z} = \begin{pmatrix}1\\ 1
\end{pmatrix},\quad\mbox{and}\quad\bar{\vec{z}} = \begin{pmatrix}1\\ -1
\end{pmatrix}.
  \end{equation}
  Then $\{\vec{z}, \bar{\vec{z}}\}$ form a basis for $\RR^{2}$.
\end{example}
\begin{proof}
  We need to show
  \begin{enumerate}
  \item $\Span\{\vec{z}, \bar{\vec{z}}\}=\RR^{2}$
  \item there is no $A\subset\{\vec{z}, \bar{\vec{z}}\}$ such that $\Span(A)=\RR^{2}$.
  \end{enumerate}
  The first claim may be proven by picking any element
  $\vec{v}\in\RR^{2}$, then showing it may be written as a linear
  combination of $\vec{z}$ and $\bar{\vec{z}}$. We see, if
  \begin{equation}
\vec{v} = \begin{pmatrix}v_{1}\\v_{2}
\end{pmatrix},
  \end{equation}
  then
  \begin{calculation}
    \displaystyle\frac{v_{1}+v_{2}}{2}\vec{z} + \frac{v_{1}-v_{2}}{2}\bar{\vec{z}}
\step{unfolding the definition of $\vec{z}$, $\bar{\vec{z}}$}
    \displaystyle\frac{v_{1}+v_{2}}{2}\begin{pmatrix}1\\1
    \end{pmatrix}
    + \frac{v_{1}-v_{2}}{2}\begin{pmatrix}1\\-1
    \end{pmatrix}
\step{scalar multiplication}
    \displaystyle\frac{1}{2}\begin{pmatrix}v_{1}+v_{2}\\v_{1}+v_{2}
    \end{pmatrix}
    + \frac{1}{2}\begin{pmatrix}v_{1}-v_{2}\\-v_{1}+v_{2}
    \end{pmatrix}
\step{distributivity}
    \displaystyle\frac{1}{2}\left[\begin{pmatrix}v_{1}+v_{2}\\v_{1}+v_{2}
    \end{pmatrix}
    + \begin{pmatrix}v_{1}-v_{2}\\-v_{1}+v_{2}
    \end{pmatrix}\right]
\step{vector addition}
    \displaystyle\frac{1}{2}\begin{pmatrix}(v_{1}+v_{2})+(v_{1}-v_{2})\\(v_{1}+v_{2})+(-v_{1}+v_{2})
    \end{pmatrix}
\step{arithmetic}
    \displaystyle\frac{1}{2}\begin{pmatrix}2v_{1}\\2v_{2}
      \end{pmatrix}
\step{scalar multiplication}
    \displaystyle\begin{pmatrix}v_{1}\\v_{2}
      \end{pmatrix} = \vec{v}
  \end{calculation}
  as desired. Hence any element of $\RR^{2}$ may be written as a linear
  combination of $\vec{z}$ and $\bar{\vec{z}}$, hence $\Span(\{\vec{z},\bar{\vec{z}}\})=\RR^{2}$.

  As to the second claim, there is no $A\subset\{\vec{z},\bar{\vec{z}}\}$,
  suppose there were such an $A$. Then either $A=\{\vec{z}\}$ or
  $A=\{\bar{\vec{z}}\}$. Pick $\vec{v}\in\{\vec{z},\bar{\vec{z}}\}$ but
  $\vec{v}\notin A$. Then we claim $\vec{v}\notin\Span(A)$.

  It suffices to show $\vec{z}$ is not a multiple of $\bar{\vec{z}}$
  (which corresponds to $A=\{\bar{\vec{z}}\}$ --- in the other case, it
  boils down to the same proof). If $\vec{z}$ were a multiple of
  $\bar{\vec{z}}$, then there is a $c\in\RR$ nonzero such that
  \begin{equation}
c\begin{pmatrix}1\\1
\end{pmatrix} = \begin{pmatrix}1\\-1
\end{pmatrix}
  \end{equation}
  This is a system of 2 equations in 1 unknown:
  \begin{equation}
c=1,\quad\mbox{and}\quad c=-1.
  \end{equation}
  But this is impossible. So $\vec{z}$ cannot be a multiple of
  $\bar{\vec{z}}$, which means $\vec{z}\notin\Span(\{\bar{\vec{z}}\})$.
  The same reasoning shows $\bar{\vec{z}}$ is not a multiple of
  $\vec{z}$, which means $\bar{\vec{z}}\notin\Span(\{\vec{z}\})$.

  Hence there is no $A\subset\{\vec{z},\bar{\vec{z}}\}$ such that $\Span(A)=\RR^{2}$.
\end{proof}

\begin{example}\label{ex:basis:canonical-basis}
Let $\vec{e}_{j}\in\RR^{n}$ have $1$ in its $j^{\text{th}}$ component
and $0$ in all other components. Then the set
$\{\vec{e}_{1},\dots,\vec{e}_{n}\}$ forms a basis of $\RR^{n}$ and is
called its \define{Canonical Basis}.
\end{example}

\N{Vector spaces have many bases}
We see that a vector space may have more than one basis. In fact, they
will have many different possible bases (plural of basis). We saw one
basis in $\RR^{2}$ given by $\vec{z}=(1,1)$ and
$\bar{\vec{z}}=(1,-1)$. We also see there is the canonical basis for
$\RR^{2}$, which is different from the first basis.

The moral of the story is that we may have many inequivalent bases for
any given vector space.

\begin{lemma}
Let $V$ be a real vector space.
Let $B=\{\vec{v}_{1},\dots,\vec{v}_{n}\}$ form a basis for $V$.
Let $T=\{\vec{w}_{1},\dots,\vec{w}_{m}\}$ be a set of linearly
independent vectors from $V$.
Then $m\leq n$ (i.e., $|T|\leq|B|$).
\end{lemma}

\begin{proof}
Since $T$ consists of linearly independent vectors, we can write
$\vec{w}_{m}$ as a linear combination of basis vectors
\begin{equation}
\vec{w}_{m} = c^{(m)}_{1}\vec{b}_{1}+c^{(m)}_{2}\vec{b}_{2}+\cdots+c^{(m)}_{n}\vec{b}_{n}.
\end{equation}
We can reindex the basis vectors such that $c^{(m)}_{1}\neq0$. In that
case, we ``swap out'' $\vec{b}_{1}$ for $\vec{w}_{m}$ since we can write
$\vec{b}_{1}$ as a linear combination:
\begin{equation}
\vec{b}_{1}=\frac{1}{c^{(m)}_{1}}\vec{w}_{m} - \frac{c^{(m)}_{2}\vec{b}_{2}+\cdots+c^{(m)}_{n}\vec{b}_{n}}{c^{(m)}_{1}}.
\end{equation}
This gives us a new set of basis vectors $B_{1}$, and we consider $T_{1}=T\setminus\{\vec{w}_{m}\}$
the collection of elements from $T$ which are not $\vec{w}_{m}$. In
particular, there are $m-1$ elements of $T_{1}$.

We can reiterate this step, swapping one element out of $T_{1}$ and
putting it into $B_{1}$ (and throwing away an element from $B_{1}$ which
has been replaced) to produce a new basis $B_{2}$. We produce $T_{2}$
from $T_{1}$ by taking the remaining elements of $T_{1}$ which are not
in $B_{2}$ into $T_{2}$. We see $T_{2}$ has $m-2$ elemeents.

Eventually one of two possibilities occurs:
\begin{enumerate}
\item We'll reach $B_{k}$ which no longer has any original basis
  elements from $B$ in it --- they are disjoint $B\cap B_{k}=\emptyset$.
  But this would imply there are elements in $T_{k}$ which cannot be
  written as a linear combination of the basis, which is a
  contradiction; or
\item We'll exhaust $T_{k}$ and have no more elements from $T$ to add to $B_{k}$.
\end{enumerate}
We iterate this until we get to $B_{m}$ and $T_{m}$, because $T_{m+1}$
will be empty.
\end{proof}

\begin{theorem}[Any two bases have same number of elements]
Let $V$ be a real vector space. Suppose there exists at least one basis
$B$ for $V$, and suppose $B$ has finitely many element.
Then any two bases for $V$ have the same number of elements as each other.
\end{theorem}

\begin{proof}
Let $B_{1}$, $B_{2}$ be any two bases for $V$. Let $m=|B_{1}|$ and $n=|B_{2}|$.
We claim
\begin{enumerate}
\item $m\leq n$ by the previous lemma, and
\item $n\leq m$ by the previous lemma.
\end{enumerate}
Hence $m=n$.
\end{proof}

\begin{definition}
Let $V$ be a real vector space, let $B$ be a basis for $V$. If $B$ has
finitely many elements, then we say $V$ is
\define{Finite-Dimensional}. In that case, we call the number of vectors
in $B$ the \define{Dimension} of $V$.
\end{definition}

\begin{remark}
Reasoning about infinite-dimensional spaces can be tricky. We have
already seen one example, $\RR[x]$ the space of polynomials. The linear
algebra of inifnite-dimensional spaces usually goes by the name
``functional analysis''. A particularly friendly subfield is ``Fourier
analysis'', where many intuitions from finite-dimensional linear algebra
carries over.
\end{remark}

\begin{definition}
Let $V$ be a finite-dimensional vector space, let
$B=\{\vec{f}_{1},\dots,\vec{f}_{n}\}$ be a basis for $V$.
We define an \define{Ordered Basis} to be a tuple $(\vec{f}_{1},\dots,\vec{f}_{n})$.

Futhermore, if the vectors $\vec{f}_{1}$, \dots, $\vec{f}_{n}$ form a
basis such that
\begin{equation}
\vec{f}_{i}\cdot\vec{f}_{j}=0\quad\mbox{if }i\neq j,
\end{equation}
then we call it an \define{Orthogonal Basis} for $V$. If, even further,
we have
\begin{equation}
\vec{f}_{i}\cdot\vec{f}_{j}=\delta_{i,j}=\begin{cases}0&\mbox{if }i\neq j\\
1 & \mbox{if }i=j
\end{cases}
\end{equation}
then we call it an \define{Orthonormal Basis}.
\end{definition}

\begin{remark}
This might seem silly (and, I guess, it is), but sets are not
ordered. There are times when we will want to specifically note the
order of basis vectors. The ordering may be arbitrary (for example, an
accidental artifact induced by the indexing), but important.
\end{remark}

\begin{remark}
In differential geometry, we sometimes see the term ``frame'' used for
an ordered basis.
\end{remark}

\subsection{Coordinates Relative to a Basis}

\M
Recall, when we began discussing vector spaces like $\RR^{1}$ and
$\RR^{2}$ (back in section~\ref{section:vectors-in-r-n}), we began by
drawing a line, picking a point $O$ (calling it the origin), and then
picking another point $P$. We identified the oriented line segment
$\overrightarrow{OP}$ as a unit vector. Then any other point $Q$ could
be identified as a multiple of $\overrightarrow{OP}$ such that
$\|\overrightarrow{OQ}\|/\|\overrightarrow{OP}\|=|x|\in\RR$ and if $Q$
is not ``in the $P$ direction'', we said
$-|x|\overrightarrow{OP}=\overrightarrow{OQ}$. In this way, we
identified $x$ as the coordinate of $Q$.

\M In $\RR^{2}$, we did the same thing, but we now have two axes. Any
point $S$ on the plane could be identified by a pair of real numbers
$(x,y)\in\RR^{2}$ by similar means.

\M
In general, this is what happens with a vector in a vector space
relative to a basis. We obtain ``coordinates'' for the vector, enabling
us to write our vector as a linear combination of basis vectors. This is
a crucial point, because an $n$-dimensional real vector space $V$ \emph{is not}
\emph{identical} to $\RR^{n}$ --- but by choosing a basis, we can
identify vectors in $V$ with tuples of real numbers in $\RR^{n}$, namely
their coordinates. This should be familiar, we've been working with
$n\times1$ matrices and calling them vectors (but really, they're just
inhabitants of $\RR^{n}$). We have been stretching the truth all this
time.

\begin{ddanger}
This is a really critical point to appreciate: vectors are ``points in
the plane'' \emph{not} an $n\times 1$ matrix. \textbf{Vectors are not matrices}.
But we can turn a vector in a finite-dimensional vector space
\emph{into} a column $n\times1$ matrix, and vice-versa, \emph{given some basis}
for $V$.
\end{ddanger}

Let us try to make things concrete.

\begin{definition}\label{defn:basis:coordinates}
Let $V$ be a finite-dimensional vector space, let
$B=(\vec{f}_{1},\dots,\vec{f}_{n})$ be an ordered basis for $V$, and let
$\vec{v}\in V$ be an arbitrary vector. Then we call the coefficients
$\lambda_{1},\dots,\lambda_{n}\in\RR$ in
\begin{equation}
\vec{v} = \lambda_{1}\vec{f}_{1} + \cdots + \lambda_{n}\vec{f}_{n}
\end{equation}
the \define{Coordinates} of $\vec{v}$ over $B$ (or ``relative to $B$'').
We may write
\begin{equation}
[\vec{v}]_{B} = \begin{pmatrix}\lambda_{1}\\\vdots\\\lambda_{n}
\end{pmatrix}
\end{equation}
for the coordinates of $\vec{v}$ relative to basis $B$.
\end{definition}

\begin{remark}
If we can write $\vec{v} = \lambda_{1}\vec{f}_{1} + \lambda_{2}\vec{f}_{2} + \cdots + \lambda_{n}\vec{f}_{n}$,
then we can identify $\vec{v}$ with the column vector of its coordinates
relative to the $\vec{f}_{j}$:
\begin{equation}
  \vec{v} \mathrel{\mbox{``=''}}
  \begin{pmatrix}\lambda_{1}\\\lambda_{2}\\\vdots\\\lambda_{n}
  \end{pmatrix}.
\end{equation}
The equality is in quotation marks because it's not an equality but an \emph{isomorphism},
a slightly weaker notion of ``the same''.
We can prove (once we formalize the notion of an isomorphism) that any
finite-dimensional real vector space $V$ is isomorphic to $\RR^{n}$ the
collection of column $n$-vectors with real entries; this is done by
identifying a vector with its coordinates relative to some basis. 
But this \emph{does not} mean $V$ is equal to $\RR^{n}$.
\end{remark}

\begin{problem}
Suppose we have a finite-dimensional vector space $V$.
Suppose we have one ordered basis $B=(\vec{e}_{1},\dots,\vec{e}_{n})$
and a distinct ordered basis $B'=(\vec{f}_{1},\dots,\vec{f}_{n})$
which share no vectors --- we have $\vec{f}_{i}\neq\vec{e}_{j}$
for all $i$, $j=1,\dots,n$.

If we have a vector $\vec{v}\in V$ and have found its coordinates
relative to $\vec{f}_{j}$,
\begin{equation}
\vec{v} = \lambda_{1}\vec{f}_{1} + \cdots + \lambda_{n}\vec{f}_{n},
\end{equation}
then how do we transform these into coordinates relative to $\vec{e}_{i}$?
\end{problem}

\subsection{Changing Bases}

\N{Change of Basis Matrix}
Let $V$ be a finite-dimensional real vector space.
Let $B=(\vec{e}_{1},\dots,\vec{e}_{n})$ and
$C=(\vec{f}_{1},\dots,\vec{f}_{n})$ be two ordered bases for $V$.
Then the change of bases from $B$ to $C$ transforms coordinates for
vectors $[\vec{v}]_{B}$ according to the matrix
\begin{equation}
  \mat{M}^{B}_{C} = \left(\begin{array}{c|c|c}
      [\vec{e}_{1}]_{C} & \dots & [\vec{e}_{n}]_{C}
  \end{array}\right),
\end{equation}
so any vector $\vec{v}\in V$ expressed in coordinates $[\vec{v}]_{B}$
relative to $B$ may be expressed in coordinates relative to $C$ as
\begin{equation}
[\vec{v}]_{C} = \mat{M}^{B}_{C}[\vec{v}]_{B}.
\end{equation}
Why would this work? Well, if we expand the right-hand side, we would
find
\begin{equation}
 \left(\begin{array}{c|c|c}
      [\vec{e}_{1}]_{C} & \dots & [\vec{e}_{n}]_{C}\\
  \end{array}\right)\begin{pmatrix}\lambda_{1}\\\vdots\\\lambda_{n}
 \end{pmatrix} =
      \lambda_{1}[\vec{e}_{1}]_{C} + \lambda_{2}[\vec{e}_{2}]_{C} +
      \dots + \lambda_{n} [\vec{e}_{n}]_{C}.
\end{equation}
But since $[\vec{e}_{j}]_{C}$ is a linear combination of the basis
vectors $\vec{f}_{1}$, \dots, $\vec{f}_{n}$, this expands out to form a
linear combination of $\vec{f}_{1}$, \dots, $\vec{f}_{n}$.

\begin{theorem}
If $V$ is a finite dimensional vector space with ordered bases $B$ and
$C$, if $[\mat{M}]^{B}_{C}$ is the change-of-coordinate matrix from $B$
to $C$, then the change-of-coordinate matrix from $C$ back to $B$
(i.e., $[\mat{M}]^{C}_{B}$) satisfies
\begin{equation}
[\mat{M}]^{C}_{B} = ([\mat{M}]^{B}_{C})^{-1}.
\end{equation}
That is to say, they are inverses of each other.
\end{theorem}

This partly motivated the bizarre superscript/subscript convention, it
resembles a fraction.

We can ketch the proof out in (\S\ref{chunk:basis:change-of-basis-matrix-among-orthonormal-bases}).

\N{Relating Coordinates Between Orthonormal Bases}
If $B$ and $C$ are both orthonormal bases for $V$, there is no reason to
believe they consist of the same basis vectors. One way to obtain a
different set of orthonormal basis vectors from $B$ is by an arbitrary
rotation, and possibly reflection about an axis (or about a nonzero
vector). These transformations produce a different basis, but do not
affect the orthonormality of the new basis.

What does the change of coordinates matrix look like between them?

\M\label{chunk:basis:change-of-basis-matrix-among-orthonormal-bases}
We can write out the matrix components explicitly for orthonormal
coordinates $\vec{e}_{1}$, \dots, $\vec{e}_{n}$ and other orthonormal
coordinates $\vec{f}_{1}$, \dots, $\vec{f}_{n}$ as
\begin{align}
  \vec{f}_{1} &= (\vec{f}_{1}\cdot\vec{e}_{1})\vec{e}_{1} + (\vec{f}_{1}\cdot\vec{e}_{2})\vec{e}_{2} + \cdots + (\vec{f}_{1}\cdot\vec{e}_{n})\vec{e}_{n}\\
  \vec{f}_{2} &= (\vec{f}_{2}\cdot\vec{e}_{1})\vec{e}_{1} + (\vec{f}_{2}\cdot\vec{e}_{2})\vec{e}_{2} + \cdots + (\vec{f}_{2}\cdot\vec{e}_{n})\vec{e}_{n}\\
  \vdots &\mathrel{\phantom{=(\vec{f}_{2}\cdot\vec{e}_{1})}}\vdots\quad\phantom{+ (\vec{f}_{1}\cdot\vec{e}_{2})}\vdots\qquad\ddots\qquad\vdots\nonumber\\
  \vec{f}_{n} &= (\vec{f}_{n}\cdot\vec{e}_{1})\vec{e}_{1} + (\vec{f}_{n}\cdot\vec{e}_{2})\vec{e}_{2} + \cdots + (\vec{f}_{n}\cdot\vec{e}_{n})\vec{e}_{n}
\end{align}
The components of the matrix
$[\mat{M}]^{B}_{C}=(\vec{f}_{i}\cdot\vec{e}_{j})$.
What is the inverse of this matrix?

We could do some complicated math, or we could wonder the simpler
problem: what is $[\mat{M}]^{B}_{C}\transpose{([\mat{M}]^{B}_{C})}$?
\begin{calculation}
  ([\mat{M}]^{B}_{C}\transpose{([\mat{M}]^{B}_{C})})_{i,k}
\step{unfold the definition of matrix multiplication}
  \sum^{n}_{j=1}([\mat{M}]^{B}_{C})_{i,j}(\transpose{([\mat{M}]^{B}_{C})})_{j,k}
\step{unfold the definition of $[\mat{M}]^{B}_{C}$ into components}
  \sum^{n}_{j=1}(\transpose{\vec{f}}_{i}\vec{e}_{j})(\transpose{\vec{e}}_{j}\vec{f}_{k})
\step{distributivity}
  \transpose{\vec{f}}_{i}\left(\sum^{n}_{j=1}\vec{e}_{j}\transpose{\vec{e}}_{j}\right)\vec{f}_{k}
\step{matrix multiplication, see Lemma~\ref{lemma:basis:outer-product-of-orthonormal-basis} below}
  \transpose{\vec{f}}_{i}\left(\mat{I}_{n}\right)\vec{f}_{k}
\step{defining property of the identity matrix, associativity of multiplication}
  \transpose{\vec{f}}_{i}\vec{f}_{k}
\step{by definition of orthonormality}
  \delta_{i,k}
\end{calculation}
In other words,
\begin{equation}
[\mat{M}]^{B}_{C}\transpose{([\mat{M}]^{B}_{C})} = \mat{I}.
\end{equation}
This implies
\begin{equation}
([\mat{M}]^{B}_{C})^{-1} = \transpose{([\mat{M}]^{B}_{C})}.
\end{equation}
In other words, the change of basis matrix is an orthogonal matrix
(c.f., Exercise~\ref{xca:matrix-algebra:orthogonal-matrix}).

\begin{lemma}\label{lemma:basis:outer-product-of-orthonormal-basis}
Let $V$ be a finite-dimensional vector space, let $\vec{e}_{1}$, \dots,
$\vec{e}_{n}$ be an orthonormal basis for $V$. Then
\begin{equation}
\sum^{n}_{j=1}\vec{e}_{j}\transpose{\vec{e}}_{j} = \mat{I}_{n}.
\end{equation}
\end{lemma}
\begin{proof}
We consider how this acts on an arbitrary vector $\vec{v}\in V$.
\begin{calculation}
  \left(\sum^{n}_{j=1}\vec{e}_{j}\transpose{\vec{e}}_{j}\right)\vec{v}
\step{expanding $\vec{v}$ in the basis}
  \left(\sum^{n}_{j=1}\vec{e}_{j}\transpose{\vec{e}}_{j}\right)\left(\sum^{n}_{k=1}c_{k}\vec{e}_{k}\right)
\step{by linearity}
  \sum^{n}_{k=1}c_{k}\left(\sum^{n}_{j=1}\vec{e}_{j}\transpose{\vec{e}}_{j}\vec{e}_{k}\right)
\step{by definition of orthonormality}
  \sum^{n}_{k=1}c_{k}\left(\sum^{n}_{j=1}\vec{e}_{j}\delta_{j,k}\right)
\step{unrolling the inner sum over $j$}
  \sum^{n}_{k=1}c_{k}\left(\vec{e}_{1}\delta_{1,k}+\dots+\vec{e}_{k-1}\delta_{k-1,k}+\vec{e}_{k}\delta_{k,k}+\vec{e}_{k+1}\delta_{k+1,k}+\dots+\vec{e}_{n}\delta_{n,k}\right)
\step{definition of $\delta_{j,k}$}
  \sum^{n}_{k=1}c_{k}\left(\vec{e}_{1}0+\dots+\vec{e}_{k-1}0+\vec{e}_{k}1+\vec{e}_{k+1}0+\dots+\vec{e}_{n}0\right)
\step{arithmetic}
  \sum^{n}_{k=1}c_{k}\left(\vec{e}_{k}\right)
\step{multiplication}
  \sum^{n}_{k=1}c_{k}\vec{e}_{k}
\step{since this is the expansion of $\vec{v}$ in the basis $\vec{e}_{k}$, ``undoing'' the first step of this chain of calculations}
  \vec{v}.
\end{calculation}
Since this was for arbitrary $\vec{v}\in V$, it follows that
\begin{equation}
\sum^{n}_{j=1}\vec{e}_{j}\transpose{\vec{e}}_{j} = \mat{I}_{n},
\end{equation}
as desired.
\end{proof}

\subsection{Graham--Schmidt Method}

\N{Puzzle}
If we have an $n$-dimensional vector space $V$ with $n$ linearly
independent [nonzero] vectors $\vec{x}_{1}$, \dots, $\vec{x}_{n}$, then is there
any way to construct an orthonormal basis out of them?

\N{Solution}
We will construct an orthonormal basis, one vector at a time.

The first step is to construct our initial vector
\begin{equation}
\widehat{\vec{v}_{1}} = \frac{\vec{x}_{1}}{\|\vec{x}_{1}\|}.
\end{equation}
This is a unit vector, and now we will use it to start our collection.
We could have easily have chosen $\vec{v}_{1}=\vec{x}_{1}$, as well, it
would just make things a little longer.

We now find the second vector $\vec{v}_{2}$. Since $\vec{x}_{2}$ and
$\vec{x}_{1}$ are linearly independent, it follows that $\vec{x}_{2}$
and $\vec{v}_{1}$ are linearly independent. Then we hope to find
coefficients $c_{1}$ and $c_{2}$ such that
\begin{equation}
\vec{v}_{2} = c_{1}\widehat{\vec{v}_{1}} + c_{2}\vec{x}_{2}
\end{equation}
is a unit vector orthogonal to $\vec{v}_{1}$. So
\begin{equation}
\widehat{\vec{v}_{1}}\cdot\vec{v}_{2} = 0,
\end{equation}
which forces us to admit
\begin{subequations}
  \begin{align}
    0 &= \widehat{\vec{v}_{1}}\cdot(c_{1}\widehat{\vec{v}_{1}} + c_{2}\vec{x}_{2})\\
    &= c_{1}\widehat{\vec{v}_{1}}\cdot\widehat{\vec{v}_{1}} + c_{2}\widehat{\vec{v}_{1}}\cdot\vec{x}_{2}\\
    &=c_{1} + c_{2}\widehat{\vec{v}_{1}}\cdot\vec{x}_{2}
  \end{align}
  hence
  \begin{equation}
c_{1} = -c_{2}\widehat{\vec{v}_{1}}\cdot\vec{x}_{2}.
  \end{equation}
  Setting $c_{2}=1$ (since it's arbitrary), we find
  \begin{equation}
\vec{v}_{2} = -(\widehat{\vec{v}_{1}}\cdot\vec{x}_{2})\widehat{\vec{v}_{1}}+\vec{x}_{1}
=\vec{x}_{2} -(\widehat{\vec{v}_{1}}\cdot\vec{x}_{2})\widehat{\vec{v}_{1}}.
  \end{equation}
\end{subequations}
We can quickly check that $\widehat{\vec{v}_{1}}\cdot\vec{v}_{2}=0$.

We now can see that $\vec{v}_{1}$, $\vec{v}_{2}$ span everything which
$\vec{x}_{1}$, $\vec{x}_{2}$ spanned. Since $\vec{x}_{3}$ was
independent of $\vec{x}_{1}$ and $\vec{x}_{2}$, we have
$\vec{x}_{3}\notin\Span(\{\vec{x}_{1},\vec{x}_{2})$. Therefore we will
use $\vec{x}_{3}$ to construct $\vec{v}_{3}$ by writing
\begin{equation}
  \vec{v}_{3} = c_{1}\vec{v}_{1} + c_{2}\vec{v}_{2} + c_{3}\vec{x}_{3}.
\end{equation}
We are trying to determine the unknown coefficients $c_{1}$, $c_{2}$,
$c_{3}$. We know $\vec{v}_{3}$ will be orthogonal to $\vec{v}_{1}$ and
$\vec{v}_{2}$:
\begin{equation}
\vec{v}_{3}\cdot\vec{v}_{2}=0,\quad\mbox{and}\quad\vec{v}_{3}\cdot\vec{v}_{1}=0.
\end{equation}
The first of these give us (recalling $\vec{v}_{2}\cdot\vec{v}_{1}=0$),
\begin{equation}
\vec{v}_{3}\cdot\vec{v}_{2} = 0 = c_{2}\vec{v}_{2}\cdot\vec{v}_{2} + c_{3}\vec{x}_{3}\cdot\vec{v}_{2},
\end{equation}
hence
\begin{equation}
c_{2} = -c_{3}\frac{\vec{x}_{3}\cdot\vec{v}_{2}}{\vec{v}_{2}\cdot\vec{v}_{2}}.
\end{equation}
Similarly, we find
\begin{equation}
\vec{v}_{3}\cdot\vec{v}_{1} = 0 = c_{1}\vec{v}_{1}\cdot\vec{v}_{1} + c_{3}\vec{x}_{3}\cdot\vec{v}_{1},
\end{equation}
give us
\begin{equation}
c_{1} = -c_{3}\frac{\vec{x}_{3}\cdot\vec{v}_{1}}{\vec{v}_{1}\cdot\vec{v}_{1}}.
\end{equation}
Setting $c_{3}=1$ give us
\begin{equation}
\vec{v}_{3} = \vec{x}_{3} - \left(\frac{\vec{x}_{3}\cdot\vec{v}_{1}}{\vec{v}_{1}\cdot\vec{v}_{1}}\right)\vec{v}_{1}
-\left(\frac{\vec{x}_{3}\cdot\vec{v}_{2}}{\vec{v}_{2}\cdot\vec{v}_{2}}\right)\vec{v}_{2}.
\end{equation}
Now we have three basis vectors in our collection.

We see that
\begin{equation}
\vec{v}_{3} = \vec{x}_{3} - (\vec{x}_{3}\cdot\widehat{\vec{v}_{1}})\widehat{\vec{v}_{1}} - (\vec{x}_{3}\cdot\widehat{\vec{v}_{2}})\widehat{\vec{v}_{2}}.
\end{equation}
The general pattern seems to be
\begin{equation}
\vec{v}_{n+1} = \vec{x}_{n+1} - \sum^{n}_{k=1}\left(\frac{\vec{x}_{n+1}\cdot\vec{v}_{k}}{\vec{v}_{k}\cdot\vec{v}_{k}}\right)\vec{v}_{k}.
\end{equation}
Is this actually true?

Well, we've proven it works for $n=2$ and $n=3$, so we can try proving
it by induction. We assume this works for arbitrary $(n+1)\in\NN$. Then
the inductive case, supposing we have $n+1$ orthogonal vectors
$\vec{v}_{1}$, $\vec{v}_{2}$, \dots, $\vec{v}_{n+1}$, and we have a
vector $\vec{x}_{n+2}\notin\Span(\{\vec{v}_{1},\dots,\vec{v}_{n+1}\})$.
Then we claim
\begin{equation}
\vec{v}_{n+2} = \vec{x}_{n+2} - \sum^{n+1}_{k=1}\left(\frac{\vec{x}_{n+1}\cdot\vec{v}_{k}}{\vec{v}_{k}\cdot\vec{v}_{k}}\right)\vec{v}_{k}
\end{equation}
is orthogonal to $\vec{v}_{j}$ for $j=1,\dots,n+1$. To see this, we
compute,
\begin{calculation}
  \vec{v}_{j}\cdot\vec{v}_{n+2}
  \step{unfolding definition of $\vec{v}_{n+2}$}
  \vec{v}_{j}\cdot\left(\vec{x}_{n+2} - \sum^{n+1}_{k=1}\left(\frac{\vec{x}_{n+1}\cdot\vec{v}_{k}}{\vec{v}_{k}\cdot\vec{v}_{k}}\right)\vec{v}_{k}\right)
  \step{distributivity of dot product}
  \vec{v}_{j}\cdot\vec{x}_{n+2} - \sum^{n+1}_{k=1}\left(\frac{\vec{x}_{n+1}\cdot\vec{v}_{k}}{\vec{v}_{k}\cdot\vec{v}_{k}}\right)\vec{v}_{j}\cdot\vec{v}_{k}
  \step{orthogonality of $\vec{v}_{j}\cdot\vec{v}_{k}=\delta_{j,k}\vec{v}_{j}\cdot\vec{v}_{j}$}
  \vec{v}_{j}\cdot\vec{x}_{n+2} - \sum^{n+1}_{k=1}\left(\frac{\vec{x}_{n+1}\cdot\vec{v}_{k}}{\vec{v}_{k}\cdot\vec{v}_{k}}\right)\delta_{j,k}\vec{v}_{j}\cdot\vec{v}_{j}
  \step{defining property of $\delta_{j,k}$}
  \vec{v}_{j}\cdot\vec{x}_{n+2} - \left(\frac{\vec{x}_{n+1}\cdot\vec{v}_{j}}{\vec{v}_{j}\cdot\vec{v}_{j}}\right)\vec{v}_{j}\cdot\vec{v}_{j}
  \step{commutativity of multiplication}
  \vec{v}_{j}\cdot\vec{x}_{n+2} - \vec{x}_{n+1}\cdot\vec{v}_{j}\left(\frac{\vec{v}_{j}\cdot\vec{v}_{j}}{\vec{v}_{j}\cdot\vec{v}_{j}}\right)
  \step{since $\vec{v}_{j}\neq\vec{0}$ hence $\vec{v}_{j}\cdot\vec{v}_{j}\neq0$}
  \vec{v}_{j}\cdot\vec{x}_{n+2} - \vec{x}_{n+1}\cdot\vec{v}_{j}
  \step{arithmetic}
  0.
\end{calculation}
Hence $\vec{v}_{n+2}$ is orthogonal to $\vec{v}_{j}$ for every
$j=1,\dots,n+1$.

\N{Graham--Schmidt Algorithm}\label{chunk:graham-schmidt}
Given $m$ linearly independent nonzero vectors $\vec{x}_{1}$, \dots,
$\vec{x}_{m}$, we can construct an orthonormal basis for
$\Span(\{\vec{x}_{1},\dots,\vec{x}_{m}\})$ as follows:

\N*{Step 1.} Set $\vec{f}_{1}=\widehat{\vec{x}_{1}}$. Then go to step 2.

\N*{Step 2.} For each $\vec{x}_{2}$, \dots, $\vec{x}_{m}$, compute
$\vec{v}_{k}$ by
\begin{equation}
\vec{v}_{k} = \vec{x}_{k} - \sum^{k}_{j=1}(\vec{x}_{k}\cdot\vec{f}_{j})\vec{f}_{j},
\end{equation}
and then set
\begin{equation}
\vec{f}_{k} = \widehat{\vec{v}_{k}} = \frac{\vec{v}_{k}}{\|\vec{v}_{k}\|}.
\end{equation}
This produces an orthonormal basis $B=\{\vec{f}_{1},\dots,\vec{f}_{m}\}$.

\begin{remark}
If we have a subspace $U\subset V$ and have found an orthonormal basis
$B_{U}$ of $B$, then we can extend it to a basis $B$ of $V$ by taking
$n=\dim(V)$ linearly independent vectors in $V$, and applying the
Graham--Schmidt algorithm to add them to $B_{V}$.
\end{remark}

\begin{definition}\label{defn:basis:projection}
Let $\vec{u},\vec{v}\in V$ be vectors. Assume $\vec{u}\neq\vec{0}_{V}$.
We define the \define{Projection} of $\vec{v}$ along the $\vec{u}$
direction to be the vector
\begin{equation}
\operatorname{Proj}_{\vec{u}}(\vec{v}) = \frac{\vec{u}\cdot\vec{v}}{\vec{u}\cdot\vec{u}}\vec{u}=(\widehat{\vec{u}}\cdot\vec{v})\widehat{\vec{u}}
\end{equation}
where $\widehat{\vec{u}}=\vec{u}/\|\vec{u}\|$ is a unit vector.
\end{definition}

\begin{remark}
We can see that the Graham--Schmidt algorithm can be rephrased as taking
$n$ linearly independent vectors $\vec{x}_{1}$, $\vec{x}_{2}$, \dots,
$\vec{x}_{n}$, and forming
\begin{subequations}
\begin{align}
  \vec{u}_{1} &= \vec{x}_{1}\\
  \vec{u}_{2} &= \vec{x}_{2} - \operatorname{Proj}_{\vec{u}_{1}}(\vec{x}_{2})\\
  \vec{u}_{3} &= \vec{x}_{3} - \operatorname{Proj}_{\vec{u}_{1}}(\vec{x}_{3}) - \operatorname{Proj}_{\vec{u}_{2}}(\vec{x}_{3})\\
  \vdots \nonumber\\
  \vec{u}_{n} &= \vec{x}_{n} - \sum^{n-1}_{k=1}\operatorname{Proj}_{\vec{u}_{k}}(\vec{x}_{n}).
\end{align}
\end{subequations}
This produces a set of $n$ orthogonal vectors, and we could normalize
them (``put hats on them'') to obtain $n$ orthonormal vectors.
\end{remark}


\begin{definition}
Let $\vec{u},\vec{v}\in V$ be vectors. Assume $\vec{u}\neq\vec{0}_{V}$.
We define the \define{Orthogonal Decomposition} of $\vec{v}$ with
respect to $\vec{u}$ as consisting of two vectors:
\begin{enumerate}
\item the parallel part $\vec{v}^{\parallel} = \operatorname{Proj}_{\vec{u}}(\vec{v})$,
and
\item the perpendicular part $\vec{v}^{\perp} = \vec{v} - \vec{v}^{\parallel}$.
\end{enumerate}
\end{definition}

\section{Continuous Functions}
\begin{prob}
One of the interesting properties we had with functions in real
analysis was the notion of continuity. Is there a way to generalize
this notion to a topological setting?
\end{prob}
\begin{defn}{(Real Analysis Definition of Continuous)}
Let $f:X\to Y$ be a function, $X$ and $Y$ be sets. We say that $f$ is
``\define{Continuous}'' if for each $\varepsilon>0$ there is a
corresponding $\delta>0$ such that
\begin{equation}
|x-x_{0}|<\delta\Rightarrow|f(x)-f(x_0)|<\varepsilon
\end{equation}
Or in other words, for each $\varepsilon$ neighborhood of $f(x_0)$,
there is a $\delta$ neighborhood of $x_0$.
\end{defn}
\begin{rmk}
Observe that what we are doing is specifying a neighborhood in the
range of a certain size, then demanding there is a corresponding
neighborhood in the domain of a certain size. We can do this in real
analysis since the reals are sufficiently nice (they are a metric
space which is a really really strong condition for a topological
space). We want to generalize this for topological spaces so when we
work with metric spaces we recover our previous notion of continuity.
\end{rmk}
\begin{defn}
Let $X$, $Y$ be topological spaces, $f:X\to Y$ be a function. We say
that ``\define{$f$ is continuous at $x_0$}'' if for every neighborhood
$V$ of $f(x)$, there is a neighborhood $U$ of $x$ such that
$f(U)\subset V$.
\end{defn}
\begin{rmk}
This notion of continuity is nearly identical to the notion we
previously introduced from real analysis. The difference is that we
are not using a metric to keep track of every open neighborhood in the
domain and the range.

But observe we demand each open neighborhood $V$ of $f(x)$ has a
corresponding neighborhood $U$ in the domain such that the image of
$U$ is contained in $V$. This is precisely what we did in the real
analysis case, we specified an $\varepsilon$ neighborhood, and
demanded that for each $\varepsilon$ neighborhood there is a
corrsponding $\delta$ neighborhood in the domain such that the image
of the $\delta$ neighborhood is contained in the $\varepsilon$
neighborhood.
\end{rmk}

\begin{thm}
Let $X$ and $Y$ be topological spaces, if $f:X\to Y$ is such
that for each open set $V\subset Y$ its preimage $f^{-1}(V)\subset X$
is open, then $f$ is continuous.
\end{thm}
\begin{proof}
Trivial.
\end{proof}
\begin{rmk}{(Inverse Functions)}
Note that contrary to appearances \emph{continuity does not demand the
  function be invertible!} The preimage of a set \emph{is not} the
same as the function's inverse. Consider $f:X\to Y$, and $Z\subset Y$
is some open set. Then
\begin{equation}
f^{-1}(Z) = \left\{x\in X:\; f(x)\in Z\right\}
\end{equation}
which does not say anything about the existence of an inverse for $f$.
\end{rmk}

\begin{defn}
Let $X$, $Z$ be topological spaces, and $Y\subset X$ be a subspace. A
map
\begin{equation}
\begin{array}{llll}
i:& Y & \hookrightarrow& X\\
  & x & \mapsto& i(x)=x
\end{array}
\end{equation}
is defined as the ``\define{Inclusion Map}''. Similarly, a continuous map
\begin{equation}
\begin{array}{llll}
r:& X & \hookrightarrow& Y\\
  & x & \mapsto& r(x)=x\;\text{if }x\in Y
\end{array}
\end{equation}
is defined as the ``\define{Retraction}'' if the retraction of $i$ to
$Y$ is the identity on $A$.
\end{defn}
\begin{rmk}
We use the inclusion map and the retraction to extend and restrict
functions (respectively) by composing them with the functions of
interest. Note that the restriction of a continuous function
(i.e. composing it with a retraction) is the composition of two
continuous functions and thus continuous.
\end{rmk}
\begin{defn}
Let $X$, $Z$ be topological spaces, $Y\subset X$ be a subspace.
If $f:X\to Z$ is a function, we can define the ``\define{Restriction
  of $f$ to $Y$}'' as
\begin{equation}
f\circ r:X\to Z
\end{equation}
the composition of the retraction map (which is just
$\operatorname{id_Y}$ on $Y$) with $f$. 
\end{defn}
\begin{rmk}
Observe that the restriction of a function is just composition of
functions.
\end{rmk}
\begin{lem}{(Pasting Lemma)}
Let $X$ and $Y$ be topological spaces, and $U,$ $V$ be open sets in
$X$ such that $X=U\cup V$. Suppose $f:U\to Y$ and $g:V\to Y$ are
continuous and 
\begin{equation}
f(x)=g(x)\quad\text{for all }x\in U\cap V
\end{equation}
Then the function $h:X\to Y$ defined by
\begin{equation}
h(x) = \begin{cases} f(x),&\text{if }x\in U\\
g(x),&\text{if }x\in V
\end{cases}
\end{equation}
is continuous.
\end{lem}

\section{Closed Sets}

\begin{prob}
We defined what it means for a set to be ``open'' by just demanding it
be in a topology on a space. So how do we define a set to be
``closed''? Intuitively, it's an ``open set that contains its
boundary'', but what is a ``boundary'' topologically?
\end{prob}

\begin{defn}
Let $X$ be a topological space, $A\subset X$ be some subset. We say
that $A$ is ``\textbf{Closed}'' if $X-A$ is open.
\end{defn}
\begin{rmk}
This is kind of weaseling out of the problem, just define a set to be
closed if its compliment is open. This doesn't seem intuitive or
immediately clear why one would want to define it this way, but we
will see why later on.
\end{rmk}

\begin{defn}
Let $X$ be a topological space, $Y\subset X$.
\begin{enumerate}
\item The \textbf{Closure} of $Y$ (denoted $\overline{Y}$ or
  $\operatorname{closure}(Y)$) is the intersection of all closed sets
  containing $Y$.
\item The \textbf{Interior} of $Y$ (denoted $Y^0$ or
  $\operatorname{int}(Y)$) is the union of all open sets contained in
  $Y$.
\end{enumerate}
\end{defn}
\begin{rmk}
Observe that the closure of a set is closed, the interior of a set is open.
\end{rmk}
\begin{rmk}
The closure of a set is the ``smallest'' closed set containing it, but
we conveniently avoided the problem of ``What do we mean by
`smallest'?'' by using intersections of closed sets. Similarly, the
interior of a set is the ``largest'' open set contained in it.
\end{rmk}
\begin{rmk}
A set $Y$ is closed iff $Y=\overline{Y}$ and it is open iff $Y=Y^0$.
\end{rmk}
\begin{defn}
Let $X$ be a topological space, $A\subset X$ be some subset. We define
the ``\textbf{Boundary}'' of $A$ (denoted $\partial A$) to be
\begin{equation}
\overline{A}\cap\overline{(X-A)}
\end{equation}
the intersection of the closure of $A$ with the closure of the
compliment of $A$.
\end{defn}
\begin{ex}
Consider the set $X=\{a,b,c\}$. Consider the topology
\begin{equation}
\mathcal{T} = \left\{\emptyset,\;\{a\},\; \{b\},\; \{a,b\},\; X\right\}
\end{equation}
We see that $\{a,c\}$ is closed since its compliment $\{b\}$ is open,
and we also see that $\{b,c\}$ is closed since its compliment $\{a\}$
is open. The intersection of these two closed sets $\{c\}$ is closed,
since its compliment is open and the intersection of closed sets is
closed.

Observe the closure of $\{a\}$ is given by the intersection of
all closed sets containing $a$:
\begin{equation}
\{a,c\}\cap\{a,b,c\}=\{a,c\}
\end{equation}
and the closure of $\{b,c\}$ (that is, the compliment of $\{a\}$) is
itself, i.e. $\{b,c\}$ is a closed set so its closure is itself. The
boundary of $\{a\}$ is then the intersection of these two sets
\begin{equation}
\partial \{a\} = \{b,c\}\cap\{a,c\} = \{c\}
\end{equation}
which we could not have found if we didn't define the boundary in a
topological way!
\end{ex}
\begin{rmk}
Let $X$ be a topological space, $A\subset X$ be a subset. The definition
of the boundary of $A$ is the same as the closure of $A$ minus its
interior
\begin{equation}
\partial A = \overline{A}-A^0.
\end{equation}
How can we see this? Well, the interior of $A$ is the union of all
open sets contained in $A$. Its compliment would be the closure of
$X-A$. We see that the intersection of $\overline{(X-A)}$ with
$\overline{A}$ is just the closure of $A$ intersected with the
compliment of its interior, i.e.
\begin{equation}
\overline{A}\cap\overline{(X-A)}=\overline{A}\cap\left(A^0\right)^C =
\overline{A}-A^0.
\end{equation}
This is by virtue of the property of compliments
\begin{equation}
(A-B)^C = A\cap B^C
\end{equation}
where $A$ and $B$ are subsets of some set $U$.
\end{rmk}

\begin{defn}
Let $X$ be a topological space, $Y\subset X$. A point $x\in X$ is 
a \textbf{Limit Point} of $Y$ if every neighborhood of $x$ intersects
$Y-\{x\}$.
\end{defn}
\begin{rmk}
This definition makes more sense given the notion of what a
``continuous function'' is, since we defined continuity in the
real analysis situation as the limit as $f(x)$ approaches
$f(x_0)$. We take this notion, and use the topological notion of
continuity to concoct a \emph{topological} notion of a limit.
\end{rmk}

\part{Act II: Construction of Topological Spaces}
\section{The Order Topology}
%%
%% orderTopology.tex
%% 
%% Made by Alex Nelson
%% Login   <alex@tomato>
%% 
%% Started on  Tue Jun  2 19:01:59 2009 Alex Nelson
%% Last update Tue Jun  2 19:01:59 2009 Alex Nelson
%%
\begin{prob}
Topologies seem more or less abstract. Isn't there some way to
simplify specifying open sets? We are going to introduce the
notion of specifying open sets using open intervals. How do we
specify open intervals? By using an order relation $a < b$, we
can specify open intervals.
\end{prob}

If $X$ is a simply ordered set, there is a standard topology for
it defined using the order relation. It's called the
\textbf{order topology}.

We should probably first generalize the notion of intervals
familiar from real analysis. Since $X$ is a simply ordered set,
there is a (simple) order relation $<$. So suppose we have
$a,b\in X$ such that $a<b$, then we have 4 possible subsets of
$X$:
\begin{subequations}
\begin{align}
(a,b) &= \{x|a<x<b\}\label{eq:openInterval}\\
(a,b] &= \{x|a<x\leq b\}\label{eq:clopenInterval}\\
[a,b) &= \{x|a\leq x<b\}\label{eq:oposedInterval}\\
[a,b] &= \{x|a\leq x\leq b\}\label{eq:closedInterval}.
\end{align}
\end{subequations}
We call the eq \eqref{eq:openInterval} an ``open interval'', eqs
\eqref{eq:clopenInterval} \eqref{eq:oposedInterval} ``clopen
intervals'', and lastly eq \eqref{eq:closedInterval} a ``closed
interval''.

\begin{defn}\label{defn:orderTopology}
Let $X$ be a set with a simple order relation. Suppose $X$ has
more than one element. Let $\mathscr{B}$ be the collection of all
sets of the following types:
\begin{enumerate}
\item All open intervals $(a,b)$ in $X$.
\item All intervals of the form $[a_{0},b)$, where $a_0$ is the
  smallest element (if any) of $X$.
\item All intervals of the form $(a,b_{0}]$, where $b_{0}$ is the
  largest element (if any) of $X$.
\end{enumerate}
The collection $\mathscr{B}$ is a basis for a topology on $X$,
which is called the \textbf{order topology}.
\end{defn}
\begin{rmk}\label{rmk:onOrderTopology}
If $X$ has no smallest element, there are no sets of type (2),
and if $X$ has no largest element there are no sets of type (3).
\end{rmk}

\begin{defn}\label{defn:rays}
If $X$ is an ordered set, and $a$ is an element of $X$, there are
four subsets of $X$ that are called the \textbf{rays} determined
by $a$. They are the following
\begin{subequations}
\begin{align}
(a,+\infty) &= \{x|x>a\}\\
(-\infty,a) &= \{x|x<a\}\\
[a,+\infty) &= \{x|x\geq a\}\\
(-\infty,a] &= \{x|x\leq a\}
\end{align}
\end{subequations}
The first two are \textbf{open rays}, the last two are
\textbf{closed rays}.
\end{defn}

\section{The Product Topology}
%%
%% productTopology.tex
%% 
%% Made by Alex Nelson
%% Login   <alex@tomato>
%% 
%% Started on  Tue Jun  2 19:41:14 2009 Alex Nelson
%% Last update Tue Jun  2 19:41:14 2009 Alex Nelson
%%
\begin{prob}
If we have two topological spaces $X$ and $Y$, can we ``glue''
them together? That is, given two topological spaces, we wish to
construct a new one using only what we know of $X$ and $Y$.
\end{prob}

\begin{defn}\label{defn:productTopology}
Let $X$, $Y$ be topological spaces. The \define{Product Topology}
on $X\times Y$ is the topology having as basis the collection
$\mathscr{B}$ of all sets of the form $U\times V$, where $U$ is
an open subset of $X$ and $V$ is an open subset of $Y$.
\end{defn}

\begin{rmk}\label{rmk:productTopologyBasis}
Being rigorous, we should probably verify that this $\mathscr{B}$
beast really is a basis. For any $x\times y\in X\times Y$, we see
that $X\times Y\in\mathscr{B}$ so the first condition is
trivially satisfied. The second condition, let $(x\times y)\in
U_{1}\times V_{1}$ and $(x\times y)\in U_{2}\times V_{2}$. We see
that $\mathscr{B}$ is the collection of the product of all open
subsets, which allows us to see that
\begin{equation}%\label{eq:}
(U_{1}\times V_{1})\cap(U_{2}\times V_{2}) = (U_{1}\cap
  U_{2})\times(V_{1}\cap V_{2})
\end{equation}
is also the product of open sets, so it's a basis element. Thus
the second property of a basis is satisfied.
\end{rmk}
\begin{thm}\label{thm:productOfBases}
If $\mathscr{B}$ is a basis for the topology of $X$ and
$\mathscr{C}$ is a basis for the topology of $Y$, then the
collection
\begin{equation}%\label{eq:}
\mathscr{D} = \{B\times C|B\in\mathscr{B},C\in\mathscr{C}\}
\end{equation}
is a basis for the topology of $X\times Y$.
\end{thm}
\begin{proof}
We will use lemma \ref{lem:findingTopologyBasis} to prove
this. That is, for some open set $W\subset X\times Y$ and each
$x\times y\in W$ there is a basis element $U\in\mathscr{B}$ such
that $x\in U$ and a basis element $V\in\mathscr{C}$ such that
$y\in V$, so $x\times y\in U\times V\subset W$ and $U\times
V\in\mathscr{D}$. Thus $\mathscr{D}$ is a basis.
\end{proof}
\begin{defn}\label{defn:projection}
Let $\pi_{1}:X\times Y\to X$ be defined by the equation
\begin{equation}%\label{eq:}
\pi_{1}(x,y)=x;
\end{equation}
let $\pi_{2}:X\times Y\to Y$ be defined by the equation
\begin{equation}%\label{eq:}
\pi_{2}(x,y) = y.
\end{equation}
The maps $\pi_{1}$ and $\pi_{2}$ are called the
\define{projections} of $X\times Y$ onto its first and second
factors, respectively.
\end{defn}
\begin{rmk}\label{rmk:projectionsAreSurjective}
Observe that projections are surjective provided that both $X$
and $Y$ are nonempty. If one is empty, $X\times Y$ is empty too,
and everything becomes trivial.
\end{rmk}
\begin{rmk}\label{rmk:inverseOfProjections}
Note that if $U\subset X$ is open, $\pi_{1}^{-1}(U)=U\times
Y$. Similarly, if $V\subset Y$ is open, $\pi_{2}^{-1}(V)=X\times
V$. Their intersection is $U\times V$.
\end{rmk}

\begin{thm}\label{thm:subBasisForProductTopology}
The collection
\begin{equation}%\label{eq:}
\mathscr{S}=\{\pi_{1}^{-1}(U)|U\text{ is open in
}X\}\cup\{\pi_{2}^{-1}(V)|V\text{ is open in }Y\}
\end{equation}
is a subbasis for the product topology on $X\times Y$.
\end{thm}
\begin{proof}
Let $\mathcal{T}$ be the topology on $X\times Y$, $\mathcal{T}'$
be the topology generated by our subbasis. We see that each
element $W\in\mathscr{S}$ is an open subset of the product
topology, which means that the topology generated by
$\mathscr{S}$ is contained in $\mathcal{T}$, i.e.
\begin{equation}%\label{eq:}
\mathcal{T}'\subset\mathcal{T}.
\end{equation}
We need to show that $\mathcal{T}\subset\mathcal{T}'$. We see
given a basis element $U\times V$ for $\mathcal{T}$ that
\begin{equation}%\label{eq:}
U\times V = \pi_{1}^{-1}(U)\cap\pi_{2}^{-1}(V)
\end{equation}
which is nothing more than a finite intersection of subbasis
elements. This implies that $\mathcal{T}\subset\mathcal{T}'$ as desired.
\end{proof}

\part{Act III: Properties of Topological Spaces}
\section{Connectedness}
%%
%% connectedSpaces.tex
%% 
%% Made by Alex Nelson
%% Login   <alex@tomato>
%% 
%% Started on  Sun Jun  7 17:58:50 2009 Alex Nelson
%% Last update Sun Jun  7 17:58:50 2009 Alex Nelson
%%
\begin{prob}
How can we tell if a space is ``connected'' or not? That is, if
it's ``disconnected'', then we should be able to ``break it up''
into independent pieces. How do we make this notion of
``disconnectedness'' rigorous?
\end{prob}
\begin{defn}\label{defn:seperation}
Let $X$ be a topological space. A \textbf{Seperation} of $X$ is a
pair $U,V$ of disjoint, nonempty, open sets whose union is
$X$. The space $X$ is said to be ``\textbf{Connected}'' if there
is no seperation in it.
\end{defn}
\begin{rmk}
Observe that we didn't really define connectedness. Instead we
defined what it means for it to have a seperation, and then
proceeded to define ``connected'' as ``not seperated''. So, to
prove a topological space is connected is equivalent to
disproving the existence of a seperation. Consequently, we will
use proof by contradiction a lot when dealing with connectedness.
\end{rmk}
\begin{rmk}
Observe that if $X$ is connected and $Y$ is homeomorphic to $X$,
then $Y$ is necessarily connected.
\end{rmk}
\begin{rmk}
A space $X$ is connected if and only if the only closed subsets
of $X$ that are both open and closed are $X$ and
$\emptyset$. Otherwise if we had some additional set $U$ that is
closed and open, then $X-U$ is also closed and open. But $U$ and
$X-U$ are disjoint, nonempty, open subsets of $X$ whose union is
$X$. This couldn't happen if $X$ were connected, so $U$ needs to
be empty or equal to all of $X$.
\end{rmk}
\begin{lem}\label{lemma:subspaceConnectedness}
If $Y$ is a subspace of $X$, a seperation of $Y$ is a pair of
disjoint nonempty sets $A$ and $B$ whose union is $Y$, neither of
which contains a limit point of the other. The space $Y$ is
connected if there exists no seperation of $Y$.
\end{lem}
\begin{proof}
Let $A$, $B$ form a seperation of $Y$. We see that
\begin{equation}%\label{eq:}
A=\overline{A}\cap Y\subset Y
\end{equation}
and
\begin{equation}%\label{eq:}
A^{C}=(\overline{A}\cap Y)^{C}=B,
\end{equation}
which implies
\begin{equation}%\label{eq:}
(\overline{A}\cap Y)\cap B=\emptyset.
\end{equation}
We can argue similarly for $(\overline{B}\cap Y)\cap A=\emptyset$.
\end{proof}
\begin{prop}%\label{prop:}
Let $\mathcal{T}$, $\mathcal{T}'$ be two topologies on $X$. If
$\mathcal{T}\subset\mathcal{T}'$, then
\begin{enumerate}
\item the existence of a seperation
in $\mathcal{T}$ implies existence of a seperation in
$\mathcal{T}'$;
\item connectedness in $\mathcal{T}'$ implies connectedness in
  $\mathcal{T}$.
\end{enumerate}
\end{prop}
\begin{proof}
\noindent\begin{enumerate}
\item Observe that if $A$, $B$ forms a seperation in the
  $\mathcal{T}$ topology, then
  $A,B\in\mathcal{T}\subset\mathcal{T}'$. So the seperation is in
  the $\mathcal{T}'$ topology too.
\item Observe that if $X$ is connected in the $\mathcal{T}'$
  topology, then there is no seperation (no pair $A,B$ that form
  a seperation of $X$ that live in $\mathcal{T}'$). This means
  that no such $A,B$ exist in $\mathcal{T}$, which implies there
  is no seperation of $X$ in the $\mathcal{T}$ topology. So $X$
  is connected in the $\mathcal{T}$ topology.
\end{enumerate}
\end{proof}
\begin{prob}
How can we \emph{construct} connected topological spaces? This
seems especially daunting since we only can really know if a
topological space is \emph{disconnected}.
\end{prob}
\begin{thm}%\label{thm:}
If the sets $C$ and $D$ form a seperation of $X$, and if $Y$ is a
connected subspace of $X$, then $Y$ lies entirely in $C$ or
entirely in $D$.
\end{thm}
\begin{proof}
Assume for contradiction that $Y$ can lie in both $C$ and
$D$. Then we can observe that $C\cap Y$ and $D\cap Y$ are subsets
of $Y$ that are both open and closed, as well as disjoint and
their union would be $Y$. How can we see this last point? Well,
we know since $C$ and $D$ form a seperation of $X$ that
\begin{equation}%\label{eq:}
C\cup D=X
\end{equation}
so it follows that
\begin{equation}%\label{eq:}
(C\cap Y)\cup(D\cap Y) = (C\cup D)\cap Y = X\cap Y = Y.
\end{equation}
We have our contradiction, it implies $Y$ has a seperation. We
reject our assumption and say that $Y$ has to lie entirely in $C$
or entirely in $D$.
\end{proof}
\begin{thm}%\label{thm:}
The union of connected subspaces of $X$ with a common shared
point is connected.
\end{thm}
\begin{proof}
Let $\{A_{\alpha}\}$ be our collection of connected subspaces,
\begin{equation}%\label{eq:}
Y=\bigcup_{\alpha} A_{\alpha}
\end{equation}
be the subspace we are trying to show is connected,
\begin{equation}%\label{eq:}
\{y_0\} = \bigcap_{\alpha}A_{\alpha} 
\end{equation}
be the point common to all connected subspaces.

Assume for contradiction $Y$ is disconnected. Then there exists a
pair $C$, $D$ of disjoint, nonempty, open subsets of $Y$ whose
union is $Y$. Since $A_\alpha$ is connected, by our previous
theorem, it must lie entirely in $C$ or $D$. 

Since all $A_\alpha$ share a common point, it implies that all
$A_\alpha$ lie entirely in $C$ or they all lie entirely in
$D$. If some lived entirely in $C$ while the rest entirely in
$D$, there would be no shared point $y_0$ since $C$ and $D$ are
disjoint. 

So it follows that their union lies entirely in one. But this
means that either $C=\emptyset$ and $D=Y$, or $C=Y$ and
$D=\emptyset$, which is a contradiction of our assumption that
$C\neq\emptyset$ and $D\neq\emptyset$.
\end{proof}
\begin{thm}%\label{thm:}
Let $A$ be a connected subspace of $X$. If $A\subset
B\subset\overline{A}$, then $B$ is connected.
\end{thm}
\begin{proof}
Assume for contradiction that $B$ has a seperation. More
precisely, there is a pair of nonempty disjoint open sets $C$,
$D$ whose union is $B$. Since $A\subset B$, and $A$ is connected,
then $A$ is contained entirely in $C$ or it is contained entirely
in $D$. Suppose it is contained entirely in $C$, then
\begin{equation}%\label{eq:}
C\cup D\subset \overline{A}\;\Rightarrow\; D \subset A'
\end{equation}
where $A'$ is the set of limit points of $A$. Recall a limit
point $x$ is such that 
\begin{equation}%\label{eq:}
x\in\overline{A-\{x\}}.
\end{equation}
But observe that
\begin{equation}%\label{eq:}
x\in\overline{A-\{x\}}\subset \overline{C-\{x\}}\subset\overline{C}
\end{equation}
but this implies that
\begin{equation}%\label{eq:}
A'\subset\overline{C}\; \Rightarrow\; B\subset\overline{A}\subset\overline{C}
\end{equation}
or equivalently, $D=\emptyset$ since we previously proven
$D\cap\overline{C}=\emptyset$, which is a contradiction.
\end{proof}
\begin{thm}%\label{thm:}
The image of a connected space under a continuous mapping is connected.
\end{thm}
\begin{proof}
Let $f:X\to Y$ be continuous, $X$ be connected. Assume for
contradiction there is a seperation (i.e. a pair $A,B$ of
nonempty, open, disjoint subsets whose union is $Y$) in
$Y$. Their preimage would be a seperation in $X$, which
contradicts our hypothesis that $X$ is connected. We reject our
assumption that $Y$ has a seperation and conclude it is connected.
\end{proof}
\begin{thm}%\label{thm:}
A finite Cartesian product of connected spaces is connected.
\end{thm}
\begin{proof}
We will do a sort of inductive proof by contradiction.

\noindent\textbf{Base Case $(n=2)$:} Lets consider two connected
spaces $X_1$ and $X_2$. Let
\begin{equation}%\label{eq:}
Y=X_1\times X_2.
\end{equation}
For contradiction, assume $Y$ is disconnected. Then there is a
pair $A$, $B$ of disjoint, nonempty subsets of $Y$ whose union is
$Y$. This means that either $\pi_{1}(A)$, $\pi_{1}(B)$ forms a
seperation of $X_1$ or $\pi_{2}(A)$, $\pi_{2}(B)$ forms a
seperation of $X_2$. But we assumed that $X_i$ ($i=1,2$) was
connected. So we have a contradiction, reject our assumption that
$Y$ is disconnected, and conclude $Y$ is connected.

\noindent\textbf{Inductive Hypothesis (arbitrary $n$):} Suppose
this works for
\begin{equation}%\label{eq:}
Y = X_1\times\cdots\times X_n.
\end{equation}

\noindent\textbf{Inductive Cast $(n+1)$:} Lets consider $X_1$,
$\ldots$, $X_{n+1}$ connected spaces, let
\begin{equation}%\label{eq:}
X = X_1\times\cdots\times X_n
\end{equation} 
and
\begin{equation}%\label{eq:}
Y = X\times X_{n+1}.
\end{equation}
Then this is just the base case, and $Y$ was shown to be connected.
\end{proof}
\begin{defn}%\label{defn:}
A space is ``\textbf{Totally Disconnected}'' if its only
connected subspaces are one-point sets.
\end{defn}
\begin{thm}%\label{thm:}
If $X$ is a topological space equipped with the discrete
topology, then it is totally disconnected.
\end{thm}
\begin{proof}
Since $X$ has the discrete topology, we see that every subset of
$X$ is open and consequently its compliment (being a subset of
$X$) is open, so every subset is also closed. The only subspaces
of $X$ which has the only subsets be both open and closed would
necessarily be the one-point sets. 

(Observe that an arbitrary set $Y\subset X$ with more than one
element could be partitioned into two subsets, which are both
open and closed. This implies that $Y$ is not connected.)

It then follows by definition that $X$ is totally disconnected.
\end{proof}
\begin{prop}%\label{prop:}
Let $Y\subset X$, let $X$ and $Y$ be connected. If $A$ and $B$
form a seperation of $X-Y$, then $A\cup Y$ and $B\cup Y$ are
connected.
\end{prop}
\begin{proof} We will prove that if $A\cup Y$ has a seperation, it
  means that $X$ has a seperation in the sense of a seperation in
  a subspace (it is more general this way -- if $X$ isn't a
  subspace, it's still a kosher seperation).
\noindent\begin{enumerate}
\item Let $A\cup B=X-Y$, assume for contradiction that
  $Z:=A\cup Y$ has a seperation $C\cup D=Z$.
\item We will show that $B$, $C\cup D$ form a seperation of
  $X$. Observe first that $C\cup D=A\cup Y$.
\item We see that $\overline{B}\cap A=B\cap\overline{A}=\emptyset$ since $A,B$ form
  a seperation of $X-Y$.
\item We see that $Y$ is closed since $X-Y=A\cup B$ is the union
  of open sets (thus open).
\item We see that $\overline{Y}\cap B=Y\cap B=\emptyset$.
\item We see that $\overline{B}\subset X-Y$ so $\overline{B}\cap
  Y=\emptyset$.
\item Observe that 
\begin{equation}%\label{eq:}
\overline{Z}\cap B=(\overline{A\cup Y})\cap B = (\overline{A}\cup\overline{Y})\cap B=\emptyset
\end{equation}
and similarly
\begin{equation}%\label{eq:}
Z\cap\overline{B}=(A\cup Y)\cap\overline{B}=\emptyset.
\end{equation}
\item We see that $B$ and $Z=A\cup Y$ form a seperation of
  $X$, implying $X$ is disconnected. This is a contradiction. We
  have to reject the assumption that $Z$ has a seperation.
\end{enumerate}
A similar argument holds for $B\cup Y$ being connected.
\end{proof}


\nocite{munkres,mccluskeyMcMaster,morris}
\bibliographystyle{plain}
\bibliography{topology}

\end{document}
