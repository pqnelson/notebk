\chapter{Groups}

\M
We should think of a group of symmetry transformations as a collection
of functions mapping an object $X$ to itself. Here ``symmetry'' means
that the transformation is ``undoable''. The composition operator
is then just the usual composition of functions.

\begin{definition}
  A \define{Group} consists of a collection $G$ of symmetry
  transformations equipped with
  \begin{enumerate}
  \item a composition operator $\circ\colon G\times G\to G$
  \item an identity element $e\in G$ (or $e\colon\singleton\to G$)
  \item an inverse operator $(-)^{-1}\colon G\to G$
  \end{enumerate}
  such that
  \begin{enumerate}
  \item composition is associative: for any $g_{1},g_{2},g_{3}\in G$,
    $(g_{1}\circ g_{2})\circ g_{3}=g_{1}\circ(g_{2}\circ g_{2})$
  \item unit laws: for any $g\in G$, $e\circ g=g\circ e=g$
  \item inverse law: for any $g\in G$, $g^{-1}\circ g=g\circ g^{-1}=e$.\popqed
  \end{enumerate}%
\end{definition}

\begin{remark}[Terminology: Stuff, Structure, Properties]
  We use the terms ``group structure'' to refer to the composition
  operator, identity element, and inverse operator. The axioms they
  satisfy are the ``properties'' of a group. We try to separate the
  ``structure'' from the ``properties'' by writing definitions using the
  generic template:

  \begin{quote}
    A ``gadget'' consists of $\langle$\emph{stuff\/}$\rangle$ equipped with
    $\langle$\emph{structure}$\rangle$ such that a bunch of
    $\langle$\emph{properties}$\rangle$ hold.
  \end{quote}
  Using the term ``group structure'' refers very technically to the
  structure we equip some collection of transformations, which satisfy
  the group properties.
\end{remark}

\begin{example}[Trivial Group]
  The most boring example is the symmetry transformation which does
  nothing, $G=\{e\}$ where $e(x)=x$ for each $x$ in the object. This
  group is called the \define{Trivial Group}.
\end{example}

\begin{example}[Dihedral Group]
  Consider the regular $n$-gon $X$ in the plane $\RR^{2}$, so
  $X\subset\RR^{2}$, for $n\in\NN$ at least 3. We explicitly note the
  vertices are at the points ($x_{k}$, $y_{k}$) = ($\cos(k\,2\pi/n)$, $\sin(k\,2\pi/n)$)
  for $k=0$, $1$, $\dots$, $n-1$.

  There are two sorts of symmetries here: rotation by $2\pi/n$ radians,
  and reflection about an axis. We can encode these transformations by
  \begin{equation}
    r\begin{pmatrix}
       x\\
       y
     \end{pmatrix}
    =
    \begin{pmatrix}
      \cos(2\pi/n) & -\sin(2\pi/n)\\
      \sin(2\pi/n) & \cos(2\pi/n)
    \end{pmatrix}
    \begin{pmatrix}
      x\\
      y
    \end{pmatrix}
  \end{equation}
  and reflections by
  \begin{equation}
    s\begin{pmatrix}x\\y
    \end{pmatrix}=
    \begin{pmatrix}1 & 0\\
      0 & -1
    \end{pmatrix}\begin{pmatrix}x\\y
    \end{pmatrix}.
  \end{equation}
  The $n$-gon $X$ remains \emph{invariant} under these transformations
  of the plane, in the sense that $r(X)=X$ and $s(X)=X$.

  In this representation of the group of symmetries, composition is
  matrix multiplication (which we know is associative) and the inverse
  law is matrix inversion. The identity element is the identity matrix.

  Usually groups are presented \emph{abstractly}, which would consider
  $s$, $s^{2}=e$, $r$, $r^{2}$, \dots, $r^{n-1}$, $r^{n}=e$, and
  $sr^{k}$ for $k=1,\dots,n-1$. We also would have $srs=r^{-1}$. Now we
  do not need to rely on matrices to reason about this symmetry group.
  The abstract symmetries of the regular $n$-gon is called the
  \define{Dihedral Group} of the [regular] $n$-gon, denoted $D_{n}$ by
  geometers and $D_{2n}$ by algebraists. We will use $\dihedral{n}$ notation.
\end{example}

\begin{example}
  Consider, for $n\geq3$, the rotations leaving the regular $n$-gon
  $X$ invariant in the plane $\RR^{2}$. These are rotations by integers
  multiples of $2\pi/n$ radians. The elements of our group are $e$, $r$,
  \dots, $r^{n-1}$, $r^{n}=e$. This is a subcollection of the Dihedral
  group, called the \define{Cyclic group} denoted $\cyclic{n}$.
\end{example}

\begin{example}[Number systems]
One class of fairly boring examples is any number system with addition
and negation. For example, $\ZZ$ equipped with ($+$, $0$, $-$) gives us a
group. Really? Let's check it:
\begin{enumerate}
\item For any $\ell,m,n\in\ZZ$ we have $(\ell + m) + n = \ell + (m+n)$
\item For any $n\in\ZZ$, we have $n+0=0+n=n$
\item For any $n\in\ZZ$, there is a unique number denoted $-n\in\ZZ$
  such that $-n+n=n+(-n)=0$.
\end{enumerate}
Thus the integers equipped with addition form a group.

Similar reasoning shows the rational numbers $\QQ$, the real numbers
$\RR$, and the complex numbers $\CC$ form a group when equipped with
addition. But the natural numbers $\NN$ do not. Why? Well, for
Americans, $0\notin\NN$; for everyone else, if $n\in\NN$ is an arbitrary
natural number, then in general $-n\notin\NN$.

A particular property of these groups is that $m+n=n+m$ the law of
composition is commutative. This is \emph{not true in general}: given a
generic group $G$, its law of composition is not necessarily
commutative. When the law is commutative, it's special and we'd like a
special name for the group. When the law of composition is commutative,
we call the group \define{Abelian}. Conventionally we write the law of
composition as ``$+$'' instead of ``$\circ$'' to stress its commutative
nature.

We can think of $\ZZ$ with this group structure as an infinite cyclic
group, and refer to it as such.
\end{example}

\begin{example}
  If we think of the dihedral group as a finite cyclic group ``with
  reflections'', then it's natural to generalize to an infinite cyclic
  group. There are two generators for the symmetries: ``rotation''
  $r^{k}$ for $k\in\ZZ$ and ``reflection'' satisfying $s^{2}=e$. Unlike
  the dihedral groups $\dihedral{n}$ which have a direct geometric
  picture, there's no simple picture for the \define{Infinite Dihedral Group}
  denoted $\dihedral{\infty}$.
\end{example}

\begin{example}
  Another ``$n\to\infty$ limit'' of the symmetries of the regular
  $n$-gon would be the symmetries of the circle $S^{1}\subset\RR^{2}$.
  This would be generated by rotations $r_{\theta}$ anti-clockwise by an angle
  $0\leq\theta<2\pi$, and reflection $s$ about the $x$-axis. The rotations
  satisfy
  \begin{equation}
    r_{\alpha}\circ r_{\beta}=r_{\alpha+\beta}
  \end{equation}
  where $\alpha$, $\beta\in\RR$. Reflections satisfy $s^{2}=e$, and as
  always $s\circ r_{\alpha}\circ s=r_{-\alpha}$. This could be viewed as
  the continuous dihedral group.
\end{example}