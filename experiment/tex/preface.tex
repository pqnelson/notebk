\preface

This is an experiment in writing with \TeX, to see if it is possible to
reproduce the code of \LaTeX\ in a minimal environment. (I will be
writing on a Raspberry Pi, where space is at a premium.) Consequently,
anyone looking at {\tt Makefile} for this project may be confused at how
convoluted the build process appears. We try to avoid using PDF\TeX,
since that produces bloated output.

Simultaneously, these notes are based on my attempt at using a
\emph{Zettelkasten} as a means to collate my math notes accumulated over
the years. Right now I'm studying finite groups, and polishing my notes
on Lie groups.

We use the ``stuff, structure, and properties'' structuralism discovered
by Baez and Dolan.

\bigbreak

A word about the style: we were inspired by Arthur Besse's beautiful
writing style, where ``chunks'' of text are grouped together and
separated by some whitespace. Each ``chunk'' begins with a number
``$c.p$'' where ``$c$'' is the chapter number, ``$p$'' is the chunk
number. Some chunks have a label (in bold) summarizing the thrust of the
chunk, or (more familiar to mathematicians) the \emph{species} of chunk
(like definition, theorem, remark, etc.).

Proofs end with the Halmos tombstone (\qedsymbol), but that's the only
``tombstone'' used. It was tempting to use something like ``\qefsymbol'' to
end examples, and have some other symbol end definitions, but due to the
writing style we could simply add another ``chunk'' without any
indicator that the preceding chunk has ended.
Unlike Besse (but like Dieudonn\'e) we allow ``subchunks'', writing
things like ``$c.p.s$'' where ``$s$'' is a positive integer numbered
sequentially. In principle, we could nest indefinitely, writing
something like ``$c.p.s_{1}.s_{2}.\cdots.s_{n}$'', though I do not know
why anyone would want to do this.
