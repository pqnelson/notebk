\input amssym
\input macros
\input notation

%% \chapter{Introduction}

%% \M
%% Paragraph 1, reporting! ${\bbb N}$ is the set of natural numbers,
%% starting at one (god damn it!). \textsc{Let us see if This works.}


%% \begin{proposition}
%%   The square of a hypoteneuse is the sum of the square of its sides.
%% \end{proposition}

%% \M
%% Paragraph 2, alert!

%% \begin{danger}%
%% \noindent\ignorespaces\textsc{{\tensc Caution:}} I do not know if this will work, but it is worth a short, no?  In math mode, you can use underbrace and overbrace, but there are no corresponding brackets macros. This is sad, because I'm more fond of brackets than I am of braces.
%% \end{danger}

%% \smallbreak\noindent\ignorespaces

%% \N[2]{Theorem}
%% {\it Paragraph {2.1} should be present.}

%% \M
%% This is the first paragraph, paragraph number 1. Uh $\square$?
%% ${\tt E}_{8}(q^{2})$

%% \begin{theorem}[Pythagoras]
%%   The square of a hypoteneuse is the sum of the square of its sides.
%% \end{theorem}

%% \begin{proof}
%%   Follows from definition of Euclidean metric.
%% \end{proof}

%% \M
%% This is paragraph number 2.

%% \section{Uh, new section}

%% \M
%% This is paragraph number 2.1

%% \M[1]
%% This is 2.10

%% \M[3]
%% This is 2.3.2

%% \M[1]
%% This is 2.2.1

%% \M[2]
%% This is, uh, 2.2.1.1.1?

%% \M[-1]
%% Wait, what does this do?

%% \M[-14]
%% This is 3

%% Spam and eggs test!
%% \begin{equation}
%% E = mc^{2} - \mathsc{Min}(T).
%% \end{equation}
\input tex/preface.tex
\input tex/groups.tex

\bye