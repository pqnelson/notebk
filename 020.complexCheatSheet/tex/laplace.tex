%%
%% laplace.tex
%% 
%% Made by Alex Nelson
%% Login   <alex@tomato>
%% 
%% Started on  Wed Jun 10 10:30:18 2009 Alex Nelson
%% Last update Wed Jun 10 10:30:18 2009 Alex Nelson
%%
\begin{defn}[Laplace Transform]
The \textbf{Laplace Transform} of a function $f(t)$ (for $t\geq
0$) is the function $\widetilde{f}(z)$ defined by
\begin{equation}%\label{eq:}
\widetilde{f}(z) = \mathcal{L}\{f\}(z) = \int^{\infty}_{0}e^{-zt}f(t)dt
\end{equation}
where $z$ is complex.
\end{defn}
\begin{prop}[Asymptotic Behavior of Laplace Transform]%\label{prop:}
Suppose $g$ is analytic in a region containing the positive real
axis and is bounded on the positve real axis. Let the Taylor
series for $g$ centered at 0 be
\begin{equation}%\label{eq:}
\sum^{\infty}_{n=0} a_{n}z^{n}
\end{equation}
and let
\begin{equation}%\label{eq:}
\widetilde{g}(z) = \int^{\infty}_{0}e^{-zt}g(t)dt.
\end{equation}
Then
\begin{equation}%\label{eq:}
\widetilde{g}(z)\sim \frac{a_0}{z}+\frac{a_1}{z^2}+\frac{2a_2}{z^3}+\cdots+\frac{n!a_n}{z^{n+1}}+\cdots
\end{equation}
as $z\to\infty$, $\operatorname{arg}(z)=0$.
\end{prop}
\begin{prop}%\label{prop:}
Suppose $g$ is infinitely differnetiable on the positive real
axis and that $g$ and each of its derivatives are of exponential
order. That is, there are constants $A_n$ and $B_n$ such that
\begin{equation}%\label{eq:}
|g^{(n)}(t)|\leq A_{n}e^{B_{n}t}
\end{equation}
for $t\geq0$. Let
\begin{equation}%\label{eq:}
\widetilde{g}(z) = \int^{\infty}_{0}e^{-zt}g(t)dt.
\end{equation}
Then
\begin{equation}%\label{eq:}
\widetilde{g}(z)\sim  \frac{g(0)}{z}+\frac{g'(0)}{z^2}+\frac{g''(0)}{z^3}+\cdots+\frac{g^{(n)}(0)}{z^{n+1}}+\cdots
\end{equation}
as $z\to\infty$, $\operatorname{arg}(z)=0$.
\end{prop}
\begin{thm}[Convergence Theorem for Laplace Transform]%\label{thm:}
Assume
\begin{equation}%\label{eq:}
f:(0,\infty)\to\mathbb{C}
\end{equation}
is of exponential order and let
\begin{equation}%\label{eq:}
\widetilde{f}(z) = \int^{\infty}_{0}e^{-zt}f(t)dt.
\end{equation}
There exists a unique number $\sigma$, $-\infty\leq\sigma<\infty$
such that this integral converges if $\re(z)>\sigma$ and diverges
if $\re(z)<\sigma$. Furthermore if $\widetilde{f}$ is analytic on
the set
\begin{equation}%\label{eq:}
A = \{z|\re(z)>\sigma\}
\end{equation}
and we have
\begin{equation}%\label{eq:}
\frac{d}{dz}\widetilde{f}(z) = -\int^{\infty}_{0}te^{-zt}f(t)dt
\end{equation}
for $\re(z)>\sigma$. The number $\sigma$ is called the
``\textbf{Abscissa of Convergence}'' and if we\marginpar{define $\rho$} define the number
$\rho$ by 
\begin{equation}%\label{eq:}
\rho=\inf\{B\in\mathbb{R}|\text{there exists an }A>0\text{ such
  that }|f(t)|\leq Ae^{Bt}\}
\end{equation}
then $\sigma\leq\rho$.
\end{thm}
\begin{thm}[Laplace Transforms]
Suppose that the functions $f$ and $h$ are continuous and that
$\widetilde{f}(z)=\widetilde{h}(z)$ for $\re(z)>\gamma_0$ for
some $\gamma_0$. Then $f(t)=h(t)$ for all $t\in(0,\infty)$.
\end{thm}
\begin{prop}%\label{prop:}
Let $f(t)($ be continuous on $(0,\infty)$ and piecewise
$C^1$. Then for $\re(z)>\rho$
\begin{equation}%\label{eq:}
\widetilde{\left(\frac{df}{dt}\right)}(z)=z\widetilde{f}(z)-f(0).
\end{equation}
\end{prop}
\begin{prop}%\label{prop:}
Let
\begin{equation}%\label{eq:}
g(t)=\int^{t}_{0}f(\tau)d\tau
\end{equation}
Then for $\re(z)>\max[0,\rho(f)]$,
\begin{equation}%\label{eq:}
\wt{g}(z) = \frac{\wt{f}(z)}{z}.
\end{equation}
\end{prop}
\begin{thm}[First Shifting Theorem]%\label{thm:}
Fix $a\in\mathbb{C}$ and let $g(t)=e^{-at}f(t)$. Then for
$\re(z)>\sigma(f)-\re(a)$, we have
\begin{equation}%\label{eq:}
\wt{g}(z)=\wt{f}(z+a).
\end{equation}
\end{thm}

\begin{thm}[Second Shifting Theorem]%\label{thm:}
Let $H(t)=0$ if $t<0$ and $H(t)=1$ if $t\geq1$ be the
\textbf{Step Function} or \textbf{Heaviside Step Function}. Let
$a\geq0$ and let $g(t)=f(t-a)H(t-a)$; that is, $g(t)=0$ if $t<a$
while $g(t)=f(t-a)$ if $t\geq a$. Then for $\re(z)>0$ we have
\begin{equation}%\label{eq:}
\wt{g}(z)=e^{-az}\wt{f}(z).
\end{equation}
\end{thm}
\begin{defn}[Convolution]%\label{defn:}
The ``\textbf{Convolution}'' of two functions $f(t)$ and $g(t)$
is defined for $t\geq0$ by
\begin{equation}%\label{eq:}
(f*g)(t)=\int^{\infty}_{0}f(t-\tau)g(\tau)d\tau
\end{equation}
where we set $f(t)=0$ if $t<0$.
\end{defn}
\begin{thm}[Convolution Theorem]%\label{thm:}
The equalities
\begin{equation}%\label{eq:}
(f*g)(t) = (g*f)(t)
\end{equation}
whenever $\re(z)>\max[\rho(f),\rho(g)]$.
\end{thm}




\subsection{Table of Properties of the Laplace Transform}
Let $u(t)$ be the Heaviside step function.
\begin{equation}%\label{eq:}
u(t) = \int^{t}_{-\infty}\delta(\tau)d\tau
\end{equation}
where $\delta$ is the delta function we all know and love.
\bigskip
\begin{tabular}{|p{4cm}|l|l|}
\hline
%&&\\
Linearity & $af\left(t\right)+bg\left(t\right)$ & $a\wt{f}\(z\)+b\wt{g}\(z\)$\\\hline
Frequency Differentiation & $tf\(t\)$ & $-\wt{f}'\(z\)$\\\hline
Frequency Differentiation & $t^nf\(t\)$ & $(-1)^{n}\wt{f}^{n}\(z\)$\\\hline
Differentiation & $f'\(t\)$ & $z\wt{f}\(z\)-f\(0\)$\\\hline
Differentiation & $f''\(t\)$ & $z^2\wt{f}\(z\)-zf\(0\)-f'\(0\)$\\\hline
Differentiation & $f^{(n)}(t)$ & $z^n\wt{f}(z) - z^{n-1}f(0) -
\cdots - f^{(n-1)}(0)$\\\hline
Frequency Integration & $f(t)/t$ & $\int^{\infty}_{z}\wt{f}(\omega)d\omega$\\\hline
Integration & $\int^{t}_{0}f(\tau)d\tau=(u*f)(t)$ & $\wt{f}(z)/z$\\\hline
Scaling & $f(at)$ & $\wt{f}(z/a)/|a|$\\\hline
Frequency Shifting & $e^{at}f(t)$ & $\wt{f}(z-a)$\\\hline
Time shifting & $f(t-a)u(t-a)$ & $e^{-az}\wt{f}(z)$\\\hline
Convolution & $(f*g)(t)$ & $\wt{f}(z)\wt{g}(z)$\\\hline
Periodic Function & $f(t)$ & $(\int^{T}_{0}e^{-zt}f(t)dt)/(1-e^{-Tz})$\\\hline
\end{tabular}
\subsection{List of Properties of the Laplace Transform}
\textbf{Definition.\quad}\ignorespaces
The Laplace transform of $f(t)$ is given by:
\begin{equation}%\label{eq:}
\wt{f}(z)=\int^{\infty}_{0}e^{-zt}f(t)dt.
\end{equation}
It is such that:
\begin{enumerate}
\item $\displaystyle \wt{g}(z)=-\frac{d}{dz}\wt{f}(z)$ where $g(t)=tf(t)$.
\item $\ds \mathcal{L}\bigl\{af+bg\bigr\} = a\wt{f}+b\wt{g}$
\item $\ds \wt{\left(\frac{df}{dt}\right)}(z)=z\wt{f}(z)-f(0)$.
\end{enumerate}
