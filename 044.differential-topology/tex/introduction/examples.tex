\N{How to Construct Manifolds}
Thanks to Theorem~\ref{thm:atlas:contained-in-unique-maximal-atlas},
we just need to construct \emph{one} finite atlas equivalent to the
differential structure for our manifold. We just need to check the
Hausdorff condition is satisfied (paracompactness follows from the atlas
being finite). In practice, ``the differential
structure'' is unobtainable, we cannot ``write it down''. Instead, we
just say it is generated by all atlases equivalent to the finite atlas
we have constructed.

Moreover, if the finite atlas is \emph{oriented}, then we can extend it
to an orientation, and conclude the manifold is orientable.

\begin{example}[Euclidean spaces]
We see that $\RR^{n}$ is an $n$-dimensional manifold. The differential
structure is \emph{determined} by the one-chart atlas $\{(\RR^{n},\id)\}$.
\end{example}

\begin{example}[$n$-Dimensional Sphere $S^{n}$]
The $n$-dimensional sphere $S^{n}=\{\,(x_{1},\dots,x_{n+1})\in\RR^{n+1}\mid{x_{1}}^{2}+\cdots+{x_{n+1}}^{2}=1\,\}$
is an $n$-dimensional smooth manifold.
\end{example}

\begin{proof}
  Let $\varphi_{\pm}\colon\RR^{n}\to S^{n}$ be defined by
  \begin{equation}
\varphi_{\pm}(x_{1},\dots,x_{n+1}) = \left(\frac{2x_{1}}{1+r^{2}},
\dots, \frac{2x_{n}}{1+r^{2}}, \pm\left(\frac{1-r^{2}}{1+r^{2}}\right)\right),
  \end{equation}
  where $r^{2} = {x_{1}}^{2}+\cdots+{x_{n+1}}^{2}$. The map
  $\varphi_{\pm}^{-1}\colon S^{n}\setminus\{0\}\to\RR^{n}$ are called
  \define{Stereographic Projections}.
  Then we claim $(\varphi_{+},\RR^{n})$ and $(\varphi_{-},\RR^{n})$ are
  $n$-dimensional charts which together form an atlas.

\smallbreak
  \textsc{Claim 1:} These are two $n$-dimensional charts.

\noindent\textit{Proof of claim 1}.
  We see these $\varphi_{\pm}$ are injective and that, if
  $(y_{1},\dots,y_{n},y_{n+1})\in S^{n}$, then
  \begin{equation}
\varphi_{\pm}^{-1}(y_{1},\dots,y_{n},y_{n+1}) = \left(\frac{y_{1}}{1 \mp y_{n+1}},\dots,\frac{y_{n}}{1 \mp y_{n+1}}\right).
  \end{equation}
  This is the left-inverse to $\varphi_{\pm}$, which implies
  $\varphi_{\pm}$ is injective.

\smallbreak
  \textsc{Claim 2:} The charts are compatible with each other.

\noindent\textit{Proof of claim 2}.
There are four criterion for proving charts are compatible, according to Definition~\ref{defn:chart:compatible}.
We need to prove:
\begin{enumerate}
\item $\varphi_{+}^{-1}\bigl(\varphi_{-}(\RR^{n})\bigr)\subset\RR^{n}$
  is open; this is obvious since $\varphi_{+}^{-1}\bigl(\varphi_{-}(\RR^{n})\bigr)=\RR^{n}\setminus\{(0,\dots,0,0)\}$;
\item $\varphi_{-}^{-1}\bigl(\varphi_{+}(\RR^{n})\bigr)\subset\RR^{n}$
  is open, for the same reason as the previous point;
\item the map $\varphi_{+}^{-1}\circ\varphi_{-}\colon\RR^{n}\setminus\{(0,\dots,0,0)\}\to\RR^{n}\setminus\{(0,\dots,0,0)\}$
  is smooth, which we can compute explicitly as acting on a point
  \begin{equation}
(x_{1},\dots,x_{n})\mapsto\left(\frac{x_{1}}{r^{2}},\dots,\frac{x_{n}}{r^{2}}\right).
  \end{equation}
  This is obviously smooth where defined; and
\item the map $\varphi_{-}^{-1}\circ\varphi_{+}\colon\RR^{n}\setminus\{(0,\dots,0,0)\}\to\RR^{n}\setminus\{(0,\dots,0,0)\}$
is smooth, for exactly the same reason as the previous point.
\end{enumerate}
Hence the two charts are compatible with each other.

\smallbreak
  \textsc{Claim 3:} The charts cover $S^{n}$, hence form an atlas.

\noindent\textit{Proof of claim 3}.  We can see that $\varphi_{\pm}(\RR^{n})=S^{n}\setminus\{(0,\dots,0,\pm1)\}$.
  Hence together their union,
  \begin{equation}
\varphi_{+}(\RR^{n})\cup\varphi_{-}(\RR^{n}) = S^{n},
  \end{equation}
cover $S^{n}$.
\end{proof}

\begin{remark}
We \emph{should} prove the Hausdorff axiom is satisfied, but since this
is the same topology as the usual sphere $S^{n}$, we know it is Hausdorff.
\end{remark}

\begin{example}[Projective spaces]\label{ex:manifold:projective-space}
The $n$-dimensional (real) projective space $\RP^{n}$ is defined as the
space of all straight lines in $\RR^{n+1}$ passing through the origin.
We claim $\RP^{n}$ is a smooth $n$-dimensional manifold.
\end{example}

Note: we will denote a ``point'' in $\RP^{n}$ by the $n+1$ tuple,
delimited by square brackets, separating components by colons, for
example:
\begin{equation}
[y_{1} : y_{2} : \cdots : y_{n+1}]\in\RP^{n}.
\end{equation}


\begin{proof}
  Define the maps $\varphi_{i}\colon\RR^{n}\to\RP^{n}$ by
sending $(x_{1},\dots,x_{n})\in\RR^{n}$ to the line passing through the
origin and the point $(x_{1},\dots,x_{i-1},1,x_{i},\dots,x_{n})$.
So, connecting our notations, we also have $\varphi_{i}(x_{1},\dots,x_{n})=[x_{1} : \dots : x_{i-1} : 1 : x_{i} : \dots : x_{n}]$.

\smallbreak
  \textsc{Claim 1:} The $(\varphi_{i},\RR^{n})$ are $n+1$ charts.

\noindent\textit{Proof of claim 1}. We know $\RR^{n}$ is an open subset
of $\RR^{n}$, so we just need to prove $\varphi_{i}$ is injective. We
see
\begin{multline}
  \varphi_{i}(x_{1},\dots,x_{n})=\varphi_{i}(y_{1},\dots,y_{n})\\
\iff (x_{1},\dots,x_{i-1},1,x_{i},\dots,x_{n})
= (y_{1},\dots,y_{i-1},1,y_{i},\dots,y_{n}),
\end{multline}
hence $x_{1}=y_{1}$, \dots, $x_{i-1}=y_{i-1}$, $x_{i}=y_{i}$, \dots, $x_{n}=y_{n}$.
Hence $\varphi_{i}$ is injective.

\smallbreak
  \textsc{Claim 2:} For any $i,j=1,\dots,n+1$ we claim the charts
$(\varphi_{i},\RR^{n})$ and $(\varphi_{j},\RR^{n})$ are compatible.

\noindent\textit{Proof of claim 2}. To prove compatibility, let us study
the preimage of charts on overlaps,
construct the transition functions, prove they are smooth, and finally
prove the preimage of overlaps under charts are open subsets of $\RR^{n}$.

\textsc{Step 1: Preimage of Overlapping Charts.}
Let $\ell\in\RP^{n}$ be a line passing through $\vec{y}=(y_{1},\dots,y_{n+1})\in\RR^{n+1}$
with $y_{i}\neq0$ and $y_{j}\neq0$ (and $i\neq j$). We have two separate
subcases, depending on if $i<j$ or if $i>j$.

\textsc{Subcase 1 ($j<i$):}
Then the line passes through the point
\begin{equation}
\vec{y}' = \left(\frac{y_{1}}{y_{i}}, \dots, \frac{y_{j-1}}{y_{i}}, \frac{1}{y_{i}},
\frac{y_{j+1}}{y_{i}}, \dots, \frac{y_{i-1}}{y_{i}}, {\color{ForestGreen}1},
\frac{y_{i+1}}{y_{i}}, \dots, \frac{y_{n}}{y_{i}}\right).
\end{equation}
We highlighted in green the component we'd discard to obtain the
preimage of the point under $\varphi_{i}^{-1}(\vec{y}')$. Explicitly,
\begin{equation}
\varphi_{i}^{-1}(\vec{y}') = \left(\frac{y_{1}}{y_{i}}, \dots, \frac{y_{j-1}}{y_{i}}, \frac{1}{y_{i}},
\frac{y_{j+1}}{y_{i}}, \dots, \frac{y_{i-1}}{y_{i}},
\frac{y_{i+1}}{y_{i}}, \dots, \frac{y_{n}}{y_{i}}\right).
\end{equation}
The collection of such lines lie in the image of
$\varphi_{i}(V_{j})$ where % and in $\varphi_{j}(V_{i})$ where
\begin{equation}
V_{k} = \{\,(x_{1},\dots,x_{n})\in\RR^{n}\mid x_{k}\neq 0\,\},
\end{equation}
for any $k=1,\dots,n$.
Hence
\begin{equation}\label{eq:pf:projective-space-is-manifold:chart-compatibility:preimage1}
\varphi_{i}^{-1}\bigl(\varphi_{i}(\RR^{n})\cap\varphi_{j}(\RR^{n})\bigr)=V_{j}\quad\mbox{if }
j < i.
\end{equation}

\textsc{Subcase 2 ($j>i$):}
We see $\varphi_{i}^{-1}\bigl(\varphi_{i}(\RR^{n})\cap\varphi_{j}(\RR^{n})\bigr)$
is determined by lines $\ell\in\RP^{n}$ passing through the point,
\begin{equation}
  \vec{y}' = \left(\frac{y_{1}}{y_{i}},
  \dots,
  \frac{y_{i-1}}{y_{i}}, {\color{ForestGreen}1}, \frac{y_{i+1}}{y_{i}},
  \dots,
  \frac{y_{j-1}}{y_{i}}, \frac{1}{y_{i}}, \frac{y_{j+1}}{y_{i}},
  \dots,
  \frac{y_{n}}{y_{i}}\right).
\end{equation}
We highlighted in dark green the component we would discard when
obtaining the preimage $\varphi_{i}^{-1}(\vec{y}')$. Explicitly,
\begin{equation}
\varphi_{i}^{-1}(\vec{y}') = \left(\frac{y_{1}}{y_{i}},
  \dots,
  \frac{y_{i-1}}{y_{i}}, \frac{y_{i+1}}{y_{i}},
  \dots,
  \frac{y_{j-1}}{y_{i}}, \frac{1}{y_{i}}, \frac{y_{j+1}}{y_{i}},
  \dots,
  \frac{y_{n}}{y_{i}}\right).
\end{equation}
In particular, $\varphi_{i}^{-1}(\vec{y}')$ has its $j-1$ component be $1/y_{i}$.
Thus this corresponds to
\begin{equation}\label{eq:pf:projective-space-is-manifold:chart-compatibility:preimage2}
\varphi_{i}^{-1}\bigl(\varphi_{i}(\RR^{n})\cap\varphi_{j}(\RR^{n})\bigr)=V_{j-1}
  \quad\mbox{for } i<j
\end{equation}

We combine
Equations~\eqref{eq:pf:projective-space-is-manifold:chart-compatibility:preimage1}
and \eqref{eq:pf:projective-space-is-manifold:chart-compatibility:preimage2}
gives us,
\begin{equation}
  \varphi_{i}^{-1}\bigl(\varphi_{i}(\RR^{n})\cap\varphi_{j}(\RR^{n})\bigr)=\begin{cases}
  V_{j-1} & \mbox{if } i < j\\
  V_{j}  & \mbox{if } i > j.
  \end{cases}
\end{equation}

\textsc{Step 2: Transition Functions.}
If $i<j$, we have the transition function be
$\varphi_{j}^{-1}\circ\varphi_{i}\colon V_{j-1}\to V_{i}$ defined
explicitly by:
\begin{subequations}
\begin{align}
&(\varphi_{j}^{-1}\circ\varphi_{i})(x_{1},\dots,x_{n})\nonumber\\
&=\varphi_{j}^{-1}([x_{1} : \cdots : x_{i-1} : 1 : x_{i+1} : \cdots : x_{j-2} : x_{j-1} : \cdots : x_{n}])\\
&=\varphi_{j}^{-1}\left(\left[\frac{x_{1}}{x_{j-1}} : \cdots :
    \frac{x_{i-1}}{x_{j-1}} : \frac{1}{x_{j-1}} : \frac{x_{i+1}}{x_{j-1}} : \cdots : \frac{x_{j-2}}{x_{j-1}} : \frac{x_{j-1}}{x_{j-1}} : \cdots : \frac{x_{n}}{x_{j-1}}
    \right]\right)\\
&=\varphi_{j}^{-1}\left(\left[\frac{x_{1}}{x_{j-1}} : \cdots :
    \frac{x_{i-1}}{x_{j-1}} : \frac{1}{x_{j-1}} : \frac{x_{i+1}}{x_{j-1}} : \cdots : \frac{x_{j-2}}{x_{j-1}} : 1 : \frac{x_{j}}{x_{j-1}} :\cdots : \frac{x_{n}}{x_{j-1}}
    \right]\right)\\
&= \left(\frac{x_{1}}{x_{j-1}}, \cdots,
    \frac{x_{i-1}}{x_{j-1}}, \frac{1}{x_{j-1}}, \frac{x_{i+1}}{x_{j-1}}, \cdots, \frac{x_{j-2}}{x_{j-1}}, \frac{x_{j}}{x_{j-1}}\cdots, \frac{x_{n}}{x_{j-1}}\right)
\end{align}
\end{subequations}
Since $V_{j-1}$ has $x_{j-1}\neq0$, it follows this transition function
is smooth.

If $i>j$, then we have
$\varphi_{j}^{-1}\circ\varphi_{i}\colon V_{j}\to V_{i}$ defined
explicitly by:
\begin{equation}
(\varphi_{j}^{-1}\circ\varphi_{i})(x_{1},\dots,x_{n})
= \left(\frac{x_{1}}{x_{j}}, \dots,
\frac{x_{j-1}}{x_{j}}, \frac{x_{j+1}}{x_{j}},
\dots,
\frac{x_{i-1}}{x_{j}}, \frac{1}{x_{j}}, \frac{x_{i+1}}{x_{j}},
\dots,
\frac{x_{n}}{x_{j}}\right).
\end{equation}
As with the previous case, this is clearly smooth.

\textsc{Step 3: $V_{j}$ are open sets.} We see that
\begin{equation}
V_{j} = \RR^{n}\setminus\underbrace{\{\,(x_{1},\dots,x_{n})\in\RR^{n}\mid
x_{j}=0\,\}}_{\text{closed}}.
\end{equation}
Hence the $V_{j}$ are open subsets of $\RR^{n}$ in the standard
topology.

Thus, combining these three steps, we see the charts are compatible by Definition~\ref{defn:chart:compatible}.

\smallbreak
  \textsc{Claim 3:} The collection $\atlas{A}=\{\,(\varphi_{i},\RR^{n})\mid i=1,\dots,n+1\,\}$
form an atlas.

\noindent\textit{Proof of claim 3}. It suffices to prove $\atlas{A}$
covers $\RP^{n}$, by Definition~\ref{defn:atlas}. Let $\ell\in\RP^{n}$
be a line passing through the origin and the point
\begin{equation}
\vec{y} = (y_{1},\dots,y_{n+1})\in\RR^{n+1}
\end{equation}
with $y_{j}\neq0$ for an arbitrary (but fixed) $j=1,\dots,n+1$.
Then $\ell\in\varphi_{j}(\RR^{n})$. Since $j$ was arbitrary, this
exhausts $\ell\in\RP^{n}$. Hence
\begin{equation}
\bigcup_{j}\varphi_{j}(\RR^{n})=\RP^{n}.
\end{equation}
Thus the collection of charts form an atlas. We should prove the maximal
atlas containing our atlas
satisfies the Hausdorff property, but it is obvious.
\end{proof}

\begin{example}[Grassmann Manifolds]
For any $m,n\in\NN$ we denote by $\Grassmann(n,m)$ the set of all
$n$-dimensional vector subspaces of the vector space $\RR^{m+n}$.
(in particular, $\Grassmann(0,m)$ and $\Grassmann(n,0)$ are one-point
sets and $\Grassmann(1,n)$ is $\RP^{n}$.) We claim $\Grassmann(n,m)$ is
an $mn$-dimensional manifold known as the \define{Grassmann Manifold}.
\end{example}

An element $\mathcal{Y}$ of $\Grassmann(n,m)$ is an $n$-dimensional
subspace of $\RR^{m+n}$. We can specify such a subspace by a choice of
$n$ basis vectors $y_{1}$, \dots, $y_{n}$ such that they span $\mathcal{Y}$.
When we pick some ordering for the basis, we can assemble them as column
vectors for a $(m+n)\times n$ matrix $Y$. The image of $Y$ is indeed
the subspace $\mathcal{Y}$, and equivalently $\mathcal{Y}$ is the
column-space for $Y$. Note there are infinitely many possible $Y$
for any given $\mathcal{Y}$.\footnote{There are at least three other
ways to think of Grassmann manifolds, for a review see
\arXiv{2011.13699}.}

Our proof that $\Grassmann(n,m)$ is an $mn$-dimensional manifold will be
to first construct a generic chart as a neighborhood for any element
$\mathcal{Y}$ of $\Grassmann(n,m)$, then construct an atlas using this scheme.

\begin{proof}[Proof (Construction of Chart for $\Grassmann(n,m)$)]
For an element $\mathcal{Y}\in\Grassmann(n,m)$, to form a chart
containing $\mathcal{Y}$ we need a map
$\varphi\colon U\subset\RR^{nm}\to\Grassmann(n,m)$. We will cheat and
use the fact that the space of linear transformations from $\mathcal{Y}$
to $\mathcal{Y}^{\bot}$ is an $mn$-dimensional real space denoted
$\mathcal{L}(\mathcal{Y},\mathcal{Y}^{\bot})$. Then our chart can be
defined using a map $\varphi_{\mathcal{Y}}\colon\mathcal{L}(\mathcal{Y},\mathcal{Y}^{\bot})\to\Grassmann(n,m)$.
We can write such a map as
\begin{equation}
\varphi_{\mathcal{Y}}(\alpha) = (\id_{\mathcal{Y}}\oplus\alpha)(\mathcal{Y})
\end{equation}
where we have $\id_{\mathcal{Y}}\oplus\alpha$ regarded as a mapping
$\mathcal{Y}\to\mathcal{Y}\oplus\mathcal{Y}^{\bot}=\RR^{m+n}$. More
bluntly, $\varphi_{\mathcal{Y}}(\alpha)$ is the graph of $\alpha$ in
$\mathcal{Y}\oplus\mathcal{Y}^{\bot}$. We see that $\varphi_{\mathcal{Y}}$
is injective since two graphs are equal implies their maps are equal.
Hence we obtain a chart $(\mathcal{L}(\mathcal{Y},\mathcal{Y}^{\bot}),\varphi_{\mathcal{Y}})$
which is a neighborhood for any element $\mathcal{Y}$ of $\Grassmann(n,m)$
as desired.
\end{proof}

\begin{proof}[Proof ($\Grassmann(n,m)$ is a manifold)]
We have, from the preceding proof, a family of charts $\atlas{A}$ for
$\Grassmann(n,m)$. We claim $\atlas{A}$ is an atlas. We see that
$\atlas{A}$ covers $\Grassmann(n,m)$ since $\mathcal{Y}=\varphi_{\mathcal{Y}}(0)$.
We just need to prove these charts are compatible. It suffices to show
an arbitrary pair of charts
$(\mathcal{L}(\mathcal{Y},\mathcal{Y}^{\bot}),\varphi_{\mathcal{Y}})$ and
$(\mathcal{L}(\mathcal{Z},\mathcal{Z}^{\bot}),\varphi_{\mathcal{Z}})$
are compatible.

We need to check
\begin{equation}
F=\varphi_{\mathcal{Z}}^{-1}\circ\varphi_{\mathcal{Y}}\colon
\varphi_{\mathcal{Y}}^{-1}\Bigl(\varphi_{\mathcal{Z}}\bigl(\mathcal{L}(\mathcal{Z},\mathcal{Z}^{\bot})\bigr)\Bigr)\to
\varphi_{\mathcal{Z}}^{-1}\Bigl(\varphi_{\mathcal{Y}}\bigl(\mathcal{L}(\mathcal{Y},\mathcal{Y}^{\bot})\bigr)\Bigr)
\end{equation}
is smooth. What does this even do? Well, let $\alpha\in\mathcal{L}(\mathcal{Y},\mathcal{Y}^{\bot})$
and consider $\beta\in\mathcal{L}(\mathcal{Z},\mathcal{Z}^{\bot})$, then
the equality $F(\alpha)=\beta$ means $\varphi_{\mathcal{Y}}(\alpha)=\varphi_{\mathcal{Z}}(\beta)$.
Let us not forget: we want to prove $F$ is smooth.

Let $\pi_{\mathcal{Z}}\colon\RR^{n+m}\to\mathcal{Z}$ be the linear
projection onto the subspace $\mathcal{Z}$. Consider $f_{\alpha}\colon\mathcal{Y}\to\mathcal{Z}$
be such that
\begin{equation}
f_{\alpha} = \pi_{\mathcal{Z}}\circ(\id_{\mathcal{Y}} + \alpha).
\end{equation}
We will prove two things:
\begin{enumerate}
\item $f_{\alpha}$ is invertible (which establishes $\varphi^{-1}_{\mathcal{Y}}\bigl(\varphi_{\mathcal{Z}}(\mathcal{L}(\mathcal{Z},\mathcal{Z}^{\bot}))\bigr)$
is an open subset of $\mathcal{L}(\mathcal{Y},\mathcal{Y}^{\bot})$), and
\item for every element $y$ in the subspace $\mathcal{Y}$,
  we have $y+\alpha(y) = f_{\alpha}(y) + \beta\bigl(f_{\alpha}(y)\bigr)$.
\end{enumerate}
We see this second condition is equivalent to stating
\begin{equation}
(\id_{\mathcal{Y}} + \alpha)(y) = \bigl((\id_{\mathcal{Z}} + \beta)\circ f_{\alpha}\bigr)(y),
\quad\mbox{or}\quad \id_{\mathcal{Y}} + \alpha = (\id_{\mathcal{Z}} + \beta)\circ f_{\alpha}.
\end{equation}
By virtue of $f_{\alpha}$ being invertible, we compose on the right by
its inverse, and find
\begin{equation}
(\id_{\mathcal{Y}} + \alpha)\circ f_{\alpha}^{-1} = \id_{\mathcal{Z}} + \beta.
\end{equation}
Solving for $\beta$ gives us
\begin{equation}
\beta = F(\alpha) = (\id_{\mathcal{Y}} + \alpha)\circ f_{\alpha}^{-1} - \id_{\mathcal{Z}},
\end{equation}
and in particular the image of $\beta$ is contained in
$\mathcal{Z}^{\bot}$.

Did we prove $F$ is smooth? Well, it is the sum of smooth functions;
$f_{\alpha}$ is the composition of two smooth functions, so
$(\id_{\mathcal{Y}} + \alpha)\circ f_{\alpha}^{-1}$ is smooth.
\end{proof}

\begin{remark}
We have given $\Grassmann(n,m)$ an infinite atlas with charts labeled by
$\mathcal{Y}\in\Grassmann(n,m)$. Do we need all these charts? In fact,
we must have a finite subatlas, by
Definition~\ref{defn:manifold:atlas-version} for a manifold (the
paracompactness condition). It is sufficient to consider $\binom{n+m}{n}$ charts
corresponding to subspaces $\mathcal{Y}$ spanned with $n$ basis vectors.
This perspective generalizes the method for constructing the atlas for
the real projective space in Example~\ref{ex:manifold:projective-space}.
\end{remark}

\endinput
