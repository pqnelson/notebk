\section{Manifold}
\subsection{Charts}

\begin{definition}
Let $M$ be a set. An \define{$n$-Dimensional Chart} on $M$
is a pair $(U,\varphi)$ consisting of
\begin{enumerate}
\item the \emph{Patch}: an open subset $U$ of $\RR^{n}$, and
\item the \emph{Parametrization}: an injective function $\varphi\colon U\to M$
  (in particular, $\varphi\colon U\to \varphi(U)$ is a bijection).
\end{enumerate}
\end{definition}

\begin{figure}[H]
  \centering
  \includegraphics{img/img.0}
  \caption{Chart on $M$}
\end{figure}

\begin{remark}\label{rmk:chart:local-coordinates}
We call the inverse function of $\varphi$ --- that is, $\varphi^{-1}$
which maps a subset of $M$ to $\RR^{n}$ --- the \define{Local Coordinates}
of the chart $(U,\varphi)$.
\end{remark}

\begin{remark}
The important aspect of the mapping $\varphi$ is that it's a bijection
$\varphi\colon U\to\varphi(U)$. Some authors\footnote{Followers of the
$V\subset M\to\RR^{n}$ convention include Bourbaki (see, e.g., \textit{Lie Groups} III \S1.1 Lemma 1),
Warner's \textit{Foundations of Differential Geometry and Lie Groups}
(Definition 1.3), Christopher Isham's
\textit{Modern Differential Geometry for Physicists} (Def.~2.1 in \S2.2),
Spivak's \textit{Comprehensive Introduction to Differential Geometry} (vol. I).

The adherents to the $U\subset\RR^{n}\to M$ convention are no less
impressive: Milnor, do Carmo's \textit{Riemannian Geometry}, Arthur Besse's
\textit{Einstein Manifolds} (\S1.41), among others.}
reverse the direction of the mapping, using ``$U\subset M$ and
$\varphi\colon U\to\RR^{n}$ is injective'' as the definition for a
chart. It really doesn't matter since, in that case, the important
aspect is that $\varphi\colon U\to\varphi(U)$ is a bijection.
\end{remark}

\begin{remark}
We may get arbitrarily abstract and general with the definition of a
chart. For example, we could replace $\RR^{n}$ with any
finite-dimensional affine space
$\mathbb{A}^{n}$. Daniel Freed~\cite{freed2022:notes} does this, for example.
Recall, a real affine space $\mathbb{A}^{n}$ is just the $n$-dimensional
real vector space with the origin forgotten --- in particular, we cannot
add elements of an affine space together, \emph{but} we can subtract
them to obtain a vector $a,b\in\mathbb{A}^{n}$, $a-b=\vec{s}\in\RR^{n}$.
We can also add a vector to a point in an affine space to translate
points $b+\vec{s}=a$. Now replace $\RR^{n}$ with a generic vector space,
and the same recipe works to obtain an affine space.
\end{remark}

\begin{remark}[Smooth? Continuous?]
It may be natural to ask if $\varphi$ is smooth or continuous, or if
we require $\varphi^{-1}$ to be smooth or continuous? But we do not have
a notion of ``smooth functions on $M$'' yet: $M$ is just ``some set''.
\end{remark}

\begin{definition}\label{defn:chart:compatible}
We call two $n$-dimensional charts on $M$, $(U,\varphi)$ and $(V,\psi)$,
\define{Compatible} if the following four conditions all hold (confer
with Figure~\ref{fig:chart:compatibility}):
\begin{enumerate}
\item the set $\varphi^{-1}\left(\psi(V)\right)\subset U$ is open (in
  $U$, hence in $\RR^{n}$);
\item the set $\psi^{-1}\left(\varphi(U)\right)\subset V$ is open (in $V$,
  hence in $\RR^{n}$);
\item\label{property:chart-compatible:transition-fun1} the map $\varphi^{-1}\left(\psi(V)\right)\to\psi^{-1}\left(\varphi(U)\right)$,
$u\mapsto(\psi^{-1}\circ\varphi)(u)$, is smooth\footnote{We will follow
mathematical tradition, and use ``smooth'' to mean $C^{\infty}$ and, in
particular, continuous. But you could have $C^{k}$ for some
$k\in\NN_{0}$, or analytic $C^{\omega}$, or whatever. These result in
$C^{k}$-atlases and $C^{k}$-manifolds, or analytic atlases and analytic
manifolds. For $C^{0}$ continuous functions, we get topological charts
and topological manifolds.}; and
\item\label{property:chart-compatible:transition-fun2} the map $\psi^{-1}\left(\varphi(U)\right)\to\varphi^{-1}\left(\psi(V)\right)$,
$v\mapsto(\varphi^{-1}\circ\psi)(v)$ is smooth.
\end{enumerate}
In particular, $(U,\varphi)$ and $(V,\psi)$ are compatible if $\varphi(U)\cap\psi(V)=\emptyset$.
\end{definition}

\begin{figure}[H]
  \centering
  \includegraphics{img/img.1}
  \caption{Compatible charts on $M$, with one particular transition function drawn.}\label{fig:chart:compatibility}
\end{figure}

\begin{remark}[Compatibility on Patches]
The intuition is that we have the first two conditions be topological
constraints on the possible choices of charts --- if the image of a
couple of charts intersect, they should have an open preimage.
This will be useful later, when we have a ``large enough'' family of
compatible charts whose images cover $M$, then we can induce a topology
on $M$ thanks to the first two criteria in our definition of
``compatible charts''.
\end{remark}

\begin{remark}[Compatibility and Transition Functions]
The latter two conditions concern transition functions
\begin{equation*}
(\psi^{-1}\circ\varphi)\colon\varphi^{-1}\left(\psi(V)\right)\to\psi^{-1}\left(\varphi(U)\right)
\end{equation*}
and
\begin{equation*}
(\varphi^{-1}\circ\psi)\colon\psi^{-1}\left(\varphi(U)\right)\to\varphi^{-1}\left(\psi(V)\right).
\end{equation*}
These describe change of coordinates on the overlap of two charts.
We require any change of coordinates to be continuous.
\end{remark}

\begin{remark}
The four conditions (of chart compatibility) form an equivalence relation
among charts.
\end{remark}

\subsection{Atlases}

\begin{definition}\label{defn:atlas}
Let $M$ be a set. We define an ($n$-dimensional) \define{Atlas} on $M$
to consist of a set of $n$-dimensional charts
$\atlas{A}=\{\,(U_{\alpha},\varphi_{\alpha})\mid\alpha\in A\,\}$
such that
\begin{enumerate}
\item the images of the charts cover $M$: $\displaystyle\bigcup_{\alpha\in{A}}\varphi_{\alpha}(U_{\alpha})=M$
\item for any $\alpha,\beta\in A$, the charts
  $(U_{\alpha},\varphi_{\alpha})$ and $(U_{\beta},\varphi_{\beta})$
  [both in $\atlas{A}$] are compatible.
\end{enumerate}
\end{definition}

\begin{remark}[Etymology]
The terminology ``atlas'' is borrowed from cartography. Back in the dark
ages, before Google Maps [i.e., before February 8, 2005], primitive humans collected paper maps and
bound them in book-form. These books were called ``atlases'', the pages
were called ``charts''.
\end{remark}

\begin{remark}[Set versus Proper Class]\label{rmk:atlas:is-a-set}
The curious reader may wonder if an atlas is really an honest set, or if
it's a proper class. A chart is an element of the collection of all
possible (subset $U$ of $\RR^{n}$, function $U\to M$) pairs, i.e., an
element of the collection:
\begin{equation}
\begin{split}
C(M) &= \{\,(U,\varphi)\mid U\in\powerset(\RR^{n}),\varphi\colon U\to M\,\}\\
&=\bigcup_{U\in\powerset(\RR^{n})}\{U\}\times\hom(U,M).
\end{split}
\end{equation}
But this is clearly a set, not a proper class: $\{U\}\times\hom(U,M)$ is the
product of sets (which is a set), and we take the union of an indexed
family of sets (which is, again, a set). The ZF axioms assure us this is
a set.

An atlas is a subcollection of $C(M)$, or equivalently an element of
$\powerset(C(M))$. By the axioms of ZF
set theory, this is a set, not a proper class. We may rest assured, an
atlas \emph{is} a set, \emph{not a proper class}.
\end{remark}

\begin{definition}
Let $\atlas{A}$, $\atlas{B}$ be two $n$-dimensional atlases of $M$.
We call $\atlas{A}$ a \define{Subatlas} of $\atlas{B}$ if $\atlas{A}\subset\atlas{B}$.
If further $\atlas{A}\neq\atlas{B}$, we call it a \emph{proper subatlas}.
\end{definition}

\begin{definition}\label{defn:atlas:equivalence}
We call two ($n$-dimensional) atlases $\atlas{A}$ and $\atlas{B}$
for $M$ \define{Equivalent} if their union $\atlas{A}\cup\atlas{B}$
is another atlas [or, equivalently, if every chart in $\atlas{A}$ is
compatible with every chart in $\atlas{B}$ and vice-versa].
\end{definition}

\begin{remark}
Although the relation of ``equivalent atlases'' is obviously symmetric
and reflexive, it is not yet established that it is a transitive relation.
We cannot say \emph{at present} that it forms an equivalence relation.
This motivates the first thing we'll do.
\end{remark}

\begin{remark}[Equivalence class of atlases is set, not proper class]\label{rmk:atlas:equivalence-class-is-proper-set}
From Remark~\ref{rmk:atlas:is-a-set}, we know that the collection of all
possible atlases forms a set, a subset of $\powerset(C(M))$.
If we have an atlas $\atlas{A}$ on $M$, the collection of atlases
equivalent to $\atlas{A}$ is a subcollection of a set, which is a set,
\emph{not a proper class}.
\end{remark}

\begin{lemma}
Let $\atlas{A}$ be an $n$-dimensional atlas for $M$, let $(U,\varphi)$ and $(V,\psi)$
be $n$-dimensional charts for $M$ (but they do not belong to $\atlas{A}$).
If $(U,\varphi)$ and $(V,\psi)$ are compatible with every chart in $\atlas{A}$,
then they are compatible with each other.
\end{lemma}

We only really need to prove properties
\ref{property:chart-compatible:transition-fun1} and
\ref{property:chart-compatible:transition-fun2} from the definition of
compatible charts, since open-ness should be obvious.

\begin{proof}
It suffices to prove $\varphi^{-1}\circ\psi$ is smooth (the proof for
$\psi\circ\varphi^{-1}$ being smooth is the same).
Let $u\in\varphi^{-1}(\psi(V))\subset U$ be arbitrary,
let $x=\varphi(u)\in M$.
Now, consider \emph{any arbitrary} chart $(W,\chi)\in\atlas{A}$ such that $x\in\chi(W)$.
Set $x=\chi(w)$ for $w\in W$. Make note that $\chi^{-1}(x)=w$. We see
\begin{equation}
\psi^{-1}(\varphi(u)) = \psi^{-1}(x) = \psi^{-1}(\chi(w))=\psi^{-1}(\chi(\chi^{-1}(\varphi(u))))
\end{equation}
which implies $\psi^{-1}\circ\varphi$ is smooth at a neighborhood of $u$,
since it is a composition of (suitable restrictions of) smooth maps
$\psi^{-1}\circ\chi$ and $\chi^{-1}\circ\varphi$.

Since we have established the claim is ``locally true'' for any
arbitrary $x\in\varphi(U)\cap\psi(V)$ and every chart $(W,\chi)$ whose
image contains $x$, then it's true that $\varphi^{-1}\circ\psi$ is
smooth on $(\psi^{-1}\circ\varphi)(U)$.
\end{proof}

\begin{corollary}
Let $\atlas{A}$, $\atlas{B}$, $\atlas{C}$ be $n$-dimensional
atlases on $M$. If $\atlas{A}$ is equivalent to $\atlas{B}$, and if
$\atlas{B}$ is equivalent to $\atlas{C}$, then $\atlas{A}$ is
equivalent to $\atlas{C}$.
\end{corollary}

\begin{proof}
Let $(U_{\alpha},\varphi_{\alpha})\in\atlas{A}$ be an arbitrary chart,
let $(V_{\gamma},\psi_{\gamma})\in\atlas{C}$ be an arbitrary chart.
Since $\atlas{A}$ is equivalent to $\atlas{B}$, we know
$(U_{\alpha},\varphi_{\alpha})$ is compatible with every chart in $\atlas{B}$.
Since $\atlas{B}$ is equivalent to $\atlas{C}$, we know
$(V_{\gamma},\psi_{\gamma})$ is compatible with every chart in $\atlas{B}$.
By the previous lemma, we conclude that any arbitrary
$(U_{\alpha},\varphi_{\alpha})\in\atlas{A}$ is compatible to any arbitrary
chart $(V_{\gamma},\psi_{\gamma})\in\atlas{C}$. But by Definition~\ref{defn:atlas:equivalence} for
equivalent atlases, this implies $\atlas{A}$ is equivalent to $\atlas{C}$.
\end{proof}

\subsection{Topology}

\begin{definition}
Let $M$ be a set with an $n$-dimensional atlas $\atlas{A}$. A subset
$B\subset M$
is called \define{Open (with respect to $\atlas{A}$)} if for any chart
$(U,\varphi)\in\atlas{A}$, the set $\varphi^{-1}(B)\subset U\subset\RR^{n}$
is open in $U$ (hence open in $\RR^{n}$).
In particular, the sets $\varphi(U)$ are open subsets of $M$.
\end{definition}

\begin{remark}
Since morphisms of topological spaces (a.k.a., ``continuous maps'')
require the preimage of open sets be open, and we began ``from scratch''
working with an arbitrary set $M$, the reader can see why charts are
defined the way they are $\varphi\colon U\subset\RR^{n}\to M$. If we
demand $\varphi$ be continuous, then we work \emph{with} the notions of
topology.
\end{remark}

\begin{remark}
It is easy to see that sets open with respect to a particular atlas
$\atlas{A}$ form a topology on $M$.
\end{remark}

\begin{proposition}
If two atlases $\atlas{A}$ and $\atlas{B}$ are equivalent, then a set
$B\subset M$ is open with respect to $\atlas{A}$ if and only if it is
open with respect to $\atlas{B}$.
\end{proposition}

\begin{proof}
  For any $B\subset M$, we have
  \begin{equation}
B = \bigcup_{(V,\psi)\in\atlas{B}}(B\cap\psi(V))
  = \bigcup_{(V,\psi)\in\atlas{B}}\psi\bigl(\psi^{-1}(B)\bigr).
  \end{equation}
If $B$ is open with respect to $\atlas{B}$, then all the sets
$\psi^{-1}(B)$ are open.

Let $(U,\varphi)$ be a chart of $\atlas{A}$. Then
\begin{subequations}
\begin{align}
\varphi^{-1}(B)
&=\varphi^{-1}\left(\bigcup_{(V,\psi)\in\atlas{B}}\psi\bigl(\psi^{-1}(B)\bigr)\right)\\
&= \bigcup_{(V,\psi)\in\atlas{B}}\varphi^{-1}\left(\psi\bigl(\psi^{-1}(B)\bigr)\right)\\
&= \bigcup_{(V,\psi)\in\atlas{B}} (\psi^{-1}\circ\varphi)^{-1}\bigl(\psi^{-1}(B)\bigr).
\end{align}
\end{subequations}
Since $\psi^{-1}\circ\varphi$ is smooth (hence continuous), the last
line shows $B$ is a union of open sets; hence it is open with respect to
$\atlas{A}$. Since this is true for arbitrary charts $(U,\varphi)\in\atlas{A}$,
it follows the set $B$ is open with respect to $\atlas{A}$.
\end{proof}

\begin{remark}
This proposition shows that an equivalence class of atlases on $M$ makes
$M$ a topological space. This permits us to speak of topological
properties of $M$ like compactness, connectedness, and so forth.
We say $M$ is \emph{compact} if every atlas contains a finite subatlas;
$M$ is \emph{connected} if for any pair of points $p,q\in M$ there
exists a finite sequence of charts $\{(U_{i},\varphi_{i}),i=1,\dots,n\}$
such that $p\in\varphi_{1}(U_{1})$, $q\in\varphi_{n}(U_{n})$, $U_{i}$ is
connected for $i=1,\dots,n$, and $\varphi_{i}(U_{i})\cap\varphi_{i+1}(U_{i+1})$
is nonempty for $i=1,\dots,n-1$.
\end{remark}

\subsection{Manifolds}

\begin{definition}[Equivalence class version]
A class $\mathfrak{D}$ of equivalent $n$-dimensional atlases on $M$ is
called a \define{Differential Structure} on $M$ if the following
conditions hold:
\begin{enumerate}
\item Paracompactness: the class $\mathfrak{D}$ contains an at-most countable atlas;
\item Hausdorff: for any distinct points $p,q\in M$, there exists
  disjoint open $U,V\subset M$ such that $p\in U$ and $q\in V$.
\end{enumerate}
The charts of atlases from $\mathfrak{D}$ are simply called ``charts of $\mathfrak{D}$''.
\end{definition}

\begin{remark}
It is rather unwieldy to work with equivalence classes, and what we are
effectively doing is working with the union of all atlases in the
equivalence class (which, by Definition~\ref{defn:atlas:equivalence}
of atlas equivalence, produces an atlas --- a \emph{maximal} atlas).
\end{remark}

\begin{definition}[Atlas version]\label{defn:manifold:atlas-version}
An $n$-dimensional atlas $\atlas{D}$ of $M$ is called a \define{Differential Structure}
on $M$ if the following three conditions all hold:
\begin{enumerate}
\item Maximal: for any atlas $\atlas{A}$ compatible with $\atlas{D}$, we
  have $\atlas{A}\subset\atlas{D}$;
\item Paracompact: there is an at most countable subatlas of $\atlas{D}$;
\item Hausdorff: for any distinct points $p,q\in M$, there exists
  disjoint open $U,V\subset M$ such that $p\in U$ and $q\in V$.
\end{enumerate}
\end{definition}

\begin{remark}
We leave it as an exercise for the reader to prove these two definitions
are logically equivalent. We know from Remark~\ref{rmk:atlas:equivalence-class-is-proper-set}
the equivalence class of atlases is a set (not a proper class).
\end{remark}

\begin{theorem}[Every atlas is contained in a maximal atlas]\label{thm:atlas:contained-in-unique-maximal-atlas}
If $\atlas{A}$ is an atlas for $M$,
then there is a unique maximal atlas containing $\atlas{A}$.
\end{theorem}

We will prove the claim without resorting to Zorn's lemma.

\begin{proof}
We know $\atlas{A}$ is contained in an equivalence class of equivalent
atlases $\mathfrak{A}$. Their union forms an atlas
\begin{equation}
\overline{\atlas{A}}=\bigcup_{\atlas{B}\in\mathfrak{A}}\atlas{B}.
\end{equation}
The union of compatible atlases form an atlas, by Definition~\ref{defn:atlas:equivalence}.
Hence $\overline{\atlas{A}}$ is an atlas, and $\atlas{A}\subset\overline{\atlas{A}}$.
We should prove $\overline{\atlas{A}}$ is maximal, but it should be obvious.

We just need to prove $\overline{\atlas{A}}$ is the \emph{unique} maximal atlas
containing $\atlas{A}$. Let $\atlas{A}'$ be any arbitrary atlas
equivalent to $\atlas{A}$ such that
\begin{equation}
\atlas{A}\properSubset\atlas{A}'.
\end{equation}
Then $\atlas{A}'$ is compatible with $\atlas{A}$ by
Definition~\ref{defn:atlas:equivalence}. Hence
$\atlas{A}'\in\mathfrak{A}$. Then
\begin{equation}
\atlas{A}'\properSubset\mathfrak{A}.
\end{equation}
Since $\atlas{A}'$ is arbitrary, this exhausts all possibilities for a
second, distinct maximal atlas $\atlas{A}'$.
\end{proof}

\begin{definition}
A \define{(Smooth) $n$-Dimensional Manifold} consists of
  a set $M$ equipped with an $n$-dimensional differential structure $\atlas{D}$.
\end{definition}

\begin{remark}
We could define a $C^{k}$ differentiable structure and $C^{k}$
$n$-dimensional manifold analogously. This works for analytic
(``$k=\omega$'') manifolds, too. For differential geometry, I believe
$k\geq2$ suffices.
\end{remark}

\begin{remark}[Non-uniqueness of differential structure]
If we may equip \emph{one} differential structure to a set $M$,
then it is natural to wonder if it is the \emph{only} differential
structure for $M$. The answer is negative.

John Milnor~\cite{milnor1956manifolds} found 7
differential structures for the sphere $S^{7}\subset\RR^{8}$ which are
not diffeomorphic to each other, but all of them are \emph{homeomorphic}
(but \emph{not} diffeomorphic) to the standard differential structure on
$S^{7}$.

In 1982, it turns out $\RR^{4}$ has uncountably many differential
structures. The first ones were published by
S.~Donaldson~\cite{donaldson1983self}, then M.~Friedman and colleagues extended
these results and produced many differential structures on $\RR^{4}$.
There are many techniques for constructing these exotic differential
structures, like Casson handles.

We call these ``unorthodox'' differential structures
\define{Exotic Differential Structures}.
\end{remark}

\subsection{Orientations}

\N{Local Coordinates}
We mentioned briefly in Remark~\ref{rmk:chart:local-coordinates}
that from a patch $(U,\varphi)$ we can construct local coordinates using
$\varphi^{-1}\colon\varphi(U)\to\RR^{n}$.
The $n$ components of $\varphi^{-1}$ are usually written as
$\{x_{1},\dots,x_{n}\}$.

If we have two overlapping charts $(U,\varphi)$ and $(V,\psi)$ from the
differential structure on $M$, with local coordinates $\{x_{1},\dots,x_{n}\}$
and $\{y_{1},\dots,y_{n}\}$ (respectively). Then
condition~\ref{property:chart-compatible:transition-fun1} in
Definition~\ref{defn:chart:compatible} amounts to stating
the $x_{j}$ are smooth functions of the $y_{i}$, and
condition~\ref{property:chart-compatible:transition-fun2} states
the $y_{i}$ are smooth functions of the $x_{j}$.
Moreover, the Jacobian matrices $[\partial x_{j}/\partial y_{i}]$
and $[\partial y_{i}/\partial x_{j}]$ are inverses of each other and, in
particular, invertible.

\begin{definition}
Let $M$ be a smooth manifold of dimension $n>0$,
let $(U,\varphi)$, $(V,\psi)$ be two charts on $M$,
and let $p\in\varphi(U)\cap\psi(V)$.
We say \define{the orientations of the charts agree at $p$} if
the determinant of the Jacobian matrix for the transformation
$(\varphi^{-1}\circ\psi)(V)\to(\psi^{-1}\circ\varphi)(U)$,
sending $x\mapsto\psi^{-1}(\varphi(x))$, at $\varphi^{-1}(p)$ is positive.
If the determinant is negative, then we say
\define{the Orientations Disagree}.
\end{definition}

\begin{definition}
The charts in the differential structure for smooth manifold $M$ of
dimension $n>0$ split into two classes, for each $p\in M$, since the
determinant of the Jacobian transformation is either positive or negative.
Within each class, the orientations agree at $p$, but any chart from one
class and any chart from the second class will disagree with the orientation
at $p$. These equivalence classes are called the \define{Orientations at $p$}.
\end{definition}

\begin{definition}
We call an atlas \define{Oriented} if the orientations of any two charts
$(U,\varphi)$ and $(V,\psi)$ agree at every point in $\varphi(U)\cap\psi(V)$.
\end{definition}

\begin{definition}
We say two oriented atlases \define{Determine the Same Orientation} if their
union is also an oriented atlas.
\end{definition}

\begin{definition}
An \define{Orientation} of a manifold is a class of atlases that
determine the same orientation. In other words, it is a \emph{maximal oriented atlas},
i.e., an oriented atlas such that it contains every chart whose
orientation agrees with that of any of its charts.
\end{definition}

\begin{remark}
There is another, more general, way to define an orientation for a
manifold. It involves sections of a fiber bundle. But since this is a
\emph{generalization} of what we have described, we will discuss it
later.
\end{remark}

\begin{definition}
For any orientation $\atlas{A}$ of an oriented manifold, there is the \define{Opposite}
orientation obtained by taking each chart $(U,\varphi)\in\atlas{A}$,
then forming $(\rho(U),\varphi\circ\rho)$ where $\rho$ is the reflection
$(x_{1}, x_{2},\dots, x_{n})\mapsto(-x_{1},x_{2},\dots,x_{n})$.
\end{definition}

\begin{proposition}
A connected orientable manifold has precisely two different
orientations, and they are opposite to each other. A disconnected
manifold is orientable if and only if all its components are orientable;
a choice of orientation of a disconnected manifold is the same as
(independent) simultaneous choices of orientations for each of its components.
\end{proposition}


\endinput
