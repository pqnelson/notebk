\section{Introduction}
\subsection{Manifold}
\subsubsection{Charts}

\begin{definition}
Let $M$ be a set. An \define{$n$-Dimensional Chart} on $M$
is a pair $(U,\varphi)$ consisting of
\begin{enumerate}
\item the \emph{Patch}: an open subset $U$ of $\RR^{n}$, and
\item the \emph{Parametrization}: an injective function $\varphi\colon U\to M$
  (in particular, $\varphi\colon U\to \varphi(U)$ is a bijection).
\end{enumerate}
\end{definition}

\begin{remark}
We call the inverse function of $\varphi$ --- that is, $\varphi^{-1}$
which maps a subset of $M$ to $\RR^{n}$ --- the \define{Local Coordinates}
of the chart $(U,\varphi)$.
\end{remark}

\begin{remark}
The important aspect of the mapping $\varphi$ is that it's a bijection
$\varphi\colon U\to\varphi(U)$. Some authors\footnote{Followers of the
$V\subset M\to\RR^{n}$ convention include Bourbaki (see, e.g., \textit{Lie Groups} III \S1.1 Lemma 1),
Warner's \textit{Foundations of Differential Geometry and Lie Groups} 
(Definition 1.3), Christopher Isham's
\textit{Modern Differential Geometry for Physicists} (Def.~2.1 in \S2.2),
Spivak's \textit{Comprehensive Introduction to Differential Geometry} (vol. I).

The adherents to the $U\subset\RR^{n}\to M$ convention are no less
impressive: Milnor, do Carmo's \textit{Riemannian Geometry}, Arthur Besse's
\textit{Einstein Manifolds} (\S1.41), among others.}
reverse the direction of the mapping, using ``$U\subset M$ and
$\varphi\colon U\to\RR^{n}$ is injective'' as the definition for a
chart. It really doesn't matter since, in that case, the important
aspect is that $\varphi\colon U\to\varphi(U)$ is a bijection.
\end{remark}

\begin{remark}
We may get arbitrarily abstract and general with the definition of a
chart. For example, we could replace $\RR^{n}$ with any
finite-dimensional affine space
$\mathbb{A}^{n}$. Daniel Freed~\cite{freed2022:notes} does this, for example.
Recall, a real affine space $\mathbb{A}^{n}$ is just the $n$-dimensional
real vector space with the origin forgotten --- in particular, we cannot
add elements of an affine space together, \emph{but} we can subtract
them to obtain a vector $a,b\in\mathbb{A}^{n}$, $a-b=\vec{s}\in\RR^{n}$.
We can also add a vector to a point in an affine space to translate
points $b+\vec{s}=a$. Now replace $\RR^{n}$ with a generic vector space,
and the same recipe works to obtain an affine space.
\end{remark}

\begin{remark}[Smooth? Continuous?]
It may be natural to ask if $\varphi$ is smooth or continuous, or if
we require $\varphi^{-1}$ to be smooth or continuous? But we do not have
a notion of ``smooth functions on $M$'' yet: $M$ is just ``some set''.
\end{remark}

\begin{definition}
We call two $n$-dimensional charts on $M$, $(U,\varphi)$ and $(V,\psi)$,
\define{Compatible} if the following four conditions all hold:
\begin{enumerate}
\item the set $\varphi^{-1}\left(\psi(V)\right)\subset U$ is open (in
  $U$, hence in $\RR^{n}$);
\item the set $\psi^{-1}\left(\varphi(U)\right)\subset V$ is open (in $V$,
  hence in $\RR^{n}$);
\item\label{property:chart-compatible:transition-fun1} the map $\varphi^{-1}\left(\psi(V)\right)\to\psi^{-1}\left(\varphi(U)\right)$,
$u\mapsto(\psi^{-1}\circ\varphi)(u)$, is smooth\footnote{We will follow
mathematical tradition, and use ``smooth'' to mean $C^{\infty}$ and, in
particular, continuous. But you could have $C^{k}$ for some
$k\in\NN_{0}$, or analytic $C^{\omega}$, or whatever. These result in
$C^{k}$-atlases and $C^{k}$-manifolds, or analytic atlases and analytic
manifolds. For $C^{0}$ continuous functions, we get topological charts
and topological manifolds.}; and
\item\label{property:chart-compatible:transition-fun2} the map $\psi^{-1}\left(\varphi(U)\right)\to\varphi^{-1}\left(\psi(V)\right)$,
$v\mapsto(\varphi^{-1}\circ\psi)(v)$ is smooth.
\end{enumerate}
In particular, $(U,\varphi)$ and $(V,\psi)$ are compatible if $\varphi(U)\cap\psi(V)=\emptyset$.
\end{definition}

\begin{remark}[Compatibility on Patches]
The intuition is that we have the first two conditions be topological
constraints on the possible choices of charts --- if the image of a
couple of charts intersect, they should have an open preimage.
This will be useful later, when we have a ``large enough'' family of
compatible charts whose images cover $M$, then we can induce a topology
on $M$ thanks to the first two criteria in our definition of
``compatible charts''.
\end{remark}

\begin{remark}[Compatibility and Transition Functions]
The latter two conditions concern transition functions
\begin{equation*}
(\psi^{-1}\circ\varphi)\colon\varphi^{-1}\left(\psi(V)\right)\to\psi^{-1}\left(\varphi(U)\right)
\end{equation*}
and
\begin{equation*}
(\varphi^{-1}\circ\psi)\colon\psi^{-1}\left(\varphi(U)\right)\to\varphi^{-1}\left(\psi(V)\right).
\end{equation*}
These describe change of coordinates on the overlap of two charts.
We require any change of coordinates to be continuous.
\end{remark}

\begin{remark}
The four conditions (of chart compatibility) form an equivalence relation
among charts.
\end{remark}

\subsubsection{Atlases}

\begin{definition}
Let $M$ be a set. We define an ($n$-dimensional) \define{Atlas} on $M$
to consist of a set of $n$-dimensional charts
$\mathcal{A}=\{\,(U_{\alpha},\varphi_{\alpha})\mid\alpha\in A\,\}$
such that
\begin{enumerate}
\item the images of the charts cover $M$: $\displaystyle\bigcup_{\alpha\in{A}}\varphi_{\alpha}(U_{\alpha})=M$
\item for any $\alpha,\beta\in A$, the charts
  $(U_{\alpha},\varphi_{\alpha})$ and $(U_{\beta},\varphi_{\beta})$
  [both in $\mathcal{A}$] are compatible.
\end{enumerate}
\end{definition}

\begin{remark}[Etymology]
The terminology ``atlas'' is borrowed from cartography. Back in the dark
ages, before Google Maps [i.e., before February 8, 2005], primitive humans collected paper maps and
bound them in book-form. These books were called ``atlases'', the pages
were called ``charts''.
\end{remark}

\begin{remark}[Set versus Proper Class]\label{rmk:atlas:is-a-set}
The curious reader may wonder if an atlas is really an honest set, or if
it's a proper class. A chart is an element of the collection of all
possible (subset $U$ of $\RR^{n}$, function $U\to M$) pairs, i.e., an
element of the collection:
\begin{equation}
\begin{split}
C(M) &= \{\,(U,\varphi)\mid U\in\powerset(\RR^{n}),\varphi\colon U\to M\,\}\\
&=\bigcup_{U\in\powerset(\RR^{n})}\{U\}\times\hom(U,M).
\end{split}
\end{equation}
But this is clearly a set, not a proper class: $\{U\}\times\hom(U,M)$ is the
product of sets (which is a set), and we take the union of an indexed
family of sets (which is, again, a set). The ZF axioms assure us this is
a set.

An atlas is a subcollection of $C(M)$, or equivalently an element of
$\powerset(C(M))$. By the axioms of ZF
set theory, this is a set, not a proper class. We may rest assured, an
atlas \emph{is} a set, \emph{not a proper class}.
\end{remark}

\begin{definition}\label{defn:atlas:equivalence}
We call two ($n$-dimensional) atlases $\mathcal{A}$ and $\mathcal{B}$
for $M$ \define{Equivalent} if their union $\mathcal{A}\cup\mathcal{B}$
is another atlas [or, equivalently, if every chart in $\mathcal{A}$ is
compatible with every chart in $\mathcal{B}$ and vice-versa].
\end{definition}

\begin{remark}
Although the relation of ``equivalent atlases'' is obviously symmetric
and reflexive, it is not yet established that it is a transitive relation.
We cannot say \emph{at present} that it forms an equivalence relation.
This motivates the first thing we'll do.
\end{remark}

\begin{remark}[Equivalence class of atlases is set, not proper class]
From Remark~\ref{rmk:atlas:is-a-set}, we know that the collection of all
possible atlases forms a set, a subset of $\powerset(C(M))$.
If we have an atlas $\mathcal{A}$ on $M$, the collection of atlases
equivalent to $\mathcal{A}$ is a subcollection of a set, which is a set,
\emph{not a proper class}.
\end{remark}

\begin{lemma}
Let $\mathcal{A}$ be an $n$-dimensional atlas for $M$, let $(U,\varphi)$ and $(V,\psi)$ 
be $n$-dimensional charts for $M$ (but they do not belong to $\mathcal{A}$).
If $(U,\varphi)$ and $(V,\psi)$ are compatible with every chart in $\mathcal{A}$,
then they are compatible with each other.
\end{lemma}

We only really need to prove properties
\ref{property:chart-compatible:transition-fun1} and
\ref{property:chart-compatible:transition-fun2} from the definition of
compatible charts, since open-ness should be obvious.

\begin{corollary}
Let $\mathcal{A}$, $\mathcal{B}$, $\mathcal{C}$ be $n$-dimensional
atlases on $M$. If $\mathcal{A}$ is equivalent to $\mathcal{B}$, and if
$\mathcal{B}$ is equivalent to $\mathcal{C}$, then $\mathcal{A}$ is
equivalent to $\mathcal{C}$.
\end{corollary}

\begin{proof}
Let $(U_{\alpha},\varphi_{\alpha})\in\mathcal{A}$ be an arbitrary chart,
let $(V_{\gamma},\psi_{\gamma})\in\mathcal{C}$ be an arbitrary chart. 
Since $\mathcal{A}$ is equivalent to $\mathcal{B}$, we know
$(U_{\alpha},\varphi_{\alpha})$ is compatible with every chart in $\mathcal{B}$.
Since $\mathcal{B}$ is equivalent to $\mathcal{C}$, we know
$(V_{\gamma},\psi_{\gamma})$ is compatible with every chart in $\mathcal{B}$.
By the previous lemma, we conclude that any arbitrary
$(U_{\alpha},\varphi_{\alpha})\in\mathcal{A}$ is compatible to any arbitrary
chart $(V_{\gamma},\psi_{\gamma})\in\mathcal{C}$. But by Definition~\ref{defn:atlas:equivalence} for
equivalent atlases, this implies $\mathcal{A}$ is equivalent to $\mathcal{C}$.
\end{proof}


