\chapter{Preface}

This introduces differential topology, which studies global aspects of
smooth manifolds. It's roughly at the level of a graduate course,
assuming the reader knows elementary differential geometry of
curves\footnote{Roughly at the level of my notes \url{http://pqnelson.github.io/assets/notebk/dg.pdf}},
real analysis, and so on. These notes are based on Dr Fuchs' course on
differential topology at UC Davis, Math 239, held Fall quarter of 2010.

The conventions are slightly old-fashioned, namely: charts are defined
as ordered pairs of maps from ``patches'' of open subsets of $\RR^{n}$
to a set along with the open subsets of $\RR^{n}$ themselves. This is
done so we can induce a topology using a collection of charts, demanding
the charts be \emph{continuous}; then their images form a topological
basis. We get, for free, a topological manifold from this demand on
maximal atlases.

Most ``modern'' books use the opposite convention, a chart on a set $M$
is a pair $(\varphi,U)$ consisting of a subset $U\subset M$ --- \emph{not}
$U\subset\RR^{n}$ (!!!) --- and a mapping $\varphi\colon U\to\RR^{n}$.
This is fine, but requires more work to induce a topology.