%%
%% lecture22.tex
%% 
%% Made by alex
%% Login   <alex@tomato>
%% 
%% Started on  Wed Oct  5 13:26:12 2011 alex
%% Last update Wed Oct  5 13:26:12 2011 alex
%%
The two fields of interest were really (1) Riemann surfaces, (2) the
theory of conformal mappings. Geometry merely refers to measuring
lengths and angles. We will briefly discuss the Riemann zeta
function. First recall the fundamental theorem of arithmetic
\begin{thm}[Arithmetic's Fundamental]
If $m$ is a positive integer, it can be written uniquely (up to
order of factors) as a product of prime numbers
\begin{equation}
m = p_{1}^{q_{1}}(\dots)p_{k}^{q_{k}}
\end{equation}
where $p_{1}$, \dots, $p_{k}$ are all distinct primes.
\end{thm}
Now, the Riemann zeta function
\begin{equation}
\zeta(s)=\sum^{\infty}_{n=1}\frac{1}{n^{s}}
\end{equation}
can be analytically continued to $\CC\setminus\{1\}$. We see
that, by the fundamental theorem of arithmetic, we may group
terms
\begin{equation}
\left(1+\frac{1}{2^{s}}+\frac{1}{2^{2s}}+\dots\right)\left(1+\frac{1}{3^{s}}+\dots\right)\left(1+\frac{1}{5^{s}}+\dots\right)(\dots)=\zeta(s)
\end{equation}
This is just from 
\begin{equation}
\sum\frac{1}{n^{s}}=\sum\frac{1}{(p_{1}^{k_{1}}\dots p_{m}^{k_{m}})^{s}}
\end{equation}
and thus\marginpar{Euler Product}
\begin{equation}
\zeta(s) = \prod_{\text{prime }p}\frac{1}{\left(1-\displaystyle\frac{1}{p^{s}}\right)}=\prod\frac{p^{s}}{p^{s}-1}=\prod\left(1+\frac{1}{p^{s}-1}\right)
\end{equation}
This is the \define{Euler Product}. We can represent it as an
integral, first note the notation $[x]$ for the integer part of
$x$, then
\begin{subequations}
\begin{align}
\zeta(s) &=\left[\int^{1}_{0}\frac{x^{s-1}}{\E^{x}-1}\D x\right]\frac{1}{\Gamma(s)}\\
&=s\int^{\infty}_{0}\frac{[x]}{x^{s+1}}\D x\\
&=\left(\frac{s}{s-1}\right)-s\int^{\infty}_{0}\frac{x-[x]}{x^{s+1}}\D
x
\end{align}
\end{subequations}
which converges for $\re(s)>0$ which is an extension of the
preceding integral. So we obtain a functional relationship
\begin{equation}\label{eq:reflectionPrincipleForZeta}
\zeta(s)=2^{s}\pi^{s-1}\sin\left(\frac{s\pi}{2}\right)\Gamma(1-s)\zeta(1-s)
\end{equation}
and with this functional relationship we may uniquely extend the
zeta function to a larger domain.
