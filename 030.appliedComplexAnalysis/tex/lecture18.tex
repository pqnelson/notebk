%%
%% lecture18.tex
%% 
%% Made by alex
%% Login   <alex@tomato>
%% 
%% Started on  Wed Oct  5 09:35:25 2011 alex
%% Last update Wed Oct  5 09:35:25 2011 alex
%%
The steepest descent method may be generalized from
\begin{equation}
f(x)=\int^{\infty}_{0}\E^{-xh(\zeta)}\D\zeta\asymptote\sqrt{\frac{2\pi}{x}}\frac{\E^{-xh''(\zeta_{0})}}{\sqrt{-h''(\zeta_{0})}}
\end{equation}
to 
\begin{equation}
f(x)=\int^{\infty}_{0}\E^{-xh(\zeta)}g(\zeta)\D\zeta
\end{equation}
where $g(\zeta)$ is bounded and $g(\zeta_{0})\not=0$.
We see that this doesn't change anything since we're working in a
small neighborhood near $\zeta_{0}$, so we have essentially
\begin{equation}
f(x)\asymptote\sqrt{\frac{2\pi}{x}}\frac{g(\zeta_{0})\E^{-xh''(\zeta_{0})}}{\sqrt{-h''(\zeta_{0})}}
\end{equation}
We make a small change in the integrand and the limits of
integration
\begin{equation}
f(x)=\int^{b}_{a}\E^{\I xh(\zeta)}g(\zeta)\D\zeta
\end{equation}
We do not insist anymore that $h(\zeta_{0})$ is a maximum, we
relax this to 
\begin{equation}
h'(\zeta_{0})=0\quad\mbox{ and }\quad h''(\zeta_{0})\not=0.
\end{equation}
Here $g(\zeta)$ is not really all that important.

\begin{wrapfigure}{l}{1in}
\vspace{-20pt}
\begin{center}
\includegraphics{img/lecture18.0}
\end{center}
\end{wrapfigure}
\noindent{}We recall from calculus that
\begin{equation}
\E^{\I xh(\zeta)}=\cos(xh(\zeta))+\I\sin(xh(\zeta))
\end{equation}
How does $\cos(xh(\zeta))$ look around $\zeta_{0}$? In some
neighborhood of $\zeta_{0}$, we have
\begin{equation}
\cos(xh(\zeta))\mbox{``$\approx$''}\cos(\mbox{constant})=\mbox{constant}
\end{equation}
locally in that neighborhood. A doodle to the left shows the
intuitive picture. The interval where it is constant is
$(\zeta_{0}-\varepsilon,\zeta_{0}+\varepsilon)$, where
$\varepsilon\sim1/\sqrt{x}$, $x$ is huge so we can think of it as
the frequency of the wave.

The contributions to the oscillations outside of this
neighborhood is $\approx0$ since destructive interference sets
everything to be negligible in comparison to the contribution
from the $(\zeta_{0}-\varepsilon,\zeta_{0}+\varepsilon)$
neighborhood.

We then have, on the one hand
\begin{subequations}
\begin{align}
f(x) &= \int^{b}_{a}\E^{\I xh(\zeta)}g(\zeta)\D\zeta\\
&\asymptote\frac{\E^{\I xh(\zeta_{0})}\sqrt{2\pi}}{\sqrt{x}\sqrt{\|h''(\zeta_{0})\|}}\E^{\sgn(h''(\zeta_{0}))\I\pi/4}
\end{align}
\end{subequations}
and on the other hand we have
\begin{subequations}
\begin{align}
f(x)
&\expequiv\int^{\zeta_{0}+\varepsilon}_{\zeta_{0}-\varepsilon}\E^{\I x[h(\zeta_{0})+h''(\zeta_{0})(\zeta-\zeta_{0})^{2}/2]}\D\zeta\\
&=\E^{\I xh(\zeta_{0})}\int^{\zeta_{0}+\varepsilon}_{\zeta_{0}-\varepsilon}\E^{\I xh''(\zeta_{0})(\zeta-\zeta_{0})^{2}/2}\D\zeta\\
\intertext{change variables to $u=\zeta-\zeta_{0}$}
&=\E^{\I xh(\zeta_{0})}\int^{\varepsilon}_{-\varepsilon}\E^{\I xh''(\zeta_{0})u^{2}/2}\D{}u\\
&=c_{0}\E^{\I xh(\zeta_{0})}\int^{+\varepsilon}_{-\varepsilon}
\E^{\I v^{2}}\D v
\end{align}
\end{subequations}
where $v=\sqrt{xh''(\zeta_{0})/2}u$. Note that we know
\begin{equation}
\int^{\infty}_{-\infty}\E^{\I v^{2}}\D v=(1+\I)\sqrt{\pi/2}
\end{equation}
so we use it!

Lets return now to the Laplace transform. We have our function
\begin{equation}
f\colon(0,\infty)\to\RR
\end{equation}
and suppose it grows slower than exponentially. More precisely,
there are numbers $A,B\in\RR$ such that
\begin{equation}
\|f(x)\|\leq A\E^{Bx}
\end{equation}
where $A,B\not=0$.

We have
\begin{equation}
\widetilde{f}(x)=\int^{\infty}_{0}\E^{-xt}f(t)\D t
\end{equation}
we require $x>B$, otherwise we don't know anything about
convergence. If we allow $x\in\CC$, then the requirement because
$\re(x)>B$. 

Now, what about asymptotics for this function? We need $f\in
C^{\infty}$ and $\|f^{(n)}(t)\|\leq A_{n}\exp(B_{n}t)$, we can
then write
\begin{subequations}
\begin{align}
\widetilde{f}(x) &= \int^{\infty}_{0}\E^{-xt}f(t)\D t\\
\intertext{then integrate by parts}
&=\left.\frac{-1}{x}\E^{-xt}f(t)\right|^{t=\infty}_{t=0}
+\frac{1}{x}\int^{\infty}_{0}\E^{-xt}f'(t)\D t\\
&=\frac{f(0)}{x}+\frac{1}{x}\int^{\infty}_{0}\E^{-xt}f'(t)\D t\\
\intertext{integration by parts again yields}
&=\frac{f(0)}{x}+\frac{f'(0)}{x}+\frac{1}{x^{2}}\int^{\infty}_{0}\E^{-xt}f''(t)\D t\\
\intertext{and inductively we obtain}
\widetilde{f}(x)&=\frac{f(0)}{x}+\dots+\frac{f^{(n)}(0)}{x^{n+1}}+\frac{1}{x^{n+1}}\int^{\infty}_{0}\E^{-xt}f^{(n+1)}(t)\D t
\end{align}
\end{subequations}
We can consider the last term
\begin{equation}
\frac{1}{x^{n+1}}\int^{\infty}_{0}\E^{-xt}f^{(n+1)}(t)\D t=:R(x)
\end{equation}
as the remainder term.


