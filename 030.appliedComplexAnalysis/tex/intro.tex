%%
%% intro.tex
%% 
%% Made by alex
%% Login   <alex@tomato>
%% 
%% Started on  Mon Oct  3 11:26:32 2011 alex
%% Last update Mon Oct  3 11:26:32 2011 alex
%%
These are a collection of notes on complex analysis. It is mostly
from the course math 185B taught spring quarter 2009, by Dmitry
Fuchs. Of course, there are several peculiarities with my notes
worth mentioning.

First, I do use diagrams. This tends to make most people cringe
(the puritans do not like pictures). But my diagrams include more
than pictures: it includes commutative diagrams! For example
\begin{equation}
\begin{diagram}[size=1pc]
a    & = & b\\
\dEq &      & \dEq \\
A    & = & B
\end{diagram}
\end{equation} 
would be used in place of four equations
\begin{subequations}
\begin{align}
a&=b\\
b&=B\\
a&=A\\
A&=B
\end{align}
\end{subequations}
which, in my modest opinion, takes up far too much room. My
diagrams litter the notes, hopefully they are useful.

Second, this is currently written in the more ``Russian''
style. That is, it glosses over a lot of material giving the
intuition underlying the theorems and few proofs. The idea is
that the reader would be mathematically intelligent enough to
supply the proofs instantly.
