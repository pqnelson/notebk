%%
%% lecture09.tex
%% 
%% Made by alex
%% Login   <alex@tomato>
%% 
%% Started on  Sun Oct  2 17:52:21 2011 alex
%% Last update Sun Oct  2 17:52:21 2011 alex
%%
\marginpar{A meromorphic map is
  $f\colon\mathcal{U}\to\CC\cup\{\infty\}$, or the ratio of two
  holomorphic functions.}We have an analytic function $f\colon\mathcal{U}\to\CC$ (or more
precisely meromorphic), $\mathcal{U}$ is simply connected, and
$\gamma$ is an oriented closed curve in $\mathcal{U}$ that
doesn't pass through poles or zeroes, we stated
\begin{equation}
\int_{\gamma}\frac{f'(z)}{f(z)}\D z=2\pi\I\left(
\sum_{{\text{zeroes }a}}I(\gamma,a)-\sum_{\mathclap{\text{poles }b}}I(\gamma,b)
\right)
\end{equation}
\begin{rmk}
Remember $\gamma$ does not pass through any zeroes or poles of $f$.
\end{rmk}
For a point $z_{0}$, we have ``locally''
\begin{equation}
f(z)=(z-z_{0})^{k}\varphi(z)
\end{equation}
for some nonzero $k\in\ZZ$ and an analytic $\varphi(z)$ with
neither pole nor zero at $z_{0}$. For $k>0$, we see that $z_{0}$ is
a zero of $f$, and for $-k<0$ that $f$ has a pole at $z_{0}$. So
we can see that
\begin{subequations}
\begin{align}
\frac{f'(z)}{f(z)} &=
\frac{k(z-z_{0})^{k-1}\varphi(z)+(z-z_{0})^{k}\varphi'(z)}{(z-z_{0})^{k}\varphi(z)}\\
&=\frac{k}{z-z_{0}}+\frac{\varphi'(z)}{\varphi(z)}
\end{align}
\end{subequations}
We see that $k$ is the residue of $f'(z)/f(z)$ at $z_{0}$, so by
our theorem we have
\begin{equation}
\int_{\gamma}\frac{f'(z)}{f(z)}\D z=2\pi\I\left(
\sum_{\substack{\text{distinct}\\\text{zeroes}\\a}}\mu(a)I(\gamma,a)
-\sum_{\substack{\text{distinct}\\\text{poles}\\b}}\mu(b)I(\gamma,b)
\right)
\end{equation}
For the winding path, we have the following rule for each intersection of
the curve:
\begin{center}
\includegraphics{img/lecture09.0}
\end{center}
\begin{comment}%%%%%%%%%%%%%%%%%%%%%%%%%%%%%%%%%%%%%%%%%%%%%%%%%%%%%%%
\begin{ex}
So for example consider
\begin{center}
\includegraphics{img/lecture09.1}
\end{center}
\end{ex}
\end{comment}%%%%%%%%%%%%%%%%%%%%%%%%%%%%%%%%%%%%%%%%%%%%%%%%%%%%%%%%%%%%%%%%%

\noindent{}We have this integral also be, by the fundamental theorem of
calculus,
\begin{equation}
\int_{\gamma}\frac{f'(z)}{f(z)}\D z=\int_{\gamma}\D\log(z)
\end{equation}
but beware! The logarithm is \emph{multivalued}. We see that
\begin{equation}
\log(\omega)=\log\|\omega\|+\I\arg(\omega).
\end{equation}
So we use a notation
\begin{equation}
\begin{split}
\int_{\gamma}\D\log(f(z))&=\Delta_{\gamma}\underbracket[0.5pt]{\log\|f(z)\|}_{\mathclap{\text{no change if $\gamma$ is closed, so it's 0}}}+\I\Delta_{\gamma}\arg(f(z))\\
&= \I\Delta_{\gamma}\arg(f(z))
\end{split}
\end{equation}
We can then write
\begin{equation}
\begin{split}
&\I\Delta_{\gamma}\arg(f(z))=2\pi\I(\sum I(\gamma,a)-\sum
  I(\gamma,b))\\
\implies&\Delta_{\gamma}\arg(f(z))=2\pi(\sum I(\gamma,a)-\sum
  I(\gamma,b))
\end{split}
\end{equation}
We can write
\begin{equation}
\gamma\colon I\to\CC
\end{equation}
which is $\gamma$ given parametrically, then write $f\circ\gamma$
or $f(\gamma(t))$.
If we integrate over $I$ we get back $2\pi$ times a number. We
can write, then, 
\begin{equation}
\begin{diagram}[small]
\Delta_{\gamma}\arg(f(z)) & \rEq & 2\pi(\sum I(\gamma,a) - \sum I(\gamma,b))\\
\dEq & & \\
2\pi I(f\circ\gamma,0) & &
\end{diagram}
\end{equation}
If $\gamma$ is a simple curve, we can write
\begin{equation}
\begin{split}
I(f\circ\gamma,0) &=
\begin{pmatrix}
\text{zeroes}\\
\text{within}\\
\text{the bdry}
\end{pmatrix}
-
\begin{pmatrix}
\text{poles}\\
\text{within}\\
\text{the bdry}
\end{pmatrix}\\
&= Z-P
\end{split}
\end{equation}
Recall a simple curve is any nonintersecting curve.

\subsection{Rouch\'e's Theorem}

Suppose that in a simply connected domain $\mathcal{U}$ we have
meromorphic functions
\begin{equation}
f,g\colon\mathcal{U}\to\CC
\end{equation}
let
\begin{equation}
\|f(z)-g(z)\|<\|f(z)\|\quad\forall z\in\gamma
\end{equation}
where $\gamma$ is an oriented closed curve in $\mathcal{U}$. Then
\begin{equation}
\Delta_{\gamma}\arg(f)=\Delta_{\gamma}\arg(g)
\end{equation}
is our claim.

\begin{wrapfigure}[12]{r}{2.8in}
\vspace{-30pt}
\begin{center}
\includegraphics{img/lecture09.4}
\end{center}
\vspace{-20pt}
\end{wrapfigure}
What this means is $f\circ\gamma$ has a curve in $\CC$ and for
some $f(z)\in f\circ\gamma$, we have $g(z)$ in a disc of radius
$\|f(z)\|$. We wish to argue that the distance from
$f(\gamma(t))$ to $g(\gamma(t))$ is less than the distance from
$f(\gamma(t))$ to the origin. So we have the inequality
$\|f(\gamma(t))-g(\gamma(t))\|<\|f(\gamma(t))\|$. This situation
is doodled to the right, where the dark red line indicates the
line from $f(\gamma(t))$ to the origin, and the dark blue line
indicates the distance from $f(\gamma(t))$ to $g(\gamma(t))$.

Now every domain in math has a cherished proof of the fundamental
theorem of algebra, we now have the tools to prove it. Let
\begin{equation}
f(z) = z^{n}+a_{n-1}z^{n-1}+\dots+a_{1}z+a_{0}
\end{equation}
and
\begin{equation}
g(z)=z^{n}.
\end{equation}
Now we take
\begin{equation}
\|f(z)-g(z)\| =
\|a_{n-1}z^{n-1}+\dots+a_{0}\|\leq\|a_{n-1}\|\cdot\|z^{n-1}\|+\dots+\|a_{0}\|
\end{equation}
then for some radius $R$ we consider the curve 
\begin{equation}
\gamma=\{z\lst \|z\|=R\}.
\end{equation}
We find on this curve
\begin{equation}
\|f(z)-g(z)\|<\|g(z)\|
\end{equation}
Then the ratio
\begin{equation}
\frac{\|f(z)-g(z)\|}{\|g(z)\|}\leq \|a_{n-1}\|\frac{1}{R}+
\|a_{n-2}\|\frac{1}{R^{2}}+\dots+\|a_{0}\|\frac{1}{R^{n}}
\end{equation}
is very interesting. We let 
\begin{equation}
h(z)=\|a_{n-1}\|z+\|a_{n-2}\|z^{2}+\dots+\|a_{0}\|z^{n}
\end{equation}
then we find a $\delta>0$ such that
\begin{equation}
\|z\|<\delta\implies\|h(z)\|<1
\end{equation}
We take $R=1/\delta$.

