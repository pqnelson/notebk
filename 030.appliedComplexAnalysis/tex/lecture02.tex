%%
%% lecture02.tex
%% 
%% Made by alex
%% Login   <alex@tomato>
%% 
%% Started on  Fri Sep 30 12:42:58 2011 alex
%% Last update Fri Sep 30 12:42:58 2011 alex
%%
We stated previously the first nontrivial theorem of the course:
\begin{riemMap}
If $\mathcal{U}\propersubset\CC$ is a simply connected domain,
and $\mathcal{U}\not=\emptyset,\CC$, then there exists a
conformal map $f\colon\mathcal{U}\to\mathcal{D}$ where
$\mathcal{D}=\{z\in\CC\lst\|z\|<1\}$ is the unit
disc. Additionally, if we fix some point $u\in\mathcal{U}$,
demand it be mapped to the origin $0\in\mathcal{D}$, and some
direction specified in $\mathcal{U}$ be mapped to some direction
in $\mathcal{D}$, then $f$ is unique.
\end{riemMap}

Existence is a more delicate matter than uniqueness. The domains
need not be finite (e.g., the upper half plane of $\CC$). There
are some particular cases we need to consider, e.g.\ when
$\mathcal{U}=\mathcal{D}$. 

\begin{wrapfigure}{r}{2.35in}
\vspace{-30pt}
\begin{center}
\includegraphics{img/lecture02.0}
\end{center}
\vspace{-20pt}
\end{wrapfigure}
Uniqueness can be constructed in two different forms. Consider
two maps $f,g\colon\mathcal{U}\to\mathcal{D}$. We see that
$h=g\circ f^{-1}\colon\mathcal{D}\to\mathcal{D}$. We also can see
that together this map takes $0\mapsto0$. We have the Schwarz
inequality $\|h(z)\|\leq\|z\|$. We can deduce this from the
Cauchy integral formula. We can apply the same process to
$h^{-1}=f\circ g^{-1}\colon\mathcal{D}\to\mathcal{D}$. We have
the inequality $\|h^{-1}(\omega)\|\leq\|\omega\|$ where
$\omega=h(z)$ and $z=h^{-1}(\omega)$. We see that this implies
$\|z\|\leq\|h(z)\|\leq\|z\|$ which implies $\|h(z)\|=\|z\|$. We
see then that this implies $h(z)=\lambda z$ where $\|\lambda\|=1$
is a root of unity. In other words, $h$ is just a rotation. By
fixing the orientation we have $g\circ f^{-1}=h=\id{}$ imply
$f=g$. This is simply just from the Schwarz inequality.

There is a notion of a \define{Fractional Linear Map} where,
given some $a,b,c,d\in\CC$, we have
\begin{equation}
|A|=\begin{vmatrix}
a & b\\ c&d
\end{vmatrix}
=|ad-bc|\not=0,
\end{equation}
we are interested in the domain and the range of the map
\begin{equation}
f_{A}(z)=\frac{az+b}{cz+d}.
\end{equation}
Observe that it is singular at $z=-d/c$, and we also see that
$f_{A}(z)\not=a/c$. 

We see that the composition of two linear fractional maps is
itself a fractional linear map given by the product of the
matrices.

We can consider the matrices as being unique (up to
multiplication by a complex number), so lets think of matrices
with determinant of 1. We should make them up to a sign, this is
\begin{equation}
\SL{2,\CC}/\{\pm1\}\iso\PSU{2}.
\end{equation}
We want to say now that every fractional linear transformation
maps circles and straight lines into circles and straight lines.

\begin{thm}
Fractional linear transformations preserves circles and lines.
\end{thm}

We can explicitly write
\begin{equation}
f(z)=\frac{az+b}{cz+d}=\frac{1}{c}\left(a+\frac{bc-ad}{cz+d}\right)
\end{equation}
by considering the following:
\begin{align*}
f_{1}(z) &= cz\\
f_{2}(z) &= z+d\\
f_{3}(z) &= 1/z\\
f_{4}(z) &= (bc-ad)z\\
f_{5}(z) &= a+z\\
f_{6}(z) &= \frac{1}{c}z\quad\mbox{assuming }c\not=0.
\end{align*}
Each of these maps---with one exception---are just translations
and multiplication by complex numbers. These preserves circles
and straight lines. The inversion operation, $f_{3}(z)$, after
some thinking can be shown to preserve ``generalized
circles''\dots that is to say, straight lines become circles and
circles become straight lines.

Consider the unit disc. What does a ``fractional linear map'' map
it to? Well, there are several possibilities, doodled below. The
boundary can be mapped to a line or a unit circle, and the
contents may be mapped to several places.
\begin{center}
\includegraphics{img/lecture02.1}
\end{center}
It depends on where the point mapped to infinity lives. It can
live on the boundary. If the singularity of the map is on the
boundary, the circle is mapped to a line, and two distinct points
on the interior of the disc is mapped to the same side of the
line, so the disc is mapped to the plane. If the singularity is
on the interior of the disc, the disc is mapped to the exterior
of the boundary's mapping in the range. The last case of interest
is if the point mapped to infinity is outside the disc, then the
disc is mapped to a disc.

Take the map
\begin{equation}
f(z)=\frac{z-a}{1-z\bar{a}}\lambda
\end{equation}
where $\|a\|<1$ and $\|\lambda\|=1$. This maps $D\to D$, and
covers all we need for the map to exist by the Riemann mapping
theorem. We see first that $f(a)=0$. We see that
\begin{subequations}
\begin{align}
f'(z) &= \lambda\left(\frac{1-\bar{a}{z}+\bar{a}(z-a)}{(1-\bar{a}z)^{2}}
\right)\\
&=\lambda\left(\frac{1-\bar{a}a}{(1-\bar{a}z)^{2}}\right)
\end{align}
\end{subequations}
Observe that
\begin{equation}
f'(0)=\lambda(1-a\bar{a})
\end{equation}
So for some orientation $+\bar{\lambda}$, we have it be mapped to
the positive real direction.

If $\|z\|=1$, then 
\begin{subequations}
\begin{align}
\|f(z)\|&=1\\
&=\|f(z)\bar{z}\|\\
&=\|\lambda\|\cdot\left\|\frac{z\bar{z}-\bar{z}a}{1-\bar{a}z}\right\|\\
&=\|\lambda\|\cdot\left\|\frac{1-\bar{z}a}{1-\bar{a}z}\right\|\\
&=\|\lambda\|=1
\end{align}
\end{subequations}
How interesting.
