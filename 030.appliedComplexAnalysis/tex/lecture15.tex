%%
%% lecture15.tex
%% 
%% Made by alex
%% Login   <alex@tomato>
%% 
%% Started on  Tue Oct  4 10:39:13 2011 alex
%% Last update Tue Oct  4 10:39:13 2011 alex
%%

A function (for us a meromorphic function of a complex variable)
may be written asymptotically as
\begin{equation}
f(z)\asymptote a_{0}+\frac{a_{1}}{z}+\frac{a_{2}}{z^{2}}+\frac{a_{3}}{z^{3}}+\dots
\end{equation}
Our example from last time was
\begin{equation}
\Gamma(z+1)\asymptote\sqrt{2\pi z}\left(\frac{z}{\E}\right)^{z}\left(1+\frac{1}{12z}+\dots\right)
\end{equation}
First, the series diverges in many important cases; and second,
the behavior at the poles is misleading.

\begin{wrapfigure}{l}{1in}
\vspace{-24pt}
\begin{center}
\includegraphics{img/lecture15.0}
\end{center}
\vspace{-20pt}
\end{wrapfigure}
We make our first assumption, $\Gamma(z)\asymptote\sqrt{2\pi
  z}(\dots)$ holds \emph{NOT} for all values of $z$ but for a
sector. That is, we confine the argument of $z$ to satisfy
\begin{equation}
\alpha\leq\arg(z)\leq\beta
\end{equation}
We then commit our second assumption that we deal with a finite
sum
\begin{equation}
S_{N}=a_{0}+\frac{a_{1}}{z}+\dots+\frac{a_{N}}{z^{N}}
\end{equation}
we demand that
\begin{equation}
\lim_{\|z\|\to\infty}\|z\|^{N}\cdot\|f(z)-S_{N}\|=0.
\end{equation}
So for each $\varepsilon>0$ there exists a $R_{N}>0$ such that if
$\alpha\leq\arg(z)\leq\beta$ and $\|z\|\geq R_{N}$ implies
\begin{equation}
\|f(z)-S_{N}\|<\frac{\varepsilon}{\|z\|^{N}}
\end{equation}

\begin{wrapfigure}{r}{1in}
\vspace{-24pt}
\begin{center}
\includegraphics{img/lecture15.1}
\end{center}
\vspace{-20pt}
\end{wrapfigure}
Within a portion of a sector the asymptotic formula is very
good. Outside of it, it becomes a bad approximation. This is
doodled on the right where the gray region is when $S_{N}$ is
very good.

Consider an example
\begin{equation}\label{eq:lecture15:theintegral}
f(x)=\int^{\infty}_{x}t^{-1}\E^{x-t}\D t,
\end{equation}
now consider the formula
\begin{equation}\label{eq:lecture15:thesum}
f(x)\asymptote\frac{1}{x}-\frac{1}{x^{2}}+\frac{2!}{x^{3}}
-\frac{3!}{x^{4}}+\frac{4!}{x^{5}}+\cdots
\end{equation}
This is the canonical example of the series with a radius of
convergence being zero.

We will now show how to relate the two, or more precisely how to
go from the integral, Eq \eqref{eq:lecture15:theintegral}, to the
sum, Eq \eqref{eq:lecture15:thesum}. We write
\begin{equation}
f(x) = \int^{\infty}_{x}t^{-1}\E^{x-t}\D t
\end{equation}
and integrate by parts letting
\begin{equation}
u=t^{-1}\quad\mbox{and}\quad \D v=\E^{x-t}\D t
\end{equation}
thus
\begin{equation}
\begin{split}
f(x)&=\left.-t^{-1}\E^{x-t}\right|^{\infty}_{x}-\int^{\infty}_{x}t^{-2}\E^{x-t}\D t\\
&=x^{-1}\E^{x-x}-\int^{\infty}_{x}t^{-2}\E^{x-t}\D t
\end{split}
\end{equation}
So we can simplify
\begin{equation}
f(x)=\frac{1}{x}-\int^{\infty}_{x} t^{-2}\E^{x-t}\D t
\end{equation}
and by iterating we obtain
\begin{equation}
f(x)=\frac{1}{x}-\frac{1}{x^{2}}+\frac{2!}{x^{3}}-\frac{3!}{x^{4}}+\dots+\frac{(-1)^{n-1}(n-1)!}{x^{n}}+\underbracket[0.5pt]{(-1)^{n}n!\int^{\infty}_{x} t^{-(n+1)}\E^{x-t}\D t}_{\text{remainder term}}
\end{equation}
We will not show, but writing $(t/x)^{-1}$ instead of $t$, we get
the remainder term being bounded by $n!/x$. So if $x$ is really
big, written
\begin{equation}
x\gg n!
\end{equation}
we have a fairly good approximation within a sector. So fro more
terms in this asymptote, the smaller the domain for convergence.

We have two things left to discuss today. First the method of
steepest descent. This may be considered a generalization of
Student's formula.
\begin{rmk}
We wrote $f(x)\asymptote a_{0}+\frac{a_{1}}{x}+\dots$, we should
write more generally that $f(x)\asymptote g(x)(a_{0}+\dots)$, so
we consider
\begin{equation}
\lim_{x\to\infty}\frac{f(x)}{g(x)}=a_{0},
\end{equation}
and this first term usually contains a lot of information.
\end{rmk}
The following has nothing to do with complex analysis: it is
absolutely real.

\begin{wrapfigure}{r}{0.9in}
\vspace{-24pt}
\begin{center}
\includegraphics{img/lecture15.2}
\end{center}
\vspace{-20pt}
\end{wrapfigure}
We have some function $h\colon(0,\infty)\to\RR$, we will consider
\begin{equation}
f(z)=\int^{\infty}_{0}\E^{zh(t)}\D t
\end{equation}
We suppose it converges. Our conditions is that $h(t)$ has a
maximum at $t_{0}$, $h'(t_{0})=0$ and $h''(t_{0})<0$.
We plot $h(t)$ to the right, with maximum at $t_{0}$. So then we
find
\begin{equation}
f(x)\asymptote\left(\frac{\E^{x-h(t_{0})}\sqrt{2\pi}}{\sqrt{x}\sqrt{-h''(t_{0})}}\right)
\end{equation}
This generalizes Stirling's formula. How? Well, observe
\begin{subequations}
\begin{align}
\Gamma(x+1) &= \int^{\infty}_{0}t^{x}\E^{-t}\D t\\
\intertext{change variables $xu=t$ so $x\D u=\D t$, thus}
&=\int^{\infty}_{0}(xu)^{x}\E^{-xu}x\D u\\
&=x^{x+1}\int^{\infty}_{0}u^{x}\E^{-xu}\D u\\
&= x^{x+1}\int^{\infty}_{0}\exp\left(x\big(\ln(u)-u\big)\right)\D
u
\end{align}
\end{subequations}
where we let $h(u):=\ln(u)-u$. We end up with
\begin{equation}
\begin{split}
\frac{\Gamma(x+1)}{x^{x+1}} &=\int^{\infty}_{0}\E^{x(\ln(u)-u)}\D u\\
&\asymptote\frac{\E^{-(x+1)}\sqrt{2\pi}}{\sqrt{x}\sqrt{1}}
\end{split}
\end{equation}
thus we get Stirling's formula back entirely.
