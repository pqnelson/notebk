%%
%% lecture01.tex
%% 
%% Made by alex
%% Login   <alex@tomato>
%% 
%% Started on  Fri Sep 30 11:33:18 2011 alex
%% Last update Fri Sep 30 11:33:18 2011 alex
%%
A \define{Conformal Map}, for us, is  a smooth map from a plane into
itself that preserves angles. Let $A\subset\RR^{2}$,
$B\subset\RR^{2}$, and
\begin{equation}
f\colon A\to B
\end{equation}
be smooth and bijective. Consider the following:
\begin{center}
\includegraphics{img/lecture01.0}
\end{center}
Additionally we have
\begin{equation}
\frac{\|f(\vec{v})\|}{\|\vec{v}\|}=\mbox{constant}.
\end{equation}
It is understood that it is an orientation preserving map, the
correct term is an ``orientation preserving conformal map''. They
precisely correspond to analytic functions of complex variables
with nonzero first derivative.

In complex analysis this is equivalent to $A,B\subset\CC$, we
have $z\in A$ and $\omega\in B$, $f$ be holomorphic. If
\begin{equation}
f'(z)\not=0
\end{equation}
for all $z\in A$, then it is conformal.

\medskip
\begin{wrapfigure}{r}{1.5in}
\vspace{-40pt}
\begin{center}
\includegraphics{img/lecture01.1}
\end{center}
\vspace{-20pt}
\end{wrapfigure}
\noindent\textbf{NON-Example }(Inversion)\textbf{.\enspace}
We have some disc of radius $R$, and it has its center be
$\mathcal{O}$. Inversion would be
$z\mapsto1/\bar{z}$ in the complex plane (if we make the origin
$\mathcal{O}$ and dilate by $1/R$).
It does not preserve orientation. In the complex plane,
$f(z)=1/z$ is a bit more interesting. This preserves angles but
it \emph{reverses} orientation. This situation is doodled on the
right. 
\medbreak

Inversion changes lines into circles if we have lines that do not
pass through the origin. How can we see this? Consider
\begin{equation}
\gamma(t) = z_{0}+z_{1}t
\end{equation}
where we have nonzero constants
$z_{0},z_{1}\in\CC\setminus\{0\}$. Then $\gamma$ is a line that
does not pass through the origin. What happens when we invert it?
It becomes
\begin{equation}
\widetilde{\gamma}(t)=\frac{1}{z_{0}+z_{1}t}=\frac{\bar{z}_{0}+\bar{z_{1}}t}{\|z_{0}+z_{1}t\|^{2}}
\end{equation}
We see that the denominator is
\begin{subequations}
\begin{align}
\|z_{0}+z_{1}t\|^{2} &= z_{0}\bar{z}_{0}+(\bar{z}_{1}z_{0}+\overline{\bar{z}_{1}z_{0}})t+\bar{z}_{1}z_{1}t^{2}\\
&= r_{0} + 2\re(\bar{z}_{1}z_{0})t+r_{1}^{2}t^{2}
\end{align}
\end{subequations}
where $r_{0},r_{1}\in\RR$ are positive real numbers. Observe,
then, that
\begin{equation}
\lim_{t\to+\infty}\widetilde{\gamma}(t)=0
\end{equation}
since it's of the form
\begin{equation}
\widetilde{\gamma}(t)\propto\frac{\mbox{const}+\bar{z}_{1}t}{\mbox{const}+\|z_{1}\|^{2}t^{2}}
\end{equation}
and the denominator vanishes quicker than the numerator
increases. Likewise, for precisely the same reason, we have
\begin{equation}
\lim_{t\to-\infty}\widetilde{\gamma}(t)=0
\end{equation}
and
\begin{equation}
\widetilde{\gamma}(0)=z_{0}^{-1}
\end{equation}
Is this convincing? Well, yes and no.

Consider the stereographic projection to the sphere. We will
consider longitudinal lines on the sphere. What happens when we
consider the inverse to the stereographic projection? We leave
this to the reader\dots

\subsection{Riemann Mapping Theorem}
Suppose that you have two domains in the plane that are not empty
and not the whole plane. Is it possible to find some conformal
map from one to the other? If so, how many are there?

\begin{thm}[Incomplete Version Riemann Mapping]
If we have two nonempty, simply connected subsets of the plane
(which are not the whole plane), if we map $z_{0}\mapsto
f(z_{0})$ and preserve orientation, then there exists a unique
conformal map from one to the other.
\end{thm}

For existence, we need some extra conditions.
