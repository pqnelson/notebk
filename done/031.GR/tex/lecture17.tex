%%
%% lecture17.tex
%% 
%% Made by alex
%% Login   <alex@tomato>
%% 
%% Started on  Wed Mar  7 10:36:17 2012 alex
%% Last update Wed Mar  7 10:36:17 2012 alex
%%

Consider the stress energy tensor for a point particle. We have
\begin{equation}
x^{\mu}=z^{\mu}(u)
\end{equation}
where $u$ is some parameter, $m$ be the particle's mass (at
rest). We should recall the geodesic action is
\begin{equation}
\begin{split}
I_{geod}
&=m\int\D s\\
&=m\int\sqrt{g_{\mu\nu}\frac{\D x^{\mu}}{\D u}
\frac{\D x^{\nu}}{\D u}}\,\D u.
\end{split}
\end{equation}
Thus the matter action is
\begin{equation}
I_{m}=m\iint\delta^{(4)}\Bigl(x-z(u)\Bigr)\sqrt{g_{\mu\nu}\frac{\D z^{\mu}}{\D
u}\frac{\D z^{\nu}}{\D u}}\,\D^{4}x\,\D u.
\end{equation}
We're fixing $z$ but varying $g_{\mu\nu}$, this is reasonably easy
to do. Also note that
\begin{equation}
\int\delta^{(4)}(x)\,\D^{4}x=1.
\end{equation}
We'll get the stress-energy tensor for may particles, the
consider the continuum limit.

First lets consider a single particle. We see that by varying the
matter action with respect to the metric we obtain
\begin{equation}
\delta I
=\frac{1}{2}\iint\left(g_{\sigma\rho}\frac{\D x^{\sigma}}{\D u}
\frac{\D x^{\rho}}{\D u}\right)^{1/2}
\delta g_{\mu\nu}\frac{\D x^{\mu}}{\D u}\frac{\D x^{\nu}}{\D u}\delta^{4}\Bigl(x-z(u)\Bigr)\,\D^{4}x\,
\D u
\end{equation}
Now we take $u=s$ as the parameter, and we find
\begin{equation}
\delta I=\frac{1}{2}\iint\delta g_{\mu\nu}\frac{\D x^{\mu}}{\D s}\frac{\D x^{\nu}}{\D s}\delta^{4}\Bigl(x-z(s)\Bigr)\,\D^{4}x\,
\D s
\end{equation}
Comparing this variation to
\begin{equation}
\begin{split}
\delta I
&=\frac{-1}{2}\int\sqrt{-g}T_{\mu\nu}\delta g^{\mu\nu}\D^{4}x\\
&=\frac{1}{2}\int\sqrt{-g}T^{\mu\nu}\delta g_{\mu\nu}\D^{4}x
\end{split}
\end{equation}
we find\marginpar{Stress-Energy tensor for a point-particle}
\begin{equation}
T^{\mu\nu}=\int m\frac{\D x^{\mu}}{\D s}\frac{\D x^{\nu}}{\D s}
\delta^{(4)}\Bigl(x-z(s)\Bigr)\frac{1}{\sqrt{-g}}\,\D s.
\end{equation}
This describes the stress-energy tensor for a point-particle.

Lets consider what happens in flat spacetime, just to get some
intuition underpinning the components of a stress-energy
tensor. We see
\begin{subequations}
\begin{align}
T^{0\mu}
&=\int m\frac{\D x^{\mu}}{\D s}\frac{\D t}{\D s}\delta^{4}\bigl(x-z(s)\bigr)\,\D s\\
&=\int m\frac{\D x^{\mu}}{\D s}\delta\bigl(t-z^{0}(s)\bigr)\delta^{3}\bigl(x^{i}-z^{i}(s)\bigr)\,\D t\\
&=\left.m\frac{\D x^{\mu}}{\D s}\delta^{3}\bigl(x^{i}-z^{i}(s)\bigr)\right|_{t=z^{0}(s)}.
\end{align}
\end{subequations}
The time-time component reads
\begin{subequations}
\begin{align}
T^{00}
&=m\frac{\D t}{\D s}\delta^{3}\bigl(x^{i}-z^{i}(s)\bigr)\\
&=\frac{m}{\sqrt{1-v^{2}}}\delta^{3}\bigl(x^{i}-z^{i}(s)\bigr)\\
&=E\delta^{3}\bigl(x^{i}-z^{i}(s)\bigr)\\
&=\begin{pmatrix}\mbox{Energy}\\\mbox{Density}
\end{pmatrix}\nonumber
\end{align}
\end{subequations}
and similarly
\begin{equation}
\begin{split}
T^{0i}
&=p^{i}\delta^{3}\bigl(x^{i}-z^{i}(s)\bigr)\\
&=\begin{pmatrix}\mbox{Momentum}\\\mbox{Density}
\end{pmatrix}
\end{split}
\end{equation}
This is just for a single particle, however.

For\marginpar{Stress-Energy Tensor for Dust} many non-interacting particles without charge (which
relativists confusingly call ``dust''), we have
\begin{equation}
T^{\mu\nu}_{(dust)}=\sum T^{\mu\nu}_{(particles)}
\end{equation}
or taking the continuum limit 
\begin{equation}
T^{\mu\nu}_{(dust)}=\rho u^{\mu}u^{\nu}
\end{equation}
where $u^{\mu}=\D x^{\mu}/\D s$. We take the continuum limit when
we have a continuous collection of noninteracting particles. For
most practical purposes in cosmology, this is good enough. 

A perfect fluid\marginpar{Stress-Energy Tensor for Perfect Fluid}
with pressure $p$ experiences a stress-energy tensor
\begin{equation}
T^{\mu\nu}=(p+\rho)u^{\mu}u^{\nu}-pg^{\mu\nu}.
\end{equation}
Observe that taking $p\to0$ recovers the dust stress-energy
tensor. The cosmological constant could be thought of as a
perfect fluid term.

Now, lets see a miracle! Consider dust
\begin{equation}
T^{\mu\nu}_{\text{dust}}=\rho u^{\mu}u^{\nu}
\end{equation}
and plug this into Einstein's field equation
\begin{equation}
G^{\mu\nu}=T^{\mu\nu}_{\text{dust}}.
\end{equation}
We see that
\begin{equation}
\nabla_{\mu}G^{\mu\nu}=0\implies\nabla_{\mu} T^{\mu\nu}_{\text{dust}}=0.
\end{equation}
We will show that dust moves along geodesics. First we note
\begin{equation}
u^{\mu}u_{\mu}=g_{\mu\nu}\frac{\D x^{\mu}}{\D s}\frac{\D x^{\nu}}{\D s}=1.
\end{equation}
Now we consider
\begin{equation}
\nabla_{\mu}T^{\mu\nu}=\nabla_{\mu}(\rho u^{\mu})u^{\nu}+\rho u^{\mu}\nabla_{\mu}
u^{\nu}.
\end{equation}
We recall the conditions for a geodesic states
\begin{equation}
u_{\mu}\nabla_{\nu}u^{\mu}=0.
\end{equation}
Thus
\begin{equation}
u_{\nu}\nabla_{\mu}T^{\mu\nu}=\underbracket[0.5pt]{u_{\nu}u^{\nu}}_{=1}\nabla_{\mu}(\rho u^{\mu})
+\rho u^{\mu}\underbracket[0.5pt]{u_{\nu}\nabla_{\mu}u^{\nu}}_{=0}
\end{equation}
which becomes
\begin{equation}
u_{\nu}\nabla_{\mu}T^{\mu\nu}=\nabla_{\mu}(\rho u^{\mu}).
\end{equation}
This is a conservation equation. More explicitly, we can rewrite
it as
\begin{equation}
\nabla_{\mu}(\rho u^{\mu})=
\frac{1}{\sqrt{-g}}\partial_{\mu}(\sqrt{-g}\rho u^{\mu})=0
\end{equation}
This is a very nice conservation law for mass which looks exactly
like the conservation of electric charge. We can now go back and
find
\begin{equation}
\rho u^{\mu}\nabla_{\mu}u^{\nu}=0.
\end{equation}
If $\rho\not=0$ (i.e.~in regions where the particles are
present), then we necessarily have
\begin{equation}
u^{\mu}\nabla_{\mu}u^{\nu}=0.
\end{equation}
Field equations always give the motion of sources.

\begin{exercises}
\begin{xca}[Conservation and equations of motion]
The tensorial description of the electromagnetic field fits the electric field $\mathbf{E}$ and the magnetic field $\mathbf{B}$ together
in an antisymmetric type 2 tensor $F_{\mu\nu}$, with field equations
\begin{equation}
\nabla_{\nu}F^{\mu\nu}=J^{\mu},\quad
\nabla_{\mu}F_{\nu\rho}+
\nabla_{\nu}F_{\rho\mu}+
\nabla_{\rho}F_{\mu\nu}=0
\end{equation}
Consider a cloud of charged particles, with a mass density $\mu$ and a charge density $\rho$. The electromagnetic current for such a system at a point $x$ is
\begin{equation}
J^{\mu} (x) = \rho(x)u^{\mu}(x)
\end{equation}
where $u^{\mu}(x)$ is the four-velocity of the particle at point $x$. The stress-energy
tensor consists of two pieces, a ``dust'' part
\begin{equation}
T^{\mu\nu}_{dust}=\mu u^{\mu}u^{\nu}
\end{equation}
as discussed and an electromagnetic part
\begin{equation}
T^{\mu\nu}_{EM}={F^{\mu}}_{\rho}F^{\nu\rho}-\frac{1}{4}g^{\mu\nu}F_{\rho\sigma}F^{\rho\sigma}
\end{equation}

\noindent\textbf{a.\quad}Show that the covariant conservation law
\begin{equation}
\nabla_{\nu}(T^{\mu\nu}_{dust} + T^{\mu\nu}_{EM}) = 0
\end{equation}
that follows from the Einstein field equations implies that
\begin{equation}
\mu u^{\nu}\nabla_{\nu}u^{\mu}=\rho{F^{\mu}}_{\rho}u^{\rho}.
\end{equation}
(Note: you will have to use both sets of Maxwell's equations and
the anti-symmetry of $F$.) 

\medbreak
\noindent\textbf{b.\quad}Suppose that the particles all have mass
$m$ and charge $e$, so $\rho/\mu = e/m$. Using the expression for
$F_{\mu\nu}$ in terms of the electric and magnetic fields, show
that in a flat spacetime, the equation derived in part (a) is
just the Lorentz force law, $F = e(E + v \times B)$.
\end{xca}
\begin{xca}[Massive Neutral Scalar Field]
The Lagrangian density for a massive neutral scalar field is
\begin{equation}
\mathcal{L}=\frac{1}{2}g^{\mu\nu}\partial_{\mu}\phi\partial_{\nu}\phi-\frac{m^{2}}{2}\phi^{2}
\end{equation}
where $m$ is the mass, and $\phi$ is the scalar field.
\begin{enumerate}
\item Find the equations of motion from the Euler-Lagrange
  equation.
\item Find the stress-energy tensor for the scalar field.
\item Find the equations of motion from Einstein's field equation.
\end{enumerate}
\end{xca}
\begin{xca}[{\cite{lightman:1975}}]
Show the stress-energy tensor for source-less electromagnetism $T^{\mu\nu}_{\text{EM}}$ has zero trace.
\end{xca}
\end{exercises}
