%%
%% lecture10.tex
%% 
%% Made by alex
%% Login   <alex@tomato>
%% 
%% Started on  Wed Feb 29 10:50:26 2012 alex
%% Last update Wed Feb 29 10:50:26 2012 alex
%%

We want differentiate quantities. Why not just differentiate ``in
the obvious way'' (i.e., take their derivatives!)?
The problem with just taking derivatives is we get under a change
of coordinates
\begin{equation}
\partial_{\mu}v^{\nu}\not=\partial_{\mu'}v^{\nu'}.
\end{equation}
What to do? Well, we should recall that a\marginpar{$p$-form} \define{$p$-Form}
is a totally antisymmetric $(0,p)$-tensor
\begin{equation}
\omega=\omega_{\mu_{1}\dots\mu_{p}}\D x^{\mu_{1}}\otimes\dots\otimes\D x^{\mu_{p}}
\end{equation}
What do we mean by totally antisymmetric? Well, it obeys
\begin{equation}
\omega_{\mu_{1}\dots\mu_{k}\mu_{k+1}\dots\mu_{p}}
=-\omega_{\mu_{1}\dots\mu_{k+1}\mu_{k}\dots\mu_{p}}.
\end{equation}
For a 2-form, the components look like
\begin{equation}
\partial_{\mu}\omega_{\nu}-\partial_{\nu}\omega_{\mu}
\end{equation}
for example.

\subsection{Exterior Calculus}
We will use the notation
\begin{equation}
A_{[\mu_{1}\dots\mu_{p}]}=\frac{1}{p!}(-1)^{\pi}A_{\pi(\mu_{1})\dots\pi(\mu_{p})}
\end{equation}
where $\pi$ is a permutation of the indices. Given a $p$-form
$A$, and a $q$-form $B$, the exterior product\marginpar{Exterior Product} is defined to be
\begin{equation}
(A\wedge B)_{\mu_{1}\dots\mu_{p+q}}
=\left(\frac{(p+q)!}{p!q!}\right)A_{[\mu_{1}\dots\mu_{p}]}B_{[\mu_{p+1}\dots\mu_{p+q}]}.
\end{equation}
We also have an exterior derivative\marginpar{Exterior Derivative}
\begin{equation}
(\D A)_{\mu_{1}\dots\mu_{p+1}}=\partial_{[\mu_{1}}A_{\mu_{2}\dots\mu_{p+1}]}.
\end{equation}
Consider concrete cases. If $A$ is a 1-form, then
\begin{equation}
\begin{split}
(\D A)_{\mu\nu}
&=\partial_{[\mu}A_{\nu]}\\
&=\frac{1}{2}(\partial_{\mu}A_{\nu}-\partial_{\nu}A_{\mu})
\end{split}
\end{equation}
Let $A$ be a $p$-form and $B$ be a $q$-form, then
\begin{equation}
\D(A\wedge B)=(\D A)\wedge B+(-1)^{p}A\wedge(\D B).
\end{equation}
Equivalently, in this formulation, we have for a function $f$
\begin{equation}
\D f=\vec{\nabla}f
\end{equation}
and
\begin{equation}
\D^{2}f=0.
\end{equation}
Whenever we have a $p$-form $A$ such that
\begin{equation}
\D A=0
\end{equation}
we call it a \define{Closed Form}\marginpar{Closed form: $\D A=0$\\ Exact Form $B=\D C$}.
We have an \define{Exact Form} be a $p$-form $B$ such that it is
of the form
\begin{equation}
B=\D C
\end{equation}
where $C$ is a $(p-1)$-form. Not all closed forms are exact.
\begin{ex}[Closed Inexact Form]
A closed but inexact form on a circle is $\D\theta$.
\end{ex}

\subsection{Differentiating Tangent Vectors}
Consider an arbitrary tangent vector
\begin{equation}
v=v^{\mu}\partial_{\mu}=v^{a}e_{a}.
\end{equation}
Lets consider what differentiation would look like in this
approach, we have
\begin{equation}
\partial_{\rho}v\;\mbox{``=''}\;(\partial_{\rho}v^{a})e_{a}+v^{a}(\partial_{\rho}e_{a}).
\end{equation}
But this second term is ambiguous? What should we have? Well, we
should write
\begin{equation}
\partial_{\rho}e_{a}=e_{b}{\Gamma^{b}}_{\rho a}-e_{b}{{\Gamma_{\rho}}^{b}}_{a}
\end{equation}
where ${{\Gamma_{\rho}}^{b}}_{a}$ is called the connection's components.
In general, we may say absolutely nothing about the connection as
it specifies the manifold. We can now write
\begin{equation}
\begin{split}
\partial_{\rho}v
&\mbox{``=''}(\partial_{\rho}v^{b})e_{b}+e_{b}{{\Gamma_{\rho}}^{b}}_{a}v^{a}\\
&\mbox{``=''}\underbracket[0.5pt]{(\partial_{\rho}v^{b}+{{\Gamma_{\rho}}^{b}}_{a}v^{a})}_{\nabla_{\rho}v^{b}}e_{b}
\end{split}
\end{equation}
where $\nabla_{\rho}v^{b}$ is the \define{Covariant Derivative}. 
The intuition is
\begin{equation}
\begin{pmatrix}
\mbox{Covariant}\\
\mbox{Derivative}
\end{pmatrix}=\begin{pmatrix}
\mbox{Derivative in}\\
\mbox{Flat Space}
\end{pmatrix}
+\begin{pmatrix}
\mbox{Corrections to}\\
\mbox{stay on the}\\\mbox{manifold}
\end{pmatrix}
\end{equation}
where the connection components are precisely these correction
terms. In a coordinate basis, the components becomes
${{\Gamma_{\rho}}^{\nu}}_{\mu}$ which are called the
\define{Christoffel Symbols}.
