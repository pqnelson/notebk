%%
%% lecture03.tex
%% 
%% Made by alex
%% Login   <alex@tomato>
%% 
%% Started on  Sun Feb 19 09:42:17 2012 alex
%% Last update Sun Feb 19 09:42:17 2012 alex
%%

We will study the Schwarzschild metric
\begin{equation}
\D s^{2}=\left(1-\frac{2m}{r}\right)\D t^{2}
-\left(1-\frac{2m}{r}\right)^{-1}\D r^{2}
-r^{2}\left(\D\theta^{2}+\sin^{2}(\theta)\D\varphi^{2}\right)
\end{equation}
where
\begin{equation}
m=\frac{GM}{c^{2}}.
\end{equation}
For the Sun, this is approximately
\begin{subequations}
\begin{equation}
m_{\odot}\sim 1.5\;\mbox{km}
\end{equation}
whereas for the Earth
\begin{equation}
m\sim 1\;\mbox{cm}.
\end{equation}
\end{subequations}
We would like to note:

(1) as $r\to\infty$, it looks like flat spacetime;

(2) at $r=2m$, one component vanishes while the other blows up;

(3) for $r<2m$, the signature ``changes''.

\noindent\ignorespaces %
This last remark means space-like curves within the region looks
time-like whereas time-like curves look space-like.

We solve the geodesic equation in the time component
\begin{equation}
\frac{\D}{\D s}\left(g_{ab}\frac{\D x^{b}}{\D s}\right)
-\frac{1}{2}(\partial_{a}g_{bc})\frac{\D x^{b}}{\D s}
\frac{\D x^{c}}{\D s}=0.
\end{equation}
Consider $a=0$, then we get
\begin{equation}
\frac{\D}{\D s}\left(g_{ab}\frac{\D x^{b}}{\D s}\right)-0=0
\end{equation}
which is a constant of motion:
\begin{equation}
\begin{split}
g_{tt}\frac{\D x^{t}}{\D s} &=\frac{\D}{\D s}
\left(1-\frac{2m}{r}\right)\frac{\D t}{\D s}\\
&=-\widetilde{E}
\end{split}
\end{equation}
This notation is tradition as it vaguely reminds us of energy.

Now for $a=3$, the $\varphi$ equation becomes
\begin{equation}
\frac{\D}{\D s}\left(g_{\varphi b}\frac{\D x^{b}}{\D s}\right)
=\frac{\D}{\D s}\left(g_{\varphi\varphi}\frac{\D x^{\varphi}}{\D s}\right)
\end{equation}
since the metric is diagonal, and
\begin{equation}
\frac{\D}{\D s}\left(g_{\varphi\varphi}\frac{\D x^{\varphi}}{\D s}\right)=0.
\end{equation}
Thus we have
\begin{equation}
r^{2}\sin^{2}(\theta)\frac{\D\varphi}{\D s}=\widetilde{L}
\end{equation}
be a constant of motion, which reminds us of angular momentum.

For the $a=2$ equation, we can set $\theta=\pi/2$ and
$\D\theta/\D s=0$ for the initial condition. This eliminates the
differential equation. We are left with radial geodesics.

The trick\marginpar{Trick: first integral of geodesic equation}
is to write the geodesic equation's first integral as 
\begin{equation}
\begin{split}
g_{ab}\frac{\D x^{a}}{\D s}\frac{\D x^{b}}{\D s} &=
\left(1-\frac{2m}{r}\right)\left(\frac{\D t}{\D s}\right)^{2}
-\left(1-\frac{2m}{r}\right)^{-1}\left(\frac{\D r}{\D s}\right)^{2}
-r^{2}\left(\frac{\D\varphi}{\D s}\right)^{2}\\
&=1
\end{split}
\end{equation}
Plugging in our constants of motion:
\begin{equation}
1=\left(1-\frac{2m}{r}\right)\widetilde{E}^{2}
-\left(1-\frac{2m}{r}\right)^{-1}\left(\frac{\D r}{\D s}\right)^{2}
-\left(\frac{\widetilde{L}}{r}\right)^{2}.
\end{equation}
This looks like the Hydrogen atom in quantum mechanics! We can
solve for
\begin{equation}
\left(\frac{\D r}{\D s}\right)^{2}=
\widetilde{E}^{2}-\left(1-\frac{2m}{r}\right)
\left(1+\frac{\widetilde{L}^{2}}{r^{2}}\right).
\end{equation}
Our equations of motion becomes
\begin{equation}
\boxed{
\begin{aligned}
\widetilde{L} &=r^{2}\frac{\D\varphi}{\D s}\\
-\widetilde{E} &= \left(1-\frac{2m}{r}\right)\frac{\D t}{\D s}\\
\left(\frac{\D r}{\D s}\right)^{2} &=
\widetilde{E}^{2}-\left(1-\frac{2m}{r}\right)
\left(1+\frac{\widetilde{L}^{2}}{r^{2}}\right)
\end{aligned}
}
\end{equation}
with the initial conditions $\theta=\pi/2$ and $\D\theta/\D
s=0$. It turns out that
\begin{equation}
s=\int\D s
\end{equation}
is a messy elliptic integral. So lets consider various
perturbative techniques to approximate it.

Lets consider
\begin{equation}
\left(\frac{\D r}{\D\varphi}\right)^{2}=\frac{(\D r/\D s)^{2}}{(\D\varphi/\D s)^{2}}
=\left(\frac{r}{\widetilde{L}}\right)^{4}[\dots].
\end{equation}
Things simplify if we write everything in terms of
\begin{equation}
u=\frac{1}{r}.
\end{equation}
We have
\begin{equation}
\left(\frac{\D u}{\D\varphi}\right)^{2}=
\frac{(\widetilde{E}^{2}-1)}{\widetilde{L}^{2}}
+\frac{2mu}{\widetilde{L}^{2}}
-u^{2}+2mu^{3}.
\end{equation}
If it weren't for the $2mu^{3}$ term, we could integrate this in
closed form. The extra term is precisely the relativistic
corrections, so we will treat it as a perturbation.

\subsection{First Approximation}
We simply ignore higher order terms. So we differentiate
\begin{equation}
\frac{\D^{2}u}{\D\varphi^{2}}=\frac{m}{\widetilde{L}^{2}}-u+3mu^{2}
\end{equation}
and throw away the $3mu^{2}$ term, obtaining
\begin{equation}
\frac{\D^{2}u}{\D\varphi^{2}}=\frac{m}{\widetilde{L}^{2}}-u.
\end{equation}
This has its solution be
\begin{equation}
u=\frac{m}{\widetilde{L}^{2}}+A\cos(\varphi).
\end{equation}
We can rewrite this as
\begin{equation}
u=\frac{1+e\cos(\varphi)}{a(1-e^{2})}
\end{equation}
where $e$ is eccentricity, and $a$ is the semimajor axis. We see
that
\begin{equation}
r^{-1}=\alpha+\beta\cos(\varphi)
\end{equation}
thus
\begin{equation}
\alpha r=1-\beta r\cos(\varphi)
\end{equation}
and using Cartesian coordinates yields
\begin{equation}
x^{2}+y^{2}=\frac{1}{\alpha^{2}}(1-\beta x)^{2}.
\end{equation}
What is this? Obviously an ellipse! This is wonderful, we recover
the Newtonian solution to the Kepler problem.
\begin{center}
  \includegraphics{img/lecture03.0}
\end{center}

\subsection{Second Approximation}
We write
\begin{equation}
u=u_{0}+y
\end{equation}
then our equation becomes
\begin{equation}
\frac{\D^{2}y}{\D\varphi^{2}}=\underbracket[0.5pt]{\frac{m}{\widetilde{L}^{2}}-u_{0}}-y+\underbracket[0.5pt]{3mu_{0}^{2}}+6mu_{0}y+3my^{2}
\end{equation}
where we choose $u_{0}$ to be such that the underbracketed terms
vanish:
\begin{equation}
\frac{m}{\widetilde{L}^{2}}-u_{0}+3mu_{0}^{2}=0.
\end{equation}
We ignore that $3my^{2}$ term as ``really small.'' Thus we have
\begin{equation}
\frac{\D^{2}y}{\D\varphi^{2}}\approx-(1-6mu_{0})y
\end{equation}
which has its solution be
\begin{equation}
y=A\cos(\sqrt{1-6mu_{0}}\varphi).
\end{equation}
We see when
\begin{equation}
\varphi\mapsto\varphi+\frac{2\pi}{\sqrt{1-6mu_{0}}}
\end{equation}
that $y\mapsto y$. For Mercury, this works out to be $(2.7\times
10^{-5})$ degrees per orbit, or roughly 0.1 arcseconds per orbit,
or again roughly 43 arcseconds per century. Although this is
small, it was observed by the end of the $19^{{\rm th}}$ century.

\subsection{Remarks on Experiments}
If the sun were oblate and the mass distribution followed this
shape, then there is a quadrapole term in Newtonian gravity.
It turns out the corrections account for roughly 4
arcseconds. After further more-precise experiments, it turns out
that General Relativity is correct to one part in a thousand. The
Messenger satellite is measuring the orbit of Mercury to great
precision. This will theoretically give the next order term.
We are also working on measuring the angular momentum of the
Sun. This will contribute to extra terms in geodesic expressions.

The next planet to think about is Mars. Thanks to the Viking
projects, there is precision to 100 meters of the orbit's
measurements. This is fairly remarkable, if you think about it!

Another test is the binary pulsar (a pulsar is a neutron star
that sends out a beam due to magnetic flux). There is a precision
of roughly 17 arcseconds per year. We can do this for a binary
star, but it is messy. There is a tidal distortions to the stars,
so it's not a sphere. For about 27--28 binary stars their orbits
agree with General Relativity (see, e.g., Kramer \emph{et al.}~\cite{Kramer:2004gj}), but for a few binary stars General
Relativity's predictions are really bad (most famously, DI Herculis~\cite{Claret:2010cj,Winn:2005ef}). It's a mystery what's
going on there! See Baumgarte, et al.,~\cite{Baumgarte:1997eg}
for simulating a binary neutron system, Laarakkers and
Poisson~\cite{Laarakkers:1997hb} for rotating neutron stars.

In Scalar-Tensor theories, the extra contribution to the
precession we get a scalar contribution which nearly cancel for
neutron stars. For other stars, it may be observable.

\begin{exercises}
\begin{xca}[Geodesics on the two-sphere]
A two-dimensional sphere of radius $R$ has a metric
\begin{equation}
\D s^2 = R^2 (\D\theta^2 + \sin^2(\theta)\D\varphi^2)
\end{equation}
Show that the geodesics of this metrics are great circles.
\end{xca}
\begin{xca}[First integral of the geodesic equation]
Show that the equation
\begin{equation}
g_{ab}\frac{\D x^{a}}{\D s}\frac{\D x^{b}}{\D s}=1
\end{equation}
is a first integral of the geodesic equation, that is, that the
$s$ derivative of this equation vanishes whenever the geodesic
equation holds. 
\end{xca}
\begin{xca}[Practice with tensors, indices, summation convention, etc.]
Consider the following problems.
\begin{enumerate}
\item Let ${\delta_{a}}^{b}$ be the Kronecker delta in an
  $n$-dimensional spacetime. Find ${\delta_{a}}^{a}$. 
\item\label{xca:lec3:probTensors:2} Suppose $S_{ab}$ is symmetric (that is, $S_{ab} = S_{ba}$)
  and $A_{ab}$ is antisymmetric (that is, $A^{ab} =
  -A^{ba}$). Show that $S_{ab}A^{ab}= 0$.
\item For $A^{ab}$ as in part \eqref{xca:lec3:probTensors:2}, and for an arbitrary tensor $T_{ab}$, show that
\begin{equation}
A^{ab} T_{ab} = A^{ab} (T_{ab} - T_{ba})
\end{equation}
\end{enumerate}
\end{xca}
\begin{xca}[{\cite[\normalfont3.18]{lightman:1975}}]
Let $Y_{\alpha\beta\gamma}$ be an arbitrary tensor, show
\begin{equation}
Y_{\alpha\beta\gamma}\not=Y_{(\alpha\beta\gamma)}+Y_{[\alpha\beta\gamma]}.
\end{equation}
\end{xca}
\end{exercises}
