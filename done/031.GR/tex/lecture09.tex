%%
%% lecture09.tex
%% 
%% Made by alex
%% Login   <alex@tomato>
%% 
%% Started on  Mon Feb 27 14:05:38 2012 alex
%% Last update Mon Feb 27 14:05:38 2012 alex
%%


\subsection{Metric Signatures, Index Gymnastics}
We can use the metric to go from covectors to vectors, and back
again. We also use it for index gymnastics:
\begin{subequations}
\begin{align}
{T^{\mu\nu}}_{\sigma}g_{\mu\rho} &=
{{T_{\rho}}^{\nu}}_{\sigma}\\
{{T_{\rho}}^{\nu}}_{\sigma}g^{\rho\tau}
&={T^{\mu\nu}}_{\sigma}g_{\mu\rho}g^{\rho\tau}\\
&={T^{\tau\nu}}_{\sigma}
\end{align}
\end{subequations}
For $g_{\mu\nu}(x)$ at a point, we can find coordinates where
this is of the form
\begin{equation}
g_{\mu\nu}(x)=\diag(\underbrace{-1,\dots,-1}_{p},\underbrace{+1,\dots,+1}_{q}),
\end{equation}
where $p+q=n$.

The signature of the metric is 
\begin{equation}
\begin{pmatrix}\mbox{metric}\\
\mbox{signature}
\end{pmatrix}
=
\begin{pmatrix}\mbox{number}\\
\mbox{of $+$}
\end{pmatrix}
-\begin{pmatrix}\mbox{number}\\
\mbox{of $-$}
\end{pmatrix}
\end{equation}
Physicists always write $+---$ or $-+++$ indicating the metric
signature, mathematicians write $(1,3)$ or $(3,1)$. If the
signature is $(+\dots+)$, the metric is called
\define{Riemannian}; and if the signature is either $(-+\dots+)$
or $(+-\dots-)$, the metric is \define{Lorentzian}.

At any point $P$ with coordinates\marginpar{Riemann Normal Coordinates} $\overline{x}$, there is a
coordinate system in which
\begin{equation}
g_{\mu\nu}(x)=\eta_{\mu\nu}+\bigO(x-\overline{x})^{2}
\end{equation}
where $\eta_{\mu\nu}$ is the flat Minkowski metric. These
coordinates are called the \define{Riemann Normal
  Coordinates}. Physically this is a freely falling frame, no
first order fictitious forces felt.


\subsection{Tensor Densities}
Suppose we have an $n$-dimensional manifold, we look at a totally
antisymmetric type $(0,n)$ tensor
\begin{equation}
\tens{T}=T_{\mu_{1}\mu_{2}\dots\mu_{n}}
\end{equation}
has only one component. We see the nonzero component is
\begin{equation}
T_{012\dots(n-1)}
\end{equation}
since if $\mu_{i}=\mu_{j}$, antisymmetry demands $\tens{T}=0$ for
the component. It looks like a function, but it doesn't transform
as such.

How does a totally antisymmetric tensor transform under a change
of coordinates? We write out the components
\begin{equation}
T_{\mu_{1}\mu_{2}\dots\mu_{n}}\D x^{\mu_{1}}\otimes\dots\otimes\D x^{\mu_{n}}
=
T'_{\nu_{1}\nu_{2}\dots\nu_{n}}\D y^{\nu_{1}}\otimes\dots\otimes\D y^{\nu_{n}}
\end{equation}
The first thing to do is act on
$(\partial'_{\rho_{1}},\dots,\partial'_{\rho_{n}})$ which obey
\begin{equation}
\<\D y^{\nu_{i}},\partial'_{\rho_{j}}\>={\delta^{\nu_{i}}}_{\rho_{j}},
\end{equation}
so
\begin{equation}
\begin{split}
T'_{\rho_{1}\dots\rho_{n}}
&=T_{\mu_{1}\dots\mu_{n}}\<\D y^{\mu_{1}},\partial'_{\rho_{1}}\>(\dots)
\<\D y^{\mu_{n}},\partial'_{\rho_{n}}\>\\
&=T_{\mu_{1}\dots\mu_{n}}
\frac{\partial x^{\mu_{1}}}{\partial y^{\rho_{1}}}
(\dots)\frac{\partial x^{\mu_{n}}}{\partial y^{\rho_{n}}}
\end{split}
\end{equation}
Wonderful, lets work in a concrete situation: 2-dimensional
manifolds. We see
\begin{subequations}
\begin{align}
T'_{01}
&=T_{\mu\nu}\frac{\partial x^{\mu}}{\partial y^{0}}
\frac{\partial x^{\nu}}{\partial y^{1}}\\
&=T_{01}\frac{\partial x^{0}}{\partial y^{0}}
\frac{\partial x^{1}}{\partial y^{1}}+
T_{10}\frac{\partial x^{1}}{\partial y^{0}}
\frac{\partial x^{0}}{\partial y^{1}}\\
&=T_{01}\left(
\frac{\partial x^{0}}{\partial y^{0}}
\frac{\partial x^{1}}{\partial y^{1}}-
\frac{\partial x^{1}}{\partial y^{0}}
\frac{\partial x^{0}}{\partial y^{1}}
\right)\\
&=T_{01}\det\left|
\frac{\partial x^{\mu}}{\partial y^{\nu}}
\right|
\end{align}
\end{subequations}
We can generalize this result
\begin{equation}
T'_{01\dots(n-1)}=T_{0\dots(n-1)}\det|\partial x/\partial y|.
\end{equation}
An object which transforms this way is called a scalar density of
weight $-1$.

Again, the general\marginpar{Tensor Density of Weight $\omega$} notion
is a tensor density of weight $\omega$ is
\begin{equation}
T'_{\mu_{1}\dots\mu_{n}}=T_{\nu_{1}\dots\nu_{n}}
\frac{\partial x^{\nu_{1}}}{\partial y^{\mu_{1}}}(\dots)
\frac{\partial x^{\nu_{n}}}{\partial y^{\mu_{n}}}
\det|\partial y/\partial x|^{\omega}.
\end{equation}
A tensor density is more general than a pseudotensor (recall: a
pseudotensor is just some quantity with indices).

\begin{ex}[Determinant of Metric Tensor]
Consider
\begin{equation}
g=\det|g_{\mu\nu}|
\end{equation}
Under a coordinate transformation $x^{\mu}\to y^{\nu}(x)$, we have
\begin{equation}
g'_{\mu\nu}=g_{\alpha\beta}\frac{\partial x^{\alpha}}{\partial y^{\mu}}
\frac{\partial x^{\beta}}{\partial y^{\nu}}
\end{equation}
thus
\begin{equation}
g'=g\det|\partial x/\partial y|^{2}
\end{equation}
So $g$ is a scalar density of weight $-2$, and moreover this
implies $\sqrt{|g|}$ is a scalar density of weight $-1$.
\end{ex}

\begin{ex}[Levi--Civita Symbol]
Consider the alternating symbol
\begin{equation}
\widetilde{\varepsilon}_{\mu_{1}\dots\mu_{n}}
=\begin{cases}
+1 & \mbox{if even permutation}\\
-1 & \mbox{if odd permutation}\\
0 & \mbox{if any two indices equal.}
\end{cases}
\end{equation}
It's a totally antisymmetric matrix, but it is not a tensor: the
alternating symbol is a tensor density of weight $+1$. We can
define a genuine tensor by
\begin{equation}
\varepsilon_{\mu_{1}\dots\mu_{n}}=\sqrt{|g|}\widetilde{\varepsilon}_{\mu_{1}\dots\mu_{n}}
\end{equation}
which is the Levi--Civita symbol\index{Levi--Civita Symbol}. We
should note, using abstract index notation, the cross product is
\begin{equation}
(A\times B)^{i}=g^{ij}\widetilde{\varepsilon}_{jk\ell}A^{k}B^{\ell}.
\end{equation}
\end{ex}
