%%
%% lecture19.tex
%% 
%% Made by alex
%% Login   <alex@tomato>
%% 
%% Started on  Fri Jul  6 11:16:20 2012 alex
%% Last update Fri Jul  6 11:16:20 2012 alex
%%

So lets consider a binary neutron star. Let $r$ be the radius of
the binary neutron star, and we are a distance $R$ away from the
center of mass. We can doodle the situation:
\begin{center}
\includegraphics{img/lecture19.0}
\end{center}
If $R\ggg r$, then we can approximate this as
\begin{equation}
\bar{h}_{\mu\nu} \approx\frac{4G}{R}\int
T_{\mu\nu}(\vec{y},t-R)\,\D^{3}y.
\end{equation}
Remember we're in a gauge where
\begin{equation}
\partial_{\mu}\bar{h}^{\mu\nu}=0
\end{equation}
which implies $h^{t\mu}$ is determined by $h^{ij}$. So for all
practical purposes, we can just compute $h^{ij}$.

Now, for some tricks:
\begin{equation}
\begin{split}
\partial_{k}(x^{i}T^{jk})
&={\delta^{i}}_{k}T^{jk} + x^{i}\partial_{k}T^{jk}\\
&=T^{ij} + x^{i}\partial_{k}T^{jk}\\
&=T^{ij} - x^{i}\partial_{t}T^{jt} + \bigO(h).
\end{split}
\end{equation}
Thus we have
\begin{equation}
T^{ij} = \partial_{t}(x^{i}T^{tj})
+ \begin{pmatrix}\mbox{total}\\\mbox{derivative}
\end{pmatrix}
+\bigO(h)
\end{equation}
Using Stoke's theorem, the integral of a total derivative is
zero, so we have
\begin{equation}
\bar{h}_{ij} = \frac{2G}{R}\frac{\D^{2}}{\D t^{2}}\int
\underbrace{y^{i}y^{j}T^{tt}(\vec{y},t-R)}_{\mathclap{\text{mass quadrapole moment}}}\,\D^{3}y
\end{equation}
which confirms the handwaving arguments from the last lecture, which
we justified with conservation laws.

Observe the quadrapole moment behaves as
\begin{equation}
I_{ij}\sim mr^{2}
\end{equation}
for a binary star with comparable masses. So we see
\begin{equation}
\ddot{I}_{ij}\sim mv^{2}
\end{equation}
If this is a gravitationally bound system, it works out\dots but
this means that
\begin{equation}
h\sim G\frac{mv^{2}}{R}.
\end{equation}
Further we know for gravitational systems
\begin{equation}
v^{2}\sim G\frac{m}{r}
\end{equation}
and thus
\begin{equation}
h\sim v^{4}\left(\frac{r}{R}\right).
\end{equation}
We discuss these things in detail in the following box. A binary
neutron star affects the distance by about $1/1000$ of the
diameter of a nucleus, though.
%%
%% simplificationsForWeakRad.tex
%% 
%% Made by alex
%% Login   <alex@tomato>
%% 
%% Started on  Sun Mar 11 14:00:37 2012 alex
%% Last update Sun Mar 11 14:00:37 2012 alex
%%


\begin{Boxed}{Some Simplifications for Weak Gravitational Radiation}
We saw that to first order in perturbation theory
\begin{equation}
\bar{h}_{\mu\nu}(\mathbf{x},t)=
4G\int\frac{T_{\mu\nu}(\mathbf{y},t-|\mathbf{y}-\mathbf{x}|)}{|\mathbf{y}-\mathbf{x}|}\,\D^{3}y
\end{equation}
Let us concentrate on the purely spatial components
$\bar{h}_{ij}$ since the remaining components $\bar{h}_{0\mu}$
may be obtained by using the harmonic gauge condition
$\partial^{\mu}\bar{h}_{\mu\nu}=0$.

First, suppose an isolated source is at a distance $R$, and has
linear size $r\lll R$. Then to a good approximation,
\begin{equation}\label{eq:box2:goodApprox}
\bar{h}_{\mu\nu}(\mathbf{x},t)=
\frac{4G}{R}\int T_{\mu\nu}(\mathbf{y},t-|\mathbf{y}-\mathbf{x}|)\,\D^{3}y
\end{equation}
Now, to lowest order in $h$, energy conservation implies that
\begin{equation}
\partial_{\mu}T^{\mu\nu}=0=\partial_{i}T^{i\nu}+\partial_{t}T^{t\nu}
\end{equation}
We can now use a trick. Note the identities
\begin{equation}\label{eq:box2:trick:id1}
\begin{split}
\partial_{k}(x^{i}T^{kj})&=\delta^{i}_{k}T^{kj}+x^{i}\partial_{k}T^{kj}\\
&=T^{ij}-x^{i}\partial_{t}T^{tj}
\end{split}
\end{equation}
\begin{equation}\label{eq:box2:trick:id2}
\begin{split}
\partial_{\ell}(x^{i}x^{j}T^{t\ell})
&=\delta^{i}_{\ell}x^{j}T^{t\ell}+\delta^{j}_{\ell}x^{i}T^{t\ell}
+x^{i}x^{j}\partial_{\ell}T^{t\ell}\\
&=x^{j}T^{ti}+x^{i}T^{tj}-x^{i}x^{j}\partial_{t}T^{t\ell}
\end{split}
\end{equation}
Solving \eqref{eq:box2:trick:id1} for $T^{ij}$, using the
symmetry of $T^{ij}$, and inserting \eqref{eq:box2:trick:id2},
we see that
\begin{align}
T^{ij}
&=x^{i}\partial_{t}T^{tj}+\partial_{k}(x^{i}T^{kj})\nonumber\\
&=\frac{1}{2}\partial_{t}(x^{i}T^{tj}+x^{j}T^{ti})+
\frac{1}{2}\partial_{k}(x^{i}T^{kj}+x^{j}T^{ik})\nonumber\\
&=\frac{1}{2}\partial_{t}\Bigl(\partial_{\ell}(x^{i}x^{j}T^{t\ell})+x^{i}x^{j}\partial_{t}T^{tt}\Bigr)
+\frac{1}{2}\partial_{k}\Bigl(x^{i}T^{kj}+x^{j}T^{ik}\Bigr)\nonumber\\
&=\frac{1}{2}\partial_{t}^{2}(x^{i}x^{j}T^{tt})
+\frac{1}{2}\partial_{\ell}\Bigl(\partial_{t}(x^{i}x^{j}T^{t\ell})+x^{i}T^{\ell j}+x^{j}T^{i\ell}\Bigr)
\end{align}
We plug this back into Equation \eqref{eq:box2:goodApprox}. By
stokes theorem, the term involving $\partial_{\ell}$ integrates
to zero---by assumption, the source is isolated, so the integral
can be converted to a surface integral over a surface
\emph{outside} the source, where $T^{\mu\nu}=0$. Hence
\begin{equation}
\begin{split}
\bar{h}_{ij}(\mathbf{x},t)
&=\frac{2G}{R}\int\partial^{2}_{t}(y^{i}y^{j}T^{tt})\,\D^{3}y\\
&=\frac{2G}{R}\frac{\D^{2}}{\D t^{2}}\int y^{i}y^{j}T^{tt}(\mathbf{y},t-|\mathbf{y}-\mathbf{x}|)\,\D^{3}y.
\end{split}
\end{equation}
The integral is the quadrupole moment; thus, the metric
perturbation goes as the second time derivative of the quadrupole
moment.

For an isolated system of a few gravitating bodies (say, stars)
with masses of order $m$ and velocities of order $v$, the
quadrupole moment is $\sim mr^{2}$, and thus $\bar{h}\sim
Gmv^{2}/R$. Furthermore, if the system is gravitationally bound,
$v^{2}\sim Gm/r$, so $\bar{h}\sim v^{4}r/R$.

For a typical binary neutron star, $r\sim 10^{7}$ km and
$v^{2}\sim10^{-7}$; for such a system at a distance of a
kiloparsec, this gives $\bar{h}\sim10^{-21}$.
\end{Boxed}


Nevertheless, in the next five years we will detect these
things. LIFO has a photon running around in a pipe with length
$L\sim 10^{3}\,\mathrm{m}$ about $10^{3}$ times which is
effectively $L_{eff}\sim 10^{9}\,\mathrm{m}$. So there would be
constructive or destructive interference.

If we are lucky, we'll see results in a year (in 2010); huge
upgrades are due in 2009. Interesting quantum effects decreasing
uncertainty in error wavelength; we are nearly saturating
uncertainty at this point.

There is something of note: the weak field approximation of the
Einstein tensor gives us
\begin{equation}
G_{\mu\nu} = \frac{-1}{2}\Box\bar{h}_{\mu\nu}+(\partial
h)(\partial h)
\end{equation}
where 
\begin{equation}
(\partial
h)(\partial h)
= \begin{pmatrix}\mbox{self-contribution}\\\mbox{term}
\end{pmatrix} =: t^{\text{(grav)}}_{\mu\nu}.
\end{equation}
It's not really a tensor! The energy carried off by the radiation
can be found by identifying power $\sim t^{0i}$. So the total
power radiated is 
\begin{equation}
P\sim t^{0i}R^{2}. 
\end{equation}
Remember that
\begin{equation}
h\sim \frac{G}{R}\ddot{I}
\end{equation}
thus
\begin{equation}
\begin{aligned}
t&\sim\frac{1}{G}\dot{h}^{2}\\
&\sim G(\dddot{I}/R)^{2}.
\end{aligned}
\end{equation}
Hence the power looks like
\begin{equation}
P\sim R^{2}t\sim G\dddot{I}^{2}.
\end{equation}
Remember
\begin{equation}
I\sim mr^{2},\quad\mbox{so}\quad
\dddot{I}\sim mva
\end{equation}
and by Newton's Laws
\begin{equation}
\dddot{I}\sim mv^{3}/r.
\end{equation}
Then the power looks like
\begin{equation}
\begin{aligned}
P & \sim G(mv^{3}/r)^{2}=Gm^{2}v^{6}/r^{2}\\
&\sim mv^{8}/r
\end{aligned}
\end{equation}
where the last manipulation is again by Newton's Laws. Remember
we set $c=1$, so the power is really small.


\begin{exercises}
\begin{xca}[Detecting gravitational radiation I]
We considered a coordinate system (``gauge'') for weak fields in
which $\partial_{\mu}\bar{h}^{\mu\nu}=0$. In a region in which the
  stress-energy tensor $T_{\mu\nu}$ is zero, we can make a
  further coordinate transformation such that $h_{0\mu}=0$ and
  $h=\eta^{\mu\nu}h_{\mu\nu}=0$. In such a coordinate system, a
  gravitational plane wave moving along the $z$ axis has a metric
  (see, e.g., Carroll~\cite{Carroll:2004st} section 7.4) 
\begin{equation}\label{eq:ex1:metricWeWannaFind}
g_{\mu\nu}=\eta_{\mu\nu}+C_{\mu\nu}\cos\bigl(\omega(t-z)\bigr),
\quad\mbox{with}\quad
C_{\mu\nu} = \begin{pmatrix} 0 & 0 & 0 & 0\\
0 & h_{+} & h_{\times} & 0\\
0 & h_{\times} & -h_{+} & 0\\
0 &0 & 0 & 0
\end{pmatrix}
\end{equation}
where $h_{+}$ and $h_{\times}$ are constants.
\begin{enumerate}
\item Under a rotation in the $x$-$y$ plane, the coordinates transform as
\begin{equation}
\begin{aligned}
x &= \bar{x}\cos\theta + \bar{y}\sin\theta\\
y &= \bar{y}\cos\theta - \bar{x}\sin\theta
\end{aligned}
\end{equation}
Find the transformations for $h_{+}$ and $h_{\times}$. For what
angle are the two polarizations interchanged? 
\item Consider a mass initially located at position $(x_{0}, 0,
  0)$ with vanishing initial velocity, $\D x^i /\D s = 0$. Find
  the geodesic equation for this object, with the metric
  \eqref{eq:ex1:metricWeWannaFind}, to first order in $h$. Show
  that the object will remain at rest at $(x_{0}, 0, 0)$. (Note
  that ``at rest'' is a coordinate-dependent statement. That's
  OK, though: for this problem, the gauge conditions have
  implicitly determined a unique coordinate system.) 
\item\label{ex1:partC:lot8} Consider two mirrors, at rest along the $x$ axis at $(0, 0,
  0)$ and $(L, 0, 0)$. Using the metric
  \eqref{eq:ex1:metricWeWannaFind}, compute the round-trip time
  $\Delta t$ for a light pulse starting at the origin at time
  $t_{0}$, moving along the $x$ axis, reflecting from the mirror
  at $x = L$, and returning to the origin. For this computation
  you can assume that $\Delta t\ll1/\omega$, so the quantity
  $\omega{t}$ can be treated as a constant. Your answer should
  depend on $t_{0}$; if it doesn't, you've made a mistake. 
\item For flat spacetime, the round-trip time $\Delta t$ of part
  \ref{ex1:partC:lot8} is $2L$. The effect of the gravitational
  wave is the same as if the light traveled a slightly different
  distance $2L +\Delta L$. For a ``strain'' of $h\sim 10^{-21}$
  (a reasonable estimate for astrophysical sources of
  gravitational radiation) and a mirror separation $L\sim 4\,{\rm
    km}$ (the length of an arm of the LIGO detector), estimate
  the maximum value of $\Delta{L}$. Compare this to the size of a
  typical atomic nucleus of about $1\,{\rm fm}$. Tiny as it is,
  this change in distance should be detectable in an
  interferometer! 
\end{enumerate}
\end{xca}
\begin{xca}[Detecting gravitational radiation II]
Another way to construct a gravitational wave detector is to use a metal bar isolated from external sources of noise. When a gravitational wave passes, the two ends of the bar will accelerate at different rates, setting up oscillations. The relative acceleration of the two ends is determined by the equation of geodesic deviation. The deviation vector $X$ can be interpreted as the distance between the two ends of the rod---since spacetime
is assumed to be nearly flat, it makes sense to talk about Cartesian coordinates and
distances. In this problem, we will model the bar of metal by two masses at the ends of
a spring.

A weak gravitational wave is given by the metric of problem 2. Consider two equal masses on a spring in the $x$-$y$ plane, initially separated by a distance L, so
\begin{equation}
X_{0} = X(t = 0) \approx (L \cos \theta, L sin \theta, 0)
\end{equation}
Say that the spring has natural frequency $\omega_{0}$, that is,
that it exerts a restoring acceleration $a = \omega_{0}^{2} (X -
X_{0})$ when the ends are displaced from their initial
positions. We shall look at the effect of a gravitational wave as
a driving force for this oscillator (ignoring its tendency to
rotate the spring). 

Start again with a gravitational wave moving along the $z$ axis, 
with polarization $h_{\times}$.
Since the wave is assumed to be weak, we can write
\begin{equation}
X(t) = X_{0} + \zeta(t)
\end{equation}
with $\zeta$ small, and work to lowest order.
\begin{enumerate}
\item From the geodesic deviation equation, find the component of
  gravitational acceleration along the direction of the spring in
  terms of $h_{\times}$, $k$, and---to lowest order---$X_{0}$. 
\item Solve the equations of motion for $\zeta$ subject to this
  acceleration and the restoring force of the spring. (Strictly
  speaking, by $\zeta$ I mean here ``the component of $\zeta$ in
  the direction of the spring,'' since we're ignoring rotation of
  the spring in the $x$-$y$ plane.) Explain the dependence on
  $\theta$, the angle at which the bar lies in the $x$-$y$
  plane. 
\end{enumerate}
\end{xca}
\end{exercises}
