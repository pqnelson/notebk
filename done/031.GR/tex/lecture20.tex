%%
%% lecture20.tex
%% 
%% Made by alex
%% Login   <alex@tomato>
%% 
%% Started on  Wed Jul 11 10:23:24 2012 alex
%% Last update Wed Jul 11 10:23:24 2012 alex
%%

The weakfield equations' advantage: reduces Einstein's equations
to linear, uncoupled differential equations. But the problem is
it doesn't tell us everything. On the other hand, the full
Einstein equation has for each component some 50000 terms if we
don't use summation. If we use symmetry, we can reduce the number
of terms.

We first look at a static (time independent) and spherically
symmetric solution. (In general, if we assume ``homogeneous'', then
we cannot assume ``static''.) 

The technical way to deal with time independence is to say there
exists a timelike vector $\zeta^{\mu}$ such that the transformation
\begin{equation}
x^{\mu}\to x^{\mu}+\zeta^{\mu}
\end{equation}
doesn't change the metric. We saw in Equation \eqref{eq:lec16:killingVec}
the metric changes under transformations of this sort as
\begin{equation}
g_{\mu\nu}\to
g_{\mu\nu}+\nabla_{\mu}\zeta_{\nu}+\nabla_{\nu}\zeta_{\mu}.
\end{equation}
We have
\begin{equation}
\nabla_{\mu}\zeta_{\nu}+\nabla_{\nu}\zeta_{\mu}=0
\end{equation}
the Killing equation, and $\zeta^{\mu}$ is the Killing vector. We
showed in Exercise \ref{xca:prob4:killing} that
\begin{equation}
\nabla_{\mu}\zeta_{\nu}+\nabla_{\nu}\zeta_{\mu}=
g_{\mu\rho}\partial_{\nu}\zeta^{\rho}
+g_{\nu\rho}\partial_{\mu}\zeta^{\rho}
+\zeta^{\rho}\partial_{\rho}g_{\mu\nu}.
\end{equation}
We just take coordinates $\zeta^{t}=1$, and $\zeta^{i}=0$ (i.e.,
rescale the time components). Then we have our Killing equation
reduce to
\begin{equation*}\tag{\mbox{stationary metric}}
\zeta^{t}\partial_{t}g_{\mu\nu}=\partial_{t}g_{\mu\nu}=0
\end{equation*}
which is a coordinate independent expression telling us the
metric is time independent. There is a subtlety here: a rotating
object doesn't appear to change. We introduce another condition,
a new symmetry as $t\to-t$. A stationary metric which satisfies
is said to be \define{Static}. In these coordinates, this is
equivalent to 
\begin{equation}
g_{it}=0.
\end{equation}
So
\begin{equation}
\D s^{2}=g_{tt}\,\D t^{2}+g_{ij}\,\D x^{i}\,\D x^{j}
\end{equation}
(If we cannot eliminate $g_{it}$, it's an indicator of a moving system.)

We will examine the static, spherically symmetric metric. The
metric shouldn't ``change'' when moving along an entire loop on a
2-sphere; in some sense there is an invariance. The angular
dependence is just
\begin{equation*}
\D\theta^{2}+\sin^{2}(\theta)\,\D\varphi^{2}
\end{equation*}
so
\begin{equation}
\D s^{2} = g_{tt}\,\D t^{2} - R^{2}(\D\theta^{2}+\sin^{2}(\theta)\,\D\varphi^{2})
-g_{rr}\,\D r^{2}
\end{equation}
if we think of the spacetime as foliated spheres, the $r$
determines which spherical disc we're on. We know that $g_{tt}$,
$R$, $g_{rr}$ depend on $r$ but not on the angles or we wouldn't
have spherical symmetry, nor does it have a dependency on $t$.
\begin{rmk}
See Wald~\cite{Wald:1984rg} for a good discussion of
$\mathrm{SO}(3)$ symmetry in General Relativity.
\end{rmk}
We still have one coordinate degree of freedom---$r$. We can still
choose many different coordinate systems.

If we fix $r$, we can do it several different ways. The laziest
way is to choose $r$ satisfying
\begin{equation}
g_{rr}=1
\end{equation}
which happens when $r$ is the proper distance. We can also choose
$r$ to satisfy instead
\begin{equation}
g_{rr} = R^{2}/r^{2}
\end{equation}
which then gives us
\begin{equation}
\D s^{2} = g_{tt}\,\D t^{2} - g_{rr}\underbrace{\bigl(\D r^{2} + r^{2}(\D\theta^{2}+\sin^{2}\theta\,\D\varphi^{2})\bigr)}_{\text{usual flat metric}}.
\end{equation}
With this choice we have isotropic coordinates. Both of these
make the field equations a wee bit complicated. But there is a
third choice! We fix
\begin{equation}
R=r
\end{equation}
which are ``areal coordinates'' describing a 2-sphere at $r$ with
area $4\pi r^{2}$.

For the Schwarzschild solution, we choose areal
coordinates. (Originally Schwarzschild chose coordinates where
$\det(g)=1$.) We then have
\begin{equation}
\D s^{2} = A(r)\,\D t^{2}-B(r)\,\D
r^{2}-r^{2}(\D\theta^{2}+\sin^{2}\theta\,\D\varphi^{2})
\end{equation}
In the Einstein vacuum equation we have
\begin{equation}
A = B^{-1} = 1-\frac{2m}{r}
\end{equation}
But really by integration, the constant term (the ``1'') is an
integration constant.

What if $r\approx 2m$? It gets mildly interesting. At $r=2m$,
something goes horribly awry since $A\to0$ but $B\to\infty$, and
$\D s^{2}\to\textbf{??}$ This was not understood for a
longtime. Back in the 1920s, Panlieve et al.~wrote papers with
novel coordinate systems but this was largely ignored. 
Is this singularity from poor choice of coordinates, or from
something deep and not easily understandable in nature?

Lets examine as an example
\begin{equation}
\begin{aligned}
\D s^{2}&= \D x^{2}+\D y^{2}, &\quad &\mbox{let}\quad
x=\frac{1}{u-1}\\
&=\frac{\D u^{2}}{(u-1)^{2}}+\D y^{2}
\end{aligned}
\end{equation}
at $u=1$ we have a singularity! 

We can try to look at coordinate independent quantities as a
first step. For example
\begin{subequations}
\begin{align}
R &= 0\\
R_{\mu\nu}R^{\mu\nu} &=0.
\end{align}
but
\begin{equation}
R_{\mu\nu\rho\sigma}R^{\mu\nu\rho\sigma}=\frac{48m^{2}}{r^{6}}.
\end{equation}
\end{subequations}
At $r=2m$, nothing scary happens! It turns out every scalar we
can form from the curvature behaves unsuspiciously at
$r=2m$. Physically, there doesn't appear locally anything new and
scary. 

On the other hand, for $r<2m$, the temporal component and radial
component switch. That is
\begin{equation}
A(r)<0,\quad\mbox{and}\quad B(r)<0
\end{equation}
so spacelike becomes timelike, and timelike becomes spacelike.
