%%
%% introMovtTwo.tex
%% 
%% Made by Alex Nelson
%% Login   <alex@tomato>
%% 
%% Started on  Sat Jul 18 12:12:48 2009 Alex Nelson
%% Last update Sat Jul 18 12:12:48 2009 Alex Nelson
%%

We introduced the basic notions in category theory (objects,
morphisms, functors, and natural transformations), and we
introduced the notion of objects as ``stuff'' equipped with
``structure'' such that ``properties'' hold. We went out of our
way to show that categories are objects. Now we are interested in
some structure we equip on categories and properties we demand
categories to obey.

The first major concept we want to tackle is the notion of
universal arrows. But this is too deep a concept to be tackled
``head on'', we need to cover a few preliminary notions first.

We introduce the notion of duality, and specifically how to find
dual properties in category theory. Intuitively, we find the
``dual'' to something by ``reversing the direction of the
arrows''. 

Following this principle, we'll move on to initial and terminal
objects. Put simply, initial objects have exactly one arrow to
every object in the category. Terminal objects are ``dual'' to
this (every object has exactly one arrow to the terminal object).


We then proceed to introduce the notion of comma categories. That
is, a category has objects and morphisms, and morphisms are
themselves (in a sense) objects. The logical question is: can we
have a category whose objects are morphisms? This is precisely
what comma categories formalize.

These notions provide sufficient structure and properties to
introduce the notion of universal arrows. That is, in math we
come across the recurring phrase ``\ldots\emph{there exists} a
\emph{unique} function such that\ldots'' which is concerned with:
(1) existence, i.e. define entities; and (2) uniqueness,
i.e. prove properties. It turns out that a universal arrow is an
initial object in a comma category. 

We'll cover a few examples of universal arrows and its
usefulness, but that concludes this movement of the
mathematician's approach.
