%%
%% pullbackPushout.tex
%% 
%% Made by Alex Nelson
%% Login   <alex@tomato>
%% 
%% Started on  Mon Jul 20 14:21:40 2009 Alex Nelson
%% Last update Mon Jul 20 14:21:40 2009 Alex Nelson
%%

We can describe pullbacks (and its dual, pushouts) using limits
(respectively colimits). We'll unfortunately have to dance around
the intuitive notion of a pullback, its rigorous definition, and
various examples (hence why it's in the Minuet chapter!). 

\subsection{Limit Point of View}

Consider the situation when we have a pair of morphisms
\begin{equation}%\label{eq:}
\vcenter{\xymatrix{
            & B\ar[d]^{g}\\
A\ar[r]_{f} & C
}}
\end{equation}
A \define{Pullback} is a universal cone with vertex $U$ over this
diagram. That means we have a pair of projection maps $p:U\to{A}$
and $q:U\to{B}$ such that
\begin{equation}%\label{eq:}
\vcenter{\xymatrix{
U\ar[d]_{p}\ar[r]^{q}   & B \ar[d]^{g}\\
A \ar[r]_{f}           & C
}}
\end{equation}
commutes. This is sort of like a product, at least intuitively,
but if we have another object $V$ and pair of projection
morphisms $t:V\to{B}$ and $s:V\to{A}$, then we see we have to demand
\begin{equation}%\label{eq:}
\vcenter{\xymatrix@C+2em@R+2em{
   V \ar@/_10pt/[ddr]_{s} \ar@/^10pt/[drr]^{t} \ar@{-->}[dr]^{\scriptstyle !} & & \\
   & U \ar[d]_(0.4){p} \ar[r]^(0.4){q} & B \ar[d]^{g}\\
   & A \ar[r]_{f} & C
}}
\end{equation}
commutes. We denote this pullback by $A\times_{C}B$. It's
sometimes known as a fibred product or a Cartesian square.
