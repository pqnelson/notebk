%%
%% duality.tex
%% 
%% Made by Alex Nelson
%% Login   <alex@tomato>
%% 
%% Started on  Sat Jul 18 12:15:08 2009 Alex Nelson
%% Last update Sat Jul 18 12:15:08 2009 Alex Nelson
%%
Duality is a very useful notion in category theory. We use it to
get ``two objects for the price of one definition''. 

\begin{defn}%\label{defn:}
Given a category $\ms{A}$, we can define a \define{dual category of $\ms{A}$}
consists of
\begin{enumerate}
\item a collection of objects $\ob{\ms{A}}$, and
\item for each pair of objects $X,Y\in\ob{\ms{A}}$ the set
  $\hom_{\ms{A}^{op}}(A,B)=\hom_{\ms{A}}(B,A)$.
\end{enumerate}
All the structure and properties of a dual category are inherited
from a category in the obvious way (loosely, ``by reversing the arrows'').
\end{defn}

\noindent Usually, we use the prefix \emph{co-} for dual objects, hence why
we use codomain --- when we reverse the direction of $f$, we get
\begin{equation}%\label{eq:}
f^{\text{op}}:y\to x
\end{equation}
where $f^{\text{op}}$ is the dual to $f$, and $y$ is the domain
of the dual to $f$ \emph{or the ``codomain''}.


\begin{rmk}
Observe that the dual some property $\mathcal{P}^{\text{op}}$ is
$\mathcal{P}$ (reversing the direction of reversed arrows returns
the original direction, kind of like squaring -1 yields 1).
\end{rmk}

We will give the general procedure for finding the dual of a
property about objects $X$ in $\ms{A}$:

\begin{ex}
Consider the property of objects $X$ in $\ms{A}$:
\begin{align*}
\mathcal{P}_{\ms{A}}(X) \equiv&\text{\emph{ For any }}\ms{A}\text{\emph{-object }}
A\text{\emph{ there exists is exactly one}}\\
 &\ms{A}\text{\emph{-morphism }}f:A\to X
\end{align*}
Step 1: In $\mathcal{P}_{\ms{A}}(X)$, replace all occurrences of
$\ms{A}$ by $\ms{A}^{op}$, thus yielding the property
\begin{align*}
\mathcal{P}_{\ms{A}^{op}}(X) \equiv&\text{\emph{ For any }}\ms{A}^{op}\text{\emph{-object }}
A\text{\emph{ there exists is exactly one}}\\
 &\ms{A}^{op}\text{\emph{-morphism }}f:A\to X
\end{align*}
Step 2: ``Translate it into the logically equivalent statment.''
That is translate it into the equivalent statement, translating
the dual category into the original category, dual morphisms into
the original ones, etc.:
\begin{align*}
\mathcal{P}^{op}_{\ms{A}}(X) \equiv&\text{\emph{ For any }}\ms{A}\text{\emph{-object }}
A\text{\emph{ there exists is exactly one}}\\
 &\ms{A}\text{\emph{-morphism }}f:X\to A.
\end{align*}
\end{ex}

Similarly, there is a procedure for finding the dual for a
property about morphisms:
\begin{ex}
Consider the property for a morphism $X\xrightarrow{\;\;f\;\;}Y$ in $\ms{A}$
\begin{align*}
\mathcal{Q}_{\ms{A}}(f) \equiv&\text{\emph{There exists an
}}\ms{A}\text{\emph{-morphism
}}Y\xrightarrow{\;\;g\;\;}X\\
&\text{\emph{ with }}X\xrightarrow{\;\;f\;\;}Y\xrightarrow{\;\;g\;\;}X=X\xrightarrow{\;\;id\;\;}X
\end{align*}
Step 1: In $\mathcal{P}_{\ms{A}}(X)$, replace all occurrences of
$\ms{A}$ by $\ms{A}^{op}$, thus yielding the property
\begin{align*}
\mathcal{Q}_{\ms{A}^{op}}(f) \equiv&\text{\emph{There exists an
}}\ms{A}^{op}\text{\emph{-morphism
}}Y\xrightarrow{\;\;g\;\;}X\\
&\text{\emph{ with }}X\xrightarrow{\;\;f\;\;}Y\xrightarrow{\;\;g\;\;}X=X\xrightarrow{\;\;id\;\;}X
\end{align*}
Step 2: Translate $\mathcal{Q}_{\ms{A}^{op}}(f)$ into the
logically equivalent statement:
\begin{align*}
\mathcal{ Q}_{\ms{A}}^{op}(f) \equiv&\text{\emph{There exists an
}}\ms{A}\text{\emph{-morphism
}}X\xrightarrow{\;\;g\;\;}Y\\
&\text{\emph{ with }}Y\xrightarrow{\;\;g\;\;}X\xrightarrow{\;\;f\;\;}Y=Y\xrightarrow{\;\;id\;\;}Y
\end{align*}
\end{ex}

We now introduce one of the most foundational concepts in
category theory:

\begin{framed}
\begin{duality-principle}\addcontentsline{toc}{section}{*** Important Concept: Duality Principle}
\label{dualityPrinciple}
Whenever a property $\mathcal{P}$ holds for all categories, then
the property $\mathcal{P}^{op}$ holds for all categories.
\end{duality-principle}
\end{framed}

We can observe several properties,
\begin{enumerate}
\item $(\ms{A}^{op})^{op}=\ms{A}$, and
\item $\mathcal{P}^{op}(\ms{A})$ holds iff $\mathcal{P}(\ms{A}^{op})$ holds.
\end{enumerate}
Further, we say a property $\mathcal{P}$ is \define{Self-Dual} if
$\mathcal{P}^{op}=\mathcal{P}$.
%% \begin{framed}
%% \begin{duality-principle}\addcontentsline{toc}{mysubsection}{*** Important Concept: Duality Principle}
%% \label{dualityPrinciple}
%% Given a category $\ms{C}$ with some property $\mathcal{P}$, 
%% we can find its dual (denoted by $\ms{C}^{\text{op}}$) by
%% simply reversing the direction of its morphisms. The dual
%% category has the dual property $\mathcal{P}^{\text{op}}$.
%% \end{duality-principle}
%% \end{framed}
