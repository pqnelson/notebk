%%
%% quaternionGroupEx.tex
%% 
%% Made by Alex Nelson
%% Login   <alex@tomato>
%% 
%% Started on  Tue Jun 30 23:51:16 2009 Alex Nelson
%% Last update Tue Jun 30 23:51:16 2009 Alex Nelson
%%
Consider the quaternions, which were characterized by the
equation
\begin{equation}%\label{eq:}
i^2 = j^2 = k^2 = ijk = -1
\end{equation}
We have the multiplication table
\begin{equation}%\label{eq:}
\begin{array}{c|cccccccc}
\times &  1  & -1  &  i  & -i  &  j  & -j  &  k  &  -k\\\hline
1      &  1  & -1  &  i  & -i  &  j  & -j  &  k  &  -k\\
-1     & -1  &  1  & -i  &  i  & -j  &  j  & -k  &   k\\
i      &  i  & -i  & -1  &  1  &  k  & -k  & -j  &   j\\
-i     & -i  &  i  &  1  & -1  & -k  &  k  &  j  &  -j\\
j      &  j  & -j  & -k  &  k  & -1  &  1  &  i  &  -i\\
-j     & -j  &  j  &  k  & -k  &  1  & -1  & -i  &   i\\
k      &  k  & -k  &  j  & -j  & -i  &  i  & -1  &   1\\
-k     & -k  &  k  & -j  &  j  &  i  & -i  &  1  &  -1
\end{array}
\end{equation}
We have various ways to encode the information in the quaternions
into diagrams, e.g.
%% \begin{equation}%\label{eq:}
%% jk=i\quad\iff\quad
%% \begindc{0}[5]
%% \obj(0,10){$*$}
%% \obj(0,0){$*$}
%% \obj(10,0){$*$}
%% \mor(0,10)(0,0){\scriptsize{$j$}}
%% \mor(0,0)(10,0){\scriptsize{$k$}}
%% \mor(0,10)(10,0){\scriptsize{$i$}}
%% \enddc
%% \end{equation}
\begin{equation}%\label{eq:}
jk=i\quad\iff\quad
\vcenter{
\xymatrix{
\txt{*}\ar[d]_{j}\ar[dr]^{i} & \\
\txt{*}\ar[r]_{k} & \txt{*}
}}
\end{equation}
We can cycle through the rest of the identities, or we can skip
ahead to the final diagram
\begin{equation}
\vcenter{
\xymatrix{            &                              & \\
\txt{*} \ar@/^2pc/[uurrdd]^{-1}
\ar[d]_{j} 
\ar[dr]_{k} 
\ar[r]^{i}       & \txt{*} \ar[r]^{i} \ar[d]_{j} & \txt{*}  \\
\txt{*}\ar@/_3pc/[dddrruuuu]_{j}             & \txt{*}\ar[l]^{i}   \ar[ur]_{k}          &
  }}
\end{equation}
%% \begin{equation}%\label{eq:}
%% \begindc{0}[5]
%% \obj(0,20){$*$}
%% \obj(10,20){$*$}
%% \obj(20,20){$*$}
%% \obj(10,10){$*$}
%% \obj(0,10){$*$}
%% \mor(0,20)(10,20){\scriptsize{$i$}}
%% \mor(10,20)(20,20){\scriptsize{$i$}}
%% \mor(0,20)(0,10){\scriptsize{$j$}}
%% \mor(0,20)(10,10){\scriptsize{$k$}}
%% \mor(10,10)(0,10){\scriptsize{$i$}}
%% \mor(10,20)(10,10){\scriptsize{$j$}}
%% \mor(10,10)(20,20){\scriptsize{$k$}}
%% \cmor((0,22)(3,25)(10,26)(17,25)(20,22))
%%   \pdown(10,27){\scriptsize{$-1$}}
%% \cmor((0,9)(3,5)(10,5)(16,9)(20,18))
%%   \pup(6,3){\scriptsize{$j$}}
%% \enddc
%% \end{equation}
With the understanding that $-1\circ-1=1\equiv\id{*}$.
%% \begin{equation}%\label{eq:}
%% \begindc{0}[5]
%% \obj(0,20){$*$}
%% \obj(10,20){$*$}
%% \obj(0,10){$*$}
%% \obj(10,10){$*$}
%% \obj(0,0){$*$}
%% \mor(0,20)(0,10){\scriptsize{$i$}}
%% \mor(0,20)(10,20){\scriptsize{$j$}}
%% \mor(0,20)(10,10){\scriptsize{$k$}}
%% \mor(10,10)(10,20){\scriptsize{$i$}}
%% \mor(0,10)(0,0){\scriptsize{$i$}}
%% \mor(10,10)(0,0){\scriptsize{$k$}}
%% \mor(0,10)(10,10){\scriptsize{$j$}}
%% \enddc
%% \end{equation}
