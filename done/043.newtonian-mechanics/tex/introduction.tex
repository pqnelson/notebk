\section{Big Picture}

\N{Goal of Mechanics}
We model physical bodies and try to predict their motion. This is the
entire goal of Newtonian mechanics (and classical mechanics more
broadly).

\N{Physical Body}
We model a physical body as a point-particle. That is to say, at any
given moment, the location of a physical body is modeled as a point. The
motion of a physical body is modeled as a [suitably smooth] curve $\gamma\colon I\to\RR^{3}$
where $I$ is some interval of ``time''. This works well as an
approximation when we use the center-of-mass as the location of the
body.

In short, using a point-particle as an idealized approximation of a
physical body is ``good enough'' for most purposes.

\begin{remark}[Extended bodies, rigidity condition]
There are also extended bodies. For classical mechanics, this is a rigid
body. We model a rigid body by a set of trajectories such that the
distance between any two points in the body is fixed. For example, a
coin is modeled as a family of trajectories $\gamma(t;\vec{x}_{0})$ with
initial position $\gamma(0;\vec{x}_{0})=\vec{x}_{0}$ such that the
coin's position at time $t$ is the region
\begin{equation}
\mathcal{C}_{t} = \{\gamma(t;\vec{x})\mid\vec{x}\in\mathcal{C}_{0}\}.
\end{equation}
The ``rigidity condition'' is: for any $\vec{x}_{1},\vec{x}_{2}\in\mathcal{C}_{0}$, we have
$\|\gamma(t;\vec{x}_{1})-\gamma(t;\vec{x}_{2})\|$ be a constant (``rigid'') with
respect to time. When this condition holds, we call an extended body a
``rigid body''.

For rigid bodies, we need to model rotational mechanics of the rigid
body in addition to the translational mechanics of the rigid body. Euler
studied this problem quite extensively. If rotational contributions to a
rigid body is negligible, then we may model it as a point-particle
located at its center of mass.
\end{remark}

\begin{remark}[Continuum mechanics]
If we remove the ``rigidity condition'' [i.e., we could allow the
distance between points in an extended body to change over time] for an
extended body, then we end up with continuum mechanics. This is an
extroardinarily difficult field. Fluid mechanics is a subfield of
continuum mechanics, and we don't know how to solve the equations of
motion in fluid mechanics.
\end{remark}

%% \N{Rigid bodies, continuum mechanics}
%% There are also extended bodies. For classical mechanics, this is a rigid
%% body. We model a rigid body by a set of trajectories such that the
%% distance between any two points in the body is fixed. For example, a
%% coin is modeled as a family of trajectories $\gamma(t;\vec{x}_{0})$ with
%% initial position $\gamma(0;\vec{x}_{0})=\vec{x}_{0}$ such that the
%% coin's position at time $t$ is the region
%% \begin{equation}
%% \mathcal{C}_{t} = \{\gamma(t;\vec{x})\mid\vec{x}\in\mathcal{C}_{0}\}.
%% \end{equation}
%% The ``rigidity condition'' is: for any $\vec{x}_{1},\vec{x}_{2}\in\mathcal{C}_{0}$, we have
%% $\|\gamma(t;\vec{x}_{1})-\gamma(t;\vec{x}_{2})\|$ be a constant (``rigid'') with
%% respect to time.

%% For rigid bodies, we need to model rotational mechanics of the rigid
%% body in addition to the translational mechanics of the rigid body. Euler
%% studied this problem quite extensively. If rotational contributions to a
%% rigid body is negligible, then we may model it as a point-particle
%% located at its center of mass. We can also weaken rigid bodies to allow
%% deformation, which is studied in the field of continuum mechanics.

\N{Equations of Motion}
Classical mechanics (and, more generally, physics) describes the motion
of bodies by means of the so-called \define{Equations of Motion} which
relates the acceleration in terms of velocities and positions. These are
second-order ordinary differential equations.

In Newtonian mechanics, the ``big insight'' is the idea that
\emph{forces} determine the ``right-hand side'' to these differential
equations. Undergraduates spend a lot of time setting up and then
solving these differential equations. But the difference in formalisms
in classical mechanics (Newtonian, Lagrangian, Hamiltonian, etc.) lies
in these two steps:
\begin{enumerate}[label=(\alph*)]
\item Setting up the equations of motion
\item Solving the equations of motion
\end{enumerate}

\M
The real skill in physics is not in solving these differential
equations, but in using physical principles [conservation laws or
  symmetries or whatever] to extract relevant and
useful information from them.
The different formalisms in classical mechanics allows us to look at a
physical system in different lighting.

\begin{remark}[``ISEE'']
If you are learning physics, I would highly encourage you to read Young
and Freedman's \emph{University Physics}. Any edition is fine (unless
you are taking a course at college or university which requires a
specified edition for its homework). The real usefulness lies in its
format for examples, which state a problem, the solves it in a template
consisting of the following steps:
\begin{description}
\item[Identify] the relevant concepts:
  \begin{itemize}
  \item Identify the physical conditions stated in the problem to help
    decide which physical concepts which are relevant
  \item Identify the target variable, the physical quantity (or quantities) you are trying to find.
  \item Identify the known quantities (stated or implied in the
    problem). (Sometimes this is done in the ``Setup'' step.)
  \end{itemize}
\item[Setup] the problem
  \begin{itemize}
  \item Given the concepts, known quantities, and target variables,
    which we have obtained from the \textsc{identify} step, now we
    choose the equations we'll use to solve the problem (and decide how
    we'll use them).
  \item Here you also choose coordinate systems, which is important and
    often overlooked.
  \item Draw sketches (if relevant) and free body diagrams (if relevant).
  \end{itemize}
\item[Execute] the solution --- that is, ``do the math'', solve the
  equations you have obtained from the \textsc{setup} step. Try to solve
  these equations \emph{algebraically} (``symbolically''), defer
  plugging in values as late as possible. This will help us in the next step
\item[Evaluate] your answer.
  \begin{itemize}
  \item We can try taking limits of the symbolic solution to recover
    previous solutions (e.g., taking the ``friction goes to zero'' limit
    should recover the frictionless solution) or special cases where we
    might divide by zero (e.g., if the answer looks like
    $f(x)/\cos(\theta)$, the $\theta=(n+1/2)\pi$ solutions lead to a
    divide-by-zero singularity resulting in an infinity)
  \item Check the order-of-magnitude of the solution, to see if it makes sense.
  \end{itemize}
\end{description}
This encourages the reader to keep a \textit{r\'{e}pertoire} of worked
problems, and nurtures a habit of ``successive refinement'' to add
layer-upon-layer of complexity and realism to a problem.
\end{remark}

\N{Observers and Reference Frames}
Before rushing along, we should discuss one subtlety to classical
mechanics, namely the idea of observers and reference frames. We can
model an observer as a physical body [i.e., a point-particle] equipped
with ``a ruler and a clock''. What does this even mean? It means we have
a set of basis spatial vectors for $\RR^{3}$ [``ruler''] and a basis for
time $\RR$ [``a clock'']. Together these are bundled together and
sometimes called a \emph{reference frame}.

The trajectories of physical bodies are expressed in coordinates
relative to an observer's reference frame.
