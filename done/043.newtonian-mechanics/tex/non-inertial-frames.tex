\section{Equations of Motion in Non-inertial Reference Frames}\label{section:eom-in-non-inertial-frame}

\M
For non-inertial reference frames, we start by considering the position
of the origin of the non-inertial frame relative to an inertial frame
\begin{equation}
  \vec{R}(t) = \begin{pmatrix}\mbox{Position of the origin for the}\\
    \mbox{non-inertial reference frame}\\
    \mbox{relative to an inertial observer}
\end{pmatrix}.
\end{equation}
We are trying to describe the trajectory of a body relative to a
noninertial reference frame $\vec{r}'(t)$ (primes indicate the quantity
is relative to the noninertial reference frame --- this is the physics
convention, not my own idiosyncratic notation) in terms of $\vec{R}(t)$
and the position of the body relative to the inertial observer
$\vec{r}(t)$. The picture we should have in mind (with subscripts
``inert'' and ``non'' to emphasize the different reference frames):
\begin{center}
  \includegraphics{img/img.0}
\end{center}
We can write this as
\begin{equation}
\vec{r}(t) = \vec{R}(t) + \vec{r}'(t).
\end{equation}
Note that $\vec{R}(t)$ is taken relative to the inertial reference frame.
Now we take derivatives of both sides to get velocities
\begin{equation}
\frac{\D\vec{r}(t)}{\D t} = \frac{\D\vec{R}(t)}{\D t} + \frac{\D\vec{r}'(t)}{\D t}
\end{equation}
and again for accelerations
\begin{equation}
\frac{\D^{2}\vec{r}(t)}{\D t^{2}} = \frac{\D^{2}\vec{R}(t)}{\D t^{2}} + \frac{\D^{2}\vec{r}'(t)}{\D t^{2}}.
\end{equation}
For accelerating reference frames, $\D^{2}\vec{R}/\D t^{2}\neq0$, and
for rotating reference frames things get more interesting.

\M
We need to start by writing the
velocity of the a body relative to an inertial frame
$\vec{v}_{i}=\D\vec{r}/\D t$ in
terms of the velocity of the body relative to a non-inertial frame
$\vec{v}_{n}$ (which is just the time derivative of the components of
the vector, since the basis vectors $\vec{e}'_{j}$ are constant relative to the
non-inertial observer),
\begin{align}
\vec{v}_{i} &=\frac{\D\vec{R}(t)}{\D t} + \frac{\D}{\D t}\left(r'_{1}(t)\vec{e}'_{1}(t) + r'_{2}(t)\vec{e}'_{2}(t) + r'_{3}(t)\vec{e}'_{3}(t)\right)\\
&=\frac{\D\vec{R}(t)}{\D t} + \frac{\D}{\D t}\sum^{3}_{j=1}r'_{j}(t)\vec{e}'_{j}(t)\\
&=\frac{\D\vec{R}(t)}{\D t} + \sum^{3}_{j=1}\frac{\D r'_{j}(t)}{\D t}\vec{e}'_{j}(t) +
r'_{j}(t)\frac{\D\vec{e}'_{j}(t)}{\D t}\\
&=\frac{\D\vec{R}(t)}{\D t} + \vec{v}_{n} + \sum^{3}_{j=1}r'_{j}(t)\frac{\D\vec{e}'_{j}(t)}{\D t}
\end{align}
where $r'_{j}(t)$ are the coordinates for the position relative to the
non-inertial frame, $\vec{e}'_{j}(t)$ are the unit vectors in the
non-inertial reference frame which vary over time (by, e.g., rotating); the
$\vec{v}_{n}$ is the velocity of the body as viewed by the non-inertial observer.
Acceleration relative to an inertial reference frame is obtained by
taking the time derivative with respect to both sides.
\begin{align}
  \frac{\D}{\D t}\vec{v}_{i}
  &= \frac{\D^{2}\vec{R}(t)}{\D t^{2}} + \frac{\D}{\D t}\sum^{3}_{j=1}\frac{\D r'_{j}(t)}{\D t}\vec{e}'_{j}(t) +
  r'_{j}(t)\frac{\D\vec{e}'_{j}(t)}{\D t}\\
  &= \frac{\D^{2}\vec{R}(t)}{\D t^{2}} + \sum^{3}_{j=1}\frac{\D}{\D t}\left(\frac{\D r'_{j}(t)}{\D t}\vec{e}'_{j}(t)\right) +
  \frac{\D}{\D t}\left(r'_{j}(t)\frac{\D\vec{e}'_{j}(t)}{\D t}\right)\\
  &= \frac{\D^{2}\vec{R}(t)}{\D t^{2}} + \sum^{3}_{j=1}\left(\frac{\D^{2} r'_{j}(t)}{\D t^{2}}\vec{e}'_{j}(t) +
  \frac{\D r'_{j}(t)}{\D t}\frac{\D\vec{e}'_{j}(t)}{\D t}\right) +
  \left(\frac{\D r'_{j}(t)}{\D t}\frac{\D\vec{e}'_{j}(t)}{\D t}
  +r'_{j}(t)\frac{\D^{2}\vec{e}'_{j}(t)}{\D t^{2}}\right)\\
&= \frac{\D^{2}\vec{R}(t)}{\D t^{2}} + \vec{a}_{n} + \sum^{3}_{j=1}2\frac{\D r'_{j}(t)}{\D t}\frac{\D\vec{e}'_{j}(t)}{\D t} +r'_{j}(t)\frac{\D^{2}\vec{e}'_{j}(t)}{\D t^{2}}
\end{align}
where $\vec{a}_{n}$ is the acceleration relative to the non-inertial frame.

\N{Rewriting the equations of motion}
Newton's second Law for a body with constant mass is then,
\begin{equation}
\vec{F} = m\frac{\D}{\D t}\vec{v}_{i}
\end{equation}
For a non-inertial observer, we just rewrite $\D\vec{v}_{i}/\D t$ by the
complicated mess
\begin{equation}
\vec{F} = m\vec{a}_{n} + m\left(\frac{\D^{2}\vec{R}(t)}{\D t^{2}} + \sum^{3}_{j=1}2\frac{\D r'_{j}(t)}{\D t}\frac{\D\vec{e}'_{j}(t)}{\D t} +r'_{j}(t)\frac{\D^{2}\vec{e}'_{j}(t)}{\D t^{2}}\right).
\end{equation}
The ``extra terms'' on the right may be interpreted as ``pseudoforces'',
and moved to the left-hand side as:
\begin{equation}
\vec{F}_{\text{pseudo}} = -m\left(\frac{\D^{2}\vec{R}(t)}{\D t^{2}} + \sum^{3}_{j=1}2\frac{\D r'_{j}(t)}{\D t}\frac{\D\vec{e}'_{j}(t)}{\D t} +r'_{j}(t)\frac{\D^{2}\vec{e}'_{j}(t)}{\D t^{2}}\right),
\end{equation}
so Newton's laws of motion for a non-inertial frame is,
\begin{equation}
\vec{F} + \vec{F}_{\text{pseudo}} = m\vec{a}_{n}.
\end{equation}
As a consistency check, we see when we make $\vec{e}'_{j}(t)$ constants
and $\vec{R}(t)$ constant, we recover Newton's laws for an inertial observer.

If you want to account for the rotation of the Earth, or the
acceleration of the universe, then we can do it. The contributions may
be small depending on what we want to do. For example, modeling a
planet's atmosphere as a fluid surrounding a rotating sphere requires
accounting for pseudoforces from the planet's rotation. But the
contribution from the acceleration of the universe is rather negligible.
