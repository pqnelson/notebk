%%
%% lecture03.tex
%% 
%% Made by alex
%% Login   <alex@tomato>
%% 
%% Started on  Sat Oct  1 13:34:38 2011 alex
%% Last update Sat Oct  1 13:34:38 2011 alex
%%
We will discuss the scheme of two proofs of existence. Given some
$\mathcal{U}\propersubset\RR^{2}$ that is simply connected,
$\mathcal{U}\not=\RR^{2}$ and $\mathcal{U}\not=\emptyset$, then
there exists some function
\begin{equation}
f\colon\mathcal{U}\xrightarrow{\iso}D
\end{equation}
is conformal (where $D$ is the open unit disc in $\CC$), or
holomorphic with $f'$ nonvanishing. We fix some
$a\in\mathcal{U}$.

Consider the set of maps 
\begin{equation}
S=\{f\colon\mathcal{U}\to D\mbox{ is conformal}\lst f(a)=0,
f'(a)>0\}
\end{equation}
First we see that $S$ is nonempty.

We suppose that $0\notin\mathcal{U}$. Since $\mathcal{U}$ is
simply connected, for each circle around $o$ we have some point
on the circle not in $\mathcal{U}$ since $\mathcal{U}$ is simply
connected we can contract it.

We consider a branch of $\sqrt{-}\colon\mathcal{U}\to\RR^{2}$,
namely we see that $\sqrt{-}(\mathcal{U})$ contains some disc. If
$\omega\in\sqrt{-}(\mathcal{U})$, then
$-\omega\notin\sqrt{-}(\mathcal{U})$. 

For every $f\in S$, $f'(a)\in\RR_{+}$ is bounded. There exists
some $M>0$ such that $f(a)<M$ for all $f\in S$ (i.e., it's
bounded).

Now we use a standard trick in analysis. Let 
\begin{equation}
M=\sup\{f'(0)\lst f\in S\}
\end{equation}
If we have some sequence $\{f_{i}\}\propersubset S$, we can find
some convergent subsequence. We wish to deduce that $\exists f\in
S$ such that $f'(a)=M$. This is our $f$ that maps $\mathcal{U}$
to $D$ injectively.

Suppose that $B\propersubset D$, $0\in B$ and $B\not=D$ is simply
connected. We claim that there is a conformal map $g\colon B\to
D$, where $f(A)=B$. Then $g\circ f\colon A\to D$ is itself
conformal. 

Suppose we have a domain $\mathcal{U}$ which we do not demand to
be simply connected. For every continuous function on
$\partial\mathcal{U}$, there is an analytic function in
$\mathcal{U}$; i.e.,
\begin{equation}
\begin{split}
&\forall h\colon\partial\mathcal{U}\to\RR\\
&\exists u\colon\bar{\mathcal{U}}\to\RR\quad\mbox{such that}\quad
h=\left.u\right|_{\partial\mathcal{U}}
\end{split}
\end{equation}
We have
\begin{equation}
h(a)=\ln|z-a|.
\end{equation}
Now this $u$ may be supplemented with $v$ such that $u+\I v$ is
holomorphic. Then we take 
\begin{equation}
f(z)=(z-a)\exp\big(-(u+\I v)\big)
\end{equation}
Our claim is that this is our function, it bijectively maps
$\mathcal{U}$ to $D$.

If $z\in\partial\mathcal{U}$, then $\|f(z)\|=1$. We see this by
direct computation:
\begin{subequations}
\begin{align}
\|f(z)\| &= \|z-a\|\cdot\|e^{-u(a)}\|\\
&=\frac{\|z-a\|}{\|z-a\|}\\
&=1
\end{align}
\end{subequations}
The only thing that requires work is that $f'(z)>0$, or at least
nonzero. We cannot use any information of the boundary of
$\mathcal{U}$. 

Is it possible to extend the Riemann mapping theorem from
$\mathcal{U}\xrightarrow{\iso}D$ to
$\bar{\mathcal{U}}\xrightarrow{\iso}\bar{D}$? Not necessarily. It
is possible if $\partial\mathcal{U}$ is a continuous closed
curved.

If $\mathcal{U}$, $\mathcal{V}$ are simply connected and bounded
domains, then what? Well, suppose we have a conformal map
\begin{equation}
f\colon\bar{\mathcal{U}}\to\RR^{2}
\end{equation}
such that $\left.f\right|_{\mathcal{U}}$ is conformal, and
additionally that $f(\partial\mathcal{U})=\partial\mathcal{V}$.
Then
\begin{equation}
f\colon\mathcal{U}\to\mathcal{V}.
\end{equation}
It is sufficient to let $\mathcal{V}=D$, then
\begin{equation}
f\colon\bar{\mathcal{U}}\to D
\end{equation}
and $z\in\bar{\mathcal{U}}$ implies $\|f(z)\|\leq1$.

Let us consider conformal maps between $D$ and the upper half
plane. Consider
\begin{equation}
z\mapsto\I\left(\frac{-z+1}{z+1}\right)
\end{equation}
which maps $1\mapsto 0$, $-1\mapsto\infty$, $\I\mapsto-1$, and
indeed we can note
\begin{equation}
\I\left(\frac{-z+1}{z+1}\right)=\I\frac{1-\|z\|^{2}}{\|1+z\|^{2}}
\end{equation}
