%%
%% lecture20.tex
%% 
%% Made by alex
%% Login   <alex@tomato>
%% 
%% Started on  Wed Oct  5 11:21:19 2011 alex
%% Last update Wed Oct  5 11:21:19 2011 alex
%%
\begin{rmk}[On Midterm]
If we wish to map the upper half of $\CC$ to the lower half of
$\CC$, we use a fractional linear transformation with real
coefficients. Also note that the product cannot be replaced by
expanding into pairwise expansion, that is
\begin{equation}
\prod\left(1+\frac{(-1)^{n}}{n}\right)\not=\prod\left(1-1\frac{1}{n(n+1)}\right)
\end{equation}
otherwise by this reasoning
\begin{equation}
\sum(-1)^{n}=\sum(1-1)=0
\end{equation}
is correct, which it most certainly is not.
\end{rmk}
So what do we know about the Laplace transform? It's something
that looks like
\begin{equation}
\widetilde{f}(z)=\int^{\infty}_{0}\E^{-zt}f(t)\D t
\end{equation}
What else can we say? First we fix notation, we will write
\begin{equation}
\laplace{f(t)}(z)=\widetilde{f}(z)=\int^{\infty}_{0}\E^{-zt}f(t)\D t.
\end{equation}
Let us list out the properties of the Laplace transform:
\begin{enumerate}
\item If $f$ is differentiable and its derivative is not growing
  too fast, then $\widetilde{(f')}(z)=z\widetilde{f}(z)-f(0)$.
\item We see that $\widetilde{(tf(t))}(z)=-(\widetilde{f})'(z)$.
\item A shift corresponds to multiplication by the exponential,
  $\laplace{\E^{-at}f(t)}(z)=\widetilde{f}(z+a)$ where $a\in\CC$
  is arbitrary.
\item We consider for $t\geq a$
  $\laplace{f(t-a)}(z)=\E^{-az}\widetilde{f}(z)$.
\item We have this strange correspondence between convolution and
  the product
  $\laplace{f*g}(z)=\widetilde{f}(z)\widetilde{g}(z)$.
\item Consider a specific example for $a>-1$:
  $\laplace{t^{a}}=\Gamma(a+1)/z^{a+1}$. 
\item An example $a\in\CC$ is arbitrary, then
  $\laplace{\E^{-at}}(z)=1/(z+a)$. 
\item For cosine we have $\laplace{\cos(at)}(z)=z/(z^{2}+a^{2})$.
\item For sine we have $\laplace{\sin(at)}(z)=a/(z^{2}+a^{2})$.
\end{enumerate}

\subsection{Some Proofs}
Let us consider some proofs of the properties just asserted.
\begin{prop}
For $a>-1$ we have
$\laplace{t^{a}}=\Gamma(a+1)/z^{a+1}$. 
\end{prop}
\begin{proof}
We see that
\begin{equation}
\laplace{t^{a}}=\int^{\infty}_{0}\E^{-zt}t^{a}\D t
\end{equation}
we let $u=zt$, so $t=u/z$ and $\D t=\D u/z$ then
\begin{equation}
\begin{split}
\int^{\infty}_{0}\E^{-zt}t^{a}\D t
&=\int^{\infty}_{0}\E^{-u}\left(\frac{u}{z}\right)^{a}\left(\frac{\D u}{z}\right)\\
&=\frac{1}{z^{a+1}}\Gamma(a+1)
\end{split}
\end{equation}
where we have used the definition of the $\Gamma$ function, specifically Eq \eqref{eq:lecture12:defnOfGammaFn}.
\end{proof}
\begin{ex}
We see $\laplace{1}=z^{-1}$, $\laplace{t}=z^{-2}$, etc. Consider
\begin{equation}
f(t)=f(0)+f'(0)t+\frac{1}{2}f''(0)t^{2}+\dots
\end{equation}
then
\begin{equation}
\laplace{f}(z)=\frac{f(0)}{z}+\frac{f'(0)}{z^{2}}+\dots+\frac{f^{(n)}(0)}{z^{n+1}}+\dots
\end{equation}
This is a bit of a joke, the approach is unreliable. The result
is correct however.
\end{ex}
\begin{prop}[Transform of Cosine]
For some $a\in\CC$, we have
\begin{equation}
\laplace{\cos(at)}(z) = \frac{z}{z^{2}+a^{2}}
\end{equation}
\end{prop}
\begin{proof}
By direct computation we see
\begin{equation}
\laplace{\cos(at)}{z}=\laplace{\frac{\E^{\I at}+\E^{-\I at}}{2}}(z)
\end{equation}
and by linearity we obtain
\begin{equation}
\laplace{\frac{\E^{\I at}+\E^{-\I at}}{2}}(z)=\frac{1}{2}\laplace{\E^{\I at}}{z}+\frac{1}{2}\laplace{\E^{-\I at}}(z).
\end{equation}
We use the result from computing $\laplace{\E^{-at}1}$ to find
\begin{equation}
\frac{1}{2}\laplace{\E^{\I at}}{z}+\frac{1}{2}\laplace{\E^{-\I at}}(z)=\frac{1}{2}\left(\frac{1}{z-\I{}a}-\frac{1}{z+\I{}a}\right)
\end{equation}
and by gathering terms we have
\begin{equation}
\frac{1}{2}\left(\frac{1}{z-\I a}-\frac{1}{z+\I a}\right)=\frac{1}{2}\left(\frac{2z}{z^{2}+a^{2}}\right)
\end{equation}
which proves the theorem.
\end{proof}
\begin{prop}[Transform of Sine]
For some $a\in\CC$, we have
\begin{equation}
\laplace{\sin(at)}(z) = \frac{a}{z^{2}+a^{2}}
\end{equation}
\end{prop}
\begin{proof}
We may use the fact
\begin{equation}
\frac{1}{a}\frac{\D}{\D z}\sin(at)=\cos(at)
\end{equation}
take the Laplace transform of both sides, and we immediately get
the result.
\end{proof}

\subsection{Applications}
It is tempting by the first property to apply this to
differential equations, we'd end up with the answer but Laplace
transformed. We'd need to inverse the transform to get the
answer. 

Let us first consider a very simply thing. If we have, e.g., 
\begin{equation}
\widetilde{f}(z)=\frac{1}{z-a},
\end{equation}
what would we expect to find for $f(t)$? We expect $f(t)=\exp(at)$.

If we instead have now
\begin{equation}
\widetilde{f}(z)=\frac{1}{(z-1)(z-2)}
\end{equation}
we'd very much like to use partial fraction decomposition
\begin{equation}
\widetilde{f}(z)=\frac{1}{z-2}-\frac{1}{z-1}
\end{equation}
to obtain the result
\begin{equation}
f(t)=e^{2t}-e^{t}.
\end{equation}
More generally, if we have a rational function
\begin{equation}
\widetilde{f}(z)=\frac{P(z)}{Q(z)}
\end{equation}
where
\begin{equation}
\deg(P)<\deg(Q)
\end{equation}
and we demand that all roots of $Q$ are different (so $Q$ has
simple roots). Let $z_{1}$, \dots, $z_{n}$ be the roots of
$Q$. We write
\begin{equation}
f(t) = \sum^{m}_{k=1}\E^{z_{i}t}\left(\frac{P(z_{i})}{Q'(z_{i})}\right).
\end{equation}
Let $F$ be some function, $F=\widetilde{f}$ or in other words $F$
is the function we will try to ``invert Laplace transform''; let 
\begin{equation}
f(t)=\sum\mbox{residues}(\E^{zt}F(z))
\end{equation}
But the sum of residues is an integral! We see that
\begin{equation}
f(t)=\frac{1}{2\pi\I}\int^{a+\I\infty}_{a-\I\infty}\E^{zt}F(z)\D z
\end{equation}
All the poles should be on the left of $a$, and there should be
only finitely many poles.
