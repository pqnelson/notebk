%%
%% lecture24.tex
%% 
%% Made by alex
%% Login   <alex@tomato>
%% 
%% Started on  Thu Oct  6 14:39:19 2011 alex
%% Last update Thu Oct  6 14:39:19 2011 alex
%%
We will cover multiple complex variables, and analytic
continuation. First the last.

\begin{wrapfigure}{l}{1.05in}
\vspace{-20pt}
\begin{center}
 \includegraphics{img/lecture24.0}
\end{center}
\vspace{-20pt}
\end{wrapfigure}
We have some boundary, and a function defined inside the
region. When can it not be extended beyond the region. Well,
consider one point, what function cannot be extended to $z_{1}$?
Well, $1/(z-z_{1})$. For $n$ points $z_{1}$, \dots, $z_{n}$ we
could have
\begin{equation}
f(z)=\sum_{k=1}^{n}\frac{1}{z-z_{k}}
\end{equation}
We could replace it by an integral on the boundary, but that'd be
hard. Why not take an infinite sum? Well, why not work with a
dense (countable) set of points on the boundary? This approach is
better, since it's countable. We take $\{z_{k}\}$ to be dense in
the boundary of the region. We can write out the sum
\begin{equation}
f(z)=\sum^{\infty}_{k=1}\frac{1}{z-z_{k}}
\end{equation}
but it won't converge. We then generalize this to
\begin{equation}
f(z)=\sum^{\infty}_{k=1}\frac{a_{k}}{z-z_{k}}
\end{equation}
where $\{a_{k}\}$ is a sequence rapidly decreasing. Although it'd
converge on the interior of the region, there is no guarantee for
convergence on the boundary.

We then specify
\begin{equation}
a_{n}=2^{-n}\min_{m<n}(1,\|z_{m}-z_{n}\|)
\end{equation}
and so on the boundary we have 1 singular point and a sum
\begin{equation}
\sum^{\infty}_{k\not=n}\frac{a_{k}}{z-z_{k}}<\sum^{\infty}_{k\not=n}2^{-k}
\end{equation}
which converges.
