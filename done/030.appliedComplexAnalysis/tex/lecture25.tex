%%
%% lecture25.tex
%% 
%% Made by alex
%% Login   <alex@tomato>
%% 
%% Started on  Thu Oct  6 15:12:43 2011 alex
%% Last update Thu Oct  6 15:12:43 2011 alex
%%

Suppose we have a region $\Gamma$ in $\CC^{2}$ which we want a holomorphic
function singular at $(z_{0},\omega_{0})\in\Gamma$. Suppose we
have it be
\begin{equation}
f(z,\omega)=\frac{1}{z+\omega-z_{0}-\omega_{0}}
\end{equation}
This is singular at
\begin{equation}
\begin{split}
z &= a + z_{0}\\
\omega &=-a+\omega_{0}
\end{split}
\end{equation}
for some $a\in\CC$. This is a complex line (since $a$ varies over
all of $\CC$). Suppose we want it to not pierce the region\dots
then we demand  the region must be convex for this to be true.

The hyperplane tangent to the point $(z_{1},\dots,z_{n})$ is
defined by the equation
\begin{equation}
a_{1}z_{1}+\dots+a_{n}z_{n}+b=0
\end{equation}
So back to the original problem in $\CC^{2}$, we write
$(a_{1}z+a_{2}\omega+b)^{-1}$ is regular in the domain and it has
a pole on the surface. We do what we did last time: take a dense
selection of such points, and so on. We can weaken the condition
of convexity to \emph{pseudoconvexity} (i.e., the tangent plane
may possibly intersect the domain).

Consider a surface described by
\begin{equation}
F(z_{1},\dots,z_{n})=0
\end{equation}
We can rewrite it as
\begin{equation}
F(x_{1},y_{1},\dots,x_{n},y_{n})=0.
\end{equation}
Let
\begin{equation}
\begin{split}
\partial_{k}=\frac{\partial}{\partial z_{k}}&=\left(\frac{\partial}{\partial x_{k}}\right)
+\I\left(\frac{\partial}{\partial y_{k}}\right)\\
\overline{\partial}_{k}=\frac{\partial}{\partial \overline{z}_{k}}&=\left(\frac{\partial}{\partial x_{k}}\right)
-\I\left(\frac{\partial}{\partial y_{k}}\right)
\end{split}
\end{equation}
For the second derivatives, we have
$[\partial_{i}\overline{\partial}_{j}F]$ be Hermitian, then the
domain is a \define{Holomorphic Domain}. They are very important!

In $\CC^{n}$ (for $n\geq2$) the region between two concentric
spheres \emph{is not} a holomorphic domain, we can extend it to
the center ball though. Recall that if $f$ is analytic inside a
domain with boundary $\gamma$, then
\begin{equation}
f(z_{0}=\frac{1}{2\pi\I}\int_{\gamma}\frac{f(\zeta)}{\zeta-z_{0}}\D\zeta.
\end{equation}
Suppose that $f$ were continuous on $\gamma$, then
\begin{equation}
g(z_{0})=\int_{\gamma}\frac{f(\zeta)}{\zeta-z_{0}}\D\zeta
\end{equation}
it is a differentiable function of $z_{0}$. Very briefly, the
idea is to take your domain, stack an infinite number of discs
there, and perform Cauchy integration on each disc. We get a
function in one variable, and this function turns out to tell us
that $f=g$ in the domain. So $f$ can be expanded. This approach
can be applied to many other domains.
