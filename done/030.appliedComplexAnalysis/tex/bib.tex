%%
%% bib.tex
%% 
%% Made by alex
%% Login   <alex@tomato>
%% 
%% Started on  Sat Oct  1 14:12:58 2011 alex
%% Last update Sat Oct  1 14:12:58 2011 alex
%%
\begin{thebibliography}{99}
\bibitem{conway1} John B.\ Conway,\newblock 
\emph{Functions of one complex variable}.\newblock 
Springer--Verlag, 1978.
\bibitem{conway2} John B.\ Conway, \newblock
\emph{Functions of one complex variable II}.\newblock
 Springer--Verlag, 1995.
\bibitem{lang} Serge Lang,\newblock
\emph{Complex Analysis}.\newblock
Springer--Verlag, Fourth edition, 1998.
\bibitem{marsden} Jerrold E.\ Marsden and Michael J.\ Hoffman,\newblock
\emph{Basic Complex Anlaysis}.\newblock
W. H. Freeman; Third Edition edition, 1999.
\bibitem{stein} Elias M. Stein, Rami Shakarchi, \newblock
\emph{Complex Analysis}.\newblock
Princeton University Press, 2003.
\bibitem{weyl} Hermann Weyl,\newblock
\emph{The Concept of a Riemann Surface}.\newblock
Dover Books on Mathematics, Third edition, 2009.
\end{thebibliography}
Note that \cite{marsden} was the ``official'' text for the course,
although we didn't touch it in math 185B. It seems like most of
the material was drawn from \cite{lang}, specifically chapters 7--16.

Some other references which may be useful:
\begin{enumerate}
\item Walter Rudin, \newblock
\emph{Real and Complex Analysis}.\newblock 
Third edition, Boston: McGraw Hill, 1987.
\end{enumerate}
The homological definition of integrals of the form
\begin{equation*}
\int g(x)\E^{f(x)}\D x
\end{equation*}
are discussed in
\begin{enumerate}\setcounter{enumi}{1}
\item Albert Schwarz, Ilya Shapiro,\newblock
``Twisted de Rham cohomology, homological definition of the
  integral and `Physics over a ring'\thinspace''.\newblock
Nucl.Phys.{\bf B809}:547--560,2009.
Eprint: \arXiv{0809.0086v1} \texttt{[math.AG]}
\end{enumerate}
