%%
%% lecture21.tex
%% 
%% Made by alex
%% Login   <alex@tomato>
%% 
%% Started on  Wed Oct  5 11:56:20 2011 alex
%% Last update Wed Oct  5 11:56:20 2011 alex
%%
Now, last time we covered the inverse Laplace transform. Let
$F(z)$ be analytic in $\CC$ with possibly only finitely many
poles. Then $F=\widetilde{f}$, and we obtain the original
function by
\begin{equation}
f(t)=\sum\left(\mbox{Residues of }\E^{tz}F(z)\right)
\end{equation}
The poles of this function comes entirely from $F(z)$ since
$\E^{tz}$ has no poles. We suppose that $F(z)$ is analytic on
$\CC$ except for a finite number of isolated singularities and
for some $\sigma\in\RR$ we have $F$ be analytic on the plane
$\{z\in\CC\lst\re(z)>\sigma\}$. 

The requirements: there are 3 positive constants $M$, $R$,
$\beta>0$ such that if $\|z\|>R$ then
\begin{equation}
\|F(z)\|<\frac{M}{\|z\|^{\beta}}=M(\|z\|^{-\beta})
\end{equation}
This is some contour integral with the requirement as
$\|z\|\to\infty$, then on the boundary $\|F(z)\|$ is ``really
small''. 

\begin{wrapfigure}{l}{1.45in}
\vspace{-24pt}
\begin{center}
\includegraphics{img/lecture21.0}
\end{center}
\vspace{-36pt}
\end{wrapfigure}
What do we do? We create a rectangle $\Gamma$ which is big enough
to contain all the singularities of $F$. This is doodled to the
left, the $\times$ indicates singularities of $F$.
We break up $\Gamma$ into two bits $\gamma$ which contains all
the singularities and $\widetilde{\gamma}$ which is everything
else. 

We see since all the singularities live inside $\gamma$ that
\begin{equation}
\int_{\gamma}\E^{zt}F(z)\D{z}=2\pi\I f(t)
\end{equation}
How can we check that this is correct?

We take its Laplace transform
\begin{equation}
2\pi\I\widetilde{f}(z)=\int^{\infty}_{0}\E^{-zt}\left[\int_{\gamma}\E^{\zeta t}F(\zeta)\D\zeta\right]\D{t}
\end{equation}
and change the order of integration
\begin{equation*}
2\pi\I\widetilde{f}(z)=\int_{\gamma}\int^{\infty}_{0}\E^{-zt}\E^{\zeta t}F(\zeta)\D\zeta\D{t}.
\end{equation*}
This is a little bit sloppy, it is really
\begin{equation}
2\pi\I\widetilde{f}(z)=\lim_{r\to\infty}\int_{\gamma}\int^{r}_{0}\E^{-zt}\E^{\zeta t}F(\zeta)\D\zeta\D{t}
\end{equation}
We then evaluate the integral and we find
\begin{equation}
\begin{split}
2\pi\I\widetilde{f}(z)&=\lim_{r\to\infty}\int_{\gamma}\left(\E^{(\zeta-z)r}-1\right)\frac{F(\zeta)}{\zeta-z}\D\zeta\\
&=\int_{\gamma}F(\zeta)\left[\frac{-1}{\zeta-z}\right]\D\zeta
\end{split}
\end{equation}
We want to show that $F(z)=\widetilde{f}(z)$. 

We use the fact that
\begin{equation}
\int_{\gamma}(\dots)=\int_{\Gamma}(\dots)+\int_{\widetilde{\gamma}}(\dots)
\end{equation}
to deduce
\begin{subequations}
\begin{align}
-2\pi\I\widetilde{f}(z)
&=-\int_{\gamma}\frac{F(\omega)}{\omega-z}\D\omega\\
&=-\int_{\Gamma}\frac{F(\omega)}{\omega-z}\D\omega-\int_{\widetilde{\gamma}}\frac{F(\omega)}{\omega-z}\D\omega\\
&=-\int_{\Gamma}\frac{F(\omega)}{\omega-z}\D\omega-2\pi\I F(z)
\end{align}
\end{subequations}
and we see that
\begin{equation}
\int_{\Gamma}\frac{F(\omega)}{\omega-z}\D\omega\approx 0
\end{equation}
when $\|\omega-z\|\sim R$ and $R$ becomes huge, we basically
divide by ``infinity''. So we have $\widetilde{f}=F$.

\begin{rmk}
The derivative is convolution with the derivative of the delta
function. 
\end{rmk}


This theorem has many corollaries. We see
\begin{equation}
f(t)=\sum(\mbox{residues }\E^{tz}F(z))
\end{equation}
so we can write this as an integral (thanks to the Residue
theorem)
\begin{equation}
f(t)=\frac{1}{2\pi\I}\int^{a+\I\infty}_{a-\I\infty}\E^{zt}F(z)\D{z}
\end{equation}
This formula is very close to the Laplace transform, and we
derived various properties of the Laplace transform using only
integration by parts (which means the inverse transform has
analogous properties).
