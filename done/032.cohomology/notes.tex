%%
%% notes.tex
%% 
%% Made by Alex Nelson
%% Login   <alex@tomato3>
%% 
%% Started on  Sun Apr 10 12:01:27 2011 Alex Nelson
%% Last update Sun Aug 14 15:37:52 2011 Alex Nelson
%%
\documentclass{article}
\pdfminorversion=4
\pdfcompresslevel9
\pdfinfo{/CreationDate (D:20110410120127)}
\usepackage{xmpincl}
\usepackage{makeidx}
\includexmp{CC_Attribution-NonCommercial-NoDerivs_3.0_Unported}
\usepackage{wrapfig}
\usepackage{notebk}
\makeindex
\diagramstyle[nohug,labelstyle=\scriptstyle]

\advance\headsep by1pc

\makeatletter
\def\lecture{\@ifnextchar[{\@lectureWith}{\@lectureWithout}}
\def\@lectureWith[#1]{\bigbreak\refstepcounter{section}\renewcommand{\leftmark}{Lecture \thesection}
  \noindent{\addcontentsline{toc}{section}{Lecture \thesection: #1\@addpunct{.}}\sectionfont Lecture \thesection. #1\@addpunct{.}}\medbreak}
\def\@lectureWithout{\bigbreak\refstepcounter{section}\renewcommand{\leftmark}{Lecture \thesection}
  \noindent{\addcontentsline{toc}{section}{Lecture \thesection}\sectionfont Lecture \thesection.}\medbreak}
\makeatother

\renewenvironment{exercise}{\refstepcounter{exercise}\medbreak%
  \noindent\llap{\manual\char'170\rm\kern.15em}% triangle in margin
  \small{\textbf{EXERCISE \theexercise}}\\%
  \noindent}{}
\def\ansno#1:{\medbreak\noindent%
  %\hbox to\parindent
  {\bf Answer to #1:\enspace}\ignorespaces}
\renewcommand{\answer}{\par\medbreak % \answer simply
  \ansno\theexercise: \\% prints directly
  } % out when it's called 
%\renewcommand{\theequation}{\arabic{equation}}

\def\CC{\mathbb{C}}
\def\NN{\mathbb{N}}
\def\QQ{\mathbb{Q}}
\def\RR{\mathbb{R}}
\def\ZZ{\mathbb{Z}}


\def\id{\mathrm{id}}
\def\dihedral#1{D_{#1}}
\def\GL{\mathrm{GL}}
\def\Mat{\mathrm{Mat}}
\def\SL{\mathrm{SL}}
%\def\normalSubgroup{\triangleleft}%{\mathrel{\unlhd}}
\def\properNormalSubgroup{\trianglelefteq}%{\mathrel{\lhd}}
\def\normalSubgroup{\mathrel{\lhd}}

\def\powerset{\mathcal{P}}
\def\hom{\mathop{\rm Hom}\nolimits}
\def\tr{\mathop{\rm tr}\nolimits}

% calculus
\def\D{\mathrm{d}}
\def\E{\mathrm{e}}
\def\I{\mathrm{i}} % = \sqrt{-1}

\DeclareMathOperator{\diag}{diag}
\DeclareMathOperator{\Span}{span}

\let\oldvec\vec
\let\vec\boldsymbol
% Apparently, the ISO 80000-2:2013 suggests using $\mathsf{T}$ for
% matrix transpose, so who am I to argue?
\newcommand\transpose[1]{#1^{\mathsf{T}}}
\newcommand\mat[1]{\mathbf{#1}}

% https://tex.stackexchange.com/a/594688
\newcommand{\svdots}{%
  \vbox{\fontsize{\sf@size}{\sf@size pt}\linespread{0.3}\selectfont
    \kern0.2\baselineskip
    \hbox{.}\hbox{.}\hbox{.}%
    \kern0.1\baselineskip
  }%
}

% Thank you, Heiko Oberdiek! Comment on https://tex.stackexchange.com/q/183686
\newcommand*{\math@rotate}[3]{%
  % #1: rotation angle       % #2: math style       % #3: symbol
  \sbox0{$#2\vcenter{}$}% location of the math axis
  \rotatebox[y=\ht0]{90}{$#2#3\m@th$}%
}
\newcommand*{\vertneq}{%
  \mathrel{%
    \mathpalette{\math@rotate{90}}{\neq}%
  }%
}
\newcommand*{\verteq}{%
  \mathrel{%
    \mathpalette{\math@rotate{90}}{=}%
  }%
}

\def\poincareDual#1{\mathcal{D}#1}

\theoremstyle{definition}
\newtheorem{xca}{\llap{\manual\char'170\rm\kern.15em}Exercise}
\title{Algebraic Topology Notes}
\date{April 10, 2011}
\begin{document}
\maketitle
\tableofcontents
\vfill\eject

\part{$K$-Theory and Characteristic Classes}
\lecture[Fibre Bundles]
%%
%% lecture01.tex
%% 
%% Made by alex
%% Login   <alex@tomato>
%% 
%% Started on  Tue Feb 14 08:29:36 2012 alex
%% Last update Tue Feb 14 08:29:36 2012 alex
%%
\subsection{General Relativity's Importance in Physics}
The first 50 years after Einstein published his field equations,
physicists held one of two opinions:

1) It was a beautiful model for how physics ought to be.

2) It was largely irrelevant unless you specialize in it.

\noindent\ignorespaces%
Most people imagine it's a model for how physics ought to be,
unless gravity's emergent. The second view is more or less
disregarded. In high energy physics, the coupling constants
converge to the same value at high enough energies where gravity
is significant (perhaps it unifies with the other forces, and
perhaps that's why it is significant). Trivially, General
Relativity is useful in cosmology.

There exists a sizeable group of people in condensed matter physics
where analog models\footnote{For a review, see Barcelo \emph{et al.}~\cite{Barcelo:2005fc}.} are used; e.g., an event horizon for sound
as an analog to Black Holes. There are attempts to make
predictions for, e.g., quantum gravity (using analogs of Hawking
radiation, etc.).

In the next 5 years, there should be experimental evidence for
gravitational radiation. In 15 years there will be more sensitive
tests available. We can ask questions like ``Does $E/c^{2}$
contribute to mass?'' There are interesting anomalies, e.g.,
measurements\footnote{See, e.g., Gundlach's measurements~\cite{gundlach},
the CODATA 2002 recommended values~\cite{CODATA2002}}
 of Newton's constant $G$ differ from each other by
$10\sigma$ to $15\sigma$, the Pioneer satellite feels
accelerations that's still not accounted for~\cite{Nieto:2007ng}, the predicted
energy level for Dark Energy is off by 120 orders of magnitude~\cite{Martin:2012bt}.

\subsection{Geometry and Physics}
Lets recall Newton's second law 
\begin{equation}\label{eq:lec01:newtonSecondLaw}
F=ma.
\end{equation}
Initially there was some contraversy whether it's a ``true
natural law'' or just a definition of force 
(see Spivak~\cite{spivak:2010} for details). We understand the
mass on the right hand side of Newton's second law describes
\emph{inertial mass}, the body's resistance to acting forces. But
we may say a couple other things.

First, since Newton's second law involves only acceleration, we
work with the second time derivative of position. Higher order time
derivative models are unstable since they have energy unbounded
from below, as Ostrogradski\index{Ostrogradski Theorem} proved\footnote{See
  Woodard~\cite[\normalfont\S2]{Woodard:2006nt} for a review of
  Ostrogradski's theorem for classical mechanics.}. 

The second thing to say is that Newton divided the world in two:
the object we are examining, and the rest of the world affecting
it. But gravity is now an exception. We consider gravitational
force of a body with mass $M$ acting on another body (with mass
$m$) in Newton's second law \eqref{eq:lec01:newtonSecondLaw},
writing
\begin{equation}
F\eqdef\frac{GmM}{r^{2}}
\end{equation}
for the gravitational force, and we invoke the second law writing
\begin{equation}\label{eq:lec01:gravity:newtonSecondLaw}
\frac{GmM}{r^{2}}=ma.
\end{equation}
We observe \emph{mathematically} the masses $m$ cancels out on
both sides. We thus obtain
\begin{equation}
\frac{GM}{r^{2}}=a.
\end{equation}
That makes gravity different from everything else, in that it
makes gravity a \emph{theory of paths}.

Almost\marginpar{Equivalence Principle gives us geometry} automatically, this makes gravity a theory of
geometry. But first note that really we have the right hand side
of \eqref{eq:lec01:gravity:newtonSecondLaw} be
\begin{equation}
ma=m_{i}a
\end{equation}
where $m_{i}$ is the \emph{inertial mass}, whereas the
gravitational force the body with mass $m$ feels is
\begin{equation}
\vec{F}_{12}=\frac{GMm_{g}}{r^{2}}\widehat{e}_{r}
\end{equation}
where $\widehat{e}_{r}$ is the unit vector from the body $M$ to
the body $m$, $m_{g}$ is the \emph{(passive) gravitational
  mass}\footnote{Note that the way to think of ``gravitational
  mass'' is that it is the \emph{``charge'' gravity feels.}}
which experiences the gravitational force, and $M$
is the \emph{(active) gravitational mass} exerting the
gravitational force. We have two conceptually different masses:
the inertial mass $m_{i}$, and the gravitational mass
$m_{g}$. The basic ingredient for gravity is the idea 
\begin{equation}\label{eq:lec01:weakEquiv}
m_{i}=m_{g}
\end{equation}
called the \define{Principle of (Weak) Equivalence}\index{Weak Equivalence Principle}\index{Equivalence Principle!Weak}. 

There is a \emph{Strong Equivalence Principle}\index{Strong Equivalence Principle}\index{Equivalence Principle!Strong}. Using
Newton's third Law, we find
\begin{equation}
\vec{F}_{21}=-\vec{F}_{12}.
\end{equation}
The role of ``active'' and ``passive'' gravitational masses swap.
%If active and passive gravitational masses were not
Active and passive gravitational masses are ``equivalent'' in the sense
\begin{equation}
\frac{m^{(a)}}{m^{(p)}}=\frac{M^{(a)}}{M^{(p)}}
\end{equation}
where $m$, $M$ are gravitational masses and the superscript
indicates whether they are active or passive gravitational
masses. This can be checked by the Earth-Moon\index{Lunar Laser Ranging} system.
Since 1968, when NASA attached lasers and reflectors (i.e., three plane mirrors
meeting mutually at right angles) to the moon, we have timed the
delay of a laser pulse sent to the Moon. If the strong
equivalence principle didn't hold, we'd expect the
Earth--Moon system's center of mass would oscillate with the
Lunar period. But we have not observed this\footnote{See Williams, et al.,~\cite{Williams:2005rv} for more data on this. Will~\cite{Will:1998dx} has a more broad discussion of experimental foundations underlying the equivalence principle in its various forms.}.

Reiterating the main point: the equality
\begin{equation*}\tag{\ref*{eq:lec01:weakEquiv}}
m_{i}=m_{g}
\end{equation*}
is what makes the geometric picture possible. The present tests
(as of 2010) suggest they're equal to parts in $10^{12}$ or $10^{13}$.

But Galileo knew the equivalence principle, did he suspect the
geometrical aspects? No, because gravity determines acceleration,
and paths depend on initial velocity too. Gravity doesn't
determine paths in space, instead it determines \emph{paths in spacetime}.
We need to articulate our vocabulary regarding paths before we can
continue discussing gravity.

First a \define{Extremal Path} between two points is referred to
as a \define{Geodesic}. We will set up the framework to discuss
geodesics, then proceed to consider calculations.

\marginpar{Flat, Intrinsic, Extrinsic Geometries}We need to know a little about what it means for a space to be
``curved'', so we will first consider what it means for space to
be ``flat''. We consider it to be the usual Euclidean
geometry. There is an important distinction between ``intrinsic geometry''\index{Geometry!Intrinsic}\index{Intrinsic Geometry}
(the curvature of space in itself without reference to higher
dimensions, e.g., the sum of angles of a triangle on Earth
doesn't add up to $\pi$, but without reference to 3 dimensions)
and ``extrinsic geometry''\index{Geometry!Extrinsic} \index{Extrinsic Geometry}
(curvature as seen in higher dimensions). We mostly care about
intrinsic curvature with General Relativity. There are times
(e.g., in the ADM formalism) when extrinsic curvature is
important. \TODO{figure out some transition motivating arclength}

With paths, we really need an idea of distance. Recall for flat
space, the Pythogoras' theorem gives us a path's length (more
or less) as
\begin{equation}
s^{2}=x^{2}+y^{2}.
\end{equation}
If we knew infinitesimal distances, that's enough: we can
integrate to get the distance
\begin{equation}
s=\int\D s,
\end{equation}
where
\begin{equation}
(\D s)^{2}=(\D x)^{2}+(\D y)^{2}.
\end{equation}
\textbf{WARNING:} the notation used is $\D s^2=(\D s)^2$, which
may confuse neophytes. The distance between $(x_0,y_0)$ and
$(x_1,y_1)$ is determined by the variation
\begin{equation}
\delta\int^{(x_1,y_1)}_{(x_0,y_0)}\D s=0.
\end{equation}
Think of it like the Euler-Lagrange equation. We will now
consider some special cases.\marginpar{Geodesic Equation: Examples}

\begin{ex}[Sphere]
Recall a sphere is described by
\begin{equation}
x^2+y^2+z^2=R^2.
\end{equation}
The distance is
\begin{equation}
\D s^2=\D x^2 + \D y^2+\D z^2
\end{equation}
with a constraint. In spherical coordinates we have
\begin{equation}
\begin{split}
r^{2}&=x^2+y^2+z^2\\
&=R^2
\end{split}
\end{equation}
Thus $r$ is constant. By substitution, we find
\begin{equation}
\D s^2=\D r^2+r^2(\D\theta^2+\sin^{2}(\theta)\,\D\varphi^2)
\end{equation}
This is in a flat 3-dimensional space. But if we set $r=R$ and
thus $\D r=0$, we obtain
\begin{equation}
\D s^2=R^2(\D\theta^2+\sin^{2}(\theta)\,\D\varphi^2)
\end{equation}
So we plug this into the variation
\begin{equation}
\delta\int\D s=0
\end{equation}
to get the geodesic equation.
\end{ex}
\begin{ex}
Consider a surface in $\RR^3$ defined by
\begin{equation}
z=f(x,y).
\end{equation}
So now we have
\begin{subequations}
\begin{align}
\D s^2 
&=\D x^2+\D y^2+\D z^2\\
&=\D x^2+\D y^2+\left(\frac{\partial f}{\partial x}\,\D x+\frac{\partial f}{\partial y}\,\D y\right)^{2}\\
&=g_{xx}\,\D x^{2}+2g_{xy}\,\D x\,\D y+g_{yy}\,\D y^{2}
\end{align}
\end{subequations}
where we have
\begin{equation}
g_{xx}=1+\left(\frac{\partial f}{\partial x}\right)^{2},\quad
g_{xy}=\frac{\partial f}{\partial x}\frac{\partial f}{\partial y},\quad
g_{yy}=1+\left(\frac{\partial f}{\partial y}\right)^{2}
\end{equation}
are the coefficients. These coefficients $g_{ab}$ are called
\define{Components of the Metric Tensor}\index{Metric Tensor!Components of ---}
These are the basic physical variables. There is one
subtlety---we can have the same geometry described by
\emph{different coordinates!} For example, in Cartesian
coordinates the plane $\RR^2$ is described by
\begin{subequations}
\begin{equation}
\D s^2=\D x^2+\D y^2,
\end{equation}
whereas in Polar coordinates it is
\begin{equation}
\D s^2=\D r^2+r^2\,\D\theta^2,
\end{equation}
\end{subequations}
and although they describe the same geometry (a flat plane), the
metric tensor is different.
\end{ex}
\begin{rmk}\index{Metric Tensor!and Rotation}\index{Rotation!and Metric Tensor}
Note that the way to tell there is an object experience rotation
in spacetime is when the metric has a nonzero
$g_{t\varphi}\not=0$ term.
\end{rmk}

\begin{ex}[Flat $\RR^2$]
Consider flat 2-dimensional space. We have
\begin{equation}
\D s^2=\D x^2+\D y^2
\end{equation}
We want to describe a path, so we parametrize it:
\begin{equation}
(x,y)=(x(u),y(u)).
\end{equation}
We want to extremize
\begin{equation}
\D s=\left[\left(\frac{\D x}{\D u}\right)^{2}+
\left(\frac{\D y}{\D u}\right)^{2}\right]^{1/2}\D u
\end{equation}
Lets call the bracketed term, say,
\begin{equation}
E\eqdef\left(\frac{\D x}{\D u}\right)^{2}+
\left(\frac{\D y}{\D u}\right)^{2}.
\end{equation}
This is just an assignment of variables. Intuitively, it plays
the role of ``kinetic energy''. We want to extremize
\begin{equation}
s=\int E^{1/2}\,\D u
\end{equation}
What to do? Well,
\begin{subequations}
\begin{align}
\delta\int E^{1/2}\D u
&=\int\frac{1}{2}E^{-1/2}\delta E\,\D u\\
&=\int\frac{1}{2}E^{-1/2}\left[2\frac{\D x}{\D u}\delta\frac{\D x}{\D u}+
\frac{\D y}{\D u}\delta\frac{\D y}{\D u}\right]^{1/2}\D u\\
&=\int\frac{1}{2}E^{-1/2}\left[2\frac{\D x}{\D u}\frac{\D }{\D u}\delta x+
\frac{\D y}{\D u}\frac{\D}{\D u}\delta y\right]^{1/2}\D u
\end{align}
\end{subequations}
We integrate by parts, and demand the variation vanishes at its
endpoints, thus
\begin{equation}
\delta\int\D s=-\int\left[
\frac{\D}{\D u}\left(E^{-1/2}\frac{\D x}{\D u}\right)\delta x
+
\frac{\D}{\D u}\left(E^{-1/2}\frac{\D y}{\D u}\right)\delta y
\right]\D u.
\end{equation}
So this is supposed to vanish, which implies for the coefficients
\begin{equation}
\frac{\D}{\D u}\left(E^{-1/2}\frac{\D x}{\D u}\right)=0\quad
\mbox{and}\quad
\frac{\D}{\D u}\left(E^{-1/2}\frac{\D y}{\D u}\right)=0
\end{equation}
We can integrate directly to find $E=$constant. There is a trick
we never specified anything about $u$. So let us choose $u=s$,
it's a perfectly kosher choice. Then
\begin{equation}
E=1
\end{equation}
which makes the equations of motion
\begin{equation}
\frac{\D^2x}{\D u^2}=0,\quad\mbox{and}\quad
\frac{\D^2y}{\D u^2}=0.
\end{equation}
This has its solution be
\begin{equation}
x(s)=as+b,\quad y(s)=\alpha s+\beta.
\end{equation}
That's the geodesic for flat space in Cartesian coordinates.
\end{ex}

\lecture
%%
%% lecture02.tex
%% 
%% Made by Alex Nelson
%% Login   <alex@tomato3>
%% 
%% Started on  Tue Jan 19 11:05:14 2010 Alex Nelson
%% Last update Thu Jan 21 10:33:23 2010 Alex Nelson
%%
The main thing of interest is Lie groups, but Lie algebras are a
useful tool to study Lie groups. We will start with Lie
algebras. First what is an algebra. Well, more or less, it's
obtained from the formula
\begin{equation}
\hbox{Vector Space}+\hbox{Ring}=\hbox{Algebra},
\end{equation}
with some compatibility conditions. %% We want the addition
%% operation in the Ring structure to be the same as the addition
%% operation in the Vector Space structure, we want the
%% multiplication operation in the Ring structure to be
%% distributive over the Vector addition operation, and so on. 
When
we speak of vector spaces, we need a field $\Bbb{F}$; a ring has 2
operations: multiplication and addition. (A ring is an Abelian
group under addition, and a magma under multiplication.)
The only only relation between addition and multiplication is
distributivity: 
\begin{subequations}
\begin{align}
a(b+c)&=ab+ac\\
(b+c)a&=ba+ca.
\end{align}
\end{subequations}
In general, a ring doesn't have a multiplicative identity, nor is
multiplication an Abelian operation.

The compatibility condition for an algebra is thus
\begin{subequations}
\begin{align}
\lambda(ab) &= (\lambda a)\cdot b\\
&= a\cdot(\lambda b)
\end{align}
\end{subequations}
where $\lambda\in\Bbb{F}$. This is associativity. 

\begin{ex}
We have $\mat_{n}(\Bbb{F})$ --- the collection of all $n\times n$
matrices over a field $\Bbb{F}$ --- be a noncommutative,
associative algebra.
\end{ex}
\begin{ex}
If we have some set $M$, the set of functions on $M$ (denoted by
$C(M)$) is an algebra with respect to point-wise addition,
multiplication, and $\Bbb{F}$-scalar multiplication. Note that if
$\Bbb{F}$ is a field, the algebra is associative. Now $C(M)$ has
a lot of subalgebras if $M$ has some additional structure. If $M$
is a topological space, we have the set $C^0(M)\subset C(M)$ of
continuous functions be a subalgebra. If $M=\Bbb{R}^{n}$ we may
consider $C^{\infty}(M)\subset C(M)$ the subalgebra of smooth functions.
\end{ex}
\begin{ex}
We can consider $C^{\infty}(S^{n})$ where $S^n$ is the
$n$-sphere. We do this by introducing local coordinates, and
define the notion of smoothness in $S^n$ by demanding tit be
smooth in every coordinate system on $S^n$. But it is possible
for a function to be smooth in one coordinate system but not
another, so we need to use the notion of a transition function.
\end{ex}

A smooth manifold $M$ is covered by smooth local coordinate
systems, and the transition function between coordinate systems
is smooth. So $C^{\infty}(M)$, for some smooth manifold $M$, is a
unital, commutative, associative algebra.\marginpar{Unital=has unit element for multiplication}

We will introduce a construction of an algebra for a group,
called the group algebra. We consider all formal linear
combinations of group elements with coefficients from a ring:
%\begin{subequations}
\begin{align}
\hbox{Group}&\to\hbox{Algebra}\nonumber\\
G&\to\Bbb{F}[G]
\end{align}
%\end{subequations}
which has an element resemble $\sum_{i}^{n}a_{i}g_{i}$ where
$g_{i}\in G$ and $a_{i}\in\Bbb{F}$ for all $i$. We have addition
be component-wise, and multiplication also be component-wise.
So for example
\begin{subequations}
\begin{align}
(g_{1}+g_{2})+(g_{2}+3g_{3}) &= g_{1}+2g_{2}+3g_{3}\\
(g_{1}+g_{2})(g_{2}+3g_{3}) &= g_{1}g_{2}+3g_{1}g_{3}+g_{2}^{2}+3g_{2}g_{3}.
\end{align}
\end{subequations}
More generally
\begin{equation}
(\sum_{i}a_{i}g_{i})(\sum_{j}b_{j}g_{j})=\sum_{i,j}a_{i}b_{j}\cdot(g_{i}g_{j}).
\end{equation}
In the language of category theory, this is a functor $\Grp\to\Alg$.

Recall a representation of a group $G\to\GL{V}$ are homomorphisms
from $G$ to automorphisms on $V$. We have very simply for a rep
$G\to\GL{n}$ a representation
$\Bbb{F}[G]\to\mathcal{L}(V,V)=\mat_{n}$ of the algebras. For
every $g\in G$ we have its representation $\varphi(g)$, so this
induces a representation
\begin{equation}
\sum a_{i}g_{i}\mapsto \sum a_{i}\varphi(g_{i}),
\end{equation}
where products go to products and sums go to sums. The opposite
direction, a representation of $\Bbb{F}[G]$ induces a
representation of $G$, is also true (by the duality
principle). Moral: representations of groups induce
representations of associative algebras (a representation of
associative algebras in general is referred to as
\define{Modules}).

Consider $\scr{A}$ an associative algebra. We will define a new
operation on $\scr{A}$, namely the bracket as a commutator
\begin{equation}
[a,b]=ab-ba
\end{equation}
for all $a,b\in\scr{A}$. So with respect to the bracket,
$\scr{A}$ is an algebra (distributivity remains, but
associativity is broken). But observe
\begin{enumerate}
\item $[a,b]=-[b,a]$ i.e. we have antisymmetry of the bracket;
\item the Jacobi identity holds.
\end{enumerate}
This newly constructed algebra is in fact a Lie algebra! So for
every associative algebra $\scr{A}$, we may construct a Lie
algebra on $\scr{A}$; this is described by a natural functor, so
algebra morphisms are mapping to Lie algebra morphisms.

\begin{rmk}
There are other ways to construct Lie algebras. 

\noindent{\bf N.B.:} subalgebras of Lie algebras are again Lie algebras.
\end{rmk}

\begin{ex}
The Lie Algebra $\mat_{n}(\Bbb{F})=\frak{gl}_{n}(\Bbb{F})$ the
Lie algebra for $\GL{n,\Bbb{F}}$.
\end{ex}
\begin{ex}
Consider
$\frak{sl}_{n}(\Bbb{F})=\{A\in\frak{gl}_{n}(\Bbb{F})\mid\tr(A)=0\}$. This
is a Lie subalgebra of $\frak{gl}_{n}(\Bbb{F})$, since
$\tr(AB)=\tr(BA)$ so $\tr([A,B])=0$ for all $A,B\in\frak{gl}_{n}(\Bbb{F})$.
\end{ex}

We can introduce the notion of an \define{Ideal} in an algebra,
especially a Lie algebra! If $I\subset R$ where $R$ is a ring,
then $IR=I$ is a left ideal, and $RI=I$ is a right ideal. This
notion may be generalized to algebras (especially Lie
algebras!). Fir Lie algebras, \emph{every ideal is a two-sided
  ideal.}

\begin{prop}
$\frak{sl}_{n}(\Bbb{F})$ is an ideal in $\frak{gl}_{n}(\Bbb{F})$.
\end{prop}

If we have a ring morphism $\varphi\colon R\to R'$, its kernel is
a two-sided ideal. Moreover we may factorize
\begin{equation}
\im(\varphi)\iso R/\ker(\varphi).
\end{equation}
This construction generalizes to algebras.

\subsection{Exercises}
\begin{exercise}\label{ex:prob1}
Check that the vector space $\RR^{3}$ is a Lie algebra with respect to cross-product of vectors. Check that this Lie algebra is simple (does not have any non-trivial ideals). Check that all derivations of this Lie algebra are inner derivations.
\end{exercise}
\begin{exercise}
Check that the Lie algebra of Problem \ref{ex:prob1} is isomorphic to the Lie algebra $\mathfrak{so}(3)$ of real antisymmetric $3\times3$ matrices and to the Lie algebra $\mathfrak{su}(2)$ of complex anti-Hermitian traceless $2\times2$ matrices.
\end{exercise}

\lecture[Classification of Principal Bundles]
%%
%% lecture03.tex
%% 
%% Made by Alex Nelson
%% Login   <alex@tomato3>
%% 
%% Started on  Thu Jan 21 10:33:51 2010 Alex Nelson
%% Last update Thu Jan 21 13:10:56 2010 Alex Nelson
%%

Let us consider $n$-dimensional space $\Bbb{R}^{n}$, with
coordinates $(x^1,\ldots,x^n)$, and we can consider either
functions or polynomials of these coordinates
$\Bbb{C}[x^1,\ldots,x^n]$ and we will consider the differential
operators on $\Bbb{C}[x^1,...,x^n]$. It is an associative
algebra, but also a Lie algebra (when the Lie bracket is the
commutator). We can consider Lie subalgebras, e.g. first order
differential operators\footnote{Note that we are using Einstein
  summation convention; when one index is upstairs and another is
  downstairs, we sum over it as a dummy index.}$\widehat{A}=A^{i}\partial_{i}$. But this
is \emph{NOT} a subalgebra of derivations, the product
$\widehat{A}\widehat{B}$ is a second order differential operator;
however note that
\begin{subequations}
\begin{align}
[\widehat{A},\widehat{B}](f)&=A^{i}\partial_{i}(B^{j}\partial_{j}f)-B^{j}\partial_{j}(A^{i}\partial_{i}f)\\
&=A^{i}(\partial_{i}B^{j})\partial_{j}f+A^{i}B^{j}\partial_{i}\partial_{j}f-B^{j}(\partial_{j}A^{i})\partial_{i}f-A^{i}B^{j}\partial_{i}\partial_{j}f\\
&=A^{i}(\partial_{i}B^{j})\partial_{j}f-B^{j}(\partial_{j}A^{i})\partial_{i}f
\end{align}
\end{subequations}
So we write
\begin{equation}
\widehat{C}=C^{k}\partial_{k}=A^{i}(\partial_{i}B^{j})\partial_{j}-B^{j}(\partial_{j}A^{i})\partial_{i}
\end{equation}
This $\widehat{C}$ is a derivation on $\Bbb{C}[x^1,...,x^n]$. We
can write
$C^k=A^{j}(\partial_{j}B^{k})-B^{j}(\partial_{j}A^{k})$. The
commutator of first order differential operators is again a first
order differential operator.

We would like to express this operator in two different
ways. First what are the coefficients $A^i$? They are the
components of a vector field. So this is really the algebra of
vector fields, the commutator of vector fields yield a Lie
Algebra. Second, we want to introduce the notion of derivation of
algebra. It is something satisfying the Leibniz rule. Suppose we
have an $\scr{A}$-algebra, and a linear map
\begin{equation}
\alpha\colon\scr{A}\to\scr{A}
\end{equation}
such that
\begin{equation}
\alpha(ab)=\alpha(a)b+a\alpha(b).
\end{equation}

First, these first order differential operators are derivations,
and moreover all derivations are first order differential
operators.

This is a bit ambiguous, the algebra considered are left
unspecified (smooth functions or polynomials!). We will prove it
for polynomials, but not for smooth functions. If we know how a
derivation behaves on the generators of the polynomial, then we
know everything. Let $\widehat{A}(x^i)=A^i(x)$ where $A^i(x)$ is
a polynomial.

\begin{rmk}
All this stuff works on smooth manifolds despite never specifying
what a ``smooth manifold'' is!
\end{rmk}

\begin{thm}
Given an algebra $\scr{A}$, then we may consider $\der(\scr{A})$
of derivations of $\scr{A}$ which form a Lie algebra.
\end{thm}
\begin{proof}
We should prove it is a vector space, but it is obvious; we
should prove the commutator of derivations
$\alpha,\beta\in\der(\scr{A})$ is a derivation
$[\alpha,\beta]\in\der(\scr{A})$. We consider
\begin{subequations}
\begin{align}
(\alpha\circ\beta)(ab)&=\alpha(\beta(ab))\\
&=\alpha(\beta(a)\cdot b+a\cdot\beta(b))\\
&=\alpha(\beta(a)\cdot b)+\alpha(a\cdot \beta(b))\qquad\hbox{by linearity}\\
&=(\alpha\circ\beta)(a)\cdot b+\beta(a)\alpha(b)+\alpha(a)\beta(b)+a\cdot(\alpha\circ\beta)(b)
\end{align}
\end{subequations}
Now we can consider the commutator expression of $\alpha$ with
$\beta$, which amounts to
\begin{subequations}
\begin{align}
[\alpha,\beta](ab) &= \bigg((\alpha\circ\beta)(a)\cdot b+\beta(a)\alpha(b)+\alpha(a)\beta(b)+a\cdot(\alpha\circ\beta)(b)\bigg)\nonumber\\
&\quad-\bigg((\beta\circ\alpha)(a)\cdot b+\beta(a)\alpha(b)+\alpha(a)\beta(b)+a\cdot(\beta\circ\alpha)(b)\bigg)\\
&=(\alpha\circ\beta)(a)b+a\cdot(\alpha\circ\beta)(b)-(\beta\circ\alpha)(a)b-a\cdot(\beta\circ\alpha)(b)\\
&=[\alpha,\beta](a)\cdot b+a\cdot[\alpha,\beta](b).
\end{align}
\end{subequations}
This concludes our proof.
\end{proof}

One last example of derivations. Consider an algebra $\scr{A}$
(either associative or Lie), take $a,x\in\scr{A}$ where $a$ is
fixed. Consider the derivation
\begin{equation}
\alpha_{a}(x)=[a,x].
\end{equation}
For Lie algebras it is absolutely trivial:
\begin{subequations}
\begin{align}
\alpha_{a}([x,y]) &= [\alpha_{a}(x),y]+[x,\alpha_{a}(y)]\\
\iff[a,[x,y]] &=[[a,x],y]+[x,[a,y]]\\
&=-[x,[y,a]]-[y,[a,x]]\qquad\hbox{Jacobi Identity!}
\end{align}
\end{subequations}
\begin{rmk}
Such derivations are called \define{Inner Derivations}.
\end{rmk}
Lets compute the commutator of two inner derivations, the answer
is the
\begin{equation}
[\alpha_a,\alpha_b]=\alpha_{[a,b]}
\end{equation}
the result is an  inner derivation. We have a homomorphism, so we
have a $\scr{G}$-Lie algebra so we get a map
$\scr{G}\to\der({\scr{G}})$ which, for all $a\in\scr{G}$, is mapped
to
\begin{equation}
\alpha_a=[a,-].
\end{equation}

\medbreak
\noindent\textbf{N.B.:} Henceforth and throughout, I will use the
term ``morphism'' and ``homomorphism'' interchangeably. 

We can consider the morphism $\scr{G}\to L(\scr{G})$ where
$L(\scr{G})$ is the linear operators on $\scr{G}$. We have a
representation of our Lie algebra $\scr{G}$, called the
\define{Adjoint Representation} where
$a\mapsto\alpha_a=[a,-]$. We write $\ad_{a}=\alpha_a$. This is
one of the simplest and most important examples of the
representation of Lie algebras.

Consider $\ker(\ad)$. Then $\alpha_a=0$. What does it mean that
$[a,x]=0$ for all $x\in\scr{G}$? This is precisely the
\define{Center of $\scr{G}$} denoted by $\ker(\alpha)=Z$.
\begin{thm}
If a finite dimensional Lie Algebra has no center (or a trivial one), then it is
isomorphic to a matrix Algebra.
\end{thm}
\begin{proof}
We see that $\im(\ad)$ consists of a subalgebra of a Lie algebra
of matrices since the Lie algebra \emph{IS} a vector space and it
is finite dimensional. Thus $\ad$ is a matrix algebra.
\end{proof}

\lecture
%%
%% lecture04.tex
%% 
%% Made by Alex Nelson
%% Login   <alex@tomato3>
%% 
%% Started on  Sun Apr 10 12:03:16 2011 Alex Nelson
%% Last update Sat May 21 13:59:02 2011 Alex Nelson
%%
We are continuing to classify principal bundles when $B=S^{n}$
and $G$ is connected. We see classes of principal fibrations over
$B$ are in one-to-one correspondence with $\homotopyClass(S^{n-1},G)$.

Let $S^{n}=D^{n}_{1}\cup D^{n}_{2}$, we assume
\begin{equation}
D^{n}_{1}\cap D^{n}_{2}=S^{n-1}
\end{equation}
which generalizes the notion of a sphere having two hemispheres
that meets at an equator. We know that on the ball the fibration
is trivial. Thus we may take
the \define{trivialization}\index{Fibration!Trivialization|textbf}\index{Trivialization!of Fibration}
of the fibration, i.e.,
\begin{equation}
p^{-1}(D^{n}_{1})=D^{n}_{1}\times G
\end{equation}
is identified as the trivial principal bundle; and trivialization
over the other hemisphere means
\begin{equation}
p^{-1}(D^{n}_{2})=D^{n}_{2}\times G
\end{equation}
But trivialization is not unique, we need to choose it in some
way. But we choose it in some way: we paste them together over
the sphere. The pasting is described by a \define{Transition function}\index{Transition Function}
a map
\begin{equation}
\varphi\colon S^{n-1}\to G
\end{equation}
The only problem is this is \emph{any} continuous map. This
establishes the one-to-one correspondence.

Really?! No, not really. We have some freedom in the choice of
trivialization. This means we have the same stuff, the preimage
of the first hemisphere may be represented in two different ways,
and we have a morphism between these two different ways (i.e., we
have the following commutative diagram)
\begin{equation}
\begin{diagram}[small]
p^{-1}(\bar{D}^{n}_{1}) & \rEq & \bar{D}^{n}_{1}\times G\\
\dEq                 & \ruTo>{\varphi} & \\
\bar{D}^{n}_{1}\times G &           & 
\end{diagram}
\end{equation}
that is, we have a morphism between these two ways
\begin{equation}
\begin{split}
\varphi\colon\bar{D}^{n}_{1}\times G\to \bar{D}^{n}_{1}\times G\\
\varphi(x,g)=(x,\alpha(x)g)
\end{split}
\end{equation}
where
\begin{equation}
\alpha\colon\bar{D}^{n}_{1}\to G.
\end{equation}
So what happens on the sphere? Well, $\alpha$ is an arbitrary
map, but on the boundary sphere $S^{n-1}$ we see that $\alpha$ is
not arbitrary because it can be extended to the ball. That means
\begin{equation}
\alpha\big|_{S^{n-1}}\homotopic\mbox{trivial map}
\end{equation}
that is, a map sending everything to a point. But we assumed $G$
is connected, so this notion of a trivial map is unambiguous.

The transition function is not uniquely defined, but only up to
homotopy. This means we have $\alpha\psi$. In any case, when we
multiply by $\alpha$ (or its analog from $\bar{D}^{n}_{2}$) we do
not change the homotopy class. This concludes the proof. But this
is only for the case when $B=S^{n}$. What happens in the
arbitrary case?

Well, we have some notions we need to first define. So suppose
that we have a principal bundle $(E_{G},B_{G},G)$ having the
property that $E_{G}$ is contractible. Then this bundle is called
a \define{Universal Bundle}\index{Universal Bundle|textbf}\index{Bundle!Universal} and $B_{G}$ is called the
\define{Classifying Space}\index{Classifying Space|textbf}. Can we construct any such bundles?
Why is this name appropriate? We will consider interesting cases
later.

\begin{thm}
Classes of principal bundles with base $B$ and group $G$ are in
one-to-one correspondence with $\homotopyClass(B,B_{G})$.
\end{thm}

Observe if
\begin{equation}
B\iso S^{n}
\end{equation}
then it is easy to prove (at least if $G$ is connected) that
\begin{equation}
\pi_{n}(B_{G})=\{S^{n},B_{G}\}
\end{equation}
since $E_{G}$ is contractible we may consider the exact homotopy
sequence that identifies
\begin{equation}
\pi_{n}(B_{G})\iso\pi_{n-1}(G)
\end{equation}
and this is the same as $\homotopyClass(S^{n-1},G)$. So this new theorem
contains the old one.

\begin{rmk}
Perhaps one should bear in mind the exact homotopy sequence of
the bundle. Don't forget we can lift $S^{n-1}$ to $\bar{D}^{n}$.
\end{rmk}

Now to prove this theorem, we need to explain some
constructions. The first thing is let us suppose we have two
principal bundles $(E,B,G)$ and $(E',B',G)$ and we have a map
\begin{equation}
f\colon (E,B,G)\to(E',B',G)
\end{equation}
which agrees with the structure
\begin{equation}
f(xg)=f(x)g.
\end{equation}
This map \emph{induces} a map of the base space
\begin{equation}
\varphi\colon B\to B'.
\end{equation}
Now there is an interesting remark: knowing $\varphi$ and the
target, we may restore the map $f$. How to do this? It's easy.

First we should explain how to recover $E$ if we know $E'$ and
$B$. Let
\begin{subequations}
\begin{align}
p\colon E\to B\\
p'\colon E'\to B'\\
f\colon E\to E'.
\end{align}
\end{subequations}
We can combine these maps, what do we get? We get a map
\begin{equation}
F\colon E\to E'\times B.
\end{equation}
How? Very simply:
\begin{equation}
F(e)=\left(f(e),p(e)\right)
\end{equation}
We may consider the image $F(E)$. We have the commutative diagram
\begin{equation}
\begin{diagram}[small]
E & \rTo^{f} & E'\\
\dTo<{p} &  & \dTo>{p'}\\
B & \rTo^{\varphi} & B'
\end{diagram}
\end{equation}
which implies
\begin{equation}
p'\circ f=\varphi\circ p.
\end{equation}
But observe that we have exactly that!

We have a space
\begin{equation}
\widetilde{E}\subset E'\times B
\end{equation}
consisting of pairs
\begin{equation}
(e',b)\in \widetilde{E}
\end{equation}
satisfying the relation
\begin{equation}
p'(e')=\varphi(b).
\end{equation}
Now what can we say? We can say the following: we have a map
\begin{equation}
F\colon E\to\widetilde{E}
\end{equation}
and it is surjective since $\widetilde{E}$ is the image, and of
course it is injective and continuous. Therefore we may say that
there is a continuous bijection
\begin{equation}
E\to\widetilde{E}.
\end{equation}
In all good cases, this is a homeomorphism.
\begin{rmk}
If a space is compact, then every continuous bijection is a
homeomorphism. 
\end{rmk}
Thus we may conclude we may restore $E$ if we know: $E'$, $f$,
$\varphi$. 

We will introduce the notion of
a \define{Pullback}\index{Pullback!of Principal Fibrations}\index{Pullback|textbf}. Consider a
space, more precisely a principal fibration
\begin{equation}
(E',B',G',p'\colon E'\to B').
\end{equation}
We consider a map
\begin{equation}
\varphi\colon B\to B'
\end{equation}
There is a way to transfer the fibration from $B'$ to $B$. We
construct 
\begin{equation}
\widetilde{E}\subset E\times B
\end{equation}
consisting of pairs $(e',b)$ that satisfy the condition
\begin{equation}
p'(e')=\varphi(b).
\end{equation}
We have a map
\begin{equation}
p\colon\widetilde{E}\to B
\end{equation}
satisfying
\begin{equation}
p(e',b)=b.
\end{equation}
This is a principal fibration. To be more general, we may
use \emph{any} fibration in this construction.

Now, suppose we have the universal bundle ($E_{G}$, $B_{G}$, $G$,
$p_{G}\colon E_{G}\to B_{G}$) with classifying space $B_{G}$. If
we have 
\begin{equation}
\varphi\colon B\to B_{G}
\end{equation}
then we may construct the pullback. The theorem that we prove
next time is if 
\begin{equation}
\varphi\homotopic\varphi',
\end{equation}
then their pullbacks are homotopic and in one-to-one
correspondence.


\lecture
%%
%% lecture05.tex
%% 
%% Made by Alex Nelson
%% Login   <alex@tomato3>
%% 
%% Started on  Thu Jan 21 11:44:35 2010 Alex Nelson
%% Last update Mon Dec 13 21:50:25 2010 Alex Nelson
%%

The first example is $\GL{n,\Bbb{R}}$ and the corresponding Lie
algebra $\frak{gl}_{n}(\Bbb{R})$ which consists of all
matrices. It is important to consider
$\frak{gl}_{n}(\Bbb{R})\subset\frak{gl}_{n}(\Bbb{C})$ the
complexification of the algebra. We will denote
$\frak{gl}_{n}(\Bbb{C})=\Bbb{C}\frak{gl}_{n}(\Bbb{R})=\frak{gl}_{n}$.

Another example ${\rm SL}_{n}(\Bbb{R})$ the group of $n$-by-$n$
matrices satisfying the property of having unit determinant. To
compute the Lie algebra, consider elements close to $I$ or more
precisely a curve
\begin{equation}
A(\tau)=I+a(\tau).
\end{equation}
Now we should like to consider the tangent vector by Taylor
expanding the curve and using the coefficient of the first order
term as the tangenet vector
\begin{equation}
A(\tau)=I+\tau X+\mathcal{O}(\tau^{2}),
\end{equation}
so we want
\begin{equation}
\det\left(A(\varepsilon)\right)=\det(I+\varepsilon X)=I+\varepsilon\tr(X)+\mathcal{O}(\varepsilon^{2}).
\end{equation}
We see immediately that the condition $\tr(X)=0$ is the condition
for elements of the Lie algebra. So we see that
\begin{equation}
\Lie(\SL{n,\Bbb{R}})=\frak{sl}_{n}(\Bbb{R})=\{X\in\frak{gl}_{n}\mid\tr(X)=0\}.
\end{equation}
We may consider the complexification
\begin{equation}
\Bbb{C}\frak{sl}_{n}(\Bbb{R})=\frak{sl}_{n}(\Bbb{C})=\frak{sl}_{n}.
\end{equation}
{\bf N.B.} $\frak{sl}_{n}$ is an ideal in $\frak{gl}_{n}$ and
$\frak{gl}_{n}=\Bbb{C}\oplus\frak{sl}_{n}$ is the direct sum of
the trivial Lie algebra $\Bbb{C}$ and $\frak{sl}_{n}$ since
$A=\alpha\cdot I+A'$ where $A'\in\frak{sl}_{n}$ so
$\tr{A}=\alpha\cdot{\rm dim}$.

We also see $\ORTH{n}=\{A\in\GL{n}\mid A^{T}A=I\}$. The Lie algebra
is obtained by considering $A=I+\varepsilon X$ where
$\varepsilon^{2}\approx 0$ is an
``infinitesimal''\footnote{Although this is a mathematically
  unrigorous notion, we can still use it for computational and
  heuristic purposes.}. The Lie algebra is obtained by
\begin{subequations}
\begin{align}
A^{T}A &= (I+\varepsilon X)^{T}(I+\varepsilon X)\\
&= I +\varepsilon(X^{T}+X)+\mathcal{O}(\varepsilon^2)\\
&= I\quad\iff\quad X^{T}+X=0.
\end{align}
\end{subequations}
This is the condition for the Lie algebra of $\ORTH{n}$ which is denoted
\begin{equation}
\frak{so}(n)=\{X\in\frak{gl}(n)\mid X+X^{T}=0\}.
\end{equation}
We are interested in
$\frak{so}(n)=\frak{so}(n,\Bbb{C})=\Bbb{C}\frak{so}(n,\Bbb{R})$. We
see the Lie group
\begin{equation}
\SO{n}=\ORTH{n}\cap\SL{n}
\end{equation}
has unit determinants. We can also quickly compute and find that
$\SO{n}$ is the connected part of $\ORTH{n}$ which contains the identity.

\begin{wrapfigure}[12]{r}{0.35\textwidth}
  \vspace{-30pt}
  \begin{center}
    \includegraphics{img/LieImg.0}
  \end{center}
  \vspace{-20pt}
  \caption{{\small The Two Seperated Components of O$(n)$.}}\label{fig:LieImg0}
  \vspace{20pt}
\end{wrapfigure}

The group $\ORTH{n}$ has elements $A\in\ORTH{n}$ such that
$\det(A)^{2}=1$, so it has two seperate components. This is seen
in figure \ref{fig:LieImg0}. This means that the group $\ORTH{n}$ is
disconnected, there is no continuous path connecting e.g. an
element $X\in\ORTH{n}$ with $\det(X)=-1$ to an element $Y\in\ORTH{n}$
with $\det(Y)=+1$, because the path would have to go through a
point with zero determinant. That is a singular matrix, which is
not contained in the group $\GL{n}$, and that would imply
$\ORTH{n}\not\subset\GL{n}$ which is a contradiction.

\begin{rmk}
Both $\SO{n}$ and $\ORTH{n}$ are both compact groups, i.e. closed
and bounded.
\end{rmk}
\begin{rmk}
Note that $\U{n}$ and $\SU{n}=\U{n}\cap\SL{n}$ are also
compact. The condition for $\U{n}=\{A\in\GL{n,\Bbb{C}}\mid
A^{\dagger}A=I\}$, and the corresponding Lie algebra is
$\frak{u}(n)$. The condition for it is
\begin{subequations}
\begin{align}
(I+\varepsilon A^{\dagger})(I+\varepsilon A)&=I+\varepsilon(A^{\dagger}+A)+\mathcal{O}(\varepsilon^{2})\\
&=I\quad\iff\quad A^{\dagger}+A=0.
\end{align}
\end{subequations}
So $\frak{u}(n)=\{X\in\frak{gl}_{n}(\Bbb{C})\mid
A^{\dagger}+A=0\}$. We have for
$\frak{su}(n)=\{A\in\frak{u}(n)\mid \tr(A)=0\}$. We see that
$\Bbb{C}=\frak{gl}_{n}(\Bbb{C})$,
$\Bbb{C}\frak{su}(n)=\frak{sl}_{n}$ are the complexifications.
\end{rmk}

The last classical group we would like to consider preserves some
skew-symmetric inner product. That is to say, $\<x,y\>=-\<y,x\>$
more generally however we will use a Bilinear form $B$ which is
antisymmetric
\begin{equation}
B(x,y)=-B(y,x).
\end{equation}
We write
\begin{equation}
B(x,y)=x^{T}By
\end{equation}
if $B$ is an antisymmetric matrix. We want to find matrices $A$
such that $B(Ax,Ay)=B(x,y)$, i.e.
\begin{equation}
(Ax)^{T}BAy=x(A^{T}BA)y=xBy
\end{equation}
or equivalently $A^{T}BA=B$. {\bf N.B.} if $B=I$ we recover the
orthogonal group. We get a group $\Sp{n}=\{A\in\GL{2n}\mid
A^{T}BA=B\}.$ the Lie algebra is of the form
\begin{equation}
(I+\varepsilon X)^{T}B(I+\varepsilon X)=B\quad\Rightarrow\quad X^{T}B+BX=0.
\end{equation}
If $B$ is nondegenerate, then 
\begin{equation}
\frak{sp}(n)=\{X|X^{T}B+BX=0\}
\end{equation}
is the Lie Algebra for $\Sp{n}$. This is a noncompact group. We
can get a compact group by examining the intersection ${\rm
  Sp}_{n}(\Bbb{C})\cap\U{n}={\rm Sp}_{n}$, and $\Bbb{C}\Lie({\rm
  Sp}_{n})=\frak{sp}(n)$ is the original Lie Algebra.

\begin{thm}
All simple Lie algebras that are Lie algebras of compact groups
are classical Lie algebras or one of 5 exceptional Lie algebras.
\end{thm}

There is a related notion of semisimple Lie algebra. A semisimple
Lie algebra is a direct sum of noncommutative simple Lie
algebras. There is an important class of \define{Solvable} Lie
algebras. 

Remember (e.g. from Lang's \emph{Algebra} chapter I \S3) that a group $G$ is
\define{Solvable} if we have a tower of groups
\begin{equation}
G\supset G_{1}\supset G_{2}\supset\cdots\supset G_{m}=\{e\}
\end{equation}
such that for $G_{n-1}\supset G_{n}$ we have $G_{n-1}/G_{n}$ be Abelian.

The notion of a solvable Lie algebra is the same except we have
$\frak{g}_{n-1}\supset \frak{g}_{n}$\marginpar{\textbf{Notational
Warning:} we will use $\frak{g}$ or $\mathscr{G}$ for the Lie
Algebra, interchangeably, and without warning!} be an ideal. We also can
doodle this cute diagram
\begin{equation}
\begin{diagram}[small]
\mbox{Group}   & \qquad & G           & \rhd    & H          &\\
               &        & \uTo<{\exp} &         & \uTo>{\exp}&\\
\mbox{Algebra} & \qquad &\Lie(G)      & \supset & \Lie(H)    &\mbox{ is an ideal}
\end{diagram}
\end{equation}

In a sense, these two notions of solvability and semisimplicity
are ``complementary'' --- an arbitrary Lie group has a semisimple
part and a reductive part. A compact Lie group has a semisimple
Lie algebra.

\subsection{Exercises}
\begin{exercise}\label{ex:prob3}
Find Lie algebras of the following matrix groups
\begin{enumerate}
\item The group of real upper triangular matrices
\item The group of real upper triangular matrices with diagonal entries equal to 1.
\item The group $T_{k}$ of real $n\times n$ matrices obeying $a_{ii} = 1$, $a_{ij} = 0$ if $j-i<k$ and $j=i$.
\end{enumerate}
\end{exercise}
\begin{exercise}
Check that the groups of Problem \ref{ex:prob3} and corresponding Lie algebras are solvable.
\end{exercise}

\lecture
%%
%% lecture06.tex
%% 
%% Made by Alex Nelson
%% Login   <alex@tomato3>
%% 
%% Started on  Sun Aug 14 14:59:43 2011 Alex Nelson
%% Last update Sun Aug 14 15:32:10 2011 Alex Nelson
%%
We described the construction involving $\GL{n,\RR}$,
$\GL{n,\CC}$ that are its classifying spaces. We are only
interested in homotopical properties. Here we may say 
\begin{equation}
\GL{n,\RR}\homotopic\ORTH{n}
\end{equation}
is homotopic, and
\begin{equation}
\GL{n,\CC}\homotopic\U{n}
\end{equation}
is also homotopic.

What is $\GL{n}$ as a space? Well, pick a basis for 
\begin{equation}
V\iso\FF^{n}
\end{equation}
then change the basis. The ``Jacobian'' is then a member of
$\GL{n}$. So $\GL{n,\RR}$ is the space of frames in $\RR^{n}$. We
can always orthonormalize a given basis by the Grahm-Schmidt
procedure. This gives us a map
\begin{equation}
\GL{n}\to\ORTH{n}
\end{equation}
and of course
\begin{equation}
\ORTH{n}\Into\GL{n}
\end{equation}
is an embedding. We merely have to prove that $\GL{n}\to\ORTH{n}$
can be done ``gradually''. We then have
\begin{equation}
\GL{n,\RR}\homotopic\ORTH{n},
\end{equation}
and similar reasoning suggests that $\GL{n,\CC}$ is homotopic to
$\ORTH{n,\CC}$. But $\ORTH{n,\CC}$ is in one-to-one
correspondence with $\U{n}$, so $\GL{n,\CC}\homotopic\U{n}$ is
homotopic.

If 
\begin{equation}
\mathrm{GL}_{+}(n,\RR)=\{X\in\GL{n,\RR}\lst\det(X)>0\}
\end{equation}
then $\mathrm{GL}_{+}(n,\RR)\homotopic\SO{n}$ is homotopic.

We come to the notion of a Stiefel manifold\index{Stiefel Manifold}, which we covered
Fall quarter. We have several different definitions of
$V_{n,k}(\RR)$ which are homotopy equivalent definitions, so we
do not distinguish between them. We can say that $V_{n,k}(\RR)$
is the space of $k$-framed in $\RR^{n}$. Observe that
\begin{equation}
V_{n,n}\homotopic\GL{n,\RR}
\end{equation}
is homotopic. We may say $V_{n,k}(\RR)$ is the space of $k$
orthonormal vectors in $\RR^{n}$. These two are homotopy
equivalent spaces. We may extend to $V_{n,k}(\CC)$.

There are various different groups that act on Stiefel
manifolds. We see that $\GL{n}$ acts on $\RR^{n}$, and thus acts
on $V_{n,k}(\RR)$ by transitivity. So $V_{n,k}$ is an orbit, thus
we may say 
\begin{equation}
V_{n,k}=\GL{n}\Big/(\text{some stable subgroup}).
\end{equation}
But it is another way to describe Stiefel manifolds.

Let $V_{n,k}^{\mathrm{orth}}(\RR)$\index{$V_{n,k}^{\mathrm{orth}}(\RR)$} be the Stiefel manifold
describing orthonormal $k$-frames. We may say $\ORTH{n}$ acts
transitively on $V_{n,k}^{\mathrm{orth}}(\RR)$, rotating one
frame into another. But in this situation, we may describe the
stabilizer in a simple way.

We look at orthogonal transformations that keep this frame in
tact. We may write
\begin{equation}
\RR^{n}=\RR^{k}\oplus\RR^{n-k}
\end{equation}
the transformations preserves $\RR^{k}$. So therefore the
stabilizer rotates $\RR^{n-k}$ into itself, and
\begin{equation}
V^{\mathrm{orth}}_{n,k}(\RR)=\ORTH{n}/\ORTH{n-k}.
\end{equation}
Precisely the same consideration goes in the complex case. The
only difference is 
\begin{equation}
V^{\mathrm{orth}}_{n,k}(\CC)=\U{n}/\U{n-k}.
\end{equation}
i.e., we work with unitary transformations.

We will look at fibrations involving Stiefel manifolds. Suppose
we had $V_{n,k}$. Consider \emph{only} $V_{n,k}^{\mathrm{orth}}$,
we will thus without loss of generality remove the
superscript. We observe
\begin{equation}
p\colon V_{n,k}\to V_{n,k-1}
\end{equation}
by forgetting the $k^{\mathrm{th}}$ vector. This map is a locally
trivial fibration. We see that
\begin{equation}
p^{-1}(e_{1},\dots,e_{k-1})=\{(e_{1},\dots,e_{k})\lst
e_{k}\in\FF\}
\end{equation}
where $\FF$ is the field we're working with. 


If
\begin{equation}
V=\Span(e_{1},\dots,e_{k-1})
\end{equation}
then $e_{k}\in V^{\bot}$ and
\begin{equation}
\dim(V^{\bot})=n-(k-1)
\end{equation}
but $e_{k}$ should be normalized and thus $e_{k}\in S^{n-k}$. We
obtain a fibration with base $V_{n,k-1}$ and fibre $S^{n-k}$ and
the total space is $V_{n,k}$.

When we work over $\CC$, we see that
\begin{equation}
\dim(V^{\bot})=2\left(n-(k-1)\right)
\end{equation}
and thus the fibre has dimension $2n-2k+1$, so
$F=S^{2n-2k+1}$. 

We would like to compute homotopy groups of the fibrations of the
Stiefel manifolds. The exact sequence of homotopy groups of the
fibration is
\begin{equation}
\pi_{i+1}(S^{\bullet})\to\pi_{i}(V_{n,k})\to\pi_{i}(V_{n,k-1})\to\pi_{i}(S^{\bullet})
\end{equation}
and we see for ``small dimensional spheres,'' we have
$\pi_{i}(S^{\bullet})=0$ which implies
\begin{equation}
\pi_{i}(V_{n,k})\iso\pi_{i}(V_{n,k-1}).
\end{equation}
We see that
\begin{equation}
V_{n,1}(\RR)\iso S^{n-1}
\end{equation}
the orthonormal (real) vector lives in the sphere, and similarly
\begin{equation}
V_{n,1}(\CC)\iso S^{2n-1}.
\end{equation}
We may inductively compute other homotopy groups. We see that
\begin{equation}
\pi_{i}(V_{\infty,k})=0
\end{equation}
which is unsurprising since $\pi_{i}(S^{\infty})=0$.

Observe that on $V_{n,k}(\RR)$ we have the action or $\ORTH{k}$
or $\SO{k}$ [and for $V_{n,k}(\CC)$ we have the action of $\U{k}$
or $\SU{k}$]. We consider an action of the form
\begin{equation}
(e_{1},\dots,e_{k})\mapsto(e_{1}',\dots,e_{k}')
\end{equation}
where
\begin{equation}
e_{i}'=\sum_{j}a_{ij}e_{j}
\end{equation}
where $(a_{ij})$ is ``invertible.'' This is a \emph{free}
action. Now we have a question: we can factorize this stuff
\begin{equation}
V_{n,k}(\CC)/\U{k}=\Gr_{n,k}(\CC)
\end{equation}
is precisely the Grassmann manifold. It is obvious. It is clear
that if we have a frame $(e_{1},\dots,e_{k})$ we can map it to 
$\Span(e_{1},\dots,e_{k})$; it is clear that this is a fibration
$V_{n,k}(\CC)\to \Gr_{n,k}(\CC)$, and the fibre is $k$-frames
living in $k$-dimensional space.

Now lets take $n=\infty$ and we find
\begin{equation}
V_{\infty,k}(\CC)/\U{k}=\Gr_{\infty,k}(\CC).
\end{equation}
We see that we have a universal bundle\index{Universal Bundle!for $\U{k}$} 
since $V_{\infty,k}$\index{$V_{\infty,k}$} is
contractible, $\U{k}$ acts freely, so $\Gr_{\infty,k}(\CC)$ is the
classifying space\index{Classifying Space!for $\U{k}$}.

Lets consider real Stiefel manifolds, we may take the
factorization
\begin{equation}
V_{n,k}(\RR)/\ORTH{k}=\Gr_{n,k}(\RR)
\end{equation}
and repeat the same arguments to find
\begin{equation}
V_{\infty,k}(\RR)/\ORTH{k}=\Gr_{\infty,k}(\RR).
\end{equation}
We may take the quotient with $\SO{k}$ which produces the space
of \emph{oriented} $k$-dimensional planes. 

In reality we did more than it seems, because we can now
construct the classifying space for \emph{any} matrix group $G$,
a closed subgroup of $\GL{k,\CC}$. We have a free action of
$\GL{k,\CC}$ on $V_{\infty,k}$. Thus we can induce a free action
of $G$ on $V_{\infty,k}$. So the classifying space is simply
$V_{\infty,k}/G$. 

\subsection*{EXERCISES}
\begin{xca}\label{xca:lec06:prob1}
Show that Moebius band can be represented as a space of fibration
having the circle $S^1$ as a base and the interval $I$ as a
fiber. Prove that this fibration is not trivial. Does it have a
section? Show that this fibration can be obtained pasting
together two trivial fibrations; find a transition function
(clutching function) taking values in the group $\ZZ_2$.
\end{xca}
\begin{xca}
The group $\ZZ_2$ acts on the circle $S^1$ (the non-trivial
transformation acts as reflection $x\to x$, $y\to-y$). Therefore
using the transition function of the
Problem \ref{xca:lec06:prob1} we can construct a fibration with
the fiber $S^1$ and the base $S^1$. Prove that the total space of
this fibration is a Klein bottle. Does this fibration have a
section?
\end{xca}
\begin{xca}\label{xca:lec06:prob3}
Let us consider principal fibrations with the group $\U{1}$,
total space $E$ and base $S^2$. Show that classes of these
fibrations are in one-to-one correspondence with integers. Let us
consider fibrations corresponding to integer numbers $n = 0, 1,
2$. Prove that we have $E = S^2\times S^1$ for $n = 0$, $E = S^3$
for $n = 1$, $E = \SO(3) = \RP^3$ for $n = 2$.

Hint. Consider Hopf fibration and tangent fibre bundle to
$S^2$. (To define Hopf fibration we consider the group $\U{1}$
acting on the $S^3$ defined by the equation $|z_{1}|^2 +
|z_{2}|^2 = 1$ in $\CC^2$; element $z\in\U{1}$ transforms $(z_1,
z_2)$ into $(zz_1, zz_2)$.) Calculate the transition function
explicitly or use exact homotopy sequence of fibration.
\end{xca}
\begin{xca}
Let us consider a principal fibration of Problem \ref{xca:lec06:prob3} corresponding to arbitrary integer
$n$. Calculate homotopy groups of total space of the fibration (for dimensions $\leq3$ you
should give an explicit answer, in dimensions $> 3$ you should express the homotopy group
of total space in terms of homotopy groups of $S^2$ ).
\end{xca}

\lecture[Characteristic Classes]
%%
%% lecture07.tex
%% 
%% Made by alex
%% Login   <alex@tomato>
%% 
%% Started on  Sat Oct  1 19:36:14 2011 alex
%% Last update Sat Oct  1 19:36:14 2011 alex
%%
We discussed the Riemann surface of
\begin{equation}
h(z)=\sqrt{(z-1)z(z+1)}.
\end{equation}
This is doodled thus:
\begin{center}
\includegraphics{img/lecture07.0}
\end{center}
Consider
\begin{equation}
\omega^{3}-\omega+z=0,
\end{equation}
we see that
\begin{equation}
z=\omega-\omega^{3},
\end{equation}
so we may write $z=z(\omega)$. We can invert this to find
$\omega=\omega(z)$. More generally we can suppose that
$p(\omega)=z$ and its inverse is $q(z)=\omega$. We can now
consider the Riemann surface of this function. Consider the graph
of this function (roughly doodled below to the left).

\begin{wrapfigure}{l}{3in}
\begin{center}
\includegraphics{img/lecture07.1}
\end{center}
\end{wrapfigure}
\noindent{}The inverse function has several values, so we get in this
complex analogue a Riemann surface. From the projection, which is
multivalued, we have the Riemann surface. In this case for
$h(z)$, the Riemann surface is homeomorphic to a torus.

The reader should make a mental note on the importance of branch
cuts in this method of constructing Riemann surfaces. Also note
that we are projecting onto \emph{Riemann spheres} which are
distinguished from the notion of \emph{Riemann surfaces}. The
Riemann sphere is $\CC$ as a sphere, obtained from Stereographic
projection. 

\begin{prop}[Fact from geometry]
An orientable, closed, compact surface is homeomorphic to a torus
(or more generally a sphere with $p$ handles).
\end{prop}

How do we find the genus of a Riemann surface? (I.e., how do we
find the value of $p$?) We have $n$ roots, we count how many
times we glue points together. We subtract the number of boundary
points of the cuts. So we have
\begin{equation}
\chi=2n-\#(\mbox{boundary points of the cuts})
\end{equation}
\marginpar{Euler Characteristic of Riemann Surface}which is precisely the Euler characteristic. The genus is
\begin{equation}
\frac{2-\chi}{2}=\mbox{genus}.
\end{equation}
This is for polynomials, however.

Riemann surfaces are defined for algebraic functions. Consider
the famous example of the logarithm function, it covers the
complex plane infinitely many times. When we consider the Riemann
surface, it's like an infinite Helix. This is the logarithmic
staircase. See Penrose's \emph{Road to Reality} for a good
doodle.

\subsection{Reflection Principle}

\begin{wrapfigure}[10]{r}{1.5in}
\vspace{-50pt}
\begin{center}
\includegraphics{img/lecture07.2}
\end{center}
\vspace{-40pt}
\end{wrapfigure}
We are nonetheless interested in extending functions. We have the
Reflection principle. We have some domain which contains on the
boundary part of the $x$ (real) axis.
We consider some function $f\colon\mathcal{U}\to\CC$, we extend
$f$ to another function $\widetilde{f}$ on $\mathcal{U}\cup I$
where $I$ is the real part of $\partial\mathcal{U}$, i.e.,
$I=\RR\cap\partial\mathcal{U}$. We demand that $\widetilde{f}$ be
continuous, and demand that $\widetilde{f}|_{I}$ be real. We
consider the complex conjugation of $\mathcal{U}$, doodled to the
right, which is $\bar{\mathcal{U}}=\{\bar{z}\lst
z\in\mathcal{U}\}$. We introduce a function $\widehat{f}$ such
that
\[
\widehat{f}(z) = \begin{cases}
f(z) & \mbox{if }z\in\mathcal{U}\\
\overline{f(\bar{z})} & \mbox{if }z\in\bar{\mathcal{U}}\\
\widetilde{f}(z) & \mbox{if }z\in I.
\end{cases}
\]
We have then $\mathcal{V}=\mathcal{U}\cup
I\cup\bar{\mathcal{U}}$, and we see that $\widehat{f}$ is
continuous on $\mathcal{V}$. We see that if $\widehat{f}$ is
analytic on $\mathcal{U}$, then $\widehat{f}$ is analytic on
$\bar{\mathcal{U}}$. 

Let $z=x+\I y\in\mathcal{U}$,
\begin{equation}
\widehat{f}(x+\I y)=u(x,y)+\I v(x,y)
\end{equation}
and for $\overline{z}\in\bar{\mathcal{U}}$ we see we have
\begin{equation}
\overline{f(\overline{z})}=u(x,-y)-\I v(x,-y)
\end{equation}
By the chain rule we see that $\overline{f(\overline{z})}$ satisfies
the Cauchy-Riemann equations.

\begin{wrapfigure}[5]{r}{1.25in}
\vspace{-30pt}
\begin{center}
\includegraphics{img/lecture07.3}
\end{center}
\vspace{-20pt}
\end{wrapfigure}
\medbreak\noindent\textbf{A General Statement.\enspace}
We will use the diagrams doodled on the right for reference.
Consider $\varphi\colon W\to\CC$ such that $\varphi$ is
continuous on $W$. If $\varphi|_{W_{1}}$ and
$\bar{\varphi}|_{W_{2}}$ are analytic, then $\varphi$ is analytic
on $W$.

\begin{lem}
Suppose we have in some domain $\mathcal{U}$ a continuous
function $$f\colon\mathcal{U}\to\CC.$$ Let $\gamma\colon
I\to\mathcal{U}$ be a continuous, closed curve, and for any such $\gamma$
that is simply connected, i.e., $$\int_{\gamma}f(z)\D z=0.$$ Then the
function is analytic.
\end{lem}

We see that the integral over a closed, simply connected path is
zero if its contained entirely in $W_{1}$ or $W_{2}$. We just
need to check for a path that crosses $\gamma$, we just treat it
by breaking it up into pieces.
\label{215c:lecture7}
\lecture
%%
%% lecture08.tex
%% 
%% Made by Alex Nelson
%% Login   <alex@tomato3>
%% 
%% Started on  Sun Feb 14 15:56:21 2010 Alex Nelson
%% Last update Sun Jun 20 11:59:29 2010 Alex Nelson
%%
%\noindent\textbf{Main Theorems}
\subsection{Main Theorems}
We have a group $G$ and we can construct the corresponding Lie
Algebra $\lie(G)$ constructed by examining the tangent space at
the identity $e=I\in G$:
\begin{equation}
\lie(G)=T_{e}G.
\end{equation}
We discussed obtaining the Lie algebra \emph{from} the Lie group,
and if we have a group morphism
\begin{equation}
\varphi\colon G\to G',
\end{equation}
we can construct a morphism of the corresponding Lie Algebras
\begin{equation}
\varphi_{*}\colon \lie(G)\to\lie(G').
\end{equation}
That is to say the following diagram commutes
\begin{equation}
\begin{diagram}
G              & \rTo & \lie(G)\\
\dTo>{\varphi} &      & \dTo>{\varphi_{*}}\\
G'             & \rTo & \lie(G')
\end{diagram}
\end{equation}
we would like to consider going the other way. That is given a
Lie algebra $\mathscr{G}$, we would like to construct a
corresponding group, and show that Lie algebra morphisms
$\mathscr{G}\to\mathscr{G}'$ generate group morphisms. But  the
group needs to be simply connected.

Consider $SU(2)/\Bbb{Z}_{2}=SO(3)$. However $SU(2)$ has the same
Lie algebra as $SO(3)$; when we identify the algebra
topologically as the sphere, this quotient identifies opposite
points as the same. This doesn't affect the Lie algebra.

If we drop the condition of being simply connected, a Lie algebra
may give rise to two different Lie groups. Simply connected
permits us to continuously deform one path to another. Every
closed curved is contractible if and only if the space is simply connected.

\begin{thm}
For each Lie algebra there exists a unique simply connected Lie group.
\end{thm}

If $g\colon[0,1]\to G$ is a differentiable curve on the group, we
may construct a curve
\begin{equation}
\gamma(t)=g(t)^{-1}\frac{\D g(t)}{\D t}=\frac{\D}{\D t}\log\big(g(t)\big)
\end{equation}
on the algebra. We have
\begin{equation}
\left.\frac{\D g(t)}{\D t}\right|_{t}\in T_{g(t)}G,
\end{equation}
which is not necessarily in the Lie algebra $T_{e}G$. However for
some $g\in G$ we have $g\cdot 1=g$, so this is a translation
which sends $1\mapsto g$. We have a map
\begin{equation}
g_{*}\colon T_{e}G\to T_{g}G,
\end{equation}
and so the right formula would be
\begin{equation}
\gamma(t)=\left(g_{*}(t)\right)^{-1}\frac{\D g(t)}{\D t}.
\end{equation}
But we will abuse notation and write
\begin{equation}
\gamma(t)=g(t)^{-1}\frac{\D g(t)}{\D t}.
\end{equation}
We have a correspondence between curves in the Lie Algebra and
curves in the Lie group. We obtain a system of differential
equations 
\begin{equation}
\frac{\D g(t)}{\D t}=g(t)\gamma(t)
\end{equation}
which has a unique solution for $g(0)=e$.\marginpar{We use $e$,
  $1$, $I$ for the identity interchanging notation without warning...}

\begin{wrapfigure}[10]{r}{6.25cm}  
%  \vspace{-20pt}
  \begin{center}
    \includegraphics{img/LieImg.1}
  \end{center}
  \vspace{-20pt}
\end{wrapfigure}

Consider a curve $g_{0}(t)$, $g_{1}(t)$ in the group, where
\begin{equation}
g_{0}(0)=g_{1}(0)=e,
\end{equation}
and we have
\begin{equation}
g_{0}(1)=g_{1}(1)=g.
\end{equation}
\begin{wrapfigure}[6]{l}{4cm}
  \vspace{-20pt}
  \begin{center}
    \includegraphics{img/LieImg.2}
  \end{center}
  \vspace{-20pt}
\end{wrapfigure}

\noindent\ignorespaces{}We assume that $G$ is simply connected
and we will deform the path, thus obtaining a family of paths
$g_{\tau}(t)=g(\tau,t)$ such that $g(0,t)=g_{0}(t)$ and
$g(1,t)=g_{1}(t)$. We can draw a diagram (seen on the left), we
have two variables to consider $\tau$ and $t$, which to
differentiate?  Both of them! Consider
\begin{equation}
\xi(\tau,t)=g(\tau,t)^{-1}\frac{\partial g(\tau,t)}{\partial t}
\end{equation}
and
\begin{equation}
\eta(\tau,t)=g(\tau,t)^{-1}\frac{\partial g(\tau,t)}{\partial\tau}.
\end{equation}
So what do we know? Well, the curves in the Lie algebra are
\begin{equation}
\gamma_{0}(t) = g_{0}(t)^{-1}\frac{\D g_{0}(t)}{\D t}=\xi(0,t)
\end{equation}
and
\begin{equation}
\gamma(1)(t) = g_{1}(t)^{-1}\frac{\D g_{1}(t)}{\D t}=\xi(1,t).
\end{equation}
We also know $g_{\tau}(0)=e=1$, and $g_{\tau}(1)=g$, for every
$\tau$. We see then that
\begin{equation}
\eta(\tau,0)=\eta(\tau,1)=0.
\end{equation}
We have
\begin{equation}
\partial_{t}g(\tau,t)=g(\tau,t)\xi(\tau,t)
\end{equation}
and
\begin{equation}
\partial_{\tau}g(\tau,t)=g(\tau,t)\eta(\tau,t).
\end{equation}
We deduce from
\begin{equation}
\partial_{\tau}\partial_{t}g(\tau,t)=\partial_{t}\partial_{\tau}g(\tau,t)
\end{equation}
that
\begin{equation}
(\partial_{\tau}g)\xi+g\partial_{\tau}\xi=(\partial_{t}g)\eta+g(\partial_{t}\eta),
\end{equation}
and by multiplying on the right by $g(\tau,t)^{-1}$ we have
\begin{equation}
\underbracket[0.5pt]{(g^{-1}\partial_{\tau}g)}_{=\eta}\xi+\partial_{\tau}\xi=\underbracket[0.5pt]{(g^{-1}\partial_{t}g)}_{=\xi}\eta+\partial_{t}\eta\quad\implies\quad
\partial_{t}\eta-\partial_{\tau}\xi=\eta\xi-\xi\eta.
\end{equation}
We have an equation of r the form
\begin{equation}
\frac{\partial\eta(\tau,t)}{\partial t}-\frac{\partial\xi(\tau,t)}{\partial\tau}
=[\xi(\tau,t),\eta(\tau,t)].
\end{equation}
Now, what should we do with this? In reality, we've done
everything already. What is our goal? We'd like to restore the
group knowing the Lie Algebra. We get points in the group by
considering equivalence classes of paths in the Lie Algebra.

\subsection{Exercises}
\subsubsection{Algebra \texorpdfstring{$\ClassicalGroup{D}_{n}$}{Dn}}

The Lie algebra $\ClassicalGroup{D}_{n}$ consists of $2n\times2n$ complex matrices
$L$ obeying
\begin{equation}
(FL)^{T}+FL=0
\end{equation}
where, in block form,
\begin{equation}
F=\begin{bmatrix}
0&1\\
1&0
\end{bmatrix}.
\end{equation}

\begin{exercise}
Check that $\ClassicalGroup{D}_{n}$ is isomorphic to the 
complexification of the Lie algebra of the orthogonal 
group $\ORTH{2n}$.
\end{exercise}
\begin{exercise}
Check that the matrices
\begin{equation}
e_{ij}:=\begin{bmatrix}E_{ij}&0\\
0&-E_{ji}
\end{bmatrix}
\end{equation}
together with the matrices
\begin{equation}
f_{pq}:=\begin{bmatrix}0&E_{pq}-E_{qp}\\
0&0
\end{bmatrix},\qquad
g_{pq}:=\begin{bmatrix}0&0\\
E_{pq}-E_{qp}&0
\end{bmatrix}
\end{equation}
form a basis of $\ClassicalGroup{D}_{n}$. 

Here $i,j=1,\ldots,n$, $1\leq p<q\leq n$, and $E_{i,j}$ has only
one nonzero entry that is equal to unity located in the $i^{th}$
row and $j^{th}$ column.
\end{exercise}
\begin{exercise}
Check that the subalgebra $\frak{h}$ of all matrices of the form
\begin{equation}
\begin{bmatrix}
A&0\\
0&-A
\end{bmatrix}
\end{equation}
(where $A$ is a diagonal matrix) is a maximal commutative
subalgebra, and prove that there exists a basis of $\ClassicalGroup{D}_{n}$
consisting of eigenvectors for elements of $\frak{h}$ acting on
$\ClassicalGroup{D}_{n}$ by means of adjoint representation. (This means that
$\frak{h}$ is a Cartan subalgebra of $\ClassicalGroup{D}_{n}$.)
\end{exercise}
\begin{exercise}
Check that $e_{i}=e_{i,i+1}$ for $i=1,...,n-1$ and
$e_{n}=f_{n-1,n}$; $f_{i}=e_{i+1,i}$ for $i=1,...,n-1$ and
$f_{n}=g_{n-1,n}$ form a system of multiplicative generators of
$\ClassicalGroup{D}_{n}$. Prove the relations
\begin{subequations}
\begin{align}
[e_{i},f_{j}]&=\delta_{ij}h_{i}\\
[h_{i},h_{j}]&=0\\
[h_{i},e_{j}]&=a_{ij}e_{j}\\
[h_{i},f_{j}]&=-a_{ij}f_{j}\\
({\rm ad}e_{i})^{1-a_{ij}}e_{j}&=0\qquad\mbox{when $i\not=j$}\\
({\rm ad}f_{i})^{1-a_{ij}}f_{j}&=0\qquad\mbox{when $i\not=j$}
\end{align}
\end{subequations}
We use here the notation $(\ad x)$ for the operator transforming $y$ into $[x,y]$.
\end{exercise}

\subsubsection{Algebra \texorpdfstring{$\ClassicalGroup{C}_{n}$}{Cn}}
Consider the Lie algebra $\ClassicalGroup{C}_{n}$ consisting of $2n\times2n$
complex matrices obeying
\begin{equation}
(FL)^{T}+FL=0
\end{equation}
where
\begin{equation}
F=\begin{bmatrix}0&1\\
-1&0
\end{bmatrix}.
\end{equation}

\begin{exercise}
Check that $\ClassicalGroup{C}_{n}$ is isomorphic to the complexification of the
Lie algebra of the compact group $\Sp{2n}\cap \U{2n}$ where
$\Sp{2n}$ stands for the group of linear transformations of
$\CC^{2n}$ preserving non-degenerate anti-symmetric bilinear
form and $\U{2n}$ denotes unitary group.
\end{exercise}
\begin{exercise}
Check that the matrices
\begin{subequations}
\begin{align}
e_{ij} &= \begin{bmatrix}E_{ij}&0\\0&-E_{ji}
\end{bmatrix}\\
f_{pq}&= \begin{bmatrix}0&E_{pq}+E_{qp}\\0&0
\end{bmatrix}\\
g_{pq}&=\begin{bmatrix}0&0\\E_{pq}+E_{qp}&0
\end{bmatrix}
\end{align}
\end{subequations}
form a basis of $\ClassicalGroup{C}_{n}$, where $i,j=1,...,n$ and $1\leq p\leq
q\leq n$.
\end{exercise}
\begin{exercise}
Check that the subalgebra $\frak{h}$ of all matrices of the form
\begin{equation}
\begin{bmatrix}
A&0\\
0&-A
\end{bmatrix}
\end{equation}
where $A$ is a diagonal matrix, is a maximal commutative
subalgebra. Prove there exists a basis of $\ClassicalGroup{C}_{n}$ consisting of
eigenvectors for elements of $\frak{h}$ acting on $\ClassicalGroup{C}_{n}$ by
means of adjoint representation.
\end{exercise}
\begin{exercise}
Check that $e_{i}=e_{i,i+1}$ ($i=1,...,n-1$) and $e_{n}=f_{n,n}$;
$f_{i}=e_{i+1,i}$ ($i=1,...,n-1$) and $f_{n}=g_{n,n}$ form a
system of generators of $\ClassicalGroup{C}_{n}$. Prove
\begin{subequations}
\begin{align}
[e_{i},f_{j}]&=\delta_{ij}h_{i},\\
[h_{i},h_{j}]&=0\\
[h_{i},e_{j}]&=a_{ij}e_{j},\\
[h_{i},f_{j}]&=-a_{ij}f_{j},\\
({\rm ad}e_{i})^{1-a_{ij}}e_{j}&=0,\qquad i\not=j \\
({\rm ad}f_{i})^{1-a_{ij}}f_{j}&=0,\qquad i\not=j
\end{align}
\end{subequations}
\end{exercise}
\subsubsection{Algebra \texorpdfstring{$\ClassicalGroup{B}_{n}$}{Bn}}

The algebra $\ClassicalGroup{B}_{n}$ consists of $(2n+1)\times(2n+1)$ complex
matrices obeying
\begin{equation}
L^{T}F+FL=0
\end{equation}
where
\begin{equation}
F=\begin{bmatrix}1&0&0\\
0&0&I_{n}\\
0&I_{n}&0
\end{bmatrix}
\end{equation}
$I_{n}$ is the $n\times n$ identity matrix, and we have written $F$ in block form.

\begin{exercise}
Show that $\ClassicalGroup{B}_{n}$ is isomorphic to the complexified Lie algebra of $\ORTH{2n+1}$.
\end{exercise}
\begin{exercise}
Check that the subalgebra $\frak{h}$ of all matrices of the form
\begin{equation}
\begin{bmatrix}0&0&0\\
0&A&0\\
0&0&-A
\end{bmatrix}
\end{equation}
(where $A$ is a diagonal matrix) is a maximal Abelian subalgebra,
and prove there is a basis of $\ClassicalGroup{B}_{n}$ consisting of eigenvectors
for elements of $\frak{h}$ acting on $\ClassicalGroup{B}_{n}$ by the adjoint
representation.
\end{exercise}
\begin{exercise}
Find a system $e_{i}$, $f_{j}$ of multiplicative generators of
$\ClassicalGroup{B}_{n}$ obeying
\begin{subequations}
\begin{align}
[e_{i},f_{j}]&=\delta_{ij}h_{i}\\
[h_{i},h_{j}]&=0\\
[h_{i},e_{j}]&=a_{ij}e_{j}\\
[h_{i},f_{j}]&=-a_{ij}f_{j}\\
({\rm ad}e_{i})^{1-a_{ij}}e_{j}&=0,\qquad i\not=j\\
({\rm ad}f_{i})^{1-a_{ij}}f_{j}&=0,\qquad i\not=j
\end{align}
\end{subequations}
for ``some'' matrix $a_{ij}$.
\end{exercise}
\begin{exercise}
Describe the roots and root vectors of $\ClassicalGroup{A}_{n}$, $\ClassicalGroup{B}_{n}$, $\ClassicalGroup{C}_{n}$, $\ClassicalGroup{D}_{n}$.
\end{exercise}

\lecture
%%
%% lecture09.tex
%% 
%% Made by Alex Nelson
%% Login   <alex@tomato3>
%% 
%% Started on  Mon Jun 14 10:11:45 2010 Alex Nelson
%% Last update Mon Jun 14 15:20:28 2010 Alex Nelson
%%

Let $G$ be a Lie group, consider $\lie(G)$ its Lie algebra. Then
there is a corresponddence between curves in the group and curves
in the Lie algebra. So if we have two curves in the Lie algebra,
we have two curves in the Lie group, then for simply connected
groups we may deform two curves $g_{1}(t)$, $g_{2}(t)$ with
\begin{equation}
g_{1}(0)=g_{2}(0)=g_{0},\quad\mbox{and}\quad
g_{1}(1)=g_{2}(1)=g_{1}
\end{equation}
by introducing a family of curves $g_{\tau}(t)$ which has a
corresponding family of curves in the Lie algebra. We have
\begin{subequations}
\begin{align}
\xi(\tau,t) &= g_{\tau}(t)^{-1}\frac{dg_{\tau}(t)}{dt}\\
\eta(\tau,t) &= g_{\tau}(t)^{-1}\frac{dg_{\tau}(t)}{d\tau}
\end{align}
\end{subequations}
and we have the relation
\begin{equation}
\frac{\partial\eta}{\partial t}-\frac{\partial\xi}{\partial t}=[\xi,\eta]
\end{equation}
which occurs in the Lie algebra. Observe that 
\begin{equation}
\xi(t,0)=\gamma_{0}(t),\quad\xi(t,1)=\gamma_{1}(t),\quad\eta(0,\tau)=\eta(1,\tau)=0.
\end{equation}
So given these conditions that, for
\begin{equation}
\frac{\partial\eta(\tau,t)}{\partial t}-\frac{\partial\xi(\tau,t)}{\partial\tau}
=[\xi(\tau,t),\eta(\tau,t)]
\end{equation}
with boundary conditions
\begin{subequations}
\begin{align}
\xi(t,0)=\gamma_{0}(t)\quad\mbox{and}\quad\xi(t,1)=\gamma_{1}(t)\\
\eta(1,\tau)=\eta(0,\tau)=0
\end{align}
\end{subequations}
can we get information induced in the group? We have
\begin{equation}\label{eq:lec09:diffEqXi}
\xi(t,\tau)=g(t,\tau)^{-1}\frac{\partial g(t,\tau)}{\partial t},
\end{equation}
where $g(0,\tau)=1$. We can restore $g(t,\tau)$ since there is a
unique solution to eq \eqref{eq:lec09:diffEqXi}.

%% So, more or less, $\widetilde{\xi}\to g\to(\xi,\eta)$. We can
%% find new $\xi$, $\eta$ satisfying the above. So
%% $\widetilde{\xi}=\xi$ by construction.

Suppose we have $\lie(G)\to\lie(G')$ be a Lie algebra morphism;
how can we induce a Lie group morphism? Well, how we do it makes
heavy use of this curve voodoo. The basic correspondence we have
is that ``points in the group'' corresponds to ``curves in the
Lie Algebra'', and ``multiplication in the group'' corresponds to
``concatenation of paths in the Lie algebra.'' Group curve
concatenation can be performed, for 
\begin{equation}
g_{1}:[0,b]\to G
\end{equation}
and
\begin{equation}
g_{2}:[b,a]\to G,
\end{equation}
as 
\begin{equation}
g(t)=\begin{cases} g_{1}(t) & t\in[0,b]\\
g_{2}(b)^{-1}g_{2}(t) & t\in[b,a].
\end{cases}
\end{equation}
If the paths $g_{1}(t)$, $g_{2}(t)$ are not loops,
i.e. $g_{1}(0)\not=g_{1}(b)$ and $g_{2}(b)\not=g_{2}(a)$, then
\begin{equation}
g(t)=\begin{cases} g_{1}(t) & t\in[0,b]\\
g_{1}(b)g_{2}(b)^{-1}g_{2}(t) & t\in[b,a].
\end{cases}
\end{equation}
We see that $g(b)$ is in the first case equal to $g_{1}(b)$, and
in the second case
\begin{equation}
g(b) = g_{1}(b)g_{2}(b)^{-1}g_{2}(b) = g_{1}(b).
\end{equation}
Thus the two cases agree on the overlap.

The corresponding curve in the Lie algebra is
\begin{equation}
\gamma(t) = \begin{cases}\displaystyle
g_{1}(t)^{-1}\frac{dg_{1}(t)}{dt} & t\in[0,b]\\
\displaystyle (g_{2}(b)^{-1}g_{2}(t))^{-1}\left(g_{2}(b)^{-1}\frac{dg_{2}(t)}{dt}\right)
& t\in[b,a]
\end{cases}
\end{equation}
up to a constant (i.e. $g_{1}(b)$) in the second case. It doesn't
play a significant role, as it is factored out. We end up with
\begin{equation}
\gamma(t) = \begin{cases}\displaystyle
g_{1}(t)^{-1}\frac{dg_{1}(t)}{dt} & t\in[0,b]\\
\displaystyle g_{2}(t)^{-1}\frac{dg_{2}(t)}{dt} & t\in[b,a]
\end{cases}
\end{equation}
We will consider the construction of Lie groups from Lie algebra
next time...

We proved there exists a one-to-one correspondence between simply
connected Lie groups and finite dimensional Lie algebras. If we
have a discrete normal subgroup $N\subset G$, then the Lie
algebra of $G/N\iso\lie(G)$. This is because there is a
neighborhood $\mathcal{U}$ of $1\in G$ such that $\mathcal{U}\cap
N=\{1\}$. 

\begin{thm}
If $G$ is simply connected, and $\lie(G)\iso\lie(G')$, then
$G'\iso G/N$ where $N$ is a discrete normal subgroup of $G$.
\end{thm}
\begin{ex}
$\Bbb{R}$ equipped with addition has trivial commutators in Lie
  algebra, but $\lie\big(U(1)\big)\iso\lie(\Bbb{R})$ so
  $U(1)\iso\Bbb{R}/\Bbb{Z}$. 
\end{ex}

\lecture
%%
%% lecture10.tex
%% 
%% Made by alex
%% Login   <alex@tomato>
%% 
%% Started on  Wed Feb 29 10:50:26 2012 alex
%% Last update Wed Feb 29 10:50:26 2012 alex
%%

We want differentiate quantities. Why not just differentiate ``in
the obvious way'' (i.e., take their derivatives!)?
The problem with just taking derivatives is we get under a change
of coordinates
\begin{equation}
\partial_{\mu}v^{\nu}\not=\partial_{\mu'}v^{\nu'}.
\end{equation}
What to do? Well, we should recall that a\marginpar{$p$-form} \define{$p$-Form}
is a totally antisymmetric $(0,p)$-tensor
\begin{equation}
\omega=\omega_{\mu_{1}\dots\mu_{p}}\D x^{\mu_{1}}\otimes\dots\otimes\D x^{\mu_{p}}
\end{equation}
What do we mean by totally antisymmetric? Well, it obeys
\begin{equation}
\omega_{\mu_{1}\dots\mu_{k}\mu_{k+1}\dots\mu_{p}}
=-\omega_{\mu_{1}\dots\mu_{k+1}\mu_{k}\dots\mu_{p}}.
\end{equation}
For a 2-form, the components look like
\begin{equation}
\partial_{\mu}\omega_{\nu}-\partial_{\nu}\omega_{\mu}
\end{equation}
for example.

\subsection{Exterior Calculus}
We will use the notation
\begin{equation}
A_{[\mu_{1}\dots\mu_{p}]}=\frac{1}{p!}(-1)^{\pi}A_{\pi(\mu_{1})\dots\pi(\mu_{p})}
\end{equation}
where $\pi$ is a permutation of the indices. Given a $p$-form
$A$, and a $q$-form $B$, the exterior product\marginpar{Exterior Product} is defined to be
\begin{equation}
(A\wedge B)_{\mu_{1}\dots\mu_{p+q}}
=\left(\frac{(p+q)!}{p!q!}\right)A_{[\mu_{1}\dots\mu_{p}]}B_{[\mu_{p+1}\dots\mu_{p+q}]}.
\end{equation}
We also have an exterior derivative\marginpar{Exterior Derivative}
\begin{equation}
(\D A)_{\mu_{1}\dots\mu_{p+1}}=\partial_{[\mu_{1}}A_{\mu_{2}\dots\mu_{p+1}]}.
\end{equation}
Consider concrete cases. If $A$ is a 1-form, then
\begin{equation}
\begin{split}
(\D A)_{\mu\nu}
&=\partial_{[\mu}A_{\nu]}\\
&=\frac{1}{2}(\partial_{\mu}A_{\nu}-\partial_{\nu}A_{\mu})
\end{split}
\end{equation}
Let $A$ be a $p$-form and $B$ be a $q$-form, then
\begin{equation}
\D(A\wedge B)=(\D A)\wedge B+(-1)^{p}A\wedge(\D B).
\end{equation}
Equivalently, in this formulation, we have for a function $f$
\begin{equation}
\D f=\vec{\nabla}f
\end{equation}
and
\begin{equation}
\D^{2}f=0.
\end{equation}
Whenever we have a $p$-form $A$ such that
\begin{equation}
\D A=0
\end{equation}
we call it a \define{Closed Form}\marginpar{Closed form: $\D A=0$\\ Exact Form $B=\D C$}.
We have an \define{Exact Form} be a $p$-form $B$ such that it is
of the form
\begin{equation}
B=\D C
\end{equation}
where $C$ is a $(p-1)$-form. Not all closed forms are exact.
\begin{ex}[Closed Inexact Form]
A closed but inexact form on a circle is $\D\theta$.
\end{ex}

\subsection{Differentiating Tangent Vectors}
Consider an arbitrary tangent vector
\begin{equation}
v=v^{\mu}\partial_{\mu}=v^{a}e_{a}.
\end{equation}
Lets consider what differentiation would look like in this
approach, we have
\begin{equation}
\partial_{\rho}v\;\mbox{``=''}\;(\partial_{\rho}v^{a})e_{a}+v^{a}(\partial_{\rho}e_{a}).
\end{equation}
But this second term is ambiguous? What should we have? Well, we
should write
\begin{equation}
\partial_{\rho}e_{a}=e_{b}{\Gamma^{b}}_{\rho a}-e_{b}{{\Gamma_{\rho}}^{b}}_{a}
\end{equation}
where ${{\Gamma_{\rho}}^{b}}_{a}$ is called the connection's components.
In general, we may say absolutely nothing about the connection as
it specifies the manifold. We can now write
\begin{equation}
\begin{split}
\partial_{\rho}v
&\mbox{``=''}(\partial_{\rho}v^{b})e_{b}+e_{b}{{\Gamma_{\rho}}^{b}}_{a}v^{a}\\
&\mbox{``=''}\underbracket[0.5pt]{(\partial_{\rho}v^{b}+{{\Gamma_{\rho}}^{b}}_{a}v^{a})}_{\nabla_{\rho}v^{b}}e_{b}
\end{split}
\end{equation}
where $\nabla_{\rho}v^{b}$ is the \define{Covariant Derivative}. 
The intuition is
\begin{equation}
\begin{pmatrix}
\mbox{Covariant}\\
\mbox{Derivative}
\end{pmatrix}=\begin{pmatrix}
\mbox{Derivative in}\\
\mbox{Flat Space}
\end{pmatrix}
+\begin{pmatrix}
\mbox{Corrections to}\\
\mbox{stay on the}\\\mbox{manifold}
\end{pmatrix}
\end{equation}
where the connection components are precisely these correction
terms. In a coordinate basis, the components becomes
${{\Gamma_{\rho}}^{\nu}}_{\mu}$ which are called the
\define{Christoffel Symbols}.

\lecture
%%
%% lecture11.tex
%% 
%% Made by Alex Nelson
%% Login   <alex@tomato3>
%% 
%% Started on  Mon Jun 14 15:56:24 2010 Alex Nelson
%% Last update Mon Jun 14 16:29:24 2010 Alex Nelson
%%

We consider representations of $\frak{sl}(2)=A_{1}$. We analyzed
completely the finite dimensional representations; the only place
where finite dimensions were used was in proving the existence of
the highest weight vector. We reasoned $h$ has eigenvectors. Then
we applied $e$ to the eigenvectors of $h$, which produced a
different eigenvector or zero.

We had the eigenvector
\begin{equation}
h\vec{v}=\lambda\vec{v},
\end{equation}
then
\begin{equation}
h(e^{k}\vec{v})=(\lambda+2k)e^{k}\vec{v},
\end{equation}
but in the finite dimensional case we get at some moment
\begin{equation}
e^{n}\vec{v}=0
\end{equation}
for some $n$. So in finite dimensions, such a vector always
exists. In the infinite dimensional case, we will assume a
highest weight vector exists. Then we will describe all
irreducible representations, we did this basically. Let $\vec{v}$
be such that
\begin{equation}
e\vec{v}=0,
\end{equation}
then let
\begin{equation}
\vec{v}_{k}=f^{k}\vec{v},
\end{equation}
so
\begin{equation}
h\vec{v}_{k}=(m-2k)\vec{v}_{k}
\end{equation}
and we have ``ladder relations''
\begin{equation}
f\vec{v}_{k}=\vec{v}_{k+1}
\end{equation}
and
\begin{equation}
e\vec{v}_{k}=\gamma_{k}\vec{v}_{k-1},
\end{equation}
where $\gamma_{k}$ is some coefficient which requires solving a
recursive formula. We have
\begin{equation}
\gamma_{k}=k(m-k+1).
\end{equation}
Lets prove this is an irreducible representation. What does this
mean? It doesn't have any nontrivial subrepresentations. Suppose
we do have some nontrivial subrepresentation, it should contain
at least one vector. Suppose this one vector is of the form
\begin{equation}
\sum c_{k}\vec{v}_{k}=\vec{w}.
\end{equation}
Lets apply to this vector $e^{s}\vec{w}$, what happens? It is pretty
clear it should be $\vec{w}=\vec{v}_{s'}$ where $s'$ is some
index, this is due to $h$ having an eigenvector in any representation.

We can now apply $e$ and $f$ to $\vec{w}$, we end up recovering
\begin{equation}
e^{s'}\vec{w}\propto\vec{v}_{0}.
\end{equation}
We made the mistake that
\begin{equation}
e\vec{v}_{k}=k(m-k+1)\vec{v}_{k-1}
\end{equation}
exists, i.e. $m-k+1\not=0$. If $m\not\in\Bbb{Z}$, then this is an
irreducible representation. But if $m\in\Bbb{Z}$, more
specifically
\begin{equation}
k=m+1,
\end{equation}
then
\begin{equation}
e\vec{v}_{m+1}=0.
\end{equation}
So we get an irreducible subrepresentation spanned by
$\vec{v}_{0}$, $\vec{v}_{1}$, ..., $\vec{v}_{m}$. So for each
$m\in\Bbb{N}$, we have on irreducible representation of dimension
$m+1$, so we have $m+1=1,2,3,...$.

Now lets discuss the group $SU(2)$, recall $\frak{su}(2)$
consists of traceless anti-Hermitian matrices; recall unitary
matrices satisfy
\begin{equation}
A^{\dagger}A=I.
\end{equation}
The rows and columns form an orthonormal basis:
\begin{equation}
A = \begin{bmatrix}a&b\\c&d\end{bmatrix}
\end{equation}
so
\begin{equation}
|a|^{2}+|b|^{2}=1\quad\mbox{and}\quad|c|^{2}+|d|^{2}=1\quad\mbox{and}\quad ad-bc=1.
\end{equation}
If we know $a$ and $b$, we can deduce that
\begin{equation}
\begin{bmatrix}
a&b\\c&d
\end{bmatrix}=
\begin{bmatrix}
a&b\\
-\overline{b}&\overline{a}
\end{bmatrix}.
\end{equation}
We can consider the topology here, if
\begin{equation}
a=a_{0}+ia_{1}\quad\mbox{and}\quad b=b_{0}+ib_{1}
\end{equation}
then we are working with a 4-dimensional sphere. We can say that
topologically SU(2) is compact and simply connected. The
representations of SU(2) may be identified by representations of
its Lie algebra. We recall
\begin{equation}
\Bbb{C}\lie\big(SU(2)\big)=\frak{sl}(2).
\end{equation}
We may describe the representations directly.

\begin{rmk}
$\frak{su}(2)$ has a scalar representation, i.e. the most boring
  representation imaginable (everything is represented by the
  unit matrix). It's trivial, and has dimension 1.
\end{rmk}

Now we have the vector or ``fundamental'' representation. It is
2-dimensional. Every matrix is repersented by itself. Let
$V=\Bbb{C}^{2}$ be the representation of SU(2). For this special
situation, we can work with polynomials of $x,y\in\Bbb{C}$.

That is to say, if $\{(x,y)\}=V$, we may take the space of
functions on $V$. We introduce
\begin{equation}
\psi_{g}\colon\varphi(z)\mapsto\varphi(g^{-1}z)
\end{equation}
which ``deforms'' $\varphi(z)$ into $\varphi(g^{-1}z)$. If we
have
\begin{equation}
\psi_{h}\colon\varphi(z)\mapsto\varphi(h^{-1}z),
\end{equation}
then we demand
\begin{equation}
\varphi_{h}\circ\varphi_{g}\colon\varphi(z)\mapsto\varphi\big((hg)^{-1}z\big).
\end{equation}
Functions of $V$ are contravariant functors.

This is a reducible representation, since we may restrict our
focus to polynomials over $V$. Is the space of polynomials an
irreducible representation? No! Why? We can have the subspace of
homogeneous polynomials of degree $m$. So it would be irreducible
and spanned by $x^{m}$, $x^{m-1}y$, ..., $xy^{m-1}$, $y^{m}$
which is of dimensions $m+1$. We can deduce the representation of
the Lie algebra. Observe for us in $\frak{su}(2)$,
\begin{equation}
h=\begin{bmatrix}
1 & 0\\
0 & -1
\end{bmatrix}
\end{equation}
so it corresponds to
\begin{equation}
\begin{bmatrix}
u & 0\\
0 & u^{-1}
\end{bmatrix}\in{\rm SU}(2).
\end{equation}
How $h$ acts on the basis is that
\begin{equation}
h\colon x\mapsto ux,\quad y\mapsto u^{-1}y.
\end{equation}
So in effect,
\begin{equation}
x^{m-k}y^{k}\mapsto u^{m-2k}x^{m-k}y^{k},
\end{equation}
with the Lie algebra $u=1+\varepsilon$ where
$|\varepsilon|^{2}\ll1$. So 
\begin{equation}
u^{m}=1+m\varepsilon
\end{equation}
and so on.

\lecture[Obstruction Theory]
%%
%% lecture12.tex
%% 
%% Made by alex
%% Login   <alex@tomato>
%% 
%% Started on  Wed Sep 28 14:11:11 2011 alex
%% Last update Wed Sep 28 14:11:11 2011 alex
%%
\index{Section!Obstructions to|(}
Today we will discuss obstruction theory\index{Obstruction Theory}. It is something that
appears in very different situations. Maybe the simplest
statement is when working with a cell complex, suppose $B$ is a
cell complex, construct a section over $B$ step-by-step taking
into account the cellular skeleton\index{Section!Extension and Obstruction|textbf}
decomposition\marginpar{Obstruction theory studies ``obstructions'' to step-by-step construction of sections}
\begin{equation}
B=\bigcup B^{k}
\end{equation}
where $B^{k}$ is the $k$-skeleton. We construct a section over
$B^{k}$, then try to extend it to $B^{k+1}$. Sometimes this is
possible; other times we get an obstruction. The most important
is the first obstruction (others afterwards are more complicated).

We may also take two sections $f$, $g$ and ask if they're
homotopic, i.e.
\begin{equation}
f\homotopic g?
\end{equation}
We may try to classify sections up to
homotopy.\index{Characteristic Class!related to Obstructions} All
characteristic classes may be obtained by obstructions.


\begin{wrapfigure}{r}{1.5in}
  \vspace{-20pt}
  \begin{center}
    \includegraphics{img/lecture12.0}
  \end{center}
  \vspace{-20pt}
\end{wrapfigure}
\marginpar{See \S\S25.6, 35, 38~\cite{steenrod}, or\break Ch.\ 12
  \cite{milnor} for details\break on obstructions}We have a fibre bundle $(E,F,B,p)$ and suppose $B$ is connected
(although that's not important). We have $B^{0}$ consists of
points and over every point we may take any point of the
fibre. This is a map $B^{0}\to E$ which is a section. Then we
would like to go to $B^{1}$, which is a little less trivial. But
if the fibre is connected, we may lift our section to $B^{1}\to E$.
Let us go to the inductive step.
Suppose we have constructed a section over
$B^{k}\xrightarrow{\sigma}E$. Then we can say that we would like
to extend this section to $B^{k+1}$, we may assume that 
\begin{equation}
B^{k+1}\setminus B^{k}=\bigcup\sigma^{k+1}_{i}
\end{equation}
but it would probably be clearer if we do the following trick: if
we have $B^{k+1}$ and we remove small balls in each
$(k+1)$-cells, then we obtain a deformation-retraction to
$B^{k}$. Thus we may extend our section to a section over
\begin{equation}
\sigma\colon B^{k+1}\setminus\bigcup(\mbox{small balls})\to E.
\end{equation}
The next step, look at the small balls. We have over our small
ball $\bar{D}^{k+1}$ the direct product $\bar{D}^{k+1}\times F$,
and it contains $S^{k}\times F$. The section we have is defined
on $S^{k}\times F$, the question is how to extend this to a
section over $\bar{D}^{k+1}$.

Suppose $F$ is simply connected. Then the maps of the sphere
\begin{equation}
\homotopyClass(S^k,F)=\pi_{k}(F),
\end{equation}
since we are working with a spheroid after all! So what do we
get? For each $(k+1)$-cell, $\sigma^{k+1}$, we obtain
$\xi(\sigma^{k})\in\pi_{k}(F_{b})$ where $b$ is the center of the
ball. The homotopy groups $\pi_{k}(F)$ form a local system over
$B$. We may say that $\xi(\sigma^{k+1})$ is a $(k+1)$-dimensional
cochain on $B$ with coefficients in a local system $\pi_{k}(F)$.

\begin{thm}
If $\pi_{k}(F)=0$, then any section over $B^{k}$ can be extended
to a section over $B^{k+1}$.
\end{thm}
This is obvious.
\begin{thm}
Suppose $\pi_{1}(F)=\dots=\pi_{n-1}(F)=0$, and
$\pi_{n}(F)\not=0$. Then we can extend the section to $B^{k}$
(the $k$-dimensional skeleton) without obstructions $\sigma\colon
B^{n}\to F$ and we have an obstruction $\xi(\sigma)$ that is a
$(n+1)$-dimensional cochain with values in $\pi_{n}(F)$ (i.e., in
local system).
\end{thm}
This is the first obstruction. Now what can we say? Two
things. First this cochain is a cocycle, i.e., its coboundary is
zero. Second this cocycle---if it is zero---is no obstruction,
but if this cocycle is \emph{homologous to zero}, then we can
change $\sigma$ in $B^{n}$ in such a way that the obstruction
becomes equal to zero.

So obstructions lie in $H^{n+1}\bigl(B,\pi_{n}(F)\bigr)$. This is
the place where we have obstructions; if this group disappears,
our obstruction disappears. We know that for local coefficient
systems, if $B$ is simply connected, then
$H^{n+1}\bigl(B,\pi_{n}(F)\bigr)=0$. But this is the notion of
obstructions.

\medbreak
\refstepcounter{thm}%
\noindent\textbf{Example \thethm} (Vector Fields on Sphere)\textbf{.}\quad
For $S^n$, a vector field is a section. For any smooth manifold
$M$, we have for each $x\in M$ the tangent space $T_{x}M$. A
vector field, therefore, is a section of a vector bundle. There
are no obstructions here.

What about nonvanishing vector fields on $S^n$? It is a section
of a bundle with a fiber $\RR^{n}\setminus\{0\}\homotopic S^{n-1}$.
This is true for any $n$-dimensional smooth manifold $M$, the
nonvanishing fields on it as a section on a bundle over $M$ with
fiber $S^{n-1}$. Why? Because what we are doing is considering
the \emph{unit vectors}. Two vectors are homotopic if and only if
they are ``pointing in the same direction'' but of different
magnitudes. Thus we are considering the collection of unit
vectors, which forms a sphere. So we are looking at sections of
this bundle
\begin{equation}
S^{n-1}\into V_{n+1,2}\to S^{n}.
\end{equation}
Observe the first homotoy group of the fiber is
$\pi_{n-1}(S^{n-1})=\ZZ$ and we have a section over $B^{n-1}$. We
want to extend this section to $B^{n}$. This is the first
obstruction, $S^{n-1}$ doesn't have any higher cells, so things
are simple.

\begin{wrapfigure}[13]{r}{2in}
  \vspace{-20pt}
    \includegraphics{img/lecture12.1}
  \vspace{-20pt}
\end{wrapfigure}

Consider the case for $S^{2}$, as doodled, where we have a unit
vector pointing towards the north pole. But in the arctic
regions, we need to change the vector field. We have for the
north pole
\begin{subequations}
\begin{equation}
\sigma\colon x\mapsto (-1)^{n-1}x,
\end{equation}
and for the South pole
\begin{equation}
\sigma\colon x\mapsto x.
\end{equation}
\end{subequations}
For $n$ even, we have an obstruction, and it is equal to 2. For
$n$ odd, we have a nonzero cochain which gives a class homologous
to zero\dots and we have no obstruction!
\medbreak

When we compute the sign, we should have taken into account
orientation of cells.

\lecture
%%
%% lecture13.tex
%% 
%% Made by alex
%% Login   <alex@tomato>
%% 
%% Started on  Mon Dec 26 21:23:33 2011 alex
%% Last update Mon Dec 26 21:23:33 2011 alex
%%

\subsection{Projective Spaces}
\index{Space!Projective|(}\index{Projective Space|(}
Recall the notion of a projective space $\PP^n$. Consider the
$(n+1)$-dimensional vector space $\FF^{n+1}$ over $\FF$. Consider
all lines in $\FF^{n+1}$ that contain the origin. We need to know
only one point---then we know the line. (Why?) If $x\in\FF^{n+1}$
and $x\not=0$, then $\lambda x$ describes the lines, for
arbitrary $\lambda\in\FF$. The set of all such lines is the
projective space over $\FF$, denoted $\FP^{n}$. We can take
\begin{equation}
(\FF^{n+1}-0)/(x\sim \alpha x)\eqdef\FP^{n}
\end{equation}
This is another description of it.

A point in projective space may be described
by \define{Homogeneous Coordinates}\index{Projective Space!Homogeneous Coordinates}\index{Homogeneous Coordinates}
denoted
\begin{equation}
(x_0:x_1:\cdots:x_n)\sim(\alpha x_0:\alpha x_1:\cdots:\alpha x_n).
\end{equation}
(In physics, we have a similar picture where wave functions are
defined up to a constant factor.) So this projective space
contains an $n$-dimensional vector space
\begin{equation}
\FF^n\propersubset\FP^n.
\end{equation}
How? Consider $x_0\not=0$. Every point with this condition is
equivalent to
\begin{equation}
(x_0:x_1:\cdots:x_n) = \left(1:\frac{x_1}{x_0}:\cdots:\frac{x_n}{x_0}\right).
\end{equation}
We take
\begin{equation}
y_i\eqdef x_i/x_0,
\end{equation}
then
\begin{equation}
(x_0:x_1:\cdots:x_n) = \left(1:y_1:\cdots:y_n\right).
\end{equation}So $\FF^n$ is sitting in projective space. The next
thing we can do is take our projective space and delete this
$\FF^n$. What do we get? Well, it's quite simple: we get all the
points where $x_0=0$. We thus get the picture that
\begin{equation}
\FP^n-\FF^n=\FP^{n-1}.
\end{equation}
So if you like, we may say that
\begin{equation}
\FP^{n}=\FF^n\sqcup\FP^{n-1}
\end{equation}
disjoint union of topological spaces.

Now we go to topology and consider two cases: $\FF=\CC$ and
$\FF=\RR$. Lets look at the simplest situations. What is
$\RP^1$?\index{$\RP^1$} It is very easy to see
\begin{equation}
\RP^1=S^1
\end{equation}
We see
\begin{equation}
\RP^1=\RR^1\sqcup\RP^0
\end{equation}
but $\RP^0$ consists of just a single point. What is
$\CP^1$?\index{$\CP^1$} Of course, we see
\begin{equation}
\begin{split}
\CP^1 &= \CC\cup\CP^0\\
&=\CC\cup\{\mbox{point}\}\\
&=S^2
\end{split}
\end{equation}
as desired.

Let us look a little bit at 
\begin{equation}
\RP^{n} = (R^{n+1}-0)/(x\sim\lambda x)
\end{equation}
But we may do something different. Namely every point on
$\RR^{n+1}-0\homotopic S^n$, why? We may divide $x$ by $\|x\|$ so
every point on $\RP^n$ may be represented by a point on a
sphere. But still we should identify $x\sim\lambda x$ where both
are on the sphere\dots but this happens when
$|\lambda|=1$. Therefore the only thing we should do is
\begin{equation}
\RP^n=S^n/(x\sim-x)
\end{equation}
This may be represented by 
\begin{equation}
\RP^n=\RR^n\sqcup\RP^{n-1},
\end{equation}
which is how we get a cellular decomposition (where we have a single
$k$-cell in every dimension $k\leq n$).

For the complex case, we see
\begin{equation}
\begin{split}
\CP^n &= (\CC^{n+1}-0)/(x\sim\lambda x)\\
&= S^{2n+1}/(x\sim\lambda x)
\end{split}
\end{equation}
where $|\lambda|=1$. Is this true? Let first note
\begin{equation}
\CC^{n+1}=\RR^{2(n+1)}
\end{equation}
but we demand $\|x\|=1$ which eliminates a dimension, giving us
\begin{equation}
\RR^{2(n+1)}/\sim = S^{2n+1}
\end{equation}
This implies $|\lambda|=1$. We can consider this set $S^{1}=\{\lambda :
|\lambda|=1\}$ as a group. This group acts on $S^{2n+1}$ simply
by
\begin{equation}
x\mapsto\lambda x
\end{equation}
\index{Hopf Fibration!Derived from $\CP^1$|(}%
One more definition of $\CP^n$. One more definition of
$\CP^n$. This is
\begin{equation}
\CP^n=S^{2n+1}/S^{1}
\end{equation}
where we mod out by this action of $S^1$; we can write similarly
\begin{equation}
\RP^n=S^n/\ZZ_2
\end{equation}
since $\lambda\in\RR$ and $|\lambda|=1$ implies
$\lambda=\pm1$. Take $n-1$ we get
\begin{equation}
\CP^1=S^3/S^1=S^2
\end{equation}
In a different way, we may say this as follows: there exists a
mapping
\begin{equation}
h\colon S^3\to S^2
\end{equation}
such that the preimage of a point $h^{-1}(\mbox{point})=S^1$. We
call $h$ the \define{Hopf Map}\index{Hopf Map}, later we will see
$h$ is not homotopic to 0, so it's very non-trivial.
\index{Hopf Fibration!Derived from $\CP^1$|)}%
\index{Space!Projective|)}\index{Projective Space|)}

\subsection{Knot Rejoinder}
How do we use $h$ to analyze the structure of knots? We did say
\begin{equation}
h^{-1}(x)=S^1
\end{equation}
for any $x\in S^2$. We can see that\index{Torus!Solid}
\begin{equation}
h^{-1}(\mbox{disc})=\begin{pmatrix}\mbox{solid}\\\mbox{torus}
\end{pmatrix}.
\end{equation}
How? Well, we could see the preimage $h^{-1}(S^{1})$ is a torus,
and $S^1$ is a boundary of a disc. So we fill in the disc
``continuously'' and we fill in the torus. 

Lets be clear here. We are looking at $S^{2}$ by constructing it
from two 2-discs 
\begin{equation}
S^{2}=(\bar{D}^{2}_{0}\sqcup\bar{D}^{2}_{1})/(\partial\bar{D}^{2}_{0}\sim\partial\bar{D}^{2}_{1})
\end{equation}
We let
\begin{subequations}
\begin{equation}
A=h^{-1}(\bar{D}^{2}_{0})
\end{equation}
and
\begin{equation}
B=h^{-1}(\bar{D}^{2}_{1})
\end{equation}
\end{subequations}
and we will use van Kampen's theorem.

So
\begin{equation}
h^{-1}(S^{2})=A\cup B
\end{equation}
where $A$, $B$ are solid tori, and
\begin{equation}
A\cap B=T^{2}
\end{equation}
is a (non-solid) torus. We obtain this since
\begin{equation}
S^{3}=\bar{D}^{2}\cup\bar{D}^{2}.
\end{equation}
We obtain this from the Hopf map.

Consider the solid torus in $\RR^3$, it is bounded by the torus
in $\RR^3$. Consider this stuff in
\begin{equation}
S^{3}=\RR^{3}\cup\infty.
\end{equation}
We consider
\begin{equation}
S^{3}-(\mbox{open solid torus}).
\end{equation}
What do we get? A solid torus! We can see this result in our
picture also. Of course, this is something with the solid torus
as a boundary. The analog of the solid torus in higher dimension
--- we take a body with handle, it's called a \define{Handle Body}.\index{Handle Body}
It's important to consider representations of them in ``Heegaard
Diagrams''\index{Heegaard Diagrams!we won't speak of ---}\dots but
we won't speak of it here.

We can take as a knot invariant $\pi_{1}(S^3-K)$. Let us take for
$K$ a \emph{trivial} knot, i.e., $K=S^1$. Then it's very simple,
because we can take a small neighborhood of $S^1$, which is a
solid torus, and 
\begin{equation}
S^{3}-K\homotopic\mbox{solid torus}\homotopic\mbox{circle}
\end{equation}
are homotopy equivalences. Therefore
\begin{equation}
\pi_{1}(S^{3}-K)\iso\pi_{1}(S^1)\iso\ZZ
\end{equation}
But we would like to distinguish two unlinked circles from two
linked circles: \includegraphics{img/lecture12.6} vs.\ \includegraphics{img/lecture12.7}.
We may take the first circle as $K$,
\begin{equation}
S^{3}-K=\begin{pmatrix}\mbox{solid}\\ \mbox{torus}
\end{pmatrix}
\end{equation}
we then consider the second circle as an elemetn of
$\pi_1(S^3-K)$. If it's trivial, the knots are unlinked.

A knot is a topological circle in $\RR^3$ or $S^3$. Now it is
possible the circle lies on a torus. Then it's called
a \define{Torus Knot}\index{Knot!Torus|textbf}\index{Torus!Knot}.
The next problem is to find
\begin{equation*}
\pi_1\bigl(S^3-(\mbox{solid torus})\bigr).
\end{equation*}
This will give us an invariant of the knot, we can classify them.

%%%%%%%%%%%%%%%%%%%%%%%%%%%%%%%%%%%%%%%%%%%%%%%%%%%%%%%%%%%%%%%%%%%%%%%%%%%
\exercises
\begin{xca}
Let $\FF_{p}$ be the finite field with $p$ elements, where $p$ is
prime. What is the cardinality of the set $\FP^{n}_{p}$?
\end{xca}
\begin{xca}
Projective space $\RP^n$ can be obtained from the sphere $S^n$ by
means of identification of antipodes $( x \sim -x)$. Describe the
cell decomposition of $\RP^n$ and use it to calculate its
fundamental group. 
\end{xca}
\begin{xca}\label{xca:lec13:prob2}
Let us consider an $n$-dimensional manifold $X$ and its subspace
$X = X \setminus D^n$ (the space $X$ with deleted open ball
$D^n$). Express the fundamental group of $X$ in terms of the
fundamental group of $X$. 
\end{xca}
\begin{xca}\label{xca:lec13:prob3}
The connected sum of two $n$-dimensional manifolds $X$ and $Y$ is
defined by means of deleting of open balls from $X$ and $Y$ and
identification of boundaries of deleted balls. (In notations of
Problem \ref{xca:lec13:prob2} we identify the boundary spheres in
$X$ and $Y$). Calculate the fundamental group of connected sum. 
\end{xca}
\begin{rmk}
In Problems \ref{xca:lec13:prob2} and \ref{xca:lec13:prob3} we
assume that $n>1$. In the case $n=2$ you can use the fact that
every two-dimensional compact manifold is a sphere with attached
handles and Moebius bands;\index{Mobius Band!and Surfaces} if the manifold is not compact one
should consider a sphere with holes (with deleted closed disks)
instead of sphere.
\end{rmk}

\lecture
%%
%% lecture14.tex
%% 
%% Made by alex
%% Login   <alex@tomato>
%% 
%% Started on  Thu Jan  5 08:18:04 2012 alex
%% Last update Thu Jan  5 08:18:04 2012 alex
%%





\exercises
\begin{xca}
Let us consider a finite cell complex $X$. Its homology
$H_k(X,\ZZ)$ can be represented as a direct sum of free abelian
group $F_k$ and torsion group $T_k$. Prove that its cohomology
$H^k(X,\ZZ)$ is isomorphic to the direct sum of $F_k$ and
$T_{k-1}$.
\end{xca}
\begin{xca}\index{K\"unneth Theorem}
Calculate the homology $H_k(\RP^2\times\RP^2, \ZZ)$ in two ways:
using cell complex and using K\"unneth theorem. (Here $\RP^2$
denotes projective plane.)
\end{xca}

\lecture
%%
%% lecture15.tex
%% 
%% Made by Alex Nelson
%% Login   <alex@tomato3>
%% 
%% Started on  Fri Dec 10 11:39:14 2010 Alex Nelson
%% Last update Fri Dec 10 12:54:35 2010 Alex Nelson
%%

Today we will start with some general examples. First some simple
constructions of representations which are quite general. We will
consider a representation
\begin{equation}
\varphi\colon G\to\GL{V}.
\end{equation}
If we have one representation, we may consider many others
related to it. We may use any natural construction, any functor,
will give you something. For example we may consider the dual
space 
\begin{equation}
V^{*}=\left\{ v\colon V\to\Bbb{F}\right\}
\end{equation}
This is a contravariant functor. Remember $\varphi(g)\in{\rm
  GL}_{n}$ if $V$ is finite dimensional; duality is related by
the transpose $\varphi(g)^{T}$, we may ask ourselves if it is a
representation?

We see immediately \emph{no it isn't!} Because we may say the
transpose
\begin{equation}
\big(\varphi(g)\varphi(h)\big)^{T}\not=\varphi(g)^{T}\varphi(h)^{T}
\end{equation}
therefore we do not have a representation. It is simple to cure
this, we take
\begin{equation}
\big(\varphi(g)^{T}\big)^{-1}=\big(\varphi(g)^{-1}\big)^{T}
\end{equation}
which is the \define{Dual Representation}, i.e.\ the
representation on the dual space. We demand then that
\begin{equation}
\big(\varphi(g)^{-1}\big)^{T}=\varphi(g^{-1})^{T}
\end{equation}
and then the character of the dual representation is
\begin{equation}
\chi_{\text{dual}}(g)=\tr\big(\varphi(g^{-1})^{T}\big)=\tr\big(\varphi(g^{-1})\big)
\end{equation}
so
\begin{equation}
\chi_{\text{dual}}(g)=\chi(g^{-1}).
\end{equation}

There is another operation that is important, namely taking the
tensor product. Suppose $V$ has basis $(v_{1},\dots,v_{m})$ and
$W$ has basis $(w_{1},\dots,w_{n})$, then $V\otimes W$ has basis 
\begin{equation}
(v_{1}\otimes w_{1},\dots, v_{1}\otimes w_{n},\dots,v_{m}\otimes
  w_{1},\dots, v_{m}\otimes w_{n})
\end{equation}
and a vector in $V\otimes W$ is of the form
\begin{equation}
z = z^{ij}v_{i}\otimes w_{j} = \sum_{i=1}^{m}\sum_{j=1}^{n}z^{ij}v_{i}\otimes w_{j}
\end{equation}
where we use Einstein summation conventions where indices
upstairs are summed over the indices downstairs. The dependence
on the choice of basis is fictitious. If we have a change of
coordinates in $V$ have the components transform by
\begin{equation}
x^{i}\mapsto\widetilde{x}^{i} = {a^{i}}_{j}x^{j}
\end{equation}
and we consider some arbitrary vector
\begin{equation}
v=x^{j}v_{j}\quad\text{in }V
\end{equation}
and if we do likewise consider a change of coordinates in $W$ by
\begin{equation}
y^{l}\mapsto\widetilde{y}^{l}={b^{l}}_{k}y^{k}
\end{equation}
where we implicitly sum over $k$, then we have
\begin{equation}
w=y^{k}w_{k}
\end{equation}
describe an arbitrary element. What is the transformation in the
coordinates of the tensor product? It is very simple. We obtain
them by
\begin{equation}
\widetilde{z}^{il}={a^{i}}_{j}{b^{l}}_{k}z^{jk}
\end{equation}
so if
\begin{equation}
a=a(g)\quad\mbox{and}\quad b=b(g)
\end{equation}
for some group element $g\in G$, then
\begin{equation}
a\otimes b=(a\otimes b)(g)
\end{equation}
depends on $g$ too. This gives rise to a tensor product of
representations, which is a representation by functoriality.

If we have
\begin{equation}
\varphi\colon G\to\GL{V}
\end{equation}
and
\begin{equation}
\psi\colon G\to\GL{W}
\end{equation}
be representations, then we have the tensor product of
representations as
\begin{equation}
(\varphi\otimes\psi)_{g}(v\otimes w)=\big(\varphi(g)v\big)\otimes\big(\psi(g)w\big).
\end{equation}
What about vectors that are not basis vectors? We can use
distributivity, if
\begin{equation}
v=x^{i}v_{i}\quad\mbox{and}\quad w=y^{j}w_{j}
\end{equation}
then by definition
\begin{equation}
v\otimes w=x^{i}y^{j}(v_{i}\otimes w_{j}).
\end{equation}
In other words, if $V$ and $W$ are $G$-modules, then $V\otimes W$
is a $G$-module. We may iterate for as many $G$-modules tensored
together as possible. We may recall 
\begin{equation}
W\otimes V\iso V\otimes W
\end{equation}
naturally.

We may construct more representations via some gadget called an
\define{Intertwiner} which is a morphism of $G$-modules
(i.e. preserves commutator, group operation). Sometimes we use
shorthand
\begin{equation}
\varphi_{g}v = gv
\end{equation}
for the group action. Then a morphism is
\begin{equation}
\alpha(gv)=g(\alpha v).
\end{equation}
If an intertwiner is invertible, we have an equivalence of
representations. 

If we consider $V\otimes V$, then we have a natural intertwiner
namely
\begin{equation}
v\otimes w\mapsto w\otimes v.
\end{equation}
This is a natural isomorphism of representations, so nothing
changes. If we have $V^{\otimes n}$, then we have the symmetric
group $S_{n}$ consisting of intertwiners. So a permutation is an
intertwiner. We may consider vectors $x$ such that
\begin{equation}
\alpha(x)=x
\end{equation}
is invariant under such permutations; they form a subspace. More
precisely
\begin{equation}
x=z^{ij}v_{i}\otimes v_{j}
\end{equation}
and the coefficients are tensors, what we do is consider
symmetric tensors which are fixed points of the intertwiner which
implies the coefficients obey
\begin{equation}
z^{ij}=z^{ji}
\end{equation}
for all $i$, $j$.

We may consider the subspace obeying
\begin{equation}
\alpha(x)=-x
\end{equation}
then the coefficients are
\begin{equation}
z^{ij}=-z^{ji}
\end{equation}
antisymmetric tensors. The symmetric one is denoted by
$\Sym^{2}V$ and the antisymmetric by $\Antisymmetric^{2}V$. We
generalize to the tensor product of $n$ spaces
\begin{equation}
V^{\otimes n}=\underbracket[0.5pt]{V\otimes\dots\otimes V}_{\text{$n$ times}}
\end{equation}
then we get guys with $n$ indices $z^{i_{1}\dots i_{n}}$. We can
apply various demands of indices. We use the notation
\begin{equation}
z^{[ij]} = \frac{1}{2!}(z^{ij}-z^{ji})
\end{equation}
and
\begin{equation}
z^{(ij)} = \frac{1}{2!}(z^{ij}+z^{ji}).
\end{equation}

We may also take tensor products including the dual space and the
vector space, for example
\begin{equation}
V^{\otimes m}\otimes (V^{*})^{\otimes n} = 
\underbracket[0.5pt]{V\otimes\dots\otimes V}_{\text{$m$ times}}\otimes
\underbracket[0.5pt]{V^{*}\otimes\dots\otimes V^{*}}_{\text{$n$ times}}
\end{equation}
which results in guys
\begin{equation}
a^{i_{1}\dots i_{m}}_{j_{1}\dots j_{n}}
\end{equation}
with mixed indices.

We have been talking about groups, but we may consider analogous
gadgetry for the Lie algebra. If we have
\begin{equation}
(\varphi_{g}\otimes\psi_{g})(u\otimes v)=\big(\varphi(g)(u)\big)\otimes\big(\psi(g)(v)\big)
\end{equation}
for the Lie group, and we take 
\begin{equation}
g=1+\gamma
\end{equation}
where $\gamma$ is ``small.'' We obtain from the Lie group
representation $\varphi_{1+\gamma}$ a Lie algebra representation
$\widetilde{\varphi}_{\gamma}$, but how does the representation
behave under the tensor product of Lie algebra representations?
We have
\begin{equation}
(\widetilde\varphi_{\gamma}\otimes\widetilde\psi_{\gamma})(u\otimes v)=
\big(\widetilde\varphi_{\gamma}(u)\big)\otimes v + u\otimes\big(\widetilde\psi_{\gamma}(v)\big).
\end{equation}
Why? Well observe that
\begin{subequations}
\begin{align}
(\varphi_{1+\gamma}\otimes\psi_{1+\gamma}) 
&= (\1+\widetilde\varphi_{\gamma})\otimes(\1+\widetilde\psi_{\gamma})\\
&= {\bf 1} +
  \underbracket[0.5pt]{\widetilde{\varphi}_{\gamma}\otimes{\bf 1}
+{\bf 1}\otimes\widetilde{\psi}_{\gamma}}_{\text{Lie Algebra rep.}} + \mathcal{O}(\varepsilon^{2})
\end{align}
\end{subequations}
where $\varepsilon$ is the ``magnitude'' of $\gamma$, which is
negligibly small in comparison to 1, and $\1$ is the
identity operator.

Consider the simplest example $\U{n}$ and its fundamental
representation $\GL{\CC^{n}}$. The maximal torus is
\begin{equation}
T = \left\{\begin{bmatrix}
 e^{i\varphi_{1}} &        &  \\
              & \ddots &  \\
              &        & e^{i\varphi_{n}}
\end{bmatrix}
\right\},
\end{equation}
this corresponds to the Cartan subalgebra consisting of diagonal
matrices. The weight vectors are the standard basis
\begin{equation}
v_{i} = e_{i}
\end{equation}
which is 1 for the $i^{th}$ component, 0 for all others. Now it
is clear what are the weights, merely the corresponding
components. We may consider the tensor product of two such
representations. The basis is by our definition $v_{i}\otimes
v_{j}$, and it is very easy to understand
\begin{equation}
v_{i}\otimes v_{j}\mapsto (\varphi_{i}+\varphi_{j})(v_{i}\otimes v_{j})
\end{equation}
is a weight vector. There is a rule for the characters
\begin{equation}
\chi_{\varphi\otimes\psi}=\chi_{\varphi}\chi_{\psi}
\end{equation}
using the characters of the ``component'' representations. We
also have a representation of symmetric tensors with the basis
\begin{equation}
v_{i}\otimes v_{j}+v_{j}\otimes v_{i}
\end{equation}
and for a representation of antisymmetric tensors
\begin{equation}
v_{i}\otimes v_{j}-v_{j}\otimes v_{i}
\end{equation}
up to some overall factor of $1/2$. Both bases have almost the
same weights $\varphi_{i}+\varphi_{j}$, but for the antisymmetric
tensors we require $i\not=j$.



%% \begin{picture}(50,7)
%% \multiput(5,1)(20,0){5}{\circle{6}}
%% \multiput(8,1)(20,0){4}{\line(1,0){14}}
%% \end{picture}

\lecture[$K$-Theory]
%%
%% lecture16.tex
%% 
%% Made by Alex Nelson
%% Login   <alex@tomato3>
%% 
%% Started on  Fri Dec 10 12:52:01 2010 Alex Nelson
%% Last update Sun Feb 20 11:01:11 2011 Alex Nelson
%%
If we have
\begin{equation}
\varphi\colon\mathscr{G}\to\mathfrak{gl}(V)
\end{equation}
a representation, and $\mathscr{H}\propersubset\mathscr{G}$ is
the Cartan subalgebra, then we recall a weight vector $v$ is an
eigenvector
\begin{equation}
\varphi(h)v = \lambda(h)v
\end{equation}
for every $h\in\mathscr{H}$. We have an adjoint representation
\begin{equation}
\ad\colon\mathscr{G}\to\mathfrak{gl}(\mathscr{G})
\end{equation}
where
\begin{equation}
(\ad x)v = [x,v]
\end{equation}
and the weight vectors for this representation are called root
vectors, and the weights are called roots. To define roots and
root vectors for $\mathscr{G}$ we are solving
\begin{equation}
[h,v] = \alpha(h)v.
\end{equation}
We can construct new root vectors from a given root vector by
applying $e_{i}$, $f_{j}$ to it. We have
\begin{subequations}
\begin{align}
[h,e] = \alpha(h)e & \iff he=eh+\alpha(h)e \\
 & \iff h(ev) = \big(eh+\alpha(h)e\big)v
\end{align}
\end{subequations}
but 
\begin{equation}
hv = \lambda(h)v \quad\implies\quad h(ev) =
\big(\lambda(h)+\alpha(h)\big) ev.
\end{equation}
This is either zero or another distinct eigenvector.

\begin{prop}
If $v$ is a weight vector with weight $\lambda$ and $e$ is a root
vector with root $\alpha$, then $ev$ is a weight vector with
weight $\lambda+\alpha$ unless $ev=0$.
\end{prop}

Now we will introduce a definition. Well several
definitions. First we introduce a notion of a Cartan Matrix which
is presented differently in different papers.
\begin{defn}
A \define{Cartan Matrix} is a matrix $A=[a_{ij}]$ such that
\begin{enumerate}
\item $a_{ii}=2$ are the diagonal components;
\item $a_{ij}\in\ZZ$ for any $i$, $j$;
\item $a_{ij}\leq 0$ for off-diagonal components;
\item although not necessarily symmetric, if $a_{ij}=0$ then
  $a_{ji}=0$;
\item it should be symmetrizable, i.e. we have a diagonal matrix
  $B$ such that $AB=D$ is also diagonal.
\end{enumerate}
\end{defn}
\begin{rmk}
Most of the time we will work with $A$ nondegenerate, i.e.
\begin{equation}
\det(A)\not=0
\end{equation}
But this is not a necessary condition, so we do not make it part
of the definition.
\end{rmk}
For every classical Lie Algebra, the matrix $a_{ij}$ % from the homework assignment,
is nondegenerate
\begin{equation}
\det(a_{ij})\not=0.
\end{equation}
We have explicitly computed this, so we should look at our
answers and nothing more.

The only thing that needs discussion is ``Why is $A$
symmetrizable?'' We know there exists a nondegenerate invariant
inner product on classical Lie algebras. The adjoint
representation is orthogonal with respect to this inner product,
i.e.\
\begin{equation}
\big\<[h,x], y\big\>+\big\<x,[h,y]\big\> = 0
\end{equation}
where $h,x,y\in\mathscr{G}$. This could be viewed as a
consequence of compact Lie groups having unitary representations
giving invariant inner product.

We can introduce the Killing form\index{Killing Form!Invariant Inner Product, Relation to} as
\begin{equation}
\<x,y\> = \tr\big(\ad_{x}\ad_{y}\big)
\end{equation}
which is an invariant inner product. %
%% %% Incorrect proof given on the day of the lecture
%% We consider the case when
%% $h\in\mathscr{H}$, $x=e_{j}$, and $y=f_{k}$, then we see that
%% \begin{subequations}
%% \begin{align}
%% \big\<[h,x], y\big\>+\big\<x,[h,y]\big\> &= \big\<[h,e_{j}],
%% f_{k}\big\>+\big\<e_{j},[h,f_{k}]\big\> =\\
%% &= \big\<a_{ij}e_{j}, f_{k}\big\>+\big\<e_{j},-a_{ik}f_{k}]\big\>
%% \end{align}
%% \end{subequations}
%% \dots
%% we should get the formula
%% \begin{equation}
%% a_{ij}\<e_{i},f_{i}\> = a_{ji}\<e_{j}, f_{j}\>
%% \end{equation}
%% %% Correct proof given in the next lecture
We may introduce an invariant inner product on the group
\begin{equation}
\<Ux,Uy\> = \<x,y\>
\end{equation}
where $U\in G$, but when $U=\1+u\varepsilon$ where $\varepsilon$
is ``small'', then we get
\begin{equation}
\<\varphi(u)x,y\> + \<x,\varphi(u)y\> = 0
\end{equation}
where $\varphi$ is a morphism. But as a representation we have
\begin{equation}
\Big\<[u,x],y\Big\> + \Big\<x,[u,y]\Big\> = 0.
\end{equation}
If we let $u=e_{i}$, $x=f_{i}$, $y=h_{j}$ we get
\begin{subequations}
\begin{align}
\Big\<[u,x],y\Big\> + \Big\<x,[u,y]\Big\>
&= \Big\<[e_{i},f_{i}],h_{j}\Big\> + \Big\<f_{i},[e_{i},h_{j}]\Big\>\\
&= \<h_{i},h_{j}\> + \<f_{i}, -a_{ji}e_{i}\>
\end{align}
\end{subequations}
This holds if and only if
\begin{equation}
\begin{diagram}[small,hug,height=13.5pt]
\<h_{i}, h_{j}\> & \rEq & a_{ji}\<f_{i},e_{i}\> \\
\dEq             &      & \dEq \\
\<h_{j}, h_{i}\> & \rEq & a_{ij}\<f_{j},e_{i}\>
\end{diagram}
\end{equation}
Using the inner product on the group, we may construct  the
matrix $B=\diag\<e_{i},f_{i}\>$ which implies $AB$ is symmetric.

For every Cartan matrix, we may construct a Lie algebra called a
Kac--Moody algebra. Really %irreducible
simple Lie Algebras are Kac--Moody algebras with additional
condition that the Cartan matrix is positive definite. We will
now describe all irreducible representations of classical Lie
Algebras; this is true for all Lie Algebras related to compact
groups, and reductive Lie Algebras.

In reality we may say for every compact group, the corresponding
Lie algebras have precisely the right generators. Moreover, we
may classify algebras of compact groups. This gives us a general
theorem for representations of compact Lie algebras. We would
like to explain the notion of a highest weight vector in this
situation. Namely the highest weight vector $v$ is such that
\begin{equation}
\varphi(e_{i})v = 0
\end{equation}
for all $e_{i}\in\mathscr{G}$. Of course this means that
\begin{equation}
\varphi(h)v = \lambda(h)v
\end{equation}
for all $h\in\mathscr{H}$, then this $\lambda$ is called the
highest weight. 

First of all, what is $\lambda(-)$? It is a linear functional
$\lambda\in\mathscr{H}^{*}$, i.e.
\begin{equation}
\lambda\colon\mathscr{H}\to\FF
\end{equation}
it is a linear functional acting on the Cartan subalgebra. Then:
\begin{enumerate}
\item Irreducible representations contain not more than one
  highest weight vector, up to a constant factor; 
\item Finite dimensional irreducible representation $\iff$ finite
  dimensional representation with one highest weight vector;
\item For every $\lambda\in\mathscr{H}^{*}$ one can construct a
  unique irreducible representation with highest weight $\lambda$
  but this representation can be infinite dimensional;
\item\label{lec16:cartan:mostImportantPoint}\marginpar{\eqref{lec16:cartan:mostImportantPoint} is the most important point!}
  This representation is finite dimensional if and only if
  $\lambda(h_{i})$ is a non-negative integer.
\end{enumerate}
This gives us a complete description of finite dimensional
irreducible representations.

\subsection{Dynkin Diagrams}
It is a very convenient way to depict Cartan matrices. Namely %it is 
first of all the dimension of the Cartan algebra is called the
\define{Rank of the Lie Algebra}. We have $\ClassicalGroup{A}_{\ell}$,
$\ClassicalGroup{B}_{\ell}$, $\ClassicalGroup{C}_{\ell}$,
$\ClassicalGroup{D}_{\ell}$ all of rank $\ell$. 

\begin{wrapfigure}[2]{r}{4cm}
  \vspace{-20pt}
  \begin{center}
    \includegraphics{img/LieImg.3}
  \end{center}
  \vspace{-20pt}
\end{wrapfigure}
The Dynkin diagram for $\ClassicalGroup{A}_{\ell}$ we draw $\ell$
vertices and we draw edges. We have the number of edges connecting vertices
$v_{i}$ to $v_{j}$ be given by the formula using the Cartan matrix 
\begin{equation}
n_{ij}=a_{ij}a_{ji}.
\end{equation}
Suppose we know $B$, its diagonal so we only need to keep track
of one index really. Since we suppose we know $B$, then
\begin{equation}
a_{ij}b_{j}=a_{ji}b_{i}
\end{equation}
implies
\begin{equation}
b_{i} = \frac{a_{ij}b_{j}}{a_{ji}}
\end{equation}
we get
\begin{equation}
n_{i}b_{i} = {a_{ij}}^{2}b_{j},
\end{equation}
or equivalently
\begin{equation}
{a_{ij}}^{2} = \frac{n_{i}b_{i}}{b_{j}}.
\end{equation}
What is the conclusion? If we know $[n_{i}]$ and $[b_{i}]$, we can
compute $[a_{ij}]$. 

Let us write down the Dynkin diagrams for the classical Lie
groups we have considered.
\begin{wrapfigure}[4]{r}{4.5cm}
  \vspace{-20pt}
  \begin{center}
    \includegraphics{img/LieImg.4}
  \end{center}
  \vspace{-20pt}
\end{wrapfigure}
\noindent{}For $\ClassicalGroup{D}_{n}$ we have the diagram drawn
on the right for the case when $n=7$ (observe there are 7
vertices). The Cartan matrix for $\ClassicalGroup{D}_{n}$ is
symmetric. One can observe this by considering the adjacency
matrix for the graph.

\begin{wrapfigure}[2]{l}{4.5cm}
  \vspace{-20pt}
  \begin{center}
    \includegraphics{img/LieImg.5}
  \end{center}
  \vspace{-20pt}
\end{wrapfigure}
\noindent{}For $\ClassicalGroup{C}_{n}$ we see the Cartan
matrix is not symmetric, but we can symmetrize it. We find that
$a_{n-1,n}a_{n,n-1}=2$.

\begin{wrapfigure}[2]{r}{4.5cm}
  \vspace{-20pt}
  \begin{center}
    \includegraphics{img/LieImg.6}
  \end{center}
  \vspace{-20pt}
\end{wrapfigure}
\noindent{}For $\ClassicalGroup{B}_{n}$ we see the Dynkin diagram
is ``the same'' as for $\ClassicalGroup{C}_{n}$ but with
different labels for the vertices.

Almost all of these groups are simple and almost all of them are
not isomorphic. But almost all. For example, in
$\ClassicalGroup{D}_{2}$ we have two disconnected vertices for
the Dynkin diagram. So $\ClassicalGroup{D}_{2}$ is not simple, it
is the direct product of $\SU{2}$ at the level of Lie Algebras,
and \emph{almost} the direct product at the level of Lie
groups. So we may examine the Dynkin diagram for $\ClassicalGroup{D}_{3}$
to find:
\begin{center}
\includegraphics{img/LieImg.7}
\end{center}
\noindent{}So we see this is the same Dynkin diagram as for
$\ClassicalGroup{A}_{3}$ which implies at the level of Lie
Algebras
\begin{equation}
\SU{4}\iso\SO{6}
\end{equation}
but only at the level of Lie Algebras. We similarly have $\ClassicalGroup{B}_{2}\iso\ClassicalGroup{C}_{2}$
by inspection of the Dynkin diagrams, but again it is an
isomorphism at the level of Lie Algebras.

%%
%% box1.tex
%% 
%% Made by Alex Nelson
%% Login   <alex@tomato3>
%% 
%% Started on  Sun Feb 20 10:55:50 2011 Alex Nelson
%% Last update Sun Feb 20 17:09:04 2011 Alex Nelson
%%
\begin{framed}
\noindent{\sectionfont Box\enspace \thesection.1 Dynkin Diagrams}
\bigskip
\noindent{}The problem we are facing is really two-fold: (a)
given a Dynkin diagram obtain the Cartan matrix, and (b) given
the Cartan matrix obtain the Dynkin diagram. This box is really
based off of \S4.7 of Kac's book \emph{Infinite Dimensional Lie Algebras}.
No secrets among friends: Kac provides the method of, given a
Cartan matrix, producing the Dynkin diagram. We review this, and
provide the algorithm going in the opposite direction. We also
consider examples. Throughout $A=[a_{ij}]$ is the Cartan matrix.

\medskip
\noindent\textbf{Given Cartan Matrix Obtain Dynkin Diagram.}\enspace
The basic idea is that we will have an $n\times n$ matrix
$A$. The Dynkin diagram is a graph that will have $n$
vertices, which are labeled by integers $i=1,\dots,n$. If
\begin{equation}
a_{ij}a_{ji}\leq4\quad\mbox{and}\quad |a_{ij}|\geq|a_{ji}|
\end{equation}
then vertices $i$ and $j$ are connected by $|a_{ij}|$ lines;
moreover if $|a_{ij}|>1$, then these lines are equipped with an
arrow pointing towards vertex $j$. 

Why do we need an arrow? The idea is that the Cartan matrix is
not symmetric, but has a weaker condition that $a_{ij}\not=0$
implies $a_{ji}\not=0$. Since we know the product by the number
of lines, we know the values by considering which direction the
arrow points.

\medskip
\noindent\textbf{Given a Dynkin Diagram Obtain Cartan Matrix.}\enspace
This occurs more often in practice (at least, for
physicists). What can we know immediately from the properties of
a Cartan matrix? Well, we know
\begin{equation}
a_{ii} = 2
\end{equation}
for all $i$. We know that the number of vertices $n$ gives
information about the number of rows, and the number of columns,
of the Cartan matrix --- i.e. $A$ is an $n\times n$ matrix. We
also know if $i\not=j$ that
\begin{equation}
a_{ij}\leq0\quad\mbox{and}\quad a_{ij}\in\Bbb{Z}.
\end{equation}
The rest we need to find from the diagram.

If vertex $i$ and $j$ are connected by $k$ lines, then $a_{ij}<0$. 
What values can this component be? Well, if $k=1$, then
\begin{equation}
a_{ij}=a_{ji}=-1
\end{equation}
since there is no arrow, it must be $-1$. If there are multiple
lines, we have an arrow to indicate which entry 

\begin{rmk}
Note that in these examples, the vertices are labeled by
\emph{indices} to keep track of which we are discussing. Usually,
the labels of a vertex are the relative (squared)
lengths of the fundamental roots as Gilmore describes
it [see Robert Gilmore, \emph{Lie Groups, Lie Algebras, and Some
    of Their Applications} Dover Publications (2002) Ch 8 \S III.2 pp 306 \emph{et seq.}].

{\bf N.B.} the method we have described are used to deduce a
\emph{generalized} Cartan matrix from a Dynkin diagram. So if we
restrict focus to Dynkin diagrams corresponding to ``strict''
Cartan matrices, we recover precisely the same information. But
we can do more! We can consider ``closed loops'' in our approach!
The only requirement we have for our considerations is that there
are less than 4 edges connecting any pair of vertices.
\end{rmk}
\begin{ex}
Consider the Dynkin diagram given by
\begin{center}
  \includegraphics{img/LieImg.11}
\end{center}
We see that there are 4 vertices, so immediately we know that the
Cartan matrix is $4\times4$ and we can write:
\begin{equation}
A = \begin{bmatrix}
2 &   &   &   \\
  & 2 &   &   \\
  &   & 2 &   \\
  &   &   & 2
\end{bmatrix}.
\end{equation}
We also see that there is one line connecting vertex 1 to vertex
2, so that means we can write
\begin{equation}
A = \begin{bmatrix}
2 &-1 &   &   \\
-1& 2 &   &   \\
  &   & 2 &   \\
  &   &   & 2
\end{bmatrix}.
\end{equation}
We then observe that there are no other edges connected to 1, so
\begin{equation}
A = \begin{bmatrix}
2 &-1 & 0 & 0 \\
-1& 2 &   &   \\
0 &   & 2 &   \\
0 &   &   & 2
\end{bmatrix}.
\end{equation}
Similar reasoning holds for vertex 4, it's connected by a single
edge to vertex 3
\begin{equation}
A = \begin{bmatrix}
2 &-1 & 0 & 0 \\
-1& 2 &   &   \\
0 &   & 2 &-1 \\
0 &   &-1 & 2
\end{bmatrix}.
\end{equation}
There are no other edges that connect vertex 4 to any other
vertex, so 
\begin{equation}
A = \begin{bmatrix}
2 &-1 & 0 & 0 \\
-1& 2 &   & 0 \\
0 &   & 2 &-1 \\
0 & 0 &-1 & 2
\end{bmatrix}.
\end{equation}
We see that there are \emph{two lines} connecting vertex 2 to
vertex 3 and there is an arrow. The arrow means that
\begin{equation}
a_{23}\not=a_{32}.
\end{equation}
The arrow points towards 3, so
\begin{equation}
|a_{32}|<|a_{23}|.
\end{equation}
Then we use the fact that there are two edges means that
\begin{equation}
|a_{23}|=2
\end{equation}
This is sufficient information to conclude
\begin{equation}
A = \begin{bmatrix}
2 &-1 & 0 & 0 \\
-1& 2 &-2 & 0 \\
0 &-1 & 2 &-1 \\
0 & 0 &-1 & 2
\end{bmatrix}.
\end{equation}
Thus we conclude our example.
\end{ex}
\begin{ex}
Consider the Dynkin diagram given by
\begin{center}
  \includegraphics{img/LieImg.12}
\end{center}
We see that there are 2 vertices, so immediately we know that the
Cartan matrix is $2\times2$ and we can write:
\begin{equation}
A = \begin{bmatrix}
 2 &   \\
   & 2 
\end{bmatrix}.
\end{equation}
The two vertices are connected by 3 edges. There is an arrow
pointing from vertex 1 to vertex 2. This implies that
\begin{equation}
A = \begin{bmatrix}
 2 &-3 \\
-1 & 2 
\end{bmatrix}.
\end{equation}
Observe that if the arrow pointed the other way, we would merely
have the transpose of our matrix.
\end{ex}
\end{framed}


\subsection{Returning to Representations}
The representations are described by means of highest weight. We
had previously
\begin{equation}
\varphi(e_{i})x = 0
\end{equation}
where $x$ is our highest weight vector, and the highest weight is
described by
\begin{equation}
\varphi(h)x = \lambda(h)x
\end{equation}
where $\lambda\in\mathscr{H}^{*}$ is the highest weight. We
should demand $\lambda(h_{i})\geq0$, and
$\lambda(h_{i})\in\ZZ$. We will now turn our attention to examples.

We will consider the fundamental representations of
$\ClassicalGroup{A}_{\ell+1}=\mathfrak{sl}(\ell+1)$. The
fundamental representation is the representation by
$(1+\ell)\times(1+\ell)$ matrices. We found
\begin{equation}
e_{i}=E_{i,i+1}
\end{equation}
where $E_{i,j}$ has zero components everywhere except at $i,j$ it
is 1. The Cartan subalgebra is
\begin{equation}
\mathscr{H}=\left\{\begin{bmatrix}\lambda_1 & & \\
 & \ddots & \\
 &        & \lambda_{n}
\end{bmatrix}\text{ such that }\lambda_{1}+\dots+\lambda_{\ell+1}=0\right\}
\end{equation}
What are the weight vectors here? It is quite clear that the
weight vectors $u_{1}$, \dots, $u_{\ell+1}$ are the standard
basis vectors. Observe
\begin{equation}
hu_{i}=\lambda_{i}u_{i}.
\end{equation}
What is the highest weight vector of this representation? We see
that
\begin{equation}
e_{i}u_{j} = 0
\end{equation}
unless $j=i+1$ we have
\begin{equation}
e_{i}u_{i+1}=u_{i}
\end{equation}
The highest weight vector is clearly $u_{1}$ because the shift
goes down and there is no way down. This implies the
representation is irreducible as the highest weight vector is
unique up to some coefficient. We see that
\begin{equation}
\lambda(h_{i}) = \delta_{i1}
\end{equation}
also holds.

What about the tensor product of representations. We find the
basis to be $u_{j}\otimes u_{k}$ and
\begin{subequations}
\begin{align}
h_{i}(u_{j}\otimes u_{k}) &= (h_{i}u_{j})\otimes u_{k} + u_{j}\otimes(h_{i}u_{k})\\
&= (\lambda_{j}+\lambda_{k})(u_{j}\otimes u_{k})
\end{align}
\end{subequations}
We found all the weight vectors\dots well not really since
$u_{1}\otimes u_{2}$ is a weight vector with the same weight as
$u_{2}\otimes u_{1}$, so $u_{1}\otimes u_{2}\pm u_{2}\otimes
u_{1}$ is again a weight vector. We find the highest weight
vector to be $u_{1}\otimes u_{1}$, but we see that
\begin{equation}
e_{1}(u_{1}\otimes u_{2}) = u_{1}\otimes u_{1}
\end{equation}
and
\begin{equation}
e_{1}(u_{2}\otimes u_{1}) = u_{1}\otimes u_{1}
\end{equation}
so it follows that $u_{1}\otimes u_{2}-u_{2}\otimes u_{1}$ is
again a highest weight vector\dots so we have 2 distinct highest
weight vectors! This cannot be an irreducible
representation. This we know, we may consider the symmetric and
antisymmetric parts of the representation.




% the next lecture was a review of the homework problems, so
% we're right on track!
\lecture
%%
%% lecture17.tex
%% 
%% Made by alex
%% Login   <alex@tomato>
%% 
%% Started on  Thu Jan  5 08:21:15 2012 alex
%% Last update Thu Jan  5 08:21:15 2012 alex
%%



\exercises
Given two disjoint connected $n$-manifolds $M$ and $N$, a
connected $n$-manifold $M\connectSum{N}$, their connected
sum\index{Connected Sum}\index{Sum!Connected}, can be constructed
by deleting the interiors of small closed balls $B_1\propersubset
M$ and $B_2\propersubset N$ and identifying the resulting
boundary spheres $\partial B_1$ and $\partial B_2$ via some 
homeomorphism between them.
\begin{xca}
Assuming that $M$ and $N$ are closed orientable manifolds prove
that $H_k(M\connectSum{N}; \ZZ)$ is isomorphic to direct sum of
$H_k(M; \ZZ)$ and $H_k(N; \ZZ)$ for $0 < k < n$.
\end{xca}
\begin{xca}
Let $M$ denote a closed $n$-dimensional connected orientable
manifold. Assuming that we know the cohomology of $M$ calculate
the cohomology with compact supports of $M \setminus A$ where
\begin{enumerate}
\item $A$ is a finite subset of $M$,
\item $A$ is a union of boundary spheres $\bdry B_1$,\dots,
  $\bdry B_n$ of non-overlapping small closed balls $B_1$, \dots,
  $B_n$ in $M$.
\end{enumerate}
\end{xca}

\lecture
%%
%% lecture18.tex
%% 
%% Made by alex
%% Login   <alex@tomato>
%% 
%% Started on  Sat Mar 10 15:25:11 2012 alex
%% Last update Sat Mar 10 15:25:11 2012 alex
%%

The weak field equations. The most important test of general
relativity is that it gives us back Newtonian gravity. Lets
consider a weak field, which can be thought of as a perturbation
of a background $\eta_{\mu\nu}$. So
\begin{equation}
g_{\mu\nu}=\eta_{\mu\nu}+h_{\mu\nu}
\end{equation}
and the inverse metric is given\footnote{We should recall that
  the Neumann series gives us
  $(I+X)^{-1}=I+X+X^{2}+\dots+X^{n}+\dots$, which is employed here.} by
\begin{equation}
g^{\mu\nu}=\eta^{\mu\nu}-h^{\mu\nu}+\bigO(h^{2}).
\end{equation}
We raise and lower indices with $\eta$ in this approximation. The
Christoffel connection
\begin{equation}
\begin{split}
\Gamma^{\rho}_{\mu\nu}
&=\frac{1}{2}g^{\rho\sigma}(\partial_{\mu}g_{\sigma\nu}
+\partial_{\nu}g_{\mu\sigma}-\partial_{\sigma}g_{\mu\nu})\\
&=\frac{1}{2}\eta^{\rho\sigma}(\partial_{\mu}h_{\sigma\nu}
+\partial_{\nu}h_{\mu\sigma}-\partial_{\sigma}h_{\mu\nu})
+\bigO(h^{2}).
\end{split}
\end{equation}
For the Ricci tensor, we only have the $\partial\Gamma$ terms to
worry about, since $\Gamma\Gamma\sim\bigO(h^{2})$. The components
of the Ricci tensor are
\begin{equation}
R_{\mu\nu}=\frac{1}{2}(\partial^{\sigma}\partial_{\mu}h_{\sigma\nu}
+\partial^{\sigma}\partial_{\nu}h_{\sigma\mu}
-\partial^{\sigma}\partial_{\sigma}h_{\mu\nu}
-\partial_{\mu}\partial_{\nu}{h^{\sigma}}_{\sigma}).
\end{equation}
If we write
\begin{equation}
h={h^{\sigma}}_{\sigma}
\end{equation}
for the trace, then the ``trace-reversed $h$'' is
\begin{equation}
\bar{h}_{\mu\nu}=h_{\mu\nu}-\frac{1}{2}\eta_{\mu\nu}h
\end{equation}
Observe
\begin{equation}
\begin{split}
\bar{h}&=\eta^{\mu\nu}\bar{h}_{\mu\nu}\\
&=h-2h=-h.
\end{split}
\end{equation}
The Einstein tensor becomes
\begin{equation}
G_{\mu\nu}=\frac{1}{2}(-\Box\bar{h}_{\mu\nu}+
\partial_{\mu}\partial^{\sigma}\bar{h}_{\sigma\nu}+
\partial_{\nu}\partial^{\sigma}\bar{h}_{\sigma\mu}-
\eta_{\mu\nu}\partial^{\sigma}\partial^{\tau}\bar{h}_{\sigma\tau}).
\end{equation}
We can choose coordinates such that
\begin{equation}
G_{\mu\nu}=\frac{-1}{2}\Box\bar{h}_{\mu\nu}.
\end{equation}
%% The basic trick is that any metric at a given point has an
%% expansion
%% \begin{equation}
%% g_{\mu\nu}=\eta_{\mu\nu}+0+(\partial_{\mu}g_{\nu\alpha}\partial_{\nu}g_{\mu\beta})\frac{g^{\alpha\beta}}{2!}+\dots
%% \end{equation}
When
\begin{equation}
x^{\mu}\to x^{\mu}+\zeta^{\mu}
\end{equation}
for infinitesimal $\zeta$, then
\begin{equation}
g_{\mu\nu}\to
g_{\mu\nu}+\nabla_{\mu}\zeta_{\nu}+\nabla_{\nu}\zeta_{\mu}
\end{equation}
and the perturbation transforms as
\begin{equation}
h_{\mu\nu}\to h_{\mu\nu}+\partial_{\mu}\zeta_{\nu}+\partial_{\nu}\zeta_{\mu}
+\bigO(h^{2}).
\end{equation}
Observe this implies
\begin{equation}
\partial^{\sigma}\bar{h}_{\sigma\nu}\to\partial^{\sigma}\bar{h}_{\sigma\nu}+\Box\zeta_{\nu}.
\end{equation}
We may choose $\zeta$ so that
\begin{equation}
\partial^{\sigma}\bar{h}_{\sigma\nu}\to0.
\end{equation}
We just have to solve
\begin{equation}
\Box\zeta_{\nu}=-\partial^{\sigma}\bar{h}^{\text{(old)}}_{\sigma\nu}
\end{equation}
which is the wave equation with a source. We know how to solve
that! See, e.g., Jackson's electrodynamics text. We can choose
coordinates such that
\begin{equation}
\partial^{\sigma}\bar{h}_{\sigma\nu}=0
\end{equation}
which we call the harmonic gauge, Fock gauge, Lorenz gauge, de
Donder gauge, etc.

\begin{rmk}[Physical Ramifications of Choice of Coordinates]
There is no physical meaning for this choice of gauge (i.e., this
particular choice of coordinates), nor does any other choice have
physical meaning unless there exists strong symmetries which
enable a canonical choice.
\end{rmk}

The harmonic gauge has Einstein's field equations read
\begin{equation}
\begin{split}
\frac{-1}{2}\Box\bar{h}_{\mu\nu}=\frac{\kappa^{2}}{2}T_{\mu\nu}.
\end{split}
\end{equation}
Among other things in life, this tells us (1) there exists
gravity waves, (2) they travel at the speed of light because of
the D'Alembertian.

\subsection{Newtonian Limit}
Lets consider the Newtonian limit, when
\begin{equation}
v/c\lll1
\end{equation}
where $v$ is the velocity of gravitating bodies.
For non-interacting matter (``dust'') we had the stress energy
tensor
\begin{equation}
T^{\mu\nu}=\rho u^{\mu}u^{\nu}
\end{equation}
which has components
\begin{equation}
T^{00}\approx\rho
\end{equation}
and
\begin{equation}
T^{ij}\approx T^{i0}\approx T^{0j}\approx 0.
\end{equation}
So we have
\begin{equation}
\Box\bar{h}^{00}=-\kappa^{2}\rho.
\end{equation}
Observe the D'Alembertian is written
\begin{equation}
\Box=\frac{1}{c^{2}}\frac{\partial^{2}}{\partial t^{2}}
-\nabla^{2}
\end{equation}
but in the Newtonian approximation
\begin{equation}
\frac{1}{c^{2}}\frac{\partial^{2}}{\partial t^{2}}\approx0.
\end{equation}
Thus
\begin{equation}
\Box\approx-\nabla^{2}.
\end{equation}
Our field equation becomes
\begin{equation}
-\nabla^{2}\bar{h}_{00}=-\kappa^{2}\rho.
\end{equation}
This is precisely Poisson's equation for Newtonian gravity! That
is
\begin{equation}
\nabla^{2}\Phi=4\pi G\rho
\end{equation}
thus by inspection
\begin{equation}
\bar{h}_{00}=\frac{\kappa^{2}}{4\pi G}\Phi
\end{equation}
We see that
\begin{equation}
h_{\mu\nu}=\bar{h}_{\mu\nu}-\frac{1}{2}\eta_{\mu\nu}\bar{h}
\end{equation}
Hence
\begin{equation}
h_{\mu\nu}\frac{\kappa^{2}}{8\pi G}\Phi\eta_{\mu\nu}.
\end{equation}
The line element in the Newtonian approximation is
\begin{equation}
\D s^{2}=\left(1+\frac{\kappa^{2}}{8\pi G}\Phi\right)\D t^{2}
-\left(1-\frac{\kappa^{2}}{8\pi G}\Phi\right)\D\mathbf{x}\cdot\D\mathbf{x}.
\end{equation}
This is good: General Relativity contains Newtonian gravity at
appropriate limits!

\begin{rmk}
When we have very light masses moving close to the speed of
light, we need to include other components of $h$; but we can
still use the weak field approximation!
\end{rmk}

We now know that the field equations are
\begin{equation}
\Box\bar{h}_{\mu\nu}=-16\pi GT_{\mu\nu}
\end{equation}
we pull out our copy of Jackson, or Afkren (or whatever), and use
the Green's function for the D'Alembertian
\begin{equation}
\begin{split}
\bar{h}_{\mu\nu}(\mathbf{x},t) &= 
4
G\int \frac{T_{\mu\nu}(\vec{y},t-|\vec{x}-\vec{y}|)}{|\vec{x}-\vec{y}|}\D^{3}y\\
&=4
G\int \frac{T_{\mu\nu}^{\text{(ret)}}}{|\vec{x}-\vec{y}|}\D^{3}y
\end{split}
\end{equation}
We can interpret the next order corrections as gravity's coupling
to the stress-energy tensor. To conclude our discussion, we will
write a table comparing the multipole expansion in
electromagnetism\footnote{C.f., Jackson's \emph{Classical
Electrodynamics}~\cite[{\normalfont p.145 \emph{et seq.}}]{jackson:1999}.} and in gravity:

\medbreak
\noindent\begin{tabular}{p{7pc}|p{11pc}p{11pc}}%{p{7pc}|p{11pc}|p{11pc}}
\toprule
Multipole Term&Electromagnetism & Gravity\\\midrule
Monopole Moment 
  & The total charge $q$; charge conserved, monopole moment is constant 
  & The total mass $m$; Newtonian limit has mass conserved, fixed
  field unchanging in time. \\    
Dipole Moment 
  & $\sum q_{i}r_{i}$, $\dot{D}=\sum q_{i}\dot{r}_{i}$; Fix two
  charges to the ends of a spring and oscillate.
  & $\sum m_{i}r_{i}$, $\sum m_{i}\dot{r}_{i} = \sum p_{i}=0$ in
  the center of mass frame. Try to oscillate total momentum, but
  this is fixed!\\
Magnetic Dipole
  & $\sum q_{i}\vec{v}_{i}\times\vec{r}_{i}$ 
  & $\sum m_{i}\vec{v}_{i}\times\vec{r}_{i}=\vec{L}=\mbox{constant}$
    Gravity has no mass dipole or magnetic dipole by conservation
    laws.\\
Quadrapole 
  & (None)
  & $\sum m_{i}{r_{i}}^{\mu}{r_{i}}^{\nu}$ This is the lowest
  order radiation for gravity, but we have to take the
  appropriate number of derivatives. The power is $\sim v/c^{8}$
  (This is a strong restriction on corrections to gravity!)\\
\bottomrule
\end{tabular}

\lecture
%%
%% lecture19.tex
%% 
%% Made by alex
%% Login   <alex@tomato>
%% 
%% Started on  Thu Jan  5 08:26:18 2012 alex
%% Last update Thu Jan  5 08:26:18 2012 alex
%%





\exercises
\begin{xca}
For finite cell complexes $X$ and $Y$ show that $\chi(X\times
Y)=\chi(X)\times\chi(Y)$.
\end{xca}
\begin{xca}
If a finite cell complex $X$ is union of subcomplexes $A$ and $B$ show that
\begin{equation}
\chi(X)=\chi(A)+\chi(B)-\chi(A\cap B).
\end{equation}
\end{xca}
\begin{xca}
Let us consider a function $\phi$ assigning a real number to
every finite cell complex. Let us assume that $\phi$ is a
topological invariant and $\phi(X)=\phi(A)+\phi(X/A)$ if $A$ is a
subcomplex of $X$. Prove that $\phi$ can be represented in the
form $\phi(X) = const(\chi(X)-1)$.
\end{xca}
\begin{xca}
Let us suppose that a closed orientable surface $M_g$ of genus
$g$ (a sphere with $g$ handles) is an $n$-sheeted covering space
of $M_h$ (of sphere with $h$ handles). Prove that $g=n(h-1)+1$.
\end{xca}

\lecture
%%
%% lecture20.tex
%% 
%% Made by Alex Nelson
%% Login   <alex@tomato3>
%% 
%% Started on  Sat Dec 11 12:35:11 2010 Alex Nelson
%% Last update Mon Dec 13 21:25:51 2010 Alex Nelson
%%
We formulated a theorem of the structure of semisimple algebras,
then considered examples. Lets go back. Recall we considered the
situation when we had $e_{\alpha}$, $f_{\alpha}$, $h_{\alpha}$
(members of the Lie Algebra) with the relations that
\begin{subequations}
\begin{align}
[h_{\alpha}, h_{\beta}] &= 0\\
[e_{\alpha}, f_{\beta}] &= h_{\alpha}\delta_{\alpha\beta}\\
[h_{\alpha}, e_{\beta}] &= a_{\alpha\beta}e_{\beta}\\
[h_{\alpha}, f_{\beta}] &=-a_{\alpha\beta}f_{\beta}.
\end{align}
\end{subequations}
We see that $e_{\alpha}$, $f_{\alpha}$ are root vectors, so
\begin{equation}
[h, e_{\alpha}] = \lambda_{\alpha}(h)e_{\alpha}
\end{equation}
and similarly
\begin{equation}
[h, f_{\beta}] = -\lambda_{\beta}(h)f_{\beta}.
\end{equation}
We recall that a mapping
\begin{equation}
\lambda\colon\mathscr{H}\to\Bbb{F}
\end{equation}
is called a \define{Root}, it's a linear functional on
$\mathscr{H}$. We see that
\begin{equation}
\lambda_{\beta}(h_{\alpha})=a_{\alpha\beta}.
\end{equation}
Moreover $a_{\alpha\beta}$ should be the Cartan matrix, so 
\begin{subequations}
\begin{align}
a_{\alpha\alpha}&=2\\
a_{\alpha\beta}&\leq0\quad\mbox{for }\alpha\not=\beta\\
a_{\alpha\beta}&\mbox{ is symmetrizable}
\end{align}
\end{subequations}
We also assume that
\begin{equation}
\det(a_{\alpha\beta})\not=0
\end{equation}
i.e. the Cartan matrix is nondegenerate. We can find the Cartan
matrix for semisimple Lie Algebras. An ideal corresponds to an
invaraint subspace of the Lie algebra under the adjoint
representation. We know a simple Lie Algebra is simple iff it has
only trivial ideals. For semisimple Lie Algebras, we can
partition roots into positive and negative roots. Positive roots
contain a subset that generates all roots, we call this subset
\define{Simple Roots}.

What may be said of representations with this data? We take
$\mathscr{G}$ a Lie algebra, we take a representation
\begin{equation}
\varphi\colon\mathscr{G}\to\mathfrak{gl}(V)
\end{equation}
for some vector space $V$, and we may consider the weights of
this Lie Algebra
\begin{equation}
\varphi(h)\vec{v}=\alpha(h)\vec{v}
\end{equation}
and weight vectors $\vec{v}$ (where we take
$h\in\mathscr{H}$). The root vectors act on weight vectors,
namely $\varphi(e_{k})\vec{v}$ is a weight vector (supposing that
$\vec{v}$ was initially a weight vector) with weight
$\alpha+\lambda_{k}$ provided that it is nonzero. Similarly
$\varphi(f_{j})\vec{v}$ is a weight vector with weight
$\alpha-\lambda_{j}$, so $\varphi(f_{j})$ lowers the weights. The
highest weight vector is annhilated by applying
$\varphi(e_{i})\vec{v}=0$ for all $i$. The highest weight vector
always exists in in finite dimensional representations, although
this is not necessarily true for infinite dimensional
representations. 

\begin{thm} {\rm(The highest weight vector exists in finite
    dimensional representations.)} If a finite dimensional
  representation is reducible, then the highest weight vector is
  not unique.
\end{thm}

Why? Well, at least one exists in the finite dimensional
case. Why? Trivially, because in linear algebra the eigenvalue
problem has no solution if the matrix is all zeroes. We cannot
have that for a nontrivial representation. 

Now suppose there exists a representation that is a
subrepresentation which will be irreducible and contains a
different highest weight vector. Let us suppose we have highest
weight vector, then we may construct a subrepresentation
consisting of
$f_{\alpha_{1}}(\cdots{})f_{\alpha_{n}}\vec{v}$\marginpar{This
may appear to be mathematically incorrect, but the weights are
integers which means at some moment these would vanish.} which
is a subrepresentation --- it is highest weight since
\begin{equation}
e_{\beta}\vec{v}=0
\end{equation}
for all $\beta$. Suppose our representation is reducible. If this
is so, there is a highest weight vector in the subrepresentation.

\begin{rmk}
To prove a representation is irreducible, it is sufficient to
prove the uniqueness of the highest weight vector.
\end{rmk}

How to classify, to describe representations (especially
irreducible representations). This is a simple thing, namely take
this highest weight vector
\begin{equation}
\varphi(h)\vec{v}=\alpha(h)\vec{v}
\end{equation}
where $\alpha\in\mathscr{H}^{*}$, and we should calculate
$\alpha(h_{1})$, \dots, $\alpha(h_{n})$ for all basis elements of
the Cartan subalgebra. We will prove that $\alpha(h_{i})\geq0$
and $\alpha(h)\in\ZZ$. To prove this is extremely simple, because
-- look -- we have these commutation relations
\begin{equation}
\Span\{h_{i},e_{i},f_{i}\}\iso\mathfrak{sl}(2)
\end{equation}
for fixed $i$. For this algebra $\mathfrak{sl}(2)$ we
know \emph{everything}, in particular all finite dimensional
irreducible representations, which is \emph{precisely} the guys
we are interested in. So the representation is characterized by
$n$ non-negative numbers. So can we take thse numbers in any way
we want? Yes we can, we'll prove it in the next lecture. We will
merely check this for $\ClassicalGroup{A}_{n}$. This is
interesting by itself. We will later check this for
$\ClassicalGroup{B}_{n}$, $\ClassicalGroup{C}_{n}$; the proof
will be constructive.

For $\ClassicalGroup{A}_{n}$, we have $e_{i}=E_{i,i+1}$,
$f_{i}=E_{i+1,i}$. We see
\begin{equation}
h_{i}=E_{i,i}-E_{i+1,i+1}
\end{equation}
What to do? Well, we have first of all $n$ numbers
$\alpha(h_{1})$, \dots, $\alpha(h_{n})$. We will prove these
numbers may be taken by considering the standard basis in
$\RR^{n}$. 

We will call the name for these
representations \define{Elementary Representations}. First it is
sufficient to find elementary representations, they represent
$\mathscr{G}$ in spaces $V_{1}$, \dots, $V_{n}$. We will take
$V^{\otimes m_{1}}_{1}\otimes\cdots\otimes V^{\otimes
m_{n}}_{n}$, and the highest weights $\alpha_{1}$, \dots,
$\alpha_{n}$ with the corresponding highest weight vectors
$\vec{v}_{1}$, \dots, $\vec{v}_{n}$, then the corresponding
weight vectors in $V^{\otimes m_{1}}_{1}\otimes\cdots\otimes V^{\otimes
m_{n}}_{n}$ have weights
$m_{1}\alpha_{1}+\cdots+m_{n}\alpha_{n}$. If we analyze these
weights, we may consider any representation constructed from the
elementary representations.

What to do? Construct the elementary representation, which is
very easy\dots we take the fundamental representation. If
$(\varphi_{1},\dots,\varphi_{n})$ are the coordinates of the
Cartan subalgebra (bear in mind because we work with
$\ClassicalGroup{A}_{n}$ we have
$\varphi_{1}+\dots+\varphi_{n}=0$ and we work with diagonal
matrices), then the weights are simply $\varphi_{1}$, \dots,
$\varphi_{n}$. The highest weight correspond to $\varphi_{1}$. We
see
\begin{equation}
\alpha_{1}(h_{k}) = \begin{cases}1 & k=1\\
0 & \text{otherwise}
\end{cases}
\end{equation}
We would like to now note this corresponds to $(1,0,\dots,0)$.

We want to consider $(0,1,0,\dots,0)$. This is constructed by
considering $\Antisymmetric^{2}(V)$, the antisymmetric part of
$V\otimes V$. The highest weight vector is $v_{1}\otimes
v_{2}-v_{2}\otimes v_{1}$, and the corresponding weight is
$\alpha_{1}+\alpha_{2}$. We see
\begin{equation}
(\alpha_{1}+\alpha_{2})(h_{k})=\begin{cases}0 & k\not=2\\
1 & k=2
\end{cases}
\end{equation}
This corresponds to the desired $(0,1,0,\dots,0)$.

The general case we have the highest weight be
$\alpha_{1}+\dots+\alpha_{k}$, which corresponds to the
representation $\Antisymmetric^{k}(V)$ --- the antismmetric part
of $V^{\otimes k}$. The highest weight vector is then $\vec{v}_{[1}\otimes\vec{v}_{2}\otimes\cdots\otimes\vec{v}_{k]}$.

\lecture
%%
%% lecture21.tex
%% 
%% Made by alex
%% Login   <alex@tomato>
%% 
%% Started on  Wed Oct  5 11:56:20 2011 alex
%% Last update Wed Oct  5 11:56:20 2011 alex
%%
Now, last time we covered the inverse Laplace transform. Let
$F(z)$ be analytic in $\CC$ with possibly only finitely many
poles. Then $F=\widetilde{f}$, and we obtain the original
function by
\begin{equation}
f(t)=\sum\left(\mbox{Residues of }\E^{tz}F(z)\right)
\end{equation}
The poles of this function comes entirely from $F(z)$ since
$\E^{tz}$ has no poles. We suppose that $F(z)$ is analytic on
$\CC$ except for a finite number of isolated singularities and
for some $\sigma\in\RR$ we have $F$ be analytic on the plane
$\{z\in\CC\lst\re(z)>\sigma\}$. 

The requirements: there are 3 positive constants $M$, $R$,
$\beta>0$ such that if $\|z\|>R$ then
\begin{equation}
\|F(z)\|<\frac{M}{\|z\|^{\beta}}=M(\|z\|^{-\beta})
\end{equation}
This is some contour integral with the requirement as
$\|z\|\to\infty$, then on the boundary $\|F(z)\|$ is ``really
small''. 

\begin{wrapfigure}{l}{1.45in}
\vspace{-24pt}
\begin{center}
\includegraphics{img/lecture21.0}
\end{center}
\vspace{-36pt}
\end{wrapfigure}
What do we do? We create a rectangle $\Gamma$ which is big enough
to contain all the singularities of $F$. This is doodled to the
left, the $\times$ indicates singularities of $F$.
We break up $\Gamma$ into two bits $\gamma$ which contains all
the singularities and $\widetilde{\gamma}$ which is everything
else. 

We see since all the singularities live inside $\gamma$ that
\begin{equation}
\int_{\gamma}\E^{zt}F(z)\D{z}=2\pi\I f(t)
\end{equation}
How can we check that this is correct?

We take its Laplace transform
\begin{equation}
2\pi\I\widetilde{f}(z)=\int^{\infty}_{0}\E^{-zt}\left[\int_{\gamma}\E^{\zeta t}F(\zeta)\D\zeta\right]\D{t}
\end{equation}
and change the order of integration
\begin{equation*}
2\pi\I\widetilde{f}(z)=\int_{\gamma}\int^{\infty}_{0}\E^{-zt}\E^{\zeta t}F(\zeta)\D\zeta\D{t}.
\end{equation*}
This is a little bit sloppy, it is really
\begin{equation}
2\pi\I\widetilde{f}(z)=\lim_{r\to\infty}\int_{\gamma}\int^{r}_{0}\E^{-zt}\E^{\zeta t}F(\zeta)\D\zeta\D{t}
\end{equation}
We then evaluate the integral and we find
\begin{equation}
\begin{split}
2\pi\I\widetilde{f}(z)&=\lim_{r\to\infty}\int_{\gamma}\left(\E^{(\zeta-z)r}-1\right)\frac{F(\zeta)}{\zeta-z}\D\zeta\\
&=\int_{\gamma}F(\zeta)\left[\frac{-1}{\zeta-z}\right]\D\zeta
\end{split}
\end{equation}
We want to show that $F(z)=\widetilde{f}(z)$. 

We use the fact that
\begin{equation}
\int_{\gamma}(\dots)=\int_{\Gamma}(\dots)+\int_{\widetilde{\gamma}}(\dots)
\end{equation}
to deduce
\begin{subequations}
\begin{align}
-2\pi\I\widetilde{f}(z)
&=-\int_{\gamma}\frac{F(\omega)}{\omega-z}\D\omega\\
&=-\int_{\Gamma}\frac{F(\omega)}{\omega-z}\D\omega-\int_{\widetilde{\gamma}}\frac{F(\omega)}{\omega-z}\D\omega\\
&=-\int_{\Gamma}\frac{F(\omega)}{\omega-z}\D\omega-2\pi\I F(z)
\end{align}
\end{subequations}
and we see that
\begin{equation}
\int_{\Gamma}\frac{F(\omega)}{\omega-z}\D\omega\approx 0
\end{equation}
when $\|\omega-z\|\sim R$ and $R$ becomes huge, we basically
divide by ``infinity''. So we have $\widetilde{f}=F$.

\begin{rmk}
The derivative is convolution with the derivative of the delta
function. 
\end{rmk}


This theorem has many corollaries. We see
\begin{equation}
f(t)=\sum(\mbox{residues }\E^{tz}F(z))
\end{equation}
so we can write this as an integral (thanks to the Residue
theorem)
\begin{equation}
f(t)=\frac{1}{2\pi\I}\int^{a+\I\infty}_{a-\I\infty}\E^{zt}F(z)\D{z}
\end{equation}
This formula is very close to the Laplace transform, and we
derived various properties of the Laplace transform using only
integration by parts (which means the inverse transform has
analogous properties).

\lecture
%%
%% lecture22.tex
%% 
%% Made by alex
%% Login   <alex@tomato>
%% 
%% Started on  Sat Dec 31 12:00:40 2011 alex
%% Last update Sat Dec 31 12:00:40 2011 alex
%%

Let $\Sigma$ be a 2-dimensional manifold\index{Manifold}. Let
$\widetilde{\Sigma}$ be the universal cover of $\Sigma$. But
there are only two choices for $\widetilde{\Sigma}$: $S^2$ or
$\RR^2$. For simplicity we will suppose that $\Sigma$ is
compact. We know how to calculate $\pi_{1}(\Sigma)$, the only
thing we need is the statement that $\pi_{1}(\Sigma)$ is finite
in two cases: $\Sigma=S^2$ or $\RP^2$. This is easy looking at
the Abelianization of the fundamental group. In both of these
cases, $\widetilde{\Sigma}=S^2$. We have
\begin{equation}
\pi_{k}(\Sigma)=\pi_{k}(\widetilde{\Sigma})
\end{equation}
for $k\geq2$. Now let us suppose $\pi_{1}(\Sigma)$ is
infinite. Then $\widetilde{\Sigma}$ is not compact. Why? Because
when we look at the covering
\begin{equation*}
\widetilde{\Sigma}\to\Sigma
\end{equation*}
the number of sheets in this covering are the number of elements
in $\pi_{1}(\Sigma)$, which is infinite. Over every disc, we have
an infinite number of discs, which is definitely noncompact. We
have only one chocie for $\widetilde{\Sigma}$. We see then that
\begin{equation}
\pi_{k}(\Sigma)=\pi_{k}(\RR^2)=0
\end{equation}
for $k\geq2$.

We would like to show 
\begin{equation}
\pi_{k}(X\times Y)\iso\pi_{k}(X)\times\pi_{k}(Y).
\end{equation}
It's a one minute proof. If we have a mapping
\begin{equation}
f\colon Z\to X\times Y=\{(x,y)\}
\end{equation}
this map means $(x,y)=f(z)$. This means
\begin{equation}
x=f_{1}(z),\quad\mbox{and}\quad y=f_{2}(z).
\end{equation}
When we apply this to spheroids, everything follows. When we
deform $f(z)$, we deform these two guys. We have, e.g., an
$n$-torus be
\begin{equation}
T^n=(S^1)^n
\end{equation}
so $\pi_{k}(T^n)\iso\pi_{k}(S^1)^n$.
 
\subsection{Relative Homotopy Groups}\index{Homotopy Group!Relative}
\index{Relative Homotopy Group}\index{Reduced Homotopy Group|see{Relative Homotopy Group}}%
\index{Relative Homotopy Group!Construction with Spheroids|(}
We will have a pair of topological spaces $X$, $A$ (so
$A\propersubset X$), and consider $*\in A\propersubset X$. For
simplicity, $A$ and $X$ are connected. We will define a
\define{Relative Homotopy Group} $\pi_{n}(X,A,*)$ or sometimes
$\pi_{n}(X,A)$. We will neglect something in $X$, namely, we
neglect $A$ --- this is the notion of ``relative''.

Recall we defined $\pi_n$ by means of spheroids
\begin{equation*}
(S^n,*)\to(X,*).
\end{equation*}
The homotopy group $\pi_{n}(X,*)$ is then the homotopy classes of
spheroids. This is nice but incomplete. We need to define an
operation. We did this by considering a
spheroid\index{Spheroid!as Map on Cube} as a map on a
cube, generalizing
concatenation\index{Concatenation!Generalization of ---}.

Lets consider something similar for relative homotopy groups. We
introduce \define{Relative Spheroids}\index{Spheroid!Relative}\index{Relative Spheroid} %
$(\bar{D}^n,S^{n-1},*)$ where 
\begin{equation}
S^{n-1}=\partial\bar{D}^{n},
\end{equation}
and relative spheroid is a map
\begin{equation}
(\bar{D}^n,S^{n-1},*)\to(X,A,*).
\end{equation}
This means we have a map
\begin{equation}
f\colon\bar{D}^{n}\to X
\end{equation}
such that
\begin{equation}
f\colon\partial\bar{D}^{n}\to A
\end{equation}
but $f(*)=*$. Such a map is a relative spheroid.

The relative homotopy group $\pi_{n}(X,A,*)$\index{Relative Homotopy Group!in Terms of Relative Spheroids}
is a set of homotopy classes of relative spheroids. It's exactly
the same for ordinary homotopy group, the only difference is we
use relative spheroids. We will discuss the operation later on.

One relation we'd like to note is we have a map
\begin{equation}
\pi_{n}(X,A,*)\to\pi_{n-1}(A,*)
\end{equation}
from the relative homotopy group to the full homotopy group. Why?
For a trivial reason that a relative spheroid
\begin{equation}
f\colon(\bar{D}^{n},S^{n-1},*)\to(X,A,*)
\end{equation}
is really a full spheroid on $A$. Thus $f\colon(S^{n-1},*)\to(A,*)$
is an $(n-1)$-spheroid.
\index{Relative Homotopy Group!Construction with Spheroids|)}

A different formulation of the relative homotopy group.
Recall we considered the space $\Omega$ of all closed loops
starting and ending at $*$. We consider 
\begin{equation*}
\pi_{n}(X,*)=\pi_{n-1}(\Omega,*)
\end{equation*}

\begin{wrapfigure}{r}{1.5in}
  \centering
  \includegraphics{img/lecture22.0}
\end{wrapfigure}\noindent\ignorespaces %
Now what about the relative groups? We have $A\propersubset X$,
consider all paths that are closed modulo $A$. That is to say
\begin{equation}
\Omega(A)=\{f\colon I\to X\mid f(0)=*, f(1)\in A\}
\end{equation}
This sort of path is doodled on the right, where it begins at the
marked point and ends anywhere inside the gray region.
We may consider its homotopy groups.
Thus we may define $\pi_{n}(X,A,*)=\pi_{n-1}(\Omega(A),*)$ where
$*(t)=*$ is the stationary path. Okay, this is a definition. We
see this is a group. Also, for $n\geq3$ we see $\pi_{n}(X,A,*)$
is Abelian. The only problem is that this is not a very good
definition. 

Let us decode this definition. Consider
$\pi_{n-1}\bigl(\Omega(A),*\bigr)$. What is this? We define this
in terms of spheroids as a map of a cube
\begin{equation}
f\colon I^{n-1}\to\Omega(A)
\end{equation}
which sends $\partial I^{n-1}\to*$. What is $\Omega(A)$? It
consists of paths. Our function $f(t_1,\dots,t_{n-1})$ itself is
a path, so really
\begin{equation}
f=f(t_1,\dots,t_{n-1},\tau)\in X.
\end{equation}
What ar ethe conditions on $f$? First of all, the condition is
\begin{equation}
f_{\tau}\colon\partial I^{n-1}\to *(\tau),
\end{equation}
so
\begin{equation}
f(\partial I^{n-1},\tau)=*(\tau).
\end{equation}
Another is that, for $\tau=0$, we have
\begin{equation}
f(\dots,0)=*
\end{equation}
be our marked point, whereas for $\tau=1$ we require
\begin{equation}
f(\dots,1)\in A
\end{equation}
and that's it!

\begin{wrapfigure}{r}{1.5in}
  \vspace{-30pt}
  \centering
  \includegraphics{img/lecture22.1}
  \vspace{-24pt}
\end{wrapfigure}
We will try to reconcile everything. Consider $f\colon I^n\to X$.
For $n=2$, we will have a square and when $\tau=1$ we go to
$A$. This is doodled on the right. When $t=0,1$ we go to $*$ and
when $\tau=0$ we also go to $*$.

\marginpar{This needs to be rewritten for clarity} 
\begin{wrapfigure}{l}{0.75in}
  \vspace{-12pt}
  \centering
  \includegraphics{img/lecture22.2}
  \vspace{-12pt}
\end{wrapfigure}
For the $n=3$ case, we have the top face go to $A$ (it is shaded
grey). This is really what is defined by Hatcher as a relative
spheroid.
But this is already defined. We should prove it is the same.
Here we have the map of a ball, which sends its boundary to
$A$. But here we have some extra stuff, namely, the rest of the
boundary. We identify it witha pouint. So really, this is
equvialent to a ball with a marked point.
Consider
\begin{equation}
\left(I^n,I^{n-1}\times\{1\},(\partial I^{n-1}\times I)\cup(I^{n-1}\times\{0\})\right)\mapsto(\bar{D}^{n},S^{n-1},*)
\end{equation}
since they both are mapped to $(X,A,*)$.
So this identifies the notion of a relative spheroid with
Hatcher's notion of a spheroid. Observe
\begin{equation*}
(\partial I^{n-1}\times I)\cup(I^{n-1}\times\{0\})
\end{equation*}
is a contractible set. When we contract, the upperface is mapped
to $S^{n-1}$ the whole boundary. So our new sense of relative
spheroid agrees with the relative spheroid in the old sense.

\lecture[Applications of $K$-Theory]
%%
%% lecture23.tex
%% 
%% Made by alex
%% Login   <alex@tomato>
%% 
%% Started on  Mon Dec 26 11:07:34 2011 alex
%% Last update Mon Dec 26 11:07:34 2011 alex
%%
Now, what about applications of $K$-theory? It has a lot of
applications, since vector bundles are considered in a lot of
brancehs of nmathematics. Perhaps the most surprising application
is to division aalgebras. We consider algebras over $\RR$. Now,
recall
\begin{equation}
\mbox{algebra} = \mbox{ring}+(\mbox{vector space})
\end{equation}
with some compatibility conditions. What are some examples? Well,
$\RR$ is an algebra over $\RR$, and $\CC$ is an algebra over
$\RR$ too (of dimensions 1 and 2, respectively). We will focus on
\define{Division Algebras}\index{Division Algebra} where, if 
\begin{equation}\label{eq:lec23:divAlgProb}
ax=b\quad\mbox{and}\quad a\not=0
\end{equation}
then we may solve this equation (``we may divide''). If we have
associativity, it is sufficient to solve
\begin{equation}
ax=1
\end{equation}
and denote the solution by $a^{-1}$. Then the solution to Eq
\eqref{eq:lec23:divAlgProb} would be
\begin{equation}
x=a^{-1}b.
\end{equation}
But this \emph{requires} associativity. In general, we don't
necessarily have associativity.

\marginpar{Quaternions}\index{Quaternions}Another division algebra is the
\define{Quaternion Algebra}, denoted by $\HH$ after its inventor
Hamilton. It is associative and noncommutative as well as
4-dimensional. It is a normed algebra, meaning
\begin{equation}
\|xy\|=\|x\|\cdot\|y\|.
\end{equation}
We may consider the situation when
\begin{equation}
\|x\|=1
\end{equation}
which is $S^{3}$ and a group isomorphic to $\SU{2}$. Note that
$\CC$ may be considered likewise, giving us an isomorphism
$S^{1}\iso\U{1}$. 

\marginpar{Octonions}The next algebra is the \define{Octonions}\index{Octonions}
which is an 8-dimensional nonassociative algebra. We may consider
a sphere $S^{7}$ but it is not a group---we don't even have associativity!
It is not terrible, the octonions are ``weakly nonassociative''
(in the sense that they are ``alternative'' --- when we attempt
to modify parentheses, it costs us a sign).
For more on the Octonions, see Baez's beautiful review paper~\cite{baez}.

Can we find other division algebras? Well, all division algebras
have dimension 1, 2, 4, and 8. This may be proven by
$K$-theory. Lets note it is a purely algebraic problem. Does it
have something to do with topology? Yes. Why? Look! Look first at
the case when we have associativity and a norm (this is too much,
we don't need it, but lets suppose). We have $n$ be the
dimension, then $S^{n-1}$ be a group. But a group has the
property that the group manifold is ``parallelizable''. What does
it mean? We have a $k$-dimensional manifold $M$, we may consider
the tangent bundle as a principal $\GL{k}$-bundle. In general, we
cannot continuously take a basis at each $T_{x}M$ for all $x\in
M$, but if we can then we have a group. It's a Lie group! Fo a
Lie group, we consider some basis at a point (e.g., the identity
element $e$), then act by $g\in M$ to smoothly map the basis to
$T_{g}M$. We did not use (nor need) associativity so far.

In reality, we may say the following thing: if the dimension of
the division algebra is $n$, then $S^{n-1}$ is parallelizable.

\bigbreak
The next application which is much more important, a revolution
during a time when analysis and topology were disjoint and
uninteracting fields. What happened? $K$-theory proved to be
useful when studying partial differential equations. We should
explain the notion of the \index{Operator Index|textbf}\index{Inder!of Operator}\define{Index} \textbf{of an Operator},
suppose $E_{1}$ and $E_{2}$ are infinite-dimensional linear
spaces and 
\begin{equation}
A\colon E_{1}\to E_{2}
\end{equation}
is a linear operator. 
We may consider $\ker(A)\propersubset E_{1}$ and
$\coker(A)=E_{2}/\im(A)$, then the index is
\begin{equation}
\ind(A)=\dim\bigl(\ker(A)\bigr)-\dim\bigl(\coker(A)\bigr).
\end{equation}
The only question is: is this well-defined? Could we get
\begin{equation}
\ind(A)=\infty-\infty?
\end{equation}
If we require $\ker(A)$ and $\coker(A)$ to both be
finite-dimensional, then we call $A$ a \define{Fredholm Operator}\index{Operator!Fredholm}\index{Fredholm Operator}
which is an isomorphism up to some one-dimensional problems, we
require $A(E_{1})$ to be closed in $E_{2}$.

In the finite-dimensional case,
\begin{equation}
\begin{split}
\dim\bigl(\coker(A)\bigr)
&=\dim(E_{2})-\dim\bigl(A(E_{1})\bigr)\\
&=\dim(E_{2})-\dim\left(\dim(E_{1})-\dim\bigl(\ker(A)\bigr)\right).
\end{split}
\end{equation}
But this won't work for infinite-dimensional cases. We will say
that the index is continuous (if we change our operator ``a
little bit'', the index ``continuously changes''). This means
that we have thus
\begin{equation}
\ind(A+\alpha)=\ind(A)
\end{equation}
if $\alpha$ is ``small'', i.e., 
\begin{equation}
\|\alpha\|<(\mbox{some number}).
\end{equation}
But we may use topology to have some information about the index.

We use this to deform an operator to a simpler operator. This is
a very topological problem. But we should stary in the domain of
Fredholm operators. Elliptic operators play an important role,
specifically elliptic differential operators (or, better,
elliptic pseudodifferential operators which lets us simplify
things better). The result of all this stuff is the
\define{Atiyah--Singer Theorem}\index{Atiyah--Singer Theorem}.

Instead of using the differences of dimension, can we consider
$\ker(A)-\coker(A)$? Why not, use formal differences! If we have
a family of operators $A_{x}$ depending on $x\in X$, then we get
a ``vector bundle''
\begin{equation}
\bigl(\ker(A)-\coker(A)\bigr)_{x}
\end{equation}
which is not quite right, but this ``index'' is an element of
$K(X)$, the $K$ group of $X$.

\appendix
\vfill\eject\renewcommand{\leftmark}{References}\phantomsection\addcontentsline{toc}{section}{References}
%%
%% bibliography.tex
%% 
%% Made by Alex Nelson
%% Login   <alex@tomato3>
%% 
%% Started on  Sat May 21 12:27:45 2011 Alex Nelson
%% Last update Sat May 21 13:42:17 2011 Alex Nelson
%%
\begin{thebibliography}{99}
\bibitem{baez} John C.\ Baez, \newblock
``The Octonions.''\newblock
Eprint: \arXiv[math.RA]{math/0105155}, 56 pages.
\bibitem{bourbaki} Nicolas Bourbaki,
\newblock \emph{General Topology: Chapters 1--4}.
\newblock Springer (1998) 452 pages.
\newblock Useful reference for topological groups.
\bibitem{fuchs} Dmitry Fuchs, \newblock
``$K$-Theory and Other Extraordinary Cohomology Theories.''\newblock
Notes, handout.
\bibitem{hatcher2006algebraic} Allen Hatcher,
\newblock\emph{Algebraic Topology}.
\newblock Cambridge University Press (2002).
\newblock Eprint: \url{http://www.math.cornell.edu/~hatcher/AT/ATpage.html}
xii+550 pages.
\bibitem{hatcher2009vector} Allen Hatcher,
\newblock\emph{Vector Bundles \& K-Theory}.
\newblock Eprint:
\url{http://www.math.cornell.edu/~hatcher/VBKT/VBpage.html} version 2.1,
\newblock May (2009) 110 pages.
\bibitem{milnor} John W.\ Milnor and James D.\ Stasheff,\newblock
\emph{Characteristic Classes}.\newblock
Princeton University Press, 1974.
\bibitem{schwarz} Albert Schwarz,
\newblock\emph{Topology for Physicists}.
\newblock Springer--Verlag (2010) 307 pages. 
\bibitem{steenrod} Norman Steenrod,
\newblock \emph{The Topology of Fibre Bundles}
\newblock Princeton Landmarks in Mathematics \textbf{14}.
\newblock Princeton University Press (1999) 224 pages.
\end{thebibliography}
Note that these items are references that I have found useful
while studying algebraic topology. 

\vfill\eject
\printindex
\end{document}
