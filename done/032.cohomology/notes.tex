%%
%% notes.tex
%% 
%% Made by Alex Nelson
%% Login   <alex@tomato3>
%% 
%% Started on  Sun Apr 10 12:01:27 2011 Alex Nelson
%% Last update Sun Aug 14 15:37:52 2011 Alex Nelson
%%
\documentclass{article}
\pdfminorversion=4
\pdfcompresslevel9
\pdfinfo{/CreationDate (D:20110410120127)}
\usepackage{xmpincl}
\usepackage{makeidx}
\includexmp{CC_Attribution-NonCommercial-NoDerivs_3.0_Unported}
\usepackage{wrapfig}
\usepackage{notebk}
\makeindex
\diagramstyle[nohug,labelstyle=\scriptstyle]

\advance\headsep by1pc

\makeatletter
\def\lecture{\@ifnextchar[{\@lectureWith}{\@lectureWithout}}
\def\@lectureWith[#1]{\bigbreak\refstepcounter{section}\renewcommand{\leftmark}{Lecture \thesection}
  \noindent{\addcontentsline{toc}{section}{Lecture \thesection: #1\@addpunct{.}}\sectionfont Lecture \thesection. #1\@addpunct{.}}\medbreak}
\def\@lectureWithout{\bigbreak\refstepcounter{section}\renewcommand{\leftmark}{Lecture \thesection}
  \noindent{\addcontentsline{toc}{section}{Lecture \thesection}\sectionfont Lecture \thesection.}\medbreak}
\makeatother

\renewenvironment{exercise}{\refstepcounter{exercise}\medbreak%
  \noindent\llap{\manual\char'170\rm\kern.15em}% triangle in margin
  \small{\textbf{EXERCISE \theexercise}}\\%
  \noindent}{}
\def\ansno#1:{\medbreak\noindent%
  %\hbox to\parindent
  {\bf Answer to #1:\enspace}\ignorespaces}
\renewcommand{\answer}{\par\medbreak % \answer simply
  \ansno\theexercise: \\% prints directly
  } % out when it's called 
%\renewcommand{\theequation}{\arabic{equation}}

\def\CC{\mathbb{C}}
\def\NN{\mathbb{N}}
\def\QQ{\mathbb{Q}}
\def\RR{\mathbb{R}}
\def\ZZ{\mathbb{Z}}


\def\id{\mathrm{id}}
\def\dihedral#1{D_{#1}}
\def\GL{\mathrm{GL}}
\def\Mat{\mathrm{Mat}}
\def\SL{\mathrm{SL}}
%\def\normalSubgroup{\triangleleft}%{\mathrel{\unlhd}}
\def\properNormalSubgroup{\trianglelefteq}%{\mathrel{\lhd}}
\def\normalSubgroup{\mathrel{\lhd}}

\def\powerset{\mathcal{P}}
\def\hom{\mathop{\rm Hom}\nolimits}
\def\tr{\mathop{\rm tr}\nolimits}

% calculus
\def\D{\mathrm{d}}
\def\E{\mathrm{e}}
\def\I{\mathrm{i}} % = \sqrt{-1}

\let\vec\boldsymbol
\newcommand\T{\mathrm{T}}
\def\deg{\operatorname{deg}}
\DeclareMathOperator{\Vect}{Vect}
%\DeclareMathOperator{\id}{id}
%\newcommand\patch[1]{\mathbf{#1}}
\newcommand\chart[1]{\mathbf{#1}}
%\DeclareMathOperator{\dim}{dim}

\def\poincareDual#1{\mathcal{D}#1}

\theoremstyle{definition}
\newtheorem{xca}{\llap{\manual\char'170\rm\kern.15em}Exercise}
\title{Algebraic Topology Notes}
\date{April 10, 2011}
\begin{document}
\maketitle
\begin{abstract}
These are my lecture notes from Albert Schwarz's ``Algebraic Topology''
graduate course Math 215C at UC Davis, taken Spring quarter 2011.
\end{abstract}
\tableofcontents
\vfill\eject

\part{$K$-Theory and Characteristic Classes}
\lecture[Fibre Bundles]
%%
%% lecture01.tex
%% 
%% Made by alex
%% Login   <alex@tomato>
%% 
%% Started on  Mon Dec 26 11:43:13 2011 alex
%% Last update Mon Dec 26 11:43:13 2011 alex
%%

We will return to fibrations, discussing pricnipal fibrations and
pricnipal bundles, characteristic classes, spectral sequences,
and equivariant homology. We may get into $G$-structures. We will
start with repeating the notion of fibrations, and some
additional stuff.

Remember if we have a map $p\colon E\to B$ which we assume is
surjective, then 
\begin{equation}
F_{b}=p^{-1}\{b\}
\end{equation}
is the preimage of $b$ and we can decompose
\begin{equation}
E = \bigsqcup_{b\in B}F_{b}
\end{equation}
as a collection of preimages called \define{Fibres}\index{Fibre|textbf}. If all the
fibres are homeomorphic to
\begin{equation}
F\iso F_{b}\quad\mbox{for all }b\in B
\end{equation}
then $p$ is called a \define{Fibration}. 

\begin{rmk}[Notation]
Some people write the fibre bundle as $(E,F,B,p)$ whereas others
write it out as $F\into E\xrightarrow{p}B$. We will use both
notations, in order to confuse the reader!
\end{rmk}

We should ask for more!
W may introduce a notion of a \define{Locally Trivial Fibration}\index{Locally Trivial Fibration}\index{Fibration!Locally Trivial|textbf}. 
First, a \define{Trivial Fibration}\index{Bundle!Trivial}\index{Trivial Bundle}\index{Trivial Fibration}\index{Fibration!Trivial|textbf} is when
\begin{equation}
E=B\times F\quad\mbox{and}\quad p(b,f)=b.
\end{equation}
For locally trivial fibrations, we have locally for ``small
sets'' $U\propersubset B$ that 
\begin{equation}
p^{-1}(U)= U\times F.
\end{equation}

\begin{wrapfigure}{r}{0.75in}
  \vspace{-12pt}
  \includegraphics{img/lecture1.0}
\end{wrapfigure}

\noindent{}For example, th Mobius band\index{Mobius Band} is a locally trivial
fibration. How can we see this? Well, it is doodled on the right
as such. This notion isn't rigorous, for example what is meant by
``equality''? We have a notion of mappings of fibrations, if
$p\colon E\to B$ and $p'\colon E'\to B'$ are fibrations, then
\begin{equation}
\begin{diagram}[small]
f\colon& E &\rTo & E'\\
       &\dTo<{p}& &\dTo>{p'}\\
\varphi\colon& B&\rTo&B'
\end{diagram}
\end{equation}
commutes, which means
\begin{subequations}
\begin{equation}
f\colon F_{b}\to F_{b'}
\end{equation}
where 
\begin{equation}
b'=\varphi(b)
\end{equation}
\end{subequations}
and
\begin{equation}
p'\circ f=\varphi\circ p.
\end{equation}
We have a map of fibrations, and if $f$ is invertible, then the
fibrations are \define{Equivalent}\index{Fibration!Equivalent|textbf} which is what we really mean
for locally trivial fibrations.

\begin{Boxed}{Fibre Bundles}
Let $E$, $F$, $B$ be topological spaces.
The idea of a fibre bundle $F\into E\to B$ is that it is a
natural generalization of the product space $E=B\times F$. Why is
this a good idea? Well, for example, in particle physics we often
generalize the notion of a tangent space to work with ``tangent
spinors''. 
\end{Boxed}

There is a notion of a \define{Section}\index{Section!of a Fibration|textbf} of a fibration $p\colon
E\to B$ is a mapping
\begin{equation}
q\colon B\to E
\end{equation}
such that
\begin{equation}
p\circ q=\id{B},
\end{equation}
i.e., $q(b)\in F_{b}$ for all $b\in B$. What are some examples of
fibrations that we know? Well, if $F$ is discrete and $E$
connected, then the notion of a ``fibration'' is the same as a
``covering space.''\index{Covering Space|textbf}

\begin{thm}[Homotopy Lifting]\index{Homotopy Lifting Theorem|textbf}
If we have a locally trivial fibration $E\to B$, and if we have a
map $\alpha\colon X\to B$, then sometimes we may lift this map,
i.e., find an $\widetilde{\alpha}\colon X\to E$ such that
$\widetilde\alpha(x)\in F_{\alpha(x)}$, i.e.,
$p\circ\widetilde{\alpha}=\alpha$.
\end{thm}

Look, a section is a lift of the identity map. We have examples
where sections don't exist, so the lifts don't always exist. What
is important is that we can do this for homotopies. So if $X$ is
a cell complex, and 
\begin{equation}
\alpha_{t}\colon X\times I\to B
\end{equation}
then we may lift this to a homotopy
\begin{equation}
\widetilde{\alpha}_{t}\colon X\times I\to E
\end{equation}
which has the property
\begin{equation}
\widetilde{\alpha}_{0}=\widetilde{\alpha}.
\end{equation}
If we may lift the homotopy ``at one point'', we may lift the
whole homotopy. 

\index{Fibration!Serre|(}
We may use this theorem to define a fibration. This is important,
as there are some fibrations which are not locally
trivial. Consider any space $B$. Lets fix a point $*\in
B$. Consider all the maps
\begin{equation}
\varphi\colon I\to B
\end{equation}
such that
\begin{equation}
\varphi(0)=*,
\end{equation}
the space of all these paths is called $\Omega$ but we will
denote this by $E$. There is a map 
\begin{equation}
p\colon E\to B
\end{equation}
which sends every path $\varphi$ to $\varphi(1)$ its end
point. What is its fibre? Lets fix some $b\in B$, then
\begin{equation}
p^{-1}(b)=\{\varphi\in E\mid \varphi(0)=*,\varphi(1)=b\} = \Omega_{*,*}=\Omega.
\end{equation}
In general, this is not a fibration (unless $B$ is a manifold),
but this theorem about lifting applies here---just use the
homotopy extension property. This shift is a \define{Serre Fibration}\index{Fibration!Serre|)}.

If we have a locally trivial fibration, or Serre fibration, then
we have an exact homotopy sequence\index{Fibration!Exact Homotopy Sequence}
\begin{equation}
\dots\to\pi_{n}(F,*)\to\pi_{n}(E,*)\to\pi_{n}(B,*)\to\pi_{n-1}(F,*)\to\dots
\end{equation}
the mapping 
\begin{equation}
\pi_{n}(B,*)\to\pi_{n-1}(F,*)
\end{equation}
requires the homotopy lifting theorem above. We also see
\begin{equation}
\pi_{n}(E,F,*)=\pi_{n}(B,*)
\end{equation}
is a consequence of the homotopy lifting theorem.
Observe
\begin{equation}
E=\{\varphi\colon I\to B\mid \varphi(0)=*\}
\end{equation}
is contractible. So this means the exact homotopy sequence gives
$\pi_{n}(B,*)\iso\pi_{n-1}(\Omega)$.

We would like to consider \define{Principal Fibrations}\index{Fibration!Principal|textbf}
by taking a topological group\index{Group!Topological} $G$ and assume $G$ acts on $E$,
then we may consider orbits of $G$ acting on $E$. The space of
orbits is denoted $E/G$, there is a natural map
\begin{equation}
E\to E/G
\end{equation}
and the fibres of this map are precisely the orbits. There is a
situation when the action is \define{Free}\index{Group!Free Action|textbf}\index{Free!Group Action|textbf}
(there are no stabilizers); then the orbits are in one-to-one
correspondence with $G$. Then we are talking about a principal
fibvration. It will be a locally trivial fibration.
\begin{prop}
If a principal fibration has a section (and when $G$ is compact),
then the fibration is trivial.
\end{prop}
The proof is trivial. If we have a section $\sigma\colon B\to E$
and $G$ acts on the right (it'll be important later), take a pair
$(b,g)$ and map it to $\sigma(b)\cdot g$. This is a map
\begin{equation}
B\times G\to E
\end{equation}
which is continuous and injective.

\lecture
%%
%% lecture02.tex
%% 
%% Made by alex
%% Login   <alex@tomato>
%% 
%% Started on  Thu Sep 29 08:34:48 2011 alex
%% Last update Thu Sep 29 08:34:48 2011 alex
%%

We will follow Hatcher's book, and Schwarz's \emph{Topology for Physicists}.

The first thing to discuss is topological
spaces\index{Topological Space}\index{Space!Topological}. We have a set
$E$ and a notion of an open set of $E$. We have a collection of
open subsets of $E$, $\{U\}$ where
\begin{subequations}
\begin{equation}
\bigcup U=E
\end{equation}
and
\begin{equation}
\bigcap_{\text{finite}} U\mbox{ is open}
\end{equation}
\end{subequations}
It is not enough to say 
\begin{subequations}
\begin{equation}
\bigcup \mbox{(open)}=\mbox{(open)}
\end{equation}
and
\begin{equation}
\bigcap_{\text{finite}} \mbox{(open)}=\mbox{(open)}
\end{equation}
\end{subequations}
We have closed sets be the complement of open sets. There is a
requirement that $E$ is both open and closed, which then implies
that $\emptyset$ is both open and closed.

We can define a continuous function $f\colon E\to E'$ such that
the preimage of open sets is open, i.e.,
\begin{equation}
f^{-1}(\mbox{open})=\mbox{open}.
\end{equation}

\begin{Boxed}{Functorial view of Topology, Continuous Functions}
\index{Topological Space!and Stuff, Structure, Properties}
\index{Continuous Function!from Stuff, Structure, Properties}
This may seem odd at first why continuous functions obey this
pre-image condition. There are a variety of explanations out
there, but I prefer this explanation. Consider the category
$\Set$. Let\index{$\hom(-,\mathbf{2})$|(}
\begin{equation}
\hom(-,\mathbf{2})\colon\Set^{\op}\to\Set
\end{equation}
be the contravariant power set functor. So in other words, we
have
\begin{equation}
\hom(X,\mathbf{2})=\begin{pmatrix}\mbox{set of indicator}\\
\mbox{functions for subsets}\\
\mbox{of the set $X$}
\end{pmatrix}
\end{equation}
We construct a topology by picking a subset of this collection of
subsets $\hom(X,\mathbf{2})$ which obey the axioms for a
topology. That is, we have $T\subset\hom(X,\mathbf{2})$ be a
topology of $X$. That is to say, $T$ consists of the indicator
functions for open subsets of $X$. It is a structure-type.
A topological space is then $(X,T)$. 

But note that we functor, so we have the immediate question:
\begin{quest}
How does $\hom(X\xrightarrow{\;f\;}Y,\mathbf{2})$ behave?
\end{quest}
If we can answer this question, then we will have some idea of
what a ``topological-space morphism'' would be like. Why? Because
we just restrict focus to the functions preserving the
``topological structure'' $T\subset\hom(X,\mathbf{2})$.

We should recall from our knowledge of category theory that the
functor $\hom(-,B)$ behaves on morphisms in the following manner:
$\hom(-,B)$ maps each morphism $h\colon X\to Y$ to the function
$\hom(h, B)\colon \hom(Y, B)\to\hom(X, B)$ given by $g \mapsto
g\circ h$ for each $g$ in $\hom(Y, B)$.  

What does this mean for our situation? Well, for each $f\colon
X\to Y$ it is mapped to the function 
\begin{equation}
\hom(f,\mathbf{2})\colon\hom(Y,\mathbf{2})\to\hom(X,\mathbf{2})
\end{equation}
given by $f\mapsto h\circ f$ where $h\in\hom(Y,\mathbf{2})$. So
$h$ is really an indicator function of an open subset of
$Y$. This is precisely the same condition as saying the preimage
$f^{-1}(\mbox{open})$ is open.\index{Open Set}\index{$\hom(-,\mathbf{2})$|)}
For a brief introduction to topology using this approach, see Nelson~\cite{nelson}.
\end{Boxed}

Usually we use the Hausdorff condition\index{Hausdorff Condition} that two distinct points
are contained in two disjoint neighborhoods.

There is a topological property of \define{Compactness}\index{Compactness} where
every open covering has a finite subcovering.

There is one more thing that is relevant. What can we do with
equivalence relations on topological spaces? We can consider
equivalence classes\index{Equivalence Classes} $E/\sim$. There is a natural map
\begin{equation}
\pi\colon E\to E/\sim
\end{equation}
What happens if $E$ is a topological space, then we would like
to have $E/\sim$ be a topological space and the map $\pi$ to be
continuous, i.e., the preimage $\pi^{-1}(\mbox{open})$ is
open. We are saying $U\propersubset E/\sim$ is open iff the
preimage $\pi^{-1}(U)$ is open in $E$. If the preimage of an open
set is open, then the preimage of a closed set is closed. We see
for a singleton $a\in E/\sim$ then the preimage $\pi^{-1}(a)$ is
an equivalence class. We have the singletons be closed, so we
require these equivalence classes be closed to avoid pathology.

\marginpar{List of topological spaces}Now why are we so interested in this construction? Because we
want to have a construction of interesting topological spaces. We
have some simple interesting topological spaces. What are they?
First of all, $\RR^{3}$ the space that surrounds us. \marginpar{$\RR^{n}$}More
generally $\RR^{n}$. Another interesting space is a ball
\begin{equation}\index{$\bar{D}^{n}$}
\overline{D}^{n}=\{\vec{x}\in\RR^{n}\lst\|\vec{x}\|\leq1\}
\end{equation}
\marginpar{Closed ball $\bar{D}^{n}$}which is closed of radius 1. The radius doesn't change anything,
balls of different radius are topologically equivalent. For
example $x\mapsto\lambda x$ for $\lambda>0$ is the topological
equivalence. We won't repeat the definition of ``topological
equivalence'' the curious reader may look it up. Another
interesting space is the open ball\index{$D^{n}$}\index{Open Ball}\marginpar{Open ball $D^{n}$}
\begin{equation}
D^{n}=\{\vec{x}\in\RR^{n}\lst\|\vec{x}\|<1\}
\end{equation}
We use the notation $\overline{D}^{n}$ to stress it is the
closure of $D^{n}$. It's an interesting space, perhaps it is
equivalent to $\overline{D}^{n}$? No! Why? Well, we see that
$D^{n}$ is not compact but $\overline{D}^{n}$ is compact, so they
cannot be topologically equivalent.

\begin{wrapfigure}{r}{2in}
  \vspace{-30pt}
  \begin{center}
    \includegraphics{img/lecture2.0}
  \end{center}
  \vspace{-20pt}
\end{wrapfigure}
But is $\RR^{n}$ topologically equivalent to $D^{n}$? Yes, we can
see this for $n=1$, we use the stereographic
projection\index{Stereographic Projection} which
gives us a one-to-one correspondence between
$S^{1}\setminus\{0\}$ and $\RR^{1}$. We can take $n=2$ and
nothing conceptually changes. The same is true for
$S^{n}\setminus\{0\}\iso\RR^{n}$. But we may say that
$S^{n}\setminus\{0\}\iso D^{n}$.

We would like to stress that the $n$-dimensional sphere is
\emph{not} a sphere in $n$-dimensional space. No, it is instead
living in $\RR^{n+1}$. In $\RR^{n}$, the sphere is characterized
by the points $\vec{x}\in\RR^{n}$ satisfying
\begin{equation}
\|\vec{x}\|=1
\end{equation}
which is an $(n-1)$-sphere. We have 
\begin{equation}
\overline{D}^{n}=S^{n-1}\cup D^{n}
\end{equation}
where $D^{n}$ are the interior points and $S^{n-1}$ is the boundary.

Now what we would like to say is that, more or less, all
interesting spaces may be constructed from the simple spaces
$D^{n}$, $\overline{D}^{n}$. We define a very general
construction, namely given a topological space $X$, and a
topological space $Y$, a closed subset $A\propersubset Y$, we'd
like to paste together $X$ and $Y$ along $A$. What does this
mean? We take any continuous map
\begin{equation}
f\colon A\to X
\end{equation}
take the disjoint union $X\sqcup Y$, and then in this disjoint
union introduce an equivalence relation that any $a\in
A\propersubset Y \sim f(a)\in X$ and no other equivalences!

\begin{wrapfigure}{l}{0.5in}
  \vspace{-10pt}
  \begin{center}
    \includegraphics{img/lecture2.1}
  \end{center}
  \vspace{-20pt}
\end{wrapfigure}
\noindent{}We require $a\in A\propersubset Y\sim f(a)\in X$ is the only
nontrivial equivalence. Lets consider some examples. The Mobius
band\index{Mobius Band!Construction of} can be defined in this way: take a rectangle (which is
topologically equivalent to $D^{2}$) and we consider the
equivalence relation that the two arrows are pasted together.

\begin{wrapfigure}{r}{2.1in}
  \vspace{-30pt}
  \begin{center}
    \includegraphics{img/lecture2.2}
  \end{center}
  \vspace{-20pt}
\end{wrapfigure}
We see that a rectangle is equivalent to a disc since both are
convex and stretch the boundary to be a rectangle which permits
us to formally write this equivalence but that won't be necessary.
We can stretch according to the gray lines doodled to the right.

\begin{wrapfigure}{l}{1.05in}
  \vspace{-25pt}
  \begin{center}
    \includegraphics{img/lecture2.3}
  \end{center}
  \vspace{-23pt}
\end{wrapfigure}
So more examples. We take the same rectangle, and paste together
points at the same height. This topologically is equivalent to a
cylinder. This is obvious, as we see in the doodle to the left.


\begin{wrapfigure}{r}{1.5in}
  \vspace{-16pt}
    \includegraphics{img/lecture2.5}
  \vspace{-20pt}
\end{wrapfigure}
But if we take our cylinder, and glue the two ends to each other
without any twisting, what do we get? Well, we have a torus. This
is doodled on the right hand side, very carefully, with colors to
show where we glued the rectangle together.
The red line indicates where we glued the rectangle to obtain a
cylinder, and the blue line indicates where we glued the cylinder
to obtain a torus. Do we really need all this information? Is
there some easier diagram which yields the relevant data? Or are
we forced to become artists to understand the topological
properties of these exotic spaces?


There is a very general construction of something called a
\define{Cell Complex}\index{Cell Complex!generalization of ---|see{Complex}}\index{Cell Complex}\index{Complex!Cellular}, we will first describe it. Take a closed
ball and some topological space $X$. Now we will take any
continuous map
\begin{subequations}
\begin{equation}
f\colon S^{n-1}\to X
\end{equation}
or in other words
\begin{equation}
f\colon\partial\overline{D}^{n}\to X
\end{equation}
\end{subequations}
and then we use the construction we just explained. That is, we
glue a closed ball along its boundary to $X$. We get a new set
\begin{subequations}
\begin{equation}
Y=X\cup\overline{D}^{n}
\end{equation}
or as sets
\begin{equation}
Y=X\sqcup\overline{D}^{n}
\end{equation}
\end{subequations}
The simplest posssible case is when $X$ is just a one point space
\begin{equation}
X=\{a\}
\end{equation}
the boundary of the ball goes to $a$. This is a trivial map.
We see in $n=2$ what do we get with identifying the boundary to
$a$? Look at the stereographic projection backwards.
\begin{center}
\includegraphics{img/lecture2.4}
\end{center}
This general construction, gluing the boundary of closed
$n$-balls to a topological space (starting with $n=0$, i.e., a
set of vertices to begin with), gives us a cell complex.

\begin{defn}
The \define{$n$-Dimensional Cell Complex}\index{Cell Complex!$n$-Dimensional|textbf} can be done inductively
by assuming we have the $(n-1)$-dimensional cell complex denoted
$X^{n-1}$ called the \define{$(n-1)$-Skeleton}\index{Skeleton!of Complex|textbf}, now we have
$k$-copies of $n$-discs and perform the same construction. We end
up with a sequence of skeletons $X^{0}\propersubset
X^{1}\propersubset X^{2}\propersubset\dots$, if we consider
$X^{n}\setminus x^{n-1}=\bigsqcup_{k} D^{n}_{k}$.
\end{defn}

\lecture[Classification of Principal Bundles]
%%
%% lecture03.tex
%% 
%% Made by alex
%% Login   <alex@tomato>
%% 
%% Started on  Thu Jan  5 07:57:17 2012 alex
%% Last update Thu Jan  5 07:57:17 2012 alex
%%

We introduced a notion of homology, for every topological space
$X$ the ntoion of the group of chains $C_{n}(X)$ consisting of
linear combinations $\sum c_n\varphi_n$ of singular cubes. We
then took
\begin{equation}
C(X)=\bigoplus_{n}C_{n}(X).
\end{equation}
We introduced the boundary operator
\begin{equation}
\bdry\colon C_{n}(X)\to C_{n-1}(X),
\end{equation}
and it obeys
\begin{equation}
\bdry^2=0.
\end{equation}
We introduced the group of cycles
\begin{equation}
Z(X)=\ker(\bdry)
\end{equation}
and the group of boundaries
\begin{equation}
B(X)=\im(X),
\end{equation}
thus the Homology group is defined as
\begin{equation}
H_{n}(X)=Z_{n}(X)/B_{n}(X).
\end{equation}
We computed $H_{0}(X)$.

\index{Fundamental Group!relation to $H_{1}(X)$|(}
Lets consider $H_{1}(X)$ and its relation with
$\pi_{1}(X)$. There exists a morphism 
\begin{equation}\label{eq:lec03:fundamentalGroupRelatedToHomology}
\pi_{1}(X)\to H_{1}(X). 
\end{equation}
Why? Well, first if we have the fundamental group, we
should mark a point to construct $\pi_{1}(X)$, but what is the
path? It is really a map
\begin{equation}
I\to X
\end{equation}
which is a singular 1-dimensional cube! So really, what is the
morphism described in Eq \eqref{eq:lec03:fundamentalGroupRelatedToHomology}?
It is a mapping
\begin{equation}
(\mbox{Paths})\to(\mbox{Chains}).
\end{equation}
Moreover, it is a mapping
\begin{equation}
(\mbox{Paths})\to(\mbox{Cycles}).
\end{equation}
Why? Well, the fundamental group is concerned with loops, which
has the starting and ending points be the same. But the boundary
of a loop vanishes. So a loop corresponds to a

\begin{wrapfigure}{r}{6pc}
  \vspace{-12pt}
  \includegraphics{B.img/lecture03.0}
  \vspace{-12pt}
\end{wrapfigure}
\noindent\ignorespaces %
cycle\index{Cycle!Relation with Loop}.
It is clear if we have the concatenation of paths, then it is
mapped to a singular cubed that may be ``divided in two''. So we
have two pictures. One gives a path divided in two pieces, and
another picture gives us a singular cube. So really, for us, it
should not be a problem to say they are the same curve. But
honestly, we should give a formula and prove formally these guys
are ``\emph{Homologous\/}''\index{Homologous}\footnote{The notion
of two guys being ``homologous'' amounts to stating \emph{they
  describe the same cycle.} What does this mean? Well, if $\alpha$ and $\beta$ are homologous $k$-chains, then $\alpha-\beta=\bdry\gamma$ for some $(k+1)$-chain $\gamma$. }. 
This is not hard to do, but we won't do it.

\begin{wrapfigure}{l}{6pc}
  \vspace{-12pt}
  \includegraphics{B.img/lecture03.1}
  \vspace{-12pt}
\end{wrapfigure}
So we have this morphism, what can we say?
We have homotopic paths. We should prove that homotopic paths
gives us homologous cycles. 
What does this mean? Well, lets draw the  homotopy. It is a
cylinder $S^1\times I$, what are the boundaries of this cylinder?
This gives us homologous cycles. Admittedly, this is somewhat
handwavy, but lets not worry about it for now.

We constructed a map $\pi_{1}(X)\to H_{1}(X)$. Now we will assume
$X$ is connected.\marginpar{If $X$ connected, we have $\pi_{1}(X)\onto H_{1}(X)$ surjective} This is a reasoanble assumption. We see that
this morphism is surjective:
\begin{equation}
\pi_{1}(X)\onto H_{1}(X).
\end{equation}
Look, suppose we have some cycle in $H_{1}(X)$, which consists of
several ``cubes'' with vanishing boundary. But really this gives
us something which is the image of a circle. Again, we see that a
cycle consists of several circles. We can always add some path
that comes from $x_{0}$, so we can cover $H_{1}(X)$ by
$\pi_{1}(X)$.
This is not a rigorous proof, just convincing handwaviness.

If $\pi_{1}(X)\onto H_{1}(X)$ is surjective, what about the
kernel of this mapping? Well, $\pi_{1}(X)$ is usually
noncommutative, so the commutator
\begin{equation}
[\pi_{1}(X),\pi_{1}(X)]\to(\mbox{Kernel}),
\end{equation}
thus
\begin{equation}
\pi_{1}(X)/[\pi_{1}(X),\pi_{1}(X)]\to H_{1}(X)
\end{equation}
and it is surjective. One can prove this map is injective. So
$H_{1}$ is simply the
\define{Abelianization}\index{Abelianization} of $\pi_{1}$,
provided $X$ is connected.
\index{Fundamental Group!relation to $H_{1}(X)$|)}

\begin{wrapfigure}{r}{4pc}
  \includegraphics{B.img/lecture03.2}
  \vspace{-12pt}
\end{wrapfigure}
Just an aside: we will give an example of something homotopic to
zero, but not homologous to zero. Consider a handle body, as
doodled to the right. Consider the path drawn in red, near the
boundary of the handle. It is homotopic to zero, but not
homologous to zero.

We\marginpar{Functorial Properties of Homology} will consider the
functorial properties of homology, which will help n performing
calculations. If
\begin{equation}
f\colon X\to Y
\end{equation}
is continuous, then it induces
\begin{equation}
H_{k}\bigl(f\colon X\to Y)\quad=\quad
f_{*}\colon H_{k}(X)\to H_{k}(Y)
\end{equation}
We may construct it by considering the behaviour on chains. We
see a chain is a linear combination $\sum c_n\varphi_n$ where
\begin{equation}
\varphi_n\colon I^k\to X,
\end{equation}
but we see by continuity
\begin{equation}
f\colon\varphi_n\colon I^k\to Y.
\end{equation}
Thus we map chains to chains by 
linearity. We have
\begin{equation}
C_{n}(X) = \left.\begin{pmatrix}\mbox{Linear}\\\mbox{Combination}
\end{pmatrix}\!\right/\!\!\begin{pmatrix}\mbox{Degenerate}\\\mbox{Chains}
\end{pmatrix}
\end{equation}
but by continuity $f_{*}$ maps degenerate cubes to degenerate
cubes. Thus $f_{*}$ maps degenerate chains to degenerate chains
by linearity.

Additionally, we have
\begin{equation}
\partial f_{*}=f_{*}\partial.
\end{equation}
As a consequence, we see it maps cycles to cycles and boundaries
to boundaries. So it maps homology to homology, by construction!

Moreover, $(f\circ g)_{*}=f_{*}\circ g_{*}$. That's obvious, and
if we have an identity map $(\id{})_{*}=\id{*}$. These are two
trivial properties.
All of these properties state we have homology be a functor
$\Top\to\Ab$.

Now, a less trivial statement. Suppose $f,g\colon X\to Y$ and we
have $f\homotopic g$ homotopic. That is, we have
\begin{equation}
F\colon X\times I\to Y
\end{equation}
such that
\begin{equation}
F(-,0)=f(-),\quad\mbox{and}\quad F(-,1)=g(-).
\end{equation}
Then the induced morphisms are the same. In other words:
\begin{thm}
If $f,g\colon X\to Y$ and $f\homotopic g$ homotopic, then they
generate the same map of homology $f_{*}=g_{*}$.
\end{thm}

\begin{wrapfigure}{r}{11pc}
  \vspace{-30pt}
  \centering
  \includegraphics{B.img/lecture03.3}
  \vspace{-24pt}
\end{wrapfigure}
\noindent\emph{Proof.\enspace}\ignorespaces %
Suppose we have a cycle in $X$, we call it $z$. We have a
cylinder obtained by $F_{*}$. This is the handwavy sketch of the
proof, and it is doodled on the right. But we will give the
rigorous details.

We have a regular cube $\varphi\colon I^{k}\to X$. Now, from this
singular cube, we have two singular cubes in $Y$, namely
\begin{subequations}
\begin{equation}
f\circ\varphi\colon I^k\to Y
\end{equation}
and
\begin{equation}
g\circ\varphi\colon I^k\to Y.
\end{equation}
\end{subequations}
But we have more! Namely
\begin{equation}
F\colon X\times I\to Y
\end{equation}
but what does this mean? We take
\begin{equation}
I^{k+1}=I^{k}\times I,
\end{equation}
and we compose
\begin{equation}
\widetilde{\varphi}=\varphi\times\id{I}\colon I^{k+1}\to X\times I
\end{equation}
with $F$, so we end up with
\begin{equation}
F\circ\widetilde{\varphi}\colon I^{k+1}\to Y.
\end{equation}

\begin{wrapfigure}{r}{7pc}
  \vspace{-12pt}
  \centering
  \includegraphics{B.img/lecture03.4}
  \vspace{-12pt}
\end{wrapfigure}      
\noindent\ignorespaces %
What will be the boundary $\bdry(F\circ\widetilde{\varphi})$?
We doodle the $k=1$ case on our right, and we immediately see
that
\begin{equation}
\bdry(F\circ\widetilde{\varphi})=g_{*}\varphi-f_{*}\varphi-F(\bdry\varphi).
\end{equation}
Then we may apply this to any chain. Apply it to $z$, we have
\begin{equation}
\bdry(F\widetilde{z})=g_{*}z-f_{*}z-F(\bdry z).
\end{equation}
In the case when $z$ is a cycle, i.e.\ $\bdry z=0$, we get
$g_{*}z-f_{*}z$ is the boundary to \emph{something!} But this
means $f_{*}=g_{*}$ for homology, which is what we wanted.\hfill\qedsymbol\break
\medskip
This proves the statement made earlier regarding homotopic paths
and homologies.

\lecture
%%
%% lecture04.tex
%% 
%% Made by alex
%% Login   <alex@tomato>
%% 
%% Started on  Sun Feb 19 11:51:19 2012 alex
%% Last update Sun Feb 19 11:51:19 2012 alex
%%
There are a few differences with lightlike geodesics and timelike
geodesics. First we use an affine parameter $\lambda$ instead of
$s$. Second $\D s^{2}=0$ between events. So we have
\begin{align*}
g_{ab}\frac{\D x^{a}}{\D s}\frac{\D x^{b}}{\D s}=1
\tag{\text{for a planet}}\\
g_{ab}\frac{\D x^{a}}{\D\lambda}\frac{\D x^{b}}{\D\lambda}=0.
\tag{\text{for light}}
\end{align*}
The only thing that changes is
\begin{equation}
\left(\frac{\D r}{\D\lambda}\right)^{2}=
E^{2}-\left(1-\frac{2m}{r}\right)\frac{L^{2}}{r^{2}}.
\end{equation}
The convention for light is to \emph{not} use tildes on $L$ and
$E$. We also have
\begin{equation}
\left(\frac{\D r}{\D\varphi}\right)^{2}=\frac{L^{2}}{r^{4}}.
\end{equation}
We can see the angle as a function of distance, instead of the
other way around:
\begin{equation}
\left(\frac{\D\varphi}{\D u}\right)^{2}=
\frac{L^{2}}{E^{2}-L^{2}u^{2}(1-2mu)}
\end{equation}
which is the same sort of problem we've seen before. In
particular, the ``Newtonian Approximation''\marginpar{Newtonian Approximation}
is
\begin{equation}
\begin{split}
\frac{\D\varphi}{\D u} 
&=\frac{1}{\sqrt{(E/L)^{2}-u^{2}}}\\
&=\frac{1}{\sqrt{b^{-2}-u^{2}}}.
\end{split}
\end{equation}
The solution for our differential equation is
\begin{equation}
\varphi-\varphi_{0}=\arcsin(bu)
\end{equation}
and thus
\begin{equation}
r\sin(\varphi-\varphi_{0})=b.
\end{equation}
This is a straight line! In this approximation, light moves in a
straight line.

\subsection{First Approximation}
It is useful to use the approximation
\begin{equation}
u^{2}-2mu^{3}\approx u^{2}(1-2mu)^{2}-m^{2}u^{4}.
\end{equation}
Let us define
\begin{equation}
y=u(1-mu),
\end{equation}
we can ignore $m^{2}y^{4}$ relative to $y^{2}$. So
\begin{equation}
\D y=\D u (1-2mu),
\end{equation}
thus
\begin{equation}
\begin{split}
\D u 
&= (1-2mu)^{-1}\D y\\
&\approx (1+2my)\D y.
\end{split}
\end{equation}
Then
\begin{equation}
\begin{split}
\D\varphi
&=\frac{\pm\D u}{\sqrt{(E/L)^{2}-u^{2}(1-2mu)}}\\
&\approx\pm\frac{(1+2my)}{\sqrt{b^{-2}-y^{2}}}\D y.
\end{split}
\end{equation}
So
\begin{equation}
\begin{split}
\frac{1}{2}\Delta\varphi
&=\int^{b^{-1}}_{0}\left(\frac{1+2my}{\sqrt{b^{-2}-y^{2}}}\right)
\D y\\
&=\pi+\frac{4m}{b}.
\end{split}
\end{equation}
This means light is bent, taking a trajectory roughly doodled
thus: 

\begin{center}
  \includegraphics{img/lecture04.0}
\end{center}

\begin{rmk}
Please avoid the temptation to Taylor expand in
\begin{equation}
\frac{\D\varphi}{\D u}=\frac{\pm1}{\sqrt{b^{-2}-u^{2}+2mu^{3}}}
\end{equation}
Do not Taylor expand the right hand side, specifically involving
the $mu^{3}$ term, about 0. We get something circuitous if we try.
\end{rmk}

\subsection{Second Approximation}
An older technique no longer taught, perhaps the most
straightforward, is
\begin{equation}\label{eq:lec3:approx2:omegaSquared}
\omega^{2}=u^{2}-2mu^{3}
\end{equation}
so we have
\begin{equation}
\varphi=\int^{1}_{0}\frac{\D u}{\sqrt{1-\omega^{2}}}.
\end{equation}
%% We have equation \eqref{eq:lec3:approx2:omegaSquared} imply
%% \begin{equation}
%% \omega=u\sqrt{1-2mu}
%% \end{equation}
%% and 
By expanding
\begin{equation}
u=\omega+\alpha_{1}\omega^{2}+\alpha_{2}\omega^{3}
\end{equation}
we get a nice systematic perturbation. Although it is possible to
solve equation \eqref{eq:lec3:approx2:omegaSquared}, that is not
the point! No, what we do is rewrite it as
\begin{equation}
\omega=u\sqrt{1-2mu}
\end{equation}
then Taylor expand the squareroot on the right hand side up to
some term. This is when we make our approximation:
\begin{equation}\label{eq:lec3:approx2:omegaAfterTaylorExpansion}
\omega\approx u-mu^{2}.
\end{equation}
We consider
\begin{equation}
m^{2}u^{4}\lll u
\end{equation}
as our approximation, so squaring equation
\eqref{eq:lec3:approx2:omegaAfterTaylorExpansion} recovers equation \eqref{eq:lec3:approx2:omegaSquared}.


Observe equation
\eqref{eq:lec3:approx2:omegaAfterTaylorExpansion} is a quadratic equation in $u$, which has its
solution
\begin{equation}
u_{\pm}=\frac{1}{2m}(1\pm\sqrt{1-4m\omega}).
\end{equation}
We Taylor expand this to third order in $\omega$, taking the
physically meaningful root $u=u_{-}$
\begin{equation}
\begin{split}
u&\approx\frac{1}{2m}\left(2m\omega+\frac{1}{8}(4m\omega)^{2}+\frac{1}{16}(4m\omega)^{3}
\right)\\
&\approx \omega+m\omega^{2}+\frac{1}{2}m^{2}\omega^{3}.
\end{split}
\end{equation}
Thus
\begin{equation}
\D u=\D\omega+2m\omega\D\omega+\frac{3}{2}m^{2}\omega^{2}\D\omega,
\end{equation}
and our integral becomes
\begin{equation}
\begin{split}
\Delta\varphi
&=\int^{1}_{0}\frac{(1+2m\omega+\frac{3}{2}m^{2}\omega^{2})}{\sqrt{1-\omega^{2}}}\D\omega\\
&=2m+\left(1+\frac{3m^{2}}{4}\right)\pi
\end{split}
\end{equation}
Observe we have an additional term involving $m^{2}$ in this approximation.
\begin{ddanger}
These calculations should be carefully double checked, and
re-examined to make certain we did everything consistently. This
is left as an exercise to you, gentle reader!
\end{ddanger}
The astute reader probably feels discomfort at $b$
disappearing. Observe that half the angle of deflection is
\begin{equation}
\begin{split}
\frac{\Delta\varphi}{2} &=
\int^{1/b}_{0}\frac{(1+2m\omega+\frac{3}{2}m^{2}\omega^{2})}{\sqrt{b^{-2}-\omega^{2}}}\D\omega\\
&=\frac{\pi}{2}+\frac{2m}{b}+\frac{3\pi m^{2}}{8b^{2}}.
\end{split}
\end{equation}
Thus the total angle of deflection is
\begin{equation}
\Delta\varphi=\pi+\frac{4m}{b}+\frac{3\pi m^{2}}{4b^{2}}.
\end{equation}
Notice this agrees, to first order in $m$, with the first
approximation we made.

\subsection{Third Approximation}
Most introductory texts perform the following approximation
\begin{equation}
\begin{split}
u^{2}-2mu^{3}&=u^{2}(1-2mu)\\
&\approx u^{2}(1-mu)^{2}
\end{split}
\end{equation}
Choose a new variable
\begin{equation}
y=u(1-mu),
\end{equation}
and then our integral becomes
\begin{equation}
\varphi=\int\frac{\D u}{\sqrt{b^{-2}-y^{2}+(\mbox{small factor})}}
\end{equation}
To lowest order, this is the same trick as the first
approximation. Higher order terms needs Newtonian corrections. We
find
\begin{equation}
\D y=(1-2mu)\D u
\end{equation}
and so
\begin{subequations}
\begin{align}
\D u
&\approx\frac{\D y}{1-2mu}\\
&\approx(1+2mu)\D y\\
&\approx(1+2my)\D y
\end{align}
\end{subequations}
thus
\begin{equation}
\varphi=\pm\int\frac{(1+2my)\D y}{\sqrt{b^{-2}-y^{2}}}.
\end{equation}
The first term is the Newtonian integral, and the second term is
straightforward. Consider half of the path
\begin{equation}
\begin{split}
\varphi
&=\int^{y=1/b}_{y=0}\frac{(1+2my)\D y}{\sqrt{b^{-2}-y^{2}}}\\
&=\frac{\pi}{2}+\frac{2m}{b}.
\end{split}
\end{equation}
So the total deflection $\Delta\varphi$ is twice this:
\begin{equation}
\Delta\varphi=\pi+4\frac{m}{b}.
\end{equation}
This is the first order correction.

\subsection{Shapiro Time Delay}

\begin{wrapfigure}{r}{10pc}
  \vspace{-1pc}
  \includegraphics{img/lecture04.1}
  \vspace{-3pc}
\end{wrapfigure}

Here's the idea: send a radio signal from the Earth to the
satellite. There is a time delay from receiving the reflection.
The physical problem is doodled on the right, with the light's
trajectory as the dashed line.

Lets assess the problem. Since this is light, we have
\begin{equation}
\begin{split}
\D s^{2} &= 0\\
&= \left(1-\frac{2m}{r}\right)\D t^{2}
-\left(1-\frac{2m}{r}\right)^{-1}\D r^{2}
-r^{2}\D\varphi^{2}
\end{split}
\end{equation}
With a particular choice of coordinates we use the Newtonian approximation
\begin{equation}
r\sin(\varphi)=b.
\end{equation}
What to do? Well, we can derive a geodesic equation for this
approximation:
\begin{equation}
\sin(\varphi)\D r+r\cos(\varphi)\D\varphi=0
\end{equation}
which is rearranged to become
\begin{equation}
\D\varphi=\frac{-1}{r}\tan(\varphi)\D r.
\end{equation}
We square both sides and use basic trigonometry
\begin{equation}
\begin{split}
r^{2}(\D\varphi)^{2}&=\tan^{2}(\varphi)\;(\D r)^{2}\\
&=\frac{b^{2}}{r^{2}-b^{2}}\D r^{2}.
\end{split}
\end{equation}
Why do this? Because we can replace the $r^{2}\D\varphi^{2}$ term
in the $\D s^{2}$ expression:
\begin{equation}
\left(1-\frac{2m}{r}\right)\D t^{2}
=
\left(1-\frac{2m}{r}\right)^{-1}\D r^{2}
+\frac{b^{2}}{r^{2}-b^{2}}\D r^{2}.
\end{equation}
Remember we want to find the \emph{time delay}, so we get rid of
$\D t^{2}$ coefficient:
\begin{equation}
\D t^{2}=\left(1-\frac{2m}{r}\right)^{-2}\D r^{2}
+\left(1-\frac{2m}{r}\right)^{-1}\frac{b^{2}}{r^{2}-b^{2}}\D r^{2}.
\end{equation}
We make the approximation
\begin{equation}
\begin{split}
\left(1-\frac{2m}{r}\right)^{-2}&\approx
\left(1+\frac{2m}{r}\right)^{2}\\
&\approx 1+\frac{4m}{r}+\underbrace{\mathcal{O}(m^{2}/r^{2})}_{\text{negligible}}
\end{split}
\end{equation}
which simplifies our expression to be
\begin{equation}
\D t^{2}
\approx
\left[1+\frac{4m}{r}+\left(1+\frac{2m}{r}\right)\frac{b^{2}}{r^{2}-b^{2}}\right]\D
r^{2}.
\end{equation}
Now, we will perform a long and tedious calculation. The
uninterested reader may skip its proof.

\begin{prop}\label{prop:lec04:simplifyingCalc:shapiroTimeDelay}
We have
\begin{equation}
\left[1+\frac{4m}{r}+\left(1+\frac{2m}{r}\right)\frac{b^{2}}{r^{2}-b^{2}}\right]
=
\frac{r^{2}}{r^{2}-b^{2}}\left[1
+\frac{4m}{r}-\frac{2m}{r}\frac{b^{2}}{r^{2}}\right]
\end{equation}
\end{prop}
\begin{proof}
We see that
\begin{equation}
\left(1+\frac{2m}{r}\right)\left(\frac{b^{2}}{r^{2}-b^{2}}\right)
=\left(\frac{b^{2}}{r^{2}-b^{2}}+\frac{2m}{r}\frac{b^{2}}{r^{2}-b^{2}}\right)
\end{equation}
Adding $(1+4m/r)$ to this yields
\begin{equation}
\begin{split}
\left(1+\frac{4m}{r}\right)+\left(\frac{b^{2}}{r^{2}-b^{2}}+\frac{2m}{r}\frac{b^{2}}{r^{2}-b^{2}}\right)
&=\left(1+\frac{4m}{r}+\frac{b^{2}}{r^{2}-b^{2}}+\frac{2m}{r}\frac{b^{2}}{r^{2}-b^{2}}\right)\\
&=\left(\frac{r^{2}}{r^{2}-b^{2}}+\frac{4m}{r}+\frac{2m}{r}\frac{b^{2}}{r^{2}-b^{2}}\right)
\end{split}
\end{equation}
Factoring out $r^{2}/(r^{2}-b^{2})$ gives us
\begin{subequations}
\begin{align}
\left(\frac{r^{2}}{r^{2}-b^{2}}+\frac{4m}{r}+\frac{2m}{r}\frac{b^{2}}{r^{2}-b^{2}}\right)
&=
\frac{r^{2}}{r^{2}-b^{2}}\left(1+\frac{4m(r^{2}-b^{2})}{r^{3}}+\frac{2m}{r}\frac{b^{2}}{r^{2}}\right)\\
&=\frac{r^{2}}{r^{2}-b^{2}}\left(1+\frac{4m}{r}-\frac{4mb^{2})}{r^{3}}+\frac{2mb^{2}}{r^{3}}\right)\\
&=\frac{r^{2}}{r^{2}-b^{2}}\left(1+\frac{4m}{r}-\frac{2mb^{2})}{r^{3}}\right).
\end{align}
\end{subequations}
This concludes the proof.
\end{proof}

Proposition \eqref{prop:lec04:simplifyingCalc:shapiroTimeDelay} yields
\begin{equation}
\D t^{2}\approx\frac{r^{2}}{r^{2}-b^{2}}\left[1
+\frac{4m}{r}-\frac{2m}{r}\frac{b^{2}}{r^{2}}\right]\D r^{2}.
\end{equation}
Thus we obtain (taking the Taylor series for the square root on
the bracketed terms)
\begin{equation}
\D t\approx
\frac{\pm r}{\sqrt{r^{2}-b^{2}}}\left[
1+\frac{2m}{r}-\frac{mb^{2}}{r^{3}}
\right]\D r.
\end{equation}
What now?

We evaluate the integral
\begin{equation}
\int\D t=\pm\biggl[
\underbracket[0.5pt]{\sqrt{r^{2}-b^{2}}}_{\substack{\text{length of line}\\\text{in flat geometry}}}
+\underbracket[0.5pt]{2m\ln\left(\frac{r}{b}+\sqrt{\frac{r^{2}}{b^{2}}-1}\right)}_{\text{correction to first term}}
-\frac{m}{r}\sqrt{r^{2}-b^{2}}
\biggr].
\end{equation}

\subsection{Time Dilation}

\begin{wrapfigure}{r}{4pc}
  \vspace{-1pc}
  \includegraphics{img/lecture04.2}
\end{wrapfigure}

This is the last experiment we will consider: gravitational time
dilation, or gravitational redshifting. Most books for the lay
person describe it as ``time runs more slowly in a gravitational
field'' (although the immediate question we should ask is:
\emph{relative to what?}).

So for that to make sense, we need to describe how we are
measuring the rate of time, and how to compare these. We will
work again with the Schwarzschild metric. An observer from clock
1 sends a signal to clock 2. We doodle a spacetime diagram, so
it's at the same angle (i.e., we assume $\theta=\varphi=$constant). 

\begin{center}
  \includegraphics{img/lecture04.3}
\end{center}

\noindent\ignorespaces%
So the metric reads
\begin{equation}
\D s^{2}=-\left(1-\frac{2m}{r}\right)^{-1}\D r^{2}
+\left(1-\frac{2m}{r}\right)\D t^{2}=0
\end{equation}
for light. After re-arranging terms we find
\begin{equation}
\D t=\left(1-\frac{2m}{r}\right)^{-1}\D r
\end{equation}
Integration yields
\begin{equation}
t_{2}-t_{1}=\int^{r_{2}}_{r_{1}}\left(1-\frac{2m}{r}\right)^{-1}\D r.
\end{equation}
But look, we also have
\begin{equation}
(t_{2}+\Delta t_{2})-(t_{1}+\Delta t_{1})=\int^{r_{2}}_{r_{1}}\left(1-\frac{2m}{r}\right)^{-1}\D r.
\end{equation}
This implies
\begin{equation}
\Delta t_{2}=\Delta t_{1}.
\end{equation}

Now, an observer measures proper time, so
\begin{subequations}
\begin{align}
\Delta s_{1}
&=\int\D s\\
&=\int\sqrt{1-\frac{2m}{r_{1}}}\D t\\
&=\sqrt{1-\frac{2m}{r_{1}}}\Delta t_{1}
\end{align}
\end{subequations}
where we consider the observer sitting at $r_{1}$ and \emph{is not}
a photon.
Similarly, the observer at clock 2 will observe the interval
between ticks as
\begin{subequations}
\begin{align}
\Delta s_{2}
&=\sqrt{1-\frac{2m}{r_{2}}}\Delta t_{2}\\
&=\left(\frac{\sqrt{1-\frac{2m}{r_{2}}}}{\sqrt{1-\frac{2m}{r_{2}}}}\right)\Delta s_{1}\\
&\approx\left(1-\frac{m}{r_{2}}+\frac{m}{r_{1}}\right)\Delta s_{1}
\end{align}
\end{subequations}
for weak gravitational fields. We see that a photon wave is
redshifted
\begin{equation}
\frac{\Delta\lambda}{\lambda}\approx
-\frac{m}{r_{2}}+\frac{m}{r_{1}}.
\end{equation}
An observer far from the black hole would see a clock on the
Black Hole's horizon stop. There is nothing deep about this, however.



\begin{exercises}
\begin{xca}[The Newtonian Approximation]
In the Newtonian approximation, the spacetime metric is
\begin{equation}
\D s^2 = (1 + 2\phi)\,\D t^2 - (1 - 2\phi)(\D x^2 + \D y^2 + \D z^2)
\end{equation}
where $\phi$ is the Newtonian gravitational potential. This
approximation holds when $\phi$ is small compared to $1$ and
velocities $v^i = \D x^i/\D t$ are also small compared to $1$,
with $\phi$ of the same order as $v^2$. 

(Notation: Latin indices from the middle of the alphabet---$i$,
$j$, $k$, \dots---are spatial indices, going from 1 to
3. Remember that we are using units $c = 1$.) 

Show that to lowest order, the geodesics are the standard paths of Newtonian gravity, that
is, $\mathbf{a} = -\nabla\phi$. 
\end{xca}
\begin{xca}[Geodesics and the Christoffel connection]
Let $g^{ab}$ be the matrix inverse of the metric tensor, that is,
$g^{ab} g_{bc} = {\delta^{a}}_{c}$. Show that the geodesic
equation can be written in the form 
\begin{equation}
\frac{\D^{2}x^{a}}{\D s^{2}}
\Gamma^{a}_{bc}\frac{\D x^{b}}{\D s}\frac{\D x^{c}}{\D s}=0
\end{equation}
where
\begin{equation}
\Gamma^{a}_{bc} = g^{ad} (\partial_{b} g_{dc} + \partial_{c}
g_{db} - \partial_{d} g_{bc}) 
\end{equation}
$\Gamma^{a}_{bc}$ is known as the Christoffel connection, or the ``Christoffel symbols.''

(Hint: you will encounter an expression of the form $\D g_{ab}/\D s$. 
Remember that in the geodesic equation, $g_{ab}$ is the metric
along the geodesic, and is therefore a function of
$x^{c}(s)$. Use the chain rule.) 
\end{xca}
\begin{xca}[Deflection of a massive particle by the Sun]
In this problem, you will (approximately) compute the deflection
of a \emph{massive} particle in the Schwarzschild metric. Note:
some of this is quite hard! 

\textbf{(a)}
Recall that for a massive particle, we defined
\begin{equation}
\widetilde{E}=-\left(
1-\frac{2m}{r}\right)\frac{\D t}{\D s}.
\end{equation}
Find the relationship between $\widetilde{E}$ and the particle speed
\begin{equation}
v^{2} =
\left|\frac{\D\mathbf{x}}{\D t}\right|^{2}
\end{equation}
at $r\to\infty$. (Hint: at infinity, the Schwarzschild metric
reduces to the spherical coordinate 
form of the flat spacetime Minkowski metric $\D s^2 = \D t^2-\D x^2-\D y^2-\D z^2$.)

\textbf{(b)}
For a massive particle, the equation of motion we derived was
\begin{equation}
\left(\frac{\D u}{\D\varphi}\right)^{2}=
\frac{\widetilde{E}^{2}-(1-2mu)(1+\widetilde{L}^{2}u^{2})}{\widetilde{L}^{2}}
\end{equation}
with $u = 1/r$. Consider a particle coming in from infinity,
being deflected, and returning to infinity. Find the deflection
$\Delta\varphi$ in the Newtonian approximation, that is,
neglecting the  term $mu^{3}$. (The solution of the equation of motion is a hyperbola, and can be derived by  a number of methods, but I suggest that you use the technique we saw in class, integrating $\D\varphi$, since this will help in part c.)

\textbf{(c)}
Find the next order approximate expression for the deflection
$\Delta\varphi$, treating the relativistic term $mu^{3}$ as a
small perturbation. You can use the same method that we did in
deriving the deflection of light, including the
definition of a new variable $y = u(1 - mu)$, although the
integral will now be somewhat different---be careful about the
slightly tricky limits of integration! As in the case of light,
assume that $mu\ll1$. 

\textbf{(d)}
The impact parameter $b$ is defined as the minimum value of $r$
on the trajectory. You should already have worked this out in
step (b) to find your integration range. (Note that $b$ is the
turning point, the value at which the derivative $\D u/\D\varphi$
changes sign.) Write $\widetilde{L}$ as a function of $b$, and
rewrite the deflection $\Delta\varphi$ in terms of
$\widetilde{E}$ and $b$. You may assume that
$\widetilde{E}^{2}-1\gg m/b$.

\textbf{(e)}
Show that for speeds near the speed of light--that is,
$v\lesssim1$---the deflection is approximately 
\begin{equation}
\Delta\varphi\approx\frac{2m}{b}\left(1+\frac{1}{v^{2}}\right)
\end{equation}
and agrees with our result for light when $v = 1$.
\end{xca}
\end{exercises}

\lecture
%%
%% lecture05.tex
%% 
%% Made by Alex Nelson
%% Login   <alex@tomato3>
%% 
%% Started on  Sat May 21 13:43:44 2011 Alex Nelson
%% Last update Sun Aug 14 14:59:39 2011 Alex Nelson

We spoke a little bit about the notions of universal bundle and
the classifying space. The universal fibration is the space
($E_{G}$, $B_{G}$, $G$) and the main property is $E_{G}$ is
contractible. Then $B_{G}$ is the base of this construction and
referred to as the classifying space.

Recall the principal fibration ($E$, $B$, $G$, $p$) are in
one-to-one correspondence with $\homotopyClass(B,B_{G})$. We will
prove this when $B$ is a cell complex. Recall we introduced the
notion of a pullback, and by considering the maps
$B\to B_{G}$ and fibrations
\begin{equation}
p_{G}\colon E_{G}\to B_{G},
\end{equation}
we can pull it back to obtain a fibration over $B$. Now what
remains is to prove this correspondence is a one-to-one
correspondence. We prove for a mapping
\begin{equation}
B\to B_{G}
\end{equation}
gives a fibration. We should prove homotopy equivalent maps
\begin{equation}
B\to B_{G}
\end{equation}
go to equivalent fibrations. Now we should prove this map is
injective and surjective. The main part is proving every
fibration may be obtained from this construction; so what should
we prove really? Every principal fibration can be mapped to a
universal fibration.

By hypothesis, we assumed $B$ is a cell complex. So 
\begin{equation}
B=\bigcup B_{k}
\end{equation}
is a union of skeletons. We have a proof by induction. The idea
is to cut out a closed ball. We have a map on the ball
$D^{n}\times G$ and on the boundary of the ball $S^{n-1}\times G$
we construct this map from
\begin{equation}
f\colon S^{n-1}\to E_{G}
\end{equation}
by
\begin{equation}
s\mapsto (s,1)
\end{equation}
Then $G$ acts on it by
\begin{equation}
(s,1)=(s,g)
\end{equation}
We should extend our map to $(b,1)$ for $b\in D^{n}$, but this
can be done since $E_{G}$ is contractible. So every map from a
sphere may be extended to a ball.

We proved if
\begin{equation}
\pi_{k}(E_{G})=0
\end{equation}
for $k\leq n$, then the map
\begin{equation}
\homotopyClass(B,B_{G})%\onto
\xtwoheadrightarrow{sur.}\{\mbox{classes of principal fibrations over }B\}
\end{equation}
is onto if $\dim(B)\leq n$. To prove this mapping is injective, well,
it's trivial.

\begin{ex}
Consider
\begin{subequations}
\begin{equation}
E_{G}=S^{n}
\end{equation}
and
\begin{equation}
G=\ZZ_{2}.
\end{equation}
\end{subequations}
In this case, the action can be defined in a very simple way, the
nontrivial element of $\ZZ_{2}$ sends
\begin{equation}
x\mapsto-x
\end{equation}
We know
\begin{equation}
B_{G}=E_{G}/G=\RP^{n}.
\end{equation}
Is this the classifying space? Not really, we have
\begin{equation}
\pi_{k}(S^{n})=0
\end{equation}
for $k<n$. It is not contractible. For 
\begin{equation}
E_{G}=S^{\infty}
\end{equation}
which is, more or less, infinite sequences such that
\begin{equation}
|x_{1}|^{2}+\dots+|x_{n}|^{2}+\dots=1.
\end{equation}
We usually set
\begin{equation}
x_{n}=0
\end{equation}
for $n\gg0$. We may say\index{$\RP^{\infty}$}\index{Classifying Space!for $G=\ZZ_{2}$}
$\RP^{\infty}$ is the classifying space for $G=\ZZ_{2}$.
\end{ex}
\begin{ex}
But we may go further. We can consider the case when
$x_{i}\in\CC$ and we have
\begin{equation}
\sum_{i\in\NN}|x_{i}|^{2}=1
\end{equation}
This is a principal fibration with
\begin{equation}
G=S^{1}=\{\lambda\in\CC\lst|\lambda|=1\}.
\end{equation}
It acts by
\begin{equation}
\lambda(x_{1},\dots,x_{n},\dots)=(\lambda x_{1},\dots,\lambda
x_{n},\dots).
\end{equation}
If we consider finite sequences we have $S^{2n-1}$ and the base
would be $\CP^{n-1}$. But if we consider infinite sequences, we
get
\begin{equation}
S^{\infty}\to\CP^{\infty}
\end{equation}
we just denote this codomain by ``$\CP^{\infty}$'' for
continuity. So $\CP^{\infty}$\index{$\CP^{\infty}$} may be viewed as the space of all
wave functions in quantum mechanics. But it is quite clear that
$\CP^{\infty}$ is the classifying space for $S^{1}$.
\end{ex}

We can think of the quaternions\index{Quaternions} as
\begin{equation}
x_{0}+x_{1}i+x_{2}j+x_{3}k=(x_{0},\vec{x}).
\end{equation}
Multiplication is then 
\begin{equation}
(x_{0},\vec{x})\star(y_{0},\vec{y})=(x_{0}y_{0}-\vec{x}\cdot\vec{y},\vec{x}\times\vec{y}).
\end{equation}
It is a normed-division algebra over $\RR$ and 4-dimensional. The
typical notation for the quaternions is $\HH$ in honor of its
inventor Hamilton.

Let $(x_{1},\dots,x_{n})\in\HH^{n}$, assume this is nonzero and
identify
\begin{equation}
(x_{1},\dots,x_{n})\sim(\lambda x_{1},\dots, \lambda x_{n})
\end{equation}
where $\lambda\in\HH\setminus\{0\}$. The result is
$\HP^{n-1}$. 

Or we may impose the condition
\begin{equation}
|x_{1}|^{2}+\dots+|x_{n}|^{2}=1.
\end{equation}
Observe that the set of $\lambda\in\HH$ such that
\begin{equation}
|\lambda|=1,
\end{equation}
i.e., $\lambda\in S^{3}$ defines a group. Only $S^{0}\iso\ZZ_{2}$,
$S^{1}\iso\U{1}$, and $S^{3}$ are groups. So we can say that
$\HP^{\infty}$\index{$\HP^{\infty}$} is the classifying
space\index{Classifying Space!for $G=S^{3}$} for $G=S^{3}$. NB
$S^{3}\iso\SU{3}$. 

We should recall Stiefel manifolds\index{Stiefel Manifold} and Grassmann manifolds\index{Grassmann manifold}
denoted by $V_{n,k}$\index{$V_{n,k}$} and $\Gr_{n,k}$\index{$\Gr_{n,k}$} (respectively). We have
\begin{equation}
\Gr_{n,k}(\FF)=\{\text{space of $k$-dim. linear subspaces of
$n$-dim. vector space}\}
\end{equation}
\begin{thm}
We have $\Gr_{\infty,k}(\RR)$\index{$\Gr_{\infty,k}(\RR)$} is the classifying
space\index{Classifying Space!for $\GL{k,\RR}$} for
$\GL{k,\RR}$, and $\Gr_{\infty,k}(\CC)$\index{$\Gr_{\infty,k}(\CC)$} is the classifying space
for $\GL{k,\CC}$.\index{Classifying Space!for $\GL{k,\CC}$}
\end{thm}



\lecture
%%
%% lecture06.tex
%% 
%% Made by Alex Nelson
%% Login   <alex@tomato3>
%% 
%% Started on  Sun Aug 14 14:59:43 2011 Alex Nelson
%% Last update Sun Aug 14 15:32:10 2011 Alex Nelson
%%
We described the construction involving $\GL{n,\RR}$,
$\GL{n,\CC}$ that are its classifying spaces. We are only
interested in homotopical properties. Here we may say 
\begin{equation}
\GL{n,\RR}\homotopic\ORTH{n}
\end{equation}
is homotopic, and
\begin{equation}
\GL{n,\CC}\homotopic\U{n}
\end{equation}
is also homotopic.

What is $\GL{n}$ as a space? Well, pick a basis for 
\begin{equation}
V\iso\FF^{n}
\end{equation}
then change the basis. The ``Jacobian'' is then a member of
$\GL{n}$. So $\GL{n,\RR}$ is the space of frames in $\RR^{n}$. We
can always orthonormalize a given basis by the Grahm-Schmidt
procedure. This gives us a map
\begin{equation}
\GL{n}\to\ORTH{n}
\end{equation}
and of course
\begin{equation}
\ORTH{n}\Into\GL{n}
\end{equation}
is an embedding. We merely have to prove that $\GL{n}\to\ORTH{n}$
can be done ``gradually''. We then have
\begin{equation}
\GL{n,\RR}\homotopic\ORTH{n},
\end{equation}
and similar reasoning suggests that $\GL{n,\CC}$ is homotopic to
$\ORTH{n,\CC}$. But $\ORTH{n,\CC}$ is in one-to-one
correspondence with $\U{n}$, so $\GL{n,\CC}\homotopic\U{n}$ is
homotopic.

If 
\begin{equation}
\mathrm{GL}_{+}(n,\RR)=\{X\in\GL{n,\RR}\lst\det(X)>0\}
\end{equation}
then $\mathrm{GL}_{+}(n,\RR)\homotopic\SO{n}$ is homotopic.

We come to the notion of a Stiefel manifold\index{Stiefel Manifold}, which we covered
Fall quarter. We have several different definitions of
$V_{n,k}(\RR)$ which are homotopy equivalent definitions, so we
do not distinguish between them. We can say that $V_{n,k}(\RR)$
is the space of $k$-framed in $\RR^{n}$. Observe that
\begin{equation}
V_{n,n}\homotopic\GL{n,\RR}
\end{equation}
is homotopic. We may say $V_{n,k}(\RR)$ is the space of $k$
orthonormal vectors in $\RR^{n}$. These two are homotopy
equivalent spaces. We may extend to $V_{n,k}(\CC)$.

There are various different groups that act on Stiefel
manifolds. We see that $\GL{n}$ acts on $\RR^{n}$, and thus acts
on $V_{n,k}(\RR)$ by transitivity. So $V_{n,k}$ is an orbit, thus
we may say 
\begin{equation}
V_{n,k}=\GL{n}\Big/(\text{some stable subgroup}).
\end{equation}
But it is another way to describe Stiefel manifolds.

Let $V_{n,k}^{\mathrm{orth}}(\RR)$\index{$V_{n,k}^{\mathrm{orth}}(\RR)$} be the Stiefel manifold
describing orthonormal $k$-frames. We may say $\ORTH{n}$ acts
transitively on $V_{n,k}^{\mathrm{orth}}(\RR)$, rotating one
frame into another. But in this situation, we may describe the
stabilizer in a simple way.

We look at orthogonal transformations that keep this frame in
tact. We may write
\begin{equation}
\RR^{n}=\RR^{k}\oplus\RR^{n-k}
\end{equation}
the transformations preserves $\RR^{k}$. So therefore the
stabilizer rotates $\RR^{n-k}$ into itself, and
\begin{equation}
V^{\mathrm{orth}}_{n,k}(\RR)=\ORTH{n}/\ORTH{n-k}.
\end{equation}
Precisely the same consideration goes in the complex case. The
only difference is 
\begin{equation}
V^{\mathrm{orth}}_{n,k}(\CC)=\U{n}/\U{n-k}.
\end{equation}
i.e., we work with unitary transformations.

We will look at fibrations involving Stiefel manifolds. Suppose
we had $V_{n,k}$. Consider \emph{only} $V_{n,k}^{\mathrm{orth}}$,
we will thus without loss of generality remove the
superscript. We observe
\begin{equation}
p\colon V_{n,k}\to V_{n,k-1}
\end{equation}
by forgetting the $k^{\mathrm{th}}$ vector. This map is a locally
trivial fibration. We see that
\begin{equation}
p^{-1}(e_{1},\dots,e_{k-1})=\{(e_{1},\dots,e_{k})\lst
e_{k}\in\FF\}
\end{equation}
where $\FF$ is the field we're working with. 


If
\begin{equation}
V=\Span(e_{1},\dots,e_{k-1})
\end{equation}
then $e_{k}\in V^{\bot}$ and
\begin{equation}
\dim(V^{\bot})=n-(k-1)
\end{equation}
but $e_{k}$ should be normalized and thus $e_{k}\in S^{n-k}$. We
obtain a fibration with base $V_{n,k-1}$ and fibre $S^{n-k}$ and
the total space is $V_{n,k}$.

When we work over $\CC$, we see that
\begin{equation}
\dim(V^{\bot})=2\left(n-(k-1)\right)
\end{equation}
and thus the fibre has dimension $2n-2k+1$, so
$F=S^{2n-2k+1}$. 

We would like to compute homotopy groups of the fibrations of the
Stiefel manifolds. The exact sequence of homotopy groups of the
fibration is
\begin{equation}
\pi_{i+1}(S^{\bullet})\to\pi_{i}(V_{n,k})\to\pi_{i}(V_{n,k-1})\to\pi_{i}(S^{\bullet})
\end{equation}
and we see for ``small dimensional spheres,'' we have
$\pi_{i}(S^{\bullet})=0$ which implies
\begin{equation}
\pi_{i}(V_{n,k})\iso\pi_{i}(V_{n,k-1}).
\end{equation}
We see that
\begin{equation}
V_{n,1}(\RR)\iso S^{n-1}
\end{equation}
the orthonormal (real) vector lives in the sphere, and similarly
\begin{equation}
V_{n,1}(\CC)\iso S^{2n-1}.
\end{equation}
We may inductively compute other homotopy groups. We see that
\begin{equation}
\pi_{i}(V_{\infty,k})=0
\end{equation}
which is unsurprising since $\pi_{i}(S^{\infty})=0$.

Observe that on $V_{n,k}(\RR)$ we have the action or $\ORTH{k}$
or $\SO{k}$ [and for $V_{n,k}(\CC)$ we have the action of $\U{k}$
or $\SU{k}$]. We consider an action of the form
\begin{equation}
(e_{1},\dots,e_{k})\mapsto(e_{1}',\dots,e_{k}')
\end{equation}
where
\begin{equation}
e_{i}'=\sum_{j}a_{ij}e_{j}
\end{equation}
where $(a_{ij})$ is ``invertible.'' This is a \emph{free}
action. Now we have a question: we can factorize this stuff
\begin{equation}
V_{n,k}(\CC)/\U{k}=\Gr_{n,k}(\CC)
\end{equation}
is precisely the Grassmann manifold. It is obvious. It is clear
that if we have a frame $(e_{1},\dots,e_{k})$ we can map it to 
$\Span(e_{1},\dots,e_{k})$; it is clear that this is a fibration
$V_{n,k}(\CC)\to \Gr_{n,k}(\CC)$, and the fibre is $k$-frames
living in $k$-dimensional space.

Now lets take $n=\infty$ and we find
\begin{equation}
V_{\infty,k}(\CC)/\U{k}=\Gr_{\infty,k}(\CC).
\end{equation}
We see that we have a universal bundle\index{Universal Bundle!for $\U{k}$} 
since $V_{\infty,k}$\index{$V_{\infty,k}$} is
contractible, $\U{k}$ acts freely, so $\Gr_{\infty,k}(\CC)$ is the
classifying space\index{Classifying Space!for $\U{k}$}.

Lets consider real Stiefel manifolds, we may take the
factorization
\begin{equation}
V_{n,k}(\RR)/\ORTH{k}=\Gr_{n,k}(\RR)
\end{equation}
and repeat the same arguments to find
\begin{equation}
V_{\infty,k}(\RR)/\ORTH{k}=\Gr_{\infty,k}(\RR).
\end{equation}
We may take the quotient with $\SO{k}$ which produces the space
of \emph{oriented} $k$-dimensional planes. 

In reality we did more than it seems, because we can now
construct the classifying space for \emph{any} matrix group $G$,
a closed subgroup of $\GL{k,\CC}$. We have a free action of
$\GL{k,\CC}$ on $V_{\infty,k}$. Thus we can induce a free action
of $G$ on $V_{\infty,k}$. So the classifying space is simply
$V_{\infty,k}/G$. 

\subsection*{EXERCISES}
\begin{xca}\label{xca:lec06:prob1}
Show that Moebius band can be represented as a space of fibration
having the circle $S^1$ as a base and the interval $I$ as a
fiber. Prove that this fibration is not trivial. Does it have a
section? Show that this fibration can be obtained pasting
together two trivial fibrations; find a transition function
(clutching function) taking values in the group $\ZZ_2$.
\end{xca}
\begin{xca}
The group $\ZZ_2$ acts on the circle $S^1$ (the non-trivial
transformation acts as reflection $x\to x$, $y\to-y$). Therefore
using the transition function of the
Problem \ref{xca:lec06:prob1} we can construct a fibration with
the fiber $S^1$ and the base $S^1$. Prove that the total space of
this fibration is a Klein bottle. Does this fibration have a
section?
\end{xca}
\begin{xca}\label{xca:lec06:prob3}
Let us consider principal fibrations with the group $\U{1}$,
total space $E$ and base $S^2$. Show that classes of these
fibrations are in one-to-one correspondence with integers. Let us
consider fibrations corresponding to integer numbers $n = 0, 1,
2$. Prove that we have $E = S^2\times S^1$ for $n = 0$, $E = S^3$
for $n = 1$, $E = \SO(3) = \RP^3$ for $n = 2$.

Hint. Consider Hopf fibration and tangent fibre bundle to
$S^2$. (To define Hopf fibration we consider the group $\U{1}$
acting on the $S^3$ defined by the equation $|z_{1}|^2 +
|z_{2}|^2 = 1$ in $\CC^2$; element $z\in\U{1}$ transforms $(z_1,
z_2)$ into $(zz_1, zz_2)$.) Calculate the transition function
explicitly or use exact homotopy sequence of fibration.
\end{xca}
\begin{xca}
Let us consider a principal fibration of Problem \ref{xca:lec06:prob3} corresponding to arbitrary integer
$n$. Calculate homotopy groups of total space of the fibration (for dimensions $\leq3$ you
should give an explicit answer, in dimensions $> 3$ you should express the homotopy group
of total space in terms of homotopy groups of $S^2$ ).
\end{xca}

\lecture[Characteristic Classes]
%%
%% lecture07.tex
%% 
%% Made by alex
%% Login   <alex@tomato>
%% 
%% Started on  Sat Oct  1 19:36:14 2011 alex
%% Last update Sat Oct  1 19:36:14 2011 alex
%%
We discussed the Riemann surface of
\begin{equation}
h(z)=\sqrt{(z-1)z(z+1)}.
\end{equation}
This is doodled thus:
\begin{center}
\includegraphics{img/lecture07.0}
\end{center}
Consider
\begin{equation}
\omega^{3}-\omega+z=0,
\end{equation}
we see that
\begin{equation}
z=\omega-\omega^{3},
\end{equation}
so we may write $z=z(\omega)$. We can invert this to find
$\omega=\omega(z)$. More generally we can suppose that
$p(\omega)=z$ and its inverse is $q(z)=\omega$. We can now
consider the Riemann surface of this function. Consider the graph
of this function (roughly doodled below to the left).

\begin{wrapfigure}{l}{3in}
\begin{center}
\includegraphics{img/lecture07.1}
\end{center}
\end{wrapfigure}
\noindent{}The inverse function has several values, so we get in this
complex analogue a Riemann surface. From the projection, which is
multivalued, we have the Riemann surface. In this case for
$h(z)$, the Riemann surface is homeomorphic to a torus.

The reader should make a mental note on the importance of branch
cuts in this method of constructing Riemann surfaces. Also note
that we are projecting onto \emph{Riemann spheres} which are
distinguished from the notion of \emph{Riemann surfaces}. The
Riemann sphere is $\CC$ as a sphere, obtained from Stereographic
projection. 

\begin{prop}[Fact from geometry]
An orientable, closed, compact surface is homeomorphic to a torus
(or more generally a sphere with $p$ handles).
\end{prop}

How do we find the genus of a Riemann surface? (I.e., how do we
find the value of $p$?) We have $n$ roots, we count how many
times we glue points together. We subtract the number of boundary
points of the cuts. So we have
\begin{equation}
\chi=2n-\#(\mbox{boundary points of the cuts})
\end{equation}
\marginpar{Euler Characteristic of Riemann Surface}which is precisely the Euler characteristic. The genus is
\begin{equation}
\frac{2-\chi}{2}=\mbox{genus}.
\end{equation}
This is for polynomials, however.

Riemann surfaces are defined for algebraic functions. Consider
the famous example of the logarithm function, it covers the
complex plane infinitely many times. When we consider the Riemann
surface, it's like an infinite Helix. This is the logarithmic
staircase. See Penrose's \emph{Road to Reality} for a good
doodle.

\subsection{Reflection Principle}

\begin{wrapfigure}[10]{r}{1.5in}
\vspace{-50pt}
\begin{center}
\includegraphics{img/lecture07.2}
\end{center}
\vspace{-40pt}
\end{wrapfigure}
We are nonetheless interested in extending functions. We have the
Reflection principle. We have some domain which contains on the
boundary part of the $x$ (real) axis.
We consider some function $f\colon\mathcal{U}\to\CC$, we extend
$f$ to another function $\widetilde{f}$ on $\mathcal{U}\cup I$
where $I$ is the real part of $\partial\mathcal{U}$, i.e.,
$I=\RR\cap\partial\mathcal{U}$. We demand that $\widetilde{f}$ be
continuous, and demand that $\widetilde{f}|_{I}$ be real. We
consider the complex conjugation of $\mathcal{U}$, doodled to the
right, which is $\bar{\mathcal{U}}=\{\bar{z}\lst
z\in\mathcal{U}\}$. We introduce a function $\widehat{f}$ such
that
\[
\widehat{f}(z) = \begin{cases}
f(z) & \mbox{if }z\in\mathcal{U}\\
\overline{f(\bar{z})} & \mbox{if }z\in\bar{\mathcal{U}}\\
\widetilde{f}(z) & \mbox{if }z\in I.
\end{cases}
\]
We have then $\mathcal{V}=\mathcal{U}\cup
I\cup\bar{\mathcal{U}}$, and we see that $\widehat{f}$ is
continuous on $\mathcal{V}$. We see that if $\widehat{f}$ is
analytic on $\mathcal{U}$, then $\widehat{f}$ is analytic on
$\bar{\mathcal{U}}$. 

Let $z=x+\I y\in\mathcal{U}$,
\begin{equation}
\widehat{f}(x+\I y)=u(x,y)+\I v(x,y)
\end{equation}
and for $\overline{z}\in\bar{\mathcal{U}}$ we see we have
\begin{equation}
\overline{f(\overline{z})}=u(x,-y)-\I v(x,-y)
\end{equation}
By the chain rule we see that $\overline{f(\overline{z})}$ satisfies
the Cauchy-Riemann equations.

\begin{wrapfigure}[5]{r}{1.25in}
\vspace{-30pt}
\begin{center}
\includegraphics{img/lecture07.3}
\end{center}
\vspace{-20pt}
\end{wrapfigure}
\medbreak\noindent\textbf{A General Statement.\enspace}
We will use the diagrams doodled on the right for reference.
Consider $\varphi\colon W\to\CC$ such that $\varphi$ is
continuous on $W$. If $\varphi|_{W_{1}}$ and
$\bar{\varphi}|_{W_{2}}$ are analytic, then $\varphi$ is analytic
on $W$.

\begin{lem}
Suppose we have in some domain $\mathcal{U}$ a continuous
function $$f\colon\mathcal{U}\to\CC.$$ Let $\gamma\colon
I\to\mathcal{U}$ be a continuous, closed curve, and for any such $\gamma$
that is simply connected, i.e., $$\int_{\gamma}f(z)\D z=0.$$ Then the
function is analytic.
\end{lem}

We see that the integral over a closed, simply connected path is
zero if its contained entirely in $W_{1}$ or $W_{2}$. We just
need to check for a path that crosses $\gamma$, we just treat it
by breaking it up into pieces.
\label{215c:lecture7}
\lecture
%%
%% lecture08.tex
%% 
%% Made by alex
%% Login   <alex@tomato>
%% 
%% Started on  Thu Feb 23 13:33:17 2012 alex
%% Last update Thu Feb 23 13:33:17 2012 alex
%%
A type $(k,l)$-tensor $T$ is a multilinear map from $k$ dual
vectors and $l$ vectors to $\RR$:
\begin{equation}
T\colon\underbrace{\mathrm{T}^{*}\mathcal{M}
\times\dots\times\mathrm{T}^{*}\mathcal{M}}_{\text{$k$ times}}\times
\underbrace{\mathrm{T}\mathcal{M}
\times\dots\times\mathrm{T}\mathcal{M}}_{\text{$l$ times}}
\to\RR.
\end{equation}
This is a linear map, so if we know what it does on the basis
vectors (and covectors), we know everything. We have
\begin{equation}
T(\D x^{\mu_{1}},\dots,\D x^{\mu_{k}},\partial_{\nu_{1}},\dots,,\partial_{\nu_{\ell}})={T^{\mu_{1}\dots\mu_{k}}}_{\nu_{1}\dots\nu_{\ell}}
\end{equation}
are the components of $T$ in a coordinate basis.

We have this method of constructing new tensors out of old ones:
the tensor product. The idea is simple, basically multiply the
components together. More formally, if we take a $(k,\ell)$
tensor and a $(m,n)$ tensor, their tensor product gives us 
a $(k+m,\ell+n)$ tensor denoted
\begin{equation}
\mathop{S}\limits^{(k,\ell)}\otimes
\mathop{T}\limits^{(m,n)}=\mathop{U}\limits^{(k+m,\ell+n)}
\end{equation}
and it has components given by
\begin{equation}
\begin{split}
(S\otimes T)&(\omega_{1},\dots,\omega_{k+m},v^{1},\dots,v^{\ell+n})=\\
&\quad S(\omega_{1},\dots,\omega_{k},v^{1},\dots,v^{\ell})T(\omega_{k+1},\dots,\omega_{k+m},v^{\ell+1},\dots,v^{\ell+n}).
\end{split}
\end{equation}
So what happens in practice? Well, consider a $(1,1)$ tensor 
\begin{equation}
T={T^{\mu}}_{\nu}\partial_{\mu}\otimes\D x^{\nu}
\end{equation}
we say
\begin{subequations}
\begin{align}
T(\omega,v)
&= T(\omega_{\rho}\D x^{\rho},v^{\sigma}\partial_{\sigma})\\
&={T^{\mu}}_{\nu}\partial_{\mu}\otimes\D x^{\nu}(\omega_{\rho}\D x^{\rho},v^{\sigma}\partial_{\sigma})\\
&={T^{\mu}}_{\nu}(\omega_{\rho}\partial_{\mu}\D x^{\rho})\otimes(v^{\sigma}\D x^{\nu}\partial_{\sigma})\\
&={T^{\mu}}_{\nu}(\omega_{\rho}{\delta_{\mu}}^{\rho})(v^{\sigma}{\delta^{\nu}}_{\sigma})\\
&=T^{\mu}_{\nu}\omega_{\mu}v^{\nu}.
\end{align}
\end{subequations}
Again, this is what physicists say. Mathematicians would be a
little more cautious, but get the same result.

\medbreak\noindent\textbf{Warning:\quad}\ignorespaces %
It looks like anything with an index is a tensor, in some sense
this is true but a tensor is independent of what basis you're using.
So lets consider a misleading non-example: a type $(2,0)$ tensor
$(\partial\omega)$ which satisfies
\begin{equation}
(\partial\omega)(\partial_{\mu},\partial_{\nu})=\partial_{\mu}\omega_{\nu}.
\end{equation}
Suppose we choose a different coordinate system, then
\begin{equation}
(\partial\omega)(\partial_{\mu},\partial_{\nu})\not=
(\partial\omega)(\partial_{\mu'},\partial_{\nu'})
\end{equation}
In other words: it is not even linear!

\noindent\textbf{Moral:\quad}\ignorespaces %
Indices don't make something an tensor!

\medbreak
\begin{ex}[Kronecker Delta]
A type $(1,1)$ tensor is the Kronecker delta
\begin{subequations}
\begin{align}
\delta
&={\delta^{\mu}}_{\nu}\partial_{\mu}\otimes\D x^{\nu}\\
&=\partial_{\mu}\otimes\D x^{\mu}.
\end{align}
\end{subequations}
So 
\begin{equation}
\delta(\omega,v)=\omega_{\mu}v^{\mu}.
\end{equation}
We can show this by showing $\delta$ is linear, or we can show
that it's independent of coordinates.
\end{ex}
\begin{ex}[Field Strength Tensor]
The field strength tensor is a $(2,0)$ tensor 
\begin{equation}
F=F_{\mu\nu}\D x^{\mu}\otimes\D x^{\nu}
\end{equation}
with components
\begin{equation}
F_{0i}\sim E_{i},\quad\mbox{and}\quad F_{ij}\sim B_{k}
\end{equation}
Usually it is written $F=\D A+A\wedge A$.
\end{ex}
\begin{ex}[Metric Tensor]
A $(2,0)$ tensor we have seen before is the metric tensor
\begin{equation}
g=g_{\mu\nu}\D x^{\mu}\otimes\D x^{\nu}
\end{equation}
which behaves on vectors as
\begin{equation}
g(v,w)=g_{\mu\nu}v^{\mu}w^{\nu}.
\end{equation}
This is a generalization of the inner product.

The metric lets us change a vector to a dual vector. Consider (in
some basis) a vector $v=v^{\mu}\partial_{\mu}$, then $g(v,-)$ is
an object taking a vector to a real number:
\begin{equation}
\begin{split}
g(v,-)\colon &\mathrm{T}\mathcal{M}\to\RR,\\
&\omega\mapsto g(v,\omega).
\end{split}
\end{equation}
Component-wise this looks like
\begin{subequations}
\begin{align}
g(v,-)
&=(g_{\mu\nu}\D x^{\mu}\otimes\D
x^{\nu})(v^{\rho}\partial_{\rho},-)\\
&=g_{\mu\nu}\<\D x^{\mu}|v^{\rho}\partial_{\rho}\>\D x^{\nu}\\
&=g_{\mu\nu}(v^{\rho}{\delta_{\rho}}^{\mu})\D x^{\nu}\\
&=(g_{\mu\nu}v^{\mu})\D x^{\nu}.
\end{align}
\end{subequations}
\textbf{NOTATION:\quad}\ignorespaces $g_{\mu\nu}v^{\mu}=v_{\nu}$.

Note that we need one more condition for $g$ to be a metric: it
must be nondegenerate. So
\begin{equation}
g(u,v)=0\quad\mbox{for all }v
\end{equation}
only when $u=0$.
Equivalently, $g_{\mu\nu}$ must be invertible. Its inverse is
denoted $g^{\mu\nu}$ so 
\begin{equation}
g_{\alpha\mu}g^{\mu\beta}={\delta_{\alpha}}^{\beta}.
\end{equation}
The metric has to be symmetric. The inverse metric tensor
\begin{equation}
g=g^{\mu\nu}\partial_{\mu}\otimes\partial_{\nu}
\end{equation}
is also an honest tensor.
\end{ex}

\lecture
%%
%% lecture09.tex
%% 
%% Made by Alex Nelson
%% Login   <alex@tomato3>
%% 
%% Started on  Mon Jun 14 10:11:45 2010 Alex Nelson
%% Last update Mon Jun 14 15:20:28 2010 Alex Nelson
%%

Let $G$ be a Lie group, consider $\lie(G)$ its Lie algebra. Then
there is a correspondence between curves in the group and curves
in the Lie algebra. So if we have two curves in the Lie algebra,
we have two curves in the Lie group, then for simply connected
groups we may deform two curves $g_{1}(t)$, $g_{2}(t)$ with
\begin{equation}
g_{1}(0)=g_{2}(0)=g_{0},\quad\mbox{and}\quad
g_{1}(1)=g_{2}(1)=g_{1}
\end{equation}
by introducing a family of curves $g_{\tau}(t)$ which has a
corresponding family of curves in the Lie algebra. We have
\begin{subequations}
\begin{align}
\xi(\tau,t) &= g_{\tau}(t)^{-1}\frac{\D g_{\tau}(t)}{\D t}\\
\eta(\tau,t) &= g_{\tau}(t)^{-1}\frac{\D g_{\tau}(t)}{\D \tau}
\end{align}
\end{subequations}
and we have the relation
\begin{equation}
\frac{\partial\eta}{\partial t}-\frac{\partial\xi}{\partial t}=[\xi,\eta]
\end{equation}
which occurs in the Lie algebra. Observe that 
\begin{equation}
\xi(t,0)=\gamma_{0}(t),\quad\xi(t,1)=\gamma_{1}(t),\quad\eta(0,\tau)=\eta(1,\tau)=0.
\end{equation}
So given these conditions that, for
\begin{equation}
\frac{\partial\eta(\tau,t)}{\partial t}-\frac{\partial\xi(\tau,t)}{\partial\tau}
=[\xi(\tau,t),\eta(\tau,t)]
\end{equation}
with boundary conditions
\begin{subequations}
\begin{align}
\xi(t,0)=\gamma_{0}(t)\quad\mbox{and}\quad\xi(t,1)=\gamma_{1}(t)\\
\eta(1,\tau)=\eta(0,\tau)=0
\end{align}
\end{subequations}
can we get information induced in the group? We have
\begin{equation}\label{eq:lec09:diffEqXi}
\xi(t,\tau)=g(t,\tau)^{-1}\frac{\partial g(t,\tau)}{\partial t},
\end{equation}
where $g(0,\tau)=1$. We can restore $g(t,\tau)$ since there is a
unique solution to eq \eqref{eq:lec09:diffEqXi}.

%% So, more or less, $\widetilde{\xi}\to g\to(\xi,\eta)$. We can
%% find new $\xi$, $\eta$ satisfying the above. So
%% $\widetilde{\xi}=\xi$ by construction.

Suppose we have $\lie(G)\to\lie(G')$ be a Lie algebra morphism;
how can we induce a Lie group morphism? Well, how we do it makes
heavy use of this curve voodoo. The basic correspondence we have
is that ``points in the group'' corresponds to ``curves in the
Lie Algebra'', and ``multiplication in the group'' corresponds to
``concatenation of paths in the Lie algebra.'' Group curve
concatenation can be performed, for 
\begin{equation}
g_{1}:[0,b]\to G
\end{equation}
and
\begin{equation}
g_{2}:[b,a]\to G,
\end{equation}
as 
\begin{equation}
g(t)=\begin{cases} g_{1}(t) & t\in[0,b]\\
g_{2}(b)^{-1}g_{2}(t) & t\in[b,a].
\end{cases}
\end{equation}
If the paths $g_{1}(t)$, $g_{2}(t)$ are not loops,
i.e. $g_{1}(0)\not=g_{1}(b)$ and $g_{2}(b)\not=g_{2}(a)$, then
\begin{equation}
g(t)=\begin{cases} g_{1}(t) & t\in[0,b]\\
g_{1}(b)g_{2}(b)^{-1}g_{2}(t) & t\in[b,a].
\end{cases}
\end{equation}
We see that $g(b)$ is in the first case equal to $g_{1}(b)$, and
in the second case
\begin{equation}
g(b) = g_{1}(b)g_{2}(b)^{-1}g_{2}(b) = g_{1}(b).
\end{equation}
Thus the two cases agree on the overlap.

The corresponding curve in the Lie algebra is
\begin{equation}
\gamma(t) = \begin{cases}\displaystyle
g_{1}(t)^{-1}\frac{\D g_{1}(t)}{\D t} & t\in[0,b]\\
\displaystyle (g_{2}(b)^{-1}g_{2}(t))^{-1}\left(g_{2}(b)^{-1}\frac{\D g_{2}(t)}{\D t}\right)
& t\in[b,a]
\end{cases}
\end{equation}
up to a constant (i.e. $g_{1}(b)$) in the second case. It doesn't
play a significant role, as it is factored out. We end up with
\begin{equation}
\gamma(t) = \begin{cases}\displaystyle
g_{1}(t)^{-1}\frac{\D g_{1}(t)}{\D t} & t\in[0,b]\\
\displaystyle g_{2}(t)^{-1}\frac{\D g_{2}(t)}{\D t} & t\in[b,a]
\end{cases}
\end{equation}
We will consider the construction of Lie groups from Lie algebra
next time...

We proved there exists a one-to-one correspondence between simply
connected Lie groups and finite dimensional Lie algebras. If we
have a discrete normal subgroup $N\subset G$, then the Lie
algebra of $G/N\iso\lie(G)$. This is because there is a
neighborhood $\mathcal{U}$ of $1\in G$ such that $\mathcal{U}\cap
N=\{1\}$. 

\begin{thm}
If $G$ is simply connected, and $\lie(G)\iso\lie(G')$, then
$G'\iso G/N$ where $N$ is a discrete normal subgroup of $G$.
\end{thm}
\begin{ex}
$\Bbb{R}$ equipped with addition has trivial commutators in Lie
  algebra, but $\lie\big(U(1)\big)\iso\lie(\Bbb{R})$ so
  $U(1)\iso\Bbb{R}/\Bbb{Z}$. 
\end{ex}

\lecture
%%
%% lecture10.tex
%% 
%% Made by Alex Nelson
%% Login   <alex@tomato3>
%% 
%% Started on  Mon Jun 14 15:20:31 2010 Alex Nelson
%% Last update Mon Jun 14 15:55:14 2010 Alex Nelson
%%
We will finish the construction of the Lie group from the Lie
algebra. There is an important formula called the
\define{Baker-Campbell-Hausdorff formula.} Recall we considered a
correspondence between a curve in the Lie group and a curve in
the Lie algebra. The question is can we only consider curves
$\exp(tA)$ one-parameter families of the group; in the
neighborhood of the identity, the correspondence is
one-to-one. It would follow that multiplication in the group
\begin{equation}
e^{A}\cdot e^{B}=e^{C(A,B)}
\end{equation}
goes to some operation in the algebra. We have
\begin{equation}
C(A,B) = \log(e^{A}\cdot e^{B})
\end{equation}
we know
\begin{equation}
e^{A}=\sum^{\infty}_{n=0}\frac{A^{n}}{n!},
\end{equation}
but there is no such series for the logarithm. There is some
expansion for the logarithm \emph{in some neighborhood.} We need
to be careful about the order of multiplication, we can express
the series in terms of the commutators
\begin{equation}
C(A,B)=A+B+\frac{1}{2}[A,B]+\frac{1}{12}[A,[A,B]]-\frac{1}{12}[B,[A,B]]-\frac{1}{24}[B,[A,[A,B]]]+\cdots.
\end{equation}
Really, the most important part of this formula is
\begin{equation}
C(A,B)=A+B+\frac{1}{2}[A,B]+\cdots
\end{equation}
We expect \emph{ab initio} that $C(A,B)\in\lie(G)$ is in the Lie
algebra. This formula permits us to construct a \define{local Lie group},
i.e. a group with induced group operation in the neighborhood of
the unit element.

There is a special situation with this being global. Consider a
nilpotent Lie algebra. The Baker-Campbell-Hausdorff formula
becomes a polynomial with finitely many terms, which gives rise
to a Lie group from the Lie algebra.

\subsection{Representations of \texorpdfstring{$\mathfrak{sl}(2)$}{sl2}}

This is really quite important in math and in physics. We know
\begin{equation}
\Bbb{C}\otimes\frak{su}(2)\iso\frak{sl}(2).
\end{equation}
So representations of $\frak{su}(2)$ may be studies by
representations of $\frak{sl}(2)$. We know
\begin{equation}
SO(3)\iso SU(2)/\Bbb{Z}_{2},
\end{equation}
so we can study representations of $\frak{so}(3)$ too!

For $\frak{sl}(2)$, we have generators $e$, $f$, $h$ with the
commutation relations
\begin{subequations}
\begin{align}
[e,f]&=h\\
[h,e]&=2e\\
[h,f]&=-2f.
\end{align}
\end{subequations}
We would like to describe all representations of
$\frak{sl}(2)$. For a general Lie Algebra, we take its Cartan
subalgebra $\frak{h}\subset\lie(G)$. Here $h$ is a generator of
the Cartan subalgebra. We will take any rep
\begin{equation}
\varphi\colon\mathscr{G}\to\frak{gl}(n).
\end{equation}
So
\begin{equation}
f\mapsto\varphi(f)=F,\quad
e\mapsto\varphi(e)=E,\quad
h\mapsto\varphi(h)=H,
\end{equation}
and the commutation relations are
\begin{subequations}
\begin{align}
[E,F]&=H\label{eq:lec10:comm1}\\
[H,E]&=2E\label{eq:lec10:comm2} \\
[H,F]&=-2F.\label{eq:lec10:comm3}
\end{align}
\end{subequations}
We need to find 3 such matrices. We will consider eigenvectors of
$H$ called \define{Weight Vectors}
\begin{equation}
H\vec{x}=\lambda\cdot\vec{x}.
\end{equation}
Once we have one weight vector $\vec{x}$, we can construct others
via use of $E$ and $F$. We have from eq \eqref{eq:lec10:comm2}
\begin{equation}
HE=E(H+2)
\end{equation}
which, when applied to the weight vector, yields
\begin{equation}
HE\vec{x}=E(H+2)\vec{x}=(\lambda+2)E\vec{x}.
\end{equation}
This implies that $E\vec{x}$ is also an eigenvector of $H$ with
eigenvalue $\lambda+2$. Thus we have infinitely many weight
vectors, right? Well, this is {\bf wrong} since $E\vec{x}$ could
vanish! If $E\vec{x}=0$, then $\vec{x}$ is called the
\define{Highest Weight Vector}.

We also have
\begin{equation}
H(F\vec{x})=(\lambda-2)F\vec{x}
\end{equation}
by the exact same reasoning. This means that $F\vec{x}$ is also a
weight vector. We will now describe all finite dimensional
representations of $\frak{sl}(2)$.

\begin{rmk}
In finite dimensional representations, $H$ always has an eigenvector.
\end{rmk}

Lets apply $E$ to $\vec{x}$ many times, so we get new
eigenvectors. Then at some moment
\begin{equation}
HE^{k}\vec{x}=0
\end{equation}
for some $k$ since we cannot have an infinite number of distinct
eigenvectors. Let
\begin{equation}
\vec{v}:=E^{k-1}\vec{x}
\end{equation}
be the highest weight vector, so
\begin{equation}
E\vec{v}=0.
\end{equation}
Let
\begin{equation}
H\vec{v}=m\vec{v}.
\end{equation}
Let
\begin{equation}
\vec{v}_{k}=F^{k}\vec{v},
\end{equation}
we know
\begin{equation}
H\vec{v}_{k}=(m-2k)\vec{v}_{k}.
\end{equation}
This is a weight vector. We have
\begin{equation}
F\vec{v}_{k}=\vec{v}_{k+1}
\end{equation}
by definition. We should apply
\begin{equation}
E\vec{v}_{k}=EF\vec{v}_{k-1}=(FE+H)\vec{v}_{k-1}.
\end{equation}
We can guess that $E$ \emph{raises} the eigenvalue. That is
\begin{equation}
E\vec{v}_{k}=\gamma_{k}\vec{v}_{k}
\end{equation}
where $\gamma_{k}$ is some factor.

We can compute
\begin{subequations}
\begin{align}
E\vec{v}_{k} &= FE\vec{v}_{k-1}+H\vec{v}_{k-1}\\
&= F(\gamma_{k-1}\vec{v}_{k-2})+(m+2-2k)\vec{v}_{k-1}\\
&= \gamma_{k-1}F\vec{v}_{k-2}+(m+2-2k)\vec{v}_{k-1}\\
&= (m+2-2k+\gamma_{k-1})\vec{v}_{k-1}\\
&= \gamma_{k}\vec{v}_{k-1}
\end{align}
\end{subequations}
This implies that
\begin{equation}
\gamma_{k}=\gamma_{k-1}+m+2-2k
\end{equation}
a recursion relation which permits us to compute $\gamma_{k}$, an
arithmetic progression./ We have our representation be
irreducible if and only if
\begin{equation}
{\rm span}\{\vec{v}_{k}\}\iso\Bbb{C}^{n}.
\end{equation}
We have everything, we just need to compute the $\gamma_{k}$
constants. It turns out that the weights range from $m$, $m-2$,
..., $-m$.

\lecture
%%
%% lecture11.tex
%% 
%% Made by alex
%% Login   <alex@tomato>
%% 
%% Started on  Thu Jan  5 08:13:35 2012 alex
%% Last update Thu Jan  5 08:13:35 2012 alex
%%





\exercises
\begin{xca}
Let us consider chain complex $C=C_k$ where $C_k=0$ for
$k\not=n,n-1$, $C_n=C_{n-1}=\ZZ$ and the boundary operator
$C_{n}\to C_{n-1}$ acts as multiplication by $m$. Calculate
corresponding homology and cohomology with coefficients in the
group $G$. In particular, consider the cases when $G$ is an
infinite or finite cyclic group.
\end{xca}
\begin{xca}
Calculate cohomology groups with integer coefficients and with
coefficients in cyclic group $\ZZ_k$ for the following spaces.
\begin{enumerate}
\item The quotient of $S^2$ obtained by identifying north and
  south poles to a point.
\item The quotient of $S^2$ under identification of $x$ with $-x$
  for $x$ in the equator $S^1$.
\end{enumerate}
\end{xca}
\begin{xca}
Calculate relative cohomology $H_k (\RP^5, \RP^2; \ZZ)$ using
cell decomposition of $\RP^5$. (The projective space $\RP^2$ is
embedded into $\RP^5$ in standard way.) Describe the exact
cohomology sequence of the pair $(\RP^5, \RP^2)$ and check that
it is exact.
\end{xca}


\lecture[Obstruction Theory]
%%
%% lecture12.tex
%% 
%% Made by alex
%% Login   <alex@tomato>
%% 
%% Started on  Mon Oct  3 12:19:17 2011 alex
%% Last update Mon Oct  3 12:19:17 2011 alex
%%
So recall our theorem
\begin{thm}
If $a_{1}$, \dots, is a sequence where $a_{i}\not=0$ for every
$i$, and $\sum\|a_{n}\|^{-2}$ converges, then 
\begin{equation}
f(z) = \prod_{n}\left(1-\frac{z}{a_{n}}\right)\E^{z/a_{n}}
\end{equation}
converges and is entire with zeroes at $a_{1}$, \dots.
\end{thm}
We know
\begin{equation}
\sin(z)=z\prod^{\infty}_{\substack{n=-\infty\\n\not=0}}\left(1-\frac{z}{\pi n}\right)
\E^{z/(\pi n)}
\end{equation}
We collect terms
\begin{subequations}
\begin{align}
\sin(z)&=z\prod^{\infty}_{n=1}\left(1-\frac{z}{\pi n}\right)\left(1+\frac{z}{\pi n}\right)
\E^{z/(\pi n)}\E^{z/(-\pi n)}\\
&=z\prod^{\infty}_{n=1}\left(1-\frac{z^{2}}{\pi^{2}n^{2}}\right)
\end{align}
\end{subequations}
If we wrote
\begin{equation}
f(z) = \prod_{n\not=0}\left(1-\frac{z}{\pi n}\right)
\end{equation}
instead, then $f(z)$ diverges entirely. This problem is similar
to how
\begin{equation}
\sum(-1)^{n}=\left(\sum 1\right)+\left(\sum -1\right)
\end{equation}
diverges but
\begin{equation}
\sum(-1)^{n}=\sum (1-1)=0
\end{equation}
converges. So if we disregard the contribution of $\exp[z/(\pi
  n)]$, the series diverges very much as the harmonic series diverges.
Consider
\begin{equation}
\sum_{n\not=0}\frac{1}{z+n}
\end{equation}
which diverges, but
\begin{equation}
\sum_{n\not=0}\left(\frac{1}{z+n}-\frac{1}{n}\right)
\end{equation}
converges! But
\begin{equation}
\sum_{n\not=0}\frac{1}{n}=0
\end{equation}
which changes nothing.

\subsection{Gamma Function}
We will begin with examining the $\Gamma$ function. What do we
know about it? Well,
\begin{equation}
\Gamma(n+1)=n!
\end{equation}
and $\Gamma(-n)$ are singular (simple poles really). We also know
\begin{equation}
\Gamma(\mu+1)=\mu\Gamma(\mu)
\end{equation}
There is a lot of information on the $\Gamma$ function in
Marsden~\cite{marsden}. 

Consider its definition: the $\Gamma$ function is defined by
\begin{equation}\label{eq:lecture12:defnOfGammaFn}
\Gamma(\mu)=\int^{\infty}_{0}x^{\mu-1}\E^{-x}\D x
\end{equation}
Consider some $x\in\RR$, $\mu\in\CC$. But are these arbitrary?
When will the integral be defined? Well, lets consider
\begin{subequations}
\begin{align}
x^{a+\I b} &= x^{a}x^{\I b}\\
&=x^{a}\E^{\I b\log(x)}\\
&=x^{a}\E^{\I\log(x^{b})}
\end{align}
\end{subequations}
Note that $\|x^{\I b}\|=1$ for $a,b\in\RR$. We are not interested
in how our integral behaves near $x\to\infty$, since $e^{-x}$
wins out (i.e., vanishes faster than $x^{\mu-1}$ explodes). So we
are interested in the behavior of the integrand near zero. The
condition is that $\re(\mu)>0$, the integral is defined.

\begin{wrapfigure}{r}{2.75in}
\vspace{-30pt}
\begin{center}
\includegraphics{img/lecture12.0}
\end{center}
\vspace{-20pt}
\end{wrapfigure}
We need to employ our favorite phrase: analytic continuation. We
see that when $\mu=0$, the integral diverges. Similarly for
$\mu=-1$, $-2$, \dots, the integral diverges too.

There are poles but no zeroes, and for this reason it is not
entire; but its inverse $1/\Gamma(z)$ is entire.

We will consider the inverse and bring in the infinite product
for the sine. Why? Well, we have zeroes for $G(z)=1/\Gamma(z)$,
which are $\NN$. They are all multiplicity zero, so we'll use
$\NN$ to indicate the zeroes. This is a little bit sloppy, but
meh, that's life. Now we know how to construct the product for
$G(z)$ since we know its zeroes. We construct it explicitly as
\begin{equation}
G(z)=\prod^{\infty}_{n=1}\left(1+\frac{z}{n}\right)\E^{-z/n}
\end{equation}
observe $G(0)=1$. We know the relation between $G(z)$ and
$\sin(z)$ by the product series. Indeed, observe
\begin{equation}
\sin(\pi z)=\pi zG(z)G(-z).
\end{equation}
Lets see to what extent $\Gamma(\mu+1)\mu\Gamma(\mu)$ is
preserved in $G(z)$, i.e. $G(z-1)=H(z)$. We see by inspecting the
zeroes of $H(z)$ that
\begin{equation}
H(z)=\E^{g(z)}z G(z)
\end{equation}
we have some unknown factor $g(z)$.
\begin{thm}
The factor $g(z)$ is a constant denoted as $\gamma$, shorthand
for
\begin{equation}
\lim_{N\to\infty}\left(\sum^{N}_{n=1}\frac{1}{n}-\log(N)\right)
\end{equation}
This is certainly one definition of $\gamma$. (This is drawn above
to the right.)
\end{thm}
Note that numerically,
\begin{equation}
\gamma\approx0.57721\; 56649\; 01532\; 86060
\end{equation}
We will now define the $\Gamma$ function as
\begin{equation}
\Gamma(z)=\left[z\E^{\gamma z}G(z)\right]^{-1}
\end{equation}
We need to prove that $g(z)=\gamma$ is constant though. We do the
following:
\begin{equation}
\frac{\D}{\D z}\log(H(z))=\frac{\D}{\D z}\log(G(z-1))
\end{equation}
We observe
\begin{subequations}
\begin{align}
\log(H(z)) &= g(z) +\log(z) + \log(G(z))\\
&= g(z) + \log(z) +
\sum^{\infty}_{n=1}\log\left(1+\frac{z}{n}\right)-\frac{z}{n}
\end{align}
\end{subequations}
We see
\begin{subequations}
\begin{align}
\frac{\D}{\D z}\log(H(z)) &=\frac{\D}{\D z}\left(g(z) + \log(z) +
\sum^{\infty}_{n=1}\log\left(1+\frac{z}{n}\right)-\frac{z}{n}\right)\\
&=g'(z)+\frac{1}{z}+\sum^{\infty}_{n=1}\left(\frac{1}{z+n}-\frac{1}{n}\right).
\end{align}
\end{subequations}
Now we consider calculations involving $G(z)$ thus
\begin{subequations}
\begin{align}
\log(G(z-1)) &=
\log(\prod^{\infty}_{n=1}\left(1+\frac{z-1}{n}\right)\E^{(1-z)/n})\\
&=\sum^{\infty}_{n=1}\left(\log\left(1+\frac{z-1}{n}\right)-\frac{z-1}{n}\right)\\
\implies\frac{\D}{\D z}\log(G(z-1)) &=
\sum^{\infty}_{n=1}\left(\frac{1}{z+n-1}-\frac{1}{n}\right)\\
&=
\frac{1}{z}+\sum^{\infty}_{n=1}\left(\frac{1}{z+n}-\frac{1}{n}\right)
\end{align}
\end{subequations}
This implies that
\begin{equation}
g'(z)=0\implies g(z)=\mbox{(constant)}
\end{equation}
which concludes the proof of the theorem.

How to find the value of $g(z)$? Take $H(z)$ and set $z=1$, so
\begin{equation}
H(1)=e^{g(1)}G(1)
\end{equation}
which gives us a new problem: what is $G(1)$? Well,
\begin{equation}
G(1)=\lim_{N\to\infty}\prod^{N}_{n=1}\left(1+\frac{1}{n}\right)\E^{-1/n}
\end{equation}
Observe that
\begin{equation}
\prod^{N}_{n=1}\left(1+\frac{1}{n}\right)=N+1
\end{equation}
which means
\begin{equation}
G(1)=\lim_{N\to\infty}(N+1)\E^{-\sum^{N}_{n=1}(1/n)}
\end{equation}
So
\begin{subequations}
\begin{align}
\log(G(1)) &=
\lim_{N\to\infty}\log(N+1)-\sum^{N}_{n=1}\frac{1}{n}\\
&=-\gamma.
\end{align}
\end{subequations}
This concludes this lecture.

\lecture
%%
%% lecture13.tex
%% 
%% Made by Alex Nelson
%% Login   <alex@tomato3>
%% 
%% Started on  Sun Jun 20 12:08:00 2010 Alex Nelson
%% Last update Sun Jun 20 12:40:21 2010 Alex Nelson
%%
Today we will talk about compact Lie groups. First a few remarks
about characters. Suppose we have
\begin{equation}
\varphi\colon G\to\GL{n}
\end{equation}
be a group morphism. We have
\begin{equation}
\tr\big(\varphi(g)\big)=\chi_{\varphi}(g)
\end{equation}
be the character at the point $g$. The character is an invariant
of a representation. If $\varphi$, $\varphi'$ are two isomorphic
representations, so 
\begin{equation}
\varphi'(g) = A\varphi(g)A'
\end{equation}
then
\begin{equation}
\chi_{\varphi}(g)=\chi_{\varphi'}(g)
\end{equation}
for every $g\in G$. The character is a class function, it doesn't
depend on conjugation
\begin{equation}
\chi_{\varphi}(aga^{-1})=\chi_{\varphi}(g).
\end{equation}
This is basically everything we need. In general, for compact
groups $G$, the characters determine everything.

\begin{thm}
If $\rho\colon G\to\GL{n}$, $\rho'\colon G\to\GL{n}$ are two
representations such that $\chi_{\rho}(g)=\chi_{\rho'}(g)$ for
each $g\in G$, then $\rho\iso\rho'$.
\end{thm}

For a compact group, we have the invariant volume be
\begin{equation}
\int 1\cdot dv = 1
\end{equation}
which can be normalized. For a finite group, this integral
averaging a function is merely a sum
\begin{subequations}
\begin{align}
\overline{f} &= \int f(v)dv\quad\mbox{for compact groups}\\
&= \frac{1}{N}\sum_{g\in G}f(g)\quad\mbox{for finite groups}
\end{align}
\end{subequations}
We have 
\begin{equation}
\<f,f_{1}\> = \int f^{*}(v)f_{1}(v)dv = \overline{f^{*}f_{1}}
\end{equation}
we can compute the norm of characters
\begin{equation}
\|\chi\| = \sqrt{\<\chi,\chi\>} = \begin{cases} 1 & \mbox{if the rep is
  irreducible}\\
0 & \mbox{otherwise}
\end{cases}
\end{equation}
So we have
\begin{equation}
\<\chi_{i},\chi_{j}\>=\delta_{ij}
\end{equation}
so we have orthonormal characters from an orthonormal
basis. Where? For class functions!

Note class functions only depend on conjugacy classes. For
example, consider $G=\U{n}$, we can diagonalize any unitary
matrix by means of unitary transformations. We have then diagonal
matrices consists of
\begin{equation}
D = \begin{bmatrix}z_{1} & & \\
 & \ddots & \\
 &  & z_{n}
\end{bmatrix}
\end{equation}
where $z_{k}=\exp(i\varphi_{k})$, so $T\iso \U{1}^{n}$ the
$n$-torus.

What's relevant is that $T$ forms a commutative subgroup of
$\U{n}$. This is the maximal commutative subgroup for $\U{n}$, or
the maximal torus. Since every element is conjugate to this
stuff, if we want to know the character, we only need to be
concerned about the character on the torus. But can the character
be an arbitrary function here? No, it can't. So the characters
$\chi(z_{1},...,z_{n})$ is a symmetric function on $T$, so it is
invariant under any permutation.

Let $G$ be a connected, compact Lie group. Le t$T$ be the maximal
torus, i.e. the maximal Abelian subgroup. It will always be a
Torus, always a product of $\U{1}$. Then every element of $g\in
G$ is conjugate (conjugated in $G$) to an element of $T$. Not
every function on the torus is a character. We should consider
elements of $G$ that form the normalizer of $T$, i.e.
\begin{equation}
N(T) = \{x\in G\mid xT=Tx\}.
\end{equation}
We know $T\subset N(T)$ trivially, so to examine the nontrivial
part of the normalizer we should consider the quotient
\begin{equation}
N(T)/T = W
\end{equation}
called the \define{Weyl Group}\index{Weyl Group}. The Weyl group
acts on the Torus. It is obvious that characters should be
invariant with respect to this group.

Now to Lie algebras. We will work with complex Lie algebras. We
will take $G$ to be a compact Lie group, $\Bbb{C}\Lie(G)$ the
complexification of its Lie algebra. The Lie algebras obtained in
this way are \define{Reductive Lie Algebras}\index{Lie Algebra!Reductive}%
\index{Reductive Lie Algebra}. They have an invariant inner
product. Every representation of a compact group is equivalent to
a unitary representation (if complex), or an orthogonal
representation (if real). Lets take a Lie group $G$, lets take
its (real) Lie algebra 
\begin{equation}
\mathscr{G}=\Lie(G)=T_{e}G
\end{equation}
which is the tangent space to the group at $e\in G$. The group
$G$ acts on $T_{e}G$ in the obvious way. For some $x\in G$, then
\begin{equation}
gxg^{-1}\in G
\end{equation}
is an inner automorphism. It maps $e\mapsto e$. So every curve
starting at the identity goes to a curve also starting at the
identity, so we can define an action of $G$ on tangent
vectors. This is called the \define{Adjoint Representation of $G$},
this is probably the most important representation. We had a
notion of the adjoint representation for Lie algebras, and really
it's related to adjoint representations of Lie groups. We should
consider
\begin{equation}
g=1+\gamma
\end{equation}
where $\gamma$ is ``small.'' Then the adjoint group
representation is
\begin{subequations}
\begin{align}
(1+\gamma)x(1+\gamma)^{-1}&=x+\gamma x-x\gamma+\cdots \\
&= x + [\gamma,x] + \cdots
\end{align}
\end{subequations}
We are concluding that the adjoint representation of the group
corresponds to the adjoint representation of the algebra. Adjoint
representation for compact group is equivalent to an orthogonal
representation. There exists an inner product $\<x,y\>$ in
$\mathscr{G}$ which is invariant under
\begin{equation}
\<{\rm Ad}_{g}x,{\rm Ad}_{g}y\> = \<x,y\>,
\end{equation}
we see
\begin{equation}
{\rm Ad}_{1+\gamma+\cdots}x = x + [\gamma,x] + \cdots = x + {\rm
  ad}_{\gamma}x + \cdots.
\end{equation}
(The convention is adjoint representation for the Lie group is
written as ``Ad'' but for the Lie algebra is ``ad''.)

We get
\begin{equation}\label{eq:orthogonalCondition}
\<[\gamma,x],y\>+\<x,[\gamma,y]\>=0.
\end{equation}
If $G$ is compact, then $\Lie(G)$ is equipped with a
nondegenerate positive invariant product (which is precisely 
the eq \eqref{eq:orthogonalCondition} condition). We can extend
this to the complexified Lie algebra $\Bbb{C}\Lie(G)$ but it is a
nondegenerate invariant inner product. This is a general result.

Examples. Consider $\mathfrak{gl}(n)$, we have an invariant inner
product $\<x,y\>=\tr(xy)$.

\lecture
%%
%% lecture14.tex
%% 
%% Made by alex
%% Login   <alex@tomato>
%% 
%% Started on  Mon Oct  3 18:44:56 2011 alex
%% Last update Mon Oct  3 18:44:56 2011 alex
%%
There are several things of the $\Gamma$ function which we still
can discuss. There are two properties of interest: computation of
its residues, and Gauss' formula.

Gauss' formula says
$\Gamma(z)\Gamma(z+\frac{1}{n})\dots\Gamma(z+\frac{n-1}{n})$ is
related to $\Gamma(n+z)$. To guess the Gauss formula, choose
$z=m\in\NN$. So we want to relate
$\Gamma(m)\Gamma(m+\frac{1}{n})\dots\Gamma(m+\frac{n-1}{n})$. To
begin with $n=1$ is uninteresting. So lets begin with $n=2$, we
have
\begin{subequations}
\begin{align}
\Gamma(m)\Gamma\left(m+\frac{1}{2}\right) &=
(m-1)!\cdot\Gamma(1/2)\frac{1}{2}\frac{3}{2}(\dots)\frac{2m-1}{2}\\
&=\Gamma(1/2)\frac{(2m-1)!}{2^{2m-1}}\\
&=\frac{(2m-1)!\sqrt{\pi}}{2^{2m-1}}
\end{align}
\end{subequations}
Now for $n=3$ what do we have? Well, we find
\begin{subequations}
\begin{align}
\Gamma(m)\Gamma\left(m+\frac{1}{3}\right)\Gamma\left(m+\frac{2}{3}\right)
&=\frac{(3m-1)!}{3^{3m-1}}\Gamma(1/3)\Gamma(2/3)\\
&=\frac{(3m-1)!}{3^{3m-1}}\frac{2\pi}{\sqrt{3}}
\end{align}
\end{subequations}
We see that for some general $n$ that
\begin{equation}
\Gamma(m)\Gamma\left(m+\frac{1}{n}\right)(\dots)\Gamma\left(m+\frac{n-1}{n}\right)=\frac{(mn-1)!}{n^{mn-1}}\Gamma(1/n)\Gamma(2/n)(\dots)\Gamma([n-1]/n)
\end{equation}
We can compute the $\Gamma$ terms by first squaring it and
rearranging terms to read:
\begin{align}
\left(\Gamma(1/n)(\dots)\Gamma([n-1]/n)\right)^{2}
&=\Gamma\left(\frac{1}{n}\right)
\Gamma\left(\frac{n-1}{n}\right)
\Gamma\left(\frac{2}{n}\right)
\Gamma\left(\frac{n-2}{n}\right)(\dots)
\Gamma\left(\frac{n-1}{n}\right)
\Gamma\left(\frac{1}{n}\right)\nonumber\\
&=\frac{\pi^{n-1}}{\sin(\pi/n)\sin(2\pi/n)(\dots)\sin([n-1]\pi/n)}
\end{align}
So we find
\begin{equation}
\sin(\pi/n)(\dots)\sin\left(\frac{(n-1)}{n}\pi\right)=\frac{n}{2^{n-1}}
\end{equation}
we can rewrite this as
\begin{equation}
\Gamma(m)\Gamma\left(m+\frac{1}{n}\right)(\dots)\Gamma\left(m+\frac{n-1}{n}\right)
=\frac{\Gamma(mn)}{n^{nm-1}}\frac{\pi^{(n-1)/2}2^{(n-1)/2}}{\sqrt{n}}.
\end{equation}
We then replace $m\to z$ and that is Gauss' formula.

Now to compute the residues, choose $(-m)$ where $m\in\NN$, the
residue is
\begin{subequations}
\begin{align}
\lim_{z\to-m}(z+m)\Gamma(z)
&=\lim_{z\to-m}(z+m)\frac{\Gamma(z+1)}{z}\\
&=\lim_{z\to-m}(z+m)\frac{\Gamma(z+2)}{z(z+1)}\\
&=\lim_{z\to-m}(z+m)\frac{\Gamma(z+m+1)}{z(z+1)(\dots)(z+m)}\\
&=\lim_{z\to-m}\frac{\Gamma(z+m+1)}{z(z+1)(\dots)(z+m-1)}\\
&=\frac{(-1)^{m}}{m!}
\end{align}
\end{subequations}
This is the residue of the $\Gamma$ function.

\subsection{Asymptotics}

Asymptotiocs are different than approximations, we will begin
with asymptotics of factorials. Consider $N!$ for some large $N$,
the number of digits of the number is more or less
$\ln(N!)$. Lets compare $n!$ to $n^{n}$, we take
\begin{equation}
\frac{n!}{n^{n}}=\frac{1}{n}\frac{2}{n}(\dots)\frac{n}{n}
\end{equation}
We take the logarithm of this to make the product into a sum
\begin{equation}
\ln\left(\frac{n!}{n^{n}}\right)=\sum^{n}_{k=1}\ln(k/n)
\end{equation}
but this sum is not well behaved. To remedy the situation, we
divide through by $N$
\begin{subequations}
\begin{align}
\frac{1}{n}\ln(n!/n^{n}) &= \frac{1}{n}\left[\ln\left(\frac{1}{n}\right)+(\dots)+\ln\left(\frac{n}{n}\right)\right]\\
&\approx\left.\int^{1}_{0}\ln(x)\D x = \lim_{\varepsilon\to0}x\ln(x)-x\right|^{1}_{\varepsilon}\\
&\phantom{\approx\int^{1}_{0}\ln(x)\D x} = (0-0)-(1-0)=-1.
\end{align}
\end{subequations}
We see that
\begin{equation}
\ln\left(\sqrt[n]{\frac{n!}{n^{n}}}\right)\approx-1
\end{equation}
so we exponentiate both sides
\begin{equation}
\sqrt[n]{\frac{n!}{n^{n}}}\approx\E^{-1}
\end{equation}
then we raise both sides to the $n^{\rm th}$ power
\begin{equation}
\frac{n!}{n^{n}}\approx\E^{-n}\implies
n!\approx\left(\frac{n}{\E}\right)^{n}
\end{equation}
which is a very rough asymptotic formula, but it was the first
asymptotic formula for the factorial.

There are more precisely asymptotic formulas, of which Stirling's
is the most famous. It states
\begin{equation}
n!\asymptote\sqrt{2\pi n}\left(\frac{n}{\E}\right)^{n}
\end{equation}
What about the relative error of using $\sqrt{2\pi n}(n/\E)^{n}$
instead of $n!$, that is the error in terms of percents. Consider
a more precise form of Stirling's approximation
\begin{equation}
n!\asymptote\sqrt{2\pi n}\left(1+\frac{1}{12n}\right)\left(\frac{n}{\E}\right)^{n}
\end{equation}
There is in fact an infinite series, then next asymptotic would be
\begin{equation}
n!\asymptote\sqrt{2\pi n}\left(1+\frac{1}{12n}+\frac{1}{288n^{2}}\right)\left(\frac{n}{\E}\right)^{n}
\end{equation}
and so on.

Consider $n=10$, then
\begin{subequations}
\begin{align}
10! &= 3\;628\;800\\
\sqrt{2\pi10}\left(\frac{10}{\E}\right)^{10} & \approx
3\;598\;695\\
\intertext{and}
\sqrt{2\pi10}\left(1+\frac{1}{120}\right)\left(\frac{10}{\E}\right)^{10} & \approx
3\;628\;685
\end{align}
\end{subequations}
and the next expansion would be precise to one digit, probably.

Consider the asymptotics for the number of primes less than
$n$.\marginpar{Prime number function $\pi(n)$} This is a special function denoted
\begin{equation}
\pi(n)=\mbox{number of primes less than $n$}
\end{equation}
We have two estimates: Euler's formula 
\begin{equation}
\pi(n)\asymptote\log(n)/n
\end{equation}
and the logarithmic integral function (``li's integral'') 
\begin{equation}
\pi(n)\asymptote\int^{n}_{0}(1/\ln(t))\D t. 
\end{equation}
The latter is better. Consider $n=10^{9}$, what happens? Well we
see (truncating to integer values) that
\begin{subequations}
\begin{align}
 n &= 10^{9}\\
\pi(n) &= 50\;847\;534\\
\frac{n}{\ln(n)} &\approx 48,429,482\\
\int^{n}_{2}\frac{\D t}{\ln(t)} &= 50,849,235
\end{align}
\end{subequations}
Can we say anything about the error? Yes and no: yes because yes,
and no because no. Most theorems about the error depends on the
Riemann zeta conjecture being true.

If the Riemann hypothesis is true, then
\begin{equation}
|\Li(n)-\pi(n)|<\frac{\sqrt{n}\log(n)}{8\pi}
\end{equation}
where
\begin{equation}
\Li(n) := \int^{n}_{2}\frac{\D t}{\log(t)}
\end{equation}
The difference $\Li(n)-\pi(n)$ changes sign infinitely many
times (John Littlewood proved this fact in 1914), the first time
is at $10^{349}$. (Although a more recent estimate puts this
around $10^{316}$.)

\marginpar{Partition function $p(n)$}The partition function $p(n)$ counts the number of ways to write
$n$ as a sum of positive numbers. So for example,
\begin{equation}
p(4)=5
\end{equation}
since we have
\begin{equation}
4,\quad 3+1,\quad 2+2,\quad 2+1+1,\quad 1+1+1+1
\end{equation}
are the five distinct sums.
\marginpar{Rademacher's formula}We have an asymptotic formula for it
\begin{equation}
p(n)\asymptote\frac{1}{4n\sqrt{3}}\E^{2\pi\sqrt{n/6}}
\end{equation}
This grows faster than any polynomial, but slower than any exponential.

\lecture
%%
%% lecture15.tex
%% 
%% Made by alex
%% Login   <alex@tomato>
%% 
%% Started on  Sat Dec 17 11:11:56 2011 alex
%% Last update Sat Dec 17 11:11:56 2011 alex
%%
We gave several definitions and we would like to repeat them and
give a couple of new ones, all of them are useful. We may prove
their equivalence later.

\marginpar{Definition 1}A \define{Chern Class}\index{Chern Class} are obstructions for complex manifolds
with structure group $\GL{n}$ or $\U{n}\propersubset\GL{n}$. We are
working with complex vector bundles. We had taken the space
$V_{n,k}$ (the Stiefel manifold\index{Stiefel Manifold}) and $\U{n}$ or $\GL{n,\CC}$
acts there. Well, they act on $V^{orth}_{n,k}$ and $V_{n,k}$
respectively. The Chern class was introduced as the first
obstruction to the corresponding bundle. So $c_{l}$ comes in
every even $k$ since $\pi_{l+1}(V_{n,l})\not=0$ for us, and we
get $H^{l+1}\bigl(B,\pi_{l}(F)\bigr)$. The $l$ can be expressed
in terms of $k$, namely $l=2k\pm1$ (the $\pm1$\dots well, look up
the sign). So $c_{k}\in H^{2k}(B,\ZZ)$.

There are other definitions.\marginpar{Definition 2} We can define the characteristic
class as $c_{k}\in H^{2k}(B_{\U{n}},\ZZ)$ for the classifying
space $B_{\U{n}}$. We use the classifying map 
\begin{equation}
\varphi\colon B\to B_{\U{n}}
\end{equation}
then use
\begin{equation}
\varphi^{*}\colon H^{2k}(B_{\U{n}},\ZZ)\to H^{2k}(B,\ZZ)
\end{equation}
to find $\varphi^{*}(c_{k})$. We know
$B_{\U{n}}=\Gr_{\infty,n}$. \hyperref[defn:maximalTorus]{\textsc{Recall}}
(page \pageref{defn:maximalTorus}) the notion of a maximal torus
$T\propersubset\U{n}$ which consists of diagonal matrices. We
have an embedding
\begin{equation}
B_{T}\to B_{\U{n}}
\end{equation}
there is a map
\begin{equation}
E/T\to E/\U{n}.
\end{equation}
This gives a morphism
\begin{equation}
H^{\bullet}(B_{\U{n}})\to H^{\bullet}(B_{T},\ZZ)
\end{equation}
but we know that if $T$ is one-dimensional then
\begin{equation}
B_{T}=\CP^{\infty},
\end{equation}
and if $T$ is $n$-dimensional, then
\begin{equation}
B_{T}=\bigl(\CP^{\infty}\bigr)^{n}.
\end{equation}
We know the cohomology ring
\begin{equation}
H^{\bullet}(B_{T})=\ZZ[\zeta_{1},\dots,\zeta_{n}]
\end{equation}
with $\dim(\zeta_{i})=2$. We may asily describe the image of
\begin{equation}
H^{\bullet}(B_{\U{n}},\ZZ)\to H^{\bullet}(B_{T},\ZZ),
\end{equation}
the image consists of symmetric polynomials. The group of
permutations act trivially on $H^{\bullet}(B_{\U{n}},\ZZ)$ but
nontrivially on $H^{\bullet}(B_{T},\ZZ)$. This is not new.

\index{Chern Class!as Symmetric Polynomial|(}
\marginpar{Chern classes as Symmetric Polynomials}But now we would like to say that this $c_{k}$ is an elementary
symmetric polynomial. This is really nice as a definition, since
$c_{k}$ are the only characteristic class; because every
symmetric polynomial may be expressed in terms of elementary
symmetric polynomials, so all characteristic classes may be
expressed in terms of $c_{k}$. We have
\begin{subequations}
\begin{align}
c_{1} &= \zeta_{1}+\dots+\zeta_{n},\\
c_{2} &= \sum_{i<j}\zeta_{i}\zeta_{j}
\end{align}
\end{subequations}
and so on. \marginpar{Usefulness of definition}This is the most useful definition if we wish to prove
any theorem about Chern classes---use this definition!
\index{Chern Class!as Symmetric Polynomial|)}

\index{Chern Class!Axiomatic Definition|(}
\marginpar{Definition 3: Axiomatic}The last definition is axiomatic in character. The first is that
\begin{enumerate}
\item $c_{k}$ is a characteristic class;
\item if we have the direct sum of bundles, then
\begin{equation}
c_{n}(E\oplus F)=\sum_{k+l=n}c_{k}(E)c_{l}(F)
\end{equation}
and these two axioms requires a normalization condition, i.e.,
another axiom:
\item $c_{n}(E)=0$ if $n>\dim(E)$
\end{enumerate}
C.f.\ Milnor and Stasheff's axiomatic framework for the
Stiefel--Whitney classes~\cite[Ch.\ 4]{milnor}.

If we have a line bundle, then we have only one Chern
class. Consider a line bundle $E$ over $\CP^{n}$. This is then
related to $B_{\U{1}}=\CP$ and we have $c_{1}$ be the generator
of $H^{2}(\CP^n,\ZZ)$.
\index{Chern Class!Axiomatic Definition|)}

\marginpar{\emph{\textbf{TODO:}} prove equivalence of definitions}So we have these three definitions of the Chern class. We should
prove the existence and uniqueness of the axiomatic version, then
prove the equivalence of all three definitions.

We will very briefly discuss the case of $\GL{n,\RR}$ (the same
with $\ORTH{n,\RR}$). We should consider $V_{n,k}(\RR)$. We have
the notion of Chern class work, but it is $w_{k}\in
H^{k}(B,\ZZ_{2})$ --- the analog of the first definition. We have
the third axiomatic definition merely change $c_{k}\mapsto
w_{k}$, nothing but notation changes. For the second definition,
don't consider the torus, but
$(\ZZ_{2})^{n}\propersubset\ORTH{n}$ the diagonal matrices have
components be $\pm1$.

Now this is not the end of the story; Stiefel--Whitney classes
are classes over $\ZZ_2$. We also have $\ZZ$-classes for the
group $\SO{n}$---the Euler characteristic is one example. But we
have more: namely, we have the \define{Pontryagin Classes}\index{Pontryagin Classes}\index{Characteristic Class!Pontryagin Classes}.
The picture is very simple, look we have an embedding
$\GL{n,\RR}\propersubset\GL{n,\CC}$. What does it mean? It means
\begin{equation}
B_{\GL{n,\RR}}\to B_{\GL{n,\CC}}
\end{equation}
or we may work with homotopy equivalent stuff
$\ORTH{n}\propersubset\U{n}$. Therefore we have the map
\begin{equation}
B_{\ORTH{n}}\to B_{\U{n}}.
\end{equation}
We may also go the other way, a $n$-dimensional complex bundle
may be considered a  $2n$-dimensional real bundle. So we have a
map
\begin{equation}
H^{\bullet}(B_{\U{n}},\ZZ)\to H^{\bullet}(B_{\ORTH{n}},\ZZ)
\end{equation}
which maps $c_{i}\mapsto\textbf{??}$. But the problem is that it is not
quite clear we have a nonzero characteristic class. We need to
take into account the Weyl group to see 
\begin{subequations}
\begin{equation}
c_{2k+1}\mapsto 0
\end{equation}
and
\begin{equation}
c_{2k}\mapsto\pm P_{k}.
\end{equation}
\end{subequations}
The $P_{k}$ are called ``\emph{Pontryagin Classes\/}'' which have
$\dim(P_{k})=4k$; Chern classes had $\dim(c_{k})=2k$. So there is
nothing magical here.

\subsection*{EXERCISES}
\begin{xca}
Calculate the first obstruction to the construction of non-vanishing vector field on a sphere with $h$ handles.
\end{xca}
\begin{xca}\label{xca:lec15:prob2}
Let us consider a principal bundle with total space $S^{2n+1}$, group $S^1$ and base $\CP^n$. Calculate the first obstruction to the construction of section of this bundle.
\end{xca}
\begin{xca}
Consider a bundle with a fiber $S^{2k-1}$ associated with the principal bundle of Problem \ref{xca:lec15:prob2}. (We assume that the sphere $S^{2k-1}$ is realized as a unit sphere in complex space $\CC^k$ and $z\in S^1$, where $z$ is a complex number having absolute value 1, transforms a point $(z_1, \dots, z_k)\in S^{2k-1}$ into $(zz_1, \dots, zz_k)$.) Calculate the first obstruction to the construction of section of this bundle.
\end{xca}
\subsection{K-Theory}
Right now we will give some main definitions and the relationship
to characteristic classes. Complex vector bundles are classified
by maps $B\to B_{\U{n}}$, so homotopy classification
$\homotopyClass(B,B_{\U{n}})$ is hard. But we may discuss stable
homotopy groups! Indeed, $K$-theory\index{K-Theory@$K$-Theory!as stable theory of vector bundles} is the stable theory of
vector bundles, if you like; it's much simpler.

Let $B$ be a connected, compact space (we will assume its
cellular decomposition is a finite polyhedron). We have trivial
bundles\index{$\varepsilon^{r}$|see{Trivial Bundle}}
\begin{equation}
E_{trivial} = \varepsilon^{r} = B\times\CC^{r}.
\end{equation}
Now what to do? We will say if we have  a vector bundle $E$ and
we take a direct sum
\begin{equation}
E\oplus\varepsilon^{r}\stabequiv E
\end{equation}
are \define{Stably Equivalent}\index{Stably Equivalent}\index{Equivalence!Stably}.
This should be transitive, so if 
\begin{equation}
E\oplus\varepsilon^r\iso E'\oplus\varepsilon^s\quad\mbox{then}\quad
E\stabequiv E'.
\end{equation}
Two guys are stably equivalent if and only if they're equal when
adding trivial bundles.

We'd lke to stress that characteristic
classes\index{Characteristic Class!and Stable-Equivalent Bundles} are equal for
stable-equivalent bundles. Why? The  trivial bundle has a
characteristic class equal to 0, and from axiom (2) it follows
immediately. 

\lecture[$K$-Theory]
%%
%% lecture16.tex
%% 
%% Made by alex
%% Login   <alex@tomato>
%% 
%% Started on  Tue Mar  6 15:06:59 2012 alex
%% Last update Tue Mar  6 15:06:59 2012 alex
%%
We will assume the cosmological constant vanishes $\Lambda=0$. We
see the field equations look like
\begin{equation}
G_{\mu\nu}=\frac{\kappa^{2}}{2}T_{\mu\nu}
\end{equation}
Ten components of the curvature tensor directly depend on $T_{\mu\nu}$.
The Weyl tensor indirectly depends on it. The $G_{\mu\nu}$
vanishes for flat space, yet the Weyl tensor describes the free
propagation of gravity waves.

Recall in electromagnetism, the source of the electric field is
charge $e$ or in the field equations charge density
$\rho_{e}$. With Lorentz transformation (viz.~length
contraction), charge density increases because volume decreases:
\begin{equation}
\begin{split}
&\rho_{e}\to\frac{1}{\sqrt{1-v^{2}}}\rho_{e}\\
\implies&\rho_{e}\sim J^{0}
\end{split}
\end{equation}
where $J^{\mu}$ is the 4-current.

We know mass is responsible for gravity, but rest mass or total
mass? Observationally, all forms of energy contributes to the
gravitational field. So the energy $E$ has energy density
$\rho_{m}$. How does this transform? Well, we see:
\begin{equation}
\begin{split}
E&\to\frac{1}{\sqrt{1-v^{2}}}E\\
\rho_{m}&\to\left(\frac{1}{1-v^{2}}\right)\rho_{m}
\end{split}
\end{equation}
This is what happens for a $00$ components of a rank-2 tensor. So
\begin{subequations}
\begin{equation}
\rho\approx T^{00}
\end{equation}
and similarly
\begin{equation}
\begin{split}
T^{0i}&\approx\mbox{``Energy Current''}\\
&\approx\mbox{``Momentum Density''}
\end{split}
\end{equation}
and
\begin{equation}
T^{ij}\approx\mbox{``Pressure''}.
\end{equation}
\end{subequations}
The field equations were thought of as
\begin{equation}
R_{\mu\nu}+g_{\mu\nu}R\propto T_{\mu\nu}
\end{equation}
just as for Newtonian gravity
\begin{equation}
\nabla^{2}\Phi=4\pi G\rho_{m}.
\end{equation}
Einstein at one point proposed
\begin{equation}
R_{\mu\nu}=T_{\mu\nu}
\end{equation}
but we can't change coordinates correctly, as these equations are
under-determined. We know in special relativity the conservation
of energy states
\begin{equation}
\partial_{\mu}T^{\mu\nu}=0
\end{equation}
So by the comma-goes-to-semicolon rule, we expect
\begin{equation}
\nabla_{\mu}T^{\mu\nu}=0.
\end{equation}
But only
\begin{equation}
\nabla_{\mu}G^{\mu\nu}=0
\end{equation}
whereas
\begin{equation}
\nabla_{\mu}R^{\mu\nu}\not=0.
\end{equation}

In general relativity, we use $I$ for the action and $S$ for the
entropy (when we Wick rotate $t\to\tau=-\I t$, the Euclidean action for a
black hole describes \emph{is} its entropy). We have the action
\begin{equation}
I=\int\sqrt{-g}L\,\D^{4}x
\end{equation}
and its variation is
\begin{equation}
\delta I=\int\sqrt{-g}E_{\mu\nu}\delta g^{\mu\nu}\,\D^{4}x
\end{equation}
This action is diffeomorphism-invariant. If $\delta g^{\mu\nu}$
is just a coordinate transformation, then $\delta I=0$ identically.
This is true ``on shell'' (when the equations of motion are
satisfied). What is $\delta g^{\mu\nu}$ under a change of
coordinates?
Consider
\begin{equation}
x^{\mu}\to x^{\mu}+\zeta^{\mu}
\end{equation}
We see then that
\begin{equation}
g_{\mu\nu}(x)\D x^{\mu}\D x^{\nu}
\to g_{\mu\nu}(x+\zeta)
\bigl(\D x^{\mu}+\partial_{\rho}\zeta^{\mu}\D x^{\rho}\bigr)
\bigl(\D x^{\nu}+\partial_{\sigma}\zeta^{\nu}\D x^{\sigma}\bigr)
\end{equation}
where we Taylor expand to first order the metric
\begin{equation}
g_{\mu\nu}(x+\zeta)=g_{\mu\nu}(x)+\zeta^{\tau}\partial_{\tau}g_{\mu\nu}(x).
\end{equation}
Observe this tells us how the metric changes, after some index
gymnastics we obtain
\begin{equation}
\begin{split}
g_{\mu\nu}&\to g_{\mu\nu}+(g_{\mu\rho}\partial_{\nu}\zeta^{\rho}+g_{\rho\nu}\partial_{\mu}\zeta^{\rho}+\zeta^{\rho}\partial_{\rho}g_{\mu\nu})\\
&\qquad=g_{\mu\nu}+\nabla_{\mu}\zeta_{\nu}+\nabla_{\nu}\zeta_{\mu}
\end{split}
\end{equation}
Thus
\begin{equation}\label{eq:lec16:killingVec}
\delta_{\zeta}g_{\mu\nu}=\nabla_{\mu}\zeta_{\nu}+\nabla_{\nu}\zeta_{\mu}.
\end{equation}
If the right hand side vanishes, we have a Killing vector (c.f.,
Exercise~\ref{xca:prob4:killing}). If we say the metric is
time-independent, then this is equivalent to stating there exists
some time-like Killing vector. Similarly, spherical symmetry
means that we have Killing vectors generate the spherical symmetries.

So, we see that
\begin{equation}
\delta_{\zeta}g^{\mu\nu}=-\nabla^{\mu}\zeta^{\nu}
-\nabla^{\nu}\zeta^{\mu}
\end{equation}
So under a coordinate transformation
\begin{equation}
\delta I=-\int\sqrt{-g}E_{\mu\nu}(\nabla^{\mu}\zeta^{\nu}+\nabla^{\nu}\zeta^{\mu})\,\D^{4}x.
\end{equation}
We have $E_{\mu\nu}=E_{\nu\mu}$ which simplifies the integrand
\begin{equation}
\delta I=-2\int\sqrt{-g}E_{\mu\nu}\nabla^{\mu}\zeta^{\nu}\,\D^{4}x
\end{equation}
since we're summing over dummy indices and $E$ is symmetric. Now
we may write this as
\begin{equation}
\delta I=-2\int\sqrt{-g}\Bigl[
\nabla^{\mu}(E_{\mu\nu}\zeta^{\nu})-(\nabla^{\mu}E_{\mu\nu})\zeta^{\nu}
\Bigr]\,\D^{4}x
\end{equation}
Recall
\begin{equation}
\nabla_{\mu}v^{\mu}=\frac{1}{\sqrt{-g}}\partial_{\mu}(\sqrt{-g}v^{\mu})
\end{equation}
thus the first term in the integrand becomes
\begin{equation}
-2\int\sqrt{-g}\nabla^{\mu}(E_{\mu\nu}\zeta^{\nu})\,\D^{4}x=-2\int\partial_{\mu}(\sqrt{-g}{E^{\mu}}_{\nu}\zeta^{\nu})\,\D^{4}x
\end{equation}
which we can always do, since the metric's covariant derivative
vanishes. This is a surface integral! Thus if $\zeta\to0$ ``fast
enough'' (or, equivalently, $\zeta=0$ on the boundary), the first
term vanishes.

Therefore
\begin{equation}
\int\sqrt{-g}(\nabla^{\mu}E_{\mu\nu})\zeta^{\nu}\,\D^{4}x=0
\end{equation}
which is true for arbitrary $\zeta$. This implies
\begin{equation}
\nabla^{\mu}E_{\mu\nu}=0
\end{equation}
which is a conservation law. But we cannot change it into
integral form. This is a special case of Noether's theorem. We
can run this backwards to get the equations of motion.

% the next lecture was a review of the homework problems, so
% we're right on track!
\lecture
%%
%% lecture17.tex
%% 
%% Made by alex
%% Login   <alex@tomato>
%% 
%% Started on  Thu Jan  5 08:21:15 2012 alex
%% Last update Thu Jan  5 08:21:15 2012 alex
%%



\exercises
Given two disjoint connected $n$-manifolds $M$ and $N$, a
connected $n$-manifold $M\connectSum{N}$, their connected
sum\index{Connected Sum}\index{Sum!Connected}, can be constructed
by deleting the interiors of small closed balls $B_1\propersubset
M$ and $B_2\propersubset N$ and identifying the resulting
boundary spheres $\partial B_1$ and $\partial B_2$ via some 
homeomorphism between them.
\begin{xca}
Assuming that $M$ and $N$ are closed orientable manifolds prove
that $H_k(M\connectSum{N}; \ZZ)$ is isomorphic to direct sum of
$H_k(M; \ZZ)$ and $H_k(N; \ZZ)$ for $0 < k < n$.
\end{xca}
\begin{xca}
Let $M$ denote a closed $n$-dimensional connected orientable
manifold. Assuming that we know the cohomology of $M$ calculate
the cohomology with compact supports of $M \setminus A$ where
\begin{enumerate}
\item $A$ is a finite subset of $M$,
\item $A$ is a union of boundary spheres $\bdry B_1$,\dots,
  $\bdry B_n$ of non-overlapping small closed balls $B_1$, \dots,
  $B_n$ in $M$.
\end{enumerate}
\end{xca}

\lecture
%%
%% lecture18.tex
%% 
%% Made by alex
%% Login   <alex@tomato>
%% 
%% Started on  Sun Dec 25 09:45:58 2011 alex
%% Last update Sun Dec 25 09:45:58 2011 alex
%%
Let $B$ be a topological space. We want to consider the
collection of vector bundles over $B$. Denote this collection of
vector bundles by $A=\Vect(B)$. Observe that $A$ is a semigroup.
But we may promote it to a group by
\begin{equation}
A\mapsto\widetilde{A}=K(B).
\end{equation}
Two elements $E,E'\in\Vect(B)$ define the same element of $K(B)$
if and only if for some $n$ we have
\begin{equation}
E\oplus\varepsilon^n=E'\oplus\varepsilon^n.
\end{equation}
We have stable equivalence
\begin{equation}
E\stabequiv E'
\end{equation}
defined by
\begin{equation}
E\stabequiv E' \iff E\oplus\varepsilon^m=E\oplus\varepsilon^n.
\end{equation}
Please note the ``exponents'' differ here. Thus we get a map
\begin{equation}
K(B)\to\widetilde{K}(B)
\end{equation}
We have a morphism
\begin{equation}
\dim\colon K(B)\to\ZZ
\end{equation}
It is clear we have a notion of dimension in $K(B)$, but it's
not an invariant in $\widetilde{K}(B)$. But we may consider
\begin{equation}
K(B)\xrightarrow{\sim}\ZZ\oplus\widetilde{K}(B)
\end{equation}
which is an isomorphism.

How do we calculate these $K$ groups? We know how to calculate
soem of them. We have seen how $\widetilde{K}(S^{n})$, the
calculations boiled down to considering
\begin{equation}
\pi_{n-1}\bigl(\U{\infty}\bigr)=\widetilde{K}(S^{n}).
\end{equation}
We know, for example,
$\widetilde{K}(S^{2})=\pi_{1}\bigl(\U{\infty}\bigr)=\ZZ$.

Consider a ``good'' subspace $A\propersubset B$ (if $B$ is a cell
complex, $A$ is a subcomplex and thus always ``good''). The
relative cohomology
\begin{equation}
H^{k}(B\bmod A)=\widetilde{H}^{k}(B/A).
\end{equation}
We have $A\propersubset B$, which implies we have a map in the
opposite direction
\begin{equation}
\dots\gets H^{k}(A)\gets H^{k}(B)\gets H^{k}(B,A)\gets\dots
\end{equation}
so we have an exact sequence. We would like to prove that for
$K$-theory we have a similar exact sequence.

This is an extraordinary sequence\index{Sequence!Extraordinary}\index{Extraordinary Sequence}
\begin{equation}
\widetilde{K}(A)\gets\widetilde{K}(B)\gets\widetilde{K}(B/A),
\end{equation}
we have the map $\widetilde{K}(A)\gets\widetilde{K}(B)$ because
we pull-back the bundle to $A$ and likewise $B/A\to B$ pulled
backed.  We have a sequence, but we should prove it is exact.

\begin{wrapfigure}{r}{7pc}
  \includegraphics{img/lecture18.0}
\end{wrapfigure}
Lets first recall some basic topological constructions\index{Obstruction!Topological}. If we
have a space $A$, then we can construct the cone over
$A$\index{Cone!Over a Space} ---
denoted $CA$ --- as doodled on the right. But from this, we can
form the suspension over $A$, which is denoted by $S(A)$ and
turns out completely equivalent to $CA/A$. Again, the
suspension\index{Suspension!Over a Space}
is doodled on the right, and we can obtain it from a continuous
mapping $CA\to S(A)$. It amounts to taking our original space $A$
(which is shaded) and contracting it to a single point.

Really, we have
\begin{equation}
\widetilde{K}(A)\gets
\widetilde{K}(B)\gets
\widetilde{K}(\underbracket[0.5pt]{B\cup CA}_{B/A})\gets
\widetilde{K}(\underbracket[0.5pt]{B\cup CA\cup CB}_{S(A)})\gets
\dots
\end{equation}

\begin{wrapfigure}{l}{7pc}
  \vspace{-30pt}
  \includegraphics{img/lecture18.1}
\end{wrapfigure}
\noindent{}Lets consider what this looks like topologically. We
have $A$ be a subcomplex of $B$, so a nice picture might be a
smaller disc contained in a larger disc. We consider the cone
$CA$ over $A$. We should observe the critically important
property that $B\cup CA\cup CB$ is the same as the suspension of
$A$, denoted by $SA$. This is the same stuff as $CA/A$, as we
have already discussed. We get another term in our exact sequence
\begin{equation}
\widetilde{K}(A)\gets
\widetilde{K}(B)\gets
\widetilde{K}(B/A)\gets
\widetilde{K}(SA)\gets
\widetilde{K}(SB)\gets
\dots
\end{equation}
So now we'd like a cohomological analog of what we have done
here. Instead of $\widetilde{K}(A)$ we will write
$\widetilde{K}^{0}(A)$. This is the same stuff, just different
notation, so $\widetilde{K}=\widetilde{K}^{0}$. Now our exact
sequence becomes
\begin{equation}
\widetilde{K}^{0}(A)\gets
\widetilde{K}^{0}(B)\gets
\widetilde{K}^{0}(B/A)\gets
\widetilde{K}^{-1}(A)\gets
\widetilde{K}^{-1}(B)\gets
\dots
\end{equation}
We use the notation
\begin{equation}
\widetilde{K}(SA)=\widetilde{K}^{-1}(A).
\end{equation}
So if we introduce this notation, we see immediately that it is a
cohomological theory---it has the same exact sequence as
cohomology. For real bundles, we use the notation $KO$. For
complex bundles we have Bott periodicity\index{Bott Periodicity}:
\begin{equation}
\widetilde{K}(S^{2}A)=\widetilde{K}(A).
\end{equation}
This is a slightly more general statement than Bott
periodicity. We had Bott periodicity describe
\begin{equation}
\widetilde{K}(S^{n+2})=\widetilde{K}(S^{n}).
\end{equation}
Our notation reads
\begin{equation}
\widetilde{K}(S^{n}A)=\widetilde{K}^{-n}(A),
\end{equation}
so $\widetilde{K}^{-n-2}(A)=\widetilde{K}^{n}(A)$.

\lecture
%%
%% lecture19.tex
%% 
%% Made by alex
%% Login   <alex@tomato>
%% 
%% Started on  Fri Jul  6 11:16:20 2012 alex
%% Last update Fri Jul  6 11:16:20 2012 alex
%%

So lets consider a binary neutron star. Let $r$ be the radius of
the binary neutron star, and we are a distance $R$ away from the
center of mass. We can doodle the situation:
\begin{center}
\includegraphics{img/lecture19.0}
\end{center}
If $R\ggg r$, then we can approximate this as
\begin{equation}
\bar{h}_{\mu\nu} \approx\frac{4G}{R}\int
T_{\mu\nu}(\vec{y},t-R)\,\D^{3}y.
\end{equation}
Remember we're in a gauge where
\begin{equation}
\partial_{\mu}\bar{h}^{\mu\nu}=0
\end{equation}
which implies $h^{t\mu}$ is determined by $h^{ij}$. So for all
practical purposes, we can just compute $h^{ij}$.

Now, for some tricks:
\begin{equation}
\begin{split}
\partial_{k}(x^{i}T^{jk})
&={\delta^{i}}_{k}T^{jk} + x^{i}\partial_{k}T^{jk}\\
&=T^{ij} + x^{i}\partial_{k}T^{jk}\\
&=T^{ij} - x^{i}\partial_{t}T^{jt} + \bigO(h).
\end{split}
\end{equation}
Thus we have
\begin{equation}
T^{ij} = \partial_{t}(x^{i}T^{tj})
+ \begin{pmatrix}\mbox{total}\\\mbox{derivative}
\end{pmatrix}
+\bigO(h)
\end{equation}
Using Stoke's theorem, the integral of a total derivative is
zero, so we have
\begin{equation}
\bar{h}_{ij} = \frac{2G}{R}\frac{\D^{2}}{\D t^{2}}\int
\underbrace{y^{i}y^{j}T^{tt}(\vec{y},t-R)}_{\mathclap{\text{mass quadrapole moment}}}\,\D^{3}y
\end{equation}
which confirms the handwaving arguments from the last lecture, which
we justified with conservation laws.

Observe the quadrapole moment behaves as
\begin{equation}
I_{ij}\sim mr^{2}
\end{equation}
for a binary star with comparable masses. So we see
\begin{equation}
\ddot{I}_{ij}\sim mv^{2}
\end{equation}
If this is a gravitationally bound system, it works out\dots but
this means that
\begin{equation}
h\sim G\frac{mv^{2}}{R}.
\end{equation}
Further we know for gravitational systems
\begin{equation}
v^{2}\sim G\frac{m}{r}
\end{equation}
and thus
\begin{equation}
h\sim v^{4}\left(\frac{r}{R}\right).
\end{equation}
We discuss these things in detail in the following box. A binary
neutron star affects the distance by about $1/1000$ of the
diameter of a nucleus, though.
%%
%% simplificationsForWeakRad.tex
%% 
%% Made by alex
%% Login   <alex@tomato>
%% 
%% Started on  Sun Mar 11 14:00:37 2012 alex
%% Last update Sun Mar 11 14:00:37 2012 alex
%%


\begin{Boxed}{Some Simplifications for Weak Gravitational Radiation}
We saw that to first order in perturbation theory
\begin{equation}
\bar{h}_{\mu\nu}(\mathbf{x},t)=
4G\int\frac{T_{\mu\nu}(\mathbf{y},t-|\mathbf{y}-\mathbf{x}|)}{|\mathbf{y}-\mathbf{x}|}\,\D^{3}y
\end{equation}
Let us concentrate on the purely spatial components
$\bar{h}_{ij}$ since the remaining components $\bar{h}_{0\mu}$
may be obtained by using the harmonic gauge condition
$\partial^{\mu}\bar{h}_{\mu\nu}=0$.

First, suppose an isolated source is at a distance $R$, and has
linear size $r\lll R$. Then to a good approximation,
\begin{equation}\label{eq:box2:goodApprox}
\bar{h}_{\mu\nu}(\mathbf{x},t)=
\frac{4G}{R}\int T_{\mu\nu}(\mathbf{y},t-|\mathbf{y}-\mathbf{x}|)\,\D^{3}y
\end{equation}
Now, to lowest order in $h$, energy conservation implies that
\begin{equation}
\partial_{\mu}T^{\mu\nu}=0=\partial_{i}T^{i\nu}+\partial_{t}T^{t\nu}
\end{equation}
We can now use a trick. Note the identities
\begin{equation}\label{eq:box2:trick:id1}
\begin{split}
\partial_{k}(x^{i}T^{kj})&=\delta^{i}_{k}T^{kj}+x^{i}\partial_{k}T^{kj}\\
&=T^{ij}-x^{i}\partial_{t}T^{tj}
\end{split}
\end{equation}
\begin{equation}\label{eq:box2:trick:id2}
\begin{split}
\partial_{\ell}(x^{i}x^{j}T^{t\ell})
&=\delta^{i}_{\ell}x^{j}T^{t\ell}+\delta^{j}_{\ell}x^{i}T^{t\ell}
+x^{i}x^{j}\partial_{\ell}T^{t\ell}\\
&=x^{j}T^{ti}+x^{i}T^{tj}-x^{i}x^{j}\partial_{t}T^{t\ell}
\end{split}
\end{equation}
Solving \eqref{eq:box2:trick:id1} for $T^{ij}$, using the
symmetry of $T^{ij}$, and inserting \eqref{eq:box2:trick:id2},
we see that
\begin{align}
T^{ij}
&=x^{i}\partial_{t}T^{tj}+\partial_{k}(x^{i}T^{kj})\nonumber\\
&=\frac{1}{2}\partial_{t}(x^{i}T^{tj}+x^{j}T^{ti})+
\frac{1}{2}\partial_{k}(x^{i}T^{kj}+x^{j}T^{ik})\nonumber\\
&=\frac{1}{2}\partial_{t}\Bigl(\partial_{\ell}(x^{i}x^{j}T^{t\ell})+x^{i}x^{j}\partial_{t}T^{tt}\Bigr)
+\frac{1}{2}\partial_{k}\Bigl(x^{i}T^{kj}+x^{j}T^{ik}\Bigr)\nonumber\\
&=\frac{1}{2}\partial_{t}^{2}(x^{i}x^{j}T^{tt})
+\frac{1}{2}\partial_{\ell}\Bigl(\partial_{t}(x^{i}x^{j}T^{t\ell})+x^{i}T^{\ell j}+x^{j}T^{i\ell}\Bigr)
\end{align}
We plug this back into Equation \eqref{eq:box2:goodApprox}. By
stokes theorem, the term involving $\partial_{\ell}$ integrates
to zero---by assumption, the source is isolated, so the integral
can be converted to a surface integral over a surface
\emph{outside} the source, where $T^{\mu\nu}=0$. Hence
\begin{equation}
\begin{split}
\bar{h}_{ij}(\mathbf{x},t)
&=\frac{2G}{R}\int\partial^{2}_{t}(y^{i}y^{j}T^{tt})\,\D^{3}y\\
&=\frac{2G}{R}\frac{\D^{2}}{\D t^{2}}\int y^{i}y^{j}T^{tt}(\mathbf{y},t-|\mathbf{y}-\mathbf{x}|)\,\D^{3}y.
\end{split}
\end{equation}
The integral is the quadrupole moment; thus, the metric
perturbation goes as the second time derivative of the quadrupole
moment.

For an isolated system of a few gravitating bodies (say, stars)
with masses of order $m$ and velocities of order $v$, the
quadrupole moment is $\sim mr^{2}$, and thus $\bar{h}\sim
Gmv^{2}/R$. Furthermore, if the system is gravitationally bound,
$v^{2}\sim Gm/r$, so $\bar{h}\sim v^{4}r/R$.

For a typical binary neutron star, $r\sim 10^{7}$ km and
$v^{2}\sim10^{-7}$; for such a system at a distance of a
kiloparsec, this gives $\bar{h}\sim10^{-21}$.
\end{Boxed}


Nevertheless, in the next five years we will detect these
things. LIFO has a photon running around in a pipe with length
$L\sim 10^{3}\,\mathrm{m}$ about $10^{3}$ times which is
effectively $L_{eff}\sim 10^{9}\,\mathrm{m}$. So there would be
constructive or destructive interference.

If we are lucky, we'll see results in a year (in 2010); huge
upgrades are due in 2009. Interesting quantum effects decreasing
uncertainty in error wavelength; we are nearly saturating
uncertainty at this point.

There is something of note: the weak field approximation of the
Einstein tensor gives us
\begin{equation}
G_{\mu\nu} = \frac{-1}{2}\Box\bar{h}_{\mu\nu}+(\partial
h)(\partial h)
\end{equation}
where 
\begin{equation}
(\partial
h)(\partial h)
= \begin{pmatrix}\mbox{self-contribution}\\\mbox{term}
\end{pmatrix} =: t^{\text{(grav)}}_{\mu\nu}.
\end{equation}
It's not really a tensor! The energy carried off by the radiation
can be found by identifying power $\sim t^{0i}$. So the total
power radiated is 
\begin{equation}
P\sim t^{0i}R^{2}. 
\end{equation}
Remember that
\begin{equation}
h\sim \frac{G}{R}\ddot{I}
\end{equation}
thus
\begin{equation}
\begin{aligned}
t&\sim\frac{1}{G}\dot{h}^{2}\\
&\sim G(\dddot{I}/R)^{2}.
\end{aligned}
\end{equation}
Hence the power looks like
\begin{equation}
P\sim R^{2}t\sim G\dddot{I}^{2}.
\end{equation}
Remember
\begin{equation}
I\sim mr^{2},\quad\mbox{so}\quad
\dddot{I}\sim mva
\end{equation}
and by Newton's Laws
\begin{equation}
\dddot{I}\sim mv^{3}/r.
\end{equation}
Then the power looks like
\begin{equation}
\begin{aligned}
P & \sim G(mv^{3}/r)^{2}=Gm^{2}v^{6}/r^{2}\\
&\sim mv^{8}/r
\end{aligned}
\end{equation}
where the last manipulation is again by Newton's Laws. Remember
we set $c=1$, so the power is really small.


\begin{exercises}
\begin{xca}[Detecting gravitational radiation I]
We considered a coordinate system (``gauge'') for weak fields in
which $\partial_{\mu}\bar{h}^{\mu\nu}=0$. In a region in which the
  stress-energy tensor $T_{\mu\nu}$ is zero, we can make a
  further coordinate transformation such that $h_{0\mu}=0$ and
  $h=\eta^{\mu\nu}h_{\mu\nu}=0$. In such a coordinate system, a
  gravitational plane wave moving along the $z$ axis has a metric
  (see, e.g., Carroll~\cite{Carroll:2004st} section 7.4) 
\begin{equation}\label{eq:ex1:metricWeWannaFind}
g_{\mu\nu}=\eta_{\mu\nu}+C_{\mu\nu}\cos\bigl(\omega(t-z)\bigr),
\quad\mbox{with}\quad
C_{\mu\nu} = \begin{pmatrix} 0 & 0 & 0 & 0\\
0 & h_{+} & h_{\times} & 0\\
0 & h_{\times} & -h_{+} & 0\\
0 &0 & 0 & 0
\end{pmatrix}
\end{equation}
where $h_{+}$ and $h_{\times}$ are constants.
\begin{enumerate}
\item Under a rotation in the $x$-$y$ plane, the coordinates transform as
\begin{equation}
\begin{aligned}
x &= \bar{x}\cos\theta + \bar{y}\sin\theta\\
y &= \bar{y}\cos\theta - \bar{x}\sin\theta
\end{aligned}
\end{equation}
Find the transformations for $h_{+}$ and $h_{\times}$. For what
angle are the two polarizations interchanged? 
\item Consider a mass initially located at position $(x_{0}, 0,
  0)$ with vanishing initial velocity, $\D x^i /\D s = 0$. Find
  the geodesic equation for this object, with the metric
  \eqref{eq:ex1:metricWeWannaFind}, to first order in $h$. Show
  that the object will remain at rest at $(x_{0}, 0, 0)$. (Note
  that ``at rest'' is a coordinate-dependent statement. That's
  OK, though: for this problem, the gauge conditions have
  implicitly determined a unique coordinate system.) 
\item\label{ex1:partC:lot8} Consider two mirrors, at rest along the $x$ axis at $(0, 0,
  0)$ and $(L, 0, 0)$. Using the metric
  \eqref{eq:ex1:metricWeWannaFind}, compute the round-trip time
  $\Delta t$ for a light pulse starting at the origin at time
  $t_{0}$, moving along the $x$ axis, reflecting from the mirror
  at $x = L$, and returning to the origin. For this computation
  you can assume that $\Delta t\ll1/\omega$, so the quantity
  $\omega{t}$ can be treated as a constant. Your answer should
  depend on $t_{0}$; if it doesn't, you've made a mistake. 
\item For flat spacetime, the round-trip time $\Delta t$ of part
  \ref{ex1:partC:lot8} is $2L$. The effect of the gravitational
  wave is the same as if the light traveled a slightly different
  distance $2L +\Delta L$. For a ``strain'' of $h\sim 10^{-21}$
  (a reasonable estimate for astrophysical sources of
  gravitational radiation) and a mirror separation $L\sim 4\,{\rm
    km}$ (the length of an arm of the LIGO detector), estimate
  the maximum value of $\Delta{L}$. Compare this to the size of a
  typical atomic nucleus of about $1\,{\rm fm}$. Tiny as it is,
  this change in distance should be detectable in an
  interferometer! 
\end{enumerate}
\end{xca}
\begin{xca}[Detecting gravitational radiation II]
Another way to construct a gravitational wave detector is to use a metal bar isolated from external sources of noise. When a gravitational wave passes, the two ends of the bar will accelerate at different rates, setting up oscillations. The relative acceleration of the two ends is determined by the equation of geodesic deviation. The deviation vector $X$ can be interpreted as the distance between the two ends of the rod---since spacetime
is assumed to be nearly flat, it makes sense to talk about Cartesian coordinates and
distances. In this problem, we will model the bar of metal by two masses at the ends of
a spring.

A weak gravitational wave is given by the metric of problem 2. Consider two equal masses on a spring in the $x$-$y$ plane, initially separated by a distance L, so
\begin{equation}
X_{0} = X(t = 0) \approx (L \cos \theta, L sin \theta, 0)
\end{equation}
Say that the spring has natural frequency $\omega_{0}$, that is,
that it exerts a restoring acceleration $a = \omega_{0}^{2} (X -
X_{0})$ when the ends are displaced from their initial
positions. We shall look at the effect of a gravitational wave as
a driving force for this oscillator (ignoring its tendency to
rotate the spring). 

Start again with a gravitational wave moving along the $z$ axis, 
with polarization $h_{\times}$.
Since the wave is assumed to be weak, we can write
\begin{equation}
X(t) = X_{0} + \zeta(t)
\end{equation}
with $\zeta$ small, and work to lowest order.
\begin{enumerate}
\item From the geodesic deviation equation, find the component of
  gravitational acceleration along the direction of the spring in
  terms of $h_{\times}$, $k$, and---to lowest order---$X_{0}$. 
\item Solve the equations of motion for $\zeta$ subject to this
  acceleration and the restoring force of the spring. (Strictly
  speaking, by $\zeta$ I mean here ``the component of $\zeta$ in
  the direction of the spring,'' since we're ignoring rotation of
  the spring in the $x$-$y$ plane.) Explain the dependence on
  $\theta$, the angle at which the bar lies in the $x$-$y$
  plane. 
\end{enumerate}
\end{xca}
\end{exercises}

\lecture
%%
%% lecture20.tex
%% 
%% Made by alex
%% Login   <alex@tomato>
%% 
%% Started on  Wed Jul 11 10:23:24 2012 alex
%% Last update Wed Jul 11 10:23:24 2012 alex
%%

The weakfield equations' advantage: reduces Einstein's equations
to linear, uncoupled differential equations. But the problem is
it doesn't tell us everything. On the other hand, the full
Einstein equation has for each component some 50000 terms if we
don't use summation. If we use symmetry, we can reduce the number
of terms.

We first look at a static (time independent) and spherically
symmetric solution. (In general, if we assume ``homogeneous'', then
we cannot assume ``static''.) 

The technical way to deal with time independence is to say there
exists a timelike vector $\zeta^{\mu}$ such that the transformation
\begin{equation}
x^{\mu}\to x^{\mu}+\zeta^{\mu}
\end{equation}
doesn't change the metric. We saw in Equation \eqref{eq:lec16:killingVec}
the metric changes under transformations of this sort as
\begin{equation}
g_{\mu\nu}\to
g_{\mu\nu}+\nabla_{\mu}\zeta_{\nu}+\nabla_{\nu}\zeta_{\mu}.
\end{equation}
We have
\begin{equation}
\nabla_{\mu}\zeta_{\nu}+\nabla_{\nu}\zeta_{\mu}=0
\end{equation}
the Killing equation, and $\zeta^{\mu}$ is the Killing vector. We
showed in Exercise \ref{xca:prob4:killing} that
\begin{equation}
\nabla_{\mu}\zeta_{\nu}+\nabla_{\nu}\zeta_{\mu}=
g_{\mu\rho}\partial_{\nu}\zeta^{\rho}
+g_{\nu\rho}\partial_{\mu}\zeta^{\rho}
+\zeta^{\rho}\partial_{\rho}g_{\mu\nu}.
\end{equation}
We just take coordinates $\zeta^{t}=1$, and $\zeta^{i}=0$ (i.e.,
rescale the time components). Then we have our Killing equation
reduce to
\begin{equation*}\tag{\mbox{stationary metric}}
\zeta^{t}\partial_{t}g_{\mu\nu}=\partial_{t}g_{\mu\nu}=0
\end{equation*}
which is a coordinate independent expression telling us the
metric is time independent. There is a subtlety here: a rotating
object doesn't appear to change. We introduce another condition,
a new symmetry as $t\to-t$. A stationary metric which satisfies
is said to be \define{Static}. In these coordinates, this is
equivalent to 
\begin{equation}
g_{it}=0.
\end{equation}
So
\begin{equation}
\D s^{2}=g_{tt}\,\D t^{2}+g_{ij}\,\D x^{i}\,\D x^{j}
\end{equation}
(If we cannot eliminate $g_{it}$, it's an indicator of a moving system.)

We will examine the static, spherically symmetric metric. The
metric shouldn't ``change'' when moving along an entire loop on a
2-sphere; in some sense there is an invariance. The angular
dependence is just
\begin{equation*}
\D\theta^{2}+\sin^{2}(\theta)\,\D\varphi^{2}
\end{equation*}
so
\begin{equation}
\D s^{2} = g_{tt}\,\D t^{2} - R^{2}(\D\theta^{2}+\sin^{2}(\theta)\,\D\varphi^{2})
-g_{rr}\,\D r^{2}
\end{equation}
if we think of the spacetime as foliated spheres, the $r$
determines which spherical disc we're on. We know that $g_{tt}$,
$R$, $g_{rr}$ depend on $r$ but not on the angles or we wouldn't
have spherical symmetry, nor does it have a dependency on $t$.
\begin{rmk}
See Wald~\cite{Wald:1984rg} for a good discussion of
$\mathrm{SO}(3)$ symmetry in General Relativity.
\end{rmk}
We still have one coordinate degree of freedom---$r$. We can still
choose many different coordinate systems.

If we fix $r$, we can do it several different ways. The laziest
way is to choose $r$ satisfying
\begin{equation}
g_{rr}=1
\end{equation}
which happens when $r$ is the proper distance. We can also choose
$r$ to satisfy instead
\begin{equation}
g_{rr} = R^{2}/r^{2}
\end{equation}
which then gives us
\begin{equation}
\D s^{2} = g_{tt}\,\D t^{2} - g_{rr}\underbrace{\bigl(\D r^{2} + r^{2}(\D\theta^{2}+\sin^{2}\theta\,\D\varphi^{2})\bigr)}_{\text{usual flat metric}}.
\end{equation}
With this choice we have isotropic coordinates. Both of these
make the field equations a wee bit complicated. But there is a
third choice! We fix
\begin{equation}
R=r
\end{equation}
which are ``areal coordinates'' describing a 2-sphere at $r$ with
area $4\pi r^{2}$.

For the Schwarzschild solution, we choose areal
coordinates. (Originally Schwarzschild chose coordinates where
$\det(g)=1$.) We then have
\begin{equation}
\D s^{2} = A(r)\,\D t^{2}-B(r)\,\D
r^{2}-r^{2}(\D\theta^{2}+\sin^{2}\theta\,\D\varphi^{2})
\end{equation}
In the Einstein vacuum equation we have
\begin{equation}
A = B^{-1} = 1-\frac{2m}{r}
\end{equation}
But really by integration, the constant term (the ``1'') is an
integration constant.

What if $r\approx 2m$? It gets mildly interesting. At $r=2m$,
something goes horribly awry since $A\to0$ but $B\to\infty$, and
$\D s^{2}\to\textbf{??}$ This was not understood for a
longtime. Back in the 1920s, Panlieve et al.~wrote papers with
novel coordinate systems but this was largely ignored. 
Is this singularity from poor choice of coordinates, or from
something deep and not easily understandable in nature?

Lets examine as an example
\begin{equation}
\begin{aligned}
\D s^{2}&= \D x^{2}+\D y^{2}, &\quad &\mbox{let}\quad
x=\frac{1}{u-1}\\
&=\frac{\D u^{2}}{(u-1)^{2}}+\D y^{2}
\end{aligned}
\end{equation}
at $u=1$ we have a singularity! 

We can try to look at coordinate independent quantities as a
first step. For example
\begin{subequations}
\begin{align}
R &= 0\\
R_{\mu\nu}R^{\mu\nu} &=0.
\end{align}
but
\begin{equation}
R_{\mu\nu\rho\sigma}R^{\mu\nu\rho\sigma}=\frac{48m^{2}}{r^{6}}.
\end{equation}
\end{subequations}
At $r=2m$, nothing scary happens! It turns out every scalar we
can form from the curvature behaves unsuspiciously at
$r=2m$. Physically, there doesn't appear locally anything new and
scary. 

On the other hand, for $r<2m$, the temporal component and radial
component switch. That is
\begin{equation}
A(r)<0,\quad\mbox{and}\quad B(r)<0
\end{equation}
so spacelike becomes timelike, and timelike becomes spacelike.

\lecture
%%
%% lecture21.tex
%% 
%% Made by alex
%% Login   <alex@tomato>
%% 
%% Started on  Thu Jul 12 10:40:45 2012 alex
%% Last update Thu Jul 12 10:40:45 2012 alex
%%

We have the Schwarzschild metric
\begin{equation}
\D s^{2} = \left(1-\frac{2m}{r}\right)\D t^{2}
-\left(1-\frac{2m}{r}\right)^{-1}\D r^{2}
-r^{2}\,\D\Omega^{2}
\end{equation}
where
\begin{equation}
\D\Omega^{2} = \D\theta^{2}+\sin^{2}\theta\D\varphi^{2}
\end{equation}
is the usual notation for the metric on $(n-2)$-sphere. 

\begin{thm}[Birkhoff]
Spherically symmetric vacuum field equations imply the
Schwarzschild solution.
\end{thm}

Physically if we have a spherically symmetric field, we can treat
it as concentrated at a point, all gravitation waves have to be
spherical, but they're really quadrapole or higher order.

We chose $t$ by demanding a time variable such that everything's
independent of it. Our solution is still perfectly good. We would
like some physical time component with some physical meaning.

We have an observer shooting off light to the cylinder of
constant radius. We could equally make this baseballs instead of
photons, which is useful for collapsing spherical shells. The
diagram is
\begin{center}
\includegraphics{img/lecture21.0}
\end{center}
We have light (null geodesics) and it's only radial (so we have
$\D\Omega^{2}=0$). Then we have 
\begin{equation}
\D s^{2} = 0 = \left(1-\frac{2m}{r}\right)\D t^{2}
-\left(1-\frac{2m}{r}\right)^{-1}\D r^{2}
\end{equation}
which implies
\begin{equation}
\pm\left(1-\frac{2m}{r}\right)^{-1}\D r = \D t.
\end{equation}
We introduce a coordinate $r_{*}$ such that
\begin{equation}
\D t= \pm\D r_{*}
\end{equation}
so
\begin{equation}
r_{*} = r + 2m\ln\left|\frac{r}{2m}-1\right|.
\end{equation}
Either $t-r_{*}=u$ or $t+r_{*}=v$ where $u$, $v$ are constants
and the same as geodesics $v$-labeling.

We\marginpar{advanced Eddington--Finkelstein Coordinates} eliminate $t$ from the metric, so we get
\begin{equation}
\D s^{2} = \left(1-\frac{2m}{r}\right)\D v^{2}-2\,\D v\,\D
r-r^{2}\,\D\Omega^{2}
\end{equation}
This is the same metric expressed in different coordinates. They
are called the \define{advanced Eddington--Finkelstein Coordinates}.
If we used $u$ instead of $v$, we'd get \emph{retarded}
Eddington--Finkelstein coordinates.

The coordinates with $v$ yields a bit of information. The null
radial geodesics satisfy
\begin{equation}
\left(1-\frac{2m}{r}\right)\D v^{2}=2\,\D r\,\D v
\end{equation}
The solutions are either
\begin{equation}
v=\mbox{const.},\quad\mbox{or}\qquad
\frac{\D r}{\D v}=\frac{1}{2}\left(1-\frac{2m}{r}\right).
\end{equation}
Outgoing geodesics asymptotically approach $r=2m$. One thing to
note is that $r=2m$ is a null geodesic (i.e., it's lightlike)!

\begin{defn}
A \define{Killing Horizon} is when a Killing vector changes from
timelike to lightlike.
\end{defn}

The\marginpar{Kruskal--Szekeres Coordinates} next thing to do is
try replacing $r$ with $u$. It's easier to first define
\begin{equation}
U=\exp(-u/4m),\quad\mbox{and}\quad V=\exp(v/4m).
\end{equation}
We find (plugging these back into the Schwarzschild solution, we
have
\begin{equation}
\D s^{2}=\frac{32m^{3}}{r}\exp(-r/2m)\,\D U\,\D V -r^{2}\,\D\Omega^{2}
\end{equation}
where $r=r(U,V)$ is defined by
\begin{equation}
\left(\frac{r}{2m}-1\right)\exp(r/2m)=-UV,\quad\mbox{and}
\quad\frac{U}{V}=\exp(-t/2m).
\end{equation}
These coordinates are called \define{Kruskal--Szekeres Coordinates}.
One of the nice things about these coordinates: nothing in
particular goes horribly awry when the metric goes to zero,
everything's nicely behaved.

\begin{wrapfigure}{r}{13pc}
\vspace{-1pc}
\includegraphics{img/lecture21.2}
\vspace{-1pc}
\end{wrapfigure}
Consider $r=2m$, in our new coordinates this is
\begin{equation}
UV=0.
\end{equation}
Our event horizon has two solutions
\begin{equation}
U=0,\quad\mbox{or}\quad V=0.
\end{equation}
These are null geodesics, so $U=0$, $V=0$ gives two lines are
$45^{\circ}$ angles.

When $r=0$, we have $UV=1$. This is a hyperboloid. The
hyperboloid is drawn to the right with dashed lines to denote a
genuine singularity (usually, it's with a ``squiggly'' line).
We also have the situation when $r$ is a constant and $r>2m$;
then $UV<0$ is also constant. Conversely, when $r<2m$ is a
constant, we have $UV>0$ be constant. These situations are drawn
in red and blue to the right.


\begin{wrapfigure}{l}{11pc}
\vspace{-1pc}
\includegraphics{img/lecture21.3}
\vspace{-1pc}
\end{wrapfigure}
We have 4 regions labeled as shown on the left. Regions I and IV
are outside of the black hole. Regions II and III are inside of
the black hole. If we enter these regions, we necessarily hit the
singularity (we'd need to travel faster than light to escape the
region). Note that a ``white hole'' is just a time-reversed black
hole. The regions relevant for black holes are I and II, whereas
I and III are relevant for white holes.

We see black holes but not white holes. Why? Well, we're working
with $T^{\mu\nu}=0$. We're working with matter collapsing, all we
really have for the vacuum is part of region I and part of region II.

We think of white hole/black hole as an eternal black hole
perhaps formed by early fluctuations of the young universe (Hsu
suggests something along these lines~\cite{Hsu:2010vp}), perhaps
this is a wrong intuition. 

There is no solution of the vacuum with an isometry which takes
region III into any other region. Presumably the white holes
radiate away.

Now\marginpar{Penrose Diagrams} we had
\begin{equation}
\D s^{2} = (\dots)\D U\,\D V - r^{2}\,\D\Omega^{2}.
\end{equation}
If we multiply by a function, it doesn't change null
geodesics. Penrose invented a trick to make $r=\infty$ into a
finite distance, a doodle called a \define{Penrose Diagram}. For
the Schwarzschild metric, we have the diagram:
\begin{center}
\includegraphics{img/lecture21.4}
\end{center}
This distorts area but preserves the causal
structure\footnote{This is because the ``causal structure'' is
  determined by the angles between intersecting curves; it's a
  conformal transformation.}. 

\begin{exercises}
\begin{xca}[Black holes and trapped surfaces]
The Schwarzschild metric in Kruskal-Szekeres coordinates is
\begin{equation}
\D s^{2} =
\frac{32m^{2}}{r}\E^{-r/2m}
\bigl(-\D T^{2}+\D X^{2}\bigr)
 + r^2 \,\D\Omega^2
\end{equation}
where $r$ is viewed as a function of $X$ and $T$: 
\[
\left(\frac{r}{2m}-1\right)\E^{r/2m}=X^{2}-T^{2}
\]
\noindent\textbf{a.\quad}\ignorespaces Show that radial null
geodesics emitted from the two-sphere $(T_{0}, X_{0}, \theta,
\varphi)$ are described by the equation of motion
\begin{equation}
X - X_{0} = \epsilon(T - T_{0}),\quad
\theta=\mbox{const.},\quad\varphi=\mbox{const.}
\end{equation}
where $\epsilon=1$ for outgoing geodesics and $\epsilon=-1$ for
ingoing geodesics.

\noindent\textbf{b.\quad}\ignorespaces Consider the new
two-sphere formed by the wave front at time $T$ of these radial
geodesics. Show that the area of this sphere is $A =
4\pi{r^{2}(X, T )}$. (Hint: this is not completely obvious; you
need to think about how area is defined in a curved spacetime.)

\noindent\textbf{c.\quad}\ignorespaces In region I, $X_{0}>0$ and
$-X_{0}<T_{0}<X_{0}$. By considering $\D{A}/\D{T}$, show that the
the area $A$ increases with $T$ for outgoing geodesics, and
decreases for ingoing geodesics. 

\noindent\textbf{d.\quad}\ignorespaces In region II (inside the
event horizon), $T_{0}>0$ and $-T_{0}<X_{0}<T_{0}$. Show that in
this region, $A$ decreases with $T$ for both ingoing and outgoing
geodesics. This is the condition that the initial sphere $(T_{0},
X_{0}, \theta, \varphi)$ is a trapped surface. 
\end{xca}
\end{exercises}



\lecture
%%
%% lecture22.tex
%% 
%% Made by alex
%% Login   <alex@tomato>
%% 
%% Started on  Sat Dec 31 12:00:40 2011 alex
%% Last update Sat Dec 31 12:00:40 2011 alex
%%

Let $\Sigma$ be a 2-dimensional manifold\index{Manifold}. Let
$\widetilde{\Sigma}$ be the universal cover of $\Sigma$. But
there are only two choices for $\widetilde{\Sigma}$: $S^2$ or
$\RR^2$. For simplicity we will suppose that $\Sigma$ is
compact. We know how to calculate $\pi_{1}(\Sigma)$, the only
thing we need is the statement that $\pi_{1}(\Sigma)$ is finite
in two cases: $\Sigma=S^2$ or $\RP^2$. This is easy looking at
the Abelianization of the fundamental group. In both of these
cases, $\widetilde{\Sigma}=S^2$. We have
\begin{equation}
\pi_{k}(\Sigma)=\pi_{k}(\widetilde{\Sigma})
\end{equation}
for $k\geq2$. Now let us suppose $\pi_{1}(\Sigma)$ is
infinite. Then $\widetilde{\Sigma}$ is not compact. Why? Because
when we look at the covering
\begin{equation*}
\widetilde{\Sigma}\to\Sigma
\end{equation*}
the number of sheets in this covering are the number of elements
in $\pi_{1}(\Sigma)$, which is infinite. Over every disc, we have
an infinite number of discs, which is definitely noncompact. We
have only one chocie for $\widetilde{\Sigma}$. We see then that
\begin{equation}
\pi_{k}(\Sigma)=\pi_{k}(\RR^2)=0
\end{equation}
for $k\geq2$.

We would like to show 
\begin{equation}
\pi_{k}(X\times Y)\iso\pi_{k}(X)\times\pi_{k}(Y).
\end{equation}
It's a one minute proof. If we have a mapping
\begin{equation}
f\colon Z\to X\times Y=\{(x,y)\}
\end{equation}
this map means $(x,y)=f(z)$. This means
\begin{equation}
x=f_{1}(z),\quad\mbox{and}\quad y=f_{2}(z).
\end{equation}
When we apply this to spheroids, everything follows. When we
deform $f(z)$, we deform these two guys. We have, e.g., an
$n$-torus be
\begin{equation}
T^n=(S^1)^n
\end{equation}
so $\pi_{k}(T^n)\iso\pi_{k}(S^1)^n$.
 
\subsection{Relative Homotopy Groups}\index{Homotopy Group!Relative}
\index{Relative Homotopy Group}\index{Reduced Homotopy Group|see{Relative Homotopy Group}}%
\index{Relative Homotopy Group!Construction with Spheroids|(}
We will have a pair of topological spaces $X$, $A$ (so
$A\propersubset X$), and consider $*\in A\propersubset X$. For
simplicity, $A$ and $X$ are connected. We will define a
\define{Relative Homotopy Group} $\pi_{n}(X,A,*)$ or sometimes
$\pi_{n}(X,A)$. We will neglect something in $X$, namely, we
neglect $A$ --- this is the notion of ``relative''.

Recall we defined $\pi_n$ by means of spheroids
\begin{equation*}
(S^n,*)\to(X,*).
\end{equation*}
The homotopy group $\pi_{n}(X,*)$ is then the homotopy classes of
spheroids. This is nice but incomplete. We need to define an
operation. We did this by considering a
spheroid\index{Spheroid!as Map on Cube} as a map on a
cube, generalizing
concatenation\index{Concatenation!Generalization of ---}.

Lets consider something similar for relative homotopy groups. We
introduce \define{Relative Spheroids}\index{Spheroid!Relative}\index{Relative Spheroid} %
$(\bar{D}^n,S^{n-1},*)$ where 
\begin{equation}
S^{n-1}=\partial\bar{D}^{n},
\end{equation}
and relative spheroid is a map
\begin{equation}
(\bar{D}^n,S^{n-1},*)\to(X,A,*).
\end{equation}
This means we have a map
\begin{equation}
f\colon\bar{D}^{n}\to X
\end{equation}
such that
\begin{equation}
f\colon\partial\bar{D}^{n}\to A
\end{equation}
but $f(*)=*$. Such a map is a relative spheroid.

The relative homotopy group $\pi_{n}(X,A,*)$\index{Relative Homotopy Group!in Terms of Relative Spheroids}
is a set of homotopy classes of relative spheroids. It's exactly
the same for ordinary homotopy group, the only difference is we
use relative spheroids. We will discuss the operation later on.

One relation we'd like to note is we have a map
\begin{equation}
\pi_{n}(X,A,*)\to\pi_{n-1}(A,*)
\end{equation}
from the relative homotopy group to the full homotopy group. Why?
For a trivial reason that a relative spheroid
\begin{equation}
f\colon(\bar{D}^{n},S^{n-1},*)\to(X,A,*)
\end{equation}
is really a full spheroid on $A$. Thus $f\colon(S^{n-1},*)\to(A,*)$
is an $(n-1)$-spheroid.
\index{Relative Homotopy Group!Construction with Spheroids|)}

A different formulation of the relative homotopy group.
Recall we considered the space $\Omega$ of all closed loops
starting and ending at $*$. We consider 
\begin{equation*}
\pi_{n}(X,*)=\pi_{n-1}(\Omega,*)
\end{equation*}

\begin{wrapfigure}{r}{1.5in}
  \centering
  \includegraphics{img/lecture22.0}
\end{wrapfigure}\noindent\ignorespaces %
Now what about the relative groups? We have $A\propersubset X$,
consider all paths that are closed modulo $A$. That is to say
\begin{equation}
\Omega(A)=\{f\colon I\to X\mid f(0)=*, f(1)\in A\}
\end{equation}
This sort of path is doodled on the right, where it begins at the
marked point and ends anywhere inside the gray region.
We may consider its homotopy groups.
Thus we may define $\pi_{n}(X,A,*)=\pi_{n-1}(\Omega(A),*)$ where
$*(t)=*$ is the stationary path. Okay, this is a definition. We
see this is a group. Also, for $n\geq3$ we see $\pi_{n}(X,A,*)$
is Abelian. The only problem is that this is not a very good
definition. 

Let us decode this definition. Consider
$\pi_{n-1}\bigl(\Omega(A),*\bigr)$. What is this? We define this
in terms of spheroids as a map of a cube
\begin{equation}
f\colon I^{n-1}\to\Omega(A)
\end{equation}
which sends $\partial I^{n-1}\to*$. What is $\Omega(A)$? It
consists of paths. Our function $f(t_1,\dots,t_{n-1})$ itself is
a path, so really
\begin{equation}
f=f(t_1,\dots,t_{n-1},\tau)\in X.
\end{equation}
What ar ethe conditions on $f$? First of all, the condition is
\begin{equation}
f_{\tau}\colon\partial I^{n-1}\to *(\tau),
\end{equation}
so
\begin{equation}
f(\partial I^{n-1},\tau)=*(\tau).
\end{equation}
Another is that, for $\tau=0$, we have
\begin{equation}
f(\dots,0)=*
\end{equation}
be our marked point, whereas for $\tau=1$ we require
\begin{equation}
f(\dots,1)\in A
\end{equation}
and that's it!

\begin{wrapfigure}{r}{1.5in}
  \vspace{-30pt}
  \centering
  \includegraphics{img/lecture22.1}
  \vspace{-24pt}
\end{wrapfigure}
We will try to reconcile everything. Consider $f\colon I^n\to X$.
For $n=2$, we will have a square and when $\tau=1$ we go to
$A$. This is doodled on the right. When $t=0,1$ we go to $*$ and
when $\tau=0$ we also go to $*$.

\marginpar{This needs to be rewritten for clarity} 
\begin{wrapfigure}{l}{0.75in}
  \vspace{-12pt}
  \centering
  \includegraphics{img/lecture22.2}
  \vspace{-12pt}
\end{wrapfigure}
For the $n=3$ case, we have the top face go to $A$ (it is shaded
grey). This is really what is defined by Hatcher as a relative
spheroid.
But this is already defined. We should prove it is the same.
Here we have the map of a ball, which sends its boundary to
$A$. But here we have some extra stuff, namely, the rest of the
boundary. We identify it witha pouint. So really, this is
equvialent to a ball with a marked point.
Consider
\begin{equation}
\left(I^n,I^{n-1}\times\{1\},(\partial I^{n-1}\times I)\cup(I^{n-1}\times\{0\})\right)\mapsto(\bar{D}^{n},S^{n-1},*)
\end{equation}
since they both are mapped to $(X,A,*)$.
So this identifies the notion of a relative spheroid with
Hatcher's notion of a spheroid. Observe
\begin{equation*}
(\partial I^{n-1}\times I)\cup(I^{n-1}\times\{0\})
\end{equation*}
is a contractible set. When we contract, the upperface is mapped
to $S^{n-1}$ the whole boundary. So our new sense of relative
spheroid agrees with the relative spheroid in the old sense.

\lecture[Applications of $K$-Theory]
%%
%% lecture23.tex
%% 
%% Made by alex
%% Login   <alex@tomato>
%% 
%% Started on  Sat Dec 31 12:01:46 2011 alex
%% Last update Sat Dec 31 12:01:46 2011 alex
%%

We gave a definition of homotopy groups $\pi_{k}(X,*)$ in various
ways. Using spheroids
\begin{equation}
f\colon(S^k,*)\to(X,*)
\end{equation}
such that $f(*)=*$, or as
\begin{equation}
f\colon I^k\to (X,*)
\end{equation}
such that $f(\partial I^k)=*$. We introduced the notion of
relative homotopy groups on $(X,A,*)$. We can have relative
spheroids
\begin{equation}
f\colon\bar{D}^{n}\to X
\end{equation}
such that
\begin{equation}
f(S^{n-1})\subset A,\quad\mbox{and}\quad f(*)=*.
\end{equation}
This is very nice but it doesn't give a group
structure. Therefore one can consider instead of maps of a ball,
well, maps of a cube. That would be
\begin{equation}
f\colon I^n\to X
\end{equation}
such that
\begin{equation}
f(I^{n-1})\propersubset A,\quad\mbox{and}\quad
f(\partial I^n-I^{n-1})=*.
\end{equation}
This is an equivalent picture, but we see how to form a binary
operation now. This definition may now be formulated as
$\pi_{n-1}\bigl(\Omega(A)\bigr)$.

The first thing to say is this definition is functorial\index{Functoriality}. 
What does this mean? Well, we have 
\begin{equation}
\alpha\colon(X,A,*)\to(Y,B,*)
\end{equation}
such that
\begin{equation}
\alpha\colon X\to Y\quad\mbox{and}\quad \alpha(A)\propersubset B
\end{equation}
and $\alpha(*)=*$. We then have ``by functoriality'' a map
\begin{equation}
\pi_{n}\bigl(\alpha\colon(X,A,*)\to(Y,B,*)\bigr)\quad=\quad \alpha_{*}\colon
\pi_{n}(X,A,*)\to\pi_{n}(Y,B,*).
\end{equation}
Moreover $\alpha_{*}$ is a morphism. We have the functoriality
property that 
\begin{equation}
(\alpha\circ\beta)_{*}=\alpha_{*}\circ\beta_{*}\quad\mbox{and}\quad
(\id{})_{*}=\id{*}.
\end{equation}
These are the functorial properties.

\subsection{Exact Homotopy Sequence of a Pair}
\index{Exact Sequence!of Homotopy Groups}
\index{Homotopy Group!Exact Sequence of ---}

We see that a spherodi in $A$ is definitely a spheroid in $X$. In
other words, functoriality acts on this inclusion
\begin{equation}
i\colon A\into X
\end{equation}
and gives us a morphism
\begin{equation}
i_{*}\colon\pi_{n}(A)\to\pi_{n}(X).
\end{equation}
Note that we do abuse notation slightly, we should write
something like $i_{*,n}$ to indicate we have $\pi_{n}(i)$, i.e.,
it depends on the $n\in\NN_{0}$.

Next we have absolute homotopy groups, we had absolute spheroids.
Now we may consider relative spheroids and relative homotopy
groups
\begin{equation}
\pi_{n}(A,*)\xrightarrow{i_{*}}\pi_{n}(X,*)\to\pi_{n}(X,A,*).
\end{equation}
Why do we have this? Well, any absolute spheroid is-a relative
spheroid. Why? Because the simple reason is $*\in A$. So a
spheroid is just a relative spheroid ``ending at $*$''.

We have one more map, which we already described
\begin{equation}
\pi_{n}(X,A,*)\to\pi_{n-1}(A,*).
\end{equation}
This comes from the fact a relative spheroid
\begin{equation}
f\colon I^n\to X
\end{equation}
can be restricted to $I^{n-1}$, but $f(I^{n-1})\propersubset
A$. This is a spheroid in $A$! So we induce this morphism.

But we also have
\begin{equation}
i_{*}\colon\pi_{n-1}(A,*)\to\pi_{n-1}(X,*).
\end{equation}
Why? Well, we saw this earlier using the inclusion and applying
the functor $\pi_{n-1}$. So we have a sequence. We have at the
end of this
\begin{equation}\label{eq:lec23:stuffNotDefined}
\dots\to\pi_{1}(A,*)\to\pi_{1}(X,*)\to
\underbracket[0.5pt]{\pi_{1}(X,A,*)\to\pi_{0}(A,*)\to\pi_{0}(X,*)}
\end{equation}
where the underlined terms are not really defined. So what to do?
Simple: \emph{define them!} We really have a problem here: we
have no group for $\pi_{0}(X)$, but we have a set. The stuff
underlined in Eq \eqref{eq:lec23:stuffNotDefined} are not groups,
but they are sets. If we consider $\pi_{0}(X)$\index{$\pi_{0}(X)$}, it may be
considered as $S^0\to X$ which maps marked point to marked
point. Please note that $S^0$ consists of two points, one of them
is marked. So what is
\begin{equation}
f\colon (S^0,*)\to(X,*)?
\end{equation}
It is a map of one point, so naively we would expect
$\pi_{0}(X,*)$ to be in one-to-one correspondence with the points
of $X$, right? Wrong: $\pi_{0}$ is the \emph{homotopy classes} of
such mappings, and two points are homotopic if they are on the
same components. So really,
\begin{equation}
\pi_{0}(X,*)=\begin{pmatrix}\mbox{components}\\
\mbox{of $X$}
\end{pmatrix}
\end{equation}
A deformation of elements of $\hom\bigl((S^0,*),(X,*)\bigr)$ is a
path. So we have really $\pi_{0}(X,*)$ be homotopy classes of
these thngs, i.e., of the components of $X$. It is a set.

A relative spheroid
\begin{equation}
f\colon I\to(X,A,*)
\end{equation}
is a path
\begin{equation}
\gamma\colon[0,1]\to X
\end{equation}
such that
\begin{equation}
\gamma(0)=*,\quad\mbox{and}\quad
\gamma(1)\in A.
\end{equation}
So a spheroid is determined where it ends. What is important is
we have a sequence of groups that ends with a sequence of groups
that ends with a sequence of sets.

\begin{thm}
This sequence
\begin{equation}
\dots\to\pi_{n}(A,*)\to\pi_{n}(X,*)\to\pi_{n}(X,A,*)\to\pi_{n-1}(A,*)\to\dots
\end{equation}
is exact.
\end{thm}
\begin{defn}\index{Sequence!Exact|(}
Consider a sequence of groups
\begin{equation}
\dots\xrightarrow{\D_n}A_{n}\xrightarrow{\D_{n-1}}A_{n-1}\xrightarrow{\D_{n-2}}A_{n-2}\to\dots
\end{equation}
it is said to be \define{Exact}\index{Exact Sequence|textbf}
if{}f $\im(\D_{k})=\ker(\D_{k-1})$.
\index{Sequence!Exact|)}
\end{defn}
Observe as a consequence in an exact sequence
\begin{equation}
\im(\D_{k-1}\circ\D_{k})=0.
\end{equation}
Can we say $\im(\D_{k})=\ker(\D_{k-1})$? Well, yes, but the
statement, for any $k$,
\begin{equation}
\im(\D_{k})\subset\ker(\D_{k-1})
\end{equation}
is sufficient to say we have an exact sequence, and
\begin{equation}
\im(\D_{k})\supset\ker(\D_{k-1})
\end{equation}
is necessary. We could write
\begin{equation}
\im(\D_{k})=\D^{-1}_{k-1}(0)
\end{equation}
for exactness conditions.

\begin{rmk}
The definition of exactness remains meaningful if we wok with
sets with marked points. We simply write $*$ for our marked
point, and $\D^{-1}_{k-1}(*)=\im(\D_{k})$, etc.
\end{rmk}

\begin{proof}[Sketch of Proof]
The full proof requires checking 6 terms showing an equality of
sets.\marginpar{\textbf{TODO:} write in the proof}
\end{proof}


\begin{wrapfigure}{l}{0.6in}
  \vspace{-20pt}
  \centering
  \includegraphics{img/lecture23.0}
  \vspace{-12pt}
\end{wrapfigure}
If we have $S^{n-1}\to A$ homotopic to the trivial path, then it
may be extended to a union of nonintersecting $S^{n-1}$, i.e., to
$D^{n}$. Consider $S^{n-1}\times I$, then on the upper part of
the cylinder (if it is homotopic to 0) is $*$, we get what is
doodled on the left.

Consider $\pi_{n-1}(A,*)\to\pi_{n-1}(X,*)$. We can extend the
sphere to a ball. But that means we have
$\pi_{n}(X,A,*)\to\pi_{n-1}(A,*)$, because this extension to a
ball is a relative spheroid. So we have the kernel included in
the image.

The best thing to do is to go home and think about it yourselves:
there are some things so simple, it cannot be explained.

\appendix
\vfill\eject\renewcommand{\leftmark}{References}\phantomsection\addcontentsline{toc}{section}{References}
%%
%% bibliography.tex
%% 
%% Made by Alex Nelson
%% Login   <alex@tomato3>
%% 
%% Started on  Sat May 21 12:27:45 2011 Alex Nelson
%% Last update Sat May 21 13:42:17 2011 Alex Nelson
%%
\begin{thebibliography}{99}
\bibitem{baez} John C.\ Baez, \newblock
``The Octonions.''\newblock
Eprint: \arXiv[math.RA]{math/0105155}, 56 pages.
\bibitem{bourbaki} Nicolas Bourbaki,
\newblock \emph{General Topology: Chapters 1--4}.
\newblock Springer (1998) 452 pages.
\newblock Useful reference for topological groups.
\bibitem{fuchs} Dmitry Fuchs, \newblock
``$K$-Theory and Other Extraordinary Cohomology Theories.''\newblock
Notes, handout.
\bibitem{hatcher2006algebraic} Allen Hatcher,
\newblock\emph{Algebraic Topology}.
\newblock Cambridge University Press (2002).
\newblock Eprint: \url{http://www.math.cornell.edu/~hatcher/AT/ATpage.html}
xii+550 pages.
\bibitem{hatcher2009vector} Allen Hatcher,
\newblock\emph{Vector Bundles \& K-Theory}.
\newblock Eprint:
\url{http://www.math.cornell.edu/~hatcher/VBKT/VBpage.html} version 2.1,
\newblock May (2009) 110 pages.
\bibitem{milnor} John W.\ Milnor and James D.\ Stasheff,\newblock
\emph{Characteristic Classes}.\newblock
Princeton University Press, 1974.
\bibitem{schwarz} Albert Schwarz,
\newblock\emph{Topology for Physicists}.
\newblock Springer--Verlag (2010) 307 pages. 
\bibitem{steenrod} Norman Steenrod,
\newblock \emph{The Topology of Fibre Bundles}
\newblock Princeton Landmarks in Mathematics \textbf{14}.
\newblock Princeton University Press (1999) 224 pages.
\end{thebibliography}
Note that these items are references that I have found useful
while studying algebraic topology. 

\vfill\eject
\printindex
\end{document}
