%%
%% lecture12.tex
%% 
%% Made by alex
%% Login   <alex@tomato>
%% 
%% Started on  Wed Sep 28 14:11:11 2011 alex
%% Last update Wed Sep 28 14:11:11 2011 alex
%%
\index{Section!Obstructions to|(}
Today we will discuss obstruction theory\index{Obstruction Theory}. It is something that
appears in very different situations. Maybe the simplest
statement is when working with a cell complex, suppose $B$ is a
cell complex, construct a section over $B$ step-by-step taking
into account the cellular skeleton\index{Section!Extension and Obstruction|textbf}
decomposition\marginpar{Obstruction theory studies ``obstructions'' to step-by-step construction of sections}
\begin{equation}
B=\bigcup B^{k}
\end{equation}
where $B^{k}$ is the $k$-skeleton. We construct a section over
$B^{k}$, then try to extend it to $B^{k+1}$. Sometimes this is
possible; other times we get an obstruction. The most important
is the first obstruction (others afterwards are more complicated).

We may also take two sections $f$, $g$ and ask if they're
homotopic, i.e.
\begin{equation}
f\homotopic g?
\end{equation}
We may try to classify sections up to
homotopy.\index{Characteristic Class!related to Obstructions} All
characteristic classes may be obtained by obstructions.


\begin{wrapfigure}{r}{1.5in}
  \vspace{-20pt}
  \begin{center}
    \includegraphics{img/lecture12.0}
  \end{center}
  \vspace{-20pt}
\end{wrapfigure}
\marginpar{See \S\S25.6, 35, 38~\cite{steenrod}, or\break Ch.\ 12
  \cite{milnor} for details\break on obstructions}We have a fibre bundle $(E,F,B,p)$ and suppose $B$ is connected
(although that's not important). We have $B^{0}$ consists of
points and over every point we may take any point of the
fibre. This is a map $B^{0}\to E$ which is a section. Then we
would like to go to $B^{1}$, which is a little less trivial. But
if the fibre is connected, we may lift our section to $B^{1}\to E$.
Let us go to the inductive step.
Suppose we have constructed a section over
$B^{k}\xrightarrow{\sigma}E$. Then we can say that we would like
to extend this section to $B^{k+1}$, we may assume that 
\begin{equation}
B^{k+1}\setminus B^{k}=\bigcup\sigma^{k+1}_{i}
\end{equation}
but it would probably be clearer if we do the following trick: if
we have $B^{k+1}$ and we remove small balls in each
$(k+1)$-cells, then we obtain a deformation-retraction to
$B^{k}$. Thus we may extend our section to a section over
\begin{equation}
\sigma\colon B^{k+1}\setminus\bigcup(\mbox{small balls})\to E.
\end{equation}
The next step, look at the small balls. We have over our small
ball $\bar{D}^{k+1}$ the direct product $\bar{D}^{k+1}\times F$,
and it contains $S^{k}\times F$. The section we have is defined
on $S^{k}\times F$, the question is how to extend this to a
section over $\bar{D}^{k+1}$.

Suppose $F$ is simply connected. Then the maps of the sphere
\begin{equation}
\homotopyClass(S^k,F)=\pi_{k}(F),
\end{equation}
since we are working with a spheroid after all! So what do we
get? For each $(k+1)$-cell, $\sigma^{k+1}$, we obtain
$\xi(\sigma^{k})\in\pi_{k}(F_{b})$ where $b$ is the center of the
ball. The homotopy groups $\pi_{k}(F)$ form a local system over
$B$. We may say that $\xi(\sigma^{k+1})$ is a $(k+1)$-dimensional
cochain on $B$ with coefficients in a local system $\pi_{k}(F)$.

\begin{thm}
If $\pi_{k}(F)=0$, then any section over $B^{k}$ can be extended
to a section over $B^{k+1}$.
\end{thm}
This is obvious.
\begin{thm}
Suppose $\pi_{1}(F)=\dots=\pi_{n-1}(F)=0$, and
$\pi_{n}(F)\not=0$. Then we can extend the section to $B^{k}$
(the $k$-dimensional skeleton) without obstructions $\sigma\colon
B^{n}\to F$ and we have an obstruction $\xi(\sigma)$ that is a
$(n+1)$-dimensional cochain with values in $\pi_{n}(F)$ (i.e., in
local system).
\end{thm}
This is the first obstruction. Now what can we say? Two
things. First this cochain is a cocycle, i.e., its coboundary is
zero. Second this cocycle---if it is zero---is no obstruction,
but if this cocycle is \emph{homologous to zero}, then we can
change $\sigma$ in $B^{n}$ in such a way that the obstruction
becomes equal to zero.

So obstructions lie in $H^{n+1}\bigl(B,\pi_{n}(F)\bigr)$. This is
the place where we have obstructions; if this group disappears,
our obstruction disappears. We know that for local coefficient
systems, if $B$ is simply connected, then
$H^{n+1}\bigl(B,\pi_{n}(F)\bigr)=0$. But this is the notion of
obstructions.

\medbreak
\refstepcounter{thm}%
\noindent\textbf{Example \thethm} (Vector Fields on Sphere)\textbf{.}\quad
For $S^n$, a vector field is a section. For any smooth manifold
$M$, we have for each $x\in M$ the tangent space $T_{x}M$. A
vector field, therefore, is a section of a vector bundle. There
are no obstructions here.

What about nonvanishing vector fields on $S^n$? It is a section
of a bundle with a fiber $\RR^{n}\setminus\{0\}\homotopic S^{n-1}$.
This is true for any $n$-dimensional smooth manifold $M$, the
nonvanishing fields on it as a section on a bundle over $M$ with
fiber $S^{n-1}$. Why? Because what we are doing is considering
the \emph{unit vectors}. Two vectors are homotopic if and only if
they are ``pointing in the same direction'' but of different
magnitudes. Thus we are considering the collection of unit
vectors, which forms a sphere. So we are looking at sections of
this bundle
\begin{equation}
S^{n-1}\into V_{n+1,2}\to S^{n}.
\end{equation}
Observe the first homotoy group of the fiber is
$\pi_{n-1}(S^{n-1})=\ZZ$ and we have a section over $B^{n-1}$. We
want to extend this section to $B^{n}$. This is the first
obstruction, $S^{n-1}$ doesn't have any higher cells, so things
are simple.

\begin{wrapfigure}[13]{r}{2in}
  \vspace{-20pt}
    \includegraphics{img/lecture12.1}
  \vspace{-20pt}
\end{wrapfigure}

Consider the case for $S^{2}$, as doodled, where we have a unit
vector pointing towards the north pole. But in the arctic
regions, we need to change the vector field. We have for the
north pole
\begin{subequations}
\begin{equation}
\sigma\colon x\mapsto (-1)^{n-1}x,
\end{equation}
and for the South pole
\begin{equation}
\sigma\colon x\mapsto x.
\end{equation}
\end{subequations}
For $n$ even, we have an obstruction, and it is equal to 2. For
$n$ odd, we have a nonzero cochain which gives a class homologous
to zero\dots and we have no obstruction!
\medbreak

When we compute the sign, we should have taken into account
orientation of cells.
