%%
%% lecture21.tex
%% 
%% Made by alex
%% Login   <alex@tomato>
%% 
%% Started on  Sun Dec 25 13:45:39 2011 alex
%% Last update Sun Dec 25 13:45:39 2011 alex
%%
There is an even part $K^{0}(B)\to H^{\text{even}}(B)$ and
multiplying by $\RR$ or $\QQ$ we get
\begin{equation}
K^{0}(B)\otimes\QQ\iso H^{\text{even}}(B,\QQ).
\end{equation}
For every vectyor bundle $E$, we have Chern classes of $E$,
$c_{k}(E)\in H^{\text{even}}(B)$ and note that everything we're
working with is complex and hence even dimensional. So we have
\begin{equation}
c_{n}(E\oplus E')=\sum_{k+\ell=n}c_{k}(E)c_{\ell}(E'),
\end{equation}
so we need something else since we have addition $E\oplus E'$ on
the left hand side, and multiplication on the right hand side. We
could consider ``some'' mapping
\begin{equation}
[E]\to f(c_1,\dots,c_k),
\end{equation}
but which mapping?

Chern classes\index{Chern Class!from Cohomology of $B_{G}$} really come from the cohomology of classifying space.
Remember we defined the classifyign space $B_{G}$ as the quotient
\begin{equation}
B_{G}\eqdef E_{G}/G
\end{equation}
where $E_{G}$ is contractlbe, and $G$ acts on $E_{G}$ without
fixed points. If we have a subgroup $H\propersubset G$, then we
may take $E_{H}=E_{G}$. We see $H$ acts on $E_{G}$ without fixed
points. The classifying space for $H$ is different from the
classifying space over $G$, we have
\begin{equation}
E_{H}/H\to E_{G}/G
\end{equation}
which corresponds to a map $B_{H}\to B_{G}$, and this induces a
map of the cohomology $H^{n}(B_{G})\to H^{n}(B_{H})$.

For complex bundles, we take $G=\U{n}$. For $H$, we take the
maximal torus $H=T^{n}\propersubset\U{n}$. We therefore have a
map of
$H^{k}\bigl(B\U{n}\bigr)\to H^{k}(BT^{n})$. We have
\begin{equation}
T^{n}=S^{1}\times\dots\times S^{1}=(S^{1})^{n},
\end{equation}
and
\begin{equation}
BS^{1}=B\U{1}=\CP^{\infty},
\end{equation}
so the cohomology of $\CP^{\infty}$\index{$\CP^{\infty}$} has a single
generator. Indeed
\begin{equation}
H^{\bullet}(\CP^{\infty})=\ZZ[\zeta]
\end{equation}
The classifying space of products is the product of classifying
spaces, so really we have\index{Classifying Space!for Maximal Torus}
\begin{equation}
H^{\bullet}(BT^{n})=\ZZ[\zeta_{1},\dots,\zeta_{n}]
\end{equation}
describe the cohomology ring for $BT^n$.

Also if we have a $\gamma\in G$ such that $\gamma
H\gamma^{-1}=G$ leaving $H$ ``in tact'' (not fixed but transforms
$H$ into itself), then if we assume that $G$ is connected, we see
that $\gamma$ acts trivially on $G$ but not necessarily trivial
on $H$. We saw if we take $W=S_{n}$\index{Weyl Group!Acting on $H^{\bullet}(T)$} the permutation group
transforming diagonal matrices into diagonal matrices, they act
nontrivially on the torus. It is easy to prove that 
\begin{equation}
H^{\bullet}\bigl(B\U{n}\bigr)\to\ZZ[\zeta_1,\dots,\zeta_n]^{S_{n}},
\end{equation}
i.e., they are mapped to symmetric polynomials. This is the easy
part of the story. The hard part is that this map is an
isomorphism!

Now, we explained that the Chern classes\index{Chern Class!as Symmetric Polynomial} are mapped to symmetric
functions:
\begin{equation}
\begin{split}
c_{1}&\mapsto\zeta_1+\cdots+\zeta_n\\
c_{2}&\mapsto\sum_{i<j}\zeta_{i}\zeta_{j}\\
c_{3}&\mapsto\sum_{i<j<k}\zeta_{i}\zeta_{j}\zeta_{k}\\
&\dots
\end{split}
\end{equation}
We need something else: we need the Chern character. We
characterize it as the mapping
\begin{equation}
\chernChar\to \E^{\zeta_{1}}+\E^{\zeta_{2}}+\dots+\E^{\zeta_{n}},
\end{equation}
i.e., it's the characteristic class that corresponds to this
sum. So how to relate this? Lets decompose!

\begin{danger}
Please note that we are working in $H\otimes\QQ$ here, so what we
are doing is alright.
\end{danger}

We see that
\begin{equation}
\chernChar\to \E^{\zeta_{1}}+\E^{\zeta_{2}}+\dots+\E^{\zeta_{n}} =
n + \chernPt_{1}+\chernPt_{2}+\dots.
\end{equation}
The first part of the Chern character is
\begin{equation}
\begin{split}
\chernPt_{1} &= \zeta_1+\dots+\zeta_{n} \\
&=c_{1}
\end{split}
\end{equation}
In general, we have
\begin{equation}
\chernPt_{k} =
\frac{1}{k!}\left(\zeta_{1}^{k}+\dots+\zeta_{n}^{k}\right).
\end{equation}
That's trivial. But this is a well-known symmetric function, and
it may be expressed in terms of elementary symmetric
functions. We find
\begin{equation}
\begin{split}
\chernPt_{2} &= \frac{1}{2}(\zeta_{1}^{2}+\dots+\zeta_{n}^{2})\\
&= \frac{1}{2}\left[(\zeta_{1}+\dots+\zeta_{n})^{2}-2\sum_{i<j}\zeta_{i}\zeta_{j}\right]\\
&= \frac{1}{2}[c_{1}^{2}-2c_{2}],
\end{split}
\end{equation}
so we may talk about the Chern character of a vector
bundle. Moreover,
\begin{equation}
\chernChar(E_{1}\oplus E_{2}) =
\chernChar(E_{1})+\chernChar(E_{2})
\end{equation}
and for the product we see
\begin{equation}
\chernChar(E_{1}\otimes E_{2}) =
\chernChar(E_{1})\chernChar(E_{2}).
\end{equation}
This means the Chern character\index{Chern Character!as Ring Morphism} gives a homomorphism 
\begin{equation}
K^{0}(B)\to H^{\text{even}}(B)
\end{equation}
between rings.

\index{Chern Class!of Line Bundles|(}
Let us look at one-dimensional bundles, so
\begin{equation}
B\to BS^{1}=B\U{1}=\CP^{\infty}
\end{equation}
and so let us take $B=S^2$. Really this is the only interesting
case for reasons we'll explain later. Remember, we've already
discussed bundles with the base $S^{2}$. The discussion was as
follows: we divided $S^{2}$ into two hemispheres, and we had a
transition function on the equator. We represent the transition
function as $z^{n}$ where $z\in\CC\homotopic S^{1}$. So the
tenros product would correspond to
\begin{equation}
z^{m}\cdot z^{n}=z^{m+n}.
\end{equation}
The Chern class is
\begin{equation}
c_{1}=n\zeta,
\end{equation}
where $n$ is the exponent of hte transition function, which has
information about the obstruction. Here
\begin{equation}
\zeta=[S^{2}].
\end{equation}
It is clear we want the tensor product to correspond to
multiplication, so it is clear that
\begin{equation}
\chernChar = \E^{c_{1}}.
\end{equation}
We got it, for a line bundle.

So lets justify why $S^{2}$ is sufficient to consider. Lets
discuss the classification of one-dimensional bundles. What
should we classify? Namely, maps
\begin{equation}
\{B\to B\U{1}=\CP^{\infty}=K(\ZZ,2)\}
\end{equation}
We have a wonderful, namely homotopy classes
$\homotopyClass(B,{K(\Pi,n)})$ are in one-to-one correspondence
with $H^{n}(B,\Pi)$. We have
\begin{equation}
B^{n-1}\to *
\end{equation}
be mapped to a point. But every $n$-dimensional cell in $B^{n}$
goes to an element in $\Pi$. This is a cochain! This cochain is
not arbitrary: it is a cocycle. It should be extendable to
$(n+1)$-dimensional guys. Higher dimensional cells don't matter,
we may always extend the map. We may extend the map how we want,
they're all homotopy equivalent. So all considerations may be
reduced to $S^{2}$.
\index{Chern Class!of Line Bundles|)}

\subsection*{EXERCISES}
\begin{xca}
Describe homomorphisms $K(B)\to\widetilde{K}(B)$, $K(B)\to\ZZ$,
$\widetilde{K}(B)\to K(B)$, $\ZZ\to K(B)$.
\end{xca}
\begin{xca}
Calculate $K^{0}(\CP^n)$ and $K^{1}(\CP^n)$.
\end{xca}
