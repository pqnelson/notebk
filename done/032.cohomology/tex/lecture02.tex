%%
%% lecture02.tex
%% 
%% Made by Alex Nelson
%% Login   <alex@tomato3>
%% 
%% Started on  Sat May 21 12:16:40 2011 Alex Nelson
%% Last update Sat May 21 13:00:16 2011 Alex Nelson
%%
The first thing to say is to consider locally trivial fibre
bundles, which means that we have a covering of the base
\begin{equation}
\{U_{\alpha}\propersubset X\lst\bigcup_{\alpha}U_{\alpha}=B\}.
\end{equation}
The fibration is trivial over each $U_{\alpha}$. The total space
is $E$. The typical fibre is $F$. We may say that over
$U_{\alpha}$ the fibration is trivial, i.e.
\begin{equation}
p^{-1}(U_{\alpha})\iso U_{\alpha}\times F
\end{equation}
This identification is \emph{not} unique. If we suppose
\begin{equation}
E=B\times F,
\end{equation}
or in other words the fibration is trivial, then there is a map
\begin{equation}
B\times F\to E
\end{equation}
which is a homeomorphism and transforms fibre to fibre. We may
consider a map
\begin{equation}
\begin{array}{c}
B\times F\to B\times F\\
(b,f)\mapsto(b,\widetilde{f})
\end{array}
\end{equation}
we may say
\begin{equation}
\widetilde{f}=\varphi(f,b).
\end{equation}
In the simplest case
\begin{equation}
B=\{b\}
\end{equation}
and this map is a homeomorphism of the fibre. But for many
fibres, the homeomorphism depends on the base point.

We may say
\begin{equation}
E=\bigcup_{\alpha}p^{-1}(U_{\alpha})
\end{equation}
the total space is the union of preimages. But we have
overlaps. So we may write
\begin{equation}
E=\bigsqcup_{\alpha}U_{\alpha}\times F/\sim
\end{equation}
where we identify the overlap, we should explain the rule that
allows us to paste this stuff together. If
\begin{equation}
U_{\alpha}\cap U_{\beta}\not=\emptyset
\end{equation}
then over this intersection we have two representations of our
fibre space, as the direct product
\begin{equation}
(U_{\alpha}\cap U_{\beta})\times F\to (U_{\alpha}\cap U_{\beta})\times F
\end{equation}
which transforms
\begin{equation}
(b,f)\mapsto (b,\widetilde{f})
\end{equation}
where
\begin{equation}
\widetilde{f}=\varphi_{\alpha\beta}(f,b)
\end{equation}
which depends on $\alpha$, $\beta$. We may say  we have a map
\begin{equation}
\varphi_{\alpha\beta}(b)\colon F\to F
\end{equation}
that depends on 
\begin{equation}
b\in U_{\alpha}\cap U_{\beta}
\end{equation}
We have the equivalence defined by
\begin{equation}
(b,f)\sim\big(b,\varphi_{\alpha\beta}(f,b)\big)
\end{equation}
where $\varphi_{\alpha\beta}$ are called the \define{Transition
Functions}\index{Transition Function|textbf} or \define{Clutching Functions}\index{Clutching Function|textbf}. Note we have two
transition functions for this overlap $\varphi_{\alpha\beta}$ and
$\varphi_{\beta\alpha}$ which satisfy
\begin{equation}
\varphi_{\alpha\beta}\circ\varphi_{\beta\alpha}=\id{}
\end{equation}
by reflexivity. The transition functions are not completely
arbitrary, for a triple intersection 
\begin{equation}
U_{\alpha}\cap U_{\beta}\cap U_{\gamma}\not=\emptyset
\end{equation}
and we should stop and think. We have reflexivity, but what comes
from this? If we take the composition of transition functions, we
get a third one --- the only challenging thing is to write it in
the correct order. We see we have
\begin{equation}
b\in U_{\alpha}\cap U_{\beta}
\end{equation}
we have
\begin{equation}
(b,f)\sim\left(b,\varphi_{\alpha\beta}(f,b)\right)\sim\left(b,\varphi_{\alpha\gamma}(f,b)\right)
\end{equation}
where
\begin{equation}
\varphi_{\alpha\gamma}(f,b)=\varphi_{\beta\gamma}(f,b)\circ\varphi_{\alpha\beta}(f,b).
\end{equation}
We may discuss the bundles pasted together by transition
functions such that some compatibility conditions hold.

We may impose some conditions, e.g., $F$ is a vector space and
the transition functions 
\begin{equation}\label{eq:transitionFnInGLF}
\varphi_{\alpha\beta}\in\GL{F}
\end{equation}
We may say that we have a \define{Vector Bundle}\index{Bundle!Vector|textbf}, in this case
the fibres are vector spaces (which means we may add elements of
the same fibre, and scalar multiply). We need the condition
\eqref{eq:transitionFnInGLF} to avoid contradiction, since
$\varphi_{\alpha\beta}$ were homeomorphisms (viz, invertible).

The question is when are two vector bundles equivalent? We will
first consider examples.
\begin{ex}
The trivial vector bundle $B\times\RR^{n}$.
\end{ex}
\begin{ex}
Consider a sphere $S^{2}$ and at each point $x\in S^{2}$ we
consider the tangent vectors
\begin{equation}
T_{x}S^{2}\iso\RR^{2},
\end{equation}
we may consider
\begin{equation}
\bigcup_{x\in S^{2}}T_{x}S^{2}=TS^{2}=E.
\end{equation}
This is the tangent bundle of $S^{2}$.
\end{ex}
\begin{ex}
Consider any submanifold $M\subset\EE^{n}$ smooth, then
\begin{equation}
\bigcup_{x\in M} T_{x}M = TM = E
\end{equation}
is a more general example. Even more generally any smooth
manifold $M$ has 
\begin{equation}
TM = \bigcup_{x\in M} T_{x}M
\end{equation}
as a tangent bundle is a vector bundle.
\end{ex}

We take any space $F$ with an action of a topological group $G$,
we can require the transition functions
\begin{equation}
\varphi_{\alpha\beta}\in G,
\end{equation}
we may say that the transition functions give a map
\begin{equation}
U_{\alpha}\cap U_{\beta}\to G.
\end{equation}
We say that we have a \define{$G$-Bundle}\index{Bundle!$G$|textbf} or a space with a
\define{Structure Group}\index{Group!Structure ---}\index{Structure Group} $G$.

Recall for principal bundles, we had a topological group $G$ and
it acts freely (so the orbits are in one-to-one correspondence
with $G$). If a principal bundle has a section, then this
principal bundle is a trivial bundle. lets repeat the
proof. Well, if there are ``local sections'' of our space $E$,
and we have
\begin{equation}
E\to E/G.
\end{equation}
Let $\sigma_{\alpha}$ be a \define{Local Section}\index{Section!Local}
\begin{equation}
\sigma_{\alpha}\colon U_{\alpha}\to E
\end{equation}
where $U_{\alpha}\subset E/G$. We want to identify
\begin{equation}
p^{-1}(U_{\alpha})\iso U_{\alpha}\times G
\end{equation}
We suppose that this has a right action of $G$, so we have
\begin{equation}
(x,g)\mapsto xg
\end{equation}
and
\begin{equation}
(xh)g=x(hg)
\end{equation}
where $g,h\in G$ and $x\in U_{\alpha}$. Now this identification
of $p^{-1}(U_{\alpha})$ in the following way, we are sending
\begin{equation}
(g,b)\mapsto\sigma_{\alpha}(b)g.
\end{equation}
Note that this implies the point
\begin{equation}
(1,b)\mapsto\sigma_{\alpha}(b)
\end{equation}
What remains is to express transition functions in terms of the
section $\sigma_{\alpha}$. That's easy. What do we have? We have
\begin{equation}
b\in U_{\alpha}\cap U_{\beta}
\end{equation}
then we can write a formula
\begin{equation}
\sigma_{\alpha}(b)g=\sigma_{\beta}(b)\widetilde{g}.
\end{equation}
We simply have
\begin{equation}
(\sigma_{\beta}^{-1}\circ\sigma_{\alpha})g=\widetilde{g}
\end{equation}
and we have
\begin{equation}
\varphi_{\alpha\beta}=\sigma_{\beta}^{-1}\circ\sigma_{\alpha}
\end{equation}
describe the transition function. So we may say that a principal
bundle is a G-bundle where the fibre is $G$ and $G$ acts on
itself by left action.

