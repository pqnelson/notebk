%%
%% lecture23.tex
%% 
%% Made by alex
%% Login   <alex@tomato>
%% 
%% Started on  Mon Dec 26 11:07:34 2011 alex
%% Last update Mon Dec 26 11:07:34 2011 alex
%%
Now, what about applications of $K$-theory? It has a lot of
applications, since vector bundles are considered in a lot of
brancehs of nmathematics. Perhaps the most surprising application
is to division aalgebras. We consider algebras over $\RR$. Now,
recall
\begin{equation}
\mbox{algebra} = \mbox{ring}+(\mbox{vector space})
\end{equation}
with some compatibility conditions. What are some examples? Well,
$\RR$ is an algebra over $\RR$, and $\CC$ is an algebra over
$\RR$ too (of dimensions 1 and 2, respectively). We will focus on
\define{Division Algebras}\index{Division Algebra} where, if 
\begin{equation}\label{eq:lec23:divAlgProb}
ax=b\quad\mbox{and}\quad a\not=0
\end{equation}
then we may solve this equation (``we may divide''). If we have
associativity, it is sufficient to solve
\begin{equation}
ax=1
\end{equation}
and denote the solution by $a^{-1}$. Then the solution to Eq
\eqref{eq:lec23:divAlgProb} would be
\begin{equation}
x=a^{-1}b.
\end{equation}
But this \emph{requires} associativity. In general, we don't
necessarily have associativity.

\marginpar{Quaternions}\index{Quaternions}Another division algebra is the
\define{Quaternion Algebra}, denoted by $\HH$ after its inventor
Hamilton. It is associative and noncommutative as well as
4-dimensional. It is a normed algebra, meaning
\begin{equation}
\|xy\|=\|x\|\cdot\|y\|.
\end{equation}
We may consider the situation when
\begin{equation}
\|x\|=1
\end{equation}
which is $S^{3}$ and a group isomorphic to $\SU{2}$. Note that
$\CC$ may be considered likewise, giving us an isomorphism
$S^{1}\iso\U{1}$. 

\marginpar{Octonions}The next algebra is the \define{Octonions}\index{Octonions}
which is an 8-dimensional nonassociative algebra. We may consider
a sphere $S^{7}$ but it is not a group---we don't even have associativity!
It is not terrible, the octonions are ``weakly nonassociative''
(in the sense that they are ``alternative'' --- when we attempt
to modify parentheses, it costs us a sign).
For more on the Octonions, see Baez's beautiful review paper~\cite{baez}.

Can we find other division algebras? Well, all division algebras
have dimension 1, 2, 4, and 8. This may be proven by
$K$-theory. Lets note it is a purely algebraic problem. Does it
have something to do with topology? Yes. Why? Look! Look first at
the case when we have associativity and a norm (this is too much,
we don't need it, but lets suppose). We have $n$ be the
dimension, then $S^{n-1}$ be a group. But a group has the
property that the group manifold is ``parallelizable''. What does
it mean? We have a $k$-dimensional manifold $M$, we may consider
the tangent bundle as a principal $\GL{k}$-bundle. In general, we
cannot continuously take a basis at each $T_{x}M$ for all $x\in
M$, but if we can then we have a group. It's a Lie group! Fo a
Lie group, we consider some basis at a point (e.g., the identity
element $e$), then act by $g\in M$ to smoothly map the basis to
$T_{g}M$. We did not use (nor need) associativity so far.

In reality, we may say the following thing: if the dimension of
the division algebra is $n$, then $S^{n-1}$ is parallelizable.

\bigbreak
The next application which is much more important, a revolution
during a time when analysis and topology were disjoint and
uninteracting fields. What happened? $K$-theory proved to be
useful when studying partial differential equations. We should
explain the notion of the \index{Operator Index|textbf}\index{Inder!of Operator}\define{Index} \textbf{of an Operator},
suppose $E_{1}$ and $E_{2}$ are infinite-dimensional linear
spaces and 
\begin{equation}
A\colon E_{1}\to E_{2}
\end{equation}
is a linear operator. 
We may consider $\ker(A)\propersubset E_{1}$ and
$\coker(A)=E_{2}/\im(A)$, then the index is
\begin{equation}
\ind(A)=\dim\bigl(\ker(A)\bigr)-\dim\bigl(\coker(A)\bigr).
\end{equation}
The only question is: is this well-defined? Could we get
\begin{equation}
\ind(A)=\infty-\infty?
\end{equation}
If we require $\ker(A)$ and $\coker(A)$ to both be
finite-dimensional, then we call $A$ a \define{Fredholm Operator}\index{Operator!Fredholm}\index{Fredholm Operator}
which is an isomorphism up to some one-dimensional problems, we
require $A(E_{1})$ to be closed in $E_{2}$.

In the finite-dimensional case,
\begin{equation}
\begin{split}
\dim\bigl(\coker(A)\bigr)
&=\dim(E_{2})-\dim\bigl(A(E_{1})\bigr)\\
&=\dim(E_{2})-\dim\left(\dim(E_{1})-\dim\bigl(\ker(A)\bigr)\right).
\end{split}
\end{equation}
But this won't work for infinite-dimensional cases. We will say
that the index is continuous (if we change our operator ``a
little bit'', the index ``continuously changes''). This means
that we have thus
\begin{equation}
\ind(A+\alpha)=\ind(A)
\end{equation}
if $\alpha$ is ``small'', i.e., 
\begin{equation}
\|\alpha\|<(\mbox{some number}).
\end{equation}
But we may use topology to have some information about the index.

We use this to deform an operator to a simpler operator. This is
a very topological problem. But we should stary in the domain of
Fredholm operators. Elliptic operators play an important role,
specifically elliptic differential operators (or, better,
elliptic pseudodifferential operators which lets us simplify
things better). The result of all this stuff is the
\define{Atiyah--Singer Theorem}\index{Atiyah--Singer Theorem}.

Instead of using the differences of dimension, can we consider
$\ker(A)-\coker(A)$? Why not, use formal differences! If we have
a family of operators $A_{x}$ depending on $x\in X$, then we get
a ``vector bundle''
\begin{equation}
\bigl(\ker(A)-\coker(A)\bigr)_{x}
\end{equation}
which is not quite right, but this ``index'' is an element of
$K(X)$, the $K$ group of $X$.
