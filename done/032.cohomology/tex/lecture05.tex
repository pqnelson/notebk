%%
%% lecture05.tex
%% 
%% Made by Alex Nelson
%% Login   <alex@tomato3>
%% 
%% Started on  Sat May 21 13:43:44 2011 Alex Nelson
%% Last update Sun Aug 14 14:59:39 2011 Alex Nelson

We spoke a little bit about the notions of universal bundle and
the classifying space. The universal fibration is the space
($E_{G}$, $B_{G}$, $G$) and the main property is $E_{G}$ is
contractible. Then $B_{G}$ is the base of this construction and
referred to as the classifying space.

Recall the principal fibration ($E$, $B$, $G$, $p$) are in
one-to-one correspondence with $\homotopyClass(B,B_{G})$. We will
prove this when $B$ is a cell complex. Recall we introduced the
notion of a pullback, and by considering the maps
$B\to B_{G}$ and fibrations
\begin{equation}
p_{G}\colon E_{G}\to B_{G},
\end{equation}
we can pull it back to obtain a fibration over $B$. Now what
remains is to prove this correspondence is a one-to-one
correspondence. We prove for a mapping
\begin{equation}
B\to B_{G}
\end{equation}
gives a fibration. We should prove homotopy equivalent maps
\begin{equation}
B\to B_{G}
\end{equation}
go to equivalent fibrations. Now we should prove this map is
injective and surjective. The main part is proving every
fibration may be obtained from this construction; so what should
we prove really? Every principal fibration can be mapped to a
universal fibration.

By hypothesis, we assumed $B$ is a cell complex. So 
\begin{equation}
B=\bigcup B_{k}
\end{equation}
is a union of skeletons. We have a proof by induction. The idea
is to cut out a closed ball. We have a map on the ball
$D^{n}\times G$ and on the boundary of the ball $S^{n-1}\times G$
we construct this map from
\begin{equation}
f\colon S^{n-1}\to E_{G}
\end{equation}
by
\begin{equation}
s\mapsto (s,1)
\end{equation}
Then $G$ acts on it by
\begin{equation}
(s,1)=(s,g)
\end{equation}
We should extend our map to $(b,1)$ for $b\in D^{n}$, but this
can be done since $E_{G}$ is contractible. So every map from a
sphere may be extended to a ball.

We proved if
\begin{equation}
\pi_{k}(E_{G})=0
\end{equation}
for $k\leq n$, then the map
\begin{equation}
\homotopyClass(B,B_{G})%\onto
\xtwoheadrightarrow{sur.}\{\mbox{classes of principal fibrations over }B\}
\end{equation}
is onto if $\dim(B)\leq n$. To prove this mapping is injective, well,
it's trivial.

\begin{ex}
Consider
\begin{subequations}
\begin{equation}
E_{G}=S^{n}
\end{equation}
and
\begin{equation}
G=\ZZ_{2}.
\end{equation}
\end{subequations}
In this case, the action can be defined in a very simple way, the
nontrivial element of $\ZZ_{2}$ sends
\begin{equation}
x\mapsto-x
\end{equation}
We know
\begin{equation}
B_{G}=E_{G}/G=\RP^{n}.
\end{equation}
Is this the classifying space? Not really, we have
\begin{equation}
\pi_{k}(S^{n})=0
\end{equation}
for $k<n$. It is not contractible. For 
\begin{equation}
E_{G}=S^{\infty}
\end{equation}
which is, more or less, infinite sequences such that
\begin{equation}
|x_{1}|^{2}+\dots+|x_{n}|^{2}+\dots=1.
\end{equation}
We usually set
\begin{equation}
x_{n}=0
\end{equation}
for $n\gg0$. We may say\index{$\RP^{\infty}$}\index{Classifying Space!for $G=\ZZ_{2}$}
$\RP^{\infty}$ is the classifying space for $G=\ZZ_{2}$.
\end{ex}
\begin{ex}
But we may go further. We can consider the case when
$x_{i}\in\CC$ and we have
\begin{equation}
\sum_{i\in\NN}|x_{i}|^{2}=1
\end{equation}
This is a principal fibration with
\begin{equation}
G=S^{1}=\{\lambda\in\CC\lst|\lambda|=1\}.
\end{equation}
It acts by
\begin{equation}
\lambda(x_{1},\dots,x_{n},\dots)=(\lambda x_{1},\dots,\lambda
x_{n},\dots).
\end{equation}
If we consider finite sequences we have $S^{2n-1}$ and the base
would be $\CP^{n-1}$. But if we consider infinite sequences, we
get
\begin{equation}
S^{\infty}\to\CP^{\infty}
\end{equation}
we just denote this codomain by ``$\CP^{\infty}$'' for
continuity. So $\CP^{\infty}$\index{$\CP^{\infty}$} may be viewed as the space of all
wave functions in quantum mechanics. But it is quite clear that
$\CP^{\infty}$ is the classifying space for $S^{1}$.
\end{ex}

We can think of the quaternions\index{Quaternions} as
\begin{equation}
x_{0}+x_{1}i+x_{2}j+x_{3}k=(x_{0},\vec{x}).
\end{equation}
Multiplication is then 
\begin{equation}
(x_{0},\vec{x})\star(y_{0},\vec{y})=(x_{0}y_{0}-\vec{x}\cdot\vec{y},\vec{x}\times\vec{y}).
\end{equation}
It is a normed-division algebra over $\RR$ and 4-dimensional. The
typical notation for the quaternions is $\HH$ in honor of its
inventor Hamilton.

Let $(x_{1},\dots,x_{n})\in\HH^{n}$, assume this is nonzero and
identify
\begin{equation}
(x_{1},\dots,x_{n})\sim(\lambda x_{1},\dots, \lambda x_{n})
\end{equation}
where $\lambda\in\HH\setminus\{0\}$. The result is
$\HP^{n-1}$. 

Or we may impose the condition
\begin{equation}
|x_{1}|^{2}+\dots+|x_{n}|^{2}=1.
\end{equation}
Observe that the set of $\lambda\in\HH$ such that
\begin{equation}
|\lambda|=1,
\end{equation}
i.e., $\lambda\in S^{3}$ defines a group. Only $S^{0}\iso\ZZ_{2}$,
$S^{1}\iso\U{1}$, and $S^{3}$ are groups. So we can say that
$\HP^{\infty}$\index{$\HP^{\infty}$} is the classifying
space\index{Classifying Space!for $G=S^{3}$} for $G=S^{3}$. NB
$S^{3}\iso\SU{3}$. 

We should recall Stiefel manifolds\index{Stiefel Manifold} and Grassmann manifolds\index{Grassmann manifold}
denoted by $V_{n,k}$\index{$V_{n,k}$} and $\Gr_{n,k}$\index{$\Gr_{n,k}$} (respectively). We have
\begin{equation}
\Gr_{n,k}(\FF)=\{\text{space of $k$-dim. linear subspaces of
$n$-dim. vector space}\}
\end{equation}
\begin{thm}
We have $\Gr_{\infty,k}(\RR)$\index{$\Gr_{\infty,k}(\RR)$} is the classifying
space\index{Classifying Space!for $\GL{k,\RR}$} for
$\GL{k,\RR}$, and $\Gr_{\infty,k}(\CC)$\index{$\Gr_{\infty,k}(\CC)$} is the classifying space
for $\GL{k,\CC}$.\index{Classifying Space!for $\GL{k,\CC}$}
\end{thm}


