%%
%% lecture01.tex
%% 
%% Made by alex
%% Login   <alex@tomato>
%% 
%% Started on  Mon Dec 26 11:43:13 2011 alex
%% Last update Mon Dec 26 11:43:13 2011 alex
%%

We will return to fibrations, discussing pricnipal fibrations and
pricnipal bundles, characteristic classes, spectral sequences,
and equivariant homology. We may get into $G$-structures. We will
start with repeating the notion of fibrations, and some
additional stuff.

Remember if we have a map $p\colon E\to B$ which we assume is
surjective, then 
\begin{equation}
F_{b}=p^{-1}\{b\}
\end{equation}
is the preimage of $b$ and we can decompose
\begin{equation}
E = \bigsqcup_{b\in B}F_{b}
\end{equation}
as a collection of preimages called \define{Fibres}\index{Fibre|textbf}. If all the
fibres are homeomorphic to
\begin{equation}
F\iso F_{b}\quad\mbox{for all }b\in B
\end{equation}
then $p$ is called a \define{Fibration}. 

\begin{rmk}[Notation]
Some people write the fibre bundle as $(E,F,B,p)$ whereas others
write it out as $F\into E\xrightarrow{p}B$. We will use both
notations, in order to confuse the reader!
\end{rmk}

We should ask for more!
W may introduce a notion of a \define{Locally Trivial Fibration}\index{Locally Trivial Fibration}\index{Fibration!Locally Trivial|textbf}. 
First, a \define{Trivial Fibration}\index{Bundle!Trivial}\index{Trivial Bundle}\index{Trivial Fibration}\index{Fibration!Trivial|textbf} is when
\begin{equation}
E=B\times F\quad\mbox{and}\quad p(b,f)=b.
\end{equation}
For locally trivial fibrations, we have locally for ``small
sets'' $U\propersubset B$ that 
\begin{equation}
p^{-1}(U)= U\times F.
\end{equation}

\begin{wrapfigure}{r}{0.75in}
  \vspace{-12pt}
  \includegraphics{img/lecture1.0}
\end{wrapfigure}

\noindent{}For example, th Mobius band\index{Mobius Band} is a locally trivial
fibration. How can we see this? Well, it is doodled on the right
as such. This notion isn't rigorous, for example what is meant by
``equality''? We have a notion of mappings of fibrations, if
$p\colon E\to B$ and $p'\colon E'\to B'$ are fibrations, then
\begin{equation}
\begin{diagram}[small]
f\colon& E &\rTo & E'\\
       &\dTo<{p}& &\dTo>{p'}\\
\varphi\colon& B&\rTo&B'
\end{diagram}
\end{equation}
commutes, which means
\begin{subequations}
\begin{equation}
f\colon F_{b}\to F_{b'}
\end{equation}
where 
\begin{equation}
b'=\varphi(b)
\end{equation}
\end{subequations}
and
\begin{equation}
p'\circ f=\varphi\circ p.
\end{equation}
We have a map of fibrations, and if $f$ is invertible, then the
fibrations are \define{Equivalent}\index{Fibration!Equivalent|textbf} which is what we really mean
for locally trivial fibrations.

\begin{Boxed}{Fibre Bundles}
Let $E$, $F$, $B$ be topological spaces.
The idea of a fibre bundle $F\into E\to B$ is that it is a
natural generalization of the product space $E=B\times F$. Why is
this a good idea? Well, for example, in particle physics we often
generalize the notion of a tangent space to work with ``tangent
spinors''. 
\end{Boxed}

There is a notion of a \define{Section}\index{Section!of a Fibration|textbf} of a fibration $p\colon
E\to B$ is a mapping
\begin{equation}
q\colon B\to E
\end{equation}
such that
\begin{equation}
p\circ q=\id{B},
\end{equation}
i.e., $q(b)\in F_{b}$ for all $b\in B$. What are some examples of
fibrations that we know? Well, if $F$ is discrete and $E$
connected, then the notion of a ``fibration'' is the same as a
``covering space.''\index{Covering Space|textbf}

\begin{thm}[Homotopy Lifting]\index{Homotopy Lifting Theorem|textbf}
If we have a locally trivial fibration $E\to B$, and if we have a
map $\alpha\colon X\to B$, then sometimes we may lift this map,
i.e., find an $\widetilde{\alpha}\colon X\to E$ such that
$\widetilde\alpha(x)\in F_{\alpha(x)}$, i.e.,
$p\circ\widetilde{\alpha}=\alpha$.
\end{thm}

Look, a section is a lift of the identity map. We have examples
where sections don't exist, so the lifts don't always exist. What
is important is that we can do this for homotopies. So if $X$ is
a cell complex, and 
\begin{equation}
\alpha_{t}\colon X\times I\to B
\end{equation}
then we may lift this to a homotopy
\begin{equation}
\widetilde{\alpha}_{t}\colon X\times I\to E
\end{equation}
which has the property
\begin{equation}
\widetilde{\alpha}_{0}=\widetilde{\alpha}.
\end{equation}
If we may lift the homotopy ``at one point'', we may lift the
whole homotopy. 

\index{Fibration!Serre|(}
We may use this theorem to define a fibration. This is important,
as there are some fibrations which are not locally
trivial. Consider any space $B$. Lets fix a point $*\in
B$. Consider all the maps
\begin{equation}
\varphi\colon I\to B
\end{equation}
such that
\begin{equation}
\varphi(0)=*,
\end{equation}
the space of all these paths is called $\Omega$ but we will
denote this by $E$. There is a map 
\begin{equation}
p\colon E\to B
\end{equation}
which sends every path $\varphi$ to $\varphi(1)$ its end
point. What is its fibre? Lets fix some $b\in B$, then
\begin{equation}
p^{-1}(b)=\{\varphi\in E\mid \varphi(0)=*,\varphi(1)=b\} = \Omega_{*,*}=\Omega.
\end{equation}
In general, this is not a fibration (unless $B$ is a manifold),
but this theorem about lifting applies here---just use the
homotopy extension property. This shift is a \define{Serre Fibration}\index{Fibration!Serre|)}.

If we have a locally trivial fibration, or Serre fibration, then
we have an exact homotopy sequence\index{Fibration!Exact Homotopy Sequence}
\begin{equation}
\dots\to\pi_{n}(F,*)\to\pi_{n}(E,*)\to\pi_{n}(B,*)\to\pi_{n-1}(F,*)\to\dots
\end{equation}
the mapping 
\begin{equation}
\pi_{n}(B,*)\to\pi_{n-1}(F,*)
\end{equation}
requires the homotopy lifting theorem above. We also see
\begin{equation}
\pi_{n}(E,F,*)=\pi_{n}(B,*)
\end{equation}
is a consequence of the homotopy lifting theorem.
Observe
\begin{equation}
E=\{\varphi\colon I\to B\mid \varphi(0)=*\}
\end{equation}
is contractible. So this means the exact homotopy sequence gives
$\pi_{n}(B,*)\iso\pi_{n-1}(\Omega)$.

We would like to consider \define{Principal Fibrations}\index{Fibration!Principal|textbf}
by taking a topological group\index{Group!Topological} $G$ and assume $G$ acts on $E$,
then we may consider orbits of $G$ acting on $E$. The space of
orbits is denoted $E/G$, there is a natural map
\begin{equation}
E\to E/G
\end{equation}
and the fibres of this map are precisely the orbits. There is a
situation when the action is \define{Free}\index{Group!Free Action|textbf}\index{Free!Group Action|textbf}
(there are no stabilizers); then the orbits are in one-to-one
correspondence with $G$. Then we are talking about a principal
fibvration. It will be a locally trivial fibration.
\begin{prop}
If a principal fibration has a section (and when $G$ is compact),
then the fibration is trivial.
\end{prop}
The proof is trivial. If we have a section $\sigma\colon B\to E$
and $G$ acts on the right (it'll be important later), take a pair
$(b,g)$ and map it to $\sigma(b)\cdot g$. This is a map
\begin{equation}
B\times G\to E
\end{equation}
which is continuous and injective.
