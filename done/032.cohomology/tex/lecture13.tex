%%
%% lecture13.tex
%% 
%% Made by alex
%% Login   <alex@tomato>
%% 
%% Started on  Fri Dec 16 10:51:06 2011 alex
%% Last update Fri Dec 16 10:51:06 2011 alex
%%
We have explained the notion of obstruction, but we did not prove
the notion we formulated. We will change the setting: instead of
working with cell complexes, we demand that
$\overline{\sigma}^{k}$ is a closed ball. Then we may simplify
the definition. We have a fibration $(E,B,F,p)$ and we construct
a section step-by-step. Remember we used induction, so if we have
a skeleton $B^{r}$ and a section 
$B^{r}\xrightarrow{\;f}E$, $b\mapsto f(b)\in F_{b}$, for small $r$
every two sections are homotopic. We would like to extend to
$(r+1)$-cells, how many different ways can we do this? Well, we
have $\overline{\sigma}^{k+1}$ and over this ball we have the
direct product $\overline{\sigma}^{k+1}\times F$. This is a
little different from last lecture, but in our new requirements
we have a direct product.

Then our section is simply a map $\partial\sigma^{r+1}\to F$
which is $\partial\sigma^{r+1}\iso S^{r}\to F$ and we want to
extend to $\sigma^{r+1}$ the rest of the cell. Then the map
$(S^{r}\to F)\homotopic 0$ homotopic. Why? Because otherwise, we
would have different approaches to extending the section to the
rest of the cell. Specifically since $F$ is connected, we should
look at $\pi_{r}(F)$. The first possibility is that
$\pi_{r}(F)=0$, then the extension is always possible. The other
possibility is that $\pi_{r}(F)\not=0$, then we have an obstruction.

Suppose that $\pi_{r}(F)=0$ for $r<k$, but
$\pi_{k}(F)\not=0$. What does this mean? We may extend our
section up to $B^{k}$ the $k$-dimensional skeleton. The extension
$B^{r-1}\to B^{r}$, for $r<k$, is essentially unique (i.e., unique
up to homotopy).
Why? Well, suppose we have $f\colon B^{r-1}\to F$ and we extend
this to $B^{r}$ in two ways $\widetilde{f}\colon B^{r}\to E$ and
$\widetilde{f}'\colon B^{r}\to E$. We have at $\sigma^{r}$
specifically $\widetilde{f}\colon\partial\sigma^{r}\to F$,
$\widetilde{f}'\colon\partial\sigma^{r}\to F$ but on the boundary
we have
\begin{equation}
f = \left.\widetilde{f}\right|_{\partial\sigma^{r}}=\left.\widetilde{f}'\right|_{\partial\sigma^{r}}
\end{equation}
This is really a map $S^{r}\to F$ and corresponds to an element of
$\pi_{r}(F)$. So for $r<k$, the two maps are homotopic, but for
$r=k$ this is wrong.

For $r=k$, and $\pi_{r}(F)=0$ for $r<k$ but $\pi_{k}(F)\not=0$,
then the extension of the section to $B^{k-1}$ is essentially
unique. If we have two extensions $f$, $f'$ to $B^{k}$, then we
have the difference cochain with values in $\pi_{k}(F)$ which is
denoted by $\D(f,f')$. Recall a cochain is a function on
cells. And for every cell we have various extensions. The
difference cochain describes the difference between the
extensions.

But we may go farther and try to extend it to $B^{k+1}$. We have
an obstruction, which gives a cochain $\zeta(f)$ with values in
$\pi_{k}(F)$. We may say that an obstruction to $f$, $\zeta(f)$,
differ from an obstruction to $f'$, $\zeta(f')$ by
\begin{equation}
\zeta(f)-\zeta(f')=\nabla\D(f,f').
\end{equation}
This is sloppy, really the obstruction is
$\zeta_{f}(\sigma^{k+1})$, so we should really make $f$ an index
and thus
\begin{equation}
\zeta_{f}(\sigma^{k+1})-\zeta_{f'}(\sigma^{k+1})=\nabla\D_{f,f'}
\end{equation}
which is very natural. Already everything follows.

First we would like to say that $\zeta$ is a
cocycle\index{Obstruction!is a cocycle}. Why? Recall
we calculate all this stuff locally, so we may work in a direct
product (i.e., take advantage of the local trivialization of the
fibration). Recall in the direct product we have a trivial
section, which means that we have a section which sends
everything to one point in the fibre. In this case, everything is
very simple. If $f'$ is a trivial section, then 
\begin{equation}
\zeta(f')=0.
\end{equation}
Thus we see
\begin{equation}
\zeta(f)=\nabla\D_{f,f'}
\end{equation}
and
\begin{equation}
\nabla\zeta_{f}=\nabla^{2}\D_{f,f'}=0.
\end{equation}
But this is only in the direct product\dots but for locally
trivial fibrations, we're done.

We have some freedom in our choice of $\zeta_{f}$. It is a
cocycle with values in $\pi_{k}(F)$. It is only a coboundary in
the direct product, but we only have that \emph{locally}. We
have
\begin{equation}
\zeta_{f}=\nabla\D_{f,f'}
\end{equation}
be a \emph{local} expression, but 
\begin{equation}
\nabla\zeta_{f}=0
\end{equation}
is a \emph{global} expression.

We constructed for every locally trivial fibration $(E,B,F,p)$
such that
\begin{equation}
\pi_{k}(F)\not=0
\end{equation}
is the first nontrivial homotopy grup, a representative in
$H^{k+1}\bigl(B,\pi_{k}(F)\bigr)$. And this was for \emph{every}
bundle. We should assume $k\geq2$ in this business\dots
otherwise the cohomology coefficients may be non-Abelian.

This construction is very natural (i.e., functorial). Take a map 
\begin{equation*}
(E,B,F,p)\to(E',B',F,p')
\end{equation*}
which is a map of fibrations. Really it is a commutative diagram
\begin{equation}
\begin{diagram}[small]
E        & \rTo^{\varphi} & E' \\
\dTo>{p} &                & \dTo>{p'} \\
B        & \rTo^{\psi}    & B'
\end{diagram}
\end{equation}
We have $\zeta\in H^{k+1}\bigl(B,\pi_{k}(F)\bigr)$ behave
``nicely'' as we go from fibration to fibration, namely we have
the following property: $\zeta$ is an obstruction of the first
fibration, $\zeta'$ is for the second. We are saying
$\zeta=\psi^{*}(\zeta')$. How to prove it? Look at the
definitions, and that's it!

Remember this was one possible definition for a characteristic
class. (See Lecture \ref{215c:lecture7}.) If the fibre is a group
$G$, then we have the principal fibration
$(E,B,G,p)\to(E_{G},B_{G},G,p_{G})$ be mapped into the universal
bundle. We proved that the characteristic classes obey this
functional property. The only difference is that we were working
with an arbitrary fibr for obstructions, whereas we work with
principal bundles for characteristic classes.

Is this too bad? Not really, we can think of a characteristic
class as an obstruction.

But we had a lot of associated bundles for a principal bundle
$(E,B,F,p)$. If $F$ is a $G$-space, then we may construct an
associated fibre bundle $(E_{F},B,F,p_{F})$ where we may
explicitly construct $E_{F}=(E\times F)/\sim$.

Consider the group $\SO{n}$ and we'd like to construct a
characteristic class of it. The simplest way is to take $\RR^n$
as an $\SO{n}$-space. SO we can take a vector bundle using
$\RR^n$ as fibre. We get an associated vector bundle. We see that
$\RR^n\homotopic(\mbox{point})$, so the obstructions are trivial.

But we may take $S^{n-1}\subset\RR^{n}$ which is an
$\SO{n}$-space. We may take sections here. We get an obstruction
$\zeta\in H^{n}\bigl(B,\pi_{n-1}(S^{n-1})\bigr)$ which is a
characteristic class.
\index{Section!Obstructions to|)}
