%%
%% lecture14.tex
%% 
%% Made by alex
%% Login   <alex@tomato>
%% 
%% Started on  Fri Dec 16 16:20:11 2011 alex
%% Last update Fri Dec 16 16:20:11 2011 alex
%%
We have a principal bundle $(E,B,G,p)$ and we'd like to construct
\index{Principal Bundle!Characteristic Classes on ---}\index{Characteristic Class!of Principal Bundle}characteristic classes of it. Consider the universal bundle
$E_{G}\xrightarrow{p_{G}}B_{G}$ where $E_{G}/G=B_{G}$. We have a
map $\psi\colon B\to B_{G}$, take $\psi^{*}(c)\in H^{\bullet}(B)$.

Consider another bundle $(E_{F}, B,F,p_{F})$ we may consider
sections in this new bundle. The first nontrivial homotopy group
$\pi_{k}(F)$, we have $\zeta\in H^{k+1}\bigl(B,\pi_{k}(F)\bigr)$
be the first obstruction to the construction of sections. Really
it is a characteristic class. For the associated bundle
$(E_{G}\times F/\sim, B_{G}, F,\dots)$ we have 
$\zeta_{G}\in H^{k+1}\bigl(B_{G},\pi_{k}(F)\bigr)$
and thus $\zeta=\psi^{*}(\zeta_{G})$.

For $G=\SO{k}$, it acts on $\RR^k$ and therefore
$S^{k-1}\subset\RR^{k}$. We may consider the characteristic class
$\zeta\in H^{k}(B,\ZZ)$. This is the \define{Euler Characteristic Class}\index{Euler Characteristic Class}\index{Characteristic Class!Euler}.
Why is it called such? Lets consider a smooth manifold of
dimension $k=\dim(M)$. We may consider the tangent bundle
$(E=TM,B=,\RR^k)$, but instead of tangent vectors we may take
frames, based in tangent spaces, $(E,M,\ORTH{k})$ where the fibre
is the frames at a given point. Or if the manifold is orientable,
and oriented, then the fibre is $\SO{k}$. We may consider unit
vectors (or if we don't have a metric, all nonzero vectors) then
the fibre is $\RR^{k}\setminus\{0\}\homotopic S^{k-1}$. Sections
correspond to\dots?
\begin{center}
\begin{tabular}{c|c|p{5cm}}
   $E$          &    $F$     & section \\ \hline
$TM$            & $\RR^k$    & Vector field\\
frames          & $\ORTH{k}$ & \\
oriented frames & $\SO{k}$   & \\
unit vectors    & $\RR^k\setminus\{0\}\homotopic S^{k-1}$ & unit
vector fields, section exists on $M$ except on finite number of points
\end{tabular}
\end{center}
\noindent{}A vector field nonzero everywhere on $M$ implies obstruction is
0; but if it cannot be constructed everywhere, finite set is
``under'' the obstruction.

So all we need is the number of singular points of the vector
field? Not really. We should take the \emph{algebraic number} of
singular points of the vector field. The algebraic number of
singularities for the vector field is an obstruction, and it is
equal to $\eulerChar(M)$. This is the first problem of the
homework! We will not solve it here in great detail.

Suppose we have a triangulation. At vertices we say that our
vector field is zero. On every edge the vector field points
towards the center.

\begin{wrapfigure}[7]{r}{1in}
  \includegraphics{img/lecture14.0}
\end{wrapfigure}

It has a singularity at the center of the edge, we decrease the
magnitude of the vector field as it tends toward the center. We
try to extend it from the edges to the faces, the vector fields
points towards the center of the faces. We get a singular point
on each cell.

But this is an algebraic sum, so we have some coefficient
$\sum(-1)^{k}\alpha_{k}$, we should recall this is an obstruction
and the obstruction lies inthe homology groups; the coefficient
is thus $(-1)^{k}$ up to a sign due to choice of orientation. The
vector field may be represented as
\begin{equation}
f(x)=Ax+(\mbox{higher order terms})
\end{equation}
We can see that the sign of the singularity is related to
$\det(A)$, provided that $A$ is nondegenerate. We may always
assume that $A$ is nondegenerate, otherwise we have a degenerate
singularity and we should look at higher order terms. The sign of
the determinant doesn't change sign as we vary $A$ continuously,
otherwise we have problems---thus the sign of $\det$ is a homotopy invariant.
We have only two possibilities since $\GL{n,\RR}$ has two
components. So $\pm\sgn\bigl(\det(A)\bigr)$, depending on whether
the vector fields point in or out (at vertices it points out; etc.).

It sits in $H^k(M,\ZZ)=\ZZ$ since $M$ is a compact oriented
manifold. The obstruction is a number, and it is simply the Euler
characteristic.

In general, the zero locus of a section is generically an
$(n-k)$-dimensional\dots something. In good cases, it is a
manifold. Well, it's an $(n-k)$-dimensional cocycle,
$H_{n-k}\bigl(M,\pi_{k}(F)\bigr)$. We need to assume $M$ is an
oriented manifold. It is very natural to \emph{conjecture} that
by Poincar\'e duality
\begin{equation}
H_{n-k}\bigl(M,\pi_{k}(F)\bigr)\iso H^{k}\bigl(M,\pi_{k}(F)\bigr).
\end{equation}
This would nicely relate many facets of obstructions. 

But we have other characteristic classes. If $G=\U{k}$ or
$\ORTH{k}$, we could construct Stiefel manifolds $V_{k,l}$. We
may define \define{Chern Classes}\index{Characteristic Class!Chern Class}\index{Chern Class} as obstructions in the case of
complex manifolds, and we may also define \define{Stiefel--Whitney Classes}\index{Stiefel--Whitney Class}\index{Characteristic Class!Stiefel--Whitney Class}
for real manifolds.
