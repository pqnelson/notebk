%%
%% lecture10.tex
%% 
%% Made by alex
%% Login   <alex@tomato>
%% 
%% Started on  Wed Sep 28 13:24:51 2011 alex
%% Last update Wed Sep 28 13:24:51 2011 alex
%%
Last time we constructed a principal bundle from any given
locally trivial bundle. We took $H_{k}(F_{b})$ or $H^{k}(F_{b})$
for each fibre, we took some field as the coefficients, and we
had a vector space over each point $b\in B$. This gives us a
locally trivial vector bundle. Locally it looks like $U\times
H_{k}(F)$. This is a special type of \textbf{vector bundle with
  \define{Flat Connection}}, we didn't really explain the notion
of a connection, but we have some notion of a \emph{flat
connection!}\index{Connection!Flat} Given some $U\propersubset B$ neighborhood, we
have the canonical identification
\begin{equation}
H_{k}(F_{b})\iso H_{k}(p^{-1}(U)),
\end{equation}
which is something we don't always have in a fibre bundle, or a
vector bundle. There is no canonical way for translating from
fibre to fibre. \textbf{But} this works \textbf{only} for $\CC$.

\begin{Boxed}{Connections and Fibre Bundles}\index{Connection!and
Fibre Bundles}
This is a confusing concept, so I'd like to review it in a bit
more detail. Koszul's ``Homologie et cohomologie des algebres de
Lie'' (Bulletin de la Soci\'et\'e Math\'ematique {\bf78} (1950) 65--127)
describes the algebraic framework for connections on vector
bundles, which are generalized and considered here.

The idea is a generalization of the geometric notion
of a ``connection.'' Recall its definition:
\begin{defn}
Let $p\colon F\to B$ be a fiber bundle. By a \define{connection}\index{Connection}
on $F$ we will mean a vector bundle splitting $TF = V\oplus H$ where
$V = \ker(Dp)$. The bundle $V$ will be called the \define{vertical tangent bundle}
and $H$ will be called the \define{horizontal tangent
bundle}\index{Tangent Bundle!Horizontal}.
\end{defn}
In this definition, we have $\left.Dp\right|_{H}\colon H\to TB$
be at each point an isomorphism. 

What we are effectively doing, however, is considering a fibre
bundle $p\colon E\to B$ that is locally trivial with fibre
$F$. We construct a finite-dimensional complex vector space
$H_{k}\bigl(p^{-1}(U),\CC\bigr)\iso H_{k}(F,\CC)$, and this is an
isomorphism because $U\subset B$ is taken to be contractible. So
$p^{-1}(U)\iso U\times F\homotopic F$ homotopic, and homology
makes this homotopy equivalence an isomorphism.

This construction gives us ``$H$'' the horizontal space. Due to
flatness, $V$ is trivial. Thus we get a flat connection. See
pages 33--34 of Atiyah and Segal's ``Twisted K-theory and
cohomology'' (\arXiv[math.KT]{math/0510674}) for uses in
$K$-theory. For some more details on this notion, see Schwarz and
Shapiro's ``Twisted de Rham cohomology, homological definition of the integral and `Physics over a ring'\thinspace'' (\arXiv[math.AG]{0809.0086})
pages 8 \emph{et seq.} The notion of \emph{D-modules} generalizes
this entire situation. Hong-Jong Kim's ``Cohomology of Flat
Vector Bundles'' (\emph{Comm. Korean Math. Soc.} {\bf11} (1996), No.2, pp. 391--405)
reviews the cohomology of flat vector bundles nicely.
\end{Boxed}

Why does this work only for $\CC$? Well, we need a field for the
group of coefficients, but only the name changes for the general case: it is a 
\define{Group Bundle}\index{Bundle!Group ---}\index{Group Bundle}.\marginpar{Group Bundle = Fibre is Abelian Group} 
In a vector bundle, we have a fibre over every point, 
and this fibre is a vector space. For a group bundle, the fibre
is an Abelian group. The Abelian property isn't really
necessary. But $F_{b}$ is an Abelian group, so consider
\begin{equation}
E=\bigsqcup_{b\in B}F_{b}
\end{equation}
Given $U\propersubset B$, we may identify
\begin{equation}
p^{-1}(U)=U\times G,
\end{equation}
then we require if $U_{\alpha},U_{\beta}\propersubset B$, then we
should have a transition function
\begin{equation}
\varphi_{\alpha\beta}\colon (U_{\alpha}\cap U_{\beta})\times G\to
(U_{\alpha}\cap U_{\beta})\times G
\end{equation}
So we have a trivial generalization of the notion of vector
bundles. We require continuity, except for discrete groups.

But we get more than a group bundle! We get a group bundle with a
flat connection---well, there is no standard notion of connection
for a group bundle. But a \define{Local Coefficient System}\index{Local Coefficient System} is a
group bundle with flat connection! That is, over small
$U\propersubset B$, we should have a canonical identification
\begin{equation}
F_{b}\iso F_{b'}
\end{equation}
of fibres, where $b,b'\in B$. This preferred isomorphism is part
of the structure. 

These are called such because we can consider homology group of
something with coefficients in a local system. For example, we
may consider the homology $H_{k}(B,H_{l}(F))$ with coefficients
being the homology group of the fibre. This occurs frequently in
spectral sequences.

Lets consider the computation of homology in any space with a
local coefficient system $\mathscr{G}$. Consider
$H_{k}(B,\mathscr{G})$. It's very easy. We consider singular
homology (it's not necessary, we may consider the cellular
homology, or the simplicial homology, or\dots). Singular cubes
are maps
\begin{equation}
\varphi_{k}\colon I^{k}\to B
\end{equation}
and we take a linear combination of them. With what coefficients?
We take the coefficients from some group $G$, so
\begin{equation}
\omega=\sum c_{k}\varphi_{k}
\end{equation}
where $c_{k}\in G$. But we have a \emph{single} group, we have
the local coefficient $\mathscr{G}$. Lets take
$c_{k}\in\mathscr{G}_{b}$ where $b$ is the center of the singular
cube $\varphi_{k}$. Then we should imitate the definition we gave
in the obvious way, define
\begin{equation}
\partial\omega=\sum c_{k}\partial\varphi_{k}
\end{equation}
but here we have a problem. The centers of the faces bounding the
cubes are \emph{not} the center of the bounded cube. What can we
do?

First we may say that we can work only with chains consisting of
small cubes. And for small cubes, the problem disappears, because
when we get small enough cubs we have our canonical
identification of fibres. 

The second remark is the $\partial^{2}=0$. Why? Because the
statement is local. Prove it only for small simplices/cubes, then
we get the usual situation. No problems at all. OK!

Therefore we define homology in the standard way. There is no
difference from our prior considerations; exact homology
sequences, etc., carry over.

Lets consider several places where this concept appears.

\marginpar{Poincar\'e duality for nonorientable manifolds}One of them is Poincar\'e duality for non-orientable\index{Poincar\'e Duality!for Nonorientable Manifolds}
manifolds. We took a manifold $M$, covered it in a complex
$\Delta$, then considered the dual to this complex
$\Delta^{\op}$. This sent
\begin{equation}
\sigma^{k}\mapsto\widetilde{\sigma}^{n-k}
\end{equation}
and we didn't know what to do about the orientation. We can
define a local coefficient system, consider a group $G$ and a
nonorientable manifold $M$. We define a local coordinate system
$\widetilde{G}$ on our manifold. How to do this? We have a notion
of the orientation of a neighborhood of a manifold, really we
have two orientations for each neighborhood $U\propersubset
M$. These orientations form a group 
\begin{equation}
\ZZ_{2}=\{\pm1\}
\end{equation}
and these are all isomorphic. Is it canonically isomorphic? We do
not have a canonical choice for orientation---if we take
$\ZZ_{2}$, then we get a canonical choice of orientation.

If we define a local coefficient in the following way: on a small
piece of the manifold, we have a canonical choice of orientation
since we have a canonical isomorphism of groups. If the manifold
is not oriented, there is a path which transforms the group in an
anti-isomorphic way: $g\mapsto-g$.

We may say that for our situation
\begin{equation}
H_{k}(M,G)=H^{n-k}(M,\widetilde{\mathscr{G}})
\end{equation}
where $\widetilde{\mathscr{G}}$ is the local system of coefficients.
