%%
%% lecture02.tex
%% 
%% Made by alex
%% Login   <alex@tomato>
%% 
%% Started on  Thu Sep 29 08:34:48 2011 alex
%% Last update Thu Sep 29 08:34:48 2011 alex
%%

We will follow Hatcher's book, and Schwarz's \emph{Topology for Physicists}.

The first thing to discuss is topological
spaces\index{Topological Space}\index{Space!Topological}. We have a set
$E$ and a notion of an open set of $E$. We have a collection of
open subsets of $E$, $\{U\}$ where
\begin{subequations}
\begin{equation}
\bigcup U=E
\end{equation}
and
\begin{equation}
\bigcap_{\text{finite}} U\mbox{ is open}
\end{equation}
\end{subequations}
It is not enough to say 
\begin{subequations}
\begin{equation}
\bigcup \mbox{(open)}=\mbox{(open)}
\end{equation}
and
\begin{equation}
\bigcap_{\text{finite}} \mbox{(open)}=\mbox{(open)}
\end{equation}
\end{subequations}
We have closed sets be the complement of open sets. There is a
requirement that $E$ is both open and closed, which then implies
that $\emptyset$ is both open and closed.

We can define a continuous function $f\colon E\to E'$ such that
the preimage of open sets is open, i.e.,
\begin{equation}
f^{-1}(\mbox{open})=\mbox{open}.
\end{equation}

\begin{Boxed}{Functorial view of Topology, Continuous Functions}
\index{Topological Space!and Stuff, Structure, Properties}
\index{Continuous Function!from Stuff, Structure, Properties}
This may seem odd at first why continuous functions obey this
pre-image condition. There are a variety of explanations out
there, but I prefer this explanation. Consider the category
$\Set$. Let\index{$\hom(-,\mathbf{2})$|(}
\begin{equation}
\hom(-,\mathbf{2})\colon\Set^{\op}\to\Set
\end{equation}
be the contravariant power set functor. So in other words, we
have
\begin{equation}
\hom(X,\mathbf{2})=\begin{pmatrix}\mbox{set of indicator}\\
\mbox{functions for subsets}\\
\mbox{of the set $X$}
\end{pmatrix}
\end{equation}
We construct a topology by picking a subset of this collection of
subsets $\hom(X,\mathbf{2})$ which obey the axioms for a
topology. That is, we have $T\subset\hom(X,\mathbf{2})$ be a
topology of $X$. That is to say, $T$ consists of the indicator
functions for open subsets of $X$. It is a structure-type.
A topological space is then $(X,T)$. 

But note that we functor, so we have the immediate question:
\begin{quest}
How does $\hom(X\xrightarrow{\;f\;}Y,\mathbf{2})$ behave?
\end{quest}
If we can answer this question, then we will have some idea of
what a ``topological-space morphism'' would be like. Why? Because
we just restrict focus to the functions preserving the
``topological structure'' $T\subset\hom(X,\mathbf{2})$.

We should recall from our knowledge of category theory that the
functor $\hom(-,B)$ behaves on morphisms in the following manner:
$\hom(-,B)$ maps each morphism $h\colon X\to Y$ to the function
$\hom(h, B)\colon \hom(Y, B)\to\hom(X, B)$ given by $g \mapsto
g\circ h$ for each $g$ in $\hom(Y, B)$.  

What does this mean for our situation? Well, for each $f\colon
X\to Y$ it is mapped to the function 
\begin{equation}
\hom(f,\mathbf{2})\colon\hom(Y,\mathbf{2})\to\hom(X,\mathbf{2})
\end{equation}
given by $f\mapsto h\circ f$ where $h\in\hom(Y,\mathbf{2})$. So
$h$ is really an indicator function of an open subset of
$Y$. This is precisely the same condition as saying the preimage
$f^{-1}(\mbox{open})$ is open.\index{Open Set}\index{$\hom(-,\mathbf{2})$|)}
For a brief introduction to topology using this approach, see Nelson~\cite{nelson}.
\end{Boxed}

Usually we use the Hausdorff condition\index{Hausdorff Condition} that two distinct points
are contained in two disjoint neighborhoods.

There is a topological property of \define{Compactness}\index{Compactness} where
every open covering has a finite subcovering.

There is one more thing that is relevant. What can we do with
equivalence relations on topological spaces? We can consider
equivalence classes\index{Equivalence Classes} $E/\sim$. There is a natural map
\begin{equation}
\pi\colon E\to E/\sim
\end{equation}
What happens if $E$ is a topological space, then we would like
to have $E/\sim$ be a topological space and the map $\pi$ to be
continuous, i.e., the preimage $\pi^{-1}(\mbox{open})$ is
open. We are saying $U\propersubset E/\sim$ is open iff the
preimage $\pi^{-1}(U)$ is open in $E$. If the preimage of an open
set is open, then the preimage of a closed set is closed. We see
for a singleton $a\in E/\sim$ then the preimage $\pi^{-1}(a)$ is
an equivalence class. We have the singletons be closed, so we
require these equivalence classes be closed to avoid pathology.

\marginpar{List of topological spaces}Now why are we so interested in this construction? Because we
want to have a construction of interesting topological spaces. We
have some simple interesting topological spaces. What are they?
First of all, $\RR^{3}$ the space that surrounds us. \marginpar{$\RR^{n}$}More
generally $\RR^{n}$. Another interesting space is a ball
\begin{equation}\index{$\bar{D}^{n}$}
\overline{D}^{n}=\{\vec{x}\in\RR^{n}\lst\|\vec{x}\|\leq1\}
\end{equation}
\marginpar{Closed ball $\bar{D}^{n}$}which is closed of radius 1. The radius doesn't change anything,
balls of different radius are topologically equivalent. For
example $x\mapsto\lambda x$ for $\lambda>0$ is the topological
equivalence. We won't repeat the definition of ``topological
equivalence'' the curious reader may look it up. Another
interesting space is the open ball\index{$D^{n}$}\index{Open Ball}\marginpar{Open ball $D^{n}$}
\begin{equation}
D^{n}=\{\vec{x}\in\RR^{n}\lst\|\vec{x}\|<1\}
\end{equation}
We use the notation $\overline{D}^{n}$ to stress it is the
closure of $D^{n}$. It's an interesting space, perhaps it is
equivalent to $\overline{D}^{n}$? No! Why? Well, we see that
$D^{n}$ is not compact but $\overline{D}^{n}$ is compact, so they
cannot be topologically equivalent.

\begin{wrapfigure}{r}{2in}
  \vspace{-30pt}
  \begin{center}
    \includegraphics{img/lecture2.0}
  \end{center}
  \vspace{-20pt}
\end{wrapfigure}
But is $\RR^{n}$ topologically equivalent to $D^{n}$? Yes, we can
see this for $n=1$, we use the stereographic
projection\index{Stereographic Projection} which
gives us a one-to-one correspondence between
$S^{1}\setminus\{0\}$ and $\RR^{1}$. We can take $n=2$ and
nothing conceptually changes. The same is true for
$S^{n}\setminus\{0\}\iso\RR^{n}$. But we may say that
$S^{n}\setminus\{0\}\iso D^{n}$.

We would like to stress that the $n$-dimensional sphere is
\emph{not} a sphere in $n$-dimensional space. No, it is instead
living in $\RR^{n+1}$. In $\RR^{n}$, the sphere is characterized
by the points $\vec{x}\in\RR^{n}$ satisfying
\begin{equation}
\|\vec{x}\|=1
\end{equation}
which is an $(n-1)$-sphere. We have 
\begin{equation}
\overline{D}^{n}=S^{n-1}\cup D^{n}
\end{equation}
where $D^{n}$ are the interior points and $S^{n-1}$ is the boundary.

Now what we would like to say is that, more or less, all
interesting spaces may be constructed from the simple spaces
$D^{n}$, $\overline{D}^{n}$. We define a very general
construction, namely given a topological space $X$, and a
topological space $Y$, a closed subset $A\propersubset Y$, we'd
like to paste together $X$ and $Y$ along $A$. What does this
mean? We take any continuous map
\begin{equation}
f\colon A\to X
\end{equation}
take the disjoint union $X\sqcup Y$, and then in this disjoint
union introduce an equivalence relation that any $a\in
A\propersubset Y \sim f(a)\in X$ and no other equivalences!

\begin{wrapfigure}{l}{0.5in}
  \vspace{-10pt}
  \begin{center}
    \includegraphics{img/lecture2.1}
  \end{center}
  \vspace{-20pt}
\end{wrapfigure}
\noindent{}We require $a\in A\propersubset Y\sim f(a)\in X$ is the only
nontrivial equivalence. Lets consider some examples. The Mobius
band\index{Mobius Band!Construction of} can be defined in this way: take a rectangle (which is
topologically equivalent to $D^{2}$) and we consider the
equivalence relation that the two arrows are pasted together.

\begin{wrapfigure}{r}{2.1in}
  \vspace{-30pt}
  \begin{center}
    \includegraphics{img/lecture2.2}
  \end{center}
  \vspace{-20pt}
\end{wrapfigure}
We see that a rectangle is equivalent to a disc since both are
convex and stretch the boundary to be a rectangle which permits
us to formally write this equivalence but that won't be necessary.
We can stretch according to the gray lines doodled to the right.

\begin{wrapfigure}{l}{1.05in}
  \vspace{-25pt}
  \begin{center}
    \includegraphics{img/lecture2.3}
  \end{center}
  \vspace{-23pt}
\end{wrapfigure}
So more examples. We take the same rectangle, and paste together
points at the same height. This topologically is equivalent to a
cylinder. This is obvious, as we see in the doodle to the left.


\begin{wrapfigure}{r}{1.5in}
  \vspace{-16pt}
    \includegraphics{img/lecture2.5}
  \vspace{-20pt}
\end{wrapfigure}
But if we take our cylinder, and glue the two ends to each other
without any twisting, what do we get? Well, we have a torus. This
is doodled on the right hand side, very carefully, with colors to
show where we glued the rectangle together.
The red line indicates where we glued the rectangle to obtain a
cylinder, and the blue line indicates where we glued the cylinder
to obtain a torus. Do we really need all this information? Is
there some easier diagram which yields the relevant data? Or are
we forced to become artists to understand the topological
properties of these exotic spaces?


There is a very general construction of something called a
\define{Cell Complex}\index{Cell Complex!generalization of ---|see{Complex}}\index{Cell Complex}\index{Complex!Cellular}, we will first describe it. Take a closed
ball and some topological space $X$. Now we will take any
continuous map
\begin{subequations}
\begin{equation}
f\colon S^{n-1}\to X
\end{equation}
or in other words
\begin{equation}
f\colon\partial\overline{D}^{n}\to X
\end{equation}
\end{subequations}
and then we use the construction we just explained. That is, we
glue a closed ball along its boundary to $X$. We get a new set
\begin{subequations}
\begin{equation}
Y=X\cup\overline{D}^{n}
\end{equation}
or as sets
\begin{equation}
Y=X\sqcup\overline{D}^{n}
\end{equation}
\end{subequations}
The simplest posssible case is when $X$ is just a one point space
\begin{equation}
X=\{a\}
\end{equation}
the boundary of the ball goes to $a$. This is a trivial map.
We see in $n=2$ what do we get with identifying the boundary to
$a$? Look at the stereographic projection backwards.
\begin{center}
\includegraphics{img/lecture2.4}
\end{center}
This general construction, gluing the boundary of closed
$n$-balls to a topological space (starting with $n=0$, i.e., a
set of vertices to begin with), gives us a cell complex.

\begin{defn}
The \define{$n$-Dimensional Cell Complex}\index{Cell Complex!$n$-Dimensional|textbf} can be done inductively
by assuming we have the $(n-1)$-dimensional cell complex denoted
$X^{n-1}$ called the \define{$(n-1)$-Skeleton}\index{Skeleton!of Complex|textbf}, now we have
$k$-copies of $n$-discs and perform the same construction. We end
up with a sequence of skeletons $X^{0}\propersubset
X^{1}\propersubset X^{2}\propersubset\dots$, if we consider
$X^{n}\setminus x^{n-1}=\bigsqcup_{k} D^{n}_{k}$.
\end{defn}
