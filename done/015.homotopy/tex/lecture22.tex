%%
%% lecture22.tex
%% 
%% Made by alex
%% Login   <alex@tomato>
%% 
%% Started on  Sat Dec 31 12:00:40 2011 alex
%% Last update Sat Dec 31 12:00:40 2011 alex
%%

Let $\Sigma$ be a 2-dimensional manifold\index{Manifold}. Let
$\widetilde{\Sigma}$ be the universal cover of $\Sigma$. But
there are only two choices for $\widetilde{\Sigma}$: $S^2$ or
$\RR^2$. For simplicity we will suppose that $\Sigma$ is
compact. We know how to calculate $\pi_{1}(\Sigma)$, the only
thing we need is the statement that $\pi_{1}(\Sigma)$ is finite
in two cases: $\Sigma=S^2$ or $\RP^2$. This is easy looking at
the Abelianization of the fundamental group. In both of these
cases, $\widetilde{\Sigma}=S^2$. We have
\begin{equation}
\pi_{k}(\Sigma)=\pi_{k}(\widetilde{\Sigma})
\end{equation}
for $k\geq2$. Now let us suppose $\pi_{1}(\Sigma)$ is
infinite. Then $\widetilde{\Sigma}$ is not compact. Why? Because
when we look at the covering
\begin{equation*}
\widetilde{\Sigma}\to\Sigma
\end{equation*}
the number of sheets in this covering are the number of elements
in $\pi_{1}(\Sigma)$, which is infinite. Over every disc, we have
an infinite number of discs, which is definitely noncompact. We
have only one chocie for $\widetilde{\Sigma}$. We see then that
\begin{equation}
\pi_{k}(\Sigma)=\pi_{k}(\RR^2)=0
\end{equation}
for $k\geq2$.

We would like to show 
\begin{equation}
\pi_{k}(X\times Y)\iso\pi_{k}(X)\times\pi_{k}(Y).
\end{equation}
It's a one minute proof. If we have a mapping
\begin{equation}
f\colon Z\to X\times Y=\{(x,y)\}
\end{equation}
this map means $(x,y)=f(z)$. This means
\begin{equation}
x=f_{1}(z),\quad\mbox{and}\quad y=f_{2}(z).
\end{equation}
When we apply this to spheroids, everything follows. When we
deform $f(z)$, we deform these two guys. We have, e.g., an
$n$-torus be
\begin{equation}
T^n=(S^1)^n
\end{equation}
so $\pi_{k}(T^n)\iso\pi_{k}(S^1)^n$.
 
\subsection{Relative Homotopy Groups}\index{Homotopy Group!Relative}
\index{Relative Homotopy Group}\index{Reduced Homotopy Group|see{Relative Homotopy Group}}%
\index{Relative Homotopy Group!Construction with Spheroids|(}
We will have a pair of topological spaces $X$, $A$ (so
$A\propersubset X$), and consider $*\in A\propersubset X$. For
simplicity, $A$ and $X$ are connected. We will define a
\define{Relative Homotopy Group} $\pi_{n}(X,A,*)$ or sometimes
$\pi_{n}(X,A)$. We will neglect something in $X$, namely, we
neglect $A$ --- this is the notion of ``relative''.

Recall we defined $\pi_n$ by means of spheroids
\begin{equation*}
(S^n,*)\to(X,*).
\end{equation*}
The homotopy group $\pi_{n}(X,*)$ is then the homotopy classes of
spheroids. This is nice but incomplete. We need to define an
operation. We did this by considering a
spheroid\index{Spheroid!as Map on Cube} as a map on a
cube, generalizing
concatenation\index{Concatenation!Generalization of ---}.

Lets consider something similar for relative homotopy groups. We
introduce \define{Relative Spheroids}\index{Spheroid!Relative}\index{Relative Spheroid} %
$(\bar{D}^n,S^{n-1},*)$ where 
\begin{equation}
S^{n-1}=\partial\bar{D}^{n},
\end{equation}
and relative spheroid is a map
\begin{equation}
(\bar{D}^n,S^{n-1},*)\to(X,A,*).
\end{equation}
This means we have a map
\begin{equation}
f\colon\bar{D}^{n}\to X
\end{equation}
such that
\begin{equation}
f\colon\partial\bar{D}^{n}\to A
\end{equation}
but $f(*)=*$. Such a map is a relative spheroid.

The relative homotopy group $\pi_{n}(X,A,*)$\index{Relative Homotopy Group!in Terms of Relative Spheroids}
is a set of homotopy classes of relative spheroids. It's exactly
the same for ordinary homotopy group, the only difference is we
use relative spheroids. We will discuss the operation later on.

One relation we'd like to note is we have a map
\begin{equation}
\pi_{n}(X,A,*)\to\pi_{n-1}(A,*)
\end{equation}
from the relative homotopy group to the full homotopy group. Why?
For a trivial reason that a relative spheroid
\begin{equation}
f\colon(\bar{D}^{n},S^{n-1},*)\to(X,A,*)
\end{equation}
is really a full spheroid on $A$. Thus $f\colon(S^{n-1},*)\to(A,*)$
is an $(n-1)$-spheroid.
\index{Relative Homotopy Group!Construction with Spheroids|)}

A different formulation of the relative homotopy group.
Recall we considered the space $\Omega$ of all closed loops
starting and ending at $*$. We consider 
\begin{equation*}
\pi_{n}(X,*)=\pi_{n-1}(\Omega,*)
\end{equation*}

\begin{wrapfigure}{r}{1.5in}
  \centering
  \includegraphics{img/lecture22.0}
\end{wrapfigure}\noindent\ignorespaces %
Now what about the relative groups? We have $A\propersubset X$,
consider all paths that are closed modulo $A$. That is to say
\begin{equation}
\Omega(A)=\{f\colon I\to X\mid f(0)=*, f(1)\in A\}
\end{equation}
This sort of path is doodled on the right, where it begins at the
marked point and ends anywhere inside the gray region.
We may consider its homotopy groups.
Thus we may define $\pi_{n}(X,A,*)=\pi_{n-1}(\Omega(A),*)$ where
$*(t)=*$ is the stationary path. Okay, this is a definition. We
see this is a group. Also, for $n\geq3$ we see $\pi_{n}(X,A,*)$
is Abelian. The only problem is that this is not a very good
definition. 

Let us decode this definition. Consider
$\pi_{n-1}\bigl(\Omega(A),*\bigr)$. What is this? We define this
in terms of spheroids as a map of a cube
\begin{equation}
f\colon I^{n-1}\to\Omega(A)
\end{equation}
which sends $\partial I^{n-1}\to*$. What is $\Omega(A)$? It
consists of paths. Our function $f(t_1,\dots,t_{n-1})$ itself is
a path, so really
\begin{equation}
f=f(t_1,\dots,t_{n-1},\tau)\in X.
\end{equation}
What ar ethe conditions on $f$? First of all, the condition is
\begin{equation}
f_{\tau}\colon\partial I^{n-1}\to *(\tau),
\end{equation}
so
\begin{equation}
f(\partial I^{n-1},\tau)=*(\tau).
\end{equation}
Another is that, for $\tau=0$, we have
\begin{equation}
f(\dots,0)=*
\end{equation}
be our marked point, whereas for $\tau=1$ we require
\begin{equation}
f(\dots,1)\in A
\end{equation}
and that's it!

\begin{wrapfigure}{r}{1.5in}
  \vspace{-30pt}
  \centering
  \includegraphics{img/lecture22.1}
  \vspace{-24pt}
\end{wrapfigure}
We will try to reconcile everything. Consider $f\colon I^n\to X$.
For $n=2$, we will have a square and when $\tau=1$ we go to
$A$. This is doodled on the right. When $t=0,1$ we go to $*$ and
when $\tau=0$ we also go to $*$.

\marginpar{This needs to be rewritten for clarity} 
\begin{wrapfigure}{l}{0.75in}
  \vspace{-12pt}
  \centering
  \includegraphics{img/lecture22.2}
  \vspace{-12pt}
\end{wrapfigure}
For the $n=3$ case, we have the top face go to $A$ (it is shaded
grey). This is really what is defined by Hatcher as a relative
spheroid.
But this is already defined. We should prove it is the same.
Here we have the map of a ball, which sends its boundary to
$A$. But here we have some extra stuff, namely, the rest of the
boundary. We identify it witha pouint. So really, this is
equvialent to a ball with a marked point.
Consider
\begin{equation}
\left(I^n,I^{n-1}\times\{1\},(\partial I^{n-1}\times I)\cup(I^{n-1}\times\{0\})\right)\mapsto(\bar{D}^{n},S^{n-1},*)
\end{equation}
since they both are mapped to $(X,A,*)$.
So this identifies the notion of a relative spheroid with
Hatcher's notion of a spheroid. Observe
\begin{equation*}
(\partial I^{n-1}\times I)\cup(I^{n-1}\times\{0\})
\end{equation*}
is a contractible set. When we contract, the upperface is mapped
to $S^{n-1}$ the whole boundary. So our new sense of relative
spheroid agrees with the relative spheroid in the old sense.
