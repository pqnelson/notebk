%%
%% lecture26.tex
%% 
%% Made by alex
%% Login   <alex@tomato>
%% 
%% Started on  Tue Dec 27 20:47:25 2011 alex
%% Last update Tue Dec 27 20:47:25 2011 alex
%%
\subsection{Aside on Principal Fibrations}
Some particular cases of fibrations are called \define{Principal
Fibrations}\index{Principal Fibrations}\index{Fibration!Index}\index{Bundle!Principal}. Let $G$ be 
a topological group (usually a Lie group; for notes on Lie
groups, see Nelson~\cite{nelson2}).  Lets consider a space $E$
where $G$ acts freely\index{Action!Free}\index{Group Action!Free}. What does it mean? Well, for any nontrivial $g\in G$
there are no fixed points, i.e.,
\begin{equation}
gx\not=x\quad\mbox{provided}\quad g\not=e.
\end{equation}
The obvious exception is the identity transformation has fixed
points, but that's a triviality. Although this is unsatisfactory,
but sufficient for us. Given such an action, every orbit $O_{x}$
is in one-to-one correspondence with $G$. We have
\begin{equation}
g\mapsto xg
\end{equation}
using the right action for notational convenience. So
\begin{equation}
G\to O_{x}
\end{equation}
is bijective and continuous. This is a topological equivalence!
Moreover, if $G$ is compact, then every continuous one-to-one
mapping has a continuous inverse. Then every orbit is
topologically equivalent to $G$. We have a map $E\to E/G$ which
is a fibration.
\begin{thm}\index{Section!Existence of a ---!Implies Trivial Principal Fibration}
A principal fibration has a section if and only if it is a
trivial fibration.
\end{thm}
We should additionally assume that $G$ is compact, and $B=E/G$ is
at least locally compact.
Lets suppose we have a section 
\begin{equation}
q\colon B\to E, 
\end{equation}
consider the mapping
\begin{subequations}
\begin{equation}
f\colon B\times G\to E
\end{equation}
defined by
\begin{equation}
f(b,g)=q(b)g
\end{equation}
\end{subequations}
which is continuous. So $F$ is one-to-one and continuous. If
everything is compact, it's a homeomorphism. It's a condition of
triviality for a principal bundle.
Also note
\begin{equation}\index{Section!Local}
\begin{pmatrix}
\mbox{Local Triviality}\\
\mbox{of Bundle}
\end{pmatrix} \implies \begin{pmatrix}\mbox{Existence
of}\\ \mbox{ Local Sections}
\end{pmatrix}.
\end{equation}
We will do the following trick: include existence of lcoal
sections into the definition of a free action.

There is an important case, namely a subgroup $H\propersubset G$
of a topological group. Itacts on the left or on the right, so
lets consider the action
\begin{equation}
(g,h)\mapsto gh
\end{equation}
for some $g\in G$ and $h\in H$. What is the space of orbits? It
is $G/H$, the space of cosets. We have
\begin{equation}
G\xrightarrow{H}G/H
\end{equation}
and it is a principal fibre bundle. We assume compactness
everywhere (otherwise, we need to worry about the existence of
sections, etc.).

Suppose $G$ acts transitively on $B$. Lets take a point $b\in B$,
then we may map $G\to B$ by considering
\begin{equation}
g\mapsto\varphi_{g}(b)
\end{equation}
where
\begin{equation}
\varphi\colon G\to\aut(B).
\end{equation}
But this is precisely the picture we had before. If we denote
$\mathrm{Stab}(b)=H$, then $B$ corresponds to $G/H$.

\subsection{Returning to Stiefel Manifolds}
Now we would like to consider two different cases: the real case,
and the complex case. But we don't really want to talk about
quaternionic case, but everything we say may be repeated for
quaternions. So what are the Stiefel Manifold $V_{n,k}$? It is
precisely $k$ column vectors in $\RR^n$ which are
orthonormal. But
\begin{equation}
V_{n,n}(\RR)=\ORTH{n}
\end{equation}
may be considered as an identity. For the complex case, we see we
get
\begin{equation}
V_{n,n}(\CC)=\U{n}
\end{equation}
the unitary group! The quaternionic case, we also have something
of this kind
\begin{equation}
V_{n,n}(\HH)=\Sp{n}
\end{equation}
the Symplectic group.

We had explained
\begin{equation}
V_{n,n-1}(\RR)=\SO{n},
\end{equation}
and some consideration gives the fact
\begin{equation}
V_{n,n-1}(\CC)=\SU{n}.
\end{equation}
In particular, what we would like to say is that 
\begin{equation}
V_{2,1}(\CC)=\SU{2}.
\end{equation}
So this is a pair of complex numbers $x,y\in\CC$ such that 
\begin{equation}
|x|^{2}+|y|^{2}=1,
\end{equation}
which is a sphere in $\CC^2\iso\RR^4$. So $\SU{2}\iso
S^{3}$.\index{$\SU{2}$!Homeomorphic to $S^3$}

Now it is easy to check that
\begin{equation}
\SO{3}\iso\SU{2}/\ZZ_{2}.
\end{equation}
This may be done in many different ways. One is to consider a
3-dimensional representation of $\SU{2}$. Or we may consider an
action of $\SU{2}$ on Hermitian matrices. We won't go into detail
here. By the way, this fact implies that
\begin{equation}
\SO{3}\iso\RP^{3}
\end{equation}
as topological spaces, and we used this before. Homotopy groups
care about the topological structure of spaces, so we have
\begin{equation}
\pi_{k}\bigl(\SO{3}\bigr)\iso\pi_{k}\bigl(\SU{2}\bigr)
\end{equation}
for any $k\in\NN$.

After these remarks, we want to construct some fibrations. We do
this in two ways. One way take $V_{n,k}$ and maps
\begin{equation}
(e_{1},\dots,e_{k})\mapsto e_{1}.
\end{equation}
Note we could have mapped it to $e_k$, it doesn't really
matter. This gives us a mapping
\begin{equation}
V_{n,k}\to V_{n,1}.
\end{equation}
What's the fibre? Well, we fix one vector $e_1$ and we have
$(k-1)$ vectors orthogonal to it. Thus we have $V_{n-1,k-1}$ be
the fibre. So we have for each $k$ a fibration of this
kind. Observe
\begin{equation}
V_{n,n}\to V_{n,1}
\end{equation}
has fibre $V_{n-1,n-1}$. But we know these guys! It's the
fibration
\begin{equation}
\ORTH{n}\to S^{n-1}
\end{equation}
with fibre $\ORTH{n-1}$, in the $\RR$ case. (For the $\CC$ case
we have $\U{n}\to S^{2n-1}$ with fibre $\U{n-1}$.) We see that
$\ORTH{n}$ acts on $\RR^n$ which preserves the scalar product and
length (likewise describes the action of $\U{n}$ on $\CC^n$). So
it follows every sphere is a quotient
\begin{equation}
\ORTH{n}/\ORTH{n-1}=S^{n-1}
\end{equation}
in the real case, and
\begin{equation}
\U{n}/\U{n-1}=S^{2n-1}
\end{equation}
for the complex case.

We can get information about the connection of $\U{n-1}$, $\U{n}$
if we know the homotopy groups for $S^{2n-1}$. We know for small
$k$ that
\begin{equation}
\pi_{k}(S^{2n-1})=0.
\end{equation}
Thus 
\begin{equation}
\pi_{k}\bigl(\U{n-1}\bigr)\iso\pi_{k}\bigl(\U{n}\bigr).
\end{equation}
We have a similar situation for the orthogonal group, for $k<n$.

\exercises
\begin{xca}
Prove that every graph (one-dimensional cell complex) has trivial
homotopy groups in dimensions $>1$. 

Hint. Every simply connected graph is contractible. (This is true
also for infinite graphs, but to solve the problem it is
sufficient to check this for finite graphs.)
\end{xca}
\begin{xca}
Calculate relative homotopy groups $\pi_{k}(S^n, S^1, ∗)$ where
$k\leq n$, $n\geq3$. Here $S^1$ stand for a circle embedded into
$n$-dimensional sphere $S^n$. 
\end{xca}
\begin{xca}
Let us consider a letter $\Phi$ as three-dimensional object (in
other words we consider $\Phi$ as a small neighborhood of a graph
in $\RR^3$). One can say also that we consider three-dimensional
body $\Phi$ bounded by a sphere with handles
$\partial\Phi$. Calculate relative homotopy groups 
$\pi_{n}(\Phi, \partial\Phi, ∗)$ where $∗\in\partial\Phi$.
\end{xca}
