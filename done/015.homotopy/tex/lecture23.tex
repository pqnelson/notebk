%%
%% lecture23.tex
%% 
%% Made by alex
%% Login   <alex@tomato>
%% 
%% Started on  Sat Dec 31 12:01:46 2011 alex
%% Last update Sat Dec 31 12:01:46 2011 alex
%%

We gave a definition of homotopy groups $\pi_{k}(X,*)$ in various
ways. Using spheroids
\begin{equation}
f\colon(S^k,*)\to(X,*)
\end{equation}
such that $f(*)=*$, or as
\begin{equation}
f\colon I^k\to (X,*)
\end{equation}
such that $f(\partial I^k)=*$. We introduced the notion of
relative homotopy groups on $(X,A,*)$. We can have relative
spheroids
\begin{equation}
f\colon\bar{D}^{n}\to X
\end{equation}
such that
\begin{equation}
f(S^{n-1})\subset A,\quad\mbox{and}\quad f(*)=*.
\end{equation}
This is very nice but it doesn't give a group
structure. Therefore one can consider instead of maps of a ball,
well, maps of a cube. That would be
\begin{equation}
f\colon I^n\to X
\end{equation}
such that
\begin{equation}
f(I^{n-1})\propersubset A,\quad\mbox{and}\quad
f(\partial I^n-I^{n-1})=*.
\end{equation}
This is an equivalent picture, but we see how to form a binary
operation now. This definition may now be formulated as
$\pi_{n-1}\bigl(\Omega(A)\bigr)$.

The first thing to say is this definition is functorial\index{Functoriality}. 
What does this mean? Well, we have 
\begin{equation}
\alpha\colon(X,A,*)\to(Y,B,*)
\end{equation}
such that
\begin{equation}
\alpha\colon X\to Y\quad\mbox{and}\quad \alpha(A)\propersubset B
\end{equation}
and $\alpha(*)=*$. We then have ``by functoriality'' a map
\begin{equation}
\pi_{n}\bigl(\alpha\colon(X,A,*)\to(Y,B,*)\bigr)\quad=\quad \alpha_{*}\colon
\pi_{n}(X,A,*)\to\pi_{n}(Y,B,*).
\end{equation}
Moreover $\alpha_{*}$ is a morphism. We have the functoriality
property that 
\begin{equation}
(\alpha\circ\beta)_{*}=\alpha_{*}\circ\beta_{*}\quad\mbox{and}\quad
(\id{})_{*}=\id{*}.
\end{equation}
These are the functorial properties.

\subsection{Exact Homotopy Sequence of a Pair}
\index{Exact Sequence!of Homotopy Groups}
\index{Homotopy Group!Exact Sequence of ---}

We see that a spherodi in $A$ is definitely a spheroid in $X$. In
other words, functoriality acts on this inclusion
\begin{equation}
i\colon A\into X
\end{equation}
and gives us a morphism
\begin{equation}
i_{*}\colon\pi_{n}(A)\to\pi_{n}(X).
\end{equation}
Note that we do abuse notation slightly, we should write
something like $i_{*,n}$ to indicate we have $\pi_{n}(i)$, i.e.,
it depends on the $n\in\NN_{0}$.

Next we have absolute homotopy groups, we had absolute spheroids.
Now we may consider relative spheroids and relative homotopy
groups
\begin{equation}
\pi_{n}(A,*)\xrightarrow{i_{*}}\pi_{n}(X,*)\to\pi_{n}(X,A,*).
\end{equation}
Why do we have this? Well, any absolute spheroid is-a relative
spheroid. Why? Because the simple reason is $*\in A$. So a
spheroid is just a relative spheroid ``ending at $*$''.

We have one more map, which we already described
\begin{equation}
\pi_{n}(X,A,*)\to\pi_{n-1}(A,*).
\end{equation}
This comes from the fact a relative spheroid
\begin{equation}
f\colon I^n\to X
\end{equation}
can be restricted to $I^{n-1}$, but $f(I^{n-1})\propersubset
A$. This is a spheroid in $A$! So we induce this morphism.

But we also have
\begin{equation}
i_{*}\colon\pi_{n-1}(A,*)\to\pi_{n-1}(X,*).
\end{equation}
Why? Well, we saw this earlier using the inclusion and applying
the functor $\pi_{n-1}$. So we have a sequence. We have at the
end of this
\begin{equation}\label{eq:lec23:stuffNotDefined}
\dots\to\pi_{1}(A,*)\to\pi_{1}(X,*)\to
\underbracket[0.5pt]{\pi_{1}(X,A,*)\to\pi_{0}(A,*)\to\pi_{0}(X,*)}
\end{equation}
where the underlined terms are not really defined. So what to do?
Simple: \emph{define them!} We really have a problem here: we
have no group for $\pi_{0}(X)$, but we have a set. The stuff
underlined in Eq \eqref{eq:lec23:stuffNotDefined} are not groups,
but they are sets. If we consider $\pi_{0}(X)$\index{$\pi_{0}(X)$}, it may be
considered as $S^0\to X$ which maps marked point to marked
point. Please note that $S^0$ consists of two points, one of them
is marked. So what is
\begin{equation}
f\colon (S^0,*)\to(X,*)?
\end{equation}
It is a map of one point, so naively we would expect
$\pi_{0}(X,*)$ to be in one-to-one correspondence with the points
of $X$, right? Wrong: $\pi_{0}$ is the \emph{homotopy classes} of
such mappings, and two points are homotopic if they are on the
same components. So really,
\begin{equation}
\pi_{0}(X,*)=\begin{pmatrix}\mbox{components}\\
\mbox{of $X$}
\end{pmatrix}
\end{equation}
A deformation of elements of $\hom\bigl((S^0,*),(X,*)\bigr)$ is a
path. So we have really $\pi_{0}(X,*)$ be homotopy classes of
these thngs, i.e., of the components of $X$. It is a set.

A relative spheroid
\begin{equation}
f\colon I\to(X,A,*)
\end{equation}
is a path
\begin{equation}
\gamma\colon[0,1]\to X
\end{equation}
such that
\begin{equation}
\gamma(0)=*,\quad\mbox{and}\quad
\gamma(1)\in A.
\end{equation}
So a spheroid is determined where it ends. What is important is
we have a sequence of groups that ends with a sequence of groups
that ends with a sequence of sets.

\begin{thm}
This sequence
\begin{equation}
\dots\to\pi_{n}(A,*)\to\pi_{n}(X,*)\to\pi_{n}(X,A,*)\to\pi_{n-1}(A,*)\to\dots
\end{equation}
is exact.
\end{thm}
\begin{defn}\index{Sequence!Exact|(}
Consider a sequence of groups
\begin{equation}
\dots\xrightarrow{\D_n}A_{n}\xrightarrow{\D_{n-1}}A_{n-1}\xrightarrow{\D_{n-2}}A_{n-2}\to\dots
\end{equation}
it is said to be \define{Exact}\index{Exact Sequence|textbf}
if{}f $\im(\D_{k})=\ker(\D_{k-1})$.
\index{Sequence!Exact|)}
\end{defn}
Observe as a consequence in an exact sequence
\begin{equation}
\im(\D_{k-1}\circ\D_{k})=0.
\end{equation}
Can we say $\im(\D_{k})=\ker(\D_{k-1})$? Well, yes, but the
statement, for any $k$,
\begin{equation}
\im(\D_{k})\subset\ker(\D_{k-1})
\end{equation}
is sufficient to say we have an exact sequence, and
\begin{equation}
\im(\D_{k})\supset\ker(\D_{k-1})
\end{equation}
is necessary. We could write
\begin{equation}
\im(\D_{k})=\D^{-1}_{k-1}(0)
\end{equation}
for exactness conditions.

\begin{rmk}
The definition of exactness remains meaningful if we wok with
sets with marked points. We simply write $*$ for our marked
point, and $\D^{-1}_{k-1}(*)=\im(\D_{k})$, etc.
\end{rmk}

\begin{proof}[Sketch of Proof]
The full proof requires checking 6 terms showing an equality of
sets.\marginpar{\textbf{TODO:} write in the proof}
\end{proof}


\begin{wrapfigure}{l}{0.6in}
  \vspace{-20pt}
  \centering
  \includegraphics{img/lecture23.0}
  \vspace{-12pt}
\end{wrapfigure}
If we have $S^{n-1}\to A$ homotopic to the trivial path, then it
may be extended to a union of nonintersecting $S^{n-1}$, i.e., to
$D^{n}$. Consider $S^{n-1}\times I$, then on the upper part of
the cylinder (if it is homotopic to 0) is $*$, we get what is
doodled on the left.

Consider $\pi_{n-1}(A,*)\to\pi_{n-1}(X,*)$. We can extend the
sphere to a ball. But that means we have
$\pi_{n}(X,A,*)\to\pi_{n-1}(A,*)$, because this extension to a
ball is a relative spheroid. So we have the kernel included in
the image.

The best thing to do is to go home and think about it yourselves:
there are some things so simple, it cannot be explained.
