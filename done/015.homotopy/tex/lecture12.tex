%%
%% lecture12.tex
%% 
%% Made by alex
%% Login   <alex@tomato>
%% 
%% Started on  Wed Dec 28 12:47:36 2011 alex
%% Last update Wed Dec 28 12:47:36 2011 alex
%%
\marginpar{\textbf{TODO:} need to carefully reconsider notation used}
We really did the main job with computing the fundamental
group. We have a set $X$, a 2-dimensional cell $\sigma^2$, we
consider
$X\cup(\sigma^2-0)$.
The intersection is $\sigma^2-0\homotopic S^1$ homotopic. Then
apply van Kampen's theorem, we have generators from $X$ and from
$\sigma^2$, we apply relations from $X$, and relations induced
from the intersection. We have
\begin{equation}
\pi_1(S^1)\to\pi_{1}(X)=\pi_1\bigl(X\cup(\sigma-0)\bigr)
\end{equation}
Note $\sigma^2$ is open, so it has a trivial fundamental
group. When we attach a 2-cell, it gives us a new operation.

Lets consider a $k$-cell: $X\cup\sigma^k$,
$X\cup(\sigma^k-0)$. If $k>2$, we have the intersection give us a
trivial fundamental group. 

\index{Cell Complex!Fundamental Group of ---}\index{Fundamental Group!for Cell Complex}%
How do we calculate $\pi_1(X)$ for some cell complex $X$?
Easy, take the fundamental group of the skeleton of the cell
complex $\pi_1(X^1)$. We have a graph, homotopic to a bouquet\index{Bouquet}
with $k$ circles. This is all if $X$ is connected. Then
\begin{equation}
\pi_1(X^1)\iso\freeGrp{k}
\end{equation}
is the free group with $k$ generators. The Euler characteristic
of the bouquet is
\begin{equation}
\chi(X^1)=1-k.
\end{equation}
We then attach $2$-cells, and each one gives us a relation. Then
higher dimensional cells, and they do nothing! So
\begin{equation}
\pi_{1}(X)\iso\pi_{1}(X^2)
\end{equation}
is a canonical isomorphism. Its generators are given by
$\pi_1(X^1)$ with a relation for each 2-cell. We don't really
know if the group is trivial or not, and we \emph{can't} know
either. For the fundamental group we can get \emph{any} group. We
can add the 2-cells in any particular order to get
relations. Furthermore, we can continue adding 2-cells to get
\emph{any} relation.

What we can really calculate is the
\define{Abelianization}\index{Fundamental Group!Abelianization}\index{Abelianization!of Fundamental Group}
of $\pi_1$ which is $\pi_1/[\pi_1,\pi_1]$. This means we have
generators and we have relations. If $a_i$ are our generators,
then we have the additional relation
\begin{equation}
a_ia_j=a_ja_i
\end{equation}
So we have an Abelian group. This is an invariant which can be
calculated efficiently. It is the first homology
group\index{Homology!First Group}\index{Homology!from Fundamental Group}.

\begin{ex}[$g$-Handled Sphere]\marginpar{TODO: draw pictures on pp.\ 33--34}
A sphere with $g$-handles. We have the same picture for a single
hand repeated several times. We have $2g$ edges and a single
2-cell. So if we want to look at the fundamental group, look at
what happens. We have $a_1$, $b_1$, \dots, $a_g$, $b_g$ and only
1 vertex. The fundamental group is a group with $2g$
generators. We should have 1 relation because we have a single
2-cell. We delete one point and consider a loop around the
deleted point. It is a circle which cannot be contracted, and
thus has the relation
\begin{equation}
a_1b_1(a_{1}^{-1})(b_{1}^{-1})(a_2b_2)(a^{-1}_{2})(b^{-1}_{2})=1
\end{equation}
in the case $g=2$. But we still do not know, maybe this group is
trivial. We go to Abelianization----all generators commute. But
then this relation becomes trivial! So in this case,
Abelianization gives us a free Abelian group with
$2g$-generators. We accidentally have a theorem.
\end{ex}

\begin{thm}
The number of handles on a surface is a topological invariant.
\end{thm}

\begin{ex}
Consider a sphere with $g$-handles and a hole. We have more
generators. For $g=2$, we have $a_1$, $b_1$, $a_2$, $b_2$,
$c$. We have a relation of the form
\begin{equation}
a_1b_1(a_{1}^{-1})(b_{1}^{-1})(a_2b_2)(a^{-1}_{2})(b^{-1}_{2})c=1
\end{equation}
We see we get a free group. We included a generator that is
\emph{not} necessary: we may express $c$ in terms of $a_i$,
$b_j$. So we end up with $2g$ generators, no relations, and thus
a free group.
\end{ex}

\subsection{Knots}

\begin{wrapfigure}{r}{12pc}
  \vspace{-20pt}
  \includegraphics{img/lecture12.5}
\end{wrapfigure}
We would like to consider the topological classification of
knots. We consider $\RR^3$ (or $S^3$, for us it'd be
equivalent). We consider a subset of $\RR^3$ topologically
equivalent to a circle. So a knot may be doodled on the
right. But this means all knots are topologically equivalent to
$S^1$. Consider two knots 
\begin{equation}
K\propersubset S^3,\quad\mbox{and}\quad L\propersubset S^3
\end{equation}
If we can find a homeomorphism
\begin{equation}
\varphi\colon S^3\to S^3
\end{equation}
such that
\begin{equation}
\varphi(K)=L
\end{equation}
then two knots are \define{Isotopic}\index{Knot!Isotopic|textbf}\index{Isotopic Knot}.

We may consider $\pi_1(\RR^3-K)$, or $\pi_1(S^3-K)$ since
$S^3=\RR^3\cup\infty$ through stereographic projection. From van
Kampen's theorem, we have
\begin{equation}
\pi_1(\RR^3-K)\iso\pi_1(S^3-K)
\end{equation}
be a knot invariant\index{Invariant!Knot}. 
We see
\begin{equation}
S^3-K\iso(\RR^3-K)\cup D^3
\end{equation}
where $D^3$ is an open ball ``near infinity''. By van Kampen's
theorem, the intersection ($D^3$) is trivial, so we have
\begin{equation}
\pi_1(\RR^3-K)\iso\pi_1(S^3-K)
\end{equation}
which is an isomorphism. (Pop quiz: is it canonical?)

We would like to introduce the notion of a \define{Link}\index{Link}
which is a disjoint union of knots (i.e., of topological
circles). For example {\centering\includegraphics{img/lecture12.6}}
is a link, but it is a trivial link. But \includegraphics{img/lecture12.7}
is two linked circles. These two pictures are topologically
different. We can take
\begin{equation*}
\pi_1(S^3-\mbox{link})
\end{equation*}
and this will be the invariant of a link. If the invariant
differs for two links, we have two topologically inequivalent links.
