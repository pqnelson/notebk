%%
%% lecture05.tex
%% 
%% Made by alex
%% Login   <alex@tomato>
%% 
%% Started on  Mon Dec 26 20:28:43 2011 alex
%% Last update Mon Dec 26 20:28:43 2011 alex
%%

Remember we consider $\hom(A,X)$ the maps of topological spaces
$A$, $X$. We will use the notation\index{$\Omega_{A}(-)$}
\begin{equation}
\hom(A,X)=\Omega_{A}(X),
\end{equation}
we would like to stress that $\Omega_{A}(-)$ is a functor. So
\begin{equation}
\Omega_{A}\left(f\colon X\to X'\right)\quad=\quad
\Omega_{A}(f)\colon\Omega_{A}(X)\to\Omega_{A}(X')
\end{equation}
which amounts to composing
\begin{equation}
A\xrightarrow{\varphi}X\xrightarrow{f}X',
\end{equation}
i.e., $f\circ\varphi$ for all $\varphi\in\Omega_{A}(X)$\index{$\Omega_{A}(-)$}. The
identity morphism, and composition of morphisms, are preserved
under action by a functor. So for every map $f\colon X\to X'$ we
have a map of ``spaces of maps''
\begin{equation}
\Omega_{A}(f)\colon\Omega_{A}(X)\to\Omega_{A}(X').
\end{equation}
What are the homotopy classes?

They are merely components of the function space\index{Space!Function}\index{Function Space}. We may say the
following: a set of homotopic classes
\begin{equation}
\homotopyClass(A,X)=\begin{pmatrix}\mbox{set of}\\
\mbox{components}\\
\mbox{of }\Omega_{A}(X)
\end{pmatrix} = \pi_{0}\bigl(\Omega_{A}(X)\bigr)
\end{equation}
where $\pi_{0}$\index{$\pi_{0}$|textbf} is an assignment to each topological space its
set of connected components, but it is also a functor! If
\begin{equation}
\psi\colon Z\to Z'
\end{equation}
then
\begin{equation}
\pi_{0}(\psi)\colon\pi_{0}(Z)\to\pi_{0}(Z').
\end{equation}
That is trivial. What can we say? We can say if
\begin{equation}
f\colon X\to X'
\end{equation}
and
\begin{equation}
\Omega_{A}(f)\colon\Omega_{A}(X)\to\Omega_{A}(X')
\end{equation}
then
\begin{equation}
\pi_{0}\bigl(\Omega_{A}(f)\bigr)\colon\pi_{0}\bigl(\Omega_{A}(X)\bigr)\to\pi_{0}\bigl(\Omega_{A}(X')\bigr).
\end{equation}
Of course, functoriality is completely irrelevant here.

If $\varphi_{0}\homotopic\varphi_{1}$ homotopic, then
$f\circ\varphi_{0}\homotopic f\circ\varphi_{1}$ homotopic
too. What did we get? A remark that if we have a map
\begin{equation}
f\colon X\to X'
\end{equation}
then we obtain a mapping
\begin{equation}
\homotopyClass(A,X)\to\homotopyClass(A,X')
\end{equation}
but that's a triviality. Is this map a one-to-one correspondence,
a bijection? We can construct a map
\begin{equation}
g\colon X'\to X
\end{equation}
and we induce
\begin{equation}
\homotopyClass(A,X')\to\homotopyClass(A,X).
\end{equation}
We would like it to compose with $f$ to give the identity. We can
require, of course, 
\begin{equation}
f\circ g=\id{X'}\quad\mbox{and}\quad
g\circ f=\id{X}
\end{equation}
but that's too much. We instead require
\begin{equation}
f\circ g\homotopic\id{X'}\quad\mbox{and}\quad
g\circ f\homotopic\id{X}
\end{equation}
both homotopic. Thus we get
\begin{equation}
\homotopyClass(A,X)=\homotopyClass(A,X').
\end{equation}
We have obtained a classification of homotopy equivalent maps.

We may do something a little bit different. We may take\index{$\Omega^{A}(-)$}
\begin{equation}
\hom(X,A)\eqdef\Omega^{A}(X)
\end{equation}
which is a contravariant functor. If we have
\begin{equation}
X'\xrightarrow{f}X,\quad\mbox{and}\quad X\xrightarrow{\varphi}A
\end{equation}
then we may consider
\begin{equation}
\varphi\circ f\colon X'\to A.
\end{equation}
We have, if $X\homotopic X'$ homotopic, we may identify
\begin{equation}
\homotopyClass(X,A)=\homotopyClass(X',A).
\end{equation}
If $X\homotopic X'$ homotopic, and $Y\homotopic Y'$ homotopic,
then $\homotopyClass(X,Y)=\homotopyClass(X',Y')$.

One more definition. We would like to ask: when $A\propersubset
X$ is a subset homotopically equivalent to the whole set?
Definitely we have a map $\iota\colon A\into X$ but we should have a map in the
opposite direction
\begin{equation}
f\colon X\to A.
\end{equation}
We require two things.
First that $\iota\circ f\homotopic\id{X}$. Second that
\begin{equation}
f\circ \iota\homotopic\id{A}
\end{equation}
is homotopic. Did we say anything new? Nothing! Now we will
require more and get a definition that implies homotopy
equivalent. We require
\begin{equation}
f\circ\iota=\id{A}.
\end{equation}
Then $f$ is called a
\define{Retraction}\index{Retraction|textbf}. By the way the
statement that a retraction exists is a non-trivial statement. It
is impossible to retract $[0,1]$ to its boundary $\{0,1\}$
without tearing. We should have a family of maps 
\begin{equation}
f_{t}\colon X\to X
\end{equation}
where $f_{0}=\id{X}$ and $f_{1}=\iota\circ f=f$. We are a little
bit sloppy here since 
\begin{equation}
f\colon X\to A\quad\mbox{and}\quad f_{0}\colon X\to X
\end{equation}
so to be completely precise
\begin{equation}
f_{1}(x)=(\iota\circ f)(x).
\end{equation}
So what does it mean? We require our retraction to be a
\define{Deformation Retraction}\index{Deformation Retraction|textbf}\index{Retraction!Deformation}
which is a very typical case of homotopy equivalence. This means
$A\propersubset X$ and $A\homotopic X$ homotopic. Now, examples!

Consider the letter ``P''. It is clear this guy is homotopically
equivalent to ``O'', i.e. $\mbox{P}\homotopic\mbox{O}$
homotopic. Let us first note that
\begin{equation}
\mbox{P}\homotopic\mbox{D}
\end{equation}
homotopic, and
\begin{equation}
\mbox{D}\iso\mbox{O}
\end{equation}
homeomorphic, thus
\begin{equation}
\mbox{P}\homotopic\mbox{O}
\end{equation}
homotopic. (Homotopic equivalence is weaker than topological
equivalence.)

\begin{wrapfigure}{l}{0.69in}
  \vspace{-14pt}
  \includegraphics{img/lecture4.8}
  \vspace{-24pt}
\end{wrapfigure}
Let us suppose we have a cell complex $X$. First of all, we
have $X^{k}\propersubset X^{k+1}$ for the $k$-dimensional and
$(k+1)$-dimensional skeletons. Suppose we deleted one point from
ever $k$-dimensional cell. Then $X^{k}$ is a deformation retract
of $X^{k+1}-\{\mbox{deleted points}\}$. We can work in every cell
separately. We can stay in one of these points, blow it up into a
larger hole, and the boundary remains in tact.

Gradually everything goes to the boundary. This is not a very
rigorous explanation. A rigorous one is available. A cell complex
comes from a ball of dimension $(k+1)$. If we remove a point
inside this ball, it is the same as a $k$-dimensional sphere
multiplied by an interval $\bar{D}^{k+1}-0=S^{k}\times[0,1)$.

\begin{thm}
If $A\propersubset X$ closed and it is ``good enough'' (not
pathological) and $A$ is contractible (i.e., homotopically
equivalent to a point), then $X\homotopic X/A$ homotopic.
\end{thm}
