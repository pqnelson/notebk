%%
%% lecture20.tex
%% 
%% Made by alex
%% Login   <alex@tomato>
%% 
%% Started on  Fri Dec 30 14:37:24 2011 alex
%% Last update Fri Dec 30 14:37:24 2011 alex
%%
Last time we gave a definition of $\pi_{2}$, the second homotopy
group of a space with a marked point. We can generalize to
$\pi_{n}(X,*)$ by considering
\begin{equation}
f\colon I^n\to X
\end{equation}
and requiring
\begin{equation}
f(\partial I^{n})=\{*\}.
\end{equation}
This guy is called an $n$-dimensional \define{Spheroid}\index{Spheroid|textbf}. We can
define the concatenation of spheroids
\begin{equation}
h=f*g
\end{equation}
by taking the domain and splitting it up (as we did for
curves). So
\begin{equation}
h(t_1,\dots,t_n) = \begin{cases}
f(2t_1,t_2,\dots,t_n) & 0\leq t_1\leq\frac{1}{2}\\
g(2t_1-1,t_2,\dots,t_n) & \frac{1}{2}\leq t_1\leq1.
\end{cases}
\end{equation}
Here $\pi_{n}(X,*)$ is the set of homotopy classes of spheroids
--- if we can deform one spheroid to another while remaining a
spheroid throughout deformation.

There is another definition which may be given through
induction. We have $\Omega$ be the sapce of paths beginning and
ending at $*$. We inductively define
$\pi_{n}(X,*)=\pi_{n-1}(\Omega,*)$. If $f\in\pi_{n-1}(\Omega,*)$,
then we see it is
\begin{equation}
f\colon I^{n-1}\to\Omega
\end{equation}
and therefore we write
\begin{equation*}
f_{\tau}(t_1,t_2,\dots,t_{n-1})=f(t_1,\dots,t_{n-1},\tau)
\end{equation*}
This is an $n$-dimensional spheroid\index{Spheroid!$n$-dimensional ---}. We see this is a
correspondence between $(n-1)$-dimensional spheroids in $\Omega$
and $n$-dimensional spheroids in $X$.
We should check concatenation is preserved. We can take
concatenation with any coordinate, so use $\tau$ and we're done.

We can give a third definition. Namely if we write the spheroid
as a map
\begin{equation}
I^{n}/\partial I^{n}\to X
\end{equation}
taking the boundary to the marked point $*\in X$. We may say
$\pi_{n}(X,*)$ is the homotopy group of spheroids; it's clearer
conceptually, but the operation is ambiguous.

The first thing to state is $\pi_{n}$ is Abelian for
$n\geq2$. The second thing is that $\pi_{n}$ is a functor. If
$f\colon X\to Y$ is a map of topological spaces such that $f(*)*$
then 
\begin{equation}
\pi_{n}\bigl(f\colon X\to Y\bigr)\quad=\quad
f_{*}\colon\pi_{n}(X,*)\to\pi_{n}(Y,*) 
\end{equation}
and it has functorial properties
\begin{equation}
(f\circ g)_{*}=f_{*}\circ g_{*},\quad\mbox{and}\quad\id{*}=\id{}
\end{equation}
as desired.

\begin{thm}
If $(X,*)\homotopic(Y,*)$ homotopy equivalent, then
$\pi_{n}(X,*)\iso\pi_{n}(Y,*)$ is an isomorphism of groups for
all $n$.
\end{thm}
\begin{proof}
We have $f\colon X\to Y$ and $g\colon Y\to X$ such that
\begin{equation}
f\circ g\homotopic\id{Y}
\end{equation}
and
\begin{equation}
g\circ f\homotopic\id{X}
\end{equation}
are homotopic. Then we may write $g_{*}\circ
f_{*}=\id{\pi_{n}(X)}$ and $f_{*}\circ
g_{*}=\id{\pi_{n}(Y)}$. Thus we may speak of $\pi_{n}$ as an invariant.
\end{proof}
There is a repetition: everything we did for $\pi_{1}$ we do for $\pi_{n}$.

Suppose $X$ is connected. Then
\begin{equation}
\pi_{n}(X,*)\iso\pi_{n}(X,\widetilde{*})
\end{equation}
is isomorphic, but it is not a canonical isomorphism: there are
many. Lets consider the $n=2$ proof, we may draw
pictures. Everything is very simple. We have a spheroid
\begin{equation}
f\colon I^2\to X,\quad f(\partial I^2)=*.
\end{equation}

\begin{wrapfigure}{l}{1.5in}
  \vspace{-12pt}
  \centering
  \includegraphics{img/lecture20.0}
  \vspace{-18pt}
\end{wrapfigure}\noindent\ignorespaces
We want to create a spheroid\index{Spheroid!Changing Base Point} which has the property 
that the boundary goes to another marked point
$\widetilde{*}$. We take a larger square, as doodled on the left,
and embed our spheroid into it. The ``bonus space'' is just
concatenation with the trivial spheroid, mapping everything to
the marked point. One could also think of this as mapping the
boundary ``thicker''\footnote{I suppose the technical term is
  ``adding a collar'' to our boundary region, if one prefers this
  exotic terminology.}. This is a bunch of paths connecting $*$
to $\widetilde{*}$, which are doodled in light gray. Our
extended version of $f$ is constructed in a nonunique way. This
$\alpha$ connects $*$ to $\widetilde{*}$, lets abuse notation to
write
\begin{equation}
\alpha\colon\pi_{n}(X,*)\to\pi_{n}(X,\widetilde{*}).
\end{equation}
We could consider $\alpha^{-1}$, or better
\begin{equation}
\beta(\tau)=\alpha(1-\tau)
\end{equation}
which goes in the opposite direction. So we may use it to
construct a map
\begin{equation}
\beta\colon\pi_{n}(X,\widetilde{*})\to\pi_{n}(X,*).
\end{equation}
Now we may state, at the level of $\pi_{n}$, we have
$\alpha\circ\beta=1$ and $\beta\circ\alpha=1$. Why? Because
\begin{subequations}
\begin{equation}
\alpha\circ\beta\homotopic\begin{pmatrix}\mbox{trivial}\\\mbox{path}
\end{pmatrix}
\end{equation}
homotopic, and
\begin{equation}
\beta\circ\alpha\homotopic\begin{pmatrix}\mbox{trivial}\\\mbox{path}
\end{pmatrix}
\end{equation}
\end{subequations}
homotopic too.

\begin{wrapfigure}{r}{3.35in}
  \vspace{-20pt}
  \centering
  \includegraphics{img/lecture20.1}
\end{wrapfigure}
We should note that we have $\alpha(f\circ g)=\alpha(f)\circ\alpha(g)$, which
we prove with a series of pictures doodled to the right.
These isomorphism are quite nontrivial. We've seen for $n=1$
they're nontrivial. In particular we may take the marked points
to be equal $*=\widetilde{*}$. By taking a nontrivial path, we
get an isomorphism
\begin{equation}
\alpha\colon\pi_{n}(X,*)\to\pi_{n}(X,*).
\end{equation}
We may say $\alpha(f)=\alpha\circ f\circ\alpha^{-1}$, so this is
the case for $*=\widetilde{*}$, we may write this for any
case. But then $\alpha^{-1}$ is not very well defined. We've
shown that $\pi_{1}(X,*)$ acts on $\pi_{n}(X,*)$ by means of
automophisms. So we have
\begin{equation}
\pi_{1}(X,*)\to\aut\bigl(\pi_{n}(X,*)\bigr)
\end{equation}
Did we prove this? Not really. We proved there exists a map, but
we need to prove it is a morphism. The proof, however, is easy.
So what did we prove? We proved if $X$ is connected, then
$\pi_{n}(X,*)$ does not depend on the marked point $*$.


Lets consider the homotopy class
\begin{equation}\index{Orbit!in Homotopy Groups}
\homotopyClass(S^{n},X)=\{\mbox{orbits of $\pi_{1}(X,*)$ in $\pi_{n}(X,*)$}\}
\end{equation}
provided $X$ is connected. This is merely a structureless
set. The proof of this is very simple. First there is a natural
map
\begin{equation}
\pi_{n}(X,*)\to\homotopyClass(S^{n},X),
\end{equation}
the only difference is we fix a point in the domain---this map is
surjective (that's obvious). Because there is a map
\begin{equation}
f\colon S^{n}\to X
\end{equation}
such that $*\mapsto f(*)=\widetilde{*}$. Is this a spheroid?
No. But we may take another path $\alpha$ from $*$ to $*$, so we
can consider $\alpha(f)$. We explained how to do this. The remark
is $\alpha(f)\homotopic f$ homotopic, as a map of spheres. We
should look at the kernel of
\begin{equation}
\pi_{n}(X,*)\to\homotopyClass(S^{n},X).
\end{equation}
It is clear $\alpha(f)\homotopic f$ homotopic as spheroids, or
more precisely as a map of spheres.



