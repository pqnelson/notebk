\phantomsection
\section*{Preface}
\addcontentsline{toc}{section}{Preface}

These are notes about elementary linear algebra, roughly at the level of
Math 22A at UC Davis. Our presentation is heavily inspired from Dr
Andrew Waldron's 2007 Spring course. Later Dr Waldron and colleagues
wrote their own textbook~\cite{waldron}.

We try to keep discussion informal and conversational, we will delimit
examples and definitions and theorems and whatnot. Towards this end, we
will use ``paragraph numbers'' with some kind of label summarizing the
paragraph chunk or mathematical register [``Theorem'', ``Definition'',
``Example'', etc.]. Propositions are statements which are either true or
false, and usually have a proof following their statement. Proofs end
with the symbol ``\qedsymbol'' called the Halmos tombstone, we read them
as ``QED'' [Latin \textit{quod erat demonstrandum}, ``that which was to
be proven'']. These are all variations on a proposition: 
\begin{description}
\item[Lemma] refers to a ``helper proposition'' used to prove a tricky
step in a later proposition, modern programmers might call lemmas
``private propositions'';
\item[Theorem] refers to an important proposition;
\item[Corollary] refers to a proposition which is an immediate
consequence of a previous proposition;
\item[Example] is a proposition asserting an object is an instance of a
gadget, or satisfies some property.
\end{description}
\medbreak\noindent%
On the other hand, there are some registers which are not propositions:
\medbreak
\begin{description}
\item[Definition] introduces a new property, a new gadget, or a new
object [or function];
\item[Remark] clarifies or comments a certain aspect of a proposition or definition;
\item[Problem] statements which drive the conversation by explicitly
providing motivation;
\item[Puzzle] statements which motivate future sections.
\end{description}

Examples and exercises are randomly generated using Lisp scripts, or
from various cited sources. The reader is encouraged to work their way
through the excercises found in, say, Kolman and Hill~\cite{kolman}.

We rely on structured derivations to explain derivations involving
chains of equations. It's explained a little in appendix A, but for the
most part it looks like
\begin{calculation}
  A
\step*{hint why $A=B$}
  B
\step*{hint why $B=C$}
  C
\step*{hint why $C=D$}
  D
\end{calculation}
From which we can conclude $A=B=C=D$ are all equal to each
other. Usually we are trying to prove $A=D$.

Occasionally we may have discussions of difficult or subtle points. We
reserve them in ``dangerous bend'' paragraphs, which begin with
\begin{center}
  \dbend\smallbreak
\end{center}
The font is smaller for such discussions, and make be skipped on first
reading.

Lastly, I have haphazardly added exercises and problems
throughout. ``Problems'' are scattered wherever I think they might be
fun or useful. ``Exercises'' are consolidated in their own subsection.
\textsc{Note:} some of the exercises are redundant --- in the sense
that, some exercise in [say] \S5 will ask you to prove something which was
proven in [say] \S3. This was originally accidental, but I realized this
is useful to ``spaced repetition'' to reinforce your understanding of
various proofs and concepts.