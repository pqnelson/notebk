
\subsection{Solution Sets}

\N{In 3-dimensions}
Recall, in $\RR^{3}$ (three-dimensional real space), we describe a plane
using a linear equation
\begin{equation}
ax + by + cz = d
\end{equation}
for some constants $a,b,c,d\in\RR$. This is a useful geometric fact,
along with ``Two planes intersect each other in a line''.

For a system of 3 linear equations in 3 unknowns,
\begin{equation}
\mat{A}\vec{x}=\vec{b}
\end{equation}
has the following possible outcomes:
\begin{enumerate}
\item \textsc{No solutions exist.} Some of the equations are
  contradictory like $x=1$ and $x=0$, which means no solution exists.
\item \textsc{Unique solution exists.} This we have seen before, when
  $\mat{A}$ is invertible.
\item \textsc{Line.} Two of the equations are multiples of each other,
  so we really have a system of 2 \emph{distinct} linear equations in 3
  unknowns. Each linear equation describes a plane. The solution to both
  equations lie in the intersection of the corresponding planes. We end
  up with our solution involving a free parameter $\lambda$, and it
  would look like $\vec{x}(\lambda) = \vec{x}_{0} + \lambda\vec{m}$.
\item \textsc{Plane.} All three equations are multiples of each
  other. In this case, we have 1 distinct linear equation, which
  describes a plane. Any point on the plane is a solution to our system
  of equations. We describe our solutions with 2 free parameters,
  $\lambda_{1}$ and $\lambda_{2}$, and the generic solution looks like
  $\vec{x}(\lambda_{1},\lambda_{2}) = \vec{x}_{0} + \lambda_{1}\vec{x}_{1}+\lambda_{2}\vec{x}_{2}$.
\item \textsc{All of $\RR^{3}$.} The matrix $\mat{A}$ is the zero
  matrix \emph{and} the constant vector $\vec{b}$ is the zero
  vector. Then every point in $\vec{x}\in\RR^{3}$ satisfies
  $\mat{0}_{n\times n}\vec{x}=\vec{0}$.
\end{enumerate}

\M
More generally, for a sytem of equation in $k$ unknowns, there are $k+2$
possible outcomes (corresponding to the number of free parameters in the
solution). 