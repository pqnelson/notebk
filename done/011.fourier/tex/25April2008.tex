%%
%% 25April2008.tex
%% 
%% Made by Alex Nelson
%% Login   <alex@tomato>
%% 
%% Started on  Sun Dec 21 12:40:34 2008 Alex Nelson
%% Last update Sun Dec 21 12:40:34 2008 Alex Nelson
%%
%% We are going to start investigating the regular
%% Sturm-Liouville problem

Let us first consider an example of a regular
Sturm-Liouville problem:
\begin{ex}
Heat diffusion through an inhomogeneous thin rod is
described by
\begin{equation}
\omega(x)\partial_{t}u(x,t) =
\partial_{x}[r(x)\partial_{x}u(x,t)] + p(x)u(x,t)
\end{equation}
with non-uniform material properties, where $r(x)$ is the
speed of propagation of heat, $p(x)$ is the source of heat
at the point $x$, but specifically we require $r(x)$ and
$\omega(x)$ be positive, and $u(x,t)$ is the temperature at
point $x$ and time $t$.
\end{ex}

We can solve this using the seperation of
variables\index{Seperation of Variables!Regular Sturm-Liouville Problem}
so we have $u(x,t)=X(x)T(t)$ and plug this back into the
equation to find
\begin{equation}
\omega(x)X(x)T'(t)=T(t)\partial_{x}[r(x)X'(x)]+p(x)X(x)T(t)
\end{equation}
we divide both sides by $\omega(x)X(x)T(t)$ and set it equal
to some constant $-\lambda$:
\begin{equation}
\frac{T'(t)}{T(t)} = \frac{\partial_{x}[r(x)X'(x)]}{\omega(x)X(x)}+\frac{p(x)}{X(x)}=-\lambda.
\end{equation}
We get two equations
\begin{equation}
T'+\lambda T=0\Rightarrow T(t)=T_{0}e^{-\lambda t}
\end{equation}
where $T_{0}$ is a constant dependent on the initial
conditions, and
\begin{equation}\label{eq:25April2008:sturmLiouville}
\partial_{x}[r(x)X'(x)]+p(x)X(x)+\lambda\omega(x)X(x)=0.
\end{equation}
Solving this equation is equivalent to solving the
Sturm-Liouville problem. Similarly, if we try to solve the
wave equation we get to boil it down to the same problem.

\begin{defn}\index{Differential Operator!Linear}
Let $L$ be a map
\begin{equation*}
L: C^{2}(a,b)\to L^{2}(a,b)
\end{equation*}
be a \textbf{linear differential operator} given by
\begin{align*}
L(f) &= [rf']' + pf\\
&= rf''+r'f' + pf
\end{align*}
where $r(x)$, $r'(x)$, $p(x)$ are all continuous and
real-valued on $[a,b]$, and $r(x)>0$ on $[a,b]$.
\end{defn}

\begin{rmk}
From \eqref{eq:25April2008:sturmLiouville} it is really
equivalent to
\begin{equation}
L(f) + \lambda\omega(x)f(x)=0
\end{equation}
We can verify that $L$ is linear, or in other words
\begin{equation}
L(c_1f+c_2g) = c_1L(f)+c_2L(g)
\end{equation}
where $c_1,c_2$ are real constants, $f,g\in L^{2}(a,b)$. So
if $f,g$ satisfy the equation, then so does $c_1f+c_2g$.
\end{rmk}
So really we have $L(f)=-\lambda\omega(x)f(x)$. This looks
remarkably like an eigenvalue problem from Linear Algebra
(ahem hint hint, wink wink, nudge nudge). If we can
show that $L$ is adjoint, then we can show that the
eigenfunctions $f$ form an orthonormal basis.

\textbf{Question:} What is the adjoint of $L$? To find out
we need to find what $\<L(f),g\>$ is, and see what
$\<f,L(g)\>$ is.

Let $f,g\in C^{2}(a,b)$, then
\begin{subequations}
\begin{align}
\<L(f),g\> &= \int^{b}_{a}\left(\partial_{x}[r(x)f'(x)]+p(x)f(x)\right)\overline{g(x)}dx\\
&=\int^{b}_{a}\partial_{x}[r(x)f'(x)]\overline{g(x)}dx + \int^{b}_{a}p(x)f(x)\overline{g(x)}dx
\end{align}
\end{subequations}
Now we can integrate by parts to find
\begin{align}
\int^{b}_{a}\partial{x}[r(x)f'(x)]\overline{g(x)}dx&=r(x)f'(x)g(x)|^{b}_{a}
- \int^{b}_{a}r(x)f'(x)\overline{g'(x)}dx
\end{align}
So we can put this back in to find
\begin{align}
\<L(f),g\> &= r(x)f'(x)g(x)|^{b}_{a}-\int^{b}_{a}r(x)f'(x)\overline{g'(x)}dx+\int^{b}_{a}p(x)f(x)\overline{g(x)}dx
\end{align}
By integration by parts again we find
\begin{multline}
 r(x)f'(x)\overline{g(x)}|^{b}_{a}-\int^{b}_{a}r(x)f'(x)\overline{g'(x)}dx+\int^{b}_{a}p(x)f(x)\overline{g(x)}dx\\
=r(x)f'(x)\overline{g(x)}|^{b}_{a}-r(x)\overline{g'(x)}f(x)|^{b}_{a}-\int^{b}_{a}[r(x)\overline{g'(x)}]'f(x)dx+\int^{b}_{a}p(x)f(x)\overline{g(x)}dx
\end{multline}
We can clean this up a little to find that
\begin{equation}
\<L(f),g\> =
\int^{b}_{a}f(x)\left(\partial_{x}[r(x)\overline{g'(x)}]+g(x)\overline{g(x)}\right)dx+\underbracket{\left[r(x)f'(x)\overline{g(x)}-r(x)f(x)\overline{g'(x)}\right]}_{\text{boundary terms!}}
\end{equation}
We can write now
\begin{equation}
\<L(f),g\> =
\int^{b}_{a}f(x)\overline{\left(\partial_{x}[r(x)\overline{g'(x)}]+g(x)\overline{g(x)}\right)}dx+\text{bdry terms}
\end{equation}
where ``bdry terms'' are the boundary terms, or more simply
\begin{equation}
\<L(f),g\>=\<f,L(g)\>+\text{bdry terms}.
\end{equation}

\begin{defn}\index{Formally Self Adjoint}\index{Self Adjoint!Formally Self Adjoint}\index{Lagrange Identity!Self Adjoint Linear Differential Operator}\index{Linear Differential Operator!Lagrange Identity}
$L$ is called \textbf{formally self-adjoint} because 
\begin{equation}
\<L(f),g\>=\<f,L(g)\>+\text{boundary terms}
\end{equation}
this is sometimes called the ``Lagrange identity''.
\end{defn}

\begin{rmk}
If the boundary conditions make
\begin{equation}
[r(x)(f'(x)\overline{g(x)}-f(x)\overline{g'(x)})]^{b}_{a}=0
\end{equation}
then $L$ is self-adjoint. Such boundary conditions we call
self-adjoint boundary conditions (we will formally define it
below).
\end{rmk}

If we expand this out we find
\begin{equation}
[r(f'\bar{g}-f\bar{g}')]^{b}_{a} = r(b)(f'(b)\bar{g}(b)-f(b)\bar{g}(b)') - r(a)(f'(a)\bar{g}(a)-f(a)\bar{g}(a)')
\end{equation}
This gives us a set of two boundary conditions
\begin{align*}
B_{1}(f)&=\alpha_1f(a)+\alpha_2f'(a)+\beta_1f(b)+\beta_2f'(b)=0\\
B_{2}(f)&=\alpha_3f(a)+\alpha_4f'(a)+\beta_3f(b)+\beta_4f'(b)=0
\end{align*}
where the $\alpha$'s and $\beta$'s are constants.

\begin{defn}\index{Self Adjoint!Boundary Conditions}\index{Boundary Conditions!Self Adjoint}
The boundary conditions $B_1$, $B_2$ are called
\textbf{self-adjoint} with respect to the operator $L$ if
\begin{equation}
[r(x)(f'(x)\overline{g(x)}-f(x)\overline{g'(x)})]^{b}_{a}=0
\end{equation}
for $f,g$ satisfy $B_{1}(f)=0$, $B_{2}(f)=0$, $B_{1}(g)=0$, $B_{2}(g)=0$.
\end{defn}
\begin{ex}
There are several famous boundary conditions used all the
time:
\begin{enumerate}
\item\index{Boundary Conditions!Dirichlet}\index{Dirichlet Boundary Conditions} The
Dirichlet boundary conditions $f(a)=f(b)=0$;
\item\index{Boundary Conditions!Neumann}\index{Neumann Boundary Conditions} The
Neumann boundary conditions $f'(a)=f'(b)=0$;
\item\index{Boundary Conditions!Periodic}\index{Periodic Boundary Conditions} The
periodic boundary conditions $f(a)=f(b)$ and $f'(a)=f'(b)$.
\end{enumerate}
\end{ex}
\begin{defn}{(Regular Sturm Liouville Problem)}\index{Sturm-Liouville!Regular Problem}
Find all solutions of the boundary value problem
\begin{align*}
L(f) + \lambda\omega(x)f(x) &= 0\\
B_{1}(f)=0,\, B_{2}(f)&=0
\end{align*}
where $\lambda$ is an arbitrary constant and
\begin{enumerate}[i)]
\item $L(f)=[rf']'+pf$ where $r(x)$, $r'(x)$, $p(x)$ are
  real and continuous on $[a,b]$, and $r(x)>0$ on $[a,b]$.
\item $B_{1}(f)=B_{2}(f)=0$ are self-adjoint with respect to
  $L$
\item $\omega(x)$ is positive and continuous on $[a,b]$.
\end{enumerate}
\end{defn}
\begin{rmk}\index{Wave Equation!From Regular Sturm-Liouville Problem}
If $r(x)=1$ and $p(x)=0$ and $\omega(x)=1$, then we get
\begin{equation}
f''(x)+\lambda f(x)=0
\end{equation}
or in other words, we have the wave equation on an elastic string.
\end{rmk}
\begin{defn}\index{Eigenvalue}\index{Eigenfunction}
For $\phi$ a solution of Sturm-Liouville problem
corresponding to some constant $\lambda$, with
$\phi\neq0$m we call $\phi$ the \textbf{eigenvector} (or
more often \textbf{eigenfunction}) corresponding to the
\textbf{eigenvalue} $\lambda$
\end{defn}
Note this definition is because
\begin{equation*}
L(f)=-\omega(x)\lambda
f(x)\Rightarrow\frac{1}{\omega(x)}L(f)=-\lambda f(x)
\end{equation*}
so $\lambda$ is an eigenvalue of the linear operator
$M(f)=L(f)/\omega(x)$ and eigenfunction $f(x)$.
\begin{defn}\index{$L^{2}$!Weighted Space}\index{Weighted $L^{2}$-Space}
Suppose that $\omega(x)$ is integrable on $[a,b]$ and
$\omega(x)>0$ almost everywhere on $[a,b]$, then the
\textbf{weighted $L^{2}$-Space} is defined as the set
\begin{equation}
L^{2}_{\omega}(a,b) = \{ f: \int^{b}_{a}|f(x)|^2\omega(x)dx < \infty\}
\end{equation}
\end{defn}
