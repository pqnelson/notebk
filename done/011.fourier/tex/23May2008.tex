%%
%% 23May2008.tex
%% 
%% Made by Alex Nelson
%% Login   <alex@tomato>
%% 
%% Started on  Sat Mar 28 16:11:02 2009 Alex Nelson
%% Last update Sat Mar 28 16:11:02 2009 Alex Nelson
%%
There is one last significant principle that we will cover in
signal processing -- Heisenberg's famous uncertainty principle.

\begin{defn}\index{Dispersion}
For $f\in L^2$, the \textbf{dispersion of $f$ about a point $a$}
is defined as
\begin{equation}
\Delta_{a}f = \frac{\int(x-a)^2|f(x)|^2dx}{\int|f(x)|^2dx}.
\end{equation}
\end{defn}
This tells us how concentrated the function $f$ is near the point
$a$. The smaller it is, the more concentrated $f$ is; the larger
it is, the less concentrated $f$ is.

\begin{ex}
Consider the rectangle function $\chi_{1/2}(t)$ which is 1 if
$t\in[-1/2,1/2]$ and 0 otherwise. We can explicitly compute
\begin{subequations}
\begin{align}
\Delta_{a}\chi_{1/2} &=
\frac{\int^{1/2}_{-1/2}(x-a)^2|\chi_{1/2}(x)|^2dx}{\int^{1/2}_{-1/2}|\chi_{1/2}(x)|^2dx}\\
&= \int^{1/2}_{-1/2}(x-a)^2dx\\
&= \frac{1}{3}(x-a)^{3}|^{x=1/2}_{x=-1/2}\\
&= a^{2}+\frac{1}{12}.
\end{align}
\end{subequations}
Note its smallest at $a=0$ and it increases as $|a|\to\infty$. 
\end{ex}

\begin{thm}{(Heisenberg's Uncertainty Principle)}\index{Uncertainty!Heisenberg}\index{Heisenberg!Uncertainty}
Given some $f\in L^2$, then
\begin{equation}
\left(\Delta_{a}f\right)\left(\Delta_{\alpha}\widehat{f}\right)\geq\frac{1}{4}
\end{equation}
for any $a,\alpha\in\mathbb{R}$.
\end{thm}
This amounts to nothing more than saying ``For any signal $f\in
L^2$, $f$ \emph{cannot} be \emph{both} time limited and band limited.''
