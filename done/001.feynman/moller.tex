\subsection{Moller Scattering}

Moller scattering is the scattering of electrons
\begin{equation}
e^{-} + e^- \to e^- + e^-.
\end{equation}
We have two diagrams to consider this time! In fact, from here on out, we will
always have two diagrams to consider (the exception being one third order example,
which is the most important third order example because it explains the 
anamolous magnetic moment of an electron -- we'll burn that bridge when we get
to it).

\textbf{Step One:} The first diagram to consider is the
following:



\strut
\begin{center}
\begin{fmffile}{mollerImg1}
  \begin{fmfgraph*}(50,25)  \fmfpen{0.2mm}
    \fmfset{arrow_len}{3mm}\fmfset{arrow_ang}{10}
    \fmfleft{i1,o1} % change i2->o1 
    \fmfright{i2,o2} % change o1->i2
    \fmflabel{$p_{1},s_{1}$}{i1}
    \fmflabel{$p_{3},s_{3}$}{o1} %
    \fmflabel{$p_{2},s_{2}$}{i2} %
    \fmflabel{$p_{4},s_{4}$}{o2}
    \fmflabel{$\mu$}{v1}
    \fmflabel{$\nu$}{v2}
    \fmf{fermion}{i1,v1} %
    \fmf{fermion}{v1,o1}
    \fmf{fermion}{i2,v2} %
    \fmf{fermion}{v2,o2} %
    \fmf{boson,label=$q$}{v1,v2}
  \end{fmfgraph*}
\end{fmffile}
\end{center}
\strut


This is precisely the electron-muon diagram with the exception that the muon
has been replaced by an electron. Thus we will simply use the \textbf{exact
same steps} we did in the first example; we will copy/paste the results here.

The integrand should take the form
\begin{eqnarray}
\quad&&[(\bar{u}(s_3,p_3))(ig_{e}\gamma^\mu)(u(s_1, p_1))]\frac{-ig_{\mu\nu}}{q^2}[(\bar{u}(s_4,p_4))(ig_{e}\gamma^\nu)(u(s_2, p_2))] (2\pi)^{8}\nonumber\\
& &\times \delta^{(4)}(p_1 - p_3 - q)\delta^{(4)}(p_2 + q - p_4) d^4q. \nonumber
\end{eqnarray}
This has the contribution to the total probability amplitude that this process
will happen of
\begin{equation}
\mathcal{M}_1 = \frac{-g_{e}^2}{(p_1 - p_3)^2} [(\bar{u}(s_3,p_3))(ig_{e}\gamma^\mu)(u(s_1, p_1))][(\bar{u}(s_4,p_4))(ig_{e}\gamma_\mu)(u(s_2, p_2))]
\end{equation}
We will add it to the probability amplitude from the other graph to get the total
probability amplitude of the process happening.

The second diagram is odd:

\strut
\begin{center}
\begin{fmffile}{mollerImg2}
  \begin{fmfgraph*}(25,50)  \fmfpen{0.2mm}
    \fmfset{arrow_len}{3mm}\fmfset{arrow_ang}{10}
    \fmfleft{i2,o2}
    \fmfright{i1,o1}
    \fmf{fermion}{i1,v1}
    \fmf{phantom}{v1,o1} % Invisible rubber band
    \fmf{fermion}{i2,v2}
    \fmf{phantom}{v2,o2} % also invisible rubber band
    \fmf{photon,label=$q$}{v1,v2}
    % These are visible, but have no tension.
    \fmf{fermion,tension=0}{v1,o2}
    \fmf{fermion,tension=0}{v2,o1}
    \fmfdot{v1,v2}
    \fmflabel{$p_2,s_2$}{i1}
    \fmflabel{$p_1,s_1$}{i2}
    \fmflabel{$p_3,s_3$}{o1}
    \fmflabel{$p_4,s_4$}{o2}
    \fmflabel{$\mu$}{v1}
    \fmflabel{$\nu$}{v2}
  \end{fmfgraph*}
\end{fmffile}
\end{center}
\strut


We make the switch of $(s_3,p_3)\iff (s_4,p_4)$ for this diagram, and low and
behold we have a rule that takes care of this!

\textbf{Step Eight:} (Yes we are hopping right along!) We have by the eighth rule
a change in signs. So the probability amplitude from this second diagram is
(when we make the switches of $p_3\mapsto p_4$, $p_4\mapsto p_3$, $s_3\mapsto s_4$, $s_4\mapsto s_3$)
\begin{equation}
\mathcal{M}_2 = \frac{g_{e}^2}{(p_1 - p_4)^2} [(\bar{u}(s_4,p_4))(ig_{e}\gamma^\mu)(u(s_1, p_1))][(\bar{u}(s_3,p_3))(ig_{e}\gamma_\mu)(u(s_2, p_2))]
\end{equation}
So the total probability amplitude is then
\begin{eqnarray*}
\mathcal{M} &=& \frac{-g_{e}^2}{(p_1 - p_3)^2} [(\bar{u}(s_3,p_3))(ig_{e}\gamma^\mu)(u(s_1, p_1))][(\bar{u}(s_4,p_4))(ig_{e}\gamma_\mu)(u(s_2, p_2))] \\
&&+ \frac{g_{e}^2}{(p_1 - p_4)^2} [(\bar{u}(s_4,p_4))(ig_{e}\gamma^\mu)(u(s_1, p_1))][(\bar{u}(s_3,p_3))(ig_{e}\gamma_\mu)(u(s_2, p_2))].
\end{eqnarray*}
