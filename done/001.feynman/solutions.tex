\subsection{Solutions to the Dirac Equation}

The easiest approach to find solutions to the Dirac equation is to insist that
the solution is independent of spatial position:
\begin{equation}
\frac{\partial\psi}{\partial x} = \frac{\partial\psi}{\partial y} = \frac{\partial\psi}{\partial z} = 0.
\end{equation}
This really describes a particle with zero momentum, since the momentum operator
is $i\hbar\partial_\mu$ and all the spatial eigenvalues vanish. The Dirac 
equation simplifies to
\begin{equation}
\frac{i\hbar}{c}\gamma^0\frac{\partial\psi}{\partial t} - mc\psi = 0
\end{equation}
or equivalently
\begin{equation}
\begin{bmatrix}
1 & 0 \\
0 & -1
\end{bmatrix}
\begin{bmatrix}
\partial\psi_A/\partial t\\
\partial\psi_B/\partial t
\end{bmatrix}
= -i\frac{mc^2}{\hbar}
\begin{bmatrix}
\psi_A \\
\psi_B
\end{bmatrix}
\end{equation}
where
\begin{equation}
\psi_A = \begin{bmatrix}
\psi_1\\
\psi_2
\end{bmatrix}
\end{equation}
carries the upper two components and 
\begin{equation}
\psi_B = \begin{bmatrix}
\psi_3\\
\psi_4
\end{bmatrix}
\end{equation}
carries the lower two components. Thus
\begin{equation}
\frac{\partial\psi_A}{\partial t} = -i\left(\frac{mc^2}{\hbar}\right)\psi_A,\quad -\frac{\partial\psi_B}{\partial t} = -i\left(\frac{mc^2}{\hbar}\right)\psi_B
\end{equation}
and the solutions are
\begin{equation}
\psi_A(t) = \exp[-i(mc^2/\hbar)t]\psi_A(0),\quad\psi_B(t)=\exp[i(mc^2/\hbar)t]\psi_B(0).
\end{equation}
We should know that in Quatum mechanics, the term
\begin{equation}
\exp(-iEt/\hbar)
\end{equation}
is the characteristic for time dependence of a quantum state with energy $E$. It
follows that at rest with $\bold{p}=0$, the energy of the particle is $E=mc^2$.
So $\psi_A$ is what we expect.

What about $\psi_B$? It has negative energy! What the heck?! This is a famous
disaster, and Dirac's response was like the Hindenberg of physics. He suggested
something called the Hole theory, we will not discuss it here. 

We interpret these ``negative'' energy particles as \emph{antiparticles} with
\emph{positive} energy. So for us in our Dirac equation, $\psi_B$ describes
positrons (or antielectrons if one prefers to be outlandish) and $\psi_A$ 
describes electrons. Each of them is a 2 component spinor (a 2 column vector).
This is ideal as such a mathematical object describes a spin 1/2 particle. So,
to sum up, we have 2 particles that are each 2 solutions for a grand total of
4 independent solutions with momentum $\bold{p}=0$:
\begin{equation}
\psi^{(1)} = \exp(i(mc^2/\hbar)t)\begin{bmatrix}
1\\
0\\
0\\
0
\end{bmatrix}\quad\psi^{(2)} = \exp(i(mc^2/\hbar)t)\begin{bmatrix}
0\\
1\\
0\\
0
\end{bmatrix}
\end{equation}
\begin{equation}
\psi^{(3)} = \exp(-i(mc^2/\hbar)t)\begin{bmatrix}
0\\
0\\
1\\
0
\end{bmatrix}\quad\psi^{(4)} = \exp(-i(mc^2/\hbar)t)\begin{bmatrix}
0\\
0\\
0\\
1
\end{bmatrix}
\end{equation}
describing (respectively) an electron with spin up, an electron with spin
down, a positron with spin up and an electron with spin down.

So to look at this from the perspective of solving differential equations, we
have a solution to the homogeneous equation and we will use the method of variation
of parameters to get solutions to the Dirac equation. What does this mean? Well,
it means we are looking for ``plane wave solutions'' that look like
\begin{equation}
\psi(\bold{r},t) = ae^{-i(Et-\bold{p}\cdot\bold{r})/\hbar}u(E,\bold{p})
\end{equation}
where $a$ is a normalization constant (so probabilities add up to 1). We want to
solve for $u(E,\bold{p})=u(p)$ (we will use $p=(E/c,\bold{p})$ which is a 4 vector, and
similarly $x=(ct,\bold{x}$), which is a mathematical object called a ``bispinor''.
We don't want any old bispinor, we want one that will solve Dirac's equation!
We have $x$ dependence only in the exponent, so we find
\begin{equation}
\partial_\mu\psi = \frac{-i}{\hbar}p_\mu a e^{-(i/hbar)x^\mu p_\mu}u
\end{equation}
By plugging this into Dirac's equation, we get
\begin{equation}
\gamma^\mu p_\mu a e^{-(i/\hbar)x\cdot p}u - mcae^{-(i/\hbar)x\cdot p}u = 0
\end{equation}
or if we want a neater and cleaner way to present it
\begin{equation}\label{momentumSpaceDiracEquation}
(\gamma^\mu p_\mu - mc)u = 0.
\end{equation}
This is the ``momentum space Dirac equation'' (which we get by taking the
Fourier Transform of the Dirac equation we all know and love). Notice this is
purely algebraic, no derivatives! That's the beauty of Fourier transforms in
solving differential equations! If $u$ satisfies (\ref{momentumSpaceDiracEquation})
then $\psi$ satisfies the Dirac equation.

Now to \emph{prove} this (because an assertion is always meaningless without a
rigorous proof -- take note of this social ``scientists'') we need to use a lot
of gamma matrix manipulations. Remember all representations are ``equivalent''
in the sense that they are related by unitary transformations. First we have
\begin{equation}
\gamma^\mu p_\mu = \gamma^0 p^0 - \bold{\gamma}\cdot\bold{p} = \frac{E}{c}\begin{bmatrix}
1 & 0 \\
0 & -1
\end{bmatrix}
- \bold{p}\cdot\begin{bmatrix}
0 & \sigma \\
-\sigma & 0 
\end{bmatrix}
=
\begin{bmatrix}
E/c & -\bold{p}\cdot\bold{\sigma} \\
\bold{p}\cdot\bold{\sigma} & -E/c
\end{bmatrix}
\end{equation}
SO it follows that
\begin{eqnarray*}
(\gamma^\mu p_\mu - mc)u &=& \begin{bmatrix}
\left(\frac{E}{c}-mc\right) & -\bold{p}\cdot\sigma \\
\bold{p}\cdot\sigma & \left(\frac{-E}{c}-mc\right)
\end{bmatrix}
\begin{bmatrix}
u_A \\
u_B
\end{bmatrix} \\
&=& \begin{bmatrix}
\left(\frac{E}{c}-mc\right)u_A & -\bold{p}\cdot\sigma u_B\\
\bold{p}\cdot\sigma u_A & \left(\frac{-E}{c}-mc\right) u_B
\end{bmatrix}
\end{eqnarray*}
where the subscript $A$ is for the upper two components and the $B$ stands for
the lower two. In order to satisfy the momentum space Dirac equation, we
must have
\begin{equation}\label{stepTowardsSolutions}
u_A = \frac{c}{E - mc^2}(\bold{p}\cdot\sigma)u_B,\quad u_B = \frac{c}{E + mc^2}(\bold{p}\cdot\sigma) u_A
\end{equation}
We substitute the second into the first to give us
\begin{equation}
u_A =\frac{c^2}{E^2 - m^2c^4}(\bold{p}\cdot\sigma)^2 u_A
\end{equation}
Observe
\begin{eqnarray*}
\bold{p}\cdot\sigma &=& p_x\begin{bmatrix}
0 & 1\\
1 & 0 
\end{bmatrix}
+ p_{y}\begin{bmatrix}
0 & -i\\
i & 0
\end{bmatrix}
+ p_{z}\begin{bmatrix}
1 & 0 \\
0 & -1
\end{bmatrix} \\
&=& \begin{bmatrix}
p_{z} & (p_{x} - ip_{y}) \\
(p_{x} + ip_{y}) & -p_{z}
\end{bmatrix}
\end{eqnarray*}
We find then by matrix multiplication (we will not calculate this out with every
detail, but we will show the result):
\begin{equation}
(\bold{p}\cdot\sigma)^2 = \begin{bmatrix} p_{z}^2 + (p_x - ip_{y})(p_{x} + ip_{y}) & p_{z}(p_{x} - ip_{y}) - p_{z}(p_{x} - ip_{y}) \\
p_{z}(p_{x} + ip_{y}) - p_{z}(p_{x} + ip_{y}) & (p_{x} + ip_{y})(p_{x} - ip_{y}) + p_{z}^2
\end{bmatrix} = \bold{p}^2 I
\end{equation}
where $I$ is the 2 by 2 identity matrix. We see then that by plugging this into our equation for $u_A$
\begin{equation}
u_A = \frac{\bold{p}^2c^2}{E^2 - m^2c^4}u_A
\end{equation}
which can be rearranged to be
\begin{eqnarray*}
(E^2 - m^2c^4)u_A &=& \bold{p}^2c^2 u_A \\
\Rightarrow (E^2 - \bold{p}^2c^2)u_A &=& m^2c^4 u_A
\end{eqnarray*}
and thus
\begin{equation}
E^2 - \bold{p}^2c^2 = m^2c^4
\end{equation}
which is the famous Einstein equation we all know and love. This tells us that
in order to satisfy the Dirac equation, we have to obey the mass shell constraint.
This admits two solutions for $E$:
\begin{equation}
E = \pm\sqrt{m^2 c^4 + \bold{p}^2c^2}
\end{equation}
where the positive root is associated with particle states, and the negative
root with antiparticle states.

Using Eq (\ref{stepTowardsSolutions}), it is straightforward to calculate out
the solutions to the Dirac equation (ignoring normalization constants):
\begin{equation*}
\mbox{Pick } u_A = \begin{bmatrix} 1\\0\end{bmatrix}\quad\mbox{then } 
u_B = \frac{c}{E + mc^2}(\bold{p}\cdot\sigma)\begin{bmatrix}1\\0\end{bmatrix} 
= \frac{c}{E + mc^2}\begin{bmatrix} p_{z}\\ p_{x}+ip_{y}\end{bmatrix}
\end{equation*}
\begin{equation*}
\mbox{Pick } u_A = \begin{bmatrix} 0\\1\end{bmatrix}\quad\mbox{then } 
u_B = \frac{c}{E + mc^2}(\bold{p}\cdot\sigma)\begin{bmatrix}0\\1\end{bmatrix} 
= \frac{c}{E + mc^2}\begin{bmatrix} p_{x}-ip_{y}\\ -p_{z}\end{bmatrix}
\end{equation*}
\begin{equation*}
\mbox{Pick } u_B = \begin{bmatrix} 1\\0\end{bmatrix}\quad\mbox{then } 
u_A = \frac{c}{E - mc^2}(\bold{p}\cdot\sigma)\begin{bmatrix}1\\0\end{bmatrix} 
= \frac{c}{E - mc^2}\begin{bmatrix} p_{z}\\ p_{x}+ip_{y}\end{bmatrix}
\end{equation*}
\begin{equation*}
\mbox{Pick } u_B = \begin{bmatrix} 0\\1\end{bmatrix}\quad\mbox{then } 
u_A = \frac{c}{E - mc^2}(\bold{p}\cdot\sigma)\begin{bmatrix}0\\1\end{bmatrix} 
= \frac{c}{E - mc^2}\begin{bmatrix} p_{x}-ip_{y}\\-p_{z}\end{bmatrix}
\end{equation*}
For the first two of these, we must use the positive energy otherwise we have
division by zero, and if you divide by zero you go to hell. For the same reason,
the energy in the latter two are negative. It is convenient to ``normalize'' these
spinors in such a way that
\begin{equation}
u^\dag u = 2|E|/c
\end{equation}
where the dagger indicates the transpose conjugate (``Hermitian conjugate'') is
used:
\begin{equation*}
u = \begin{bmatrix}a\\b\\c\\d\end{bmatrix}\Rightarrow\quad u^\dag = (a^*, b^*, c^*, d^*)
\end{equation*}
so that
\begin{equation}
u^\dag u = |a|^2 + |b|^2 + |c|^2 + |d|^2.
\end{equation}
So we find that the four solutions are:
\begin{equation}
u^{(1)} = N\begin{bmatrix}1\\0\\\frac{\displaystyle cp_{z}}{\displaystyle E+mc^2}\\ \frac{\displaystyle c(p_{x}+ip_{y})}{\displaystyle E+mc^2}\end{bmatrix}
\end{equation}
\begin{equation}
u^{(2)} = N\begin{bmatrix}0\\1\\ \frac{\displaystyle c(p_{x}-ip_{y})}{\displaystyle E+mc^2}\\\frac{\displaystyle -cp_{z}}{\displaystyle E+mc^2}\end{bmatrix}
\end{equation}
with $E = +\sqrt{m^2c^4 + \bold{p}^2c^2}$
\begin{equation}
u^{(3)} = N\begin{bmatrix}\frac{\displaystyle cp_{z}}{\displaystyle E-mc^2}\\ \frac{\displaystyle c(p_{x}+ip_{y})}{\displaystyle E-mc^2}\\1\\0\end{bmatrix}
\end{equation}
\begin{equation}
u^{(4)} = N\begin{bmatrix} \frac{\displaystyle c(p_{x}-ip_{y})}{\displaystyle E-mc^2}\\\frac{\displaystyle c(-p_{z})}{\displaystyle E-mc^2}\\ 0\\1\end{bmatrix}
\end{equation}
with $E = -\sqrt{m^2c^4 + \bold{p}^2c^2}$, and the normalization constant is
\begin{equation}
N = \sqrt{(|E|+mc^2)/c}.
\end{equation}
Now we are really tempted to say that $u^{(1)}$ is an electron with spin up,
and $u^{(2)}$ is an electron with spin down, and so on, but this is not quite so.
For Dirac Particles, the spin matrices are
\begin{equation}
S = \frac{\hbar}{2}\Sigma\quad\mbox{with }\Sigma\equiv\begin{bmatrix}\sigma & 0\\0 & \sigma\end{bmatrix}
\end{equation}
and it's easy to check that $u^{(1)}$ is \emph{not} an eigenstate of $\Sigma$.
However, if we orient the $z$ axis so it points along the direction of motion
(in which case $p_x = p_y = 0$) then $u^{(1)}$, $u^{(2)}$, $u^{(3)}$, and $u^{(4)}$
are eigenspinors of $S_z$; $u^{(1)}$ and $u^{(3)}$ are spin up, and
$u^{(2)}$ and $u^{(4)}$ are spin down\footnote{It is actually mathematically
impossible to construct spinors that satisfies the momentum Dirac equation and
are simultaneously eigenspinors of $S_z$ (except for the special case $\bold{p}$ = $p_z\hat{z}$).
The reason is that $S$ by itself is \emph{not a conserved quantity.} Only the
\emph{total} angular momentum $L+S$ is conserved. It is possible to construct
eigenspinors of \emph{helicity}, $\Sigma\cdot\hat{p}$ (there's no \emph{orbital}
angular momentum about the direction of motion), but these are cumbersome and in
practice we like to work with the spinors we have constructed, even though it is
difficult to have a physical intuition to what they mean. In the end, all that
really matters is that we have a complete set of solutions.}

Now we have to discuss the importance of $E$ and $\bold{p}$, which are mathematical
parameters which correspond physically to energy and momentum. At least, this is
true for the electron states $u^{(1)}$ and $u^{(2)}$; but in $u^{(3)}$ and 
$u^{(4)}$ the $E<0$...so it \emph{cannot} represent positron energy. All free
particles -- electrons and positrons alike -- carry \emph{positive} energy.
The ``negative-energy'' solutions must be reinterpreted as \emph{positive}
energy \emph{antiparticle} states. To express these solutions in terms of 
the \emph{physical} energy and momentum of the positron, we flip the signs of
$E$ and $\bold{p}$:
\begin{equation}
\psi(\bold{r},t) = ae^{i/\hbar(Et - \bold{p}\cdot\bold{r})}u(-E,-\bold{p})
\end{equation}
for solutions (3) and (4) of course. These are the same solutions, we just have
changed the signs of two parameters so it is physically appealing. It is 
customary to use $v$ for positron states, expressed in terms of the physical
energy and momentum:
\begin{equation}
v^{(1)}(E,\bold{p}) = u^{(4)}(-E,-\bold{p}) = N\begin{bmatrix} 
\frac{\displaystyle c(p_x - ip_y)}{\displaystyle E + mc^2}\\
\frac{\displaystyle c(-p_z)}{\displaystyle E + mc^2}\\
0\\
1\end{bmatrix}
\end{equation}
\begin{equation}
v^{(2)}(E,\bold{p}) = u^{(4)}(-E,-\bold{p}) = N\begin{bmatrix} 
\frac{\displaystyle c(p_z)}{\displaystyle E + mc^2}\\
\frac{\displaystyle c(p_x + ip_y)}{\displaystyle E + mc^2}\\
1\\
0\end{bmatrix}
\end{equation}
(with $E=\sqrt{m^2c^4 + \bold{p}^2c^2}$).

So we will no longer be working with $u^{(3)}$ and $u^{(4)}$; instead, the
set of solutions we will be working with are $u^{(1)}$, $u^{(2)}$ (representing
the two spin states of an electron with energy $E$ and momentum $\bold{p}$),
and $v^{(1)}$, $v^{(2)}$ (representing the two spin states of a positron with
energy $E$ and momentum $\bold{p}$). Notice that whereas the $u$'s satisfy the
momentum space Dirac equation in the form
\begin{equation}
(\gamma^\mu p_\mu - mc)u = 0
\end{equation}
the $v$'s obey the equation with the sign of $p_\mu$ reversed:
\begin{equation}
(\gamma^\mu p_\mu + mc)v = 0.
\end{equation}
Sure this is interesting, but it's only the special case of plane waves. Why
bother? Well, they are of interest because they describe particles with 
specified energies and momenta, and in a typical experiment that's what we
control and measure.
