\section{Decay Rates and Feynman Diagrams}

Remember if we have some collection of particles (e.g. Muons) and they decay,
the decay rate $\Gamma$ (the probability per unit time that any given muon will
disintegrate) satisfies a particular relation. If $N(t)$ is the number of particles
at time $t$, the infinitesmal change in $N$ from $t$ to $t+dt$ is
\begin{equation}
dN = -\Gamma N(t)dt
\end{equation}
which tells us the number is decreasing when we move forward in time. It follows
that
\begin{eqnarray*}
\frac{1}{N}dN &=& -\Gamma dt\\
\int\frac{1}{N}dN &=& -\int\Gamma dt\\
\ln(N(t)) &=& -\Gamma t + C \\
N(t) &=& \exp(-\Gamma t)\exp(C) \\
&=& N(0)\exp(-\Gamma t)
\end{eqnarray*}
where $N(0)$ is the initial number of particles, and $C$ is the constant of
integration. 

The \textbf{mean lifetime} of the particle is simply the reciprocal of the 
decay rate
\begin{equation}
\tau = \frac{1}{\Gamma}.
\end{equation}
If there are several different ways for the particle decay, each with different
decay rates, the total decay rate is given by the sum of the individual rates:
\begin{equation}
\Gamma_\textrm{tot} = \sum_{j=1}^{n} \Gamma_{j}
\end{equation}
and the mean lifetime is the reciprocal of this quantity
\begin{equation}
\tau = \frac{1}{\Gamma_\textrm{tot}}.
\end{equation}

\subsection{Fermi's Golden Rule}

Suppose we have one particle that decays into several others
\begin{equation}
1\to 2+3+\cdots+n
\end{equation}
If $\mathcal{M}$ is the total probability amplitude from the various Feynman
diagram representations of this process, then the infinitesmal decay rate is
given by
\begin{equation}
d\Gamma = |\mathcal{M}|^2 \frac{S}{2\hbar m_1}\left[\left(\frac{cd^3p_2}{(2\pi)^32E_2}\right)\left(\frac{cd^3p_3}{(2\pi)^32E_3}\right) \cdots \left(\frac{cd^3p_n}{(2\pi)^32E_n}\right) \right]\times (2\pi)^4 \delta^{(4)}(p_1 - (p_2 + p_3 + \cdots + p_n))
\end{equation}
