%%
%% basis.tex
%% 
%% Made by Alex Nelson
%% Login   <alex@tomato>
%% 
%% Started on  Sun May 31 12:44:15 2009 Alex Nelson
%% Last update Sun May 31 12:44:15 2009 Alex Nelson
%%
\begin{prob}
Oftentimes, a topology is cumbersome to work with, so we want to
make life easier. In linear algebra, we don't work with vectors
aribtrarily, we work with linear combinations of basis
vectors. We use an analogous concept for topological spaces. We
work with basis elements when discussing topological properties,
it simplifies life a bit. For us, however, our basis elements are
not vectors! Instead, they are a selected collection of open
subsets of a given topological space $X$ with some extra
properties. They are the ``basic building blocks'' for a
topology, so intuitively if all the ``building blocks'' share a
property, the topology should have it too.
\end{prob}
\begin{defn}\label{defn:basisForTopology}
  If $X$ is a set, a \textbf{basis for a topology} on $X$ is a
  collection $\mathscr{B}$ of subsets of $X$ (or ``\textbf{basis elements}'') 
  such that 
\begin{enumerate}
\item For each $x\in X$, there is at least one basis element $B$
  containing $x$.
\item If $x$ belongs to the intersection of two basis elements
  $B_{1}$, $B_{2}$, then there is a basis element $B_{3}$ such
  that
\begin{equation}
x\in B_{3}\subset B_{1}\cap B_{2}.
\end{equation}
\end{enumerate}
If $\mathscr{B}$ satisfies these two conditions, then we define
the \textbf{topology generated by the basis $\mathscr{B}$} by
first saying an open set $U\subseteq X$ is in the ``topology generated'' by
$\mathscr{B}$ iff $\forall x\in U$, there is a $B\in\mathscr{B}$
such that $x\in B\subseteq U$.
\end{defn}
\begin{rmk}\label{rmk:topologyGenerated}
We can alternatively say that if an open set $U$ in the ``topology
generated'' by $\mathscr{B}$, then
\begin{equation}
U = \bigcup_{\alpha\in J}B_{\alpha}
\end{equation}
where $J$ is some indexing set, $B_{\alpha}\in\mathscr{B}$ are
basis elements.
\end{rmk}
\begin{rmk}\label{rmk:propertiesOfBasis}
A few notes on the properties of the basis for a topology.
\begin{enumerate}
\item The first property is basically saying the union of all our
  basis elements covers $X$. There are no ``holes'' in our
  topology.
\item The second property is a bit more abstract. We want
  intersections to be covered by some number of basis
  elements. If we didn't have this property, the intersection of
  two basis elements wouldn't be an open set, which implies that
  basis elements are not open sets. This is bad, it would imply
  that the ``topology generated'' by a basis isn't really a
  topology!
\end{enumerate}
\end{rmk}
\begin{prob}
Note that the definition of the ``topology generated'' is
  not yet shown to be a topology, we need to prove that it is a
  topology.
\end{prob}
\begin{thm}\label{thm:topologyGeneratedIsTopology}
  Let $\mathscr{B}$ be a basis for a topology on $X$. Then
  $\mathcal{T}$, the ``topology generated'' by $\mathscr{B}$, is
  a topology.
\end{thm}
\begin{proof}
We need to prove that the ``topology generated'' by a given basis
satisfies the properties of a topology. We will do this property
by property. Let $\mathscr{B}$ be the basis for a topology on $X$,
$\mathcal{T}$ by the ``topology generated'' by $\mathscr{B}$.
\begin{enumerate}
\item (Contains $\emptyset$, $X$) \begin{enumerate}
\item We see that $\emptyset\in\mathcal{T}$ is vacuously true.
\item We see that for each $x\in X$ there is a $B_{x}\in\mathscr{B}$
  such that $x\in B_{x}$. It follows that
\begin{equation}%\label{eq:}
 X=\bigcup_{x\in X}\{x\} \subseteq \bigcup_{x\in X}B_{x} 
\end{equation}
Since each $B_{x}\subseteq X$, it follows that 
\begin{equation}%\label{eq:}
  \bigcup_{x\in X}B_{x}\subseteq X
\end{equation}
Thus the union of all basis elements is $X$.
\end{enumerate}
So both $\emptyset\in\mathcal{T}$ and $X\in\mathcal{T}$.
\item (Closure under arbitrary unions) We see that if
  $\{U_{\alpha}\}_{\alpha\in J}$ is a collection of open sets in
  $\mathcal{T}$ and $J$ is some index set, then we can write
\begin{equation}%\label{eq:}
  \bigcup_{\beta\in I_{\alpha}} B_{\beta} = U_{\alpha}
\end{equation}
by virtue of the definition of an open set in the ``topology
generated'' by $\mathscr{B}$. If 
\begin{equation}
I = \bigcup_{\alpha\in J}I_{\alpha}
\end{equation}
then
\begin{equation}%\label{eq:}
  \bigcup_{\beta\in I}B_{\beta} = \bigcup_{\alpha\in
    J}\left(\bigcup_{\beta\in I_{\alpha}} B_{\beta}\right) =
  \bigcup_{\alpha\in J}U_{\alpha}
\end{equation}
is also, by definition, a set in $\mathcal{T}$.
\item (Closure under finite intersections) Let
  $\{U_{\alpha}\}^{n}_{\alpha=1}$ be some finite collection of
  open sets in $\mathcal{T}$. We see that by definition of an
  open set in the ``topology generated'' by $\mathscr{B}$
\begin{equation}%\label{eq:}
  U_{\alpha} = \bigcup_{\beta\in I_{\alpha}}B_{\beta}
\end{equation}
for some index set $I_{\alpha}$, and basis elements
  $B_{\alpha}\in\mathscr{B}$. We see then that the finite
  intersection is then
\begin{equation}%\label{eq:}
  \bigcap_{\alpha=1}^{n}U_{\alpha} = \bigcap_{\alpha=1}^{n}\left(\bigcup_{\beta\in I_{\alpha}}B_{\beta}\right).
\end{equation}
Let
\begin{equation}%\label{eq:}
  I = \bigcap_{\alpha=1}^{n}I_{\alpha}
\end{equation}
then
\begin{equation}%\label{eq:}
  \bigcap_{\alpha=1}^{n}U_{\alpha} = \bigcup_{\beta\in I}B_{\beta}.
\end{equation}
But this is by definition an open set in the ``topology
generated'' by $\mathscr{B}$.
\end{enumerate}
Thus the ``topology generated'' $\mathcal{T}$ by $\mathscr{B}$
satisfies all the properties of a topology. 
\end{proof}

\begin{lem}\label{lem:basisTopologyUnions}
  Let $X$ be a set, let $\mathscr{B}$ be a basis for a topology
  $\mathcal{T}$ on $X$. Then $\mathcal{T}$ equals the collection
  of all unions of elements of $\mathscr{B}$.
\end{lem}
\begin{proof}
Given any arbitrary $U\in\mathcal{T}$, we can write it as
\begin{equation}%\label{eq:}
  U = \bigcup_{x\in U}B_{x}
\end{equation}
where $x\in B_{x}\subseteq U$ is a basis element. We know such an
element exists by property 1 of a basis element, and we know it
is contained in $U$ by property 2 of a basis element. 

We have that any element of $\mathcal{T}$ can be written as an
arbitrary union of basis elements, we need to show that it
contains every union of basis elements. We know by definition,
every basis element is open. We also know that arbitrary union of
open sets is open. By definition, open sets are elements of a specified
topology. Bases are defined for a given topology, which implies
the arbitrary union of basis elements is in the given
topology. So every union of basis elements is in the topology
$\mathcal{T}$.
\end{proof}
\begin{rmk}\label{rmk:basisNonuniqueness}
This lemma basically says that any open set $U$ can be written as
a union of basis elements. Note that it does not specify this
union is necessarily \emph{unique}. This stands in stark contrast
to linear algebra, where a vector \emph{is} a unique linear
combination of basis elements. It is an unfortunate fact of life
that topologists choose poor definitions.
\end{rmk}
\begin{rmk}
So, just to review, if we have a basis $\mathscr{B}$ for
the topological space $(X,\mathcal{T})$, if $U\subset X$, and if
for each $x\in U$ there is a $B_{x}\in\mathscr{B}$ such that
$x\in B_{x}\subset U$, then $U$ is a union of basis elements.
\end{rmk}
\subsection{Given a Topological Space, Finding a Basis.}
\begin{prob}
But now that we have some notion of a ``basis'' for a topology,
the question is ``How can we find a basis for a given topology?''
\end{prob}
\begin{lem}\label{lem:findingTopologyBasis}
Let $(X,\mathcal{T})$ be a topological space. Suppose that $\mathscr{C}$ is a
collection of open sets of $X$ such that for each open set $U$ of
$X$ and each $x\in U$, there is an element $C\in\mathscr{C}$ such
that
$x\in C\subset U$. Then $\mathscr{C}$ is a basis of $X$.
\end{lem}
\begin{proof}
We need to do two things. First we need to show that
$\mathscr{C}$ is a basis. Second, we need to show that the
topology generated by $\mathscr{C}$, call it $\mathcal{T}'$, is
really just $\mathcal{T}$.
\begin{enumerate}
\item We see that given each $x\in X$, there is at least one
  $C\in\mathscr{C}$ containing it. This is, by hypothesis,
  true. We need to show that if $x\in C_{1}\cap C_{2}$, then
  there is a $C_{3}\in\mathscr{C}$ such that $x\in C_{3}\subseteq
  C_{1}\cap C_{2}$. We see that $C_{1}\cap C_{2}$ is open, so by
  hypothesis there is a $C_{3}\subseteq C_{1}\cap C_{2}$ such
  that $x\in C_{3}$. This means that $\mathscr{C}$ is a basis.
\item We need to show that $\mathcal{T}'=\mathcal{T}$.
\begin{enumerate}
\item We see that any element $U\in\mathcal{T}$ is such that for
  some $x\in U$, there is a $C\in\mathscr{C}$ such that $x\in
  C\subset U$ so $U\in\mathcal{T}'$ by definition. 
\item We also see that any element $V\in\mathcal{T}'$ is such
  that it is the union of basis elements in $\mathscr{C}$, by
  lemma \eqref{lem:basisTopologyUnions}. But each
  $C\in\mathscr{C}$ is such that $C\in\mathcal{T}$. This means
  that $V$ (as a union of elements of $\mathcal{T}$) is also in
  $\mathcal{T}$.
\end{enumerate}
This implies that $\mathcal{T}=\mathcal{T}'$, as desired.
\end{enumerate}
Thus $\mathscr{C}$ is a basis for the topology $\mathcal{T}$ on $X$.
\end{proof}
\subsection{Example of Usefulness of Bases.}
\begin{prob}
In the beginning of this section, we mentioned that it is useful
to use bases instead of topologies since they allow us to prove
properties faster. That is, if we can prove some topological
property holds on each basis element (or equivalently, an
arbitrary basis element), then it holds for the topology. We will
illustrate the use of bases in making such a claim.
\end{prob}
\begin{lem}\label{lem:finerBasesEqualsFinerTopologies}
Let $\mathscr{B}$, $\mathscr{B}'$ be bases for the topologies
$\mathcal{T}$, $\mathcal{T}'$ (respectively) on $X$. Then the
following are equivalent:
\begin{enumerate}
\item $\mathcal{T}'$ is finer than $\mathcal{T}$.
\item For each $x\in X$ and each basis element $B\in\mathscr{B}$
  containing $x$, there is a basis element $B'\in{\mathscr{B}}'$
  such that $x\in B'\subset B$.
\end{enumerate}
\end{lem}
\begin{proof}
$(2)\Rightarrow(1).$ We see that an element $U\in T$ can be
  written as the union of basis elements
  $\{B_{\alpha}\}_{\alpha\in J}$. But since, for each $x\in U$,
  there is a $B'_{\alpha}\in\mathscr{B}'$ such that
  $x\in B'_{\alpha}\subset B_{\alpha}$. Then $B'_{\alpha}\subset
  U$, which implies $U\in\mathcal{T}'$.

\noindent$(1)\Rightarrow(2).$ Given $x\in X$, and a
$B\in\mathscr{B}$ that contains $x$. We see that
$B\in\mathcal{T}$. But by (1), $\mathcal{T}\subset\mathcal{T}'$
so $B\in\mathcal{T}'$. So consider $B$ as an open set that
coincidentally contains a point $x$. Then by definition of
$\mathscr{B}'$ as a basis, there is a basis element
$B'\in\mathscr{B}'$ such that $x\in B'\subset B$.
\end{proof}
\begin{rmk}
The gist of this lemma boils down to this: given two bases on a
given set, and one is ``finer'' than the other, the topology
generated by the ``finer'' basis is finer than the topology
generated by the ``coarser'' basis.
\end{rmk}
\subsection{A ``Sub-Basis''.}
\begin{defn}\label{defn:subbasis}
Let $(X,\mathcal{T})$ be a topological space. A \textbf{Subbasis}
for the topology on $X$ is the collection of subsets of $X$ whose
union equals $X$. The \textbf{Topology Generated by a Subbasis} is
defined as the collection $\mathcal{T}$ of all unions of finite
intersections of subbasis elements.
\end{defn}
\begin{rmk}\label{rmk:subbasisAsBadDefn}
In the author's opinion, the previous definition is a bad one. It
appeals too much to linear algebraic intuition of a subbasis as a
basis of a subspace, and that is not the case here. Instead, it
should really have the intuition of a \emph{pre}basis. That is,
after a bit of manipulation we can get a basis. But it is not as
though anyone can change this definition now!
\end{rmk}
