\lecture

The first paper on quantum gravity was written by Rosenfeld in
1930.\footnote{I believe this is, in fact, two papers by Leon Rosenfeld:
\begin{enumerate}
\item ``Zur Quantelung der Wellenfelder'',
\journal{Ann.Phys.} \volume{397} (1930) 113--152. An English translation
may be found thanks to D.~Salisbury, Max Planck
Institute for the History of Science, Preprint 381 (2009) \url{https://pure.mpg.de/rest/items/item_2274368_1/component/file_2274366/content}.
\item ``\"{U}ber die Gravitationswirkungen des Lichtes''. \journal{Z.~Phys.} \volume{65} (1930) 589--599.
\end{enumerate}
The curious reader may peruse Peruzzi and Rocci's ``Tales from the prehistory of Quantum Gravity. L\'eon Rosenfeld's earliest contribution'' \arXiv{1802.08878} for a summary of
Rosenfeld's contributions to quantum gravity.}

A small aside on if gravity needs to be quantized. ``Well everything
else is [quantized].'' True, but gravity is slightly different. The
proper answer is: \emph{we don't know for certain but it seems
likely}. Lets consider a few thought experiments.

We will try to cover the collapse of the wave function without getting
into what it really means. Let us ask two questions:
\begin{enumerate}
\item Does gravity collapse the wave function?
\item Do other measurements collapse the wave function?
\end{enumerate}
There are four possible answers.

\bigbreak\noindent\textbf{Answer 1: No, No.}  This is the Everett
interpretation of quantum mechanics. This is fine if everything is
quantum mechanical, but what if gravity is not quantum mechanical? A
classical gravitational field coupled to the quantum mechanical matter
results in observable inconsistencies.

\begin{itemize}
\item Don N.~Page and C.D.~Geilker,
``Indirect Evidence for Quantum Gravity''.
\journal{Phys. Rev. Lett.} \volume{47} (1981) pp.979 \emph{et seq.}
{\tt\doi{10.1103/PhysRevLett.47.979}}
\end{itemize}
Page and Geilker experiment testing if gravity is classical and matter
is quantum mechanical.

\bigbreak\noindent\textbf{Answer 2: No, Yes.}
The paper for this perspective:

\begin{itemize}
\item Kenneth Eppley and Eric Hannah,
``The necessity of quantizing the gravitational field''.
\journal{Foundations of Physics} \volume{7} (1977) pp.51--68
{\tt\doi{10.1007/BF00715241}}
\end{itemize}

Eppley and Hannah argue if this were the case, we could send information
faster than light. Their argument is a tad elaborate.

\begin{wrapfigure}{R}{7pc}
\centering
\includegraphics{img/2009-04-01.0}
\end{wrapfigure}

Consider a particle in a box symmetric in the middle. We lower some
barrier in the middle (the dashed line to the right), split the box in
two. Send one to Pluto, the other remains here. Measure the
gravitational field. The measured field shouldn't be that of a whole
electron since that violates the conservation of energy, and such a
violation is bad. We are assuming that gravity is classical, so both
observers should measure the gravitational field for half of an
electron. Open the box [on Earth]. If the electron is present, the wave
function collapses, and information instantaneously changes --- the
gravitational field of Pluto's box \emph{instantaneously}
changes. That's bad.

What if we try to weaken causality? Well, causality is either there or
not, it's like pregnancy.

One may be able to weasel out of it by supposing that measurements may
be generalized a bit.

\bigbreak\noindent\textbf{Answer 3: Yes, No.}  This is Roger Penrose's
idea. We modify Schrodinger's equation to include some ``weak
nonlinearities'' from gravity.

\bigbreak\noindent\textbf{Answer 4: Yes, Yes.}  Gravity --- albeit
classical --- causes collapse of the wave function and measurement does
as well. This leads to violation of uncertainty, or the conservation of
energy(?).


\begin{wrapfigure}{R}{15pc}\vskip-2pc
\centering
\includegraphics{img/2009-04-01.1}
\end{wrapfigure}\bigbreak\noindent\textbf{Example} (Heisenberg microscope)\textbf{.}
Consider a microscope and an electron some distance $f$ from the
lense. We shine some photon to see the electron. We ignore factors and
use small-angle approximations. Also we set $c=1$.

Lets look at the uncertainty in momentum.  The electron receives
momentum from the collision of the photon with it. Suppose the energy of
the electron is $E$. We have
\begin{equation}
\Delta p_{x}\sim E\sin(\theta).
\end{equation}
What about the uncertainty in position? This comes from the diffraction
limit, we can approximate
\begin{equation}
\theta_{c}\sim\lambda/D,
\end{equation}
we can find the exact calculations from Jackson [\textit{Classical
    Electrodynamics}]. We have
\begin{equation}
\Delta x\sim f\theta_{c}\sim(f\lambda/D)\sim\lambda/\theta.
\end{equation}
So we find
\begin{subequations}
\begin{equation}
\Delta x\,\Delta p_{x}\sim E\lambda,
\end{equation}
then using the de Broglie relation $E\lambda\sim h$ gives us
\begin{equation}
\Delta x\,\Delta p_{x}\sim h.
\end{equation}
\end{subequations}
Classically, for a gravitational wave, we can have $E$ as low as we
want, and $\lambda$ as large as we want. This violates the uncertainty
principle.

If we violate the uncertainty principle, presumably all of quantum
mechanics is undermined. On the other hand, momentum conservation is
violated if the uncertainty principle is preserved.

There are limits to how accurately we can measure low energy
gravitational waves. The apparatus has to be smaller and more massive,
but that may collapse into a black hole. This may be a loophole to the
aforementioned [Heisenberg microscope] argument.
%\end{example}

\bigbreak
\noindent Although none of these are conclusive, they seem to
\emph{imply} that gravity is quantized.

\subsection{Semiclassical Gravity}

Suppose we have classical gravity and quantum fields. The Einstein field
equations become
\begin{equation}
\widehat{T}_{\mu\nu}\mid\psi\rangle = \frac{1}{8\pi} G_{\mu\nu}\mid\psi\rangle,
\end{equation}
which may be a bit too restrictive since $\widehat{T}_{\mu\nu}$ may have
noncommuting elements. On the other hand, we could make it
\begin{equation}
\langle\widehat{T}_{\mu\nu}\rangle = \frac{1}{8\pi} G_{\mu\nu}.
\end{equation}
The metric now depends on the matter field, and the matter field depends
on the metric. This becomes nonlinear, albeit a ``weak'' nonlinearity.

We can look at the Newtonian version of this:
\begin{equation}
\begin{split}
  \I\hbar\frac{\partial}{\partial t}\mid\psi\rangle
  = \left(\frac{-\hbar^{2}}{2m}\nabla^{2} + V\right)\mid\psi\rangle,\\
  \nabla^{2}V = 4\pi Gm\rho = 4\pi Gm\sum_{j}m_{j}|\psi_{j}|^{2}.
\end{split}
\end{equation}
There is a paper on this:
\begin{itemize}
\item P.J.~Salzman, S.~Carlip, ``A possible experimental test of
  quantized gravity''. \arXiv{gr-qc/0606120}, 9 pages.
\end{itemize}
Suppose we start with a single particle with a Gaussian wave
function. For small mass, it behaves like a free particle. For a large
mass, the width narrows since gravitational collapse ``wins out''. For
somewhere in between, there is nonlinear wiggling.

If we neglect the self-gravitating part, we recover the Hartree
approximation.

There is another potential problem that the covariant divergence of the
quantum stress-energy tensor is not conserved. We need to include in the
stress-energy tensor the contribution of the measurement apparatus.

\subsection{Positive Aspects of Quantizing Gravity}

There are some positive aspects of the quantization of gravity!

\begin{enumerate}
\item There are singularities in general relativity which need to be
  dealt with. This is similar to back when quantum mechanics was
  starting and we were answering questions like, ``Why doesn't the
  electron fall into the nucleus?''

\item Quantum gravity may deal with the problem of infinities in quantum
  field theory. Consider the renormalization of mass,
\begin{equation}
m(\varepsilon) = m_{0} + \frac{e^{2}}{\varepsilon},
\end{equation}
where we include the electric self-energy (which looks like
$e^{2}/\varepsilon$). If we include the classical self-energy to this,
we have,
\begin{equation}
  m(\varepsilon)
  = m_{0} + \frac{e^{2}}{\varepsilon} - \frac{Gm(\varepsilon)^{2}}{\varepsilon}.
\end{equation}
We can solve for $m(\varepsilon)$ to find that this is finite, it is
something like the Planck mass times $137$ or $1/137$.

We can also see the sum of Feynman diagrams of the gravitational
self-interaction of the electron is a finite sum,
\begin{equation}
  \eqgraph{5pt}{5pt}{  \begin{fmfgraph}(120,20)
    \fmthinpen
    %% \fmfcmd{save arrow_len; arrow_len := ahlength;
    %% save arrow_ang; arrow_ang := ahangle;}
    \fmfcmd{arrow_len := 2ahlength;% arrow_ang := ahangle;
    arrow_len := 7pt;}
\fmfleft{i1,d1}
\fmfright{o1,d2}
\fmf{fermion}{i1,i2}
\fmf{fermion}{i2,o2}
\fmf{fermion}{o2,o1}
\fmffreeze
\fmf{graviton,left}{i2,o2}
  \end{fmfgraph}
}\mspace{-24mu}
+\mspace{-24mu}
  \eqgraph{5pt}{5pt}{
\begin{fmfgraph}(132,20)
    \fmthinpen
\fmfleft{i1,d1}
\fmfright{o1,d2}
\fmfn{fermion}{i}{3}
\fmf{fermion}{i3,o3}
\fmf{fermion}{o3,o2,o1}
\fmffreeze
\fmf{graviton,left}{i2,o2}
\fmf{graviton,left}{i3,o3}
  \end{fmfgraph}
}\mspace{-24mu}
+\mspace{6mu} \cdots = \mbox{finite},
\end{equation}
despite each term being divergent! (People are finding sets of finite sums of Feynman
diagrams in supergravity. There is no proof yet.)

\item There are a few physical systems we would like to understand that
  only quantum gravity can answer. The very early universe when quantum
  effects were present as well as gravity being the dominant
  force. Black holes also may be better understood with the quantization
  of gravity.
\end{enumerate}

\subsection{Why not Quantum Gravity?}

Well, why not quantize gravity? In ordinary quantum theory, the basic
observables are local.  Consider a scalar field $\widehat{\varphi}(x)$,
the value of the field at point $x$, the axiomatic formulations of
quantum field theory these are observables.  This does not make sense,
since $x$ does not make sense.  There is no background. The symmetry of
general relativity is diffeomorphism invariance, i.e., invariance under
change of coordinates.  If $x\to x+a$, then
$\widehat{\varphi}(x)\to\widehat{\varphi}(x+a)$ which does not make
sense.

This is already an issue in classical general relativity.  We need to be
careful not to write ``the position $x$ of \emph{blah}'', but instead
``the time an atomic clock reads for a laser to reach some location''.
This is nonlocal, but what about this treatment in quantum theory?  It's
fine in classical general relativity, but we have problems in quantum
mechanics with nonlocal stuff.
