\lecture

Now the massless state for the open string
$\alpha^{i}_{-1}\mid0,k\rangle$ acting on the vacuum, this corresponds
to $A^{i}$ ($i=2,\dots,D$). This either is a massless field, or we've
broken Lorentz invariance.

For the closed string, the procedure is pretty much the same. We have
the Fourier expansion be,
\begin{equation}
X^{i} = \bar{X}^{i} + \left(\frac{p^{i}}{p^{+}}\right)\tau
+\I\sqrt{\frac{\alpha'}{2}}\sum^{\infty}_{n\neq0}\left[
\frac{\alpha^{i}_{n}}{n}\E^{-\I2\pi n(\sigma+\tau)/\ell}
+\frac{\widetilde{\alpha}^{i}_{n}}{n}\E^{-\I2\pi n(\sigma-\tau)/\ell}
  \right].
\end{equation}
If we're fixing the endpoints, the modes obey this boundary condition.
If we've got a closed string, then the waves can propagate.

Now, we find the mass squared as,
\begin{equation}
m^{2}=\frac{2}{\alpha'}\left(N+\widetilde{N}+2\left(\frac{2-D}{24}\right)\right).
\end{equation}
There are two number operators now. There is an extra symmetry, the
zero-point for $\sigma$ is arbitrary. Working through the constraints we
find that,
\begin{equation}
P = -\int\pi^{i}\partial_{\sigma}X^{i}\,\D\sigma = -\frac{2\pi}{\ell}(N-\widetilde{N})=0.
\end{equation}
For the excited state above the vacuum,
\begin{equation}
\alpha^{i}_{-1}\widetilde{\alpha}^{j}_{-1}\mid0,0,k\rangle\qquad m^{2}=\frac{2}{\alpha'}\left(2+2\left(\frac{2-D}{24}\right)\right)
\end{equation}
we see $m^{2}=0$ if $D=26$.

\marginnote{String spectrum:\\$G^{ij}$ graviton\\$\Phi$ dilaton\\$B^{ij}$ axion}
Since $\alpha$, $\widetilde{\alpha}$ are not symmetric, we can break
their product up into a symmetric traceless part (which transforms as a
spin-2 particle) $G^{ij}$, the trace which is just a scalar $\Phi$
called the dilaton, and there is also the antisymmetric part $B^{ij}$
(sometimes referred to as the axion). The axion acts like a sort of
gauge potential,
\begin{subequations}
\begin{equation}
B_{ij}\to B_{ij} +\partial_{i}\Lambda_{j}-\partial_{j}\Lambda_{i}.
\end{equation}
The analog of the field-strength tensor would be,
\begin{equation}
H_{ijk} = \partial_{i}B_{jk} + \partial_{j}B_{ki} + \partial_{k}B_{ij}.
\end{equation}
\end{subequations}
This is the string spectrum (or particles present) in our action.

Now, let's look at a string interaction. We will examing the asymptotic
behaviour of incoming and outgoing strings, something like:
\begin{center}
  $\vcenter{\hbox{\includegraphics{img/2009-05-15.1}}}$\quad$\longrightarrow$\quad$\vcenter{\hbox{\includegraphics{img/2009-05-22.0}}}$
\end{center}
Consider the metric of a 2-dimensional cylinder, labeling the height
with the $r$ coordinate,
\begin{equation}
\D s^{2} = \E^{2\sigma}(\D r^{2} + \D\theta^{2})
\end{equation}
Let $z=\exp(-r+\I\theta)$ and $\bar{z}=\exp(-r-\I\theta)$, then as
$r\to\infty$ we have $z\to0$ and $\bar{z}\to0$. We similarly find, as
$r\to0$ that $z\to\exp(+\I\theta)$ (that is, $z$ and $\bar{z}$ go to the unit circle)
This gives us a way to describe interactions using the unit disk.
\begin{center}\includegraphics{img/2009-05-22.1}
\end{center}
The state-operator correspondence specifies the outcome of a string
interaction by examining an operator near the center of the disk. We
have
\begin{equation}
\begin{split}
\alpha^{\mu}_{-m} &= \sqrt{\frac{2}{\alpha'}}\int
z^{-m}\partial_{z}X^{m}\frac{\D z}{2\pi}\\
&\sim (\partial_{z}^{m}X^{\mu})(0).
\end{split}
\end{equation}
This is a vertex operator. In practice we rarely work with string
diagrams, we usually have vertex operators acting on some [Riemann] sphere.
We see that
\begin{equation}
\mid0,k\rangle = \normalorder{\E^{\I kx}}\mid0,0\rangle.
\end{equation}
Then we have a \define{Vertex Operator},
\begin{equation}
V = (\mbox{const.})\int\left[
(\gamma^{ab}S_{\mu\nu}+\I\varepsilon^{ab}a_{\mu\nu})\partial_{a}X^{\mu}\partial_{b}X^{\nu}
+\alpha'\phi\,{}^{(2)}\!R\E^{\I kx}
  \right]\sqrt{-\gamma}\,\D^{2}\sigma
\end{equation}
where $S_{\mu\nu}$ is symmetric, $a_{\mu\nu}$ is antisymmetric which
corresponds to the graviton and axion vertex operators (respectively).
The last term corresponds to the dilaton. We can work backwords starting
from
\begin{equation}
G_{\mu\nu} = \eta_{\mu\nu}+S_{\mu\nu},
\end{equation}
then the action
\begin{equation}
\action = \int \gamma^{ab}\partial_{a}X^{\mu}\partial_{b}X^{\nu}G_{\mu\nu}\sqrt{-\gamma}\,\D^{2}\sigma
\end{equation}
has its path-integral's integrand expands in powers of $S_{\mu\nu}$ like
\begin{equation}
\E^{\I\action}=\E^{\I\action(G=\eta)}\left(1 + \int\gamma^{ab}\partial_{a}X^{\mu}\partial_{b}X^{\nu}S_{\mu\nu}\sqrt{-\gamma}\,\D^{2}\sigma+\dots\right).
\end{equation}
