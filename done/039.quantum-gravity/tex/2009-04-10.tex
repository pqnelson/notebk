\lecture

I was too sick to attend, but I have been told: Professor Carlip argued
the gauge symmetries of general relativity are isometries described by
Killing equation, derived ADM coordinates including lapse and shift
functions, described extrinsic curvature in terms of lapse and shift,
rewrote the Einstein--Hilbert action in ADM coordinates, derived canonically
conjugate momentum to metric, wrote first-order form of the action,
argued lapse and shift functions are Lagrange multipliers.

In second-order formalism positions $x$ and velocities $\dot{x}$ are treated as independent
variables, but first-order formalism treats positions $x$ and momenta $p$
as independent variables. Also, Professor Waldron refers to the ADM
action's terms as:
\begin{equation}
\action_{ADM} = \int(\underbrace{\pi^{ij}\dot{q}_{ij}}_{\substack{\text{symplectic}\\\text{term}}}
- \underbrace{N_{i}\mathcal{H}^{i} - N\mathcal{H}}_{\text{constraints}})\,\D^{4}x.
\end{equation}
Also notation: $D_{i}$ determined using the spatial metric $q_{ij}$
such that $D_{i}q_{jk}=0$.

\textbf{Caveat: these are notes I've written, not based on Dr Carlip's lectures,
but from what I've learned over the years.}

We start by choosing some coordinate $t$ and foliate spacetime with
spacelike hypersurfaces $\Sigma_{t}$ indexed by $t$. We have the unit normal
$n^{\mu}$ on each hypersurface $\Sigma_{t}$, as well as the induced
3-metric $q_{ij}$. For spacelike hypersurfaces
\begin{equation}
n^{\mu}n_{\mu}=+1.
\end{equation}
We then have a projection of tensors onto their spatial components
\begin{equation}
{h^{\mu}}_{\nu} = {\delta^{\mu}}_{\nu} - n^{\mu}n_{\nu}.
\end{equation}
We define the extrinsic curvature as the spatial projection of the
covariant derivative for the unit normal,
\begin{equation}
K_{\mu\nu} = {h_{\mu}}^{\rho}\nabla_{\rho}n_{\nu}.
\end{equation}
It's not hard to see $n^{\mu}K_{\mu\nu}=0$ and $n^{\nu}K_{\mu\nu}=0$
(since it's a spatial tensor).

Now, we have the ADM decomposition of the metric. We begin with the line
element, using the Lorentzian analog of the Pythagorean theorem.
Intuitively, we should imagine something like the picture:
\begin{center}
\includegraphics{img/2009-04-10.0}
\end{center}
We start off with a point on a hypersurface $\Sigma_{t}$. If we
translate along the time dimension from $t\to t+\D t$, then we end up at
$x^{i}+N\,\D t$ --- this is because the coordinates are arbitrary, but
$N\,\D t$ should be the ``infinitesimal proper time'' not the
``infinitesimal coordinate time''. Similarly, we could have some
rotational effect, which we would account for by adding a translation on
$\Sigma_{t+\D t}$ by $-N^{i}\,\D t$. This gives us the line element
\begin{equation}
\D s^{2} = N^{2}\,\D t^{2} - q_{ij}(\D x^{i} + N^{i}\,\D t)(\D x^{j} + N^{j}\,\D t).
\end{equation}
Here $N$ is called the ``Lapse function'', the $N^{i}$ are called the
``Shift vector''.

It's not too hard (I think it's an exercise
in \hyperref[section:hw2]{homework 2}) to show that
\begin{equation}
K_{ij} = \frac{1}{2N}(\partial_{t}q_{ij} - D_{i}N_{j}-D_{j}N_{i})
\end{equation}
where $D_{j}$ is the spatial covariant derivative (compatible with
$q_{ij}$, i.e., $D_{i}q_{jk}=0$). We also find the inverse 4-metric
decomposes like
\begin{equation}
g^{ab} = \begin{pmatrix}\frac{1}{N^{2}} & -\frac{N^{i}}{N^{2}}\\-\frac{N^{j}}{N^{2}} & -q^{ij} + \frac{N^{i}N^{j}}{N^{2}} \end{pmatrix},
\end{equation}
where $N^{i}=q^{ij}N_{j}$, and $q^{ij}$ is the inverse of $q_{ij}$
(i.e., $q^{ij}q_{jk}=\delta^{i}_{k}$).

We rewrite the Lagrangian using the Gauss--Codazzi equations
\begin{equation}
{}^{(4)}R = {}^{(3)}R + K_{ij}K^{ij} - K^{2} - 2\nabla_{\mu}(n^{\mu}\nabla_{\nu}n^{\nu}-n^{\nu}
\nabla_{\mu}n^{\mu}).
\end{equation}
Then the action
\begin{subequations}
\begin{align}
\action_{EH} &= \frac{1}{16\pi G}\int {}^{(4)}R\,\sqrt{-g}\,\D^{4}x\\
&= \frac{1}{16\pi G}\iint [{}^{(3)}R + K_{ij}K^{ij} - K^{2}](N\sqrt{q})\,\D^{3}x\,\D t
+ \mbox{(boundary terms)}.
\end{align}
\end{subequations}
We find the conjugate momenta to the 3-metric $q_{ij}$ are
\begin{equation}
\pi^{ij} = \frac{\partial L}{\partial(\partial_{t}q_{ij})}
= \frac{1}{16\pi G}(K^{ij} - q^{ij}K).
\end{equation}
Using this, we can write the canonical action
\begin{equation}
\action = \frac{1}{16\pi G}\iint(\pi^{ij}\partial_{t}q_{ij} - \mathcal{H}_{\text{can}})\,\D^{3}x\,\D t,
\end{equation}
where $\mathcal{H}_{\text{can}} = \pi^{ij}\partial_{t}q_{ij} - \mathcal{L}$.

We should expect there to be constraints, since the components of the
metric $N$ and $N_{i}$ do not enter the action with any time derivatives.
In fact, it turns out we have
\begin{subequations}
\begin{align}
\mathcal{H} &= \frac{16\pi G}{\sqrt{q}}(\pi_{ij}\pi^{ij} - \frac{1}{2}\pi^{2})
-\frac{1}{16\pi G}\sqrt{q}\,{}^{(3)}\!R\\
\intertext{and}
\mathcal{H}^{i} &= -2D_{j}\pi^{ij}
\end{align}
\end{subequations}
are the two constraints, called the Diffeomorphism constraint (or
Hamiltonian constraint) and the Momentum constraints, respectively.

The Poisson brackets would be defined on a spatial hypersurface (so
$t=\mbox{constant}$) as
\begin{equation}
\{q_{ij}(\vec{x}), \pi^{k\ell}(\vec{x}')\}
= \frac{1}{2}(\delta^{k}_{i}\delta^{\ell}_{j} + 
\delta^{k}_{j}\delta^{\ell}_{i})\widetilde{\delta}^{(3)}(\vec{x}-\vec{x}'),
\end{equation}
where we use the densitized delta $\delta^{(3)}(\vec{x})\sqrt{q}=\widetilde{\delta}^{(3)}(\vec{x})$
since it satisfies:
\begin{equation}
\int\widetilde{\delta}^{(3)}(\vec{x})\,\D^{3}x=1.
\end{equation}
A number of exercises concerning the Poisson bracket may be found
in \hyperref[section:hw3]{Homework 3}. In particular, the Poisson
bracket of the constraints generate diffeomorphisms (morally speaking).