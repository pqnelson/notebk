\lecture

(Remark: If we can write the constraints in two independent groups, then
we can do a mixture of Dirac quantization and reduced phase-space
quantization.)

Last time we ended up with a kind of gauge-like field,
\begin{equation}
A^{(\immirzi)\widehat{I}}_{i} =  {\Gamma_{i}}^{\widehat{I}} + \immirzi{K_{i}}^{\widehat{I}}.
\end{equation}
We can write this gauge-like field in terms of the spin connection as:
\begin{equation}
A^{(\immirzi)\widehat{I}}_{i} =  \frac{1}{2}\epsilon^{0\widehat{I}\widehat{J}\widehat{K}}\omega_{i\widehat{J}\widehat{K}}
+\immirzi {\omega_{i}}^{0\widehat{I}}.
\end{equation}
We can think of this as a canonical transformation. In the ADM
formalism, ${K_{i}}^{\widehat{I}}$ is more or less the canonical
conjugate momentum, and we're adding some terms involving derivatives of
the tetrad to it.

The next step is slightly dodgy, but makes the math easier. We gauge fix
Lorentz-boosts:
\begin{equation}
{e^{t}}_{\widehat{I}}=0.
\end{equation}
If we don't do this, then we get second-class constraints. This may give
a different representation (there is some evidence of it yielding a
different representation\footnote{Unfortunately, I didn't ask for
references on this.}). We can now define
\begin{subequations}
\begin{align}
{e^{t}}_{\widehat{0}} &= 1/\lapse\\
{e^{i}}_{\widehat{0}} &= -\shift^{i}/\lapse,
\end{align}
\end{subequations}
where $\lapse$ is the Lapse function and $\shift^{i}$ is the shift function
both from the ADM formalism.We have
\begin{equation}
q^{ij} = {e^{i}}_{\widehat{I}}e^{j\widehat{I}},
\end{equation}
so we can write
\begin{subequations}
\begin{align}
g^{ij} &= q^{ij} - \frac{\shift^{i}\shift^{j}}{\lapse^{2}}\\
  &={e^{i}}_{\widehat{0}}e^{j\widehat{0}} + {e^{i}}_{\widehat{I}}e^{j\widehat{I}}.
\end{align}
\end{subequations}
With this gauge fixing, we recover the ADM decomposition of the metric.

\begin{notation}
  Let's define
  \begin{equation}
{\widetilde{E}^{i}}_{\widehat{I}} := \sqrt{q}{e^{i}}_{\widehat{I}}.
  \end{equation}
It is a tensor density, and it is a triad on a spatial hypersurface.
\end{notation}

Given all of this, we can go back to the Einstein--Hilbert action, do
all the computations, we find:
\begin{equation}
  \action = \frac{1}{8\pi G}\int\left(
  \frac{1}{\immirzi}{A_{i}}^{\widehat{I}}\frac{\D}{\D t} {{\widetilde{E}}^{i}}_{\phantom{i}\widehat{I}}
-\underbrace{\I A_{0\widehat{I}}G^{\widehat{I}} + \I\shift^{i}V_{i} - \frac{1}{2}\frac{\lapse}{\sqrt{q}}S}_{\text{constraints}}
  \right)\,\D^{3}x\,\D t.
\end{equation}
If we didn't impose our gauge-fixing condition, we'd have a more
complicated constraint algebra and one more constraint. Now, let us
examine these constraints:
\begin{enumerate}[nosep,label=(\arabic*)]
\item We have $G^{\widehat{I}} = D_{i}\widetilde{E}^{i\widehat{I}}$
  where $D_{i}$ is the gauge covariant derivative treating this as a
  gauge theory.
\item $V_{j} = \widetilde{E}^{i}_{\phantom{i}\widehat{I}}{F_{ij}}^{\widehat{I}}$
  where ${F_{ij}}^{\widehat{I}} = \partial_{i}{A_{j}}^{\widehat{I}} - \partial_{j}{A_{i}}^{\widehat{I}}+\epsilon^{\widehat{I}\widehat{J}\widehat{K}}A_{i\widehat{J}}A_{j\widehat{K}}$.
  (This should ring a bell as the field strength tensor for a nonabelian
  gauge theory.) Observe these two do not involve the Immirzi parameter
  $\immirzi$ directlry.
\item The remaining constraint is a monster:
  \begin{equation}
S = \epsilon^{\widehat{I}\widehat{J}\widehat{K}}\widetilde{E}^{i}_{\widehat{I}}\widetilde{E}^{j}_{\widehat{J}}F_{ij\widehat{K}}
-2\left(\frac{1+\immirzi^{2}}{\immirzi^{2}}\right)\widetilde{E}^{i}_{[\widehat{I}}\widetilde{E}^{j}_{\widehat{J}]}(A^{(\immirzi)\widehat{I}}_{i}-\Gamma_{i}^{\phantom{i}\widehat{I}})
(A^{(\immirzi)\widehat{J}}_{j}-\Gamma_{j}^{\phantom{j}\widehat{J}}).
  \end{equation}
  The factor of $A^{(\immirzi)\widehat{I}}_{i}-\Gamma_{i}^{\phantom{i}\widehat{I}}$
  should remind us of the extrinsic curvature.
\end{enumerate}\medbreak

Let us consider the Poisson brackets of quantities.
\begin{equation}
\{\widetilde{E}^{i}_{\widehat{I}}(x),A^{(\immirzi)\widehat{J}}_{j}(x')\}
= -8\pi G\immirzi{\delta_{\widehat{I}}}^{\widehat{J}}\widetilde{\delta}^{(3)}(x-x').
\end{equation}
If we look at this as a nonabelian gauge theory, the Poisson bracket
looks like an electric field and potential. The constraint
$D_{i}\widetilde{E}^{i\widehat{I}}=G^{\widehat{I}}$ looks like Gauss's
law.\index{Gauss Law}

It looks like the physical phase space of an $\SU(2)$ gauge theory.
We're then imposing two additional constraints, and calling the result
quantum gravity. The natural thing to do is treat the
$A^{(\immirzi)\widehat{J}}_{j}$ as positions and
the $\widetilde{E}^{i}_{\widehat{I}}$ as momenta.

If we work at $G^{\widehat{I}}$, it tells us the wave functions are
gauge invariant, the $V^{i}$ constraints generate spatial
diffeomorphisms, and the $S$ is the Hamiltonian constraint. We have:
  \begin{equation}
S = \underbrace{\epsilon^{\widehat{I}\widehat{J}\widehat{K}}\widetilde{E}^{i}_{\widehat{I}}\widetilde{E}^{j}_{\widehat{J}}F_{ij\widehat{K}}}_{\text{scalar curvature term}}
    -\underbrace{2\left(\frac{1+\immirzi^{2}}{\immirzi^{2}}\right)\widetilde{E}^{i}_{[\widehat{I}}\widetilde{E}^{j}_{\widehat{J}]}(A^{(\immirzi)\widehat{I}}_{i}-\Gamma_{i}^{\phantom{i}\widehat{I}})
(A^{(\immirzi)\widehat{J}}_{j}-\Gamma_{j}^{\phantom{j}\widehat{J}})}_{\text{the $\pi^{2}$ term}}.
  \end{equation}
The $\pi^{2}$ term is ugly since the $\Gamma$ term depend on $E$, so we
have a constraint with quadratic terms in $E$.

There is a trick here, discovered originally by Thiemann, called the
Thiemann trick\index{Thiemann trick}, where we represent the ugly term
in terms of nested Poisson brackets. So it is possible to make it really
pretty.

Let's forget the Hamiltonian constraint for the time being. Let's try to
solve the other constraints, beginning with the Gauss's Law constraint.
The constraints
\begin{equation}
G^{\widehat{I}} = 0
\end{equation}
implies the wave functionals $\Psi$ are gauge-invariant. We will write
\begin{equation}
A = {A^{\widehat{I}}}_{i}\,\D x^{i}\,\tau_{\widehat{I}}
\end{equation}
where $\tau_{\widehat{I}}$ are the generators of $\SU(2)$ or $\SO(3)$
depending on gauge, let $g\in\SU(2)$ then the $A$ field transforms like:
\begin{equation}\label{eq:2009-05-01:gauge-transform-A}
A\to g^{-1}\,\D g + g^{-1}Ag.
\end{equation}
(For electromagnetism, $g^{-1}Ag = g^{-1}gA=A$ and $g^{-1}\,\D g =\D\Lambda$.)
Recall the field strength 2-form is then,
\begin{equation}
F = \D A + A\wedge A.
\end{equation}
We see, from $\D(gg^{-1})=0$ we have,
\begin{equation}
\D(g^{-1}) = -g^{-1}\,(\D g) g^{-1}.
\end{equation}
In particular,
\begin{equation}
\D(g^{-1}\,\D g + g^{-1}Ag) = - g^{-1}\,\D g\, g^{-1}\,\D g - g^{-1}\,\D g\,g^{-1}Ag +
g^{-1}\,\D A\,g - g^{-1}A\,\D g.
\end{equation}
Then applying this and Eq~\eqref{eq:2009-05-01:gauge-transform-A} to the field
strength 2-form gives us,
\begin{subequations}
\begin{align}
F\to F' &= \D(g^{-1}\,\D g + g^{-1}Ag) + (g^{-1}\,\D g + g^{-1}A g)\wedge(g^{-1}\,\D g + g^{-1}A g)\\
&= g^{-1}\,\D A\, g + (g^{-1}Ag)\wedge(g^{-1}Ag)\\
&= g^{-1}(\D A + A\wedge A)g\\
&= g^{-1}Fg.
\end{align}
\end{subequations}
This isn't terribly surprising, it's basic differential geometry.

The kinetic term is
\begin{equation}
\Tr(F^{2}) = F^{\widehat{I}}_{\mu\nu}F^{\mu\nu}_{\widehat{I}}
\end{equation}
for the gauge field. It's invariant under gauge transformations.

\index{Holonomy|(}
Now we would like to construct a basis of gauge invariant quantities
(easier said than done). But we can consider the parallel transport of a
gauge field on a cloed curve on a surface, the holonomy is gauge
invariant! We have the parallel transport,
\begin{equation}
\frac{\D v^{\widehat{I}}}{\D s}
+ \frac{\D x^{i}}{\D s}\left({A_{i}}^{\widehat{I}}{{\epsilon_{\widehat{K}}}^{\widehat{I}}}_{\widehat{J}}\right)v^{\widehat{J}} = 0.
\end{equation}
This is the equation for parallel transport, it's basic differential
geometry.\footnote{See, e.g., \S13 of my notes on general relativity
\url{http://pqnelson.github.io/assets/notebk/GR.pdf}.} The result is that
\begin{equation}
v^{\widehat{I}}(s) = \mathcal{U}^{\widehat{I}}_{\phantom{I}\widehat{J}}(s,s_{0})v^{\widehat{J}}(s_{0}),
\end{equation}
where we have the path-ordering exponential,
\begin{equation}
\mathcal{U}^{\widehat{I}}_{\phantom{I}\widehat{J}}(s,s_{0}) =
\mathcal{P}\exp\left(-\int^{s}_{s_{0}}{A_{i}}^{\widehat{K}}\underbrace{\epsilon^{\widehat{I}}_{\phantom{I}\widehat{J}\widehat{K}}}_{\text{generators}}\,\D x^{i}\right).
\end{equation}
We can generalize to any representation of $\SU(2)$, just replace the
$\epsilon^{\widehat{I}}_{\phantom{I}\widehat{J}\widehat{K}}$
``generators'' factor with $\tau_{\widehat{K}}$.

This is more general than curvature, we're not doomed to infinitesimal nightmares.

We see, taking care with ordering due to noncommutativity, that:
\begin{equation}\label{eq:2009-05-01:enlightenment}
\frac{\D}{\D s}\mathcal{U}(s,s_{0}) = -A(s)\,\mathcal{U},
\end{equation}
and similarly,
\begin{equation}
\frac{\D}{\D s_{0}}\mathcal{U}(s,s_{0}) = -\mathcal{U}A(s_{0}).
\end{equation}
Let $\widetilde{\mathcal{U}} = g(s)^{-1}\mathcal{U}g(s_{0})$, so we
have:
\begin{subequations}
\begin{align}
\frac{\D}{\D s}\widetilde{\mathcal{U}}
&= - g^{-1}(s)\frac{\D g(s)}{\D s}\widetilde{\mathcal{U}}-g^{-1}(s)A\mathcal{U}g(s_{0})\\
&= \left(-g^{-1}Ag - g^{-1}\frac{\D g}{\D s}g\right)\widetilde{\mathcal{U}}\\
&= - \widetilde{A}\,\widetilde{\mathcal{U}}.
\end{align}
\end{subequations}
We read this backwards, when
\begin{equation}
A\to g^{-1}\,\D g + g^{-1}Ag,
\end{equation}
we simply have
\begin{equation}
\mathcal{U}(s,s_{0})\to g^{-1}(s)\mathcal{U}(s,s_{0})g(s_{0}).
\end{equation}
\textbf{Note:} redo the computations starting from Eq~\eqref{eq:2009-05-01:enlightenment}
for enlightening insight.
\index{Holonomy|)}

In particular, for a closed curve $C$, we find
$\Tr(\mathcal{U}(s,s_{0}))$ is gauge invariant. This is the Wilson
loop,\index{Wilson Loop} and it gives us a complete set of gauge
invariant variables for a gauge theory.

There is one problem with this, the Wilson loops give an overcomplete
set of variables (i.e., they're not all linearly independent of each
other, due to the Mandelstam identities):
\begin{equation}
\mathcal{U}_{C_{1}}\mathcal{U}_{C_{2}}=\mathcal{U}_{C_{1}\circ C_{2}} + \mathcal{U}_{C_{1}\circ C_{2}^{-1}},
\end{equation}
where $C_{1}$, $C_{2}$ are closed curves sharing a point, as doodled
below:
\begin{center}
  \includegraphics{img/2009-05-01.0}
\end{center}
There's a nice basis called the \define{Spin Network Basis}, and we can
claim to solved 3 of the constraints of quantum gravity.
