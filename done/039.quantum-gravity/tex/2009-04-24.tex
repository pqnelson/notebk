\lecture

Spacelike hypersurfaces defined by metric, but in general we don't know
the metric in quantum gravity, so we're out of luck. (We are assuming
that $\mathcal{M}=\mathbb{R}\times\Sigma$ where $\mathbb{R}$ is time,
$\Sigma$ is a spatial 3-manifold, at least in the canonical approach.)
Anyways, back to the Dirac approach.

We're imposing the constraints as operators on the wave function. We
interpret the momentum constraint
\begin{equation}
\widehat{\mathcal{H}}^{i}\Psi[q] = 0
\end{equation}
as telling us the wave function is invariant under spatial
diffeomorphisms. We should be able to, at least have the urge to assume
$\widetilde{H}$ is telling us the wave function is invariant under
temporal diffeomorphism but realize that this is meaningless. We're on a
spatial hypersurfaces, after all!

The DeWitt supermetric\index{DeWitt Supermetric}\index{Supermetric}
\begin{equation}
G_{ijk\ell} = \frac{1}{2}\frac{1}{\sqrt{q}}(q_{ik}q_{j\ell} +
q_{i\ell}q_{jk} - q_{ij}q_{k\ell}).
\end{equation}
This is like a metric of metrics. We have the deformation of a metric
$\delta q^{ij}$ have the length
\begin{equation}
\|\delta q^{ij}\|^{2} = \int G_{ijk\ell}\delta q^{ij}\delta q^{k\ell}\,\D^{3}x.
\end{equation}
This defines the distance on the space of metrics. (The signature of the
supermetric is $(-+++++)$, we take each pair of indices as a single
index resulting in a 6-by-6 matrix.)

We introduce
\begin{equation}
\widehat{\pi}^{ij}=-\I\frac{\delta}{\delta q_{ij}}.
\end{equation}
We plug it into the Hamiltonian constraint, and write:
\begin{equation}
\widehat{\mathcal{H}} = 16\pi G G_{ijk\ell}\frac{\delta}{\delta q_{ij}}\frac{\delta}{\delta q_{k\ell}}
+\frac{1}{16\pi G}\sqrt{q}\,\threeRicci.
\end{equation}
Resist the urge to make the first term a Laplacian. The Ricci 3-scalar
$\threeRicci$ is intuitively a sort of potential term, when viewed as a
function of $q_{ij}$. So then we plug it back into
\begin{equation}
\widehat{\mathcal{H}}\Psi[q]=0.
\end{equation}
This is the Wheeler--DeWitt Equation.\index{Wheeler--DeWitt Equation}

We need an inner product, wave functions alone do not suffice for a
quantum theory. There are 2 obvious thing to try to do.

The first thing, the ordinary Schrodinger picture using the 3-metric
\begin{equation}
\int\Psi^{*}\Phi\,[\D q] = \infty
\end{equation}
always since the Hamiltonian constraint, we need to gauge fix the inner
product, like a path integral with some extra symmetry.

\begin{itemize}
\item R.~P.~Woodard,
``Enforcing the Wheeler-de Witt Constraint the Easy Way''.
\journal{Class.~Quant.~Grav.} \volume{10} (1993), 483--496.\\
{\tt\doi{10.1088/0264-9381/10/3/008}}
\end{itemize}

We can think of $\widehat{\mathcal{H}}\Psi=0$ as a sort of Klein--Gordon
equation, and the correct inner product there is:
\begin{equation}
\int\Psi^{*}\overleftrightarrow{\frac{\delta}{\delta q}}\Phi\,[\D q].
\end{equation}
There is some ambiguity here, we have a number of inner products since
$\delta/\delta q$ is nonunique.

We could quantize the Wheeler-DeWitt equation, which is a third
quantization. This creates and annihilates metrics (which correspond to
universes), which we do not really observe.

There is another approach to finding an inner product which Woodard
proposes, which for simple models looks like the Klein-Godon inner
product. The wave function encodes some information about the placement
of the spatial hypersurface in the universe, which has some information
about time.

Another technical problem is the first term of $\widehat{\mathcal{H}}$
has two functional derivatives, which is problematic. We could try to
put in some regulator, so the first term looks like:
\begin{equation}
\widehat{\mathcal{H}} = \lim_{x\to x'}\widetilde{G}_{ijk\ell}\frac{\delta}{\delta q_{ij}(x)}\frac{\delta}{\delta q_{k\ell}(x')}+\dots.
\end{equation}
We need to show the result is independent of regularization. We also
need to be conscious of the Poisson bracket
$\{\mathcal{H},\mathcal{H}^{i}\}$ must be recovered from the commutator
$[\widehat{\mathcal{H}},\widehat{\mathcal{H}}^{i}]$ with our own
regularization.

\subsection{Perturbative Expansion}

The other thing we could try is a perturbative expansion, which is
natural if we cannot get an exact solution.\footnote{A good reference
for this subsection is Claus Kiefer~\cite{Kiefer:2004xyv}, \emph{Quantum Gravity}, third
edition, section 5.4.} We assume the wave
functional $\Psi$ satisfies the momentum constraints.

We can do what is roughly the Born--Openheimer approximation, wherein we
couple gravity and matter. (Basic idea of the Born--Openheimer
approximation is we have 2 independent processes, e.g., there is some
background on which matter moves slowly, but there is some
backreaction.)

Let us write:
\begin{equation}
\left(16\pi G\hbar G_{ijk\ell}\frac{\delta}{\delta q_{ij}}\frac{\delta}{\delta q_{k\ell}}
+\frac{1}{16\pi G\hbar}\sqrt{q}\,\threeRicci + \mathcal{H}_{m}\right)\Psi = 0,
\end{equation}
where we have the matter Hamiltonian $\mathcal{H}_{m}\approx T_{00}$.
Let us do a sort of WKB approximation:
\begin{equation}
\Psi = A\exp\left(\frac{\I}{32\pi G\hbar}S_{0}\right).
\end{equation}
We can expand in powers of the Planck length.
The lowest order expansion is just
\begin{equation}\label{eq:2009-04-24:perturbative-expansion-wd:zeroeth-order}
\frac{-1}{4}G_{ijk\ell}\frac{\delta S_{0}}{\delta q_{ij}}\frac{\delta S_{0}}{\delta k\ell}
+\sqrt{q}\,\threeRicci = 0.
\end{equation}
(This is the Hamilton--Jacobi equation for gravity uncoupled to matter.)
So at this level we have some background that's fixed and looks
classical.

The next order:
\begin{equation}\label{eq:2009-04-24:perturbative-expansion-wd:first-order}
\I G_{ijk\ell}\frac{\delta S_{0}}{\delta q_{ij}}\frac{\delta A}{\delta q_{k\ell}}
+\frac{\I}{2} G_{ijk\ell}\left(\frac{\delta^{2} S_{0}}{\delta q_{ij}\,\delta q_{k\ell}}\right)A
+\mathcal{H}_{m}A = 0.
\end{equation}
If we are clever, we can choose the functional derivative of $A$
to \emph{look like}:
\begin{subequations}
\begin{equation}
\frac{\delta A}{\delta q_{k\ell}}\sim\mbox{``}\frac{\D}{\D t}A\mbox{''}.
\end{equation}
Remember $A$ is the coefficient for our wave functional $\Psi$.
Explicitly,
\begin{equation}
A = D[q]\widetilde{\Psi},
\end{equation}
choose $D$ such that
\begin{equation}
\frac{\I}{2}G_{ijk\ell}\left(\frac{\delta^{2} S_{0}}{\delta q_{ij}\,\delta q_{k\ell}}\right)D
+ \I G_{ijk\ell}\frac{\delta D}{\delta q_{ij}}\frac{\delta A}{\delta q_{k\ell}}=0.
\end{equation}
\end{subequations}
Then,
\begin{equation}
\I G_{ijk\ell}\frac{\delta S_{0}}{\delta
q_{ij}}\frac{\delta\widetilde{\Psi}}{\delta q_{k\ell}} + \mathcal{H}_{m}\widetilde{\Psi}=0.
\end{equation}
This looks a lot like the Schrodinger equation. We have
\begin{equation}
G_{ijk\ell}\frac{\delta S_{0}}{\delta
q_{ij}} \mathrel{\mbox{``$\sim$''}}\frac{\delta q_{k\ell}}{\delta T}
\end{equation}
where $T$ is ``time'' (we don't know exactly what this is in quantum
gravity). We can be far more rigorous in certain midisuperspace models.

We can go to higher orders, where we get backreaction, where $S_{0}$
gets corrections from $S_{2}$ (the effects of gravity self-gravitating).
Barvinsky has worked out a systematic formalism using doodles that look
like Feynman diagrams.\footnote{Although a reference was not given, I
believe it is Barvinsky and Kiefer~\cite{Barvinsky:1997hp}.} It's not known if the approximation is renormalizable.

The zeroeth order
Eq~\eqref{eq:2009-04-24:perturbative-expansion-wd:zeroeth-order}
describes how spacetime curves, the first-order corrections in
Eq~\eqref{eq:2009-04-24:perturbative-expansion-wd:first-order} tells matter
how to move, the second-order correction tells spacetime curves due to matter,
then the third-order correction tells matter how to react to third-order
corrections, and so on.

We can do cosmology in this formalism. (Halliwell(?) did some old work
here.)\footnote{I think this refers to Jonathan
Halliwell~\arXiv{gr-qc/9208001}, possibly other papers.} Time has sort of emerged, which is nice, but this tells us that
time emerges when the universe is approximately classical. (What about
in other universes?)

\subsection{Strong Coupling Limit}

There's another approximation that has appeal. That is to take $\hbar G$
as large (the so-called \define{Strong Coupling Approximation}). This
might be good to tell us about the small scale structure of
spacetime.\footnote{Professor Carlip wrote a review paper with a good
discussion of this approximation in \S2 of \arXiv{1009.1136}.}
The leading order contribution in the Wheeler-DeWitt equation is the
first term. This tells u s that the metric ``decouples'' at each point.
To lowest order we have ``almost independent'' metrics at each point.

Classically, at each point, the general solution is the Kasner universe
\begin{equation}
\D s^{2} = -\D t^{2} + \E^{2p_{1}}\D x^{2} + \E^{2p_{2}}\D y^{2}
+ \E^{2p_{3}}\D z^{2},
\end{equation}
where, at lowest order, the terms $p_{j}$ are constants satisfying:
\begin{subequations}
\begin{equation}
p_{1}+p_{2}+p_{3}=1,
\end{equation}
and
\begin{equation}
p_{1}^{2}+p_{2}^{2}+p_{3}^{2}=1.
\end{equation}
\end{subequations}
The next order correction treats the $p_{j}$ as slowly-varying terms.

\begin{center}
\includegraphics{img/2009-04-24.0}
\end{center}

Misner called this the Mixmaster universe. There's a huge literature on
this (lookup the Belinskii-Khalatnikov-Lifshitz [BKL] model). It is
conjectured that near the Big Bang, the universe behaved this way.

At higher-order corrections, the oscillations look like:

\begin{center}
\includegraphics{img/2009-04-24.1}
\end{center}

The constraints imply $p_{i}>0$, $p_{j}>0$ and $p_{k}<0$ where $i\neq j\neq k$
and $i,j,k=1,2,3$.
