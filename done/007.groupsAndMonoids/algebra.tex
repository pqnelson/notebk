%%
%% algebra.tex
%% 
%% Made by Alex Nelson
%% Login   <alex@tomato>
%% 
%% Started on  Mon Dec 22 11:28:40 2008 Alex Nelson
%% Last update Mon Dec 22 11:28:40 2008 Alex Nelson
%%
\documentclass[10pt,draft]{article}
\usepackage{fly}
% for usuage of conditionals see http://zyliu2005.blogspot.com/2007/06/latex-how-to-use-conditional.html
\newboolean{useIndex}
\setboolean{useIndex}{false}
%\setboolean{useIndex}{true}
\usepackage{fly}
\ifthenelse{\boolean{useIndex}}%
{%
\usepackage{makeidx} % for the index e.g. \index{key}
\makeindex%
}%
{\renewcommand{\index}[1]{ }}
\def\slug{\hbox{\kern1.5pt\vrule width2.5pt height6pt depth1.5pt\kern1.5pt}}
%\def\qef{~\rule[-1.25pt]{3pt}{8pt}}
\let\qef=\slug
\let\imag=\im
\title{Notes on Abstract Algebra}
\author{Alex Nelson}
\date{December 22, 2008}
\begin{document}
\maketitle
\tableofcontents
\listoffigures
\section*{Introduction}
\begin{rmk}
Just a prefatory note, to indicate an example has ended I
will use either `QEF' or \qef ~and for the end of a proof, I
will use an empty square; to show that a sketch of a proof
has ended I will use a spade $\spadesuit$.
\end{rmk}
%%
%% algebraIndex.tex
%% 
%% Made by Alex Nelson
%% Login   <alex@tomato>
%% 
%% Started on  Mon Dec 22 11:31:48 2008 Alex Nelson
%% Last update Fri Jan 30 12:35:29 2009 alex
%%
\graphicspath{img}
\section{Groups}
\subsection{Monoids}
%%
%% monoids.tex
%% 
%% Made by Alex Nelson
%% Login   <alex@tomato>
%% 
%% Started on  Mon Dec 22 11:22:31 2008 Alex Nelson
%% Last update Mon Dec 22 11:22:31 2008 Alex Nelson
%%

Let $S$ be some set. A mapping
\begin{equation}
S\times S\to S
\end{equation}
is sometimes called a \textbf{law of composition}\index{Law Of Composition} (of $S$
into itself). If $x,y$ are elements of $S$, the image of the
pair $(x,y)$ under this mapping is also called their
\textbf{product}\index{Product} under the law of composition. It will be
denoted as $xy$.
\begin{ex}
In $\mathbb{R}^3$, our favorite vector space from calculus
21 C, an example of a noncommutative law of composition is
the \textbf{cross product}\index{Cross Product}
\begin{subequations}
\begin{align}
\vec{u}\times\vec{v} &= \begin{vmatrix}\hat{x} & \hat{y} & \hat{z}\\
u_{x} & u_{y} & u_{z}\\
v_{x} & v_{y} & v_{z}
\end{vmatrix}\\
&= (u_{y}v_{z}-u_{z}v_{y})\hat{x} +
(u_{z}v_{x}-u_{x}v_{z})\hat{y} + (u_{x}v_{y}-u_{y}v_{x})\hat{z}
\end{align}
\end{subequations}
Observe that the dot product does not really satisfy the
criteria for the composition, because the dot product
``takes in'' two vectors and it ``spits out'' a scalar. But
for a composition, it needs to ``take in'' two of the same
object, and it ``spits out'' the same object. The cross
product does this, it ``takes in'' two 3-vectors and it
``spits out'' a 3-vector. \qef
\end{ex}
\begin{ex}
Consider multiplication of complex numbers
\begin{equation}
(t + iu)(x + iy) = (xt - yu) + i(ux + ty)
\end{equation}
It takes in two complex numbers, and it returns a complex
number. It is commutative. \qef
\end{ex}
\begin{ex}
Consider an arbitrary vector space $V$ over some field
$\mathbb{F}$. Given any two vectors $\vec{u}$, $\vec{v}\in
V$, we have the commutative composition operation of vector
addition
\begin{equation}
(\vec{u}+\vec{v})\in V
\end{equation}
which means that we have a law of composition for vector
spaces. \qef
\end{ex}
\begin{ex}
Let $W$ be a vector space over the field
$\mathbb{F}$. Consider two linear operators: 
\begin{equation}
T,U:W\to W
\end{equation}
then we may compose the linear operators into a new linear
operator
\begin{equation}
V = T\circ U
\end{equation}
so when acting on some element $\vec{w}\in W$ we have
\begin{equation}
V(\vec{w}) = T\Big(U(\vec{w})\Big)
\end{equation}
which is also a linear operator. \qef
\end{ex}
Similarly, for more general (not necessarily commutative, but possibly
commutative) operations, we use the symbol `$\cdot$' and
write $x\cdot y$ (or more generally just $xy$).

Now in general, if the composition is commutative (so
$xy=yx$) then we usually use the symbol `+' to indicate this
and generically call it ``\textbf{addition}''\index{Addition}. So when
$x+y=y+x$ (i.e. when the operation is commutative) we use
the generic term ``addition'' and the symbol `+' to indicate
this. 

But to be fully general, a generic law of composition is
dubbed ``multiplication'', and if it is commutative we call
it ``addition''. We use the corresponding terminology
(e.g. ``the sum of $x$ and $y$ is $x+y$'', ``given $u$ and
$v$, their product is $uv$'').

Now let us consider a set $S$ with a law of composition. If
$x,y,z$ are elements of $S$, we can write their product in
two ways: $(xy)z$ and $x(yz)$. \marginpar{Screw with the
  parenthese, i.e. order of operating doesn't matter,
  operation is associative}If $(xy)z=x(yz)$ for all
$x,y,z\in S$, then we say that their law of composition is ``\textbf{associative}''\index{Associativity}.

An element $e$ of $S$ such that
\begin{equation}
ex = xe = x
\end{equation}
for all $x\in S$ is called a \textbf{unit element}\index{Unit Element}.
(When the law of composition is written additively, the unit
element is denoted by 0, and is \textbf{zero element}\index{Zero Element}.) 
A unit element is unique, suppose that there is another unit
element $e'$ distinct from $e$, then
\begin{equation}
e = ee' = e'
\end{equation}
by assumption. In most cases, the unit element is written
simply as 1 (instead of $e$). It is a generalization of the
notion of the identity element with respect to some given
``Law of Composition''. For most of this section, we'll use
$e$ for clarity when specifying the basic properties.

Now, a \textbf{monoid}\index{Monoid} is a set $G$ with a law
\marginpar{Monoid: a set of elements equipped with some law of composition, that is closed under said law of composition}of composition which is associative, and
having a unit element (which implies that $G$ is never
empty). Note that there is nothing about being finite or
infinite, nor is there any specification about the law of
composition possessing an inverse.

\begin{ex}
Consider the natural numbers with 0\footnote{The author, not
  being French, doesn't believe 0 is either natural or a number.},
that is
$\mathbb{N}\cup\{0\}$. We have the law of composition
defined by addition in the usual way (so $1+0=1$, $2+4=6$,
etc.) and we have the additive identity $0$. If we didn't
have 0, we'd be a bit out of luck as there is no additive
identity in $\mathbb{N}$. So all by itself, $\mathbb{N}$ is
not a monoid, but the union $\mathbb{N}\cup\{0\}$ is a
monoid as it has the additive identity. \qef
\end{ex}
\begin{ex}
Let $V$ be a vector space over the field $\mathbb{F}$. Let
$\mathcal{L}(V)$ be the set of linear operators on
(``endomorphisms of'') $V$. Then $\mathcal{L}(V)$ is a
monoid in two different ways: one is with the law of
composition being matrix addition with the unit element
being the zero matrix; the other is with the law of
composition being matrix multiplication with the unit
element being the identity matrix. \qef
\end{ex}

Let $G$ be a monoid, and $x_1,\ldots,x_n$ be elements of $G$
\marginpar{Remember, we have the law of composition generically referred to as ``multiplication''}(where $n>1$ is some integer). We can define their product
inductively:
\begin{equation}
\prod^{n}_{\nu=1}x_{\nu}=x_1\cdots x_n = (x_1\cdots
x_{n-1})x_n.
\end{equation}
Now, don't jump to conclusions! It is very tempting to think
of this as just the usual product series as we learned from
analysis, but this is more general due to this just being a
way to iterate the law of composition on a sequence of
elements in $G$. Note that we did define it to be careful
about the order of operating. 

\marginpar{Extended associativty for an arbitary number of elements}We can also observe that we have the following rule
\begin{equation}
\prod^{m}_{\mu=1}x_{\mu}\prod^{n}_{\nu=1}x_{m+\nu}=\prod^{m+n}_{\nu=1}x_{\nu}
\end{equation}
which essentially asserts \emph{that we can insert
  parentheses in any manner in our product without changing
  its value}. This is actually relatively trivial since
monoids have their law of composition be associative.
\begin{sketch}
We can do this proof by induction on $m$. So for an
arbitrary positive integer, we have
\textbf{Base Case:} $m=2$ Observe that
\begin{subequations}
\begin{align}
\prod^{m}_{\mu=1}x_{\mu}\cdot\prod^{n}_{\nu=1}x_{m+\nu}
&= (x_1\cdot x_2)\cdot(x_3\cdot(\cdots)\cdot x_{2+n})\\
&= x_1\cdot x_2 \cdot x_3 \cdot (\cdots)\cdot x_{2+n}\text{ (by Associativity)}\\
&= \prod^{2+n}_{\nu=1}x_{\nu}
\end{align}
\end{subequations}
where we justify the last step by just grouping terms by
associativity of the law of composition.

\textbf{Inductive Hypthotesis:} Assume this works for
arbitrary $m$.

\textbf{Inductive Case:} For $m+1$ we have
\begin{subequations}
\begin{align}
\prod^{m+1}_{\mu=1}x_{\mu}\cdot\prod^{n}_{\nu=1}x_{\nu+m+1} &= 
\left(\prod^{m}_{\mu=1}x_{\mu}\right)\cdot x_{m+1}\cdot\left(\prod^{n}_{\mu=1}x_{\mu+m+1}\right)\\
&= \left(\prod^{m}_{\mu=1}x_{\mu}\right)\cdot\left(\prod^{n}_{\mu=1}x_{\mu+m}\right)
\end{align}
\end{subequations}
where we justify that last step by, again, the associativity
of the law of composition, since we're using a monoid. Then
we see that this is precisely the case for $m$ which we
assumed worked! So that concludes our proof by induction. $\spadesuit$
\end{sketch}

Also note just a few standards\index{Product!Conventions} we have with the product. For
instance, we have
\begin{equation}
\prod^{m+n}_{m+1}x_{\nu}\text{  instead of  }
\prod^{n}_{\nu}x_{m+\nu}
\end{equation}
and we \emph{define}
\begin{equation}
\prod^{0}_{\nu=1}x_\nu = e.
\end{equation}

It should be possible to define more general laws of
composition, that is to say maps $S_{1}\times S_{2}\to
S_{3}$ where $S_1,S_2,S_3$ are arbitrary sets. One can then
express associativity and commutativity in any setting where
they make sense to express them (it will make more sense
that than sentence just written). For instance, to have
commutativity we need the law of composition take the form
of 
\begin{equation}
f: S\times S\to T
\end{equation}
where the two sets that are ``eaten'' by $f$ are necessarily
the same. \textbf{Commutativity}\index{Commutativity} then
means that $f(x,y)=f(y,x)$ (or omitting $f$, $xy=yx$). 

For associativity, when would it make sense? There are 8 possible
cases one can imagine
\begin{equation*}
\begin{array}{ccc}
S\times S\to S & S\times T\to S & T\times S\to S\\
T\times T\to T & S\times T\to T & T\times S\to T\\
S\times S\to T & T\times T\to S & 
\end{array}
\end{equation*}
To have associativity, we need to have 
\begin{equation}
f(f(a,b),c) = f(f(a,c),b)\text{ or } f(a,f(b,c))=f(b,f(a,c))
\end{equation}
so that means that $b,c$ are both elements of the same
set. That automatically rules out two possibilities
\begin{equation*}
S\times S\to T\quad\text{and}\quad T\times T\to S.
\end{equation*}
By symmetry, we can see that if $S\times S\to S$ is
associative, then $T\times T\to T$ is also
associative. Similarly, if $S\times T\to S$ is associative,
then $T\times S\to T$ is also associative. Commuting the
position of the arguments doesn't change anything either, so
if $S\times T\to S$ is associative then $T\times S\to S$ is
also associative. So the only cases that can be associative
\marginpar{When associativity could make sense}are
\begin{equation}
\begin{array}{ccc}
S\times S\to S & S\times T\to S & T\times S\to S\\
T\times T\to T & S\times T\to T & T\times S\to T
\end{array}
\end{equation}
and all others cannot be associative.

Now, if the law of composition of $G$ is commutative, we say
that $G$ is \textbf{commutative} (or more often \textbf{Abelian}\index{Abelian!Module}\index{Module!Abelian}).

\begin{prop}
Let $G$ be a commutative monoid, and $x_1$, $\ldots$, $x_n$
be elements of $G$. Let $\psi$ be a bijection of the set of
integers $(1,\ldots,n)$ onto itself (in other words, it's a
permutation of $(1,\ldots,n)$). Then
\begin{equation}
\prod^{n}_{\nu=1}x_{\psi(\nu)} = \prod^{n}_{\nu=1}x_{\nu}
\end{equation}
\end{prop}
\begin{proof}
First let us show by induction that we can reorder a product
of a sequence of elements in arbitrary order.
\textbf{Base Case:} $n=2$ So
\begin{equation}
xy = yx
\end{equation}
which is trivially true from the definition of a commutative
monoid.

\textbf{Inductive Hypothesis:} Assume this works for
arbitrary $n$.

\textbf{Inductive Step:} We can show that since the law of
composition maps $G\times G\to G$ that
\begin{equation}
\prod^{n}_{\nu=1}x_{\psi(\nu)} = y
\end{equation}
and by the inductive hypothesis we have
\begin{equation}
\prod^{n}_{\nu=1}x_{\nu} = y
\end{equation}
thus
\begin{equation}
yx_{n+1} = x_{n+1}y
\end{equation}
by commuting $x_{n+1}$ through all of the elements in the
product. Thus we cover arbitrary permutations of
$(1,\ldots,n+1)$ onto itself.
\end{proof}

Let $G$ be a commutative monoid, let $I$ be a set, and let
$f:I\to G$ be a mapping such that $f(i)=e$ for almost all
$i\in I$. (Here and thereafter, \textbf{almost all}\index{Almost All} 
means \emph{all but a finite number}.) Let $I_{0}$ be the
subset of $I$ consisting of those $i$ such that $f(i)\neq
e$. By
\begin{equation}
\prod_{i\in I}f(i)
\end{equation}
we shall mean the product
\begin{equation}
\prod_{i\in I_{0}}f(i)
\end{equation}
taken in any order. 

(When $G$ is written additively, then instead of a product
sign, we write the sum $\sum$ instead of the product $\prod$.)

We can continue rattling off a grocery list of properties
that the product has. But it would be a pain in the rear to
do, and not all that educational. \marginpar{note notation}
We will, out of laziness, write
\begin{equation}
\prod f(i)
\end{equation}
omitting the $i\in I$ bit, since we have just seen the order
doesn't matter. Another matter of convention, we use the
exponent to indicate the number of times the operation of a
set is used on a given element. So
\begin{equation*}
x^n = \prod^{n}_{i=1}x
\end{equation*}
This allows us to have $x^0=e$ and $x^1=x$. We are content
with this convention, since it allows us to use familiar
notation making anything to the zeroeth power be the
identity element.

\begin{ex}
Consider the set of all invertibe matrices of a given vector
space $V$ on a field $\mathbb{F}$. This is typically denoted
as $GL(V)$, and for an arbitrary element $X\in GL(V)$ we
have
\begin{equation}
X^0 = I
\end{equation}
where $I$ is the identity matrix. \qef
\end{ex}

Let $S$, and $S'$ be two subsets of a monoid $G$, then we
define $SS'$ to be the subset consisting of all elements
$xy$ with $x\in S$ and $y\in S'$. Inductively, we can define
the product of a finite number of subsets, and we have
associativity. Why? How can we see this assertion? Well,
consider three subsets $S,S',S''$ of $G$. Then we can write
$(SS')S''=S(S'S'')$. \marginpar{Remember law of compositions map $G\times G\to G$}Observe that we can say $GG=G$ since
$G$ has a unit element. If $x\in G$, then we can define $xS$
to be $\{x\}S$ where $\{x\}$ is the set with a single
element $x$ (this is sort of like a coset). Thus $xS$
consists of all elements $xy$ with $y\in S$.

\begin{ex}
Consider all the square matrices on the vector space
$V=\mathbb{C}^2$ over the scalar field $\mathbb{C}$. Let 
\begin{equation}
z = \begin{bmatrix}1&0\\0&-1\end{bmatrix}
\end{equation}
and let $S$ be the set of all traceless operators acting on
$V$. Then
\begin{equation}
zS = \{ zy:y\in S\} = \left\{\begin{bmatrix}0&\beta\\-\alpha&0\end{bmatrix}:\alpha,\beta\in\mathbb{C} \right\}
\end{equation}
is the, uh, ``comonoid''.\qef
\end{ex}

We can specify a monoid which is a subset of a given monoid.
We call it a \textbf{submonoid}\index{Monoid!Submonoid}\index{Submonoid}
of $G$. To be fully clear, a submonoid of $G$ is a subset
$H$ of $G$ containing the unit element $e$, and such that --
if $x,y\in H$ then $xy\in H$ -- or in other words, $H$ is
\textbf{closed} under the law of composition. Well, by the
definition of a monoid, we see that $H$ is also a monoid.

We can see a trivial example of a submonoid given a monoid
is just the powers of an arbitrary element. What does this
mean? Well, choose some element $x$. What can we do with
this? Well, we have the law of composition, so we can
multiply it to itself to get $x^2$. We can then multiply
this by $x$ again to get $x^3$. We can keep going and going
for arbitrary $n\in\mathbb{N}$ have the set of $x^n$. But
this alone is not a monoid. Why? There is no identity
element, which isn't nice. So we copy the French, and have
$n\in\{0\}\cup\mathbb{N}$, then $\{x^n:
n\in\{0\}\cup\mathbb{N}\}$ is a monoid and its closed under
the law of composition. We see that it is commutative and
associative. So it's a really nice monoid! 

\subsection{Groups}
%%
%% groups.tex
%% 
%% Made by Alex Nelson
%% Login   <alex@tomato>
%% 
%% Started on  Wed Dec 24 14:35:51 2008 Alex Nelson
%% Last update Wed Dec 24 14:35:51 2008 Alex Nelson
%%

\begin{figure}[ht!]
\begin{center}
\includegraphics{img/img.1}
\end{center}
\caption[Relation between Monoids and Groups]{A Venn Diagram
  to illustrate the relation of the monoids and the
  groups. Observe all groups are monoids, but not all
  monoids are groups. Similarly, all trout are fish, but not
  all fish are trouts.}
\end{figure}
A \textbf{group}\index{Group} $G$ is a monoid with some
extra structure. Namely, for each element $x\in G$ there is
an element $y\in G$ such that
\begin{equation}
xy = yx = e.
\end{equation}
This element $y$ is called an
\textbf{inverse}\index{Inverse} for $x$. Such an inverse is
unique, a simple proof to illustrate this: suppose that
there are two inverses $y,y'$ of $x$, then
\begin{equation}
y' = y'e = y'(xy) = (y'x)y = ey = y.
\end{equation}
For multiplication, we denote the inverse of $x$ by
$x^{-1}$, and for addition the inverse of $x$ is denoted by
$-x$. The \textbf{order}\index{Group!Order} of a group is
the number of elements in the group, the size of the set so
to speak.

It is probably worth going on without saying that $e$ is its
own inverse. But if we have (an identity $e$) an element $x$ and its inverse
$x^{-1}$ and a given law of composition, then we can create
a group consisting of the elements
\begin{equation}
G = \{ (x^{-1})^{n}:n\in\mathbb{N}\}\cup\{(x)^{n}:n\in\mathbb{N}\}\cup\{e\}
\end{equation}
which is trivial. 

\begin{ex}
Consider the matrix
\begin{equation}
z = \begin{bmatrix}1 & 0 \\0 & -1\end{bmatrix}
\end{equation}
observe that $z^2=I$. That tells us that $\{I,z\}$ is a
matrix group under matrix multiplication. \qef
\end{ex}

Notice we are being a bit ambiguous here in specifying
``inverses'' and ``unit elements''. E.g. with matrices, it makes
a difference if we multiply by the right or by the left
(e.g. $X$ multiplied on the right by $Y$ is $XY$ but
multiplied on the left is $YX$, and in general $XY-YX\neq0$
-- if it is zero, then $X$, $Y$ are diagonal matrices). But
we have been sufficiently general so left inverses and left
identity elements are also inverses and identity
elements. We can be precise in specification and prove it
too:
\begin{prop}
Let $G$ be a set with an associative law of composition, let
$e$ be a left unit for that law, and assume that every
element has a left inverse. Then $e$ is a unit and each left
inverse is also an inverse. In particular, $G$ is a group.
\end{prop}
\begin{proof}
Let $b\in G$ and let $a\in G$ be such that $ba=e$. Then
\begin{equation}
bab = eb = b.
\end{equation} 
This much should be trivial, it's just substitution. We can
see that
\begin{equation}
abab = a(ba)b = aeb = ab
\end{equation}
which is equivalent to
\begin{equation}
(ab)^2 = ab.
\end{equation}
We can multiply both sides by $(ab)^{-1}$ on the left to see
that
\begin{equation}
(ab) = e
\end{equation}
which implies that $a$ is the left inverse for $b$, or
equivalently that $b$ is the right inverse for $a$. We then
see that
\begin{equation}
aba = a(e) = (e)a
\end{equation}
which implies the left identity $e$ is a ``bi''-identity
(i.e. both a left and right identity).
\end{proof}
\begin{ex}
Let $G$ be a group, and $S$ be some nonempty set. The set of
maps from $S$ to $G$, denoted $M(S,G)$, is itself a
group. That is, for two maps $f,g:S\to G$, we define $fg$ to
be the map such that
\begin{equation}\label{eq:groups:exLawOfComposition}
(fg)(x) = f(x)g(x)
\end{equation}
The inverse $f^{-1}$ multiplied by $f$ is equal to the
identity element in the group ${\mathbbm 1}$. That means,
$f^{-1}(x)=f(x)^{-1}$. If $f(x)^{-1}$ is well defined,
meaning that $f(x)$ -- an element of the group $G$ -- has an
inverse, which is necessarily true because that's the
definition of a group, then each element of $M(S,G)$ has an
inverse under the law of composition defined by Eq
\eqref{eq:groups:exLawOfComposition}. We also have for
arbitrary $f\in M(S,G)$ the product of it well defined too
since $f(x)$ is an element in the group, so $f(x)^n$ is an
element in the group raised to the $n^{th}$ power -- which
is well defined because it's a monoid! So it is a group.\qef
\end{ex}
\begin{ex}
Let $S$ be a non-empty set. Let $G$ be the set of bijections
from $S$ to $S$. Then we see that, making the composition of
mappings the law of composition, $G$ is a group. (Remember a
bijection is invertible, so each map has an inverse and we
have the identity defined in the obvious way as the identity
map of $S$.) The elements of $G$ are called
\textbf{permutations}\index{Permutations} of $S$. We denote
$G$ by $\operatorname{Perm}(S)$. \qef
\end{ex}
\begin{ex}
Consider the group $\mathbb{Z}_{2}$, which consists of two
elements $0$ and $1$. One can imagine this as a ``bit'' or a
binary digit. It has the operation of addition
\begin{equation}
0+0=1+1=0,\quad 0+1=1+0=1
\end{equation}
which is commutative. It is a cyclic group.\index{Group!Cyclic}\index{Cyclic Group} We can further
generalize this to $\mathbb{Z}_{3}$ with 3 elements: 0, 1,
and 2. It has the laws of addition
\begin{equation}
0+0=1+2=2+1=0,\quad 0+1=1+0=2+2=1,\quad 0+2=2+0=1+1=2
\end{equation}
which is also commutative. We have inverses well defined,
etc. etc. etc. We can generalize this to $\mathbb{Z}_{n}$
where $n\in\mathbb{N}$. This is a family of finite groups. \qef
\end{ex}
\begin{rmk}
Typically we ``represent'' a number $p$ in a cyclic group
$\mathbb{Z}_{n}$ (where $0\leq p<n$ is some integer) by
$\exp[i2\pi (p/n)]$. In this ``representation'', we have
instead of addition simple multiplication. Note that the
``generator'' of the group (the only element we really need
that we can subject to the law of composition of the group
to) is the primitive $n^{th}$ root of unity
$\exp[i2\pi/n]$. So $p$``=''$(\exp[i2\pi/n])^p$.
\end{rmk}
\begin{ex}
Let $G_1$, $G_2$ be groups. Let $G_1\times G_2$ be the
\textbf{direct product}\index{Direct Product!Groups}\index{Group!Direct Product}
which is similar to the direct product of sets, so
$G_1\times G_2$ is the set of all pairs $(x_1, x_2)$ with
$x_1\in G_1$ and $x_2\in G_2$. We define the product
componentwise by
\begin{equation}
(x_1,x_2)(y_1,y_2) = (x_1y_1,x_2y_2).
\end{equation}
We see that $G_1\times G_2$ is a group, with unit element
$(e_1, e_2)$ (where $e_1$ is the unit of $G_1$, and $e_2$ is
the unit of $G_2$). We can do this for $n$ groups, with
componentwise multiplication. \qef
\end{ex}

Let $G$ be a group. A \textbf{subgroup}\index{Group!Subgroup}\index{Subgroup} $H$
of $G$ is a subset of $G$ containing theunit element, and
such that $H$ is closed under the law of composition and
inverse (or equivalently, it is a submonoid such that if
$x\in H$ then $x^{-1}\in H$). A subgroup is
\textbf{trivial}\index{Subgroup!Trivial} if it consists of
the unit element alone. And trivially, the intersection of
an arbitrary non-empty family of subgroups is a subgroup.

Let $G$ be a group and $S\subset G$ be a subset. We say that
$S$ \textbf{generates} $G$ or that $S$ is a set of
\textbf{generators} for $G$ if every element of $G$ can be
expressed as a product of elements of $S$ or inverses of
elements of $S$, i.e. as a product $x_1\cdots x_n$ where
each $x_i$ or $x_{i}^{-1}$ is in $S$. It is clear that the
set of al such products is a subgroup of $G$ (remember,
$x^0=e$) and is the smallest subgroup of $G$ containing
$S$. What the hell does this mean? Well, $S$ generates $G$
if and only if the smallest subgroup of $G$ containing $S$
is $G$ itself. 

Let's stop and reiterate what we have just introduced. We
have a subgroup $S$ of a group $G$ is a submonoid that is a
group. So the submonoid contains inverses for each element.
We have some finite subset $S'$ of a group $G$ which is such that
any element of $G$ can be written as a product of any finite
number of elements of $S'$. This is a bit vague, so perhaps
an example is needed. 

\begin{ex}
Consider the quaternions, which have elements
\begin{equation}
i^2=j^2=k^2=ijk=-1
\end{equation}
We see that
\begin{align*}
i(ijk) &= i(-1)\\
\Rightarrow jk &= i\\
\Rightarrow (jk)k &= ik\\
\Rightarrow -j &= ik\\
\Rightarrow ij &= k
\end{align*}
So from $i$ and $j$, we can get $1=(-1)^2=(i)^4=(j)^4$ as
well as $k=ij$. All we need are two elements to have the
rest of the quaternions. \qef
\end{ex}

Note typically, if $G$ is a group, and $S$ is a set of
generators, then we write $G=\<S\>$. \marginpar{cyclic group has one generator}By 
definition, a cyclic group is a group which has one
generator.

\begin{ex} 
There are two non-abelian groups of order 8. One is the
quaternions which we already say, the other is the
\textbf{symmetries of the square},\index{Symmetries of Square}\index{Symmetry!Square}\index{Square!Symmetry} generated
by two elements $\sigma$, $\tau$ such that
\begin{equation}
\sigma^4 = \tau^2 = e,\quad\text{and }\tau\sigma\tau^{-1} =
\sigma^{3}
\end{equation}
We see that
\begin{align*}
(\tau\sigma\tau^{-1})\tau &=
\sigma^{3}\tau\\
&= \tau\sigma
\end{align*}
but also that
\begin{equation}
\sigma^{-1} = \sigma^{3}\Rightarrow \sigma^{-1}\sigma = e =
\sigma^4
\end{equation}
and
\begin{equation}
\tau^{-1} = \tau\Rightarrow \tau\sigma\tau=\sigma^3.
\end{equation}
So that means that
\begin{equation}
\tau\sigma\tau = \sigma^{-1}
\end{equation}
and
\begin{equation}
\tau\sigma^{-1}\tau = \sigma.
\end{equation}
But doesn't this imply
\begin{equation}
\tau = \sigma\tau\sigma
\end{equation}
by multiplying by the right by $\tau\sigma$. So we can write
the inverses in terms of $\sigma$ and $\tau$ alone. \qef
\end{ex}
\begin{ex}
The quaternion group is more generally the group generated
by two elements $i$, $j$ and setting $ij=k$ and $m=i^2$, we
have
\begin{equation}
i^4 = j^4 = k^4 = e,\quad\text{and }i^2=j^2=k^2=m,\; ij=mji
\end{equation}
but observe that as we introduced it before, it works
perfectly fine setting $m=-1$ and $e=1$. \qef
\end{ex}

Let $G,G'$ be monoids. A \textbf{monoid-homomorphism} (or
simply \textbf{homomorphism}\index{Homomorphism}) of $G$
into $G'$ is a mapping $f:G\to G'$ such that
$f(xy)=f(x)f(y)$ for all $x,y\in G$ and mapping the unit
element of $G$ into that of $G'$. If additionally $G,G'$ are
groups, the mapping is given a special name called a
\textbf{group-homomorphism}. 

\begin{ex}
Consider the monoid $\mathbb{N}$, a map
\begin{equation}
f:\mathbb{N}\to\mathbb{Z}_2
\end{equation}
is a homomorphism defined such that
\begin{equation}
f(n) = \begin{cases}0 &\text{$n$ is even}\\
1 &\text{otherwise}\end{cases}
\end{equation}
Then $f$ is a homomorphism. It maps the 0 element to the 0
element, and it maps 
\begin{align*}
f(m+n) &= f(m)+f(n)
\end{align*} 
trivially (odd+odd=even, even+even=even, even+odd=odd, and
1+1=0, 0+0=0, 0+1=1 respectively). \qef
\end{ex}
\begin{ex}
Let $V$, $W$ be arbitrary vector spaces. Let
\begin{equation}
f:V\to W
\end{equation}
be a linear transformation. Trivially it is a homomorphism,
as
\begin{equation}
f(u+v)=f(u)+f(v)
\end{equation} 
and by the definition of a linear transformation, we have
\begin{equation}
f(0)=0.
\end{equation}
Why? Well, it's trivial, observe
\begin{equation}
f(0+0)=f(0)+f(0)=f(0)\Rightarrow f(0)=0.
\end{equation}
So it preserves vector addition, and it maps the identity of
vector addition to the identity of vector addition. \qef
\end{ex}

Observe that for a group homomorphism $f:G\to G'$, we have
\begin{align*}
f(xx^{-1}) &= f(e)\text{ since }xx^{-1}=e\\
&= f(x)f(x^-1)\text{ since $f$ is a homomorphism}\\
&= e'\text{ since $f$ is a homomorphism}\\
\Rightarrow \left(f(x)\right)^{-1}f(e) &= f(x^{-1})\\
&= f(x)^{-1}.
\end{align*}
It's trivial.

\begin{ex}
Let $G$ be a commutative group. Then for $x\in G$,
\begin{equation}
x\mapsto x^n
\end{equation}
for some fixed integer $n$, is a homomorphism called the
\textbf{$n$-th power map}\index{Homomorphism!$n$-th Power Map}. \qef
\end{ex}
\begin{ex}
Let $G_i$ be some collection of groups, and $i\in I$ be an
element of some indexing set. Let
\begin{equation}
G = \prod G_{i}
\end{equation}
be direct product of all $G_i$, for all $i\in I$. So an
element of $G$ would be a tuple consisting of components
from each of the $G_i$. Let
\begin{equation}
p_i:G\to G_i
\end{equation}
be the projection of the $i^{th}$ factor. It selects the
component of the tuples in $G$ which corresponds to the
contribution from the group $G_i$. Then $p_i$ is a
homomorphism. \qef
\end{ex}\begin{quote}\begin{thm}
Let $G$, $G'$ be groups, and $S$  be a set of generators of
$G$. Let
\begin{equation}
f:S\to G'
\end{equation}
be a map. If there exists a homomorphism $\widetilde{f}:G\to
G'$ such that when we restrict $G$ to be $S$,
$\widetilde{f}=f$, then there exists only one
$\widetilde{f}$. 
\end{thm}
\end{quote}
In other words, $f$ has at most one extension to a
homomorphism of $G$ into $G'$. 

Let $f:G\to G'$ and $g:G'\to G''$ be two
group-homomorphisms. Then the composition $g\circ f$ is also
a group homomorphism. If $f,g$ are isomorphisms then so is
$g\circ f$. Further, $f^{-1}:G'\to G$ is also an
isomorphism. In particular, the set of all automorphisms of
$G$ is itself a group, denoted as $\aut(G)$.
\begin{defn}
Let $f:G\to G'$ be a group homomorphism. Let $e,e'$ be the
respective unit elements of $G,G'$. We can then define the
\textbf{kernel}\index{Kernel!Group Homomorphism} of $f$ to
be the subset of $G$ consisting of all elements $x$ such
that $f(x)=e'$.
\end{defn}
We immediately see that the kernel forms a subgroup of $G$,
since for any two $x,y\in\ker(f)$, we have
\begin{align*}
f(x+y) &= f(x)+f(y)\\
&= e'+e'\\
&= e'.
\end{align*}
Furthermore, since $f$ is a homomorphism, we have $f(e)=e'$
and
\begin{equation}
f(e) = f(xx^{-1}) = f(x)f(x)^{-1} = e'f(x^{-1})=e'
\end{equation}
which implies $f(x^{-1})=e'$. So $x^{-1}$ is in $\ker(f)$.
\begin{prop}
Let $f:G\to G'$ be a group homomorphism. Let $H'$ be the
\textbf{image}\index{Homomorphism!Image} of $f$. Then $H'$
is a subgroup of $G'$.
\end{prop}
\begin{proof}
Observe that for $x,y\in G$, that
\begin{equation}
f(xy)=f(x)f(y)\in H'.
\end{equation}
Further, since $f(e)=e'\in H'$ (so $H'$ has an identity
element), we see
\begin{equation}
f(xx^{-1})=f(x)f(x^{-1})=e'\Rightarrow f(x^{-1})\in H'.
\end{equation}
So the inverse of an arbitrary element is in $H'$, as is the
identity element, which is sufficient for $H'$ to be a
subgroup. 
\end{proof}
\begin{rmk}
The kernel and image of $f$ are denoted by $\ker(f)$ and
$\imag(f)$ respectively.
\end{rmk}

\section{Examples of Character Tables}
%%
%% characterTableEx.tex
%% 
%% Made by alex
%% Login   <alex@ubuntu>
%% 
%% Started on  Fri Jan 30 12:21:55 2009 alex
%% Last update Fri Jan 30 15:07:59 2009 alex
%%

\textbf{NOTE!} We will be working with irreps over a finite
dimensional vector space $V$ over the field $\mathbb{C}$ of
complex numbers.

%% \subsection{The Character Table for $S_{3}$}

%% Consider the symmetric group $S_3$. We have 
%% \begin{equation}

%% \end{equation}

\subsection{The Character Table for $D_3$}

Consider the group $D_3$ the symmetries of a regular
triangle. It has 6 elements, and the presentation
\begin{equation}
\<x,y|x^3=y^2=1,yx=x^{-1}y\>.
\end{equation}
It has, as mentioned, 6 elements
\begin{equation}
D_{3} = \{1,x,x^2,y,yx,yx^2\}.
\end{equation}
It has the multiplication table
\begin{equation}
\begin{array}{c|cccccc}
\times & 1   & x  & x^2 & y & yx & yx^2\\
\hline
1      & 1   & x  & x^2 & y & yx & yx^2\\
x^2    & x^2  & 1 & x   & yx & yx^2 & y\\ 
x      & x   & x^2 & 1 & yx^2 & y & yx\\
y      & y   & yx & yx^2 & 1 & x & x^2\\
yx     & yx  & yx^2 & y & x^2 & 1 & x\\
yx^2   & yx^2 & y & yx & x & x^2 & 1
\end{array}
\end{equation}
We find the cosets of the group
\begin{subequations}
\begin{align}
C(x) &= \{x^{k}xx^{-k}=x,
yxy=x^{-1},yx^{k}xyx^{k}=x^{-1}\}\\
&=\{x,x^2\}\\
C(y) &= \{x^kyx^{-k}=yx^{-2k}, yyy=y,
yx^{k}yyx^{k}=x^{-k}yx^{k}=yx^{2k}\}\\
&=\{y,yx,yx^2\}
\end{align}
\end{subequations}
Observe that $(x^2)^2=x^4=x$ in our group. Refer to the
multiplication table if in doubt. We can set up the
character table. We expect there to be some number of
irreducible representations $k$ such that the sum of the
squares of the dimensions of the irrep $\rho_k$ is the size
of the group. That is to say, 
\begin{equation}
d_{1}^{2} + d_{2}^{2} + \cdots + d_{k}^{2} = 6
\end{equation}
How many ways are there to do this? Well, observe $1^2=1<6$,
so $d_i$ can be at least 1. Further, $2^2=4<6$, so 2 is a
possible dimension. But $3^2=9>6$, so we can only have 1 or
2 dimensional irreps. How many different combinations are
there to write this? Trivially, two different ways
\begin{equation}
1^2+1^2+1^2+1^2+1^2+1^2=6,\quad\text{or }1^2+1^2+2^2=6.
\end{equation}
Those are the only two ways to write it! Further, we expect
the character table to be a square matrix, we have three
cosets: $C(1)$ (or as I denote it $C(e)$), $C(x)$, and
$C(y)$. That means we expect there to be three terms in our
sum of squares, and the only formula with three terms is
\begin{equation}
1^2+1^2+2^2=1+1+4=6.
\end{equation} 
So we have a good idea of how many irreps there are for
$D_3$. Let us now construct the character table:
\begin{equation}
\begin{array}{c|cccc}
\text{size of coset} & (1) & (2) & (3) \\
\text{coset rep} & 1 & x & y\\ \hline
\chi_1           &   &   &  \\ 
\chi_2           &   &   &  \\
\chi_3           &   &   & 
\end{array}
\end{equation}
Observe that we have the number of elements, which is
\emph{not standard} in character tables. I just like to add
them in as a reminder for quick reference. Now the first
column is the character of the irrep $\rho_i$. The first
irrep $\rho_1$ is always the trivial representation, it maps
everything to the identity in 1 dimension.
\begin{equation}
\begin{array}{c|cccc}
\text{size of coset} & (1) & (2) & (3) \\
\text{coset rep} & 1 & x & y\\ \hline
\chi_1           & 1 & 1 & 1\\ 
\chi_2           &   &   &  \\
\chi_3           &   &   & 
\end{array}
\end{equation}
Now we should remember that the character of the identity is
always the number of dimensions of the irrep. So we can fill
in the second column.
\begin{equation}
\begin{array}{c|cccc}
\text{size of coset} & (1) & (2) & (3) \\
\text{coset rep} & 1 & x & y\\ \hline
\chi_1           & 1 & 1 & 1\\ 
\chi_2           & 1 &   &  \\
\chi_3           & 2 &   & 
\end{array}
\end{equation}
We also have orthogonality of the rows (in a weighted inner
product), and orthonormality of the columns. So we can
\textbf{guess} the second row will be something of the form
\begin{equation}
\begin{array}{c|cccc}
\text{size of coset} & (1) & (2) & (3) \\
\text{coset rep} & 1 & x & y\\ \hline
\chi_1           & 1 & 1 & 1\\ 
\chi_2           & 1 & a & b\\
\chi_3           & 2 &   & 
\end{array}
\end{equation}
where $a,b\in\mathbb{C}$ are to be determined. We know that
the inner product between two characters is
\begin{equation}
\<\chi_i,\chi_j\> =
\frac{1}{n}\sum_{g}\overline{\chi_{j}(g)}\chi_{i}(g) = \delta_{ij}
\end{equation}
where $g$ is a coset representative and $\delta_{ij}$ is the
Kronecker delta, $\chi_i$ is the character for the irrep $\rho_i$. So we have
\begin{subequations}
\begin{align}
\<\chi_1,\chi_2\> = 1(1\cdot1) + 2(1\cdot\bar{a})+3(1\cdot\bar{b}) &= 0\\
\<\chi_2,\chi_1\> = 1(1)+2(a)+3(b) &= 0\\
\<\chi_2,\chi_2\> = 1(1\cdot1) + 2(a\cdot\bar{a})+3(b\cdot\bar{b}) &= 6.
\end{align}
\end{subequations}
So from the first two of these equations, we find $a,b$ are
real. We know that characters of irreps of finite groups are
integer combinations of roots of unity, so that means that
$a,b\in\mathbb{Z}$. From the condition that
\begin{equation}
1+2a+3b=0
\end{equation}
we find that
\begin{equation}
a=1,\;\; b=-1.
\end{equation}
So, we plug these into our character table
\begin{equation}
\begin{array}{c|cccc}
\text{size of coset} & (1) & (2) & (3) \\
\text{coset rep} & 1 & x & y\\ \hline
\chi_1           & 1 & 1 & 1\\ 
\chi_2           & 1 & 1 & -1\\
\chi_3           & 2 &   & 
\end{array}
\end{equation}
The orthonormality of the columns then demand that the rest
of the character table is trivially
\begin{equation}
\begin{array}{c|cccc}
\text{size of coset} & (1) & (2) & (3) \\
\text{coset rep} & 1 & x & y\\ \hline
\chi_1           & 1 & 1 & 1\\ 
\chi_2           & 1 & 1 & -1\\
\chi_3           & 2 & -1  & 0
\end{array}
\end{equation}
How do we see this? Take the first two columns and take
their inner product
\begin{equation}
1\cdot1+1\cdot1+2\cdot(-1)=0
\end{equation}
and similarly for the first and third column
\begin{equation}
1\cdot1-1\cdot1+0\cdot2=1-1=0.
\end{equation}
The second and third column are trivially orthogonal.

This is then the full character table for $D_3$. 

%% Let us try to construct the two dimensional irrep for
%% $D_3$. It is sufficient to do so for the coset
%% representatives I think. So we have $\rho_{3}(e)=I_{2}$ be
%% the two by two identity matrix, and for $\rho_{3}(x)$,
%% $\rho_{3}(y)$ -- since we do not know what they are! -- we
%% set up the matrices with unknown quantities we are solving
%% for. We know that $\chi_{3}(x)=-1$ and $\chi_{3}(y)=0$, so
%% that tells us that $\rho_{3}(x)$ has a trace of -1 and
%% $\rho_{3}(y)$ has a trace of 0 (it's traceless!). We set up
%% the matrices 
%% \begin{equation}
%% x = \begin{bmatrix}a & b\\ c & -(1+a)\end{bmatrix},\quad
%% y = \begin{bmatrix}\alpha & \beta\\ \gamma & -\alpha\end{bmatrix}.
%% \end{equation}
%% We can calculate the other elements of the group in our
%% representation by matrix multiplication
%% \begin{equation}
%% x^2 = \begin{bmatrix}a^2+bc & -b\\ -c & bc+(1+a)^2\end{bmatrix}
%% \end{equation}
%% so we find
%% \begin{equation}
%% yx = \begin{bmatrix} a\alpha+c\beta & b\alpha - (1+a)\beta\\
%% a\gamma-c\alpha & b\gamma+(1+a)\alpha\end{bmatrix}
%% \end{equation}
%% and
%% \begin{equation}
%% yx^2=xy=\begin{bmatrix}a\alpha+b\gamma & a\beta-b\alpha\\
%% c\alpha-(1+a)\gamma & c\beta+(1+a)\gamma\end{bmatrix}.
%% \end{equation}
%% From our character table, we demand that $yx$ and $yx^2$ are
%% traceless, and $x^2$ has a trace of -1. We have one more
%% condition worth mentioning, namely that $\rho_{3}(y)^2=I$
%% and $\rho_{3}(x)^3=I$, or in equation form
%% \begin{equation}
%% y^2 = \begin{bmatrix}\alpha^2+\beta\gamma & 0\\0 &
%%   \alpha^2+\beta\gamma\end{bmatrix} = I
%% \end{equation}
%% Similarly, an equation is derived from $x^3=I$ but we will
%% simplify our expression for $x$ first.

%% So we have a system of equations we need to solve
%% \begin{subequations}
%% \begin{align}
%% \tr(x^2) = 1+2a+2a^2+2bc &= -1\\
%% \tr(yx) = (1+2a)\alpha+(\beta+\gamma)c &= 0\\
%% \tr(yx^2) = (1+2a)\alpha+b\gamma+c\beta &= 0\\
%% (y^2=I) \;\; \alpha^2+\beta\gamma &= 1
%% \end{align}
%% \end{subequations} 
%% We can subtract the second from the third to find
%% \begin{equation}
%% b\gamma-c\gamma=0\Rightarrow b=c
%% \end{equation}
%% Plug this into the first equation we find
%% \begin{equation}
%% 1+a+a^2+b^2=0\Rightarrow b^2=a-(1+a)^2
%% \end{equation}
%% Things are getting slightly better, but now we can write the
%% matrix  representation of $x$ in a more convenient form
%% \begin{equation}
%% x = \begin{bmatrix}a & b\\b & -(1+a)\end{bmatrix}
%% \end{equation}
%% so
%% \begin{equation}
%% x^3 = \begin{bmatrix} a(a^2+b^2)-b^3 & -ab+b^3+b(1+a)^2\\
%% b(a^2+b^2)+(1+a)b & -b^2-(1+a)^3-b^{2}(1+a)\end{bmatrix} = I.
%% \end{equation}
%% This gives us an additional 4 equations.



\ifthenelse{\boolean{useIndex}}{
\printindex}{}
\nocite{*}
\bibliographystyle{elements}
\bibliography{algebra}
\end{document}
