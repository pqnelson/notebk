%%
%% glossary.tex
%% 
%% Made by Alex Nelson
%% Login   <alex@tomato>
%% 
%% Started on  Sat Aug 15 13:42:56 2009 Alex Nelson
%% Last update Sat Aug 15 13:42:56 2009 Alex Nelson
%%
A small glossary of various terms used in quantum field theory.
\begin{description}
\item[Abelian Group] when the group is commutative, we call it
  ``Abelian''.
\item[BRST Method] a generalization of the Faddeev-Popov
  procedure for non-Abelian gauge symmetries, it involves Ghosts.
\item[Correlation Function] the two-point correlation function
  corresponds to the amplitude for propagation of a
  particle or excitation between $y$ and $x$; Wick's theorem
  tells us we can express higher order correlation functions in
  terms of two-point correlation functions.
\item[Faddeev-Popov Procedure] a method to avoid redundant infinities in
  the functional integral when we have (Abelian) gauge symmetries.
\item[Gauge Symmetry] a change in coordinates which leaves the
  equations of motion invariant.
\item[Ghosts] a sort of ``particle'' which serves as negative
  degrees of freedom to cancel the effects of the unphysical
  timelike and longitudinal polarization states of gauge bosons.
\item[LSZ Reduction Formula] the multiparticle generalization of
  the Lorentz-invariant formula $\<k|k'\>=(2\pi)^{3}2\omega\delta^{(3)}(\bar{k}-\bar{k}')$.
\item[Non-Abelian Group] when the group's multiplication
  operation is not commutative, it is Non-Abelian. Contrast to an
  Abelian Group.
\end{description}
