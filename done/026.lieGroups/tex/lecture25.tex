%%
%% lecture25.tex
%% 
%% Made by Alex Nelson
%% Login   <alex@tomato3>
%% 
%% Started on  Mon Dec 13 13:00:13 2010 Alex Nelson
%% Last update Mon Dec 13 14:20:58 2010 Alex Nelson
%%
The goal here on out is twofold: (1) the Virasoro algebra is
important, (2) we will be forced to reconsider the relevant
theorems for semisimple Lie algebras. We will touch upon the
notion of central extensions explicitly.

\begin{defn}
Lets suppose we have a group $(G/N)$ and $N\subset G$ is a normal
subgroup, or more precisely $N\subset Z(G)$. Let $H:=G/N$. Then
$G$ is a \define{Central Extension} of $H$.
\end{defn}

\begin{ex}
Consider $\GL{V}$, consider its center which consists of scalar
matrices 
\begin{equation}
N = \{\, cI : c\in\CC\,\}.
\end{equation}
We have $\PGL{V}$ be the ``projectivization'' of $\GL{V}$, and
$\GL{V}$ is the central extension of $\PGL{V}=\GL{V}/N$.
\end{ex}

The notion is very close to the notion of a \define{Projective
  Representation} where we have
\begin{equation}
U_{a}U_{b}=c(a,b)U_{ab}
\end{equation}
the product be up to a factor $c(a,b)$ which may depend on $a$
and/or $b$. We have $U_{a}\in\GL{V}$, denote
\begin{equation}
[U_{a}]\in\PGL{V}
\end{equation}
Therefore we may write
\begin{equation}
[U_{a}][U_{b}]=[U_{ab}]
\end{equation}
We may alternatively think of a projective representation as a
morphism
\begin{equation}
\rho\colon G\to\PGL{V}.
\end{equation}

One more thing to say is we may consider the situation when $G$
is a central extension of $H$. This means
\begin{equation}
H=G/N.
\end{equation}
Then we may say an irreducible (complex) representation $\rho$ of
a group $G$ gives us a projective representation of $H$. The
image of $\rho(N)$ commutes with the image $\rho(G)$, so by
Schur's lemma 
\begin{equation}
\rho(N) = \{cI\}.
\end{equation}
for each scalar $c\in\CC$. So then $\rho(H)$ is ``up to
constants'' $\rho(G)$. We have this situation be described by
demanding the diagram
\begin{equation}
\begin{diagram}
  G      & \rTo^{\varphi}              & \GL{V} \\
\dTo\!\!\!\uTo & \ruTo{}{\widetilde{\varphi}} & \\
H        &                        &
\end{diagram}
\end{equation}
We lift $h$ to $G$ in many ways, we get a multivalued map
\begin{equation}
\widetilde{\varphi}\colon H\to\GL{V}
\end{equation}
This is precisely a projective representation.

There is a parallel notion for Lie Algebras. We have
\begin{equation}
\mathscr{G}/\mathscr{N}=\mathscr{H},
\end{equation}
and we have that $\mathscr{N}\subset\mbox{center of
}\mathscr{G}$. Then we have $\mathscr{G}$ be the central
extension of $\mathscr{H}$. Everything may be repeated, with
small changes of course. For groups it is difficult to lift a
quotient group $H$ to the group $G$; for Lie algebras, we may
lift it as linear spaces. If we have a surjective morphism $V\to W$
of vector spaces, we may lift it in th following way: find
$\widetilde{W}\subset V$ such that $\widetilde{W}\iso W$. Then we
have
\begin{equation}
V=\widetilde{W}\oplus(\mbox{something})
\end{equation}
this is something special for vector spaces. So therefore for Lie
Algebras we may say that 
\begin{equation}
\mathscr{G}=\mathscr{H}\oplus\mathscr{N}
\end{equation}
\textbf{as vector spaces!!} Does it work for the direct sum as
Lie Algebras?

For simplicity we will consider $\mathscr{N}=\{\lambda c\}$ where
$\lambda c\in\CC$ and $c$ is one single generator. We can now say
that $g\in\mathscr{G}$ may be represented as
\begin{equation}
g = h + \lambda c
\end{equation}
for some $h\in\mathscr{H}$. The general properties are really
clear. Suppose we have the commutator 
\begin{subequations}
\begin{align}
[g,g'] &= [h+\lambda c, h'+\lambda'c]\\
&= [h,h'] + \underbrace{[\lambda c, h']+[h,\lambda' c] + [\lambda
    c,\lambda'c]}_{=0\text{ since $c$ is in the center}}
\end{align}
\end{subequations}
So we find
\begin{equation}
[h,h']_{\text{new}} =
\underbracket[0.5pt]{\varphi(h,h')}_{\mathclap{\text{element in
    $\mathscr{H}$}}} +
\overbracket[0.5pt]{\lambda(h,h')}^{\mathclap{\text{Element in $\mathscr{N}$}}}\cdot c
\end{equation}
This is our new commutator, if we factorize with respect to
$\mathscr{N}$, we get a morphism
\begin{equation}
\mathscr{G}\to\mathscr{H}
\end{equation}
The formula is as follows
\begin{equation}
[h,h']_{\text{new}} = [h,h']_{\text{old}} + \lambda(h,h')\cdot c
\end{equation}
which is precisely our central extension. Note that
$[\cdot,\cdot]_{\text{new}}$ is the commutator in $\mathscr{G}$,
and $[\cdot,\cdot]_{\text{old}}$ is the commutator in $\mathscr{H}$.
Can we use any $\lambda(h,h')$? Of course not, otherwise it'd be
too easy. We want the new bracket to form a Lie Algebra. So it
needs to obey the condition of antisymmetry (which is easy), but
also the Jacobi identity.

\subsection{Spinor Representations}
How to construct spinor representations in the language of Lie
Algebras? Remember the Clifford algebra is an algebra with
generators $\gamma_{\mu}$ and the anticommutator of these
generators is
\begin{equation}
\gamma_{\mu}\gamma_{\nu}+\gamma_{\nu}\gamma_{\mu}=2\eta_{\mu\nu}\cdot1
\end{equation}
where $\eta_{\mu\nu}$ is a symmetric, nondegenerate matrix. So a
representation of the Clifford algebra is given by the Dirac
matrices
\begin{equation}
\Gamma_{\mu}\Gamma_{\nu}+\Gamma_{\nu}\Gamma_{\mu}=2\eta_{\mu\nu}\cdot1
\end{equation}
We know something about the representations of Clifford
algebras. We will denote $\cliff(n)$ as the $n$-dimensional
Clifford algebra. Now what we would like to do, since $\cliff(n)$
is an associative algebra, is to note it is also a Lie algebra:
\begin{equation}
[\Gamma_{a},\Gamma_{b}] =
\Gamma_{a}\Gamma_{b}-\Gamma_{b}\Gamma_{a}
\end{equation}
We will divide our Clifford algebra into parts, a standard trick!
The general element $a\in\cliff(n)$ can be written as
\begin{equation}
a = a^{0}\cdot\1 + a^{\mu}\Gamma_{\mu} + a^{\mu\nu}\Gamma_{\mu}\Gamma_{\nu}+\cdots
\end{equation}
We will now consider elements only up to quadratic terms. Is it a
subalgebra? Not exactly\dots but if we use the Lie algebra
operation, we get a Lie subalgebra. Why is this true?

First of all, constant terms drop out of the commutator, and
\begin{equation}
[\Gamma_{\mu},\Gamma_{\nu}] = \mbox{something quadratic}
\end{equation}
But what about
\begin{equation}
[\Gamma_{\mu}\Gamma_{\nu},\Gamma_{\rho}\Gamma_{\sigma}] = ?
\end{equation}
Well, we see first of all that
\begin{subequations}
\begin{align}
[\Gamma_{\alpha},\Gamma_{\beta}\Gamma_{\gamma}] 
&=\Gamma_{\alpha}\Gamma_{\beta}\Gamma_{\gamma}-\Gamma_{\beta}\Gamma_{\gamma}\Gamma_{\alpha}\\
&=\Gamma_{\alpha}\Gamma_{\beta}\Gamma_{\gamma}-\Gamma_{\beta}(-\Gamma_{\alpha}\Gamma_{\gamma}+2\eta_{\alpha\gamma}\cdot\1)\\
&=\Gamma_{\alpha}\Gamma_{\beta}\Gamma_{\gamma}+\Gamma_{\beta}\Gamma_{\alpha}\Gamma_{\gamma}-2\eta_{\alpha\gamma}\Gamma_{\beta}\\
&=(\Gamma_{\alpha}\Gamma_{\beta}+\Gamma_{\beta}\Gamma_{\alpha})\Gamma_{\gamma}-2\eta_{\alpha\gamma}\Gamma_{\beta}\\
&=2\eta_{\alpha\beta}\Gamma_{\gamma}-2\eta_{\alpha\gamma}\Gamma_{\beta}
\end{align}
\end{subequations}
Then we take advantage of the identity of the commutator
\begin{equation}
[AB,C] = A[B,C] + [A,C]B
\end{equation}
to find
\begin{subequations}
\begin{align}
[\Gamma_{\mu}\Gamma_{\nu},\Gamma_{\alpha}\Gamma_{\beta}] &=
\Gamma_{\mu}[\Gamma_{\nu},\Gamma_{\alpha}\Gamma_{\beta}]+[\Gamma_{\mu},\Gamma_{\alpha}\Gamma_{\beta}]\Gamma_{\nu}\\
&=2\Gamma_{\mu}(\eta_{\nu\alpha}\Gamma_{\beta}-\eta_{\nu\beta}\Gamma_{\alpha})
+2(\eta_{\mu\alpha}\Gamma_{\beta}-\eta_{\mu\beta}\Gamma_{\alpha})\Gamma_{\nu}\\
&=2\eta_{\nu\alpha}(-\Gamma_{\beta}\Gamma_{\mu}+2\eta_{\mu\beta}\1)-2\eta_{\nu\beta}(-\Gamma_{\alpha}\Gamma_{\mu}+2\eta_{\alpha\mu}\1)+\cdots\\
&=4(\eta_{\nu\alpha}\eta_{\mu\beta}-\eta_{\nu\beta}\eta_{\mu\alpha})\1\nonumber\\
&\phantom{=4}\quad+ 2(\eta_{\mu\alpha}\Gamma_{\beta}\Gamma_{\mu}-\eta_{\mu\beta}\Gamma_{\alpha}\Gamma_{\nu}-\eta_{\nu\alpha}\Gamma_{\beta}\Gamma_{\mu}+\eta_{\nu\beta}\Gamma_{\alpha}\Gamma_{\mu})\\
&\sim a\Gamma_{\sigma}\Gamma_{\tau}+b\cdot\1
\end{align}
\end{subequations}
We find that $a\cdot\1+\Gamma_{\mu}\Gamma_{\nu}a^{\mu\nu}$ is
itself a Lie subalgebra. What is this Lie algebra? We may say
that $a^{\mu\nu}$ may be restricted, due to the anticommutation
relation any symmetric combination disappears. So
\begin{equation}
a^{\mu\nu}=-a^{\nu\mu}
\end{equation}
is an antisymmetric matrix. It is pretty clear it is the
orthogonal Lie algebra, as it is described by means of
antisymmetric matrices. But we have this extra stuff, so we have
no chance for it to be an orthogonal Lie Algebra, but it is the
central extension to the orthogonal Lie algebra. Moreover it is
the trivial central extension.
