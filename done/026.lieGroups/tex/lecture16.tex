%%
%% lecture16.tex
%% 
%% Made by Alex Nelson
%% Login   <alex@tomato3>
%% 
%% Started on  Fri Dec 10 12:52:01 2010 Alex Nelson
%% Last update Sun Feb 20 11:01:11 2011 Alex Nelson
%%
If we have
\begin{equation}
\varphi\colon\mathscr{G}\to\mathfrak{gl}(V)
\end{equation}
a representation, and $\mathscr{H}\propersubset\mathscr{G}$ is
the Cartan subalgebra, then we recall a weight vector $v$ is an
eigenvector
\begin{equation}
\varphi(h)v = \lambda(h)v
\end{equation}
for every $h\in\mathscr{H}$. We have an adjoint representation
\begin{equation}
\ad\colon\mathscr{G}\to\mathfrak{gl}(\mathscr{G})
\end{equation}
where
\begin{equation}
(\ad x)v = [x,v]
\end{equation}
and the weight vectors for this representation are called root
vectors, and the weights are called roots. To define roots and
root vectors for $\mathscr{G}$ we are solving
\begin{equation}
[h,v] = \alpha(h)v.
\end{equation}
We can construct new root vectors from a given root vector by
applying $e_{i}$, $f_{j}$ to it. We have
\begin{subequations}
\begin{align}
[h,e] = \alpha(h)e & \iff he=eh+\alpha(h)e \\
 & \iff h(ev) = \big(eh+\alpha(h)e\big)v
\end{align}
\end{subequations}
but 
\begin{equation}
hv = \lambda(h)v \quad\implies\quad h(ev) =
\big(\lambda(h)+\alpha(h)\big) ev.
\end{equation}
This is either zero or another distinct eigenvector.

\begin{prop}
If $v$ is a weight vector with weight $\lambda$ and $e$ is a root
vector with root $\alpha$, then $ev$ is a weight vector with
weight $\lambda+\alpha$ unless $ev=0$.
\end{prop}

Now we will introduce a definition. Well several
definitions. First we introduce a notion of a Cartan Matrix which
is presented differently in different papers.
\begin{defn}
A \define{Cartan Matrix} is a matrix $A=[a_{ij}]$ such that
\begin{enumerate}
\item $a_{ii}=2$ are the diagonal components;
\item $a_{ij}\in\ZZ$ for any $i$, $j$;
\item $a_{ij}\leq 0$ for off-diagonal components;
\item although not necessarily symmetric, if $a_{ij}=0$ then
  $a_{ji}=0$;
\item it should be symmetrizable, i.e. we have a diagonal matrix
  $B$ such that $AB=D$ is also diagonal.
\end{enumerate}
\end{defn}
\begin{rmk}
Most of the time we will work with $A$ nondegenerate, i.e.
\begin{equation}
\det(A)\not=0
\end{equation}
But this is not a necessary condition, so we do not make it part
of the definition.
\end{rmk}
For every classical Lie Algebra, the matrix $a_{ij}$ % from the homework assignment,
is nondegenerate
\begin{equation}
\det(a_{ij})\not=0.
\end{equation}
We have explicitly computed this, so we should look at our
answers and nothing more.

The only thing that needs discussion is ``Why is $A$
symmetrizable?'' We know there exists a nondegenerate invariant
inner product on classical Lie algebras. The adjoint
representation is orthogonal with respect to this inner product,
i.e.\
\begin{equation}
\big\<[h,x], y\big\>+\big\<x,[h,y]\big\> = 0
\end{equation}
where $h,x,y\in\mathscr{G}$. This could be viewed as a
consequence of compact Lie groups having unitary representations
giving invariant inner product.

We can introduce the Killing form\index{Killing Form!Invariant Inner Product, Relation to} as
\begin{equation}
\<x,y\> = \tr\big(\ad_{x}\ad_{y}\big)
\end{equation}
which is an invariant inner product. %
%% %% Incorrect proof given on the day of the lecture
%% We consider the case when
%% $h\in\mathscr{H}$, $x=e_{j}$, and $y=f_{k}$, then we see that
%% \begin{subequations}
%% \begin{align}
%% \big\<[h,x], y\big\>+\big\<x,[h,y]\big\> &= \big\<[h,e_{j}],
%% f_{k}\big\>+\big\<e_{j},[h,f_{k}]\big\> =\\
%% &= \big\<a_{ij}e_{j}, f_{k}\big\>+\big\<e_{j},-a_{ik}f_{k}]\big\>
%% \end{align}
%% \end{subequations}
%% \dots
%% we should get the formula
%% \begin{equation}
%% a_{ij}\<e_{i},f_{i}\> = a_{ji}\<e_{j}, f_{j}\>
%% \end{equation}
%% %% Correct proof given in the next lecture
We may introduce an invariant inner product on the group
\begin{equation}
\<Ux,Uy\> = \<x,y\>
\end{equation}
where $U\in G$, but when $U=\1+u\varepsilon$ where $\varepsilon$
is ``small'', then we get
\begin{equation}
\<\varphi(u)x,y\> + \<x,\varphi(u)y\> = 0
\end{equation}
where $\varphi$ is a morphism. But as a representation we have
\begin{equation}
\Big\<[u,x],y\Big\> + \Big\<x,[u,y]\Big\> = 0.
\end{equation}
If we let $u=e_{i}$, $x=f_{i}$, $y=h_{j}$ we get
\begin{subequations}
\begin{align}
\Big\<[u,x],y\Big\> + \Big\<x,[u,y]\Big\>
&= \Big\<[e_{i},f_{i}],h_{j}\Big\> + \Big\<f_{i},[e_{i},h_{j}]\Big\>\\
&= \<h_{i},h_{j}\> + \<f_{i}, -a_{ji}e_{i}\>
\end{align}
\end{subequations}
This holds if and only if
\begin{equation}
\begin{diagram}[small,hug,height=13.5pt]
\<h_{i}, h_{j}\> & \rEq & a_{ji}\<f_{i},e_{i}\> \\
\dEq             &      & \dEq \\
\<h_{j}, h_{i}\> & \rEq & a_{ij}\<f_{j},e_{i}\>
\end{diagram}
\end{equation}
Using the inner product on the group, we may construct  the
matrix $B=\diag\<e_{i},f_{i}\>$ which implies $AB$ is symmetric.

For every Cartan matrix, we may construct a Lie algebra called a
Kac--Moody algebra. Really %irreducible
simple Lie Algebras are Kac--Moody algebras with additional
condition that the Cartan matrix is positive definite. We will
now describe all irreducible representations of classical Lie
Algebras; this is true for all Lie Algebras related to compact
groups, and reductive Lie Algebras.

In reality we may say for every compact group, the corresponding
Lie algebras have precisely the right generators. Moreover, we
may classify algebras of compact groups. This gives us a general
theorem for representations of compact Lie algebras. We would
like to explain the notion of a highest weight vector in this
situation. Namely the highest weight vector $v$ is such that
\begin{equation}
\varphi(e_{i})v = 0
\end{equation}
for all $e_{i}\in\mathscr{G}$. Of course this means that
\begin{equation}
\varphi(h)v = \lambda(h)v
\end{equation}
for all $h\in\mathscr{H}$, then this $\lambda$ is called the
highest weight. 

First of all, what is $\lambda(-)$? It is a linear functional
$\lambda\in\mathscr{H}^{*}$, i.e.
\begin{equation}
\lambda\colon\mathscr{H}\to\FF
\end{equation}
it is a linear functional acting on the Cartan subalgebra. Then:
\begin{enumerate}
\item Irreducible representations contain not more than one
  highest weight vector, up to a constant factor; 
\item Finite dimensional irreducible representation $\iff$ finite
  dimensional representation with one highest weight vector;
\item For every $\lambda\in\mathscr{H}^{*}$ one can construct a
  unique irreducible representation with highest weight $\lambda$
  but this representation can be infinite dimensional;
\item\label{lec16:cartan:mostImportantPoint}\marginpar{\eqref{lec16:cartan:mostImportantPoint} is the most important point!}
  This representation is finite dimensional if and only if
  $\lambda(h_{i})$ is a non-negative integer.
\end{enumerate}
This gives us a complete description of finite dimensional
irreducible representations.

\subsection{Dynkin Diagrams}
It is a very convenient way to depict Cartan matrices. Namely %it is 
first of all the dimension of the Cartan algebra is called the
\define{Rank of the Lie Algebra}. We have $\ClassicalGroup{A}_{\ell}$,
$\ClassicalGroup{B}_{\ell}$, $\ClassicalGroup{C}_{\ell}$,
$\ClassicalGroup{D}_{\ell}$ all of rank $\ell$. 

\begin{wrapfigure}[2]{r}{4cm}
  \vspace{-20pt}
  \begin{center}
    \includegraphics{img/LieImg.3}
  \end{center}
  \vspace{-20pt}
\end{wrapfigure}
The Dynkin diagram for $\ClassicalGroup{A}_{\ell}$ we draw $\ell$
vertices and we draw edges. We have the number of edges connecting vertices
$v_{i}$ to $v_{j}$ be given by the formula using the Cartan matrix 
\begin{equation}
n_{ij}=a_{ij}a_{ji}.
\end{equation}
Suppose we know $B$, its diagonal so we only need to keep track
of one index really. Since we suppose we know $B$, then
\begin{equation}
a_{ij}b_{j}=a_{ji}b_{i}
\end{equation}
implies
\begin{equation}
b_{i} = \frac{a_{ij}b_{j}}{a_{ji}}
\end{equation}
we get
\begin{equation}
n_{i}b_{i} = {a_{ij}}^{2}b_{j},
\end{equation}
or equivalently
\begin{equation}
{a_{ij}}^{2} = \frac{n_{i}b_{i}}{b_{j}}.
\end{equation}
What is the conclusion? If we know $[n_{i}]$ and $[b_{i}]$, we can
compute $[a_{ij}]$. 

Let us write down the Dynkin diagrams for the classical Lie
groups we have considered.
\begin{wrapfigure}[4]{r}{4.5cm}
  \vspace{-20pt}
  \begin{center}
    \includegraphics{img/LieImg.4}
  \end{center}
  \vspace{-20pt}
\end{wrapfigure}
\noindent{}For $\ClassicalGroup{D}_{n}$ we have the diagram drawn
on the right for the case when $n=7$ (observe there are 7
vertices). The Cartan matrix for $\ClassicalGroup{D}_{n}$ is
symmetric. One can observe this by considering the adjacency
matrix for the graph.

\begin{wrapfigure}[2]{l}{4.5cm}
  \vspace{-20pt}
  \begin{center}
    \includegraphics{img/LieImg.5}
  \end{center}
  \vspace{-20pt}
\end{wrapfigure}
\noindent{}For $\ClassicalGroup{C}_{n}$ we see the Cartan
matrix is not symmetric, but we can symmetrize it. We find that
$a_{n-1,n}a_{n,n-1}=2$.

\begin{wrapfigure}[2]{r}{4.5cm}
  \vspace{-20pt}
  \begin{center}
    \includegraphics{img/LieImg.6}
  \end{center}
  \vspace{-20pt}
\end{wrapfigure}
\noindent{}For $\ClassicalGroup{B}_{n}$ we see the Dynkin diagram
is ``the same'' as for $\ClassicalGroup{C}_{n}$ but with
different labels for the vertices.

Almost all of these groups are simple and almost all of them are
not isomorphic. But almost all. For example, in
$\ClassicalGroup{D}_{2}$ we have two disconnected vertices for
the Dynkin diagram. So $\ClassicalGroup{D}_{2}$ is not simple, it
is the direct product of $\SU{2}$ at the level of Lie Algebras,
and \emph{almost} the direct product at the level of Lie
groups. So we may examine the Dynkin diagram for $\ClassicalGroup{D}_{3}$
to find:
\begin{center}
\includegraphics{img/LieImg.7}
\end{center}
\noindent{}So we see this is the same Dynkin diagram as for
$\ClassicalGroup{A}_{3}$ which implies at the level of Lie
Algebras
\begin{equation}
\SU{4}\iso\SO{6}
\end{equation}
but only at the level of Lie Algebras. We similarly have $\ClassicalGroup{B}_{2}\iso\ClassicalGroup{C}_{2}$
by inspection of the Dynkin diagrams, but again it is an
isomorphism at the level of Lie Algebras.

%%
%% box1.tex
%% 
%% Made by Alex Nelson
%% Login   <alex@tomato3>
%% 
%% Started on  Sun Feb 20 10:55:50 2011 Alex Nelson
%% Last update Sun Feb 20 17:09:04 2011 Alex Nelson
%%
\begin{framed}
\noindent{\sectionfont Box\enspace \thesection.1 Dynkin Diagrams}
\bigskip
\noindent{}The problem we are facing is really two-fold: (a)
given a Dynkin diagram obtain the Cartan matrix, and (b) given
the Cartan matrix obtain the Dynkin diagram. This box is really
based off of \S4.7 of Kac's book \emph{Infinite Dimensional Lie Algebras}.
No secrets among friends: Kac provides the method of, given a
Cartan matrix, producing the Dynkin diagram. We review this, and
provide the algorithm going in the opposite direction. We also
consider examples. Throughout $A=[a_{ij}]$ is the Cartan matrix.

\medskip
\noindent\textbf{Given Cartan Matrix Obtain Dynkin Diagram.}\enspace
The basic idea is that we will have an $n\times n$ matrix
$A$. The Dynkin diagram is a graph that will have $n$
vertices, which are labeled by integers $i=1,\dots,n$. If
\begin{equation}
a_{ij}a_{ji}\leq4\quad\mbox{and}\quad |a_{ij}|\geq|a_{ji}|
\end{equation}
then vertices $i$ and $j$ are connected by $|a_{ij}|$ lines;
moreover if $|a_{ij}|>1$, then these lines are equipped with an
arrow pointing towards vertex $j$. 

Why do we need an arrow? The idea is that the Cartan matrix is
not symmetric, but has a weaker condition that $a_{ij}\not=0$
implies $a_{ji}\not=0$. Since we know the product by the number
of lines, we know the values by considering which direction the
arrow points.

\medskip
\noindent\textbf{Given a Dynkin Diagram Obtain Cartan Matrix.}\enspace
This occurs more often in practice (at least, for
physicists). What can we know immediately from the properties of
a Cartan matrix? Well, we know
\begin{equation}
a_{ii} = 2
\end{equation}
for all $i$. We know that the number of vertices $n$ gives
information about the number of rows, and the number of columns,
of the Cartan matrix --- i.e. $A$ is an $n\times n$ matrix. We
also know if $i\not=j$ that
\begin{equation}
a_{ij}\leq0\quad\mbox{and}\quad a_{ij}\in\Bbb{Z}.
\end{equation}
The rest we need to find from the diagram.

If vertex $i$ and $j$ are connected by $k$ lines, then $a_{ij}<0$. 
What values can this component be? Well, if $k=1$, then
\begin{equation}
a_{ij}=a_{ji}=-1
\end{equation}
since there is no arrow, it must be $-1$. If there are multiple
lines, we have an arrow to indicate which entry 

\begin{rmk}
Note that in these examples, the vertices are labeled by
\emph{indices} to keep track of which we are discussing. Usually,
the labels of a vertex are the relative (squared)
lengths of the fundamental roots as Gilmore describes
it [see Robert Gilmore, \emph{Lie Groups, Lie Algebras, and Some
    of Their Applications} Dover Publications (2002) Ch 8 \S III.2 pp 306 \emph{et seq.}].

{\bf N.B.} the method we have described are used to deduce a
\emph{generalized} Cartan matrix from a Dynkin diagram. So if we
restrict focus to Dynkin diagrams corresponding to ``strict''
Cartan matrices, we recover precisely the same information. But
we can do more! We can consider ``closed loops'' in our approach!
The only requirement we have for our considerations is that there
are less than 4 edges connecting any pair of vertices.
\end{rmk}
\begin{ex}
Consider the Dynkin diagram given by
\begin{center}
  \includegraphics{img/LieImg.11}
\end{center}
We see that there are 4 vertices, so immediately we know that the
Cartan matrix is $4\times4$ and we can write:
\begin{equation}
A = \begin{bmatrix}
2 &   &   &   \\
  & 2 &   &   \\
  &   & 2 &   \\
  &   &   & 2
\end{bmatrix}.
\end{equation}
We also see that there is one line connecting vertex 1 to vertex
2, so that means we can write
\begin{equation}
A = \begin{bmatrix}
2 &-1 &   &   \\
-1& 2 &   &   \\
  &   & 2 &   \\
  &   &   & 2
\end{bmatrix}.
\end{equation}
We then observe that there are no other edges connected to 1, so
\begin{equation}
A = \begin{bmatrix}
2 &-1 & 0 & 0 \\
-1& 2 &   &   \\
0 &   & 2 &   \\
0 &   &   & 2
\end{bmatrix}.
\end{equation}
Similar reasoning holds for vertex 4, it's connected by a single
edge to vertex 3
\begin{equation}
A = \begin{bmatrix}
2 &-1 & 0 & 0 \\
-1& 2 &   &   \\
0 &   & 2 &-1 \\
0 &   &-1 & 2
\end{bmatrix}.
\end{equation}
There are no other edges that connect vertex 4 to any other
vertex, so 
\begin{equation}
A = \begin{bmatrix}
2 &-1 & 0 & 0 \\
-1& 2 &   & 0 \\
0 &   & 2 &-1 \\
0 & 0 &-1 & 2
\end{bmatrix}.
\end{equation}
We see that there are \emph{two lines} connecting vertex 2 to
vertex 3 and there is an arrow. The arrow means that
\begin{equation}
a_{23}\not=a_{32}.
\end{equation}
The arrow points towards 3, so
\begin{equation}
|a_{32}|<|a_{23}|.
\end{equation}
Then we use the fact that there are two edges means that
\begin{equation}
|a_{23}|=2
\end{equation}
This is sufficient information to conclude
\begin{equation}
A = \begin{bmatrix}
2 &-1 & 0 & 0 \\
-1& 2 &-2 & 0 \\
0 &-1 & 2 &-1 \\
0 & 0 &-1 & 2
\end{bmatrix}.
\end{equation}
Thus we conclude our example.
\end{ex}
\begin{ex}
Consider the Dynkin diagram given by
\begin{center}
  \includegraphics{img/LieImg.12}
\end{center}
We see that there are 2 vertices, so immediately we know that the
Cartan matrix is $2\times2$ and we can write:
\begin{equation}
A = \begin{bmatrix}
 2 &   \\
   & 2 
\end{bmatrix}.
\end{equation}
The two vertices are connected by 3 edges. There is an arrow
pointing from vertex 1 to vertex 2. This implies that
\begin{equation}
A = \begin{bmatrix}
 2 &-3 \\
-1 & 2 
\end{bmatrix}.
\end{equation}
Observe that if the arrow pointed the other way, we would merely
have the transpose of our matrix.
\end{ex}
\end{framed}


\subsection{Returning to Representations}
The representations are described by means of highest weight. We
had previously
\begin{equation}
\varphi(e_{i})x = 0
\end{equation}
where $x$ is our highest weight vector, and the highest weight is
described by
\begin{equation}
\varphi(h)x = \lambda(h)x
\end{equation}
where $\lambda\in\mathscr{H}^{*}$ is the highest weight. We
should demand $\lambda(h_{i})\geq0$, and
$\lambda(h_{i})\in\ZZ$. We will now turn our attention to examples.

We will consider the fundamental representations of
$\ClassicalGroup{A}_{\ell+1}=\mathfrak{sl}(\ell+1)$. The
fundamental representation is the representation by
$(1+\ell)\times(1+\ell)$ matrices. We found
\begin{equation}
e_{i}=E_{i,i+1}
\end{equation}
where $E_{i,j}$ has zero components everywhere except at $i,j$ it
is 1. The Cartan subalgebra is
\begin{equation}
\mathscr{H}=\left\{\begin{bmatrix}\lambda_1 & & \\
 & \ddots & \\
 &        & \lambda_{n}
\end{bmatrix}\text{ such that }\lambda_{1}+\dots+\lambda_{\ell+1}=0\right\}
\end{equation}
What are the weight vectors here? It is quite clear that the
weight vectors $u_{1}$, \dots, $u_{\ell+1}$ are the standard
basis vectors. Observe
\begin{equation}
hu_{i}=\lambda_{i}u_{i}.
\end{equation}
What is the highest weight vector of this representation? We see
that
\begin{equation}
e_{i}u_{j} = 0
\end{equation}
unless $j=i+1$ we have
\begin{equation}
e_{i}u_{i+1}=u_{i}
\end{equation}
The highest weight vector is clearly $u_{1}$ because the shift
goes down and there is no way down. This implies the
representation is irreducible as the highest weight vector is
unique up to some coefficient. We see that
\begin{equation}
\lambda(h_{i}) = \delta_{i1}
\end{equation}
also holds.

What about the tensor product of representations. We find the
basis to be $u_{j}\otimes u_{k}$ and
\begin{subequations}
\begin{align}
h_{i}(u_{j}\otimes u_{k}) &= (h_{i}u_{j})\otimes u_{k} + u_{j}\otimes(h_{i}u_{k})\\
&= (\lambda_{j}+\lambda_{k})(u_{j}\otimes u_{k})
\end{align}
\end{subequations}
We found all the weight vectors\dots well not really since
$u_{1}\otimes u_{2}$ is a weight vector with the same weight as
$u_{2}\otimes u_{1}$, so $u_{1}\otimes u_{2}\pm u_{2}\otimes
u_{1}$ is again a weight vector. We find the highest weight
vector to be $u_{1}\otimes u_{1}$, but we see that
\begin{equation}
e_{1}(u_{1}\otimes u_{2}) = u_{1}\otimes u_{1}
\end{equation}
and
\begin{equation}
e_{1}(u_{2}\otimes u_{1}) = u_{1}\otimes u_{1}
\end{equation}
so it follows that $u_{1}\otimes u_{2}-u_{2}\otimes u_{1}$ is
again a highest weight vector\dots so we have 2 distinct highest
weight vectors! This cannot be an irreducible
representation. This we know, we may consider the symmetric and
antisymmetric parts of the representation.



