%%
%% lecture05.tex
%% 
%% Made by Alex Nelson
%% Login   <alex@tomato3>
%% 
%% Started on  Thu Jan 21 11:44:35 2010 Alex Nelson
%% Last update Mon Dec 13 21:50:25 2010 Alex Nelson
%%

The first example is $\GL{n,\Bbb{R}}$ and the corresponding Lie
algebra $\frak{gl}_{n}(\Bbb{R})$ which consists of all
matrices. It is important to consider
$\frak{gl}_{n}(\Bbb{R})\subset\frak{gl}_{n}(\Bbb{C})$ the
complexification of the algebra. We will denote
$\frak{gl}_{n}(\Bbb{C})=\Bbb{C}\frak{gl}_{n}(\Bbb{R})=\frak{gl}_{n}$.

Another example ${\rm SL}_{n}(\Bbb{R})$ the group of $n$-by-$n$
matrices satisfying the property of having unit determinant. To
compute the Lie algebra, consider elements close to $I$ or more
precisely a curve
\begin{equation}
A(\tau)=I+a(\tau).
\end{equation}
Now we should like to consider the tangent vector by Taylor
expanding the curve and using the coefficient of the first order
term as the tangenet vector
\begin{equation}
A(\tau)=I+\tau X+\mathcal{O}(\tau^{2}),
\end{equation}
so we want
\begin{equation}
\det\left(A(\varepsilon)\right)=\det(I+\varepsilon X)=I+\varepsilon\tr(X)+\mathcal{O}(\varepsilon^{2}).
\end{equation}
We see immediately that the condition $\tr(X)=0$ is the condition
for elements of the Lie algebra. So we see that
\begin{equation}
\Lie(\SL{n,\Bbb{R}})=\frak{sl}_{n}(\Bbb{R})=\{X\in\frak{gl}_{n}\mid\tr(X)=0\}.
\end{equation}
We may consider the complexification
\begin{equation}
\Bbb{C}\frak{sl}_{n}(\Bbb{R})=\frak{sl}_{n}(\Bbb{C})=\frak{sl}_{n}.
\end{equation}
{\bf N.B.} $\frak{sl}_{n}$ is an ideal in $\frak{gl}_{n}$ and
$\frak{gl}_{n}=\Bbb{C}\oplus\frak{sl}_{n}$ is the direct sum of
the trivial Lie algebra $\Bbb{C}$ and $\frak{sl}_{n}$ since
$A=\alpha\cdot I+A'$ where $A'\in\frak{sl}_{n}$ so
$\tr{A}=\alpha\cdot{\rm dim}$.

We also see $\ORTH{n}=\{A\in\GL{n}\mid A^{T}A=I\}$. The Lie algebra
is obtained by considering $A=I+\varepsilon X$ where
$\varepsilon^{2}\approx 0$ is an
``infinitesimal''\footnote{Although this is a mathematically
  unrigorous notion, we can still use it for computational and
  heuristic purposes.}. The Lie algebra is obtained by
\begin{subequations}
\begin{align}
A^{T}A &= (I+\varepsilon X)^{T}(I+\varepsilon X)\\
&= I +\varepsilon(X^{T}+X)+\mathcal{O}(\varepsilon^2)\\
&= I\quad\iff\quad X^{T}+X=0.
\end{align}
\end{subequations}
This is the condition for the Lie algebra of $\ORTH{n}$ which is denoted
\begin{equation}
\frak{so}(n)=\{X\in\frak{gl}(n)\mid X+X^{T}=0\}.
\end{equation}
We are interested in
$\frak{so}(n)=\frak{so}(n,\Bbb{C})=\Bbb{C}\frak{so}(n,\Bbb{R})$. We
see the Lie group
\begin{equation}
\SO{n}=\ORTH{n}\cap\SL{n}
\end{equation}
has unit determinants. We can also quickly compute and find that
$\SO{n}$ is the connected part of $\ORTH{n}$ which contains the identity.

\begin{wrapfigure}[12]{r}{0.35\textwidth}
  \vspace{-30pt}
  \begin{center}
    \includegraphics{img/LieImg.0}
  \end{center}
  \vspace{-20pt}
  \caption{{\small The Two Seperated Components of O$(n)$.}}\label{fig:LieImg0}
  \vspace{20pt}
\end{wrapfigure}

The group $\ORTH{n}$ has elements $A\in\ORTH{n}$ such that
$\det(A)^{2}=1$, so it has two separate components. This is seen
in figure \ref{fig:LieImg0}. This means that the group $\ORTH{n}$ is
disconnected, there is no continuous path connecting e.g. an
element $X\in\ORTH{n}$ with $\det(X)=-1$ to an element $Y\in\ORTH{n}$
with $\det(Y)=+1$, because the path would have to go through a
point with zero determinant. That is a singular matrix, which is
not contained in the group $\GL{n}$, and that would imply
$\ORTH{n}\not\subset\GL{n}$ which is a contradiction.

\begin{rmk}
Both $\SO{n}$ and $\ORTH{n}$ are both compact groups, i.e. closed
and bounded.
\end{rmk}
\begin{rmk}
Note that $\U{n}$ and $\SU{n}=\U{n}\cap\SL{n}$ are also
compact. The condition for $\U{n}=\{A\in\GL{n,\Bbb{C}}\mid
A^{\dagger}A=I\}$, and the corresponding Lie algebra is
$\frak{u}(n)$. The condition for it is
\begin{subequations}
\begin{align}
(I+\varepsilon A^{\dagger})(I+\varepsilon A)&=I+\varepsilon(A^{\dagger}+A)+\mathcal{O}(\varepsilon^{2})\\
&=I\quad\iff\quad A^{\dagger}+A=0.
\end{align}
\end{subequations}
So $\frak{u}(n)=\{X\in\frak{gl}_{n}(\Bbb{C})\mid
A^{\dagger}+A=0\}$. We have for
$\frak{su}(n)=\{A\in\frak{u}(n)\mid \tr(A)=0\}$. We see that
$\Bbb{C}=\frak{gl}_{n}(\Bbb{C})$,
$\Bbb{C}\frak{su}(n)=\frak{sl}_{n}$ are the complexifications.
\end{rmk}

The last classical group we would like to consider preserves some
skew-symmetric inner product. That is to say, $\<x,y\>=-\<y,x\>$
more generally however we will use a Bilinear form $B$ which is
antisymmetric
\begin{equation}
B(x,y)=-B(y,x).
\end{equation}
We write
\begin{equation}
B(x,y)=x^{T}By
\end{equation}
if $B$ is an antisymmetric matrix. We want to find matrices $A$
such that $B(Ax,Ay)=B(x,y)$, i.e.
\begin{equation}
(Ax)^{T}BAy=x(A^{T}BA)y=xBy
\end{equation}
or equivalently $A^{T}BA=B$. {\bf N.B.} if $B=I$ we recover the
orthogonal group. We get a group $\Sp{n}=\{A\in\GL{2n}\mid
A^{T}BA=B\}.$ the Lie algebra is of the form
\begin{equation}
(I+\varepsilon X)^{T}B(I+\varepsilon X)=B\quad\Rightarrow\quad X^{T}B+BX=0.
\end{equation}
If $B$ is nondegenerate, then 
\begin{equation}
\frak{sp}(n)=\{X|X^{T}B+BX=0\}
\end{equation}
is the Lie Algebra for $\Sp{n}$. This is a noncompact group. We
can get a compact group by examining the intersection ${\rm
  Sp}_{n}(\Bbb{C})\cap\U{n}={\rm Sp}_{n}$, and $\Bbb{C}\Lie({\rm
  Sp}_{n})=\frak{sp}(n)$ is the original Lie Algebra.

\begin{thm}
All simple Lie algebras that are Lie algebras of compact groups
are classical Lie algebras or one of 5 exceptional Lie algebras.
\end{thm}

There is a related notion of semisimple Lie algebra. A semisimple
Lie algebra is a direct sum of noncommutative simple Lie
algebras. There is an important class of \define{Solvable} Lie
algebras. 

Remember (e.g. from Lang's \emph{Algebra} chapter I \S3) that a group $G$ is
\define{Solvable} if we have a tower of groups
\begin{equation}
G\supset G_{1}\supset G_{2}\supset\cdots\supset G_{m}=\{e\}
\end{equation}
such that for $G_{n-1}\supset G_{n}$ we have $G_{n-1}/G_{n}$ be Abelian.

The notion of a solvable Lie algebra is the same except we have
$\frak{g}_{n-1}\supset \frak{g}_{n}$\marginpar{\textbf{Notational
Warning:} we will use $\frak{g}$ or $\mathscr{G}$ for the Lie
Algebra, interchangeably, and without warning!} be an ideal. We also can
doodle this cute diagram
\begin{equation}
\begin{diagram}[small]
\mbox{Group}   & \qquad & G           & \rhd    & H          &\\
               &        & \uTo<{\exp} &         & \uTo>{\exp}&\\
\mbox{Algebra} & \qquad &\Lie(G)      & \supset & \Lie(H)    &\mbox{ is an ideal}
\end{diagram}
\end{equation}

In a sense, these two notions of solvability and semisimplicity
are ``complementary'' --- an arbitrary Lie group has a semisimple
part and a reductive part. A compact Lie group has a semisimple
Lie algebra.

\subsection{Exercises}
\begin{exercise}\label{ex:prob3}
Find Lie algebras of the following matrix groups
\begin{enumerate}
\item The group of real upper triangular matrices
\item The group of real upper triangular matrices with diagonal entries equal to 1.
\item The group $T_{k}$ of real $n\times n$ matrices obeying $a_{ii} = 1$, $a_{ij} = 0$ if $j-i<k$ and $j=i$.
\end{enumerate}
\end{exercise}
\begin{exercise}
Check that the groups of Problem \ref{ex:prob3} and corresponding Lie algebras are solvable.
\end{exercise}
