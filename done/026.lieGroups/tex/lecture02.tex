%%
%% lecture02.tex
%% 
%% Made by Alex Nelson
%% Login   <alex@tomato3>
%% 
%% Started on  Tue Jan 19 11:05:14 2010 Alex Nelson
%% Last update Thu Jan 21 10:33:23 2010 Alex Nelson
%%
The main thing of interest is Lie groups, but Lie algebras are a
useful tool to study Lie groups. We will start with Lie
algebras. First what is an algebra. Well, more or less, it's
obtained from the formula
\begin{equation}
\hbox{Vector Space}+\hbox{Ring}=\hbox{Algebra},
\end{equation}
with some compatibility conditions. %% We want the addition
%% operation in the Ring structure to be the same as the addition
%% operation in the Vector Space structure, we want the
%% multiplication operation in the Ring structure to be
%% distributive over the Vector addition operation, and so on. 
When
we speak of vector spaces, we need a field $\Bbb{F}$; a ring has 2
operations: multiplication and addition. (A ring is an Abelian
group under addition, and a magma under multiplication.)
The only only relation between addition and multiplication is
distributivity: 
\begin{subequations}
\begin{align}
a(b+c)&=ab+ac\\
(b+c)a&=ba+ca.
\end{align}
\end{subequations}
In general, a ring doesn't have a multiplicative identity, nor is
multiplication an Abelian operation.

The compatibility condition for an algebra is thus
\begin{subequations}
\begin{align}
\lambda(ab) &= (\lambda a)\cdot b\\
&= a\cdot(\lambda b)
\end{align}
\end{subequations}
where $\lambda\in\Bbb{F}$. This is associativity. 

\begin{ex}
We have $\mat_{n}(\Bbb{F})$ --- the collection of all $n\times n$
matrices over a field $\Bbb{F}$ --- be a noncommutative,
associative algebra.
\end{ex}
\begin{ex}
If we have some set $M$, the set of functions on $M$ (denoted by
$C(M)$) is an algebra with respect to point-wise addition,
multiplication, and $\Bbb{F}$-scalar multiplication. Note that if
$\Bbb{F}$ is a field, the algebra is associative. Now $C(M)$ has
a lot of subalgebras if $M$ has some additional structure. If $M$
is a topological space, we have the set $C^0(M)\subset C(M)$ of
continuous functions be a subalgebra. If $M=\Bbb{R}^{n}$ we may
consider $C^{\infty}(M)\subset C(M)$ the subalgebra of smooth functions.
\end{ex}
\begin{ex}
We can consider $C^{\infty}(S^{n})$ where $S^n$ is the
$n$-sphere. We do this by introducing local coordinates, and
define the notion of smoothness in $S^n$ by demanding tit be
smooth in every coordinate system on $S^n$. But it is possible
for a function to be smooth in one coordinate system but not
another, so we need to use the notion of a transition function.
\end{ex}

A smooth manifold $M$ is covered by smooth local coordinate
systems, and the transition function between coordinate systems
is smooth. So $C^{\infty}(M)$, for some smooth manifold $M$, is a
unital, commutative, associative algebra.\marginpar{Unital=has unit element for multiplication}

We will introduce a construction of an algebra for a group,
called the group algebra. We consider all formal linear
combinations of group elements with coefficients from a ring:
%\begin{subequations}
\begin{align}
\hbox{Group}&\to\hbox{Algebra}\nonumber\\
G&\to\Bbb{F}[G]
\end{align}
%\end{subequations}
which has an element resemble $\sum_{i}^{n}a_{i}g_{i}$ where
$g_{i}\in G$ and $a_{i}\in\Bbb{F}$ for all $i$. We have addition
be component-wise, and multiplication also be component-wise.
So for example
\begin{subequations}
\begin{align}
(g_{1}+g_{2})+(g_{2}+3g_{3}) &= g_{1}+2g_{2}+3g_{3}\\
(g_{1}+g_{2})(g_{2}+3g_{3}) &= g_{1}g_{2}+3g_{1}g_{3}+g_{2}^{2}+3g_{2}g_{3}.
\end{align}
\end{subequations}
More generally
\begin{equation}
(\sum_{i}a_{i}g_{i})(\sum_{j}b_{j}g_{j})=\sum_{i,j}a_{i}b_{j}\cdot(g_{i}g_{j}).
\end{equation}
In the language of category theory, this is a functor $\Grp\to\Alg$.

Recall a representation of a group $G\to\GL{V}$ are homomorphisms
from $G$ to automorphisms on $V$. We have very simply for a rep
$G\to\GL{n}$ a representation
$\Bbb{F}[G]\to\mathcal{L}(V,V)=\mat_{n}$ of the algebras. For
every $g\in G$ we have its representation $\varphi(g)$, so this
induces a representation
\begin{equation}
\sum a_{i}g_{i}\mapsto \sum a_{i}\varphi(g_{i}),
\end{equation}
where products go to products and sums go to sums. The opposite
direction, a representation of $\Bbb{F}[G]$ induces a
representation of $G$, is also true (by the duality
principle). Moral: representations of groups induce
representations of associative algebras (a representation of
associative algebras in general is referred to as
\define{Modules}).

Consider $\scr{A}$ an associative algebra. We will define a new
operation on $\scr{A}$, namely the bracket as a commutator
\begin{equation}
[a,b]=ab-ba
\end{equation}
for all $a,b\in\scr{A}$. So with respect to the bracket,
$\scr{A}$ is an algebra (distributivity remains, but
associativity is broken). But observe
\begin{enumerate}
\item $[a,b]=-[b,a]$ i.e. we have antisymmetry of the bracket;
\item the Jacobi identity holds.
\end{enumerate}
This newly constructed algebra is in fact a Lie algebra! So for
every associative algebra $\scr{A}$, we may construct a Lie
algebra on $\scr{A}$; this is described by a natural functor, so
algebra morphisms are mapping to Lie algebra morphisms.

\begin{rmk}
There are other ways to construct Lie algebras. 

\noindent{\bf N.B.:} subalgebras of Lie algebras are again Lie algebras.
\end{rmk}

\begin{ex}
The Lie Algebra $\mat_{n}(\Bbb{F})=\frak{gl}_{n}(\Bbb{F})$ the
Lie algebra for $\GL{n,\Bbb{F}}$.
\end{ex}
\begin{ex}
Consider
$\frak{sl}_{n}(\Bbb{F})=\{A\in\frak{gl}_{n}(\Bbb{F})\mid\tr(A)=0\}$. This
is a Lie subalgebra of $\frak{gl}_{n}(\Bbb{F})$, since
$\tr(AB)=\tr(BA)$ so $\tr([A,B])=0$ for all $A,B\in\frak{gl}_{n}(\Bbb{F})$.
\end{ex}

We can introduce the notion of an \define{Ideal} in an algebra,
especially a Lie algebra! If $I\subset R$ where $R$ is a ring,
then $IR=I$ is a left ideal, and $RI=I$ is a right ideal. This
notion may be generalized to algebras (especially Lie
algebras!). Fir Lie algebras, \emph{every ideal is a two-sided
  ideal.}

\begin{prop}
$\frak{sl}_{n}(\Bbb{F})$ is an ideal in $\frak{gl}_{n}(\Bbb{F})$.
\end{prop}

If we have a ring morphism $\varphi\colon R\to R'$, its kernel is
a two-sided ideal. Moreover we may factorize
\begin{equation}
\im(\varphi)\iso R/\ker(\varphi).
\end{equation}
This construction generalizes to algebras.

\subsection{Exercises}
\begin{exercise}\label{ex:prob1}
Check that the vector space $\RR^{3}$ is a Lie algebra with respect to cross-product of vectors. Check that this Lie algebra is simple (does not have any non-trivial ideals). Check that all derivations of this Lie algebra are inner derivations.
\end{exercise}
\begin{exercise}
Check that the Lie algebra of Problem \ref{ex:prob1} is isomorphic to the Lie algebra $\mathfrak{so}(3)$ of real antisymmetric $3\times3$ matrices and to the Lie algebra $\mathfrak{su}(2)$ of complex anti-Hermitian traceless $2\times2$ matrices.
\end{exercise}
