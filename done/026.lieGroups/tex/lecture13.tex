%%
%% lecture13.tex
%% 
%% Made by Alex Nelson
%% Login   <alex@tomato3>
%% 
%% Started on  Sun Jun 20 12:08:00 2010 Alex Nelson
%% Last update Sun Jun 20 12:40:21 2010 Alex Nelson
%%
Today we will talk about compact Lie groups. First a few remarks
about characters. Suppose we have
\begin{equation}
\varphi\colon G\to\GL{n}
\end{equation}
be a group morphism. We have
\begin{equation}
\tr\big(\varphi(g)\big)=\chi_{\varphi}(g)
\end{equation}
be the character at the point $g$. The character is an invariant
of a representation. If $\varphi$, $\varphi'$ are two isomorphic
representations, so 
\begin{equation}
\varphi'(g) = A\varphi(g)A'
\end{equation}
then
\begin{equation}
\chi_{\varphi}(g)=\chi_{\varphi'}(g)
\end{equation}
for every $g\in G$. The character is a class function, it doesn't
depend on conjugation
\begin{equation}
\chi_{\varphi}(aga^{-1})=\chi_{\varphi}(g).
\end{equation}
This is basically everything we need. In general, for compact
groups $G$, the characters determine everything.

\begin{thm}
If $\rho\colon G\to\GL{n}$, $\rho'\colon G\to\GL{n}$ are two
representations such that $\chi_{\rho}(g)=\chi_{\rho'}(g)$ for
each $g\in G$, then $\rho\iso\rho'$.
\end{thm}

For a compact group, we have the invariant volume be
\begin{equation}
\int 1\cdot dv = 1
\end{equation}
which can be normalized. For a finite group, this integral
averaging a function is merely a sum
\begin{subequations}
\begin{align}
\overline{f} &= \int f(v)dv\quad\mbox{for compact groups}\\
&= \frac{1}{N}\sum_{g\in G}f(g)\quad\mbox{for finite groups}
\end{align}
\end{subequations}
We have 
\begin{equation}
\<f,f_{1}\> = \int f^{*}(v)f_{1}(v)dv = \overline{f^{*}f_{1}}
\end{equation}
we can compute the norm of characters
\begin{equation}
\|\chi\| = \sqrt{\<\chi,\chi\>} = \begin{cases} 1 & \mbox{if the rep is
  irreducible}\\
0 & \mbox{otherwise}
\end{cases}
\end{equation}
So we have
\begin{equation}
\<\chi_{i},\chi_{j}\>=\delta_{ij}
\end{equation}
so we have orthonormal characters from an orthonormal
basis. Where? For class functions!

Note class functions only depend on conjugacy classes. For
example, consider $G=\U{n}$, we can diagonalize any unitary
matrix by means of unitary transformations. We have then diagonal
matrices consists of
\begin{equation}
D = \begin{bmatrix}z_{1} & & \\
 & \ddots & \\
 &  & z_{n}
\end{bmatrix}
\end{equation}
where $z_{k}=\exp(i\varphi_{k})$, so $T\iso \U{1}^{n}$ the
$n$-torus.

What's relevant is that $T$ forms a commutative subgroup of
$\U{n}$. This is the maximal commutative subgroup for $\U{n}$, or
the maximal torus. Since every element is conjugate to this
stuff, if we want to know the character, we only need to be
concerned about the character on the torus. But can the character
be an arbitrary function here? No, it can't. So the characters
$\chi(z_{1}$, \dots, $z_{n})$ is a symmetric function on $T$, so it is
invariant under any permutation.

Let $G$ be a connected, compact Lie group. Let $T$ be the maximal
torus, i.e., the maximal Abelian subgroup. It will always be a
Torus, always a product of $\U{1}$. Then every element of $g\in
G$ is conjugate (conjugated in $G$) to an element of $T$. Not
every function on the torus is a character. We should consider
elements of $G$ that form the normalizer of $T$, i.e.
\begin{equation}
N(T) = \{x\in G\mid xT=Tx\}.
\end{equation}
We know $T\subset N(T)$ trivially, so to examine the nontrivial
part of the normalizer we should consider the quotient
\begin{equation}
N(T)/T = W
\end{equation}
called the \define{Weyl Group}\index{Weyl Group}. The Weyl group
acts on the Torus. It is obvious that characters should be
invariant with respect to this group.

Now to Lie algebras. We will work with complex Lie algebras. We
will take $G$ to be a compact Lie group, $\Bbb{C}\Lie(G)$ the
complexification of its Lie algebra. The Lie algebras obtained in
this way are \define{Reductive Lie Algebras}\index{Lie Algebra!Reductive}%
\index{Reductive Lie Algebra}. They have an invariant inner
product. Every representation of a compact group is equivalent to
a unitary representation (if complex), or an orthogonal
representation (if real). Lets take a Lie group $G$, lets take
its (real) Lie algebra 
\begin{equation}
\mathscr{G}=\Lie(G)=T_{e}G
\end{equation}
which is the tangent space to the group at $e\in G$. The group
$G$ acts on $T_{e}G$ in the obvious way. For some $x\in G$, then
\begin{equation}
gxg^{-1}\in G
\end{equation}
is an inner automorphism. It maps $e\mapsto e$. So every curve
starting at the identity goes to a curve also starting at the
identity, so we can define an action of $G$ on tangent
vectors. This is called the \define{Adjoint Representation of $G$},
this is probably the most important representation. We had a
notion of the adjoint representation for Lie algebras, and really
it's related to adjoint representations of Lie groups. We should
consider
\begin{equation}
g=1+\gamma
\end{equation}
where $\gamma$ is ``small.'' Then the adjoint group
representation is
\begin{subequations}
\begin{align}
(1+\gamma)x(1+\gamma)^{-1}&=x+\gamma x-x\gamma+\cdots \\
&= x + [\gamma,x] + \cdots
\end{align}
\end{subequations}
We are concluding that the adjoint representation of the group
corresponds to the adjoint representation of the algebra. Adjoint
representation for compact group is equivalent to an orthogonal
representation. There exists an inner product $\<x,y\>$ in
$\mathscr{G}$ which is invariant under
\begin{equation}
\<{\rm Ad}_{g}x,{\rm Ad}_{g}y\> = \<x,y\>,
\end{equation}
we see
\begin{equation}
{\rm Ad}_{1+\gamma+\cdots}x = x + [\gamma,x] + \cdots = x + {\rm
  ad}_{\gamma}x + \cdots.
\end{equation}
(The convention is adjoint representation for the Lie group is
written as ``Ad'' but for the Lie algebra is ``ad''.)

We get
\begin{equation}\label{eq:orthogonalCondition}
\<[\gamma,x],y\>+\<x,[\gamma,y]\>=0.
\end{equation}
If $G$ is compact, then $\Lie(G)$ is equipped with a
nondegenerate positive invariant product (which is precisely 
the Eq \eqref{eq:orthogonalCondition} condition). We can extend
this to the complexified Lie algebra $\Bbb{C}\Lie(G)$ but it is a
nondegenerate invariant inner product. This is a general result.

Examples. Consider $\mathfrak{gl}(n)$, we have an invariant inner
product $\<x,y\>=\tr(xy)$.
