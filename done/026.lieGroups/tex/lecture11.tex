%%
%% lecture11.tex
%% 
%% Made by Alex Nelson
%% Login   <alex@tomato3>
%% 
%% Started on  Mon Jun 14 15:56:24 2010 Alex Nelson
%% Last update Mon Jun 14 16:29:24 2010 Alex Nelson
%%

We consider representations of $\frak{sl}(2)=A_{1}$. We analyzed
completely the finite dimensional representations; the only place
where finite dimensions were used was in proving the existence of
the highest weight vector. We reasoned $h$ has eigenvectors. Then
we applied $e$ to the eigenvectors of $h$, which produced a
different eigenvector or zero.

We had the eigenvector
\begin{equation}
h\vec{v}=\lambda\vec{v},
\end{equation}
then
\begin{equation}
h(e^{k}\vec{v})=(\lambda+2k)e^{k}\vec{v},
\end{equation}
but in the finite dimensional case we get at some moment
\begin{equation}
e^{n}\vec{v}=0
\end{equation}
for some $n$. So in finite dimensions, such a vector always
exists. In the infinite dimensional case, we will assume a
highest weight vector exists. Then we will describe all
irreducible representations, we did this basically. Let $\vec{v}$
be such that
\begin{equation}
e\vec{v}=0,
\end{equation}
then let
\begin{equation}
\vec{v}_{k}=f^{k}\vec{v},
\end{equation}
so
\begin{equation}
h\vec{v}_{k}=(m-2k)\vec{v}_{k}
\end{equation}
and we have ``ladder relations''
\begin{equation}
f\vec{v}_{k}=\vec{v}_{k+1}
\end{equation}
and
\begin{equation}
e\vec{v}_{k}=\gamma_{k}\vec{v}_{k-1},
\end{equation}
where $\gamma_{k}$ is some coefficient which requires solving a
recursive formula. We have
\begin{equation}
\gamma_{k}=k(m-k+1).
\end{equation}
Lets prove this is an irreducible representation. What does this
mean? It doesn't have any nontrivial subrepresentations. Suppose
we do have some nontrivial subrepresentation, it should contain
at least one vector. Suppose this one vector is of the form
\begin{equation}
\sum c_{k}\vec{v}_{k}=\vec{w}.
\end{equation}
Lets apply to this vector $e^{s}\vec{w}$, what happens? It is pretty
clear it should be $\vec{w}=\vec{v}_{s'}$ where $s'$ is some
index, this is due to $h$ having an eigenvector in any representation.

We can now apply $e$ and $f$ to $\vec{w}$, we end up recovering
\begin{equation}
e^{s'}\vec{w}\propto\vec{v}_{0}.
\end{equation}
We made the mistake that
\begin{equation}
e\vec{v}_{k}=k(m-k+1)\vec{v}_{k-1}
\end{equation}
exists, i.e. $m-k+1\not=0$. If $m\not\in\Bbb{Z}$, then this is an
irreducible representation. But if $m\in\Bbb{Z}$, more
specifically
\begin{equation}
k=m+1,
\end{equation}
then
\begin{equation}
e\vec{v}_{m+1}=0.
\end{equation}
So we get an irreducible subrepresentation spanned by
$\vec{v}_{0}$, $\vec{v}_{1}$, ..., $\vec{v}_{m}$. So for each
$m\in\Bbb{N}$, we have on irreducible representation of dimension
$m+1$, so we have $m+1=1,2,3,...$.

Now lets discuss the group $SU(2)$, recall $\frak{su}(2)$
consists of traceless anti-Hermitian matrices; recall unitary
matrices satisfy
\begin{equation}
A^{\dagger}A=I.
\end{equation}
The rows and columns form an orthonormal basis:
\begin{equation}
A = \begin{bmatrix}a&b\\c&d\end{bmatrix}
\end{equation}
so
\begin{equation}
|a|^{2}+|b|^{2}=1\quad\mbox{and}\quad|c|^{2}+|d|^{2}=1\quad\mbox{and}\quad ad-bc=1.
\end{equation}
If we know $a$ and $b$, we can deduce that
\begin{equation}
\begin{bmatrix}
a&b\\c&d
\end{bmatrix}=
\begin{bmatrix}
a&b\\
-\overline{b}&\overline{a}
\end{bmatrix}.
\end{equation}
We can consider the topology here, if
\begin{equation}
a=a_{0}+ia_{1}\quad\mbox{and}\quad b=b_{0}+ib_{1}
\end{equation}
then we are working with a 4-dimensional sphere. We can say that
topologically SU(2) is compact and simply connected. The
representations of SU(2) may be identified by representations of
its Lie algebra. We recall
\begin{equation}
\Bbb{C}\lie\big(SU(2)\big)=\frak{sl}(2).
\end{equation}
We may describe the representations directly.

\begin{rmk}
$\frak{su}(2)$ has a scalar representation, i.e. the most boring
  representation imaginable (everything is represented by the
  unit matrix). It's trivial, and has dimension 1.
\end{rmk}

Now we have the vector or ``fundamental'' representation. It is
2-dimensional. Every matrix is represented by itself. Let
$V=\Bbb{C}^{2}$ be the representation of SU(2). For this special
situation, we can work with polynomials of $x,y\in\Bbb{C}$.

That is to say, if $\{(x,y)\}=V$, we may take the space of
functions on $V$. We introduce
\begin{equation}
\psi_{g}\colon\varphi(z)\mapsto\varphi(g^{-1}z)
\end{equation}
which ``deforms'' $\varphi(z)$ into $\varphi(g^{-1}z)$. If we
have
\begin{equation}
\psi_{h}\colon\varphi(z)\mapsto\varphi(h^{-1}z),
\end{equation}
then we demand
\begin{equation}
\varphi_{h}\circ\varphi_{g}\colon\varphi(z)\mapsto\varphi\big((hg)^{-1}z\big).
\end{equation}
Functions of $V$ are contravariant functors.

This is a reducible representation, since we may restrict our
focus to polynomials over $V$. Is the space of polynomials an
irreducible representation? No! Why? We can have the subspace of
homogeneous polynomials of degree $m$. So it would be irreducible
and spanned by $x^{m}$, $x^{m-1}y$, ..., $xy^{m-1}$, $y^{m}$
which is of dimensions $m+1$. We can deduce the representation of
the Lie algebra. Observe for us in $\frak{su}(2)$,
\begin{equation}
h=\begin{bmatrix}
1 & 0\\
0 & -1
\end{bmatrix}
\end{equation}
so it corresponds to
\begin{equation}
\begin{bmatrix}
u & 0\\
0 & u^{-1}
\end{bmatrix}\in{\rm SU}(2).
\end{equation}
How $h$ acts on the basis is that
\begin{equation}
h\colon x\mapsto ux,\quad y\mapsto u^{-1}y.
\end{equation}
So in effect,
\begin{equation}
x^{m-k}y^{k}\mapsto u^{m-2k}x^{m-k}y^{k},
\end{equation}
with the Lie algebra $u=1+\varepsilon$ where
$|\varepsilon|^{2}\ll1$. So 
\begin{equation}
u^{m}=1+m\varepsilon
\end{equation}
and so on.
