%%
%% lecture09.tex
%% 
%% Made by Alex Nelson
%% Login   <alex@tomato3>
%% 
%% Started on  Mon Jun 14 10:11:45 2010 Alex Nelson
%% Last update Mon Jun 14 15:20:28 2010 Alex Nelson
%%

Let $G$ be a Lie group, consider $\lie(G)$ its Lie algebra. Then
there is a correspondence between curves in the group and curves
in the Lie algebra. So if we have two curves in the Lie algebra,
we have two curves in the Lie group, then for simply connected
groups we may deform two curves $g_{1}(t)$, $g_{2}(t)$ with
\begin{equation}
g_{1}(0)=g_{2}(0)=g_{0},\quad\mbox{and}\quad
g_{1}(1)=g_{2}(1)=g_{1}
\end{equation}
by introducing a family of curves $g_{\tau}(t)$ which has a
corresponding family of curves in the Lie algebra. We have
\begin{subequations}
\begin{align}
\xi(\tau,t) &= g_{\tau}(t)^{-1}\frac{\D g_{\tau}(t)}{\D t}\\
\eta(\tau,t) &= g_{\tau}(t)^{-1}\frac{\D g_{\tau}(t)}{\D \tau}
\end{align}
\end{subequations}
and we have the relation
\begin{equation}
\frac{\partial\eta}{\partial t}-\frac{\partial\xi}{\partial t}=[\xi,\eta]
\end{equation}
which occurs in the Lie algebra. Observe that 
\begin{equation}
\xi(t,0)=\gamma_{0}(t),\quad\xi(t,1)=\gamma_{1}(t),\quad\eta(0,\tau)=\eta(1,\tau)=0.
\end{equation}
So given these conditions that, for
\begin{equation}
\frac{\partial\eta(\tau,t)}{\partial t}-\frac{\partial\xi(\tau,t)}{\partial\tau}
=[\xi(\tau,t),\eta(\tau,t)]
\end{equation}
with boundary conditions
\begin{subequations}
\begin{align}
\xi(t,0)=\gamma_{0}(t)\quad\mbox{and}\quad\xi(t,1)=\gamma_{1}(t)\\
\eta(1,\tau)=\eta(0,\tau)=0
\end{align}
\end{subequations}
can we get information induced in the group? We have
\begin{equation}\label{eq:lec09:diffEqXi}
\xi(t,\tau)=g(t,\tau)^{-1}\frac{\partial g(t,\tau)}{\partial t},
\end{equation}
where $g(0,\tau)=1$. We can restore $g(t,\tau)$ since there is a
unique solution to eq \eqref{eq:lec09:diffEqXi}.

%% So, more or less, $\widetilde{\xi}\to g\to(\xi,\eta)$. We can
%% find new $\xi$, $\eta$ satisfying the above. So
%% $\widetilde{\xi}=\xi$ by construction.

Suppose we have $\lie(G)\to\lie(G')$ be a Lie algebra morphism;
how can we induce a Lie group morphism? Well, how we do it makes
heavy use of this curve voodoo. The basic correspondence we have
is that ``points in the group'' corresponds to ``curves in the
Lie Algebra'', and ``multiplication in the group'' corresponds to
``concatenation of paths in the Lie algebra.'' Group curve
concatenation can be performed, for 
\begin{equation}
g_{1}:[0,b]\to G
\end{equation}
and
\begin{equation}
g_{2}:[b,a]\to G,
\end{equation}
as 
\begin{equation}
g(t)=\begin{cases} g_{1}(t) & t\in[0,b]\\
g_{2}(b)^{-1}g_{2}(t) & t\in[b,a].
\end{cases}
\end{equation}
If the paths $g_{1}(t)$, $g_{2}(t)$ are not loops,
i.e. $g_{1}(0)\not=g_{1}(b)$ and $g_{2}(b)\not=g_{2}(a)$, then
\begin{equation}
g(t)=\begin{cases} g_{1}(t) & t\in[0,b]\\
g_{1}(b)g_{2}(b)^{-1}g_{2}(t) & t\in[b,a].
\end{cases}
\end{equation}
We see that $g(b)$ is in the first case equal to $g_{1}(b)$, and
in the second case
\begin{equation}
g(b) = g_{1}(b)g_{2}(b)^{-1}g_{2}(b) = g_{1}(b).
\end{equation}
Thus the two cases agree on the overlap.

The corresponding curve in the Lie algebra is
\begin{equation}
\gamma(t) = \begin{cases}\displaystyle
g_{1}(t)^{-1}\frac{\D g_{1}(t)}{\D t} & t\in[0,b]\\
\displaystyle (g_{2}(b)^{-1}g_{2}(t))^{-1}\left(g_{2}(b)^{-1}\frac{\D g_{2}(t)}{\D t}\right)
& t\in[b,a]
\end{cases}
\end{equation}
up to a constant (i.e. $g_{1}(b)$) in the second case. It doesn't
play a significant role, as it is factored out. We end up with
\begin{equation}
\gamma(t) = \begin{cases}\displaystyle
g_{1}(t)^{-1}\frac{\D g_{1}(t)}{\D t} & t\in[0,b]\\
\displaystyle g_{2}(t)^{-1}\frac{\D g_{2}(t)}{\D t} & t\in[b,a]
\end{cases}
\end{equation}
We will consider the construction of Lie groups from Lie algebra
next time...

We proved there exists a one-to-one correspondence between simply
connected Lie groups and finite dimensional Lie algebras. If we
have a discrete normal subgroup $N\subset G$, then the Lie
algebra of $G/N\iso\lie(G)$. This is because there is a
neighborhood $\mathcal{U}$ of $1\in G$ such that $\mathcal{U}\cap
N=\{1\}$. 

\begin{thm}
If $G$ is simply connected, and $\lie(G)\iso\lie(G')$, then
$G'\iso G/N$ where $N$ is a discrete normal subgroup of $G$.
\end{thm}
\begin{ex}
$\Bbb{R}$ equipped with addition has trivial commutators in Lie
  algebra, but $\lie\big(U(1)\big)\iso\lie(\Bbb{R})$ so
  $U(1)\iso\Bbb{R}/\Bbb{Z}$. 
\end{ex}
