%%
%% lecture08.tex
%% 
%% Made by Alex Nelson
%% Login   <alex@tomato3>
%% 
%% Started on  Sun Feb 14 15:56:21 2010 Alex Nelson
%% Last update Sun Jun 20 11:59:29 2010 Alex Nelson
%%
%\noindent\textbf{Main Theorems}
\subsection{Main Theorems}
We havea group $G$ and we can construct the corresponding Lie
Algebra $\lie(G)$ constructed by examining the tangent space at
the identity $e=I\in G$:
\begin{equation}
\lie(G)=T_{e}G.
\end{equation}
We discussed obtaining the Lie algebra \emph{from} the Lie group,
and if we have a group morphism
\begin{equation}
\varphi\colon G\to G',
\end{equation}
we can construct a morphism of the corresponding Lie Algebras
\begin{equation}
\varphi_{*}\colon \lie(G)\to\lie(G').
\end{equation}
That is to say the following diagram commutes
\begin{equation}
\begin{diagram}
G              & \rTo & \lie(G)\\
\dTo>{\varphi} &      & \dTo>{\varphi_{*}}\\
G'             & \rTo & \lie(G')
\end{diagram}
\end{equation}
we would like to consider going the other way. That is given a
Lie algebra $\mathscr{G}$, we would like to construct a
corresponding group, and show that Lie algebra morphisms
$\mathscr{G}\to\mathscr{G}'$ generate group morphisms. But  the
group needs to be simply connected.

Consider $SU(2)/\Bbb{Z}_{2}=SO(3)$. However $SU(2)$ has the same
Lie algebra as $SO(3)$; when we identify the algebra
topologically as the sphere, this quotient identifies opposite
points as the same. This doesn't affect the Lie algebra.

If we drop the condition of being simply connected, a Lie algebra
may give rise to two different Lie groups. Simply connected
permits us to continuously deform one path to another. Every
closed curved is contractible iff the space is simply connected.

\begin{thm}
For each Lie algebra there exists a uniquie simply connected Lie group.
\end{thm}

If $g\colon[0,1]\to G$ is a differentiable curve on the group, we
may construct a curve
\begin{equation}
\gamma(t)=g(t)^{-1}\frac{dg(t)}{dt}=\frac{d}{dt}\log\big(g(t)\big)
\end{equation}
on the algebra. We have
\begin{equation}
\left.\frac{dg(t)}{dt}\right|_{t}\in T_{g(t)}G,
\end{equation}
which is not necessarily in the Lie algebra $T_{e}G$. However for
some $g\in G$ we have $g\cdot 1=g$, so this is a translation
which sends $1\mapsto g$. We have a map
\begin{equation}
g_{*}\colon T_{e}G\to T_{g}G,
\end{equation}
and so the right formula would be
\begin{equation}
\gamma(t)=\left(g_{*}(t)\right)^{-1}\frac{dg(t)}{dt}.
\end{equation}
But we will abuse notation and write
\begin{equation}
\gamma(t)=g(t)^{-1}\frac{dg(t)}{dt}.
\end{equation}
We have a correspondence between curves in the Lie Algebra and
curves in the Lie group. We obtain a system of differential
equations 
\begin{equation}
\frac{dg(t)}{dt}=g(t)\gamma(t)
\end{equation}
which has a unique solution for $g(0)=e$.\marginpar{We use $e$,
  $1$, $I$ for the identity interchanging notation without warning...}

\begin{wrapfigure}[10]{r}{6.25cm}  
%  \vspace{-20pt}
  \begin{center}
    \includegraphics{img/LieImg.1}
  \end{center}
  \vspace{-20pt}
\end{wrapfigure}

Consider a curve $g_{0}(t)$, $g_{1}(t)$ in the group, where
\begin{equation}
g_{0}(0)=g_{1}(0)=e,
\end{equation}
and we have
\begin{equation}
g_{0}(1)=g_{1}(1)=g.
\end{equation}
\begin{wrapfigure}[6]{l}{4cm}
  \vspace{-20pt}
  \begin{center}
    \includegraphics{img/LieImg.2}
  \end{center}
  \vspace{-20pt}
\end{wrapfigure}

\noindent\ignorespaces{}We assume that $G$ is simply connected
and we will deform the path, thus obtaining a family of paths
$g_{\tau}(t)=g(\tau,t)$ such that $g(0,t)=g_{0}(t)$ and
$g(1,t)=g_{1}(t)$. We can draw a diagram (seen on the left), we
have two variables to consider $\tau$ and $t$, which to
differentiate?  Both of them! Consider
\begin{equation}
\xi(\tau,t)=g(\tau,t)^{-1}\frac{\partial g(\tau,t)}{\partial t}
\end{equation}
and
\begin{equation}
\eta(\tau,t)=g(\tau,t)^{-1}\frac{\partial g(\tau,t)}{\partial\tau}.
\end{equation}
So what do we know? Well, the curves in the Lie algebra are
\begin{equation}
\gamma_{0}(t) = g_{0}(t)^{-1}\frac{dg_{0}(t)}{dt}=\xi(0,t)
\end{equation}
and
\begin{equation}
\gamma(1)(t) = g_{1}(t)^{-1}\frac{dg_{1}(t)}{dt}=\xi(1,t).
\end{equation}
We also know $g_{\tau}(0)=e=1$, and $g_{\tau}(1)=g$, for every
$\tau$. We see then that
\begin{equation}
\eta(\tau,0)=\eta(\tau,1)=0.
\end{equation}
We have
\begin{equation}
\partial_{t}g(\tau,t)=g(\tau,t)\xi(\tau,t)
\end{equation}
and
\begin{equation}
\partial_{\tau}g(\tau,t)=g(\tau,t)\eta(\tau,t).
\end{equation}
We deduce from
\begin{equation}
\partial_{\tau}\partial_{t}g(\tau,t)=\partial_{t}\partial_{\tau}g(\tau,t)
\end{equation}
that
\begin{equation}
(\partial_{\tau}g)\xi+g\partial_{\tau}\xi=(\partial_{t}g)\eta+g(\partial_{t}\eta),
\end{equation}
and by multiplying on the right by $g(\tau,t)^{-1}$ we have
\begin{equation}
\underbracket[0.5pt]{(g^{-1}\partial_{\tau}g)}_{=\eta}\xi+\partial_{\tau}\xi=\underbracket[0.5pt]{(g^{-1}\partial_{t}g)}_{=\xi}\eta+\partial_{t}\eta\quad\implies\quad
\partial_{t}\eta-\partial_{\tau}\xi=\eta\xi-\xi\eta.
\end{equation}
We have an equation of r the form
\begin{equation}
\frac{\partial\eta(\tau,t)}{\partial t}-\frac{\partial\xi(\tau,t)}{\partial\tau}
=[\xi(\tau,t),\eta(\tau,t)].
\end{equation}
Now, what should we do with this? In reality, we've done
everything already. What is our goal? We'd like to restore the
group knowing the Lie Algebra. We get points in the group by
considering equivalence classes of paths in the Lie Algebra.

\subsection{Exercises}
\subsubsection{Algebra \texorpdfstring{$\ClassicalGroup{D}_{n}$}{Dn}}

The Lie algebra $\ClassicalGroup{D}_{n}$ consists of $2n\times2n$ complex matrices
$L$ obeying
\begin{equation}
(FL)^{T}+FL=0
\end{equation}
where, in block form,
\begin{equation}
F=\begin{bmatrix}
0&1\\
1&0
\end{bmatrix}.
\end{equation}

\begin{exercise}
Check that $\ClassicalGroup{D}_{n}$ is isomorphic to the complexification of the
Lie algebra of the orthogonal group $\ORTH(2n)$.
\end{exercise}
\begin{exercise}
Check that the matrices
\begin{equation}
e_{ij}:=\begin{bmatrix}E_{ij}&0\\
0&-E_{ji}
\end{bmatrix}
\end{equation}
together with the matrices
\begin{equation}
f_{pq}:=\begin{bmatrix}0&E_{pq}-E_{qp}\\
0&0
\end{bmatrix},\qquad
g_{pq}:=\begin{bmatrix}0&0\\
E_{pq}-E_{qp}&0
\end{bmatrix}
\end{equation}
form a basis of $\ClassicalGroup{D}_{n}$. 

Here $i,j=1,\ldots,n$, $1\leq p<q\leq n$, and $E_{i,j}$ has only
one nonzero entry that is equal to unity located in the $i^{th}$
row and $j^{th}$ column.
\end{exercise}
\begin{exercise}
Check that the subalgebra $\frak{h}$ of all matrices of the form
\begin{equation}
\begin{bmatrix}
A&0\\
0&-A
\end{bmatrix}
\end{equation}
(where $A$ is a diagonal matrix) is a maximal commutative
subalgebra, and prove that there exists a basis of $\ClassicalGroup{D}_{n}$
consisting of eigenvectors for elements of $\frak{h}$ acting on
$\ClassicalGroup{D}_{n}$ by means of adjoint representation. (This means that
$\frak{h}$ is a Cartan subalgebra of $\ClassicalGroup{D}_{n}$.)
\end{exercise}
\begin{exercise}
Check that $e_{i}=e_{i,i+1}$ for $i=1,...,n-1$ and
$e_{n}=f_{n-1,n}$; $f_{i}=e_{i+1,i}$ for $i=1,...,n-1$ and
$f_{n}=g_{n-1,n}$ form a system of multiplicative generators of
$\ClassicalGroup{D}_{n}$. Prove the relations
\begin{subequations}
\begin{align}
[e_{i},f_{j}]&=\delta_{ij}h_{i}\\
[h_{i},h_{j}]&=0\\
[h_{i},e_{j}]&=a_{ij}e_{j}\\
[h_{i},f_{j}]&=-a_{ij}f_{j}\\
({\rm ad}e_{i})^{1-a_{ij}}e_{j}&=0\qquad\mbox{when $i\not=j$}\\
({\rm ad}f_{i})^{1-a_{ij}}f_{j}&=0\qquad\mbox{when $i\not=j$}
\end{align}
\end{subequations}
We use here the notation $(\ad x)$ for the operator transforming $y$ into $[x,y]$.
\end{exercise}

\subsubsection{Algebra \texorpdfstring{$\ClassicalGroup{C}_{n}$}{Cn}}
Consider the Lie algebra $\ClassicalGroup{C}_{n}$ consisting of $2n\times2n$
complex matrices obeying
\begin{equation}
(FL)^{T}+FL=0
\end{equation}
where
\begin{equation}
F=\begin{bmatrix}0&1\\
-1&0
\end{bmatrix}.
\end{equation}

\begin{exercise}
Check that $\ClassicalGroup{C}_{n}$ is isomorphic to the complexification of the
Lie algebra of the compact group $\Sp{2n}\cap \U{2n}$ where
$\Sp{2n}$ stands for the group of linear transformations of
$\CC^{2n}$ preserving non-degerate anti-symmetric bilinear
form and $\U{2n}$ denotes unitary group.
\end{exercise}
\begin{exercise}
Check that the matrices
\begin{subequations}
\begin{align}
e_{ij} &= \begin{bmatrix}E_{ij}&0\\0&-E_{ji}
\end{bmatrix}\\
f_{pq}&= \begin{bmatrix}0&E_{pq}+E_{qp}\\0&0
\end{bmatrix}\\
g_{pq}&=\begin{bmatrix}0&0\\E_{pq}+E_{qp}&0
\end{bmatrix}
\end{align}
\end{subequations}
form a basis of $\ClassicalGroup{C}_{n}$, where $i,j=1,...,n$ and $1\leq p\leq
q\leq n$.
\end{exercise}
\begin{exercise}
Check that the subalgebra $\frak{h}$ of all matrices of the form
\begin{equation}
\begin{bmatrix}
A&0\\
0&-A
\end{bmatrix}
\end{equation}
where $A$ is a diagonal matrix, is a maximal commutative
subalgebra. Prove there exists a basis of $\ClassicalGroup{C}_{n}$ consisting of
eigenvectors for elements of $\frak{h}$ acting on $\ClassicalGroup{C}_{n}$ by
means of adjoint representation.
\end{exercise}
\begin{exercise}
Check that $e_{i}=e_{i,i+1}$ ($i=1,...,n-1$) and $e_{n}=f_{n,n}$;
$f_{i}=e_{i+1,i}$ ($i=1,...,n-1$) and $f_{n}=g_{n,n}$ form a
system of generators of $\ClassicalGroup{C}_{n}$. Prove
\begin{subequations}
\begin{align}
[e_{i},f_{j}]&=\delta_{ij}h_{i},\\
[h_{i},h_{j}]&=0\\
[h_{i},e_{j}]&=a_{ij}e_{j},\\
[h_{i},f_{j}]&=-a_{ij}f_{j},\\
({\rm ad}e_{i})^{1-a_{ij}}e_{j}&=0,\qquad i\not=j \\
({\rm ad}f_{i})^{1-a_{ij}}f_{j}&=0,\qquad i\not=j
\end{align}
\end{subequations}
\end{exercise}
\subsubsection{Algebra \texorpdfstring{$\ClassicalGroup{B}_{n}$}{Bn}}

The algebra $\ClassicalGroup{B}_{n}$ consists of $(2n+1)\times(2n+1)$ complex
matrices obeying
\begin{equation}
L^{T}F+FL=0
\end{equation}
where
\begin{equation}
F=\begin{bmatrix}1&0&0\\
0&0&I_{n}\\
0&I_{n}&0
\end{bmatrix}
\end{equation}
$I_{n}$ is the $n\times n$ identity matrix, and we have written $F$ in block form.

\begin{exercise}
Show that $\ClassicalGroup{B}_{n}$ is isomorphic to the complexified Lie algebra of $\ORTH{2n+1}$.
\end{exercise}
\begin{exercise}
Check that the subalgebra $\frak{h}$ of all matrices of the form
\begin{equation}
\begin{bmatrix}0&0&0\\
0&A&0\\
0&0&-A
\end{bmatrix}
\end{equation}
(where $A$ is a diagonal matrix) is a maximal Abelian subalgebra,
and prove there is a basis of $\ClassicalGroup{B}_{n}$ consisting of eigenvectors
for elements of $\frak{h}$ acting on $\ClassicalGroup{B}_{n}$ by the adjoint
representation.
\end{exercise}
\begin{exercise}
Find a system $e_{i}$, $f_{j}$ of multiplicative generators of
$\ClassicalGroup{B}_{n}$ obeying
\begin{subequations}
\begin{align}
[e_{i},f_{j}]&=\delta_{ij}h_{i}\\
[h_{i},h_{j}]&=0\\
[h_{i},e_{j}]&=a_{ij}e_{j}\\
[h_{i},f_{j}]&=-a_{ij}f_{j}\\
({\rm ad}e_{i})^{1-a_{ij}}e_{j}&=0,\qquad i\not=j\\
({\rm ad}f_{i})^{1-a_{ij}}f_{j}&=0,\qquad i\not=j
\end{align}
\end{subequations}
for ``some'' matrix $a_{ij}$.
\end{exercise}
\begin{exercise}
Describe the roots and root vectors of $\ClassicalGroup{A}_{n}$, $\ClassicalGroup{B}_{n}$, $\ClassicalGroup{C}_{n}$, $\ClassicalGroup{D}_{n}$.
\end{exercise}
