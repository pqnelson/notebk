%%
%% lecture06.tex
%% 
%% Made by Alex Nelson
%% Login   <alex@tomato3>
%% 
%% Started on  Thu Jan 21 12:28:44 2010 Alex Nelson
%% Last update Thu Jan 21 12:55:27 2010 Alex Nelson
%%
\begin{thm}
Consider a matrix group $G\subset\GL{n}$. Consider $\scr{G}$ the
tangent space to $G$ at 1. Then $\scr{G}$ is a Lie algebra with
respect to commutator of elements.
\end{thm}
\begin{proof}
Consider a curve
\begin{equation}
x\colon[0,1]\to G,\qquad x(\tau)=1+a\tau+\mathcal{O}(\tau^{2}),
\end{equation}
we obtain the tangent vector
\begin{equation}
x'(0)=\lim_{\tau\to0}\frac{x(\tau)-x(0)}{\tau}=a.
\end{equation}
We are proving this is a vector space and a ring.

\noindent\textbf{Vector Space.} If we have two
curves
\begin{equation}
x(\tau)=1+a\tau+\cdots,\qquad y(\tau)=1+b\tau+\cdots,
\end{equation}
we multiply together to find
\begin{equation}
x(\tau)y(\tau)=1+(a+b)\tau+\mathcal{O}(\tau^{2}),
\end{equation}
and its derivative at 0 is
\begin{equation}
\left.\frac{d}{d\tau}(x(\tau)y(\tau))\right|_{\tau=0}=a+b.
\end{equation}
This implies that $a+b\in\scr{G}$. We can similarly show
\begin{equation}
\left.\frac{d}{d\tau}x(\lambda\tau)\right|_{\tau=0}=\lambda a\in\scr{G}
\end{equation}
for all $\lambda\in\Bbb{R}$. Thus it's a vector space.

\noindent\textbf{Ring.} The commutator $[a,b]$ is obtained by the
group commutator. We first let
\begin{equation}
x(\tau)=1+a\tau+\alpha\tau^2+\cdots,\qquad y(\tau)=1+b\tau+\beta\tau^{2}+\cdots,
\end{equation}
and for the inverses use primed coefficients
\begin{equation}
x^{-1}(\tau)=1+a'\tau+\alpha'\tau^2+\cdots,\qquad y^{-1}(\tau)=1+b'\tau+\beta'\tau^{2}+\cdots.
\end{equation}
Then we plug it into the definition of the group commutator:
\begin{equation}
x(\tau)y(\tau)x^{-1}(\tau)y^{-1}(\tau)=(1+a\tau+\cdots)(1+b\tau+\cdots)(1+\alpha\tau+\cdots)(1+\beta\tau+\cdots)
\end{equation}
and we demand that
\begin{equation}
x(\tau)x^{-1}(\tau)=1
\end{equation}
produces the conditions that $a+a'=0$, and $aa'+\alpha'+\alpha=0$
from the first and second order terms respectively. We see
similar reasoning for $y(\tau)y^{-1}(\tau)=1$ produces $b+b'=0$
and $bb'+\beta'+\beta=0$ as conditions on the coefficients. By
carrying out multiplication, we find
\begin{align}
&(1+a\tau+\alpha\tau^2+\cdots)(1+b\tau+\beta\tau^{2}+\cdots)
(1+a'\tau+\alpha'\tau^2+\cdots)(1+b'\tau+\beta'\tau^{2}+\cdots)\nonumber\\
&=1+(a+b+a'+b')\tau\nonumber\\
&\qquad+(\alpha+\beta+\alpha'+\beta'+ab+aa'+ab'+ba'+bb'+a'b')\tau^{2}+\cdots
\end{align}
The first order term vanishes identically. The second order terms
can be factored as
\begin{equation}
[(\alpha+\alpha'+aa')+(\beta+\beta'+bb')+ab+ab'+ba'+a'b']\tau^{2}
\end{equation}
which can be rewritten as
\begin{equation}
[ab+(-a)(-b)+a(-b)+(-a)b]\tau^{2}=[ab-ba]\tau^{2}.
\end{equation}
Thus
\begin{equation}
x(\tau)y(\tau)x(\tau)^{-1}y(\tau)^{-1}=1+[a,b]\tau^{2}+\mathcal{O}(\tau^{3}).
\end{equation}
We can introduce a new parameter $\sigma=\sqrt{\tau}$ and rewrite
our equations to first order in $\sigma$ which implies $[a,b]\in\scr{G}$.
\end{proof}

\subsection{Derivations as Infinitesimal Automorphisms}
We have an algebra $\scr{A}$, all atuomorphisms of $\scr{A}$ form
a group $\aut(\scr{A})$. We may say that derivations are
``tangent vectors'' to automorphisms, or in other words are
infinitesimal automorphisms. Consider a continuous family of
automorphisms $x(\tau)$, then 
\begin{equation}
x(\tau)(ab)=x(\tau)(a)\cdot x(\tau)(b).
\end{equation}
We take its derivative, if $x'(\tau)=\alpha_{\tau}$, then
\begin{equation}
\alpha_{\tau}(ab)=\alpha_{\tau}(a)x(\tau)(b)+x(\tau)(a)\alpha_{\tau}(b).
\end{equation}
We assume $x(0)=1$, then
\begin{equation}
\alpha_{0}(ab)=\alpha_{0}(a)b+a\alpha_{0}(b),
\end{equation}
in other words derivations are infinitesimal automorphisms. We
can say that derivations form the tangent space to
$e\in\aut(\scr{A})$.

\noindent\textbf{N.B.} \emph{We cheated!} We assumed $\scr{A}$
was equipped with a topology, so a curve being ``continuous''
(much less differentiable!) made sense. 

Consider a smooth manifold $M$, consider $C^{\infty}(M)$ the
algebra of smooth functions on $M$. Any change of variables,
i.e. smooth map, $\varphi\colon M\to M$ will generate an
automorphism. Diffeomorphisms of $M$ may be considered as
automorphisms of $C^{\infty}(M)$. Then $\diff(M)\subset\aut(C^{\infty}(M))$.

To be completely precise we should write $f(\varphi^{-1}(x))$. We
may say the diffeomorphisms form an infinite dimensional
topological group. If $M$ is compact, there is a natural notion
of topology on $\diff(M)$. If we have a family of diffeomorphisms
$\varphi_{t}(x)$, differentiation with respect to $t$ is
obvious. Vector fields are derivations of this algebra; that is
$\der=\lie\aut\scr{A}$ and
\begin{equation}
\vect(M)\subset\lie\diff(M).
\end{equation}
We are of course a bit sloppy here....we won't go into details on
infinite dimensional groups.

Suppose that $G$ acts on $M$, its action can be a smooth
action. That is we have a morphism of $G\to\diff(M)$ and at the
level of Lie algebras consider $\lie(G)\to\vect(M)$.
