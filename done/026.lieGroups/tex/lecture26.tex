%%
%% lecture26.tex
%% 
%% Made by Alex Nelson
%% Login   <alex@tomato3>
%% 
%% Started on  Mon Dec 13 14:22:29 2010 Alex Nelson
%% Last update Mon Dec 13 18:24:22 2010 Alex Nelson
%%
Last time we considered Clifford algebras and its representation
by gamma matrices
\begin{equation}
\Gamma^{\mu}\Gamma^{\nu}+\Gamma^{\nu}\Gamma^{\mu}=2\eta^{\mu\nu}\cdot\1
\end{equation}
We can consider part of this business which is spanned by
\begin{equation}
a\1 + b_{\mu\nu}\Gamma^{\mu}\Gamma^{\nu}
\end{equation}
zero order terms and quadratic terms. This is a subalgebra of the
associative algebra. We may consider it with an induced Lie
algebra structure. We consider the coefficients of the quadratic
term
\begin{equation}
b_{\mu\nu}=-b_{\nu\mu}
\end{equation}
to be antisymmetric. Moreover this is an ``almost'' orthogonal
Lie algebra.

We will introduce new notation
\begin{equation}
\Gamma^{\mu\nu} =
\frac{1}{2}(\Gamma^{\mu}\Gamma^{\nu}-\Gamma^{\nu}\Gamma^{\mu}).
\end{equation}
This implies that
\begin{equation}
\Gamma^{\mu\nu}=-\Gamma^{\nu\mu}.
\end{equation}
We also observe that
\begin{subequations}
\begin{align}
\Gamma^{\mu\nu} &=
\frac{1}{2}(\Gamma^{\mu}\Gamma^{\nu}+\Gamma^{\mu}\Gamma^{\nu}-2\eta^{\mu\nu}\1)\\
&= \Gamma^{\mu}\Gamma^{\nu}-\eta^{\mu\nu}\cdot\1
\end{align}
\end{subequations}
This is something created in such a way it gives us an orthogonal
Lie algebra. We may take $a_{\mu\nu}\Gamma^{\mu\nu}$
requiring\footnote{Although this requirement is not really
  necessary. We can write $a_{\mu\nu}$ as the sum of the
  symmetric and antisymmetric part. But if $s_{\mu\nu}$ is
  symmetric, then $s_{\mu\nu}\Gamma^{\mu\nu}=0$ always.}
\begin{equation}
a_{\mu\nu}=-a_{\nu\mu}.
\end{equation}
If we take the commutator of
\begin{equation}
[a_{\mu\nu}\Gamma^{\mu\nu},b_{\rho\sigma}\Gamma^{\rho\sigma}]=C_{\alpha\beta}\Gamma^{\alpha\beta}
\end{equation}
where
\begin{equation}
C=[a,b].
\end{equation}
The only thing we should check, that's not so easy, is that the
scalar part goes away. We see by adding a scalar part, we get a
central extension of $\mathfrak{so}(n)$. This is a trivial
central extension, we managed to separate it into 2 parts. A
representation of Clifford algebra gives us a representation of
the orthogonal algebra called the \define{Spinor
Representation}. For $\mathfrak{so}(2n)$,
i.e. $\ClassicalGroup{D}_{n}$, we get a reducible representation
--- alternatively, for public relations reasons, we say we get 2
irreducible representations. The spinor representation
corresponds to the Dynkin diagram:
\begin{center}
\includegraphics{img/LieImg.10}
\end{center}

So lets consider all this stuff in the case when the Lie algebra
is $\mathfrak{so}(2n)$. In this case, remember what did we do? We
constructed the Clifford algebra in such a way
\begin{equation}
\eta = \begin{bmatrix} 0 & 1\\ 1& 0
\end{bmatrix},
\end{equation}
and instead of the $\Gamma$ we took $a^{\dagger}$ and $a$ with
the anti-commutation relations
\begin{subequations}
\begin{align}
[a_{i},a_{j}]_{+} = [a^{\dagger}_{i},a^{\dagger}_{j}]_{+} = 0\\
[a_{i}, a^{\dagger}_{j}]_{+} = 2\delta_{ij}
\end{align}
\end{subequations}
Of course this describes the canonical anticommutation relations;
this is of course a Clifford algebra, or more precisely a
particularly case of it.

We have proven in any finite dimensional representation there is
(for Clifford algebras) a vacuum vector $\phi$ such that
\begin{equation}
a_{i}\phi = 0
\end{equation}
for all $a_{i}$. We have a basis by applying the creation
operators to $\phi$ our vacuum:
$a^{\dagger}_{i_{1}}(\cdots)a^{\dagger}_{i_{k}}\phi$ (where
$i_{1}<\cdots<i_{k}$), it follows
\begin{equation}
(a^{\dagger}_{i})^{2}=0
\end{equation}
for all $i$, so we need the indices to be strictly ordered. We
have a spinor representation here, we should consider these guys
$\Gamma^{\mu\nu}$. But really what will we do? We will only write
$\Gamma^{\mu}\Gamma^{\nu}$, we \emph{should} write the constant
part
\begin{equation}
a^{\dagger}_{i}a^{\dagger}_{j}-\eta_{ij}\1
\end{equation}
but 
\begin{equation}
\eta_{ij}=0
\end{equation}
in this instance, so we have terms of the form
\begin{subequations}
\begin{align}
a^{\dagger}_{i}a^{\dagger}_{j}\quad\mbox{and}\quad a_{i}a_{j}\\
\frac{1}{2}(a^{\dagger}_{i}a_{j}-a_{j}a^{\dagger}_{i})=a^{\dagger}_{i}a_{j}-\frac{1}{2}\delta_{ij}
\end{align}
\end{subequations}
These are the generators of the spinor representation acting on
the Fock space; it is reducible. Look at the number of
$a^{\dagger}$'s and $a$'s --- they're always even. The parity is
preserved. The Fock space is the sum of two parts
\begin{equation}
\mathcal{F}=\mathcal{F}_{\text{odd}}\oplus\mathcal{F}_{\text{even}}
\end{equation}
where even/odd refer to the number of $a^{\dagger}$'s. So we have
2 representations, one is called the \define{Left Spinor
  Representation} and the other is surprisingly enough the
\define{Right Spinor Representation}. 

We want to check these representations are irreducible. How to do
this? Take the Cartan subalgebra of this stuff. The Cartan
subalgebra is generated by
\begin{equation}
h_{i}=a^{\dagger}_{i}a_{i}-\frac{1}{2}.
\end{equation}
What we see here is something similar to one of the exercises,
with $\ClassicalGroup{D}_{n}$ we had elements of the form
\begin{equation}
x = \begin{bmatrix}A & B \\ C & D
\end{bmatrix}
\end{equation}
written in block form, where
\begin{subequations}
\begin{align}
B &= -B^{T}\\
C &= -C^{T}\\
D &= -A^{T}
\end{align}
\end{subequations}
describe the $n\times n$ block components. We know the Cartan
subalgebra, we should take the simple roots, although this is
unnecessary. We decompose the algebra into three parts:
\begin{equation}
\mbox{algebra} = (\mbox{positive})\oplus(\mbox{negative})\oplus(\mbox{Cartan}).
\end{equation}
Let us take
\begin{equation}
a^{\dagger}_{i}a^{\dagger}_{j}\sim B
\end{equation}
be the negative part,
\begin{equation}
a_{i}a_{j}\sim C
\end{equation}
for the positive part. We have additionally
\begin{equation}
\alpha_{ij}(a^{\dagger}_{i}a_{j}-\frac{1}{2}\delta_{ij})
\end{equation}
where $\alpha_{ij}$ is the Cartan matrix. There are two places to
make a choice: positive guys form a Lie algebra, and the negative
guys form a Lie algebra too. These choices should be compatible,
how they form Lie algebras should be done compatibly. Now the
highest weight vectors must be determined. We have $\phi$ as the
highest weight vector, but we see that
\begin{equation}
a_{j}(a^{\dagger}_{i}\phi) = 0
\end{equation}
for all $a^{\dagger}_{i}$ and $a_{j}$. But only one of them is
the highest weight vector, really. Since we can decompose
$\mathcal{F}$ into two irreducible representations, we have
\emph{two} highest weight vectors.

What we would like to do now is consider infinite-dimensional
Clifford algebra. This is taught all the time in physics, there
are an infinite number of creation/annihilation operators for
fermions. What to do? We have creation and annihilation operators
$a^{\dagger}_{i}$, $a_{i}$ which obey the canonical
anticommutation relations
\begin{subequations}
\begin{align}
[a_{i},a_{j}]_{+} = [a^{\dagger}_{i}, a^{\dagger}_{j}]_{+} = 0\\
[a_{i}, a^{\dagger}_{j}]_{+} = \delta_{ij}
\end{align}
\end{subequations}
We may try to consider infinite dimensional representations of
this algebra, constructed in exactly the same way. Consider a
vector $\phi$ such that
\begin{equation}
a_{i}\phi = 0
\end{equation}
for all $a_{i}$. We may consider
\begin{equation}
a^{\dagger}_{i_{1}}(\cdots)a^{\dagger}_{i_{k}}\phi
\end{equation}
where $i_{1}<\cdots<i_{k}$. We can act by means of annihilation
operators $a_{i}$ on this stuff. Something new happens, namely:
\begin{itemize}
\item First of all, we do not know if this vector $\phi$ exists
  at all. Maybe it doesn't exist!
\item Second we may introduce $a^{\dagger}_{i}$, $a_{i}$ in many
  ways. These operators are on completely equal footing, and we
  may interchange them! Or parts of them! Why not? If we do this
  in finite dimensions, we have a vector annihilated by all
  $a^{\dagger}$'s, and we have a vector annihilated by all
  $a$'s. They the operators are on completely equal
  footing. Physicists are brave, they apply an infinite number of
  $a^{\dagger}$'s. Mathematicians are not so brave.
\end{itemize}
This Fock space may be definite in the infinite-dimensional case.

We would like to consider analogous generators for an
infinite-dimensional spinor representation. We have some
problems. We consider
\begin{equation}
A = \sum_{i,j} \alpha_{ij}a^{\dagger}_{i}a^{\dagger}_{j}+\beta\cdot\1
\end{equation}
we consider operators of this form (we can do it for finite
dimensions). Then this is a representation of $\GL{n}$, it is
sitting inside of $\SO{n}$. Well $\mathfrak{gl}(n)$ to be
precise. We may consider something like this here. We get a
nontrivial central extension for infinite-dimensional
representations. We will describe an infinite-dimensional general
linear Lie Algebra $\mathfrak{gl}(\infty)$. We cannot consider
them all, but we will restrict our focus. These guys appear in
physics all the time during quantization. We have projective
representations as our tool in quantum theory.
