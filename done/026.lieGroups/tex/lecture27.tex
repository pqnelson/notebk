%%
%% lecture27.tex
%% 
%% Made by Alex Nelson
%% Login   <alex@tomato3>
%% 
%% Started on  Mon Dec 13 18:25:25 2010 Alex Nelson
%% Last update Mon Dec 13 21:17:49 2010 Alex Nelson
%%
We will discuss some stuff that generalizes what we
considered. Last time we considered infinite dimensional Clifford
algebras with generators $a^{\dagger}_{k}$, $a_{k}$ such that
\begin{subequations}
\begin{align}
[a_{i},a_{j}]_{+} = [a^{\dagger}_{i}, a^{\dagger}_{j}]_{+} = 0\\
[a_{i}, a^{\dagger}_{j}]_{+} = \delta_{ij}
\end{align}
\end{subequations}
We can construct a representation in analogy to the finite
dimensional case: take a vector $\phi$ such that
\begin{equation}
a_{i}\phi = 0
\end{equation}
for all $a_{i}$. We consider
\begin{equation}
a^{\dagger}_{k_{1}}(\cdots)a^{\dagger}_{k_{n}}\phi
\end{equation}
where $k_{1}<\cdots<k_{n}$. This gives us a space called the
\define{Fock Space} and the vector $\phi$ is called the
\define{Vaccum Vector}. We may swap
\begin{equation}
a^{\dagger}\iff a
\end{equation}
we get another Fock space; we may exchange only parts of the
operator, and we still get another Fock space. Maybe these Fock
spaces are equivalent, maybe not.

Consider terms of the form $\alpha_{kl}a^{\dagger}_{k}a_{l}$ with
Euclidean summation convention\footnote{If two indices are
  repeated, either upstairs or downstairs, sum over it. So
  e.g. $\alpha_{kl}a_{k}a_{l}=\sum_{k,l}\alpha_{kl}a_{k}a_{l}$. The
  term ``Euclidean summation convention'' is due to Misner,
  Thorne and Wheeler's \emph{Gravitation} Chapter 12.3:
  ``(`Euclidean' index convention: repeated space indices to be
  summed even if both are down; dot denotes time derivative.)''.}. We consider
\begin{subequations}
\begin{align}
[\alpha_{kl}a^{\dagger}_{k}a_{l}, \beta_{rs}a^{\dagger}_{r}a_{s}]
&= \alpha_{kl}a^{\dagger}_{k}a_{l}\beta_{rs}a^{\dagger}_{r}a_{s}
- \beta_{rs}a^{\dagger}_{r}a_{s}\alpha_{kl}a^{\dagger}_{k}a_{l}\\
&= \alpha_{kl}\beta_{rs}a^{\dagger}_{k}a_{l}a^{\dagger}_{r}a_{s}
- a^{\dagger}_{r}a_{s}a^{\dagger}_{k}a_{l}
\end{align}
\end{subequations}
Well, we deduce
\begin{equation}
a_{l}a^{\dagger}_{r} = -a^{\dagger}_{r}a_{l}+\delta_{rl}.
\end{equation}
We thus get something of the form
\begin{equation}
[\alpha_{kl}a^{\dagger}_{k}a_{l}, \beta_{rs}a^{\dagger}_{r}a_{s}]
\sim \gamma_{mn}a^{\dagger}_{m}a_{n}+\tr[\alpha,\beta]\cdot\1,
\end{equation}
the trace of hte commutator is nonzero for infinite dimensional
matrices $\alpha$, $\beta$.

What we would like to do is move from formal considerations to
reality. When working with infinite sums, it is not terribly
clear. But with representations, we may ask questions regarding
the sum
\begin{equation}
\left(\sum_{kl}\alpha_{kl}a^{\dagger}_{k}a_{l}\right)(a^{\dagger}_{i_{1}}\cdots
a^{\dagger}_{i_{s}}\phi)
\end{equation}
we may questions. If the sum is infinite, then sometimes it is
still well-defined. Suppose for every $l$, there are finitely
many $k$'s which have nonzero components
\begin{equation}
\alpha_{kl}\not=0
\end{equation}
(if $k,l\in\ZZ$, we could say $\alpha_{kl}$ if $|k-l|<N$ for some
fixed $N$). We have finite number of indices; therefore if $l$ is
sufficiently large, then we can obtain a vanishing
answer. Thererfore only finitely many cases are possible, but the
expression is well-defined! So everything is fine! But such
operators appear in such matrices.

In particular with the Virasoro algebra, we have
\begin{equation}
\ell_{k} = -z^{k+1}\frac{d}{dz}
\end{equation}
If we take the basis $z^{n}$, the derivative shifts the index
\begin{equation}
\frac{d}{dz}\colon z^{k}\mapsto kz^{k-1},
\end{equation}
multiplication by $z$ shifts the index too:
\begin{equation}
z\colon z^{k}\mapsto z^{k+1}.
\end{equation}
We have a nontrivial central extension of some algebra
$\mathfrak{gl}(\infty)$ the Lie algebra of \emph{good} infinite
dimensional matrices (but not \emph{all} infinite dimensional
matrices). We can embed the Witt algebra into $\mathfrak{gl}_{\infty}$.

\begin{rmk}
The infinite dimensional Clifford algebra appears to be the limit
or colimit of all finite dimensional Clifford algebras; Schwarz
says this is not so for the infinite dimensional case; there
appears to be cannonical embeddings
\begin{equation}
\CC^{n}\hookrightarrow\CC^{n+1}\hookrightarrow\CC^{n+2}\hookrightarrow\cdots
\end{equation}
which is the limit of this sort of represntation. There are some
side-effects with this, especially regarding the central
extension --- it becomes trivial in this limit!
\end{rmk}

We would like to describe a very interesting situation when we
get an affine algebra, a situation of the Kac-Moody algebra. Take
any Lie Algebra $\mathscr{G}$. Now take all Laurent series $\sum
a_{n}z^{n}$ where $a_{n}\in\mathscr{G}$ and $n\in\ZZ$, but we
restrict ourselves specifically to Laurent \emph{polynomials} so
this sum is \emph{finite}. We can obtain a Lie algebra, defining
the Lie bracket as
\begin{equation}
[az^{m},bz^{n}] = [a,b]z^{m+n},
\end{equation}
and by demanding linearity, distributivity, etc. This is not the
most interesting one; the most important thing here is the
central extension. One of the ways is by means of matrices and
infinite matrices, then perform the central extension.

We will merely write the answer. (This is related to the answer
to one of the problems on the final!) We will define a \emph{new}
Lie Algebra denoted $\widehat{\mathscr{G}}$ in the following way:
take the central extension of the algebra in the following
manner. Generators are of the form $az^{n}$, the central
extension $c$. We assume
\begin{equation}
[az^{n},c] = 0
\end{equation}
We don't really have a choice! Define the new commutator by
\begin{equation}
[az^{m},bz^{n}]_{\text{new}}=[a,b]_{\text{old}}z^{m+n}+\underbracket[0.5pt]{m\<a,b\>\delta_{m+n,0}c}_{\mathclap{\text{central term with coefficient}}}
\end{equation}
This is the answer, we get a central extension this way. Well, do
we? Prove it! Check it is a Lie Algebra, check the Jacobi
identity:
\begin{subequations}
\begin{align}
[[az^{m},bz^{n}]_{\text{new}},rz^{s}]_{\text{new}} &=[[a,b]_{\text{old}}z^{m+n},rz^{s}]_{\text{new}}\\
&=[[a,b]_{\text{old}},r]_{\text{old}}z^{m+n+s} +
(m+n)\<[a,b]_{\text{old}},r\>\delta_{m+n+s,0}c
\end{align}
\end{subequations}
By taking cyclic permutations, the first term vanishes
identically (we borrow the Jacobi identity from the old Lie
bracket). What about the second term? Well, we use the invariant
inner product, i.e. for a representation $\varphi$ we have
\begin{equation}
\<y,\varphi(z)\>+\<\varphi(y),z\> = 0;
\end{equation}
the adjoint representation has
\begin{equation}
\<[x,y],z\>+\<y,[x,z]\> = 0
\end{equation}
So really the cyclic permutations of $\<[a,b],r\>$ are equal up
to a sign (or more closely examined, we see it is identical). We
can repeat more or less everything from finite-dimensional Lie
algebras.

\begin{rmk}
This is useful in physics, if symmetry coincides with Lie
algebra, we need projective representations, thus we need central
extensions; the cohomology of Lie algebras gives us the central
extension being unique.
\end{rmk}
