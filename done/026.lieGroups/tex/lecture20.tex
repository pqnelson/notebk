%%
%% lecture20.tex
%% 
%% Made by Alex Nelson
%% Login   <alex@tomato3>
%% 
%% Started on  Sat Dec 11 12:35:11 2010 Alex Nelson
%% Last update Mon Dec 13 21:25:51 2010 Alex Nelson
%%
We formulated a theorem of the structure of semisimple algebras,
then considered examples. Lets go back. Recall we considered the
situation when we had $e_{\alpha}$, $f_{\alpha}$, $h_{\alpha}$
(members of the Lie Algebra) with the relations that
\begin{subequations}
\begin{align}
[h_{\alpha}, h_{\beta}] &= 0\\
[e_{\alpha}, f_{\beta}] &= h_{\alpha}\delta_{\alpha\beta}\\
[h_{\alpha}, e_{\beta}] &= a_{\alpha\beta}e_{\beta}\\
[h_{\alpha}, f_{\beta}] &=-a_{\alpha\beta}f_{\beta}.
\end{align}
\end{subequations}
We see that $e_{\alpha}$, $f_{\alpha}$ are root vectors, so
\begin{equation}
[h, e_{\alpha}] = \lambda_{\alpha}(h)e_{\alpha}
\end{equation}
and similarly
\begin{equation}
[h, f_{\beta}] = -\lambda_{\beta}(h)f_{\beta}.
\end{equation}
We recall that a mapping
\begin{equation}
\lambda\colon\mathscr{H}\to\Bbb{F}
\end{equation}
is called a \define{Root}, it's a linear functional on
$\mathscr{H}$. We see that
\begin{equation}
\lambda_{\beta}(h_{\alpha})=a_{\alpha\beta}.
\end{equation}
Moreover $a_{\alpha\beta}$ should be the Cartan matrix, so 
\begin{subequations}
\begin{align}
a_{\alpha\alpha}&=2\\
a_{\alpha\beta}&\leq0\quad\mbox{for }\alpha\not=\beta\\
a_{\alpha\beta}&\mbox{ is symmetrizable}
\end{align}
\end{subequations}
We also assume that
\begin{equation}
\det(a_{\alpha\beta})\not=0
\end{equation}
i.e. the Cartan matrix is nondegenerate. We can find the Cartan
matrix for semisimple Lie Algebras. An ideal corresponds to an
invaraint subspace of the Lie algebra under the adjoint
representation. We know a simple Lie Algebra is simple iff it has
only trivial ideals. For semisimple Lie Algebras, we can
partition roots into positive and negative roots. Positive roots
contain a subset that generates all roots, we call this subset
\define{Simple Roots}.

What may be said of representations with this data? We take
$\mathscr{G}$ a Lie algebra, we take a representation
\begin{equation}
\varphi\colon\mathscr{G}\to\mathfrak{gl}(V)
\end{equation}
for some vector space $V$, and we may consider the weights of
this Lie Algebra
\begin{equation}
\varphi(h)\vec{v}=\alpha(h)\vec{v}
\end{equation}
and weight vectors $\vec{v}$ (where we take
$h\in\mathscr{H}$). The root vectors act on weight vectors,
namely $\varphi(e_{k})\vec{v}$ is a weight vector (supposing that
$\vec{v}$ was initially a weight vector) with weight
$\alpha+\lambda_{k}$ provided that it is nonzero. Similarly
$\varphi(f_{j})\vec{v}$ is a weight vector with weight
$\alpha-\lambda_{j}$, so $\varphi(f_{j})$ lowers the weights. The
highest weight vector is annhilated by applying
$\varphi(e_{i})\vec{v}=0$ for all $i$. The highest weight vector
always exists in in finite dimensional representations, although
this is not necessarily true for infinite dimensional
representations. 

\begin{thm} {\rm(The highest weight vector exists in finite
    dimensional representations.)} If a finite dimensional
  representation is reducible, then the highest weight vector is
  not unique.
\end{thm}

Why? Well, at least one exists in the finite dimensional
case. Why? Trivially, because in linear algebra the eigenvalue
problem has no solution if the matrix is all zeroes. We cannot
have that for a nontrivial representation. 

Now suppose there exists a representation that is a
subrepresentation which will be irreducible and contains a
different highest weight vector. Let us suppose we have highest
weight vector, then we may construct a subrepresentation
consisting of
$f_{\alpha_{1}}(\cdots{})f_{\alpha_{n}}\vec{v}$\marginpar{This
may appear to be mathematically incorrect, but the weights are
integers which means at some moment these would vanish.} which
is a subrepresentation --- it is highest weight since
\begin{equation}
e_{\beta}\vec{v}=0
\end{equation}
for all $\beta$. Suppose our representation is reducible. If this
is so, there is a highest weight vector in the subrepresentation.

\begin{rmk}
To prove a representation is irreducible, it is sufficient to
prove the uniqueness of the highest weight vector.
\end{rmk}

How to classify, to describe representations (especially
irreducible representations). This is a simple thing, namely take
this highest weight vector
\begin{equation}
\varphi(h)\vec{v}=\alpha(h)\vec{v}
\end{equation}
where $\alpha\in\mathscr{H}^{*}$, and we should calculate
$\alpha(h_{1})$, \dots, $\alpha(h_{n})$ for all basis elements of
the Cartan subalgebra. We will prove that $\alpha(h_{i})\geq0$
and $\alpha(h)\in\ZZ$. To prove this is extremely simple, because
-- look -- we have these commutation relations
\begin{equation}
\Span\{h_{i},e_{i},f_{i}\}\iso\mathfrak{sl}(2)
\end{equation}
for fixed $i$. For this algebra $\mathfrak{sl}(2)$ we
know \emph{everything}, in particular all finite dimensional
irreducible representations, which is \emph{precisely} the guys
we are interested in. So the representation is characterized by
$n$ non-negative numbers. So can we take thse numbers in any way
we want? Yes we can, we'll prove it in the next lecture. We will
merely check this for $\ClassicalGroup{A}_{n}$. This is
interesting by itself. We will later check this for
$\ClassicalGroup{B}_{n}$, $\ClassicalGroup{C}_{n}$; the proof
will be constructive.

For $\ClassicalGroup{A}_{n}$, we have $e_{i}=E_{i,i+1}$,
$f_{i}=E_{i+1,i}$. We see
\begin{equation}
h_{i}=E_{i,i}-E_{i+1,i+1}
\end{equation}
What to do? Well, we have first of all $n$ numbers
$\alpha(h_{1})$, \dots, $\alpha(h_{n})$. We will prove these
numbers may be taken by considering the standard basis in
$\RR^{n}$. 

We will call the name for these
representations \define{Elementary Representations}. First it is
sufficient to find elementary representations, they represent
$\mathscr{G}$ in spaces $V_{1}$, \dots, $V_{n}$. We will take
$V^{\otimes m_{1}}_{1}\otimes\cdots\otimes V^{\otimes
m_{n}}_{n}$, and the highest weights $\alpha_{1}$, \dots,
$\alpha_{n}$ with the corresponding highest weight vectors
$\vec{v}_{1}$, \dots, $\vec{v}_{n}$, then the corresponding
weight vectors in $V^{\otimes m_{1}}_{1}\otimes\cdots\otimes V^{\otimes
m_{n}}_{n}$ have weights
$m_{1}\alpha_{1}+\cdots+m_{n}\alpha_{n}$. If we analyze these
weights, we may consider any representation constructed from the
elementary representations.

What to do? Construct the elementary representation, which is
very easy\dots we take the fundamental representation. If
$(\varphi_{1},\dots,\varphi_{n})$ are the coordinates of the
Cartan subalgebra (bear in mind because we work with
$\ClassicalGroup{A}_{n}$ we have
$\varphi_{1}+\dots+\varphi_{n}=0$ and we work with diagonal
matrices), then the weights are simply $\varphi_{1}$, \dots,
$\varphi_{n}$. The highest weight correspond to $\varphi_{1}$. We
see
\begin{equation}
\alpha_{1}(h_{k}) = \begin{cases}1 & k=1\\
0 & \text{otherwise}
\end{cases}
\end{equation}
We would like to now note this corresponds to $(1,0,\dots,0)$.

We want to consider $(0,1,0,\dots,0)$. This is constructed by
considering $\Antisymmetric^{2}(V)$, the antisymmetric part of
$V\otimes V$. The highest weight vector is $v_{1}\otimes
v_{2}-v_{2}\otimes v_{1}$, and the corresponding weight is
$\alpha_{1}+\alpha_{2}$. We see
\begin{equation}
(\alpha_{1}+\alpha_{2})(h_{k})=\begin{cases}0 & k\not=2\\
1 & k=2
\end{cases}
\end{equation}
This corresponds to the desired $(0,1,0,\dots,0)$.

The general case we have the highest weight be
$\alpha_{1}+\dots+\alpha_{k}$, which corresponds to the
representation $\Antisymmetric^{k}(V)$ --- the antismmetric part
of $V^{\otimes k}$. The highest weight vector is then $\vec{v}_{[1}\otimes\vec{v}_{2}\otimes\cdots\otimes\vec{v}_{k]}$.
