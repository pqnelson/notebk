%%
%% oneParticleState.tex
%% 
%% Made by Alex Nelson
%% Login   <alex@tomato>
%% 
%% Started on  Tue Jul 21 12:54:12 2009 Alex Nelson
%% Last update Tue Jul 21 12:54:12 2009 Alex Nelson
%%

The simplest system to consider is a single particle. The
function space used to model quantum-mechanical states is a
Hilbert Space $\mathcal{H}$ of square integrable functions on the
physical space (denoted by $\mathcal{C}$):
\begin{equation}%\label{eq:}
L^{2}(\mathcal{C}) = \left\{f\,:\;\int_{\mathcal{C}}|f(\overline{x})|^{2}d^{3}\overline{x}<\infty\right\}
\end{equation}
Note that in all fairness, $\mathcal{H}$ can be written in either
position coordinates $\overline{x}$ or momentum coordinates
$\overline{p}$. The relationship between the position-space and
momentum-space is precisely the familiar Fourier transform:
\begin{equation}%\label{eq:}
\mathcal{F}(f)(\overline{p}) \stackrel{\text{def}}{=} \int e^{i\overline{x}\cdot\overline{p}}f(\overline{x})d^{3}\overline{x}
\end{equation}
Despite the change of variables, $\mathcal{F}$ sends
$\mathcal{H}$ to itself, so both $f$ and its Fourier transform
$\mathcal{F}(f)$ are in $\mathcal{H}$.
\begin{rmk}
It should be emphasized that if $f$ is square-integrable, then
$e^{i\overline{x}\cdot\overline{p}}f(\overline{x})$ is
square-integrable \emph{but not necessarily integrable!} That is,
we have no guarantee that
$e^{i\overline{x}\cdot\overline{p}}f(\overline{x})\in L^{1}(\mathcal{C})$.

To define the Fourier transform on $\mathcal{H}$, we should first
define it on some suitably nice subspace of $\mathcal{H}$
(e.g. the space of smooth functions with ``compact support'' ---
i.e. they are zero outside of a compact subset of their
domain). Then we observe that the Fourier transform is an
isometry (up to some scale factor) on our nice subspace, so we
extend this isometry from our nice subspace to all of $\mathcal{H}$.
\end{rmk}

We represent the observables by operators. More relevantly, the
position operators $\widehat{x}_{m}$ and momentum operators
$\widehat{p}_{m}$ are represented in position-space by
multiplication by the coordinate fuhnctions $x_{m}$ and the
partial derivative operators $-i\partial_{m}$
(respectively). Observe also that the Fourier transform converts
multiplication by $x_{m}$ on functions of $\overline{x}$ into the
differential operators $-i\partial_{m}$ on functions of
$\overline{p}$:
\begin{equation}%\label{eq:}
\mathcal{F}(x_{m}f)(\overline{p})=-i\partial_{m}\mathcal{F}(f)(\overline{p}).
\end{equation}

The natural question to ask is ``What are the eigenstates of
these operators?'' Well, in position space, we find the position
eigenstates are just delta functions
\begin{subequations}
\begin{align}
(\widehat{x}_{m}\delta_{\overline{q}})(\overline{x}) &=
  \widehat{x}_{m}\delta(\overline{x}-\overline{q})\\
&= q_{m}\delta(\overline{x}-\overline{q})\\
&= (q_{m}\delta_{\overline{q}})(\overline{x})
\end{align}
\end{subequations}
Similarly, for the eigenstates of the momentum operators
$\widehat{p}_{m}$, we see that the eigenstates in position-space
are $e_{\overline{p}}(\overline{x})$:
\begin{subequations}
\begin{align}
(\widehat{p}_{m}e_{\overline{p}})(\overline{x}) &= -i\partial_{m}\exp(i\overline{p}\cdot\overline{x})\\
&= p_{m}\exp(i\overline{p}\cdot\overline{x})\\
&= (p_{m}e_{\overline{p}})(\overline{x}).
\end{align}
\end{subequations}

But we have just two minor problems: \begin{inparaenum}
\item neither $\widehat{x}_{m}$ nor $\widehat{p}_{n}$ act on all
  of $\mathcal{H}$, and
\item $\mathcal{H}$ doesn't contain the eigenstates of either
  operators.
\end{inparaenum}
We can solve the first problem fairly easily --- we'll consider
the subspace $S\subset{\mathcal{H}}$ where the operators map $S$
to itself. Similarly, we resolve the second problem by defining
the kets as elements of $S^{*}$, the space of continuous
antilinear functionals on $S$. Since $\widehat{p}_{n}$ acts on
all functions of $S$, these functions must be infinitely
differentiable, and so $S^{*}$ will contain the
$\delta$-functions and all their derivatives. Similarly, by
taking the Fourier Transform, since $\widehat{x}_{m}$ acts on
$S$, it follows that $S^{*}$ will contain exponential functions
$\exp(i\overline{p}\cdot\overline{x})$. 

Instead of a single Hilbert space, we end up with a triple
\begin{equation}%\label{eq:}
S\subset{\mathcal{H}}\subset{S^{*}}
\end{equation}
The physical states live in $S$, and the operator eigenstates
live in $S^{*}$. With appropriate demands on the space $S$, 
this triple ends up being a \emph{Rigged Hilbert Space}~\cite{delamadrid}~\cite{Madrid:2004zy}. 
In this context ``Rigged'' \emph{is  not} in the sense of ``This
game is rigged'' but rather in the sense of ``equipped'' --- like
how a boat is ``rigged'' or ``equipped to sail''. \linebreak


\begin{ddanger}
In fact, the triple $S\subset\mathcal{H}\subset{S^{*}}$ is a
rigged Hilbert space if $S$ is a nuclear subspace of
$\mathcal{H}$. See Gelfand~\cite{gelfandgeneralized} or
Maurin~\cite{maurin} for rigorous details about the notion of
nuclear spaces. We'll discuss one such criteria for $S$ to be
nuclear. Specifically,
\begin{enumerate}
\item there exists a countable family $\|\cdot\|_{k}$ of norms on
$S$ with respect to which convergence is defined
by 
\begin{equation}
f_{n}\to{f}\quad\iff\quad\|f_{n}-f\|_{k}\to0\;\;\forall k\geq0;
\end{equation}
\item $S$ is complete with respect to this notion of
  convergence; and 
\item there exists a Hilbert-Schmidt operator on
  $S$ with a continuous inverse.
\end{enumerate}
We'll leave the interested reader to refer to the cited sources.
\end{ddanger}

In a rigged Hilbert Space we have eigenfunction expansions. More
precisely, consider a state $|f\>$ represented by the function
$f$, let $|\overline{x}\>$ be the position eigenstate represented
by the distribution $\delta_{\overline{x}}$. We assume the
relationship between the functions and the kets is such that
\begin{equation}%\label{eq:}
f(\overline{x}) = \<\overline{x}|f\>.
\end{equation}
We can then expand the state $|f\>$ in terms of the position
eigenstate $|\overline{x}\>$ which should be of the form
\begin{equation}%\label{eq:}
|f\> = \int |\overline{x}\>\,\<\overline{x}|f\>\,d^{3}\overline{x} = \int f(\overline{x})\,|\overline{x}\>\,d^{3}\overline{x}.
\end{equation}
The conditions on $S$ in a rigged Hilbert Space ensure that
$f(\overline{x})|\overline{x}\>$ is integrable for all $f\in{S}$.
