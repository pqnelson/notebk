%%
%% actionPoincareGroup.tex
%% 
%% Made by Alex Nelson
%% Login   <alex@tomato>
%% 
%% Started on  Sat Jul 25 14:01:05 2009 Alex Nelson
%% Last update Sat Jul 25 14:01:05 2009 Alex Nelson
%%

We really want to find a unitary representation of the Poincar\'e
group, which is the Lorentz group plus spacetime translations
(i.e. rotations, Lorentz boosts, and space-time translations). We
have the representation condition $U(gh)=U(g)U(h)$ must hold for
all $g,h$ in the Poincar\'e group. We've seen what happens when
both $g,h$ are in the Lorentz group, and when both $g,h$ are
space-time translations. We now need to ask: what happens when
one is a translation and the other is a boost?

We can uniquely factor any element $g$ of the Poincar\'e group as
the product
\begin{equation}%\label{eq:}
g = \Delta_{a}\Lambda
\end{equation}
where $\Lambda$ is in the Lorentz group, and $\Delta_a$ is a
translation. Multiplication in the Poincar\'e group depends on
multiplication in the Lorentz group and addition of translations
through an interchange in the order of the two facts:
\begin{subequations}
\begin{align}
gh &= \Delta_{a}\Lambda\Delta_{b}M\\
&= \Delta_{a}(\Lambda\Delta_{b}\Lambda^{-1})\Lambda M\\
&= \Delta_{a}\Delta_{\Lambda b}\Lambda M
\end{align}
\end{subequations}
where we have used the identity
\begin{equation}%\label{eq:}
\Lambda\Delta_{b}\Lambda^{-1} = \Delta_{\Lambda b}
\end{equation}
a relation trivially verified when we act on a 4-vector $x$.

Our definition of $U$ so far covers translations and Lorentz
group elements only; when we extend to the Poincar\'e group, we
do so through the definition
\begin{equation}%\label{eq:}
U(\Delta_{a}\Lambda) \stackrel{\text{def}}{=} U(\Delta_{a})U(\Lambda)
\end{equation}
We can now see that $U$ is a representation of the Poincar\'e
group if and only if $U$ preserves the action
$\Lambda\Delta_{b}\Lambda^{-1}=\Delta_{\Lambda b}$ of Lorentz
group elements on translations:
\begin{subequations}
\begin{align}
 & U(\Delta_{a}\Lambda)U(\Delta_{b}M) =
  U(\Delta_{a}\Delta_{\Lambda b}\Lambda M) \\
\iff & U(\Delta_{a})U(\Lambda)U(\Delta_{b})U(M) =
U(\Delta_{a})U(\Delta_{\Lambda b})U(\Lambda)U(M)\\
\iff & U(\Lambda)U(\Delta_{b}) = U(\Delta_{\Lambda b})U(\Lambda)\\
\iff & U(\Lambda)U(\Delta_{b})U(\Lambda)^{\dag} = U(\Delta_{\Lambda b})
\end{align}
\end{subequations}
We verify the final condition by evaluating both sides on some
test state $|k\>$. From the right hand side, we have
\begin{equation}%\label{eq:}
U(\Delta_{\Lambda b})|k\> = \exp(i\Lambda b^{\mu}k_{\mu})|k\>
\end{equation}
and from the left hand side
\begin{subequations}
\begin{align}
U(\Lambda)U(\Delta_{b})U(\Lambda)^{\dag}|k\> &=
U(\Lambda)U(\Delta_{b})|\Lambda^{-1}k\>\\
&=
U(\Lambda)\exp(ib^{\mu}{\Lambda_{\mu}}^{\nu}k_{\nu})|\Lambda^{-1} k\>\\
&=\exp(ib^{\mu}{\Lambda_{\mu}}^{\nu}k_{\nu})|k\>.
\end{align}
\end{subequations}
The equality of the two sides follows from the Lorentz-invariance
of the inner product.

We can now summarize our results of $U$ in the following theorem:
\begin{thm}%\label{thm:}
The map $U$ from the Poincar\'e group to operators on the state
space defined by
\begin{subequations}
\begin{align}
U(\Delta_{a})|k\> &= e^{ia^{\mu}k_{\mu}}|k\>\\
U(\Lambda)|k\> &= |\Lambda k\>\\
U(\Delta_{a}\Lambda) &= U(\Delta_{a})U(\Lambda) 
\end{align}
\end{subequations}
is a unitary representation of the Poincar\'e group.
\end{thm}

The unitary representation $U$ is often boasted to successfully
combines the principle (as represented by the Poincar\'e group)
with the principles of quantum mechanics (as represented by
unitary operators and state-space formalisms). This combined
structure of a one-particle state space provides the foundation
for the many-particle state space used in all quantum field theories.
