%%
%% oneParticleSystemPhysics.tex
%% 
%% Made by Alex Nelson
%% Login   <alex@tomato>
%% 
%% Started on  Tue Jul 21 15:55:44 2009 Alex Nelson
%% Last update Tue Jul 21 15:55:44 2009 Alex Nelson
%%

We're interested in a toy model of relativistic quantum
mechanics, so we begin with a single particle. All we really
need, truth be told, is a state space plus a Hamiltonian
operator. We should remember, from Special Relativity, the
energy-momentum four-vector $\widehat{p}_{\mu}$ has as its time component the
Hamiltonian $\widehat{p}_{0}=\widehat{H}$. For convenience, we'll
work in the momentum space with the momentum operator eigenbasis
\begin{equation}%\label{eq:}
\widehat{p}_{m}|\overline{k}\>=k_{m}|\overline{k}\>
\end{equation}
We assume the states are normalized thus
\begin{equation}%\label{eq:}
\<\overline{k}|\overline{k}'\>=\delta^{(3)}(\overline{k}-\overline{k}').
\end{equation}
This means that the length of a ket is undefined. It is,
nonetheless, a normalization suitable for integration over
momentum. As an added bonus, we also get the resolution of the
identity
\begin{equation}%\label{eq:}
\mathbf{1}=\int\,|\overline{k}\>\,\<\overline{k}|\,d^{3}\overline{k}
\end{equation}

Since energy-momentum is a four-vector, we demand that
\begin{equation}%\label{eq:}
\widehat{p}^{\mu}\widehat{p}_{\mu} = \widehat{H}^{2}-|\widehat{p}_{m}\widehat{p}^{m}|
\end{equation}
needs to be constant on the orbits of the Poincar\'e
group. Further if $|\overline{k}\>$ and $|\overline{k}\,'\>$ are
two states of a single particle, then there exists a Lorentz
boost from one to the other (up to scale). Hence we assume the
existence of a scalar quantity $\mu$ (the particle mass) which
satisfies
\begin{equation}%\label{eq:}
(\widehat{H}^{2} - \widehat{p}_{m}\widehat{p}^{m})|\overline{k}\>
= \mu^{2}|\overline{k}\>
\end{equation}
This implies that the Hamiltonian operator $\widehat{H}$ is
diagonal in the momentum eigenbasis (i.e. the basis of
eigenstates of the momentum operator):
\begin{equation}%\label{eq:}
\widehat{H}|\overline{k}\> = \left(\|\overline{k}\|^{2}+\mu^{2}\right)^{1/2}|\overline{k}\>
\end{equation}
The eigenvalues of the Hamiltonian operator come up enough times
that we introduce the shorthand for it:
\begin{equation}%\label{eq:}
\omega(\overline{k}) \stackrel{\text{def}}{=} \left(\|\overline{k}\|^{2}+\mu^{2}\right)^{1/2}
\end{equation}
(This should be vaguely reminiscent of the de Broglie relations
$E=\hbar\omega$.)

\begin{rmk}
Observe that this entire scheme we've devised is equivalent to
taking the limit of the state space for a cube of side $L$ under
periodic boundary conditions, i.e. the particle in a box
situation. In such a cube, we should recall the spectrum of the
momentum operator is discrete and the normalization is given by
the Kronecker delta:
\begin{equation}%\label{eq:}
\overline{k}=\frac{2\pi}{L}(n_x,n_y,n_z),\quad\text{and}\quad\<\overline{k}|\overline{k}\,'\>=\delta_{\,\overline{k}\, ,\,\overline{k}\,'}
\end{equation}
This observation is taken advantage of when deriving the
differential transition probability per unit time for particle
scattering.
\end{rmk}
