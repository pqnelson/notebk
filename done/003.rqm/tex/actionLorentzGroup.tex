%%
%% actionLorentzGroup.tex
%% 
%% Made by Alex Nelson
%% Login   <alex@tomato>
%% 
%% Started on  Wed Jul 22 13:01:36 2009 Alex Nelson
%% Last update Wed Jul 22 13:01:36 2009 Alex Nelson
%%

The space of particle states is three dimensional. The energy
$k_0$ of a particle with momentum $\overline{k}$ is constrained
by
\begin{equation}%\label{eq:}
k_{0}\geq0
\end{equation}
and
\begin{equation}%\label{eq:}
k^{2} = k_{\mu}k^{\mu} = \mu^{2}.
\end{equation}
Therefore the possible energy-momentum vectors lie on a
hyperbolic sheet in $k$-space, the mass hyperboloid. We need an
integration measure on this hyperboloid if we want to do Lorentz
invariant computations.

Let $\theta(t)$ be the Heaviside step function
\begin{equation}%\label{eq:}
\theta(t) = \begin{cases} 0 &\text{if }t<0\\
1 & \text{if }t>0
\end{cases}
\end{equation}
Define an integration $d\lambda(k)$ on the positive hyperboloid
as follows:
\begin{equation}%\label{eq:}
d\lambda(k) \stackrel{\text{def}}{=} d^{4}k \delta(k^{2}-\mu^{2})\theta(k_{0})
\end{equation}
The Lebesgue measure $d^{4}k$ is Lorentz invariant due to the
Lorentz transformation having unit determinant. Here since
$k^{2}-\mu^{2}$ is Lorentz invariant, the $\delta$ function is
Lorentz invariant. Similar reasoning holds for $\theta(k_{0})$
being Lorentz invariant.

We can take advantage of the identity
\begin{equation}%\label{eq:}
\delta(f(k)) = \sum_{\{k:f(k)=0\}}\frac{1}{\|f'(k)\|}\delta(k)
\end{equation}
and  the fact that
\begin{subequations}
\begin{align}
k^{2}-\mu^{2} &= (k_{0}^{2}-\|\overline{k}\|^{2})-\mu^{2}\\
&= k_{0}^{2} - (\|\overline{k}\|^{2} + \mu^{2}) \\
&= k_{0}^{2} - \omega(\overline{k})^{2} \\
&= (k_{0} - \omega(\overline{k}))(k_{0} + \omega(\overline{k}))
\end{align}
\end{subequations}
to deduce that
\begin{subequations}
\begin{align}
\delta(k^{2}-\mu^{2})\theta(k_{0}) &= \delta\left((k_{0} - \omega(\overline{k}))(k_{0} + \omega(\overline{k}))\right)\theta(k_0)\\
&=\frac{1}{2\omega(\overline{k})}(\delta(k_0-\omega(\overline{k}))\theta(k_0)+\delta(k_0+\omega(\overline{k}))\theta(k_0))\\
&=\frac{1}{2\omega(\overline{k})}\delta(k_0-\omega(\overline{k}))\theta(k_0)
\end{align}
\end{subequations}
since $\delta(k_0+\omega(\overline{k}))$ requires $k_0<0$ which
then demands that $\theta(k_0)=0$, so that term drops out completely.
Observe that this means we can effectively eliminate $k_0$ from
any integral with respect to $\omega(\overline{k})$ as follows:
\begin{subequations}
\begin{align}
\int f(k)d\lambda(k) &= \int f(k)\left(\frac{\delta(k_{0}-\omega(\overline{k}))}{2\omega(\overline{k})}\theta(k_{0})d^{3}\overline{k}dk_{0}\right)\\
&= \int f\left(\omega(\overline{k}),\overline{k}\right)\frac{d^{3}\overline{k}}{2\omega(\overline{k})}
\end{align}
\end{subequations}
This integral and the arbitrary function $f$ are commonly
eliminated from this result, leaving an equality of measures
\begin{equation}%\label{eq:}
d\lambda(k) = \frac{d^{3}\overline{k}}{2\omega(\overline{k})}
\end{equation}
and
\begin{equation}%\label{eq:}
k_{0} = \omega(\overline{k}).
\end{equation}

\begin{comment}
\begin{ddanger}We can now ask if the measure
%\begin{equation}%\label{eq:}
$d^{3}\overline{k}/2\omega(\overline{k})$
%\end{equation}
is Lorentz invariant or not. We expect it to be so, but lets try
to demonstrate it explicitly by computing the Jacobian of a
Lorentz boost $\Lambda$. Without loss of generality, we can
assume that we are working with Cartesian coordinates. Note we
can factorize an Lorentz boost as
\begin{equation}%\label{eq:}
{\Lambda^{\mu}}_{\nu} = {L^{\mu}}_{\alpha}{R^{\alpha}}_{\nu}
\end{equation}
where $L$ is a rotation in the $t-x$ plane, and $R$ is an
arbitrary spatial rotation. We know from Euler's theorem that
\begin{equation}%\label{eq:}
\det(R)=1
\end{equation}
so that means that
\begin{equation}%\label{eq:}
\det(\Lambda)=\det(L).
\end{equation}
But we can write in block form
\begin{equation}%\label{eq:}
{L^{\mu}}_{\alpha} = \begin{bmatrix} L & 0\\ 0 & I_{2} \end{bmatrix}
\end{equation}
which means that 
\begin{equation}%\label{eq:}
\det({L^{\mu}}_{\alpha}) = {L^{0}}_{0}{L^{1}}_{1} - {L^{1}}_{0}{L^{0}}_{1}.
\end{equation}
This means, without loss of generality, we can write
${\Lambda^{\mu}}_{\nu}={L^{\mu}}_{\nu}$. We make the switch
$k^{0}=\omega(\overline{k})$, so our transformation
yields the coordinates
\begin{subequations}
\begin{align}
l^{0} &= {L^{0}}_{0}\omega(\overline{k}) - {L^{0}}_{1}k^{1}\\
l^{1} &= {L^{1}}_{0}\omega(\overline{k}) - {L^{1}}_{1}k^{1}\\
l^{2} &= k^{2}\\
l^{3} &= k^{3}
\end{align}
\end{subequations}
One may be at first alarmed by the switch to
$\omega(\overline{k})$ but it is invariant under spatial
rotations, and we've seen that $k^0=\omega(\overline{k})$ so it
is kosher. By Lorentz invariance, we demand further that
\begin{equation}%\label{eq:}
l^{\mu}l_{\mu} = k^{\mu}k_{\mu} = \mu^{2}
\end{equation}
which in turn implies that
\begin{equation}%\label{eq:}
l^{0}l_{0} = \|\overline{l}\|^{2} + \mu^{2}\;\Rightarrow\; l^{0}
= \omega(\overline{l})
\end{equation}
all by Lorentz invariance.
\end{ddanger}
\end{comment}

If we define Lorentz-normalized kets $|k\>$ by
\begin{equation}%\label{eq:}
|k\> = \left(2\omega(\overline{k})\right)^{1/2}(2\pi)^{3/2}|\overline{k}\>
\end{equation}
with $k_{0}=\omega(\overline{k})$, then the new normalization
conditions is
\begin{equation}%\label{eq:}
\<k|k'\> = 2\omega(\overline{k})(2\pi)^{3}\delta^{(3)}(\overline{k}-\overline{k}')
\end{equation}
and the resolution of the identity is based on the Lorentz
invariant measure:
\begin{equation}%\label{eq:}
\mathbf{1} = \int|k\>\<k|\frac{d^{3}\overline{k}}{(2\pi)^{3}2\omega(\overline{k})}.
\end{equation}
With these Lorentz-normalized states, we can define the unitary
representation of the Lorentz group simply:
\begin{thm}%\label{thm:}
If we define $U(\Lambda)$ by $U(\Lambda)|k\>=|\Lambda k\>$, then
$U$ is a unitary representation of the Lorentz group.
\end{thm}
\begin{proof}
The multiplications property
$U(\Lambda\Lambda')=U(\Lambda)U(\Lambda')$ follows immediately
from definition. To show that the representation is unitary, we
use the resolution of the identity
\begin{subequations}
\begin{align}
U(\Lambda)U(\Lambda)^{\dag} &= \int U(\Lambda)|k\>\<k|U(\Lambda)^{\dag}\frac{d^{3}\overline{k}}{(2\pi)^{3}2\omega(\overline{k})}\\
&= \int |\Lambda k\>\<\Lambda k|\frac{d^{3}\overline{k}}{(2\pi)^{3}2\omega(\overline{k})}\\
&= \mathbf{1}
\end{align}
\end{subequations}
since the measure is Lorentz-invariant.
\end{proof}
It is mildly surprising that $U(\Lambda)$ defined in our theorem
is a unitary operator due to $|k\>$ and $|\Lambda k\>$ appear to
have different lengths when $\Lambda$ is a boost. \emph{However,}
$\delta^{(3)}(0)$ is undefined, so the normalization of the kets
does not determine a length. We regard the uniformly unlocalized
state described by $|k\>$ as \emph{unphysical}. The physical
states have the form 
\begin{equation}%\label{eq:}
|\psi\>\stackrel{\text{def}}{=} \int \psi(\overline{k})|k\>\frac{d^{3}\overline{k}}{(2\pi)^{3}2\omega(\overline{k})}
\end{equation}
where the measure is structured so
$\<k|\psi\>=\psi(\overline{k})$. We can check that the length of
$|\psi\>$ is well defined whenever $\psi(\overline{k})$ is
square-integrable and that our definition of $U(\Lambda)$ makes
the representation unitary on the space of physical states.
