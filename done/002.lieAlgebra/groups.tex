\section{A Few Remarks On Groups}

We will follow Dummit and Foote~\cite{dummitFoote} here when introducing groups
to be as general as possible. 

\begin{defn}
A \textbf{binary operation} $\star$ on a set $G$ is a function
\begin{equation}
\star: G\times G\to G.
\end{equation}
For any $a,b\in G$ we shall write $a\star b$ instead of $\star(a,b)$.
\end{defn}

\begin{ex}
The operator of addition defined on integers is a binary operation.
\end{ex}

\begin{defn}
A binary operation $\star$ on $G$ is \textbf{associative} if for all $a,b,c\in G$
we have 
\begin{equation}
a\star(b\star c)=(a\star b)\star c.
\end{equation}
\end{defn}
\begin{rmk}
Associativity means we can screw around with the order of evaluation, or more to
the point we can screw around with parentheses as we please. Most things (the
vast majority of things) are associative in math.
\end{rmk}

\begin{defn}
If $\star$ is a binary operation on a set $G$ we say elements $a,b$ of $G$
\textbf{commute} if
\begin{equation}
a\star b = b\star a.
\end{equation}
We say $\star$ (or $G$) is \textbf{commutative} if this holds for all $a,b\in G$.
\end{defn}

\begin{rmk}
Let $\star$ be a binary operation defined on $G$ and $H\subset G$ be a subset.
If for all $a,b\in H$ we have $a\star b\in H$, then $H$ is said to be \textbf{closed}
under $\star$. Observe further that if $\star$ is associative on $G$, then
necessarily it is associative on $H$.
\end{rmk}

\begin{defn}
A \textbf{group} is an ordered pair $(G,\star)$ where $G$ is a set and $\star$ 
is a binary operation on $G$ satisfying:
\begin{enumerate}
\item $(a\star b)\star c = a\star(b\star c)$ for all $a,b,c\in G$, i.e. $\star$ is
associative;
\item There exists an element $e\in G$ called the \textbf{identity} of $G$ such
that for all $a\in G$ we have $a\star e = e\star a = a$;
\item For each $a\in G$ there is an element $a^{-1}\in G$ called the \textbf{inverse} of $a$ such that $a\star a^{-1} = a^{-1}\star a = e$.
\end{enumerate}
\end{defn}

\begin{defn}
Let $(G,\star)$ be a group. If 
\begin{equation}
a\star b = b\star a
\end{equation}
holds for all $a,b\in G$, then the group is called \textbf{abelian}.
\end{defn}

\begin{rmk}
Abelian merely means the group has a commutative binary operation, or equivalently
the group is commutative.
\end{rmk}

\begin{thm}
Let $G$ be a group with the operation $\star$, then
\begin{enumerate}
\item The identity of $G$ is unique;
\item For each $a\in G$, $a^{-1}$ is unique;
\item $(a^{-1})^{-1}=a$ for all $a\in G$;
\item $(a\star b)^{-1}=(b^{-1})\star(a^{-1})$
\item For any $a_1,\ldots,a_n\in G$, the expression $a_1\star\ldots\star a_n$ is
independent of how the expression is bracketed (``generalized associativity law'').
\end{enumerate}
\end{thm}

\marginpar{Change of Notation}Note that the notation $a\star b$ gets tedious to write, so we instead just
write $a\cdot b$ or $ab$. 

\begin{thm}
Let $G$ be a group and $a,b\in G$. The equations $ax = b$ and $ya = b$ have
unique solutions for $x,y\in G$. In particular, the left and right cancellation
laws hold in $G$, i.e.
\begin{enumerate}
\item if $au = av$, then $u=v$, and
\item if $ub = vb$, then $u=v$.
\end{enumerate}
\end{thm}

\begin{defn}
For $G$ a group and $x\in G$, the \textbf{order} of $x$ is defined to be the
smallest positive integer $n$ such that 
\begin{equation}
x^n = 1
\end{equation}
and denote this integer by $|x|$. In this case, $x$ is said to be of order $n$.
If no positive power of $x$ is the identity, the order of $x$ is defined to be 
infinity and $x$ is said to be of infinite order.
\end{defn}

This last definition is useful only when dealing with finite groups, which we
may possibly deal with.

\begin{defn}
Let $G = \{g_1,\ldots,g_n\}$ be a finite group with $g_1=e$. The \textbf{multiplication table} or \textbf{group table} of $G$ is the $n\times n$ matrix
whose $i,j$ entry is the group element $g_ig_j$.
\end{defn}

\begin{ex}
Let our group be $\mathbb{Z}_2 = \{0,1\}$ with the usual multiplication. Then
the multiplication table is
\begin{equation}
\begin{array}{c|cc}
\times & 0 & 1 \\ \hline
0 & 0 & 0 \\
1 & 0 & 1 
\end{array}
\end{equation}
\end{ex}

\begin{defn}
Let $G$ be a group. If $H\subset G$ is closed under the binary operator $\star$,
and $H$ satisfies all the axioms of a group, then $H$ is called a \textbf{subgroup}.
\end{defn}

\begin{defn}
The binary operation on $G$ defined by the map $G\times G\to G$ defined by
\begin{equation}
(g,x)\mapsto gxg^{-1}
\end{equation}
is called \textbf{conjugation}.
\end{defn}

\begin{defn}
A subgroup $N$ of $G$ is called a \textbf{normal subgroup} if it has the
following property: For every $a\in N$ and every $b\in G$, the conjugate
$bab^{-1}$ is in $N$.
\end{defn}

\subsection{Homomorphisms and Isomorphisms}

\begin{defn}
Let $(G,\star)$ and $(H,\Diamond)$ be groups. A map $\varphi:G\to H$ such that
\begin{equation}
\varphi(x\star y) = \varphi(x)\Diamond\varphi(y)
\end{equation}
(for all $x,y\in G$) is called a \textbf{homomorphism}.
\end{defn}

\begin{ex}
The map
\begin{equation}
\{x^i,p_j\}=\delta^{i}_{j} \mapsto [\hat{x}^i,\hat{p}_j] = i\hbar\delta^{i}_{j}
\end{equation}
used in quantization of classical systems is an example of a homomorphism.
\end{ex}

\begin{defn}
A Homomorphism $\varphi:G\to H$ is a \textbf{isomorphism} if and only if $\varphi$ is bijective.
We say $G$ and $H$ are \textbf{isomorphic} and denote it as $G\cong H$.
\end{defn}

\begin{rmk}
An isomorphism is a homorphism that can be undone. We can think of it as a sort
of equivalence relation. That is, if $G\cong H$ then we can intuitively think
of $G$ and $H$ as being ``equivalent''.
\end{rmk}

\begin{defn}
The \textbf{kernel of a linear operator} $T$ is the set of all arguments which
are mapped to zero, i.e. the additive identity.
\end{defn}

\begin{defn}
The \textbf{kernel of a homomorphism} $\varphi: G\to G'$ is the set of elements
of $G$ mapped to the identity in $G'$:
\begin{equation}
\ker(\varphi) = \{ a\in G : \varphi(a) = 1 \}
\end{equation}
which can be described as $\varphi^{-1}(1)$ -- the inverse image of the identity.
The kernel is a subgroup because (for $a,b\in G$)
\begin{equation}
\varphi(ab)=\varphi(a)\varphi(b)=1\cdot 1 = 1
\end{equation}
thus $ab\in\ker(\varphi)$.
\end{defn}

\subsection{Centralizers, Normalizers, Stabilizers, etc.}

\begin{defn}
Let $H$ be a subgroup of $G$. A \textbf{left coset} is a subset of the form
\begin{equation}
aH = \{ ah : h\in H \}.
\end{equation}
Observe that the subgroup $H$ itself is a coset, i.e. $H=1H$.
\end{defn}

\begin{ex}
If we have a collection of vectors over the field $\mathbb{F}_2$ such that they
all satisfy some system of equations (they are eigenvectors of a given operator),
then we may construct a coset by adding some constant vector $\vec{v}$ to all
of these vectors.
\end{ex}

\begin{cor}
The left cosets of a subgroup partition the group.
\end{cor}



It is customary to denote the set of cosets of a normal subgroup $N$ of $G$ by
the notation
\begin{equation}
G/N = \textrm{set of cosets of }N\textrm{ in }G.
\end{equation}
We will also use the symbols $\bar{G} = G/N$.

We can now study a set $S$ on which the group $G$ operates, then decompose
$S$ into ``orbits'' which we will now define.

\begin{defn}
The \textbf{orbit} of $s\in S$ is the set
\begin{equation}
O_{s} = \{ gs : g\in G\}.
\end{equation}
It is a subset of $S$.
\end{defn}

\begin{defn}
The \textbf{stabilizer} of an element $s\in S$ is the subgroup $G_{s}$ of $G$ of
elements leaving $s$ fixed
\begin{equation}
G_{s} = \{ g\in G : gs = s \}.
\end{equation}
\end{defn}





\begin{defn}
The stabilizer of an element $x\in G$ for the operation of conjugation is called
the \textbf{centralizer} of $x$ and is denoted $Z(x)$:
\begin{equation}
Z(x) = \{ g\in G: gxg^{-1} = x\} = \{ g\in G: gx = xg\}.
\end{equation}
The centralizer is the set of group elements which commute with $x$.
\end{defn}
