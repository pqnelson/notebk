\section{Lie Algebras}

The notion of a Lie algebra is fairly straightforward, but to reiterate we just
generalized the notion of a tangent vector to a point $x$ in an algebraic set
$S$ generated by the zeros of the function $f$ to be the vector $\vec{v}$ such
that
\begin{equation}
f(x + \varepsilon\vec{v}) = 0.
\end{equation}
We will use this notion to find the ``tangent vectors'' of the matrix groups
tangent to the ``point'' $I$ (the identity matrix).

\begin{ex}
We have certain conditions on the groups which we have outlined in function form, 
e.g. the orthogonal group requires that
\begin{equation}
A^\textrm{T} = A^{-1}.
\end{equation}
We have the tangent vectors then satisfy
\begin{equation}
(I + \varepsilon V)^\textrm{T} = (I + \varepsilon V)^{-1}
\end{equation}
which reduces to
\begin{equation}
\varepsilon V^\textrm{T} = -\varepsilon V.
\end{equation}
How did we do this step? Well, with matrices if the spectral radius is less than
$\pm1$, then we can do the following
\begin{equation}
(I - X)^{-1} = I + X + \frac{X^2}{2!} + \ldots
\end{equation}
and since $\varepsilon^2 = 0$ we have it only to first order. Thus
\begin{equation}
(I + \varepsilon V)^{-1} = I - \varepsilon V
\end{equation}
and we are content.

This implies that
\begin{equation}
\varepsilon(V^\textrm{T} + V) = 0
\end{equation}
and since 
\begin{equation}
\varepsilon\neq 0
\end{equation}
we have \emph{necessarily}
\begin{equation}\label{orthogonalLieAlgebra}
(V^\textrm{T} + V) = 0\Rightarrow V^\textrm{T}=-V.
\end{equation}
Thus we are dealing with traceless matrices. QEF.
\end{ex}

It often occurs that the same infinitesmal tangents can correspond to different groups.
For example, for the special linear group we have the condition
\begin{equation}
\det(X) = 1
\end{equation}
but we have the infinitesmal tangents determined by
\begin{equation}
\det(I + \varepsilon A) = 1.
\end{equation}
We know for ``small'' $\varepsilon$ that
\begin{equation}
\det(I + \varepsilon A) = 1 + \varepsilon \operatorname{tr}(A)
\end{equation}
and to have this be equal to one we need
\begin{equation}
\operatorname{tr}(A) = 0
\end{equation}
which is satisfied by Eq (\ref{orthogonalLieAlgebra}).

\begin{thm}
Let $A$ be a real, $n\times n$ matrix. Then the following are equivalent
\begin{enumerate}
\item $\operatorname{tr}(A)=0$;
\item $\exp(tA)$ is a one-parameter subgroup of $SL_{n}(\mathbb{R})$;
\item $A$ is in the Lie Algebra of $SL_{n}(\mathbb{R})$;
\item $A$ is an infinitesimal tangent to $SL_{n}(\mathbb{R})$ at $I$.
\end{enumerate}
\end{thm}

Note note note that the infinitesmal tangents includes as a subset the Lie
Algebras. We denote the Lie Algebra of a Matrix Group $G$ as $\lie{G}$. We also
denote the infinitesmal tangents of $G$ by $\textrm{Inf}(G)$.  We denote the
one-parameter subgroup of $G$ as $\textrm{Exp}(G)$. We have
\begin{equation}
\textrm{Exp}(G)\subset\lie(G)\subset\textrm{Inf}(G).
\end{equation}
So we have the most general set being the tangents of $I$ in $G$.

\subsection{Lie Bracket}

The Lie Algebra of a linear group has additional structure, an operation called
a \textbf{Lie Bracket}. We formally induce a Lie Bracket on these collections 
of matrices by the expression
\begin{equation}
[A,B] = AB - BA
\end{equation}
where $A$ and $B$ are matrices. This satisfies the \textbf{Jacobi Identity}
\begin{equation}
[A,[B,C]] + [B,[C,A]] + [C,[A,B]] = 0.
\end{equation}
We see also that if $A,B$ are skew-symmetric (antisymmetric), then
\begin{equation}
[A,B]^\textrm{T} = (AB-BA)^\textrm{T} = B^\textrm{T}A^\textrm{T} - A^\textrm{T}B^\textrm{T} = BA-AB = -[A,B].
\end{equation}
That is, it results in a skew-symmetric matrix.

The bracket is important because it is the infinitesmal version of the commutator in Lie \emph{groups}
$PQP^{-1}Q^{-1}$. To see this, we introduce two infinitesimal parameters $\epsilon$
and $\delta$ such that
\begin{equation}
\epsilon^2 = \delta^2 = 0
\end{equation}
and
\begin{equation}
\epsilon\delta = \delta\epsilon.
\end{equation}
Recall that the inverse of $I+A\epsilon$ is $I-A\epsilon$. So if $P=I+A\epsilon$
and $Q=I+B\delta$, the commutator expands to
\begin{equation}
(I+A\epsilon)(I+B\delta)(I-A\epsilon)(I-B\delta) = I + (AB - BA)\epsilon\delta.
\end{equation}
Intuitively we can see this because the commutator of infinitesmal elements in
$G$ is still in $G$!

\begin{defn}
A \textbf{Lie Algebra} $V$ over a field $\mathbb{F}$ is a vector space together
with a law of composition
\begin{eqnarray*}
V\times V &\to& V \\
v,w & \mapsto & [v,w]
\end{eqnarray*}
called the \textbf{bracket} having the following properties:
\begin{enumerate}
\item (Bilinearity) $[v_1 + v_2,w] = [v_1,w] + [v_2,w]$, $[cv,w] = c[v,w]$, $[v,w_1 + w_2] = [v,w_1] + [v,w_2]$, $[v,cw] = c[v,w]$;
\item (Antisymmetry) $[v,v]=0$;
\item (Jacobi Identity) $[u,[v,w]] + [v,[w,u]] + [w,[u,v]] = 0$;
\end{enumerate}
for all $u,v,w\in V$ and all $c\in\mathbb{F}$.
\end{defn}
