\section{An Introduction to Infinitesimals}

\subsection{A Brief Aside on Complex Numbers}

Recall with complex numbers (the set $\mathbb{C}$) we are dealing with a new
``number''
\begin{equation}\label{root}
i^2 + 1 = 0\quad\Rightarrow\quad i = \sqrt{-1}
\end{equation}
and so we have a new number system with elements of the form
\begin{equation}
(x + iy)\in\mathbb{C}\quad\textrm{where }x,y\in\mathbb{R}.
\end{equation}
We can formally define multiplication using the polynomial multiplication:
\begin{equation}
(a + ib)(x + iy) = (ax - by) + i(ay + bx)
\end{equation}
which results in a complex number.

\subsection{And Now For Something Slightly Different}

Suppose instead of Eq (\ref{root}) we worked with
\begin{equation}
\varepsilon^2 = 0\quad\textrm{and}\quad\varepsilon\neq 0
\end{equation}
which means we have elements called \textbf{infinitesimals} of the form
\begin{equation*}
x + \varepsilon y.
\end{equation*}
We can formally define multiplication in the same way we did for complex 
numbers
\begin{equation}
(a + b\varepsilon)(x+y\varepsilon) = ax + (xb + ya)\varepsilon + \cancelto{0}{by\varepsilon^2}.
\end{equation}
Note that $\varepsilon^2=0$ is the defining characteristic of our infinitesimal element!

Observe what mayhem we may do with this. We have some favorite Taylor series
\begin{equation}
f(x + h) = \sum_{n=0}^{\infty}\frac{f^{(n)}(x)}{n!}h^n
\end{equation}
which becomes with infinitesimals
\begin{equation}\label{classicTaylor}
f(x + \varepsilon) = f(x) + f'(x)\varepsilon + \cancelto{0}{\mathcal{O}(\varepsilon^2)}.
\end{equation}
Similarly for
\begin{equation}
\frac{1}{1 + x} = \sum_{n=0} (-x)^n = 1 - x + x^2 - \ldots
\end{equation}
we get
\begin{equation}
\frac{1}{1 + \varepsilon} = 1 - \varepsilon.
\end{equation}
For the famous euler's constant
\begin{equation}
e^x = \sum_{n=0}\frac{x^n}{n!}
\end{equation}
we get
\begin{equation}
e^\varepsilon = 1 + \varepsilon + \cancelto{0}{\mathcal{O}(\varepsilon^2/2)}.
\end{equation}
For trigonometric functions, which can be obtained by Euler's identity
\begin{equation}
e^{ix} = \cos(x) + i\sin(x)
\end{equation}
we deduce both
\begin{equation}
\cos(\varepsilon) = 1 + \cancelto{0}{\mathcal{O}(\varepsilon^2)}
\end{equation}
and
\begin{equation}
\sin(\varepsilon) = \varepsilon + \cancelto{0}{\mathcal{O}(\varepsilon^3)}.
\end{equation}
Thus we find that in this instance
\begin{equation}
\tan(\varepsilon) = \frac{\sin(\varepsilon)}{\cos(\varepsilon)} = \varepsilon/1
\end{equation}
which is remarkable!
