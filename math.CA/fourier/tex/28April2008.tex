%%
%% 28April2008.tex
%% 
%% Made by Alex Nelson
%% Login   <alex@tomato>
%% 
%% Started on  Sat Dec 20 17:48:34 2008 Alex Nelson
%% Last update Sat Dec 20 17:48:34 2008 Alex Nelson
%%
%% This section just covers two things: 1) weighted
%% $L^{2}_{\omega}(a,b)$ space; 2) eigenvalues,
%% eigenfunctions of regular Sturm-Liouville problem

Recall we just introduced $L^{2}_{\omega}(a,b) = \{
f:(a,b)\to\mathbb{R} |
\int^{b}_{a}|f(x)|^{2}\omega(x)dx<\infty\}$. For us, in
practice, we'll often have $\omega(x)>0$ and it's continuous
on $[a,b]$. We have weighted inner products and weighted
norms on $L^{2}_{\omega}(a,b)$ is defined as follows:
\begin{eqnarray}
\<f,g\>_{\omega} &=& \int^{b}_{a}f(x)\overline{g(x)}\omega(x)dx\\
\|f\|_{\omega} &=& \sqrt{\<f,f\>_{\omega}} = \left(\int^{b}_{a}|f(x)|^{2}\omega(x)dx\right)^{1/2}
\end{eqnarray}

Let demonstrate the weighted inner product\index{Weighted Inner Product}\index{Inner Product!Weighted} 
and weighted norm\index{Norm!Weighted Norm}\index{Weighted Norm}
satisfy the fundamental properties of the inner product and
norm (respectively).
\begin{enumerate}[(i)]
\item $\<f,0\>_{\omega}=0$, this is trivial (it really is,
  unlike all those other times I said something was trivial)
\item Linear in the first slot \marginpar{Linearity in First Slot}
\begin{align*}
\<\alpha f+\beta g,h\>_{\omega} &= \int^{b}_{a} (\alpha f(x)+\beta g(x))\overline{h(x)}\omega(x)dx\\
&=\int^{b}_{a} \alpha f(x)\overline{h(x)}\omega(x)dx +
\int^{b}_{a} \beta g(x)\overline{h(x)}\omega(x)dx\\
&= \alpha\<f,g\>_{\omega} + \beta\<g,h\>_{\omega}
\end{align*}
\item Antilinear in the second slot \marginpar{Antilinear in Second Slot} \begin{align*}
\<f,\alpha g+\beta h\>_{\omega} &=
\int^{b}_{a}f(x)\overline{[\alpha g(x)+\beta h(x)]}\omega(x)dx\\
&=\int^{b}_{a} f(x)\overline{\alpha h(x)}\omega(x)dx +
\int^{b}_{a} g(x)\overline{\beta h(x)}\omega(x)dx\\
&=\bar{\alpha}\<f,g\>_{\omega} + \bar{\beta}\<f,g\>_{\omega}
\end{align*}
\item Hermitian symmetric
\begin{align*}
\<f,g\>_{\omega} &=\int^{b}_{a}f(x)\overline{g(x)}\omega(x)dx\\
&=\overline{\int^{b}_{a}\overline{f(x)}g(x)\omega{x}dx},\qquad\omega:\mathbb{C}\to\mathbb{R}\\
&=\overline{\<g,f\>}_{\omega}
\end{align*}
\end{enumerate}

Those are the fundamental properties of the inner product,
so let us now consider the fundamental properties of the norm.

\begin{enumerate}[(a)]
\item{(Homogeneity)} $\|\alpha f\|_{\omega} =
  |\alpha|\|f\|_{\omega}$ (this is trivial).
\item{(Positivity)} $\|f\|_{\omega}\geq 0$ and
  $\|f\|_{\omega}=0$ if and only if $f=0$.
\end{enumerate}
Observe
\begin{align}
\|f\|^{2}_{\omega} =
\int^{b}_{a}&\underbracket[0.5pt]{|f(x)|^{2}\omega(x)}dx\\
&\text{always positive }\Rightarrow\text{ nonzero integral}\nonumber
\end{align}
If $\omega(x)=0$ on a small interval from
$[\alpha,\beta]\subset[a,b]$, then if $f(x)\neq0$ on
$f[\alpha,\beta]$ and $f(x)=0$ otherwise, but the norm
vanishes
\begin{equation}
\|f\|_{\omega} = \int^{b}_{a}|f(x)|^{2}\omega(x)dx = \int^{\beta}_{\alpha}|f(x)|^{2}\omega(x)dx=0.
\end{equation}
So
\begin{equation}
\omega(x)>0\qquad\forall x\in[a,b]
\end{equation}
is needed for the positivity of the norm. Therefore all
properties of the norm also hold. 

\begin{rmk}
The weighted inner product
\begin{equation}
\<f,g\>_{\omega} = \int^{b}_{a}f(x)\overline{g(x)}\omega(x)dx
\end{equation}
can be translated into the standard inner product
\begin{equation}
\<f\omega,g\>=\<f,\omega g\>=\<f,g\>_{\omega}
\end{equation}
because $\omega$ is real valued,
i.e. $\overline{\omega(x)}=\omega(x)$ for all
$x\in[a,b]$. We see that
\begin{equation}
0<m=\min_{x\in[a,b]}\omega(x)\leq\omega(x)\leq M=\max_{x\in[a,b]}\omega(x)
\end{equation}
which implies
\begin{equation}
0<m|f(x)|^2\leq\omega(x)|f(x)|^{2}\leq M|f(x)|^2
\end{equation}
and
\begin{align}
\overbracket[0.5pt]{\hphantom{m\int^{b}_{a}|f(x)|^{2}dx\leq\int^{b}_{a}|f(x)|^{2}\omega(x)dx}}^{\text{if $f\in L^{2}_{\omega}$, then the $L^2$ norm is bounded by this}} \hphantom{\leq M\int^{b}_{a}|f(x)|^{2}dx}\nonumber\\
m\int^{b}_{a}|f(x)|^{2}dx\leq\int^{b}_{a}|f(x)|^{2}\omega(x)dx\leq M\int^{b}_{a}|f(x)|^{2}dx\\
\hphantom{m\int^{b}_{a}|f(x)|^{2}dx}\underbracket[0.5pt]{\hphantom{\int^{b}_{a}|f(x)|^{2}\omega(x)dx\leq M\int^{b}_{a}|f(x)|^{2}dx}}_{\text{if $f\in L^{2}_{\omega}$, then the norm is finite}}\nonumber
\end{align}
Thus $L^{2}_{\omega}(a,b) = L^{2}(a,b)$. In fact, if
$\{\phi_{n}\}^{\infty}_{1}$ is an orthonormal basis for
$L^{2}(a,b)$, then we may obtain a basis
$\{\psi_{n}\}^{\infty}_{1}$ for $L^{2}_{\omega}(a,b)$ where 
\begin{equation}
\psi_{n}(x) = \frac{1}{\sqrt{\omega(x)}}\phi_{n}(x).
\end{equation}
The other way holds too, given $\psi_{n}$ we find
\begin{equation}
\phi_{n} = \sqrt{\omega(x)}\psi_{n}
\end{equation}
as an orthonormal basis for $L^{2}$.
\end{rmk}
We can use this to solve the regular Sturm-Liouville
problem. We want to prove the Sturm-Liouville problme has
its eigenfunctions form an orthonormal basis of
$L^{2}(a,b)$. Recall the linear operator we're working with is
\begin{align}
L(f)+\lambda\underbracket[0.5pt]{\omega(x)}&f(x)=[r(x)f'(x)]'+p(x)f(x)+\lambda\underbracket[0.5pt]{\omega(x)}f(x)=0\\
 &\!\!\!\!\!\!\!\uparrow \qquad\qquad\qquad\qquad\qquad\qquad\qquad\qquad\quad \!\!\!\!\uparrow \nonumber\\
 &\!\!\!\!\!\!\!\!\! \text{$\omega(x)$\;  used\;  in\;  the\;  weighted\;  inner\;  product} \nonumber
\end{align}
The self-adjoint boundary conditions:
\begin{subequations}
\begin{align}
B_{1}(f) &= \alpha_{1}f(a) + \alpha_{2}f'(a) + \beta_{1}f(b)+ \beta_{2}f'(b) = 0\\
B_{2}(f) &= \alpha_{3}f(a) + \alpha_{4}f'(a) + \beta_{3}f(b)+ \beta_{4}f'(b) = 0
\end{align}
\end{subequations}
For any $\lambda$, $f$ is a solution (a nontrivial solution,
read: nonzero solution) of the Sturm-Liouville problem, then
$\lambda$ is called an
\textbf{eigenvalue}\index{Eigenvalue}\index{Sturm-Liouville!Eigenvalues}\index{Differential Operator!Eigenvalues} 
and $f$ is called the
\textbf{eigenfunction}\index{Eigenfunction}\index{Sturm-Liouville!Eigenfunction}\index{Differential Operator!Eigenfunction} of the Sturm-Liouville problem.

\begin{rmk}
Note that we have
\begin{equation}
\<L(f),f\>=\<f,L(f)\>
\end{equation}
The boundary conditions make all other terms vanish. If both
$f$ and $g$ satisfy the boundary conditions, then 
\begin{equation}
\<L(f),g\> = \<f,L(g)\>.
\end{equation}
\end{rmk}

\subsection[Properties of Sturm-Liouville Problem]{Some Properties of Eigenvalues and Eigenfunctions of the Sturm-Liouville Problem}

\begin{thm}\label{thm:28April2008:thm3.9}
For the regular Sturm-Liouville problem, there are three
properties that hold:
\begin{enumerate}
\item All eigenvalues are real
\item Eigenfunctions corresponding to distinct eigenvalues
  are orthogonal with respect to the weighted inner product
\item For any eigenvalue $\lambda$, the eigenspace is at
  most two-dimensional (there are at most 2 linearly
  independent eigenfunctions associated with $\lambda$).
\end{enumerate}
Moreover, if the boundary conditions are seperated
(i.e. $B_{1}(f)$ focuses on one endpoint and $B_{2}(f)$
focuses on the other, e.g.
\begin{align*}
B_{1}(f) &= \alpha_{1}f(a) + \alpha_{2}f'(a)\\
B_{2}(f) &= \beta_{1}f(a) + \beta_{2}f'(a)
\end{align*}
then for any eigenvalue $\lambda$, the eigenspace associated
to $\lambda$ is one dimensional.
\end{thm}
\begin{proof}
\begin{enumerate}
\item We showed for Hermitian matrices, eigenvalues are
  real, so lets take the same steps. Let $\lambda$, $f$ be
  an eigenvalue/eigenfunction pair, then
\begin{align*}
\lambda\|f\|^{2}_{\omega} &= \<\lambda
f,f\>_{\omega}\quad\text{$\lambda$ is real}\\
&=\<\lambda\omega f,f\>\quad\text{by relation of $L^2$ to $L^{2}_{\omega}$}\\
&=\<-L(f),f\>\quad\text{by Sturm-Liouville equation}\\
&=\<-f,L(f)\>\quad\text{by Self-Adjointness of $L$}\\
&=\<f,-L(f)\>\\
&=\<f,\lambda\omega f\>\quad\text{by S-L eigenproblem}\\
&=\<f,\lambda f\>_{\omega}\quad\text{by relation of $L^2$ to $L^{2}_{\omega}$}\\
&=\bar{\lambda}\|f\|^{2}_{\omega}\quad\text{by Antilinearity}\\
\lambda=\bar{\lambda}&\Rightarrow\lambda\in\mathbb{R}
\end{align*}
\item Suppose $f,g$ are eigenfunctions corresponding to
  eigenvalues $\lambda,\mu$ where $\lambda\neq\mu$. Then we
  take the inner product
\begin{align*}
\lambda\<f,g\>_{\omega} &= \<\lambda\omega f,g\> \\
&= \<-L(f),g\>\\
&= \<f,-L(g)\>\quad\text{both are eigenfunctions, boundary and terms vanish}\\
&= \<f,\mu\omega g\>\quad\text{by S-L problem}\\
&= \mu\<f,g\>_{\omega}\\
\Rightarrow (\lambda-\mu)\<f,g\>_{\omega} &=0
\end{align*}
but $\lambda-\mu\neq0$ so it implies $\<f,g\>_{\omega}=0$.
\item Apply the existence of solution theorem to the
  Sturm-Liouville problem (For $L(f)+\lambda\omega f=0$ is a
  second order ODE with boundary conditions $f(a)=c_1$,
  $f(b)=c_2$, then there exists a unique solution.) For us,
\begin{align*}
B_{1}(f) &= \alpha_{1}f(a) + \alpha_{2}f'(a) +\beta_{1}f(b)+\beta_{2}f'(b)=0\\
B_{2}(f) &= \alpha_{3}f(a) + \alpha_{4}f'(a) +\beta_{3}f(b)+\beta_{4}f'(b)=0
\end{align*}
with two free variables (i.e. two degrees of freedom). There
are 4 unknowns and two equations imposed on them, so we have
2 degrees of freedom. There are then at most 2 linearly
independent eigenfunctions. When the boundary conditions are
set:
\begin{equation}
\begin{array}{l}
B_{1}(f) = \alpha_{1}f(a) + \alpha_{2}f'(a)\\
B_{2}(f) = \beta_{1}f(b) + \beta_{2}f'(b)
\end{array}\bigg\}\Rightarrow \text{only one degree of freedom}
\end{equation}
\end{enumerate}
Which concludes our proof.
\end{proof}
