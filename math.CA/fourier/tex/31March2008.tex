%%
%% 31March2008.tex
%% 
%% Made by Alex Nelson
%% Login   <alex@tomato>
%% 
%% Started on  Wed Dec 17 12:33:29 2008 Alex Nelson
%% Last update Sat Jan  3 16:35:11 2009 Alex Nelson
%%

\begin{ex}{(Diffusion of heat in solid body)}

The initial motivation behind Fourier series was to solve the
heat equation. The specific problem was trying to model the flow
of heat in a thin metal rod, such as the one seen in figure
\eqref{fig:31March2008img1}. The equation describing it requires
us first to consider several variables. For simplicity, consider
only a one dimensional ``thing'' insulated metal rod with length
$L$. 

\begin{figure}[h!]
  \begin{center}
    %%
%% 31March2008img1.tex
%% 
%% Made by Alex Nelson
%% Login   <alex@tomato>
%% 
%% Started on  Wed Dec 17 12:41:25 2008 Alex Nelson
%% Last update Wed Dec 17 12:41:25 2008 Alex Nelson
%% 2.1875in
\setlength{\unitlength}{0.0125in}
\begin{picture}(175,50)(0,-20)
\thinlines
\put(20,20){\circle{9}} % one end of the rod
\put(150,20){\arc{10}{5}{1}} % the other end of the rod
\drawline(0,0)(165,0) % the x axis
\drawline(21,24)(152,25) % the top of the rod
\drawline(21,16)(152,15) % the bottom of the rod
\dottedline{3}(20,16)(20,0) % the dashed bit to $x=0$
\dottedline{3}(152,16)(152,0) % dashed bit to $x=L$
\put(17,-10){\makebox(0,0)[lb]{\raisebox{0pt}[0pt][0pt]{\twltt $0$}}}
\put(147,-10){\makebox(0,0)[lb]{\raisebox{0pt}[0pt][0pt]{\twltt $L$}}}
\put(170,-3){\makebox(0,0)[lb]{\raisebox{0pt}[0pt][0pt]{\twltt $x$}}}
\end{picture}

  \end{center}
\caption{A thin insulated metal rod of length $L$}
\label{fig:31March2008img1}
\end{figure}

We can describe the problem with some conditions\index{Heat Equation}
\begin{subequations}
\begin{align}
u(x,t) &= \text{temperature at point $x$ and time $t$} \\
u(x,0) &= f(x)\quad (\text{temperature at time $t=0$, initial
  condition}) \\
u(0,t) = u(L,t) &= 0 \quad(\text{boundary condition})
\end{align}
\end{subequations}
Observe that we hold the temperature at the endpoints fixed to be
zero at all time. This is weird physically, because we rarely see
such a system. So we can describe the heat flow in a thin metal
rod by the equation
\begin{equation}\label{eq:31March2008:HeatEqn}
\frac{\partial u}{\partial t} = k\frac{\partial^{2}u}{\partial
  x^{2}}
\end{equation}
where $k$ is the diffusitivity of the rod (how quickly heat can
spread), and this is known as the heat equation. Now that we have
described the dynamics of the heat flow for this thin metal rod,
solving it should be trivial...right? 

Well, without
the boundary conditions and initial conditions, one can struggle
\marginpar{Skip work, here's answer}for a long long long time to find that
\begin{equation}\label{eq:31March2008:solution}
u(x,t) = t^{-1/2}\exp\left(\frac{-x^{2}}{4kt}\right)
\end{equation}
is a solution for $t>0$. So we would just need to solve the
boundary value problem. This is not as trivial as it seems.
\end{ex}

Our toolkit from taking basic differential equations (which is:
``Guess really really good!'') is inadequate to solve this
problem. We need to introduce a new
technique\marginpar{seperation of variables} which we call
\textbf{seperation of variables}. What we do is we assume we can
write the function $u(x,t)$ as
\begin{equation}\label{eq:31March2008:seperationOfVariables}
u(x,t)=X(x)T(t)
\end{equation}
This usually works but not always\footnote{It should be noted
  that this is a coordinate-choice, so there are a finite number
  of coordinate systems we can do this in (e.g. Morse et
  al.~\cite{morseVolTwo} point out that the Hemholtz equation,
  for example, can only be solved via seperation of variables in
  11 coordinate systems).}. We will not investigate it here but
it should be noted that it may not always work.

The second step to the ``seperation of
variables''\index{Seperation of Variables} technique is
to plug in \eqref{eq:31March2008:seperationOfVariables} into the
differential equation
\begin{subequations}
\begin{align}
\partial_{t} u(x,t) &= k\partial_{x}^{2}u(x,t) \\
\partial_{t} (X(x)T(t)) &= k\partial_{x}^{2}(X(x)T(t)) \\
X(x)T'(t) &= kX''(x)T(t)
\end{align}
\end{subequations}
and at first this seems useless. What do we do with this result?
Well, we divide both sides by $kX(x)T(t)$ and set it to be equal
to some constant\marginpar{Don't believe it? Note the left
  equation is independent of $x$, the middle equation is
  independent of $t$, so take the $t$ derivative of
  everything and you get the middle equation being 0,
  similar results for the $x$ derivative imply that they're
  equal to some constant.}
\begin{equation}
\frac{T'(t)}{kT(t)} = \frac{X''(x)}{X(x)}
= \begin{pmatrix}$some$\ $constant$\\$to$\ $be$\ $determined$\end{pmatrix}
\end{equation}
Observe that we have on the left hand side a function of $t$
only, it is independent of $x$. Similarly, the middle term is
independent of $t$. That is, we note
\begin{equation*}
\underbracket[0.5pt]{\frac{T'(t)}{kT(t)}}_{\text{independent of $x$}} =
\underbracket[0.5pt]{ \frac{X''(x)}{X(x)}}_{\text{independent of $t$}} =  \begin{pmatrix}$some$\ $constant$\\$to$\ $be$\ $determined$\end{pmatrix}
\end{equation*}
Now, we can solve for the general solution of $T(t)$
\begin{equation}
\frac{T'(t)}{T(t)}=kA\Rightarrow \log|T(t)|=kAt +
c_{1}\Rightarrow T(t)=T_{0}e^{kAt}
\end{equation}
where $T_{0}=T(0)$ is some constant.                            

Now, for solving $X(x)$, we have a bit more difficult of a
problem. The differential equation of relevance is
\begin{equation}
\frac{X''(x)}{X(x)} = A\Rightarrow X''(x) - AX(x) = 0
\end{equation}
Recall for general second order differential equations
\begin{equation}
y'' + ay' + by = 0
\end{equation}
we have the \textbf{characteristic equation}\index{Characteristic Equation!Differential Equations}
\begin{equation}
r^{2} + ar + b = 0.
\end{equation}
It has two roots, by the fundamental theorem of
Algebra\marginpar{Fundamental Thm Of Algebra: A polynomial
  of degree $n$ has exactly $n$ complex roots}. So we have
the two roots $r_{1}$ and $r_{2}$ which gives us that
general solution
\begin{equation}
y(x) = c_{1}e^{r_{1}x} + c_{2}e^{r_{2}x}.
\end{equation}
For us, we have a simple characteristic equation
\begin{equation}
r^{2} - A = 0\Rightarrow r = \pm\sqrt{A}
\end{equation}
thus the general solution is
\begin{equation}
X(x) = c_{1}e^{x\sqrt{A}} + c_{2}e^{-x\sqrt{A}}
\end{equation}
\marginpar{What's $A$?}So now we have a new problem: what
values of $A$ give nonzero solutions? That is, which ones
satisfy the boundary conditions
\begin{equation}
X(0)=X(L)=0
\end{equation}
without $X(x)$ being trivial (i.e. $X(x)=0$).

If $A>0$, then we have by one boundary condition
\begin{equation}
X(0)=c_1 + c_2=0\Rightarrow c_1 = -c_2.
\end{equation}
And by the other boundary condition
\begin{equation}
X(L) = c_{1}e^{L\sqrt{A}}+c_{2}e^{-L\sqrt{A}} = 0\Rightarrow
c_{1}e^{2L\sqrt{A}}+c_{2}=0
\end{equation}
and by putting these two together we find
\begin{equation}
c_{1}(1 - e^{2L\sqrt{A}}) = 0
\end{equation}
which requires either $c_1=0$ or $\exp(2L\sqrt{A})=1$. The
latter implies either $L=0$ or $A=0$. We assumed that $L>0$
and $A>0$. So that implies $c_1=0$. Thus the general
solution is
\begin{equation}
X(x) = c_{2}e^{-x\sqrt{A}}
\end{equation}
which is really just exponential decay.

For the $A<0$ case, the general solution can be written as
\begin{equation}
X(x) = c_{1}\cos(\lambda x) + c_{2}\sin(\lambda x)
\end{equation}
where $\lambda=\sqrt{-A}$. We plug in the boundary
conditions
\begin{equation}
X(0) = c_1 = 0
\end{equation}
and
\begin{equation}
X(L) = c_{2}\sin(\lambda L) = 0\Rightarrow \lambda L = n\pi
\Rightarrow \lambda = \frac{n\pi}{L}
\end{equation}
where $n\in\mathbb{Z}$. So here we have $A=-n^{2}\pi^{2}/L^{2}$.

Putting this all together, the general solution to the
boundary value problem is
\begin{equation}
u_{n}(x,t) =
\left(c_{0}\exp(-k\frac{n^{2}\pi^{2}}{L^{2}}t)\right)\left(c_{1}\sin(\frac{n\pi}{L}x\right)
\end{equation}
But also observe that \emph{any linear combinations} of
these solutions \emph{are also solutions} to the boundary
value problem. So the general solution is
\begin{equation}\label{eq:31March2008:generalSolutionUsingSeries}
u(x,t) =
\sum^{\infty}_{n=1}a_{n}e^{-kn^{2}\pi^{2}t/L^{2}}\sin\left(\frac{n\pi}{L}x\right)
\end{equation}
where $a_{n}$ are arbitrary constants.

``\emph{How to make $u(x,t)$ satisfy the initial condition
  $u(x,0)=f(x)$?}'' Well, we have a small substitution to
make, we just set
\eqref{eq:31March2008:generalSolutionUsingSeries} equal to
the given boundary condition
\begin{equation}
\sum^{\infty}_{n=1}a_{n}\sin\left(\frac{n\pi}{L}x\right) =
f(x)
\end{equation}
\emph{This is where Fourier works his magic, expanding a
  function in terms of sines and cosines.} \marginpar{Point
  to ponder}Is it possible to find an expansion for \emph{any} heat distribution?
