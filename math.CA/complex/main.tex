%%
%% main.tex
%% 
%% Made by Alex Nelson
%% Login   <alex@tomato>
%% 
%% Started on  Wed Jun 10 10:28:56 2009 Alex Nelson
%% Last update Wed Jun 10 10:28:56 2009 Alex Nelson
%%
\documentclass{amsart}
\usepackage{url}
\usepackage{manfnt}
\usepackage{amsthm}
\usepackage{amsmath}
\usepackage{amsthm}
\usepackage{amssymb}
\usepackage{amsfonts}
\usepackage{amscd}
\usepackage{graphicx}
\usepackage{mathrsfs}
\usepackage{array}
\numberwithin{equation}{section}

\theoremstyle{definition}
\newtheorem{defn}{Definition}
\newtheorem{thm}{Theorem}
\newtheorem{rmk}{Remark}
\newtheorem{lem}{Lemma}
\newtheorem{cor}{Corollary}
\newtheorem{ex}{Example}
\newtheorem{prop}{Proposition}
\newtheorem{sch}{Scholium}
\newtheorem{axm}{Axiom}
\newtheorem*{prob}{Problem}

\def\re{\operatorname{Re}}
\def\tr{\operatorname{Tr}}
\def\<{\langle}
\def\>{\rangle}

%%
% This macro header is what controls the ``dangerous bend''
% paragraph
%%
\def\rd{\noindent\begingroup\hangindent=3pc\hangafter=-2\def\par{\endgraf\endgroup}\hbox
  to0pt{\hskip-\hangindent\dbend\hfill}\ignorespaces}
%%
% This command allows you to write stuff in small font size and
% use the
% bourbaki ``dangerous bend'' so it's great when you want to
% ramble on 
% about some extra stuff!
%%
\newcommand{\danger}[1] {\rd{\small {#1}}}

%%
% This macro header is what controls the ``dangerous bend''
% paragraph
%%
\def\ddbend{\dbend\kern1pt\dbend}

\def\rdd{\noindent\begingroup\hangindent=4pc\hangafter=-2\def\par{\endgraf\endgroup}\hbox
  to0pt{\hskip-\hangindent\ddbend\hfill}\ignorespaces}

% make the margin sexy
\setlength{\marginparwidth}{1.2in}
\let\oldmarginpar\marginpar
\renewcommand\marginpar[1]{\-\oldmarginpar[\raggedleft\footnotesize #1]%
{\raggedright\footnotesize #1}}

% macros
\def\({\left(}
\def\){\right)}
\newcommand{\ds}{\displaystyle}
\newcommand{\ddanger}[1] {\rdd{\small {#1}}}
\newcommand{\wt}[1]{\widetilde{#1}}
\newcommand{\define}[1] {\textbf{#1}\index{#1}}
\setlength{\extrarowheight}{0.15cm}
\title{Complex Analysis Cheat Sheet Cont'd}
\date{June 10, 2009}
\email{pqnelson@gmail.com}
\author{Alex Nelson}
\begin{document}
\maketitle
\tableofcontents
\section{Laplace Transform}
%%
%% laplace.tex
%% 
%% Made by Alex Nelson
%% Login   <alex@tomato>
%% 
%% Started on  Wed Jun 10 10:30:18 2009 Alex Nelson
%% Last update Wed Jun 10 10:30:18 2009 Alex Nelson
%%
\begin{defn}[Laplace Transform]
The \textbf{Laplace Transform} of a function $f(t)$ (for $t\geq
0$) is the function $\widetilde{f}(z)$ defined by
\begin{equation}%\label{eq:}
\widetilde{f}(z) = \mathcal{L}\{f\}(z) = \int^{\infty}_{0}e^{-zt}f(t)dt
\end{equation}
where $z$ is complex.
\end{defn}
\begin{prop}[Asymptotic Behavior of Laplace Transform]%\label{prop:}
Suppose $g$ is analytic in a region containing the positive real
axis and is bounded on the positve real axis. Let the Taylor
series for $g$ centered at 0 be
\begin{equation}%\label{eq:}
\sum^{\infty}_{n=0} a_{n}z^{n}
\end{equation}
and let
\begin{equation}%\label{eq:}
\widetilde{g}(z) = \int^{\infty}_{0}e^{-zt}g(t)dt.
\end{equation}
Then
\begin{equation}%\label{eq:}
\widetilde{g}(z)\sim \frac{a_0}{z}+\frac{a_1}{z^2}+\frac{2a_2}{z^3}+\cdots+\frac{n!a_n}{z^{n+1}}+\cdots
\end{equation}
as $z\to\infty$, $\operatorname{arg}(z)=0$.
\end{prop}
\begin{prop}%\label{prop:}
Suppose $g$ is infinitely differnetiable on the positive real
axis and that $g$ and each of its derivatives are of exponential
order. That is, there are constants $A_n$ and $B_n$ such that
\begin{equation}%\label{eq:}
|g^{(n)}(t)|\leq A_{n}e^{B_{n}t}
\end{equation}
for $t\geq0$. Let
\begin{equation}%\label{eq:}
\widetilde{g}(z) = \int^{\infty}_{0}e^{-zt}g(t)dt.
\end{equation}
Then
\begin{equation}%\label{eq:}
\widetilde{g}(z)\sim  \frac{g(0)}{z}+\frac{g'(0)}{z^2}+\frac{g''(0)}{z^3}+\cdots+\frac{g^{(n)}(0)}{z^{n+1}}+\cdots
\end{equation}
as $z\to\infty$, $\operatorname{arg}(z)=0$.
\end{prop}
\begin{thm}[Convergence Theorem for Laplace Transform]%\label{thm:}
Assume
\begin{equation}%\label{eq:}
f:(0,\infty)\to\mathbb{C}
\end{equation}
is of exponential order and let
\begin{equation}%\label{eq:}
\widetilde{f}(z) = \int^{\infty}_{0}e^{-zt}f(t)dt.
\end{equation}
There exists a unique number $\sigma$, $-\infty\leq\sigma<\infty$
such that this integral converges if $\re(z)>\sigma$ and diverges
if $\re(z)<\sigma$. Furthermore if $\widetilde{f}$ is analytic on
the set
\begin{equation}%\label{eq:}
A = \{z|\re(z)>\sigma\}
\end{equation}
and we have
\begin{equation}%\label{eq:}
\frac{d}{dz}\widetilde{f}(z) = -\int^{\infty}_{0}te^{-zt}f(t)dt
\end{equation}
for $\re(z)>\sigma$. The number $\sigma$ is called the
``\textbf{Abscissa of Convergence}'' and if we\marginpar{define $\rho$} define the number
$\rho$ by 
\begin{equation}%\label{eq:}
\rho=\inf\{B\in\mathbb{R}|\text{there exists an }A>0\text{ such
  that }|f(t)|\leq Ae^{Bt}\}
\end{equation}
then $\sigma\leq\rho$.
\end{thm}
\begin{thm}[Laplace Transforms]
Suppose that the functions $f$ and $h$ are continuous and that
$\widetilde{f}(z)=\widetilde{h}(z)$ for $\re(z)>\gamma_0$ for
some $\gamma_0$. Then $f(t)=h(t)$ for all $t\in(0,\infty)$.
\end{thm}
\begin{prop}%\label{prop:}
Let $f(t)($ be continuous on $(0,\infty)$ and piecewise
$C^1$. Then for $\re(z)>\rho$
\begin{equation}%\label{eq:}
\widetilde{\left(\frac{df}{dt}\right)}(z)=z\widetilde{f}(z)-f(0).
\end{equation}
\end{prop}
\begin{prop}%\label{prop:}
Let
\begin{equation}%\label{eq:}
g(t)=\int^{t}_{0}f(\tau)d\tau
\end{equation}
Then for $\re(z)>\max[0,\rho(f)]$,
\begin{equation}%\label{eq:}
\wt{g}(z) = \frac{\wt{f}(z)}{z}.
\end{equation}
\end{prop}
\begin{thm}[First Shifting Theorem]%\label{thm:}
Fix $a\in\mathbb{C}$ and let $g(t)=e^{-at}f(t)$. Then for
$\re(z)>\sigma(f)-\re(a)$, we have
\begin{equation}%\label{eq:}
\wt{g}(z)=\wt{f}(z+a).
\end{equation}
\end{thm}

\begin{thm}[Second Shifting Theorem]%\label{thm:}
Let $H(t)=0$ if $t<0$ and $H(t)=1$ if $t\geq1$ be the
\textbf{Step Function} or \textbf{Heaviside Step Function}. Let
$a\geq0$ and let $g(t)=f(t-a)H(t-a)$; that is, $g(t)=0$ if $t<a$
while $g(t)=f(t-a)$ if $t\geq a$. Then for $\re(z)>0$ we have
\begin{equation}%\label{eq:}
\wt{g}(z)=e^{-az}\wt{f}(z).
\end{equation}
\end{thm}
\begin{defn}[Convolution]%\label{defn:}
The ``\textbf{Convolution}'' of two functions $f(t)$ and $g(t)$
is defined for $t\geq0$ by
\begin{equation}%\label{eq:}
(f*g)(t)=\int^{\infty}_{0}f(t-\tau)g(\tau)d\tau
\end{equation}
where we set $f(t)=0$ if $t<0$.
\end{defn}
\begin{thm}[Convolution Theorem]%\label{thm:}
The equalities
\begin{equation}%\label{eq:}
(f*g)(t) = (g*f)(t)
\end{equation}
whenever $\re(z)>\max[\rho(f),\rho(g)]$.
\end{thm}




\subsection{Table of Properties of the Laplace Transform}
Let $u(t)$ be the Heaviside step function.
\begin{equation}%\label{eq:}
u(t) = \int^{t}_{-\infty}\delta(\tau)d\tau
\end{equation}
where $\delta$ is the delta function we all know and love.
\bigskip
\begin{tabular}{|p{4cm}|l|l|}
\hline
%&&\\
Linearity & $af\left(t\right)+bg\left(t\right)$ & $a\wt{f}\(z\)+b\wt{g}\(z\)$\\\hline
Frequency Differentiation & $tf\(t\)$ & $-\wt{f}'\(z\)$\\\hline
Frequency Differentiation & $t^nf\(t\)$ & $(-1)^{n}\wt{f}^{n}\(z\)$\\\hline
Differentiation & $f'\(t\)$ & $z\wt{f}\(z\)-f\(0\)$\\\hline
Differentiation & $f''\(t\)$ & $z^2\wt{f}\(z\)-zf\(0\)-f'\(0\)$\\\hline
Differentiation & $f^{(n)}(t)$ & $z^n\wt{f}(z) - z^{n-1}f(0) -
\cdots - f^{(n-1)}(0)$\\\hline
Frequency Integration & $f(t)/t$ & $\int^{\infty}_{z}\wt{f}(\omega)d\omega$\\\hline
Integration & $\int^{t}_{0}f(\tau)d\tau=(u*f)(t)$ & $\wt{f}(z)/z$\\\hline
Scaling & $f(at)$ & $\wt{f}(z/a)/|a|$\\\hline
Frequency Shifting & $e^{at}f(t)$ & $\wt{f}(z-a)$\\\hline
Time shifting & $f(t-a)u(t-a)$ & $e^{-az}\wt{f}(z)$\\\hline
Convolution & $(f*g)(t)$ & $\wt{f}(z)\wt{g}(z)$\\\hline
Periodic Function & $f(t)$ & $(\int^{T}_{0}e^{-zt}f(t)dt)/(1-e^{-Tz})$\\\hline
\end{tabular}
\subsection{List of Properties of the Laplace Transform}
\textbf{Definition.\quad}\ignorespaces
The Laplace transform of $f(t)$ is given by:
\begin{equation}%\label{eq:}
\wt{f}(z)=\int^{\infty}_{0}e^{-zt}f(t)dt.
\end{equation}
It is such that:
\begin{enumerate}
\item $\displaystyle \wt{g}(z)=-\frac{d}{dz}\wt{f}(z)$ where $g(t)=tf(t)$.
\item $\ds \mathcal{L}\bigl\{af+bg\bigr\} = a\wt{f}+b\wt{g}$
\item $\ds \wt{\left(\frac{df}{dt}\right)}(z)=z\wt{f}(z)-f(0)$.
\end{enumerate}

\section{Gamma Function}
%%
%% gamma.tex
%% 
%% Made by Alex Nelson
%% Login   <alex@tomato>
%% 
%% Started on  Wed Jun 10 11:35:15 2009 Alex Nelson
%% Last update Wed Jun 10 11:35:15 2009 Alex Nelson
%%
So for $n$ a positive integer, we have
\begin{equation}%\label{eq:}
\Gamma(n)=(n-1)!
\end{equation}
\subsection{List of Properties of the Gamma Function}
Remember that it is defined as
\begin{equation}%\label{eq:}
\Gamma(z)=\int^{\infty}_{0}t^{z-1}e^{-t}dt.
\end{equation}
or equivalently as an infinite product
\begin{equation}%\label{eq:}
\Gamma(z) = \frac{1}{ze^{\gamma z}\left[\prod^{\infty}_{n=1}\(1+\frac{z}{n}\)e^{-z/n}\right]}
\end{equation}
where
\begin{equation}%\label{eq:}
\begin{split}
\gamma &=
\lim_{n\to\infty}\(1+\frac{1}{2}+\cdots+\frac{1}{n}-\ln(n)\)\\
&\approx 0.577215664901532860606512090082
\end{split}
\end{equation}

It has the following properties:
\begin{enumerate}
\item $\Gamma$ is meromorphic with simple poles at $0,$ $-1,$
  $-2,$ \ldots
\item $\ds\Gamma(z+1)=z\Gamma(z)$ for $z\neq 0,-1,-2,\ldots$
\item $\Gamma(n+1)=n!$ for $n=0,1,\ldots$
\item $\ds \Gamma(z)\Gamma(1-z) = \frac{\pi}{\sin(\pi z)}$
\item $\ds \Gamma(z)\neq 0$ for all $z$
\item $\ds \Gamma\(\frac{1}{2}\)=\sqrt{\pi}$,
  $\ds\Gamma\(n+\frac{1}{2}\)=\frac{1\cdot3\cdot(\cdots)\cdot(2n-1)}{2n}\sqrt{\pi}$
\item $\ds\Gamma(z)=\frac{1}{z}\prod^{\infty}_{n=1}\left[\(1+\frac{1}{n}\)^z\(1+\frac{z}{n}\)^{-1}\right]$
\item
  $\ds\Gamma(z)=\lim_{n\to\infty}\frac{n!n^z}{z(z+1)(\cdots)(z+n)}$
\item $\ds\Gamma(z)\Gamma\(z+\frac{1}{n}\)(\cdots)\Gamma\(z+\frac{n-1}{n}\)=(2\pi)^{(n-1)/2}n^{(1/2)-nz}\Gamma(nz)$
\item $\ds 2^{2z-1}\Gamma(z)\Gamma\(z+\frac{1}{2}\)=\sqrt{\pi}\Gamma(2z)$
\item The residue of $\Gamma(z)$ at $z=-m$ is equals
  $(-1)^{m}/m!$
\item (Euler's Integral) $\ds \Gamma(z)=\int^{\infty}_{0}t^{z-1}e^{-t}dt$ for $\re(z)>0$. The
  convergence is uniform and absolute for
  $-\pi/2+\delta\leq\arg(z)\leq\pi/2+\delta$ ($\delta>0$) and for
  $\varepsilon\leq|z|\leq R$ where $0<\varepsilon<R$.
\item $\ds\frac{\Gamma'(z)}{\Gamma(z)}=-\gamma-\frac{1}{z}+\sum_{n=1}^{\infty}\(\frac{1}{n}-\frac{1}{z+n}\)=\int^{\infty}_{0}\(\frac{e^{-t}}{t}-\frac{e^{-zt}}{1-e^{-t}}\)dt.$
\item $\ds \pi^{-z/2} \; \Gamma\left(\frac{z}{2}\right) \zeta(z) = \pi^{-\frac{1-z}{2}} \; \Gamma\left(\frac{1-z}{2}\right) \; \zeta(1-z).$
(Where $\zeta(s)$ is the Riemann zeta function)
\item $\ds \zeta(z) \; \Gamma(z) = \int_{0}^{\infty} \frac{u^{z-1}}{e^u - 1} \; du $ which holds for $\re(z)>1$.
\end{enumerate}


\section{Zeta Function}
%%
%% zeta.tex
%% 
%% Made by Alex Nelson
%% Login   <alex@tomato>
%% 
%% Started on  Wed Jun 10 11:57:19 2009 Alex Nelson
%% Last update Wed Jun 10 11:57:19 2009 Alex Nelson
%%
The definition for the Riemann zeta function is
\begin{equation}%\label{eq:}
\zeta(s)=\sum_{n=1}^{\infty}\frac{1}{n^s}=\frac{1}{1^s} + \frac{1}{2^s} + \frac{1}{3^s} + \cdots
\end{equation}
It is holomorphic everwhere except for a simple pole at $s=1$
with residue 1.

For any positive even integer $2n$, we have
\begin{equation}%\label{eq:}
\zeta(2n) = (-1)^{n+1}\frac{B_{2n}(2\pi)^{2n}}{2(2n)!}
\end{equation}
where $B_{2n}$ is a Bernoulli number, and for negative integers
we have
\begin{equation}%\label{eq:}
\zeta(-n)=\frac{-B_{n+1}}{n+1}
\end{equation}
for $n\geq1$.

Let
\begin{equation}%\label{eq:}
f(x) = \frac{x}{e^x-1}
\end{equation}
then the Bernoulli numbers may be found from
\begin{equation}%\label{eq:}
 B_n=\lim_{x\to0}\frac{d^n}{dx^n}\frac{x}{(e^x-1)}. 
\end{equation}
Observe that for $n=1$
\begin{equation}%\label{eq:}
f'(x) = \left({{1}\over{e^{x}-1}}\right)\left(1-\frac{f(x)}{e^x-1}\right)
\end{equation}
and now observe that
\begin{equation}%\label{eq:}
\frac{d}{dx}\left(\frac{1}{e^x-1}\right) = -e^x\left({{1}\over{e^{x}-1}}\right)^2
\end{equation}
and we can use the product rule to find all of our favorite
Bernoulli numbers.

We have a table of the first few Bernoulli numbers:
\begin{center}
\begin{tabular}{|c|c|}
\hline
$n$ & $B_n$\\\hline
0 & 1\\\hline
1 & -1/2\\\hline
2 & 1/6\\\hline
4 & -1/30\\\hline
6 & 1/42\\\hline
8 & -1/30\\\hline
10& 5/66$\approx$ 0.07575757576\\\hline
12& -691/2730$\approx$-0.25311355311\\\hline
14& 7/6\\\hline
16& -3617/510$\approx$ -7.09125686275\\\hline
18& 43867/798$\approx$ 54.9711779448\\\hline
\end{tabular}
\end{center}

The zeta function satisfies the functional equation
\begin{equation}%\label{eq:}
\zeta(s) = 2^s\pi^{s-1}\ \sin\left(\frac{\pi s}{2}\right)\ \Gamma(1-s)\ \zeta(1-s) \!,
\end{equation}
valid for all $s\in\mathbb{C}$. An equivalent relationship may be
expressed as a sum
\begin{equation}%\label{eq:}
\zeta(s)(1-{2^{1-s}})= \sum_{n=1}^\infty \frac{(-1)^{n+1}}{n^s}.\!
\end{equation}


\subsection{Mellin Transform}

The Mellin transform of a function $f(x)$ is defined as
\begin{equation}%\label{eq:}
 \int_0^\infty f(x)x^{s-1}\, dx,\!
\end{equation}
when defined. We can relate the zeta function to one million and
one things this way, we have
\begin{equation}%\label{eq:}
\Gamma(s)\zeta(s) =\int_0^\infty\frac{x^{s-1}}{\exp(x)-1}dx,\!
\end{equation}
where $\Gamma$ is our favorite gamma function, and
\begin{equation}%\label{eq:}
2\sin(\pi s)\Gamma(s)\zeta(s) =i\oint_{C}\frac{(-x)^{s-1}}{\exp(x)-1}dx\!
\end{equation}
for all $s$ where the contour $C$ begins and ends at $+\infty$
and circles the origin once.

\subsection{Laurent Series}

Since the zeta function has a single simple pole at $s=1$ we can
expand it around the singular point. The series is
\begin{equation}%\label{eq:}
\zeta(s)=\frac{1}{s-1}+\sum_{n=0}^\infty \frac{(-1)^n}{n!} \gamma_n \; (s-1)^n.
\end{equation}
where $\gamma_n$ are the Stieltjes constants, defined by the
limit
\begin{equation}%\label{eq:}
 \gamma_n = \lim_{m \rightarrow \infty} {\left(\left(\sum_{k = 1}^m \frac{(\log k)^n}{k}\right) - \frac{(\log m)^{n+1}}{n+1}\right)}.
\end{equation}
where the constant $n=0$ term in the Laurent series is just
$\gamma_0$ the Euler-Mascheroni constant.

\nocite{*}
\bibliographystyle{utcaps}
\bibliography{main}
\end{document}
