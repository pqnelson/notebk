\documentclass{article}
\usepackage{macros}

\title{Quantum Gravity}
\author{Alex Nelson\thanks{This is a page from \homeurl{}\hfil\break\indent\;\, Compiled:\enspace\today\ at \currenttime\ (PST)}}
\date{October 12, 2022}

\begin{document}
\begin{fmffile}{img/feyngraph}
\fmfcustomparticles
\maketitle

\section*{Introduction}
These are my notes on quantum gravity taken from Steve Carlip's course during Spring Quarter of 2010 at UC Davis. Lectures were on Wednesdays and Fridays.
Any errors or typos are mine. I have opted to include the inline citations, which Carlip gave in class, and collected them in the end in the references section.

I have tried to correct some small idiosyncracies in my notes, like
referring to the wave functional as a wave function.

% Barvinsky and Krykhtin 1993 for Feynman diagrams?
% Barvinsky and Kiefer (1998
\lecture

The first paper on quantum gravity was written by Rosenfeld in
1930.\footnote{I believe this is, in fact, two papers by Leon Rosenfeld:
\begin{enumerate}
\item ``Zur Quantelung der Wellenfelder'',
\journal{Ann.Phys.} \volume{397} (1930) 113--152. An English translation
may be found thanks to D.~Salisbury, Max Planck
Institute for the History of Science, Preprint 381 (2009) \url{https://pure.mpg.de/rest/items/item_2274368_1/component/file_2274366/content}.
\item ``\"{U}ber die Gravitationswirkungen des Lichtes''. \journal{Z.~Phys.} \volume{65} (1930) 589--599.
\end{enumerate}
The curious reader may peruse Peruzzi and Rocci's ``Tales from the prehistory of Quantum Gravity. L\'eon Rosenfeld's earliest contribution'' \arXiv{1802.08878} for a summary of
Rosenfeld's contributions to quantum gravity.}

A small aside on if gravity needs to be quantized. ``Well everything
else is [quantized].'' True, but gravity is slightly different. The
proper answer is: \emph{we don't know for certain but it seems
likely}. Lets consider a few thought experiments.

We will try to cover the collapse of the wave function without getting
into what it really means. Let us ask two questions:
\begin{enumerate}
\item Does gravity collapse the wave function?
\item Do other measurements collapse the wave function?
\end{enumerate}
There are four possible answers.

\bigbreak\noindent\textbf{Answer 1: No, No.}  This is the Everett
interpretation of quantum mechanics. This is fine if everything is
quantum mechanical, but what if gravity is not quantum mechanical? A
classical gravitational field coupled to the quantum mechanical matter
results in observable inconsistencies.

\begin{itemize}
\item Don N.~Page and C.D.~Geilker,
``Indirect Evidence for Quantum Gravity''.
\journal{Phys. Rev. Lett.} \volume{47} (1981) pp.979 \emph{et seq.}
\doi{10.1103/PhysRevLett.47.979}
\end{itemize}
Page and Geilker experiment testing if gravity is classical and matter
is quantum mechanical.

\bigbreak\noindent\textbf{Answer 2: No, Yes.}
The paper for this perspective:

\begin{itemize}
\item Kenneth Eppley and Eric Hannah,
``The necessity of quantizing the gravitational field''.
\journal{Foundations of Physics} \volume{7} (1977) pp.51--68
\doi{10.1007/BF00715241}
\end{itemize}

Eppley and Hannah argue if this were the case, we could send information
faster than light. Their argument is a tad elaborate.

\begin{wrapfigure}{R}{7pc}
\centering
\includegraphics{img/2009-04-01.0}
\end{wrapfigure}

Consider a particle in a box symmetric in the middle. We lower some
barrier in the middle (the dashed line to the right), split the box in
two. Send one to Pluto, the other remains here. Measure the
gravitational field. The measured field shouldn't be that of a whole
electron since that violates the conservation of energy, and such a
violation is bad. We are assuming that gravity is classical, so both
observers should measure the gravitational field for half of an
electron. Open the box [on Earth]. If the electron is present, the wave
function collapses, and information instantaneously changes --- the
gravitational field of Pluto's box \emph{instantaneously}
changes. That's bad.

What if we try to weaken causality? Well, causality is either there or
not, it's like pregnancy.

One may be able to weasel out of it by supposing that measurements may
be generalized a bit.

\bigbreak\noindent\textbf{Answer 3: Yes, No.}  This is Roger Penrose's
idea. We modify Schrodinger's equation to include some ``weak
nonlinearities'' from gravity.

\bigbreak\noindent\textbf{Answer 4: Yes, Yes.}  Gravity --- albeit
classical --- causes collapse of the wave function and measurement does
as well. This leads to violation of uncertainty, or the conservation of
energy(?).


\begin{wrapfigure}{R}{15pc}\vskip-2pc
\centering
\includegraphics{img/2009-04-01.1}
\end{wrapfigure}\bigbreak\noindent\textbf{Example} (Heisenberg microscope)\textbf{.}
Consider a microscope and an electron some distance $f$ from the
lense. We shine some photon to see the electron. We ignore factors and
use small-angle approximations. Also we set $c=1$.

Lets look at the uncertainty in momentum.  The electron receives
momentum from the collision of the photon with it. Suppose the energy of
the electron is $E$. We have
\begin{equation}
\Delta p_{x}\sim E\sin(\theta).
\end{equation}
What about the uncertainty in position? This comes from the diffraction
limit, we can approximate
\begin{equation}
\theta_{c}\sim\lambda/D,
\end{equation}
we can find the exact calculations from Jackson [\textit{Classical
    Electrodynamics}]. We have
\begin{equation}
\Delta x\sim f\theta_{c}\sim(f\lambda/D)\sim\lambda/\theta.
\end{equation}
So we find
\begin{subequations}
\begin{equation}
\Delta x\,\Delta p_{x}\sim E\lambda,
\end{equation}
then using the de Broglie relation $E\lambda\sim h$ gives us
\begin{equation}
\Delta x\,\Delta p_{x}\sim h.
\end{equation}
\end{subequations}
Classically, for a gravitational wave, we can have $E$ as low as we
want, and $\lambda$ as large as we want. This violates the uncertainty
principle.

If we violate the uncertainty principle, presumably all of quantum
mechanics is undermined. On the other hand, momentum conservation is
violated if the uncertainty principle is preserved.

There are limits to how accurately we can measure low energy
gravitational waves. The apparatus has to be smaller and more massive,
but that may collapse into a black hole. This may be a loophole to the
aforementioned [Heisenberg microscope] argument.
%\end{example}

\bigbreak
\noindent Although none of these are conclusive, they seem to
\emph{imply} that gravity is quantized.

\subsection{Semiclassical Gravity}

Suppose we have classical gravity and quantum fields. The Einstein field
equations become
\begin{equation}
\widehat{T}_{\mu\nu}\mid\psi\rangle = \frac{1}{8\pi} G_{\mu\nu}\mid\psi\rangle,
\end{equation}
which may be a bit too restrictive since $\widehat{T}_{\mu\nu}$ may have
noncommuting elements. On the other hand, we could make it
\begin{equation}
\langle\widehat{T}_{\mu\nu}\rangle = \frac{1}{8\pi} G_{\mu\nu}.
\end{equation}
The metric now depends on the matter field, and the matter field depends
on the metric. This becomes nonlinear, albeit a ``weak'' nonlinearity.

We can look at the Newtonian version of this:
\begin{equation}
\begin{split}
  \I\hbar\frac{\partial}{\partial t}\mid\psi\rangle
  = \left(\frac{-\hbar^{2}}{2m}\nabla^{2} + V\right)\mid\psi\rangle,\\
  \nabla^{2}V = 4\pi Gm\rho = 4\pi Gm\sum_{j}m_{j}|\psi_{j}|^{2}.
\end{split}
\end{equation}
There is a paper on this:
\begin{itemize}
\item P.J.~Salzman, S.~Carlip, ``A possible experimental test of
  quantized gravity''. \arXiv{gr-qc/0606120}, 9 pages.
\end{itemize}
Suppose we start with a single particle with a Gaussian wave
function. For small mass, it behaves like a free particle. For a large
mass, the width narrows since gravitational collapse ``wins out''. For
somewhere in between, there is nonlinear wiggling.

If we neglect the self-gravitating part, we recover the Hartree
approximation.

There is another potential problem that the covariant divergence of the
quantum stress-energy tensor is not conserved. We need to include in the
stress-energy tensor the contribution of the measurement apparatus.

\subsection{Positive Aspects of Quantizing Gravity}

There are some positive aspects of the quantization of gravity!

\begin{enumerate}
\item There are singularities in general relativity which need to be
  dealt with. This is similar to back when quantum mechanics was
  starting and we were answering questions like, ``Why doesn't the
  electron fall into the nucleus?''

\item Quantum gravity may deal with the problem of infinities in quantum
  field theory. Consider the renormalization of mass,
\begin{equation}
m(\varepsilon) = m_{0} + \frac{e^{2}}{\varepsilon},
\end{equation}
where we include the electric self-energy (which looks like
$e^{2}/\varepsilon$). If we include the classical self-energy to this,
we have,
\begin{equation}
  m(\varepsilon)
  = m_{0} + \frac{e^{2}}{\varepsilon} - \frac{Gm(\varepsilon)^{2}}{\varepsilon}.
\end{equation}
We can solve for $m(\varepsilon)$ to find that this is finite, it is
something like the Planck mass times $137$ or $1/137$.

We can also see the sum of Feynman diagrams of the gravitational
self-interaction of the electron is a finite sum, despite each term
being divergent! (People are finding sets of finite sums of Feynman
diagrams in supergravity. There is no proof yet.)

\item There are a few physical systems we would like to understand that
  only quantum gravity can answer. The very early universe when quantum
  effects were present as well as gravity being the dominant
  force. Black holes also may be better understood with the quantization
  of gravity.
\end{enumerate}

\subsection{Why not Quantum Gravity?}

Well, why not quantize gravity? In ordinary quantum theory, the basic
observables are local.  Consider a scalar field $\widehat{\varphi}(x)$,
the value of the field at point $x$, the axiomatic formulations of
quantum field theory these are observables.  This does not make sense,
since $x$ does not make sense.  There is no background. The symmetry of
general relativity is diffeomorphism invariance, i.e., invariance under
change of coordinates.  If $x\to x+a$, then
$\widehat{\varphi}(x)\to\widehat{\varphi}(x+a)$ which does not make
sense.

This is already an issue in classical general relativity.  We need to be
careful not to write ``the position $x$ of \emph{blah}'', but instead
``the time an atomic clock reads for a laser to reach some location''.
This is nonlocal, but what about this treatment in quantum theory?  It's
fine in classical general relativity, but we have problems in quantum
mechanics with nonlocal stuff.
\lecture

In classical general relativity, there are no local observables, so we
do not know what the right operators should be.  For a proof of the
absence of local observables, see:
\begin{itemize}
\item C.G.~Torre
``Gravitational Observables and Local Symmetries''.
\journal{Phys.~Rev.} \volume{D48} (1993) R2373--R2376(R); \arXiv{gr-qc/9306030}.\newline
{\tt\doi{10.1103/PhysRevD.48.R2373}}
\end{itemize}
A particular example of this is the ``problem of time''.

Consider a free scalar field in flat Minkowski spacetime, pick an
initial time slice and a final time slice to be the same in two
different foliations.  Is the time evolution in one foliation equivalent
to the time evolution in the other?
\begin{center}
\includegraphics{img/2009-04-03.0}
\includegraphics{img/2009-04-03.1}
\end{center}
We \emph{should} be able to ask if we have
\begin{equation}
|\psi_{1}(t)\rangle = \mathcal{U}|\psi_{2}(t)\rangle,
\end{equation}
where $\mathcal{U}$ is a unitary matrix indicating a change of bases.

Torre and Varadarajan~\cite{Torre:1997zs,Torre:1998eq} show, in general,
these are not related by a unitary matrix.  But we \emph{can} relate two
operators by orderings, which hold in the classical limit.

Determining time by spatial hypersurfaces requires using the metric.
Perhaps we can use the expectation value of the metric while demanding
it to be spatial but this depends on the wave function which we're
trying to find.

A lot of these problems come from thinking in the Schrodinger picture,
perhaps using the Heisenberg picture fixes it.  There are indications
from lower dimensional approaches that this may be correct.

\subsection{Quantization}

If we want to quantize general relativity, we need to talk about what it
means to quantize something. This is---for physicsts---the wrong
question. There may be more than one way to go to the classical limit,
but we work with the ones that are experimentally correct.

We start with some classical phase space with coordinates $(q,p)$ and
some Poisson bracket
\begin{equation}
\{p,q\}=1.
\end{equation}
We want to put hats on everything
\begin{equation}
[\widehat{p},\widehat{q}]=\I\hbar.
\end{equation}
We look for unitary irreducible representations on a Hilbert space, and
so on. That's the quantum theory. We have the rule
\begin{equation}
\{q,p\}\mapsto\frac{1}{\I\hbar}[\widehat{q},\widehat{p}],
\end{equation}
and for any observables $A$ and $B$ we have:
\begin{equation}
\{A,B\}\mapsto\frac{1}{\I\hbar}[\widehat{A},\widehat{B}].
\end{equation}
In general, this is impossible. There's a ``no go theorem'' from van
Hove proving there's no consistent way to do this.

We have to choose some subset of functions on the phase space, some set
of preferred phase space functions, that is ``small enough'' that this
mapping from Poisson brackets to commutators is consistent. But it must
be ``large enough'' so that any other function can be expressed in terms
of the preferred set. This is what we do when we quantize the Hydrogen
atom.

Suppose we have a phase space with a symmetry group $G$ which relates
any point with any other point. So we have the Poisson bracket be
preserved
\begin{equation}
\{gA,gB\} = \{A,B\}.
\end{equation}
If $H$ is the stabilizer of $x_{0}$ --- so $h\in H$ implies
$hx_{0}=x_{0}$ --- then $G/H$ is the phase space. In this case, we
choose the generators of the action of the group on the symmetric space
for the preferred functions to quantize.

The Stone--von Neumann theorem ensures the representation of
translations is unique up to unitary equivalence. But this theorem does
not hold in infinite-dimensions [i.e., for field theories].

There is another approach to quantization called ``deformation
quantization''. We have a quantization map,
\begin{equation}
\mathcal{Q}\colon\mbox{phase space}\to\mbox{operators}
\end{equation}
such that
\begin{enumerate}
\item Linearity: $\mathcal{Q}(c_{1}f_{1}+c_{2}f_{2}) = c_{1}\mathcal{Q}(f_{1}) + c_{2}\mathcal{Q}(f_{2})$
\item Preserves identity: $\mathcal{Q}(1)=\mathbf{1}$
\item $\mathcal{Q}(x)$, $\mathcal{Q}(p)$ are represented irreducibly
\item $\mathcal{Q}(\{f,g\}) = \frac{\I}{\hbar}[\mathcal{Q}(f),\mathcal{Q}(g)] + \mathcal{O}(1)$
\end{enumerate}
See:
\begin{enumerate}
\item P.~Tillman, ``Deformation Quantization, Quantization, and the
Klein-Gordon Equation''.
\journal{J.Phys.~A} \volume{40} (2007) 7017--7024; \arXiv{gr-qc/0610141}.\\
{\tt\doi{10.1088/1751-8113/40/25/S55}}
\item P.~Tillman, ``Deformation Quantization: From Quantum Mechanics to
Quantum Field Theory''. \arXiv{gr-qc/0610159}
\item S.~Twareque Ali, Miroslav Engli\v{s},
``Quantization Methods: A Guide for Physicists and Analysts''.
\journal{Rev.Math.Phys.} \volume{17} (2005) pp.391--490;
\arXiv{math-ph/0405065}.\\
{\tt\doi{10.1142/S0129055X05002376}}
\end{enumerate}

There is the path integral, which is just the continuous sum over the
paths, we write this formally as:
\begin{equation}
\int[\D q]\E^{\I S}.
\end{equation}
We can get different answers depending on how we define the derivative,
and we get extra terms of order $\hbar$. We think of these ambiguities
as normalization.

\lecture

We spoke about what it means to quantize a system. This time we will
discuss naive quantization of a system with constraints.

\textbf{An addendum from last time:} Take a one-dimensional particle
moving along a line. We have $q$, $p$ be the canonical coordinates and
we make the Poisson bracket into commutators
\begin{equation}
\{q,p\}\mapsto\frac{\I}{\hbar}[\widehat{q},\widehat{p}].
\end{equation}
The operator $\exp(\I a\widehat{p}/\hbar)$ generates translations in
position, so:
\begin{equation}
\E^{\I a\widehat{p}/\hbar}\widehat{q}\E^{-\I a\widehat{p}/\hbar}=\widehat{q}\pm a.
\end{equation}
Hence $\widehat{q}$ could take on any value.

Suppose we move on the positive real line, not the entire line. We can
use the affine commutation relations. We use $\widehat{q}$ and
\begin{equation}
\widehat{D} = \widehat{qp}.
\end{equation}
Classically we have
\begin{equation}
\{q,D\}=q,
\end{equation}
yet quantum mechanically,
\begin{equation}
[\widehat{q},\widehat{D}]=\I\hbar\widehat{q}.
\end{equation}
This is a different representation than the first set of commutators.

We have
\begin{equation}
\E^{\I a\widehat{D}/\hbar}\widehat{q}\E^{-\I a\widehat{D}/\hbar} = \E^{a}\widehat{q}.
\end{equation}
So this $\widehat{D}$ operation is just dilation.

This ought to be important since the ``position'' [in general
  relativity] is the metric on a spatial hypersurface, it should be
positive definite. In the naive way, we can get timelike directions or
nondefinite values, etc. We probably ought to use affine commutators.

(The Poisson bracket is unique. If we used the Heisenberg brackets, there
would be an ordering problem, though not a serious one since we could
use a symmetrized product.)

\subsection{Quantization of Constrained Systems}

In general, we have some action (we use $\action$ for the action since,
in Euclidean quantum gravity, the action is minus the entropy and $S$ is
used for entropy)
\begin{equation}
\action = \int L(q,\dot{q})\,\D t.
\end{equation}
In general, higher-order derivatives in the Lagrangian generically leads
to unbounded energies. For a review paper on this, see:
\begin{itemize}
\item R.P.~Woodard,
  ``Avoiding Dark Energy with $1/R$ Modifications of Gravity''.
  \journal{Lect.~Notes Phys.} \volume{720} (2007) pp.403--433; \arXiv{astro-ph/0601672}.\\
{\tt\doi{10.1007/978-3-540-71013-4_14}}
\end{itemize}
There are some exceptions, for example, when there are an infinite
number of derivatives (but this is a sort of nonlocality, and there have
been some papers recently on a nonlocal generalization of the
Einstein--Hilbert action), or when we can integrate by parts to get to
first-order. This is what happens with Einstein--Hilbert action.

The action of a system with constraints looks like
\begin{equation}
\action = \int(L(q,\dot{q}) - \lambda C(q,\dot{q}))\,\D t,
\end{equation}
where $\lambda$ is the Lagrange multiplier and $C(q,\dot{q})$ is the
constraint. By varying the action with respect to $\lambda$, we obtain a
constraint on the initial data
\begin{equation}
C(q,\dot{q})=0.
\end{equation}
We can rewrite the action to take into account constraints; if there are
no second-order time derivatives of certain variables, then there will
be constraints.

For general relativity, from the conservation laws, we have
\begin{subequations}
  \begin{align}
    \nabla_{\mu}G^{\mu\nu} &= \partial_{\mu}G^{\mu\nu} + \dots\\
    &= \partial_{t}G^{t\nu} + \partial_{i}G^{i\nu} + \dots.
  \end{align}
\end{subequations}
This means that each term has to have one time derivative and one
spatial derivative (in the Einstein tensor).

We can look at it one way and say the space of initial data is smaller
than we thought. Usually constraints ``generate'' gauge transformations,
meaning we can look at it as:
\begin{subequations}
  \begin{align}
    \delta q &= \{\varepsilon C,q\}\\
    \delta p &= \{\varepsilon C,p\},
  \end{align}
\end{subequations}
where $\varepsilon$ is an arbitrary function of time. This is a
generator of canonical transformations. In general, they're gauge
transformations.

Here's the sketch of the basic idea (see Henneaux and Teitelboim's
\emph{Quantization of Gauge Systems} for further details). We want to
consider the variation of the action $\delta\action$. Let's consider the
action in Hamiltonian form:
\begin{equation}
\action = \int(p\dot{q} - H -\lambda C)\,\D t.
\end{equation}
We often write
\begin{equation}
H^{*} := H + \lambda C,
\end{equation}
and refer to it as the ``Extended Hamiltonian''. Let's consider the
variation of the kinetic term:
\begin{subequations}
  \begin{align}
    \{\varepsilon C,p\dot{q}\}
    &=\{\varepsilon C,p\}\dot{q} + p\frac{\D}{\D t}\{\varepsilon C,q\}\\
    &=\left(\varepsilon \frac{\partial C}{\partial q}\right)\dot{q}
       +p\frac{\D}{\D t}\left(-\varepsilon\frac{\partial C}{\partial p}\right)\\
    &= \varepsilon \frac{\partial C}{\partial q}\dot{q}
       -\frac{\D}{\D t}\left(\varepsilon p\frac{\partial C}{\partial p}\right)
       +\varepsilon\frac{\partial C}{\partial p}\dot{p}\\
    &= \varepsilon \left(\frac{\partial C}{\partial q}\dot{q}+\frac{\partial C}{\partial p}\dot{p}\right)
       -\frac{\D}{\D t}\left(\varepsilon p\frac{\partial C}{\partial p}\right)\\
    &= \varepsilon \frac{\D C}{\D t}
       -\frac{\D}{\D t}\left(\varepsilon p\frac{\partial C}{\partial p}\right)\\
    &= -\dot{\varepsilon}C
       +\frac{\D}{\D t}\left(\varepsilon C - \varepsilon p\frac{\partial C}{\partial p}\right).
  \end{align}
\end{subequations}
The next term we need to examine is $\{\varepsilon C,H\}$ which, in
general, could be anything. If $C$ remains a constraint under time
translations, then the bracket with the Hamiltonian $H$ is also a
constraint. In general this is true, the commutator between the
Hamiltonian and a constraint is another constraint. Let
\begin{equation}
\{H,C\}=vC
\end{equation}
where $v$ is some function, then
\begin{equation}
\{\varepsilon C,H\} = -\varepsilon vC.
\end{equation}
Putting all of this together, we find
\begin{equation}\label{eq:2009-04-08:pb-of-constraint-and-lagrangian}
  \{\varepsilon C, p\dot{q}-H-\lambda C\}
  = -\dot{\varepsilon}C + \frac{\D}{\D t}\left(\varepsilon C - \varepsilon p\frac{\partial C}{\partial p}\right)
+\varepsilon vC-\{\varepsilon C,\lambda C\}.
\end{equation}
If
\begin{equation}
\delta\lambda = -(\dot{\varepsilon}-\varepsilon v),
\end{equation}
then the variation of the action is zero. This is because $\delta C
=\{\varepsilon C,C\} = 0$, so
\begin{equation}
\{\varepsilon C,\lambda C\} = \{\varepsilon C,\lambda\}C = (\delta\lambda)C.
\end{equation}
Plugging this back into
Eq~\eqref{eq:2009-04-08:pb-of-constraint-and-lagrangian} makes
$\{\varepsilon C, p\dot{q}-H-\lambda C\}$ into a total derivative, which
contributes nothing to the action.

The moral of the story is that constraints generate gauge
transformations and, in general (with the exception of some pathological
counterexamples), the converse holds too. Note: if $\{C,C\}\propto C$,
then the results still hold.

Now we use the results, and generalize to multiple constraints. We need
\begin{equation}
\{C_{i},C_{j}\} = {f_{ij}}^{k}C_{k}
\end{equation}
where ${f_{ij}}^{k}$ are ``structure constants'' and these are called
``first-class constraints''. (If we change Poisson brackets to
commutators, these are the generators of the gauge algebra --- or, at
least, the structure constants are those from the Lie algebra of the
gauge group.)

There are also ``second-class constraints'' which do not generate gauge
transformations.

There are various ways to handle constraints in the quantization
process. There is a constraint surface in the phase space when the
constraints are satisfied. We have this gauge invariance which
takes\dots\marginpar{TODO: insert diagram from page 9 here}

The space of orbits is what is interesting. We take the physical degrees
of freedom by taking some subsurface which cuts through the phase space
orbits only once for each orbit.

There are times when a section may not contain an orbit (or some other
unpleasant problem), which is the Gribov ambiguity.

For electromagnetism, we have $A_{\mu}\to
A_{\mu}+\partial_{\mu}\Lambda$.

The approaches to quantizing (``canonically'') systems with constraints:

\bigbreak
\textbf{Approach 1:} Reduced phase space quantization, the recipe is:
\begin{enumerate}[nosep,label=(\arabic*)]
\item Clasically solve the constraints.
\item Choose a section (``gauge fix'').
\item Insert into the action $\action$ and quantize.
\end{enumerate}
Part of the problem is that, well, sometimes solving the constraints
is fairly hard. For example, for classical general relativity, we do
not know the general solution for Einstein's field equations.

A second problem is when fixing a gauge, when we jump to the third step,
the resulting field is typically nonlocal. Consider electromagnetism. We
have $A^{\mu}$ which is ambiguous, we could have
\begin{equation}
A^{\mu} = \widetilde{A}^{\mu} + \partial^{\mu}\Lambda.
\end{equation}
We can gauge fix using the Lorenz gauge
$\partial_{\mu}\overline{A}^{\mu}=0$ for some gauge fixed potential
$\overline{A}^{\mu}$. We can expand this to be:
\begin{subequations}
\begin{align}
\partial_{\mu}\overline{A}^{\mu}
&= \partial_{\mu}(A^{\mu} + \partial^{\mu}\Lambda)\\
&=\partial_{\mu}A^{\mu} + \square\Lambda = 0.
\end{align}
\end{subequations}
Then we have
\begin{equation}
\Lambda = -\square^{-1}\partial_{\mu}A^{\mu},
\end{equation}
and then
\begin{equation}
\overline{A}^{\mu} = A^{\mu} - \partial^{\mu}\square^{-1}\partial_{\nu}A^{\nu},
\end{equation}
or
\begin{equation}
A^{\mu} = \overline{A}^{\mu}  - \partial^{\mu}\square^{-1}\partial_{\nu}A^{\nu}.
\end{equation}
The second term on the right-hand side is horribly nonlocal. BRST says
that sticking a differential gauge transformation back into the system
when solved is illegal. There are particular cases when this works; but
the more complicated the theory, the harder this approach becomes.

\bigbreak
\textbf{Approach 2:} Dirac quantization. The basic recipe is:
\begin{enumerate}[nosep,label=(\arabic*)]
\item Quantize the whole system.
\item Impose constraints as operator conditions. That is, we define the
  physical states as
  \begin{equation}
\widehat{C}\mid\mbox{physical}\rangle = 0,
  \end{equation}
  the kernel of a ``constraint operator'' (or the intersection of
  kernels of constraint operators). The states are automatically gauge
  invariant this way.
\item Define the inner product on physical states (intuitively: ``gauge
  fixing the inner product''). How to do this is less obvious and usually hard.
\item Find physical operators $\widehat{\mathcal{O}}_{\text{phys}}$ that take physical states to physical
  states (so if one realizes this, then it's equivalent to the operators
  which commutes with the constraints). That is, we need to find
  \begin{equation}
[\widehat{\mathcal{O}}_{\text{phys}},\widehat{C}]=0.
  \end{equation}
(For general relativity, these physical operators are in general
  nonlocal and we don't know what they are.)
\end{enumerate}

\bigbreak\noindent\textbf{Example.}
The parametrized particle. For a one-dimensional particle subjected to
some potential, then
\begin{equation}
\action = \int(p\dot{q} - H)\,\D t.
\end{equation}
If we change $t$ to some monotonic function of time, then $p\dot{q}\,\D t$
remains invariant \emph{but} the Hamiltonian contribution $H\,\D t$
doesn't quite remain invariant. Let's define a parameter $\tau$ such
that,
\begin{equation}
\action = \int\left(p\frac{\D q}{\D \tau} - H\frac{\D t}{\D\tau}\right)\D t.
\end{equation}
Let us write $q^{0}=t$ and $p_{0}=H$, so we can write the action as:
\begin{equation}
\action = \int\left(p_{\mu}\frac{\D q^{\mu}}{\D\tau}\right)\D\tau.
\end{equation}
But to do this, we observe $H=H(p,q)$, so we need to introduce a
constraint:
\begin{equation}
\action = \int\left(p_{\mu}\frac{\D q^{\mu}}{\D\tau} - \lambda(p_{0}+H(p,q))\right)\D\tau.
\end{equation}
The constraint generates parametrization invariance under
\begin{equation}
\tau\to\tau+\delta\tau.
\end{equation}
So far, so good.

The reduced phase space approach solves the constraint, which is
trivially $p_{0}=-H$. We plug this back in:
\begin{subequations}
  \begin{align}
    p_{\mu}\frac{\D q^{\mu}}{\D\tau}
    &= p_{1}\frac{\D q^{1}}{\D\tau} + p_{0}\frac{\D q^{0}}{\D\tau}\\
    &= p_{1}\frac{\D q^{1}}{\D\tau} + (-H)\frac{\D q^{0}}{\D\tau}.
  \end{align}
  We plug in our gauge-fixing $q^{0}=t$, then
  \begin{equation}
p_{\mu}\frac{\D q^{\mu}}{\D\tau}= p_{1}\frac{\D q^{1}}{\D\tau} + (-H)\frac{\D t}{\D\tau}.
  \end{equation}
\end{subequations}

The Dirac approach where the wave function $\Psi[q^{\mu}]$, the
commutators $[p_{\mu},q^{\nu}]=\I\hbar{\delta_{\mu}}^{\nu}$, the
constraint operator is
\begin{subequations}
  \begin{align}
    (\widehat{p}_{0}+\widehat{H}_{\text{phys}})\psi_{\text{phys}}
    &= 0\\
    &= \left(-\I\hbar\frac{\partial}{\partial q^{0}} + \widehat{H}_{\text{phys}}\right)\psi_{\text{phys}}.
  \end{align}
\end{subequations}
We need to gauge fix the inner product
\begin{equation}
\begin{split}
  \int\Psi^{*}_{1}(q^{\mu})\Psi_{2}(q^{\mu})\,\D q^{i}\D q^{0}
  &=\int\langle\Psi_{1}\mid\Psi_{2}\rangle_{\text{phys}}\,\D t\\
  &=\infty
\end{split}
\end{equation}
where $\langle\Psi_{1}\mid\Psi_{2}\rangle_{\text{phys}}$ is the usual
old-school inner product of quantum mechanics. This integral over time
generates infinities, we use a rigged Hilbert space to define the inner
product---roughly speaking, we ``divide out by infinity''.

\lecture

I was too sick to attend, but I have been told: Professor Carlip argued
the gauge symmetries of general relativity are isometries described by
Killing equation, derived ADM coordinates including lapse and shift
functions, described extrinsic curvature in terms of lapse and shift,
rewrote the Einstein--Hilbert action in ADM coordinates, derived canonically
conjugate momentum to metric, wrote first-order form of the action,
argued lapse and shift functions are Lagrange multipliers.

In second-order formalism positions $x$ and velocities $\dot{x}$ are treated as independent
variables, but first-order formalism treats positions $x$ and momenta $p$
as independent variables. Also, Professor Waldron refers to the ADM
action's terms as:
\begin{equation}
\action_{ADM} = \int(\underbrace{\pi^{ij}\dot{q}_{ij}}_{\substack{\text{symplectic}\\\text{term}}}
- \underbrace{N_{i}\mathcal{H}^{i} - N\mathcal{H}}_{\text{constraints}})\,\D^{4}x.
\end{equation}
Also notation: $D_{i}$ determined using the spatial metric $q_{ij}$
such that $D_{i}q_{jk}=0$.

\textbf{Caveat: these are notes I've written, not based on Dr Carlip's lectures,
but from what I've learned over the years.}

We start by choosing some coordinate $t$ and foliate spacetime with
spacelike hypersurfaces $\Sigma_{t}$ indexed by $t$. We have the unit normal
$n^{\mu}$ on each hypersurface $\Sigma_{t}$, as well as the induced
3-metric $q_{ij}$. For spacelike hypersurfaces
\begin{equation}
n^{\mu}n_{\mu}=+1.
\end{equation}
We then have a projection of tensors onto their spatial components
\begin{equation}
{h^{\mu}}_{\nu} = {\delta^{\mu}}_{\nu} - n^{\mu}n_{\nu}.
\end{equation}
We define the extrinsic curvature as the spatial projection of the
covariant derivative for the unit normal,
\begin{equation}
K_{\mu\nu} = {h_{\mu}}^{\rho}\nabla_{\rho}n_{\nu}.
\end{equation}
It's not hard to see $n^{\mu}K_{\mu\nu}=0$ and $n^{\nu}K_{\mu\nu}=0$
(since it's a spatial tensor).

Now, we have the ADM decomposition of the metric. We begin with the line
element, using the Lorentzian analog of the Pythagorean theorem.
Intuitively, we should imagine something like the picture:
\begin{center}
\includegraphics{img/2009-04-10.0}
\end{center}
We start off with a point on a hypersurface $\Sigma_{t}$. If we
translate along the time dimension from $t\to t+\D t$, then we end up at
$x^{i}+N\,\D t$ --- this is because the coordinates are arbitrary, but
$N\,\D t$ should be the ``infinitesimal proper time'' not the
``infinitesimal coordinate time''. Similarly, we could have some
rotational effect, which we would account for by adding a translation on
$\Sigma_{t+\D t}$ by $-N^{i}\,\D t$. This gives us the line element
\begin{equation}
\D s^{2} = N^{2}\,\D t^{2} - q_{ij}(\D x^{i} + N^{i}\,\D t)(\D x^{j} + N^{j}\,\D t).
\end{equation}
Here $N$ is called the ``Lapse function'', the $N^{i}$ are called the
``Shift vector''.

It's not too hard (I think it's an exercise
in \hyperref[section:hw2]{homework 2}) to show that
\begin{equation}
K_{ij} = \frac{1}{2N}(\partial_{t}q_{ij} - D_{i}N_{j}-D_{j}N_{i})
\end{equation}
where $D_{j}$ is the spatial covariant derivative (compatible with
$q_{ij}$, i.e., $D_{i}q_{jk}=0$). We also find the inverse 4-metric
decomposes like
\begin{equation}
g^{ab} = \begin{pmatrix}\frac{1}{N^{2}} & -\frac{N^{i}}{N^{2}}\\-\frac{N^{j}}{N^{2}} & -q^{ij} + \frac{N^{i}N^{j}}{N^{2}} \end{pmatrix},
\end{equation}
where $N^{i}=q^{ij}N_{j}$, and $q^{ij}$ is the inverse of $q_{ij}$
(i.e., $q^{ij}q_{jk}=\delta^{i}_{k}$).

We rewrite the Lagrangian using the Gauss--Codazzi equations
\begin{equation}
{}^{(4)}R = {}^{(3)}R + K_{ij}K^{ij} - K^{2} - 2\nabla_{\mu}(n^{\mu}\nabla_{\nu}n^{\nu}-n^{\nu}
\nabla_{\mu}n^{\mu}).
\end{equation}
Then the action
\begin{subequations}
\begin{align}
\action_{EH} &= \frac{1}{16\pi G}\int {}^{(4)}R\,\sqrt{-g}\,\D^{4}x\\
&= \frac{1}{16\pi G}\iint [{}^{(3)}R + K_{ij}K^{ij} - K^{2}](N\sqrt{q})\,\D^{3}x\,\D t
+ \mbox{(boundary terms)}.
\end{align}
\end{subequations}
We find the conjugate momenta to the 3-metric $q_{ij}$ are
\begin{equation}
\pi^{ij} = \frac{\partial L}{\partial(\partial_{t}q_{ij})}
= \frac{1}{16\pi G}(K^{ij} - q^{ij}K).
\end{equation}
Using this, we can write the canonical action
\begin{equation}
\action = \frac{1}{16\pi G}\iint(\pi^{ij}\partial_{t}q_{ij} - \mathcal{H}_{\text{can}})\,\D^{3}x\,\D t,
\end{equation}
where $\mathcal{H}_{\text{can}} = \pi^{ij}\partial_{t}q_{ij} - \mathcal{L}$.

We should expect there to be constraints, since the components of the
metric $N$ and $N_{i}$ do not enter the action with any time derivatives.
In fact, it turns out we have
\begin{subequations}
\begin{align}
\mathcal{H} &= \frac{16\pi G}{\sqrt{q}}(\pi_{ij}\pi^{ij} - \frac{1}{2}\pi^{2})
-\frac{1}{16\pi G}\sqrt{q}\,{}^{(3)}\!R\\
\intertext{and}
\mathcal{H}^{i} &= -2D_{j}\pi^{ij}
\end{align}
\end{subequations}
are the two constraints, called the Diffeomorphism constraint (or
Hamiltonian constraint) and the Momentum constraints, respectively.

The Poisson brackets would be defined on a spatial hypersurface (so
$t=\mbox{constant}$) as
\begin{equation}
\{q_{ij}(\vec{x}), \pi^{k\ell}(\vec{x}')\}
= \frac{1}{2}(\delta^{k}_{i}\delta^{\ell}_{j} + 
\delta^{k}_{j}\delta^{\ell}_{i})\widetilde{\delta}^{(3)}(\vec{x}-\vec{x}'),
\end{equation}
where we use the densitized delta $\delta^{(3)}(\vec{x})\sqrt{q}=\widetilde{\delta}^{(3)}(\vec{x})$
since it satisfies:
\begin{equation}
\int\widetilde{\delta}^{(3)}(\vec{x})\,\D^{3}x=1.
\end{equation}
A number of exercises concerning the Poisson bracket may be found
in \hyperref[section:hw3]{Homework 3}. In particular, the Poisson
bracket of the constraints generate diffeomorphisms (morally speaking).
% Carlip was out of town the following week, so the next lecture is:
\lecture

The action of general relativity in Hamiltonian form,
\begin{equation}
\action_{ADM} = \int[\pi^{ij}\dot{q}_{ij} - N\mathcal{H}-N_{i}\mathcal{H}^{i}]\,\D^{3}x\,\D t.
\end{equation}
The sign conventions varies, but the Hamiltonian is:
\begin{subequations}
\begin{equation}
\mathcal{H} = \frac{16\pi G}{\sqrt{q}}(\pi^{ij}\pi_{ij}-\pi^{2}) -
\frac{1}{16\pi G}\sqrt{q}\,\threeRicci,
\end{equation}
and the momentum constraints,
\begin{equation}
\mathcal{H}^{i} = -2D_{j}\pi^{ij}.
\end{equation}
\end{subequations}
This is a Hamiltonian for general relativity based on a certain set of
variables: the metric for a spatial hypersurfaces as the position
variable and its time derivative for its conjugate momenta.

We can consider the Poisson brackets for this field:
\begin{equation}
\{q_{ij}(x),\pi^{k\ell}(x')\} =
\frac{1}{2}(\delta^{k}_{i}\delta^{\ell}_{j}
+\delta^{k}_{j}\delta^{\ell}_{i})\widetilde{\delta}^{(3)}(x-x'),
\end{equation}
where the tilde indicates a densitized delta function, so
\begin{equation}
\int\widetilde{\delta}^{(3)}(x)\,\D^{3}x=1.
\end{equation}
In particular, we do not need to explicitly write out $\sqrt{q}$. Using
a densitized delta should make intuitive sense, since $\pi^{k\ell}$ is a
tensor density.

This is a completely constrained system, with the momentum constraints
generating spatial change of coordinates. Consider:
\begin{subequations}
  \begin{align}
    \{\int\xi^{i}\mathcal{H}_{i}(x)\,\D^{3}x, q_{k\ell}(x')\}
    &=\{-2\int\xi^{i}D^{j}\pi_{ij}(x)\,\D^{3}x, q_{k\ell}(x')\}\\
    &=\{\int(\xi_{i}D_{j} + \xi_{j}D_{i})\pi^{ij}(x)\,\D^{3}x, q_{k\ell}(x')\}\\
    &= - (D_{k}\xi_{\ell}+D_{\ell}\xi_{k})\\
    &= -\mathcal{L}_{\xi}q_{k\ell}.
  \end{align}
\end{subequations}
This means that $\mathcal{H}_{i}$ are generators of spatial coordinate
transformations. The Poisson bracket for the momentum constraints and
the $\pi^{ij}$ are a bit more complicated.

We are working on spatial hypersurfaces, so there is a question of
what ``$\mathcal{H}$ generates time translations'' even means. The easy
bracket is with $q_{ij}$, technically what these yield are ``surface
deformations''.\index{Surface Deformation}
(On the horizon of a black hole, surface deformations are not equivalent
to changes of coordinates which could be bad\dots)

\subsection{Reduced Phase Space Quantization}

We have two ways to quantize this system: reduced phase space approach,
and the Dirac approach. Lets begin with the reduced phase space
approach.

Recall the basic idea with the reduced phase space quantization is to
solve the constraints, find new variables, then quantize. The problem
with this is solving the constraints, which is roughly the same as
solving Einstein's field equations. We don't have it, or anything near
it. We need to assume some sort of symmetry (e.g., cylindrical
symmetry). This leads to ``minisuperspace'' or ``midisuperspace''.\index{Minisuperspace}\index{Midisuperspace}
In some sense, this is the wrong thing to do because assuming symmetry
at this level assumes that quantum states have this symmetry too.

An alternative approach is to change variables that change 4 nightmarish
PDEs into 4 simpler equations.

This is the York time-slicing\index{York Time-Slicing}\index{Time-Slicing!York},
the work on this was done predominantly by Fischer and Moncrief.
We start with
\begin{equation}
q_{ij} = \phi^{4}\widetilde{q}_{ij}
\end{equation}
where $\phi$ is the conformal factor, $\widetilde{q}_{ij}$ is such that
the Ricci scalar is
\begin{equation}
\threeRicci[\widetilde{q}]\in\{0,\pm1\}.
\end{equation}
This is the Yamabe condition. We can always do this for any Riemannian
manifold.
(The reason why $\threeRicci[\widetilde{q}]=0$ or $\pm1$ is due to the
topological properties of the spatial hypersurface; it is some deep
result in topology that is not immediately obvious.) This is for
spatially compact universes (or asymptotically flat ones).

We need the decomposition of the canonical momentum:
\begin{equation}
\pi^{ij} = \frac{1}{16\pi G}[\phi^{-4}p^{ij} -
  \underbrace{\frac{2}{3}K\phi^{2}\widetilde{q}^{ij}\sqrt{\widetilde{q}}}_{\text{trace part}}
  + (\underbrace{\widetilde{D}^{i}Y^{j} + \widetilde{D}^{j}Y^{i}}_{\text{symmetrized covariant derivative of a vector}}
  -\frac{2}{3}\widetilde{q}^{ij}D_{k}Y^{k})]
\end{equation}
where $Y^{i}$ is a density, $p^{ij}$ is a density, and $\widetilde{D}_{i}\widetilde{q}_{jk}=0$.
(Locally any vector
in 3 dimensions can be written as $\vec{\nabla}\phi + \vec{\nabla}\times\vec{A}$.)
We have
\begin{subequations}
\begin{equation}
\widetilde{D}_{i}p^{ij} = 0
\end{equation}
and
\begin{equation}
\widetilde{q}_{ij}p^{ij}=0.
\end{equation}
\end{subequations}
The momentum constraint $D_{i}\pi^{ij}=0$ can be translated into a
covariant derivative with respect to $\widetilde{q}$, we have
\begin{equation}
D_{i}\pi^{ij} = (\dots)\widetilde{D}_{i}p^{ij} + (\dots)\partial^{j}K + (\dots)\widetilde{D}_{i}(\widetilde{D}^{i}Y^{j}+\dots)=0.
\end{equation}
\marginpar{Crucial Step}%
The crucial step: we choose
\begin{equation}
t = -K,
\end{equation}
constant mean extrinsic curvature. This is not an obvious choice, but
there are proofs that this is neat, nice, and consistent. For a black
hole, the hypersurfaces curve around the singularity.

For a large class of solutions, Anderson and Moncrief have recent proofs
this is kosher.

We have
\begin{equation}
K = \frac{1}{N}\partial_{t}(\ln\sqrt{q}),
\end{equation}
some signs vary.

This choice tremendously simplifies things, we are left with
\begin{equation}
D_{i}\pi^{ij} = (\dots)\widetilde{D}_{i}(\widetilde{D}^{i}Y^{j}+\dots) = 0.
\end{equation}
If we assume spatial compactness or $Y$ falls off at infinity, we get
$Y^{i}=0$. So we have simplified the conjugate momenta to be:
\begin{equation}
\pi^{ij} = \frac{1}{16\pi G}[\phi^{-4}p^{ij} - \frac{2}{3}K\phi^{2}\widetilde{q}^{ij}\sqrt{\widetilde{q}}].
\end{equation}
Here $K$ is the proportional time rate of change of the local volume.

We solved the momentum constraints, $p$ is freely specified provided it
satisfies certain conditions. Now, the Hamiltonian constraint, which is
hard. We are left with really 2 independent components in
$\widetilde{q}$ and in $p$. We are left with the conformal factor $\phi$
to determine. The Hamiltonian constraint determines it! The Hamiltonian
constraint puts the condition on $\phi$:
\begin{equation}
  \widetilde{\Delta}\phi - \frac{1}{8}\phi + \frac{1}{12}t^{2}\phi^{5}
  -\frac{1}{8}\left(\frac{\widetilde{q}_{ij}\widetilde{q}_{k\ell}p^{ik}p^{j\ell}}{\widetilde{q}^{2}}\right)\phi^{-7} = 0.
\end{equation}
This is a second-order elliptic PDE.

Plugging this back into the action, we get (combining \emph{everything}
back together):
\begin{equation}
\action = \left(\frac{1}{16\pi G}\right)^{2}\int[p^{ij}\dot{\widetilde{q}}_{ij}-\frac{4}{3}\sqrt{\widetilde{q}}\phi^{6}]\D^{3}x\,\D t.
\end{equation}
With the Hamiltonian constraint implying we can write the conformal
factor as a function of $p$ and $\widetilde{q}$,
$\phi=\phi(p,\widetilde{q})$. In our notion of time, that Hamiltonian is
very nonlocal. It is effectively
\begin{equation}
\mathcal{H} = \frac{4}{3}\sqrt{\widetilde{q}}\phi^{6}.
\end{equation}
Due to this nightmarish nonlocality, we don't know how to put hats on stuff.

(We have been working with a zero cosmological constant $\Lambda=0$,
there should be some contribution from it in $\phi$ for nonzero
$\Lambda$.)

In $2+1$ dimensional gravity, the $q^{2}p^{2}\phi^{-7}$ term goes away,
and we have a local Hamiltonian, and everything's nice.

This was based on a particular decomposition, we'd like to keep
something similar to the decomposition of $\pi$.

\subsection{Dirac Quantization}

Let's begin Dirac quantization of the system. We basically impose the
constraints at the quantum level. We have our wave function $\Psi[q]$,
so we have
\begin{subequations}
\begin{equation}
\widehat{\mathcal{H}}^{i}\Psi[q]=0
\end{equation}
and
\begin{equation}
\widehat{\mathcal{H}}\Psi[q]=0.
\end{equation}
\end{subequations}
We use the Schrodinger picture to have
\begin{equation}
\widehat{\pi}^{ij} = -\I\frac{\delta}{\delta q_{ij}}.
\end{equation}
The momentum constraint smeared by some vector $\zeta^{i}$ is
\begin{equation}
\int\zeta_{j}D_{i}\frac{\delta}{\delta q_{ij}}\Psi[q]\,\D^{3}x=0.
\end{equation}
Integration by parts gives us,
\begin{equation}
\int(D_{i}\zeta_{j}+D_{j}\zeta_{i})\frac{\delta}{\delta q_{ij}}\Psi[q]\,\D^{3}x=0.
\end{equation}
By functional Taylor expansion, we have
\begin{equation}
\Psi[q_{ij} + D_{i}\zeta_{j}+D_{j}\zeta_{i}] - \Psi[q_{ij}]=0.
\end{equation}
So $\Psi$ is invariant under such coordinate transformations. This is
not as easy as it seems.

\lecture

Spacelike hypersurfaces defined by metric, but in general we don't know
the metric in quantum gravity, so we're out of luck. (We are assuming
that $\mathcal{M}=\mathbb{R}\times\Sigma$ where $\mathbb{R}$ is time,
$\Sigma$ is a spatial 3-manifold, at least in the canonical approach.)
Anyways, back to the Dirac approach.

We're imposing the constraints as operators on the wave function. We
interpret the momentum constraint
\begin{equation}
\widehat{\mathcal{H}}^{i}\Psi[q] = 0
\end{equation}
as telling us the wave function is invariant under spatial
diffeomorphisms. We should be able to, at least have the urge to assume
$\widetilde{H}$ is telling us the wave function is invariant under
temporal diffeomorphism but realize that this is meaningless. We're on a
spatial hypersurfaces, after all!

The DeWitt supermetric\index{DeWitt Supermetric}\index{Supermetric}
\begin{equation}
G_{ijk\ell} = \frac{1}{2}\frac{1}{\sqrt{q}}(q_{ik}q_{j\ell} +
q_{i\ell}q_{jk} - q_{ij}q_{k\ell}).
\end{equation}
This is like a metric of metrics. We have the deformation of a metric
$\delta q^{ij}$ have the length
\begin{equation}
\|\delta q^{ij}\|^{2} = \int G_{ijk\ell}\delta q^{ij}\delta q^{k\ell}\,\D^{3}x.
\end{equation}
This defines the distance on the space of metrics. (The signature of the
supermetric is $(-+++++)$, we take each pair of indices as a single
index resulting in a 6-by-6 matrix.)

We introduce
\begin{equation}
\widehat{\pi}^{ij}=-\I\frac{\delta}{\delta q_{ij}}.
\end{equation}
We plug it into the Hamiltonian constraint, and write:
\begin{equation}
\widehat{\mathcal{H}} = 16\pi G G_{ijk\ell}\frac{\delta}{\delta q_{ij}}\frac{\delta}{\delta q_{k\ell}}
+\frac{1}{16\pi G}\sqrt{q}\,\threeRicci.
\end{equation}
Resist the urge to make the first term a Laplacian. The Ricci 3-scalar
$\threeRicci$ is intuitively a sort of potential term, when viewed as a
function of $q_{ij}$. So then we plug it back into
\begin{equation}
\widehat{\mathcal{H}}\Psi[q]=0.
\end{equation}
This is the Wheeler--DeWitt Equation.\index{Wheeler--DeWitt Equation}

We need an inner product, wave functions alone do not suffice for a
quantum theory. There are 2 obvious thing to try to do.

The first thing, the ordinary Schrodinger picture using the 3-metric
\begin{equation}
\int\Psi^{*}\Phi\,[\D q] = \infty
\end{equation}
always since the Hamiltonian constraint, we need to gauge fix the inner
product, like a path integral with some extra symmetry.

\begin{itemize}
\item R.~P.~Woodard,
``Enforcing the Wheeler-de Witt Constraint the Easy Way''.
\journal{Class.~Quant.~Grav.} \volume{10} (1993), 483--496.\\
{\tt\doi{10.1088/0264-9381/10/3/008}}
\end{itemize}

We can think of $\widehat{\mathcal{H}}\Psi=0$ as a sort of Klein--Gordon
equation, and the correct inner product there is:
\begin{equation}
\int\Psi^{*}\overleftrightarrow{\frac{\delta}{\delta q}}\Phi\,[\D q].
\end{equation}
There is some ambiguity here, we have a number of inner products since
$\delta/\delta q$ is nonunique.

We could quantize the Wheeler-DeWitt equation, which is a third
quantization. This creates and annihilates metrics (which correspond to
universes), which we do not really observe.

There is another approach to finding an inner product which Woodard
proposes, which for simple models looks like the Klein-Godon inner
product. The wave function encodes some information about the placement
of the spatial hypersurface in the universe, which has some information
about time.

Another technical problem is the first term of $\widehat{\mathcal{H}}$
has two functional derivatives, which is problematic. We could try to
put in some regulator, so the first term looks like:
\begin{equation}
\widehat{\mathcal{H}} = \lim_{x\to x'}\widetilde{G}_{ijk\ell}\frac{\delta}{\delta q_{ij}(x)}\frac{\delta}{\delta q_{k\ell}(x')}+\dots.
\end{equation}
We need to show the result is independent of regularization. We also
need to be conscious of the Poisson bracket
$\{\mathcal{H},\mathcal{H}^{i}\}$ must be recovered from the commutator
$[\widehat{\mathcal{H}},\widehat{\mathcal{H}}^{i}]$ with our own
regularization.

\subsection{Perturbative Expansion}

The other thing we could try is a perturbative expansion, which is
natural if we cannot get an exact solution.\footnote{A good reference
for this subsection is Claus Kiefer~\cite{Kiefer:2004xyv}, \emph{Quantum Gravity}, third
edition, section 5.4.} We assume the wave
functional $\Psi$ satisfies the momentum constraints.

We can do what is roughly the Born--Openheimer approximation, wherein we
couple gravity and matter. (Basic idea of the Born--Openheimer
approximation is we have 2 independent processes, e.g., there is some
background on which matter moves slowly, but there is some
backreaction.)

Let us write:
\begin{equation}
\left(16\pi G\hbar G_{ijk\ell}\frac{\delta}{\delta q_{ij}}\frac{\delta}{\delta q_{k\ell}}
+\frac{1}{16\pi G\hbar}\sqrt{q}\,\threeRicci + \mathcal{H}_{m}\right)\Psi = 0,
\end{equation}
where we have the matter Hamiltonian $\mathcal{H}_{m}\approx T_{00}$.
Let us do a sort of WKB approximation:
\begin{equation}
\Psi = A\exp\left(\frac{\I}{32\pi G\hbar}S_{0}\right).
\end{equation}
We can expand in powers of the Planck length.
The lowest order expansion is just
\begin{equation}\label{eq:2009-04-24:perturbative-expansion-wd:zeroeth-order}
\frac{-1}{4}G_{ijk\ell}\frac{\delta S_{0}}{\delta q_{ij}}\frac{\delta S_{0}}{\delta k\ell}
+\sqrt{q}\,\threeRicci = 0.
\end{equation}
(This is the Hamilton--Jacobi equation for gravity uncoupled to matter.)
So at this level we have some background that's fixed and looks
classical.

The next order:
\begin{equation}\label{eq:2009-04-24:perturbative-expansion-wd:first-order}
\I G_{ijk\ell}\frac{\delta S_{0}}{\delta q_{ij}}\frac{\delta A}{\delta q_{k\ell}}
+\frac{\I}{2} G_{ijk\ell}\left(\frac{\delta^{2} S_{0}}{\delta q_{ij}\,\delta q_{k\ell}}\right)A
+\mathcal{H}_{m}A = 0.
\end{equation}
If we are clever, we can choose the functional derivative of $A$
to \emph{look like}:
\begin{subequations}
\begin{equation}
\frac{\delta A}{\delta q_{k\ell}}\sim\mbox{``}\frac{\D}{\D t}A\mbox{''}.
\end{equation}
Remember $A$ is the coefficient for our wave functional $\Psi$.
Explicitly,
\begin{equation}
A = D[q]\widetilde{\Psi},
\end{equation}
choose $D$ such that
\begin{equation}
\frac{\I}{2}G_{ijk\ell}\left(\frac{\delta^{2} S_{0}}{\delta q_{ij}\,\delta q_{k\ell}}\right)D
+ \I G_{ijk\ell}\frac{\delta D}{\delta q_{ij}}\frac{\delta A}{\delta q_{k\ell}}=0.
\end{equation}
\end{subequations}
Then,
\begin{equation}
\I G_{ijk\ell}\frac{\delta S_{0}}{\delta
q_{ij}}\frac{\delta\widetilde{\Psi}}{\delta q_{k\ell}} + \mathcal{H}_{m}\widetilde{\Psi}=0.
\end{equation}
This looks a lot like the Schrodinger equation. We have
\begin{equation}
G_{ijk\ell}\frac{\delta S_{0}}{\delta
q_{ij}} \mathrel{\mbox{``$\sim$''}}\frac{\delta q_{k\ell}}{\delta T}
\end{equation}
where $T$ is ``time'' (we don't know exactly what this is in quantum
gravity). We can be far more rigorous in certain midisuperspace models.

We can go to higher orders, where we get backreaction, where $S_{0}$
gets corrections from $S_{2}$ (the effects of gravity self-gravitating).
Barvinsky has worked out a systematic formalism using doodles that look
like Feynman diagrams.\footnote{Although a reference was not given, I
believe it is Barvinsky and Kiefer~\cite{Barvinsky:1997hp}.} It's not known if the approximation is renormalizable.

The zeroeth order
Eq~\eqref{eq:2009-04-24:perturbative-expansion-wd:zeroeth-order}
describes how spacetime curves, the first-order corrections in
Eq~\eqref{eq:2009-04-24:perturbative-expansion-wd:first-order} tells matter
how to move, the second-order correction tells spacetime curves due to matter,
then the third-order correction tells matter how to react to third-order
corrections, and so on.

We can do cosmology in this formalism. (Halliwell(?) did some old work
here.)\footnote{I think this refers to Jonathan
Halliwell~\arXiv{gr-qc/9208001}, possibly other papers.} Time has sort of emerged, which is nice, but this tells us that
time emerges when the universe is approximately classical. (What about
in other universes?)

\subsection{Strong Coupling Limit}

There's another approximation that has appeal. That is to take $\hbar G$
as large (the so-called \define{Strong Coupling Approximation}). This
might be good to tell us about the small scale structure of
spacetime.\footnote{Professor Carlip wrote a review paper with a good
discussion of this approximation in \S2 of \arXiv{1009.1136}.}
The leading order contribution in the Wheeler-DeWitt equation is the
first term. This tells u s that the metric ``decouples'' at each point.
To lowest order we have ``almost independent'' metrics at each point.

Classically, at each point, the general solution is the Kasner universe
\begin{equation}
\D s^{2} = -\D t^{2} + \E^{2p_{1}}\D x^{2} + \E^{2p_{2}}\D y^{2}
+ \E^{2p_{3}}\D z^{2},
\end{equation}
where, at lowest order, the terms $p_{j}$ are constants satisfying:
\begin{subequations}
\begin{equation}
p_{1}+p_{2}+p_{3}=1,
\end{equation}
and
\begin{equation}
p_{1}^{2}+p_{2}^{2}+p_{3}^{2}=1.
\end{equation}
\end{subequations}
The next order correction treats the $p_{j}$ as slowly-varying terms.

\begin{center}
\includegraphics{img/2009-04-24.0}
\end{center}

Misner called this the Mixmaster universe. There's a huge literature on
this (lookup the Belinskii-Khalatnikov-Lifshitz [BKL] model). It is
conjectured that near the Big Bang, the universe behaved this way.

At higher-order corrections, the oscillations look like:

\begin{center}
\includegraphics{img/2009-04-24.1}
\end{center}

The constraints imply $p_{i}>0$, $p_{j}>0$ and $p_{k}<0$ where $i\neq j\neq k$
and $i,j,k=1,2,3$.

\lecture

For reduced phase space quantization, we are left with one horrible
equation---as opposed to many horrible equations in the Dirac approach.

There's also the problem that we chose $t=-K$. It's sometimes not the
obvious choice for certain problems, for example the Schwarzschild
solution is completely scary. Do we get different quantum theories with
different time slicings? We don't know, this is kind of an anomaly
problem---is the quantum theory generally covariant in the reduced phase
space approach?

The Dirac approach has a few problems. We need to gauge fix the inner
product, but in practice we don't know how to do this. Another problem
is that the Wheeler-DeWitt equation has a piece that looks like the
product of two functional derivatives at a point, and this results in a
$\delta(0)$ contribution. This is a standard problem in quantum field
theory, regularization is needed. We could regulate it in theory as
\begin{equation}
\frac{\delta}{\delta g(x)}\frac{\delta}{\delta g(x')}\to
\frac{\delta}{\delta g(x)}K_{\varepsilon}(x,x')\frac{\delta}{\delta g(x')}
\end{equation}
where $K_{\varepsilon}(x,x')$ is some regulator invariant under spatial
diffeomorphisms, preserves the Poisson brackets, and becomes a $\delta$
function. No one has a proof that results are independent of how we
regulate. It could be possible it makes sense, we just don't know enough
about functional differential equations.

Even if this all worked out, the problem remains how to make sense of
basic variables. We have physical states be annihilated by the
constraints
\begin{equation}
\widehat{\mathcal{H}}\Psi_{\text{phy}}=0.
\end{equation}
We want a physical operator $\widehat{\mathcal{O}}$ to map physical
states to physical states
\begin{equation}
\widehat{\mathcal{O}}\Psi_{\text{phy}}=\Psi_{\text{phy}}'.
\end{equation}
This requires
\begin{equation}
[\widehat{\mathcal{H}},\widehat{\mathcal{O}}]\approx 0.
\end{equation}
We know no operators that do this. There have been proofs that such
operators are necessarily nonlocal, which we don't know how to deal
with. There's been work by some to make the Hamiltonian constraint
``almost local''.

This is where things stood roughly in the 1980s. There are some
simplified models where the Wheeler-DeWitt equation simplifies, just
freeze out degrees of freedom, very simplified settings. The
Wheeler-DeWitt equation becomes an ordinary differential equation.

In the early 1980s, two new approaches emerged:
\begin{enumerate}
\item Loop Quantum Gravity (which sought to simplify the Wheeler-DeWitt
  equation), and
\item String Theory (possibly contains quantum gravity).
\end{enumerate}
Then in the 1990s there was a new approach called dynamical
triangulations. We'll cover these three for the rest of the quarter.

\subsection{Loop Quantum Gravity}

We'll begin with gravity in the first-order formulation; i.e., a
tetrad/vierbein/frame field ${e^{I}}_{\mu}$. The capital Latin indices
track the basis vector, the Greek indices track the components of the
vector. We have
\begin{equation}
g^{\mu\nu}{e^{I}}_{\mu}{e^{J}}_{\nu}=\eta^{IJ}.
\end{equation}
It follows that
\begin{equation}
\eta_{IJ}{e^{I}}_{\mu}{e^{J}}_{\nu} = g_{\mu\nu}.
\end{equation}
We have an additional symmetry: local Lorentz symmetry.

Given such a tetrad, we can introduce the covariant derivative
\begin{equation}
\nabla_{\mu}A^{I} = \partial_{\mu}A^{I} + {{\omega_{\mu}}^{I}}_{J}A^{J},
\end{equation}
where ${{\omega_{\mu}}^{I}}_{J}$ is the spin connection.\index{Spin Connection}
Spin connections came about when people tried to introduce the spinor to
general relativity. We could demand metric compatibility to specify the
spin connection. The notation gets difficult, but let
$\widetilde{\nabla}_{\mu}$ be th eordinary covariant derivative for
tensors. The demand is that
\begin{equation}
\widetilde{\nabla}_{\mu}{e_{\nu}}^{J} +
          {{\omega_{\mu}}^{I}}_{J}{e_{\nu}}^{J} = 0
\end{equation}
determines the spin connection $\omega$ in terms of the frame $e$ and
Christoffel connection.

We can now do ordinary general relativity with this. So
\begin{equation}
[\nabla_{\mu},\nabla_{\nu}]A^{I} = {{R_{\mu\nu}}^{I}}_{J}A^{J},
\end{equation}
where
\begin{equation}
{{R_{\mu\nu}}^{\alpha}}_{\beta}{e^{I}}_{\alpha}{e_{J}}^{\beta} = {{R_{\mu\nu}}^{I}}_{J}.
\end{equation}
We write
\begin{equation}
A^{I} = {e_{\mu}}^{I}A^{\mu},
\end{equation}
and by our specification of the covariant derivative (specifically, the
spin connection) permits us to write the commutator.

The Einstein field equations are derived from the action:
\begin{equation}
\action_{EH} = \frac{1}{16\pi G}\int|e|e^{\mu I}e^{\nu J}R_{\mu\nu I J}\,\D^{4}x,
\end{equation}
where $|e| = \det|{e_{\mu}}^{I}|=\sqrt{-g}$ is the determinant of the
tetrad. We can express $R$ in terms of the spin connection, computed
directly from the commutator, as:
\begin{equation}
  {{R_{\mu\nu}}^{I}}_{J} = \partial_{\mu}{{\omega_{\nu}}^{I}}_{J}
  +{{\omega_{\mu}}^{I}}_{K}{{\omega_{\nu}}^{K}}_{J} - (\mu\leftrightarrow\nu).
\end{equation}
We can also treat the tetrad and connection as independent variables.
This isn't new: Palatini showed this holds for the metric and $\Gamma$
back in the 1930s.

The variation of the action, when treating tetrad and connection as
independent variables, gives us:
\begin{equation}
  \begin{split}
    \delta e: &\qquad e^{\nu I}R_{\mu\nu I J} = 0 = R_{\mu J}\\
    \delta\omega : &\qquad \nabla_{\mu}(e(e^{\mu I}e^{\nu J} - e^{\mu J}e^{\nu I})) = 0.
  \end{split}
\end{equation}
(The second variation is just the same as $\nabla^{\text{total}}_{\mu}{e_{\nu}}^{J}=0$.)
The Wheeler-DeWitt equation isn't more interesting, difficult, or
simple. But we can do interesting stuff!

We can write this in terms of forms
\begin{subequations}
\begin{align}
  e^{I} &= {e_{\mu}}^{I}\,\D x^{\mu},\\
  {\omega^{I}}_{J} &= {{\omega_{\mu}}^{I}}_{J}\,\D x^{\mu}.
\end{align}
\end{subequations}
We can write down the curvature 2-form
\begin{equation}
{\mathcal{R}^{I}}_{J} = \D{\omega^{I}}_{J} + {\omega^{I}}_{K}\wedge{\omega^{K}}_{J}.
\end{equation}
The action becomes (up to some sign error):
\begin{equation}
\action = \pm\frac{1}{64\pi G}\int\epsilon_{IJKL}e^{I}\wedge e^{J}\wedge\mathcal{R}^{KL}.
\end{equation}
This makes it \emph{look} neater.

Let us call
\begin{equation}
\mathcal{B}^{IJ} := e^{I}\wedge e^{J}.
\end{equation}
Then the action looks like
\begin{equation}
\action = \int\epsilon_{IJKL}\mathcal{B}^{IJ}\wedge\mathcal{R}^{KL}.
\end{equation}
We impose the condition $\mathcal{B}^{IJ} = e^{I}\wedge e^{J}$ (e.g.,
$\mathcal{B}^{IJ}\wedge\mathcal{B}^{KL} = e\varepsilon^{IJKL}$). The
converse (having $\mathcal{B}^{IJ}$ defined  by the condition
$\mathcal{B}^{IJ}\wedge\mathcal{B}^{KL} = e\varepsilon^{IJKL}$) is
\emph{almost} true. We then have:
\begin{equation}
  \action = \frac{1}{64\pi G}\int\left(
  \underbrace{\epsilon_{IJKL}\mathcal{B}^{IJ}\wedge\mathcal{R}^{KL}}_{\text{a ``BF'' theory}}
  + \phi_{IJKL}\underbrace{(\mathcal{B}^{IJ}\wedge\mathcal{B}^{KL} - e\varepsilon^{IJKL})}_{\text{constraint}}
  \right),
\end{equation}
and the $\phi_{IJKL}$ are Lagrange multipliers. If we didn't have tje
constraint, we'd have a flat spacetime with a sort of gauge theory
living on it. % (This is where topological QFT comes into play methinks.)

So writing things in new variables suggests new approaches.
Let us try some new variables.

\subsection{Self-Dual 2-Forms}

First, we define a $*$ operator on a 2-form:
\begin{subequations}
\begin{align}
F^{*}_{IJ} &= \frac{-\I}{2}\epsilon_{IJKL}F^{KL},\\
F^{**} &= F,
\end{align}
\end{subequations}
where $F_{[IJ]}=0$ (i.e., $F$ is antisymmetric). So this is a dual of
$F$ (there are many notions of ``duality''). We say $F$ is
\define{Self-Dual} if
\begin{equation}
F^{*}=F,
\end{equation}
and $F$ is \define{Anti-Self-Dual} if
\begin{equation}
F^{*}=-F.
\end{equation}
We can write, for an arbitrary 2-form $F$,
\begin{equation}
F^{\pm IJ} = \frac{1}{2}\left(F^{IJ}\pm F^{* IJ}\right).
\end{equation}
We can define a self-dual connection,
\begin{equation}
{A_{\mu}}^{IJ} = \frac{1}{2}\left({\omega_{\mu}}^{IJ} - \frac{\I}{2}{\epsilon^{IJ}}_{KL}{\omega_{\mu}}^{KL}\right).
\end{equation}
We define the self-dual curvature as:
\begin{subequations}
\begin{align}
{F_{\mu\nu}}^{IJ}
&= \partial_{\mu}{A_{\nu}}^{IJ} + {A_{\mu}}^{I}_{K}{A_{\nu}}^{KL} - (\mu\leftrightarrow\nu)\\
&= \frac{1}{2}({R_{\mu\nu}}^{IJ} + {R_{\mu\nu}}^{IJ*}).
\end{align}
\end{subequations}
We complexified, doubling the degrees of freedom, roughly speaking the
self-dual and antiself-dual splits the degrees of freedom.

The Ashtekar--Sen Action is then:
\begin{equation}
\action_{AS} = \frac{1}{8\pi G}\int e e^{\mu I}e^{\nu J}F_{\mu\nu IJ}\,\D^{4}x.
\end{equation}
By treating the self-dual connection as separate [independent] from the
tetrad, we get the Einstein field equations. (There is actually an extra
term like $\sim e^{\mu I}e^{\nu J}{R_{\mu\nu}}^{KL} = \epsilon_{IJKL}R^{IJKL}=0$.)
The constraints simplify \emph{dramatically}.

We need to have \define{Reality Conditions} so we don't have anything
imaginary. Classically, they are:
\begin{subequations}
\begin{align}
{\omega_{\mu}}^{IJ} &= {A_{\mu}}^{IJ} + {A_{\mu}}^{IJ*}\\
\frac{-\I}{2}{\epsilon^{IJ}}_{KL}{\omega_{\mu}}^{KL}
&={A_{\mu}}^{IJ} - {A_{\mu}}^{IJ*}\\
&=\frac{-\I}{2}{\epsilon^{IJ}}_{KL}({A_{\mu}}^{IJ} + {A_{\mu}}^{IJ*}).
\end{align}
\end{subequations}
The statement is that
\begin{equation}
{A_{\mu}}^{IJ} - {A_{\mu}}^{IJ*} = \frac{-\I}{2}{\epsilon^{IJ}}_{KL}({A_{\mu}}^{IJ} + {A_{\mu}}^{IJ*}).
\end{equation}
This is a second-class condition, which relates the real and imaginary
parts of the connection.

Given the change of variables to the Ashtekar--Sen action, we can do a
$3+1$ dimensional split. We will introduce new indices ($\widehat{I}$,
$\widehat{J}$, \dots = 1, 2, 3) for tetrad indices and ($i$, $j$, \dots =
1, 2, 3) for coordinate indices. Let's look at the components:
\begin{subequations}
\begin{align}
{A_{\mu}}^{0\widehat{L}} &= \frac{1}{2\I}{\epsilon^{0\widehat{L}}}_{\widehat{I}\widehat{J}}{A_{\mu}}^{\widehat{I}\widehat{J}}=\frac{1}{2\I}{A_{\mu}}^{\widehat{L}},\\
{A_{\mu}}^{\widehat{I}\widehat{J}} &= \frac{1}{2}{\epsilon_{0}}^{\widehat{I}\widehat{J}\widehat{K}}A_{\mu\widehat{L}}.
\end{align}
\end{subequations}
If we went back to the original spin connection, we find it is related
to the extrinsic curvature\index{Spin Connection!And Extrinsic Curvature}
\begin{equation}
{\omega_{i}}^{0\widehat{I}} = {K_{i}}^{\widehat{I}}.
\end{equation}
We can define
\begin{equation}
{\Gamma_{i}}^{\widehat{I}} = \frac{1}{2}{\epsilon_{0}}^{\widehat{I}\widehat{J}\widehat{K}}\omega_{i\widehat{J}\widehat{K}},
\end{equation}
which is basically the connection on the spatial hypersurface ignoring
the embedding. We do this so we can write the self-dual connection as
\begin{align}
{A_{i}}^{\widehat{I}} &= {\Gamma_{i}}^{\widehat{I}} + \I{K_{i}}^{\widehat{I}} \\
&= \begin{pmatrix}\mbox{Ordinary Connection}\\
\mbox{On the Slice}\end{pmatrix} + \I\begin{pmatrix}\mbox{Extrinsic Curvature}\\
\mbox{On the Slice}\end{pmatrix}\nonumber
\end{align}
We can generalize, letting $\immirzi$ be ``some parameter''
\begin{equation}
A^{(\immirzi)\widehat{I}}_{i} =  {\Gamma_{i}}^{\widehat{I}} + \immirzi{K_{i}}^{\widehat{I}}
\end{equation}
where $\immirzi$ is the \define{Immirzi--Barbero Parameter}. The
self-dual connection is really just a canonical transformation.

\lecture

(Remark: If we can write the constraints in two independent groups, then
we can do a mixture of Dirac quantization and reduced phase-space
quantization.)

Last time we ended up with a kind of gauge-like field,
\begin{equation}
A^{(\immirzi)\widehat{I}}_{i} =  {\Gamma_{i}}^{\widehat{I}} + \immirzi{K_{i}}^{\widehat{I}}.
\end{equation}
We can write this gauge-like field in terms of the spin connection as:
\begin{equation}
A^{(\immirzi)\widehat{I}}_{i} =  \frac{1}{2}\epsilon^{0\widehat{I}\widehat{J}\widehat{K}}\omega_{i\widehat{J}\widehat{K}}
+\immirzi {\omega_{i}}^{0\widehat{I}}.
\end{equation}
We can think of this as a canonical transformation. In the ADM
formalism, ${K_{i}}^{\widehat{I}}$ is more or less the canonical
conjugate momentum, and we're adding some terms involving derivatives of
the tetrad to it.

The next step is slightly dodgy, but makes the math easier. We gauge fix
Lorentz-boosts:
\begin{equation}
{e^{t}}_{\widehat{I}}=0.
\end{equation}
If we don't do this, then we get second-class constraints. This may give
a different representation (there is some evidence of it yielding a
different representation\footnote{Unfortunately, I didn't ask for
references on this.}). We can now define
\begin{subequations}
\begin{align}
{e^{t}}_{\widehat{0}} &= 1/\lapse\\
{e^{i}}_{\widehat{0}} &= -\shift^{i}/\lapse,
\end{align}
\end{subequations}
where $\lapse$ is the Lapse function and $\shift^{i}$ is the shift function
both from the ADM formalism.We have
\begin{equation}
q^{ij} = {e^{i}}_{\widehat{I}}e^{j\widehat{I}},
\end{equation}
so we can write
\begin{subequations}
\begin{align}
g^{ij} &= q^{ij} - \frac{\shift^{i}\shift^{j}}{\lapse^{2}}\\
  &={e^{i}}_{\widehat{0}}e^{j\widehat{0}} + {e^{i}}_{\widehat{I}}e^{j\widehat{I}}.
\end{align}
\end{subequations}
With this gauge fixing, we recover the ADM decomposition of the metric.

\begin{notation}
  Let's define
  \begin{equation}
{\widetilde{E}^{i}}_{\widehat{I}} := \sqrt{q}{e^{i}}_{\widehat{I}}.
  \end{equation}
It is a tensor density, and it is a triad on a spatial hypersurface.
\end{notation}

Given all of this, we can go back to the Einstein--Hilbert action, do
all the computations, we find:
\begin{equation}
  \action = \frac{1}{8\pi G}\int\left(
  \frac{1}{\immirzi}{A_{i}}^{\widehat{I}}\frac{\D}{\D t} {{\widetilde{E}}^{i}}_{\phantom{i}\widehat{I}}
-\underbrace{\I A_{0\widehat{I}}G^{\widehat{I}} + \I\shift^{i}V_{i} - \frac{1}{2}\frac{\lapse}{\sqrt{q}}S}_{\text{constraints}}
  \right)\,\D^{3}x\,\D t.
\end{equation}
If we didn't impose our gauge-fixing condition, we'd have a more
complicated constraint algebra and one more constraint. Now, let us
examine these constraints:
\begin{enumerate}[nosep,label=(\arabic*)]
\item We have $G^{\widehat{I}} = D_{i}\widetilde{E}^{i\widehat{I}}$
  where $D_{i}$ is the gauge covariant derivative treating this as a
  gauge theory.
\item $V_{j} = \widetilde{E}^{i}_{\phantom{i}\widehat{I}}{F_{ij}}^{\widehat{I}}$
  where ${F_{ij}}^{\widehat{I}} = \partial_{i}{A_{j}}^{\widehat{I}} - \partial_{j}{A_{i}}^{\widehat{I}}+\epsilon^{\widehat{I}\widehat{J}\widehat{K}}A_{i\widehat{J}}A_{j\widehat{K}}$.
  (This should ring a bell as the field strength tensor for a nonabelian
  gauge theory.) Observe these two do not involve the Immirzi parameter
  $\immirzi$ directlry.
\item The remaining constraint is a monster:
  \begin{equation}
S = \epsilon^{\widehat{I}\widehat{J}\widehat{K}}\widetilde{E}^{i}_{\widehat{I}}\widetilde{E}^{j}_{\widehat{J}}F_{ij\widehat{K}}
-2\left(\frac{1+\immirzi^{2}}{\immirzi^{2}}\right)\widetilde{E}^{i}_{[\widehat{I}}\widetilde{E}^{j}_{\widehat{J}]}(A^{(\immirzi)\widehat{I}}_{i}-\Gamma_{i}^{\phantom{i}\widehat{I}})
(A^{(\immirzi)\widehat{J}}_{j}-\Gamma_{j}^{\phantom{j}\widehat{J}}).
  \end{equation}
  The factor of $A^{(\immirzi)\widehat{I}}_{i}-\Gamma_{i}^{\phantom{i}\widehat{I}}$
  should remind us of the extrinsic curvature.
\end{enumerate}\medbreak

Let us consider the Poisson brackets of quantities.
\begin{equation}
\{\widetilde{E}^{i}_{\widehat{I}}(x),A^{(\immirzi)\widehat{J}}_{j}(x')\}
= -8\pi G\immirzi{\delta_{\widehat{I}}}^{\widehat{J}}\widetilde{\delta}^{(3)}(x-x').
\end{equation}
If we look at this as a nonabelian gauge theory, the Poisson bracket
looks like an electric field and potential. The constraint
$D_{i}\widetilde{E}^{i\widehat{I}}=G^{\widehat{I}}$ looks like Gauss's
law.\index{Gauss Law}

It looks like the physical phase space of an $\SU(2)$ gauge theory.
We're then imposing two additional constraints, and calling the result
quantum gravity. The natural thing to do is treat the
$A^{(\immirzi)\widehat{J}}_{j}$ as positions and
the $\widetilde{E}^{i}_{\widehat{I}}$ as momenta.

If we work at $G^{\widehat{I}}$, it tells us the wave functions are
gauge invariant, the $V^{i}$ constraints generate spatial
diffeomorphisms, and the $S$ is the Hamiltonian constraint. We have:
  \begin{equation}
S = \underbrace{\epsilon^{\widehat{I}\widehat{J}\widehat{K}}\widetilde{E}^{i}_{\widehat{I}}\widetilde{E}^{j}_{\widehat{J}}F_{ij\widehat{K}}}_{\text{scalar curvature term}}
    -\underbrace{2\left(\frac{1+\immirzi^{2}}{\immirzi^{2}}\right)\widetilde{E}^{i}_{[\widehat{I}}\widetilde{E}^{j}_{\widehat{J}]}(A^{(\immirzi)\widehat{I}}_{i}-\Gamma_{i}^{\phantom{i}\widehat{I}})
(A^{(\immirzi)\widehat{J}}_{j}-\Gamma_{j}^{\phantom{j}\widehat{J}})}_{\text{the $\pi^{2}$ term}}.
  \end{equation}
The $\pi^{2}$ term is ugly since the $\Gamma$ term depend on $E$, so we
have a constraint with quadratic terms in $E$.

There is a trick here, discovered originally by Thiemann, called the
Thiemann trick\index{Thiemann trick}, where we represent the ugly term
in terms of nested Poisson brackets. So it is possible to make it really
pretty.

Let's forget the Hamiltonian constraint for the time being. Let's try to
solve the other constraints, beginning with the Gauss's Law constraint.
The constraints
\begin{equation}
G^{\widehat{I}} = 0
\end{equation}
implies the wave functionals $\Psi$ are gauge-invariant. We will write
\begin{equation}
A = {A^{\widehat{I}}}_{i}\,\D x^{i}\,\tau_{\widehat{I}}
\end{equation}
where $\tau_{\widehat{I}}$ are the generators of $\SU(2)$ or $\SO(3)$
depending on gauge, let $g\in\SU(2)$ then the $A$ field transforms like:
\begin{equation}\label{eq:2009-05-01:gauge-transform-A}
A\to g^{-1}\,\D g + g^{-1}Ag.
\end{equation}
(For electromagnetism, $g^{-1}Ag = g^{-1}gA=A$ and $g^{-1}\,\D g =\D\Lambda$.)
Recall the field strength 2-form is then,
\begin{equation}
F = \D A + A\wedge A.
\end{equation}
We see, from $\D(gg^{-1})=0$ we have,
\begin{equation}
\D(g^{-1}) = -g^{-1}\,(\D g) g^{-1}.
\end{equation}
In particular,
\begin{equation}
\D(g^{-1}\,\D g + g^{-1}Ag) = - g^{-1}\,\D g\, g^{-1}\,\D g - g^{-1}\,\D g\,g^{-1}Ag +
g^{-1}\,\D A\,g - g^{-1}A\,\D g.
\end{equation}
Then applying this and Eq~\eqref{eq:2009-05-01:gauge-transform-A} to the field
strength 2-form gives us,
\begin{subequations}
\begin{align}
F\to F' &= \D(g^{-1}\,\D g + g^{-1}Ag) + (g^{-1}\,\D g + g^{-1}A g)\wedge(g^{-1}\,\D g + g^{-1}A g)\\
&= g^{-1}\,\D A\, g + (g^{-1}Ag)\wedge(g^{-1}Ag)\\
&= g^{-1}(\D A + A\wedge A)g\\
&= g^{-1}Fg.
\end{align}
\end{subequations}
This isn't terribly surprising, it's basic differential geometry.

The kinetic term is
\begin{equation}
\Tr(F^{2}) = F^{\widehat{I}}_{\mu\nu}F^{\mu\nu}_{\widehat{I}}
\end{equation}
for the gauge field. It's invariant under gauge transformations.

\index{Holonomy|(}
Now we would like to construct a basis of gauge invariant quantities
(easier said than done). But we can consider the parallel transport of a
gauge field on a cloed curve on a surface, the holonomy is gauge
invariant! We have the parallel transport,
\begin{equation}
\frac{\D v^{\widehat{I}}}{\D s}
+ \frac{\D x^{i}}{\D s}\left({A_{i}}^{\widehat{I}}{{\epsilon_{\widehat{K}}}^{\widehat{I}}}_{\widehat{J}}\right)v^{\widehat{J}} = 0.
\end{equation}
This is the equation for parallel transport, it's basic differential
geometry.\footnote{See, e.g., \S13 of my notes on general relativity
\url{http://pqnelson.github.io/assets/notebk/GR.pdf}.} The result is that
\begin{equation}
v^{\widehat{I}}(s) = \mathcal{U}^{\widehat{I}}_{\phantom{I}\widehat{J}}(s,s_{0})v^{\widehat{J}}(s_{0}),
\end{equation}
where we have the path-ordering exponential,
\begin{equation}
\mathcal{U}^{\widehat{I}}_{\phantom{I}\widehat{J}}(s,s_{0}) =
\mathcal{P}\exp\left(-\int^{s}_{s_{0}}{A_{i}}^{\widehat{K}}\underbrace{\epsilon^{\widehat{I}}_{\phantom{I}\widehat{J}\widehat{K}}}_{\text{generators}}\,\D x^{i}\right).
\end{equation}
We can generalize to any representation of $\SU(2)$, just replace the
$\epsilon^{\widehat{I}}_{\phantom{I}\widehat{J}\widehat{K}}$
``generators'' factor with $\tau_{\widehat{K}}$.

This is more general than curvature, we're not doomed to infinitesimal nightmares.

We see, taking care with ordering due to noncommutativity, that:
\begin{equation}\label{eq:2009-05-01:enlightenment}
\frac{\D}{\D s}\mathcal{U}(s,s_{0}) = -A(s)\,\mathcal{U},
\end{equation}
and similarly,
\begin{equation}
\frac{\D}{\D s_{0}}\mathcal{U}(s,s_{0}) = -\mathcal{U}A(s_{0}).
\end{equation}
Let $\widetilde{\mathcal{U}} = g(s)^{-1}\mathcal{U}g(s_{0})$, so we
have:
\begin{subequations}
\begin{align}
\frac{\D}{\D s}\widetilde{\mathcal{U}}
&= - g^{-1}(s)\frac{\D g(s)}{\D s}\widetilde{\mathcal{U}}-g^{-1}(s)A\mathcal{U}g(s_{0})\\
&= \left(-g^{-1}Ag - g^{-1}\frac{\D g}{\D s}g\right)\widetilde{\mathcal{U}}\\
&= - \widetilde{A}\,\widetilde{\mathcal{U}}.
\end{align}
\end{subequations}
We read this backwards, when
\begin{equation}
A\to g^{-1}\,\D g + g^{-1}Ag,
\end{equation}
we simply have
\begin{equation}
\mathcal{U}(s,s_{0})\to g^{-1}(s)\mathcal{U}(s,s_{0})g(s_{0}).
\end{equation}
\textbf{Note:} redo the computations starting from Eq~\eqref{eq:2009-05-01:enlightenment}
for enlightening insight.
\index{Holonomy|)}

In particular, for a closed curve $C$, we find
$\Tr(\mathcal{U}(s,s_{0}))$ is gauge invariant. This is the Wilson
loop,\index{Wilson Loop} and it gives us a complete set of gauge
invariant variables for a gauge theory.

There is one problem with this, the Wilson loops give an overcomplete
set of variables (i.e., they're not all linearly independent of each
other, due to the Mandelstam identities):
\begin{equation}
\mathcal{U}_{C_{1}}\mathcal{U}_{C_{2}}=\mathcal{U}_{C_{1}\circ C_{2}} + \mathcal{U}_{C_{1}\circ C_{2}^{-1}},
\end{equation}
where $C_{1}$, $C_{2}$ are closed curves sharing a point, as doodled
below:
\begin{center}
  \includegraphics{img/2009-05-01.0}
\end{center}
There's a nice basis called the \define{Spin Network Basis}, and we can
claim to solved 3 of the constraints of quantum gravity.

\lecture

Consider a wave functional $\Psi[\varphi(x)]$ in a Schrodinger type
picture in quantum field theory. Consider infinitesimal deformations of
the field
\begin{equation}
\Psi[\varphi(x)+\varepsilon(x)] = \Psi[\varphi(x)] + \int\frac{\delta\Psi}{\delta\varphi}(x_{1})\varepsilon(x_{1})\,\D^{n}
x_{1} + \bigOh(\varepsilon^{2}).
\end{equation}
If the field is invariant under $\varphi\to\varphi+\widetilde{\varepsilon}$,
then
\begin{equation}
\Psi[\varphi+\widetilde{\varepsilon}\,]=\Psi[\varphi],
\end{equation}
and moreover
\begin{equation}
\int\frac{\delta\Psi}{\delta\varphi}(x_{1})\widetilde{\varepsilon}(x_{1})\,\D^{n} x_{1} = 0.
\end{equation}
This is useful for computing vacuum expectation values.

Suppose we have a constraint. For us, we have the Gauss Law constraint
\begin{equation}
D_{i}\widetilde{E}^{i\widehat{I}}
=\partial_{i}\widetilde{E}^{i\widehat{I}} + \epsilon^{\widehat{I}\widehat{J}\widehat{K}}A_{i\widehat{J}}\widetilde{E}^{i}_{\phantom{i}\widehat{K}}=0,
\end{equation}
where in the Schrodinger picture we have,
\begin{equation}
\widetilde{E}^{i\widehat{I}} = -8\pi\immirzi G_{N}\hbar\frac{\delta}{\delta A_{i\widehat{I}}}.
\end{equation}
The constraint is linear in functional derivatives. What we can do is
look at the integral of the constraint against a test function,
\begin{equation}
\int\lambda_{\widehat{I}}D_{i}\widetilde{E}^{i\widehat{I}}\,\D^{n}x=0.
\end{equation}
If this is true for arbitrary $\lambda_{\widehat{I}}$, then integration
by parts
\begin{equation}
\int\lambda_{\widehat{I}}D_{i}\widetilde{E}^{i\widehat{I}}\,\D^{n}x
= 8\pi G_{N}\hbar\immirzi\int
D_{i}\lambda_{\widehat{I}}\frac{\delta}{\delta A_{i\widehat{I}}}\,\D^{n}x.
\end{equation}
Our constraint is then, when applied to a wave functional,
\begin{equation}
8\pi G_{N}\hbar\immirzi\int
D_{i}\lambda_{\widehat{I}}\frac{\delta}{\delta A_{i\widehat{I}}}\,\D^{n}x\,\Psi[A]=0.
\end{equation}
This is the first term in a Taylor expansion
\begin{equation}
\Psi[{A_{i}}^{\widehat{I}} + D_{i}\lambda^{\widehat{I}}] = \Psi[{A_{i}}^{\widehat{I}}].
\end{equation}
We can also have gauge transformations not built up from infinitesimal
transformations (e.g., time reversal) called \define{Large Gauge Transformations}.

We get to the Wilson line (a.k.a., the parallel propagator), we have the
holonomy
\begin{equation}
\mathcal{U}^{\widehat{J}}_{\phantom{J}\widehat{K}}
  = \mathcal{P}\exp\left[-\int_{C}A_{i}^{\phantom{i}\widehat{J}}{{\epsilon_{\widehat{I}}}^{\,\widehat{J}}}_{\!\widehat{K}}\, \D x^{i}\right],
\end{equation}
or suppressing indices and letting $\tau_{\widehat{I}}$ be the
generators of the gauge algebra,
\begin{equation}
\mathcal{U}
= \mathcal{P}\exp\left[-\int_{C}A_{i}^{\phantom{i}\widehat{I}}\tau_{\widehat{I}}\, \D x^{i}\right].
\end{equation}
Observe this transforms under change of ``coordinates'' as
\begin{equation}
\mathcal{U}\to g^{-1}(s_{2})\,\mathcal{U}\, g(s_{1}).
\end{equation}
We wish to construct invariants, so we construct closed loops then take
the trace of the holonomy $\mathcal{U}$ over the loop. This is an
overcomplete set of variables.

\begin{wrapfigure}{R}{7pc}
\centering\vskip-1pc
\includegraphics{img/2009-05-06.0}
\end{wrapfigure}
The way out is to consider the intersection, as doodled to the right.
We wish to consider this in detail. Give each edge of the graph
depicting the intersection a representation of $\SU(2)$.
We assign the vertex a Clebsch-Gordon coefficient. If we generalize this
to $n$-edges meeting at a vertex, we can use an intertwiner instead of a
Clebsch--Gordon coefficient. For $\SU(2)$, we have a neatway to combine
things as vertices:
\begin{center}
  \includegraphics{img/2009-05-06.1}
\end{center}
This works for $\SU(2)$, it may not necessarily work for an arbitrary
gauge group. So in short:
\begin{itemize}
\item At each node, we have an intertwiner;
\item At each edge, we have a representation of $\SU(2)$.
\end{itemize}

\begin{wrapfigure}{l}{8pc}
\centering %\vskip-1pc
\includegraphics{img/2009-05-06.2}
\end{wrapfigure}
The spin network (generically doodled to the left) gives a complete (but
not overcomplete) basis of states. Loop quantum gravity theorists like
to say a spin network is a state. What they mean is: the spin network is
a function of the connection $A$, which is all a state \emph{is} in
quantum gravity. A spin network eats in a value of $A$ and spits out a
complex number. One could use this for computation in, e.g., QCD (this
is group field theory).

We don't need a smooth connection, we can \emph{generalize} the
connection so it gives Wilson lines along finite parts of space (it
could be only where the edges are in fact). If we take the space of all
connections modulo gauge transformations, and complete it (so it's a
Hilbert space), then that's the Hilbert space we use.

Since a spin network is a state, we should probably define an inner
product between two spin networks. We should consider the usual way to
define the inner product on the Hilbert space just described as
something like
\begin{equation}
\langle\Psi\mid\Phi\rangle\sim\int_{\mathcal{A}/\mathcal{G}}\Psi^{*}[A]\Phi[A]\,[\D A].
\end{equation}
This is the only gauge-invariant inner product. We can get close to a
delta function, specifying geomtries down to the Planck length, using
weave states.

\bigbreak
Note: the spin networks doodled below,
\begin{center}
  \includegraphics{img/2009-05-06.3}
\end{center}
are distinct, since the lines are Wilson lines, the integral \emph{changes}.

\lecture

We need the spatial topology to not change, otherwise we can end up with
a number of baby universes (``polymer topology'').

Let us consider the simplest spin network, we need 2 nodes and 3
edges. (If we had 2 edges, then we would obtain the identity spin
network.)
We don't have spin 0, as the propagator would be the identity.
\begin{center}
  \includegraphics{img/2009-05-08.1}
\end{center}
For each of these lines we have the Wilson line
\begin{equation}
\mathcal{U} = \mathcal{P}\exp[-\int A].
\end{equation}
We have three Wilson lines ${\mathcal{U}^{m_{1}}_{1}}_{n_{1}}$,
${\mathcal{U}^{m_{2}}_{2}}_{n_{2}}$, and ${\mathcal{U}^{m_{3}}_{3}}_{n_{3}}$.
Conceptually, the $\mathcal{U}$'s tell spin-$1/2$ objects rotate in spin
space. We have $m_{1}=1/2, -1/2$ for spin up and spin down (respectively).
We see $m_{2}$ also describes a spin-$1/2$ object, but $m_{3}$ describes
a spin-1 object with possible values $m_{3}=-1,0,+1$. We can now use the
Clebsch--Gordon coefficients $\langle j\,m\mid j_{1}\,m_{1}\,, j_{2}\,m_{2}\rangle$
and find
\begin{equation}
  \sum_{\substack{m_{1},m_{2},m_{3}\\ n_{1},n_{2},n_{3}}}
  {\mathcal{U}^{m_{1}}_{1}}_{n_{1}}{\mathcal{U}^{m_{2}}_{2}}_{n_{2}}{\mathcal{U}^{m_{3}}_{3}}_{n_{3}}
\langle1\,m_{3}\mid\frac{1}{2}\,m_{1}\,\frac{1}{2}m_{2}\rangle
\langle1\,n_{3}\mid\frac{1}{2}\,n_{1}\,\frac{1}{2}n_{2}\rangle,
\end{equation}
which is a function of $A$.

\subsection{Area Operator}

We take a surface $\Sigma$, we can ask ``What is the area of the
surface?'' Suppose we have some spin network that ``goes through'' our
surface $\Sigma$:
\begin{center}
  \includegraphics{img/2009-05-08.2}
\end{center}
We won't consider an edge of the spin network ``grazing'' the surface,
or lies inside the surface: we care about puncturing edges.

We will only really consider a simple example choosing a surface where
$x^{3}=0$, the area of the surface would be classically
\begin{equation}
A = \int\sqrt{{}^{(2)}g}\D^{2}x.
\end{equation}
We see
\begin{equation}
{}^{(2)}g = g_{11}g_{22} - 2(g_{12})^{2} = \widetilde{E}^{3}_{\widehat{I}} \widetilde{E}^{3\widehat{I}}.
\end{equation}
So the area is
\begin{equation}
A = \int\sqrt{\widetilde{E}^{3}_{\widehat{I}} \widetilde{E}^{3\widehat{I}}}\D^{2}x.
\end{equation}
Consider a more general surface with coordinates $\sigma^{1}$, $\sigma^{2}$.
Then our considerations change by
\begin{equation}
\widetilde{E}^{3}_{\phantom{3}\widehat{I}}\to\epsilon_{ijk}\frac{\partial x^{i}}{\partial\sigma^{1}}\frac{\partial x^{j}}{\partial\sigma^{2}}\widetilde{E}^{k}_{\phantom{k}\widehat{I}}.
\end{equation}
In the classical arena, the criteria for a ``small region'' is not
really well-defined; in the quantum arena, we just require a single
piercing of a spin network with the surface (as doodled in the margin).\marginpar{\includegraphics{img/2009-05-08.3}} We can turn this now into an
operator
\begin{equation}
\widetilde{E}_{\widehat{I}} = -8\pi G_{N}\immirzi\int\epsilon_{ijk}\frac{\partial x^{i}}{\partial\sigma^{1}}\frac{\partial x^{j}}{\partial\sigma^{2}}\frac{\delta}{\delta
A^{\widehat{I}}_{k}}\,\D\sigma^{1}\D\sigma^{2}.
\end{equation}
We need to consider $\delta\mathcal{U}/\delta A^{\widehat{I}}_{k}$. If
we didn't have path-ordering, then this would be trivial. But we must be
careful, since things do not commute. Consider the path doodled below:
\begin{center}
  \includegraphics{img/2009-05-08.4}
\end{center}
We have
\begin{subequations}
  \begin{align}
    \mathcal{U}(s_{2},s_{1})
    &= \mathcal{U}(s_{2},s)\mathcal{U}(s,s_{1})\\
    &= \mathcal{U}(s_{2},s+\varepsilon)\mathcal{U}(s+\varepsilon,s-\varepsilon)\mathcal{U}(s-\varepsilon,s_{1}).
  \end{align}
\end{subequations}
If $\varepsilon$ is small enough, we have
\begin{equation}
\frac{\delta}{\delta A^{\widehat{I}}_{i}(s)}\mathcal{U}(s_{2},s_{1})
= \mathcal{U}(s_{2},s)\left(-\tau_{\widehat{I}}\frac{\D x^{i}}{\D s}\right)\mathcal{U}(s,s_{1}).
\end{equation}
More generally,
\begin{equation}
\frac{\delta}{\delta A^{\widehat{I}}_{i}(x)}\mathcal{U}(s_{2},s_{1})
= \int\delta^{(3)}(C(s)-x)\,\mathcal{U}(s_{2},s)\left(-\tau_{\widehat{I}}\frac{\D x^{i}}{\D s}\right)\mathcal{U}(s,s_{1})\,\D s,
\end{equation}
and is zero if $x$ does not lie on the curve $C$.
We now see that
\begin{equation}
E_{\widehat{I}}\mathcal{U}(s_{2},s_{1}) = 8\pi\immirzi
G_{N}\int\epsilon_{ijk}\frac{\partial
    x^{i}}{\partial\sigma^{1}}\frac{\partial
    x^{j}}{\partial\sigma^{2}}\frac{\partial x^{k}}{\partial s}\delta^{(3)}(C(s)-x)\,\mathcal{U}(s_{2},s)\tau_{\widehat{I}}\,\mathcal{U}(s,s_{1})\,\D\sigma^{1}\D\sigma^{2}\D s.
\end{equation}
We see that
\begin{equation*}
\int\epsilon_{ijk}\frac{\partial
    x^{i}}{\partial\sigma^{1}}\frac{\partial
    x^{j}}{\partial\sigma^{2}}\frac{\partial x^{k}}{\partial s}\delta^{(3)}(C(s)-x)\,\D\sigma^{1}\D\sigma^{2}\D s
\end{equation*}
is called the ``oriented intersection number'' (it's $\pm1$ if $C(s)$
intersects $\Sigma$, and $0$ otherwise).

The moral of the story is that the oriented intersection number
$I(C,\Sigma)$ is used to find
\begin{equation}
E_{\widehat{I}}\,\mathcal{U}(s_{2},s_{1}) =
8\pi G_{N}\immirzi I(C,\Sigma)\mathcal{U}(s_{2},s)\tau_{\widehat{I}}\,\mathcal{U}(s,s_{1}),
\end{equation}
where $C(s)$ is the point of intersection.

\lecture

The Hamiltonian constraint gives surface deformations. So far we have
not looked at the momentum or Hamiltonian constraints. It's like
Einstein's field equations are missing. We have
\begin{equation}
V_{i} = F^{\widehat{I}}_{ij}\widetilde{E}^{j}_{\widehat{I}}=0.
\end{equation}
Let us first consider a trick useful in many circumstances. Consider
\begin{subequations}
\begin{align}
\xi^{i}F_{ij}^{\widehat{I}}
&= \xi^{i}(\partial_{i}A^{\widehat{I}}_{j} -
    \partial_{j}A^{\widehat{I}}_{i} + \epsilon^{\widehat{I}\widehat{M}\widehat{N}}A_{i\widehat{M}}A_{j\widehat{N}})\\
&= \xi^{i}\partial_{i}A^{\widehat{I}}_{j} - \partial_{j}(\xi^{i}A^{\widehat{I}}_{i}) +
    (\partial_{j}\xi^{i})A^{\widehat{I}}_{i} + \epsilon^{\widehat{I}\widehat{M}\widehat{N}}\xi^{i}A_{i\widehat{M}}A_{j\widehat{N}}\\
&=
    \underbrace{\xi^{i}\partial_{i}A^{\widehat{I}}_{j}+(\partial_{j}\xi^{i})A^{\widehat{I}}_{i}}_{\substack{\text{change of }
        A\text{ under an}\\
        \text{infinitesimal change of}\\
        \text{coordinates}}} - D_{j}(\xi^{i}A_{i}^{\widehat{I}})\\
&= \underbrace{\delta_{\xi}A^{\widehat{I}}_{i}}_{\text{Lie derivative}} - D_{j}(\xi^{i}A_{i}^{\widehat{I}}).
\end{align}
\end{subequations}
The moral is that a vector contracted with the field strength tensor
looks liek a covariant derivative and a diffeomorphism (gauge
transformation). So we have
\begin{subequations}
\begin{equation}
\int\xi^{i}V_{i}\,\D^{3}x\sim\int(-D_{i}(\xi^{i}A^{\widehat{I}}_{i}) +
\delta_{\xi}A^{\widehat{I}}_{i})\frac{\delta}{\delta A^{\widehat{I}}_{i}}\D^{3}x,
\end{equation}
which acts on fields like:
\begin{equation}
A\to A - D\varepsilon + \delta_{\xi}A,
\end{equation}
\end{subequations}
where $\varepsilon^{\widehat{I}} = \xi^{i}A^{\widehat{I}}_{i}$.
This is exactly analogous to the statement in quantum mechanics that
$\widehat{p}$ generates translations in position.

\lecture

References for loop quantum gravity.
\begin{itemize}
\item About Spin Foams\begin{itemize}
\item J.~C.~Baez,
``An Introduction to Spin Foam Models of $BF$ Theory and Quantum Gravity''.
\journal{Lect. Notes Phys.} \volume{543} (2000), 25-93;
\arXiv{gr-qc/9905087}\\
{\tt\doi{10.1007/3-540-46552-9_2}}
\item A.~Perez,
``Spin foam models for quantum gravity''.
\journal{Class. Quant. Grav.} \textbf{20} (2003) R43;
\arXiv{gr-qc/0301113}\\
{\tt\doi{10.1088/0264-9381/20/6/202}}
\end{itemize}
\item Group Field Theory
  \begin{itemize}
  \item D.~Oriti,
``The Group field theory approach to quantum gravity''.
\arXiv{gr-qc/0607032}.
  \end{itemize}
\item Critique of Loop Quantum Gravity:
  \begin{itemize}
  \item H.~Nicolai and K.~Peeters,
``Loop and spin foam quantum gravity: A Brief guide for beginners''.
\journal{Lect.\ Notes Phys.} \textbf{721} (2007) 151-184;
\arXiv{hep-th/0601129}.\\
{\tt\doi{10.1007/978-3-540-71117-9_9}}
  \end{itemize}
\item Covariant Canonical Quantization
  \begin{itemize}
  \item A.~Ashtekar, L.~Bombelli and O.~Reula,
``The Covariant Phase Space of Asymptotically Flat Gravitational Fields''.
In \textit{Mechanics, Analysis and Geometry: 200 Years After Lagrange},
pp.417--450, Elsevier, 1991. \\
{\tt\doi{10.1016/B978-0-444-88958-4.50021-5}}.
  \end{itemize}
\end{itemize}

\subsection{String Theory}

A very brief introduction to string theory, but we'll focus on its
relevance to gravity. There are several points to consider
\begin{enumerate}[nosep,label=(\arabic*)]
\item Closed loops have a massless spin-2 excitation (``graviton'')
\item Strings propagate only in a spacetime satisfying the Einstein
  field equations (plus some negligible corrections)
\end{enumerate}\medbreak\noindent\ignorespaces%
(Any theory with self-interacting spin-2 massless excitations is a hint
that gravity is in the game.)\medbreak
\begin{enumerate}[resume*]
\item Background spacetime of the second point is equivalent to a
  coherent state of excitations of the first point.
\end{enumerate}\medbreak

%\marginnote{{\footnotesize Review of classical spin-2 ``gravitons''}}
\marginnote{Review of classical spin-2 ``gravitons''}
Let us examine the first point. We have the basic field be some tensor
with two indices $h_{\mu\nu}$ and the field equations look like:
\begin{equation}
\Box h_{\mu\nu} + (\mbox{terms involving }\partial_{\rho}h^{\rho\sigma})
= T_{\mu\nu}.
\end{equation}
The most general result is that we end up with a spin-2 part, a vector
(spin-1) part and a scalar (spin-0) part. We need these extra (vector
and scalar) parts vanish, which is a gauge choice (analogous to a spin-1
field $\Box A_{\mu} + k\partial_{\mu}\partial_{\nu}A^{\nu}=J_{\mu}$
requires $\partial^{\mu}J_{\mu}$, otherwise we do not have
electromagnetism). The gauge invariance for us is
\begin{equation}
h_{\mu\nu}\to h_{\mu\nu} + \partial_{\mu}\xi_{\nu} + \partial_{\nu}\xi_{\mu}
\end{equation}
which demands
\begin{equation}
\partial_{\mu}T^{\mu\nu}=0
\end{equation}
for consistency.
We can choose gauge $\partial_{\rho}h^{\rho\sigma}=0$ (Lorenz gauge,
Harmonic gauge, de Donder gauge, Fock gauge, Harmonic gauge, Feynman
gauge, Lorenz gauge, etc.). The reason harmonic is sometimes we write $\Box X^{\mu}$
where $X^{\mu}$ is some parametrized version of the coordinates which
individually transform as scalars (and $\Box$ uses derivatives with
respect to the parameters).

For consistency, after choosing the gauge, we add contributions of
order $h^{2}$ to the stress-energy tensor to the right-hand side
\begin{equation}
\Box h_{\mu\nu} = T_{\mu\nu} + T^{(h)}_{\mu\nu}
\end{equation}
but then we will need to add cubic interactions, and then quartic
interactions, etc. Deser showed (reprinted as \arXiv{gr-qc/0411023}),
using an incredibly clever choice of variables, the series terminates.
Damour and Henneaux (\arXiv{hep-th/0007220} and \arXiv{hep-th/0009109})
used clever arguments to show there are some extra terms using
cohomological techniques. This work done by Deser, Damour and Henneaux,
are entirely classical, but there are some soft graviton theorems.

\marginnote{Strings, Worldsheet}%\marginnote{{\footnotesize Strings, Worldsheet}}
Let's start with string theory, we will start by talking about strings.
Open strings trace out a 2-dimensional surface with intrinsic
coordinates $(\sigma,\tau)=(\sigma^{0},\sigma^{1})$. We can write the
4-coordinates of the surface in terms of $\sigma$ and $\tau$.
\begin{center}
\includegraphics{img/2009-05-15.0}
\end{center}
We can model interactions using pant diagrams, for example:
\begin{center}
\includegraphics{img/2009-05-15.1}\includegraphics{img/2009-05-15.2}\includegraphics{img/2009-05-15.3}
\end{center}
These are the only possible interactions. These are, of course, in a
fixed background. The common statistic is that there are $10^{500}$
possible backgrounds.
We write the metric for this background as $G_{\mu\nu}$ and the induced
metric on the worldsheet is,
\begin{equation}
h_{ab} = G_{\mu\nu}\frac{\partial X^{\mu}}{\partial\sigma^{a}}\frac{\partial X^{\nu}}{\partial\sigma^{b}}.
\end{equation}
The generalization of the action for a world line is the area of the
worldsheet,\marginnote{Nambu--Goto Action\\Polyakov Action}
\begin{equation}
\action = \frac{-1}{2\pi\alpha'}\int\sqrt{-h}\,\D^{2}\sigma.
\end{equation}
This is the Nambu--Goto action.

Quantizing the Nambu--Goto action turns out to be quite difficult, which
we should expect with any Lagrangian involving the squareroot of
quadratic terms. This leads us to consider the generalization of this
action, the Polyakov action, which requires us to use a new induced
metric $\gamma_{ab}$,
\begin{equation}
\action=\frac{-1}{4\pi\alpha'}\int\gamma^{ab}\partial_{a}X^{\mu}\partial_{b}X^{\nu}G_{\mu\nu}\sqrt{-\gamma}\,\D^{2}\sigma.
\end{equation}
Classically this is equivalent to the Nambu--Goto action, where the
metric $\gamma$ behaves as a Lagrange multiplier (since its derivatives
do not appear in the action). We see varying the action with respect to
$\gamma$ yields
\begin{equation}
\partial_{a}X^{\mu}\partial_{b}X^{\nu}G_{\mu\nu} - \frac{1}{2}\gamma_{ab}\gamma^{cd}\partial_{c}X^{\mu}\partial_{d}X^{\nu}G_{\mu\nu}=0.
\end{equation}
We can solve this equation for $\gamma_{ab}$ to write the induced metric
as
\begin{equation}
\gamma_{ab}=2f(\sigma^{0},\sigma^{1})\partial_{a}X^{\mu}\partial_{b}X^{\nu}G_{\mu\nu}
\end{equation}
where
\begin{equation}
\frac{1}{f(\sigma^{0},\sigma^{1})} = \gamma^{cd}\partial_{c}X^{\mu}\partial_{d}X^{\nu}G_{\mu\nu}.
\end{equation}
This action has Weyl symmetry,
\begin{equation}
\gamma_{ab}\to \Omega(\sigma^{c})\gamma_{ab},
\end{equation}
where $\Omega$ is everywhere positive. We see that the Polyakov action
may be rewritten as,
\begin{equation}
\action=\frac{-1}{4\pi\alpha'}\int\frac{\sqrt{-\gamma}}{f(\sigma^{0},\sigma^{1})}\,\D^{2}\sigma.
\end{equation}
Taking advantage of the fact $f$ is everywhere positive, we see we can
classically recover the Nambu--Goto action by the Weyl symmetry
transformation $\gamma_{ab}\to\gamma_{ab}/\sqrt{f}$. Polchinski proved
the two actions are the same quantum mechanically. %% We can look at its
%% four-dimensional position of the worldsheet $X^{\mu}$
We can look at the string worldsheet as fundamental, then view the
$X^{\mu}$ living on the worldsheet, and four-dimensional spacetime
``emerges''.\footnote{Ed Witten's ``Reflections on the Fate of
Spacetime'' (\journal{Physics Today}, April 1996, pp.24--30) argues this
heuristically; Nick Huggett and Christian W\"{u}thrich's ``Out of
Nowhere: The `emergence' of spacetime in string theory''
(\arXiv{2005.10943}) reviews this general subject.}


% \includegraphics{img.0}
\appendix\let\oldvec\vec
\let\vec\mathbf\setlength{\parskip}{1ex}
\section[Homework I]{Introduction to Quantum Gravity: Homework I\footnote{This was handed out April 2, 2009.}}


I. \textbf{Planck scale}

The planck length is $\ell_{Pl} = \sqrt{\hbar G/c^{3}}$.
\begin{enumerate}[label=(\alph*),nosep]
\item Suppose you wish to probe an area of characteristic size $R$ with
  a relativistic particle (that is, one for which $E\sim pc$). Consider
  the following two restrictions
  \begin{itemize}
  \item[--] uncertainty relation: $\Delta x\Delta p\gtrsim\hbar$
  \item[--] no black hole formed by problem: $G\Delta E/R c^{4}\lesssim 1$.
  \end{itemize}
  Find an estimate of the smallest possible value of $R$. How would this
  change if you allow a \emph{nonrelativistic} probe (that is, a massive
  probe with $mc^{2}\gg pc$)?
\item Consider a piece of matter of energy $E$ that is not already a
  black hole. Its size must be greater than its Compton wavelength
  (quantum mechanics) and also large enough that it is not a black hole
  (general relativity). Approximately what is its minimum size?
\item Recall that for any two quantum mechanical observables
  $\widehat{A}$ and $\widehat{B}$ ann uncertainty principle holds
  \begin{equation*}
\Delta\widehat{A}\Delta\widehat{B}\geq\frac{1}{2}|\langle[\widehat{A},\widehat{B}]\rangle|
  \end{equation*}
  For a free particle (in the Heisenberg picture) with position operator
  $\widehat{x}(t)$ and momentum operator $\widehat{p}$,
  \begin{equation*}
    \widehat{x}(t) = \widehat{x}(0) + \frac{t}{m}\widehat{p}
  \end{equation*}
  and $[\widehat{x}(0),\widehat{p}]=\I\hbar$. Assuming that $\Delta x(t)$
  is of the same order as $\Delta x(0)$, find its minimum value as a
  function of $t$ and $m$. (This is closely related to what is known as
  the ``standard quantum limit''.)

  Now consider measuring a distance $L$ between two points by sending a
  particle from one to the other and timing its motion. By relativity,
  we must have $L\leq ct$. If the particle is too massive, the two
  points we are measuring will be inside a black hole; to avoid this we
  need $Gm/Lc^{2}\lesssim1$. Find the resulting limit on $\Delta x$
  on the accuracy to which we can measure $L$.
\item Find loopholes in these arguments.
\end{enumerate}
\medbreak\noindent{}II. \textbf{Van Hove's theorem}\medbreak

In the Hamiltonian formalism, a classical dynamical system typically has
a phase space that is (at least locally) parametrized by (generalized)
positions $\vec{q}$ and momenta $\vec{p}$. The basic rule in
quantization is that ``Poisson brackets become commutators.'' One way to
express this is by a quantization map $Q$ from functions of the phase
space ($f(\vec{q},\vec{p})$, $g(\vec{q},\vec{p})$, etc.) to operators on
a Hilbert space. Since we're physicists, we'll denote the action of $Q$
by adding a ``hat'': $Q(f)=\widehat{f}$. An obvious set of conditions
for $Q$ is:
\begin{enumerate}[nosep,label=(\arabic*)]
\item $Q(af+bg)=aQ(f)+bQ(g)$ (linearity)
\item $Q(1)=1$
\item $Q(\vec{x})$ and $Q(\vec{p})$ are represented irreducibly
\item $[Q(f),Q(g)]=\I\hbar Q(\{f,g\})$ (where $\{f,g\}$ is the Poisson bracket)
\end{enumerate}

Van Hove's theorem say that this is not possible for a particle moving
in one dimension. Prove this.

Hint: In the phase space, one has hat $p^{2}q^{2} = -\frac{1}{9}\{p^{3},q^{3}\}$
and $p^{2}q^{2} = -\frac{1}{3}\{p^{2}q,q^{2}p\}$. Show that these give
different values for $Q(p^{2}q^{2})$. (To do this mathematically
rigorously, you will have to use the irreducibility condition (3), which
implies that if $[\widehat{q},\widehat{O}]=0$ and
$[\widehat{p},\widehat{O}]=0$ for some operator $\widehat{O}$, then
$\widehat{O}$ is proportional to the identity, that is, $\widehat{O}$ is
a number. Most ``physicists' proofs'' don't pay too much attention to
this.)

Note: ``deformation quantization'' replaces condition 4 by
\begin{itemize}
\item[($4'$)] $[Q(f),Q(g)]=\I\hbar Q(\{f,g\}) + (\mbox{terms of order }\hbar^{2})$
\end{itemize}


\medbreak\noindent{}III. \textbf{Affine commutators}\medbreak

\begin{enumerate}[label=(\alph*),nosep]
\item Show that if $[\widehat{q},\widehat{p}]=\I\hbar$, then
  $\widehat{q}$ generates translations in $\widehat{p}$, that is,
  \begin{equation*}
\E^{-\I a\widehat{q}/\hbar}\widehat{p}\E^{\I a\widehat{q}/\hbar}=\widehat{p}+a.
  \end{equation*}
\item Suppose that $[\widehat{q},\widehat{p}]=\I\hbar$. Show that if
  there is any state that is an eigenfunction (or a generalized
  eigenfunction---that is, the state need not be normalizable) of
  $\widehat{p}$, then \emph{all} real numbers appear as eigenvalues of $\widehat{p}$.
\item Suppose the fundamental operators are instead $\widehat{q}$ and
  $\widehat{D}=\widehat{qp}$ with
  $[\widehat{q},\widehat{D}]=\I\hbar\widehat{q}$. Show that
  $\widehat{D}$ generates dilatations, that is,
  \begin{equation*}
\E^{-\I a\widehat{D}/\hbar}\widehat{p}\E^{\I a\widehat{D}/\hbar}=\E^{a}\widehat{p}.
  \end{equation*}
\item With affine commutators, show that if there is any state that is
  an eigenfunction of $\widehat{p}$ with positive eigenvalue, then all
  positive real numbers appear as an eigenvalue of $\widehat{p}$.
\end{enumerate}

\section[Homework II]{Introduction to Quantum Gravity: Homework II\footnote{This was handed out April 10, 2009.}}



I.~Electrodynamics as a constrained system.
\medbreak
Classical electrodynamics is described by a four-vector potential
$A_{\mu}$ and an antisymmetric field strength tensor
$F_{\mu\nu}=\partial_{\mu}A_{\nu}-\partial_{\nu}A_{\mu}$. The Lagrangian
is
\begin{equation*}
L = \frac{1}{4}F_{\mu\nu}F^{\mu\nu} + A_{\mu}\mathcal{J}^{\mu}
\end{equation*}
where $\mathcal{J}^{\mu}$ is the current four-vector, and the ordinary
electric and magnetic fields are
\begin{equation*}
E^{i}=F^{0i},\qquad B_{i}=\frac{1}{2}\epsilon_{ijk}F^{jk}
\end{equation*}
where $i$, $j$, $k$, \dots run from $1$ to $3$ and $\mu$, $\nu$, \dots
run from $0$ to $3$.

The tensor $F_{\mu\nu}$ is invariant under gauge transformations
\begin{equation*}
A_{\mu}\to A_{\mu}+\partial_{\mu}\Lambda
\end{equation*}
This suggests the system has constraints (to generate the gauge
transformations).

\begin{enumerate}[label=(\alph*),nosep]
  \item If you are not familiar with this formalism, convince yourself
    that it really does give you Maxwell's equations.
  \item Show the canonical variables (``generalized positions and
    momenta'') are $A_{i}$ and $E^{j}$, with Poisson brackets
    $\{A_{i}(\vec{x}),E^{j}(\vec{y})\}=\delta^{j}_{i}\delta^{3}(\vec{x}-\vec{y})$.
  \item Show that $A_{0}$ is a Lagrange multiplier, and the
    corresponding constraint is the Gauss law $C=\vec{\nabla}\cdot\vec{E}=0$.
  \item Show the constraint generates the gauge transformations of
    $A_{i}$, that is, $\{\int\Lambda(\vec{x})C(\vec{x})\,\D^{3}x, A_{i}(\vec{x}')\}\sim(A_{i}+\partial_{i}\Lambda)(\vec{x}')$.
  \item Correct any mistakes in algebra I have made.
\end{enumerate}


\medbreak\noindent{}II. \textbf{Extrinsic Curvature: Symmetry}\medbreak

Let $n^{a}$ be the unit normal to a $t=const.$ hypersurface. (Note this
implies $n_{a}=f\nabla_{a}t$ for some function $f$. Why?) Recall the
extrinsic curvature tensor is
\begin{equation}
K_{ab} = {q_{a}}^{c}\nabla_{c}n_{b}
\end{equation}
where $g_{ab}=q_{ab}+n_{a}n_{b}$.

\begin{enumerate}[label=(\alph*),nosep]
\item Show $K_{ab}=K_{ba}$.

  (Hint: use the fact $n_{a}=f\nabla_{a}t$ and find
  $n^{c}\nabla_{c}n^{b}$ in terms of $f$. The vector
  $a^{b}=n^{c}\nabla_{c}n^{b}$ is sometimes called the ``acceleration''.)
\item Show $K=\nabla_{a}n^{a}$ (where $K={K^{a}}_{a}$).
\end{enumerate}

\medbreak\noindent{}III. \textbf{Gauss--Codazzi}\medbreak

A ``spatial'' tensor ${T^{ab\dots}}_{cd\dots}$ is one with no normal
components, i.e.,
\begin{equation*}
n_{a}{T^{ab\dots}}_{cd\dots} = n_{b}{T^{ab\dots}}_{cd\dots} =
n^{c}{T^{ab\dots}}_{cd\dots} = \dots = 0
\end{equation*}
${q^{a}}_{b}$ is a projection operator, that is, it projects any index
into a ``purely spatial'' one. (Why?)

The three-dimensional (``spatial'') covariant derivative $D_{a}$ of a
spatial tensor can be defined as
\begin{equation*}
D_{e}{T^{ab\dots}}_{cd\dots} = {q^{a}}_{g}{q^{b}}_{h}\dots{q_{c}}^{i}{q_{d}}^{j}\dots{q_{e}}^{f}\nabla_{f}{T^{gh\dots}}_{ij\dots}
\end{equation*}
(that is, we take the ordinary four-dimensional covariant derivative and
then project all indices onto the $t=const.$ slice). The spatial
curvature is defined by the condition
\begin{equation*}
[D_{a},D_{b}]v_{c}={}^{(3)}\!{R_{abc}}^{d}v_{d}
\end{equation*}
for any spatial vector $v_{d}$. Using these facts, show that
\begin{enumerate}[label=(\alph*),nosep]
\item ${}^{(3)}\!{R_{abcd}} = {q_{a}}^{e}{q_{b}}^{f}{q_{c}}^{g}{q_{d}}^{h}R_{efgh}+K_{ac}K_{bd}-K_{ad}K_{bc}$
\item $R_{abcd}n^{d}=\nabla_{a}K_{bc} - \nabla_{b}K_{ac} - n_{a}(\nabla_{b}a_{c}-a_{b}a_{c}) + n_{b}(\nabla_{a}a_{c}-a_{a}a_{c})$
\item $R = {}^{(3)}\!R - K_{ab}K^{ab} + K^{2} + 2\nabla_{a}(n^{b}\nabla_{b}n^{a} - n^{a}\nabla_{b}n^{b})$
\end{enumerate}
Note that sign conventions differ reference to reference.

\medbreak\noindent{}IV. \textbf{ADM metric and extrinsic curvature}\medbreak

Suppose we decompose the metric in ADM form as
\begin{equation*}
\D s^{2} = N^{2}\,\D t^{2} - q_{ij}(\D x^{i} + N^{i}\,\D t)(\D x^{j} + N^{j}\,\D t)
\end{equation*}
\begin{enumerate}[label=(\alph*),nosep]
\item Confirm that the inverse metric is
  $$g^{ab} = \begin{pmatrix}\frac{1}{N^{2}} & -\frac{N^{i}}{N^{2}}\\-\frac{N^{j}}{N^{2}} & -q^{ij} + \frac{N^{i}N^{j}}{N^{2}} \end{pmatrix}$$
  where $q^{ij}$ is the inverse of $q_{ij}$ and I raise and lower
  indices with the spatial metric tensor $q_{ij}$.
\item Show that the extrinsic curvature tensor is
  \begin{equation*}
K_{ij} = \frac{1}{2N}\left(\partial_{t}q_{ij} - D_{i}N_{j} - D_{j}N_{i}\right)
  \end{equation*}
  Note that sign conventions differ from reference to reference.

  (Hint: as noted in problem II, the unit normal is $n_{a} = f\nabla_{a}t$. What is this in these coordinates? What is $f$?)
\end{enumerate}

\section[Homework III]{Introduction to Quantum Gravity: Homework III\footnote{I do not recall when this was handed out. This was never posted to the course website, unlike the other homework assignments.}}


I.~\textbf{Lie Derivative of the Metric.}
\bigbreak
The Lie derivative of a metric along a vector $\xi^{a}$ is
\begin{equation*}
\mathcal{L}_{\xi}g_{ab} = g_{ac}\partial_{b}\xi^{c}+g_{bc}\partial_{a}\xi^{c}+\xi^{c}\partial_{c}g_{ab}
\end{equation*}
Show this may be rewritten as
\begin{equation*}
\mathcal{L}_{\xi}g_{ab} = \nabla_{a}\xi_{b}+\nabla_{b}\xi_{a}
\end{equation*}
where $\nabla$ is the standard covariant derivative.

\bigbreak\noindent{}II. \textbf{Constraints generate diffeomorphism}\bigbreak

Recall that the Hamiltonian and momentum constraints are
\begin{equation*}
\mathcal{H} = \frac{16\pi G}{\sqrt{q}}\left(\pi_{ij}\pi^{ij}-\frac{1}{2}\pi^{2}\right),\quad
\mathcal{H}^{i}=-2D_{j}\pi^{ij}
\end{equation*}
and $\pi^{ij} = \frac{1}{16\pi G}\sqrt{q}(K^{ij}-q^{ij}q)$
with $K_{ij} = \frac{1}{2N}(\partial_{t}q_{ij} - D_{i}N_{j} - D_{j}N_{i})$. Let
\begin{equation*}
H[\widehat{\xi}] = \int\left[\widehat{\xi}^{\bot}\mathcal{H}+\widehat{\xi}^{i}\mathcal{H}_{i}\right]\D^{3}x.
\end{equation*}
Show that $H[\widehat{\xi}]$ generates (spacetime) diffeomorphisms of
$q_{ij}$, that is,
\begin{equation*}
\left\{H[\widehat{\xi}], q_{ij}\right\} = (\mathcal{L}_{\xi}q)_{ij}
\end{equation*}
where $\mathcal{L}_{\xi}$ is the full spacetime Lie derivative and the
spacetime vector field $\xi^{\mu}$ is given by the full spacetime Lie
derivative and the spacetime vector field $\xi^{\mu}$ is given by
\begin{equation*}
\widehat{\xi}^{\bot} = N\xi^{0},\quad\widehat{\xi}^{i} = \xi^{i}+N^{i}\xi^{0}
\end{equation*}
The parameters $(\widehat{\xi}^{\bot},\widehat{\xi}^{i})$ are known as
``surface deformation'' parameters.

(Hint: use problem 1 and express the Lie derivative of the spacetime
metric in terms of the ADM decomposition.)

\bigbreak\noindent{}III. \textbf{Surface deformation algebra}\bigbreak

Show that
\begin{equation*}
\left\{H[\widehat{\xi}], H[\widehat{\eta}]\right\} = H[\{\widehat{\xi},\widehat{\eta}\}_{SD}]
\end{equation*}
where the ``surface deformation bracket'' $\{-,-\}_{SD}$ is
\begin{align*}
  \{\widehat{\xi},\widehat{\eta}\}_{SD}^{\bot} &= \widehat{\xi}^{i}\partial_{i}\widehat{\eta}^{\bot}-\widehat{\eta}^{i}\partial_{i}\widehat{\xi}^{\bot}\\
  \{\widehat{\xi},\widehat{\eta}\}_{SD}^{i} &=
  \widehat{\eta}^{j}\partial_{j}\widehat{\xi}^{i}
  \widehat{\xi}^{j}\partial_{j}\widehat{\eta}^{i}
  +q^{ij}\left(\widehat{\xi}^{\bot}\partial_{j}\widehat{\eta}^{\bot} - \widehat{\eta}^{\bot}\partial_{j}\widehat{\xi}^{\bot}\right)
\end{align*}
Show that for purely spatial deformations ($\xi^{0}=\eta^{0}=0$), the
surface deformation bracket is equal to the ordinary commutator.

(The surface deformation bracket is a ``canonical'' bracket, defined at
one moment of time. For deformations with $\xi^{0}$ or $\eta^{0}$
nonzero, the commutator involves time derivatives; it can be shown that
the time derivatives of $\xi^{\mu}$ and $\eta^{\mu}$ can be chosen so
that the surface deformation brackt is again equal to the commutator.)

\section[Final Exam]{Final Exam\footnote{This was a handout given a few weeks before the final exam. It is transcribed verbatim.}}

The final exam for ``Quantum Gravity'' will take place on Tuesday, June
9, from 8--10 am in Roessler 158. The format of the exam will be as
follows:

I have listed below twelve topics that have been central themes of the
course. On the exam, I will list six of these. You will choose four of
the six, and write a short (3--4 paragraph) essay describing each of the
four you have chosen.

Your essays do not have to be heavily mathematical, but some
mathematics---equations and simple derivations---are appropriate for
most of these topics. The goal of the essays is to demonstrate the basic
concepts, well enough to (for example) explain the fundamental ideas to
another student or read and roughly follow a paper in which they are
used. I have included a sample essay on the opposite side of this paper.

Topics:
\begin{enumerate}
\item Ambiguities in quantizing a classical system
\item Quantization of constrained systems: Dirac and reduced phase space methods
\item Observables and the ``problem of time'' in quantum gravity
\item The ADM form of the metric
\item The diffeomorphism and Hamiltonian constraints in general relativity
\item The Wheeler--DeWitt equation
\item The parallel transport matrix, holonomies, and gauge-invariant observables
\item Spin networks
\item The area operator in loop quantum gravity
\item The Nambu--Goto and Polyakov actions in string theory
\item How string theory contains gravity
\item Regge calculus and dynamical triangulations
\end{enumerate}

\vfill\eject\setlength{\parskip}{1ex}

\noindent\textbf{Sample essay on the theme ``constraints and
  symmetries''}

(Note that this example is more heavily mathematical than would be
appropriate for some of the other topics. Include derivations like this
when you can make them simple enough; otherwise, describe them and give
a few steps. Of course, this sample is also a bit more polished than I
would expect on the exam.)

A constraint is an equation of motion that involves only first time
derivatives (in the Lagrangian formalism) or no time derivatives (in the
Hamiltonian formalism). It therefore does not describe time evolution,
but rather restricts (``constrains'') the initial data. In the action, a
constraint is most easily described with a Lagrange multiplier. In the
Hamiltonian form, for instance,
\begin{equation*}\tag{1}
I = \int(p\dot{q}-H-\lambda C)\,\D t
\end{equation*}
where the Lagrange multiplier $\lambda$ enforces the constraint
$C(p,q)=0$. In order to be preserved under time evolution, the
constraint must effectively have a vanishing Poisson bracket with the
Hamiltonian:
\begin{equation*}\tag{2}
\{C,H\}=VC \implies \frac{\D C}{\D t}=0\mbox{ when }C=0.
\end{equation*}

With some technical exceptions, a constraint generates a symmetry of the
classical action. That is, under the transformation
\begin{equation*}\tag{3}
  \delta q=\epsilon\{C,q\}=-\epsilon\frac{\D C}{\D p},\quad
  \delta p=\epsilon\{C,p\}=\epsilon\frac{\D C}{\D q}
\end{equation*}
the action remains invariant. To see this, note that
\begin{align*}
\delta I
&= \int\left[\delta q\left(\frac{\D p}{\D t}-\frac{\D H}{\D q}\right)
  + \delta p\left(-\frac{\D q}{\D t}-\frac{\D H}{\D p}\right) -
 \delta\lambda\;C\right]\D t\\
&= \int\left[-\epsilon\frac{\D C}{\D p}\left(\frac{\D p}{\D t}-\frac{\D H}{\D q}\right)
  + \epsilon\frac{\D C}{\D q}\left(-\frac{\D q}{\D t}-\frac{\D H}{\D p}\right) -
 \delta\lambda\;C\right]\D t\\
&= \int\left[-\epsilon\frac{\D C}{\D t} - \epsilon\{C,H\} + \delta\lambda\;C\right]\D t
= \int\left(\frac{\D\epsilon}{\D t}-\epsilon V -
\delta\lambda\right)C\,\D t
\end{align*}
which is zero if we choose $\delta\lambda = (\D\epsilon/\D t) - \epsilon V$.

An example of a constrained system is electromagnetism. For an
electromagnetic system, the vector potential $\vec{A}$ and the electric
field $\vec{E}$ are canonically conjugate, and the Gauss law
$G=\nabla\cdot\vec{E}-\rho=0$ is a constraint. It is easy to check that
$G$ generates gauge transformations of $\vec{A}$. General relativity is
also a constrained system, in which the constraints generate
diffeomorphisms of space and ``surface deformations'' that are equialent
to diffeomorphisms involving time when the equations of motion are satisfied.

\setlength{\parskip}{0pt plus1pt}

\begin{thebibliography}{99}
\footnotesize%  \small

\bibitem{adams1996:ex}
J.F.\ Adams,
\textit{Lectures on Exceptional Lie Groups}.
University of Chicago Press, 1996.

%\cite{Baez:2001dm}
\bibitem{Baez:2001dm}
John C.~Baez,
``The Octonions''.
\journal{Bull.Am.Math.Soc.} \textbf{39} (2002) 145--205
[erratum: \journal{Bull.Am.Math.Soc.} \textbf{42} (2005) 213]
{\tt\doi{10.1090/S0273-0979-01-00934-X}}
[\arXiv{math/0105155} [math.RA]].
%396 citations counted in INSPIRE as of 30 May 2023

%\cite{Ekins:1975yu}
\bibitem{Ekins:1975yu}
J.~M.~Ekins and J.~F.~Cornwell,
``Semisimple Real Subalgebras of Noncompact Semisimple Real Lie Algebras. 5.,''
\journal{Rept.Math.Phys.} \textbf{7} (1975) 167--203
{\tt\doi{10.1016/0034-4877(75)90026-9}}
%4 citations counted in INSPIRE as of 02 Jun 2023


%\cite{Figueroa-OFarrill:2007jcv}
\bibitem{Figueroa-OFarrill:2007jcv}
Jos\'e Figueroa-O'Farrill,
``A Geometric construction of the exceptional Lie algebras $F_{4}$ and $E_{8}$''.
\journal{Commun.Math.Phys.} \textbf{283} (2008) 663--674
{\tt\doi{10.1007/s00220-008-0581-7}}
[\arXiv{0706.2829} [math.DG]].
%14 citations counted in INSPIRE as of 02 Jun 2023

%\cite{Garling:2011zz}
\bibitem{Garling:2011zz}
D.~J.~H.~Garling,
\textit{Clifford Algebras: An Introduction}.
Cambridge University Press, 2011.
%4 citations counted in INSPIRE as of 02 Jun 2023

%\cite{Gogberashvili:2019ojg}
\bibitem{Gogberashvili:2019ojg}
Merab Gogberashvili and Alexandre Gurchumelia,
``Geometry of the Non-Compact $G(2)$''.
\journal{J.Geom.Phys.} \textbf{144} (2019) 308--313
{\tt\doi{10.1016/j.geomphys.2019.06.015}}
[\arXiv{1903.04888} [physics.gen-ph]].
%3 citations counted in INSPIRE as of 02 Jun 2023

%\cite{Gunaydin:2001bt}
\bibitem{Gunaydin:2001bt}
M.~Gunaydin, K.~Koepsell and H.~Nicolai,
``The Minimal unitary representation of $\mathtt{E}_{8(8)}$''.
\journal{Adv.Theor.Math.Phys.} \textbf{5} (2002) 923--946
{\tt\doi{10.4310/ATMP.2001.v5.n5.a3}}
[\arXiv{hep-th/0109005} [hep-th]].
%59 citations counted in INSPIRE as of 02 Jun 2023

\bibitem{1212.3182}
Aaron Wangberg, Tevian Dray,
``$E_{6}$, the Group: The structure of $\SL(3,\OO)$''.
\arXiv{1212.3182}

%\cite{Zhang:2011ym}
\bibitem{Zhang:2011ym}
R.B.~Zhang,
``Serre presentations of Lie superalgebras''.
[\arXiv{1101.3114} [math.RT]].
%4 citations counted in INSPIRE as of 30 May 2023

\end{thebibliography}

% Tits construction of the exceptional simple Lie algebras
% https://arxiv.org/abs/0907.3789


\end{fmffile}
\end{document}
