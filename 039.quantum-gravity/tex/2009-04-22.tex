\lecture

The action of general relativity in Hamiltonian form,
\begin{equation}
\action_{ADM} = \int[\pi^{ij}\dot{q}_{ij} - N\mathcal{H}-N_{i}\mathcal{H}^{i}]\,\D^{3}x\,\D t.
\end{equation}
The sign conventions varies, but the Hamiltonian is:
\begin{subequations}
\begin{equation}
\mathcal{H} = \frac{16\pi G}{\sqrt{q}}(\pi^{ij}\pi_{ij}-\pi^{2}) -
\frac{1}{16\pi G}\sqrt{q}\,{{}^{(3)}\!R},
\end{equation}
and the momentum constraints,
\begin{equation}
\mathcal{H}^{i} = -2D_{j}\pi^{ij}.
\end{equation}
\end{subequations}
This is a Hamiltonian for general relativity based on a certain set of
variables: the metric for a spatial hypersurfaces as the position
variable and its time derivative for its conjugate momenta.

We can consider the Poisson brackets for this field:
\begin{equation}
\{q_{ij}(x),\pi^{k\ell}(x')\} =
\frac{1}{2}(\delta^{k}_{i}\delta^{\ell}_{j}
+\delta^{k}_{j}\delta^{\ell}_{i})\widetilde{\delta}^{(3)}(x-x'),
\end{equation}
where the tilde indicates a densitized delta function, so
\begin{equation}
\int\widetilde{\delta}^{(3)}(x)\,\D^{3}x=1.
\end{equation}
In particular, we do not need to explicitly write out $\sqrt{q}$. Using
a densitized delta should make intuitive sense, since $\pi^{k\ell}$ is a
tensor density.

This is a completely constrained system, with the momentum constraints
generating spatial change of coordinates. Consider:
\begin{subequations}
  \begin{align}
    \{\int\xi^{i}\mathcal{H}_{i}(x)\,\D^{3}x, q_{k\ell}(x')\}
    &=\{-2\int\xi^{i}D^{j}\pi_{ij}(x)\,\D^{3}x, q_{k\ell}(x')\}\\
    &=\{\int(\xi_{i}D_{j} + \xi_{j}D_{i})\pi^{ij}(x)\,\D^{3}x, q_{k\ell}(x')\}\\
    &= - (D_{k}\xi_{\ell}+D_{\ell}\xi_{k})\\
    &= -\mathcal{L}_{\xi}q_{k\ell}.
  \end{align}
\end{subequations}
This means that $\mathcal{H}_{i}$ are generators of spatial coordinate
transformations. The Poisson bracket for the momentum constraints and
the $\pi^{ij}$ are a bit more complicated.

We are working on spatial hypersurfaces, so there is a question of
what ``$\mathcal{H}$ generates time translations'' even means. The easy
bracket is with $q_{ij}$, technically what these yield are ``surface
deformations''.\index{Surface Deformation}
(On the horizon of a black hole, surface deformations are not equivalent
to changes of coordinates which could be bad\dots)

\subsection{Reduced Phase Space Quantization}

We have two ways to quantize this system: reduced phase space approach,
and the Dirac approach. Lets begin with the reduced phase space
approach.

Recall the basic idea with the reduced phase space quantization is to
solve the constraints, find new variables, then quantize. The problem
with this is solving the constraints, which is roughly the same as
solving Einstein's field equations. We don't have it, or anything near
it. We need to assume some sort of symmetry (e.g., cylindrical
symmetry). This leads to ``minisuperspace'' or ``midisuperspace''.\index{Minisuperspace}\index{Midisuperspace}
In some sense, this is the wrong thing to do because assuming symmetry
at this level assumes that quantum states have this symmetry too.

An alternative approach is to change variables that change 4 nightmarish
PDEs into 4 simpler equations.

This is the York time-slicing\index{York Time-Slicing}\index{Time-Slicing!York},
the work on this was done predominantly by Fischer and Moncrief.
We start with
\begin{equation}
q_{ij} = \phi^{4}\widetilde{q}_{ij}
\end{equation}
where $\phi$ is the conformal factor, $\widetilde{q}_{ij}$ is such that
the Ricci scalar is
\begin{equation}
{{}^{(3)}}R[\widetilde{q}]\in\{0,\pm1\}.
\end{equation}
This is the Yamabe condition. We can always do this for any Riemannian
manifold.
(The reason why ${{}^{(3)}}R[\widetilde{q}]=0$ or $\pm1$ is due to the
topological properties of the spatial hypersurface; it is some deep
result in topology that is not immediately obvious.) This is for
spatially compact universes (or asymptotically flat ones).

We need the decomposition of the canonical momentum:
\begin{equation}
\pi^{ij} = \frac{1}{16\pi G}[\phi^{-4}p^{ij} -
  \underbrace{\frac{2}{3}K\phi^{2}\widetilde{q}^{ij}\sqrt{\widetilde{q}}}_{\text{trace part}}
  + (\underbrace{\widetilde{D}^{i}Y^{j} + \widetilde{D}^{j}Y^{i}}_{\text{symmetrized covariant derivative of a vector}}
  -\frac{2}{3}\widetilde{q}^{ij}D_{k}Y^{k})]
\end{equation}
where $Y^{i}$ is a density, $p^{ij}$ is a density, and $\widetilde{D}_{i}\widetilde{q}_{jk}=0$.
(Locally any vector
in 3 dimensions can be written as $\vec{\nabla}\phi + \vec{\nabla}\times\vec{A}$.)
We have
\begin{subequations}
\begin{equation}
\widetilde{D}_{i}p^{ij} = 0
\end{equation}
and
\begin{equation}
\widetilde{q}_{ij}p^{ij}=0.
\end{equation}
\end{subequations}
The momentum constraint $D_{i}\pi^{ij}=0$ can be translated into a
covariant derivative with respect to $\widetilde{q}$, we have
\begin{equation}
D_{i}\pi^{ij} = (\dots)\widetilde{D}_{i}p^{ij} + (\dots)\partial^{j}K + (\dots)\widetilde{D}_{i}(\widetilde{D}^{i}Y^{j}+\dots)=0.
\end{equation}
\marginpar{Crucial Step}%
The crucial step: we choose
\begin{equation}
t = -K,
\end{equation}
constant mean extrinsic curvature. This is not an obvious choice, but
there are proofs that this is neat, nice, and consistent. For a black
hole, the hypersurfaces curve around the singularity.

For a large class of solutions, Anderson and Moncrief have recent proofs
this is kosher.

We have
\begin{equation}
K = \frac{1}{N}\partial_{t}(\ln\sqrt{q}),
\end{equation}
some signs vary.

This choice tremendously simplifies things, we are left with
\begin{equation}
D_{i}\pi^{ij} = (\dots)\widetilde{D}_{i}(\widetilde{D}^{i}Y^{j}+\dots) = 0.
\end{equation}
If we assume spatial compactness or $Y$ falls off at infinity, we get
$Y^{i}=0$. So we have simplified the conjugate momenta to be:
\begin{equation}
\pi^{ij} = \frac{1}{16\pi G}[\phi^{-4}p^{ij} - \frac{2}{3}K\phi^{2}\widetilde{q}^{ij}\sqrt{\widetilde{q}}].
\end{equation}
Here $K$ is the proportional time rate of change of the local volume.

We solved the momentum constraints, $p$ is freely specified provided it
satisfies certain conditions. Now, the Hamiltonian constraint, which is
hard. We are left with really 2 independent components in
$\widetilde{q}$ and in $p$. We are left with the conformal factor $\phi$
to determine. The Hamiltonian constraint determines it! The Hamiltonian
constraint puts the condition on $\phi$:
\begin{equation}
  \widetilde{\Delta}\phi - \frac{1}{8}\phi + \frac{1}{12}t^{2}\phi^{5}
  -\frac{1}{8}\left(\frac{\widetilde{q}_{ij}\widetilde{q}_{k\ell}p^{ik}p^{j\ell}}{\widetilde{q}^{2}}\right)\phi^{-7} = 0.
\end{equation}
This is a second-order elliptic PDE.

Plugging this back into the action, we get (combining \emph{everything}
back together):
\begin{equation}
\action = \left(\frac{1}{16\pi G}\right)^{2}\int[p^{ij}\dot{\widetilde{q}}_{ij}-\frac{4}{3}\sqrt{\widetilde{q}}\phi^{6}]\D^{3}x\,\D t.
\end{equation}
With the Hamiltonian constraint implying we can write the conformal
factor as a function of $p$ and $\widetilde{q}$,
$\phi=\phi(p,\widetilde{q})$. In our notion of time, that Hamiltonian is
very nonlocal. It is effectively
\begin{equation}
\mathcal{H} = \frac{4}{3}\sqrt{\widetilde{q}}\phi^{6}.
\end{equation}
Due to this nightmarish nonlocality, we don't know how to put hats on stuff.

(We have been working with a zero cosmological constant $\Lambda=0$,
there should be some contribution from it in $\phi$ for nonzero
$\Lambda$.)

In $2+1$ dimensional gravity, the $q^{2}p^{2}\phi^{-7}$ term goes away,
and we have a local Hamiltonian, and everything's nice.

This was based on a particular decomposition, we'd like to keep
something similar to the decomposition of $\pi$.

\subsection{Dirac Quantization}

Let's begin Dirac quantization of the system. We basically impose the
constraints at the quantum level. We have our wave function $\Psi[q]$,
so we have
\begin{subequations}
\begin{equation}
\widehat{\mathcal{H}}^{i}\Psi[q]=0
\end{equation}
and
\begin{equation}
\widehat{\mathcal{H}}\Psi[q]=0.
\end{equation}
\end{subequations}
We use the Schrodinger picture to have
\begin{equation}
\widehat{\pi}^{ij} = -\I\frac{\delta}{\delta q_{ij}}.
\end{equation}
The momentum constraint smeared by some vector $\zeta^{i}$ is
\begin{equation}
\int\zeta_{j}D_{i}\frac{\delta}{\delta q_{ij}}\Psi[q]\,\D^{3}x=0.
\end{equation}
Integration by parts gives us,
\begin{equation}
\int(D_{i}\zeta_{j}+D_{j}\zeta_{i})\frac{\delta}{\delta q_{ij}}\Psi[q]\,\D^{3}x=0.
\end{equation}
By functional Taylor expansion, we have
\begin{equation}
\Psi[q_{ij} + D_{i}\zeta_{j}+D_{j}\zeta_{i}] - \Psi[q_{ij}]=0.
\end{equation}
So $\Psi$ is invariant under such coordinate transformations. This is
not as easy as it seems.
