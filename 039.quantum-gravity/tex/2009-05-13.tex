\lecture

The Hamiltonian constraint gives surface deformations. So far we have
not looked at the momentum or Hamiltonian constraints. It's like
Einstein's field equations are missing. We have
\begin{equation}
V_{i} = F^{\widehat{I}}_{ij}\widetilde{E}^{j}_{\widehat{I}}=0.
\end{equation}
Let us first consider a trick useful in many circumstances. Consider
\begin{subequations}
\begin{align}
\xi^{i}F_{ij}^{\widehat{I}}
&= \xi^{i}(\partial_{i}A^{\widehat{I}}_{j} -
    \partial_{j}A^{\widehat{I}}_{i} + \epsilon^{\widehat{I}\widehat{M}\widehat{N}}A_{i\widehat{M}}A_{j\widehat{N}})\\
&= \xi^{i}\partial_{i}A^{\widehat{I}}_{j} - \partial_{j}(\xi^{i}A^{\widehat{I}}_{i}) +
    (\partial_{j}\xi^{i})A^{\widehat{I}}_{i} + \epsilon^{\widehat{I}\widehat{M}\widehat{N}}\xi^{i}A_{i\widehat{M}}A_{j\widehat{N}}\\
&=
    \underbrace{\xi^{i}\partial_{i}A^{\widehat{I}}_{j}+(\partial_{j}\xi^{i})A^{\widehat{I}}_{i}}_{\substack{\text{change of }
        A\text{ under an}\\
        \text{infinitesimal change of}\\
        \text{coordinates}}} - D_{j}(\xi^{i}A_{i}^{\widehat{I}})\\
&= \underbrace{\delta_{\xi}A^{\widehat{I}}_{i}}_{\text{Lie derivative}} - D_{j}(\xi^{i}A_{i}^{\widehat{I}}).
\end{align}
\end{subequations}
The moral is that a vector contracted with the field strength tensor
looks liek a covariant derivative and a diffeomorphism (gauge
transformation). So we have
\begin{subequations}
\begin{equation}
\int\xi^{i}V_{i}\,\D^{3}x\sim\int(-D_{i}(\xi^{i}A^{\widehat{I}}_{i}) +
\delta_{\xi}A^{\widehat{I}}_{i})\frac{\delta}{\delta A^{\widehat{I}}_{i}}\D^{3}x,
\end{equation}
which acts on fields like:
\begin{equation}
A\to A - D\varepsilon + \delta_{\xi}A,
\end{equation}
\end{subequations}
where $\varepsilon^{\widehat{I}} = \xi^{i}A^{\widehat{I}}_{i}$.
This is exactly analogous to the statement in quantum mechanics that
$\widehat{p}$ generates translations in position.

We have these spin network states. We would like them to be invariant
under gauge transformations. Forget about position and think about a
spin network in the graph theoretic notion of a network. For instance
consider the two spin networks doodled below:
\begin{center}
\includegraphics{img/2009-05-13.0}
\end{center}
We now treat them as the same spin network (despite in our earlier
treatment, we would have treated them as distinct spin networks).
That is to say, \emph{before} we would have found the inner product of
these two spin states would vanish, \emph{but now} we will find their
inner product is unity (i.e., 1). There is an obvious problem here,
though: spin networks are not ``functions'' of connections since thre is
no longer any background space. This is fine on the one hand, but it's
harder to find how to get physics from this. The trick is to define a
diffeomorphism invariant notion of what a surface is, so whn we perform
some diffeomorphism of a surface, and the surface $\Sigma$ is punctured
by a spin network, dragging the surface $\Sigma$ would cause a
diffeomorphism in the spin network; i.e., everything is dragged along.
\begin{center}
\includegraphics{img/2009-05-13.1}
\end{center}
Here we need some extra parameters, e.g., if we had a scalar field,
there would be some extra information that needs to be preserved under
diffeomorphisms. (This is what we do in practice.)

Thus far we have left out the Hamiltonian constraint, but for agood
reason: no one knows how to deal with it. Recall it looks like
\begin{equation}
S\sim FEE + (1+\immirzi^{2})(\mbox{big mess}).
\end{equation}
Here the Hamiltonian is quadratic in functional derivatives. There are
some tricks that may possibly work; e.g., we can see the volume operator
looks like $V\sim E^{3}$, so we have the Poisson bracket $\{V,A\}\sim E^{2}$,
which permits us to rewrite the first term as
\begin{equation}
FEE\sim F\{V,A\}.
\end{equation}

\begin{wrapfigure}{R}{5pc}
\centering\vskip-1pc
\includegraphics{img/2009-05-13.2}
\end{wrapfigure}
Recall, to get $F$ we found the path ordered exponential integral
(a.k.a., the holonomy)
%\begin{equation*}
$\displaystyle \mathcal{P}\E^{-\int A}\sim 1 + \int F + \cdots$
%\end{equation*}
But to do that for a spin network, we work around a node and take the
dashed line (doodled to the right) to go to the node (its length
vanishing). But diffeomorphism invariance allows us towiggle the line,
so its vanishing no longer really matters. The regulation becomes
independent of the regulator.

As far as the ``big mess'' term in the Hamiltonian constraint, Thomas
Thiemann has done a lot of work trying to simplify it.

\begin{wrapfigure}{r}{10pc}
\centering\vskip-1pc
\includegraphics{img/2009-05-13.3}
\end{wrapfigure}
Another problem is that the Hamiltonian acts node by node, so it may be
too local (resulting in nothing propagating).

It may be these are not problems, we just don't know. We'd need to know
information about the solution to the Hamiltonian constraint.

Spin foams may give an alternative point of view to time evolution of
spin networks. If we want to consider how a spin network changes, we can
change the spin label or the intertwiners (which is hard) or we can add
edges (or remove edges). We have several ingredients: the node of a spin
network is promoted to an edge in the spin foam, the edge becomes a
plane, and we need some way to deal with the vertex in the spin foam.%\wrapfill


\begin{wrapfigure}{l}{10pc}
\centering\vskip-1pc
\includegraphics{img/2009-05-13.4}
\end{wrapfigure}
Recall the notion of duals to simplices. In two dimensions, the dual is
doodled to the left in red, and in three dimensions it is generalized
accordingly. The trick is for the spin foam, we have a sort of
dictionary identifying various things to arious other parts of the dual
4-simplex. The use of a spin foam is to give transition amplitudes to
evolving spin networks.

As a closing remark, there are field theories which use group elements
on the edges of its Feynman diagrams, but these appear to be ``dual'' to
spin foams---so in a sense, gravity ``emerges''.\footnote{I believe this
remark refers to group field theory.}
