\section[Final Exam]{Final Exam\footnote{This was a handout given a few weeks before the final exam. It is transcribed verbatim.}}

The final exam for ``Quantum Gravity'' will take place on Tuesday, June
9, from 8--10 am in Roessler 158. The format of the exam will be as
follows:

I have listed below twelve topics that have been central themes of the
course. On the exam, I will list six of these. You will choose four of
the six, and write a short (3--4 paragraph) essay describing each of the
four you have chosen.

Your essays do not have to be heavily mathematical, but some
mathematics---equations and simple derivations---are appropriate for
most of these topics. The goal of the essays is to demonstrate the basic
concepts, well enough to (for example) explain the fundamental ideas to
another student or read and roughly follow a paper in which they are
used. I have included a sample essay on the opposite side of this paper.

Topics:
\begin{enumerate}
\item Ambiguities in quantizing a classical system
\item Quantization of constrained systems: Dirac and reduced phase space methods
\item Observables and the ``problem of time'' in quantum gravity
\item The ADM form of the metric
\item The diffeomorphism and Hamiltonian constraints in general relativity
\item The Wheeler--DeWitt equation
\item The parallel transport matrix, holonomies, and gauge-invariant observables
\item Spin networks
\item The area operator in loop quantum gravity
\item The Nambu--Goto and Polyakov actions in string theory
\item How string theory contains gravity
\item Regge calculus and dynamical triangulations
\end{enumerate}

\vfill\eject\setlength{\parskip}{1ex}

\noindent\textbf{Sample essay on the theme ``constraints and
  symmetries''}

(Note that this example is more heavily mathematical than would be
appropriate for some of the other topics. Include derivations like this
when you can make them simple enough; otherwise, describe them and give
a few steps. Of course, this sample is also a bit more polished than I
would expect on the exam.)

A constraint is an equation of motion that involves only first time
derivatives (in the Lagrangian formalism) or no time derivatives (in the
Hamiltonian formalism). It therefore does not describe time evolution,
but rather restricts (``constrains'') the initial data. In the action, a
constraint is most easily described with a Lagrange multiplier. In the
Hamiltonian form, for instance,
\begin{equation*}\tag{1}
I = \int(p\dot{q}-H-\lambda C)\,\D t
\end{equation*}
where the Lagrange multiplier $\lambda$ enforces the constraint
$C(p,q)=0$. In order to be preserved under time evolution, the
constraint must effectively have a vanishing Poisson bracket with the
Hamiltonian:
\begin{equation*}\tag{2}
\{C,H\}=VC \implies \frac{\D C}{\D t}=0\mbox{ when }C=0.
\end{equation*}

With some technical exceptions, a constraint generates a symmetry of the
classical action. That is, under the transformation
\begin{equation*}\tag{3}
  \delta q=\epsilon\{C,q\}=-\epsilon\frac{\D C}{\D p},\quad
  \delta p=\epsilon\{C,p\}=\epsilon\frac{\D C}{\D q}
\end{equation*}
the action remains invariant. To see this, note that
\begin{align*}
\delta I
&= \int\left[\delta q\left(\frac{\D p}{\D t}-\frac{\D H}{\D q}\right)
  + \delta p\left(-\frac{\D q}{\D t}-\frac{\D H}{\D p}\right) -
 \delta\lambda\;C\right]\D t\\
&= \int\left[-\epsilon\frac{\D C}{\D p}\left(\frac{\D p}{\D t}-\frac{\D H}{\D q}\right)
  + \epsilon\frac{\D C}{\D q}\left(-\frac{\D q}{\D t}-\frac{\D H}{\D p}\right) -
 \delta\lambda\;C\right]\D t\\
&= \int\left[-\epsilon\frac{\D C}{\D t} - \epsilon\{C,H\} + \delta\lambda\;C\right]\D t
= \int\left(\frac{\D\epsilon}{\D t}-\epsilon V -
\delta\lambda\right)C\,\D t
\end{align*}
which is zero if we choose $\delta\lambda = (\D\epsilon/\D t) - \epsilon V$.

An example of a constrained system is electromagnetism. For an
electromagnetic system, the vector potential $\vec{A}$ and the electric
field $\vec{E}$ are canonically conjugate, and the Gauss law
$G=\nabla\cdot\vec{E}-\rho=0$ is a constraint. It is easy to check that
$G$ generates gauge transformations of $\vec{A}$. General relativity is
also a constrained system, in which the constraints generate
diffeomorphisms of space and ``surface deformations'' that are equialent
to diffeomorphisms involving time when the equations of motion are satisfied.
