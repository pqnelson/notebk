\lecture

We spoke about what it means to quantize a system. This time we will
discuss naive quantization of a system with constraints.

\textbf{An addendum from last time:} Take a one-dimensional particle
moving along a line. We have $q$, $p$ be the canonical coordinates and
we make the Poisson bracket into commutators
\begin{equation}
\{q,p\}\mapsto\frac{\I}{\hbar}[\widehat{q},\widehat{p}].
\end{equation}
The operator $\exp(\I a\widehat{p}/\hbar)$ generates translations in
position, so:
\begin{equation}
\E^{\I a\widehat{p}/\hbar}\widehat{q}\E^{-\I a\widehat{p}/\hbar}=\widehat{q}\pm a.
\end{equation}
Hence $\widehat{q}$ could take on any value.

Suppose we move on the positive real line, not the entire line. We can
use the affine commutation relations. We use $\widehat{q}$ and
\begin{equation}
\widehat{D} = \widehat{qp}.
\end{equation}
Classically we have
\begin{equation}
\{q,D\}=q,
\end{equation}
yet quantum mechanically,
\begin{equation}
[\widehat{q},\widehat{D}]=\I\hbar\widehat{q}.
\end{equation}
This is a different representation than the first set of commutators.

We have
\begin{equation}
\E^{\I a\widehat{D}/\hbar}\widehat{q}\E^{-\I a\widehat{D}/\hbar} = \E^{a}\widehat{q}.
\end{equation}
So this $\widehat{D}$ operation is just dilation.

This ought to be important since the ``position'' [in general
  relativity] is the metric on a spatial hypersurface, it should be
positive definite. In the naive way, we can get timelike directions or
nondefinite values, etc. We probably ought to use affine commutators.

(The Poisson bracket is unique. If we used the Heisenberg brackets, there
would be an ordering problem, though not a serious one since we could
use a symmetrized product.)

\subsection{Quantization of Constrained Systems}

In general, we have some action (we use $\action$ for the action since,
in Euclidean quantum gravity, the action is minus the entropy and $S$ is
used for entropy)
\begin{equation}
\action = \int L(q,\dot{q})\,\D t.
\end{equation}
In general, higher-order derivatives in the Lagrangian generically leads
to unbounded energies. For a review paper on this, see:
\begin{itemize}
\item R.P.~Woodard,
  ``Avoiding Dark Energy with $1/R$ Modifications of Gravity''.
  \journal{Lect.~Notes Phys.} \volume{720} (2007) pp.403--433; \arXiv{astro-ph/0601672}.\\
{\tt\doi{10.1007/978-3-540-71013-4_14}}
\end{itemize}
There are some exceptions, for example, when there are an infinite
number of derivatives (but this is a sort of nonlocality, and there have
been some papers recently on a nonlocal generalization of the
Einstein--Hilbert action), or when we can integrate by parts to get to
first-order. This is what happens with Einstein--Hilbert action.

The action of a system with constraints looks like
\begin{equation}
\action = \int(L(q,\dot{q}) - \lambda C(q,\dot{q}))\,\D t,
\end{equation}
where $\lambda$ is the Lagrange multiplier and $C(q,\dot{q})$ is the
constraint. By varying the action with respect to $\lambda$, we obtain a
constraint on the initial data
\begin{equation}
C(q,\dot{q})=0.
\end{equation}
We can rewrite the action to take into account constraints; if there are
no second-order time derivatives of certain variables, then there will
be constraints.

For general relativity, from the conservation laws, we have
\begin{subequations}
  \begin{align}
    \nabla_{\mu}G^{\mu\nu} &= \partial_{\mu}G^{\mu\nu} + \dots\\
    &= \partial_{t}G^{t\nu} + \partial_{i}G^{i\nu} + \dots.
  \end{align}
\end{subequations}
This means that each term has to have one time derivative and one
spatial derivative (in the Einstein tensor).

We can look at it one way and say the space of initial data is smaller
than we thought. Usually constraints ``generate'' gauge transformations,
meaning we can look at it as:
\begin{subequations}
  \begin{align}
    \delta q &= \{\varepsilon C,q\}\\
    \delta p &= \{\varepsilon C,p\},
  \end{align}
\end{subequations}
where $\varepsilon$ is an arbitrary function of time. This is a
generator of canonical transformations. In general, they're gauge
transformations.

Here's the sketch of the basic idea (see Henneaux and Teitelboim's
\emph{Quantization of Gauge Systems} for further details). We want to
consider the variation of the action $\delta\action$. Let's consider the
action in Hamiltonian form:
\begin{equation}
\action = \int(p\dot{q} - H -\lambda C)\,\D t.
\end{equation}
We often write
\begin{equation}
H^{*} := H + \lambda C,
\end{equation}
and refer to it as the ``Extended Hamiltonian''. Let's consider the
variation of the kinetic term:
\begin{subequations}
  \begin{align}
    \{\varepsilon C,p\dot{q}\}
    &=\{\varepsilon C,p\}\dot{q} + p\frac{\D}{\D t}\{\varepsilon C,q\}\\
    &=\left(\varepsilon \frac{\partial C}{\partial q}\right)\dot{q}
       +p\frac{\D}{\D t}\left(-\varepsilon\frac{\partial C}{\partial p}\right)\\
    &= \varepsilon \frac{\partial C}{\partial q}\dot{q}
       -\frac{\D}{\D t}\left(\varepsilon p\frac{\partial C}{\partial p}\right)
       +\varepsilon\frac{\partial C}{\partial p}\dot{p}\\
    &= \varepsilon \left(\frac{\partial C}{\partial q}\dot{q}+\frac{\partial C}{\partial p}\dot{p}\right)
       -\frac{\D}{\D t}\left(\varepsilon p\frac{\partial C}{\partial p}\right)\\
    &= \varepsilon \frac{\D C}{\D t}
       -\frac{\D}{\D t}\left(\varepsilon p\frac{\partial C}{\partial p}\right)\\
    &= -\dot{\varepsilon}C
       +\frac{\D}{\D t}\left(\varepsilon C - \varepsilon p\frac{\partial C}{\partial p}\right).
  \end{align}
\end{subequations}
The next term we need to examine is $\{\varepsilon C,H\}$ which, in
general, could be anything. If $C$ remains a constraint under time
translations, then the bracket with the Hamiltonian $H$ is also a
constraint. In general this is true, the commutator between the
Hamiltonian and a constraint is another constraint. Let
\begin{equation}
\{H,C\}=vC
\end{equation}
where $v$ is some function, then
\begin{equation}
\{\varepsilon C,H\} = -\varepsilon vC.
\end{equation}
Putting all of this together, we find
\begin{equation}\label{eq:2009-04-08:pb-of-constraint-and-lagrangian}
  \{\varepsilon C, p\dot{q}-H-\lambda C\}
  = -\dot{\varepsilon}C + \frac{\D}{\D t}\left(\varepsilon C - \varepsilon p\frac{\partial C}{\partial p}\right)
+\varepsilon vC-\{\varepsilon C,\lambda C\}.
\end{equation}
If
\begin{equation}
\delta\lambda = -(\dot{\varepsilon}-\varepsilon v),
\end{equation}
then the variation of the action is zero. This is because $\delta C
=\{\varepsilon C,C\} = 0$, so
\begin{equation}
\{\varepsilon C,\lambda C\} = \{\varepsilon C,\lambda\}C = (\delta\lambda)C.
\end{equation}
Plugging this back into
Eq~\eqref{eq:2009-04-08:pb-of-constraint-and-lagrangian} makes
$\{\varepsilon C, p\dot{q}-H-\lambda C\}$ into a total derivative, which
contributes nothing to the action.

The moral of the story is that constraints generate gauge
transformations and, in general (with the exception of some pathological
counterexamples), the converse holds too. Note: if $\{C,C\}\propto C$,
then the results still hold.

Now we use the results, and generalize to multiple constraints. We need
\begin{equation}
\{C_{i},C_{j}\} = {f_{ij}}^{k}C_{k}
\end{equation}
where ${f_{ij}}^{k}$ are ``structure constants'' and these are called
``first-class constraints''. (If we change Poisson brackets to
commutators, these are the generators of the gauge algebra --- or, at
least, the structure constants are those from the Lie algebra of the
gauge group.)

There are also ``second-class constraints'' which do not generate gauge
transformations.

There are various ways to handle constraints in the quantization
process. There is a constraint surface in the phase space when the
constraints are satisfied. We have this gauge invariance which
takes \addenda{a physical state on this constraint surface and produces
another distinct point in the phase space, but is physically
indistinguishable from the original physical state.}
\begin{center}
\includegraphics{img/2009-04-08.0}
\end{center}
The space of orbits is what is interesting. We take the physical degrees
of freedom by taking some subsurface which cuts through the phase space
orbits only once for each orbit.

There are times when a section may not contain an orbit (or some other
unpleasant problem), which is the Gribov ambiguity.

For electromagnetism, we have $A_{\mu}\to
A_{\mu}+\partial_{\mu}\Lambda$.

The approaches to quantizing (``canonically'') systems with constraints:

\bigbreak
\textbf{Approach 1:} Reduced phase space quantization, the recipe is:
\begin{enumerate}[nosep,label=(\arabic*)]
\item Clasically solve the constraints.
\item Choose a section (``gauge fix'').
\item Insert into the action $\action$ and quantize.
\end{enumerate}
Part of the problem is that, well, sometimes solving the constraints
is fairly hard. For example, for classical general relativity, we do
not know the general solution for Einstein's field equations.

A second problem is when fixing a gauge, when we jump to the third step,
the resulting field is typically nonlocal. Consider electromagnetism. We
have $A^{\mu}$ which is ambiguous, we could have
\begin{equation}
A^{\mu} = \widetilde{A}^{\mu} + \partial^{\mu}\Lambda.
\end{equation}
We can gauge fix using the Lorenz gauge
$\partial_{\mu}\overline{A}^{\mu}=0$ for some gauge fixed potential
$\overline{A}^{\mu}$. We can expand this to be:
\begin{subequations}
\begin{align}
\partial_{\mu}\overline{A}^{\mu}
&= \partial_{\mu}(A^{\mu} + \partial^{\mu}\Lambda)\\
&=\partial_{\mu}A^{\mu} + \square\Lambda = 0.
\end{align}
\end{subequations}
Then we have
\begin{equation}
\Lambda = -\square^{-1}\partial_{\mu}A^{\mu},
\end{equation}
and then
\begin{equation}
\overline{A}^{\mu} = A^{\mu} - \partial^{\mu}\square^{-1}\partial_{\nu}A^{\nu},
\end{equation}
or
\begin{equation}
A^{\mu} = \overline{A}^{\mu}  - \partial^{\mu}\square^{-1}\partial_{\nu}A^{\nu}.
\end{equation}
The second term on the right-hand side is horribly nonlocal. BRST says
that sticking a differential gauge transformation back into the system
when solved is illegal. There are particular cases when this works; but
the more complicated the theory, the harder this approach becomes.

\bigbreak
\textbf{Approach 2:} Dirac quantization. The basic recipe is:
\begin{enumerate}[nosep,label=(\arabic*)]
\item Quantize the whole system.
\item Impose constraints as operator conditions. That is, we define the
  physical states as
  \begin{equation}
\widehat{C}\mid\mbox{physical}\rangle = 0,
  \end{equation}
  the kernel of a ``constraint operator'' (or the intersection of
  kernels of constraint operators). The states are automatically gauge
  invariant this way.
\item Define the inner product on physical states (intuitively: ``gauge
  fixing the inner product''). How to do this is less obvious and usually hard.
\item Find physical operators $\widehat{\mathcal{O}}_{\text{phys}}$ that take physical states to physical
  states (so if one realizes this, then it's equivalent to the operators
  which commutes with the constraints). That is, we need to find
  \begin{equation}
[\widehat{\mathcal{O}}_{\text{phys}},\widehat{C}]=0.
  \end{equation}
(For general relativity, these physical operators are in general
  nonlocal and we don't know what they are.)
\end{enumerate}

\bigbreak\noindent\textbf{Example.}
The parametrized particle. For a one-dimensional particle subjected to
some potential, then
\begin{equation}
\action = \int(p\dot{q} - H)\,\D t.
\end{equation}
If we change $t$ to some monotonic function of time, then $p\dot{q}\,\D t$
remains invariant \emph{but} the Hamiltonian contribution $H\,\D t$
doesn't quite remain invariant. Let's define a parameter $\tau$ such
that,
\begin{equation}
\action = \int\left(p\frac{\D q}{\D \tau} - H\frac{\D t}{\D\tau}\right)\D t.
\end{equation}
Let us write $q^{0}=t$ and $p_{0}=H$, so we can write the action as:
\begin{equation}
\action = \int\left(p_{\mu}\frac{\D q^{\mu}}{\D\tau}\right)\D\tau.
\end{equation}
But to do this, we observe $H=H(p,q)$, so we need to introduce a
constraint:
\begin{equation}
\action = \int\left(p_{\mu}\frac{\D q^{\mu}}{\D\tau} - \lambda(p_{0}+H(p,q))\right)\D\tau.
\end{equation}
The constraint generates parametrization invariance under
\begin{equation}
\tau\to\tau+\delta\tau.
\end{equation}
So far, so good.

The reduced phase space approach solves the constraint, which is
trivially $p_{0}=-H$. We plug this back in:
\begin{subequations}
  \begin{align}
    p_{\mu}\frac{\D q^{\mu}}{\D\tau}
    &= p_{1}\frac{\D q^{1}}{\D\tau} + p_{0}\frac{\D q^{0}}{\D\tau}\\
    &= p_{1}\frac{\D q^{1}}{\D\tau} + (-H)\frac{\D q^{0}}{\D\tau}.
  \end{align}
  We plug in our gauge-fixing $q^{0}=t$, then
  \begin{equation}
p_{\mu}\frac{\D q^{\mu}}{\D\tau}= p_{1}\frac{\D q^{1}}{\D\tau} + (-H)\frac{\D t}{\D\tau}.
  \end{equation}
\end{subequations}

The Dirac approach where the wave function $\Psi[q^{\mu}]$, the
commutators $[p_{\mu},q^{\nu}]=\I\hbar{\delta_{\mu}}^{\nu}$, the
constraint operator is
\begin{subequations}
  \begin{align}
    (\widehat{p}_{0}+\widehat{H}_{\text{phys}})\psi_{\text{phys}}
    &= 0\\
    &= \left(-\I\hbar\frac{\partial}{\partial q^{0}} + \widehat{H}_{\text{phys}}\right)\psi_{\text{phys}}.
  \end{align}
\end{subequations}
We need to gauge fix the inner product
\begin{equation}
\begin{split}
  \int\Psi^{*}_{1}(q^{\mu})\Psi_{2}(q^{\mu})\,\D q^{i}\D q^{0}
  &=\int\langle\Psi_{1}\mid\Psi_{2}\rangle_{\text{phys}}\,\D t\\
  &=\infty
\end{split}
\end{equation}
where $\langle\Psi_{1}\mid\Psi_{2}\rangle_{\text{phys}}$ is the usual
old-school inner product of quantum mechanics. This integral over time
generates infinities, we use a rigged Hilbert space to define the inner
product---roughly speaking, we ``divide out by infinity''.
