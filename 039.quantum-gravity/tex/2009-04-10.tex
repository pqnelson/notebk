\lecture

I was too sick to attend, but I have been told: Professor Carlip argued
the gauge symmetries of general relativity are isometries described by
Killing equation, derived ADM coordinates including lapse and shift
functions, described extrinsic curvature in terms of lapse and shift,
rewrote the Einstein--Hilbert action in ADM coordinates, derived canonically
conjugate momentum to metric, wrote first-order form of the action,
argued lapse and shift functions are Lagrange multipliers.

In second-order formalism positions $x$ and velocities $\dot{x}$ are treated as independent
variables, but first-order formalism treats positions $x$ and momenta $p$
as independent variables. Also, Professor Waldron refers to the ADM
action's terms as:
\begin{equation}
\action_{ADM} = \int(\underbrace{\pi^{ij}\dot{q}_{ij}}_{\text{symplectic term}}
- \underbrace{N^{i}\mathcal{H}_{i} - N\mathcal{H}}_{\text{constraints}})\,\D^{4}x.
\end{equation}
Also notation: $D_{i}$ determined using the spatial metric $q_{ij}$
such that $D_{i}q_{jk}=0$.
