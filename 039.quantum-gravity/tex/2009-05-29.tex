\lecture[AdS/CFT Correspondence, Causal Dynamical Triangulations]

The AdS/CFT [anti de Sitter Space, Conformal Field Theory]
correspondence is a really interesting, nonperturbative string theory
requiring a negative cosmological constant. Although unphysical, it
could contain interesting analogies for our purposes.
Recall Anti-De Sitter space is a flat solution to the Einstein field
equations with a negative cosmological constant. We usually write the
cosmological constant as
\begin{equation}
\Lambda=\frac{-1}{\ell^{2}}
\end{equation}
where $\ell$ is the radius of the universe. The metric is
\begin{equation}
\D s^{2} = \frac{\D r^{2}}{1 + (r/\ell)^{2}} + r^{2}\,\D\Omega^{2} -
\left(1 + \frac{r^{2}}{\ell^{2}}\right)\,\D t^{2}.
\end{equation}
There is another set of coordinates $r=\ell\sinh(\rho)$, so
\begin{equation}
  \D s^{2} = \ell^{2}\left(
  \D\rho^{2} + \sinh^{2}(\rho)\,\D\Omega^{2} - \cosh^{2}(\rho)\,\D t^{2}
  \right).
\end{equation}
Observe that $\rho$ is just the proper distance. If we look at the limit
as $\rho\to\infty$, the asymptotic limit, we have
\begin{equation}
\D s^{2}\sim\ell^{2}\,\D\rho^{2} +
\frac{\ell^{2}}{4}\E^{2\rho}(\D\Omega^{2}-\D t^{2}).
\end{equation}
Near infinity, AdS space looks like a flat cylinder with a radial
coordinate added. If we look at geodesics, we can doodle timelike
geodesics [solid line] and lightlike geodesics [dashed line] in the
Penrose diagram:
\begin{center}
  \includegraphics{img/2009-05-29.0}
\end{center}
Timelike geodesics which
start at the center of the cylinder will try to ``move out'', but
the cosmological constant then acts like an attractive potential. For
lightlike geodesics, such geodesics will reflect off of the boundary. In
theory, an observer could receive back a pulse of light they sent
provided they live long enough. The isometry group for AdS is
$\SO(d-1,2)$.

If we have a pair of $D$-branes, we can have the endpoints of a string be both on
one $D$-brane, or one endpoint on each $D$-brane; something like
sketched below:
\begin{center}
  \includegraphics{img/2009-05-29.1}
\end{center}
On 2 $D$-branes, open string states have 2 indices varying from $1$, $2$.
For $N$ $D$-branes we have $N$ indices with values being $1$, $2$. This
may be oriented, not necessarily symmetrized. It turns out these indices
have an $\SU(N)$ symmetry. There is an $N\to\infty$ limit where we may
take to decouple gravitons, giving us an $\SU(N)$ Yang--Mills theory
with a supersymmetry. We can look at the strong coupling limit, we end
up with charged black branes.

We can look at things quite simply: these are dual to each other. That
is, they each describe the entire picture at various limits. What we
find is the near horizon geometry of charged black branes is
$\AdS\times{(compact)}$. If we look at $N$ coincident $3$-branes (3
spatial dimensions $+$ time), this gives $\AdS_{5}\times S^{5}$ and its
boundary is a four-dimensional flat cylinder.

Maldacena AdS/CFT conjecture: string theory in bulk ($\AdS_{5}\times S^{5}$)
$\iff$ $N=4$ Supersymmetric $\SU(N)$ Yang--Mills theory on a flat
4-dimensional cylinder.
That is, ``in bulk'' means ``in spacetime that's asymptotically $\AdS_{5}\times S^{5}$''.

\subsection{Causal Dynamical Triangulations}
A lattice approach to quantum gravity. Nonrenormalizability may be a
statement about the perturbative approach, we're doing the perturbation
wrong---there's nothing wrong with the theory. We have the path integral
approach, which has of supreme importance the partition function
\begin{equation}
Z = \int\E^{\I(\action[g] + \int g_{\mu\nu}J^{\mu\nu})}\,[\D g]
\end{equation}
where $J^{\mu\nu}$ is a fixed source. We approximate this as a sum over
finite geometries, so we havea discrete approach using lattices (e.g.,
Regge calculus approach). The defecit angle $\delta$, the Ricci scalar
is determined by parallel transport $R =
\delta\cdot\delta^{(2)}(\vec{x})$. For three dimensions, we work with
3-simplices, so imagine an edge sticking out of the paper at the vertex
doodled below and making each triangle a face on the tetrahedron.
\begin{center}
  \includegraphics{img/2009-05-29.2}
\end{center}
We find that, in three-dimensions,
\begin{equation}
\int R\sqrt{|g|}\,\D^{3}x = \sum_{i}\delta_{i}\ell_{i}.
\end{equation}
In four-dimensions, we have,
\begin{equation}
\int R\sqrt{|g|}\,\D^{4}x = \sum_{\text{2-d hinges}}\delta_{i}A_{i}.
\end{equation}
Regge figured this out in 1961.

So we end up with a discretization of the path integral
\begin{equation}
Z\approx\sum_{\substack{\text{discrete}\\\text{geometries}}}\E^{\I(\action_{\text{regge}}+\Lambda\sum_{n}V_{n})}
\end{equation}
There are several approaches to summing over geometries.

Regge advocated fixing ttriangulation, summing over lengths (and
angles). The other approach is dynamical triangulations fix edge-lengths
and sum over triangulations. In principle, either approach approximates
smooth surfaces. The dynamical triangulations approach walks over the
space of possible states. There are two phases in the computation
\begin{enumerate}
\item The crumpled phase
\item The branched polymer phase.
\end{enumerate}
Dynamical triangulations is modified to become Causal Dynamical
Triangulations, prohibiting polymer phases. We have a sort of timelike
foliation, restricting the sort of triangulations. It's not entirely
clear how to recover the Newtonian limit.
