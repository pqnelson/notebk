\lecture

References for loop quantum gravity.
\begin{itemize}
\item About Spin Foams\begin{itemize}
\item J.~C.~Baez,
``An Introduction to Spin Foam Models of $BF$ Theory and Quantum Gravity''.
\journal{Lect. Notes Phys.} \volume{543} (2000), 25-93;
\arXiv{gr-qc/9905087}\\
{\tt\doi{10.1007/3-540-46552-9_2}}
\item A.~Perez,
``Spin foam models for quantum gravity''.
\journal{Class. Quant. Grav.} \textbf{20} (2003) R43;
\arXiv{gr-qc/0301113}\\
{\tt\doi{10.1088/0264-9381/20/6/202}}
\end{itemize}
\item Group Field Theory
  \begin{itemize}
  \item D.~Oriti,
``The Group field theory approach to quantum gravity''.
\arXiv{gr-qc/0607032}.
  \end{itemize}
\item Critique of Loop Quantum Gravity:
  \begin{itemize}
  \item H.~Nicolai and K.~Peeters,
``Loop and spin foam quantum gravity: A Brief guide for beginners''.
\journal{Lect.\ Notes Phys.} \textbf{721} (2007) 151-184;
\arXiv{hep-th/0601129}.\\
{\tt\doi{10.1007/978-3-540-71117-9_9}}
  \end{itemize}
\item Covariant Canonical Quantization
  \begin{itemize}
  \item A.~Ashtekar, L.~Bombelli and O.~Reula,
``The Covariant Phase Space of Asymptotically Flat Gravitational Fields''.
In \textit{Mechanics, Analysis and Geometry: 200 Years After Lagrange},
pp.417--450, Elsevier, 1991. \\
{\tt\doi{10.1016/B978-0-444-88958-4.50021-5}}.
  \end{itemize}
\end{itemize}

\subsection{String Theory}

A very brief introduction to string theory, but we'll focus on its
relevance to gravity. There are several points to consider
\begin{enumerate}[nosep,label=(\arabic*)]
\item Closed loops have a massless spin-2 excitation (``graviton'')
\item Strings propagate only in a spacetime satisfying the Einstein
  field equations (plus some negligible corrections)
\end{enumerate}\medbreak\noindent\ignorespaces%
(Any theory with self-interacting spin-2 massless excitations is a hint
that gravity is in the game.)\medbreak
\begin{enumerate}[resume*]
\item Background spacetime of the second point is equivalent to a
  coherent state of excitations of the first point.
\end{enumerate}\medbreak

Let us examine the first point. We have the basic field be some tensor
with two indices $h_{\mu\nu}$ and the field equations look like:
\begin{equation}
\Box h_{\mu\nu} + (\mbox{terms involving }\partial_{\rho}h^{\rho\sigma})
= T_{\mu\nu}.
\end{equation}
The most general result is that we end up with a spin-2 part, a vector
(spin-1) part and a scalar (spin-0) part. We need these extra (vector
and scalar) parts vanish, which is a gauge choice (analogous to a spin-1
field $\Box A_{\mu} + k\partial_{\mu}\partial_{\nu}A^{\nu}=J_{\mu}$
requires $\partial^{\mu}J_{\mu}$, otherwise we do not have
electromagnetism). The gauge invariance for us is
\begin{equation}
h_{\mu\nu}\to h_{\mu\nu} + \partial_{\mu}\xi_{\nu} + \partial_{\nu}\xi_{\mu}
\end{equation}
which demands
\begin{equation}
\partial_{\mu}T^{\mu\nu}=0
\end{equation}
for consistency.
We can choose gauge $\partial_{\rho}h^{\rho\sigma}=0$ (Lorenz gauge,
Harmonic gauge, de Donder gauge, Fock gauge, Harmonic gauge, Feynman
gauge, Lorenz gauge, etc.). The reason harmonic is sometimes we write $\Box X^{\mu}$
where $X^{\mu}$ is some parametrized version of the coordinates which
individually transform as scalars (and $\Box$ uses derivatives with
respect to the parameters).
