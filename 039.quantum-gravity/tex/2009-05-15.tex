\lecture

References for loop quantum gravity.
\begin{itemize}
\item About Spin Foams\begin{itemize}
\item J.~C.~Baez,
``An Introduction to Spin Foam Models of $BF$ Theory and Quantum Gravity''.
\journal{Lect. Notes Phys.} \volume{543} (2000), 25-93;
\arXiv{gr-qc/9905087}\\
{\tt\doi{10.1007/3-540-46552-9_2}}
\item A.~Perez,
``Spin foam models for quantum gravity''.
\journal{Class. Quant. Grav.} \textbf{20} (2003) R43;
\arXiv{gr-qc/0301113}\\
{\tt\doi{10.1088/0264-9381/20/6/202}}
\end{itemize}
\item Group Field Theory
  \begin{itemize}
  \item D.~Oriti,
``The Group field theory approach to quantum gravity''.
\arXiv{gr-qc/0607032}.
  \end{itemize}
\item Critique of Loop Quantum Gravity:
  \begin{itemize}
  \item H.~Nicolai and K.~Peeters,
``Loop and spin foam quantum gravity: A Brief guide for beginners''.
\journal{Lect.\ Notes Phys.} \textbf{721} (2007) 151-184;
\arXiv{hep-th/0601129}.\\
{\tt\doi{10.1007/978-3-540-71117-9_9}}
  \end{itemize}
\item Covariant Canonical Quantization
  \begin{itemize}
  \item A.~Ashtekar, L.~Bombelli and O.~Reula,
``The Covariant Phase Space of Asymptotically Flat Gravitational Fields''.
In \textit{Mechanics, Analysis and Geometry: 200 Years After Lagrange},
pp.417--450, Elsevier, 1991. \\
{\tt\doi{10.1016/B978-0-444-88958-4.50021-5}}.
  \end{itemize}
\end{itemize}

\subsection{String Theory}

A very brief introduction to string theory, but we'll focus on its
relevance to gravity. There are several points to consider
\begin{enumerate}[nosep,label=(\arabic*)]
\item Closed loops have a massless spin-2 excitation (``graviton'')
\item Strings propagate only in a spacetime satisfying the Einstein
  field equations (plus some negligible corrections)
\end{enumerate}\medbreak\noindent\ignorespaces%
(Any theory with self-interacting spin-2 massless excitations is a hint
that gravity is in the game.)\medbreak
\begin{enumerate}[resume*]
\item Background spacetime of the second point is equivalent to a
  coherent state of excitations of the first point.
\end{enumerate}\medbreak

%\marginnote{{\footnotesize Review of classical spin-2 ``gravitons''}}
\marginnote{Review of classical spin-2 ``gravitons''}
Let us examine the first point. We have the basic field be some tensor
with two indices $h_{\mu\nu}$ and the field equations look like:
\begin{equation}
\Box h_{\mu\nu} + (\mbox{terms involving }\partial_{\rho}h^{\rho\sigma})
= T_{\mu\nu}.
\end{equation}
The most general result is that we end up with a spin-2 part, a vector
(spin-1) part and a scalar (spin-0) part. We need these extra (vector
and scalar) parts vanish, which is a gauge choice (analogous to a spin-1
field $\Box A_{\mu} + k\partial_{\mu}\partial_{\nu}A^{\nu}=J_{\mu}$
requires $\partial^{\mu}J_{\mu}$, otherwise we do not have
electromagnetism). The gauge invariance for us is
\begin{equation}
h_{\mu\nu}\to h_{\mu\nu} + \partial_{\mu}\xi_{\nu} + \partial_{\nu}\xi_{\mu}
\end{equation}
which demands
\begin{equation}
\partial_{\mu}T^{\mu\nu}=0
\end{equation}
for consistency.
We can choose gauge $\partial_{\rho}h^{\rho\sigma}=0$ (Lorenz gauge,
Harmonic gauge, de Donder gauge, Fock gauge, Harmonic gauge, Feynman
gauge, Lorenz gauge, etc.). The reason harmonic is sometimes we write $\Box X^{\mu}$
where $X^{\mu}$ is some parametrized version of the coordinates which
individually transform as scalars (and $\Box$ uses derivatives with
respect to the parameters).

For consistency, after choosing the gauge, we add contributions of
order $h^{2}$ to the stress-energy tensor to the right-hand side
\begin{equation}
\Box h_{\mu\nu} = T_{\mu\nu} + T^{(h)}_{\mu\nu}
\end{equation}
but then we will need to add cubic interactions, and then quartic
interactions, etc. Deser showed (reprinted as \arXiv{gr-qc/0411023}),
using an incredibly clever choice of variables, the series terminates.
Damour and Henneaux (\arXiv{hep-th/0007220} and \arXiv{hep-th/0009109})
used clever arguments to show there are some extra terms using
cohomological techniques. This work done by Deser, Damour and Henneaux,
are entirely classical, but there are some soft graviton theorems.

\marginnote{Strings, Worldsheet}%\marginnote{{\footnotesize Strings, Worldsheet}}
Let's start with string theory, we will start by talking about strings.
Open strings trace out a 2-dimensional surface with intrinsic
coordinates $(\sigma,\tau)=(\sigma^{0},\sigma^{1})$. We can write the
4-coordinates of the surface in terms of $\sigma$ and $\tau$.
\begin{center}
\includegraphics{img/2009-05-15.0}
\end{center}
We can model interactions using pant diagrams, for example:
\begin{center}
\includegraphics{img/2009-05-15.1}\includegraphics{img/2009-05-15.2}\includegraphics{img/2009-05-15.3}
\end{center}
These are the only possible interactions. These are, of course, in a
fixed background. The common statistic is that there are $10^{500}$
possible backgrounds.
We write the metric for this background as $G_{\mu\nu}$ and the induced
metric on the worldsheet is,
\begin{equation}
h_{ab} = G_{\mu\nu}\frac{\partial X^{\mu}}{\partial\sigma^{a}}\frac{\partial X^{\nu}}{\partial\sigma^{b}}.
\end{equation}
The generalization of the action for a world line is the area of the
worldsheet,\marginnote{Nambu--Goto Action\\Polyakov Action}
\begin{equation}
\action = \frac{-1}{2\pi\alpha'}\int\sqrt{-h}\,\D^{2}\sigma.
\end{equation}
This is the Nambu--Goto action.

Quantizing the Nambu--Goto action turns out to be quite difficult, which
we should expect with any Lagrangian involving the squareroot of
quadratic terms. This leads us to consider the generalization of this
action, the Polyakov action, which requires us to use a new induced
metric $\gamma_{ab}$,
\begin{equation}
\action=\frac{-1}{4\pi\alpha'}\int\gamma^{ab}\partial_{a}X^{\mu}\partial_{b}X^{\nu}G_{\mu\nu}\sqrt{-\gamma}\,\D^{2}\sigma.
\end{equation}
Classically this is equivalent to the Nambu--Goto action, where the
metric $\gamma$ behaves as a Lagrange multiplier (since its derivatives
do not appear in the action). We see varying the action with respect to
$\gamma$ yields
\begin{equation}
\partial_{a}X^{\mu}\partial_{b}X^{\nu}G_{\mu\nu} - \frac{1}{2}\gamma_{ab}\gamma^{cd}\partial_{c}X^{\mu}\partial_{d}X^{\nu}G_{\mu\nu}=0.
\end{equation}
We can solve this equation for $\gamma_{ab}$ to write the induced metric
as
\begin{equation}
\gamma_{ab}=2f(\sigma^{0},\sigma^{1})\partial_{a}X^{\mu}\partial_{b}X^{\nu}G_{\mu\nu}
\end{equation}
where
\begin{equation}
\frac{1}{f(\sigma^{0},\sigma^{1})} = \gamma^{cd}\partial_{c}X^{\mu}\partial_{d}X^{\nu}G_{\mu\nu}.
\end{equation}
This action has Weyl symmetry,
\begin{equation}
\gamma_{ab}\to \Omega(\sigma^{c})\gamma_{ab},
\end{equation}
where $\Omega$ is everywhere positive. We see that the Polyakov action
may be rewritten as,
\begin{equation}
\action=\frac{-1}{4\pi\alpha'}\int\frac{\sqrt{-\gamma}}{f(\sigma^{0},\sigma^{1})}\,\D^{2}\sigma.
\end{equation}
Taking advantage of the fact $f$ is everywhere positive, we see we can
classically recover the Nambu--Goto action by the Weyl symmetry
transformation $\gamma_{ab}\to\gamma_{ab}/\sqrt{f}$. Polchinski proved
the two actions are the same quantum mechanically. %% We can look at its
%% four-dimensional position of the worldsheet $X^{\mu}$
We can look at the string worldsheet as fundamental, then view the
$X^{\mu}$ living on the worldsheet, and four-dimensional spacetime
``emerges''.\footnote{Ed Witten's ``Reflections on the Fate of
Spacetime'' (\journal{Physics Today}, April 1996, pp.24--30) argues this
heuristically; Nick Huggett and Christian W\"{u}thrich's ``Out of
Nowhere: The `emergence' of spacetime in string theory''
(\arXiv{2005.10943}) reviews this general subject.}
