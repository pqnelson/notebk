\lecture

Consider a wave functional $\Psi[\varphi(x)]$ in a Schrodinger type
picture in quantum field theory. Consider infinitesimal deformations of
the field
\begin{equation}
\Psi[\varphi(x)+\varepsilon(x)] = \Psi[\varphi(x)] + \int\frac{\delta\Psi}{\delta\varphi}(x_{1})\varepsilon(x_{1})\,\D^{n}
x_{1} + \bigOh(\varepsilon^{2}).
\end{equation}
If the field is invariant under $\varphi\to\varphi+\widetilde{\varepsilon}$,
then
\begin{equation}
\Psi[\varphi+\widetilde{\varepsilon}\,]=\Psi[\varphi],
\end{equation}
and moreover
\begin{equation}
\int\frac{\delta\Psi}{\delta\varphi}(x_{1})\widetilde{\varepsilon}(x_{1})\,\D^{n} x_{1} = 0.
\end{equation}
This is useful for computing vacuum expectation values.

Suppose we have a constraint. For us, we have the Gauss Law constraint
\begin{equation}
D_{i}\widetilde{E}^{i\widehat{I}}
=\partial_{i}\widetilde{E}^{i\widehat{I}} + \epsilon^{\widehat{I}\widehat{J}\widehat{K}}A_{i\widehat{J}}\widetilde{E}^{i}_{\phantom{i}\widehat{K}}=0,
\end{equation}
where in the Schrodinger picture we have,
\begin{equation}
\widetilde{E}^{i\widehat{I}} = -8\pi\immirzi G_{N}\hbar\frac{\delta}{\delta A_{i\widehat{I}}}.
\end{equation}
The constraint is linear in functional derivatives. What we can do is
look at the integral of the constraint against a test function,
\begin{equation}
\int\lambda_{\widehat{I}}D_{i}\widetilde{E}^{i\widehat{I}}\,\D^{n}x=0.
\end{equation}
If this is true for arbitrary $\lambda_{\widehat{I}}$, then integration
by parts
\begin{equation}
\int\lambda_{\widehat{I}}D_{i}\widetilde{E}^{i\widehat{I}}\,\D^{n}x
= 8\pi G_{N}\hbar\immirzi\int
D_{i}\lambda_{\widehat{I}}\frac{\delta}{\delta A_{i\widehat{I}}}\,\D^{n}x.
\end{equation}
Our constraint is then, when applied to a wave functional,
\begin{equation}
8\pi G_{N}\hbar\immirzi\int
D_{i}\lambda_{\widehat{I}}\frac{\delta}{\delta A_{i\widehat{I}}}\,\D^{n}x\,\Psi[A]=0.
\end{equation}
This is the first term in a Taylor expansion
\begin{equation}
\Psi[{A_{i}}^{\widehat{I}} + D_{i}\lambda^{\widehat{I}}] = \Psi[{A_{i}}^{\widehat{I}}].
\end{equation}
We can also have gauge transformations not built up from infinitesimal
transformations (e.g., time reversal) called \define{Large Gauge Transformations}.

We get to the Wilson line (a.k.a., the parallel propagator), we have the
holonomy
\begin{equation}
\mathcal{U}^{\widehat{J}}_{\phantom{J}\widehat{K}}
  = \mathcal{P}\exp\left[-\int_{C}A_{i}^{\phantom{i}\widehat{J}}{{\epsilon_{\widehat{I}}}^{\,\widehat{J}}}_{\!\widehat{K}}\, \D x^{i}\right],
\end{equation}
or suppressing indices and letting $\tau_{\widehat{I}}$ be the
generators of the gauge algebra,
\begin{equation}
\mathcal{U}
= \mathcal{P}\exp\left[-\int_{C}A_{i}^{\phantom{i}\widehat{I}}\tau_{\widehat{I}}\, \D x^{i}\right].
\end{equation}
Observe this transforms under change of ``coordinates'' as
\begin{equation}
\mathcal{U}\to g^{-1}(s_{2})\,\mathcal{U}\, g(s_{1}).
\end{equation}
We wish to construct invariants, so we construct closed loops then take
the trace of the holonomy $\mathcal{U}$ over the loop. This is an
overcomplete set of variables.

\begin{wrapfigure}{R}{7pc}
\centering\vskip-1pc
\includegraphics{img/2009-05-06.0}
\end{wrapfigure}
The way out is to consider the intersection, as doodled to the right.
We wish to consider this in detail. Give each edge of the graph
depicting the intersection a representation of $\SU(2)$.
We assign the vertex a Clebsch-Gordon coefficient. If we generalize this
to $n$-edges meeting at a vertex, we can use an intertwiner instead of a
Clebsch--Gordon coefficient. For $\SU(2)$, we have a neatway to combine
things as vertices:
\begin{center}
  \includegraphics{img/2009-05-06.1}
\end{center}
This works for $\SU(2)$, it may not necessarily work for an arbitrary
gauge group. So in short:
\begin{itemize}
\item At each node, we have an intertwiner;
\item At each edge, we have a representation of $\SU(2)$.
\end{itemize}

\begin{wrapfigure}{l}{8pc}
\centering %\vskip-1pc
\includegraphics{img/2009-05-06.2}
\end{wrapfigure}
The spin network (generically doodled to the left) gives a complete (but
not overcomplete) basis of states. Loop quantum gravity theorists like
to say a spin network is a state. What they mean is: the spin network is
a function of the connection $A$, which is all a state \emph{is} in
quantum gravity. A spin network eats in a value of $A$ and spits out a
complex number. One could use this for computation in, e.g., QCD (this
is group field theory).

We don't need a smooth connection, we can \emph{generalize} the
connection so it gives Wilson lines along finite parts of space (it
could be only where the edges are in fact). If we take the space of all
connections modulo gauge transformations, and complete it (so it's a
Hilbert space), then that's the Hilbert space we use.

Since a spin network is a state, we should probably define an inner
product between two spin networks. We should consider the usual way to
define the inner product on the Hilbert space just described as
something like
\begin{equation}
\langle\Psi\mid\Phi\rangle\sim\int_{\mathcal{A}/\mathcal{G}}\Psi^{*}[A]\Phi[A]\,[\D A].
\end{equation}
This is the only gauge-invariant inner product. We can get close to a
delta function, specifying geomtries down to the Planck length, using
weave states.

\bigbreak
Note: the spin networks doodled below,
\begin{center}
  \includegraphics{img/2009-05-06.3}
\end{center}
are distinct, since the lines are Wilson lines, the integral \emph{changes}.
