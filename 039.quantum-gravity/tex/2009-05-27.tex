\lecture

Remember, if we look at the massless states for open and closed strings
we get $G_{\mu\nu}$, $B_{\mu\nu}$, $\Phi$, and $A_{\mu}$.

We can also add fermions in two ways. We can look at spacetime fermions
(described not just by $x$ but also by fermionic coordinates). We end up
with the Green--Schwartz model.

The other way is to add world sheet fermions. We have
$X^{\mu}(\sigma,\tau)$ be our coordinates, $\psi^{\mu}(\sigma,\tau)$ be
our fermion.
To do this, it's easier to use $2$-spinors:
\begin{equation}
  \psi^{\mu} = \begin{pmatrix}
    \psi^{\mu}\\ \widetilde{\psi}^{\mu}
\end{pmatrix}
\end{equation}
There is a nice supersymmetry of the form
\begin{subequations}
\begin{align}
\delta X^{\mu} &\sim\eta\psi^{\mu}-\eta^{*}\widetilde{\psi}^{\mu}\\
\delta\psi^{\mu}&\sim\eta\partial_{z}X^{\mu}\\
\delta\widetilde{\psi}^{\mu}&\sim\eta^{*}\partial_{\bar{z}}X^{\mu},
\end{align}
\end{subequations}
where $z=\tau+\I\sigma$.
The Lagrangian looks like
\begin{equation}
  \mathcal{L}\sim\partial_{z}X^{\mu}\partial_{\bar{z}}X_{\mu}
  + \psi^{\mu}\partial_{\bar{z}}\psi_{\mu}
  + \widetilde{\psi}^{\mu}\partial_{z}\widetilde{\psi}_{\mu}.
\end{equation}
For a closed string, we can have boundary conditions
$\psi(\sigma+\ell)=\psi(\sigma)$ or $\psi(\sigma+\ell)=-\psi(\sigma)$.
We can make either choice consistently, the former is known as Ramond
boundary conditions, the latter is Neveu--Schwartz boundary condition.
We can consider a mode expansion
\begin{equation}
  \begin{split}
  R\colon\qquad\psi^{\mu} &= \sum_{m\in\ZZ} d^{\mu}_{m}\E^{\I 2\pi m\sigma/\ell}\\
NS\colon\qquad\psi^{\mu} &= \sum_{m\in\ZZ+\frac{1}{2}}b^{\mu}_{m}\E^{\I2\pi m\sigma/\ell}
  \end{split}
\end{equation}
We can look at the coefficients as creation/annihilation
operators\marginnote{Vacuum states}
\begin{equation}
  \begin{split}
  R\colon\qquad d^{\mu}_{r}\mid0\rangle &= 0\mbox{ for }r\geq1\\
NS\colon\qquad b^{\mu}_{r}\mid 0\rangle &= 0\mbox{ for } r\geq\frac{1}{2}
  \end{split}
\end{equation}
We can work out the commutators and anticommutators
\begin{equation}
\{d^{\mu}_{0},d^{\nu}_{0}\}=\eta^{\mu\nu},
\end{equation}
so the $d_{0}$ are $\Gamma$ matrices. The Ramond boundary conditions
yields spacetime spinors.

We can now ask what are the consistency conditions. There are five
types:
\begin{center}\let\oldarraystretch\arraystretch\renewcommand{\arraystretch}{1.5}
  \begin{tabular}{cccc}
    type & fields & strings allowed & boundary conditions imposed\\\hline
IIA & $\psi_{+}$, $\widetilde{\psi}_{-}$ & closed & $\psi$ has Ramond, $\widetilde{\psi}$ has NS\\
IIB & $\psi_{+}$, $\widetilde{\psi}_{+}$ & closed & both have Ramond\\
I & $\SO(32)$ & open and closed & \\
heterotic & $\SO(32)$ & half closed, half open and closed & \\
heterotic & $\mathtt{E}_{8}$ & half closed, half open and closed &
\end{tabular}\renewcommand{\arraystretch}{\oldarraystretch}
\end{center}\marginnote{T-duality}
A few more string theory miracles. Suppose $X^{9}\sim X^{9}+2\pi R$ (so
we have a cylinder). There are two implications:
\begin{enumerate}
\item The momentum is quantized $p^{9}=n/R$,
\item If we fix time, $X^{9}(\sigma+\ell)=X^{9}(\sigma) + 2\pi R\omega$
  where $\omega\in\ZZ$ (and it is, in fact, the winding number).
\end{enumerate}
We Fourier expand it,
\begin{equation}
X^{9} = \frac{2\pi R\omega}{\ell}\sigma + \frac{n}{R}\tau + \mbox{(oscillators)}.
\end{equation}
We can work out the mass spectrum
\begin{equation}
m^{2} = \frac{n^{2}}{R^{2}} + \frac{\omega^{2}R^{2}}{(\alpha')^{2}} +
\mbox{(oscillator contributions)}.
\end{equation}
Observe as $R\to\alpha'/R$ and $n\leftrightarrow\omega$, we end up with
exactly the same mass. The winding modes and momentum modes get switched.

This switching of $n$ and $\omega$ is equivalent to switching $\tau$ and
$\sigma$. The left moving momentum $p_{L}\to p_{L}$ but the right moving
momentum $p_{R}\to -p_{R}$. The fermions also change sign
$\widetilde{\psi}\to-\widetilde{\psi}$. This means
$\mbox{IIA}\leftrightarrow\mbox{IIB}$. This is the gist of T-duality.

For open strings, $R\to0$ corresponds to the center-of-mass motion as
$m\to\infty$. If we consider $\sigma\leftrightarrow\tau$, then Neumann
boundary conditions $\partial_{\sigma}X^{9}=0$ become Dirichlet boundary
conditions $\partial_{\tau}X^{9}=0$. The endpoints of the string sees
$D-1$ dimensions, the rest of the string sees all $D$ dimensions.
