\lecture

Today we will discuss only a crude quantization of strings, just to see
how a picture of quantum gravity might look like.

We will first consider an open string in flat Minkowski spacetime. Its
action,
\begin{equation}
\action=\frac{-1}{4\pi\alpha'}\int\gamma^{ab}\partial_{a}X^{\mu}\partial_{b}X^{\nu}\eta_{\mu\nu}\sqrt{-\gamma}\,\D^{2}\sigma.
\end{equation}
The equations of motion are roughly
\begin{equation}
\partial^{a}\partial_{a}X^{\mu}\sim0.
\end{equation}
We need to impose boundary conditions; the two obvious ones are Neumann
$\vec{n}^{a}\partial_{a}X^{\mu}=0$ at the endpoints (where $\vec{n}^{a}$
is the normal vector at the endpoints), and the Dirichlet
$X^{\mu}=\mbox{fixed}$ at the endpoints.

Observe the graviton has no such boundary conditions, \addenda{and open
  strings cannot describe massless particles of spin greater than 1, so we must
  use a closed string for the graviton.}\footnote{For more
on open strings, see Carlo Angelantonj and Augusto Sagnotti's review article \arXiv{hep-th/0204089}.}
We have 3 gauge invariances (2 diffeomorphisms, 1 Weyl invariance). This
is simple enough that we can choose 3 gauge conditions (technically we
should verify the consistency with the constraints, but it doesn't get
to the interesting point):
\begin{enumerate}[start=0]
\item Define $X^{\pm} = (X^{0}\pm X^{1})/\sqrt{2}$ (observe: this breaks
  Lorentz invariance),
\item Choose $X^{+}=\tau$ ``light-front coordinates'' (or ``light-cone coordinates'' or ``light-cone gauge''),
\item Choose $\partial_{\sigma}\gamma_{\sigma\sigma}=0$, and
\item Choose $\det(\gamma_{ab})=-1$.
\end{enumerate}
These choices have no deep physical meaning, they just simplify the
mathematics. The first two gauge conditions ($X^{+}=\tau$ and
$\partial_{\sigma}\gamma_{\sigma\sigma}=0$) deal with the
diffeomorphisms, whereas the last gauge condition
($\det(\gamma_{ab})=-1$) deals with Weyl invariance.
Before going further, we use the notation
\begin{equation}
\bar{X}^{-} = \frac{1}{\ell}\int^{\ell}_{0}X^{-}\,\D\sigma
\end{equation}
for the center-of-mass for the $X^{-}$ coordinate.

Plugging these choices into the action gives us,
\begin{multline}
  \action=\frac{-1}{4\pi\alpha'}\iint%\int\int^{\ell}_{0}
  \left[
\gamma_{\sigma\sigma}(2\partial_{\tau}\bar{X}^{-}-\partial_{\tau}X^{'}\partial_{\tau}X^{'})\phantom{\frac{1}{\gamma_{\sigma\sigma}}}\right.\\
-2\gamma_{\sigma\tau}(\partial_{\sigma}Y^{-}-\partial_{\tau}X^{i}\partial_{\sigma}X^{i})\\
+\left.\frac{1}{\gamma_{\sigma\sigma}}(1-\gamma_{\sigma\tau}^{2})\partial_{\sigma}X^{i}\partial_{\sigma}X^{i}
\right]\D\sigma\D\tau
\end{multline}
where $i=2,\dots,D$ and $X^{-}=\bar{X}^{-}+Y^{-}$.

By varying the action with respect to $Y^{-}$ gives us the equations of
motion
\begin{equation}
\partial_{\sigma}\gamma_{\sigma\tau}=0.
\end{equation}
Observe, since $Y^{-}$ doesn't appear in the action with a time
derivative, it acts like a Lagrange multiplier. Since
$\gamma_{\sigma\tau}=0$ at the boundary, and
$\partial_{\sigma}\gamma_{\sigma\tau}=0$ at the boundary, it follows
that $\gamma_{\sigma\tau}=0$ everywhere. This simplifies the action to
two pieces
\begin{equation}\label{eq:2009-05-20:reduced-action}
  \action=\frac{-1}{4\pi\alpha'}\iint%\int\int^{\ell}_{0}
  \left[
\gamma_{\sigma\sigma}(2\partial_{\tau}\bar{X}^{-}-\partial_{\tau}X^{'}\partial_{\tau}X^{'})+\frac{1}{\gamma_{\sigma\sigma}}\partial_{\sigma}X^{i}\partial_{\sigma}X^{i}
\right]\D\sigma\D\tau.
\end{equation}
We've eliminated $X^{+}$ and $Y^{-}$, so roughly speaking we have $D-2$
components. The transverse fluctuations are described by the second
term, the motion of the center-of-mass is described by the first term.
We write for the first term's momentum,
\begin{subequations}
\begin{equation}
p_{-} = \frac{-\ell}{2\pi\alpha'}\gamma_{\sigma\sigma} = -p^{+},
\end{equation}
and the second term's momentum,
\begin{equation}
\pi^{i} = \frac{p^{+}}{\ell}\partial_{\tau}X^{i}.
\end{equation}
\end{subequations}
We see the first term in the action
Eq~\eqref{eq:2009-05-20:reduced-action} is just the relativistic
particle, and the second term is just a harmonic oscillator. We can now write,
\begin{equation}
H = \frac{\ell}{4\pi\alpha' p^{+}}\int\left[2\pi\alpha'(\pi^{i})^{2}+\frac{1}{2\pi\alpha'}(\partial_{\sigma}X^{i})^{2}\right]\D\sigma
\end{equation}
We see
\begin{subequations}
  \begin{align}
    \partial_{\tau}p^{+} &= \frac{\partial H}{\partial\bar{X}^{-}}\quad\mbox{by Hamilton's equations}\\
    &= 0
  \end{align}
\end{subequations}
hence $p^{+}$ is a constant of motion. We can choose units $\ell/(2\pi\alpha'p^{+})=1$.

We now want to quantize it, we see that $\bar{X}^{-}$ and $p^{+}$ are
conjugate variables, so
\begin{equation}
[\bar{X}^{-},p^{+}]=\I.
\end{equation}
We can now write
\begin{equation}
X^{i}(\sigma,\tau) = \bar{X}^{i}
+ \underbrace{\left(\frac{p^{i}}{p^{+}}\right)\tau}_{\mathclap{\substack{\text{arbitrary choice}\\\text{makes life}\\\text{easier}}}}
+
\underbrace{\I\sqrt{2\alpha'}\sum_{n\neq0}\frac{1}{n}\alpha^{i}_{n}\E^{-\I\pi n\tau/\ell}\cos\left(\frac{\pi n\sigma}{\ell}\right)}_{\text{just a Fourier series expansion for } X}
\end{equation}
If we plug this in and find the conjugate momentum, we find the
commutation relations
\begin{equation}
[\bar{X}^{i},p^{j}]=\I\delta^{ij},
\end{equation}
and similarly
\begin{equation}
[\alpha^{i}_{m},\alpha^{j}_{n}]=m\delta^{ij}\delta_{0,m+n}.
\end{equation}
This shouldn't be too surprising,
$\alpha^{i}_{-n}=(\alpha^{i}_{n})^{\dagger}$ for $X$ to be real. (This
should be familiar: it \emph{is} a simple harmonic oscillator.)

Let's look at the states of the string, there is a vacuum
$\mid0,k\rangle$ with
\begin{subequations}
\begin{align}
p^{+}\mid0,k\rangle &= k^{+}\mid0,k\rangle\\
p^{i}\mid0,k\rangle &= k^{i}\mid0,k\rangle\\
\alpha^{i}_{m}\mid0,k\rangle &= 0\quad\mbox{for $m\geq0$}.
\end{align}
\end{subequations}
We basically have a bunch of harmonic oscillators.

Now, we can work out the Hamiltonian, and we will find,
\begin{equation}
H = \frac{1}{2p^{+}}(p^{i})^{2} +
\frac{1}{2p^{+}\alpha'}\sum^{\infty}_{n=1}\alpha^{i}_{-n}\alpha^{i}_{n}+\underbrace{A}_{\mathclap{\text{constant, zero-point energy}}}
\end{equation}
This is just the harmonic oscillator Hamiltonian. The constant term is
thus
\begin{equation}
A = (D - 2)\sum^{\infty}_{n=1}\frac{1}{2}n
\end{equation}
As everyone knows
\begin{equation}
\sum^{\infty}_{n=1}n = \frac{-1}{12}.
\end{equation}
There are two ways to see this. The first way is to use the Riemann zeta
function $\zeta(s) = \sum^{\infty}_{n=1}n^{-s}$. We analytically
continue it, and its value at $s=-1$ is $\zeta(-1)=-1/12$.

The second way, the physicist's way, is to consider
\begin{equation}
\sum^{\infty}_{n=1}n\E^{-\varepsilon n} = -\frac{\D}{\D\varepsilon}\sum^{\infty}_{n=1}\E^{-\varepsilon n}=-\frac{\D}{\D\varepsilon}\left(\frac{\E^{-\varepsilon}}{1-\E^{-\varepsilon}}\right)
=\frac{1}{\varepsilon^{2}}-\frac{1}{12}+\bigOh(\varepsilon).
\end{equation}
Being physicists, we throw away the divergent part, then take $\varepsilon\to0$.

Either way, we plug this into our Hamiltonian, we find:
\begin{equation}
\begin{split}
H & =\frac{1}{2p^{+}}(p^{i})^{2} +
\frac{1}{2p^{+}\alpha'}\sum^{\infty}_{n=1}N_{n} - \frac{D-2}{24}\\
&= p^{-}
\end{split}
\end{equation}
since $p^{-}$ generates time translations and we chose $X^{+}=\tau$. We
can use creation operators on the vacuum which gives excited states on
the string. We can ask what is the value for the mass squared (since
mass squared is a Lorentz invariant quantity),
\begin{equation}
\begin{split}
m^{2} &= p_{\mu}p^{\mu} = 2p^{+}p^{-} - (p^{i})^{2}\\
&= \frac{1}{\alpha'}\left(N-\frac{D-2}{24}\right)
\end{split}
\end{equation}
This means for the states of the string, $N=0$ (vacuum) has $m^{2}<0$,
so it's Tachyonic (which is bad!). For $N=1$,
\begin{equation}
m^{2} = \frac{1}{\alpha'}\left(1 - \frac{D-2}{24}\right)
\end{equation}
and the states are just $\alpha^{i}_{-1}\mid0,k\rangle$.

Lorentz invariance requires $m^{2}=0$, which requires
\begin{equation}
\frac{D-2}{24}=1\implies D=24+2=26.
\end{equation}
We can use the Chan--Paton generators to do some fancy tricks.
