\section[Homework II]{Introduction to Quantum Gravity: Homework II\footnote{This was handed out April 10, 2009.}}



I.~Electrodynamics as a constrained system.
\medbreak
Classical electrodynamics is described by a four-vector potential
$A_{\mu}$ and an antisymmetric field strength tensor
$F_{\mu\nu}=\partial_{\mu}A_{\nu}-\partial_{\nu}A_{\mu}$. The Lagrangian
is
\begin{equation*}
L = \frac{1}{4}F_{\mu\nu}F^{\mu\nu} + A_{\mu}\mathcal{J}^{\mu}
\end{equation*}
where $\mathcal{J}^{\mu}$ is the current four-vector, and the ordinary
electric and magnetic fields are
\begin{equation*}
E^{i}=F^{0i},\qquad B_{i}=\frac{1}{2}\epsilon_{ijk}F^{jk}
\end{equation*}
where $i$, $j$, $k$, \dots run from $1$ to $3$ and $\mu$, $\nu$, \dots
run from $0$ to $3$.

The tensor $F_{\mu\nu}$ is invariant under gauge transformations
\begin{equation*}
A_{\mu}\to A_{\mu}+\partial_{\mu}\Lambda
\end{equation*}
This suggests the system has constraints (to generate the gauge
transformations).

\begin{enumerate}[label=(\alph*),nosep]
  \item If you are not familiar with this formalism, convince yourself
    that it really does give you Maxwell's equations.
  \item Show the canonical variables (``generalized positions and
    momenta'') are $A_{i}$ and $E^{j}$, with Poisson brackets
    $\{A_{i}(\vec{x}),E^{j}(\vec{y})\}=\delta^{j}_{i}\delta^{3}(\vec{x}-\vec{y})$.
  \item Show that $A_{0}$ is a Lagrange multiplier, and the
    corresponding constraint is the Gauss law $C=\vec{\nabla}\cdot\vec{E}=0$.
  \item Show the constraint generates the gauge transformations of
    $A_{i}$, that is, $\{\int\Lambda(\vec{x})C(\vec{x})\,\D^{3}x, A_{i}(\vec{x}')\}\sim(A_{i}+\partial_{i}\Lambda)(\vec{x}')$.
  \item Correct any mistakes in algebra I have made.
\end{enumerate}


\medbreak\noindent{}II. \textbf{Extrinsic Curvature: Symmetry}\medbreak

Let $n^{a}$ be the unit normal to a $t=const.$ hypersurface. (Note this
implies $n_{a}=f\nabla_{a}t$ for some function $f$. Why?) Recall the
extrinsic curvature tensor is
\begin{equation}
K_{ab} = {q_{a}}^{c}\nabla_{c}n_{b}
\end{equation}
where $g_{ab}=q_{ab}+n_{a}n_{b}$.

\begin{enumerate}[label=(\alph*),nosep]
\item Show $K_{ab}=K_{ba}$.

  (Hint: use the fact $n_{a}=f\nabla_{a}t$ and find
  $n^{c}\nabla_{c}n^{b}$ in terms of $f$. The vector
  $a^{b}=n^{c}\nabla_{c}n^{b}$ is sometimes called the ``acceleration''.)
\item Show $K=\nabla_{a}n^{a}$ (where $K={K^{a}}_{a}$).
\end{enumerate}

\medbreak\noindent{}III. \textbf{Gauss--Codazzi}\medbreak

A ``spatial'' tensor ${T^{ab\dots}}_{cd\dots}$ is one with no normal
components, i.e.,
\begin{equation*}
n_{a}{T^{ab\dots}}_{cd\dots} = n_{b}{T^{ab\dots}}_{cd\dots} =
n^{c}{T^{ab\dots}}_{cd\dots} = \dots = 0
\end{equation*}
${q^{a}}_{b}$ is a projection operator, that is, it projects any index
into a ``purely spatial'' one. (Why?)

The three-dimensional (``spatial'') covariant derivative $D_{a}$ of a
spatial tensor can be defined as
\begin{equation*}
D_{e}{T^{ab\dots}}_{cd\dots} = {q^{a}}_{g}{q^{b}}_{h}\dots{q_{c}}^{i}{q_{d}}^{j}\dots{q_{e}}^{f}\nabla_{f}{T^{gh\dots}}_{ij\dots}
\end{equation*}
(that is, we take the ordinary four-dimensional covariant derivative and
then project all indices onto the $t=const.$ slice). The spatial
curvature is defined by the condition
\begin{equation*}
[D_{a},D_{b}]v_{c}={}^{(3)}\!{R_{abc}}^{d}v_{d}
\end{equation*}
for any spatial vector $v_{d}$. Using these facts, show that
\begin{enumerate}[label=(\alph*),nosep]
\item ${}^{(3)}\!{R_{abcd}} = {q_{a}}^{e}{q_{b}}^{f}{q_{c}}^{g}{q_{d}}^{h}R_{efgh}+K_{ac}K_{bd}-K_{ad}K_{bc}$
\item $R_{abcd}n^{d}=\nabla_{a}K_{bc} - \nabla_{b}K_{ac} - n_{a}(\nabla_{b}a_{c}-a_{b}a_{c}) + n_{b}(\nabla_{a}a_{c}-a_{a}a_{c})$
\item $R = {}^{(3)}\!R - K_{ab}K^{ab} + K^{2} + 2\nabla_{a}(n^{b}\nabla_{b}n^{a} - n^{a}\nabla_{b}n^{b})$
\end{enumerate}
Note that sign conventions differ reference to reference.

\medbreak\noindent{}IV. \textbf{ADM metric and extrinsic curvature}\medbreak

Suppose we decompose the metric in ADM form as
\begin{equation*}
\D s^{2} = N^{2}\,\D t^{2} - q_{ij}(\D x^{i} + N^{i}\,\D t)(\D x^{j} + N^{j}\,\D t)
\end{equation*}
\begin{enumerate}[label=(\alph*),nosep]
\item Confirm that the inverse metric is
  $$g^{ab} = \begin{pmatrix}\frac{1}{N^{2}} & -\frac{N^{i}}{N^{2}}\\-\frac{N^{j}}{N^{2}} & -q^{ij} + \frac{N^{i}N^{j}}{N^{2}} \end{pmatrix}$$
  where $q^{ij}$ is the inverse of $q_{ij}$ and I raise and lower
  indices with the spatial metric tensor $q_{ij}$.
\item Show that the extrinsic curvature tensor is
  \begin{equation*}
K_{ij} = \frac{1}{2N}\left(\partial_{t}q_{ij} - D_{i}N_{j} - D_{j}N_{i}\right)
  \end{equation*}
  Note that sign conventions differ from reference to reference.

  (Hint: as noted in problem II, the unit normal is $n_{a} = f\nabla_{a}t$. What is this in these coordinates? What is $f$?)
\end{enumerate}
