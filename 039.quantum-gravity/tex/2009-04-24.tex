\lecture

Spacelike hypersurfaces defined by metric, but in general we don't know
the metric in quantum gravity, so we're out of luck. (We are assuming
that $\mathcal{M}=\mathbb{R}\times\Sigma$ where $\mathbb{R}$ is time,
$\Sigma$ is a spatial 3-manifold, at least in the canonical approach.)
Anyways, back to the Dirac approach.

We're imposing the constraints as operators on the wave function. We
interpret the momentum constraint
\begin{equation}
\widehat{\mathcal{H}}^{i}\Psi[q] = 0
\end{equation}
as telling us the wave function is invariant under spatial
diffeomorphisms. We should be able to, at least have the urge to assume
$\widetilde{H}$ is telling us the wave function is invariant under
temporal diffeomorphism but realize that this is meaningless. We're on a
spatial hypersurfaces, after all!

The DeWitt supermetric\index{DeWitt Supermetric}\index{Supermetric}
\begin{equation}
G_{ijk\ell} = \frac{1}{2}\frac{1}{\sqrt{q}}(q_{ik}q_{j\ell} +
q_{i\ell}q_{jk} - q_{ij}q_{k\ell}).
\end{equation}
This is like a metric of metrics. We have the deformation of a metric
$\delta q^{ij}$ have the length
\begin{equation}
\|\delta q^{ij}\|^{2} = \int G_{ijk\ell}\delta q^{ij}\delta q^{k\ell}\,\D^{3}x.
\end{equation}
This defines the distance on the space of metrics. (The signature of the
supermetric is $(-+++++)$, we take each pair of indices as a single
index resulting in a 6-by-6 matrix.)

We introduce
\begin{equation}
\widehat{\pi}^{ij}=-\I\frac{\delta}{\delta q_{ij}}.
\end{equation}
We plug it into the Hamiltonian constraint, and write:
\begin{equation}
\widehat{\mathcal{H}} = 16\pi G G_{ijk\ell}\frac{\delta}{\delta q_{ij}}\frac{\delta}{\delta q_{k\ell}}
+\frac{1}{16\pi G}\sqrt{q}\,{{}^{(3)}R}.
\end{equation}
Resist the urge to make the first term a Laplacian. The Ricci 3-scalar
${{}^{(3)}R}$ is intuitively a sort of potential term, when viewed as a
function of $q_{ij}$. So then we plug it back into
\begin{equation}
\widehat{\mathcal{H}}\Psi[q]=0.
\end{equation}
This is the Wheeler--DeWitt Equation.\index{Wheeler--DeWitt Equation}

We need an inner product, wave functions alone do not suffice for a
quantum theory. There are 2 obvious thing to try to do.

The first thing, the ordinary Schrodinger picture using the 3-metric
\begin{equation}
\int\Psi^{*}\Phi\,[\D q] = \infty
\end{equation}
always since the Hamiltonian constraint, we need to gauge fix the inner
product, like a path integral with some extra symmetry.

\begin{itemize}
\item R.~P.~Woodard,
``Enforcing the Wheeler-de Witt Constraint the Easy Way''.
\journal{Class.~Quant.~Grav.} \volume{10} (1993), 483--496.\\
{\tt\doi{10.1088/0264-9381/10/3/008}}
\end{itemize}

We can think of $\widehat{\mathcal{H}}\Psi=0$ as a sort of Klein--Gordon
equation, and the correct inner product there is:
\begin{equation}
\int\Psi^{*}\overleftrightarrow{\frac{\delta}{\delta q}}\Phi\,[\D q].
\end{equation}
There is some ambiguity here, we have a number of inner products since
$\delta/\delta q$ is nonunique.
