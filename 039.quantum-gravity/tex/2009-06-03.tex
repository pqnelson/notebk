\lecture[Black Hole Thermodynamics]

The reference for today's lecture will be:
\begin{itemize}
\item Steven Carlip,
  ``Black Hole Thermodynamics and Statistical Mechanics''.
  \arXiv{0807.4520}, 35 pages.
\end{itemize}
Black holes are not black but thermal objects
\begin{equation}
kT=\frac{\hbar \kappa}{2\pi}
\end{equation}
where $\kappa$ is the ``surface gravity'' (which is $1/4 GM$ for
Schwarzschild black holes). They also have entropy
\begin{equation}
S_{BH} = \frac{A_{\text{horizon}}}{4\hbar G}.
\end{equation}
In a sense, this is part of quantum gravity, since it involves $\hbar$,
$G$.

Suppose we have a black hole with mass $M$, there is a ``box of gas''
with [characteristic] length $L$, temperature $T$, and mass $m$. We see
that the change in entropy is
\begin{equation}
\Delta S = \frac{\Delta E}{T} = -\frac{m}{T}.
\end{equation}
We better have a corresponding change in entropy for the black hole, or
else a black hole could be used for a perpetual motion machine. Suppose
the black hole is Schwarzschild, so its metric is
\begin{equation}
\D s^{2} = \left(1 - \frac{2GM}{r}\right)\D t^{2} - \left(1 -
\frac{2GM}{r}\right)^{-1}\D r^{2} - r^{2}\,\D\Omega^{2}.
\end{equation}
The proper distance $\rho$ in the Schwarzschild metric for the box to be
on the surface of the black hole is
\begin{equation}
\rho =\int^{2GM+\delta r}_{2GM}\frac{\D r}{\sqrt{1 - 2GM/r}}
\sim\sqrt{GM\,\delta r}
\end{equation}
We have $\rho=L$ when $\delta r\sim L^{2}/GM$. The mass $m$ is
redshifted when w eare far from the black hole, so the change of mass
for the black hole would be,
\begin{equation}
\Delta M\sim m\sqrt{1 - \frac{2GM}{2GM+\delta r}}\sim\frac{m L}{GM}.
\end{equation}
To maximize the loss of entropy, we need $L\sim\hbar/T$. We find then
\begin{equation}
\Delta S\sim\frac{GM\Delta M}{\hbar}\sim\frac{\Delta A}{\hbar G}.
\end{equation}
One of the laws of Black Hole Thermodynamics is $\Delta M=\frac{\kappa}{8\pi G}\Delta A$.

By the uncertainty principle, a virtual pair with energy $E$ can exist
for time $\sim \hbar/E$. Fort a virtual pair, with the negative energy
particle entering the black hole, locally it's as though the black hole
swallowed a particle of positive energy. Far away, it looks as though it
emitted a particle of energy $E$.
\begin{center}
  \includegraphics{img/2009-06-03.0}
\end{center}
The proper time for the particle to fall into the black hole would be
\begin{equation}
\tau\sim\sqrt{GM\,\delta r}\sim\frac{\hbar}{E}
\end{equation}
hence
\begin{equation}
E\sim\frac{\hbar}{\sqrt{GM\,\delta r}}.
\end{equation}
So to an observer far away, the energy is redshifted, so as seen from
infinity,
\begin{equation}
E_{\infty}\sim\frac{\hbar}{\sqrt{GM\,\delta r}}\sqrt{1 -
  \frac{2GM}{2GM+\delta r}}\sim\frac{\hbar}{GM}.
\end{equation}
This is precisely the energy corresponding to the black hole temperature.
(Usually temperature is derived first, then standard thermodynamics is
used to derive entropy.)

\marginnote{Review: QFT in curved spacetime}
Let us briefly review quantum field theory, using discrete momentum. Let
\begin{equation}
u_{k} = \exp(\I\vec{k}\cdot\vec{x}-\I\omega_{k}t),
\end{equation}
where $\omega_{k}=+\sqrt{|k|^{2}+m^{2}}$. The Fourier decomposition of
the free field,
\begin{equation}
\varphi = \sum_{k}(a_{k}u_{k} + a^{\dagger}_{k}u^{*}_{k}),
\end{equation}
we observe the commutation relations are such that
\begin{subequations}
\begin{align}
[a^{\dagger}_{k},a_{k'}] &= \delta_{k,k'},\\
a_{k}\mid0\rangle &= 0.
\end{align}
\end{subequations}
We generalize to curved spacetime,
\begin{equation}
(\Box + m^{2})u_{k}=0
\end{equation}
where $\Box=\nabla_{\mu}\nabla^{\mu}$ uses covariant derivatives.
There's an infinite number of orthogonal solutions (this happens in flat
spacetimes, too, e.g., Bessel functions, Fourier decomposition, etc.\
etc.\ etc.). We can choose two different modes
\begin{subequations}
\begin{equation}
\varphi = \sum_{k}(a_{k}u_{k} + a^{\dagger}_{k}u^{*}_{k})
 = \sum_{k}(\bar{a}_{k}\bar{u}_{k} + \bar{a}^{\dagger}_{k}\bar{u}^{*}_{k})
\end{equation}
where
\begin{equation}
\bar{u}_{k} =
\sum(\underbrace{\alpha_{ki}}_{\mkern-60mu\mathrlap{\text{Bogoliubov coefficients}}}u_{i} +\underbrace{\beta_{ki}}u^{*}_{i})
\end{equation}
\end{subequations}
We have two number operators, $N_{k}$ and $\bar{N}_{k}$. Using the
orthogonality of mode functions, we can prove,
\begin{equation}
\langle\bar{0}\mid N_{k}\mid\bar{0}\rangle=\sum_{j}|\beta_{jk}|^{2},
\end{equation}
so how do we choose what vacuum to work in? Hawking argued if we're far
away looking at the black hole region prior to collapse, there is a
natural vacuum: the Minkowski vacuum.

There is another derivation from Parikh and Wilczek~\cite{Parikh:1999mf}. It's a tunneling
approach, where particles tunnel from inside the black hole to the
outside. The particle position doesn't move outward, the position of the
event horizon moves inward. Then apply the WKB approximation. We have
\begin{equation}
\Gamma = \E^{-2\Im(\action/\hbar)}
\end{equation}
where $\action$ is the action.
Now we just write
\begin{equation}
\action = \int^{r_{\text{out}}}_{r_{\text{in}}}p_{r}\,\D r.
\end{equation}
We use the metric and there is a pole $\sqrt{1-2GM/r}$.

In ordinary quantum mechanics, time translation operator is
$\langle\mbox{out}\mid\exp(\I tH)\mid\mbox{in}\rangle$, for
thermodynamics it is $\operatorname{Tr}(\E^{-\beta H})$, if we make time imaginary we
get a partition function.

There are at least a dozen different modern ways to derive black hole entropy.
