\lecture

In classical general relativity, there are no local observables, so we
do not know what the right operators should be.  For a proof of the
absence of local observables, see:
\begin{itemize}
\item C.G.~Torre
``Gravitational Observables and Local Symmetries''.
\journal{Phys.~Rev.} \volume{D48} (1993) R2373--R2376(R); \arXiv{gr-qc/9306030}.\newline
{\tt\doi{10.1103/PhysRevD.48.R2373}}
\end{itemize}
A particular example of this is the ``problem of time''.

Consider a free scalar field in flat Minkowski spacetime, pick an
initial time slice and a final time slice to be the same in two
different foliations.  Is the time evolution in one foliation equivalent
to the time evolution in the other?
\begin{center}
\includegraphics{img/2009-04-03.0}
\includegraphics{img/2009-04-03.1}
\end{center}
We \emph{should} be able to ask if we have
\begin{equation}
|\psi_{1}(t)\rangle = \mathcal{U}|\psi_{2}(t)\rangle,
\end{equation}
where $\mathcal{U}$ is a unitary matrix indicating a change of bases.

Torre and Varadarajan~\cite{Torre:1997zs,Torre:1998eq} show, in general,
these are not related by a unitary matrix.  But we \emph{can} relate two
operators by orderings, which hold in the classical limit.

Determining time by spatial hypersurfaces requires using the metric.
Perhaps we can use the expectation value of the metric while demanding
it to be spatial but this depends on the wave function which we're
trying to find.

A lot of these problems come from thinking in the Schrodinger picture,
perhaps using the Heisenberg picture fixes it.  There are indications
from lower dimensional approaches that this may be correct.

\subsection{Quantization}

If we want to quantize general relativity, we need to talk about what it
means to quantize something. This is---for physicsts---the wrong
question. There may be more than one way to go to the classical limit,
but we work with the ones that are experimentally correct.

We start with some classical phase space with coordinates $(q,p)$ and
some Poisson bracket
\begin{equation}
\{p,q\}=1.
\end{equation}
We want to put hats on everything
\begin{equation}
[\widehat{p},\widehat{q}]=\I\hbar.
\end{equation}
We look for unitary irreducible representations on a Hilbert space, and
so on. That's the quantum theory. We have the rule
\begin{equation}
\{q,p\}\mapsto\frac{1}{\I\hbar}[\widehat{q},\widehat{p}],
\end{equation}
and for any observables $A$ and $B$ we have:
\begin{equation}
\{A,B\}\mapsto\frac{1}{\I\hbar}[\widehat{A},\widehat{B}].
\end{equation}
In general, this is impossible. There's a ``no go theorem'' from van
Hove proving there's no consistent way to do this.

We have to choose some subset of functions on the phase space, some set
of preferred phase space functions, that is ``small enough'' that this
mapping from Poisson brackets to commutators is consistent. But it must
be ``large enough'' so that any other function can be expressed in terms
of the preferred set. This is what we do when we quantize the Hydrogen
atom.

Suppose we have a phase space with a symmetry group $G$ which relates
any point with any other point. So we have the Poisson bracket be
preserved
\begin{equation}
\{gA,gB\} = \{A,B\}.
\end{equation}
If $H$ is the stabilizer of $x_{0}$ --- so $h\in H$ implies
$hx_{0}=x_{0}$ --- then $G/H$ is the phase space. In this case, we
choose the generators of the action of the group on the symmetric space
for the preferred functions to quantize.

The Stone--von Neumann theorem ensures the representation of
translations is unique up to unitary equivalence. But this theorem does
not hold in infinite-dimensions [i.e., for field theories].

There is another approach to quantization called ``deformation
quantization''. We have a quantization map,
\begin{equation}
\mathcal{Q}\colon\mbox{phase space}\to\mbox{operators}
\end{equation}
such that
\begin{enumerate}
\item Linearity: $\mathcal{Q}(c_{1}f_{1}+c_{2}f_{2}) = c_{1}\mathcal{Q}(f_{1}) + c_{2}\mathcal{Q}(f_{2})$
\item Preserves identity: $\mathcal{Q}(1)=\mathbf{1}$
\item $\mathcal{Q}(x)$, $\mathcal{Q}(p)$ are represented irreducibly
\item $\mathcal{Q}(\{f,g\}) = \frac{\I}{\hbar}[\mathcal{Q}(f),\mathcal{Q}(g)] + \mathcal{O}(1)$
\end{enumerate}
See:
\begin{enumerate}
\item P.~Tillman, ``Deformation Quantization, Quantization, and the
Klein-Gordon Equation''.
\journal{J.Phys.~A} \volume{40} (2007) 7017--7024; \arXiv{gr-qc/0610141}.\\
{\tt\doi{10.1088/1751-8113/40/25/S55}}
\item P.~Tillman, ``Deformation Quantization: From Quantum Mechanics to
Quantum Field Theory''. \arXiv{gr-qc/0610159}
\item S.~Twareque Ali, Miroslav Engli\v{s},
``Quantization Methods: A Guide for Physicists and Analysts''.
\journal{Rev.Math.Phys.} \volume{17} (2005) pp.391--490;
\arXiv{math-ph/0405065}.\\
{\tt\doi{10.1142/S0129055X05002376}}
\end{enumerate}

There is the path integral, which is just the continuous sum over the
paths, we write this formally as:
\begin{equation}
\int[\D q]\E^{\I S}.
\end{equation}
We can get different answers depending on how we define the derivative,
and we get extra terms of order $\hbar$. We think of these ambiguities
as normalization.
