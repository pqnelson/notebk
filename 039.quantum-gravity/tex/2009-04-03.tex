\lecture

In classical general relativity, there are no local observables, so we
do not know what the right operators should be.  For a proof of the
absence of local observables, see:
\begin{itemize}
\item C.G.~Torre
``Gravitational Observables and Local Symmetries''.
\journal{Phys.~Rev.} \volume{D48} (1993) R2373--R2376(R); \arXiv{gr-qc/9306030}.\newline
\doi{10.1103/PhysRevD.48.R2373}
\end{itemize}
A particular example of this is the ``problem of time''.

Consider a free scalar field in flat Minkowski spacetime, pick an
initial time slice and a final time slice to be the same in two
different foliations.  Is the time evolution in one foliation equivalent
to the time evolution in the other?
\begin{center}
\includegraphics{img/2009-04-03.0}
\includegraphics{img/2009-04-03.1}
\end{center}
We \emph{should} be able to ask if we have
\begin{equation}
|\psi_{1}(t)\rangle = \mathcal{U}|\psi_{2}(t)\rangle,
\end{equation}
where $\mathcal{U}$ is a unitary matrix indicating a change of bases.

Torre and Varadarajan~\cite{Torre:1997zs,Torre:1998eq} show, in general,
these are not related by a unitary matrix.  But we \emph{can} relate two
operators by orderings, which hold in the classical limit.

Determining time by spatial hypersurfaces requires using the metric.
Perhaps we can use the expectation value of the metric while demanding
it to be spatial but this depends on the wave function which we're
trying to find.

A lot of these problems come from thinking in the Schrodinger picture,
perhaps using the Heisenberg picture fixes it.  There are indications
from lower dimensional approaches that this may be correct.