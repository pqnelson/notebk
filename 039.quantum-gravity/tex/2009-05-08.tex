\lecture

We need the spatial topology to not change, otherwise we can end up with
a number of baby universes (``polymer topology'').

Let us consider the simplest spin network, we need 2 nodes and 3
edges. (If we had 2 edges, then we would obtain the identity spin
network.)
We don't have spin 0, as the propagator would be the identity.
\begin{center}
  \includegraphics{img/2009-05-08.1}
\end{center}
For each of these lines we have the Wilson line
\begin{equation}
\mathcal{U} = \mathcal{P}\exp[-\int A].
\end{equation}
We have three Wilson lines ${\mathcal{U}^{m_{1}}_{1}}_{n_{1}}$,
${\mathcal{U}^{m_{2}}_{2}}_{n_{2}}$, and ${\mathcal{U}^{m_{3}}_{3}}_{n_{3}}$.
Conceptually, the $\mathcal{U}$'s tell spin-$1/2$ objects rotate in spin
space. We have $m_{1}=1/2, -1/2$ for spin up and spin down (respectively).
We see $m_{2}$ also describes a spin-$1/2$ object, but $m_{3}$ describes
a spin-1 object with possible values $m_{3}=-1,0,+1$. We can now use the
Clebsch--Gordon coefficients $\langle j\,m\mid j_{1}\,m_{1}\,, j_{2}\,m_{2}\rangle$
and find
\begin{equation}
  \sum_{\substack{m_{1},m_{2},m_{3}\\ n_{1},n_{2},n_{3}}}
  {\mathcal{U}^{m_{1}}_{1}}_{n_{1}}{\mathcal{U}^{m_{2}}_{2}}_{n_{2}}{\mathcal{U}^{m_{3}}_{3}}_{n_{3}}
\langle1\,m_{3}\mid\frac{1}{2}\,m_{1}\,\frac{1}{2}m_{2}\rangle
\langle1\,n_{3}\mid\frac{1}{2}\,n_{1}\,\frac{1}{2}n_{2}\rangle,
\end{equation}
which is a function of $A$.

\subsection{Area Operator}

We take a surface $\Sigma$, we can ask ``What is the area of the
surface?'' Suppose we have some spin network that ``goes through'' our
surface $\Sigma$:
\begin{center}
  \includegraphics{img/2009-05-08.2}
\end{center}
We won't consider an edge of the spin network ``grazing'' the surface,
or lies inside the surface: we care about puncturing edges.

We will only really consider a simple example choosing a surface where
$x^{3}=0$, the area of the surface would be classically
\begin{equation}
A = \int\sqrt{{}^{(2)}g}\D^{2}x.
\end{equation}
We see
\begin{equation}
{}^{(2)}g = g_{11}g_{22} - 2(g_{12})^{2} = \widetilde{E}^{3}_{\widehat{I}} \widetilde{E}^{3\widehat{I}}.
\end{equation}
So the area is
\begin{equation}
A = \int\sqrt{\widetilde{E}^{3}_{\widehat{I}} \widetilde{E}^{3\widehat{I}}}\D^{2}x.
\end{equation}
Consider a more general surface with coordinates $\sigma^{1}$, $\sigma^{2}$.
Then our considerations change by
\begin{equation}
\widetilde{E}^{3}_{\phantom{3}\widehat{I}}\to\epsilon_{ijk}\frac{\partial x^{i}}{\partial\sigma^{1}}\frac{\partial x^{j}}{\partial\sigma^{2}}\widetilde{E}^{k}_{\phantom{k}\widehat{I}}.
\end{equation}
In the classical arena, the criteria for a ``small region'' is not
really well-defined; in the quantum arena, we just require a single
piercing of a spin network with the surface (as doodled in the margin).\marginpar{\includegraphics{img/2009-05-08.3}} We can turn this now into an
operator
\begin{equation}
\widetilde{E}_{\widehat{I}} = -8\pi G_{N}\immirzi\int\epsilon_{ijk}\frac{\partial x^{i}}{\partial\sigma^{1}}\frac{\partial x^{j}}{\partial\sigma^{2}}\frac{\delta}{\delta
A^{\widehat{I}}_{k}}\,\D\sigma^{1}\D\sigma^{2}.
\end{equation}
We need to consider $\delta\mathcal{U}/\delta A^{\widehat{I}}_{k}$. If
we didn't have path-ordering, then this would be trivial. But we must be
careful, since things do not commute. Consider the path doodled below:
\begin{center}
  \includegraphics{img/2009-05-08.4}
\end{center}
We have
\begin{subequations}
  \begin{align}
    \mathcal{U}(s_{2},s_{1})
    &= \mathcal{U}(s_{2},s)\mathcal{U}(s,s_{1})\\
    &= \mathcal{U}(s_{2},s+\varepsilon)\mathcal{U}(s+\varepsilon,s-\varepsilon)\mathcal{U}(s-\varepsilon,s_{1}).
  \end{align}
\end{subequations}
If $\varepsilon$ is small enough, we have
\begin{equation}
\frac{\delta}{\delta A^{\widehat{I}}_{i}(s)}\mathcal{U}(s_{2},s_{1})
= \mathcal{U}(s_{2},s)\left(-\tau_{\widehat{I}}\frac{\D x^{i}}{\D s}\right)\mathcal{U}(s,s_{1}).
\end{equation}
More generally,
\begin{equation}
\frac{\delta}{\delta A^{\widehat{I}}_{i}(x)}\mathcal{U}(s_{2},s_{1})
= \int\delta^{(3)}(C(s)-x)\,\mathcal{U}(s_{2},s)\left(-\tau_{\widehat{I}}\frac{\D x^{i}}{\D s}\right)\mathcal{U}(s,s_{1})\,\D s,
\end{equation}
and is zero if $x$ does not lie on the curve $C$.
We now see that
\begin{equation}
E_{\widehat{I}}\mathcal{U}(s_{2},s_{1}) = 8\pi\immirzi
G_{N}\int\epsilon_{ijk}\frac{\partial
    x^{i}}{\partial\sigma^{1}}\frac{\partial
    x^{j}}{\partial\sigma^{2}}\frac{\partial x^{k}}{\partial s}\delta^{(3)}(C(s)-x)\,\mathcal{U}(s_{2},s)\tau_{\widehat{I}}\,\mathcal{U}(s,s_{1})\,\D\sigma^{1}\D\sigma^{2}\D s.
\end{equation}
We see that
\begin{equation*}
\int\epsilon_{ijk}\frac{\partial
    x^{i}}{\partial\sigma^{1}}\frac{\partial
    x^{j}}{\partial\sigma^{2}}\frac{\partial x^{k}}{\partial s}\delta^{(3)}(C(s)-x)\,\D\sigma^{1}\D\sigma^{2}\D s
\end{equation*}
is called the ``oriented intersection number'' (it's $\pm1$ if $C(s)$
intersects $\Sigma$, and $0$ otherwise).

The moral of the story is that the oriented intersection number
$I(C,\Sigma)$ is used to find
\begin{equation}
E_{\widehat{I}}\,\mathcal{U}(s_{2},s_{1}) =
8\pi G_{N}\immirzi I(C,\Sigma)\mathcal{U}(s_{2},s)\tau_{\widehat{I}}\,\mathcal{U}(s,s_{1}),
\end{equation}
where $C(s)$ is the point of intersection.
