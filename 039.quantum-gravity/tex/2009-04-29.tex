\lecture

For reduced phase space quantization, we are left with one horrible
equation---as opposed to many horrible equations in the Dirac approach.

There's also the problem that we chose $t=-K$. It's sometimes not the
obvious choice for certain problems, for example the Schwarzschild
solution is completely scary. Do we get different quantum theories with
different time slicings? We don't know, this is kind of an anomaly
problem---is the quantum theory generally covariant in the reduced phase
space approach?

The Dirac approach has a few problems. We need to gauge fix the inner
product, but in practice we don't know how to do this. Another problem
is that the Wheeler-DeWitt equation has a piece that looks like the
product of two functional derivatives at a point, and this results in a
$\delta(0)$ contribution. This is a standard problem in quantum field
theory, regularization is needed. We could regulate it in theory as
\begin{equation}
\frac{\delta}{\delta g(x)}\frac{\delta}{\delta g(x')}\to
\frac{\delta}{\delta g(x)}K_{\varepsilon}(x,x')\frac{\delta}{\delta g(x')}
\end{equation}
where $K_{\varepsilon}(x,x')$ is some regulator invariant under spatial
diffeomorphisms, preserves the Poisson brackets, and becomes a $\delta$
function. No one has a proof that results are independent of how we
regulate. It could be possible it makes sense, we just don't know enough
about functional differential equations.

Even if this all worked out, the problem remains how to make sense of
basic variables. We have physical states be annihilated by the
constraints
\begin{equation}
\widehat{\mathcal{H}}\Psi_{\text{phy}}=0.
\end{equation}
We want a physical operator $\widehat{\mathcal{O}}$ to map physical
states to physical states
\begin{equation}
\widehat{\mathcal{O}}\Psi_{\text{phy}}=\Psi_{\text{phy}}'.
\end{equation}
This requires
\begin{equation}
[\widehat{\mathcal{H}},\widehat{\mathcal{O}}]\approx 0.
\end{equation}
We know no operators that do this. There have been proofs that such
operators are necessarily nonlocal, which we don't know how to deal
with. There's been work by some to make the Hamiltonian constraint
``almost local''.

This is where things stood roughly in the 1980s. There are some
simplified models where the Wheeler-DeWitt equation simplifies, just
freeze out degrees of freedom, very simplified settings. The
Wheeler-DeWitt equation becomes an ordinary differential equation.

In the early 1980s, two new approaches emerged:
\begin{enumerate}
\item Loop Quantum Gravity (which sought to simplify the Wheeler-DeWitt
  equation), and
\item String Theory (possibly contains quantum gravity).
\end{enumerate}
Then in the 1990s there was a new approach called dynamical
triangulations. We'll cover these three for the rest of the quarter.

\subsection{Loop Quantum Gravity}

We'll begin with gravity in the first-order formulation; i.e., a
tetrad/vierbein/frame field ${e^{I}}_{\mu}$. The capital Latin indices
track the basis vector, the Greek indices track the components of the
vector. We have
\begin{equation}
g^{\mu\nu}{e^{I}}_{\mu}{e^{J}}_{\nu}=\eta^{IJ}.
\end{equation}
It follows that
\begin{equation}
\eta_{IJ}{e^{I}}_{\mu}{e^{J}}_{\nu} = g_{\mu\nu}.
\end{equation}
We have an additional symmetry: local Lorentz symmetry.

Given such a tetrad, we can introduce the covariant derivative
\begin{equation}
\nabla_{\mu}A^{I} = \partial_{\mu}A^{I} + {{\omega_{\mu}}^{I}}_{J}A^{J},
\end{equation}
where ${{\omega_{\mu}}^{I}}_{J}$ is the spin connection.\index{Spin Connection}
Spin connections came about when people tried to introduce the spinor to
general relativity. We could demand metric compatibility to specify the
spin connection. The notation gets difficult, but let
$\widetilde{\nabla}_{\mu}$ be th eordinary covariant derivative for
tensors. The demand is that
\begin{equation}
\widetilde{\nabla}_{\mu}{e_{\nu}}^{J} +
          {{\omega_{\mu}}^{I}}_{J}{e_{\nu}}^{J} = 0
\end{equation}
determines the spin connection $\omega$ in terms of the frame $e$ and
Christoffel connection.

We can now do ordinary general relativity with this. So
\begin{equation}
[\nabla_{\mu},\nabla_{\nu}]A^{I} = {{R_{\mu\nu}}^{I}}_{J}A^{J},
\end{equation}
where
\begin{equation}
{{R_{\mu\nu}}^{\alpha}}_{\beta}{e^{I}}_{\alpha}{e_{J}}^{\beta} = {{R_{\mu\nu}}^{I}}_{J}.
\end{equation}
We write
\begin{equation}
A^{I} = {e_{\mu}}^{I}A^{\mu},
\end{equation}
and by our specification of the covariant derivative (specifically, the
spin connection) permits us to write the commutator.

The Einstein field equations are derived from the action:
\begin{equation}
\action_{EH} = \frac{1}{16\pi G}\int|e|e^{\mu I}e^{\nu J}R_{\mu\nu I J}\,\D^{4}x,
\end{equation}
where $|e| = \det|{e_{\mu}}^{I}|=\sqrt{-g}$ is the determinant of the
tetrad. We can express $R$ in terms of the spin connection, computed
directly from the commutator, as:
\begin{equation}
  {{R_{\mu\nu}}^{I}}_{J} = \partial_{\mu}{{\omega_{\nu}}^{I}}_{J}
  +{{\omega_{\mu}}^{I}}_{K}{{\omega_{\nu}}^{K}}_{J} - (\mu\leftrightarrow\nu).
\end{equation}
We can also treat the tetrad and connection as independent variables.
This isn't new: Palatini showed this holds for the metric and $\Gamma$
back in the 1930s.

The variation of the action, when treating tetrad and connection as
independent variables, gives us:
\begin{equation}
  \begin{split}
    \delta e: &\qquad e^{\nu I}R_{\mu\nu I J} = 0 = R_{\mu J}\\
    \delta\omega : &\qquad \nabla_{\mu}(e(e^{\mu I}e^{\nu J} - e^{\mu J}e^{\nu I})) = 0.
  \end{split}
\end{equation}
(The second variation is just the same as $\nabla^{\text{total}}_{\mu}{e_{\nu}}^{J}=0$.)
The Wheeler-DeWitt equation isn't more interesting, difficult, or
simple. But we can do interesting stuff!

We can write this in terms of forms
\begin{subequations}
\begin{align}
  e^{I} &= {e_{\mu}}^{I}\,\D x^{\mu},\\
  {\omega^{I}}_{J} &= {{\omega_{\mu}}^{I}}_{J}\,\D x^{\mu}.
\end{align}
\end{subequations}
We can write down the curvature 2-form
\begin{equation}
{\mathcal{R}^{I}}_{J} = \D{\omega^{I}}_{J} + {\omega^{I}}_{K}\wedge{\omega^{K}}_{J}.
\end{equation}
The action becomes (up to some sign error):
\begin{equation}
\action = \pm\frac{1}{64\pi G}\int\epsilon_{IJKL}e^{I}\wedge e^{J}\wedge\mathcal{R}^{KL}.
\end{equation}
This makes it \emph{look} neater.

Let us call
\begin{equation}
\mathcal{B}^{IJ} := e^{I}\wedge e^{J}.
\end{equation}
Then the action looks like
\begin{equation}
\action = \int\epsilon_{IJKL}\mathcal{B}^{IJ}\wedge\mathcal{R}^{KL}.
\end{equation}
We impose the condition $\mathcal{B}^{IJ} = e^{I}\wedge e^{J}$ (e.g.,
$\mathcal{B}^{IJ}\wedge\mathcal{B}^{KL} = e\varepsilon^{IJKL}$). The
converse (having $\mathcal{B}^{IJ}$ defined  by the condition
$\mathcal{B}^{IJ}\wedge\mathcal{B}^{KL} = e\varepsilon^{IJKL}$) is
\emph{almost} true. We then have:
\begin{equation}
  \action = \frac{1}{64\pi G}\int\left(
  \underbrace{\epsilon_{IJKL}\mathcal{B}^{IJ}\wedge\mathcal{R}^{KL}}_{\text{a ``BF'' theory}}
  + \phi_{IJKL}\underbrace{(\mathcal{B}^{IJ}\wedge\mathcal{B}^{KL} - e\varepsilon^{IJKL})}_{\text{constraint}}
  \right),
\end{equation}
and the $\phi_{IJKL}$ are Lagrange multipliers. If we didn't have tje
constraint, we'd have a flat spacetime with a sort of gauge theory
living on it. % (This is where topological QFT comes into play methinks.)

So writing things in new variables suggests new approaches.
Let us try some new variables.

\subsection{Self-Dual 2-Forms}

First, we define a $*$ operator on a 2-form:
\begin{subequations}
\begin{align}
F^{*}_{IJ} &= \frac{-\I}{2}\epsilon_{IJKL}F^{KL},\\
F^{**} &= F,
\end{align}
\end{subequations}
where $F_{[IJ]}=0$ (i.e., $F$ is antisymmetric). So this is a dual of
$F$ (there are many notions of ``duality''). We say $F$ is
\define{Self-Dual} if
\begin{equation}
F^{*}=F,
\end{equation}
and $F$ is \define{Anti-Self-Dual} if
\begin{equation}
F^{*}=-F.
\end{equation}
We can write, for an arbitrary 2-form $F$,
\begin{equation}
F^{\pm IJ} = \frac{1}{2}\left(F^{IJ}\pm F^{* IJ}\right).
\end{equation}
We can define a self-dual connection,
\begin{equation}
{A_{\mu}}^{IJ} = \frac{1}{2}\left({\omega_{\mu}}^{IJ} - \frac{\I}{2}{\epsilon^{IJ}}_{KL}{\omega_{\mu}}^{KL}\right).
\end{equation}
We define the self-dual curvature as:
\begin{subequations}
\begin{align}
{F_{\mu\nu}}^{IJ}
&= \partial_{\mu}{A_{\nu}}^{IJ} + {A_{\mu}}^{I}_{K}{A_{\nu}}^{KL} - (\mu\leftrightarrow\nu)\\
&= \frac{1}{2}({R_{\mu\nu}}^{IJ} + {R_{\mu\nu}}^{IJ*}).
\end{align}
\end{subequations}
We complexified, doubling the degrees of freedom, roughly speaking the
self-dual and antiself-dual splits the degrees of freedom.

The Ashtekar--Sen Action is then:
\begin{equation}
\action_{AS} = \frac{1}{8\pi G}\int e e^{\mu I}e^{\nu J}F_{\mu\nu IJ}\,\D^{4}x.
\end{equation}
By treating the self-dual connection as separate [independent] from the
tetrad, we get the Einstein field equations. (There is actually an extra
term like $\sim e^{\mu I}e^{\nu J}{R_{\mu\nu}}^{KL} = \epsilon_{IJKL}R^{IJKL}=0$.)
The constraints simplify \emph{dramatically}.

We need to have \define{Reality Conditions} so we don't have anything
imaginary. Classically, they are:
\begin{subequations}
\begin{align}
{\omega_{\mu}}^{IJ} &= {A_{\mu}}^{IJ} + {A_{\mu}}^{IJ*}\\
\frac{-\I}{2}{\epsilon^{IJ}}_{KL}{\omega_{\mu}}^{KL}
&={A_{\mu}}^{IJ} - {A_{\mu}}^{IJ*}\\
&=\frac{-\I}{2}{\epsilon^{IJ}}_{KL}({A_{\mu}}^{IJ} + {A_{\mu}}^{IJ*}).
\end{align}
\end{subequations}
The statement is that
\begin{equation}
{A_{\mu}}^{IJ} - {A_{\mu}}^{IJ*} = \frac{-\I}{2}{\epsilon^{IJ}}_{KL}({A_{\mu}}^{IJ} + {A_{\mu}}^{IJ*}).
\end{equation}
This is a second-class condition, which relates the real and imaginary
parts of the connection.

Given the change of variables to the Ashtekar--Sen action, we can do a
$3+1$ dimensional split. We will introduce new indices ($\widehat{I}$,
$\widehat{J}$, \dots = 1, 2, 3) for tetrad indices and ($i$, $j$, \dots =
1, 2, 3) for coordinate indices. Let's look at the components:
\begin{subequations}
\begin{align}
{A_{\mu}}^{0\widehat{L}} &= \frac{1}{2\I}{\epsilon^{0\widehat{L}}}_{\widehat{I}\widehat{J}}{A_{\mu}}^{\widehat{I}\widehat{J}}=\frac{1}{2\I}{A_{\mu}}^{\widehat{L}},\\
{A_{\mu}}^{\widehat{I}\widehat{J}} &= \frac{1}{2}{\epsilon_{0}}^{\widehat{I}\widehat{J}\widehat{K}}A_{\mu\widehat{L}}.
\end{align}
\end{subequations}
If we went back to the original spin connection, we find it is related
to the extrinsic curvature\index{Spin Connection!And Extrinsic Curvature}
\begin{equation}
{\omega_{i}}^{0\widehat{I}} = {K_{i}}^{\widehat{I}}.
\end{equation}
We can define
\begin{equation}
{\Gamma_{i}}^{\widehat{I}} = \frac{1}{2}{\epsilon_{0}}^{\widehat{I}\widehat{J}\widehat{K}}\omega_{i\widehat{J}\widehat{K}},
\end{equation}
which is basically the connection on the spatial hypersurface ignoring
the embedding. We do this so we can write the self-dual connection as
\begin{align}
{A_{i}}^{\widehat{I}} &= {\Gamma_{i}}^{\widehat{I}} + \I{K_{i}}^{\widehat{I}} \\
&= \begin{pmatrix}\mbox{Ordinary Connection}\\
\mbox{On the Slice}\end{pmatrix} + \I\begin{pmatrix}\mbox{Extrinsic Curvature}\\
\mbox{On the Slice}\end{pmatrix}\nonumber
\end{align}
We can generalize, letting $\immirzi$ be ``some parameter''
\begin{equation}
A^{(\immirzi)\widehat{I}}_{i} =  {\Gamma_{i}}^{\widehat{I}} + \immirzi{K_{i}}^{\widehat{I}}
\end{equation}
where $\immirzi$ is the \define{Immirzi--Barbero Parameter}. The
self-dual connection is really just a canonical transformation.
