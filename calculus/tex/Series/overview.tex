%%
%% overview.tex
%% 
%% Made by Alex Nelson
%% Login   <alex@black-cherry>
%% 
%% Started on  Fri Jun  1 10:29:47 2012 Alex Nelson
%% Last update Fri Jun  1 10:46:31 2012 Alex Nelson
%%

\section{Overview of Series}
\begin{defn}
We have a \define{Series} be an infinite sum
\begin{equation}
\sum^{\infty}_{k=0}s_{k}
\end{equation}
which may or may not be finite. 
\end{defn}

\N{Naive Properties}
What are some properties we expect to hold? Well, we expect three
main properties. If we consider the sequence of numbes
$p_{k+1}=p_{k}+s_{k+1}$ which are partial sums (i.e.,
$p_{k}=s_{0}+\dots+s_{k}$), then we want the following properties
to hold:
\begin{description}
\item[Regularity:] If the sequence of partial sums $p_{k}$
  converges to $x$, then the series should be equal to $x$.
\item[Linearity:] We expect
  $\sum_{n}(a_{n}+b_{n})=(\sum_{n}a_{n})+(\sum_{n}b_{n})$ and,
  for any constant $c$, $\sum_{n}ca_{n}=c\sum_{n}a_{n}$.
\item[Stability:] This lets us do things like
  $\sum_{n=0}a_{n}=a_{0}+\sum_{n=0}a_{n+1}$. 
\end{description}
So lets consider an example.

\begin{ex}
Lets consider the classic geometric series
\begin{equation}
\begin{split}
f(x) &= 1 + x + x^{2} + x^{3} + \dots + x^{n} + \dots \\
&=\sum^{\infty}_{n=0}x^{n}.
\end{split}
\end{equation}
Observe that stability lets us write
\begin{equation}
\begin{split}
f(x) &= \sum_{n=0}^{\infty}x^{n}\\
&= 1 + \sum^{\infty}_{n=0}x^{n+1}.
\end{split}
\end{equation}
Linearity lets us factor out an $x$:
\begin{equation}
1+\sum^{\infty}_{n=0}x^{n+1} = 1+x\sum^{\infty}_{n=0}x^{n}
\end{equation}
We also may observe the series on the right hand side is
precisely $f(x)$, we may plug this in:
\begin{equation}
1+\sum^{\infty}_{n=0}x^{n+1} = 1+xf(x).
\end{equation}
But tracing back through our equalities, this is equal to
$f(x)$. In other words, we have the relationship
\begin{equation}
f(x) = 1 + xf(x).
\end{equation}
So we subtract $xf(x)$ from both sides
\begin{equation}
(1-x)f(x) = 1
\end{equation}
Then divide through by $1-x$ obtaining
\begin{equation}
f(x)=\frac{1}{1-x}.
\end{equation}
Thus we have deduced the geometric series ``converges'' to $(1-x)^{-1}$.
\end{ex}
