%%
%% derivatives.tex
%% 
%% Made by alex
%% Login   <alex@tomato>
%% 
%% Started on  Thu Nov  3 14:31:59 2011 alex
%% Last update Wed May 30 10:17:56 2012 Alex Nelson
%%

\subsection{Derivative}
\M
We still have these bonus parts when considering the slope. That
is, for some nonzero $\Delta x$ and arbitrary $f(x)$, we have 
\begin{equation}
f(x+\Delta x)-f(x)=h(x)\Delta x+\bigO\left((\Delta x)^{2}\right)
\end{equation}
which gives us
\begin{equation}
\frac{\Delta f(x)}{\Delta x}=h(x)+\bigO(\Delta x).
\end{equation}
We want to get rid of that $\Delta x$ on the right hand side. How
to do this?

Lets be absolutely clear before moving on. We want to consider
the slope of our function $f$. To do this we considered a nonzero
$\Delta x$, and then constructed
\begin{equation}
\Delta f(x)=f(x+\Delta x)-f(x).
\end{equation}
This function described the difference between the values of $f$
at $x+\Delta x$ and at $x$. So, to describe the \emph{rate of change}
we take
\begin{equation}
\frac{\Delta f(x)}{\Delta x}= h(x)+\bigO(\Delta x).
\end{equation}
But we want to describe the \emph{instantaneous} rate of
change. Although this sounds scary, it really means we don't want
to work with some extra parameter $\Delta x$. We want to consider
the rate of change and describe it in such a way that it doesn't
depend on $\Delta x$. 

So what do we do? Well, the first answer is to set $\Delta x$ to
be 0. This is tempting, but wrong, because we end up with
\begin{equation}
\frac{f(x+\Delta x)-f(x)}{\Delta x}\mapsto \frac{f(x+0)-f(x)}{0}
\end{equation}
which is not well-defined. The second answer is to consider the
limit $\Delta x\to 0$, so we can avoid division-by-zero
errors. This is better, and we write
\begin{equation}
\lim_{\Delta x\to 0}\frac{\Delta f(x)}{\Delta x}=\frac{\D f(x)}{\D x}
\end{equation}
following Leibniz's notation. This is the definition of the
derivative of $f$.

%\subsection{Divide by Zero, and You Go To Hell!}
\N{Divide by Zero, and You Go To Hell!}
Well, formally, we need to take the limit $\Delta x\to 0$. What
does that mean for the left hand side? Could we accidentally be
dividing by $\Delta x$ and get infinities? This is a problem we
have to seriously consider.

The first claim is that
\begin{equation}\label{eq:deltaFisLinearInDeltaX}
f(x+\Delta x)=f(x)+\bigO(\Delta x).
\end{equation}
This would imply that
\begin{equation}
\frac{\Delta f(x)}{\Delta x} = h(x)+\bigO(\Delta x)
\end{equation}
for some function $h(x)$. There would be no division by zero
errors, but still we have to prove that equation
\eqref{eq:deltaFisLinearInDeltaX} is true \emph{in general},
i.e. for \emph{every} function $f(x)$. We have seen it is true
only for polynomials.

So, let us consider a function
\begin{equation}
F(x)=\frac{1}{x^{n}}
\end{equation}
for some $n\in\NN$. What to do? Well, lets consider what happens
when $x\mapsto x+\Delta x$, we change $x$ to be $x+\Delta x$. We have
\begin{equation}
F(x+\Delta x)=\frac{1}{(x+\Delta x)^{n}}
\end{equation}
by definition of $F$. We would expect then
\begin{equation}
\Delta F(x)=F(x+\Delta x)-F(x)=\frac{1}{(x+\Delta x)^{n}}-\frac{1}{x^{n}}.
\end{equation}
What to do? Well, lets gather the terms together
\begin{equation}
\Delta F(x)=\frac{x^{n}}{x^{n}(x+\Delta x)^{n}}-\frac{(x+\Delta
x)^{n}}{x^{n}(x+\Delta x)^{n}}
\end{equation}
which we can do, since we multiply both terms by 1 (the first
term is $x^{n}/x^{n}$, the second term is $(x+\Delta
x)^{n}/(x+\Delta x)^{n}$). We can then add the fractions together
\begin{equation}
\Delta F(x)=\frac{x^{n}-(x+\Delta x)^{n}}{x^{n}(x+\Delta x)^{n}}
\end{equation}
and consider expanding the numerator and denominators out. We see
that to first order, we have
\begin{equation}
x^{n}-(x+\Delta x)^{n}=-nx^{n-1}\Delta x+\bigO\left((\Delta
x)^{2}\right)
\end{equation}
which shouldn't be surprising (we've proven this many times so
far!). The denominator expands out to be
\begin{equation}
x^{n}(x+\Delta x)^{n}=x^{n}(x^{n}+\bigO(\Delta x))
\end{equation}
which, for nonzero $x$, cannot be made 0.

We combine these results to write
\begin{equation}
\Delta F(x)=\frac{-nx^{n-1}\Delta x+\bigO\left((\Delta x)^{2}\right)}{x^{n}(x^{n}+\bigO(\Delta x))}.
\end{equation}
We observe that we can factor out a $\Delta x$ in the numerator
(the upstairs part of the fraction) and then we can divide both
sides by it:
\begin{equation}
\frac{\Delta F(x)}{\Delta x}=\frac{-nx^{n-1}+\bigO(\Delta x)}{x^{n}(x^{n}+\bigO(\Delta x))}.
\end{equation}
So what happens if we set $\Delta x=0$ on the right hand side? Do
we run into problems? Well, we run into problems on the left hand
side, but not on the right hand side.

So what to do? Well, the formal mathematical procedure is to take
the \emph{limit} $\Delta x\to0$, which then lets us write
\begin{equation}
\lim_{\Delta x\to0}\frac{\Delta F(x)}{\Delta x}=\frac{\D F(x)}{\D
x}
\end{equation}
for the left hand side. For the right hand side, we can
symbolically just set $\Delta x=0$. This is sloppy, because it's
not quite true. But this is what's done in practice. We get
\begin{equation}
\lim_{\Delta x\to0}\frac{-nx^{n-1}+\bigO(\Delta x)}{x^{n}(x^{n}+\bigO(\Delta x))}
= \frac{-nx^{n-1}}{x^{n}(x^{n}+0)}.
\end{equation}
Observe that we can combine these results to write
\begin{equation}
\frac{\D F(x)}{\D x}=\frac{-n}{x^{2n-(n-1)}}=-nx^{-n-1}.
\end{equation}
There was no risk of dividing by zero anywhere.

%\subsection{Product Rule}
\N{Product Rule}
Suppose we have two arbitrary functions $f(x)$ and $g(x)$. Lets
define a new function
\begin{equation}\label{eq:whereWeDefinedH}
h(x)=f(x)g(x),
\end{equation}
then what's 
\begin{equation}
\Delta h(x)=?
\end{equation}
I don't know, let us look. We see that we first pick some nonzero
$\Delta x$ and then consider
\begin{equation}
\Delta h(x)=h(x+\Delta x)-h(x).
\end{equation}
Now we plug in this expression to equation
\eqref{eq:whereWeDefinedH}, the equation where we defined $h$,
and we find
\begin{equation}
h(x+\Delta x)-h(x)=f(x+\Delta x)g(x+\Delta x)-f(x)g(x).
\end{equation}
We do the following trick: add to both sides
\begin{equation}
0=f(x)g(x+\Delta x)-f(x)g(x+\Delta x)
\end{equation}
and we obtain
\begin{equation}
\begin{split}
\Delta h(x)=&f(x+\Delta x)g(x+\Delta x)-f(x)g(x) \\
&\quad+f(x)g(x+\Delta x)-f(x)g(x+\Delta x).
\end{split}
\end{equation}
We can gather terms together
\begin{equation}
\Delta h(x)=(f(x+\Delta x)-f(x))g(x+\Delta
x)+f(x)(-g(x)+g(x+\Delta x))
\end{equation}
which simplifies to
\begin{equation}
\Delta h(x)=\Delta f(x)\cdot g(x+\Delta x)+f(x)\cdot\Delta g(x).
\end{equation}
As usual, we divide both sides by $\Delta x$
\begin{equation}
\frac{\Delta h(x)}{\Delta x}=\frac{\Delta f(x)}{\Delta x}g(x+\Delta x)
+f(x)\frac{\Delta g(x)}{\Delta x}.
\end{equation}
By taking the limit $\Delta x\to 0$ we end up with
\begin{equation}
\frac{\D h(x)}{\D x}=\frac{\D f(x)}{\D x}g(x)+f(x)\frac{\D g(x)}{\D x}.
\end{equation}
Notice that we implicitly noted
\begin{equation}
\lim_{\Delta x\to 0}g(x+\Delta x)=g(x).
\end{equation}
Of course, we assume that $g$ is continuous at $x$, which turns
out to be correct since differentiability implies continuity (we
will prove this at some other time).

\begin{thm}[Product Rule]
Let $f$, $g$ be differentiable and
\begin{equation}
h(x)=f(x)g(x),
\end{equation}
then
\begin{equation}
\frac{\D h(x)}{\D x}=\frac{\D f(x)}{\D x}g(x)+f(x)\frac{\D g(x)}{\D x}
\end{equation}
is the derivative.
\end{thm}

We've already proven this. So lets consider an example.
\begin{equation}
f(x)=x^{n-1}
\end{equation}
where $(n-1)\in\NN$, and
\begin{equation}
g(x)=x.
\end{equation}
Thus
\begin{equation}
h(x)=x^{n}.
\end{equation}
The claim is that
\begin{equation}
\frac{\D h(x)}{\D x}=nx^{n-1}.
\end{equation}
Is this surprising? No, but the surprising part is that it is a
\emph{consequence} of the product rule. How to prove this? Well,
we need to do it by induction on $n$. 

\textbf{Base Case} ($n=2$) we see that
\begin{equation}
f(x)=x
\end{equation}
and we can see immediately that
\begin{equation}
\frac{\D h(x)}{\D x}=\frac{\D f(x)}{\D x}g(x)+f(x)\frac{\D g(x)}{\D x}=1\cdot x+x\cdot 1=2x.
\end{equation}
So this proves the base case.

\textbf{Inductive Hypothesis:} suppose this will work for
arbitrary $n$.

\textbf{Inductive Case:} for $n+1$, we have
\begin{equation}
\frac{\D h(x)}{\D x}=\frac{\D (x^{n-1}x)}{\D x}g(x)+x^{n}\frac{\D g(x)}{\D x}
\end{equation}
Observe we can consider the first term and apply the base case
\begin{equation}
\frac{\D (x^{n-1}x)}{\D x}g(x)=\frac{\D (x^{n-1})}{\D x}xg(x)+x^{n-1}\frac{\D (x)}{\D x}g(x)
\end{equation}
which is then
\begin{equation}
\frac{\D (x^{n-1}x)}{\D x}g(x)=(n-1)(x^{n-2})xg(x)+x^{n-1}g(x)=nx^{n-1}g(x).
\end{equation}
The second term is (recall $g(x)=x$) simpler
\begin{equation}
x^{n}\frac{\D g(x)}{\D x}=x^{n}.
\end{equation}
We add both of these together to find
\begin{equation}
\frac{\D h(x)}{\D x}=nx^{n}+x^{n}=(n+1)x^{n}.
\end{equation}
But this is precisely what we wanted! And that concludes the inductive proof.

%\subsection{Chain Rule}
\N{Chain Rule}
We can combine functions together through composition. This looks
like
\begin{equation}
h(x)=g(f(x)).
\end{equation}
The question is: what's the derivative (rate of change) of $h$ in
terms of the derivatives of $g$ and $f$?

Here we really take advantage of big-O notation. Observe for some
nonzero $\Delta x$ we have
\begin{equation}
h(x+\Delta x)=g(f(x+\Delta x))
\end{equation}
Let us write
\begin{equation}
u = f(x)
\end{equation}
thus
\begin{equation}
h(x+\Delta x)=g(u+\Delta u).
\end{equation}
Its finite difference is then
\begin{subequations}\label{eq:derivatives:intermediateResult}
\begin{align}
\Delta h(x) &= h(x+\Delta x)-h(x)\\
&=g(u+\Delta u)-g(u)
\end{align}
\end{subequations}
Observe
\begin{subequations}\label{eq:chainRule:defnOfDeltaf}
\begin{align}
\Delta u &= \Delta f(x)\\
&= f(x+\Delta x)-f(x)
\end{align}
\end{subequations}
implies
\begin{equation}
g(u+\Delta u)=g(f(x+\Delta x)).
\end{equation}
We plug this back into Eq \eqref{eq:derivatives:intermediateResult}
and obtain
\begin{equation}
\Delta h(x) = g(f(x+\Delta x))-g(f(x)).
\end{equation}
Divide both sides by $\Delta x$, in order to consider the
derivative
\begin{equation}
\frac{\Delta h(x)}{\Delta x}=\frac{g\left(f(x+\Delta x)\right)-g\left(f(x)\right)}{\Delta{}x}.
\end{equation}
Now what do we do?
%% \begin{comment}
%% but we argued that
%% \begin{equation}\label{eq:chainRule:defnOfDeltaf}
%% f(x+\Delta x)=f(x)+F(x)\Delta x +\bigO\left((\Delta x)^{2}\right).
%% \end{equation}
%% Lets plug this in
%% \begin{equation}
%% h(x+\Delta x)=g(f(x)+F(x)\Delta x +\bigO\left((\Delta x)^{2}\right)).
%% \end{equation}
%% So we conclude that
%% \begin{equation}
%% \Delta h(x)=g(f(x)+F(x)\Delta x +\bigO\left((\Delta x)^{2}\right))-g(f(x)).
%% \end{equation}
%% We can divide both sides by $\Delta x$ simply
%% \begin{equation}
%% \frac{\Delta h(x)}{\Delta x}=\frac{g\left(f(x)+F(x)\Delta x+\bigO\left((\Delta x)^{2}\right)\right)-g(f(x))}{\Delta x}.
%% \end{equation}
%% Now what to do?
%% \end{comment}

Well, we can do the following trick: multiply both sides by
\begin{equation}
1=\frac{\Delta f(x)}{\Delta f(x)}.
\end{equation}
This would give us
\begin{equation}
\frac{\Delta h(x)}{\Delta x}=\frac{g\left(f(x+\Delta x)\right))-g\left(f(x)\right)}{\Delta f(x)}
\frac{\Delta f(x)}{\Delta x}.
\end{equation}
But what is $\Delta f(x)$? We recall equation \eqref{eq:chainRule:defnOfDeltaf}
and write
\begin{equation}
\Delta f(x) = f(x+\Delta x)-f(x)=F(x)\Delta x +\bigO\left((\Delta x)^{2}\right).
\end{equation}
Using this, we can simplify our equation
\begin{equation}
\frac{\Delta h(x)}{\Delta x}=\frac{g(f(x)+\Delta f(x))-g(f(x))}{\Delta f(x)}
\frac{\Delta f(x)}{\Delta x}.
\end{equation}
Observe that we may take the limit as $\Delta x\to0$, which gives
us
\begin{equation}
\frac{\D h(x)}{\D x}=\frac{\D g(f(x))}{\D f(x)}\frac{\D f(x)}{\D x}
\end{equation}
which intuitively looks like fractions cancelling out to give the
right answer. Although this is the intuitive idea, \emph{DO NOT}
cancel terms!

Moreover, we should really clarify what is meant by
\begin{equation}
\frac{\D g(f(x))}{\D f(x)}=\dots
\end{equation}
Let us first consider
\begin{equation}
u=f(x).
\end{equation}
Then really
\begin{equation}
\frac{\D g(f(x))}{\D f(x)}=\frac{\D g(u)}{\D u}
\end{equation}
describes what we should do. Namely, first take the derivative of
$g$ and then evaluate it at $u=f(x)$.

\begin{thm}
Let $f$, $g$ be differentiable at $x$, and let
\begin{equation}
h(x)=g(f(x)).
\end{equation}
Then
\begin{equation}
\frac{\D h(x)}{\D x}=\frac{\D g(f(x))}{\D f(x)}\frac{\D f(x)}{\D x}
\end{equation}
describes the derivative of $h$ at $x$.
\end{thm}

Again, we also proved this, which concludes this section.


\begin{exercises}
\suggestedExercise [14] Prove or find a counter-example: if
$f'(x)=0$, then $f(x)$ is constant.

\exercise [19] Find the derivative of $p(x)=(3x^{5}+x^{-1})(x^{9}+3x^{4}+x+4)$.

\subsection*{Calculation Exercises}
\noindent{}In these exercises, take the derivatives of the
following expressions.
\exercise [22] $\displaystyle - 20 x - \frac{9}{x} - \frac{20}{x^{8}} - \frac{25}{x^{7}} + \frac{7}{x^{2}} - 19 x^{6}$
\exercise [20] $\displaystyle 22 x + \frac{14}{x^{9}} - 29 x^{2} + 26 x^{9} + 28 x^{10}$
\exercise [28] $\displaystyle - 13 x + \frac{10}{x^{8}} + \frac{16}{x^{6}} + \frac{16}{x^{5}} + 18 x^{2} + 29 x^{4} + 14 x^{5} + 10 x^{9} - 20 x^{10}$
\exercise [20] $\displaystyle - 18 x - \frac{24}{x^{4}} + 3 x^{2} - 26 x^{6} + 27 x^{8}$
\exercise [19] $\displaystyle -19 + 28 x - \frac{9}{x^{4}} - \frac{18}{x^{2}} + 27 x^{10}$
\exercise [19] $\displaystyle 28 - 25 x + 17 x^{2} - 17 x^{4} + 14 x^{7}$
\exercise [24] $\displaystyle \frac{21}{x^{8}} - \frac{7}{x^{6}} - \frac{29}{x^{5}} + 9 x^{2} + 24 x^{3} + 29 x^{4} - 19 x^{7}$
\exercise [18] $\displaystyle - 9 x - \frac{16}{x} - \frac{14}{x^{8}} + \frac{20}{x^{4}}$
\exercise [23] $\displaystyle 2 - \frac{5}{x^{8}} - \frac{8}{x^{7}} + \frac{23}{x^{3}} + 25 x^{2} + 10 x^{6} + 4 x^{9}$
\exercise [12] $\displaystyle - \frac{13}{x^{8}}$
\exercise [17] $\displaystyle 26 - \frac{20}{x^{7}} - \frac{6}{x^{6}} - 24 x^{6}$
\exercise [25] $\displaystyle -8 - \frac{3}{x^{8}} - \frac{29}{x^{7}} + 24 x^{2} - 18 x^{4} + 17 x^{5} - 28 x^{9} + 24 x^{10}$
\begin{comment}
\exercise [23] $\displaystyle 25 - \frac{9}{x} - \frac{28}{x^{7}} + \frac{13}{x^{3}} + 10 x^{4} + 14 x^{6} + 23 x^{9}$
\exercise [14] $\displaystyle - \frac{23}{x} + 9 x^{10}$
\exercise [28] $\displaystyle - \frac{7}{x^{10}} + \frac{17}{x^{9}} + \frac{6}{x^{8}} + \frac{28}{x^{7}} + \frac{5}{x^{2}} - 30 x^{2} + 12 x^{4} + 26 x^{5} + 13 x^{10}$
\exercise [13] $\displaystyle -20 + 12 x^{2}$
\exercise [26] $\displaystyle 26 x - \frac{9}{x^{10}} + \frac{15}{x^{7}} - \frac{17}{x^{6}} + \frac{8}{x^{4}} - \frac{12}{x^{2}} + 26 x^{7} + 12 x^{9}$
\exercise [12] $\displaystyle 18 x^{8}$
\exercise [12] $\displaystyle - \frac{4}{x^{7}}$
\exercise [14] $\displaystyle \frac{7}{x^{5}} + 14 x^{5}$
\exercise [29] $\displaystyle 18 + 24 x - \frac{26}{x^{9}} - \frac{17}{x^{5}} + \frac{13}{x^{4}} - \frac{6}{x^{2}} + 12 x^{2} - 22 x^{7} + 20 x^{8} - 17 x^{9}$
\exercise [23] $\displaystyle 1 - \frac{4}{x} + \frac{17}{x^{10}} - \frac{14}{x^{7}} - \frac{11}{x^{2}} - 21 x^{2} + 19 x^{9}$
\exercise [17] $\displaystyle 11 + \frac{4}{x^{10}} + 13 x^{4} + 6 x^{8}$
\exercise [27] $\displaystyle -13 - \frac{28}{x} - \frac{7}{x^{10}} + \frac{25}{x^{4}} + 23 x^{3} - 21 x^{4} - 25 x^{6} - 20 x^{7} + 23 x^{8}$
\exercise [24] $\displaystyle - \frac{25}{x^{10}} + \frac{10}{x^{8}} - \frac{20}{x^{3}} - 19 x^{3} - 15 x^{4} + 12 x^{5} - 8 x^{10}$
\exercise [21] $\displaystyle 16 - \frac{11}{x^{9}} + \frac{5}{x^{6}} + 29 x^{2} + 26 x^{3} + 15 x^{5}$
\exercise [20] $\displaystyle - \frac{11}{x^{8}} - \frac{30}{x^{7}} - \frac{4}{x^{4}} + 28 x^{2} + 12 x^{7}$
\exercise [20] $\displaystyle - \frac{24}{x^{9}} - \frac{10}{x^{5}} - 12 x^{5} + 25 x^{6} + 19 x^{9}$
\exercise [19] $\displaystyle 15 - 6 x - \frac{29}{x^{8}} - 7 x^{6} + 7 x^{7}$
\exercise [29] $\displaystyle 21 - 29 x + \frac{21}{x} + \frac{13}{x^{6}} + \frac{21}{x^{3}} + \frac{27}{x^{2}} - 27 x^{3} - 4 x^{4} + 21 x^{8} - 5 x^{9}$
\exercise [19] $\displaystyle 8 + \frac{13}{x} - \frac{27}{x^{7}} - \frac{15}{x^{3}} - 25 x^{9}$
\exercise [21] $\displaystyle 7 - 11 x + \frac{4}{x} - \frac{22}{x^{7}} - \frac{15}{x^{3}} - 19 x^{10}$
\exercise [16] $\displaystyle - \frac{24}{x^{10}} - \frac{7}{x^{7}} - 9 x^{9}$
\exercise [27] $\displaystyle -29 + \frac{3}{x} - \frac{4}{x^{8}} - \frac{30}{x^{7}} + \frac{4}{x^{4}} + \frac{27}{x^{3}} + 2 x^{3} + 28 x^{4} - 15 x^{9}$
\exercise [29] $\displaystyle 21 - 14 x + \frac{6}{x^{9}} - \frac{3}{x^{7}} + \frac{7}{x^{6}} + \frac{8}{x^{3}} + 15 x^{5} - 13 x^{7} + 29 x^{8} - 3 x^{9}$
\exercise [28] $\displaystyle \frac{17}{x^{6}} - 15 x^{3} - 3 x^{4} - 9 x^{5} + 29 x^{6} - 30 x^{7} + 21 x^{8} - 13 x^{9} + 11 x^{10}$
\exercise [26] $\displaystyle \frac{11}{x^{10}} + \frac{23}{x^{9}} + \frac{16}{x^{7}} + \frac{23}{x^{6}} - \frac{23}{x^{4}} + 20 x^{7} - 19 x^{8} - 27 x^{10}$
\exercise [14] $\displaystyle \frac{20}{x^{2}} - 2 x^{7}$
\exercise [18] $\displaystyle \frac{15}{x^{7}} + \frac{11}{x^{5}} + \frac{23}{x^{3}} + 12 x^{4}$
\exercise [16] $\displaystyle \frac{19}{x} + \frac{2}{x^{6}} - 29 x^{10}$
\exercise [12] $\displaystyle \frac{27}{x^{4}}$
\exercise [24] $\displaystyle \frac{16}{x^{10}} - \frac{8}{x^{9}} + \frac{28}{x^{7}} + \frac{26}{x^{5}} - \frac{15}{x^{3}} + 7 x^{6} + 28 x^{9}$
\exercise [18] $\displaystyle - \frac{20}{x^{10}} + 21 x^{3} - 25 x^{8} - 14 x^{10}$
\exercise [25] $\displaystyle -17 - \frac{24}{x} + \frac{20}{x^{9}} - \frac{5}{x^{6}} + 14 x^{3} - 14 x^{7} + 25 x^{9} + 17 x^{10}$
\exercise [27] $\displaystyle 2 + \frac{28}{x^{10}} - \frac{27}{x^{8}} + \frac{18}{x^{7}} + \frac{23}{x^{2}} - 6 x^{2} - 15 x^{3} + 20 x^{4} - 20 x^{5}$
\exercise [16] $\displaystyle - \frac{21}{x^{8}} + \frac{15}{x^{5}} - \frac{19}{x^{2}}$
\exercise [15] $\displaystyle -13 + \frac{15}{x^{6}} + \frac{29}{x^{3}}$
\exercise [15] $\displaystyle 7 + \frac{18}{x^{8}} + 25 x^{3}$
\exercise [22] $\displaystyle - \frac{26}{x^{4}} + \frac{8}{x^{2}} - 6 x^{3} - 19 x^{5} + 20 x^{8} + 28 x^{10}$
\exercise [28] $\displaystyle - 24 x - \frac{24}{x} - \frac{4}{x^{6}} - \frac{10}{x^{3}} - 3 x^{3} - 3 x^{4} + 19 x^{5} + 25 x^{7} + 16 x^{8}$
\exercise [16] $\displaystyle - \frac{27}{x} - 26 x^{8} - 3 x^{10}$
\exercise [17] $\displaystyle -28 + \frac{13}{x} - \frac{16}{x^{9}} + \frac{5}{x^{3}}$
\exercise [23] $\displaystyle -5 - \frac{18}{x^{9}} - \frac{11}{x^{7}} - \frac{28}{x^{5}} + \frac{29}{x^{2}} - 8 x^{5} + 4 x^{9}$
\exercise [20] $\displaystyle \frac{23}{x^{10}} - \frac{3}{x^{9}} + \frac{3}{x^{5}} - 6 x^{4} + 18 x^{5}$
\exercise [15] $\displaystyle -17 - 24 x^{5} + 29 x^{8}$
\exercise [16] $\displaystyle - 24 x - \frac{17}{x^{5}} - 12 x^{8}$
\exercise [22] $\displaystyle \frac{5}{x^{8}} + \frac{26}{x^{3}} + 11 x^{2} - 22 x^{7} - 7 x^{9} - 15 x^{10}$
\exercise [18] $\displaystyle \frac{11}{x^{9}} + \frac{6}{x^{7}} - \frac{24}{x^{3}} + 27 x^{2}$
\exercise [25] $\displaystyle -14 + \frac{10}{x^{10}} + \frac{29}{x^{8}} + 18 x^{4} - 5 x^{5} + 21 x^{6} + 5 x^{8} - 11 x^{10}$
\exercise [22] $\displaystyle \frac{28}{x^{9}} + \frac{12}{x^{3}} - 19 x^{6} - 7 x^{8} + 27 x^{9} + 29 x^{10}$
\exercise [15] $\displaystyle -17 - \frac{20}{x^{7}} + 3 x^{10}$
\exercise [24] $\displaystyle - \frac{26}{x^{6}} + x^{-4} - 17 x^{2} + 5 x^{4} - 18 x^{6} - 10 x^{7} - 28 x^{9}$
\exercise [15] $\displaystyle 20 + \frac{16}{x^{3}} + 2 x^{3}$
\exercise [16] $\displaystyle 4 x + 22 x^{3} - 26 x^{10}$
\exercise [17] $\displaystyle -14 - 24 x + 6 x^{3} - 26 x^{10}$
\exercise [28] $\displaystyle - 11 x + \frac{3}{x} - \frac{25}{x^{10}} - \frac{30}{x^{9}} + \frac{7}{x^{7}} - \frac{2}{x^{5}} - 10 x^{3} + 5 x^{4} + 3 x^{7}$
\exercise [28] $\displaystyle \frac{29}{x} + \frac{26}{x^{10}} - \frac{20}{x^{9}} + \frac{24}{x^{8}} - \frac{22}{x^{7}} - \frac{16}{x^{6}} + \frac{22}{x^{5}} - \frac{28}{x^{4}} + 28 x^{6}$
\exercise [17] $\displaystyle -27 + \frac{9}{x^{9}} + \frac{25}{x^{5}} + 20 x^{3}$
\exercise [17] $\displaystyle 11 + \frac{17}{x^{9}} + \frac{7}{x^{8}} + \frac{17}{x^{4}}$
\exercise [27] $\displaystyle 9 + \frac{21}{x^{10}} + \frac{7}{x^{7}} - \frac{11}{x^{3}} + x^{-2} + 10 x^{2} - 16 x^{3} + 22 x^{4} + 27 x^{7}$
\exercise [20] $\displaystyle - \frac{10}{x^{10}} - \frac{29}{x^{7}} - \frac{29}{x^{3}} - 30 x^{4} + 25 x^{9}$
\exercise [19] $\displaystyle 10 + \frac{3}{x^{6}} + \frac{25}{x^{4}} - 8 x^{3} - 21 x^{5}$
\exercise [17] $\displaystyle 10 - \frac{6}{x^{5}} - \frac{3}{x^{2}} - 16 x^{5}$
\exercise [22] $\displaystyle - \frac{24}{x} + \frac{4}{x^{3}} - 30 x^{2} + 2 x^{5} + 11 x^{7} - 18 x^{8}$
\exercise [18] $\displaystyle - \frac{19}{x^{10}} + \frac{22}{x^{9}} - 28 x^{4} + 14 x^{8}$
\exercise [27] $\displaystyle -8 - \frac{7}{x} - \frac{20}{x^{8}} - \frac{25}{x^{5}} + \frac{13}{x^{4}} - \frac{18}{x^{2}} + 18 x^{8} - 24 x^{9} - 2 x^{10}$
\exercise [20] $\displaystyle - \frac{15}{x^{8}} - \frac{14}{x^{5}} - \frac{11}{x^{4}} - 15 x^{4} + 10 x^{8}$
\exercise [22] $\displaystyle - \frac{6}{x^{8}} + \frac{12}{x^{5}} - \frac{29}{x^{4}} + 24 x^{7} + 6 x^{9} - 25 x^{10}$
\exercise [23] $\displaystyle -10 + \frac{3}{x^{10}} - \frac{7}{x^{9}} + \frac{17}{x^{6}} - \frac{9}{x^{5}} + 9 x^{4} - 8 x^{8}$
\exercise [21] $\displaystyle -4 + 11 x + \frac{4}{x} - \frac{6}{x^{10}} + \frac{21}{x^{5}} + 18 x^{8}$
\exercise [16] $\displaystyle - 16 x + \frac{10}{x^{5}} - 17 x^{6}$
\exercise [16] $\displaystyle - 11 x - 5 x^{7} - 5 x^{9}$
\exercise [21] $\displaystyle -20 + 20 x - \frac{25}{x^{9}} - \frac{16}{x^{7}} - \frac{17}{x^{6}} + 12 x^{3}$
\exercise [13] $\displaystyle 16 - \frac{17}{x^{10}}$
\exercise [23] $\displaystyle -7 + \frac{8}{x^{7}} - \frac{3}{x^{2}} - 12 x^{2} + 15 x^{3} + 19 x^{5} + 5 x^{10}$
\exercise [26] $\displaystyle - \frac{23}{x^{9}} - \frac{5}{x^{5}} + \frac{14}{x^{3}} + 7 x^{3} + 27 x^{4} + 7 x^{5} + 14 x^{7} + 18 x^{8}$
\exercise [19] $\displaystyle 7 + \frac{8}{x^{10}} - \frac{11}{x^{9}} - 16 x^{7} - 26 x^{9}$
\exercise [17] $\displaystyle 10 - \frac{18}{x^{8}} + \frac{9}{x^{5}} - 6 x^{4}$
\exercise [23] $\displaystyle -26 - 22 x - \frac{7}{x^{10}} + \frac{8}{x^{6}} - \frac{6}{x^{3}} - 18 x^{7} + 23 x^{10}$
\exercise [13] $\displaystyle 15 + 26 x^{10}$
\exercise [29] $\displaystyle 27 - \frac{12}{x^{10}} + \frac{7}{x^{6}} + \frac{14}{x^{4}} - \frac{15}{x^{3}} - \frac{3}{x^{2}} + 12 x^{2} - 12 x^{3} + 8 x^{6} - 25 x^{8}$
\exercise [27] $\displaystyle 16 - \frac{25}{x} - \frac{30}{x^{6}} - \frac{6}{x^{5}} - 24 x^{2} + 22 x^{4} - 29 x^{7} + 13 x^{9} + 24 x^{10}$
\exercise [12] $\displaystyle - \frac{2}{x^{2}}$
\exercise [22] $\displaystyle - 18 x + \frac{21}{x} - \frac{23}{x^{9}} + \frac{27}{x^{2}} + 2 x^{4} - 23 x^{10}$
\exercise [20] $\displaystyle - \frac{2}{x} - \frac{27}{x^{3}} - 9 x^{2} + 14 x^{4} + x^{6}$
\exercise [15] $\displaystyle -2 - 13 x - 20 x^{7}$
\exercise [14] $\displaystyle 28 x^{2} - 18 x^{8}$
\exercise [27] $\displaystyle -26 + \frac{9}{x} - \frac{7}{x^{7}} + \frac{28}{x^{6}} + \frac{5}{x^{5}} - \frac{21}{x^{2}} - 13 x^{7} - 6 x^{8} + 16 x^{10}$
\exercise [28] $\displaystyle \frac{6}{x^{10}} + \frac{28}{x^{6}} + x^{-5} - \frac{15}{x^{3}} + 14 x^{2} + 9 x^{6} - 26 x^{7} + 23 x^{8} + 12 x^{9}$
\exercise [17] $\displaystyle 6 - \frac{9}{x^{6}} + \frac{7}{x^{5}} + \frac{19}{x^{2}}$
\end{comment}
\end{exercises}
