%%
%% todo.tex
%% 
%% Made by alex
%% Login   <alex@tomato>
%% 
%% Started on  Tue Feb 28 13:34:25 2012 alex
%% Last update Wed May 30 13:24:22 2012 Alex Nelson
%%
\section{To Do List}

\N{Logarithms}
It'd be nice to have a method of calculating $\ln(n)$ for
$n\in\NN$. Why? Because we can always approximate any real number
by a rational one, and the logarithm of a rational number is
\begin{equation}
\ln(m/n)=\ln(m)-\ln(n).
\end{equation}
This could be computed if we had some method of computing
$\ln(n)$. Again, this can be simplified if we consider only prime
numbers by the fundamental theorem of arithmetic.

Another approach would be to take any positive real number $x$
and write it out as
\begin{equation}
x=x^{*}\times 10^{n}
\end{equation}
where $n\in\NN_{0}$ and $0\leq x^{*}<10$. Thus one needs only
compute $\ln(10)$ and $\ln(x^{*})$, which may or may not be
terribly easy to do.
