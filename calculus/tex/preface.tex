%%
%% preface.tex
%% 
%% Made by alex
%% Login   <alex@tomato>
%% 
%% Started on  Thu Nov  3 14:45:53 2011 alex
%% Last update Wed May 30 10:11:45 2012 Alex Nelson
%%
%\phantomsection\addcontentsline{toc}{chapter}{Preface} 
%\chapter*{Preface}
This manuscript is intended to teach calculus using big O
notation. Originally, Knuth suggested it in a letter~\cite{knuth}. It seems
to explain the intuitive idea, which is my aim. This is done in a
slightly less-than-orthodox manner, thinking ``microscopically''
(``infinitesimally'') and ``macroscopically''. I found this
really helps explain the concepts of differentiation and
integration.

As this is a calculus text, all the concerns are purely
\emph{symbolic}. We do not care if something is well-defined or
not, everything we do is symbolic manipulation. When we stop,
think, and wonder if it makes sense, then we begin the field of
real analysis. 

Depending on how far things get, this text may be focused on
calculus of a single variable only or more. If we got farther, we
need to discuss linear algebra (even a provisional presentation
of it will be necessary). 

The focus varies, but consistently we are application
oriented. The ``applications'' I am speaking of are computational
in nature (e.g., how can we compute $\sin(1)$ to 7 digits of
precision by hand?). This is useful when you forget your
calculator, and wait for your plane at the airport.

But really, this is a stepping stone towards other
applications. Texts which will follow from this are my notes on
classical physics; my notes on complex analysis; and so on.

\bigskip
A note when reading. There will be ``problems'' which come up in
the text. They are sometimes exercises, other times they are
motivating questions to guide the text. The reader should spend a
moment thinking about them before moving on.

Exercises appear in the ``exercises'' sections. They are numbered
to indicate their difficulty on a logarithmic scale from 0
(easiest) to 50 (open research problems). For a rough
approximation of difficulty and explanation, consider the
following table:

\medskip
\begin{tabular}{c | p{10cm}}
\toprule
Scale & Meaning \\
\midrule 
00 & ``Warmups.'' Every reader should try to do these when reading.\\
10 & ``Basics.'' Exercises to develop facts that should be done
from one's own derivations rather than glancing at someone else's.\\
20 & ``Homework Exercises.'' Intended to deepen understanding of
the material covered in the current section or chapter. \\
30 & ``Exam Problems.'' These typically involve multiple sections
of the book to gain a better insight into mathematics. \\
40 & ``Bonus Problems.'' These problems extend the text in
interesting ways. \\
50 & ``Open Research Problems.'' These are open problems that may
or may not have a solution. \\
\bottomrule
\end{tabular}

\medskip
\noindent{}It should be noted that sometimes it is necessary to just do
calculations in calculus. A python program was written to
generate exercises, just for the sake of practicing calculations,
and will be sectioned off into a subsection ``Calculator Exercises.''
Here ``calculator'' refers to the historical connotation of a
person who is skilled at calculations, not the electronic
pocket-sized gadget.

\medskip
As far as the level of rigor is concerned, as stated, everything
is purely symbolic manipulation. Proofs are given, but not always
within the ``\emph{Proof}. \dots \slug'' pattern mathematicians
have come to love.

However, we are following Euler's example in applying the
``Generality of Algebra'' to our thinking. That is to say, we are
not being rigorous in the modern sense of the word because we are
following symbolic rules that work in particular cases. We just
assume that the rules work in \emph{any} situation.

Although we are following Euler, our presentation of material
differs slightly. We will stick to the outline in most modern
calculus textbooks, namely: derivatives first and integrals
second, single variables before multiple. We will also use
convergence of series and sequences as a segue from single to
many variable calculus.

\vfill
\begin{quotes}
The heart of mathematics consists of 
concrete examples and concrete problems.
\Author P.\ R.\ Halmos, {\sl How to write mathematics} (1973)

\bigskip
It is downright sinful to teach 
the abstract before the concrete.
\Author Z.\ A.\ Melzak, {\sl Companion to Concrete Mathematics} (1973)

\bigskip
The advanced reader who skips parts that appear 
too elementary may miss more than the less advanced reader 
who skips parts that appear too complex.
\Author G.\ Polya, {\sl Induction and Analogy in Mathematics} (1954)

\bigskip
The material of concrete mathematics may seem at first to be a disparate
bag of tricks, but practice makes it into a disciplined set of tools.
\Author D.\ E.\ Knuth, {\sl Concrete Mathematics} (1994)
\end{quotes}
