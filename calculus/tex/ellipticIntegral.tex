%%
%% ellipticIntegral.tex
%%
%% Made by alex
%% Login   <alex@tomato>
%%
%% Started on  Mon May 21 17:08:22 2012 alex
%% Last update Wed May 30 13:26:26 2012 Alex Nelson
%%

\section{Elliptic Integrals and Elliptic Functions}
\M
As a warm up, lets try to consider the properties of a function
\begin{equation}
s(x)=\int^{x}_{0}\frac{\D u}{\sqrt{1-u^{2}}}
\end{equation}
We should recall this is the arc-sine function. How do we study
its properties? Well, through its inverse function $\sin(u)$.

\M
Consider the following function
\begin{equation}
F(\varphi) =
\int^{\varphi}_{0}\frac{\D u}{\sqrt{(1-u^{2})(1-k^{2}u^{2})}}
\end{equation}
We should observe immediately that
\begin{equation}
F(0)=0.
\end{equation}
But observe this is just a \emph{deformation} of the warm-up!

\M
Note that the Binomial series can be extended to non-integer exponents.
We would have
\begin{equation}
(1 + x)^{-1/2} = 1 - \frac{1}{2}x + \frac{3}{8}x^{2} - \frac{5}{16}x^{3} +\dots
= \sum^{\infty}_{n=0}\frac{(-1)^{n}(2n)!}{4^{n}(n!)^{2}}x^{n}.
\end{equation}
If we plug in $x=-k^{2}u^{2}$, we obtain
\begin{equation}
\frac{1}{\sqrt{1 - k^{2}u^{2}}} = \sum^{\infty}_{n=0}\frac{(2n)!}{4^{n}(n!)^{2}}k^{2n}u^{2n}.
\end{equation}
For ``small $\varphi$'' and ``small $ku$'', we would have
\begin{equation}
F(\varphi) = \int^{\varphi}_{0}\frac{1}{\sqrt{1-u^{2}}}\left(\sum^{\infty}_{n=0}\frac{(2n)!}{4^{n}(n!)^{2}}k^{2n}u^{2n}\right)\,\D u.
\end{equation}
Now we just have to evaluate integrals of the form
\begin{equation}
I_{n}(\varphi) = \int^{\varphi}_{0}\frac{u^{2n}}{\sqrt{1 - u^{2}}}\,\D u,
\end{equation}
reducing our problem to evaluating the sum,
\begin{equation}
F(\varphi) = \sum^{\infty}_{n=0}\frac{(2n)!}{4^{n}(n!)^{2}}k^{2n}I_{n}(\varphi).
\end{equation}
This approach boils down to writing $F(\varphi)$ in terms of Gauss's
hypergeometric functions.
