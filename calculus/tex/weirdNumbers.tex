%%
%% weirdNumbers.tex
%%
%% Made by alex
%% Login   <alex@tomato>
%%
%% Started on  Thu Nov  3 14:44:05 2011 alex
%% Last update Wed May 30 10:17:08 2012 Alex Nelson
%%

\section{Weird Numbers}
%\subsection{Slope}

\N{Slope}
Consider some linear function
\begin{equation}
g(x)=mx+b
\end{equation}
for some nonzero real number $m$, and an arbitrary real number
$b$. We can calculate the slope by considering
\begin{equation}
 \Delta x\not=0
\end{equation}
some constant ``shift'' in $x$, and using this to figure out the
change in $g$
\begin{equation}
\Delta g(x)=g(x+\Delta x)-g(x).
\end{equation}
What is this? Well, we plug in the definition of $g$ to find
\begin{equation}
\Delta g(x)=(m(x+\Delta x)+b)-(mx+b)
\end{equation}
which reduces to
\begin{equation}
\Delta g(x)=m\Delta x.
\end{equation}
Thus we may write the slope of $g$ as
\begin{equation}
m=\frac{\Delta g(x)}{\Delta x}
\end{equation}
which is independent of both $x$ and the choice of $\Delta x$.

Can we do this in general for a polynomial
\begin{equation}\label{eq:definitionOfF}
f(x)=x^{n}
\end{equation}
for some $n\in\NN$? Let us try! We consider some nonzero $\Delta
x$ term, and we write (for $n>1$)
\begin{equation}
f(x+\Delta x)=(x+\Delta x)^{n}=x^{n}+nx^{n-1}\Delta
x+(\Delta x)^{2}(\mbox{bonus parts})
\end{equation}
where the ``bonus parts'' are other stuff. Actually by the binomial
theorem, it would have to be a polynomial in $\Delta x$ with a
nonzero constant term. This information is really encoded in
\begin{equation}
(\Delta x)^{2}(\mbox{bonus parts})=\bigO\left((\Delta x)^{2}\right)
\end{equation}
where $\bigO(\dots)$ is a more rigorous way of saying ``bonus
parts at least quadratic in $\Delta x$.'' This gives us a more
precise way to specify the error when writing out terms at most
linear in $\Delta x$.

\begin{problem}
What is $\Delta x\bigO(\Delta x)$? What is $(\Delta
x)^{-1}\bigO\left((\Delta x)^{2}\right)$?
\end{problem}

We see that we are abusing notation and writing
\begin{equation}
h(x)\Delta x+(\Delta x)^{2}(\mbox{bonus parts})=h(x)\Delta x+\bigO\left((\Delta x)^{2}\right)
\end{equation}
for some $h(x)$. So by dividing through by $\Delta x$ we obtain
\begin{equation}
h(x)+(\Delta x)(\mbox{bonus parts})=h(x)+\bigO(\Delta x).
\end{equation}
This implies
\begin{equation}
(\Delta x)^{-1}\bigO\left((\Delta x)^{2}\right)=\bigO(\Delta x)
\end{equation}
and similar reasoning suggests
\begin{equation}
(\Delta x)\bigO\left((\Delta x)^{2}\right)=\bigO\left((\Delta x)^{3}\right).
\end{equation}
So let us go on with our considerations.

We then have
\begin{equation}
f(x+\Delta x)=f(x)+nx^{n-1}\Delta x+\bigO\left((\Delta x)^{2}\right),
\end{equation}
where we were slick and noted the definition of $f$ in order to
plug it in. So, we can rewrite this as
\begin{equation}
f(x+\Delta x)-f(x)=nx^{n-1}\Delta x+\bigO\left((\Delta x)^{2}\right)
\end{equation}
and we want to divide both sides by $\Delta x$. But we know how
to do this now! First we will write
\begin{equation}
\Delta f(x)=f(x+\Delta x)-f(x)
\end{equation}
as shorthand, and rewrite our equation as
\begin{equation}
\Delta f(x)=nx^{n-1}\Delta x+\bigO\left((\Delta x)^{2}\right).
\end{equation}
We divide both sides by $\Delta x$
\begin{equation}
\frac{\Delta f(x)}{\Delta x}=nx^{n-1}+\bigO(\Delta x).
\end{equation}
But we have a problem that we didn't have before: the slope
depends on $\Delta x$ and $x$.

Historically, people noted that we were working with a term
$\bigO((\Delta x)^{2})$. If we could make that term equal to 0,
then everything would work out nicely. How do we do this? Well,
we formally invent a number $\varepsilon$ and use it instead of a
finite nonzero number $\Delta x$.

%\subsection{$\varepsilon$ and $\sqrt{-1}$}
\N{$\varepsilon$ and $\sqrt{-1}$}
We know that we have a ``number'' $\I$ satisfying
\begin{equation}
\I^{2}=-1.\label{eq:definingPropertyForI}
\end{equation}
There is no real number which satisfies this, but we can ``adjoin''
$\I$ to $\RR$. That is, we pretend that $\I$ is a variable
satisfying equation \eqref{eq:definingPropertyForI}, then we have
polynomials of the form
\begin{equation}
p(x,y)=x+\I\cdot y.
\end{equation}
Of course, we can formally multiply these polynomials together,
and we end up with the number system (``ring'') of complex numbers
$\CC$ (we would have to prove that $1/\I$ exists to make it a
field).
\begin{problem}
Why do we not have higher order terms in $\I$? That is, a general
polynomial $x+\I\cdot y+\I^{2}\cdot z + \dots$?
\end{problem}
Lets consider it. Suppose we did have
\begin{equation}
p(x,y)=x+\I\cdot y+\I^{2}\cdot z.
\end{equation}
Then we plug in \eqref{eq:definingPropertyForI} to find
\begin{equation}
p(x,y)=x+\I\cdot y+(-1) z
\end{equation}
which simplifies to merely
\begin{equation}
p(x,y)=(x-z)+\I\cdot y.
\end{equation}
But this is precisely of the form we described: there is some
term which is a multiple of $\I$ (the \emph{imaginary} term) and
another independent of $\I$ (the \emph{real} term).

Lets consider a similar problem. We want a nonzero ``number''
$\varepsilon$ which is the ``smallest'' number possible. What would
this mean? Suppose we have a ``small'' finite number
\begin{equation}
0<x<1.
\end{equation}
Then we see the property specifying that $x$ is small would be
\begin{equation}
0<x^{2}<x.
\end{equation}
But if we had the \emph{smallest} number, then the general
argument is we expect
\begin{equation}
\varepsilon^2=0.
\end{equation}
We call such a $\varepsilon$ an \define{Infinitesimal} number.
If we formally consider such an $\varepsilon$ (i.e., pretend it
exists and obeys this relationship), then we can run into some
problems. For example: what is $\varepsilon^{-1}$?


%\subsection{Division by Zero?}
\N{Division by Zero?}
The problem is: what is $\varepsilon^{-1}$? The answer is: we
don't know.

However, why would $\varepsilon$ ever be useful? We can consider
\begin{equation}
f(x)=x^{n}
\end{equation}
for some $n\in\NN$. Then
\begin{equation}
f(x+\varepsilon)=(x+\varepsilon)^{n}
\end{equation}
can be simplified to what? Lets consider the $n=2$ case:
\begin{equation}
(x+\varepsilon)^{2}=x^{2}+2\varepsilon\cdot x+\varepsilon^{2}.
\end{equation}
But the $\varepsilon^{2}$ term vanishes, so
\begin{equation}
(x+\varepsilon)^{2}=x^{2}+2\varepsilon\cdot x.
\end{equation}
We see that
\begin{equation}
(x+\varepsilon)^{3}=(x^{2}+2\varepsilon\cdot x)(x+\varepsilon)
\end{equation}
can be carried out as if it were polynomial multiplication. We
then obtain
\begin{equation}
x^{2}(x+\varepsilon)+2\varepsilon\cdot x(x+\varepsilon)=
x^{3}+x^{2}\varepsilon+2\varepsilon x^{2}+2\varepsilon^{2}x
\end{equation}
and again, the $\varepsilon^{2}$ term vanishes. We thus obtain
\begin{equation}
(x+\varepsilon)^{3} = x^{3}+3\varepsilon\cdot x.
\end{equation}
Indeed  the general pattern appears to be
\begin{equation}
(x+\varepsilon)^{n}=(x)^{n}+n\varepsilon\cdot x^{n-1}.
\end{equation}
We would like to write
\begin{equation}
(x+\varepsilon)^{n}-(x)^{n}=n\varepsilon\cdot x^{n-1}.
\end{equation}
Notice the difference this time: we don't have any
$\bigO(\varepsilon^{2})$ terms. The only price we paid is we
cannot get rid of the factor $\varepsilon$.

%\subsection{Big O for the Bonus Parts}
\N{Big O for the Bonus Parts}
The take home moral is that $\bigO(\dots)$ enables us to
\emph{rigorously} consider infinitesimals. How? Well, the most
significant terms are written out explicitly, and the rest are
swept under the rug with $\bigO(\dots)$. For our example of
\begin{equation}
f(x)=x^{n}
\end{equation}
we saw we could write
\begin{equation}
f(x+\Delta x)-f(x)=nx^{n-1}\Delta x+\bigO\left((\Delta
x)^{2}\right)
\end{equation}
which tells us the error of ``truncating,'' or cutting off the
polynomial to be explicitly first order plus some change. This
change we consider to be \emph{in effect} ``infinitesimal'' in
comparison to the $\Delta x$ term.
