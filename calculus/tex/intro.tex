%%
%% intro.tex
%% 
%% Made by alex
%% Login   <alex@tomato>
%% 
%% Started on  Mon May 21 15:09:58 2012 alex
%% Last update Wed May 30 10:11:39 2012 Alex Nelson
%%
\chapter*{\phantomsection\addcontentsline{toc}{chapter}{Introduction}Introduction}

\textbf{Warning:\quad\ignorespaces} This introduction may be
skipped for most readers. The intended audience are logicians
interested in the foundations used throughout, and people
attempting to write ``formal mathematics''.

We will be working without much ``formal'' framework. That is, we
are largely working symbolically and not rigorously. The logical
foundation made could be described as ``High School Algebra''. We
use variables (uncontroversial for logicians). Our operations
are:
\begin{enumerate}
\item Addition. We have $a+b$ be commutative, so $a+b=b+a$ and
  associatve $(a+b)+c=a+(b+c)$. There is an identity element $0$
  such that $0+a=a$ for any $a$. Negation produces the additive
  inverse $-(a)=-a$. 
\item Subtraction. This is just adding by the additive inverse:
  $a-b = a+(-(b))$. 
\item Multiplication. Written $a\cdot{b}$, it's commutative
  $a\cdot{b}=b\cdot{a}$ and associative
  $a\cdot(b\cdot{c})=(a\cdot{b})\cdot{c}$. Its identity element
  is denoted $1\cdot{a}=a$ for any $a$. Multiplicative inverse is
  denoted $1/a$ for any nonzero $a$ (i.e., $a\not=0$).
\item Division. This, like subtraction, multiplies by the
  multiplicative inverse.
\item Exponentiation. We write $a^b$. It is the first
  noncommutative operation $a^b\not=b^a$ and it is not
  associative $a^{(b^{c})}\not=(a^{b})^{c}$. It has an identity
  element in the sense that $a^{1}=a$ for any $a$.
\end{enumerate}
Tacitly, we have an ordering of numbers:
\begin{equation}
a<b\iff b-a\mbox{ is positive}
\end{equation}
We have an ordered field. 

The astute student would realize that exponentiation could have
two inverses: the logarithm and the $n^{\rm th}$-root. We say
\begin{equation}
\log_{b}(x)=y\iff b^{y}=x.
\end{equation}
The root-approach specifies for any number $a$ another number is
produced $a^{1/n}$ satisfying
\begin{equation}
(a^{1/n})^{n}=a.
\end{equation}
What about $\sqrt{-1}$? We run into difficulties: the root
approach gives us unreal (or \emph{imaginary}) numbers. Really,
we work with real numbers ``embedded'' in an ambient complex
number system.

