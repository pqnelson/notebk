%%
%% inverseFn.tex
%% 
%% Made by alex
%% Login   <alex@tomato>
%% 
%% Started on  Tue May 29 14:16:57 2012 alex
%% Last update Tue May 29 14:16:57 2012 alex
%%

\section{Differentiating Inverse Functions}

Let $y=f(x)$ be a function. Recall the inverse function $f^{-1}$
is defined to satisfy
\begin{equation}
f^{-1}(y)=x.
\end{equation}
It basically ``undoes'' $f$. What is its derivative? Well, take
the derivative with respect to $x$ on both sides:
\begin{equation}
\frac{\D f^{-1}(y)}{\D x}=\frac{\D x}{\D x}
\end{equation}
Observe the left hand side requires the chain rule, whereas the
right hand side is 1. Thus
\begin{equation}
\frac{\D f^{-1}(y)}{\D y}\frac{\D y}{\D x}=1.
\end{equation}
Divide both sides by $(\D y/\D x)$:
\begin{equation}
\frac{\D f^{-1}(y)}{\D y} = \frac{1}{\D y/\D x}.
\end{equation}
This gives us the rule for considering differentiating inverse
functions. We can substitute $y=f(x)$ making it
\begin{equation}
\frac{\D f^{-1}(y)}{\D y}=\frac{1}{\D f(x)/\D x}.
\end{equation}
Lets consider some examples.

\N{Example (Sine Function)}
Let
\begin{equation}
\phi=\sin(\theta).
\end{equation}
We recall that
\begin{equation}
\arcsin\bigl(\sin(\theta)\bigr)=\theta.
\end{equation}
So by our calculations, we expect
\begin{equation}\label{eq:ex-arcsin:der}
\frac{\D\arcsin(\phi)}{\D\phi}=\frac{1}{\cos(\theta)}.
\end{equation}
The problem is we need to rewrite $\cos(\theta)$ in terms of
\begin{equation}
\phi=\sin(\theta).
\end{equation}
Well, it's simple when we recall
\begin{equation}
\begin{split}
\cos^{2}(\theta)&=1-\sin^{2}(\theta)\\
&=1-\phi^{2}.
\end{split}
\end{equation}
Taking the squareroot of both sides gives us
\begin{equation}
\cos(\theta)=\sqrt{1-\phi^{2}}.
\end{equation}
Thus we plug this back into our derivative expression, Eq \eqref{eq:ex-arcsin:der},
obtaining
\begin{equation}
\frac{\D\arcsin(\phi)}{\D\phi}=\frac{1}{\sqrt{1-\phi^{2}}}.
\end{equation}
That concludes our example.

\N{Example (Cosine)}
We can do likewise
\begin{equation}
\phi=\cos(\theta)
\end{equation}
thus
\begin{equation}
\arccos(\phi) = \theta.
\end{equation}
Taking its derivative produces
\begin{equation}
\frac{\D\arccos(\phi)}{\D\phi}=\frac{1}{-\sin(\theta)}
\end{equation}
We see
\begin{equation}
\sin(\theta) = \sqrt{1-\cos^{2}(\theta)}=\sqrt{1-\phi^{2}}.
\end{equation}
Thus
\begin{equation}
\frac{\D\arccos(\phi)}{\D\phi}=\frac{-1}{\sqrt{1-\phi^{2}}}.
\end{equation}
This gives us the derivative of arc-cosine.
