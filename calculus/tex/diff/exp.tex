%%
%% exp.tex
%% 
%% Made by Alex Nelson
%% Login   <alex@black-cherry>
%% 
%% Started on  Wed May 30 13:29:13 2012 Alex Nelson
%% Last update Fri Jun  1 10:53:26 2012 Alex Nelson
%%
\section{Exponential}
\N{Problem:}Consider a curve $y=y(x)$ such that at any point
$(x,y)$ on the curve, the tangent line has its slope be
\begin{equation}
\frac{\D y}{\D x}=y.
\end{equation}
What does $y(x)$ look like?

Lets write its tangent line at $x_{0}$ as
\begin{equation}
\begin{split}
t(x) &= y(x_{0})+y'(x_{0})(x-x_{0})\\
&=y(x_{0})(1+x-x_{0}).
\end{split}
\end{equation}
We see that
\begin{equation}
t(a)=0\iff a=x_{0}-1.
\end{equation}
Well we should note that if we change $x$ to $x+b$ (for a ``small
$b$''), then $y(x)$ changes to
\begin{equation}
y(x+b)\approx y(x)+y'(x)b
\end{equation}
Note that 

\M
Consider the following function of positive integers
\begin{equation}
e(n) = \left(1+\frac{1}{n}\right)^{n}=\frac{(1+n)^{n}}{n^{n}}.
\end{equation}
Lets calculate some values: $e(1)=2$, 
\begin{equation*}
\begin{split}
e(10) &= \frac{25937424601}{10^{10}}=2.5937424601\\
e(20) &\approx 2.65329\\
e(100) &\approx 2.7048
\end{split}
\end{equation*}
If $e(n)$ converges to some constant, then it does so slowly as
$n\to\infty$. 

