%%
%% linear.tex
%% 
%% Made by Alex Nelson
%% Login   <alex@black-cherry>
%% 
%% Started on  Fri Jun  1 10:03:00 2012 Alex Nelson
%% Last update Fri Jun  1 10:26:21 2012 Alex Nelson
%%
\section{Linear Approximations, Tangent Lines}
\N{Tangent Lines}
So the derivative gives us the ``instantaneous'' slope of a curve
at a point. In fact, we can construct the tangent line to a curve
$y=y(x)$ using the following function
\begin{equation}
t(x) = y(x_{0})+y'(x_{0})\cdot(x-x_{0}).
\end{equation}
Lets look at this a moment. The general structure looks like
\begin{equation}
t(x) = b + m(x-x_{0})
\end{equation}
where $b=y(x_{0})$ guarantees the line passes through the point
$(x_{0},y_{0})$ for $y_{0}=y(x_{0})$. The slope is the derivative
of the curve evaluated at $x_{0}$. We also make certain that the
second term $m(x-x_{0})$ vanishes at $x_{0}$, so
$t(x_{0})=y_{0}$.

\begin{ex}
Consider the polynomial
\begin{equation}
f(x) = x^{5}+3x^{2}+2.
\end{equation}
What is its tangent line at $x_{0}=2$? 

We first find its derivative
\begin{equation}
\frac{\D f}{\D x}= 5x^{4} + 6x
\end{equation}
Then we consider the tangent line
\begin{equation}
t(x) = f(x_{0}) + f'(x_{0})\cdot(x-x_{0}).
\end{equation}
Observe
\begin{equation}
\begin{split}
f(2) &= 2^{5} + 3(2^{2})+2\\
&= 32 + 12 + 2 = 46
\end{split}
\end{equation}
and
\begin{equation}
\begin{split}
f'(2) &= 5(2^{4}) + 6(2)\\
&=5(16) + 12 = 92.
\end{split}
\end{equation}
Thus the tangent line to $y=f(x)$ at $x=2$ is
\begin{equation}
t(x) = 46 + 92(x-2).
\end{equation}
\end{ex}

\begin{ex}
We can also use the tangent line as a good approximation for
$x\approx x_{0}$. Suppose we wanted to find
\begin{equation}
q = \sqrt{4.1}
\end{equation}
What can we do? Well, consider the function
\begin{equation}
g(x)=\sqrt{4+x}.
\end{equation}
We take its linear approximation at $x=0$, so
\begin{equation}
t(x)=g(0) + g'(0)\cdot x
\end{equation}
Observe that $g(0)=2$ and
\begin{equation}
g'(x)=\frac{1}{2\sqrt{4+x}}
\end{equation}
thus $g'(0)=1/4$. We find
\begin{equation}
t(x) = 2 + \frac{x}{4}.
\end{equation}
We approximate the solution
\begin{equation}
\begin{split}
q &\approx t(0.1)\\
&=2+\frac{1}{40}.
\end{split}
\end{equation}
Is this a good approximation? Well, lets check!

Observe
\begin{equation*}
2+\frac{1}{40} = \frac{81}{40}.
\end{equation*}
Then we square it
\begin{equation}
\begin{split}
\left(\frac{81}{40}\right)^{2} &= \frac{6561}{1600}\\
&=\frac{6400}{1600}+\frac{161}{1600}.
\end{split}
\end{equation}
What's the error in this approximation? Well, we see
\begin{equation}
\left|q^{2}-\left(\frac{81}{40}\right)^{2}\right| =
\frac{1}{1600} = 0.000625,
\end{equation}
so it's correct until the fourth decimal place.
\end{ex}
\begin{rmk}
The general scheme is, if we know what $f(x_{0})$ is exactly, and
we want to consider $f(x_{0}+\Delta x)$ for ``small enough''
$\Delta x$, what to do? We approximate $f(x_{0}+\Delta
x)=t(x+\Delta x)$ where $t(x)$ is the tangent to $f$ at $x_{0}$. 
\end{rmk}

\begin{xca}
Recall that $\sin(\pi/4)=1/\sqrt{2}$. What is $\sin(3\pi/16)$?
\end{xca}
