%%
%% trig.tex
%% 
%% Made by Alex Nelson
%% Login   <alex@black-cherry>
%% 
%% Started on  Wed May 30 10:19:45 2012 Alex Nelson
%% Last update Wed May 30 13:20:15 2012 Alex Nelson
%%
\section{Differentiating Trigonometric Functions}

\N{Review of Two Identities}
We will remind the reader of two identities:
\begin{subequations}
\begin{align}
\sin(\theta+\phi) &= \cos(\theta)\sin(\phi)+\sin(\theta)\cos(\phi)\\
\cos(\theta+\phi) &= \cos(\theta)\cos(\phi)-\sin(\theta)\sin(\phi).
\end{align}
\end{subequations}
It is also worthwhile to mention
\begin{equation}
\cos(-\theta)=\cos(\theta),\quad\mbox{and}\quad
\sin(-\theta)=-\sin(\theta)
\end{equation}
for any $\theta$.

\N{Derivative of Sine}
Now consider, for some $\Delta\theta$,
\begin{equation}
\sin(\theta+\Delta\theta)=
\cos(\theta)\sin(\Delta\theta)+\sin(\theta)\cos(\Delta\theta).
\end{equation}
We find then that
\begin{equation}
\begin{split}
\Delta\sin(\theta)
&=\sin(\theta+\Delta\theta)-\sin(\theta)\\
&=\cos(\theta)\sin(\Delta\theta)+\sin(\theta)\bigl(\cos(\Delta\theta)-1\bigr).
\end{split}
\end{equation}
The slope thus is
\begin{equation}
\frac{\Delta\sin(\theta)}{\Delta\theta}
=\frac{\cos(\theta)\sin(\Delta\theta)+\sin(\theta)\bigl(\cos(\Delta\theta)-1\bigr)}{\Delta\theta}.
\end{equation}
What happens as $\Delta\theta\to0$? Well
\begin{equation}
\lim_{\Delta\theta\to0}\frac{\sin(\Delta\theta)}{\Delta\theta}=1,\quad
\lim_{\Delta\theta\to0}\frac{\cos(\Delta\theta)-1}{\Delta\theta}=0.
\end{equation}
Thus the derivative of sine is
\begin{equation}
\frac{\D\sin(\theta)}{\D\theta}=\cos(\theta).
\end{equation}

\N{Derivative of Cosine}
Likewise, we see that
\begin{equation}
\cos(\theta+\Delta\theta)=\cos(\theta)\cos(\Delta\theta)-\sin(\theta)\sin(\Delta\theta)
\end{equation}
thus
\begin{equation}
\begin{split}
\Delta\cos(\theta)
&=\cos(\theta+\Delta\theta)-\cos(\theta)\\
&=\cos(\theta)\bigl(\cos(\Delta\theta)-1\bigr)-\sin(\theta)\sin(\Delta\theta).
\end{split}
\end{equation}
The rate of change would be:
\begin{equation}
\frac{\Delta\cos(\theta)}{\Delta\theta}
=\cos(\theta)\left(\frac{\cos(\Delta\theta)-1}{\Delta\theta}\right)-\sin(\theta)\frac{\sin(\Delta\theta)}{\Delta\theta}.
\end{equation}
We see that
\begin{equation}
\left(\frac{\cos(\Delta\theta)-1}{\Delta\theta}\right)=\bigO(\Delta\theta)
\end{equation}
and
\begin{equation}
\frac{\sin(\Delta\theta)}{\Delta\theta}=1-\bigO\bigl(\Delta\theta^{2}\bigr).
\end{equation}
We understand that $\Delta x^{2}=(\Delta x)^{2}$ and not $\Delta(x^{2})$.

So when we take the limit $\Delta\theta\to0$ we obtain
\begin{equation}
\lim_{\Delta\theta\to0}\frac{\Delta\cos(\theta)}{\Delta\theta}
=
\lim_{\Delta\theta\to0}\cos(\theta)(\bigO(\Delta\theta))-\sin(\theta)+\bigO(\Delta\theta^{2}).
\end{equation}
This gives us the derivative
\begin{equation}
\frac{\D\cos(\theta)}{\D\theta}=-\sin(\theta).
\end{equation}

\N{Derivative of Tangent}
We should recall
\begin{equation}
\tan(\theta)=\frac{\sin(\theta)}{\cos(\theta)}.
\end{equation}
What is its derivative? Well, we can use the product rule:
\begin{equation}
\frac{\D\tan(\theta)}{\D\theta}=
\frac{1}{\cos(\theta)}\frac{\D\sin(\theta)}{\D\theta}
+\sin(\theta)\frac{\D}{\D\theta}\left(\frac{1}{\cos(\theta)}\right).
\end{equation}
We see that the first term on the right hand side is $1$, and the
second term we use the power rule and chain rule:
\begin{equation}
\frac{\D}{\D\theta}\frac{1}{\cos(\theta)}=\frac{-1}{\cos^{2}(\theta)}(-\sin(\theta))
\end{equation}
Thus we obtain for the derivative of tangent
\begin{equation}
\begin{split}
\frac{\D\tan(\theta)}{\D\theta}&=1+\tan^{2}(\theta)\\
&=\frac{\cos^{2}(\theta)+\sin^{2}(\theta)}{\cos^{2}(\theta)}\\
&=\frac{1}{\cos^{2}(\theta)}=\sec^{2}(\theta).
\end{split}
\end{equation}


