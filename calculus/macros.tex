%%
%% macros.tex
%% 
%% Made by alex
%% Login   <alex@tomato>
%% 
%% Started on  Fri May 25 10:50:22 2012 alex
%% Last update Fri Jun  1 10:23:53 2012 Alex Nelson
%%

\makeatletter
\hyphenpenalty=500
\vbadness=200
\widowpenalty=10000
\clubpenalty=10000
% automatically generate the author information
%\renewcommand{\thefootnote}{\fnsymbol{footnote}}
\usepackage[12hr,us]{datetime}

\author{Alex Nelson\footnote{This is a page from \url{http://code.google.com/p/notebk/}\hfil\break\indent\;\, Compiled:\enspace\today\ at \currenttime\ (PST)}\\\texttt{Email:\enspace\href{mailto:pqnelson@gmail.com}{pqnelson@gmail.com}}}

\usepackage{amsmath,amsfonts,amscd,amssymb,amsthm}
\usepackage{booktabs}
\usepackage{float,framed}
\usepackage{ifpdf}
\usepackage{graphicx}
\ifpdf
\DeclareGraphicsRule{*}{mps}{*}{}
\usepackage{microtype}
\fi
\usepackage{color}
\usepackage{marginnote}
\definecolor{BrickRed}{rgb}{0.45,0,0}
\usepackage[final,colorlinks=true, 
            hyperindex=true,
            citecolor=BrickRed,
            filecolor=BrickRed,
            menucolor=BrickRed,
            linkcolor=BrickRed,
            urlcolor=BrickRed,
            bookmarksopen=true,
            pdfauthor={Alex Nelson},
            unicode=false]{hyperref}
\usepackage[all]{hypcap}
\font\chapterfont=cmssbx10 scaled\magstep2
\font\twelvess=cmss12
\def\sectionfont{\large\sffamily\bfseries} %=cmssbx10 at 12pt
\def\subsectionfont{\large\sffamily\bfseries} %\font\subsectionfont=cmssbx10
\font\manual=manfnt
\font\titlefont=cmssbx10 scaled\magstep4
\font\eightss=cmssq8
\font\eightssi=cmssqi8
\font\sixss=cmssq8 scaled 800

\setlength\parindent{20pt}
\setlength\parskip{0pt plus 1pt}
\newskip\smallskipamount \smallskipamount=3pt plus 1pt minus 1pt
\newskip\medskipamount \medskipamount=6pt plus 2pt minus 2pt
\newskip\bigskipamount \bigskipamount=12pt plus 4pt minus 4pt

\def\smallskip{\vskip\smallskipamount}
\def\medskip{\vskip\medskipamount}
\def\bigskip{\vskip\bigskipamount}
\def\removelastskip{\ifdim\lastskip=\z@\else\vskip-\lastskip\fi}
\def\smallbreak{\par\ifdim\lastskip<\smallskipamount
  \removelastskip\penalty-50\smallskip\fi}
\def\medbreak{\par\ifdim\lastskip<\medskipamount
  \removelastskip\penalty-100\medskip\fi}
\def\bigbreak{\par\ifdim\lastskip<\bigskipamount
  \removelastskip\penalty-200\bigskip\fi}

\numberwithin{equation}{section}

\def\specialsection{\@startsection{section}{1}%
  \z@{\baselineskip\@plus\baselineskip}{.25\baselineskip}%
  {\sectionfont}}
\def\section{\@startsection{section}{1}%
  \z@{.7\baselineskip\@plus\baselineskip}{.125\baselineskip}%
  {\sectionfont}}
\def\subsection{\@startsection{subsection}{2}%
  \z@{.7\baselineskip\@plus\baselineskip}{.125\baselineskip}%
%  \z@{.5\linespacing\@plus.7\linespacing}{-.5em}%
  {\subsectionfont}}
\def\subsubsection{\@startsection{subsubsection}{3}%
  \z@{.7\baselineskip\@plus\baselineskip}{.125\baselineskip}%
%  \z@{.5\linespacing\@plus.7\linespacing}{-.5em}%
  {\subsectionfont}}

%%%%%%%%%%%%%%%%%%%%%
% Author Quotes
%%%%%%%%%%%%%%%%%%%%%
\def\quoteformat{
  \baselineskip 10pt
  \parfillskip \z@
  \interlinepenalty 10000
  \leftskip \z@ plus 40pc minus \parindent
  \let\rm=\eightss \let\sl=\eightssi \let\adbcfont=\sixss
  \everypar{\sl}
  \def\\{\hskip.05em} % can say 3\\:\\16
  \obeylines}
\newenvironment{quotes}{
  \baselineskip 10pt
  \parfillskip \z@
  \interlinepenalty 10000
  \leftskip \z@ plus 40pc minus \parindent
  \let\rm=\eightss \let\sl=\eightssi \let\adbcfont=\sixss
  \everypar{\sl}
  \def\\{\hskip.05em} % can say 3\\:\\16
  \obeylines}{}
\def\Author#1(#2){\par\nobreak\smallskip\noindent\rm--- #1\unskip\enspace(#2)}

%%%%%%%%%%%%%%%%%%%%%%%%%%%%%%%%%%%%%%%%%%%%%%%%%%%%%%%%%%%%%%%%%%%%%%%%%%%%%%%%

\def\dbend{{\manual\char127}} % dangerous bend sign
\let\dangersize\footnotesize
% Danger, Will Robinson!
\newenvironment{danger}{\medbreak\noindent\hangindent=2pc\hangafter=-2%
  \clubpenalty=10000%
  \hbox to0pt{\hskip-\hangindent\dbend\hfill}\dangersize\ignorespaces}%
  {\medbreak\par}

% Danger! Danger!
\newenvironment{ddanger}{\medbreak\noindent\hangindent=3pc\hangafter=-2%
  \clubpenalty=10000%
  \hbox to0pt{\hskip-\hangindent\dbend\kern2pt\dbend\hfill}\dangersize\ignorespaces}%
  {\medbreak\par}

%%%%%%%%%%%%%%%%%%%%%%%%%%%%%%%%%%%%%%%%%%%%%%%%%%%%%%%%%%%%%%%%%%%%%%%%%%%%%%%%

\def\textindent#1{\indent \llap {#1\enspace }\ignorespaces}
\newcounter{exercise}[section]
\renewcommand{\theexercise}{\thesection.\arabic{exercise}}
\def\exercise [#1]{\refstepcounter{exercise}\ifnum\value{exercise}>1 \smallbreak\fi
  \textindent{\textbf{\arabic{exercise}.}}[\textit{#1\/}]\kern6pt}
\def\suggestedExercise [#1]{\refstepcounter{exercise}\ifnum\value{exercise}>1 \smallbreak\fi
  \textindent{\llap{\manual x\hskip3pt}\bf{\hbox to
     \ifnum \value{exercise}>99 1.5em\else 1em\fi{\hfil\arabic{exercise}}}.}[\textit{#1\/}]\kern6pt}

\def\HM{H\kern-.1em M} % used for "higher math" exercise ratings
\def\MN{M\kern-.1em N} % used in Section 4.3.1 when $MN$ appears frequently

\newenvironment{exercises}{\medskip\phantomsection%\addcontentsline{toc}{section}{Exercises}%
        \noindent\ignorespaces\section*{EXERCISES}%
        \parindent=0pt%
        \setcounter{exercise}{0}}{}

\newwrite\ans%
\immediate\openout\ans=tex/answers % file for answers to exercises
\def\ansSec#1{\immediate\write\ans{\detokenize{\section*}{#1}}}
\def\ansSec#1{\immediate\write\ans{\detokenize{\section*{#1}}}}
\def\ansno#1:{\smallbreak\textindent{\textbf{#1.\enspace}}}
\newenvironment{answer}%
 {\par\medbreak
    \immediate\write\ans{}%
    \immediate\write\ans{\string\indent\string\ansno\arabic{exercise}:}%
  \@bsphack
  \let\do\@makeother\dospecials\catcode`\^^M\active
  \def\verbatim@processline{%
    \immediate\write\ans{\the\verbatim@line}}%
  \verbatim@start}%
 {\@esphack} 

%% Sometimes there's an alternate way to solve the problem
%% TODO: Make this include an optional argument specifying how it
%%       was done, so one can use the following code snippet
%% \begin{altAns}[Using Complex Analysis]
%% ...
%% \end{altAns}
\newenvironment{altAns}%
 {\par\medbreak
    \immediate\write\ans{}%
    \immediate\write\ans{\string\ansno\arabic{exercise}:}%
    \immediate\write\ans{\detokenize{\textbf{(Alternate)}}}%
  \@bsphack
  \let\do\@makeother\dospecials\catcode`\^^M\active
  \def\verbatim@processline{%
    \immediate\write\ans{\the\verbatim@line}}%
  \verbatim@start}%
 {\@esphack} 

%%%%%%%%%%%%%%%%%%%%%%%%%%%%%%%%%%%%%%%%%%%%%%%%%%%%%%%%%%%%%%%%%%%%%%%%%%%%%%%%

\def\arXiv#1{\href{http://arxiv.org/abs/#1}{\texttt{arXiv:#1}}}

%{\MakeFramed {\advance\hsize-\width \FrameRestore}}%
%  {\endMakeFramed}
\newcounter{boxed}[section]
\newenvironment{Boxed}[1]{\refstepcounter{boxed}\MakeFramed {\advance\hsize 2pc\advance \leftskip 0.5pc \advance \rightskip 0.5pc\FrameRestore}\noindent{\subsectionfont Box \theboxed. #1}\vskip .7\baselineskip}{\endMakeFramed}

%%%%%%%%%%%%%%%%%%%%%%%%%%%%%%%%%%%%%%%%%%%%%%%%%%%%%%%%%%%%%%%%%%%%%%%%%%%%%%%%
\newcommand{\slug}{\hbox{\kern1.5pt\vrule width2.5pt height6pt depth1.5pt\kern1.5pt}}

\theoremstyle{plain}
\newtheorem{thm}{Theorem}[section]
\newtheorem{prop}[thm]{Proposition}
\newtheorem{lem}[thm]{Lemma}
\newtheorem{cor}[thm]{Corollary}
\newtheorem{conjecture}[thm]{Conjecture}
\newtheorem*{ExtensionPrinciple}{Extension Principle}
\newtheorem*{DiracConjecture}{Dirac's Conjecture}
\newtheorem*{bessel}{Bessel Inequality}
\newtheorem*{parseval}{Parseval Equality}
\newtheorem*{riemleb}{Riemann-Lebesgue Theorem}
\newtheorem*{invFourier}{Fourier Inversion Theorem}
\newtheorem*{plancherel}{Plancherel Theorem}
\newtheorem*{samplingthm}{Sampling Theorem}
\newtheorem*{dual}{Dual Principle for Categories}
\newtheorem*{yoneda}{Yoneda Lemma}
\newtheorem*{schur}{Schur's Lemma}
\newtheorem*{implicitFunctionThm}{Implicit Function Theorem}
\newtheorem*{minNum}{Minimal Number Principle}
\newtheorem*{continuumHypothesis}{Continuum Hypothesis}
\newtheorem*{archimedeanProperty}{Archimedean Property}
\theoremstyle{definition}
\newtheorem{xca}{Exercise}[section]
\newtheorem{defn}[thm]{Definition}
\newtheorem{ex}[thm]{Example}
\newtheorem{nonex}[thm]{NON-Example}
\newtheorem{futureEx}[thm]{Future Example}
\newtheorem{axiom}[thm]{Axiom}
\newtheorem{fact}{Experimental Fact}
\newtheorem{prob}[thm]{Problem}
\newtheorem{con}[thm]{Conjecture}
\newtheorem{notation}[thm]{Notation}
\newtheorem{problem}[thm]{Problem}
\newtheorem{construction}[thm]{Construction}
\newtheorem*{assume}{Assumption}
\newtheorem*{quest}{Question}
\theoremstyle{remark}
\newtheorem{rmk}[thm]{Remark}
\newtheorem{sch}[thm]{Scholium}

\renewcommand{\qedsymbol}{\slug}

\providecommand{\sketchname}{Sketch of Proof}
\newenvironment{sketch}[1][\sketchname]{\par
  \pushQED{\qed}%
  \normalfont \topsep6\p@\@plus6\p@\relax
  \trivlist
  \item[\hskip\labelsep
        \itshape
    #1\@addpunct{.}]\ignorespaces
}{%
  \popQED\endtrivlist\@endpefalse
}

%%%%%%%%%%%%%%%%%%%%%%%%%%%%%%%%%%%%%%%%%%%%%%%%%%%%%%%%%%%%%%%%%%%%%%%%%%%%%%%%


\def\overbracket{\@ifnextchar [ {\@overbracket} {\@overbracket
[\@bracketheight]}}
\def\@overbracket[#1]{\@ifnextchar [ {\@over@bracket[#1]}
{\@over@bracket[#1][0.3em]}}
\def\@over@bracket[#1][#2]#3{%\message {Overbracket: #1,#2,#3}
\mathop {\vbox {\m@th \ialign {##\crcr \noalign {\kern 3\p@
\nointerlineskip }\downbracketfill {#1}{#2}
                              \crcr \noalign {\kern 3\p@ }
                              \crcr  $\hfil \displaystyle {#3}\hfil $%
                              \crcr} }}\limits}
\def\downbracketfill#1#2{$\m@th \setbox \z@ \hbox {$\braceld$}
                  \edef\@bracketheight{\the\ht\z@}\downbracketend{#1}{#2}
                  \leaders \vrule \@height #1 \@depth \z@ \hfill
                  \leaders \vrule \@height #1 \@depth \z@ \hfill
\downbracketend{#1}{#2}$}
\def\downbracketend#1#2{\vrule depth #2 width #1\relax}


\def\underbracket{%
  \@ifnextchar[{\@underbracket}{\@underbracket [\@bracketheight]}%
}
\def\@underbracket[#1]{%
  \@ifnextchar[{\@under@bracket[#1]}{\@under@bracket[#1][0.4em]}%
}
\def\@under@bracket[#1][#2]#3{%\message {Underbracket: #1,#2,#3}
 \mathop{\vtop{\m@th \ialign {##\crcr $\hfil \displaystyle {#3}\hfil $%
 \crcr \noalign {\kern 3\p@ \nointerlineskip }\upbracketfill {#1}{#2}
       \crcr \noalign {\kern 3\p@ }}}}\limits}
\def\upbracketfill#1#2{$\m@th \setbox \z@ \hbox {$\braceld$}
                    \edef\@bracketheight{\the\ht\z@}\bracketend{#1}{#2}
                    \leaders \vrule \@height #1 \@depth \z@ \hfill
                    \leaders \vrule \@height #1 \@depth \z@ \hfill \bracketend
                      {#1}{#2}$}
\def\bracketend#1#2{\vrule height #2 width #1\relax}

\allowdisplaybreaks
\def\mathllap{\mathpalette\mathllapinternal}
\def\mathllapinternal#1#2{\llap{$\mathsurround=0pt#1{#2}$}}
\def\clap#1{\hbox to 0pt{\hss#1\hss}}
\def\mathclap{\mathpalette\mathclapinternal}
\def\mathclapinternal#1#2{\clap{$\mathsurround=0pt#1{#2}$}}
\def\mathrlap{\mathpalette\mathrlapinternal}
\def\mathrlapinternal#1#2{\rlap{$\mathsurround=0pt#1{#2}$}}

%%%%%%%%%%%%%%%%%%%%%%%%%%%%%%%%%%%%%%%%%%%%%%%%%%%%%%%%%%%%%%%%%%%%%%%%%%%%%%%%
%%%%%%%%%%%%%%%%%%%%%%%%%%%%%%%%%%%%%%%%%%%%%%%%%%%%%%%%%%%%%%%%%%%%%%%%%%%%%%%%

\def\xskip{\hskip 7pt plus 3pt minus 4pt}
\newdimen\algindent
\newif\ifitempar \itempartrue % normally true unless briefly set false
\def\algindentset#1{\setbox0\hbox{{\bf #1.\kern.25em}}\algindent=\wd0\relax}
\def\algbegin #1 #2{\algindentset{#21}\alg #1 #2} % when steps all have 1 digit
\def\aalgbegin #1 #2{\algindentset{#211}\alg #1 #2} % when 10 or more steps
\def\alg#1(#2). {\medbreak % Usage: \algbegin Algorithm A (algname). This...
  \noindent{\bf#1}({\it#2\/}).\xskip\ignorespaces}
\def\algstep#1.{\ifitempar\smallskip\noindent\else\itempartrue
  \hskip-\parindent\fi
  \hbox to\algindent{\bf\hfil #1.\kern.25em}%
  \hangindent=\algindent\hangafter=1\ignorespaces}
% end of borrowed macros

% For Example, with less than 10 steps:
% \algbegin X (Multiplication). blah blah blah blah...
% \algstep X1. [{\it Do stuff\/}] blah blah blah
% \algstep X2. Terminate the algorithm.\quad\slug


\newcommand{\re}{\mathop{\mathrm{Re}}\nolimits}
\newcommand{\im}{\mathop{\mathrm{Im}}\nolimits}
\newcommand{\tr}{\mathop{\mathrm{Tr}}\nolimits}
\newcommand{\aut}{\mathop{\mathrm{Aut}}\nolimits}
\renewcommand{\ker}{\mathop{\mathrm{Ker}}\nolimits}
\newcommand{\coker}{\mathop{\mathrm{Coker}}\nolimits}
\newcommand{\erfi}{\mathop{\mathrm{erfi}}\nolimits}
\newcommand{\erf}{\mathop{\mathrm{erf}}\nolimits}
\let\<\langle
\let\>\rangle
\let\iso\cong
\newcommand{\equalsdef}{\stackrel{\text{def}}{=}}
\let\eqdef\equalsdef
\newcommand{\define}[1] {``\/\textbf{#1}\/\/''}%\index{#1}}
\newcommand{\lie}{\mathop{\mathrm{Lie}}\nolimits}
\let\Lie\lie
\renewcommand{\arctan}{\mathop{\mathrm{ArcTan}}\nolimits}
\renewcommand{\arcsin}{\mathop{\mathrm{ArcSin}}\nolimits}
\renewcommand{\arccos}{\mathop{\mathrm{ArcCos}}\nolimits}
\renewcommand{\hom}{\mathop{\mathrm{Hom}}\nolimits}
\newcommand\fun{\mathop{\mathrm{Fun}}\nolimits} % like \hom but for categories
\newcommand\nat{\mathop{\mathrm{Nat}}\nolimits} % like \hom but for functors
\let\Hom\hom
\def\noparskip{}
\newcommand{\id}[1]{\mathop{\mathrm{id}}\nolimits_{#1}}
\newcommand{\ob}[1]{\mathrm{Ob}(#1)}
\newcommand{\cat}[1]{\mathbf{#1}} % notation for a category, e.g. $\cat{Set}$
\newcommand\powerset[1]{\mathcal{P}(#1)}
\newcommand{\sgn}{\mathop{\mathrm{sgn}}\nolimits}
\newcommand{\mat}{\mathop{\mathrm{Mat}}\nolimits}
\newcommand{\der}{\mathop{\mathrm{Der}}\nolimits}
\newcommand{\Ad}{\mathop{\mathrm{Ad}}\nolimits}
\newcommand{\ad}{\mathop{\mathrm{ad}}\nolimits}
%% Frequently used sets
\let\logic\texttt
%\def\logic#1{\relax\ifmmode\mathtt{#1}\else\texttt{#1}\fi}
\newcommand\CC{\mathbb{C}}
\newcommand\FF{\mathbb{F}}
\newcommand\NN{\mathbb{N}}
\newcommand\OO{\mathbb{O}}
\newcommand\QQ{\mathbb{Q}}
\newcommand\RR{\mathbb{R}}
\newcommand\ZZ{\mathbb{Z}}
\let\oldsetminus\setminus
\renewcommand\setminus{-}
\newcommand\universe{\mathcal{U}}
\newcommand\domain{\mathop{\mathrm{dom}}\nolimits}
\newcommand\codomain{\mathop{\mathrm{cod}}\nolimits}
\let\propersubset\subset
\let\propersupset\supset
\let\oldsubset\subset
\let\oldsupset\supset
\let\subset\subseteq
\let\supset\supseteq
\newcommand\1{\mathbf{1}}
\newcommand\cardinality[1]{\left|#1\right|}
\newcommand\continuumCardinality{\mathfrak{c}}
%% Lie groups frequently used
\def\GL#1{\mathrm{GL}(#1)}
\def\SL#1{\mathrm{SL}(#1)}
\def\SO#1{\mathrm{SO}(#1)}
\def\ORTH#1{\mathrm{O}(#1)}
\def\U#1{\mathrm{U}(#1)}
\def\SU#1{\mathrm{SU}(#1)}
\def\Sp#1{\mathrm{Sp}(#1)}

%% Differential Geometry refers to these quantities frequently as well
\def\diff{\mathrm{Diff}}
\def\vect{\mathrm{Vect}}

%% Categories frequently used
\newcommand\Set{\cat{Set}} 		% category of sets
\newcommand\FinSet{\cat{FinSet}} 	% category of finite sets
\newcommand\Cat{\cat{Cat}} 		% category of categories
\newcommand\Vect{\cat{Vect}} 		% category of vector spaces
\newcommand\Top{\cat{Top}} 		% category of topological spaces
\newcommand\Grp{\cat{Grp}} 		% category of groups
\newcommand\Ab{\cat{Ab}}   		% category of Abelian groups
\newcommand\Gpd{\cat{Gpd}} 		% category of groupoids
\newcommand\Alg{\cat{Alg}} 		% category of algebras over a field
\newcommand\Magm{\cat{Magm}} 		% category of Magmas
\newcommand\Ord{\cat{Ord}} 		% category of ordinals
\newcommand\FinOrd{\cat{FinOrd}} 	% category of FINITE ordinals

\def\weak={\approx} % weak equality when working in classical gauge theory
\def\PB(#1,#2){\left\{#1,#2\right\}} % the Poisson bracket for field theory and mechanics

\newcommand\ClassicalGroup[1]{\mathsf{#1}}
\newcommand\Antisymmetric{{\textstyle\bigwedge\nolimits}}

%\let\oldemptyset\emptyset
\let\varemptyset\varnothing

% number theoretic notation
\let\totient\varphi
\DeclareFontFamily{U}{matha}{\hyphenchar\font45}
\DeclareFontShape{U}{matha}{m}{n}{
      <5> <6> <7> <8> <9> <10> gen * matha
      <10.95> matha10 <12> <14.4> <17.28> <20.74> <24.88> matha12
      }{}
\DeclareSymbolFont{matha}{U}{matha}{m}{n}
\DeclareMathSymbol{\divides}{3}{matha}{"17}
\DeclareMathSymbol{\notdivides}{3}{matha}{"1F} 
%\newcommand{\divides}{\mathrel{\vert}}
%\newcommand{\notdivides}{\negthinspace\mathrel{\not\mskip-0.5mu\vert}}

% Notation used for algebraic topology
\let\homotopic\sim

% Notation for mathematical logic
\def\tee{\logic{true}}
\def\falsum{\logic{false}}
\let\entails\implies
\def\lst{\texttt{\thinspace{}:\thinspace{}}} % logical ``such that'' notation

\def\bigO{\mathcal{O}}
\newcommand\identify{\noindent\llap{\rm1.\kern.5em}\textbf{Identify:}}
\newcommand\setup{\noindent\llap{\rm2.\kern.5em}\textbf{Set up:}}
\newcommand\execute{\noindent\llap{\rm3.\kern.5em}\textbf{Execute:}}
\newcommand\evaluate{\noindent\llap{\rm4.\kern.5em}\textbf{Evaluate:}}

\newcommand\D{\mathrm{d}}
\newcommand\E{\mathrm{e}}
\let\eul=\E
\newcommand\I{\mathrm{i}}


\newcounter{@chunk}[section]
\renewcommand\the@chunk{\thesection.\arabic{@chunk}}

\def\M{\medbreak
  \noindent\refstepcounter{@chunk}%
  \textbf{\the@chunk\@addpunct{.}\quad}\ignorespaces}
\def\N#1{\M\textbf{#1\@addpunct{.}\quad}\ignorespaces}


\makeatother
