\usepackage{sidecap}
\usepackage{cancel}
\usepackage{xmpincl}
\let\oldmessage\message
\def\message#1{}
\usepackage[nohug]{diagrams}
\let\message\oldmessage
\usepackage[12hr,us]{datetime}


\author{Alex Nelson\footnote{This is a page from \url{http://code.google.com/p/notebk/}\hfil\break\indent\;\, Compiled:\enspace\today\ at \currenttime\ (PST)}\\\texttt{Email:\enspace\href{mailto:pqnelson@gmail.com}{pqnelson@gmail.com}}}

\usepackage{booktabs}
\usepackage[top=6pc,textheight=8.9in,textwidth=33pc,inner=6pc,marginparwidth=9pc,marginparsep=1.5pc]{geometry}
\usepackage{amsmath,amsfonts,amscd,amssymb}
\usepackage{extpfeil,manfnt}
%\usepackage{epic,eepic}
\usepackage{arydshln} % for the dashed lines in the ADM metric
\usepackage{float}
\usepackage{framed}
\usepackage{ifpdf}
\usepackage[final]{graphicx}
\ifpdf
\DeclareGraphicsRule{*}{mps}{*}{}
\usepackage[kerning]{microtype}
\fi
\usepackage{mathrsfs}
\usepackage{comment}
\usepackage{verbatim}
\usepackage{slashed}
\usepackage{paralist}
\usepackage{marginnote}
\usepackage{subfigure}
\usepackage{wrapfig} % used for wrapping figures!
\usepackage{amsthm}
\usepackage{fancyhdr}
\usepackage[final,colorlinks=true, 
            hyperindex=true,
            citecolor=black,
            filecolor=black,
            menucolor=black,
            linkcolor=black,
            urlcolor=black,
            bookmarksopen=true,
            pdfauthor={Alex Nelson}]{hyperref}
\usepackage[all]{hypcap}
\makeatletter
%%%%%%%%%%%%%%%%%%%%%%%%%%%%%%%%%%%%%%%%%%%%%%%%%%%%%%%%%%%%%%%%%%%%%%%%%%%%%%%
% Font used...
%%%%%%%%%%%%%%%%%%%%%%%%%%%%%%%%%%%%%%%%%%%%%%%%%%%%%%%%%%%%%%%%%%%%%%%%%%%%%%%
\font\manual=manfnt
\font\chapterfont=cmssbx10 scaled\magstep2
\font\twelvess=cmss12
\font\sectionfont=cmssbx10 at 12pt
\font\subsectionfont=cmssbx10 %at 12pt
\def\upbf#1#2{\textbf{#1\uppercase{{\footnotesize#2}}}}
\font\eightss=cmssq8
\font\eightssi=cmssqi8
\font\sixss=cmssq8 scaled 800
%%%%%%%%%%%%%%%%%%%%%%%%%%%%%%%%
% Fix the line skip amounts
%%%%%%%%%%%%%%%%%%%%%%%%%%%%%%%%
\setlength\parindent{20pt}
\setlength\parskip{0pt plus 1pt}
\newskip\smallskipamount \smallskipamount=3pt plus 1pt minus 1pt
\newskip\medskipamount \medskipamount=6pt plus 2pt minus 2pt
\newskip\bigskipamount \bigskipamount=12pt plus 4pt minus 4pt
\def\smallskip{\vskip\smallskipamount}
\def\medskip{\vskip\medskipamount}
\def\bigskip{\vskip\bigskipamount}
\def\removelastskip{\ifdim\lastskip=\z@\else\vskip-\lastskip\fi}
\def\smallbreak{\par\ifdim\lastskip<\smallskipamount
  \removelastskip\penalty-50\smallskip\fi}
\def\medbreak{\par\ifdim\lastskip<\medskipamount
  \removelastskip\penalty-100\medskip\fi}
\def\bigbreak{\par\ifdim\lastskip<\bigskipamount
  \removelastskip\penalty-200\bigskip\fi}


\newenvironment{quotes}{
  \baselineskip 10pt
  \parfillskip \z@
  \interlinepenalty 10000
  \leftskip \z@ plus 40pc minus \parindent
  \let\rm=\eightss \let\sl=\eightssi \let\adbcfont=\sixss
  \everypar{\sl}
  \def\\{\hskip.05em} % can say 3\\:\\16
  \obeylines}{}
\def\Author#1(#2){\par\nobreak\smallskip\noindent\rm--- #1\unskip\enspace(#2)}
%%%%%%%%%%%%%%%%%%%%%%%%%%%%%%%%%%%%%%%%%%%%%%%%%%%%%%%%%%%%%%%%%%%%%%%%%%%%%%%
% We customize theorem styles...
%%%%%%%%%%%%%%%%%%%%%%%%%%%%%%%%%%%%%%%%%%%%%%%%%%%%%%%%%%%%%%%%%%%%%%%%%%%%%%%

\let\Bbb=\mathbb
\let\frak=\mathfrak

\theoremstyle{plain}
\newtheorem{thm}{Theorem}[chapter]
\newtheorem{prop}[thm]{Proposition}
\newtheorem{lem}[thm]{Lemma}
\newtheorem{cor}[thm]{Corollary}
\newtheorem{conjecture}[thm]{Conjecture}
\newtheorem*{ClassificationCliff}{Classification Theorem for Representations of Clifford Algebras}
\newtheorem*{DiracConjecture}{Dirac's Conjecture}
\newtheorem*{bessel}{Bessel Inequality}
\newtheorem*{parseval}{Parseval Equality}
\newtheorem*{riemleb}{Riemann-Lebesgue Theorem}
\newtheorem*{riemMap}{Riemann Mapping Theorem}
\newtheorem*{invFourier}{Fourier Inversion Theorem}
\newtheorem*{plancherel}{Plancherel Theorem}
\newtheorem*{samplingthm}{Sampling Theorem}
\newtheorem*{dual}{Dual Principle for Categories}
\newtheorem*{yoneda}{Yoneda Lemma}
\newtheorem*{schur}{Schur's Lemma}
\newtheorem*{implicitFunctionThm}{Implicit Function Theorem}
\newtheorem*{minNum}{Minimal Number Principle}
\theoremstyle{definition}
\newtheorem{defn}[thm]{Definition}
\newtheorem{ex}[thm]{Example}
\def\example{\ex}
\def\endexample{\hfill$\square$\endex}
\newtheorem{nonex}[thm]{NON-Example}
\newtheorem{axiom}{Axiom}
\newtheorem{fact}{Experimental Fact}
\newtheorem{problem}[thm]{Problem}
\newtheorem{construction}[thm]{Construction}
\def\construct{\construction}
\def\endconstruct{\hfill$\square$\endconstruction}
\newtheorem{con}[thm]{Conjecture}
\newtheorem*{notation}{Notation}
\newtheorem*{assume}{Assumption}
\newtheorem*{quest}{Question}
\theoremstyle{remark}
\newtheorem{rmk}[thm]{Remark}
\newtheorem{sch}[thm]{Scholium}



\newcommand{\slug}{\hbox{\kern1.5pt\vrule width2.5pt height6pt depth1.5pt\kern1.5pt}}
\newcommand{\slugonright}{\vrule width0pt\nobreak\hfill\slug}
\renewcommand{\qedsymbol}{\slug}


\providecommand{\sketchname}{Sketch of Proof}
\newenvironment{sketch}[1][\sketchname]{\par
  \pushQED{\qed}%
  \normalfont \topsep6\p@\@plus6\p@\relax
  \trivlist
  \item[\hskip\labelsep
        \itshape
    #1\@addpunct{.}]\ignorespaces
}{%
  \popQED\endtrivlist\@endpefalse
}

\allowdisplaybreaks

%\numberwithin{figure}{section}
\numberwithin{equation}{chapter}

\pagestyle{fancy}
\if@twoside
\fancyhead[LE,RO]{\thepage}
\fancyhead[LO,RE]{\nouppercase{\leftmark}}
\fancyheadoffset[OR,EL]{8pc}
\else
\fancyhead[L]{\nouppercase{\leftmark}}
\fancyhead[R]{\thepage}
\fancyheadoffset[R]{8pc}
\fi
\cfoot{}
\renewcommand{\headrulewidth}{0.4pt}
\headsep=5pt

%%%%%%%%%%%%%%%%%%%%%%%%%%%%%%%%%%%%%%%%%%%%%%%%%%%%%%%%%%%%%%%%%%%%%%%%%%%%%%%
% We also make the margin notes look prettier
%%%%%%%%%%%%%%%%%%%%%%%%%%%%%%%%%%%%%%%%%%%%%%%%%%%%%%%%%%%%%%%%%%%%%%%%%%%%%%%
\if@twoside
\renewcommand\marginpar[1]{\-\marginnote{\footnotesize{{#1}}}}
\else
\renewcommand\marginpar[1]{\-\marginnote{\raggedright\footnotesize{{#1}}}}%
\fi

%%%%%%%%%%%%%%%%%%%%%%%%%%%%%%%%%%%%%%%%%%%%%%%%%%%%%%%%%%%%%%%%%%%%%%%%%%%%%%%
% Bracket macros
%%%%%%%%%%%%%%%%%%%%%%%%%%%%%%%%%%%%%%%%%%%%%%%%%%%%%%%%%%%%%%%%%%%%%%%%%%%%%%%
\def\overbracket{\@ifnextchar [ {\@overbracket} {\@overbracket
[\@bracketheight]}}
\def\@overbracket[#1]{\@ifnextchar [ {\@over@bracket[#1]}
{\@over@bracket[#1][0.3em]}}
\def\@over@bracket[#1][#2]#3{%\message {Overbracket: #1,#2,#3}
\mathop {\vbox {\m@th \ialign {##\crcr \noalign {\kern 3\p@
\nointerlineskip }\downbracketfill {#1}{#2}
                              \crcr \noalign {\kern 3\p@ }
                              \crcr  $\hfil \displaystyle {#3}\hfil $%
                              \crcr} }}\limits}
\def\downbracketfill#1#2{$\m@th \setbox \z@ \hbox {$\braceld$}
                  \edef\@bracketheight{\the\ht\z@}\downbracketend{#1}{#2}
                  \leaders \vrule \@height #1 \@depth \z@ \hfill
                  \leaders \vrule \@height #1 \@depth \z@ \hfill
\downbracketend{#1}{#2}$}
\def\downbracketend#1#2{\vrule depth #2 width #1\relax}


\def\underbracket{%
  \@ifnextchar[{\@underbracket}{\@underbracket [\@bracketheight]}%
}
\def\@underbracket[#1]{%
  \@ifnextchar[{\@under@bracket[#1]}{\@under@bracket[#1][0.4em]}%
}
\def\@under@bracket[#1][#2]#3{%\message {Underbracket: #1,#2,#3}
 \mathop{\vtop{\m@th \ialign {##\crcr $\hfil \displaystyle {#3}\hfil $%
 \crcr \noalign {\kern 3\p@ \nointerlineskip }\upbracketfill {#1}{#2}
       \crcr \noalign {\kern 3\p@ }}}}\limits}
\def\upbracketfill#1#2{$\m@th \setbox \z@ \hbox {$\braceld$}
                    \edef\@bracketheight{\the\ht\z@}\bracketend{#1}{#2}
                    \leaders \vrule \@height #1 \@depth \z@ \hfill
                    \leaders \vrule \@height #1 \@depth \z@ \hfill \bracketend
	              {#1}{#2}$}
\def\bracketend#1#2{\vrule height #2 width #1\relax}

\def\mathllap{\mathpalette\mathllapinternal}
\def\mathllapinternal#1#2{\llap{$\mathsurround=0pt#1{#2}$}}
\def\clap#1{\hbox to 0pt{\hss#1\hss}}
\def\mathclap{\mathpalette\mathclapinternal}
\def\mathclapinternal#1#2{\clap{$\mathsurround=0pt#1{#2}$}}
\def\mathrlap{\mathpalette\mathrlapinternal}
\def\mathrlapinternal#1#2{\rlap{$\mathsurround=0pt#1{#2}$}}

%%%%%%%%%%%%%%%%%%%%%%%%%%%%%%%%%%%%%%%%%%%%%%%%%%%%%%%%%%%%%%%%%%%%%%%%%%%%%%%
% Dangerous bend macros
%%%%%%%%%%%%%%%%%%%%%%%%%%%%%%%%%%%%%%%%%%%%%%%%%%%%%%%%%%%%%%%%%%%%%%%%%%%%%%%
%%
% This macro header is what controls the ``dangerous bend''
% paragraph
%%
\def\dbend{{\manual\char127}} % dangerous bend sign

% Danger, Will Robinson!
\newenvironment{danger}{\medbreak\noindent\hangindent=2pc\hangafter=-2%
  \clubpenalty=10000%
  \hbox to0pt{\hskip-\hangindent\dbend\hfill}\small\ignorespaces}%
  {\medbreak\par}

% Danger! Danger!
\newenvironment{ddanger}{\medbreak\noindent\hangindent=3pc\hangafter=-2%
  \clubpenalty=10000%
  \hbox to0pt{\hskip-\hangindent\dbend\kern2pt\dbend\hfill}\small\ignorespaces}%
  {\medbreak\par}

%% % Danger, Will Robinson!
%% \def\danger{\begin{footnotesize}\noindent%
%% \hangindent=20pt\hangafter=-2%
%% \hbox to0pt{\hskip-\hangindent\dbend\hfill}\ignorespaces}
%% \def\enddanger{\par\end{footnotesize}\medbreak}
%%
%% % Danger! Danger!
%% \def\ddanger{\medbreak\begin{footnotesize}\noindent%
%% \hangindent=36pt\hangafter=-2%
%% \hbox to0pt{\hskip-\hangindent\dbend\kern2pt\dbend\hfill}\ignorespaces}
%% \def\endddanger{\par\end{footnotesize}\medbreak}

%==============================================================================
% section commands
%\iffalse
\renewcommand\chapter{\clearpage%\if@openright\cleardoublepage\else\clearpage\fi
                    \thispagestyle{plain}%
                    \global\@topnum\z@
                    \@afterindentfalse
                    \secdef\@chapter\@schapter}
\def\@makechapterhead#1{%
  \vspace*{50\p@}%
  {\parindent \z@ \raggedright \normalfont
    \ifnum \c@secnumdepth >\m@ne
        \twelvess \@chapapp\space \thechapter
        \par\nobreak
        \vskip 20\p@
    \fi
    \interlinepenalty\@M
    \chapterfont\hfill #1\par\nobreak
    \vskip 40\p@
    \gdef\@chaptername{#1}\gdef\@currentSection{\thechapter}
  }}
\def\@makeschapterhead#1{%
  \vspace*{50\p@}%
  {\parindent \z@ \raggedright
    \normalfont
    \interlinepenalty\@M
    \chapterfont \hfill #1\par\nobreak
    \vskip 40\p@
  }}

\def\specialsection{\@startsection{section}{1}%
  \z@{\baselineskip\@plus\baselineskip}{.25\baselineskip}%
  {\sectionfont}}
\def\section{\@startsection{section}{1}%
  \z@{.7\baselineskip\@plus\baselineskip}{.125\baselineskip}%
  {\sectionfont}}
\def\subsection{\@startsection{subsection}{2}%
  \z@{.7\baselineskip\@plus\baselineskip}{.125\baselineskip}%
%  \z@{.5\linespacing\@plus.7\linespacing}{-.5em}%
  {\subsectionfont}}
\def\subsubsection{\@startsection{subsubsection}{3}%
  \z@{.7\baselineskip\@plus\baselineskip}{.125\baselineskip}%
%  \z@{.5\linespacing\@plus.7\linespacing}{-.5em}%
  {\subsectionfont}}
\begin{comment}
\renewenvironment{thebibliography}[1]
     {\pagebreak\section*{\refname}\addcontentsline{toc}{section}{\refname} %
      \@mkboth{\MakeUppercase\refname}{\MakeUppercase\refname}%
      \begin{footnotesize}\list{\@biblabel{\@arabic\c@enumiv}}%
           {\settowidth\labelwidth{\@biblabel{#1}}%
            \leftmargin\labelwidth
            \advance\leftmargin\labelsep
            \@openbib@code
            \usecounter{enumiv}%
            \let\p@enumiv\@empty
            \renewcommand\theenumiv{\@arabic\c@enumiv}}%
      \sloppy
      \clubpenalty4000
      \@clubpenalty \clubpenalty
      \widowpenalty4000%
      \sfcode`\.\@m}
     {\def\@noitemerr
       {\@latex@warning{Empty `thebibliography' environment}}%
      \endlist\end{footnotesize}}
\end{comment}

%%%%%%%%%%
% Exercise Macros
%%%%%%%%%%
\def\textindent#1{\indent \llap {#1\enspace }\ignorespaces}
\newcounter{exercise}
\def\exercise [#1]{\refstepcounter{exercise}\ifnum\value{exercise}>1 \smallbreak\fi
  \textindent{\textbf{\theexercise.}}[\textit{#1\/}]\kern6pt}
\def\suggestedExercise [#1]{\refstepcounter{exercise}\ifnum\value{exercise}>1 \smallbreak\fi
  \textindent{\llap{\manual x\hskip3pt}\bf{\hbox to
     \ifnum \value{exercise}>99 1.5em\else 1em\fi{\hfil\theexercise}}.}[\textit{#1\/}]\kern6pt}

\def\HM{H\kern-.1em M} % used for "higher math" exercise ratings
\def\MN{M\kern-.1em N} % used in Section 4.3.1 when $MN$ appears frequently

\newenvironment{exercises}{\medskip\phantomsection\addcontentsline{toc}{section}{Exercises}%
        \noindent\ignorespaces\section*{Exercises}%
        \parindent=0pt%
        \setcounter{exercise}{0}}{}

\newwrite\ans%
\immediate\openout\ans=tex/answers % file for answers to exercises
\def\ansSec#1{\immediate\write\ans{\detokenize{\section*}{#1}}}
\def\ansno#1:{\smallbreak\textindent{\textbf{#1.\enspace}}}
\newenvironment{answer}%
 {\par\medbreak
    \immediate\write\ans{}%
    \immediate\write\ans{\string\ansno\arabic{exercise}:}%
  \@bsphack
  \let\do\@makeother\dospecials\catcode`\^^M\active
  \def\verbatim@processline{%
    \immediate\write\ans{\the\verbatim@line}}%
  \verbatim@start}%
 {\@esphack} 

%% Sometimes there's an alternate way to solve the problem
%% TODO: Make this include an optional argument specifying how it
%%       was done, so one can use the following code snippet
%% \begin{altAns}[Using Complex Analysis]
%% ...
%% \end{altAns}
\newenvironment{altAns}%
 {\par\medbreak
    \immediate\write\ans{}%
    \immediate\write\ans{\string\ansno\theexercise (Alternate):}%
  \@bsphack
  \let\do\@makeother\dospecials\catcode`\^^M\active
  \def\verbatim@processline{%
    \immediate\write\ans{\the\verbatim@line}}%
  \verbatim@start}%
 {\@esphack} 


%%%%%%%%%%%%%%%%%%%%%%%%%%%%%%%%%%%%%%%%%%%%%%%%%%%%%%%%%%%%%%%%%%%%%%%%%%%%%%%%
%%%%%%%%%%%%%%%%%%%%%%%%%%%%%%%%%%%%%%%%%%%%%%%%%%%%%%%%%%%%%%%%%%%%%%%%%%%%%%%
% We also customize commands
%%%%%%%%%%%%%%%%%%%%%%%%%%%%%%%%%%%%%%%%%%%%%%%%%%%%%%%%%%%%%%%%%%%%%%%%%%%%%%%

\def\noparskip{}
\newcommand\Sym{\mathop{\rm Sym}\nolimits}
\newcommand\diag{\mathop{\rm diag}\nolimits}
\newcommand\cliff{C\ell}
\newcommand\rank{\mathop{\rm rank}\nolimits}
\newcommand\Span{\mathop{\rm span}\nolimits}


\newcommand{\re}{\mathop{\mathrm{Re}}\nolimits}
\newcommand{\im}{\mathop{\mathrm{Im}}\nolimits}
\newcommand{\tr}{\mathop{\mathrm{Tr}}\nolimits}
\newcommand{\aut}{\mathop{\mathrm{Aut}}\nolimits}
\renewcommand{\ker}{\mathop{\mathrm{Ker}}\nolimits}
\newcommand{\coker}{\mathop{\mathrm{Coker}}\nolimits}
\newcommand{\erfi}{\mathop{\mathrm{erfi}}\nolimits}
\newcommand{\erf}{\mathop{\mathrm{erf}}\nolimits}
\let\<\langle
\let\>\rangle
\let\iso\cong
\newcommand{\equalsdef}{\stackrel{\text{def}}{=}}
\let\eqdef\equalsdef
\newcommand{\define}[1] {``\/\textbf{#1}\/\/''}%\index{#1}}
\newcommand{\lie}{\mathop{\mathrm{Lie}}\nolimits}
\let\Lie\lie
\renewcommand{\arctan}{\mathop{\mathrm{ArcTan}}\nolimits}
\renewcommand{\hom}{\mathop{\mathrm{Hom}}\nolimits}
\newcommand\fun{\mathop{\mathrm{Fun}}\nolimits} % like \hom but for categories
\newcommand\nat{\mathop{\mathrm{Nat}}\nolimits} % like \hom but for functors
\let\Hom\hom
\def\noparskip{}
\newcommand{\id}[1]{\mathop{\mathrm{id}}\nolimits_{#1}}
\newcommand{\ob}[1]{\mathrm{Ob}(#1)}
\newcommand{\cat}[1]{\mathbf{#1}} % notation for a category, e.g. $\cat{Set}$
\newcommand\powerset[1]{\mathcal{P}(#1)}
\newcommand{\sgn}{\mathop{\mathrm{sgn}}\nolimits}
\newcommand{\mat}{\mathop{\mathrm{Mat}}\nolimits}
\newcommand{\der}{\mathop{\mathrm{Der}}\nolimits}
\newcommand{\Ad}{\mathop{\mathrm{Ad}}\nolimits}
\newcommand{\ad}{\mathop{\mathrm{ad}}\nolimits}
\newcommand\op{\mathrm{op}}
%% Frequently used sets
\let\logic\texttt
%\def\logic#1{\relax\ifmmode\mathtt{#1}\else\texttt{#1}\fi}
\newcommand\CC{\mathbb{C}}
\newcommand\CP{\mathbb{CP}}
\newcommand\EE{\mathbb{E}}
\newcommand\FF{\mathbb{F}}
\newcommand\HH{\mathbb{H}}
\newcommand\HP{\mathbb{HP}}
\newcommand\NN{\mathbb{N}}
\newcommand\OO{\mathbb{O}}
\newcommand\OP{\mathbb{OP}}
\newcommand\QQ{\mathbb{Q}}
\newcommand\RR{\mathbb{R}}
\newcommand\RP{\mathbb{RP}}
\newcommand\ZZ{\mathbb{Z}}
\let\oldsetminus\setminus
\renewcommand\setminus{-}
\newcommand\universe{\mathcal{U}}
\newcommand\domain{\mathop{\mathrm{dom}}\nolimits}
\newcommand\codomain{\mathop{\mathrm{cod}}\nolimits}
\let\propersubset\subset
\let\propersupset\supset
\let\oldsubset\subset
\let\oldsupset\supset
\let\subset\subseteq
\let\supset\supseteq
\let\into\hookrightarrow
\let\onto\twoheadrightarrow
\let\xonto\xtwoheadrightarrow
\newcommand\1{\mathbf{1}}
\newcommand\cardinality[1]{\left|#1\right|}
\newcommand\continuumCardinality{\mathfrak{c}}
%% Lie groups frequently used
\newcommand\GL[2][\relax]{\ifx#1\relax\mathrm{GL}(#2)\else\mathrm{GL}_{#1}(#2)\fi}
\newcommand\SL[2][\relax]{\ifx#1\relax\mathrm{SL}(#2)\else\mathrm{SL}_{#1}(#2)\fi}
\newcommand\SO[2][\relax]{\ifx#1\relax\mathrm{SO}(#2)\else\mathrm{SO}_{#1}(#2)\fi}
\newcommand\ORTH[2][\relax]{\ifx#1\relax\mathrm{O}(#2)\else\mathrm{O}_{#1}(#2)\fi}
\newcommand\U[2][\relax]{\ifx#1\relax\mathrm{U}(#2)\else\mathrm{U}_{#1}(#2)\fi}
\newcommand\SU[2][\relax]{\ifx#1\relax\mathrm{SU}(#2)\else\mathrm{SU}_{#1}(#2)\fi}
\newcommand\Sp[2][\relax]{\ifx#1\relax\mathrm{Sp}(#2)\else\mathrm{Sp}_{#1}(#2)\fi}
\newcommand\PGL[2][\relax]{\ifx#1\relax\mathrm{PGL}(#2)\else\mathrm{PGL}_{#1}(#2)\fi}
\newcommand\PSU[2][\relax]{\ifx#1\relax\mathrm{PSU}(#2)\else\mathrm{PSU}_{#1}(#2)\fi}

%% Differential Geometry refers to these quantities frequently as well
\def\diff{\mathrm{Diff}}
\def\vect{\mathrm{Vect}}

%% Categories frequently used
\newcommand\Set{\cat{Set}} % category of sets
\newcommand\FinSet{\cat{FinSet}} % category of finite sets
\newcommand\Cat{\cat{Cat}} % category of categories
\newcommand\Vect{\cat{Vect}} % category of vector spaces
\newcommand\Top{\cat{Top}} % category of topological spaces
\newcommand\Grp{\cat{Grp}} % category of groups
\newcommand\Gpd{\cat{Gpd}} % category of groupoids
\newcommand\Alg{\cat{Alg}} % category of algebras over a field
\newcommand\Magm{\cat{Magm}} % category of Magmas
\newcommand\Ord{\cat{Ord}} % category of ordinals
\newcommand\FinOrd{\cat{FinOrd}} % category of FINITE ordinals

\def\weak={\approx} % weak equality when working in classical gauge theory
\def\PB(#1,#2){\left\{#1,#2\right\}} % the Poisson bracket for field theory and mechanics

\newcommand\ClassicalGroup[1]{\mathsf{#1}}
\newcommand\Antisymmetric{{\textstyle\bigwedge\nolimits}}

%\let\oldemptyset\emptyset
\let\varemptyset\varnothing

% Notation used for algebraic topology
\let\homotopic\sim

% Notation for mathematical logic
\def\tee{\logic{true}}
\def\falsum{\logic{false}}
\let\entails\implies
\def\lst{\mbox{\texttt{:}}} % logical ``such that'' notation
\def\lst{\;\mid\;}
%
%  special signs and characters
%\newcommand{\D}{\mathrm{d}}
%\newcommand{\E}{\mathrm{e}}
%\let\eul=\E
%\newcommand{\I}{{\rm i}}
%\let\imag=\I


\newcounter{boxed}[section]
\newenvironment{Boxed}[1]{\refstepcounter{boxed}\MakeFramed {\advance\hsize 2pc\FrameRestore}\noindent{\subsectionfont Box \theboxed. #1}\vskip .125\baselineskip}{\endMakeFramed}
\def\arXiv#1{\href{http://arxiv.org/abs/#1}{\texttt{arXiv:#1}}}


\def\bigO{\mathcal{O}}
\newcommand\units[1]{\ensuremath{\, \mathrm{#1}}}

\newcommand\identify{\noindent\llap{\rm1.\kern.5em}\textbf{Identify:}}
\newcommand\setup{\noindent\llap{\rm2.\kern.5em}\textbf{Set up:}}
\newcommand\execute{\noindent\llap{\rm3.\kern.5em}\textbf{Execute:}}
\newcommand\evaluate{\noindent\llap{\rm4.\kern.5em}\textbf{Evaluate:}}
\makeatother

