%%
%% measureTheory.tex
%% 
%% Made by Alex Nelson
%% Login   <alex@tomato>
%% 
%% Started on  Fri Aug  7 11:54:31 2009 Alex Nelson
%% Last update Fri Aug  7 11:54:31 2009 Alex Nelson
%%

\begin{defn}%\label{defn:}
Let $X$ be a set. A \define{$\sigma$-Algebra} over $X$ consists of
\begin{itemize}
\item a collection $\Sigma$ of subsets of $X$
\end{itemize}
such that
\begin{itemize}
\item $\Sigma$ is nonempty,
\item if $x\in\Sigma$, the $x^{C}\in\Sigma$,
\item let $I$ be a finite indexing set, then
$$\left(\bigcup_{i \in I} E_i\right) \in\Sigma$$
 for countably many $E_i\in\Sigma$.
\end{itemize}
\end{defn}

It allows us to introduce the notion of a measure:

\begin{defn}%\label{defn:}
Let $X$ be some set, $\Sigma$ be a $\sigma$-algebra over $X$. A
\define{measure} $\mu$ consists of
\begin{itemize}
\item a function $\mu:\Sigma\to[-\infty,\infty]$
\end{itemize}
such that
\begin{itemize}
\item $\mu(E)\geq0$ for all $E\in\Sigma$;
\item $\mu(\emptyset)=0$;
\item if $\{E_{i}\}_{i\in{I}}$ is a countable collection of
  pairwise disjoint sets in $\Sigma,$ then
$$ \mu\left(\bigcup_{i \in I} E_i\right) = \sum_{i \in I} \mu(E_i).$$
\end{itemize}
\end{defn}
