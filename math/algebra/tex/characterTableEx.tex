%%
%% characterTableEx.tex
%% 
%% Made by alex
%% Login   <alex@ubuntu>
%% 
%% Started on  Fri Jan 30 12:21:55 2009 alex
%% Last update Fri Jan 30 15:07:59 2009 alex
%%

\textbf{NOTE!} We will be working with irreps over a finite
dimensional vector space $V$ over the field $\mathbb{C}$ of
complex numbers.

%% \subsection{The Character Table for $S_{3}$}

%% Consider the symmetric group $S_3$. We have 
%% \begin{equation}

%% \end{equation}

\subsection{The Character Table for $D_3$}

Consider the group $D_3$ the symmetries of a regular
triangle. It has 6 elements, and the presentation
\begin{equation}
\<x,y|x^3=y^2=1,yx=x^{-1}y\>.
\end{equation}
It has, as mentioned, 6 elements
\begin{equation}
D_{3} = \{1,x,x^2,y,yx,yx^2\}.
\end{equation}
It has the multiplication table
\begin{equation}
\begin{array}{c|cccccc}
\times & 1   & x  & x^2 & y & yx & yx^2\\
\hline
1      & 1   & x  & x^2 & y & yx & yx^2\\
x^2    & x^2  & 1 & x   & yx & yx^2 & y\\ 
x      & x   & x^2 & 1 & yx^2 & y & yx\\
y      & y   & yx & yx^2 & 1 & x & x^2\\
yx     & yx  & yx^2 & y & x^2 & 1 & x\\
yx^2   & yx^2 & y & yx & x & x^2 & 1
\end{array}
\end{equation}
We find the cosets of the group
\begin{subequations}
\begin{align}
C(x) &= \{x^{k}xx^{-k}=x,
yxy=x^{-1},yx^{k}xyx^{k}=x^{-1}\}\\
&=\{x,x^2\}\\
C(y) &= \{x^kyx^{-k}=yx^{-2k}, yyy=y,
yx^{k}yyx^{k}=x^{-k}yx^{k}=yx^{2k}\}\\
&=\{y,yx,yx^2\}
\end{align}
\end{subequations}
Observe that $(x^2)^2=x^4=x$ in our group. Refer to the
multiplication table if in doubt. We can set up the
character table. We expect there to be some number of
irreducible representations $k$ such that the sum of the
squares of the dimensions of the irrep $\rho_k$ is the size
of the group. That is to say, 
\begin{equation}
d_{1}^{2} + d_{2}^{2} + \cdots + d_{k}^{2} = 6
\end{equation}
How many ways are there to do this? Well, observe $1^2=1<6$,
so $d_i$ can be at least 1. Further, $2^2=4<6$, so 2 is a
possible dimension. But $3^2=9>6$, so we can only have 1 or
2 dimensional irreps. How many different combinations are
there to write this? Trivially, two different ways
\begin{equation}
1^2+1^2+1^2+1^2+1^2+1^2=6,\quad\text{or }1^2+1^2+2^2=6.
\end{equation}
Those are the only two ways to write it! Further, we expect
the character table to be a square matrix, we have three
cosets: $C(1)$ (or as I denote it $C(e)$), $C(x)$, and
$C(y)$. That means we expect there to be three terms in our
sum of squares, and the only formula with three terms is
\begin{equation}
1^2+1^2+2^2=1+1+4=6.
\end{equation} 
So we have a good idea of how many irreps there are for
$D_3$. Let us now construct the character table:
\begin{equation}
\begin{array}{c|cccc}
\text{size of coset} & (1) & (2) & (3) \\
\text{coset rep} & 1 & x & y\\ \hline
\chi_1           &   &   &  \\ 
\chi_2           &   &   &  \\
\chi_3           &   &   & 
\end{array}
\end{equation}
Observe that we have the number of elements, which is
\emph{not standard} in character tables. I just like to add
them in as a reminder for quick reference. Now the first
column is the character of the irrep $\rho_i$. The first
irrep $\rho_1$ is always the trivial representation, it maps
everything to the identity in 1 dimension.
\begin{equation}
\begin{array}{c|cccc}
\text{size of coset} & (1) & (2) & (3) \\
\text{coset rep} & 1 & x & y\\ \hline
\chi_1           & 1 & 1 & 1\\ 
\chi_2           &   &   &  \\
\chi_3           &   &   & 
\end{array}
\end{equation}
Now we should remember that the character of the identity is
always the number of dimensions of the irrep. So we can fill
in the second column.
\begin{equation}
\begin{array}{c|cccc}
\text{size of coset} & (1) & (2) & (3) \\
\text{coset rep} & 1 & x & y\\ \hline
\chi_1           & 1 & 1 & 1\\ 
\chi_2           & 1 &   &  \\
\chi_3           & 2 &   & 
\end{array}
\end{equation}
We also have orthogonality of the rows (in a weighted inner
product), and orthonormality of the columns. So we can
\textbf{guess} the second row will be something of the form
\begin{equation}
\begin{array}{c|cccc}
\text{size of coset} & (1) & (2) & (3) \\
\text{coset rep} & 1 & x & y\\ \hline
\chi_1           & 1 & 1 & 1\\ 
\chi_2           & 1 & a & b\\
\chi_3           & 2 &   & 
\end{array}
\end{equation}
where $a,b\in\mathbb{C}$ are to be determined. We know that
the inner product between two characters is
\begin{equation}
\<\chi_i,\chi_j\> =
\frac{1}{n}\sum_{g}\overline{\chi_{j}(g)}\chi_{i}(g) = \delta_{ij}
\end{equation}
where $g$ is a coset representative and $\delta_{ij}$ is the
Kronecker delta, $\chi_i$ is the character for the irrep $\rho_i$. So we have
\begin{subequations}
\begin{align}
\<\chi_1,\chi_2\> = 1(1\cdot1) + 2(1\cdot\bar{a})+3(1\cdot\bar{b}) &= 0\\
\<\chi_2,\chi_1\> = 1(1)+2(a)+3(b) &= 0\\
\<\chi_2,\chi_2\> = 1(1\cdot1) + 2(a\cdot\bar{a})+3(b\cdot\bar{b}) &= 6.
\end{align}
\end{subequations}
So from the first two of these equations, we find $a,b$ are
real. We know that characters of irreps of finite groups are
integer combinations of roots of unity, so that means that
$a,b\in\mathbb{Z}$. From the condition that
\begin{equation}
1+2a+3b=0
\end{equation}
we find that
\begin{equation}
a=1,\;\; b=-1.
\end{equation}
So, we plug these into our character table
\begin{equation}
\begin{array}{c|cccc}
\text{size of coset} & (1) & (2) & (3) \\
\text{coset rep} & 1 & x & y\\ \hline
\chi_1           & 1 & 1 & 1\\ 
\chi_2           & 1 & 1 & -1\\
\chi_3           & 2 &   & 
\end{array}
\end{equation}
The orthonormality of the columns then demand that the rest
of the character table is trivially
\begin{equation}
\begin{array}{c|cccc}
\text{size of coset} & (1) & (2) & (3) \\
\text{coset rep} & 1 & x & y\\ \hline
\chi_1           & 1 & 1 & 1\\ 
\chi_2           & 1 & 1 & -1\\
\chi_3           & 2 & -1  & 0
\end{array}
\end{equation}
How do we see this? Take the first two columns and take
their inner product
\begin{equation}
1\cdot1+1\cdot1+2\cdot(-1)=0
\end{equation}
and similarly for the first and third column
\begin{equation}
1\cdot1-1\cdot1+0\cdot2=1-1=0.
\end{equation}
The second and third column are trivially orthogonal.

This is then the full character table for $D_3$. 

%% Let us try to construct the two dimensional irrep for
%% $D_3$. It is sufficient to do so for the coset
%% representatives I think. So we have $\rho_{3}(e)=I_{2}$ be
%% the two by two identity matrix, and for $\rho_{3}(x)$,
%% $\rho_{3}(y)$ -- since we do not know what they are! -- we
%% set up the matrices with unknown quantities we are solving
%% for. We know that $\chi_{3}(x)=-1$ and $\chi_{3}(y)=0$, so
%% that tells us that $\rho_{3}(x)$ has a trace of -1 and
%% $\rho_{3}(y)$ has a trace of 0 (it's traceless!). We set up
%% the matrices 
%% \begin{equation}
%% x = \begin{bmatrix}a & b\\ c & -(1+a)\end{bmatrix},\quad
%% y = \begin{bmatrix}\alpha & \beta\\ \gamma & -\alpha\end{bmatrix}.
%% \end{equation}
%% We can calculate the other elements of the group in our
%% representation by matrix multiplication
%% \begin{equation}
%% x^2 = \begin{bmatrix}a^2+bc & -b\\ -c & bc+(1+a)^2\end{bmatrix}
%% \end{equation}
%% so we find
%% \begin{equation}
%% yx = \begin{bmatrix} a\alpha+c\beta & b\alpha - (1+a)\beta\\
%% a\gamma-c\alpha & b\gamma+(1+a)\alpha\end{bmatrix}
%% \end{equation}
%% and
%% \begin{equation}
%% yx^2=xy=\begin{bmatrix}a\alpha+b\gamma & a\beta-b\alpha\\
%% c\alpha-(1+a)\gamma & c\beta+(1+a)\gamma\end{bmatrix}.
%% \end{equation}
%% From our character table, we demand that $yx$ and $yx^2$ are
%% traceless, and $x^2$ has a trace of -1. We have one more
%% condition worth mentioning, namely that $\rho_{3}(y)^2=I$
%% and $\rho_{3}(x)^3=I$, or in equation form
%% \begin{equation}
%% y^2 = \begin{bmatrix}\alpha^2+\beta\gamma & 0\\0 &
%%   \alpha^2+\beta\gamma\end{bmatrix} = I
%% \end{equation}
%% Similarly, an equation is derived from $x^3=I$ but we will
%% simplify our expression for $x$ first.

%% So we have a system of equations we need to solve
%% \begin{subequations}
%% \begin{align}
%% \tr(x^2) = 1+2a+2a^2+2bc &= -1\\
%% \tr(yx) = (1+2a)\alpha+(\beta+\gamma)c &= 0\\
%% \tr(yx^2) = (1+2a)\alpha+b\gamma+c\beta &= 0\\
%% (y^2=I) \;\; \alpha^2+\beta\gamma &= 1
%% \end{align}
%% \end{subequations} 
%% We can subtract the second from the third to find
%% \begin{equation}
%% b\gamma-c\gamma=0\Rightarrow b=c
%% \end{equation}
%% Plug this into the first equation we find
%% \begin{equation}
%% 1+a+a^2+b^2=0\Rightarrow b^2=a-(1+a)^2
%% \end{equation}
%% Things are getting slightly better, but now we can write the
%% matrix  representation of $x$ in a more convenient form
%% \begin{equation}
%% x = \begin{bmatrix}a & b\\b & -(1+a)\end{bmatrix}
%% \end{equation}
%% so
%% \begin{equation}
%% x^3 = \begin{bmatrix} a(a^2+b^2)-b^3 & -ab+b^3+b(1+a)^2\\
%% b(a^2+b^2)+(1+a)b & -b^2-(1+a)^3-b^{2}(1+a)\end{bmatrix} = I.
%% \end{equation}
%% This gives us an additional 4 equations.

