% Time-stamp: "2022-11-20T18:53:23-08 Alex"
\documentclass{report}
\usepackage{danger}
\usepackage{macros}
\usepackage{makeidx}
\usepackage{deriv}
\usepackage{notation}

% Think about: https://tex.stackexchange.com/questions/627357/how-to-get-an-index-entry-pointing-to-the-exact-location-rather-than-the-page
\makeindex

\pdfinfo{/CreationDate (D:20221120185323)}
\title{Quantum Field Theory}
\date{November 20, 2022}
\begin{document}

\maketitle

\begin{abstract}
This is a summary of the conceptual aspects of quantum field theory. A
lot of calculations will be omitted or sketched out. Many calculations
may be found explicitly in Hatfield~\cite{Hatfield:1992rz}. We instead
focus on the usefulness of Lie algebras and Lie groups in quantum field
theory.

Since most of the \emph{conceptual} aspects may be discussed in the
classical setting, we spend a lot of time reviewing classical field
theory and analogous phenomena at the classical level.
\end{abstract}

\tableofcontents

% When I get to Spinor fields,
% - Poisson bracket conventions https://physics.stackexchange.com/a/186988/9290

%%
%% preface.tex
%% 
%% Made by alex
%% Login   <alex@tomato>
%% 
%% Started on  Thu Nov  3 14:45:53 2011 alex
%% Last update Wed May 30 10:11:45 2012 Alex Nelson
%%
%\phantomsection\addcontentsline{toc}{chapter}{Preface} 
%\chapter*{Preface}
This manuscript is intended to teach calculus using big O
notation. Originally, Knuth suggested it in a letter~\cite{knuth}. It seems
to explain the intuitive idea, which is my aim. This is done in a
slightly less-than-orthodox manner, thinking ``microscopically''
(``infinitesimally'') and ``macroscopically''. I found this
really helps explain the concepts of differentiation and
integration.

As this is a calculus text, all the concerns are purely
\emph{symbolic}. We do not care if something is well-defined or
not, everything we do is symbolic manipulation. When we stop,
think, and wonder if it makes sense, then we begin the field of
real analysis. 

Depending on how far things get, this text may be focused on
calculus of a single variable only or more. If we got farther, we
need to discuss linear algebra (even a provisional presentation
of it will be necessary). 

The focus varies, but consistently we are application
oriented. The ``applications'' I am speaking of are computational
in nature (e.g., how can we compute $\sin(1)$ to 7 digits of
precision by hand?). This is useful when you forget your
calculator, and wait for your plane at the airport.

But really, this is a stepping stone towards other
applications. Texts which will follow from this are my notes on
classical physics; my notes on complex analysis; and so on.

\bigskip
A note when reading. There will be ``problems'' which come up in
the text. They are sometimes exercises, other times they are
motivating questions to guide the text. The reader should spend a
moment thinking about them before moving on.

Exercises appear in the ``exercises'' sections. They are numbered
to indicate their difficulty on a logarithmic scale from 0
(easiest) to 50 (open research problems). For a rough
approximation of difficulty and explanation, consider the
following table:

\medskip
\begin{tabular}{c | p{10cm}}
\toprule
Scale & Meaning \\
\midrule 
00 & ``Warmups.'' Every reader should try to do these when reading.\\
10 & ``Basics.'' Exercises to develop facts that should be done
from one's own derivations rather than glancing at someone else's.\\
20 & ``Homework Exercises.'' Intended to deepen understanding of
the material covered in the current section or chapter. \\
30 & ``Exam Problems.'' These typically involve multiple sections
of the book to gain a better insight into mathematics. \\
40 & ``Bonus Problems.'' These problems extend the text in
interesting ways. \\
50 & ``Open Research Problems.'' These are open problems that may
or may not have a solution. \\
\bottomrule
\end{tabular}

\medskip
\noindent{}It should be noted that sometimes it is necessary to just do
calculations in calculus. A python program was written to
generate exercises, just for the sake of practicing calculations,
and will be sectioned off into a subsection ``Calculator Exercises.''
Here ``calculator'' refers to the historical connotation of a
person who is skilled at calculations, not the electronic
pocket-sized gadget.

\medskip
As far as the level of rigor is concerned, as stated, everything
is purely symbolic manipulation. Proofs are given, but not always
within the ``\emph{Proof}. \dots \slug'' pattern mathematicians
have come to love.

However, we are following Euler's example in applying the
``Generality of Algebra'' to our thinking. That is to say, we are
not being rigorous in the modern sense of the word because we are
following symbolic rules that work in particular cases. We just
assume that the rules work in \emph{any} situation.

Although we are following Euler, our presentation of material
differs slightly. We will stick to the outline in most modern
calculus textbooks, namely: derivatives first and integrals
second, single variables before multiple. We will also use
convergence of series and sequences as a segue from single to
many variable calculus.

\vfill
\begin{quotes}
The heart of mathematics consists of 
concrete examples and concrete problems.
\Author P.\ R.\ Halmos, {\sl How to write mathematics} (1973)

\bigskip
It is downright sinful to teach 
the abstract before the concrete.
\Author Z.\ A.\ Melzak, {\sl Companion to Concrete Mathematics} (1973)

\bigskip
The advanced reader who skips parts that appear 
too elementary may miss more than the less advanced reader 
who skips parts that appear too complex.
\Author G.\ Polya, {\sl Induction and Analogy in Mathematics} (1954)

\bigskip
The material of concrete mathematics may seem at first to be a disparate
bag of tricks, but practice makes it into a disciplined set of tools.
\Author D.\ E.\ Knuth, {\sl Concrete Mathematics} (1994)
\end{quotes}

\part{Mechanics}
\chapter{Quantum Mechanics}

\M We will review the pertinent aspects of \emph{nonrelativistic} quantum
mechanics. This is pedagogical, in the sense we start with ``simple but wrong''
rules, and later on revise the rules as needed.

\section{Basic Rules}\label{sec:qm:basic-rules}

\M We will review the basic rules for describing a quantum system using
a wave function. We will revise the rules later, for greater generality,
but we begin with relevant versions for physical systems.

\N{Position--Momentum Commutator Relation}
We have
\begin{equation}\label{eq:position-momentum-uncertainty-commutator}
[\widehat{x},\widehat{p}] = \I\hbar.
\end{equation}


\subsection{Wave Functions}

\M Following Isham~\cite{Isham:1995lq}, we summarize the rules for
interpreting wave functions.

\N{First rule}
The quantum state of a point particle moving in one-dimension is
represented by a complex-valued wave function $\psi(x)$ which is
normalized to one:
\begin{equation}
\int^{\infty}_{-\infty}|\psi(x)|^{2}\,\D x = 1.
\end{equation}
We will say that ``$\psi$ is normalized to unity'' when this condition holds.

\N{Superposition principle}
We can superimpose any two wave functions $\psi_{1}(x)$ and
$\psi_{2}(x)$ with arbitrary complex coordinates $\alpha_{1}$, $\alpha_{2}$.
That is to say, we get a new wave function
$\alpha_{1}\psi_{1}(x) + \alpha_{2}\psi_{2}(x)$, provided that
$\alpha_{1}$ and $\alpha_{2}$ are chosen to make the new wave function
normalized to unity.

\N{Second rule}
Any physical quantity which we can measure (i.e., any observable) is
represented by a linear differential operator that acts on wave functions
and is self-adjoint.

\begin{remark}[Eigenvalues of self-adjoint operators]
``Self-adjoint'' operators have real eigenvalues. We measure
eigenvalues. If we are measuring a complex-valued quantity, then we
need an anti-self-adjoint operator.
\end{remark}

\begin{remark}[Completeness of eigenfunctions for self-adjoint operators]
The eigenfunctions for a self-adjoint operator are ``complete'' in the
sense that they form a basis. Further, normalized eigenfunctions satisfy
the orthogonality condition
\begin{equation}
\int^{\infty}_{-\infty}f_{m}^{*}(x)f_{n}(x)\,\D x = \delta_{m,n}
\end{equation}
where $\delta_{m,n}=1$ if $m=n$ and $\delta_{m,n}=0$ otherwise.
\end{remark}

\N{Third rule}
The only possible result of measuring an observable $A$ is one of the
eigenvalues of the self-adjoint operator $\widehat{A}$ which represents
it.

\begin{remark}
This rule implies the long-term average value of the results of repeated
measurements of an observable $A$ is
\begin{equation}
\langle A\rangle_{\psi} = \langle\psi\mid A\mid\psi\rangle = \int^{\infty}_{-\infty}\psi^{*}(x)(\widehat{A}\psi)(x)\,\D x.
\end{equation}
\end{remark}

\M
For nondegenerate operators (i.e., any two eigenfunctions of
$\widehat{A}$ with the same eigenvalue are proportional to each other)
and with a discrete set of eigenvalues $a_{1}$, $a_{2}$, \dots
corresponding to eigenfunctions $f_{1}$, $f_{2}$, \dots; if the state is
$\psi(x)$, then the probability that a measurement of $A$ will yield a
particular eigenvalue $a_{n}$ is
\begin{equation}
\Pr(A = a_{n}; \psi) = |\psi_{n}|^{2},
\end{equation}
assuming the wave function is written as a linear combination of the
eigenfunctions,
\begin{equation}
\psi(x) = \sum_{n}\psi_{n}f_{n}(x)
\end{equation}
and the $\psi_{n}\in\CC$ are constants. We also assume $\psi(x)$ is
normalized to unity.

\begin{remark}
As a consequence of this, a wave function being normalized to unity is
the same as the ``sum of all probabilities equal 1''.
\end{remark}

\N{Fourth Rule}
The state function evolved in time according to the time-dependent
Schr\"{o}dinger equation
\begin{equation}
\I\hbar\frac{\partial\psi(x,t)}{\partial t} = \widehat{H}\psi(x,t),
\end{equation}
where the Hamiltonian operator $\widehat{H}$ is obtained from the
classical enerygy expression
\begin{equation}
H = \frac{p^{2}}{2m} + V(x),
\end{equation}
replacing the momentum $p$ and position $x$ by their corresponding
operators $\widehat{p}$, $\widehat{x}$.

\N{Incompleteness of Rules}
As we have presented things, these rules are incomplete since they give
no information on how to construct the operator which represents any
specific observable for a given physical system. In practice, this is
done by the ``substitution rule'' putting hats on things: for any
classical observable $A(x, p)$ the analogous operator is
$A(\widehat{x},\widehat{p})$. 

\begin{danger}
The astute reader will note we never defined ``observable'',
``measurement'', ``physical system'', ``property'', ``state'',
``causality'', or ``determinism''. We also didn't define ``physical quantity''.
This ambiguity is seldom addressed directly, instead most texts shift
the meaning of these words as needed. What is an ``object''? What is a
``property''? Or a ``physical quantity''? Is the significance of these
terms really so obvious? We're confronted by a deceptive question: ``What
is a `thing'?''
\end{danger}

\section{Quantum Harmonic Oscillator}

\N{Problem} Given the simple Harmonic oscillator's Hamiltonian
\begin{equation}\label{eq:quantum-harmonic-oscillator:defn:hamiltonian}
\widehat{H} = \frac{1}{2m}\left(\widehat{p}^{2} + (m\omega \widehat{x})^{2}\right),
\end{equation}
determine the eigenstates for the Schrodinger equation.

\M The trick is to use the identity
\begin{equation}
a^{2} + b^{2} = (a + \I b)(a -\I b)
\end{equation}
to introduce the operators:
\begin{equation}\label{eq:quantum-harmonic-oscillator:defn:ladder-operators}
a_{\pm} := \frac{1}{\sqrt{2\hbar m\omega}}\left(m\omega \widehat{x} \mp\I\widehat{p}\right).
\end{equation}
We will use this to rewrite the Hamiltonian operator.

\begin{exercise}
Is $a_{+}$ a self-adjoint operator? Is $a_{-}$ a self-adjoint operator?
\end{exercise}

\begin{lemma}\label{lemma:quantum-harmonic-oscillator:ladder-operator-hamiltonian-relation}
We find $a_{+}a_{-} = (\widehat{H}/\hbar\omega) - 1/2$.
\end{lemma}
\begin{proof}[Proof (lengthy calculation)]
  By direct computation
  \begin{calculation}
    a_{+}a_{-}
\step{by Eq~\eqref{eq:quantum-harmonic-oscillator:defn:ladder-operators}}
    \frac{1}{2\hbar m\omega}\left(m\omega \widehat{x}-\I\widehat{p}\right)\left(m\omega \widehat{x}+\I\widehat{p}\right)
\step{distributivity}
    \frac{1}{2\hbar m\omega}\left((m\omega \widehat{x})^{2}-\I\widehat{p}m\omega \widehat{x}+m\omega \widehat{x}\I\widehat{p}-\I^{2}\widehat{p}^{2}\right)
\step{linearity of operators}
    \frac{1}{2\hbar m\omega}\left((m\omega \widehat{x})^{2}-\I m\omega \widehat{p}\widehat{x}+\I m\omega \widehat{x}\widehat{p}+\widehat{p}^{2}\right)
\step{explicitly insert a commutator}
    \frac{1}{2\hbar m\omega}\left((m\omega \widehat{x})^{2}+\I m\omega[\widehat{x},\widehat{p}]+\widehat{p}^{2}\right)
\step{commutator relation Eq~\eqref{eq:position-momentum-uncertainty-commutator}}
    \frac{1}{2\hbar m\omega}\left((m\omega \widehat{x})^{2}+\I m\omega (\I\hbar)+\widehat{p}^{2}\right)
\step{associativity of addition}
    \frac{1}{2\hbar m\omega}\left((m\omega \widehat{x})^{2}+\widehat{p}^{2}\right)
+\frac{\I m\omega (\I\hbar)}{2\hbar m\omega}
\step{folding the definition of the Hamiltonian Eq~\eqref{eq:quantum-harmonic-oscillator:defn:hamiltonian}}
    \frac{1}{\hbar\omega}\widehat{H} +\frac{\I m\omega (\I\hbar)}{2\hbar m\omega}
\step{algebra}
    \frac{1}{\hbar\omega}\widehat{H} -\frac{1}{2}\qedhere
  \end{calculation}
\end{proof}

\begin{lemma}\label{lemma:quantum-harmonic-oscillator:ladder-operator-hamiltonian-relation2}
We find $a_{-}a_{+} = (\widehat{H}/\hbar\omega) + 1/2$.
\end{lemma}

\begin{proof}
Our calculation is exactly the same, except the sign of the commutator
$[\widehat{x},\widehat{p}]$ changes in our consideration, which is the
source of the $1/2$ contribution.
\end{proof}

\N{Commutator of Ladder Operators}
We have $[a_{-},a_{+}]=1$.

\begin{proof}
  We have, by our previous two lemmas,
  \begin{equation}
a_{-}a_{+}-a_{+}a_{-} = \left(\frac{\widehat{H}}{\hbar\omega}+\frac{1}{2}\right)-\left(\frac{\widehat{H}}{\hbar\omega}-\frac{1}{2}\right).
  \end{equation}
Hence the result.
\end{proof}

\N{Ladder Operators Acting on Eigenstates}
Suppose $\psi(x)$ is an eigenfunction for the Hamiltonian operator, with
energy eigenvalue $E$. We claim $a_{+}\psi(x)$ is an eigenfunction with
energy eigenvalue $E+\hbar\omega$.

\begin{proof} By direct computation,
\begin{calculation}
  \widehat{H}(a_{+}\psi(x))
\step{using $\widehat{H} = \hbar\omega(a_{+}a_{-}+1/2)$}
  \hbar\omega\left(a_{+}a_{-}+\frac{1}{2}\right)(a_{+}\psi(x))
\step{move $a_{+}$ inside the parentheses using right distributivity}
  \hbar\omega\left(a_{+}a_{-}a_{+}+\frac{1}{2}a_{+}\right)(\psi(x))
\step{associativity}
  \hbar\omega\left(a_{+}(a_{-}a_{+})+\frac{1}{2}a_{+}\right)(\psi(x))
\step{using Lemma~\ref{lemma:quantum-harmonic-oscillator:ladder-operator-hamiltonian-relation2}}
  \hbar\omega\left(a_{+}\left(\frac{\widehat{H}}{\hbar\omega}+\frac{1}{2}\right)+\frac{1}{2}a_{+}\right)(\psi(x))
\step{algebra}
  (a_{+}\widehat{H}+a_{+}\hbar\omega)\psi(x)
\step{pull out a factor of $a_{+}$ using left distributivity}
  a_{+}(\widehat{H} + \hbar\omega)\psi(x)
\step{since $\psi$ is an energy eigenstate}
  a_{+}(E + \hbar\omega)\psi(x)
\step{commutativity of numbers and operators}
  (E + \hbar\omega)(a_{+}\psi(x))
\end{calculation}
Hence $a_{+}\psi(x)$ is an energy eigenstate with eigenvalue $E + \hbar\omega$.
\end{proof}

\M
Similarly, for any energy eigenstate $\psi(x)$ with eigenvalue $E$, we
have $a_{-}\psi(x)$ is another energy eigenstate with eigenvalue
$E-\hbar\omega$.

\begin{proof} By direct computation,
\begin{calculation}
  \widehat{H}(a_{-}\psi(x))
\step{using $\widehat{H} = \hbar\omega(a_{-}a_{+}-1/2)$}
  \hbar\omega\left(a_{-}a_{+}-\frac{1}{2}\right)(a_{-}\psi(x))
\step{move $a_{-}$ inside using right distributivity}
  \hbar\omega\left(a_{-}a_{+}a_{-}-\frac{1}{2}a_{-}\right)(\psi(x))
\step{associativity}
  \hbar\omega\left(a_{-}(a_{+}a_{-})-\frac{1}{2}a_{+}\right)(\psi(x))
\step{using Lemma~\ref{lemma:quantum-harmonic-oscillator:ladder-operator-hamiltonian-relation}}
  \hbar\omega\left(a_{-}\left(\frac{\widehat{H}}{\hbar\omega}-\frac{1}{2}\right)-\frac{1}{2}a_{-}\right)(\psi(x))
\step{algebra}
  (a_{-}\widehat{H}-a_{-}\hbar\omega)\psi(x)
\step{pull out $a_{-}$ using left distributivity}
  a_{-}(\widehat{H} - \hbar\omega)\psi(x)
\step{since $\psi$ is an energy eigenstate}
  a_{-}(E - \hbar\omega)\psi(x)
\step{commutativity of numbers and operators}
  (E - \hbar\omega)(a_{-}\psi(x))
\end{calculation}
Hence $a_{-}\psi(x)$ is an energy eigenstate with eigenvalue $E - \hbar\omega$.
\end{proof}

\N{Ground State}
We can't keep decreasing the energy eigenstate forever, there must be
some groundstate $\psi_{0}(x)$ such that
\begin{equation}
a_{-}\psi_{0}(x) =0.
\end{equation}
Solve this differential equation for $\psi_{0}(x)$.

\begin{proof}[Solution]
We recall the definition of the ladder operator $a_{-}$ from Eq~\eqref{eq:quantum-harmonic-oscillator:defn:ladder-operators}
\begin{equation}
\frac{1}{\sqrt{2\hbar m\omega}}\left(m\omega \widehat{x}+\I\widehat{p}\right)\psi_{0}(x)=0.
\end{equation}
Doing some algebra, this becomes
\begin{equation}
(m\omega\widehat{x} + \I\widehat{p})\psi_{0}(x) = 0,
\end{equation}
which in position-space becomes
\begin{equation}
\left(m\omega x + \I(-\I)\hbar\frac{\partial}{\partial x}\right)\psi_{0}(x)=0.
\end{equation}
This becomes
\begin{equation}
\psi'_{0}(x) = -\frac{m\omega}{\hbar}x\psi_{0}(x).
\end{equation}
Divide both sides by $\psi_{0}$, then integrating with respect to $x$
gives
\begin{equation}
\int\frac{\D\psi_{0}}{\psi_{0}} = -\frac{m\omega}{\hbar}\int x\,\D x.
\end{equation}
Hence
\begin{equation}
\ln(\psi_{0}(x)) = -\frac{m\omega}{\hbar}\frac{x^{2}}{2} + c_{0},
\end{equation}
where $c_{0}$ is some integration constant.
Then we find
\begin{equation}
\psi_{0}(x) = C\exp\left(-\frac{m\omega}{\hbar}\frac{x^{2}}{2}\right).
\end{equation}
We can determine $C$ by demanding $\psi_{0}(x)$ be normalizable. This
gives us
\begin{equation}
C^{2}\int\exp\left(-\frac{m\omega}{\hbar}x^{2}\right)\,\D x = 1.
\end{equation}
This is a Gaussian integral (\S\ref{cor:math:general-gaussian-integral-in-one-dim}), which means
\begin{equation}
C^{2}\sqrt{\frac{\pi\hbar}{m\omega}}=1\implies C=\left(\frac{m\omega}{\pi\hbar}\right)^{1/4}.
\end{equation}
Thus
\begin{equation}
\boxed{\psi_{0}(x) = \left(\frac{m\omega}{\pi\hbar}\right)^{1/4}\exp\left(-\frac{m\omega}{\hbar}\frac{x^{2}}{2}\right).}
\end{equation}
Up to phase factor (i.e., a factor of $\E^{\I c_{1}}$ for any
arbitrary constant $c_{1}\in\RR$).
\end{proof}

\N{Other Eigenstates}
We can apply the ladder operator $a_{+}$ finitely many times to get
other eigenstates $\psi_{n}(x)\sim(a_{+})^{n}\psi_{0}(x)$. More
precisely,
\begin{equation}
\psi_{n}(x) = C_{n}(a_{+})^{n}\psi_{0}(x)
\end{equation}
where $C_{n}$ is some normalization constant.

\section{Experimental Background}

\M Around the turn of the $20^{\text{th}}$ century, physics faced some
puzzling experimental results which formed the motivation for deriving
quantum mechanics. Let us review superficially what was known at the
time, then in each subsection we shall consider an experiment, why it
produced unexplainable results, and how physicists resolved things.

\N{Status of Light}\label{chunk:qm:experimental-background:status-of-light}
We should stress that at the time, physicists believed light propagated
as a wave. Thomas Young first tested this in 1800 with his famous
double-slit experiment, where light propagated through two slits and
these two light sources appeared to ``interfer'' with each other. This
has a description using geometric optics and waves, which undergraduates
learn routinely --- see, e.g., Young and Freedman's
\textit{University Physics}.

\N{Atomic Theory}\label{chunk:qm:experimental-background:atomic-theory}
Around the turn of the $20^{\text{th}}$ century, physicists began to
recognize that matter can be described as consisting of
atoms. Rutherford supervised Geiger and Marsden's 1909 experiment, then
later in 1911 explained that the experiment implied an atom consists of
a nucleus (which contains most of the mass and is positively charged)
and electrons which orbit the nucleus, just as planets orbit a star. We
stress that this interprets an electron as a particle.

Millikan's oil drop experiment (conducted in 1908, reported in 1913)
determined the electric charge for the elementary electric charge
$e\approx1.602\times10^{-19}~\mathrm{C}$.  Each electron has a charge of
$-e$, and the nucleus has a charge of $+ne$ where $n$ is the atomic
number.

\subsection{Black Body Radiation}

\begin{definition}
A \define{Black Body} is an idealized object which absorbs all electromagnetic
radiation which hits it.
\end{definition}

\begin{remark}
Just because a black body \emph{absorbs} all radiation which hits
\emph{does not mean} a black body cannot emit any radiation. A black
body \emph{can} and \emph{does} emit radiation.
\end{remark}

\begin{remark}
It turns out that ``thermal radiation'' is precisely ``electromagnetic radiation'',
so I may use the terms interchangeably. But at the end of the day,
they're just photons.
\end{remark}

\N{Problem Statement}
Suppose we make a hollow cube out of a black body. Then the problem we
want to answer: describe the distribution of the frequencies of the
[emitted] radiation inside the cube.

\N{Classical Expectations}
Inside the body, the radiation will constantly be absorbed and re-emitted,
eventually reaching thermal equilibrium.
At each frequency, the absorption and emission of radiation will be
perfectly balanced.

We can invoke the ``Equipartition Theorem'' (of classical statistical
mechanics) which states the energy in any given mode of electromagnetic
radiation should be exponentially distributed with an average value
equal to $k_{B}T$ where $T$ is the temperature in Kelvin and $k_{B}$ is
the Boltzmann constant.

\N{Ultraviolet Catastrophe}
The difficulty is that the average amount of energy is the same for
every mode\footnote{Physicists use the term ``mode'' and ``frequency''
interchangeably.}. So when we add up the energy for each mode (for which
there are infinitely many), we get an infinite amount of energy for the
radiation in our black body oven. This is referred to as the
\define{Ultraviolet Catastrophe}, since the infinity comes from the
ultraviolet [high-frequency] end of the spectrum.

This contradicts our observation of a finite amount of energy inside a
black body oven.

\N{Planck's Solution}
In 1900, Max Planck offered an alternative prediction for this problem. The
key step is that Planck postulated energy in the electromagnetic field
at a frequency $\omega$ is ``quantized'', meaning it comes in an integer
multiple of a certain basic unit $\hbar\omega$ where $\hbar$ is a
constant we now recognize as Planck's constant.\marginpar{{\footnotesize Planck's Constant $\hbar$}}

Planck postulated the energy is then exponentially distributed over
integer multiples of $\hbar\omega$. At low frequencies, this will
coincide with classical statistical mechanics. But at high frequencies
(where $\hbar\omega$ is comparable to $k_{B}T$), Planck's theory
predicts a rapid fall-off of the average energy.

\begin{exercise}
Let $c>0$ be real. Prove $\displaystyle\sum^{\infty}_{n=0}n\E^{-cn}=\frac{\E^{c}}{(1-\E^{-c})^{2}}$.
\end{exercise}

\begin{exercise}
In Planck's model, the energies for electromagnetic radiation in a
frequency $\omega$ (in units of $[\mbox{time}]^{-1}$) is
distributed randomly over all numbers $n\hbar\omega$ for $n=0,1,2,\dots$.
We postulate the probability of finding energy $n\hbar\omega$ is
\begin{equation}
\Pr(E = n\hbar\omega) = \frac{1}{Z}\E^{\beta n\hbar\omega}
\end{equation}
where $\beta = 1/(k_{B}T)$ and $Z$ is a normalization constant. Then the
expected value for energy is
\begin{equation}
\langle E\rangle = \sum^{\infty}_{n=0}(n\hbar\omega)\Pr(E=n\hbar\omega)= \frac{1}{Z}\sum^{\infty}_{n=0}(n\hbar\omega)\E^{-\beta n\hbar\omega}.
\end{equation}
Now you come in:
\begin{enumerate}
\item Find $Z$ by demanding $\sum_{n=0}^{\infty}\Pr(E = n\hbar\omega)=1$.
\item Prove $\displaystyle\langle E\rangle = \frac{\hbar\omega}{\E^{\beta\hbar\omega}-1}$.
\item Show $\langle E\rangle$ behaves like $1/\beta = k_{B}T$ for small $\omega$,
but $\langle E\rangle$ decays exponentially as $\omega\to\infty$.
\end{enumerate}
\end{exercise}

\subsection{Photoelectric Effect}

\N{Problem Statement}
Suppose we have a metal surface. We can observe as electromagnetic
radiation (``incident light'') hits the surface, electrons will be emitted from the surface.

Einstein found as we increase the \emph{intensity} of the incident
light, the \emph{number} of electrons increases but the \emph{energy} of
each electron does not change. This is bizarre for a number of reasons.

If electromagnetic radiation were waves, then low-frequency light at
high intensity  should ``build up'' the energy necessary to produce
electrons, but we do not observe this\dots which suggests
electromagnetic radiation (``light'') is not a wave but consists of
particles. More precisely, light comes in discrete energy packages which
Einstein called \define{Photons}.

\N{Einstein's Solution}
In 1905, Einstein proposed that the electromagnetic radiation travels in
discrete energy packets called photons. For a frequency $\omega$, the
packet has energy $\hbar\omega$. (At the time, Einstein did not realize
$\hbar$ was Planck's constant, and used some constant of proportionality.)

The highest kinetic energy $K_{\text{max}}$ for the electron removed
from their atomic bindings by the absorption of a photon with energy
$\hbar\omega$ is then
\begin{equation}
K_{\text{max}} = \hbar\omega - W,
\end{equation}
where $W$ is the minimum energy required to remove an electron from the
metal surface. The literature refers to $W$ as the ``work function'' of
the surface. If we write the work function as:
\begin{equation}
W = \hbar\omega_{0},
\end{equation}
then the kinetic energy upper bound is
\begin{equation}
K_{\text{max}} = \hbar(\omega - \omega_{0}).
\end{equation}
Obviously emission can only occur when $K_{\text{max}}$ is positive,
requiring $\omega>\omega_{0}$.

\N{Quantum?}
Observe this solution requires $\hbar$ (a quantum quantity) and
classical physics fails to predict observed phenomena. More explicitly
the wave description of light fails to predict what we observe,
suggesting light consists of discrete packets (``particles'').

\subsection{Double-Slit Experiment}

\N{Problem statement}
We have assumed (\S\ref{chunk:qm:experimental-background:status-of-light})
light is a wave, since that appears to be supported by Young's
Double-Slit experiment. But if Einstein is correct and light propagated
in discrete packets, then how can we explain the double-slit phenomenon?

\N{Solutions}
J.J.~Thomson\footnote{J.J.~Thomson, ``On the ionization of Gases by
Ultra- Violet Light and on the evidence as to the structure of light
afforded by its Electrical Effect''. \journal{Prof.Cam.Phil.Soc.}
\volume{14} (1907) 417--424.} sketched out an experimental test in 1907, 
suggesting the results observed in the double-slit experiment could be
explained by the photons somehow interacting with each other.
Geoffrey Taylor\footnote{G.I.~Taylor, ``Interference Fringes with Feeble
Light''. \journal{Prof.Cam.Phil.Soc.} \volume{15} (1909) 114--115.}
first performed a low-intensity double-slit experiment
in 1909 by reducing the level of incident light (to the point where the
experiment took roughly 2000 hours to form an interference pattern ---
or 83 days and 12 hours). Taylor observed interference occurring still
at such low intensities of light.

Taylor's experimental results are interpreted as interference remains
even when photons are widely separated from each other (which is
weird!), but the photons are not interferring \emph{with each other}.
Instead, as Dirac writes in his book \textit{Principles of Quantum Mechanics},
``Each photon interfers only with itself.''

\subsection{Hydrogen Spectrum}

\N{Experiment}
If we pass electricity through a tube of hydrogen gas, then light will
be emitted. We can pass that light through a prism and four visible
``bands'' of light will form (red, cyan, blue, violet). This indicates
light is not a continuous spectrum but consists of discrete
frequencies. From the perspective of photons, this suggests only certain
energy packets are emitted. But this is surprising from a classical
perspective where light is a wave, and has no classical explanation.

\N{(Phenomenological) Explanation}
The Hydrogen atom consists of one proton and one electron. As we pass
electricity through the Hydrogen gas, the orbital electron ``gets
excited'' and moves farther away from the nucleus. Eventually the
electron will return to lower states, emitting a photon in the
process. This emitted photon takes on certain discrete
frequencies. Johannes Rydberg concluded in 1888 based on empirical data
that the energies of the emitted photon satisfy
\begin{equation}
E_{n} = -\frac{R}{n^{2}}
\end{equation}
where $n\in\NN$ and $R$ is the Rydberg constant
\begin{equation}
R = \frac{m_{e}Q^{4}}{2\hbar^{2}}.
\end{equation}
Here $m_{e}$ is the mass of the electron, $Q=-e$ is the charge of the
electron.
Note that we should use the reduced mass $\mu=m_{e}m_{p}/(m_{e}+m_{p})$
where $m_{p}$ is the mass of the proton,
but since $m_{p}\gg m_{e}$ ($m_{p}\sim 2\times10^{3}m_{e}$), we find
$\mu\approx m_{e}$ (the error would be something like $10^{-3}m_{e}$).

The frequencies for the emitted photon are then of the form
\begin{equation}
\omega = \frac{1}{\hbar}(E_{n} - E_{m})
\end{equation}
for some $n>m$, which agrees with observation.

But this lacks theoretical basis, and is rather unsatisfactory.

\N{Bohr Model}
Niels Bohr was as unsatisfied as I feel, and sought a theoretical
explanation for the Hydrogen spectrum. We can recall uniform circular
motion in the plane is described by the trajectory
\begin{equation}
(x(t),y(t)) = (r\cos(\omega t), r\sin(\omega t)).
\end{equation}
Its acceleration is obtained by taking the second derivative with
respect to time,
\begin{equation}
\vec{a}(t) = (-\omega^{2}r\cos(\omega t), -\omega^{2}r\sin(\omega t)).
\end{equation}
Observe the magnitude of the acceleration vector is $\omega^{2}r$. If
the only force acting on the electron (considered as a point-particle in
uniform circular motion) is Coulomb's law describing electromagnetism,
\begin{equation}
F = \frac{Q^{2}}{r^{2}}
\end{equation}
(up to some proportionality constant depending on our system of units),
then Newton's second Law gives us the equation of motion for our electron:
\begin{equation}
m_{e}\omega^{2}r = \frac{Q^{2}}{r^{2}}.
\end{equation}
Now we find the frequency $\omega$ by simple algebra
\begin{equation}\label{eq:qm:experimental-background:bohr-omega}
\omega = \sqrt{\frac{Q^{2}}{m_{e}r^{3}}}.
\end{equation}
Happy?

Well, we can do a few more things. We know the magnitude of velocity
$|\vec{v}|=\omega r$, multiplying by mass gives us momentum
$p=m_{e}\omega r$. When we plug in
Eq~\eqref{eq:qm:experimental-background:bohr-omega},
\begin{equation}\label{eq:qm:experimental-background:bohr:momenta}
p = \sqrt{\frac{m_{e}Q^{2}}{r}}.
\end{equation}
We can find the angular momentum, since its magnitude $J=pr$, giving us
\begin{equation}
J = \sqrt{m_{e}rQ^{2}}.
\end{equation}
So far, we haven't introduced anything new.

\N{Quantization Condition}
Bohr now introduces a quantization condition, namely that angular
momentum is an integer multiple of $\hbar$:
\begin{equation}
J = n\hbar = \sqrt{m_{e}rQ^{2}}.
\end{equation}
Solving for $r$ gives us
\begin{equation}
r_{n} = \frac{n^{2}\hbar^{2}}{m_{e}Q^{2}}.
\end{equation}
If we compute the energy for the electron with this radius, then we
recover observed Hydrogen spectrum.

\begin{remark}
What Bohr actually did was a bit more complicated, but if we were more
faithful to Bohr, we would make the big picture rather opaque.
\end{remark}

\begin{exercise}
Recall kinetic energy is $K=\frac{1}{2}m_{e}\vec{v}\cdot\vec{v}$ and the
potential energy for the electron would be $V = -Q^{2}/r$. Compute the
total energy $E = K + V$ with $r_{n}$ and $v=\omega r_{n}$. [Hint: you
  should recover $E_{n}=-R/n^{2}$.]
\end{exercise}

\N{De Broglie Condition}
Louis de Broglie [pronounced ``Broy-Lee''] proposed in 1924 that we
should interpret the Bohr quantization condition of the angular momentum
as a condition on the wave. That is, we should expect
\begin{equation}
2\pi r = n\lambda_{B}
\end{equation}
where $\lambda_{B}$ is the de Broglie wavelength (in units of length), so
the angular momentum 
\begin{equation}
J = rp = n\hbar
\end{equation}
is quantized when we have
\begin{equation}
\lambda_{B} = \frac{h}{p}.
\end{equation}
We write $p = h/\lambda_{B}$ (where $h=2\pi\hbar$) but it is more useful
to introduce a quantity $k=\lambda_{B}/2\pi$ satisfying
\begin{equation}
p = \hbar k.
\end{equation}
For vector quantities,
\begin{equation}
\vec{p} = \hbar\vec{k}.
\end{equation}
We call $\vec{k}$ the \define{Angular Wave Vector} (but the terminology
may vary depending on the reference) and it has units of
$[\mbox{length}]^{-1}$. We can work backwards, starting with this
condition, and derive Bohr's work. 

Specifically we have
\begin{calculation}
2\pi r
\step{de Broglie condition that an orbit is an integer number of periods}
n \frac{2\pi}{k}
\step{since $k=p/\hbar$}
n \frac{2\pi}{p/\hbar}
\step{using Eq~\eqref{eq:qm:experimental-background:bohr:momenta}}
n 2\pi\hbar\sqrt{\frac{r}{m_{e}Q^{2}}}.
\end{calculation}
We can solve this for $r$
\begin{equation}
  \sqrt{r} = \frac{n\hbar}{\sqrt{m_{e}Q^{2}}}\implies
r = \frac{n^{2}\hbar^{2}}{m_{e}Q^{2}}.
\end{equation}
This is precisely the result Bohr obtained.

\begin{remark}
Just to review, we have seen two \emph{different} theoretical
derivations for the energy spectrum of the Hydrogen atom. They are
equivalent, but de Broglie's relation $\vec{p}=\hbar\vec{k}$ turns out
to be an important relation which will play a critical role in quantum
theory. 
\end{remark}

\subsection{Electron Wave-Like Behaviour}

\M
There are a number of experiments which test the wave-like behaviour of
electrons, I will give two. They are rather ``hairy'' and technical, so
I'll skip the usual format and jump to the punch lines.

\N{Low Energy Electron Diffraction}
Around 1926,
Clinton Davisson and Lester Germer shot a beam of low energy electrons
at a thin sheet of nickel. There was an error in the experimental setup
and the nickel sheet heated up to extraordinary temperatures, forming a
different crystal structure than expected, which changed the scattering
behaviour of the electrons. The intensity of the backscattered electrons
had an angular dependence which suggested there was some interference
pattern which X-ray waves experience when scattering off metal sheets
(which Bragg determined a couple decades earlier).

But this only makes sense if electrons behaved like waves following the
de Broglie relations. This was the earliest strong evidence supporting
de Broglie's conjecture.

\N{Hitachi Double-Slit Experiment}
Akira Tonomura led a team of experimentalists at Hitachi in 1989
performing the double-slit experiment with electrons. The setup had a
screen which marked the point where an individual electron particle hits
it. After running $N$ electrons through this (for various $N$ ranging
from 160 to thousands), a clear interference pattern emerges which
resembles the double-slit experiment for light. This corroborates the
hypothesis that electrons have a wave-like behaviour, despite being
particles. 


\chapter{Special Relativity}

\M
There's a heuristic taught to graduate students:
\begin{equation}
\begin{pmatrix}\mbox{Quantum}\\
\mbox{Field}\\
\mbox{Theory}
\end{pmatrix}
=
\begin{pmatrix}\mbox{Special}\\
\mbox{Relativity}
\end{pmatrix}
+\begin{pmatrix}\mbox{Quantum}\\
\mbox{Mechanics}
\end{pmatrix}.
\end{equation}
This is partially true, but we should review special relativity (if only
to specify our conventions).

Historically, 
\begin{equation}
\begin{pmatrix}\mbox{Special}\\
\mbox{Relativity}
\end{pmatrix}
=
\begin{pmatrix}\mbox{Newtonian}\\
\mbox{Mechanics}
\end{pmatrix}
+
(\mbox{Electromagnetism}).
\end{equation}
This is because electromagnetism determines that the speed of light is
constant in a vacuum, and Newtonian mechanics assumes Galilean
relativity. But what happens if an inertial observer is moving at the
speed of light? Addition of velocities would suggest that bodies could
move faster than light\footnote{Einstein's thought experiment: suppose
you were riding on a bicycle going $0.999c$, and then you turn on your
headlights. What speed will the photons emitted from your headlights
travel?}, which violates results from electromagnetism.

\section{Foundations}

\begin{axiom}[Principle of Relativity]
The laws of physics are identical in all inertial frames. That is to
say, the outcome of any physical experiment is the same when performed
with identical initial conditions relative to any inertial frame.
\end{axiom}

\begin{axiom}
There exists an inertial reference frame in which light signals in
vacuum always travel rectilinearly at constant speed $c$, in all
directions, independently of the motion of the source.
\end{axiom}

\begin{remark}
These are the axioms as formulated in Rinder~\cite[\S4]{Rindler:1991sr}.
Traditionally, introductory texts to special relativity use some version
of these axioms, then ``derive'' results in special relativity.

A more intuitive approach uses $K$-calculus, as first formulated by
Bondi, using spacetime diagrams.
\end{remark}

\section{Kinematics}

\N{Four-Vectors} Special relativity differs from Newtonian physics by
working with 4-vectors. The usual vectors we encountered in physics
$\vec{x}$ are 3-vectors in physical space. Now we will use 4-component
vectors of the form $(t, \vec{x})$ or more generally using indices
$(x^{0}, x^{1}, x^{2}, x^{3})$. We can write this using basis vectors
\begin{subequations}
\begin{align}
  A &= (A^{0}, A^{1}, A^{2}, A^{3})\\
&= A^{0}\vec{e}_{0} + A^{1}\vec{e}_{1} + A^{2}\vec{e}_{2} + A^{3}\vec{e}_{3}\\
&= A^{0}\vec{e}_{0} + A^{i}\vec{e}_{i}\\
&= A^{\mu}\vec{e}_{\mu}
\end{align}
\end{subequations}
where $\vec{e}_{\mu}$ are basis vectors, we implicitly sum over repeated
indices, $i = 1,2,3$, and $\mu=0,1,2,3$.

The zeroth component is the time component, the remaining components are
the spatial components.

\begin{remark}
When the components of the four-vector are upstairs $A^{\mu}$,
these are \define{Contravariant} vectors.
\end{remark}

\N{Covectors}
We can also consider ``covectors'' or ``dual vectors'' (which eat in a
4-vector and produce a [real] number). These have components with
downstairs indices $B_{\mu}\vec{f}^{\mu}$ where $\vec{f}^{\mu}$ are the
co-basis covectors.

\M
Normally we can transform the components of a four-vector
$\vec{A}=A^{\mu}\vec{e}_{\mu}$ into the components of a covector by
using the metric $g_{\mu\nu}$ in the usual way:
\begin{equation}
A_{\nu}\vec{f}^{\nu} = (g_{\mu\nu}A^{\mu})\vec{f}^{\nu}.
\end{equation}
We can do the same, but for special relativity the metric is denoted
$\eta_{\mu\nu}$.\marginpar{\footnotesize Minkowski metric $\eta_{\mu\nu}$}
We call this the \define{Minkowski metric} and in Cartesian coordinates
has components
\begin{equation}
\eta_{\mu\nu} = \begin{pmatrix}-1 & 0 & 0 & 0\\
0 & 1 & 0 & 0\\
0 & 0 & 1 & 0\\
0 & 0 & 0 & 1
\end{pmatrix}.
\end{equation}
This is the so-called \define{East-Coast Convention} (particle
physicists multiply this by $-1$ and that's the \emph{West-coast convention}).

\N{Magnitude of Four-Vectors}
When $a^{\mu}$ is any four-vector, its magnitude is the scalar quantity
given by:
\begin{equation}
\|a^{\mu}\|^{2} := \eta_{\mu\nu}a^{\mu}a^{\nu}.
\end{equation}

\begin{definition}
The \define{Four-Position} is the four-vector with components
\begin{equation*}
x^{\mu} = (ct, x^{1}, x^{2}, x^{3}).
\end{equation*}
In Cartesian coordinates, $x^{\mu} = (ct, x, y, z)$.
\end{definition}

\N{Events}
Events in spacetime are described using 4-position vectors. This is an
idealization that events have no ``duration''. We could handle events
with some duration by having one 4-position for the ``start'' and
another 4-position for the ``end''.

\begin{definition}
When we have two 4-position vectors $\vec{A}_{1}=(ct_{1},\vec{r}_{1})$
and $\vec{A}_{2}=(ct_{2},\vec{r}_{2})$, we define the
\define{Displacement Four-Vector} as the four-vector
\begin{equation*}
\Delta\vec{A} = (c\,\Delta t,\Delta\vec{r}) = \vec{A}_{2} - \vec{A}_{1}.
\end{equation*}
For an \define{Infinitesimal Displacement Four-Vector} (or
\emph{Differential Four-Position}), we write
$\D\vec{A}$.
\end{definition}

\M
Suppose we have two events in spacetime separated by an infinitesimal
displacement 4-vector $\D x^{\mu}$. Special relativity demands the
infinitesimal interval,
\begin{equation}
(\D s)^{2} := \eta_{\mu\nu}\,\D x^{\mu}\,\D x^{\nu} = -c^{2}(\D t)^{2} + (\D x)^{2} + (\D y)^{2} + (\D z)^{2},
\end{equation}
must be the same for all inertial observers. Here we use summation
conventions where, when we have an index downstairs and upstairs, we sum
over it (so in our equation, we sum over $\mu$ and $\nu$).

\begin{definition} Let $a^{\mu}$ be a 4-vector.
  \begin{enumerate}
  \item If $\eta_{\mu\nu}a^{\mu}a^{\nu} < 0$, then we call $a^{\mu}$ \define{Time-like}.
  \item If $\eta_{\mu\nu}a^{\mu}a^{\nu} = 0$, then we call $a^{\mu}$ \define{Light-like}.
  \item If $\eta_{\mu\nu}a^{\mu}a^{\nu} > 0$, then we call $a^{\mu}$ \define{Space-like}.
  \end{enumerate}
\end{definition}

\begin{remark}
To see the motivation for these definitions, consider the trajectory of
a photon moving along the $x$-axis in Cartesian coordinates. It would be
$\vec{\gamma}(t)=(ct,ct,0,0)$ and its displacement from the origin would
be always zero for any $t\in\RR$. Therefore, the displacement would be
light-like.

For time-like vectors, consider the displacement 4-vector
$\vec{\alpha}(t)=(ct,0,0,0)$ which stays at the spatial origin. We see
its magnitude is $-c^{2}t^{2}<0$.

For space-like vectors, consider the displacement 4-vector
$\vec{\beta}(t)=(0,ct,0,0)$. Its magnitude is $c^{2}t^{2}>0$.
\end{remark}

\begin{definition}[Minkowski spacetime]
We write $\RR^{3,1}$ for \define{Minkowski Spacetime}, i.e., $\RR^{4}$
equipped with the Minkowski metric $\eta$.
\end{definition}

\begin{remark}
If we were using $+---$ signature conventions, Minkowski space would be
$\RR^{1,3}$. 
\end{remark}

\begin{definition}\label{defn:relativity:light-cone}
Let $a^{\mu}$ be any event. We define the \define{Lightcone} for $a^{\mu}$
to be the set of events light-like separated from $a^{\mu}$,
\begin{equation}
\mathscr{C} = \{\,b^{\mu}\in\RR^{3,1} \mid b^{\mu}~\mbox{is light-like separated from}~a^{\mu}\,\}.
\end{equation}
We can separate the light-cone in two: events in the future and events
in the past, giving us the \define{Future Light Cone}
\begin{equation}
\mathscr{C}^{+} = \{\,b^{\mu}\in\mathscr{C} \mid b^{0} > a^{0}\,\},
\end{equation}
and the \define{Past Light Cone} consisting of events preceding $a^{\mu}$,
\begin{equation}
\mathscr{C}^{-} = \{\,b^{\mu}\in\mathscr{C} \mid b^{0} < a^{0}\,\}.
\end{equation}
We can take the \define{Closed Light Cone} to consist of all light-like
separated events \emph{and} all time-like separated events, and denote
it by
\begin{equation}
\overline{\mathscr{C}} = \{\,b^{\mu}\in\RR^{3,1} \mid \eta_{\mu\nu}(b^{\mu}-a^{\mu})(b^{\nu}-a^{\nu})\leq0\,\}.
\end{equation}
\end{definition}

\M
The only possible causal influences for an event $a^{\mu}$
\emph{must lie within the past causal light cone} for the event, since
nothing can travel faster than light. For this reason, we call any
four-vector $b^{\mu}$ \define{Causal-like} if it is not space-like
separated from $a^{\mu}$, i.e., if the displacement $b^{\mu} - a^{\mu}$
four-vector is either time-like or light-like.

\N{World lines}
A curve $\gamma$ in spacetime is called a \define{World Line}
if its tangent vector is future time-like at each point along the
curve. More generally, we could weaken the condition, and allow tangent
vectors to be causal-like.

Particles move along world lines in special relativity.

\M
If we have a curve $\gamma$ in spacetime such that its tangent vector is
space-like at each point along the curve, we call the curve
\define{Space-like}.

If we have a curve in spacetime such that its tangent vector is
ligh-like at each point along the curve, we call the curve
\define{Light-like}. 


\N{Proper Time}
The time between two events along a world line, according to the
observer moving along the trajectory, is precisely the
\define{Proper Time} and denoted $\tau$. For an infinitesimal
displacement along the trajectory, the infinitesimal change in proper
time is given by
\begin{equation}
c^{2}\,(\D\tau)^{2} = -(\D s)^{2}.
\end{equation}
The proper time interval along a trajectory is given by the integral
\begin{equation}
\Delta\tau = \int_{\gamma}\D\tau =
\int_{\gamma}\frac{\sqrt{-\eta_{\mu\nu}\,\D x^{\mu}\,\D x^{\nu}}}{c}.
\end{equation}

\N{Parametrizing World Lines}
We parametrize a world line by its proper time, and write it as
$x^{\mu}(\tau)$. This gives us a smooth family of four-positions for a
physical body.

\begin{definition}
Let $x^{\mu}(\tau)$ be a world line. We define its \define{Four-Velocity}
as the four-vector
\begin{equation*}
U^{\mu}(\tau) = \frac{\D x^{\mu}(\tau)}{\D\tau}.
\end{equation*}
\end{definition}

\begin{remark}
Whenever we are tempted to take the time derivative of a quantity, we
really want to take the derivative with respect to \emph{proper time}.
\end{remark}

\N{Lorentz Factor}
We have the familiar Newtonian 3-velocity given by
\begin{equation}
v^{i} = \frac{\D x^{i}}{\D t}.
\end{equation}
We can relate the coordinate time $x^{0}=ct$ with proper time $\tau$ by
the \define{Lorentz Factor} (which is a function of the Newtonian 3-velocity),
\begin{equation}
  \begin{split}
\gamma(\vec{v}) &= \frac{\D t}{\D\tau} = \left(1 - \frac{\eta_{ij}v^{i}v^{j}}{c^{2}}\right)^{-1/2}\\
&=\frac{c}{\sqrt{-\eta_{\mu\nu}U^{\mu}U^{\nu}}}.
  \end{split}
\end{equation}
This allows us to relate the familiar Newtonian 3-velocity
to the four-velocity by
\begin{equation}
U^{\mu} = (c, \gamma(\vec{v})\vec{v}).
\end{equation}

\section{Dynamics}

\M
Often the dynamics of particles are omitted in discussions of special
relativity, because things getting complicated conceptually.

\begin{definition}
Let $x^{\mu}(\tau)$ be the world line for a massive body.
Then the \define{Rest Mass} (or \emph{invariant mass}) for the body is
the mass $m_{0}$ as measured in its reference frame.
\end{definition}

\begin{remark}
Some authors use a notion of \emph{relativistic mass}
$m = \gamma(\vec{v})m_{0}$, which is mildly controversial. 
\end{remark}

\N{Four-Momentum}
For a massive particle of rest mass $m_{0}$, its \define{Four-Momentum}
$\vec{P}$ is defined as the product of the rest mass and its
four-velocity, i.e.,
\begin{equation}
\vec{P} := m_{0}\vec{U}.
\end{equation}
We can write out its explicit components
\begin{equation}
\vec{P} = m_{0}\gamma(\vec{v})(c, \vec{v}) = (E/c, \vec{p}).
\end{equation}
Here the total energy of the moving particle is given by
\begin{equation}
E = \gamma(\vec{v})m_{0}c^{2},
\end{equation}
and the total (relativistic) 3-momentum is
\begin{equation}
\vec{p} = m_{0}\gamma(\vec{v})\vec{v}.
\end{equation}

\N{Energy--Momentum Relation}
We have the energy--momentum relation (or \define{Mass--Shell Relation})
be
\begin{equation}
E^{2} = c^{2}\vec{p}\cdot\vec{p} + \bigl(m_{0}c^{2}\bigr)^{2}.
\end{equation}
Equivalently, we have this relation describe the magnitude of the
four-momentum as a constant:
\begin{equation}
\eta_{\mu\nu}p^{\mu}p^{\nu} = -m_{0}^{2}c^{2}.
\end{equation}

\begin{ddanger}
In special relativity, we can meaningfully talk about the Center-of-Mass
reference frame for a system of particles. This is especially useful for
scattering problems. However, if we tried to carry this notion over to
General Relativity, then we run into problems because it is a nonlocal
concept. 
\end{ddanger}

\begin{exercise}
From the mass-shell relation and $\vec{P} = (E/c, \vec{p})$, deduce
$E = \gamma(\vec{v})m_{0}c^{2}$ and 
$\vec{p}\cdot\vec{p}=\pm(\gamma(\vec{v})^{2}-1)m_{0}^{2}c^{2}$.
Determine the correct sign in that second relation.
\end{exercise}

\begin{exercise}
Suppose we have two massive particles with four-momenta $\vec{P}_{1}$
and $\vec{P}_{2}$ and relative speed $v$. Determine
$\vec{P}_{1}\cdot\vec{P}_{2}=\eta_{\mu\nu}P^{\mu}_{(1)}P^{\nu}_{(2)}$ in
terms of their rest mass $m_{01}$ and $m_{02}$ and relative speed $v$.
Hint: if $\vec{P}_{1}=\vec{P}_{2}$ and $v=0$, then you should recover
the mass--shell relation.
\end{exercise}

\N{Four-Force}
We can define the four-force as the four-vector
\begin{equation}
\vec{F} = \frac{\D\vec{P}}{\D\tau}.
\end{equation}
As an immediate consequence of the mass--shell relation, we find
\begin{equation}
\vec{F}\cdot\vec{P} = \eta_{\mu\nu}F^{\mu}P^{\nu} = 0.
\end{equation}

\N{Lagrangian for Point-Particle}
We can write the Lagrangian for a massive point-particle with rest mass
$m_{0}$ as
\begin{equation}
L = cm_{0}\sqrt{-\eta_{\mu\nu}\dot{x}^{\mu}\dot{x}^{\nu}} - V
\end{equation}
where $\dot{x}^{\mu} = \D x^{\mu}/\D\tau = U^{\mu}$, and $V$ is the potential
energy term. The action is then 
\begin{equation}
I = \int L\,\D\tau.
\end{equation}
Varying the action with respect to $\delta x^{\mu}$ then gives us the
equations of motion.

For massless particles, care must be taken with the parametrization, as
well as using its four-momentum to write
\begin{equation}
L = c\sqrt{-\eta_{\mu\nu}P^{\mu}P^{\nu}} - V.
\end{equation}
We can use the relation (which holds for both massive and massless particles):
\begin{equation}
\frac{\D x^{\mu}}{\D t} = \frac{P^{\mu}}{P^{0}}.
\end{equation}

\begin{danger}
This is the correct Lagrangian to work with, especially if we want to
quantize it. There is some subtlety with it, which we should confess
openly: it is a constrained system. To see this, compute the Hamiltonian
for a free massive relativistic particle. It will vanish. This is
because time is a coordinate (proper time is a parameter), and its
conjugate momentum is ``the Hamiltonian''. So we end up with a
constraint. 
\end{danger}

\begin{ddanger}
Some authors insist that canonical mechanics for special relativistic
systems ``breaks Lorentz invariance'', which is not really true. If
you've picked $\mu=0$ to be the time component for four-vectors, then
you've also ``broken Lorentz invariance'' just as much as canonical
mechanics has. We can describe a Lorentz boost as a canonical
transformation (which preserves Lorentz invariance as much as anything
else). This is just offered as a lazy and sloppy justification for using
the path integral formalism, which makes no coherent sense.
\end{ddanger}

\N{References}
For the uninitiated, Taylor and Wheeler~\cite{Taylor:1992sp} is a great
introduction. Rindler~\cite{Rindler:1991sr} is a good review.

\N{TODO: Lorentz Invariance}
I should specifically spell out what Lorentz invariance means, since
it's implicit but not done anywhere.

\N{TODO: Group Theory}
I should also mention the Lorentz and Poincar\'e groups, and how they
capture the essence of Lorentz invariance\dots because they're going to
be important in quantum field theory.

\N{TODO: Scattering}
It is probably good to discuss $2\to2$ scattering in special relativity,
since that's the basis of a lot of particle physics experiments.

\N{TODO: Electromagnetism}
I should also take some time to discuss the basics of electromagnetism
using a four-potential $A^{\mu}$ and the Maxwell field tensor
$F^{\mu\nu}$. After all, this serves as the basis for Yang--Mills theory
later on.
%%
%% rqm.tex
%% 
%% Made by Alex Nelson
%% Login   <alex@tomato>
%% 
%% Started on  Tue Jul 21 10:59:59 2009 Alex Nelson
%% Last update Tue Jul 21 10:59:59 2009 Alex Nelson
%%
\documentclass[final]{amsart}
\usepackage{fly}
\usepackage{danger}
\usepackage{paralist}


\title{Notes on Relativistic Quantum Mechanics}
\date{July 21, 2009}
\email{pqnelson@gmail.com}
\author{Alex Nelson}
\begin{document}
\maketitle\footnote{Note we are using the West Coast convention, i.e. + - - - metric signature, and setting $c=1$ and $\hbar=1$.}
\tableofcontents

\section{One Particle Systems: Mathematical Formalism}
%%
%% oneParticleState.tex
%% 
%% Made by Alex Nelson
%% Login   <alex@tomato>
%% 
%% Started on  Tue Jul 21 12:54:12 2009 Alex Nelson
%% Last update Tue Jul 21 12:54:12 2009 Alex Nelson
%%

The simplest system to consider is a single particle. The
function space used to model quantum-mechanical states is a
Hilbert Space $\mathcal{H}$ of square integrable functions on the
physical space (denoted by $\mathcal{C}$):
\begin{equation}%\label{eq:}
L^{2}(\mathcal{C}) = \left\{f\,:\;\int_{\mathcal{C}}|f(\overline{x})|^{2}d^{3}\overline{x}<\infty\right\}
\end{equation}
Note that in all fairness, $\mathcal{H}$ can be written in either
position coordinates $\overline{x}$ or momentum coordinates
$\overline{p}$. The relationship between the position-space and
momentum-space is precisely the familiar Fourier transform:
\begin{equation}%\label{eq:}
\mathcal{F}(f)(\overline{p}) \stackrel{\text{def}}{=} \int e^{i\overline{x}\cdot\overline{p}}f(\overline{x})d^{3}\overline{x}
\end{equation}
Despite the change of variables, $\mathcal{F}$ sends
$\mathcal{H}$ to itself, so both $f$ and its Fourier transform
$\mathcal{F}(f)$ are in $\mathcal{H}$.
\begin{rmk}
It should be emphasized that if $f$ is square-integrable, then
$e^{i\overline{x}\cdot\overline{p}}f(\overline{x})$ is
square-integrable \emph{but not necessarily integrable!} That is,
we have no guarantee that
$e^{i\overline{x}\cdot\overline{p}}f(\overline{x})\in L^{1}(\mathcal{C})$.

To define the Fourier transform on $\mathcal{H}$, we should first
define it on some suitably nice subspace of $\mathcal{H}$
(e.g. the space of smooth functions with ``compact support'' ---
i.e. they are zero outside of a compact subset of their
domain). Then we observe that the Fourier transform is an
isometry (up to some scale factor) on our nice subspace, so we
extend this isometry from our nice subspace to all of $\mathcal{H}$.
\end{rmk}

We represent the observables by operators. More relevantly, the
position operators $\widehat{x}_{m}$ and momentum operators
$\widehat{p}_{m}$ are represented in position-space by
multiplication by the coordinate fuhnctions $x_{m}$ and the
partial derivative operators $-i\partial_{m}$
(respectively). Observe also that the Fourier transform converts
multiplication by $x_{m}$ on functions of $\overline{x}$ into the
differential operators $-i\partial_{m}$ on functions of
$\overline{p}$:
\begin{equation}%\label{eq:}
\mathcal{F}(x_{m}f)(\overline{p})=-i\partial_{m}\mathcal{F}(f)(\overline{p}).
\end{equation}

The natural question to ask is ``What are the eigenstates of
these operators?'' Well, in position space, we find the position
eigenstates are just delta functions
\begin{subequations}
\begin{align}
(\widehat{x}_{m}\delta_{\overline{q}})(\overline{x}) &=
  \widehat{x}_{m}\delta(\overline{x}-\overline{q})\\
&= q_{m}\delta(\overline{x}-\overline{q})\\
&= (q_{m}\delta_{\overline{q}})(\overline{x})
\end{align}
\end{subequations}
Similarly, for the eigenstates of the momentum operators
$\widehat{p}_{m}$, we see that the eigenstates in position-space
are $e_{\overline{p}}(\overline{x})$:
\begin{subequations}
\begin{align}
(\widehat{p}_{m}e_{\overline{p}})(\overline{x}) &= -i\partial_{m}\exp(i\overline{p}\cdot\overline{x})\\
&= p_{m}\exp(i\overline{p}\cdot\overline{x})\\
&= (p_{m}e_{\overline{p}})(\overline{x}).
\end{align}
\end{subequations}

But we have just two minor problems: \begin{inparaenum}
\item neither $\widehat{x}_{m}$ nor $\widehat{p}_{n}$ act on all
  of $\mathcal{H}$, and
\item $\mathcal{H}$ doesn't contain the eigenstates of either
  operators.
\end{inparaenum}
We can solve the first problem fairly easily --- we'll consider
the subspace $S\subset{\mathcal{H}}$ where the operators map $S$
to itself. Similarly, we resolve the second problem by defining
the kets as elements of $S^{*}$, the space of continuous
antilinear functionals on $S$. Since $\widehat{p}_{n}$ acts on
all functions of $S$, these functions must be infinitely
differentiable, and so $S^{*}$ will contain the
$\delta$-functions and all their derivatives. Similarly, by
taking the Fourier Transform, since $\widehat{x}_{m}$ acts on
$S$, it follows that $S^{*}$ will contain exponential functions
$\exp(i\overline{p}\cdot\overline{x})$. 

Instead of a single Hilbert space, we end up with a triple
\begin{equation}%\label{eq:}
S\subset{\mathcal{H}}\subset{S^{*}}
\end{equation}
The physical states live in $S$, and the operator eigenstates
live in $S^{*}$. With appropriate demands on the space $S$, 
this triple ends up being a \emph{Rigged Hilbert Space}~\cite{delamadrid}~\cite{Madrid:2004zy}. 
In this context ``Rigged'' \emph{is  not} in the sense of ``This
game is rigged'' but rather in the sense of ``equipped'' --- like
how a boat is ``rigged'' or ``equipped to sail''. 

\bigskip
\begin{ddanger}
In fact, the triple $S\subset\mathcal{H}\subset{S^{*}}$ is a
rigged Hilbert space if $S$ is a nuclear subspace of
$\mathcal{H}$. See Gelfand~\cite{gelfandgeneralized} or
Maurin~\cite{maurin} for rigorous details about the notion of
nuclear spaces. We'll discuss one such criteria for $S$ to be
nuclear. Specifically,
\begin{enumerate}
\item there exists a countable family $\|\cdot\|_{k}$ of norms on
$S$ with respect to which convergence is defined
by 
\begin{equation}
f_{n}\to{f}\quad\iff\quad\|f_{n}-f\|_{k}\to0\;\;\forall k\geq0;
\end{equation}
\item $S$ is complete with respect to this notion of
  convergence; and 
\item there exists a Hilbert-Schmidt operator on
  $S$ with a continuous inverse.
\end{enumerate}
We'll leave the interested reader to refer to the cited sources.
\end{ddanger}
\bigskip

In a rigged Hilbert Space we have eigenfunction expansions. More
precisely, consider a state $|f\>$ represented by the function
$f$, let $|\overline{x}\>$ be the position eigenstate represented
by the distribution $\delta_{\overline{x}}$. We assume the
relationship between the functions and the kets is such that
\begin{equation}%\label{eq:}
f(\overline{x}) = \<\overline{x}|f\>.
\end{equation}
We can then expand the state $|f\>$ in terms of the position
eigenstate $|\overline{x}\>$ which should be of the form
\begin{equation}%\label{eq:}
|f\> = \int |\overline{x}\>\,\<\overline{x}|f\>\,d^{3}\overline{x} = \int f(\overline{x})\,|\overline{x}\>\,d^{3}\overline{x}.
\end{equation}
The conditions on $S$ in a rigged Hilbert Space ensure that
$f(\overline{x})|\overline{x}\>$ is integrable for all $f\in{S}$.

\section{One Particle Systems: Physical Aspects}
%%
%% oneParticleSystemPhysics.tex
%% 
%% Made by Alex Nelson
%% Login   <alex@tomato>
%% 
%% Started on  Tue Jul 21 15:55:44 2009 Alex Nelson
%% Last update Tue Jul 21 15:55:44 2009 Alex Nelson
%%

We're interested in a toy model of relativistic quantum
mechanics, so we begin with a single particle. All we really
need, truth be told, is a state space plus a Hamiltonian
operator. We should remember, from Special Relativity, the
energy-momentum four-vector $\widehat{p}_{\mu}$ has as its time component the
Hamiltonian $\widehat{p}_{0}=\widehat{H}$. For convenience, we'll
work in the momentum space with the momentum operator eigenbasis
\begin{equation}%\label{eq:}
\widehat{p}_{m}|\overline{k}\>=k_{m}|\overline{k}\>
\end{equation}
We assume the states are normalized thus
\begin{equation}%\label{eq:}
\<\overline{k}|\overline{k}'\>=\delta^{(3)}(\overline{k}-\overline{k}').
\end{equation}
This means that the length of a ket is undefined. It is,
nonetheless, a normalization suitable for integration over
momentum. As an added bonus, we also get the resolution of the
identity
\begin{equation}%\label{eq:}
\mathbf{1}=\int\,|\overline{k}\>\,\<\overline{k}|\,d^{3}\overline{k}
\end{equation}

Since energy-momentum is a four-vector, we demand that
\begin{equation}%\label{eq:}
\widehat{p}^{\mu}\widehat{p}_{\mu} = \widehat{H}^{2}-|\widehat{p}_{m}\widehat{p}^{m}|
\end{equation}
needs to be constant on the orbits of the Poincar\'e
group. Further if $|\overline{k}\>$ and $|\overline{k}\,'\>$ are
two states of a single particle, then there exists a Lorentz
boost from one to the other (up to scale). Hence we assume the
existence of a scalar quantity $\mu$ (the particle mass) which
satisfies
\begin{equation}%\label{eq:}
(\widehat{H}^{2} - \widehat{p}_{m}\widehat{p}^{m})|\overline{k}\>
= \mu^{2}|\overline{k}\>
\end{equation}
This implies that the Hamiltonian operator $\widehat{H}$ is
diagonal in the momentum eigenbasis (i.e. the basis of
eigenstates of the momentum operator):
\begin{equation}%\label{eq:}
\widehat{H}|\overline{k}\> = \left(\|\overline{k}\|^{2}+\mu^{2}\right)^{1/2}|\overline{k}\>
\end{equation}
The eigenvalues of the Hamiltonian operator come up enough times
that we introduce the shorthand for it:
\begin{equation}%\label{eq:}
\omega(\overline{k}) \stackrel{\text{def}}{=} \left(\|\overline{k}\|^{2}+\mu^{2}\right)^{1/2}
\end{equation}
(This should be vaguely reminiscent of the de Broglie relations
$E=\hbar\omega$.)

\begin{rmk}
Observe that this entire scheme we've devised is equivalent to
taking the limit of the state space for a cube of side $L$ under
periodic boundary conditions, i.e. the particle in a box
situation. In such a cube, we should recall the spectrum of the
momentum operator is discrete and the normalization is given by
the Kronecker delta:
\begin{equation}%\label{eq:}
\overline{k}=\frac{2\pi}{L}(n_x,n_y,n_z),\quad\text{and}\quad\<\overline{k}|\overline{k}\,'\>=\delta_{\,\overline{k}\, ,\,\overline{k}\,'}
\end{equation}
This observation is taken advantage of when deriving the
differential transition probability per unit time for particle
scattering.
\end{rmk}

\section{Unitary Representation of Poincar\'e Group}
\subsection{Action of Translation on States}
%%
%% poincareRep.tex
%% 
%% Made by Alex Nelson
%% Login   <alex@tomato>
%% 
%% Started on  Wed Jul 22 12:09:17 2009 Alex Nelson
%% Last update Wed Jul 22 12:09:17 2009 Alex Nelson
%%

The Lorentz transformation is usually ``represented'' by a matrix
$\Lambda$ which, when written explicitly, is
\begin{equation}%\label{eq:}
(\Lambda x)^{\mu} = {\Lambda^{\mu}}_{\nu}x^{\nu}
\end{equation}
where Einstein convention is used (implicit sum over $\nu$
occurs). We have that the matrix ${\Lambda^{\mu}}_{\nu}$ must
satisfy
\begin{equation}%\label{eq:}
{\Lambda^{\lambda}}_{\mu}{\Lambda_{\lambda}}_{\nu} = \eta_{\mu\nu}
\end{equation}
where $\eta_{\mu\nu}$ is the Minkowski metric (metric for flat
spacetime).

Now, the Poincar\'e group is the set of Lorentz transformations
and space-time translations, so the element of the group would be
$(\Lambda,a)$ such that
\begin{equation}%\label{eq:}
x^{\mu}\to y^{\mu} = {\Lambda^{\mu}}_{\nu}x^{\nu}+a^{\mu}.
\end{equation}
The group multiplication law is then just
\begin{equation}%\label{eq:}
(\Lambda_{2},a_{2})(\Lambda_{1},a_{1}) = (\Lambda_{2}\Lambda_{1},\Lambda_{2}a_{1}+a_{2}).
\end{equation}
We are interested in irreducible unitary representations $U(\cdot)$ of our
group, all we need to worry about are the generators.

The translations, rotations, and boosts of the Poincar\'e group
must act on the space of states. A Poincar\'e group element $g$
acts as a unitaruy operator $U(g)$ on the state space. The action
must satisfy a multipication condition
\begin{equation}%\label{eq:}
U(gh)=U(g)U(h)
\end{equation}
for all $g,h$ in the Poincar\'e group.

Translation of spacetime by a four-vector $a^{\mu}$ is defined by
\begin{equation}%\label{eq:}
\Delta_{a}(x) = x+a.
\end{equation}
Translation of a state $\psi$, on the other hand, should be
moving the graph by $a$. This means that
$\Delta_{a}\psi(x)=\psi(x-a)$. The unitary representation
$U(\Delta_{a})$ of $\Delta_{a}$ must thus be defined by
\begin{equation}%\label{eq:}
U(\Delta_{a})|\psi\>=|\Delta_{a}\psi\>.
\end{equation}
We'd like to find an expression for $U(\Delta_a)$ in terms of the
energy-momentum four-vector $\widehat{p}_{\mu}$.

Evolution in time is translation of the observer forward in time,
or (equivalently) translation of the system backwards in time:
\begin{equation}%\label{eq:}
\exp(-it\widehat{H})|\psi(x)\> = |\psi(x_{0}+t,\overline{x})\>.
\end{equation}
Let $\tau^{\mu}=(-t,\vec{0})$ be a four-vector, then we can
rewrite our translation in time as
\begin{equation}%\label{eq:}
\exp(i\tau^{\mu}\widehat{p}_{\mu})|\psi\> = |\Delta_{\tau}\psi\>.
\end{equation}
Lorentz invariance implies that this equation is true whenever
$\tau$ is timelike, and the additivity of translations then shows
this to be true for all four-vectors $\tau$. From this definition
of $U(\Delta_{a})$ we can therefore deduce that
\begin{equation}%\label{eq:}
U(\Delta_a) = \exp(ia^{\mu}\widehat{p}_{\mu}).
\end{equation}

Although this unitary representation is derived in the
position-space formulation of quantum mechanics, it works equally
well in the momentum-space formulation. We can deduce that the
unitary representation of translations on momentum eigenstates is
given by
\begin{equation}%\label{eq:}
U(\Delta_{a})|\overline{k}\> =
\exp(ia^{\mu}\widehat{p}_{\mu})|\overline{k}\> = \exp(ia^{\mu}k_{\mu})|\overline{k}\>
\end{equation}
where $k_{0} = \omega(\overline{k})$.

\begin{rmk}
Recall Taylor's theorem in real analysis can be formulated as
\begin{equation}%\label{eq:}
f(x+h) =
\left(\sum_{n=0}^{\infty}h^{n}\frac{d^{n}}{dx^{n}}\right)f(x) = \exp\left(h\frac{d}{dx}\right)f(x)
\end{equation}
which should look familiar: we just deduced the unitary
representation of spacetime translations should be
\begin{equation}%\label{eq:}
\exp(i\tau^{\mu}\widehat{p}_{\mu})|\psi\> = U(\Delta_{\tau})|\psi\>.
\end{equation}
If we don't distinguish $|\psi\>$ from $\psi(x)$, we see that
Taylor's theorem guarantees our representation to be of spacetime
translations.
\end{rmk}

\subsection{Action of the Lorentz Group}
%%
%% actionLorentzGroup.tex
%% 
%% Made by Alex Nelson
%% Login   <alex@tomato>
%% 
%% Started on  Wed Jul 22 13:01:36 2009 Alex Nelson
%% Last update Wed Jul 22 13:01:36 2009 Alex Nelson
%%

The space of particle states is three dimensional. The energy
$k_0$ of a particle with momentum $\overline{k}$ is constrained
by
\begin{equation}%\label{eq:}
k_{0}\geq0
\end{equation}
and
\begin{equation}%\label{eq:}
k^{2} = k_{\mu}k^{\mu} = \mu^{2}.
\end{equation}
Therefore the possible energy-momentum vectors lie on a
hyperbolic sheet in $k$-space, the mass hyperboloid. We need an
integration measure on this hyperboloid if we want to do Lorentz
invariant computations.

Let $\theta(t)$ be the Heaviside step function
\begin{equation}%\label{eq:}
\theta(t) = \begin{cases} 0 &\text{if }t<0\\
1 & \text{if }t>0
\end{cases}
\end{equation}
Define an integration $d\lambda(k)$ on the positive hyperboloid
as follows:
\begin{equation}%\label{eq:}
d\lambda(k) \stackrel{\text{def}}{=} d^{4}k \delta(k^{2}-\mu^{2})\theta(k_{0})
\end{equation}
The Lebesgue measure $d^{4}k$ is Lorentz invariant due to the
Lorentz transformation having unit determinant. Here since
$k^{2}-\mu^{2}$ is Lorentz invariant, the $\delta$ function is
Lorentz invariant. Similar reasoning holds for $\theta(k_{0})$
being Lorentz invariant.

We can take advantage of the identity
\begin{equation}%\label{eq:}
\delta(f(k)) = \sum_{\{k:f(k)=0\}}\frac{1}{\|f'(k)\|}\delta(k)
\end{equation}
and  the fact that
\begin{subequations}
\begin{align}
k^{2}-\mu^{2} &= (k_{0}^{2}-\|\overline{k}\|^{2})-\mu^{2}\\
&= k_{0}^{2} - (\|\overline{k}\|^{2} + \mu^{2}) \\
&= k_{0}^{2} - \omega(\overline{k})^{2} \\
&= (k_{0} - \omega(\overline{k}))(k_{0} + \omega(\overline{k}))
\end{align}
\end{subequations}
to deduce that
\begin{subequations}
\begin{align}
\delta(k^{2}-\mu^{2})\theta(k_{0}) &= \delta\left((k_{0} - \omega(\overline{k}))(k_{0} + \omega(\overline{k}))\right)\theta(k_0)\\
&=\frac{1}{2\omega(\overline{k})}(\delta(k_0-\omega(\overline{k}))\theta(k_0)+\delta(k_0+\omega(\overline{k}))\theta(k_0))\\
&=\frac{1}{2\omega(\overline{k})}\delta(k_0-\omega(\overline{k}))\theta(k_0)
\end{align}
\end{subequations}
since $\delta(k_0+\omega(\overline{k}))$ requires $k_0<0$ which
then demands that $\theta(k_0)=0$, so that term drops out completely.
Observe that this means we can effectively eliminate $k_0$ from
any integral with respect to $\omega(\overline{k})$ as follows:
\begin{subequations}
\begin{align}
\int f(k)d\lambda(k) &= \int f(k)\left(\frac{\delta(k_{0}-\omega(\overline{k}))}{2\omega(\overline{k})}\theta(k_{0})d^{3}\overline{k}dk_{0}\right)\\
&= \int f\left(\omega(\overline{k}),\overline{k}\right)\frac{d^{3}\overline{k}}{2\omega(\overline{k})}
\end{align}
\end{subequations}
This integral and the arbitrary function $f$ are commonly
eliminated from this result, leaving an equality of measures
\begin{equation}%\label{eq:}
d\lambda(k) = \frac{d^{3}\overline{k}}{2\omega(\overline{k})}
\end{equation}
and
\begin{equation}%\label{eq:}
k_{0} = \omega(\overline{k}).
\end{equation}

\begin{comment}
\begin{ddanger}We can now ask if the measure
%\begin{equation}%\label{eq:}
$d^{3}\overline{k}/2\omega(\overline{k})$
%\end{equation}
is Lorentz invariant or not. We expect it to be so, but lets try
to demonstrate it explicitly by computing the Jacobian of a
Lorentz boost $\Lambda$. Without loss of generality, we can
assume that we are working with Cartesian coordinates. Note we
can factorize an Lorentz boost as
\begin{equation}%\label{eq:}
{\Lambda^{\mu}}_{\nu} = {L^{\mu}}_{\alpha}{R^{\alpha}}_{\nu}
\end{equation}
where $L$ is a rotation in the $t-x$ plane, and $R$ is an
arbitrary spatial rotation. We know from Euler's theorem that
\begin{equation}%\label{eq:}
\det(R)=1
\end{equation}
so that means that
\begin{equation}%\label{eq:}
\det(\Lambda)=\det(L).
\end{equation}
But we can write in block form
\begin{equation}%\label{eq:}
{L^{\mu}}_{\alpha} = \begin{bmatrix} L & 0\\ 0 & I_{2} \end{bmatrix}
\end{equation}
which means that 
\begin{equation}%\label{eq:}
\det({L^{\mu}}_{\alpha}) = {L^{0}}_{0}{L^{1}}_{1} - {L^{1}}_{0}{L^{0}}_{1}.
\end{equation}
This means, without loss of generality, we can write
${\Lambda^{\mu}}_{\nu}={L^{\mu}}_{\nu}$. We make the switch
$k^{0}=\omega(\overline{k})$, so our transformation
yields the coordinates
\begin{subequations}
\begin{align}
l^{0} &= {L^{0}}_{0}\omega(\overline{k}) - {L^{0}}_{1}k^{1}\\
l^{1} &= {L^{1}}_{0}\omega(\overline{k}) - {L^{1}}_{1}k^{1}\\
l^{2} &= k^{2}\\
l^{3} &= k^{3}
\end{align}
\end{subequations}
One may be at first alarmed by the switch to
$\omega(\overline{k})$ but it is invariant under spatial
rotations, and we've seen that $k^0=\omega(\overline{k})$ so it
is kosher. By Lorentz invariance, we demand further that
\begin{equation}%\label{eq:}
l^{\mu}l_{\mu} = k^{\mu}k_{\mu} = \mu^{2}
\end{equation}
which in turn implies that
\begin{equation}%\label{eq:}
l^{0}l_{0} = \|\overline{l}\|^{2} + \mu^{2}\;\Rightarrow\; l^{0}
= \omega(\overline{l})
\end{equation}
all by Lorentz invariance.
\end{ddanger}
\end{comment}

If we define Lorentz-normalized kets $|k\>$ by
\begin{equation}%\label{eq:}
|k\> = \left(2\omega(\overline{k})\right)^{1/2}(2\pi)^{3/2}|\overline{k}\>
\end{equation}
with $k_{0}=\omega(\overline{k})$, then the new normalization
conditions is
\begin{equation}%\label{eq:}
\<k|k'\> = 2\omega(\overline{k})(2\pi)^{3}\delta^{(3)}(\overline{k}-\overline{k}')
\end{equation}
and the resolution of the identity is based on the Lorentz
invariant measure:
\begin{equation}%\label{eq:}
\mathbf{1} = \int|k\>\<k|\frac{d^{3}\overline{k}}{(2\pi)^{3}2\omega(\overline{k})}.
\end{equation}
With these Lorentz-normalized states, we can define the unitary
representation of the Lorentz group simply:
\begin{thm}%\label{thm:}
If we define $U(\Lambda)$ by $U(\Lambda)|k\>=|\Lambda k\>$, then
$U$ is a unitary representation of the Lorentz group.
\end{thm}
\begin{proof}
The multiplications property
$U(\Lambda\Lambda')=U(\Lambda)U(\Lambda')$ follows immediately
from definition. To show that the representation is unitary, we
use the resolution of the identity
\begin{subequations}
\begin{align}
U(\Lambda)U(\Lambda)^{\dag} &= \int U(\Lambda)|k\>\<k|U(\Lambda)^{\dag}\frac{d^{3}\overline{k}}{(2\pi)^{3}2\omega(\overline{k})}\\
&= \int |\Lambda k\>\<\Lambda k|\frac{d^{3}\overline{k}}{(2\pi)^{3}2\omega(\overline{k})}\\
&= \mathbf{1}
\end{align}
\end{subequations}
since the measure is Lorentz-invariant.
\end{proof}
It is mildly surprising that $U(\Lambda)$ defined in our theorem
is a unitary operator due to $|k\>$ and $|\Lambda k\>$ appear to
have different lengths when $\Lambda$ is a boost. \emph{However,}
$\delta^{(3)}(0)$ is undefined, so the normalization of the kets
does not determine a length. We regard the uniformly unlocalized
state described by $|k\>$ as \emph{unphysical}. The physical
states have the form 
\begin{equation}%\label{eq:}
|\psi\>\stackrel{\text{def}}{=} \int \psi(\overline{k})|k\>\frac{d^{3}\overline{k}}{(2\pi)^{3}2\omega(\overline{k})}
\end{equation}
where the measure is structured so
$\<k|\psi\>=\psi(\overline{k})$. We can check that the length of
$|\psi\>$ is well defined whenever $\psi(\overline{k})$ is
square-integrable and that our definition of $U(\Lambda)$ makes
the representation unitary on the space of physical states.

\subsection{Representing the Poincar\'e Group}
%%
%% actionPoincareGroup.tex
%% 
%% Made by Alex Nelson
%% Login   <alex@tomato>
%% 
%% Started on  Sat Jul 25 14:01:05 2009 Alex Nelson
%% Last update Sat Jul 25 14:01:05 2009 Alex Nelson
%%

We really want to find a unitary representation of the Poincar\'e
group, which is the Lorentz group plus spacetime translations
(i.e. rotations, Lorentz boosts, and space-time translations). We
have the representation condition $U(gh)=U(g)U(h)$ must hold for
all $g,h$ in the Poincar\'e group. We've seen what happens when
both $g,h$ are in the Lorentz group, and when both $g,h$ are
space-time translations. We now need to ask: what happens when
one is a translation and the other is a boost?

We can uniquely factor any element $g$ of the Poincar\'e group as
the product
\begin{equation}%\label{eq:}
g = \Delta_{a}\Lambda
\end{equation}
where $\Lambda$ is in the Lorentz group, and $\Delta_a$ is a
translation. Multiplication in the Poincar\'e group depends on
multiplication in the Lorentz group and addition of translations
through an interchange in the order of the two facts:
\begin{subequations}
\begin{align}
gh &= \Delta_{a}\Lambda\Delta_{b}M\\
&= \Delta_{a}(\Lambda\Delta_{b}\Lambda^{-1})\Lambda M\\
&= \Delta_{a}\Delta_{\Lambda b}\Lambda M
\end{align}
\end{subequations}
where we have used the identity
\begin{equation}%\label{eq:}
\Lambda\Delta_{b}\Lambda^{-1} = \Delta_{\Lambda b}
\end{equation}
a relation trivially verified when we act on a 4-vector $x$.

Our definition of $U$ so far covers translations and Lorentz
group elements only; when we extend to the Poincar\'e group, we
do so through the definition
\begin{equation}%\label{eq:}
U(\Delta_{a}\Lambda) \stackrel{\text{def}}{=} U(\Delta_{a})U(\Lambda)
\end{equation}
We can now see that $U$ is a representation of the Poincar\'e
group if and only if $U$ preserves the action
$\Lambda\Delta_{b}\Lambda^{-1}=\Delta_{\Lambda b}$ of Lorentz
group elements on translations:
\begin{subequations}
\begin{align}
 & U(\Delta_{a}\Lambda)U(\Delta_{b}M) =
  U(\Delta_{a}\Delta_{\Lambda b}\Lambda M) \\
\iff & U(\Delta_{a})U(\Lambda)U(\Delta_{b})U(M) =
U(\Delta_{a})U(\Delta_{\Lambda b})U(\Lambda)U(M)\\
\iff & U(\Lambda)U(\Delta_{b}) = U(\Delta_{\Lambda b})U(\Lambda)\\
\iff & U(\Lambda)U(\Delta_{b})U(\Lambda)^{\dag} = U(\Delta_{\Lambda b})
\end{align}
\end{subequations}
We verify the final condition by evaluating both sides on some
test state $|k\>$. From the right hand side, we have
\begin{equation}%\label{eq:}
U(\Delta_{\Lambda b})|k\> = \exp(i\Lambda b^{\mu}k_{\mu})|k\>
\end{equation}
and from the left hand side
\begin{subequations}
\begin{align}
U(\Lambda)U(\Delta_{b})U(\Lambda)^{\dag}|k\> &=
U(\Lambda)U(\Delta_{b})|\Lambda^{-1}k\>\\
&=
U(\Lambda)\exp(ib^{\mu}{\Lambda_{\mu}}^{\nu}k_{\nu})|\Lambda^{-1} k\>\\
&=\exp(ib^{\mu}{\Lambda_{\mu}}^{\nu}k_{\nu})|k\>.
\end{align}
\end{subequations}
The equality of the two sides follows from the Lorentz-invariance
of the inner product.

We can now summarize our results of $U$ in the following theorem:
\begin{thm}%\label{thm:}
The map $U$ from the Poincar\'e group to operators on the state
space defined by
\begin{subequations}
\begin{align}
U(\Delta_{a})|k\> &= e^{ia^{\mu}k_{\mu}}|k\>\\
U(\Lambda)|k\> &= |\Lambda k\>\\
U(\Delta_{a}\Lambda) &= U(\Delta_{a})U(\Lambda) 
\end{align}
\end{subequations}
is a unitary representation of the Poincar\'e group.
\end{thm}

The unitary representation $U$ is often boasted to successfully
combines the principle (as represented by the Poincar\'e group)
with the principles of quantum mechanics (as represented by
unitary operators and state-space formalisms). This combined
structure of a one-particle state space provides the foundation
for the many-particle state space used in all quantum field theories.


\section{Notes on a Position Operator}
%%
%% positionOperator.tex
%% 
%% Made by Alex Nelson
%% Login   <alex@tomato>
%% 
%% Started on  Sat Jul 25 14:17:02 2009 Alex Nelson
%% Last update Sat Jul 25 14:17:02 2009 Alex Nelson
%%

The astute reader would probably have realized by now we
``implemented'' relativity in the momentum space. The question
that naturally presents itself is ``Why not try to implement
relativity in position-space, as we usually do when introducing
relativity classically?'' In this section, we'll answer that
question. 

The short answer is that it turns out to be inconsistent. We can
sketch out the general scheme and its problem in this paragraph
too. Consider putting a particle (of mass $m$) into a box whose
sides are small compared to the Compton wavelength $\lambda$,
then the uncertainty in position satisfies
\begin{equation}%\label{eq:}
\Delta x\lll\lambda
\end{equation}
and the uncertainty in momentum satisfies
\begin{equation}%\label{eq:}
\Delta p\ggg m.
\end{equation}
But this makes the range of energies so large that pair
production becomes possible. Hence, from first principles, the
position of a one-particle system is not so well defined. We'll
show (slightly more rigorously) that the notion of Lorentz
causality is violated by measuring the position operator.

We first set up the axioms for (properties satisfied by) the
position operator $\widehat{x}^{m}$. We want:
\begin{description}
\item[Axiom 1] $\widehat{x}=\widehat{x}^{\dag}$ (i.e. it's
  self-adjoint, so it has real eigenvalues);
\item[Axiom 2] If $\Delta_{a}$ is a spatial translation, then
  $U(\Delta_{a})^{\dag}\widehat{x}^{m}U(\Delta_{a}) = \widehat{x}^{m}+a^{m}$
\item[Axiom 3] If $R$ is a spatial rotation, then
  $U(R)^{\dag}\widehat{x}^{m}U(R) = {R^{m'}}_{m}\widehat{x}^{m}$.
\end{description}
From axiom 2 and $U(\Delta_{a}) = \exp(ia^{m}\widehat{P}_{m})$,
we deduce
\begin{equation}%\label{eq:}
e^{ia^{m}\widehat{p}_{m}}\widehat{x}^{n}e^{-ia^{m}\widehat{p}_{m}}=\widehat{x}^{n}+a^{n}.
\end{equation}
(Note that the sign in the exponent reflects the relationship
between the Lorentz dot product and the Euclidean dot product of
3-vectors.) Differentiating both sides with respect to the
component $a^{n}$ of $a$ then setting $a^{m}=\vec{0}$, we recover
the usual commutation relations:
\begin{equation}\label{eq:recoveryCanonicalCommutationRelations}
[i\widehat{p}_{n},\widehat{x}^{m}] = {\delta^{m}}_{n}.
\end{equation}

\begin{comment}
Although $\widehat{p}_{n}$ is unbdefined on $|x^{m}\>$, the
axioms for the position operator imply that the exponential
$\exp(-ia^{m}\widehat{p}_{m})$ must be defined on these states:
\begin{equation}%\label{eq:}
e^{-ia^{m}\widehat{p}_{m}}|x^{n}\> = |x^{n}+a^{n}\>.
\end{equation}
Also, if $\<\overline{k}|\overline{x}=\overline{0}\>=1$, then
$\<\overline{k}|\overline{a}\>=\exp(-ia^{m}k_{m})$. 
\end{comment}

\begin{rmk}
The position operator is ``essentially'' unique. That is to say,
it's unique up to unitarity. Suppose we have two operators
$\widehat{y}^{m'}$, $\widehat{x}^{m}$ that satisfy our
axioms. We'll demonstrate that there exists a unitary operator
$U$ such that $\widehat{y}^{m}=U^{\dag}\widehat{x}^{m}U$.

Assume that $\widehat{y}^{m}$ is the position operator with
respect to the basis $|\overline{k}\>$. The canonical commutation
relations eq \eqref{eq:recoveryCanonicalCommutationRelations}
shows that $\widehat{p}_{n}$ commutes with
$\widehat{x}^{n}-\widehat{y}^{n}$. Therefore, supposing any operator can
be expressed using $\widehat{x}^{m}$ and $\widehat{p}_{n}$, we
have
\begin{equation}%\label{eq:}
\widehat{y}^{m} = \widehat{x}^{m}+f^{m}(\widehat{p}).
\end{equation}
Axiom 3 however implies that $f^{m}(\widehat{p})\sim
g(\|\widehat{p}\|^{2})\widehat{p}_{m}$. This vector-valued
function of a vector has zero curl and thus may be written as the
gradient of a scalar function. Lets denote this scalar function
as $\phi(\|\widehat{p}\|^{2})$ where 
\begin{equation}%\label{eq:}
\phi(\xi)\stackrel{\text{def}}{=}\int^{\xi}_{0}g(\eta)d\eta
\end{equation}
If we define our new kets using a unitary operator $U$ to change
phases
\begin{equation}%\label{eq:}
|\overline{k}\>_{\text{new}}\stackrel{\text{def}}{=}U|\overline{k}\>\stackrel{\text{def}}{=}\exp(-i\phi(\|\overline{k}\|^{2}))|\overline{k}\>,
\end{equation}
then since
\begin{equation}%\label{eq:}
\<\psi'|\widehat{y}^{m}|\psi\> = {}_\text{new}\<\psi'|U\widehat{y}^{m}U^{\dag}|\psi\>_\text{new}
\end{equation}
the new operators are $U\widehat{y}^{m}U^{\dag}$. Writing
$U^{\dag}=e^{A}$ we find
\begin{subequations}
\begin{align}
U\widehat{y}^{m}U^{\dag} &= U\left(U^{\dag}\widehat{y}^{m}+[\widehat{y}^{m},U^{\dag}]\right)\\
&= \widehat{y}^{m} +
U[\widehat{y}^{m},1+A+\frac{1}{2}A^{2}+\cdots]\\
&= \widehat{y}^{m} + U(1+A+\frac{1}{2}A^{2}+\cdots)[\widehat{y}^{m},A]\\
&= \widehat{y}^{m} +
[\widehat{y}^{m},i\phi(\|\widehat{p}\|^{2})]\\
&= \widehat{y}^{m} - g(\|\widehat{p}\|^{2})\widehat{p}_{m}\\
&= \widehat{x}^{m}.
\end{align}
\end{subequations}
We therefore conclude that any two sets of position operators
$\widehat{x}^{m}$, $\widehat{y}^{n}$ are related by a change of
basis. We also note since $U$ is a function of the momentum
operators, the new momentum operators $U\widehat{p}_{m}U^{\dag}$
are precisely the old ones $\widehat{p}_{m}$. This shows that the
axioms determining the position operator uniquely up to a choice
of phase in the momentum eigenstates, and this concludes our remark.
\end{rmk}

The simplest inconsistency emerges when we consider a state
initially localized at the origin and see whether it can be
detected outside the forward lightcone of the origin.

Suppose we have a position operator $\widehat{x}^{m}$. Let
$|\overline{x}\>$ be a basis of position eigenstates. Then, form
our knowledge of nonrelativistic quantum mechanics, we can choose
the normalization of these kets to be such that
\begin{equation}%\label{eq:}
\<\overline{x}|\overline{k}\> = \exp(i\overline{x}\cdot\overline{k}).
\end{equation}
Now consider the evolution $|\psi\>$ of a state $|\psi_{0}\>$
initially localized at the origin:
\begin{equation}%\label{eq:}
\psi_{0}(\overline{x})\stackrel{\text{def}}{=}(2\pi)^{3}\delta^{(3)}(\overline{x})\,\Rightarrow\,\widehat{\psi}_{0}(\overline{k})=1\,\Rightarrow\,|\psi_{0}\>=\int|k\>d^{3}\overline{k},
\end{equation}
where $\widehat{\psi}_{0}$ is the Fourioer transform of
$\psi_{0}$. The evolution of this state is given by:
\begin{subequations}
\begin{align}
\psi(t,\overline{x}) &= \<\overline{x}|e^{-iHt}|\psi_{0}\>\\
&= \int \<\overline{x}|e^{-iHt}|\overline{k}\>d^{3}\overline{k}\\
&= \int
\<\overline{x}|e^{-i\omega(\overline{k})t}|\overline{k}\>d^{3}\overline{k}\\
&= \int e^{-i\omega(\overline{k})t}e^{i\overline{x}\cdot\overline{k}}d^{3}\overline{k}.
\end{align}
\end{subequations}
If the theorey is relativistic, then a state initially localized
at the origin should have zero amplitude outside the lightcone
(otherwise, there is a positive probability that something could
travel faster than light). We therefore proceed to estimate
$\psi(t,\overline{x})$ outside the light cone. Using spherical
coordinates, letting $k=\|\overline{k}\|$, $r=\|\overline{x}\|$,
we find that
\begin{subequations}
\begin{align}
\psi(t,\overline{x}) &=
\int^{1}_{-1}d(\cos\theta)\int^{2\pi}_{0}d\phi\int^{\infty}_{0}k^{2}e^{-it\sqrt{k^{2}+\mu^{2}}}e^{ikr\cos\theta}dk\\
&=\frac{2\pi}{ir}\int^{\infty}_{0}ke^{-it\sqrt{k^{2}+\mu^{2}}}(e^{ikr}-e^{-ikr})dk\\
&=\frac{2\pi}{ir}\int^{\infty}_{-\infty}ke^{-it\sqrt{k^{2}+\mu^{2}}}e^{ikr}dk.
\end{align}
\end{subequations}
We can use complex analysis to evaluate this integral when $r>t$,
we deform the contour of integration from $\mathbb{R}$ to the
first principal branch cut from $i\mu$ to $i\infty$. Substituting
$k=iz$, we find
\begin{equation}%\label{eq:}
\psi(t,\overline{x}) = \frac{4\pi i}{r}\int^{\infty}_{\mu}z\sinh(t\sqrt{z^{2}-\mu^{2}})e^{-zr}dz
\end{equation}
which is clearly nonzero.

\begin{rmk}
The integral we've been manipulating is actually divergent. This
is a consequence of the extreme nature of the initial state
$|\psi_{0}\>$. If we had started with a physical state instaead
of a position eigenstate, there would be no convergence
problem. The moral of the story is to treat integrals which arise
in such situations as defining distributions.
\end{rmk}

The outcome is that a position operator is inconsistent with
relativity. This compels us to find another way of modeling
localization of events. In field theory, we do this by making
observable operators dependent on position in spacetime.


\section{Conclusion}
%%
%% conclusion.tex
%% 
%% Made by Alex Nelson
%% Login   <alex@tomato>
%% 
%% Started on  Sat Jul 25 14:56:27 2009 Alex Nelson
%% Last update Sat Jul 25 14:56:27 2009 Alex Nelson
%%

We've reviewed some notions from quantum mechanics, such as the
Rigged Hilbert Space and using unitary operators for
observables. When using representation theory, we need a unitary
representation of a group for use in quantum theory.

We've introduced various aspects of making quantum mechanics
relativistic. The main approach is to take advantage of the fact
that special relativity is basically ``just'' the Poincar\'e
group. We then proceeded to find a unitary representation of the
Lorentz group and the group of spacetime translations, then
combined them in a suitably nice way.

We've considered the situation of making the position operator
relativistic, and concluded after a few naive attempts that it
wouldn't work. 

The interested reader is free to peruse the resources cited in
the bibliography for further reading (specifically, the notion of
measurement relative to an observer is tackled beautifully in
Gambini and Porto~\cite{Gambini:2001pq}).


\nocite{*}
\bibliographystyle{utcaps}
\bibliography{rqm}
\end{document}

\chapter{Constrained Hamiltonian Mechanics}

\section{Constraints}

\M
Recall we can write down the Euler--Lagrange equations for a mechanical
system of point particles as:
\begin{equation}
\frac{\partial L}{\partial q^{i}} - \frac{\D}{\D t}\frac{\partial L}{\partial\dot{q}^{i}}=0,
\end{equation}
where $i=1,\dots,N$.
We can expand the total time derivative as
\begin{equation}
\frac{\partial L}{\partial q^{i}}
- \left(\ddot{q}^{j}\frac{\partial}{\partial\dot{q}^{j}}\frac{\partial L}{\partial\dot{q}^{i}}
+\dot{q}^{j}\frac{\partial}{\partial q^{j}}\frac{\partial L}{\partial\dot{q}^{i}}\right)=0.
\end{equation}
Therefore we can write this as:
\begin{equation}
\ddot{q}^{j}\frac{\partial}{\partial\dot{q}^{j}}\frac{\partial L}{\partial\dot{q}^{i}}
=\frac{\partial L}{\partial q^{i}}
-\dot{q}^{j}\frac{\partial}{\partial q^{j}}\frac{\partial L}{\partial\dot{q}^{i}}.
\end{equation}
If the matrix $\partial^{2}L/\partial\dot{q}^{i}\partial\dot{q}^{j}$ is
invertible, then we can uniquely determine the accelerations as a
function of positions and velocities.

\M When this matrix is singular
(i.e., non-invertible), what happens? Well, recall the definition of the
canonically conjugate momenta:
\begin{equation}
p_{i} = \frac{\partial L}{\partial\dot{q}^{i}}.
\end{equation}
Its Jacobian matrix is
\begin{equation}
\frac{\partial p_{i}}{\partial\dot{q}^{j}} = \frac{\partial^{2} L}{\partial\dot{q}^{j}\partial\dot{q}^{i}}.
\end{equation}
When our desired matrix is not invertible, then it's logically
equivalent to having the Jacobian for the canonically conjugate momenta
being non-invertible. This means the momenta are not all independent of
each other; instead we have some relations
\begin{equation}
\phi_{m}(q,p)=0
\end{equation}
for $m=1,\dots,M$.

\begin{remark}
We will assume that the $\phi_{m}$ are independent of each other.
\end{remark}

\begin{definition}
The conditions $\phi_{m}(q,p)=0$ obtained from the definition of the
conjugate momenta are called \define*{Primary Constraints}.\index{Constraint!Primary|textbf}\index{Primary constraint|see{Constraint}}
The submanifold defined by these equations in the phase space is called
the \define*{Primary Constraint Surface}\index{Constraint!Primary!Surface}\index{Constraint!Surface}.
\end{definition}

\begin{remark}
If we permute/reindex the velocities so the first $N-M$ canonically
conjugate momenta \emph{are} independent of each other, then the primary
constraints will take the form:
\[\phi_{m}=p_{N-M+m}-f_{m}(q^{i},p_{j'}),\]
where $j'=1,\dots,N-M$ ranges over the independent momenta.
\end{remark}

\begin{remark}
The word ``primary'' here refers to the fact that we obtained them from
trying to obtain the canonically conjugate momenta. Later we will find
``secondary'' constraints by demanding, roughly, $\PB{\phi_{m}}{H}=0$ on
the constraint surface. This invokes the equations of motion on the
primary constraints, which gives us ``secondary'' constraints if the
Poisson bracket is nonzero.
\end{remark}

\N{Canonical Hamiltonian}
The next step in the Hamiltonian analysis is to determine the
Hamiltonian by means of the Legendre transform:
\begin{equation}
H = \dot{q}^{i}p_{i} - L.
\end{equation}

\begin{lemma}
The canonical Hamiltonian $H$ has velocities $\dot{q}^{i}$ enter through
the momenta $p_{i}(q,\dot{q})$ alone.
\end{lemma}

\begin{proof}
We consider the change $\variation H$ induced by arbitrary independent
variations of the positions and velocities
\begin{calculation}
\variation H
\step{linearity of $\variation$}
\variation(\dot{q}^{i}p_{i}) - \variation L
\step{product rule}
\dot{q}^{i}\,\variation p_{i} + p_{i}\,\variation\dot{q}^{i} - \variation L
\step{expanding $\variation L$ using partial derivatives}
\dot{q}^{i}\,\variation p_{i} + p_{i}\,\variation\dot{q}^{i} 
- \variation\dot{q}^{i}\frac{\partial L}{\partial\dot{q}^{i}}
- \variation q^{i}\frac{\partial L}{\partial q^{i}}
\step{collecting coefficients of $\variation\dot{q}^{i}$}
\dot{q}^{i}\,\variation p_{i} + \left(p_{i} 
- \frac{\partial L}{\partial\dot{q}^{i}}\right)\variation\dot{q}^{i}
- \variation q^{i}\frac{\partial L}{\partial q^{i}}
\step{using the definition of the conjugate momentum}
\dot{q}^{i}\,\variation p_{i} + 
- \frac{\partial L}{\partial q^{i}}\variation q^{i}.
\end{calculation}
Hence
\begin{equation}
\variation H = 
\dot{q}^{i}\,\variation p_{i} + 
- \frac{\partial L}{\partial q^{i}}\variation q^{i},
\end{equation}
establishing the result, since $\variation p_{i}$ is not an independent
variation but really is a combination of the $\variation q$'s and
$\variation\dot{q}$'s (and therefore only through $\variation p_{i}$ can
$\variation\dot{q}$ can enter into $\variation H$). 
\end{proof}

\N{Problem: Non-Uniqueness of Hamiltonian}
We should observe that the momenta are not all independent of each
other, so the variations $\variation p_{i}$ are not all independent of
each other. Instead the $\variation p_{i}$ are restricted to preserve
the primary constraints $\phi_{m} = 0$.

We are forced to conclude the canonical Hamiltonian is well-defined only
on the submanifold defined by the primary constraints, but can be
extended arbitrarily off that manifold.

It follows that the formalism should remain unchanged by the replacement
\begin{equation}\label{eq:constrained-dynamics:constraints:hamiltonian-remains-unchanged-by-replacement}
H\to H + c^{m}(q,p)\phi_{m}
\end{equation}
for arbitrary functions $c^{m}$.

\begin{theorem}[{Henneaux and Teitelboim~\cite[Th1.2]{Henneaux:1992ig}}]
If $\lambda_{i}\,\variation q^{i} + \mu^{i}\,\variation p_{i}=0$ for
arbitrary variations $\variation q^{i}$, $\variation p_{i}$ tangent to
the constraint surface, then we have
\begin{subequations}
\begin{align}
\lambda_{i} &= u^{m}\frac{\partial\phi_{m}}{\partial q^{i}}\\
\mu^{i} &= u^{m}\frac{\partial\phi_{m}}{\partial p_{i}}
\end{align}
\end{subequations}
for some $u^{m}$. The equalities here are equalities on the constraint surface.
\end{theorem}

\M
Working on Eq~\eqref{eq:constrained-dynamics:constraints:hamiltonian-remains-unchanged-by-replacement},
we can rewrite it as:
\begin{equation}
\left(\frac{\partial H}{\partial q^{i}} + \frac{\partial L}{\partial q^{i}}\right)\variation q^{i}
+\left(\frac{\partial H}{\partial p_{i}} -\dot{q}^{i}\right)\variation p_{i}=0,
\end{equation}
from which we now apply our preceding theorem to
find:\marginnote{Introduce $u^{m}$}
\begin{subequations}
\begin{align}
\dot{q}^{i} &= \frac{\partial H}{\partial p_{i}} + u^{m}\frac{\partial\phi_{m}}{\partial p_{i}}\\
\left.-\frac{\partial L}{\partial q^{i}}\right|_{\dot{q}}
&=\left.\frac{\partial H}{\partial q^{i}}\right|_{p}+u^{m}\frac{\partial\phi_{m}}{\partial q^{i}}.
\end{align}
\end{subequations}
The first of these equations allows us to recover the velocities
$\dot{q}^{i}$ from knowing (1) the momenta $p_{i}$ (obeying the primary
constraints $\phi_{m}=0$) and (2) the extra parameters $u^{m}$. These
extra parameters can be thought of as coordinates on the surface of the
inverse images of a given $p_{i}$.

\N{Determining $u$'s}.
If the constraints are independent, then the vectors
$\partial\phi_{m}/\partial p_{i}$ are also independent on the constraint
surface $\phi_{m}=0$ because of the regularity conditions. Hence no two
different sets of the $u$'s can yield the same velocities. This means
that the $u$'s can be expressed, in principle, as functions of the
coordinates and velocities by solving the equations
\begin{equation}
\dot{q}^{i}
= \frac{\partial H}{\partial p_{i}}\bigl(q,p(q,\dot{q})\bigr)
+ u^{m}(q,\dot{q})\frac{\partial\phi_{m}}{\partial p_{i}}\bigl(q,p(q,\dot{q})\bigr).
\end{equation}

% Exercise 1.1(a)

\N{Equations of Motion using Poisson Bracket}\label{chunk:constrained-hamiltonian:constraint:initial-form-of-equations-of-motion-using-poisson-brackets}
We have, for any phase-space function $F(q,p)$, its time derivative:
\begin{equation}
\dot{F} = \PB{F}{H} + u^{m}\PB{F}{\phi_{m}}.
\end{equation}
This is the usual Poisson bracket.

\N{Consistency of Constraints}
We will want the primary constraints be preserved in time
$\dot{\phi}_{m}=0$, which can be expressed using the Poisson brackets as
\begin{equation}
\PB{\phi_{m}}{H} + u^{m'}\PB{\phi_{m}}{\phi_{m'}}=0.
\end{equation}
There are two possible consequences to this condition:
\begin{enumerate}
\item it reduces to a relation independent of the $u$'s (thus involving
  only the $q$'s and $p$'s), or
\item it may impose a restriction on the $u$'s.
\end{enumerate}
The first case, if the relation is independent of the primary
constraints, then we have a \define*{Secondary Constraint}\index{Constraint!Secondary}.
\index{Secondary constraint|see {Constraint}}

\M
If there is a secondary constraint, call it $C(q,p)=0$ for illustration,
then we must impose a new consistency condition
\begin{equation}
\PB{C}{H} + u^{m}\PB{C}{\phi_{m}}=0.
\end{equation}
As before, we must check whether this implies a new secondary constraint
or whether it restricts the $u$'s. After this process is finished, we
are left with a number of secondary constraints, which we denote
by\marginnote{Secondary constraints $\phi_{k}$}
\begin{equation}
\phi_{k}=0
\end{equation}
for $k=M+1$, \dots, $M+K$ where $K$ is the total number of secondary
constraints. 

We then have all the constraints (primary and secondary) written as
\begin{equation}
\phi_{j} = 0,
\end{equation}
for $j=1$, \dots, $J=M+K$.

\begin{remark}
Some authors are painfully pedantic, calling ``secondary constraints''
only those constraints emerging from the consistency of ``primary constraints''.
The constraints emerging from the consistency of primary and secondary
constraints are dubbed ``tertiary constraints''. Presumably demanding
consistency of the primary, secondary, and tertiary constraints gives us
quaternary constraints. We then end up with quinary, senary, septenary,
octonary, nonary, denary constraints (and hopefully no more).

However, as we will see soon, this division into primary, secondary,
etc., constraints is not physically fundamental. There is another way to
classify constraints due to Dirac.
\end{remark}

\begin{definition}
Let $F$, $G$ be two phase space functions. Then we say $F$ and $G$ are
\define{Weakly Equal} and write $F\weakEq G$ if and only if they
coincide on the constraint manifold $\phi_{j}=0$. We refer to the symbol
``$\weakEq$'' as the \emph{weak equality symbol}.

If $F$, $G$ coincide \emph{everywhere} in the phase space, then we say
$F$ and $G$ are \define{Strongly Equal} and indicate this as $F = G$
with the usual equality symbol.
\end{definition}

\begin{theorem}[{Henneaux and Teitelboim~\cite[Th1.2]{Henneaux:1992ig}}]
If $F$ and $G$ are two phase space functions, then
\begin{equation}
F\weakEq G\iff F-G = c^{j}(q,p)\phi_{j}.
\end{equation}
\end{theorem}

\N{Re-phrasing Consistency Condition: Restricting $u^{m}$}
We have a complete set of constraints $\phi_{j}\weakEq0$, and we want to
study the restrictions on the Lagrange multipliers $u^{m}$. These
restrictions are
\begin{equation}
\PB{\phi_{j}}{H} + u^{m}\PB{\phi_{j}}{\phi_{m}}\weakEq0,
\end{equation}
where we sum $m$ from $1$ to $M$, and $j$ is an index taking any value
from $1$ to $J$. This describes a system of $J$ linear equations in the
$M\leq J$ unknowns $u^{m}$ with coefficients which are functions of the
$q$'s and $p$'s. These equations should have solutions, otherwise the
Lagrangian description would be inconsistent.

The general solution to our system of linear equations is of the form
\begin{equation}
u^{m} = U^{m} + V^{m}
\end{equation}
where $U^{m}$ is a particular solution of the inhomogeneous system of
equations, and $V^{m}$ is the most general solution to the associated
homogeneous system
\begin{equation}
V^{m}\PB{\phi_{j}}{\phi_{m}}\weakEq0.
\end{equation}
The most general $V^{m}$ is a linear combination of linearly independent
solutions ${V_{a}}^{m}$ for $a=1$, \dots, $A$. The number $A$ of
independent solutions is the same for all $q$, $p$ on the constraint
surface, because we assumed the rank of $\PB{\phi_{j}}{\phi_{m}}$ is
constant on the constraint surface.
Then we can write
\begin{equation}
u^{m}\weakEq U^{m} + v^{a}{V_{a}}^{m},
\end{equation}
where the coefficients $v^{a}$ \emph{are totally arbitrary}.

Hence we have separated $u^{m}$ into the part $U^{m}$ determined by the
consistency conditions and separately the totally arbitrary part.

\begin{definition}\index{Hamiltonian!First-class}\index{First-class!Hamiltonian}
We define the \define*{First-Class Hamiltonian} to be the function
\begin{equation}
H' := H + U^{m}\phi_{m}.
\end{equation}
\end{definition}

\N{Equations of Motion}
We can revisit the equations of motion (\S\ref{chunk:constrained-hamiltonian:constraint:initial-form-of-equations-of-motion-using-poisson-brackets})
to write the time derivative of any phase space function $F$ as
\begin{equation}
\dot{F}\weakEq\PB{F}{H' + v^{a}\phi_{a}}
\end{equation}
where $\phi_{a} = {V_{a}}^{m}\phi_{m}$.

To see this, we have used
\begin{equation}
\PB{F}{U^{m}\phi_{m}} = U^{m}\PB{F}{\phi_{m}}+\PB{F}{U^{m}}\phi_{m}
\weakEq U^{m}\PB{F}{\phi_{m}},
\end{equation}
and similar manipulations for $\PB{F}{{V_{a}}^{m}\phi_{m}}\weakEq{V_{a}}^{m}\PB{F}{\phi_{m}}$.

\begin{definition}\index{Hamiltonian!Total|textbf}
We define the \define*{Total Hamiltonian} $H_{T} = H' + v^{a}\phi_{a} = H + u^{m}\phi_{m}$.
\end{definition}

\begin{remark}
The total Hamiltonian recovers the Lagrangian dynamics, but in terms of
canonical variables.
\end{remark}

\M
We can then describe the equations of motion using the total Hamiltonian
as
\begin{equation}
\dot{F}\weakEq\PB{F}{H_{T}}.
\end{equation}
These equations are equivalent, by construction, to the original
Lagrangian equations of motion.

\begin{definition}[{Henneaux and Teitelboim~\cite[\S1.1.10]{Henneaux:1992ig}}]
We call a phase-space function $F$ \define*{First-Class}\index{First-class!Function|textbf} if its Poisson
bracket with every constraint weakly
vanishes: $\PB{F}{\phi_{j}}\weakEq0$ for every $j=1,\dots,J$.
Otherwise, if $F$ is not first-class, we call it \define*{Second-Class}\index{Second-class!Function|textbf}.
\end{definition}

\M Observe that the total Hamiltonian is the sum of the first-class
Hamiltonian $H'$ and the first-class primary constraints multiplied by
arbitrary coefficients. This splitting of $H_{T}$ into $H'$ and
$v^{a}\phi_{a}$ is not unique because $U^{m}$ appearing in this
decomposition can be \emph{any solution} to a (possibly underdetermined)
system of linear equations.

\begin{theorem}
If $F$ and $G$ are first-class functions on the phase space, then their
Poisson bracket $\PB{F}{G}$ is also a first-class function.
\end{theorem}

\begin{proof}[Proof sketch]
The proof is a straightforward calculation, using
$\PB{F}{\phi_{j}}={f_{j}}^{j'}\phi_{j'}$
and
$\PB{G}{\phi_{j}}={g_{j}}^{j'}\phi_{j'}$, then using the Jacobi identity
to write $\PB{\PB{F}{G}}{\phi_{j}}=\PB{F}{\PB{G}{\phi_{j}}}-\PB{G}{\PB{F}{\phi_{j}}}$.
This will turn out to be a linear combination of constraints, which
proves the theorem.
\end{proof}

\N{Gauge Transformations}
We can transform a function $F$ using an arbitrary variation of the
arbitrary functions $\variation v^{a}=\overline{v}^{a}-v^{a}$ as
\begin{equation}
\variation F = \variation v^{a}\PB{F}{\phi_{a}}.
\end{equation}
These are precisely gauge transformations, i.e., transformations which
do not change the physical state of the system.

\part{Fields}
\chapter{Classical Fields}

\M
The general idea is that we will review/introduce classical fields, then
appeal to an heuristic quantization procedure to obtain quantum fields.
The motivation is a careful derivation of the linear chain, where we
have an infinite number of identical point-masses connected by identical
massless springs. Taking the continuum limit gives us the scalar field.

%\includegraphics{img/img.0}
\section{Linear Chain}

\N{Problem Statement}
Consider $N\in\NN$ identical point-masses (each with rest mass $m$)
each of which are connected to two neighboring point masses by identical
massless springs with spring constant $k$ and equilibrium length $a$.
The point-masses then form a line segment.

We will take $N\to\infty$ limit in such a way that at any point-mass
there are infinitely many point-masses in either direction.
This specifically is to let us ignore the boundary conditions.

What are the equations of motion for a point-mass in this chain? What is
the Lagrangian for this system?

Take the continuum limit where $a\to0$ while $a^{2}k/m$ is held
constant. What happens to the equations of motion and the Lagrangian?

\begin{exercise}
What dimensions does $a^{2}k/m$ have?
\end{exercise}

\N{Coordinates}
Since we have the point-masses form a one-dimensional system, we will
write $x_{j}$ for the position of the point-mass with $j\in\ZZ$.

\N{Free Body Diagram}
Suppose we examine the free-body diagram for the point-mass. The only
forces acting on a point-mass $x_{j}$ are the spring forces:
\begin{center}
\includegraphics{img/img.0}
\end{center}

\N{Equations of Motion}
Then we see the force acting on $x_{j}$ is
\begin{equation}
\begin{split}
  F_{j} &= -k(x_{j}-x_{j-1}-a) + k(x_{j+1}-x_{j}-a)\\
  &= k(x_{j+1}-2x_{j}+x_{j}).
\end{split}
\end{equation}
Using Newton's second Law,
\begin{equation}
m\ddot{x}_{j} = F_{j} = k(x_{j+1}-2x_{j}+x_{j}).
\end{equation}
We will rearrange this to:
\begin{equation}\label{eq:classical-field-theory:linear-chain:newton-eom}
\ddot{x}_{j} = \frac{k}{m}(x_{j+1}-2x_{j}+x_{j}).
\end{equation}


\N{Lagrangian}
We can then write the Lagrangian for this system,
\begin{equation}
L = \sum_{j\in\ZZ}\frac{m}{2}\dot{x}^{2}_{j} - \frac{k}{2}(x_{j+1}-x_{j})^{2}.
\end{equation}
Since the sum is over all integers, the forces acting on $x_{j}$ come
from the $j-1$ term and the $j$ term.

\N{Continuum Limit}
Now care must be taken, because as $a\to 0$ the index $j$ labeling
particles will become a real number indicating the position of the
particle. To avoid ambiguity, we will write $q_{j}(t)$ for the position
of particle $j$.

We observe as $a\to0$, we have $q_{j}(t)\to q(x,t)$
\begin{equation}
\frac{x_{j+1}-2x_{j}+x_{j-1}}{a^{2}}\xrightarrow{a\to0}\frac{\partial^{2}}{\partial x^{2}}q(x,t).
\end{equation}
Then the continuum limit of the equations of motion,
Eq~\eqref{eq:classical-field-theory:linear-chain:newton-eom}, (first
dividing through by $m$) is:
\begin{equation}
\ddot{x}_{j}\xrightarrow{a\to0}\frac{\partial^{2}}{\partial t^{2}}q(x,t),
\quad\mbox{and}\quad\frac{ka^{2}}{m}\frac{x_{j+1}-2x_{j}+x_{j-1}}{a^{2}}
\xrightarrow{a\to0}v^{2}
\frac{\partial^{2}}{\partial x^{2}}q(x,t).
\end{equation}
Then equating both limits gives us:
\begin{equation}
\frac{\partial^{2}}{\partial t^{2}}q(x,t) = v^{2}
\frac{\partial^{2}}{\partial x^{2}}q(x,t).
\end{equation}
This is precisely the wave equation for an elastic string.
Here $v=\sqrt{a^{2}k/m}$ is the velocity of propagation.

\begin{exercise}
We have been working with one spatial dimension, assuming it is $\RR$.
Suppose space is a circle $S^{1}$ and our linear chain forms a
ring. Perform the continuum limit analysis for this situation.
\end{exercise}

\begin{exercise}
If we took space to be a closed interval $[a,b]$ instead of $\RR$,
then what boundary conditions do we need to impose for things to work
out in the continuum limit?
\end{exercise}

\N{Canonical Analysis}
We can perform the Legendre transform of the Lagrangian, first finding
the conjugate momenta
\begin{equation}
p_{j} = \frac{\partial L}{\partial\dot{q}_{j}} = m\dot{q}_{j}.
\end{equation}
Then
\begin{equation}
H = \sum_{j\in\ZZ}\frac{p_{j}^{2}}{2m} + k(q_{j+1}-q_{j})^{2}.
\end{equation}

\chapter{Outline of Quantum Field Theory}

\M
The big idea is that we want to describe scattering using quantum
calculations, involving different types of particles and different types
of interactions [fields], using different ``mathematical toolkits'' and
pictures [sum over histories, functional Schr\"{o}dinger, etc.].
Consequently any thorough textbook would walk through 9 calculations:
for each of the three toolkits [Heisenberg/interaction, path integral,
functional Schr\"{o}dinger], we compute the propagator and scattering
of the three particles [scalar, ``spinor'' (spin-$1/2$), and vector/gauge].


\section{Scattering}

\N{Scattering}
The basic idea is we have $2\to2$ scattering\footnote{Decay may be
interpreted as $1\to n$, and other situations may be described
analogously.}  (i.e., we collide two particles towards each other, and
then two particles emerge from the ``collision'') and we try to measure
something in the laboratory. Usually this is the scattering angle
$\theta$, which is related to the cross-sectional area $\sigma$ by some
equation of the form
\begin{equation}
\frac{\D\sigma}{\D\cos\theta}\sim f(\cos\theta).
\end{equation}
Quantum theory permits us to write an equation relating $\D\sigma$ to
entries of the $S$-matrix. The problem for the theorist is to compute
these $S$-matrix components.

\M
The LSZ formula relates $S$-matrix theory to quantum field theory.
Specifically, the components of the $S$-matrix
\begin{equation}
S_{f,i} = \langle f|S|i\rangle
\end{equation}
may be related to the asymptotic free field via the LSZ formula.

\N{Adiabatic ``Theorem''}
The vacuum state used in the LSZ formula is $|\Omega\rangle$, the vacuum
state for the interacting theory --- compared to the $|0\rangle$ vacuum
state for the free theory. 

The assumption is that $|\Omega\rangle$ is a perturbation of
$|0\rangle$, so they can be related. This is carried out in Peskin and
Schroeder (chapter 4, section 2; see esp.\ Eq~(4.27)).

\N{Feynman Diagrams}
We expand the LSZ formula perturbatively, and organize the terms using
Feynman diagrams describing different interactions.

\section{Particles and Fields}

\N{Deriving the Scalar Field}
We derive the scalar field by considering point masses (of identical
mass $m$) connected by an array of identical springs in each
dimension. When we take the ``spacing goes to zero'' limit, we obtain a
continuum expression which corresponds to the Lagrangian density for the
scalar field. This is the intuition for what a field looks like. When we
``quantize'' the field, we use the quantum harmonic oscillator, and
obtain the Klein--Gordon [free scalar] field.

This is cute, but usually particles in quantum field theory can be
neatly derived from studying irreducible representations of the Poincar\'{e} Group.

\N{Lorentz Group and Algebra}\marginpar{In $-+++$ signature}
Consider proper orthochronous Lorentz transformations
$\Lambda\in\ISO(1,3)\subset\O(1,3)$ such that $\det(\Lambda)=+1$ and
${\Lambda^{0}}_{0}=+1$. Then we can write any element of this group as
\begin{equation}
{\Lambda^{\mu}}_{\nu} = [\exp\left(\frac{-\I}{2}\omega_{\kappa\lambda}M^{\kappa\lambda}\right)]{{}^{\mu}}_{\nu}
\end{equation}
where $\omega_{\kappa\lambda}=-\omega_{\lambda\kappa}$ are ``rotation
angles'' (real constants parametrizing the symmetry) and
$M^{\kappa\lambda}$ is an indexed family of matrices (i.e., fix a value
of $\kappa$ and $\lambda$, and you get a $4\times4$ matrix). These $M^{\kappa\lambda}$ are
generators of the Lie algebra for the Lorentz group. Explicitly
\begin{equation}
(M^{\kappa\lambda})_{\mu\nu} = \I(\delta^{\kappa}_{\mu}\delta^{\lambda}_{\nu}-\delta^{\kappa}_{\nu}\delta^{\lambda}_{\mu})
\end{equation}
Now the trick is that we can write the generators of the Lorentz Lie
algebra using
\begin{subequations}
\begin{align}
L^{i} &= \frac{1}{2}\epsilon^{ijk}M_{jk}\\
\intertext{for spatial rotations, and}
K^{i} &= M^{0i}\\
\intertext{for Lorentz boosts. We define}
\vec{J}_{\pm} &= \frac{1}{2}(\vec{L}\pm\I\vec{K}).
\end{align}
\end{subequations}
The reader may verify the commutation relations become
\begin{equation}
[J^{i}_{\pm}, J^{j}_{\pm}] = \I\epsilon^{ijk}J^{k}_{\pm}.
\end{equation}
But now look, this is precisely two copies of $\su(2)$ (more precisely,
it is $\sl(2,\CC)$).

The punchline, however, is: \textit{Each irreducible representation of $\so(1,3)$
is characterized by a pair of half-integers $(j_{+}, j_{-})$.} We can
interpret these irreducible representations as particles, summarized by
the handy-dandy table:

\begin{center}
\begin{tabular}{c|c|c}
  $(j_{+}, j_{-})$ & Name of Field & Dimension of Rep \\\hline
  $(0, 0)$ &	Scalar  &	1\\
$(1/2, 0)$ & 	Left-handed Weyl Spinor &	2\\
$(0, 1/2)$ &	Right-handed Weyl Spinor &	2\\
$(1, 0)$ &	(Imaginary) Self-dual 2-form &	3\\
$(0, 1)$ &	(Imaginary) Anti-self-dual 2-form &	3\\
$(1/2, 1/2)$ &	Vector (gauge field) &	4\\
$(1/2, 1)$ & 	Left-Handed Rarita-Schwinger field &	6\\
$(1, 1/2)$ &	Right-Handed Rarita-Schwinger field &	6\\
$(1, 1)$ &	Graviton (spin-2 field) &	9
\end{tabular}
\end{center}

\M
We study the scalar, the Dirac spinor $(1/2, 0)\oplus(0, 1/2)$, and
Vector fields specifically, since these are the necessary ingredients
for the Standard Model (and they are renormalizable fields).

\subsection{Yang--Mills Theory}

\N{Global and Local Symmetries}
Textbooks usually begin by studying ``global symmetries'', which do not
depend on spacetime coordinates. For example, if we have $N$ real scalar
fields, then we may put them into a column vector, and rotate by some
orthogonal $N\times N$ matrix. This works because the kinetic and
potential terms of the Lagrangian involve the norm squared of these
$N$-vectors, which are invariant under such rotations.

Physicists then ``gauge'' these symmetries and make them ``local''. But
then the kinetic terms will end up with derivatives of the rotation
matrix. These are then ``gauged away'' by changing the differential
operator.

\N{Yang--Mills Theory}
Another way to approach this is to start with electromagnetism, which
involves the electromagnetic 4-potential $A_{\mu}$. Then we consider
some Lie algebra $\mathfrak{g}$ and work with Lie algebra-valued
4-potentials $A_{\mu}^{I}T_{I}$ where $T_{I}$ are the generators of the
Lie algebra. We compute the field tensor:
\begin{equation}
F_{\mu\nu}^{I} = \partial_{\mu}A^{I}_{\nu} - \partial_{\nu}A^{I}_{\mu}
+g{f^{I}}_{JK}A^{J}_{\mu}A^{K}_{\nu}
\end{equation}
where we use the Lie bracket to determine the structure constants
${f^{I}}_{JK}$ by:
\begin{equation}
[T_{J}, T_{K}] = \I {f^{I}}_{JK}T_{I}.
\end{equation}
We usually work with $\su(n)$ as our Lie algebra, since $\su(3)$
describes the strong force, and $\su(2)\times\mathfrak{u}(1)$ describes
the electroweak forces.

\textsc{Cautionary Note}: a lot of books get confused over indices of
the Lie algebra, and use a bizarre Euclidean summation convention for
Lie algebra indices (but Einstein summation convention for spacetime
indices). Weinberg carefully works through the correct summation
conventions in his book \textit{The Quantum Theory of Fields}
(volume II, \S15.1; see also volume I, \S2.2).

\N{Problems with Massive Yang--Mills}
If we try to add a nonzero mass to a non-Abelian Yang--Mills theory,
then we sacrifice either renormalizability or unitarity; Delbourgo and
friends argued this first~\cite{Delbourgo:1987np}.
Ellwanger and Wschebor~\cite{Ellwanger:2002sj} constructed a small
counter-example in $\su(2)$ by modifying BRST variations, working in a
particular gauge.

If we give up unitarity, we basically give up probabilities adding up to
$100\%$. On the other hand, nonrenormalizable fields have ``runaway
self-interactions'' which lead to infinities.

It's also worth mentioning that the mass term is not gauge-invariant,
which causes its own special Hell.

\N{Confinement} For quarks in the strong force, they experience a
phenomenon called ``confinement'' which we define as stating \emph{there
is a force between quarks which do not decrease with distance.} See
Esprieu~\cite{Espriu:1994br}, especially \S7. As a consequence, we will
not be able to find an isolated quark.

\N{BRST Symmetry}
When quantizing the Standard Model (really, Yang--Mills theory) we run
into problems. The trick is to extend the phase space, embedding it as
the even part of a supermanifold, then we replace the gauge symmetry
with a rigid fermionic symmetry. This is the heart of BRST quantization
and handles technical difficulties when quantizing non-abelian Yang--Mills.
This is how Henneaux and Teitelboim~\cite[\S18.1]{Henneaux:1992ig}
describe the BRST procedure.

\chapter{Grand Unified Theories}

\M
Graham Ross's \textit{Grand Unified Theories} is one of the only books
on the subject; it was originally written in 1985, and much has changed
since then.

There is also Rabindra Mohapatra's \textit{Unification and Supersymmetry: The Frontiers of Quark-Lepton Physics},
which is more recent.

Baez and Heurta~\cite{Baez:2009dj} review of the algebra behind grand
unified theories.

\appendix
\part{Appendices}
\chapter{Calculus}
\section{Integrals}

\begin{theorem}[Gaussian Integral]
$\displaystyle\int^{\infty}_{-\infty}\E^{-x^{2}}\,\D x = \sqrt{\pi}$.
\end{theorem}

\begin{proof}
  Let
  \begin{equation}
I = \int^{\infty}_{-\infty}\E^{-x^{2}}\,\D x.
  \end{equation}
  Then
\begin{calculation}
  I^{2}
  \step{unfold}
\int^{\infty}_{-\infty} \int^{\infty}_{-\infty}\E^{-x^{2}-y^{2}}\,\D x\,\D y
\step{change to polar coordinates}
\int^{\infty}_{0}\int^{2\pi}_{0}\E^{-r^{2}}r\,\D\theta\,\D r
\step{since integrand is constant with respect to $\theta$}
2\pi\int^{\infty}_{0}\E^{-r^{2}}r\,\D r
\step{change coordinates with $u=r^{2}$, $\D u = 2r\,\D r$}
\pi\int^{r=\infty}_{r=0}\E^{-u}\,\D u
\step{simple integration}
\left.-\pi\E^{-r^{2}}\right|^{r=\infty}_{r=0}
\step{substitution}
-\pi\E^{-\infty}+\pi\E^{-0} = \pi.
\end{calculation}
Therefore $I=\pm\sqrt{\pi}$. Since the integrand is always positive, we
conclude $I=\sqrt{\pi}$ as desired.
\end{proof}

\begin{corollary}\label{cor:math:general-gaussian-integral-in-one-dim}
Let $a\in\RR$ and $b\in\CC$. Then $\displaystyle\int^{\infty}_{-\infty}\E^{-a(x-b)^{2}}\,\D x = \sqrt{\frac{\pi}{a}}$
\end{corollary}

\begin{remark}
We can let $a\in\CC$ provided its real part is positive (so $\Re(-ax^{2})\leq0$).
\end{remark}

\begin{corollary}
Let $a\in\RR$ be positive $a>0$, and $b$, $c\in\CC$. Then $\displaystyle\int^{\infty}_{-\infty}\E^{-ax^{2}+bx+c} \,\D x = \sqrt{\frac{\pi}{a}}\exp\left(\frac{b^{2}}{4a}+c\right)$.
\end{corollary}

\begin{corollary}
Let $a\in\RR$ be positive $a>0$, $n\in\NN$. Then
\begin{equation*}
\int^{\infty}_{-\infty}x^{2n}\E^{-ax^{2}/2}\,\D x = \sqrt{\frac{2\pi}{a}}\frac{(2n-1)!!}{a^{n}}.
\end{equation*}
The odd moments all vanish.
\end{corollary}

\begin{proof}
  By induction on $n\in\NN$. We see the base case $n=1$ is obtained from:
  \begin{calculation}
    \int^{\infty}_{-\infty}x^{2}\E^{-ax^{2}/2}\,\D x
\step{differentiating under the integral sign}
    -2\frac{\D}{\D a}\int^{\infty}_{-\infty}\E^{-ax^{2}/2}\,\D x
\step{using the Gaussian integral result}
    -2\frac{\D}{\D a}\left(\frac{2\pi}{a}\right)^{1/2}
\step{power rule for derivatives}
    -2\frac{-1}{2}\frac{1}{a}\left(\frac{2\pi}{a}\right)^{1/2}
\step{algebra}
    \frac{1}{a}\left(\frac{2\pi}{a}\right)^{1/2}.
  \end{calculation}
  This establishes the result.

  The inductive hypothesis is that, for arbitrary $n\in\NN$, 
\begin{equation}
\int^{\infty}_{-\infty}x^{2n}\E^{-ax^{2}/2}\,\D x = \sqrt{\frac{2\pi}{a}}\frac{(2n-1)!!}{a^{n}}.
\end{equation}
Now we prove the $n+1$ case
\begin{calculation}
\int^{\infty}_{-\infty}x^{2(n+1)}\E^{-ax^{2}/2}\,\D x
\step{rewrite using $x^{2(n+1)}=x^{2n}x^{2}$}
\int^{\infty}_{-\infty}x^{2}x^{2n}\E^{-ax^{2}/2}\,\D x
\step{using differentiation under the integral sign}
-2\frac{\D}{\D a}\int^{\infty}_{-\infty}x^{2n}\E^{-ax^{2}/2}\,\D x
\step{using the inductive hypothesis}
-2\frac{\D}{\D a}\left(\sqrt{\frac{2\pi}{a}}\frac{(2n-1)!!}{a^{n}}\right)
\step{using the derivative of $a^{-(2n+1)/2}$}
-2\frac{-(2n+1)}{2}\frac{1}{a}\left(\sqrt{\frac{2\pi}{a}}\frac{(2n-1)!!}{a^{n}}\right)
\step{since $(2n+1)!! = (2n + 1)\cdot (2n-1)!!$}
-2\frac{-1}{2}\frac{1}{a}\left(\sqrt{\frac{2\pi}{a}}\frac{(2n+1)!!}{a^{n}}\right)
\step{simple algebra}
\left(\sqrt{\frac{2\pi}{a}}\frac{(2n+1)!!}{a^{n+1}}\right).
\end{calculation}
Hence the result.
\end{proof}

\begin{theorem}
Let $n\in\NN$, let $A$ be a symmetric positive-definite $n\times n$ matrix.
Then
\begin{equation*}
\int_{\RR^{n}}\exp\left(\frac{-1}{2}\sum^{n}_{i,j=1}A_{i,j}x_{i}x_{j}\right)\D^{n}x
=\sqrt{\frac{(2\pi)^{n}}{\det(A)}}.
\end{equation*}
\end{theorem}

\begin{theorem}
Let $n\in\NN$, let $A$ be a symmetric positive-definite $n\times n$ matrix,
let $\vec{b}$ be an $n$-vector (of constants).
Then
\begin{equation*}
\int_{\RR^{n}}\exp\left(\frac{-1}{2}\sum^{n}_{i,j=1}A_{i,j}x_{i}x_{j}+\sum^{n}_{i}b_{i}x_{i}\right)\D^{n}x
=\sqrt{\frac{(2\pi)^{n}}{\det(A)}}\exp\left(\frac{1}{2}\transpose{\vec{b}}A^{-1}\vec{b}\right).
\end{equation*}
\end{theorem}


\chapter{Linear Algebra}
\section{Linear Algebra}

\begin{definition}
Let $\mathcal{H}$ be a Hilbert space and
Let $L\colon\mathcal{H}\to\mathcal{H}$ be a linear operator.
Then the \define{Adjoint} of $L$ is a linear operator denoted
$L^{\dagger}\colon\mathcal{H}\to\mathcal{H}$ defined by, for any
$\vec{u}$, $\vec{v}\in\mathcal{H}$ we have
\begin{equation}
\langle L\vec{u},\vec{v}\rangle = \langle \vec{u},L^{\dagger}\vec{v}\rangle.
\end{equation}
\end{definition}

\begin{definition}
Let $\mathcal{H}$ be a Hilbert space and
Let $L\colon\mathcal{H}\to\mathcal{H}$ be a linear operator.
We call $L$ \define{Self-Adjoint} if
\begin{equation}
L^{\dagger} = L.
\end{equation}
\end{definition}


\begin{theorem}
Let $\mathcal{H}$ be a finite-dimensional Hilbert space and
$L\colon \mathcal{H}\to\mathcal{H}$ be a self-adjoint operator. Then:
\begin{enumerate}
\item The eigenvalues of $L$ are all real;
\item The eigenvectors of $L$ form a complete set;
\item The normalized eigenvectors of $L$ are orthonormal.
\end{enumerate}
\end{theorem}

\begin{proof}[Proof (Eigenvalues of $L$ are real)]
Let $\vec{u}\in\mathcal{H}$ be an eigenvector for $L$ with eigenvalue
$\lambda\in\CC\setminus\{0\}$. Then we have
\begin{equation}
  \begin{array}{ccc}
    \langle L\vec{u},\vec{u}\rangle & = & \langle \vec{u},L\vec{u}\rangle\\
= &  & =\\
\lambda\langle\vec{u},\vec{u}\rangle & & \overline{\lambda}\langle\vec{u},\vec{u}\rangle
  \end{array}
\end{equation}
since $L$ is self-adjoint, $L^{\dagger}=L$ which gives us the top line,
and then linearity in the first slot (and anti-linearity in the second
slot) of the inner product gives us the second line. This implies
\begin{equation}
\lambda=\overline{\lambda},
\end{equation}
hence $\lambda\in\RR$.
\end{proof}

\begin{proof}[Proof (Eigenvectors of $L$ are orthogonal)]
Let $\vec{u}_{1}, \vec{u}_{2}\in\mathcal{H}$ be eigenvectors of $L$ with
distinct eigenvalues $\lambda_{1}$, $\lambda_{2}$ respectively.
Then
\begin{equation}
  \begin{array}{ccc}
    \langle L\vec{u}_{1},\vec{u}_{2}\rangle & = & \langle \vec{u}_{1},L\vec{u}_{2}\rangle\\
= &  & =\\
\lambda_{1}\langle\vec{u}_{1},\vec{u}_{2}\rangle & & \overline{\lambda}_{2}\langle\vec{u}_{1},\vec{u}_{2}\rangle.
  \end{array}
\end{equation}
Since the eigenvalues are real, $\lambda_{2}=\overline{\lambda}_{2}$,
and since $\lambda_{1}\neq\lambda_{2}$, we must have
\begin{equation}
\langle\vec{u}_{1},\vec{u}_{2}\rangle=0.
\end{equation}
Further, if the eigenvectors $\vec{u}_{i}$ are unit vectors, then they
form an orthonormal basis.
\end{proof}

\section{Unitary Operators}

\begin{definition}
Let $\mathcal{H}$ be a Hilbert space and
let $L\colon\mathcal{H}\to\mathcal{H}$ be a linear operator.
We call $L$ \define{Unitary} if
\begin{enumerate}
\item it is an isometry: $L^{\dagger}L=\id$
\item it is a co-isometry: $LL^{\dagger}=\id$
\end{enumerate}
\end{definition}

\begin{theorem}
Let $\mathcal{H}$ be a Hilbert space and
let $U\colon\mathcal{H}\to\mathcal{H}$ be a unitary operator.
Then $U$ is a bijection and $U^{-1}=U^{\dagger}$.
\end{theorem}

\chapter{Functional Analysis}
\section{Dirac Delta Function}

\begin{theorem}\label{thm:math:dirac-delta:single-dimension:alpha-x}
For any non-zero scalar $\alpha\neq0$, we have
\begin{equation}
\delta(\alpha x) = \frac{\delta(x)}{|\alpha|}.
\end{equation}
\end{theorem}

\begin{proof}
By direct calculation,
\begin{calculation}
\int^{\infty}_{-\infty}\delta(\alpha x)\,\D x
\step{change of variables, $u=\alpha x$, $\D u=\alpha\,\D x$}
\int^{\infty}_{-\infty}\delta(u)\frac{\D u}{|\alpha|}
\step{defining property of delta function}
\frac{1}{|\alpha|}.\qedhere
\end{calculation}
\end{proof}

\begin{lemma}
Let $g$ be continuously differentiable, assume $g'(x)$ is nowhere zero,
and there is only one simple root to $g$ at $g(x_{0})=0$. Then
\begin{equation}
\delta(g(x)) = \frac{\delta(x-x_{0})}{|g'(x_{0})|}.
\end{equation}
\end{lemma}

\begin{theorem}\label{thm:math:delta-function-of-polynomials}
Let $g$ be continuously differentiable and $g'(x)$ is nowhere zero.
Then
\begin{equation}
\delta(g(x)) = \sum_{\{x_{0}\mid g(x_{0})=0\}}\frac{1}{|g'(x_{0})|}\delta(x-x_{0}),
\end{equation}
where the sum extends to all distinct (non-repeated) simple roots.
\end{theorem}

\begin{corollary}\label{cor:math:dirac-delta-function-of-quadratic-polynomial}
  Let $\alpha$ be a nonzero scalar. Then we have
  \begin{equation}
\delta(x^{2} - \alpha^{2}) = \frac{1}{2|\alpha|}\left[\delta(x+\alpha)+\delta(x-\alpha)\right].
  \end{equation}
\end{corollary}

\begin{theorem}
Formally, for any $\alpha$, $\displaystyle\delta(x-\alpha)=\frac{1}{2\pi}\int^{\infty}_{-\infty}\E^{\I k(x-\alpha)}\,\D k$.
\end{theorem}

This follows from Fourier analysis.

\N{In higher dimensions}
When we work in $n$-dimensions and $\vec{a}\in\RR^{n}$, we write
\begin{equation}
\delta^{(n)}(\vec{x}-\vec{a}) = \prod^{n}_{j=1}\delta(x_{j}-a_{j}).
\end{equation}
Then for sufficiently nice $f\colon\RR^{n}\to\RR$, we have
\begin{equation}
\int_{\RR^{n}}f(\vec{x})\delta^{(n)}(\vec{x}-\vec{a})\,\D^{n}x=f(\vec{a}).
\end{equation}

\begin{theorem}
Let $\rho\colon\RR^{n}\to\RR^{n}$ be a reflection or rotation,
then $\delta^{(n)}(\rho\vec{x})=\delta^{(n)}(\vec{x})$.
\end{theorem}

\begin{theorem}
Let $\alpha\in\CC$ be nonzero $\alpha\neq0$.
Then $\delta^{(n)}(\alpha\vec{x})=|\alpha|^{-n}\delta^{(n)}(\vec{x})$.
\end{theorem}

This follows from repeatedly applying
Theorem~\ref{thm:math:dirac-delta:single-dimension:alpha-x}
to $\delta^{(n)}$.
\section{Functional Derivatives}\label{section:math:functional-derivatives}

\M
The intuition physicists use is to work directly analogous to discrete
indices. Just as we have
\begin{subequations}
\begin{equation}
\frac{\partial}{\partial x_{i}}x_{j} = \delta_{ij}
\end{equation}
we expect functional derivatives to obey similarly (for functions of
$n\in\NN$ variables),
\begin{equation}
\frac{\delta}{\delta f(x)}f(y)=\delta^{(n)}(x-y).
\end{equation}
\end{subequations}
Likewise, if we had a dot product
\begin{subequations}
\begin{equation}
x\cdot k = \sum_{j}x_{j}k_{j},
\end{equation}
this would correspond to the integral
\begin{equation}
\int f(x)k(y)\,\D^{n}y.
\end{equation}
Differentiation of the dot product
\begin{equation}
\frac{\partial}{\partial x_{i}}\sum_{j}x_{j}k_{j} = k_{i},
\end{equation}
corresponds to the analogous result,
\begin{equation}
\frac{\delta}{\delta f(x)}\int f(y)g(y)\,\D^{n}y = g(x).
\end{equation}
\end{subequations}
When we try to take the functional derivative with respect to $f(x)$ of
an integral involving the derivative of $f(x)$, we integrate by parts
before doing anything (and discard the boundary terms).

Physicists assume the functional derivative obeys the chain rule as well
as the Leibniz product rule. This turns out to be related to the
Fr\'echet derivative.

\begin{definition}
Let $F[\phi]$ be a functional, which is a mapping from a normed linear
space of functions [Banach space] $M = \{\phi(x)\mid x\in\RR\}$ to the
field of real or complex functions, $F\colon M\to\RR$ or $F\colon M\to\CC$.
Then we define the \define{Fr\'echet derivative} of $F$ in the direction
of $\lambda\in M$ to be the linear mapping $L\colon M\to\FF$ (where
$\FF=\RR$ if $F$ is real-valued, and $\FF=\CC$ if $F$ is complex-valued)
such that
\begin{equation}
F[\phi + \varepsilon\lambda]-F[\phi]=\varepsilon L[\phi,\lambda]
+o(\varepsilon).
\end{equation}
Physicists often write $\delta\phi$ instead of $\lambda$, and $\delta F$
instead of $L$.
\end{definition}

\N{Notation}
We write $\delta_{y}(x)=\delta(y-x)$.

\N{Physicist's Functional Derivative}
If we take $\delta\phi(x)=\varepsilon\delta_{y}(x)$, then
\begin{calculation}
\delta F[\phi]
\step{definition of Fr\'echet derivative}
F[\phi+\varepsilon\delta_{y}] - F[\phi]
\step{in analogy to differential form}
\int\frac{\delta F[\phi]}{\delta\phi(x)}\varepsilon\delta(y-x)\,\D x.
\end{calculation}
In the limit of vanishing $\varepsilon$, we obtain what physicists call
the functional derivative:
\begin{equation}
\boxed{\frac{\delta F[\phi]}{\delta\phi(y)}=\lim_{\varepsilon\to0}\frac{F[\phi+\varepsilon\delta_{y}]-F[\phi]}{\varepsilon}.}
\end{equation}
We call this the \define{Physicist's Functional Derivative} and it is a
generalized function.

\begin{example}\label{ex:math:functional-derivative:delta-f-over-delta-f-is-delta-fun}
  Let
  \begin{equation}
F[\phi] = \int \phi(y')\delta(y'-x)\,\D y' = \phi(x).
  \end{equation}
  Then
  \begin{equation}
\frac{\delta\phi(x)}{\delta\phi(y)}=\frac{\delta F[\phi]}{\delta\phi(y)}=\delta(y-x).
  \end{equation}
  The proof is by direct calculation:
\begin{calculation}
\frac{\delta F[\phi]}{\delta\phi(y)}
\step{definition of physicist's functional derivative}
\lim_{\varepsilon\to0}\frac{F[\phi + \varepsilon\delta_{y}]-F[\phi]}{\varepsilon}
\step{unfolding definition of $F[-]$}
\lim_{\varepsilon\to0}\frac{\int\bigl(\phi(y')+\varepsilon\delta(y-y')\bigr)\delta(y'-x)\,\D y' -\int \phi(y')\delta(y'-x)\,\D y'}{\varepsilon}
\step{arithmetic in the numerator}
\lim_{\varepsilon\to0}\frac{\varepsilon\int\delta(y-y')\delta(y'-x)\,\D y'}{\varepsilon}
\step{algebra, taking the limit}
\int\delta(y-y')\delta(y'-x)\,\D y'
\step{defining property of the delta function}
\delta(y-x).
\end{calculation}
\end{example}

\begin{example}
Let $\alpha\in\CC$ be nonzero $\alpha\neq0$.
Then
\begin{equation}
\frac{\delta\phi(x)^{\alpha}}{\delta\phi(y)}=\delta(x-y)\alpha\phi(x)^{\alpha-1}.
\end{equation}
\begin{proof}
  We begin by defining the functional
  \begin{equation}
F[\phi] = \int\phi(y')^{\alpha}\delta(y'-x)\,\D y'.
  \end{equation}
  Then by direct calculation, we find
\begin{calculation}
\frac{\delta F[\phi]}{\delta\phi(y)}
\step{definition of physicist's functional derivative}
\lim_{\varepsilon\to0}\frac{F[\phi + \varepsilon\delta_{y}]-F[\phi]}{\varepsilon}
\step{unfolding definition of $F[-]$}
\lim_{\varepsilon\to0}\frac{\int\bigl(\phi(y')+\varepsilon\delta(y-y')\bigr)^{\alpha}\delta(y'-x)\,\D y' -\int \phi(y')^{\alpha}\delta(y'-x)\,\D y'}{\varepsilon}
\step{Taylor expand to first order in $\varepsilon$ in the first term}
\lim_{\varepsilon\to0}\frac{\int\bigl(\phi(y')^{\alpha}+\alpha\varepsilon\delta(y-y')\phi(y')^{\alpha-1}+\bigOh(\varepsilon^{2})\bigr)\delta(y'-x)\,\D y' -\int \phi(y')^{\alpha}\delta(y'-x)\,\D y'}{\varepsilon}
\step{arithmetic in the numerator, factoring out a power of $\varepsilon$}
\lim_{\varepsilon\to0}\frac{\varepsilon\int\bigl(\delta(y-y')\alpha\phi(y')^{\alpha-1}+\bigOh(\varepsilon)\bigr)\delta(y'-x)\,\D y'}{\varepsilon}
\step{algebra, taking the limit}
\int\delta(y-y')\alpha\phi(y')^{\alpha-1}\delta(y'-x)\,\D y'
\step{defining property of the delta function}
\delta(y-x)\alpha\phi(y)^{\alpha-1}.\qedhere
\end{calculation}
\end{proof}
\end{example}

\begin{theorem}
Let $f\colon\RR\to\RR$ be a sufficiently smooth function, $F[-]$ be a
functional defined by
\begin{subequations}
\begin{equation}
F[\phi] = \int f\bigl(\phi(x)\bigr)\,\D x.
\end{equation}
Then
\begin{equation}
\frac{\delta F[\phi]}{\delta\phi(y)} = f'\bigl(\phi(y)\bigr),
\end{equation}
where $f'$ is the first derivative of $f$.
\end{subequations}
\end{theorem}

\begin{proof}
We see that, Taylor expanding $F[\phi+\varepsilon\delta_{y}]$ to first
order in $\varepsilon$,
\begin{equation}
F[\phi+\varepsilon\delta_{y}]=\int\left(f\bigl(\phi(x)\bigr)+\varepsilon\delta(x-y)f'\bigl(\phi(x)\bigr)+\bigOh(\varepsilon^{2})\right)\D x.
\end{equation}
Then
\begin{equation}
F[\phi+\varepsilon\delta_{y}]-F[\phi]=\int\left(\varepsilon\delta(x-y)f'\bigl(\phi(x)\bigr)+\bigOh(\varepsilon^{2})\right)\D x.
\end{equation}
Dividing both sides by $\varepsilon$ gives us
\begin{equation}
\frac{F[\phi+\varepsilon\delta_{y}]-F[\phi]}{\varepsilon}=\int\left(\delta(x-y)f'\bigl(\phi(x)\bigr)+\bigOh(\varepsilon)\right)\D x.
\end{equation}
Taking the $\varepsilon\to0$ limit gives us the physicist's functional derivative
\begin{equation}
\frac{\delta F[\phi]}{\delta\phi(y)}
=\int\left(\delta(x-y)f'\bigl(\phi(x)\bigr)\right)\D x.
\end{equation}
Then invoking the defining property of the delta function produces the result.
\end{proof}

\begin{theorem}[Linearity]
Let $F[-]$, $G[-]$ be functionals, $c_{1}$, $c_{2}$ be (real or complex)
constants, then
\begin{equation}
\frac{\delta(c_{1}F[\phi]+c_{2}G[\phi])}{\delta\phi(x)}
=c_{1}\frac{\delta F[\phi]}{\delta\phi(x)}
+c_{2}\frac{\delta G[\phi]}{\delta\phi(x)}.
\end{equation}
\end{theorem}

\begin{theorem}[Leibniz ``product'' rule]
Let $F[\phi]=G[\phi]H[\phi]$ be the product of two functionals.
Then
\begin{equation}
\frac{\delta F[\phi]}{\delta\phi(x)}
=\frac{\delta G[\phi]}{\delta\phi(x)}
H[\phi] + G[\phi]\frac{\delta H[\phi]}{\delta\phi(x)}.
\end{equation}
\end{theorem}

\begin{theorem}
[Function chain rule]
Let $g$ be a sufficiently smooth function, $F[-]$ be a functional.
Then
\begin{equation}
\frac{\delta F\bigl[g(\phi)\bigr]}{\delta\phi(y)}
=\frac{\delta F\bigl[g(\phi)\bigr]}{\delta g\bigl(\phi(y)\bigr)}g'\bigl(\phi(y)\bigr).
\end{equation}
\end{theorem}

\begin{remark}
The chain rule for functional differentiation holds, but it requires
$F\colon U\to Y$ to be differentiable at $\phi\in U$, and $G\colon Y\to W$
is differentiable at $\psi=F[\phi]$, then $G\circ F$ is differentiable
at $\phi$. But since $Y$ is usually taken to be the complex or real
numbers, this amounts to making $G$ ``just'' a function $G[-]=g(-)$, and we are
considering $(G\circ F)[\phi] = g\bigl(F[\phi]\bigr)$. The chain rule
would be
\begin{equation}
\frac{\delta g\bigl(F[\phi]\bigr)}{\delta\phi(x)} = g'\bigl(F[\phi]\bigr)
\frac{\delta F[\phi]}{\delta\phi(x)},
\end{equation}
where $g'$ is the derivative of $g$.
\end{remark}

\N{Functional Taylor Series}
We can perform something analogous to the Taylor series expansion for a
functional using functional derivatives, writing
\begin{equation}
F[\phi + \psi] =\sum^{\infty}_{k=0}\frac{1}{k!}\int\frac{\delta^{k}F[\phi]}{\delta\phi(x_{1})\cdots\delta\phi(x_{k})}\psi(x_{1})\cdots\psi(x_{k})\,\D^{n}x_{1}\cdots\D^{n}x_{k}.
\end{equation}
This will prove important when working with the effective action of a
quantum theory.

\N{References}
Peskin and Schroeder~\cite{Peskin:1995ev}, in chapter 9 section 2, give
the heuristic presented initially as the defining axioms for the
functional derivative. Greiner~\cite{Greiner:1996zu}, section 2.3, discusses
functional derivatives in greater detail but is sloppy with his
discussion of the chain-rule.

%%
%% bib.tex
%% 
%% Made by alex
%% Login   <alex@tomato>
%% 
%% Started on  Thu Nov  3 14:28:43 2011 alex
%% Last update Wed May 30 10:14:48 2012 Alex Nelson
%%


\vfill\eject
\phantomsection\addcontentsline{toc}{section}{References} 
\begin{thebibliography}{99}
\bibitem{knuth} Donald Knuth,\newblock
``Teach Calculus with Big O.''\newblock
Eprint: \url{http://www-cs-staff.stanford.edu/~uno/ocalc.tex}\newblock
\emph{Not.\ of the AMS} {\bf45} (6): 687
\bibitem{livshits} Michael Livshits, \newblock
``You could simplify calculus.''\newblock
Eprint: \arXiv{0905.3611} \texttt{[math.HO]}
\bibitem{shchepin}
E.\ V.\ Shechepin,\newblock
``Gateway to Calculus: On Euler's Footsteps.''\newblock
Eprint: \url{http://www.mi.ras.ru/~scepin/uppsala.ps}
\end{thebibliography}


% TOC entry will give wrong page for index unless this phantomsection is
% added; plus the hyperref link requires this.
\phantomsection 
\printindex
\end{document}
