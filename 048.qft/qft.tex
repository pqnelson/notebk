% Time-stamp: "2022-11-20T18:53:23-08 Alex"
\documentclass{report}
\usepackage{danger}
\usepackage{macros}
\usepackage{deriv}
\usepackage{notation}

\pdfinfo{/CreationDate (D:20221120185323)}
\title{Quantum Field Theory}
\date{November 20, 2022}
\begin{document}

\maketitle

\begin{abstract}
This is a summary of the conceptual aspects of quantum field theory. A
lot of calculations will be omitted or sketched out. Many calculations
may be found explicitly in Hatfield~\cite{Hatfield:1992rz}. We instead
focus on the usefulness of Lie algebras and Lie groups in quantum field
theory.
\end{abstract}

\tableofcontents

\chapter{Preface}

This introduces differential topology, which studies global aspects of
smooth manifolds. It's roughly at the level of a graduate course,
assuming the reader knows elementary differential geometry of
curves\footnote{Roughly at the level of my notes \url{http://pqnelson.github.io/assets/notebk/dg.pdf}},
real analysis, and so on. These notes are based on Dr Fuchs' course on
differential topology at UC Davis, Math 239, held Fall quarter of 2010.

The conventions are slightly old-fashioned, namely: charts are defined
as ordered pairs of maps from ``patches'' of open subsets of $\RR^{n}$
to a set along with the open subsets of $\RR^{n}$ themselves. This is
done so we can induce a topology using a collection of charts, demanding
the charts be \emph{continuous}; then their images form a topological
basis. We get, for free, a topological manifold from this demand on
maximal atlases.

Most ``modern'' books use the opposite convention, a chart on a set $M$
is a pair $(\varphi,U)$ consisting of a subset $U\subset M$ --- \emph{not}
$U\subset\RR^{n}$ (!!!) --- and a mapping $\varphi\colon U\to\RR^{n}$.
This is fine, but requires more work to induce a topology.
\chapter{Quantum Mechanics}

\M We will review the pertinent aspects of \emph{nonrelativistic} quantum
mechanics. This is pedagogical, in the sense we start with ``simple but wrong''
rules, and later on revise the rules as needed.

\section{Basic Rules}\label{sec:qm:basic-rules}

\M We will review the basic rules for describing a quantum system using
a wave function. We will revise the rules later, for greater generality,
but we begin with relevant versions for physical systems.

\N{Position--Momentum Commutator Relation}
We have
\begin{equation}\label{eq:position-momentum-uncertainty-commutator}
[\widehat{x},\widehat{p}] = \I\hbar.
\end{equation}


\subsection{Wave Functions}

\M Following Isham~\cite{Isham:1995lq}, we summarize the rules for
interpreting wave functions.

\N{First rule}
The quantum state of a point particle moving in one-dimension is
represented by a complex-valued wave function $\psi(x)$ which is
normalized to one:
\begin{equation}
\int^{\infty}_{-\infty}|\psi(x)|^{2}\,\D x = 1.
\end{equation}
We will say that ``$\psi$ is normalized to unity'' when this condition holds.

\N{Superposition principle}
We can superimpose any two wave functions $\psi_{1}(x)$ and
$\psi_{2}(x)$ with arbitrary complex coordinates $\alpha_{1}$, $\alpha_{2}$.
That is to say, we get a new wave function
$\alpha_{1}\psi_{1}(x) + \alpha_{2}\psi_{2}(x)$, provided that
$\alpha_{1}$ and $\alpha_{2}$ are chosen to make the new wave function
normalized to unity.

\N{Second rule}
Any physical quantity which we can measure (i.e., any observable) is
represented by a linear differential operator that acts on wave functions
and is self-adjoint.

\begin{remark}[Eigenvalues of self-adjoint operators]
``Self-adjoint'' operators have real eigenvalues. We measure
eigenvalues. If we are measuring a complex-valued quantity, then we
need an anti-self-adjoint operator.
\end{remark}

\begin{remark}[Completeness of eigenfunctions for self-adjoint operators]
The eigenfunctions for a self-adjoint operator are ``complete'' in the
sense that they form a basis. Further, normalized eigenfunctions satisfy
the orthogonality condition
\begin{equation}
\int^{\infty}_{-\infty}f_{m}^{*}(x)f_{n}(x)\,\D x = \delta_{m,n}
\end{equation}
where $\delta_{m,n}=1$ if $m=n$ and $\delta_{m,n}=0$ otherwise.
\end{remark}

\N{Third rule}
The only possible result of measuring an observable $A$ is one of the
eigenvalues of the self-adjoint operator $\widehat{A}$ which represents
it.

\begin{remark}
This rule implies the long-term average value of the results of repeated
measurements of an observable $A$ is
\begin{equation}
\langle A\rangle_{\psi} = \langle\psi\mid A\mid\psi\rangle = \int^{\infty}_{-\infty}\psi^{*}(x)(\widehat{A}\psi)(x)\,\D x.
\end{equation}
\end{remark}

\M
For nondegenerate operators (i.e., any two eigenfunctions of
$\widehat{A}$ with the same eigenvalue are proportional to each other)
and with a discrete set of eigenvalues $a_{1}$, $a_{2}$, \dots
corresponding to eigenfunctions $f_{1}$, $f_{2}$, \dots; if the state is
$\psi(x)$, then the probability that a measurement of $A$ will yield a
particular eigenvalue $a_{n}$ is
\begin{equation}
\Pr(A = a_{n}; \psi) = |\psi_{n}|^{2},
\end{equation}
assuming the wave function is written as a linear combination of the
eigenfunctions,
\begin{equation}
\psi(x) = \sum_{n}\psi_{n}f_{n}(x)
\end{equation}
and the $\psi_{n}\in\CC$ are constants. We also assume $\psi(x)$ is
normalized to unity.

\begin{remark}
As a consequence of this, a wave function being normalized to unity is
the same as the ``sum of all probabilities equal 1''.
\end{remark}

\N{Fourth Rule}
The state function evolved in time according to the time-dependent
Schr\"{o}dinger equation
\begin{equation}
\I\hbar\frac{\partial\psi(x,t)}{\partial t} = \widehat{H}\psi(x,t),
\end{equation}
where the Hamiltonian operator $\widehat{H}$ is obtained from the
classical enerygy expression
\begin{equation}
H = \frac{p^{2}}{2m} + V(x),
\end{equation}
replacing the momentum $p$ and position $x$ by their corresponding
operators $\widehat{p}$, $\widehat{x}$.

\N{Incompleteness of Rules}
As we have presented things, these rules are incomplete since they give
no information on how to construct the operator which represents any
specific observable for a given physical system. In practice, this is
done by the ``substitution rule'' putting hats on things: for any
classical observable $A(x, p)$ the analogous operator is
$A(\widehat{x},\widehat{p})$. 

\begin{danger}
The astute reader will note we never defined ``observable'',
``measurement'', ``physical system'', ``property'', ``state'',
``causality'', or ``determinism''. We also didn't define ``physical quantity''.
This ambiguity is seldom addressed directly, instead most texts shift
the meaning of these words as needed. What is an ``object''? What is a
``property''? Or a ``physical quantity''? Is the significance of these
terms really so obvious? We're confronted by a deceptive question: ``What
is a `thing'?''
\end{danger}

\section{Quantum Harmonic Oscillator}

\N{Problem} Given the simple Harmonic oscillator's Hamiltonian
\begin{equation}\label{eq:quantum-harmonic-oscillator:defn:hamiltonian}
\widehat{H} = \frac{1}{2m}\left(\widehat{p}^{2} + (m\omega \widehat{x})^{2}\right),
\end{equation}
determine the eigenstates for the Schrodinger equation.

\M The trick is to use the identity
\begin{equation}
a^{2} + b^{2} = (a + \I b)(a -\I b)
\end{equation}
to introduce the operators:
\begin{equation}\label{eq:quantum-harmonic-oscillator:defn:ladder-operators}
a_{\pm} := \frac{1}{\sqrt{2\hbar m\omega}}\left(m\omega \widehat{x} \mp\I\widehat{p}\right).
\end{equation}
We will use this to rewrite the Hamiltonian operator.

\begin{lemma}\label{lemma:quantum-harmonic-oscillator:ladder-operator-hamiltonian-relation}
We find $a_{+}a_{-} = (\widehat{H}/\hbar\omega) - 1/2$.
\end{lemma}
\begin{proof}[Proof (lengthy calculation)]
  By direct computation
  \begin{calculation}
    a_{+}a_{-}
\step{by Eq~\eqref{eq:quantum-harmonic-oscillator:defn:ladder-operators}}
    \frac{1}{2\hbar m\omega}\left(m\omega \widehat{x}-\I\widehat{p}\right)\left(m\omega \widehat{x}+\I\widehat{p}\right)
\step{distributivity}
    \frac{1}{2\hbar m\omega}\left((m\omega \widehat{x})^{2}-\I\widehat{p}m\omega \widehat{x}+m\omega \widehat{x}\I\widehat{p}-\I^{2}\widehat{p}^{2}\right)
\step{linearity of operators}
    \frac{1}{2\hbar m\omega}\left((m\omega \widehat{x})^{2}-\I m\omega \widehat{p}\widehat{x}+\I m\omega \widehat{x}\widehat{p}+\widehat{p}^{2}\right)
\step{explicitly insert a commutator}
    \frac{1}{2\hbar m\omega}\left((m\omega \widehat{x})^{2}+\I m\omega[\widehat{x},\widehat{p}]+\widehat{p}^{2}\right)
\step{commutator relation Eq~\eqref{eq:position-momentum-uncertainty-commutator}}
    \frac{1}{2\hbar m\omega}\left((m\omega \widehat{x})^{2}+\I m\omega (\I\hbar)+\widehat{p}^{2}\right)
\step{associativity of addition}
    \frac{1}{2\hbar m\omega}\left((m\omega \widehat{x})^{2}+\widehat{p}^{2}\right)
+\frac{\I m\omega (\I\hbar)}{2\hbar m\omega}
\step{folding the definition of the Hamiltonian Eq~\eqref{eq:quantum-harmonic-oscillator:defn:hamiltonian}}
    \frac{1}{\hbar\omega}\widehat{H} +\frac{\I m\omega (\I\hbar)}{2\hbar m\omega}
\step{algebra}
    \frac{1}{\hbar\omega}\widehat{H} -\frac{1}{2}\qedhere
  \end{calculation}
\end{proof}

\begin{lemma}\label{lemma:quantum-harmonic-oscillator:ladder-operator-hamiltonian-relation2}
We find $a_{-}a_{+} = (\widehat{H}/\hbar\omega) + 1/2$.
\end{lemma}

\begin{proof}
Our calculation is exactly the same, except the sign of the commutator
$[\widehat{x},\widehat{p}]$ changes in our consideration, which is the
source of the $1/2$ contribution.
\end{proof}

\N{Commutator of Ladder Operators}
We have $[a_{-},a_{+}]=1$.

\begin{proof}
  We have, by our previous two lemmas,
  \begin{equation}
a_{-}a_{+}-a_{+}a_{-} = \left(\frac{\widehat{H}}{\hbar\omega}+\frac{1}{2}\right)-\left(\frac{\widehat{H}}{\hbar\omega}-\frac{1}{2}\right).
  \end{equation}
Hence the result.
\end{proof}

\N{Ladder Operators Acting on Eigenstates}
Suppose $\psi(x)$ is an eigenfunction for the Hamiltonian operator, with
energy eigenvalue $E$. We claim $a_{+}\psi(x)$ is an eigenfunction with
energy eigenvalue $E+\hbar\omega$.

\begin{proof} By direct computation,
\begin{calculation}
  \widehat{H}(a_{+}\psi(x))
\step{using $\widehat{H} = \hbar\omega(a_{+}a_{-}+1/2)$}
  \hbar\omega\left(a_{+}a_{-}+\frac{1}{2}\right)(a_{+}\psi(x))
\step{move $a_{+}$ inside the parentheses using right distributivity}
  \hbar\omega\left(a_{+}a_{-}a_{+}+\frac{1}{2}a_{+}\right)(\psi(x))
\step{associativity}
  \hbar\omega\left(a_{+}(a_{-}a_{+})+\frac{1}{2}a_{+}\right)(\psi(x))
\step{using Lemma~\ref{lemma:quantum-harmonic-oscillator:ladder-operator-hamiltonian-relation2}}
  \hbar\omega\left(a_{+}\left(\frac{\widehat{H}}{\hbar\omega}+\frac{1}{2}\right)+\frac{1}{2}a_{+}\right)(\psi(x))
\step{algebra}
  (a_{+}\widehat{H}+a_{+}\hbar\omega)\psi(x)
\step{pull out a factor of $a_{+}$ using left distributivity}
  a_{+}(\widehat{H} + \hbar\omega)\psi(x)
\step{since $\psi$ is an energy eigenstate}
  a_{+}(E + \hbar\omega)\psi(x)
\step{commutativity of numbers and operators}
  (E + \hbar\omega)(a_{+}\psi(x))
\end{calculation}
Hence $a_{+}\psi(x)$ is an energy eigenstate with eigenvalue $E + \hbar\omega$.
\end{proof}

\M
Similarly, for any energy eigenstate $\psi(x)$ with eigenvalue $E$, we
have $a_{-}\psi(x)$ is another energy eigenstate with eigenvalue
$E-\hbar\omega$.

\begin{proof} By direct computation,
\begin{calculation}
  \widehat{H}(a_{-}\psi(x))
\step{using $\widehat{H} = \hbar\omega(a_{-}a_{+}-1/2)$}
  \hbar\omega\left(a_{-}a_{+}-\frac{1}{2}\right)(a_{-}\psi(x))
\step{move $a_{-}$ inside using right distributivity}
  \hbar\omega\left(a_{-}a_{+}a_{-}-\frac{1}{2}a_{-}\right)(\psi(x))
\step{associativity}
  \hbar\omega\left(a_{-}(a_{+}a_{-})-\frac{1}{2}a_{+}\right)(\psi(x))
\step{using Lemma~\ref{lemma:quantum-harmonic-oscillator:ladder-operator-hamiltonian-relation}}
  \hbar\omega\left(a_{-}\left(\frac{\widehat{H}}{\hbar\omega}-\frac{1}{2}\right)-\frac{1}{2}a_{-}\right)(\psi(x))
\step{algebra}
  (a_{-}\widehat{H}-a_{-}\hbar\omega)\psi(x)
\step{pull out $a_{-}$ using left distributivity}
  a_{-}(\widehat{H} - \hbar\omega)\psi(x)
\step{since $\psi$ is an energy eigenstate}
  a_{-}(E - \hbar\omega)\psi(x)
\step{commutativity of numbers and operators}
  (E - \hbar\omega)(a_{-}\psi(x))
\end{calculation}
Hence $a_{-}\psi(x)$ is an energy eigenstate with eigenvalue $E - \hbar\omega$.
\end{proof}

\N{Ground State}
We can't keep decreasing the energy eigenstate forever, there must be
some groundstate $\psi_{0}(x)$ such that
\begin{equation}
a_{-}\psi_{0}(x) =0.
\end{equation}
Solve this differential equation for $\psi_{0}(x)$.

\begin{proof}[Solution]
We recall the definition of the ladder operator $a_{-}$ from Eq~\eqref{eq:quantum-harmonic-oscillator:defn:ladder-operators}
\begin{equation}
\frac{1}{\sqrt{2\hbar m\omega}}\left(m\omega \widehat{x}+\I\widehat{p}\right)\psi_{0}(x)=0.
\end{equation}
Doing some algebra, this becomes
\begin{equation}
(m\omega\widehat{x} + \I\widehat{p})\psi_{0}(x) = 0,
\end{equation}
which in position-space becomes
\begin{equation}
\left(m\omega x + \I(-\I)\hbar\frac{\partial}{\partial x}\right)\psi_{0}(x)=0.
\end{equation}
This becomes
\begin{equation}
\psi'_{0}(x) = -\frac{m\omega}{\hbar}x\psi_{0}(x).
\end{equation}
Divide both sides by $\psi_{0}$, then integrating with respect to $x$
gives
\begin{equation}
\int\frac{\D\psi_{0}}{\psi_{0}} = -\frac{m\omega}{\hbar}\int x\,\D x.
\end{equation}
Hence
\begin{equation}
\ln(\psi_{0}(x)) = -\frac{m\omega}{\hbar}\frac{x^{2}}{2} + c_{0},
\end{equation}
where $c_{0}$ is some integration constant.
Then we find
\begin{equation}
\psi_{0}(x) = C\exp\left(-\frac{m\omega}{\hbar}\frac{x^{2}}{2}\right).
\end{equation}
We can determine $C$ by demanding $\psi_{0}(x)$ be normalizable. This
gives us
\begin{equation}
C^{2}\int\exp\left(-\frac{m\omega}{\hbar}x^{2}\right)\,\D x = 1.
\end{equation}
This is a Gaussian integral (\S\ref{cor:math:general-gaussian-integral-in-one-dim}), which means
\begin{equation}
C^{2}\sqrt{\frac{\pi\hbar}{m\omega}}=1\implies C=\left(\frac{m\omega}{\pi\hbar}\right)^{1/4}.
\end{equation}
Thus
\begin{equation}
\boxed{\psi_{0}(x) = \left(\frac{m\omega}{\pi\hbar}\right)^{1/4}\exp\left(-\frac{m\omega}{\hbar}\frac{x^{2}}{2}\right).}
\end{equation}
Up to phase factor (i.e., a factor of $\E^{\I c_{1}}$ for any
arbitrary constant $c_{1}\in\RR$).
\end{proof}

\N{Other Eigenstates}
We can apply the ladder operator $a_{+}$ finitely many times to get
other eigenstates $\psi_{n}(x)\sim(a_{+})^{n}\psi_{0}(x)$. More
precisely,
\begin{equation}
\psi_{n}(x) = C_{n}(a_{+})^{n}\psi_{0}(x)
\end{equation}
where $C_{n}$ is some normalization constant.

\section[*Experimental Background]{*Experimental Background\footnote{This section can be skipped on first reading, it just summarizes the
experimental problems which classical physics could not answer (which
forced physicists to invent quantum mechanics).}}

\M Around the turn of the $20^{\text{th}}$ century, physics faced some
puzzling experimental results which formed the motivation for deriving
quantum mechanics. Let us review superficially what was known at the
time, then in each subsection we shall consider an experiment, why it
produced unexplainable results, and how physicists resolved things.

\N{Status of Light}\label{chunk:qm:experimental-background:status-of-light}
We should stress that at the time, physicists believed light propagated
as a wave. Thomas Young first tested this in 1800 with his famous
double-slit experiment, where light propagated through two slits and
these two light sources appeared to ``interfer'' with each other. This
has a description using geometric optics and waves, which undergraduates
learn routinely --- see, e.g., Young and Freedman's
\textit{University Physics}.

\N{Atomic Theory}\label{chunk:qm:experimental-background:atomic-theory}
Around the turn of the $20^{\text{th}}$ century, physicists began to
recognize that matter can be described as consisting of
atoms. Rutherford supervised Geiger and Marsden's 1909 experiment, then
later in 1911 explained that the experiment implied an atom consists of
a nucleus (which contains most of the mass and is positively charged)
and electrons which orbit the nucleus, just as planets orbit a star. We
stress that this interprets an electron as a particle.

Millikan's oil drop experiment (conducted in 1908, reported in 1913)
determined the electric charge for the elementary electric charge
$e\approx1.602\times10^{-19}~\mathrm{C}$.  Each electron has a charge of
$-e$, and the nucleus has a charge of $+ne$ where $n$ is the atomic
number.

\subsection{Black Body Radiation}

\begin{definition}
A \define{Black Body} is an idealized object which absorbs all electromagnetic
radiation which hits it.
\end{definition}

\begin{remark}
Just because a black body \emph{absorbs} all radiation which hits
\emph{does not mean} a black body cannot emit any radiation. A black
body \emph{can} and \emph{does} emit radiation.
\end{remark}

\begin{remark}
It turns out that ``thermal radiation'' is precisely ``electromagnetic radiation'',
so I may use the terms interchangeably. But at the end of the day,
they're just photons.
\end{remark}

\N{Problem Statement}
Suppose we make a hollow cube out of a black body. Then the problem we
want to answer: describe the distribution of the frequencies of the
[emitted] radiation inside the cube.

\N{Classical Expectations}
Inside the body, the radiation will constantly be absorbed and re-emitted,
eventually reaching thermal equilibrium.
At each frequency, the absorption and emission of radiation will be
perfectly balanced.

We can invoke the ``Equipartition Theorem'' (of classical statistical
mechanics) which states the energy in any given mode of electromagnetic
radiation should be exponentially distributed with an average value
equal to $k_{B}T$ where $T$ is the temperature in Kelvin and $k_{B}$ is
the Boltzmann constant.

\N{Ultraviolet Catastrophe}
The difficulty is that the average amount of energy is the same for
every mode\footnote{Physicists use the term ``mode'' and ``frequency''
interchangeably.}. So when we add up the energy for each mode (for which
there are infinitely many), we get an infinite amount of energy for the
radiation in our black body oven. This is referred to as the
\define{Ultraviolet Catastrophe}, since the infinity comes from the
ultraviolet [high-frequency] end of the spectrum.

This contradicts our observation of a finite amount of energy inside a
black body oven.

\N{Planck's Solution}
In 1900, Max Planck offered an alternative prediction for this problem. The
key step is that Planck postulated energy in the electromagnetic field
at a frequency $\omega$ is ``quantized'', meaning it comes in an integer
multiple of a certain basic unit $\hbar\omega$ where $\hbar$ is a
constant we now recognize as Planck's constant.\marginpar{{\footnotesize Planck's Constant $\hbar$}}

Planck postulated the energy is then exponentially distributed over
integer multiples of $\hbar\omega$. At low frequencies, this will
coincide with classical statistical mechanics. But at high frequencies
(where $\hbar\omega$ is comparable to $k_{B}T$), Planck's theory
predicts a rapid fall-off of the average energy.

\begin{exercise}
Let $c>0$ be real. Prove $\displaystyle\sum^{\infty}_{n=0}n\E^{-cn}=\frac{\E^{c}}{(1-\E^{-c})^{2}}$.
\end{exercise}

\begin{exercise}
In Planck's model, the energies for electromagnetic radiation in a
frequency $\omega$ (in units of $[\mbox{time}]^{-1}$) is
distributed randomly over all numbers $n\hbar\omega$ for $n=0,1,2,\dots$.
We postulate the probability of finding energy $n\hbar\omega$ is
\begin{equation}
\Pr(E = n\hbar\omega) = \frac{1}{Z}\E^{\beta n\hbar\omega}
\end{equation}
where $\beta = 1/(k_{B}T)$ and $Z$ is a normalization constant. Then the
expected value for energy is
\begin{equation}
\langle E\rangle = \sum^{\infty}_{n=0}(n\hbar\omega)\Pr(E=n\hbar\omega)= \frac{1}{Z}\sum^{\infty}_{n=0}(n\hbar\omega)\E^{-\beta n\hbar\omega}.
\end{equation}
Now you come in:
\begin{enumerate}
\item Find $Z$ by demanding $\sum_{n=0}^{\infty}\Pr(E = n\hbar\omega)=1$.
\item Prove $\displaystyle\langle E\rangle = \frac{\hbar\omega}{\E^{\beta\hbar\omega}-1}$.
\item Show $\langle E\rangle$ behaves like $1/\beta = k_{B}T$ for small $\omega$,
but $\langle E\rangle$ decays exponentially as $\omega\to\infty$.
\end{enumerate}
\end{exercise}

\subsection{Photoelectric Effect}

\N{Problem Statement}
Suppose we have a metal surface. We can observe as electromagnetic
radiation (``incident light'') hits the surface, electrons will be emitted from the surface.

Einstein found as we increase the \emph{intensity} of the incident
light, the \emph{number} of electrons increases but the \emph{energy} of
each electron does not change. This is bizarre for a number of reasons.

If electromagnetic radiation were waves, then low-frequency light at
high intensity  should ``build up'' the energy necessary to produce
electrons, but we do not observe this\dots which suggests
electromagnetic radiation (``light'') is not a wave but consists of
particles. More precisely, light comes in discrete energy packages which
Einstein called \define*{Photons}\index{Photon!Photoelectric effect}.

\N{Einstein's Solution}
In 1905, Einstein proposed that the electromagnetic radiation travels in
discrete energy packets called photons. For a frequency $\omega$, the
packet has energy $\hbar\omega$. (At the time, Einstein did not realize
$\hbar$ was Planck's constant, and used some constant of proportionality.)

The highest kinetic energy $K_{\text{max}}$ for the electron removed
from their atomic bindings by the absorption of a photon with energy
$\hbar\omega$ is then
\begin{equation}
K_{\text{max}} = \hbar\omega - W,
\end{equation}
where $W$ is the minimum energy required to remove an electron from the
metal surface. The literature refers to $W$ as the ``work function'' of
the surface. If we write the work function as:
\begin{equation}
W = \hbar\omega_{0},
\end{equation}
then the kinetic energy upper bound is
\begin{equation}
K_{\text{max}} = \hbar(\omega - \omega_{0}).
\end{equation}
Obviously emission can only occur when $K_{\text{max}}$ is positive,
requiring $\omega>\omega_{0}$.

\N{Quantum?}
Observe this solution requires $\hbar$ (a quantum quantity) and
classical physics fails to predict observed phenomena. More explicitly
the wave description of light fails to predict what we observe,
suggesting light consists of discrete packets (``particles'').

\subsection{Double-Slit Experiment}

\N{Problem statement}
We have assumed (\S\ref{chunk:qm:experimental-background:status-of-light})
light is a wave, since that appears to be supported by Young's
Double-Slit experiment. But if Einstein is correct and light propagated
in discrete packets, then how can we explain the double-slit phenomenon?

\N{Solutions}
J.J.~Thomson\footnote{J.J.~Thomson, ``On the ionization of Gases by
Ultra- Violet Light and on the evidence as to the structure of light
afforded by its Electrical Effect''. \journal{Prof.Cam.Phil.Soc.}
\volume{14} (1907) 417--424.} sketched out an experimental test in 1907, 
suggesting the results observed in the double-slit experiment could be
explained by the photons somehow interacting with each other.
Geoffrey Taylor\footnote{G.I.~Taylor, ``Interference Fringes with Feeble
Light''. \journal{Prof.Cam.Phil.Soc.} \volume{15} (1909) 114--115.}
first performed a low-intensity double-slit experiment
in 1909 by reducing the level of incident light (to the point where the
experiment took roughly 2000 hours to form an interference pattern ---
or 83 days and 12 hours). Taylor observed interference occurring still
at such low intensities of light.

Taylor's experimental results are interpreted as interference remains
even when photons are widely separated from each other (which is
weird!), but the photons are not interferring \emph{with each other}.
Instead, as Dirac writes in his book \textit{Principles of Quantum Mechanics},
``Each photon interfers only with itself.''

\subsection{Hydrogen Spectrum}

\N{Experiment}
If we pass electricity through a tube of hydrogen gas, then light will
be emitted. We can pass that light through a prism and four visible
``bands'' of light will form (red, cyan, blue, violet). This indicates
light is not a continuous spectrum but consists of discrete
frequencies. From the perspective of photons, this suggests only certain
energy packets are emitted. But this is surprising from a classical
perspective where light is a wave, and has no classical explanation.

\N{(Phenomenological) Explanation}
The Hydrogen atom consists of one proton and one electron. As we pass
electricity through the Hydrogen gas, the orbital electron ``gets
excited'' and moves farther away from the nucleus. Eventually the
electron will return to lower states, emitting a photon in the
process. This emitted photon takes on certain discrete
frequencies. Johannes Rydberg concluded in 1888 based on empirical data
that the energies of the emitted photon satisfy
\begin{equation}
E_{n} = -\frac{R}{n^{2}}
\end{equation}
where $n\in\NN$ and $R$ is the Rydberg constant
\begin{equation}
R = \frac{m_{e}Q^{4}}{2\hbar^{2}}.
\end{equation}
Here $m_{e}$ is the mass of the electron, $Q=-e$ is the charge of the
electron.
Note that we should use the reduced mass $\mu=m_{e}m_{p}/(m_{e}+m_{p})$
where $m_{p}$ is the mass of the proton,
but since $m_{p}\gg m_{e}$ ($m_{p}\sim 2\times10^{3}m_{e}$), we find
$\mu\approx m_{e}$ (the error would be something like $10^{-3}m_{e}$).

The frequencies for the emitted photon are then of the form
\begin{equation}
\omega = \frac{1}{\hbar}(E_{n} - E_{m})
\end{equation}
for some $n>m$, which agrees with observation.

But this lacks theoretical basis, and is rather unsatisfactory.

\N{Bohr Model}
Niels Bohr was as unsatisfied as I feel, and sought a theoretical
explanation for the Hydrogen spectrum. We can recall uniform circular
motion in the plane is described by the trajectory
\begin{equation}
(x(t),y(t)) = (r\cos(\omega t), r\sin(\omega t)).
\end{equation}
Its acceleration is obtained by taking the second derivative with
respect to time,
\begin{equation}
\vec{a}(t) = (-\omega^{2}r\cos(\omega t), -\omega^{2}r\sin(\omega t)).
\end{equation}
Observe the magnitude of the acceleration vector is $\omega^{2}r$. If
the only force acting on the electron (considered as a point-particle in
uniform circular motion) is Coulomb's law describing electromagnetism,
\begin{equation}
F = \frac{Q^{2}}{r^{2}}
\end{equation}
(up to some proportionality constant depending on our system of units),
then Newton's second Law gives us the equation of motion for our electron:
\begin{equation}
m_{e}\omega^{2}r = \frac{Q^{2}}{r^{2}}.
\end{equation}
Now we find the frequency $\omega$ by simple algebra
\begin{equation}\label{eq:qm:experimental-background:bohr-omega}
\omega = \sqrt{\frac{Q^{2}}{m_{e}r^{3}}}.
\end{equation}
Happy?

Well, we can do a few more things. We know the magnitude of velocity
$|\vec{v}|=\omega r$, multiplying by mass gives us momentum
$p=m_{e}\omega r$. When we plug in
Eq~\eqref{eq:qm:experimental-background:bohr-omega},
\begin{equation}\label{eq:qm:experimental-background:bohr:momenta}
p = \sqrt{\frac{m_{e}Q^{2}}{r}}.
\end{equation}
We can find the angular momentum, since its magnitude $J=pr$, giving us
\begin{equation}
J = \sqrt{m_{e}rQ^{2}}.
\end{equation}
So far, we haven't introduced anything new.

\N{Quantization Condition}
Bohr now introduces a quantization condition, namely that angular
momentum is an integer multiple of $\hbar$:
\begin{equation}
J = n\hbar = \sqrt{m_{e}rQ^{2}}.
\end{equation}
Solving for $r$ gives us
\begin{equation}
r_{n} = \frac{n^{2}\hbar^{2}}{m_{e}Q^{2}}.
\end{equation}
If we compute the energy for the electron with this radius, then we
recover observed Hydrogen spectrum.

\begin{remark}
What Bohr actually did was a bit more complicated, but if we were more
faithful to Bohr, we would make the big picture rather opaque.
\end{remark}

\begin{exercise}
Recall kinetic energy is $K=\frac{1}{2}m_{e}\vec{v}\cdot\vec{v}$ and the
potential energy for the electron would be $V = -Q^{2}/r$. Compute the
total energy $E = K + V$ with $r_{n}$ and $v=\omega r_{n}$. [Hint: you
  should recover $E_{n}=-R/n^{2}$.]
\end{exercise}

\N{De Broglie Condition}
Louis de Broglie [pronounced ``Broy-Lee''] proposed in 1924 that we
should interpret the Bohr quantization condition of the angular momentum
as a condition on the wave. That is, we should expect
\begin{equation}
2\pi r = n\lambda_{B}
\end{equation}
where $\lambda_{B}$ is the de Broglie wavelength (in units of length), so
the angular momentum 
\begin{equation}
J = rp = n\hbar
\end{equation}
is quantized when we have
\begin{equation}
\lambda_{B} = \frac{h}{p}.
\end{equation}
We write $p = h/\lambda_{B}$ (where $h=2\pi\hbar$) but it is more useful
to introduce a quantity $k=\lambda_{B}/2\pi$ satisfying
\begin{equation}
p = \hbar k.
\end{equation}
For vector quantities,
\begin{equation}
\vec{p} = \hbar\vec{k}.
\end{equation}
We call $\vec{k}$ the \define{Angular Wave Vector} (but the terminology
may vary depending on the reference) and it has units of
$[\mbox{length}]^{-1}$. We can work backwards, starting with this
condition, and derive Bohr's work. 

Specifically we have
\begin{calculation}
2\pi r
\step{de Broglie condition that an orbit is an integer number of periods}
n \frac{2\pi}{k}
\step{since $k=p/\hbar$}
n \frac{2\pi}{p/\hbar}
\step{using Eq~\eqref{eq:qm:experimental-background:bohr:momenta}}
n 2\pi\hbar\sqrt{\frac{r}{m_{e}Q^{2}}}.
\end{calculation}
We can solve this for $r$
\begin{equation}
  \sqrt{r} = \frac{n\hbar}{\sqrt{m_{e}Q^{2}}}\implies
r = \frac{n^{2}\hbar^{2}}{m_{e}Q^{2}}.
\end{equation}
This is precisely the result Bohr obtained.

\begin{remark}
Compare the de Broglie relations to Exercise~\ref{xca:qm:free-particle:de-broglie-relations}.
\end{remark}

\begin{remark}
Just to review, we have seen two \emph{different} theoretical
derivations for the energy spectrum of the Hydrogen atom. They are
equivalent, but de Broglie's relation $\vec{p}=\hbar\vec{k}$ turns out
to be an important relation which will play a critical role in quantum
theory. 
\end{remark}

\subsection{Electron Wave-Like Behaviour}

\M
There are a number of experiments which test the wave-like behaviour of
electrons, I will give two. They are rather ``hairy'' and technical, so
I'll skip the usual format and jump to the punch lines.

\N{Low Energy Electron Diffraction}
Around 1926,
Clinton Davisson and Lester Germer shot a beam of low energy electrons
at a thin sheet of nickel. There was an error in the experimental setup
and the nickel sheet heated up to extraordinary temperatures, forming a
different crystal structure than expected, which changed the scattering
behaviour of the electrons. The intensity of the backscattered electrons
had an angular dependence which suggested there was some interference
pattern which X-ray waves experience when scattering off metal sheets
(which Bragg determined a couple decades earlier).

But this only makes sense if electrons behaved like waves following the
de Broglie relations. This was the earliest strong evidence supporting
de Broglie's conjecture.

\N{Hitachi Double-Slit Experiment}
Akira Tonomura led a team of experimentalists at Hitachi in 1989
performing the double-slit experiment with electrons. The setup had a
screen which marked the point where an individual electron particle hits
it. After running $N$ electrons through this (for various $N$ ranging
from 160 to thousands), a clear interference pattern emerges which
resembles the double-slit experiment for light. This corroborates the
hypothesis that electrons have a wave-like behaviour, despite being
particles. 


\chapter{Special Relativity}

\M
There's a heuristic taught to graduate students:
\begin{equation}
\begin{pmatrix}\mbox{Quantum}\\
\mbox{Field}\\
\mbox{Theory}
\end{pmatrix}
=
\begin{pmatrix}\mbox{Special}\\
\mbox{Relativity}
\end{pmatrix}
+\begin{pmatrix}\mbox{Quantum}\\
\mbox{Mechanics}
\end{pmatrix}.
\end{equation}
This is partially true, but we should review special relativity (if only
to specify our conventions).

Historically, 
\begin{equation}
\begin{pmatrix}\mbox{Special}\\
\mbox{Relativity}
\end{pmatrix}
=
\begin{pmatrix}\mbox{Newtonian}\\
\mbox{Mechanics}
\end{pmatrix}
+
(\mbox{Electromagnetism}).
\end{equation}
This is because electromagnetism determines that the speed of light is
constant in a vacuum, and Newtonian mechanics assumes Galilean
relativity. But what happens if an inertial observer is moving at the
speed of light? Addition of velocities would suggest that bodies could
move faster than light\footnote{Einstein's thought experiment: suppose
you were riding on a bicycle going $0.999c$, and then you turn on your
headlights. What speed will the photons emitted from your headlights
travel?}, which violates results from electromagnetism.

\section{Foundations}

\begin{axiom}[Principle of Relativity]
The laws of physics are identical in all inertial frames. That is to
say, the outcome of any physical experiment is the same when performed
with identical initial conditions relative to any inertial frame.
\end{axiom}

\begin{axiom}
There exists an inertial reference frame in which light signals in
vacuum always travel rectilinearly at constant speed $c$, in all
directions, independently of the motion of the source.
\end{axiom}

\begin{remark}
These are the axioms as formulated in Rinder~\cite[\S4]{Rindler:1991sr}.
Traditionally, introductory texts to special relativity use some version
of these axioms, then ``derive'' results in special relativity.

A more intuitive approach uses $K$-calculus, as first formulated by
Bondi, using spacetime diagrams.
\end{remark}

\M
The best approach to discussing special relativity is the geometric
approach, using spacetime diagrams. However, the point of this chapter
is not to teach special relativity, but to review the Lorentz group as
encoding Lorentz invariance.

The geometric approach uses a 4-dimensional affine space describing
physical space-time, consisting of ``events'' [points]. We do the usual
trick setting up a reference frame as consisting of a point (or trajectory)
which we think of as the ``origin'', then take 4 other distinct points
to construct unit vectors (by taking the displacement from the
``origin'' to each of these points). This gives us a coordinate system.

We can describe time intervals in difference reference frames by
``shooting off'' one photon at the start of the interval (and then
another at the end of the interval). This translates to an interval
$k\,\Delta t$ in the other reference frame, where $k$ is a factor to be
determined. This is done in D'Inverno's \textit{Introducing Einstein's Relativity},
chapters 2 and 3.

\N{On ``Paradoxes''}
One last word, a lot of ``paradoxes'' encountered in physics (like the
``Twin paradox'' in special relativity) are not actually paradoxes: they
are just physical situations where our intuitions fail to describe the
phenomenon properly.

\section{Kinematics}

\N{Four-Vectors} Special relativity differs from Newtonian physics by
working with 4-vectors. The usual vectors we encountered in physics
$\vec{x}$ are 3-vectors in physical space. Now we will use 4-component
vectors of the form $(t, \vec{x})$ or more generally using indices
$(x^{0}, x^{1}, x^{2}, x^{3})$. We can write this using basis vectors
\begin{subequations}
\begin{align}
  A &= (A^{0}, A^{1}, A^{2}, A^{3})\\
&= A^{0}\vec{e}_{0} + A^{1}\vec{e}_{1} + A^{2}\vec{e}_{2} + A^{3}\vec{e}_{3}\\
&= A^{0}\vec{e}_{0} + A^{i}\vec{e}_{i}\\
&= A^{\mu}\vec{e}_{\mu}
\end{align}
\end{subequations}
where $\vec{e}_{\mu}$ are basis vectors, we implicitly sum over repeated
indices, $i = 1,2,3$, and $\mu=0,1,2,3$.

The zeroth component is the time component, the remaining components are
the spatial components.

\begin{remark}
When the components of the four-vector are upstairs $A^{\mu}$,
these are \define{Contravariant} vectors.
\end{remark}

\N{Covectors}
We can also consider ``covectors'' or ``dual vectors'' (which eat in a
4-vector and produce a [real] number). These have components with
downstairs indices $B_{\mu}\vec{f}^{\mu}$ where $\vec{f}^{\mu}$ are the
co-basis covectors.

\M
Normally we can transform the components of a four-vector
$\vec{A}=A^{\mu}\vec{e}_{\mu}$ into the components of a covector by
using the metric $g_{\mu\nu}$ in the usual way:
\begin{equation}
A_{\nu}\vec{f}^{\nu} = (g_{\mu\nu}A^{\mu})\vec{f}^{\nu}.
\end{equation}
We can do the same, but for special relativity the metric is denoted
$\eta_{\mu\nu}$.\marginpar{\footnotesize Minkowski metric $\eta_{\mu\nu}$}
We call this the \define{Minkowski metric} and in Cartesian coordinates
has components
\begin{equation}
\eta_{\mu\nu} = \begin{pmatrix}-1 & 0 & 0 & 0\\
0 & 1 & 0 & 0\\
0 & 0 & 1 & 0\\
0 & 0 & 0 & 1
\end{pmatrix}.
\end{equation}
This is the so-called \define{East-Coast Convention} (particle
physicists multiply this by $-1$ and that's the \emph{West-coast convention}).

\N{Magnitude of Four-Vectors}
When $a^{\mu}$ is any four-vector, its magnitude is the scalar quantity
given by:
\begin{equation}
\|a^{\mu}\|^{2} := \eta_{\mu\nu}a^{\mu}a^{\nu}.
\end{equation}

\begin{definition}
The \define{Four-Position} is the four-vector with components
\begin{equation*}
x^{\mu} = (ct, x^{1}, x^{2}, x^{3}).
\end{equation*}
In Cartesian coordinates, $x^{\mu} = (ct, x, y, z)$.
\end{definition}

\N{Events}
Events in spacetime are described using 4-position vectors. This is an
idealization that events have no ``duration''. We could handle events
with some duration by having one 4-position for the ``start'' and
another 4-position for the ``end''.

\begin{definition}
When we have two 4-position vectors $\vec{A}_{1}=(ct_{1},\vec{r}_{1})$
and $\vec{A}_{2}=(ct_{2},\vec{r}_{2})$, we define the
\define{Displacement Four-Vector} as the four-vector
\begin{equation*}
\Delta\vec{A} = (c\,\Delta t,\Delta\vec{r}) = \vec{A}_{2} - \vec{A}_{1}.
\end{equation*}
For an \define{Infinitesimal Displacement Four-Vector} (or
\emph{Differential Four-Position}), we write
$\D\vec{A}$.
\end{definition}

\M
Suppose we have two events in spacetime separated by an infinitesimal
displacement 4-vector $\D x^{\mu}$. Special relativity demands the
infinitesimal interval,
\begin{equation}
(\D s)^{2} := \eta_{\mu\nu}\,\D x^{\mu}\,\D x^{\nu} = -c^{2}(\D t)^{2} + (\D x)^{2} + (\D y)^{2} + (\D z)^{2},
\end{equation}
must be the same for all inertial observers. Here we use summation
conventions where, when we have an index downstairs and upstairs, we sum
over it (so in our equation, we sum over $\mu$ and $\nu$).

\begin{definition} Let $a^{\mu}$ be a 4-vector.
  \begin{enumerate}
  \item If $\eta_{\mu\nu}a^{\mu}a^{\nu} < 0$, then we call $a^{\mu}$ \define{Time-like}.
  \item If $\eta_{\mu\nu}a^{\mu}a^{\nu} = 0$, then we call $a^{\mu}$ \define{Light-like}.
  \item If $\eta_{\mu\nu}a^{\mu}a^{\nu} > 0$, then we call $a^{\mu}$ \define{Space-like}.
  \end{enumerate}
\end{definition}

\begin{remark}
To see the motivation for these definitions, consider the trajectory of
a photon moving along the $x$-axis in Cartesian coordinates. It would be
$\vec{\gamma}(t)=(ct,ct,0,0)$ and its displacement from the origin would
be always zero for any $t\in\RR$. Therefore, the displacement would be
light-like.

For time-like vectors, consider the displacement 4-vector
$\vec{\alpha}(t)=(ct,0,0,0)$ which stays at the spatial origin. We see
its magnitude is $-c^{2}t^{2}<0$.

For space-like vectors, consider the displacement 4-vector
$\vec{\beta}(t)=(0,ct,0,0)$. Its magnitude is $c^{2}t^{2}>0$.
\end{remark}

\begin{definition}[Minkowski spacetime]
We write $\RR^{3,1}$ for \define{Minkowski Spacetime}, i.e., $\RR^{4}$
equipped with the Minkowski metric $\eta$.
\end{definition}

\begin{remark}
If we were using $+---$ signature conventions, Minkowski space would be
$\RR^{1,3}$. 
\end{remark}

\subsection{Invariance of Differential Displacement}

\M We have the obvious spatial rotations and spatial translations leave
the differential displacement $\D s^{2}$ invariant, since this is a
carry-over from Euclidean geometry. The more interesting case is the
``rotation'' of time and space. We will restrict focus to $\RR^{1,1}$
for simplicity, but the reasoning generalizes to $\RR^{n,1}$ by
composing with rotations to make the ``direction of motion'' a spatial
direction.

%% \M Let us restrict focus to $\RR^{1,1}$ for simplicity. Suppose we
%% change coordinates
%% \begin{equation}
%% \begin{pmatrix}c\,\D t'\\\D x'
%% \end{pmatrix}
%% = \begin{pmatrix}a & b\\k & d
%% \end{pmatrix}
%% \begin{pmatrix}c\,\D t\\\D x
%% \end{pmatrix}.
%% \end{equation}
%% Then writing $(\D s)^{2}$ in the new coordinates:
%% \begin{calculation}
%%   (\D s)^{2}
%% \step{by definition}
%%   -c^{2}(\D t')^{2} + (\D x)^{2}
%% \step{plugging in the values of $\D t'$ and $\D x'$}
%%   -c^{2}(a\,\D t + b\,\D x)^{2} + (k\,\D t + d\,\D x)^{2}
%% \step{expanding terms}
%%   -c^{2}[a^{2}\,(\D t)^{2} + 2ab\,\D t\D x + b^{2}\,(\D x)^{2}]
%%   +[k^{2}\,(\D t)^{2} + 2kd\,\D t\D x + d^{2}\,(\D x)^{2}]
%% \step{collecting terms}
%%   -(c^{2}a^{2} - k^{2})(\D t)^{2} - 2(c^{2}ab - kd)\,\D t\D x + (-c^{2}b^{2}+d^{2})(\D x)^{2}.
%% \end{calculation}
%% Comparing to $(\D s)^{2} = -(\D t)^{2} + (\D x)^{2}$, we have the
%% conditions:
%% \begin{subequations}
%% \begin{align}
%% (c^{2}a^{2} - k^{2}) &= 1\\
%% (c^{2}ab - kd) &= 0\\
%% (-c^{2}b^{2}+d^{2}) &= 1.
%% \end{align}
%% \end{subequations}

%% \M
%% Specifically, when changing from one inertial frame (at rest) to another
%% (moving at speed $v$ in the $x$-direction relative to the first frame),
%% we expect
%% \begin{equation}
%% x' = x - vt.
%% \end{equation}
%% There may be a relativistic correction, a factor $\alpha(v)$ such that
%% for $v\ll c$ we have $\alpha(v)\approx 1$. Then we use
%% \begin{equation}
%% x' = \alpha(v)(x - vt).
%% \end{equation}
%% This gives us
%% \begin{equation}
%% \begin{pmatrix}c\,\D t'\\\D x'
%% \end{pmatrix}
%% = \begin{pmatrix}a & b\\-\alpha v & \alpha
%% \end{pmatrix}
%% \begin{pmatrix}c\,\D t\\\D x
%% \end{pmatrix}.
%% \end{equation}
%% This lets us substitute $k=-\alpha(v)v/c$ and $d=\alpha(v)$. This gives us
%% the system of equations
%% \begin{subequations}
%% \begin{align}
%% (c^{2}a^{2} - v^{2}\alpha^{2}) &= c^{2}\\
%% (c^{2}ab + \alpha^{2}v) &= 0\\
%% (-c^{2}b^{2}+\alpha^{2}) &= 1.
%% \end{align}
%% \end{subequations}
%% This is 3 equations in 3 unknowns, which has a solution.
%% \begin{subequations}
%% \begin{align}
%% \alpha(v) &= \frac{-v/c}{\sqrt{1 - v^{2}/c^{2}}}\\
%% \intertext{and}
%% a = b &= \frac{1}{\sqrt{1 - v^{2}/c^{2}}}.
%% \end{align}
%% \end{subequations}

\N{Problem statement}
Given an inertial observer at rest, and another observer moving with
constant speed $v$, how do the coordinate change from the observer at
rest to the moving observer?

\N{Desiderata}
We want, for $v^{2}\ll c^{2}$, to recover Galilean transformations. That
is, $x' \approx x - vt$ (where primed coordinates are those relative to
the moving observer).

\M Let me try this again. We consider an inertial reference frame moving
with velocity $v$ in the $x$-direction relative to an inertial reference
frame at rest. Then the new coordinates are
\begin{subequations}
\begin{align}
ct' &= a_{11}ct + a_{12}x\\
x' &= a_{22}(x - vt).
\end{align}
\end{subequations}
This reduces to the familiar Galilean transformation for $v\ll c$, which
we demand $a_{22}\approx 1$. Now, for an infinitesimal displacement, its
components are
\begin{subequations}
\begin{align}
c\,\D t' &= a_{11}c\,\D t + a_{12}\,\D x\\
\D x' &= a_{22}(\D x - v\,\D t).
\end{align}
\end{subequations}
Demanding the invariance of the infinitesimal displacement is the same
as the condition:
\begin{equation}
-c^{2}(\D t')^{2} + (\D x')^{2} =-c^{2}(\D t)^{2} + (\D x)^{2}.
\end{equation}
Expanding the primed quantities gives us
\begin{equation}
-(a_{11}^{2}c^{2} - a_{22}^{2}v^{2})(\D t)^{2}
+ 2 (a_{22}^{2}v - a_{11}a_{12}c)\,\D x\,\D t
+ (a_{22}^{2} - a_{12}^{2})(\D x)^{2} =-c^{2}(\D t)^{2} + (\D x)^{2}.
\end{equation}
This gives us the system of 3 equations in 3 unknowns:
\begin{subequations}
\begin{align}
-(a_{11}^{2}c^{2} - a_{22}^{2}v^{2}) &= -c^{2}\\
2 (a_{22}^{2}v - a_{11}a_{12}c) &= 0\\
(a_{22}^{2} - a_{12}^{2}) &= 1.
\end{align}
\end{subequations}
We can solve this system:
\begin{subequations}
\begin{align}
a_{11} = a_{22} &= \frac{1}{\sqrt{1 - v^{2}/c^{2}}}\\
a_{12} &= \frac{-v/c}{\sqrt{1 - v^{2}/c^{2}}}.
\end{align}
\end{subequations}
The standard notation is
\begin{equation}
\gamma(v) = \frac{1}{\sqrt{1 - v^{2}/c^{2}}}
\end{equation}
and
\begin{equation}
\beta(v) = \frac{v}{c}.
\end{equation}
Then our transformation is
\begin{equation}
\begin{pmatrix}ct'\\ x'
\end{pmatrix}
=\begin{pmatrix}\gamma & -\gamma\beta\\
-\gamma\beta & \gamma
\end{pmatrix}
\begin{pmatrix}ct\\ x
\end{pmatrix}.
\end{equation}
This transformation is called a \define{Lorentz Boost}.

\M
Observe when $\beta=1/2$ that
\begin{equation}
\gamma = \frac{2}{\sqrt{3}}\approx 1.1547.
\end{equation}
In other words, for an inertial observer moving half the speed of light,
there is a deviation approximately 15\% from non-relativistic values.

At the Large Hadron Collider, protons are accelerated to speeds of about
$99.9999991\%$ the speed of light --- that is, $1-\beta\approx9\times10^{-9}$.
This gives us $\gamma\approx7453$.

\begin{definition}
We call a velocity \define{Ultra-Relativistic} when $1-\beta\ll1$.
\end{definition}

\begin{remark}
For ultra-relativistic velocities, we see that $1+\beta\approx2$, and
therefore
\begin{equation}
\gamma = \frac{1}{\sqrt{(1 + \beta)(1 - \beta)}}
\approx\frac{1}{\sqrt{2(1 - \beta)}}.
\end{equation}
Simple algebra gives us
\begin{equation}
1 - \beta\approx\frac{1}{2\gamma^{2}}.
\end{equation}
\end{remark}

\begin{exercise}
What is the error in the approximation
$1-\beta\approx(2\gamma^{2})^{-1}$ for $\beta=9/10$? For $\gamma=2$? For $\beta=99/100$?
\end{exercise}

\begin{exercise}
For $0\leq\beta\leq0.9$, find a linear approximation $L(\beta)$ for
$\gamma(\beta)$ such that $L(0)=\gamma(0)$ and
$L(0.9)=\gamma(0.9)$. What is the $L^{2}$-norm of the residual for this approximation?
Is there a better linear approximation?
\end{exercise}

\subsection{Light Cones and Causality}

\begin{definition}\label{defn:relativity:light-cone}
Let $a^{\mu}$ be any event. We define the \define{Lightcone} for $a^{\mu}$
to be the set of events light-like separated from $a^{\mu}$,
\begin{equation}
\mathscr{C} = \{\,b^{\mu}\in\RR^{3,1} \mid b^{\mu}~\mbox{is light-like separated from}~a^{\mu}\,\}.
\end{equation}
We can separate the light-cone in two: events in the future and events
in the past, giving us the \define{Future Light Cone}
\begin{equation}
\mathscr{C}^{+} = \{\,b^{\mu}\in\mathscr{C} \mid b^{0} > a^{0}\,\},
\end{equation}
and the \define{Past Light Cone} consisting of events preceding $a^{\mu}$,
\begin{equation}
\mathscr{C}^{-} = \{\,b^{\mu}\in\mathscr{C} \mid b^{0} < a^{0}\,\}.
\end{equation}
We can take the \define{Closed Light Cone} to consist of all light-like
separated events \emph{and} all time-like separated events, and denote
it by
\begin{equation}
\overline{\mathscr{C}} = \{\,b^{\mu}\in\RR^{3,1} \mid \eta_{\mu\nu}(b^{\mu}-a^{\mu})(b^{\nu}-a^{\nu})\leq0\,\}.
\end{equation}
\end{definition}

\M
The only possible causal influences for an event $a^{\mu}$
\emph{must lie within the past causal light cone} for the event, since
nothing can travel faster than light. For this reason, we call any
four-vector $b^{\mu}$ \define{Causal-like} if it is not space-like
separated from $a^{\mu}$, i.e., if the displacement $b^{\mu} - a^{\mu}$
four-vector is either time-like or light-like.

\subsection{World Lines and Trajectories}

\N{World lines}
A curve $\gamma$ in spacetime is called a \define{World Line}
if its tangent vector is future time-like at each point along the
curve. More generally, we could weaken the condition, and allow tangent
vectors to be causal-like.

Particles move along world lines in special relativity.

\M
If we have a curve $\gamma$ in spacetime such that its tangent vector is
space-like at each point along the curve, we call the curve
\define{Space-like}.

If we have a curve in spacetime such that its tangent vector is
ligh-like at each point along the curve, we call the curve
\define{Light-like}. 


\N{Proper Time}
The time between two events along a world line, according to the
observer moving along the trajectory, is precisely the
\define{Proper Time} and denoted $\tau$. For an infinitesimal
displacement along the trajectory, the infinitesimal change in proper
time is given by
\begin{equation}
c^{2}\,(\D\tau)^{2} = -(\D s)^{2}.
\end{equation}
The proper time interval along a trajectory is given by the integral
\begin{equation}
\Delta\tau = \int_{\gamma}\D\tau =
\int_{\gamma}\frac{\sqrt{-\eta_{\mu\nu}\,\D x^{\mu}\,\D x^{\nu}}}{c}.
\end{equation}

\N{Parametrizing World Lines}
We parametrize a world line by its proper time, and write it as
$x^{\mu}(\tau)$. This gives us a smooth family of four-positions for a
physical body.

Care must be taken when working with world lines for photons (or other
bodies moving at the speed of light), since $\D s = 0$ and therefore
$\D\tau=0$. We need to work with an affine parameter $\lambda$, but
physicists usually just reason along the lines of
\begin{equation}
(\D s)^{2} = 0 = -c^{2}\,(\D t)^{2} + (\D\vec{r})^{2}\implies c^{2}(\D t)^{2}=(\D\vec{r})^{2}.
\end{equation}
This is ``morally right'' but mathematically wrong.

\begin{definition}
Let $x^{\mu}(\tau)$ be a world line. We define its \define{Four-Velocity}
as the four-vector
\begin{equation*}
U^{\mu}(\tau) = \frac{\D x^{\mu}(\tau)}{\D\tau}.
\end{equation*}
\end{definition}

\begin{remark}
Whenever we are tempted to take the time derivative of a quantity, we
really want to take the derivative with respect to \emph{proper time}.
\end{remark}

\N{Lorentz Factor}
We have the familiar Newtonian 3-velocity given by
\begin{equation}
v^{i} = \frac{\D x^{i}}{\D t}.
\end{equation}
We can relate the coordinate time $x^{0}=ct$ with proper time $\tau$ by
the \define{Lorentz Factor} (which is a function of the Newtonian 3-velocity),
\begin{equation}
  \begin{split}
\gamma(\vec{v}) &= \frac{\D t}{\D\tau} = \left(1 - \frac{\eta_{ij}v^{i}v^{j}}{c^{2}}\right)^{-1/2}\\
&=\frac{c}{\sqrt{-\eta_{\mu\nu}U^{\mu}U^{\nu}}}.
  \end{split}
\end{equation}
This allows us to relate the familiar Newtonian 3-velocity
to the four-velocity by
\begin{equation}
U^{\mu} = (c, \gamma(\vec{v})\vec{v}).
\end{equation}

\N{Four-Acceleration}
We can define the four-acceleration analogous to how we defined the
four-velocity as
\begin{equation}
A^{\mu} := \frac{\D U^{\mu}}{\D\tau}.
\end{equation}

\begin{definition}
The \define{Instantaneous Rest Frame} is the frame in which the 3-speed vanishes $v=0$.
\end{definition}

\M
In the instantaneous rest frame we have $v=0$ so $\gamma=1$. The
four-velocity has components $V^{\mu}=(c,\vec{0})$ and the
four-acceleration has components $A^{\mu}=(0,\vec{a})$. Then:
\begin{enumerate}
\item $\eta_{\mu\nu}A^{\mu}A^{\nu}=a^{2}$ where $a$ is the acceleration
  measured in the instantaneous rest frame (that is, the acceleration
  felt by the body);
\item $\eta_{\mu\nu}U^{\mu}U^{\nu}=c^{2}$;
\item $\eta_{\mu\nu}A^{\mu}U^{\nu}=0$ (which can be obtained by
  differentiating the previous result with respect to $\tau$).
\end{enumerate}

\begin{exercise}
Let $v$ be the Newtonian speed of an object, $v = |\D\vec{x}/\D t|$.
Prove
\begin{equation*}
\frac{\D\gamma}{\D t} = \frac{\gamma^{3}v}{c^{2}}\frac{\D v}{\D t}.
\end{equation*}
\end{exercise}

\begin{exercise}
Let $X^{\mu}$ be the position 4-vector, $t$ be coordinate time. Prove
the 4-acceleration may be written as:
\begin{equation*}
A^{\mu} = \gamma\frac{\D}{\D t}\left(\gamma\frac{\D}{\D t}X^{\mu}\right)
\end{equation*}
\end{exercise}

\N{Constant Acceleration}
Consider a particle moving along the $x$-axis with constant acceleration
$a$ as measured in its rest frame at each point.

Let $U^{\mu}=(c\dot{t},\dot{x},0,0)$ where $\dot{x}=\D x/\D\tau$ and
$\dot{t}=\D t/\D\tau$. Then $A^{\mu}=(c\ddot{t},\ddot{x},0,0)$. We have
\begin{subequations}
\begin{align}
-c^{2} &= \eta_{\mu\nu}U^{\mu}U^{\nu} = -c^{2}\dot{t}^{\,2} + \dot{x}^{2}\\
\intertext{and}
a^{2} &= \eta_{\mu\nu}A^{\mu}A^{\nu} = -c^{2}\ddot{t}^{\,2} + \ddot{x}^{2}.
\end{align}
\end{subequations}
Differentiating the first equation gives us $c^{2}\dot{t}\ddot{t}=\dot{x}\ddot{x}$,
then square both sides
$c^{4}\dot{t}^{\,2}\ddot{t}^{\,2}=\dot{x}^{2}\ddot{x}^{2}$, and then
we can eliminate $\dot{x}^{2}=c^{2}\dot{t}^{2}-c^{2}$ and
$\ddot{x}^{\,2}=a^{2}+c^{2}\ddot{t}^{\,2}$ to give us
\begin{equation}
c^{4}\dot{t}^{\,2}\ddot{t}^{\,2}=(c^{2}\dot{t}^{\,2}-c^{2})(a^{2}+c^{2}\ddot{t}^{\,2}).
\end{equation}
We subtract $c^{2}\ddot{t}^{\,2}(\dot{t}^{\,2}-1)$ from both sides to
obtain
\begin{equation}
c^{2}\ddot{t}^{\,2} = a^{2}(\dot{t}^{\,2} - 1)\implies\ddot{t}=\frac{a}{c}\sqrt{\dot{t}^{\,2} - 1}.
\end{equation}
We can solve this differential equation with the initial condition
$t(\tau=0)=0$ and $x(\tau=0)=0$, first integrating to find
\begin{equation}
\dot{t} = \cosh(a\tau/c),
\end{equation}
then integrating again to find
\begin{equation}
t = \frac{c}{a}\sinh(a\tau/c).
\end{equation}
Then from $\dot{x}^{2} = c^{2}(\dot{t}^{\,2}-1)$ we find $\dot{x} = c\sqrt{\cosh^{2}(a\tau/c)-1}=c\sinh(a\tau/c)$
which can be integrated to give us
\begin{equation}
x = \frac{c^{2}}{a}\cosh(a\tau/c).
\end{equation}

\begin{remark}
  Observe that the magnitude of the spatial components for the
  4-velocity (the ``Newtonian speed'') is:
  \begin{equation}
v(\tau) = \frac{\D x}{\D t} = \frac{\dot{x}}{\dot{t}} = c\tanh(a\tau/c)\leq c.
  \end{equation}
\end{remark}

\N{Example}
How do we interpret this? Well, if a space ship leaves a planet at
constant acceleration, and the planet is at rest in the $(x,t)$
coordinates at $x=c^{2}/a$ at $t=0$, then afer some proper time $\tau$
has elapsed the space ship will be located at $x=(c^{2}/a)\cosh(a\tau/c)$.
The time elapsed relative to a clock on the planet will read $t=(c/a)\sinh(a\tau/c)$
where $\tau$ is the time elapsed relative to a clock on the space ship.
For $a\approx g$ and $\tau\approx 10~\mbox{years}$, then $(c/a)=1.03092~\mbox{year}\approx 1~\mbox{year}$
and
\begin{equation}
t\approx\frac{1}{2}\exp(10)~\mbox{years}\approx 11,000~\mbox{years}.
\end{equation}
Observe that
\begin{equation}
x\approx\frac{1}{2}\exp(10)~\mbox{light years}\approx11,000~\mbox{light years}.
\end{equation}
Thus, although only 10 years elapsed on the space ship moving at 1~g,
relative to clocks on the planet which launched the space ship it would
appear to have been 11,000 years. In Newtonian mechanics, the space ship
would be about $50$ light years away (since $a\approx c~\mbox{year}^{-1}$).

\begin{remark}
This may seem counter-intuitive (and it is), but we should emphasize it
is difficult to have a space ship move at a constant 1~g acceleration.
Indeed, \emph{constant acceleration} is the bit belonging to science
fiction (at the moment).
\end{remark}

\begin{exercise}
In Robert Rath's \textit{The Infinite and the Divine}, a super-advanced
species of aliens (who existed hundreds of millions of years ago)
transferred their consciousness into robots. This problem concerns one
of their spaceship's trajectory using their advanced technology.

We are told the initial velocity of the spaceship is $1000$ leagues per
hour, but after a decade [literally 10 years] the spaceship reaches a
billion [$1.00\times10^{9}$] leagues per hour. Note: 1 league per hour
is approximately 1.54333 meters per second.
\begin{enumerate}
\item Assume constant acceleration. Find the constant acceleration,
  assuming the universe is Newtonian, and forces acting on the spaceship
  are negligible.
\item Using the solution to the previous problem, how far did the
  spaceship travel? The story suggests it is the distance between
  several stars. If there is 5 light years between stars in the galaxy,
  is this reasonable? If not, how fast would the spaceship have to
  travel?
\item If we allow for special relativity to apply, then time dilation
  presumably occurs. Suppose the ten years in our problem refers to the
  proper time as measured by the ship's chronometer [clock]. Does this
  affect the distance the ship travels? If so, how far does this ship
  travel?
\end{enumerate}
\end{exercise}

\section{Dynamics}

\M
Often the dynamics of particles are omitted in discussions of special
relativity, because things getting complicated conceptually.

\begin{definition}
Let $x^{\mu}(\tau)$ be the world line for a massive body.
Then the \define{Rest Mass} (or \emph{invariant mass}) for the body is
the mass $m_{0}$ as measured in its reference frame.
\end{definition}

\begin{remark}
Some authors use a notion of \emph{relativistic mass}
$m = \gamma(\vec{v})m_{0}$, which is mildly controversial. 
\end{remark}

\N{Four-Momentum}
For a massive particle of rest mass $m_{0}$, its \define{Four-Momentum}
$\vec{P}$ is defined as the product of the rest mass and its
four-velocity, i.e.,
\begin{equation}
\vec{P} := m_{0}\vec{U}.
\end{equation}
We can write out its explicit components
\begin{equation}
\vec{P} = m_{0}\gamma(\vec{v})(c, \vec{v}) = (E/c, \vec{p}).
\end{equation}
Here the total energy of the moving particle is given by
\begin{equation}
E = \gamma(\vec{v})m_{0}c^{2},
\end{equation}
and the total (relativistic) 3-momentum is
\begin{equation}
\vec{p} = m_{0}\gamma(\vec{v})\vec{v}.
\end{equation}

\N{Energy--Momentum Relation}
We have the energy--momentum relation (or \define{Mass--Shell Relation})
be
\begin{equation}
E^{2} = c^{2}\vec{p}\cdot\vec{p} + \bigl(m_{0}c^{2}\bigr)^{2}.
\end{equation}
Equivalently, we have this relation describe the magnitude of the
four-momentum as a constant:
\begin{equation}
\eta_{\mu\nu}p^{\mu}p^{\nu} = -m_{0}^{2}c^{2}.
\end{equation}

\begin{ddanger}
In special relativity, we can meaningfully talk about the Center-of-Mass
reference frame for a system of particles. This is especially useful for
scattering problems. However, if we tried to carry this notion over to
General Relativity, then we run into problems because it is a nonlocal
concept. 
\end{ddanger}

\begin{exercise}
From the mass-shell relation and $\vec{P} = (E/c, \vec{p})$, deduce
$E = \gamma(\vec{v})m_{0}c^{2}$ and 
$\vec{p}\cdot\vec{p}=\pm(\gamma(\vec{v})^{2}-1)m_{0}^{2}c^{2}$.
Determine the correct sign in that second relation.
\end{exercise}

\begin{exercise}
Suppose we have two massive particles with four-momenta $\vec{P}_{1}$
and $\vec{P}_{2}$ and relative speed $v$. Determine
$\vec{P}_{1}\cdot\vec{P}_{2}=\eta_{\mu\nu}P^{\mu}_{(1)}P^{\nu}_{(2)}$ in
terms of their rest mass $m_{01}$ and $m_{02}$ and relative speed $v$.
Hint: if $\vec{P}_{1}=\vec{P}_{2}$ and $v=0$, then you should recover
the mass--shell relation.
\end{exercise}

\N{Four-Force}
We can define the four-force as the four-vector
\begin{equation}
\vec{F} = \frac{\D\vec{P}}{\D\tau}.
\end{equation}
As an immediate consequence of the mass--shell relation, we find
\begin{equation}
\vec{F}\cdot\vec{P} = \eta_{\mu\nu}F^{\mu}P^{\nu} = 0.
\end{equation}

\N{Lagrangian for Point-Particle}
We can write the Lagrangian for a massive point-particle with rest mass
$m_{0}$ as
\begin{equation}
L = cm_{0}\sqrt{-\eta_{\mu\nu}\dot{x}^{\mu}\dot{x}^{\nu}} - V
\end{equation}
where $\dot{x}^{\mu} = \D x^{\mu}/\D\tau = U^{\mu}$, and $V$ is the potential
energy term. The action is then 
\begin{equation}
I = \int L\,\D\tau.
\end{equation}
Varying the action with respect to $\delta x^{\mu}$ then gives us the
equations of motion.

For massless particles, care must be taken with the parametrization, as
well as using its four-momentum to write
\begin{equation}
L = c\sqrt{-\eta_{\mu\nu}P^{\mu}P^{\nu}} - V.
\end{equation}
We can use the relation (which holds for both massive and massless particles):
\begin{equation}
\frac{\D x^{\mu}}{\D t} = \frac{P^{\mu}}{P^{0}}.
\end{equation}

\begin{danger}
This is the correct Lagrangian to work with, especially if we want to
quantize it. There is some subtlety with it, which we should confess
openly: it is a constrained system. To see this, compute the Hamiltonian
for a free massive relativistic particle. It will vanish. This is
because time is a coordinate (proper time is a parameter), and its
conjugate momentum is ``the Hamiltonian''. So we end up with a
constraint. 
\end{danger}

\begin{ddanger}
Some authors insist that canonical mechanics for special relativistic
systems ``breaks Lorentz invariance'', which is not really true. If
you've picked $\mu=0$ to be the time component for four-vectors, then
you've also ``broken Lorentz invariance'' just as much as canonical
mechanics has. We can describe a Lorentz boost as a canonical
transformation (which preserves Lorentz invariance as much as anything
else). This is just offered as a lazy and sloppy justification for using
the path integral formalism, which makes no coherent sense.
\end{ddanger}

\section{Group Theoretic Description}

\begin{definition}
The \define{(Homogeneous) Lorentz Group} for $\RR^{n,1}$ is the group
generated by spatial rotations and Lorentz boosts --- that is, the indefinite
orthogonal group $\O(n,1)$. A generic element of the Lorentz group is
called a \define{Lorentz Transformation}.
\end{definition}

\begin{exercise}
Prove that this actually is a group. That is, the composition of Lorentz
transformation are a Lorentz transformation; inverses of Lorentz
transformations are Lorentz transformations.
\end{exercise}

\M
There are 4 connected components to the Lorentz group $\O(3,1)$ which
are related by parity operator $P$ and time reversal operator $T$, which
are represented by the matrices (acting on the ``obvious'' fundamental
representation as):
\begin{subequations}
\begin{align}
P &= \diag(1, -1, -1, -1)\\
T &= \diag(-1, 1, 1, 1).
\end{align}
\end{subequations}

\begin{definition}
The \define{Poncar\'e Group} for $\RR^{n,1}$ is the group generated by
translations in spacetime [by a constant displacement], spatial
rotations, and Lorentz boosts; it is given by the group
$G = \RR^{n,1}\rtimes\O(n,1)$.
\end{definition}

\begin{remark}
Weinberg calls the Poncar\'e Group the \emph{Inhomogeneous Lorentz Group}.
\end{remark}

\M
We can always consider how an element of $g\in\O(3,1)$ will act on any
4-position $x\in\RR^{3,1}$ ``in the obvious way'' as a Lorentz
transformation. We will write this as $g\cdot x$. Specifically this acts
by means of a $4\times4$ matrix ${\Lambda^{\alpha'}}_{\mu}$ sending
$x^{\mu}\to x^{\alpha'} = f^{\alpha'}(x^{\mu}) = (\Lambda x)^{\alpha'}$.
Explicitly, using Einstein summation convention,
\begin{equation*}
x^{\alpha'} = {\Lambda^{\alpha'}}_{\mu}x^{\mu}.
\end{equation*}

\N{Inverse Transformation}
We can compose Lorentz transformations to find the inverse
transformation ${\Lambda^{\nu}}_{\alpha'}$ which satisfies
\begin{equation}
{\Lambda^{\nu}}_{\alpha'}{\Lambda^{\alpha'}}_{\mu}={\delta^{\nu}}_{\mu}.
\end{equation}
For covariant vectors, we need to use
% We then have
${[\transpose{(\Lambda^{-1})}]^{\mu}}_{\alpha'} = {\Lambda_{\alpha'}}^{\mu}$
be the transpose of the inverse Lorentz transformation for $\Lambda$.
This is because covariant vectors transform under the dual
representation. 


\M
The Lorentz group acts on a scalar field $\varphi(x)$ by
\begin{equation}
g\cdot\varphi(x) = \varphi\bigl(g^{-1}\cdot x\bigr).
\end{equation}
This is because a scalar field is ``just a function'', and this is how
groups act on functions.

\N{Action on Vector Fields}
For a vector field with components $A^{\mu}(x)$, a Lorentz
transformation $g\in\O(3,1)$ acts on this in a ``mixed'' way. We have
the matrix ${\Lambda^{\alpha'}}_{\mu}$ describe the action $(g\cdot x)^{\alpha'} = {\Lambda^{\alpha'}}_{\nu}x^{\nu}$,
which is the change of coordinates $x^{\mu}\to x^{\alpha'}=f^{\alpha'}(x^{\mu})$,
so therefore a contravariant vector field transforms as:
\begin{equation}
[(g\cdot \vec{A})(x)]^{\alpha'} = {\Lambda^{\alpha'}}_{\nu}A^{\nu}(g^{-1}\cdot x).
\end{equation}
It acts on the vector as a whole as we would expect, and on each
component as a function.

For a covariant vector $A_{\mu}(x)$, we see it transforms as
\begin{equation}
[(g\cdot \vec{A})(x)]_{\alpha'} = {\Lambda_{\alpha'}}^{\mu}A_{\mu}(g^{-1}\cdot x)
\end{equation}
where ${\Lambda_{\alpha'}}^{\mu}$ is the matrix inverse of the Lorentz
transformation. It satisfies
\begin{equation}
{\Lambda_{\nu}}^{\alpha'}{\Lambda^{\nu}}_{\beta'} = {\delta^{\alpha'}}_{\beta'}.
\end{equation}

\N{Action on Tensor Fields}
More generally, for a rank $n$ tensor with components $[\mathsf{T}(x)]^{\mu_{1}\cdots\mu_{n}}=T^{\mu_{1}\cdots\mu_{n}}(x)$,
the element $g\in\O(3,1)$ acts as
\begin{equation}
[(g\cdot \tens{T})(x)]^{\alpha_{1}'\cdots\alpha_{n}'}
= {\Lambda^{\alpha_{1}'}}_{\nu_{1}}(\cdots){\Lambda^{\alpha_{n}'}}_{\nu_{n}}
T^{\nu_{1}\cdots\nu_{n}}(g^{-1}\cdot x).
\end{equation}
For covariant indices, we multiply by matrix inverses of $\Lambda$.
These actions are all basic representation theory.

\N{Lorentz Invariance}
We call a quantity $Q(x)$ \define{Lorentz Invariant} if for each $g\in\O(3,1)$
we have
\begin{equation}
g\cdot Q(x) = Q(x).
\end{equation}
For example $\D s^{2}$ is Lorentz invariant.

\section{Electromagnetism}

\N{Maxwell's Equations}
Maxwell's equations which we learn in Physics 101 are, in Gaussian
units, for statics:
\begin{subequations}
\begin{align}
\nabla\cdot\vec{E} &= 4\pi\rho_{e}\\
\nabla\cdot\vec{B} &= 0\\
\intertext{and for dynamics:}
-\frac{\partial\vec{E}}{\partial t} + c\nabla\times\vec{B} &= 4\pi\vec{j}\\
\frac{\partial\vec{B}}{\partial t} + c\nabla\times\vec{E} &= 0.
\end{align}
\end{subequations}
Here $\rho_{e}$ is the electric charge density, and $\vec{j}$ is
the electric current density.
The exact form of these equations depend on the units used. In SI units,
the coefficients need to be altered:
\begin{subequations}
\begin{align}
\nabla\cdot\vec{E} &= \frac{\rho_{e}}{\varepsilon_{0}}\\
\nabla\cdot\vec{B} &= 0
\end{align}
\end{subequations}
\begin{subequations}
\begin{align}
-\mu_{0}\varepsilon_{0}\frac{\partial\vec{E}}{\partial t} + \nabla\times\vec{B} &= \mu_{0}\vec{j}\\
\frac{\partial\vec{B}}{\partial t} + \nabla\times\vec{E} &= 0,
\end{align}
\end{subequations}
where $\varepsilon_{0}$ is the permittivity of free space, $\mu_{0}$ is
the permeability of free space, and $c=1/\sqrt{\varepsilon_{0}\mu_{0}}$
is the speed of light in a vacuum.

\N{Potentials}
It is common to introduce the electric potential $\varphi(\vec{x})$
and magnetic potential $\vec{A}(\vec{x}, t)$. Then the electric field is
taken to be:
\begin{equation}
\vec{E} = -\vec{\nabla}\varphi - \frac{\partial\vec{A}}{\partial t}.
\end{equation}
We call $\varphi$ the \define{Electric Potential} and $\vec{A}$ the
\define{Magnetic Potential}.
The magnetic field is,
\begin{equation}
\vec{B} = \nabla\times\vec{A}.
\end{equation}

\begin{remark}
Any time-dependence the electric field enjoys may be traced back to the
magnetic potential contribution.
\end{remark}

\N{Gauge Invariance}
These potentials are not unique. We could, for any function
$\lambda(\vec{x},t)$, consider the potentials
\begin{subequations}
\begin{align}
\varphi' &= \varphi - \frac{\partial\lambda}{\partial t}\\
\vec{A}' &= \vec{A} + \vec{\nabla}\lambda.
\end{align}
\end{subequations}
This arbitrariness is an example of a gauge symmetry, a redundancy
describing the same physical conditions. We can eliminate this
redundancy (a process called \define{Gauge-Fixing}) by adding an
additional condition.

\N{Lorenz Gauge}
We impose the condition
\begin{equation}
\vec{\nabla}\cdot\vec{A} + \frac{1}{c}\frac{\partial\varphi}{\partial t}=0.
\end{equation}
This is the \emph{Lorenz Gauge} (note the lack of ``t'', because it is
named after the Dutch physicist Ludvig Lorenz, not to be confused with
the other Dutch physicist Hendrik Lorentz [of ``Lorentz transformation''
fame]). Maxwell's equations
simplify to 
\begin{subequations}
\begin{align}
\nabla^{2}\varphi - \frac{1}{c^{2}}\frac{\partial^{2}\varphi}{\partial t^{2}}
&= -\frac{\rho_{e}}{\varepsilon_{0}}\\
\nabla^{2}\vec{A} - \frac{1}{c^{2}}\frac{\partial^{2}\vec{A}}{\partial t^{2}}
&= -\mu_{0}\vec{j}.
\end{align}
\end{subequations}
This makes manifest the wave structure of electric and magnetic fields,
since these are wave equations.

\begin{remark}
There are other gauge choices, and the exact form of the equations
depend on the gauge choice.
\end{remark}

\N{Four-Potential}
The first step towards describing electromagnetism in special relativity
is to use four-vectors. The potentials together form a four-vector
called the \define{Four-Potential} $A^{\mu} = (\varphi, \vec{A})$.
Conventions vary, some authors take $A^{t}=c^{-1}\varphi$. This implies
\begin{equation}
A_{\mu} = (-\varphi, \vec{A}).
\end{equation}

\begin{remark}
We can assemble a one-form called the Potential one-form
$A = \eta_{\mu\nu}A^{\mu}\,\D x^{\nu}$.
\end{remark}

\N{Partial Derivatives}
We have, in our conventions,
\begin{equation}
\partial^{\mu} := (-c^{-1}\partial_{t}, \vec{\nabla}).
\end{equation}
Then, in Cartesian coordinates,
\begin{equation}
\partial_{\nu} = \eta_{\mu\nu}\partial^{\mu} = (c^{-1}\partial_{t}, \vec{\nabla}).
\end{equation}

\N{Field-Strength Tensor}
We can form the field-strength tensor from the four-potential as
\begin{subequations}
\begin{equation}
F^{\mu\nu} := \partial^{\mu}A^{\nu} - \partial^{\nu}A^{\mu},
\end{equation}
or with indices downstairs
\begin{equation}
F_{\mu\nu} := \partial_{\mu}A_{\nu} - \partial_{\nu}A_{\mu}.
\end{equation}
\end{subequations}
The field strength tensor's components are then
\begin{subequations}
\begin{align}
F_{0j} &= \partial_{0}A_{j} - \partial_{j}A_{0}\\
&= \partial_{t}A_{j} - \partial_{j}(-\varphi)\\
&= -E_{j},
\end{align}
\end{subequations}
and
\begin{equation}
F_{ij} = \partial_{i}A_{j} - \partial_{j}A_{i} = \begin{pmatrix}
 0     &  B_{z} & -B_{y}\\
-B_{z} &  0     & B_{x}\\
 B_{y} & -B_{x} & 0
\end{pmatrix}
\end{equation}
(where $j$ indexes the columns, $i$ the rows). Then
\begin{equation}
F_{\mu\nu} = \begin{pmatrix}F_{00} & F_{0j}\\
F_{i0} & F_{ij}
\end{pmatrix} = \begin{pmatrix}
 0       & -E_{x}/c & -E_{y}/c & -E_{z}/c\\
 E_{x}/c &  0       &  B_{z}   & -B_{y}\\
 E_{y}/c & -B_{z}   &  0       &  B_{x}\\
 E_{z}/c &  B_{y}   & -B_{x}   &  0
\end{pmatrix}.
\end{equation}

\begin{exercise}
Compute the components of $F^{\mu\nu}=\eta^{\alpha\mu}\eta^{\beta\nu}F_{\alpha\beta}$.
\end{exercise}

\begin{exercise}
Expression $F_{\mu\nu}F^{\mu\nu}$ in terms of $\vec{E}$ and $\vec{B}$.
\end{exercise}

\N{Recovering Maxwell's Equations}
We see that
\begin{calculation}
  \partial^{\nu}F_{0\nu}
\step{since $F_{00}=0$}
  \partial^{j}F_{0j}
\step{since $F_{0j}=-E_{j}$}
  -\partial^{j}E_{j}
\step{by Gauss's Law}
  -4\pi\rho_{e}.
\end{calculation}
This gives us the first of Maxwell's equations.
We also see
\begin{calculation}
  \partial^{\nu}F_{i\nu}
\step{breaking $\mu$ up into $(0,j$)}
  \partial^{0}F_{i,t} + \partial^{j}F_{i,j}
\step{since $F_{i,t} = E_{i}$ and $F_{i,j} = \epsilon_{ijk}B^{k}$}
  \partial^{0}E_{i} + \partial^{j}\epsilon_{ijk}B^{k}
\step{since $\partial^{0}=-c^{-1}\partial_{t}$ and
    $\partial^{j}\epsilon_{ijk}B^{k} = (\nabla\times\vec{B})_{i}$}
  \frac{-1}{c}\partial_{t}E_{i} + \partial^{j}\epsilon_{ijk}B^{k}
\step{using Electrodynamic equation of motion}
  4\pi J_{i}.
\end{calculation}
This motivates a 4-vector for the current density called the \define{Four-Current}:
\begin{equation}
J_{\mu} = (-\rho_{e}, \vec{j}).
\end{equation}
Therefore two of Maxwell's equations for the electric field (both
electrostatics and electrodynamics) are encoded by
\begin{equation}
\boxed{\partial^{\nu}F_{\mu\nu} = 4\pi J_{\mu}.}
\end{equation}
We just need to express the remaining equations using the field strength
tensor.

\N{Equations for Magnetism}
Magnetodynamics may be written as
\begin{equation}
\partial_{t}\epsilon_{ijk}B^{k} + \partial_{i}E_{j} - \partial_{j}E_{i}
= 0
\end{equation}
for \emph{any} $i$, $j$, $k=1,2,3$. Then
\begin{calculation}
0
\step{Magnetodynamics equation}
\partial_{t}\epsilon_{ijk}B^{k} + \partial_{i}E_{j} - \partial_{j}E_{i}
\step{since $F_{ij} = \epsilon_{ijk}B^{k}$}
\partial_{t}F_{ij} + \partial_{i}E_{j} - \partial_{j}E_{i}
\step{since $E_{j}=F_{j0}$ and $-E_{i}=F_{0i}$}
\partial_{t}F_{ij} + \partial_{i}F_{j0} + \partial_{j}F_{0i}.
\end{calculation}
This suggests more generally (replacing the spatial indices with
4-indices and $t$ with another distinct index)
\begin{equation}\label{eq:relativity:electromagnetism:guessed-magnetic-equations}
0 \stackrel{???}{=} \partial_{\alpha}F_{\beta\gamma} + \partial_{\beta}F_{\gamma\alpha} + \partial_{\gamma}F_{\alpha\beta}.
\end{equation}
For $\alpha=0$, $\beta=i$, $\gamma=j$, this recovers Magnetodynamics.
Does this recover the magnetostatic Maxwell equation?

If any two indices are repeated (say $\alpha=\beta$), then $F_{\mu\nu}$
being antisymmetric implies Eq~\eqref{eq:relativity:electromagnetism:guessed-magnetic-equations}
holds trivially. Similarly, if $\alpha=\beta=\gamma$, then the equation
holds. Therefore, the only remaining cases are spatial indices. We see
\begin{calculation}
\partial_{x}F_{yz} + \partial_{y}F_{zx} + \partial_{z}F_{xy}
\step{since $F_{yz}=B_{x}$, $F_{zx}=B_{y}$, $F_{xy}=B_{z}$}
\partial_{x}B_{x} + \partial_{y}B_{y} + \partial_{z}B_{z}
\step{this is the divergence of the magnetic field}
\nabla\cdot\vec{B}
\step{by magnetostatics}
0.
\end{calculation}
Permuting the indices will just multiply through by $-1$, which doesn't
change the result. Therefore we conclude
\begin{equation}
\partial_{i}F_{jk} + \partial_{j}F_{ki} + \partial_{k}F_{ij} = 0,
\end{equation}
and moreover this encodes the magnetostatic Maxwell equation. Combining
things together, we find the magnetic Maxwell equations are:
\begin{equation}\label{eq:relativity:magnetic-eom}
\boxed{\partial_{\alpha}F_{\beta\gamma} + \partial_{\beta}F_{\gamma\alpha} + \partial_{\gamma}F_{\alpha\beta}
= 0.}
\end{equation}

\begin{exercise}
Prove Eq~\eqref{eq:relativity:magnetic-eom} holds identically for $F_{\mu\nu}=\partial_{\mu}A_{\nu}-\partial_{\nu}A_{\mu}$,
and therefore we don't need to worry about it when writing down a
Lagrangian for the electromagnetic field.
\end{exercise}

\subsection{Lagrangian for the Electromagnetic Field}

\begin{exercise}
Prove $\displaystyle\frac{\partial \left(\partial_{\mu} A_{\nu}\right)}{\partial\left(\partial_{\rho} A_{\sigma}\right)}= \delta_{\mu}^{\rho} \delta_{\nu}^{\sigma}$.
\end{exercise}

\begin{exercise}
Compute $\displaystyle\frac{\partial(F_{\alpha\beta}F^{\alpha\beta})}{\partial(\partial_{\mu}A_{\nu})}$.
Remember: $F^{\alpha\beta} = \eta^{\alpha\rho}\eta^{\beta\sigma}F_{\rho\sigma}$.
\end{exercise}

\begin{exercise}
Is it true or not that 
$\displaystyle\frac{\partial(F_{\alpha\beta}F^{\alpha\beta})}{\partial(A_{\nu})}=0$?
\end{exercise}

\M
Combining the results from these exercises, you can show the Lagrangian
density for Electromagnetism is
\begin{equation}\label{eq:relativity:lagrangian-for-em}
\mathcal{L} = \frac{-1}{4}F_{\mu\nu}F^{\mu\nu} + A_{\mu}J^{\mu}.
\end{equation}
The Euler--Lagrange equations for this would be (summing over $\beta$):
\begin{equation}
\partial_{\beta}\left[\frac{\partial\mathcal{L}}{\partial(\partial_{\beta}A_{\alpha})}\right]
-\frac{\partial\mathcal{L}}{\partial A_{\alpha}} = 0.
\end{equation}
This should recover $\partial_{\alpha}F^{\alpha\beta}=J^{\beta}$.

\begin{exercise}
Prove the Lagrangian density in Eq~\eqref{eq:relativity:lagrangian-for-em}
is Lorentz invariant.
\end{exercise}

\begin{exercise}
\begin{enumerate}
\item Find the canonically conjugate momenta density $\pi_{i} = \partial\mathcal{L}/\partial(\partial_{t}A^{i})$.
\item Rewrite $\mathcal{L}$ using $A^{i}$, $\pi_{j}$
\item Compute the Hamiltonian density $\mathcal{H} = \pi_{i}\partial_{t}A^{i}-\mathcal{L}$.
\end{enumerate}
\noindent \textsc{Hint}: you should have a constrained system, since the
Gauss law does not involve time derivatives. (Therefore we should expect
to find a constraint equivalent to the Gauss law.)
\end{exercise}


\N{References}
For the uninitiated, Taylor and Wheeler~\cite{Taylor:1992sp} is a great
introduction. Rindler~\cite{Rindler:1991sr} is a good review.
Misner, Thorne, and Wheeler's \textit{Gravitation} (ch.2) discusses
special relativity and we rely on its index notation conventions.
We also have double checked calculations against Srednicki~\cite{Srednicki:2007qs}.

There are a lot of subtleties to the Hamiltonian analysis of
electromagnetism and, more generally, any theory with symmetries. The
best review of the topic is the first 5 or so chapters of Henneaux and Teitelboim~\cite{Henneaux:1992ig}.

\N{TODO: Index Correct?}
I should double check the indices are correct for the Lorentz
transformation ${\Lambda^{\mu}}_{\nu}$ --- or is this the inverse of the
Lorentz transformation? After some investigation, we should have
$(\Lambda x)^{\mu} = {\Lambda^{\mu}}_{\nu}x^{\nu}$.
Rows are indexed by covariant indices, columns are indexed by
contravariant indices. MTW \S2.9 uses the notation
$x^{\alpha'} = {\Lambda^{\alpha'}}_{\mu}x^{\mu}$ and the inverse
transformation by $x^{\mu} = {\Lambda^{\mu}}_{\alpha'}x^{\alpha'}$.
Therefore composing these guys gives us
${\Lambda^{\mu}}_{\alpha'}{\Lambda^{\alpha'}}_{\nu}={\delta^{\mu}}_{\nu}$
and ${\Lambda^{\alpha'}}_{\mu}{\Lambda^{\mu}}_{\beta'} = {\delta^{\alpha'}}_{\beta'}$.

\N{TODO: Scattering}
It is probably good to discuss $2\to2$ scattering in special relativity,
since that's the basis of a lot of particle physics experiments.

%%
%% rqm.tex
%% 
%% Made by Alex Nelson
%% Login   <alex@tomato>
%% 
%% Started on  Tue Jul 21 10:59:59 2009 Alex Nelson
%% Last update Tue Jul 21 10:59:59 2009 Alex Nelson
%%
\documentclass[final]{amsart}
\usepackage{fly}
\usepackage{danger}
\usepackage{paralist}


\title{Notes on Relativistic Quantum Mechanics}
\date{July 21, 2009}
\email{pqnelson@gmail.com}
\author{Alex Nelson}
\begin{document}
\maketitle\footnote{Note we are using the West Coast convention, i.e. + - - - metric signature, and setting $c=1$ and $\hbar=1$.}
\tableofcontents

\section{One Particle Systems: Mathematical Formalism}
%%
%% oneParticleState.tex
%% 
%% Made by Alex Nelson
%% Login   <alex@tomato>
%% 
%% Started on  Tue Jul 21 12:54:12 2009 Alex Nelson
%% Last update Tue Jul 21 12:54:12 2009 Alex Nelson
%%

The simplest system to consider is a single particle. The
function space used to model quantum-mechanical states is a
Hilbert Space $\mathcal{H}$ of square integrable functions on the
physical space (denoted by $\mathcal{C}$):
\begin{equation}%\label{eq:}
L^{2}(\mathcal{C}) = \left\{f\,:\;\int_{\mathcal{C}}|f(\overline{x})|^{2}d^{3}\overline{x}<\infty\right\}
\end{equation}
Note that in all fairness, $\mathcal{H}$ can be written in either
position coordinates $\overline{x}$ or momentum coordinates
$\overline{p}$. The relationship between the position-space and
momentum-space is precisely the familiar Fourier transform:
\begin{equation}%\label{eq:}
\mathcal{F}(f)(\overline{p}) \stackrel{\text{def}}{=} \int e^{i\overline{x}\cdot\overline{p}}f(\overline{x})d^{3}\overline{x}
\end{equation}
Despite the change of variables, $\mathcal{F}$ sends
$\mathcal{H}$ to itself, so both $f$ and its Fourier transform
$\mathcal{F}(f)$ are in $\mathcal{H}$.
\begin{rmk}
It should be emphasized that if $f$ is square-integrable, then
$e^{i\overline{x}\cdot\overline{p}}f(\overline{x})$ is
square-integrable \emph{but not necessarily integrable!} That is,
we have no guarantee that
$e^{i\overline{x}\cdot\overline{p}}f(\overline{x})\in L^{1}(\mathcal{C})$.

To define the Fourier transform on $\mathcal{H}$, we should first
define it on some suitably nice subspace of $\mathcal{H}$
(e.g. the space of smooth functions with ``compact support'' ---
i.e. they are zero outside of a compact subset of their
domain). Then we observe that the Fourier transform is an
isometry (up to some scale factor) on our nice subspace, so we
extend this isometry from our nice subspace to all of $\mathcal{H}$.
\end{rmk}

We represent the observables by operators. More relevantly, the
position operators $\widehat{x}_{m}$ and momentum operators
$\widehat{p}_{m}$ are represented in position-space by
multiplication by the coordinate fuhnctions $x_{m}$ and the
partial derivative operators $-i\partial_{m}$
(respectively). Observe also that the Fourier transform converts
multiplication by $x_{m}$ on functions of $\overline{x}$ into the
differential operators $-i\partial_{m}$ on functions of
$\overline{p}$:
\begin{equation}%\label{eq:}
\mathcal{F}(x_{m}f)(\overline{p})=-i\partial_{m}\mathcal{F}(f)(\overline{p}).
\end{equation}

The natural question to ask is ``What are the eigenstates of
these operators?'' Well, in position space, we find the position
eigenstates are just delta functions
\begin{subequations}
\begin{align}
(\widehat{x}_{m}\delta_{\overline{q}})(\overline{x}) &=
  \widehat{x}_{m}\delta(\overline{x}-\overline{q})\\
&= q_{m}\delta(\overline{x}-\overline{q})\\
&= (q_{m}\delta_{\overline{q}})(\overline{x})
\end{align}
\end{subequations}
Similarly, for the eigenstates of the momentum operators
$\widehat{p}_{m}$, we see that the eigenstates in position-space
are $e_{\overline{p}}(\overline{x})$:
\begin{subequations}
\begin{align}
(\widehat{p}_{m}e_{\overline{p}})(\overline{x}) &= -i\partial_{m}\exp(i\overline{p}\cdot\overline{x})\\
&= p_{m}\exp(i\overline{p}\cdot\overline{x})\\
&= (p_{m}e_{\overline{p}})(\overline{x}).
\end{align}
\end{subequations}

But we have just two minor problems: \begin{inparaenum}
\item neither $\widehat{x}_{m}$ nor $\widehat{p}_{n}$ act on all
  of $\mathcal{H}$, and
\item $\mathcal{H}$ doesn't contain the eigenstates of either
  operators.
\end{inparaenum}
We can solve the first problem fairly easily --- we'll consider
the subspace $S\subset{\mathcal{H}}$ where the operators map $S$
to itself. Similarly, we resolve the second problem by defining
the kets as elements of $S^{*}$, the space of continuous
antilinear functionals on $S$. Since $\widehat{p}_{n}$ acts on
all functions of $S$, these functions must be infinitely
differentiable, and so $S^{*}$ will contain the
$\delta$-functions and all their derivatives. Similarly, by
taking the Fourier Transform, since $\widehat{x}_{m}$ acts on
$S$, it follows that $S^{*}$ will contain exponential functions
$\exp(i\overline{p}\cdot\overline{x})$. 

Instead of a single Hilbert space, we end up with a triple
\begin{equation}%\label{eq:}
S\subset{\mathcal{H}}\subset{S^{*}}
\end{equation}
The physical states live in $S$, and the operator eigenstates
live in $S^{*}$. With appropriate demands on the space $S$, 
this triple ends up being a \emph{Rigged Hilbert Space}~\cite{delamadrid}~\cite{Madrid:2004zy}. 
In this context ``Rigged'' \emph{is  not} in the sense of ``This
game is rigged'' but rather in the sense of ``equipped'' --- like
how a boat is ``rigged'' or ``equipped to sail''. 

\bigskip
\begin{ddanger}
In fact, the triple $S\subset\mathcal{H}\subset{S^{*}}$ is a
rigged Hilbert space if $S$ is a nuclear subspace of
$\mathcal{H}$. See Gelfand~\cite{gelfandgeneralized} or
Maurin~\cite{maurin} for rigorous details about the notion of
nuclear spaces. We'll discuss one such criteria for $S$ to be
nuclear. Specifically,
\begin{enumerate}
\item there exists a countable family $\|\cdot\|_{k}$ of norms on
$S$ with respect to which convergence is defined
by 
\begin{equation}
f_{n}\to{f}\quad\iff\quad\|f_{n}-f\|_{k}\to0\;\;\forall k\geq0;
\end{equation}
\item $S$ is complete with respect to this notion of
  convergence; and 
\item there exists a Hilbert-Schmidt operator on
  $S$ with a continuous inverse.
\end{enumerate}
We'll leave the interested reader to refer to the cited sources.
\end{ddanger}
\bigskip

In a rigged Hilbert Space we have eigenfunction expansions. More
precisely, consider a state $|f\>$ represented by the function
$f$, let $|\overline{x}\>$ be the position eigenstate represented
by the distribution $\delta_{\overline{x}}$. We assume the
relationship between the functions and the kets is such that
\begin{equation}%\label{eq:}
f(\overline{x}) = \<\overline{x}|f\>.
\end{equation}
We can then expand the state $|f\>$ in terms of the position
eigenstate $|\overline{x}\>$ which should be of the form
\begin{equation}%\label{eq:}
|f\> = \int |\overline{x}\>\,\<\overline{x}|f\>\,d^{3}\overline{x} = \int f(\overline{x})\,|\overline{x}\>\,d^{3}\overline{x}.
\end{equation}
The conditions on $S$ in a rigged Hilbert Space ensure that
$f(\overline{x})|\overline{x}\>$ is integrable for all $f\in{S}$.

\section{One Particle Systems: Physical Aspects}
%%
%% oneParticleSystemPhysics.tex
%% 
%% Made by Alex Nelson
%% Login   <alex@tomato>
%% 
%% Started on  Tue Jul 21 15:55:44 2009 Alex Nelson
%% Last update Tue Jul 21 15:55:44 2009 Alex Nelson
%%

We're interested in a toy model of relativistic quantum
mechanics, so we begin with a single particle. All we really
need, truth be told, is a state space plus a Hamiltonian
operator. We should remember, from Special Relativity, the
energy-momentum four-vector $\widehat{p}_{\mu}$ has as its time component the
Hamiltonian $\widehat{p}_{0}=\widehat{H}$. For convenience, we'll
work in the momentum space with the momentum operator eigenbasis
\begin{equation}%\label{eq:}
\widehat{p}_{m}|\overline{k}\>=k_{m}|\overline{k}\>
\end{equation}
We assume the states are normalized thus
\begin{equation}%\label{eq:}
\<\overline{k}|\overline{k}'\>=\delta^{(3)}(\overline{k}-\overline{k}').
\end{equation}
This means that the length of a ket is undefined. It is,
nonetheless, a normalization suitable for integration over
momentum. As an added bonus, we also get the resolution of the
identity
\begin{equation}%\label{eq:}
\mathbf{1}=\int\,|\overline{k}\>\,\<\overline{k}|\,d^{3}\overline{k}
\end{equation}

Since energy-momentum is a four-vector, we demand that
\begin{equation}%\label{eq:}
\widehat{p}^{\mu}\widehat{p}_{\mu} = \widehat{H}^{2}-|\widehat{p}_{m}\widehat{p}^{m}|
\end{equation}
needs to be constant on the orbits of the Poincar\'e
group. Further if $|\overline{k}\>$ and $|\overline{k}\,'\>$ are
two states of a single particle, then there exists a Lorentz
boost from one to the other (up to scale). Hence we assume the
existence of a scalar quantity $\mu$ (the particle mass) which
satisfies
\begin{equation}%\label{eq:}
(\widehat{H}^{2} - \widehat{p}_{m}\widehat{p}^{m})|\overline{k}\>
= \mu^{2}|\overline{k}\>
\end{equation}
This implies that the Hamiltonian operator $\widehat{H}$ is
diagonal in the momentum eigenbasis (i.e. the basis of
eigenstates of the momentum operator):
\begin{equation}%\label{eq:}
\widehat{H}|\overline{k}\> = \left(\|\overline{k}\|^{2}+\mu^{2}\right)^{1/2}|\overline{k}\>
\end{equation}
The eigenvalues of the Hamiltonian operator come up enough times
that we introduce the shorthand for it:
\begin{equation}%\label{eq:}
\omega(\overline{k}) \stackrel{\text{def}}{=} \left(\|\overline{k}\|^{2}+\mu^{2}\right)^{1/2}
\end{equation}
(This should be vaguely reminiscent of the de Broglie relations
$E=\hbar\omega$.)

\begin{rmk}
Observe that this entire scheme we've devised is equivalent to
taking the limit of the state space for a cube of side $L$ under
periodic boundary conditions, i.e. the particle in a box
situation. In such a cube, we should recall the spectrum of the
momentum operator is discrete and the normalization is given by
the Kronecker delta:
\begin{equation}%\label{eq:}
\overline{k}=\frac{2\pi}{L}(n_x,n_y,n_z),\quad\text{and}\quad\<\overline{k}|\overline{k}\,'\>=\delta_{\,\overline{k}\, ,\,\overline{k}\,'}
\end{equation}
This observation is taken advantage of when deriving the
differential transition probability per unit time for particle
scattering.
\end{rmk}

\section{Unitary Representation of Poincar\'e Group}
\subsection{Action of Translation on States}
%%
%% poincareRep.tex
%% 
%% Made by Alex Nelson
%% Login   <alex@tomato>
%% 
%% Started on  Wed Jul 22 12:09:17 2009 Alex Nelson
%% Last update Wed Jul 22 12:09:17 2009 Alex Nelson
%%

The Lorentz transformation is usually ``represented'' by a matrix
$\Lambda$ which, when written explicitly, is
\begin{equation}%\label{eq:}
(\Lambda x)^{\mu} = {\Lambda^{\mu}}_{\nu}x^{\nu}
\end{equation}
where Einstein convention is used (implicit sum over $\nu$
occurs). We have that the matrix ${\Lambda^{\mu}}_{\nu}$ must
satisfy
\begin{equation}%\label{eq:}
{\Lambda^{\lambda}}_{\mu}{\Lambda_{\lambda}}_{\nu} = \eta_{\mu\nu}
\end{equation}
where $\eta_{\mu\nu}$ is the Minkowski metric (metric for flat
spacetime).

Now, the Poincar\'e group is the set of Lorentz transformations
and space-time translations, so the element of the group would be
$(\Lambda,a)$ such that
\begin{equation}%\label{eq:}
x^{\mu}\to y^{\mu} = {\Lambda^{\mu}}_{\nu}x^{\nu}+a^{\mu}.
\end{equation}
The group multiplication law is then just
\begin{equation}%\label{eq:}
(\Lambda_{2},a_{2})(\Lambda_{1},a_{1}) = (\Lambda_{2}\Lambda_{1},\Lambda_{2}a_{1}+a_{2}).
\end{equation}
We are interested in irreducible unitary representations $U(\cdot)$ of our
group, all we need to worry about are the generators.

The translations, rotations, and boosts of the Poincar\'e group
must act on the space of states. A Poincar\'e group element $g$
acts as a unitaruy operator $U(g)$ on the state space. The action
must satisfy a multipication condition
\begin{equation}%\label{eq:}
U(gh)=U(g)U(h)
\end{equation}
for all $g,h$ in the Poincar\'e group.

Translation of spacetime by a four-vector $a^{\mu}$ is defined by
\begin{equation}%\label{eq:}
\Delta_{a}(x) = x+a.
\end{equation}
Translation of a state $\psi$, on the other hand, should be
moving the graph by $a$. This means that
$\Delta_{a}\psi(x)=\psi(x-a)$. The unitary representation
$U(\Delta_{a})$ of $\Delta_{a}$ must thus be defined by
\begin{equation}%\label{eq:}
U(\Delta_{a})|\psi\>=|\Delta_{a}\psi\>.
\end{equation}
We'd like to find an expression for $U(\Delta_a)$ in terms of the
energy-momentum four-vector $\widehat{p}_{\mu}$.

Evolution in time is translation of the observer forward in time,
or (equivalently) translation of the system backwards in time:
\begin{equation}%\label{eq:}
\exp(-it\widehat{H})|\psi(x)\> = |\psi(x_{0}+t,\overline{x})\>.
\end{equation}
Let $\tau^{\mu}=(-t,\vec{0})$ be a four-vector, then we can
rewrite our translation in time as
\begin{equation}%\label{eq:}
\exp(i\tau^{\mu}\widehat{p}_{\mu})|\psi\> = |\Delta_{\tau}\psi\>.
\end{equation}
Lorentz invariance implies that this equation is true whenever
$\tau$ is timelike, and the additivity of translations then shows
this to be true for all four-vectors $\tau$. From this definition
of $U(\Delta_{a})$ we can therefore deduce that
\begin{equation}%\label{eq:}
U(\Delta_a) = \exp(ia^{\mu}\widehat{p}_{\mu}).
\end{equation}

Although this unitary representation is derived in the
position-space formulation of quantum mechanics, it works equally
well in the momentum-space formulation. We can deduce that the
unitary representation of translations on momentum eigenstates is
given by
\begin{equation}%\label{eq:}
U(\Delta_{a})|\overline{k}\> =
\exp(ia^{\mu}\widehat{p}_{\mu})|\overline{k}\> = \exp(ia^{\mu}k_{\mu})|\overline{k}\>
\end{equation}
where $k_{0} = \omega(\overline{k})$.

\begin{rmk}
Recall Taylor's theorem in real analysis can be formulated as
\begin{equation}%\label{eq:}
f(x+h) =
\left(\sum_{n=0}^{\infty}h^{n}\frac{d^{n}}{dx^{n}}\right)f(x) = \exp\left(h\frac{d}{dx}\right)f(x)
\end{equation}
which should look familiar: we just deduced the unitary
representation of spacetime translations should be
\begin{equation}%\label{eq:}
\exp(i\tau^{\mu}\widehat{p}_{\mu})|\psi\> = U(\Delta_{\tau})|\psi\>.
\end{equation}
If we don't distinguish $|\psi\>$ from $\psi(x)$, we see that
Taylor's theorem guarantees our representation to be of spacetime
translations.
\end{rmk}

\subsection{Action of the Lorentz Group}
%%
%% actionLorentzGroup.tex
%% 
%% Made by Alex Nelson
%% Login   <alex@tomato>
%% 
%% Started on  Wed Jul 22 13:01:36 2009 Alex Nelson
%% Last update Wed Jul 22 13:01:36 2009 Alex Nelson
%%

The space of particle states is three dimensional. The energy
$k_0$ of a particle with momentum $\overline{k}$ is constrained
by
\begin{equation}%\label{eq:}
k_{0}\geq0
\end{equation}
and
\begin{equation}%\label{eq:}
k^{2} = k_{\mu}k^{\mu} = \mu^{2}.
\end{equation}
Therefore the possible energy-momentum vectors lie on a
hyperbolic sheet in $k$-space, the mass hyperboloid. We need an
integration measure on this hyperboloid if we want to do Lorentz
invariant computations.

Let $\theta(t)$ be the Heaviside step function
\begin{equation}%\label{eq:}
\theta(t) = \begin{cases} 0 &\text{if }t<0\\
1 & \text{if }t>0
\end{cases}
\end{equation}
Define an integration $d\lambda(k)$ on the positive hyperboloid
as follows:
\begin{equation}%\label{eq:}
d\lambda(k) \stackrel{\text{def}}{=} d^{4}k \delta(k^{2}-\mu^{2})\theta(k_{0})
\end{equation}
The Lebesgue measure $d^{4}k$ is Lorentz invariant due to the
Lorentz transformation having unit determinant. Here since
$k^{2}-\mu^{2}$ is Lorentz invariant, the $\delta$ function is
Lorentz invariant. Similar reasoning holds for $\theta(k_{0})$
being Lorentz invariant.

We can take advantage of the identity
\begin{equation}%\label{eq:}
\delta(f(k)) = \sum_{\{k:f(k)=0\}}\frac{1}{\|f'(k)\|}\delta(k)
\end{equation}
and  the fact that
\begin{subequations}
\begin{align}
k^{2}-\mu^{2} &= (k_{0}^{2}-\|\overline{k}\|^{2})-\mu^{2}\\
&= k_{0}^{2} - (\|\overline{k}\|^{2} + \mu^{2}) \\
&= k_{0}^{2} - \omega(\overline{k})^{2} \\
&= (k_{0} - \omega(\overline{k}))(k_{0} + \omega(\overline{k}))
\end{align}
\end{subequations}
to deduce that
\begin{subequations}
\begin{align}
\delta(k^{2}-\mu^{2})\theta(k_{0}) &= \delta\left((k_{0} - \omega(\overline{k}))(k_{0} + \omega(\overline{k}))\right)\theta(k_0)\\
&=\frac{1}{2\omega(\overline{k})}(\delta(k_0-\omega(\overline{k}))\theta(k_0)+\delta(k_0+\omega(\overline{k}))\theta(k_0))\\
&=\frac{1}{2\omega(\overline{k})}\delta(k_0-\omega(\overline{k}))\theta(k_0)
\end{align}
\end{subequations}
since $\delta(k_0+\omega(\overline{k}))$ requires $k_0<0$ which
then demands that $\theta(k_0)=0$, so that term drops out completely.
Observe that this means we can effectively eliminate $k_0$ from
any integral with respect to $\omega(\overline{k})$ as follows:
\begin{subequations}
\begin{align}
\int f(k)d\lambda(k) &= \int f(k)\left(\frac{\delta(k_{0}-\omega(\overline{k}))}{2\omega(\overline{k})}\theta(k_{0})d^{3}\overline{k}dk_{0}\right)\\
&= \int f\left(\omega(\overline{k}),\overline{k}\right)\frac{d^{3}\overline{k}}{2\omega(\overline{k})}
\end{align}
\end{subequations}
This integral and the arbitrary function $f$ are commonly
eliminated from this result, leaving an equality of measures
\begin{equation}%\label{eq:}
d\lambda(k) = \frac{d^{3}\overline{k}}{2\omega(\overline{k})}
\end{equation}
and
\begin{equation}%\label{eq:}
k_{0} = \omega(\overline{k}).
\end{equation}

\begin{comment}
\begin{ddanger}We can now ask if the measure
%\begin{equation}%\label{eq:}
$d^{3}\overline{k}/2\omega(\overline{k})$
%\end{equation}
is Lorentz invariant or not. We expect it to be so, but lets try
to demonstrate it explicitly by computing the Jacobian of a
Lorentz boost $\Lambda$. Without loss of generality, we can
assume that we are working with Cartesian coordinates. Note we
can factorize an Lorentz boost as
\begin{equation}%\label{eq:}
{\Lambda^{\mu}}_{\nu} = {L^{\mu}}_{\alpha}{R^{\alpha}}_{\nu}
\end{equation}
where $L$ is a rotation in the $t-x$ plane, and $R$ is an
arbitrary spatial rotation. We know from Euler's theorem that
\begin{equation}%\label{eq:}
\det(R)=1
\end{equation}
so that means that
\begin{equation}%\label{eq:}
\det(\Lambda)=\det(L).
\end{equation}
But we can write in block form
\begin{equation}%\label{eq:}
{L^{\mu}}_{\alpha} = \begin{bmatrix} L & 0\\ 0 & I_{2} \end{bmatrix}
\end{equation}
which means that 
\begin{equation}%\label{eq:}
\det({L^{\mu}}_{\alpha}) = {L^{0}}_{0}{L^{1}}_{1} - {L^{1}}_{0}{L^{0}}_{1}.
\end{equation}
This means, without loss of generality, we can write
${\Lambda^{\mu}}_{\nu}={L^{\mu}}_{\nu}$. We make the switch
$k^{0}=\omega(\overline{k})$, so our transformation
yields the coordinates
\begin{subequations}
\begin{align}
l^{0} &= {L^{0}}_{0}\omega(\overline{k}) - {L^{0}}_{1}k^{1}\\
l^{1} &= {L^{1}}_{0}\omega(\overline{k}) - {L^{1}}_{1}k^{1}\\
l^{2} &= k^{2}\\
l^{3} &= k^{3}
\end{align}
\end{subequations}
One may be at first alarmed by the switch to
$\omega(\overline{k})$ but it is invariant under spatial
rotations, and we've seen that $k^0=\omega(\overline{k})$ so it
is kosher. By Lorentz invariance, we demand further that
\begin{equation}%\label{eq:}
l^{\mu}l_{\mu} = k^{\mu}k_{\mu} = \mu^{2}
\end{equation}
which in turn implies that
\begin{equation}%\label{eq:}
l^{0}l_{0} = \|\overline{l}\|^{2} + \mu^{2}\;\Rightarrow\; l^{0}
= \omega(\overline{l})
\end{equation}
all by Lorentz invariance.
\end{ddanger}
\end{comment}

If we define Lorentz-normalized kets $|k\>$ by
\begin{equation}%\label{eq:}
|k\> = \left(2\omega(\overline{k})\right)^{1/2}(2\pi)^{3/2}|\overline{k}\>
\end{equation}
with $k_{0}=\omega(\overline{k})$, then the new normalization
conditions is
\begin{equation}%\label{eq:}
\<k|k'\> = 2\omega(\overline{k})(2\pi)^{3}\delta^{(3)}(\overline{k}-\overline{k}')
\end{equation}
and the resolution of the identity is based on the Lorentz
invariant measure:
\begin{equation}%\label{eq:}
\mathbf{1} = \int|k\>\<k|\frac{d^{3}\overline{k}}{(2\pi)^{3}2\omega(\overline{k})}.
\end{equation}
With these Lorentz-normalized states, we can define the unitary
representation of the Lorentz group simply:
\begin{thm}%\label{thm:}
If we define $U(\Lambda)$ by $U(\Lambda)|k\>=|\Lambda k\>$, then
$U$ is a unitary representation of the Lorentz group.
\end{thm}
\begin{proof}
The multiplications property
$U(\Lambda\Lambda')=U(\Lambda)U(\Lambda')$ follows immediately
from definition. To show that the representation is unitary, we
use the resolution of the identity
\begin{subequations}
\begin{align}
U(\Lambda)U(\Lambda)^{\dag} &= \int U(\Lambda)|k\>\<k|U(\Lambda)^{\dag}\frac{d^{3}\overline{k}}{(2\pi)^{3}2\omega(\overline{k})}\\
&= \int |\Lambda k\>\<\Lambda k|\frac{d^{3}\overline{k}}{(2\pi)^{3}2\omega(\overline{k})}\\
&= \mathbf{1}
\end{align}
\end{subequations}
since the measure is Lorentz-invariant.
\end{proof}
It is mildly surprising that $U(\Lambda)$ defined in our theorem
is a unitary operator due to $|k\>$ and $|\Lambda k\>$ appear to
have different lengths when $\Lambda$ is a boost. \emph{However,}
$\delta^{(3)}(0)$ is undefined, so the normalization of the kets
does not determine a length. We regard the uniformly unlocalized
state described by $|k\>$ as \emph{unphysical}. The physical
states have the form 
\begin{equation}%\label{eq:}
|\psi\>\stackrel{\text{def}}{=} \int \psi(\overline{k})|k\>\frac{d^{3}\overline{k}}{(2\pi)^{3}2\omega(\overline{k})}
\end{equation}
where the measure is structured so
$\<k|\psi\>=\psi(\overline{k})$. We can check that the length of
$|\psi\>$ is well defined whenever $\psi(\overline{k})$ is
square-integrable and that our definition of $U(\Lambda)$ makes
the representation unitary on the space of physical states.

\subsection{Representing the Poincar\'e Group}
%%
%% actionPoincareGroup.tex
%% 
%% Made by Alex Nelson
%% Login   <alex@tomato>
%% 
%% Started on  Sat Jul 25 14:01:05 2009 Alex Nelson
%% Last update Sat Jul 25 14:01:05 2009 Alex Nelson
%%

We really want to find a unitary representation of the Poincar\'e
group, which is the Lorentz group plus spacetime translations
(i.e. rotations, Lorentz boosts, and space-time translations). We
have the representation condition $U(gh)=U(g)U(h)$ must hold for
all $g,h$ in the Poincar\'e group. We've seen what happens when
both $g,h$ are in the Lorentz group, and when both $g,h$ are
space-time translations. We now need to ask: what happens when
one is a translation and the other is a boost?

We can uniquely factor any element $g$ of the Poincar\'e group as
the product
\begin{equation}%\label{eq:}
g = \Delta_{a}\Lambda
\end{equation}
where $\Lambda$ is in the Lorentz group, and $\Delta_a$ is a
translation. Multiplication in the Poincar\'e group depends on
multiplication in the Lorentz group and addition of translations
through an interchange in the order of the two facts:
\begin{subequations}
\begin{align}
gh &= \Delta_{a}\Lambda\Delta_{b}M\\
&= \Delta_{a}(\Lambda\Delta_{b}\Lambda^{-1})\Lambda M\\
&= \Delta_{a}\Delta_{\Lambda b}\Lambda M
\end{align}
\end{subequations}
where we have used the identity
\begin{equation}%\label{eq:}
\Lambda\Delta_{b}\Lambda^{-1} = \Delta_{\Lambda b}
\end{equation}
a relation trivially verified when we act on a 4-vector $x$.

Our definition of $U$ so far covers translations and Lorentz
group elements only; when we extend to the Poincar\'e group, we
do so through the definition
\begin{equation}%\label{eq:}
U(\Delta_{a}\Lambda) \stackrel{\text{def}}{=} U(\Delta_{a})U(\Lambda)
\end{equation}
We can now see that $U$ is a representation of the Poincar\'e
group if and only if $U$ preserves the action
$\Lambda\Delta_{b}\Lambda^{-1}=\Delta_{\Lambda b}$ of Lorentz
group elements on translations:
\begin{subequations}
\begin{align}
 & U(\Delta_{a}\Lambda)U(\Delta_{b}M) =
  U(\Delta_{a}\Delta_{\Lambda b}\Lambda M) \\
\iff & U(\Delta_{a})U(\Lambda)U(\Delta_{b})U(M) =
U(\Delta_{a})U(\Delta_{\Lambda b})U(\Lambda)U(M)\\
\iff & U(\Lambda)U(\Delta_{b}) = U(\Delta_{\Lambda b})U(\Lambda)\\
\iff & U(\Lambda)U(\Delta_{b})U(\Lambda)^{\dag} = U(\Delta_{\Lambda b})
\end{align}
\end{subequations}
We verify the final condition by evaluating both sides on some
test state $|k\>$. From the right hand side, we have
\begin{equation}%\label{eq:}
U(\Delta_{\Lambda b})|k\> = \exp(i\Lambda b^{\mu}k_{\mu})|k\>
\end{equation}
and from the left hand side
\begin{subequations}
\begin{align}
U(\Lambda)U(\Delta_{b})U(\Lambda)^{\dag}|k\> &=
U(\Lambda)U(\Delta_{b})|\Lambda^{-1}k\>\\
&=
U(\Lambda)\exp(ib^{\mu}{\Lambda_{\mu}}^{\nu}k_{\nu})|\Lambda^{-1} k\>\\
&=\exp(ib^{\mu}{\Lambda_{\mu}}^{\nu}k_{\nu})|k\>.
\end{align}
\end{subequations}
The equality of the two sides follows from the Lorentz-invariance
of the inner product.

We can now summarize our results of $U$ in the following theorem:
\begin{thm}%\label{thm:}
The map $U$ from the Poincar\'e group to operators on the state
space defined by
\begin{subequations}
\begin{align}
U(\Delta_{a})|k\> &= e^{ia^{\mu}k_{\mu}}|k\>\\
U(\Lambda)|k\> &= |\Lambda k\>\\
U(\Delta_{a}\Lambda) &= U(\Delta_{a})U(\Lambda) 
\end{align}
\end{subequations}
is a unitary representation of the Poincar\'e group.
\end{thm}

The unitary representation $U$ is often boasted to successfully
combines the principle (as represented by the Poincar\'e group)
with the principles of quantum mechanics (as represented by
unitary operators and state-space formalisms). This combined
structure of a one-particle state space provides the foundation
for the many-particle state space used in all quantum field theories.


\section{Notes on a Position Operator}
%%
%% positionOperator.tex
%% 
%% Made by Alex Nelson
%% Login   <alex@tomato>
%% 
%% Started on  Sat Jul 25 14:17:02 2009 Alex Nelson
%% Last update Sat Jul 25 14:17:02 2009 Alex Nelson
%%

The astute reader would probably have realized by now we
``implemented'' relativity in the momentum space. The question
that naturally presents itself is ``Why not try to implement
relativity in position-space, as we usually do when introducing
relativity classically?'' In this section, we'll answer that
question. 

The short answer is that it turns out to be inconsistent. We can
sketch out the general scheme and its problem in this paragraph
too. Consider putting a particle (of mass $m$) into a box whose
sides are small compared to the Compton wavelength $\lambda$,
then the uncertainty in position satisfies
\begin{equation}%\label{eq:}
\Delta x\lll\lambda
\end{equation}
and the uncertainty in momentum satisfies
\begin{equation}%\label{eq:}
\Delta p\ggg m.
\end{equation}
But this makes the range of energies so large that pair
production becomes possible. Hence, from first principles, the
position of a one-particle system is not so well defined. We'll
show (slightly more rigorously) that the notion of Lorentz
causality is violated by measuring the position operator.

We first set up the axioms for (properties satisfied by) the
position operator $\widehat{x}^{m}$. We want:
\begin{description}
\item[Axiom 1] $\widehat{x}=\widehat{x}^{\dag}$ (i.e. it's
  self-adjoint, so it has real eigenvalues);
\item[Axiom 2] If $\Delta_{a}$ is a spatial translation, then
  $U(\Delta_{a})^{\dag}\widehat{x}^{m}U(\Delta_{a}) = \widehat{x}^{m}+a^{m}$
\item[Axiom 3] If $R$ is a spatial rotation, then
  $U(R)^{\dag}\widehat{x}^{m}U(R) = {R^{m'}}_{m}\widehat{x}^{m}$.
\end{description}
From axiom 2 and $U(\Delta_{a}) = \exp(ia^{m}\widehat{P}_{m})$,
we deduce
\begin{equation}%\label{eq:}
e^{ia^{m}\widehat{p}_{m}}\widehat{x}^{n}e^{-ia^{m}\widehat{p}_{m}}=\widehat{x}^{n}+a^{n}.
\end{equation}
(Note that the sign in the exponent reflects the relationship
between the Lorentz dot product and the Euclidean dot product of
3-vectors.) Differentiating both sides with respect to the
component $a^{n}$ of $a$ then setting $a^{m}=\vec{0}$, we recover
the usual commutation relations:
\begin{equation}\label{eq:recoveryCanonicalCommutationRelations}
[i\widehat{p}_{n},\widehat{x}^{m}] = {\delta^{m}}_{n}.
\end{equation}

\begin{comment}
Although $\widehat{p}_{n}$ is unbdefined on $|x^{m}\>$, the
axioms for the position operator imply that the exponential
$\exp(-ia^{m}\widehat{p}_{m})$ must be defined on these states:
\begin{equation}%\label{eq:}
e^{-ia^{m}\widehat{p}_{m}}|x^{n}\> = |x^{n}+a^{n}\>.
\end{equation}
Also, if $\<\overline{k}|\overline{x}=\overline{0}\>=1$, then
$\<\overline{k}|\overline{a}\>=\exp(-ia^{m}k_{m})$. 
\end{comment}

\begin{rmk}
The position operator is ``essentially'' unique. That is to say,
it's unique up to unitarity. Suppose we have two operators
$\widehat{y}^{m'}$, $\widehat{x}^{m}$ that satisfy our
axioms. We'll demonstrate that there exists a unitary operator
$U$ such that $\widehat{y}^{m}=U^{\dag}\widehat{x}^{m}U$.

Assume that $\widehat{y}^{m}$ is the position operator with
respect to the basis $|\overline{k}\>$. The canonical commutation
relations eq \eqref{eq:recoveryCanonicalCommutationRelations}
shows that $\widehat{p}_{n}$ commutes with
$\widehat{x}^{n}-\widehat{y}^{n}$. Therefore, supposing any operator can
be expressed using $\widehat{x}^{m}$ and $\widehat{p}_{n}$, we
have
\begin{equation}%\label{eq:}
\widehat{y}^{m} = \widehat{x}^{m}+f^{m}(\widehat{p}).
\end{equation}
Axiom 3 however implies that $f^{m}(\widehat{p})\sim
g(\|\widehat{p}\|^{2})\widehat{p}_{m}$. This vector-valued
function of a vector has zero curl and thus may be written as the
gradient of a scalar function. Lets denote this scalar function
as $\phi(\|\widehat{p}\|^{2})$ where 
\begin{equation}%\label{eq:}
\phi(\xi)\stackrel{\text{def}}{=}\int^{\xi}_{0}g(\eta)d\eta
\end{equation}
If we define our new kets using a unitary operator $U$ to change
phases
\begin{equation}%\label{eq:}
|\overline{k}\>_{\text{new}}\stackrel{\text{def}}{=}U|\overline{k}\>\stackrel{\text{def}}{=}\exp(-i\phi(\|\overline{k}\|^{2}))|\overline{k}\>,
\end{equation}
then since
\begin{equation}%\label{eq:}
\<\psi'|\widehat{y}^{m}|\psi\> = {}_\text{new}\<\psi'|U\widehat{y}^{m}U^{\dag}|\psi\>_\text{new}
\end{equation}
the new operators are $U\widehat{y}^{m}U^{\dag}$. Writing
$U^{\dag}=e^{A}$ we find
\begin{subequations}
\begin{align}
U\widehat{y}^{m}U^{\dag} &= U\left(U^{\dag}\widehat{y}^{m}+[\widehat{y}^{m},U^{\dag}]\right)\\
&= \widehat{y}^{m} +
U[\widehat{y}^{m},1+A+\frac{1}{2}A^{2}+\cdots]\\
&= \widehat{y}^{m} + U(1+A+\frac{1}{2}A^{2}+\cdots)[\widehat{y}^{m},A]\\
&= \widehat{y}^{m} +
[\widehat{y}^{m},i\phi(\|\widehat{p}\|^{2})]\\
&= \widehat{y}^{m} - g(\|\widehat{p}\|^{2})\widehat{p}_{m}\\
&= \widehat{x}^{m}.
\end{align}
\end{subequations}
We therefore conclude that any two sets of position operators
$\widehat{x}^{m}$, $\widehat{y}^{n}$ are related by a change of
basis. We also note since $U$ is a function of the momentum
operators, the new momentum operators $U\widehat{p}_{m}U^{\dag}$
are precisely the old ones $\widehat{p}_{m}$. This shows that the
axioms determining the position operator uniquely up to a choice
of phase in the momentum eigenstates, and this concludes our remark.
\end{rmk}

The simplest inconsistency emerges when we consider a state
initially localized at the origin and see whether it can be
detected outside the forward lightcone of the origin.

Suppose we have a position operator $\widehat{x}^{m}$. Let
$|\overline{x}\>$ be a basis of position eigenstates. Then, form
our knowledge of nonrelativistic quantum mechanics, we can choose
the normalization of these kets to be such that
\begin{equation}%\label{eq:}
\<\overline{x}|\overline{k}\> = \exp(i\overline{x}\cdot\overline{k}).
\end{equation}
Now consider the evolution $|\psi\>$ of a state $|\psi_{0}\>$
initially localized at the origin:
\begin{equation}%\label{eq:}
\psi_{0}(\overline{x})\stackrel{\text{def}}{=}(2\pi)^{3}\delta^{(3)}(\overline{x})\,\Rightarrow\,\widehat{\psi}_{0}(\overline{k})=1\,\Rightarrow\,|\psi_{0}\>=\int|k\>d^{3}\overline{k},
\end{equation}
where $\widehat{\psi}_{0}$ is the Fourioer transform of
$\psi_{0}$. The evolution of this state is given by:
\begin{subequations}
\begin{align}
\psi(t,\overline{x}) &= \<\overline{x}|e^{-iHt}|\psi_{0}\>\\
&= \int \<\overline{x}|e^{-iHt}|\overline{k}\>d^{3}\overline{k}\\
&= \int
\<\overline{x}|e^{-i\omega(\overline{k})t}|\overline{k}\>d^{3}\overline{k}\\
&= \int e^{-i\omega(\overline{k})t}e^{i\overline{x}\cdot\overline{k}}d^{3}\overline{k}.
\end{align}
\end{subequations}
If the theorey is relativistic, then a state initially localized
at the origin should have zero amplitude outside the lightcone
(otherwise, there is a positive probability that something could
travel faster than light). We therefore proceed to estimate
$\psi(t,\overline{x})$ outside the light cone. Using spherical
coordinates, letting $k=\|\overline{k}\|$, $r=\|\overline{x}\|$,
we find that
\begin{subequations}
\begin{align}
\psi(t,\overline{x}) &=
\int^{1}_{-1}d(\cos\theta)\int^{2\pi}_{0}d\phi\int^{\infty}_{0}k^{2}e^{-it\sqrt{k^{2}+\mu^{2}}}e^{ikr\cos\theta}dk\\
&=\frac{2\pi}{ir}\int^{\infty}_{0}ke^{-it\sqrt{k^{2}+\mu^{2}}}(e^{ikr}-e^{-ikr})dk\\
&=\frac{2\pi}{ir}\int^{\infty}_{-\infty}ke^{-it\sqrt{k^{2}+\mu^{2}}}e^{ikr}dk.
\end{align}
\end{subequations}
We can use complex analysis to evaluate this integral when $r>t$,
we deform the contour of integration from $\mathbb{R}$ to the
first principal branch cut from $i\mu$ to $i\infty$. Substituting
$k=iz$, we find
\begin{equation}%\label{eq:}
\psi(t,\overline{x}) = \frac{4\pi i}{r}\int^{\infty}_{\mu}z\sinh(t\sqrt{z^{2}-\mu^{2}})e^{-zr}dz
\end{equation}
which is clearly nonzero.

\begin{rmk}
The integral we've been manipulating is actually divergent. This
is a consequence of the extreme nature of the initial state
$|\psi_{0}\>$. If we had started with a physical state instaead
of a position eigenstate, there would be no convergence
problem. The moral of the story is to treat integrals which arise
in such situations as defining distributions.
\end{rmk}

The outcome is that a position operator is inconsistent with
relativity. This compels us to find another way of modeling
localization of events. In field theory, we do this by making
observable operators dependent on position in spacetime.


\section{Conclusion}
We introduced a different action which is based off of Weyl's attempt
to unify gravity and electromagnetism. Instead of attempting such a
unified field theory, we observed that it has interesting
gravitational properties. 

The vacuum satisfies the Schwarzschild solution for general relativity
with a nonzero cosmological constant, plus some nonzero term and a
term linear in $r$ negligibly small at the ``local'' scale. Due to
these extra terms, the scale invariance was spontaneously broken. This
was purely accidental.

We also observed that when we solve the fourth order field equations
for the isotropic and homogeneous case, we end up breaking symmetry
again. But in doing so, we recover the standard cosmological model,
and we explained why gravity is accelerating within the framework of
the Conformal gravity model. Further, we have an effective
gravitational constant that is scale dependent which allows gravity to
be repulsive globally but (due to inhomogeneities in the scalar field)
is locally attractive. This is consistent with the first investigation
of spontaneous symmetry breaking in solving the static, spherically
symmetric body's gravitational field as locally (``for small enough
$r$'') resembling Schwarzschild's solution.

Observe that this is really nothing surprising, since this is just
another version of the Brans-Dicke theory. The Brans-Dicke action is
\begin{equation}
I = \frac{1}{16\pi}\int d^{4}x\sqrt{-g}\left(\phi R
- \omega \frac{\partial_{\mu}\phi\partial^{\mu}\phi}{\phi} + L_{matter}\right)
\end{equation}
one can rearrange it by introducing $\Phi^2=\phi$ to look like
\begin{equation}
I = \frac{1}{16\pi}\int d^{4}x\sqrt{-g}\left(\Phi^2R -
4\omega\partial_\mu\Phi\partial^{\mu}\Phi + L_{matter}\right)
\end{equation}
which resembles the action in Eq \eqref{symmetryBreakingAction}. What
the Brans-Dicke theory effectively does is replace $k=16\pi G/c^4$
with a scalar field $\phi$. We did something similar, except our
scalar field spontaneously broke the scale invariance (so,
analogously, we had a bare minimum value for $k$) which gave rise
to a cosmological constant in addition to recovering the standard
cosmological model. Further, we used covariant derivatives instead of
partial derivatives, so we would need to include in the $L_{matter}$
the extra terms, the $\Phi^4$ term, and the coupling to
matter. Nonetheless, the cosmological constant naturally emerges when
we break symmetry. 


\nocite{*}
\bibliographystyle{utcaps}
\bibliography{rqm}
\end{document}

\chapter{Classical Fields}

\M
The general idea is that we will review/introduce classical fields, then
appeal to an heuristic quantization procedure to obtain quantum fields.
The motivation is a careful derivation of the linear chain, where we
have an infinite number of identical point-masses connected by identical
massless springs. Taking the continuum limit gives us the scalar field.

Most fields can intuitively be imagined as a ``tuple of scalar fields''.
For example, electromagnetism has its field $A^{\mu}$ be a 4-tuple of
scalar fields which are related in some manner (e.g., enjoying Lorentz
invariance and satisfies the equations of motion). This is a lie, but a
comforting lie that physicists tell themselves.

\M
There is another, less discussed, derivation of classical fields from
demanding certain ``background independence'' properties. We can derive
gauge theory in this manner, too. Teitelboim~\cite{Teitelboim:1980hs}
wrote a review of this approach, and it appears to be folklore among
quantum gravity researchers (especially those working in the canonical
approaches). This requires accepting a few innocent axioms and working
within the Hamiltonian framework.

%\includegraphics{img/img.0}
\section{Linear Chain}

\N{Problem Statement}
Consider $N\in\NN$ identical point-masses (each with rest mass $m$)
each of which are connected to two neighboring point masses by identical
massless springs with spring constant $k$ and equilibrium length $a$.
The point-masses then form a line segment.

We will take $N\to\infty$ limit in such a way that at any point-mass
there are infinitely many point-masses in either direction.
This specifically is to let us ignore the boundary conditions.

What are the equations of motion for a point-mass in this chain? What is
the Lagrangian for this system?

Take the continuum limit where $a\to0$ while $a^{2}k/m$ is held
constant. What happens to the equations of motion and the Lagrangian?

\begin{exercise}
What dimensions does $a^{2}k/m$ have?
\end{exercise}

\N{Coordinates}
Since we have the point-masses form a one-dimensional system, we will
write $x_{j}$ for the position of the point-mass with $j\in\ZZ$.

\N{Free Body Diagram}
Suppose we examine the free-body diagram for the point-mass. The only
forces acting on a point-mass $x_{j}$ are the spring forces:
\begin{center}
\includegraphics{img/img.0}
\end{center}

\N{Equations of Motion}
Then we see the force acting on $x_{j}$ is
\begin{equation}
\begin{split}
  F_{j} &= -k(x_{j}-x_{j-1}-a) + k(x_{j+1}-x_{j}-a)\\
  &= k(x_{j+1}-2x_{j}+x_{j}).
\end{split}
\end{equation}
Using Newton's second Law,
\begin{equation}
m\ddot{x}_{j} = F_{j} = k(x_{j+1}-2x_{j}+x_{j}).
\end{equation}
We will rearrange this to:
\begin{equation}\label{eq:classical-field-theory:linear-chain:newton-eom}
\ddot{x}_{j} = \frac{k}{m}(x_{j+1}-2x_{j}+x_{j}).
\end{equation}


\N{Lagrangian}
We can then write the Lagrangian for this system,
\begin{equation}
L = \sum_{j\in\ZZ}\frac{m}{2}\dot{x}^{2}_{j} - \frac{k}{2}(x_{j+1}-x_{j})^{2}.
\end{equation}
Since the sum is over all integers, the forces acting on $x_{j}$ come
from the $j-1$ term and the $j$ term.

\N{Continuum Limit}
Now care must be taken, because as $a\to 0$ the index $j$ labeling
particles will become a real number indicating the position of the
particle. To avoid ambiguity, we will write $q_{j}(t)$ for the position
of particle $j$.

We observe as $a\to0$, we have $q_{j}(t)\to q(x,t)$
\begin{equation}
\frac{x_{j+1}-2x_{j}+x_{j-1}}{a^{2}}\xrightarrow{a\to0}\frac{\partial^{2}}{\partial x^{2}}q(x,t).
\end{equation}
Then the continuum limit of the equations of motion,
Eq~\eqref{eq:classical-field-theory:linear-chain:newton-eom}, (first
dividing through by $m$) is:
\begin{equation}
\ddot{x}_{j}\xrightarrow{a\to0}\frac{\partial^{2}}{\partial t^{2}}q(x,t),
\quad\mbox{and}\quad\frac{ka^{2}}{m}\frac{x_{j+1}-2x_{j}+x_{j-1}}{a^{2}}
\xrightarrow{a\to0}v^{2}
\frac{\partial^{2}}{\partial x^{2}}q(x,t).
\end{equation}
Then equating both limits gives us:
\begin{equation}
\frac{\partial^{2}}{\partial t^{2}}q(x,t) = v^{2}
\frac{\partial^{2}}{\partial x^{2}}q(x,t).
\end{equation}
This is precisely the wave equation for an elastic string.
Here $v=\sqrt{a^{2}k/m}$ is the velocity of propagation.

\begin{exercise}
We have been working with one spatial dimension, assuming it is $\RR$.
Suppose space is a circle $S^{1}$ and our linear chain forms a
ring. Perform the continuum limit analysis for this situation.
\end{exercise}

\begin{exercise}
If we took space to be a closed interval $[a,b]$ instead of $\RR$,
then what boundary conditions do we need to impose for things to work
out in the continuum limit?
\end{exercise}

\N{Canonical Analysis}
We can perform the Legendre transform of the Lagrangian, first finding
the conjugate momenta
\begin{equation}
p_{j} = \frac{\partial L}{\partial\dot{q}_{j}} = m\dot{q}_{j}.
\end{equation}
Then
\begin{equation}
H = \sum_{j\in\ZZ}\frac{p_{j}^{2}}{2m} + k(q_{j+1}-q_{j})^{2}.
\end{equation}

\section{Scalar Field}

\M
We have the Klein--Gordon field\index{Klein--Gordon!Field}
be described by the Lagrangian in
Eq~\eqref{eq:classical-field-theory:linear-chain:continuum-limit:mass-term:lagrangian}.
More generally, since we could add an arbitrary potential term
$V(\varphi)$, the Lagrangian looks like:
\begin{equation}\label{eq:classical-field-theory:scalar-field:lagrangian}
\begin{split}
  \mathcal{L} &= -\frac{1}{2}\partial^{\mu}\varphi\partial_{\mu}\varphi
-\frac{1}{2}\frac{m^{2}c^{2}}{\hbar^{2}}\varphi^{2} - V(\varphi)\\
&=\frac{1}{2}c^{-2}(\partial_{t}\varphi)^{2}-\frac{1}{2}(\nabla\varphi)^{2}
-\frac{1}{2}\frac{m^{2}c^{2}}{\hbar^{2}}\varphi^{2} - V(\varphi).
\end{split}
\end{equation}
Our physical intuition should be that of a linear chain of
point-particles connected by massless identical springs, as we have
analyzed in the previous section.

\begin{exercise}
Can we interpret the $\frac{1}{2}(\nabla\varphi)^{2}$ term in the
Lagrangian as a contribution to the potential? If so, what is its
physical interpretation?
\end{exercise}

\begin{definition}\index{Klein--Gordon!Field}\index{Field!Scalar!Free}\index{Scalar Field!Free}
When $V=0$ in the Lagrangian, we call the type of field a
\define{Klein--Gordon Field} (or \emph{Free Scalar Field}).
\end{definition}

\begin{remark}
We could have $V(\varphi) = a + b\varphi + c\varphi^{2}$ and reabsorb
$a$, $b$, $c$ into the mass and elsewhere in the Lagrangian, producing
an equivalent free scalar field. And physicists will be a little sloppy
in their language, referring to such Lagrangians with
nonzero-but-quadratic potentials as ``free''.
\end{remark}

\begin{definition}
When $V(\varphi)\neq0$ (and specifically $V(\varphi)$ is not a constant,
or a linear or quadratic polynomial), we say we have a
\define{Self-Interacting Scalar Field}.
\end{definition}

\begin{definition}
When we have a linear $\varphi$ term in the Lagrangian's potential term
(something like $\sigma\varphi$), we refer to it as an
\define{External Source}.
\end{definition}

\begin{remark}
This terminology may seem bizarre at first, but it generalizes the
4-current which we used to recover Maxwell's equations (\S\ref{chunk:relativity:electromagnetism:recovering-maxwell-equations}).
The 4-current captured our intuition of coupling electromagnetism to
matter (``charged particles''), and the external source couples the
field to ``generic matter''.

We will also find, however, that it's useful to stick in an external
source into the Lagrangian for mathematical purposes. It's a
mathematical trick where we will differentiate with respect to external
sources to compute moments of integrals, then set the external source to
zero. 
\end{remark}

\N{Goals}
We will study the \emph{free} scalar field in this section. 

\N{Variational Analysis}
We consider a region $\mathcal{R}\subset\RR^{3,1}$, usually taken to be
$\RR^{3}\times[t_{1},t_{2}]$ for some $t_{1}<t_{2}$. Now we will
consider the action
\begin{equation}
\action[\varphi] = \int_{\mathcal{R}}\left(\frac{1}{2}c^{-2}(\partial_{t}\varphi)^{2}-\frac{1}{2}(\nabla\varphi)^{2}
-\frac{1}{2}\frac{m^{2}c^{2}}{\hbar^{2}}\varphi^{2}\right)\D^{3}\vec{x}\,\D t.
\end{equation}
As usual, we will try to find the critical points of the action, and
argue these are the physical solutions to the equations of motion.

Now, we take a variation of this action with respect to $\varphi$. This
is done by writing
\begin{equation}
\varphi_{\lambda} = \varphi + \lambda\psi,
\end{equation}
where $\lambda$ is a real parameter, $\psi|_{\partial\mathcal{R}}=0$ is
an (otherwise) arbitrary function. Physicists write
\begin{equation}
\delta\varphi = \lambda\psi,
\end{equation}
and pretend $\lambda$ is an infinitesimal quantity $\lambda^{2}\ll1$.
Then expanding the integrand in the action to first-order in $\lambda\psi$,
we demand the coefficient to $\lambda\psi$ vanish. We find,
\begin{calculation}
\action[\varphi_{\lambda}]
\step{plugging in the definition of the action}
\int_{\mathcal{R}}\left(\frac{1}{2}c^{-2}(\partial_{t}\varphi_{\lambda})^{2}-\frac{1}{2}(\nabla\varphi_{\lambda})^{2}
-\frac{1}{2}\frac{m^{2}c^{2}}{\hbar^{2}}\varphi_{\lambda}^{2}\right)\D^{3}\vec{x}\,\D t
\step{unfold the definition of $\varphi_{\lambda}$}
\int_{\mathcal{R}}\left(\frac{1}{2}c^{-2}(\partial_{t}[\varphi + \lambda\psi])^{2}-\frac{1}{2}(\nabla[\varphi + \lambda\psi])^{2}
-\frac{1}{2}\frac{m^{2}c^{2}}{\hbar^{2}}[\varphi + \lambda\psi]^{2}\right)\D^{3}\vec{x}\,\D t
%% \step{expand}
%% \int_{\mathcal{R}}\left(\frac{1}{2}c^{-2}
%% (\partial_{t}\varphi)^{2}+c^{-2}\partial_{t}\varphi\partial_{t}\psi+\frac{1}{2}c^{-2}(\partial_{t}\psi)^{2}
%% -\frac{1}{2}(\nabla\varphi)^{2}
%% -(\nabla\varphi)\cdot(\nabla(\lambda\psi))
%% -\frac{1}{2}(\nabla(\lambda\psi))\cdot(\nabla(\lambda\psi))
%% -\frac{1}{2}\frac{m^{2}c^{2}}{\hbar^{2}}[\varphi^{2} + 2\lambda\psi\varphi + \lambda^{2}\psi^{2}]\right)\D^{3}\vec{x}\,\D t
\step{expanding and collecting coefficients of $\lambda$}
\action[\varphi] + \int_{\mathcal{R}}\left(
c^{-2}\partial_{t}\varphi\partial_{t}(\lambda\psi)
-(\nabla\varphi)\cdot(\nabla(\lambda\psi))
-\frac{m^{2}c^{2}}{\hbar^{2}}\varphi\cdot(\lambda\psi)\right)\D^{3}\vec{x}\,\D t
+\action[\lambda\psi].
\end{calculation}
We write the first variation of the action as:
\begin{equation}
\delta\action[\varphi_{\lambda}] = \int_{\mathcal{R}}\left(
c^{-2}\partial_{t}\varphi\partial_{t}(\lambda\psi)
-(\nabla\varphi)\cdot(\nabla(\lambda\psi))
-\frac{m^{2}c^{2}}{\hbar^{2}}\varphi\cdot\lambda\psi\right)\D^{3}\vec{x}\,\D t.
\end{equation}

\N{Initial conditions}
We need to specify the initial and final configuration for the scalar
field, so for $\mathcal{R}=\mathcal{R}_{3}\times[t_{1},t_{2}]\subset\RR^{3,1}$,
we need functions
\begin{equation}
\varphi_{1},\varphi_{2}\colon\mathcal{R}_{3}\to\RR,
\end{equation}
such that
\begin{equation}
\varphi|_{\mathcal{R}_{3}\times\{t_{1}\}}=\varphi_{1},\quad\mbox{and}\quad
\varphi|_{\mathcal{R}_{3}\times\{t_{2}\}}=\varphi_{2}.
\end{equation}

\M
We can integrate $\delta\action[\varphi_{\lambda}]$ by parts (with
respect to time) to get
\begin{equation}
\delta\action[\varphi_{\lambda}] = \int_{\mathcal{R}}\left(
-c^{-2}(\lambda\psi)\partial_{t}^{2}\varphi
-(\nabla\varphi)\cdot(\nabla(\lambda\psi))
-\frac{m^{2}c^{2}}{\hbar^{2}}\varphi\cdot\lambda\psi\right)\D^{3}\vec{x}\,\D t.
\end{equation}
The boundary terms from this integration-by-parts vanishes since
$\lambda\psi|^{t_{2}}_{t_{1}}=0$.

\M
When we have $\mathcal{R}=\RR^{3}\times[t_{1},t_{2}]$, we treat
$\RR^{3}$ as a sphere with radius $r\to\infty$. Doing so allows us to
use the divergence theorem to further rewrite the first variation of the
action as, when $\vec{n}$ is the outward-pointing unit normal vector,
\begin{equation}
\begin{split}
\delta\action[\varphi_{\lambda}] &= \int_{\mathcal{R}}\left(
-c^{-2}(\lambda\psi)\partial_{t}^{2}\varphi
+(\lambda\psi)\nabla^{2}\varphi
-\frac{m^{2}c^{2}}{\hbar^{2}}\varphi\cdot\lambda\psi\right)\D^{3}\vec{x}\,\D t\\
&\qquad-\int^{t_{2}}_{t_{1}}\int_{r\to\infty}(\lambda\psi)\vec{n}\cdot(\nabla\varphi)%
\,\D A\,\D t,
\end{split}
\end{equation}
where $\D A$ is the differential area for the sphere. (In $n+1$
dimensions, $\D A\sim r^{n-2}\,\D r$.)
We have implicitly used the fact that
\begin{equation}
(\nabla\varphi)\cdot\nabla(\lambda\psi)=\nabla\cdot((\nabla\varphi)\lambda\psi)
-(\lambda\psi)\nabla^{2}\varphi,
\end{equation}
before using the divergence theorem.

\N{Assumption on growth of field}
We need to assume that, at a fixed time $t$, the scalar field $\varphi$
(and therefore both $\varphi_{\lambda}$ and $\lambda\psi$) fall off as
$r\to\infty$ sufficiently fast so as to make the boundary term in the
first variation of the action $\delta\action$ vanishes. One possibility
is to work with fields with compact support.

The usual assumption physicists make is that, since the area eleement
$\D A$ grows like $r^{2}$, the integrand must fall faster than $1/r^{2}$
as $r\to\infty$ for the boundary term to vanish.

More generally, in $n+1$ dimensional spacetime, $\D A$ grows like $r^{n-1}$,
requiring the integrand to fall faster than $1/r^{n-1}$ as $r\to\infty$
for the boundary terms to vanish.

\M
Now we see that the first variation of the action is just
\begin{subequations}
\begin{equation}
\delta\action[\varphi] = \int_{\mathcal{R}}\left(
-c^{-2}(\lambda\psi)\partial_{t}^{2}\varphi
+(\lambda\psi)\nabla^{2}\varphi
-\frac{m^{2}c^{2}}{\hbar^{2}}\varphi\cdot\lambda\psi\right)\D^{3}\vec{x}\,\D t,
\end{equation}
or factoring out the $\delta\varphi=\lambda\psi$,
\begin{equation}
\delta\action[\varphi] = \int_{\mathcal{R}}\left(
-c^{-2}\partial_{t}^{2}\varphi
+\nabla^{2}\varphi
-\frac{m^{2}c^{2}}{\hbar^{2}}\varphi\right)\delta\varphi\,\D^{3}\vec{x}\,\D t.
\end{equation}
\end{subequations}
Now we need to find the critical points of the action, which is to say,
we demand $\delta\action[\varphi]=0$. This gives us a rather tricky
differential-integral equation, but fortunately the fundamental lemma of
variational calculus\marginnote{TODO: cite fundamental lemma of variational calculus} says the condition is exactly demanding the
integrand's coefficient of $\delta\varphi$ vanishes, i.e.,
\begin{equation}
\boxed{-c^{-2}\partial_{t}^{2}\varphi
+\nabla^{2}\varphi
-\frac{m^{2}c^{2}}{\hbar^{2}}\varphi=0.}
\end{equation}
Solving this differential equation gives us the critical point for the
action.

Usually we expedite this whole process of variational analysis, and just
use the Euler--Lagrange equations.

\begin{exercise}
Re-perform this analysis with an arbitrary potential contribution
$V(\varphi)$ and observe how the equations of motion change (i.e., what
extra term will be added to the equations of motion, something involving
$V'(\varphi)$ we expect).
\end{exercise}

\N{Equations of Motion}
We can now derive the equations of motion using the Euler--Lagrange
equations, which are:
\begin{equation}
c^{-2}\partial_{t}^{2}\varphi - \nabla^{2}\varphi + \frac{m^{2}c^{2}}{\hbar^{2}}\varphi+V'(\varphi)=0.
\end{equation}
Also note that $V(\varphi)$ is usually some polynomial in $\varphi$, and
we can discard the terms lower than quadratic order (and absorb the
quadratic term into the mass term).

\begin{proof}
  We have
  \begin{subequations}
  \begin{equation}
\frac{\partial\mathcal{L}}{\partial(\partial_{\mu}\varphi)}
=-\partial^{\mu}\varphi,
  \end{equation}
  so the ``acceleration'' part of the equations of motion:
  \begin{equation}
\partial_{\mu}\frac{\partial\mathcal{L}}{\partial(\partial_{\mu}\varphi)}
=-\partial_{\mu}\partial^{\mu}\varphi.
  \end{equation}
  Then the ``force'' part of the equations of motion
\begin{equation}
\frac{\partial\mathcal{L}}{\partial\varphi} = -\frac{m^{2}c^{2}}{\hbar^{2}}\varphi
- V'(\varphi).
\end{equation}
Taken altogether, the equations of motion read:
\begin{equation}
-\partial_{\mu}\partial^{\mu}\varphi = -\frac{m^{2}c^{2}}{\hbar^{2}}\varphi
- V'(\varphi).
\end{equation}
  \end{subequations}
  Some gentle messaging yields the result.
\end{proof}

\N{Solutions}
We can solve the equations of motion for the \emph{free} scalar field by
taking its Fourier transform. We have
\begin{equation}
\varphi(t,\vec{x}) = \iint\E^{-\I(\omega t-\vec{k}\cdot\vec{x})}\widetilde{\varphi}(\omega,\vec{k})\frac{\D^{3}\vec{k}}{(2\pi\hbar)^{3}}\frac{\D\omega}{2\pi\hbar}
\end{equation}
where $\widetilde{\varphi}$ is the Fourier transform, $\omega=E/\hbar$,
and $\vec{k}=\vec{p}/\hbar$.

\begin{exercise}
Explicitly work this out. I mean, it's \emph{obvious}, but you should
check that it's \emph{true}.
\end{exercise}

\begin{exercise}
Let us consider the Fourier transform in the spatial variables only for
the scalar field,
\begin{equation}
\widetilde{\varphi}(\vec{k},t) = \int\E^{\I(\vec{k}\cdot\vec{x})}\varphi(\vec{x},t)\D^{3}\vec{x}.
\end{equation}
Prove or find a counter-example: the complex conjugate of the
\emph{spatially} Fourier
transformed scalar field $\widetilde{\varphi}(\vec{k},t)^{*}$
satisfies $\widetilde{\varphi}(\vec{k},t)^{*}=\widetilde{\varphi}(-\vec{k},t)$.

Also: do we need to include a factor of $(2\pi)^{-3/2}$ in the spatial Fourier
transform, or will things work out fine as we presented it? 
\end{exercise}

\subsection{Aside: Variational Calculus}

\M
We will typically use some heuristics in physics when doing variational
calculus, rather than working with the level of rigour anyone would hope
for. Given some functional of the form
\begin{equation}
F[\varphi] = \int_{\mathcal{R}}f(\varphi(x), \partial_{\mu}\varphi(x))\,\D^{4}x,
\end{equation}
we consider its first variation with respect to $\varphi$ by pretending:
\begin{enumerate}
\item $\delta$ is a linear operator obeying the Leibniz product rule (i.e., it
is a derivation), and
\item variations commute with differentiation (e.g., $\delta(\partial_{\mu}\varphi)=\partial_{\mu}(\delta\varphi)$).
\end{enumerate}
So to be clear, in the integrand, the $\delta$ operator acts like:
\begin{equation}
\delta f(\varphi(x)) 
= \frac{\partial f(\varphi)}{\partial\varphi}\delta\varphi.
\end{equation}
When there are also derivatives of $\varphi$, we have
\begin{subequations}
\begin{align}
\delta f(\varphi(x),\partial_{\mu}\varphi) 
&= \frac{\partial f(\varphi,\partial_{\mu}\varphi)}{\partial\varphi}\delta\varphi
+ \frac{\partial f(\varphi,\partial_{\mu}\varphi)}{\partial(\partial_{\mu}\varphi)}\delta(\partial_{\mu}\varphi)\\
\intertext{or, commuting variation with partial derivatives in the second term,}
\delta f(\varphi(x),\partial_{\mu}\varphi) &= \frac{\partial f(\varphi,\partial_{\mu}\varphi)}{\partial\varphi}\delta\varphi
+ \frac{\partial f(\varphi,\partial_{\mu}\varphi)}{\partial(\partial_{\mu}\varphi)}\partial_{\mu}(\delta\varphi)
\end{align}
\end{subequations}
When computing these partial derivatives, we pretend
\begin{equation}
\frac{\partial(\partial_{\mu}\varphi)}{\partial\varphi}
=0,\quad\mbox{and}\quad
\frac{\partial\varphi}{\partial(\partial_{\mu}\varphi)}=0.
\end{equation}

Then we have
\begin{equation}
\delta F[\varphi] = \int\left(\frac{\partial f}{\partial\varphi}\delta\varphi
+\frac{\partial f}{\partial(\partial_{\mu}\varphi)}\delta(\partial_{\mu}\varphi)\right)\,\D^{4}x.
\end{equation}
We try to integrate by parts to rewrite the integrand as
\begin{equation}
\delta F[\varphi] = \begin{pmatrix}\mbox{boundary}\\\mbox{terms}
\end{pmatrix}
+ \int\mbox{(something)}\delta\varphi\,\D^{4}x,
\end{equation}
then using the fundamental lemma of
variational calculus to set $\mbox{(something)}=0$. For most problems in
physics, we discard the boundary terms.\footnote{General Relativity is a
notable example where boundary terms are important.} This will give us a
differential equation whose solution is a critical point of the
functional.

\M
For multiple fields $\varphi_{a}$ with $a=1,\dots,N$, we need to take
the variation with respect to each of these fields. The first variation
would require summing over $a$, giving us $N$ coefficients of
$\delta\varphi_{a}$ (one for each $a$). We need each coefficient to
separately vanish, giving us a system of $N$ equations.

\M
These heuristics aren't ``wrong'', they just sweep details under the
rug. Specifically we would have
$\varphi_{a}^{(\lambda)}=\varphi_{a}+\lambda_{a}\psi_{a}$ where
$\psi_{a}$ are arbitrary functions which vanish at the initial and final
time slices. Then for any functional $F[\varphi_{a}]$ we find its first
variation as
\begin{equation}
\left.\frac{\D}{\D\lambda}F[\varphi^{(\lambda)}_{a}]\right|_{\lambda=0}=\delta
F[\varphi_{a},\psi_{a}]=\int\sum_{a}\frac{\delta F}{\delta\varphi_{a}}\psi_{a}\,\D^{4}x,
\end{equation}
where we just abuse notation left and right, using
$\delta F/\delta\varphi_{a}$ for the integrand, and sometimes writing
$\delta\varphi_{a}$ instead of $\psi_{a}$, and so on.

\begin{ddanger}
The notation used here is horribly sloppy in the physics literature, and
we have chosen to be consistent with that literature as much as possible.
The notation for a functional derivative coincides with a variational
derivative, which is horribly unfortunate. Some physicists try to relate
the two, but it's not quite the same thing.
\end{ddanger}

\N{Conditions for a minimum}
We recall from calculus in a single variable that not all critical
points of a function are minima (sometimes they are maxima, or
inflection points, or something else). This should caution us from being
too optimistic about critical points of functionals: the critical points
are \emph{candidates} for the minima, but we must check they are minima.

In calculus, this involves examining the sign of
$f''(x_{\text{crit}})$. When $f''(x_{\text{crit}})>0$, we have a
[possibly local] minimum.

For functionals, we need to examine the second variation $\delta^{2}F$
at the critical point. Specifically, for variation $\varphi_{\text{crit}}+\lambda\psi$,
we need
\begin{equation}
\delta^{2}F[\varphi_{\text{crit}}+\psi]\geq k\|\psi\|^{2}
\end{equation}
for all $\psi$ and some constant $k>0$. This is completely analogous to
the situation in calculus. Here we have (Taylor expanding in $\lambda\psi$, integrating by parts,
so the integrand is a Taylor polynomial in $\lambda\psi$):
\begin{equation}
F[\varphi+\lambda\psi] = F[\varphi] + \int\left(\frac{\delta F}{\delta\varphi}\lambda\psi
+\frac{\delta^{2} F}{\delta\varphi^{2}}\lambda^{2}\psi^{2} + \bigOh(\lambda^{3}\psi^{3})\right)\D^{n}x.
\end{equation}
Then we have 
\begin{equation}
\delta^{2}F[\varphi+\lambda\psi] = \int\frac{\delta^{2} F}{\delta\varphi^{2}}\lambda^{2}\psi^{2}\,\D^{n}x.
\end{equation}
This is the \define{Second Variation} of $F$.

However, this is all rather tedious, and in practice physicists drop the
``Mission Accomplished'' banner upon finding critical points for
functionals.

\begin{ddanger}
Not all functionals have a second variation. We need it to be twice
differentiable. For the functionals we care about in physics, which is
just the integral of a Lagrangian, we can compute the second variation.
\end{ddanger}

\N{References}
The quartic self-interacting scalar field (i.e., with
$V(\varphi)=\lambda\varphi^{4}/4!$ where $\lambda$ is the coupling
constant) has a few known exact solutions, which are presented in
Frasca~\cite{Frasca:2009bc}. These are quite tricky, since it requires
knowledge of Jacobi elliptical functions.
\section{Noether's Theorem}

\M
Noether's theorems tell us something about symmetries of the equations
of motion (or field equations) and conserved quantities.

\begin{danger}
Physicists have two notions of symmetries: (1) Symmetries of \textsc{laws}
governing the field (i.e., field equations), and (2) Symmetries of
\textsc{states} of the field (i.e., solutions to the field equations).
Usually the first type of symmetry is what is meant. However, in the
case of spontaneous symmetry breaking, it refers to the latter sense of
symmetries. 
\end{danger}

\subsection{For Mechanics}

\M
Consider a mechanical system consisting of $N$ particles with positions
$q^{i}(t)$ for $i=1,\dots,N$. We describe it by its action
\begin{equation}
\action[q^{i}(t)] = \int^{t_{2}}_{t_{1}}L(q^{i},\dot{q}^{i})\,\D t.
\end{equation}
Now we suppose the dynamics is invariant under
\begin{equation}
q^{i}(t)\to q^{i}(t)+\delta q^{i}(t)
\end{equation}
where
\begin{equation}\label{eq:classical-field-theory:noether:epsilon-variation}
\delta q^{i}(t) = \epsilon^{a}(t)F^{i}_{a}(q,\dot{q}) + \dot{\epsilon}^{a}
G^{i}_{a}(q,\dot{q}),
\end{equation}
the $\epsilon^{a}$ are arbitrary functions of time (except possibly at
the endpoints) and $a=1,\dots,n$.

\M The action varies like:
\begin{equation}
\delta\action = \int^{t_{2}}_{t_{1}}L_{i}\,\delta q^{i}\,\D t
  + \left.\frac{\partial L}{\partial\dot{q}^{i}}\delta q^{i}\right|^{t_{2}}_{t_{1}},
\end{equation}
where the ``Euler'' derivatives are
\begin{equation}
L_{i} := \frac{\partial L}{\partial q^{i}}
- \frac{\D}{\D t}\frac{\partial L}{\partial\dot{q}^{i}}.
\end{equation}
The equations of motion are satisfied if and only if $L_{i}=0$ (these
are the Euler--Lagrange equations).

\N{Noether's Second Theorem}
Now, the variation of the action using
Eq~\eqref{eq:classical-field-theory:noether:epsilon-variation}
gives us
\begin{calculation}
\delta\action
\step{definition of variation of action}
\int^{t_{2}}_{t_{1}}L_{i}\,\delta q^{i}\,\D t
  + \left.\frac{\partial L}{\partial\dot{q}^{i}}\delta q^{i}\right|^{t_{2}}_{t_{1}}
\step{unfolding $\delta q^{i}$}
\int^{t_{2}}_{t_{1}}L_{i}\bigl(\epsilon^{a}(t)F^{i}_{a}(q,\dot{q}) + \dot{\epsilon}^{a} G^{i}_{a}(q,\dot{q})\bigr)\,\D t
  + \left.\frac{\partial L}{\partial\dot{q}^{i}}\bigl(\epsilon^{a}(t)F^{i}_{a}(q,\dot{q}) + \dot{\epsilon}^{a} G^{i}_{a}(q,\dot{q})\bigr)\right|^{t_{2}}_{t_{1}}
\step{integrate by parts to eliminate $\dot\epsilon^{a}$}
\int^{t_{2}}_{t_{1}}\epsilon^{a}(t)\left(L_{i}F^{i}_{a}(q,\dot{q})
- \frac{\D}{\D t}\bigl(L_{i} G^{i}_{a}(q,\dot{q})\bigr)\right)\D t
+ \left[\epsilon^{a}(t)L_{i}G^{i}_{a}
+ \frac{\partial L}{\partial\dot{q}^{i}}\bigl(\epsilon^{a}(t)F^{i}_{a}(q,\dot{q}) + \dot{\epsilon}^{a} G^{i}_{a}(q,\dot{q})\bigr)\right]^{t_{2}}_{t_{1}}
\step{associativity applied to boundary terms}
\int^{t_{2}}_{t_{1}}\epsilon^{a}(t)\left(L_{i}F^{i}_{a}(q,\dot{q})
- \frac{\D}{\D t}\bigl(L_{i} G^{i}_{a}(q,\dot{q})\bigr)\right)\D t
+ \left[\epsilon^{a}(t)\left(L_{i}G^{i}_{a}
+\frac{\partial L}{\partial\dot{q}^{i}}F^{i}_{a}\right)
+ \dot{\epsilon}^{a}\frac{\partial L}{\partial\dot{q}^{i}}G^{i}_{a}(q,\dot{q})\right]^{t_{2}}_{t_{1}}
\end{calculation}
As usual, we ignore boundary terms, then invariance of the action (up to
boundary terms) demands that:
\begin{equation}\label{eq:classical-field-theory:noether:second-theorem}
\boxed{L_{i}F^{i}_{a}(q,\dot{q}) - \frac{\D}{\D t}\bigl(L_{i} G^{i}_{a}(q,\dot{q})\bigr) = 0.}
\end{equation}
These $n$ are identities are known as \define*{Noether's Second Theorem}\index{Noether!Second theorem}
(or the \emph{generalized Bianchi identities}\index{Bianchi identity!Generalized}).

\N{Example, Noether's First Theorem}\index{Noether!First theorem}
Consider the special case when
\begin{equation}
\epsilon^{a}(t) = \epsilon^{a} = \mbox{constant}.
\end{equation}
Then
\begin{equation}
\delta q^{i} = \epsilon^{a}F^{i}_{a}.
\end{equation}
We then have the variation of the action, under this variation,
\begin{equation}
\delta\action = \int^{t_{2}}_{t_{1}}\epsilon^{a}F^{i}_{a}L_{i}\,\D t + \left.\vphantom{\frac{1}{1}}\epsilon^{a}F^{i}_{a}p_{i}\right|^{t_{2}}_{t_{1}},
\end{equation}
where we introduce the canonical momentum $p_{i}$. Define
\begin{equation}
C_{a} := F^{i}_{a}p_{i},
\end{equation}
then invariance of the action \emph{including the boundary terms}
leads to the condition
\begin{equation}
\left.\vphantom{\frac{1}{1}}F^{i}_{a}p_{i}-C_{a}\right|_{t_{2}}
=
\left.\vphantom{\frac{1}{1}}F^{i}_{a}p_{i}-C_{a}\right|_{t_{1}}.
\end{equation}
An invariance under a group with a finite number of parameters thus
leads to \emph{conservation laws}. This is Noether's first theorem.

\begin{remark}
Noether ends the first section of her paper by stating (as translated by
Traver), ``With these supplementary remarks, Theorem I comprises all
theorems on first integrals known to mechanics etc., while Theorem II
may be described as the utmost possible generalization of the `general
theory of relativity' in group theory.''
\end{remark}

\begin{remark}
More generally, when the general invariance under a group is
parametrized by functions of time, this will lead to \emph{constraints}
instead of conservation laws. This is what leads to the study of
constrained Hamiltonian systems and underpins the canonical formalism of
gauge theory.
\end{remark}

\subsection{For Fields}

\M
The results still hold, we just assume invariance under
\begin{subequations}
\begin{equation}
\varphi^{A}(x)\to\varphi^{A}+\delta\varphi^{A}(x)
\end{equation}
with
\begin{equation}
\delta\varphi^{A}(x) = \epsilon^{A}(x)F^{A}_{a}(x) + (\partial_{\mu}\epsilon^{a}(x))G^{A\mu}_{a}(\varphi,x),
\end{equation}
\end{subequations}
and we now have arbitrary functions $\epsilon^{A}(x)$ of spacetime (not
just time).

\begin{exercise}[{Sundermeyer~\cite{Sundermeyer:1982gv}}]
Let $L_{A} = (\partial\mathcal{L}/\partial\varphi^{A})-\partial_{\mu}(\partial\mathcal{L}/\partial(\partial_{\mu}\varphi^{A}))$.
Prove the field theory version of Noether's second theorem Eq~\eqref{eq:classical-field-theory:noether:second-theorem}
is
\begin{subequations}
\begin{equation}
\epsilon^{a}\bigl(L_{A}F^{a}_{A} - \partial_{\mu}(L_{A}G^{A\mu}_{a})\bigr)=0.
\end{equation}
If further we consider $x^{\mu}\to x^{\mu}+\delta x^{\mu}$ with
$\variation x^{\mu} = \epsilon^{a}(x)\xi_{a}^{\mu}(x)$, then
\begin{equation}
\epsilon^{a}\bigl(L_{A}F^{a}_{A} 
- L_{A}(\partial_{\mu}\varphi^{A})\xi^{\mu}_{a}(x)
- \partial_{\mu}(L_{A}G^{A\mu}_{a})\bigr)=0.
\end{equation}
\end{subequations}
\end{exercise}

\M
The more common presentation of Noether's first theorem begins by
supposing we have some infinitesimal symmetry transformation of the field
\begin{equation}
\varphi(x)\to\varphi'(x)=\varphi(x) + \alpha\,\Delta\varphi(x),
\end{equation}
where $\alpha$ is an infinitesimal quantity and $\Delta\varphi(x)$ is
the change in the field. We will hope the Lagrangian density transforms
like
\begin{equation}
\mathcal{L}(x)\to\mathcal{L}(x) + \alpha\partial_{\mu}\mathcal{J}^{\mu}(x)
\end{equation}
where $\mathcal{J}^{\mu}$ is ``something''. This is the same as adding
some boundary contribution to the action, which will not affect the
equations of motion.

Now, if we plug in the infinitesimally transformed field into the
Lagrangian
\begin{equation}
\mathcal{L}(\varphi + \alpha\,\Delta\varphi) = \mathcal{L} + \alpha\,\Delta\mathcal{L},
\end{equation}
where $\alpha\,\Delta\mathcal{L}$ may be found by Taylor expanding to
first-order:
\begin{calculation}
\alpha\,\Delta\mathcal{L}
\step{Taylor expansion to first order}
\frac{\partial\mathcal{L}}{\partial\varphi}(\alpha\,\Delta\varphi)
+\frac{\partial\mathcal{L}}{\partial(\partial_{\mu}\varphi)}\partial_{\mu}(\alpha\,\Delta\varphi)
\step{since $A\partial_{\mu}B=\partial_{\mu}(AB)-B\partial_{\mu}A$}
\frac{\partial\mathcal{L}}{\partial\varphi}(\alpha\,\Delta\varphi)
+\partial_{\mu}\left(\frac{\partial\mathcal{L}}{\partial(\partial_{\mu}\varphi)}\alpha\,\Delta\varphi\right)
-\left(\partial_{\mu}\frac{\partial\mathcal{L}}{\partial(\partial_{\mu}\varphi)}\right)(\alpha\,\Delta\varphi)
\step{collecting terms, factoring out $\alpha$}
\alpha\partial_{\mu}\left(\frac{\partial\mathcal{L}}{\partial(\partial_{\mu}\varphi)}\Delta\varphi\right)
+\alpha\left[\frac{\partial\mathcal{L}}{\partial\varphi}
-\partial_{\mu}\left(\frac{\partial\mathcal{L}}{\partial(\partial_{\mu}\varphi)}\right)\right]\Delta\varphi
\step{when the Euler--Lagrange equations hold}
\alpha\partial_{\mu}\left(\frac{\partial\mathcal{L}}{\partial(\partial_{\mu}\varphi)}\Delta\varphi\right)
+\alpha\cdot0\cdot\Delta\varphi
\step{multiplying the second term by zero makes it vanish}
\alpha\partial_{\mu}\left(\frac{\partial\mathcal{L}}{\partial(\partial_{\mu}\varphi)}\Delta\varphi\right).
\end{calculation}
This is precisely the sort of thing we're looking for: we want
$\alpha\,\Delta\mathcal{L}=\alpha\partial_{\mu}\mathcal{J}^{\mu}$. We
then define the \define*{Noether Current}\index{Noether!Current},
\begin{equation}
j^{\mu}(x) := \frac{\partial\mathcal{L}}{\partial(\partial_{\mu}\varphi)}\Delta\varphi-\mathcal{J}^{\mu}(x),
\end{equation}
and our demands for symmetry invariance amounts to the conservation of
the Noether current:
\begin{equation}
\partial_{\mu}j^{\mu}(x) = 0.
\end{equation}
We see that this means
\begin{equation}
\partial_{\mu}j^{\mu}(x) = 0\iff\partial_{\mu}\left(\frac{\partial\mathcal{L}}{\partial(\partial_{\mu}\varphi)}\Delta\varphi\right)=\partial_{\mu}\mathcal{J}^{\mu},
\end{equation}
and therefore the two terms are interchangeable, allowing us to recover
our desired symmetry.

\begin{remark}[Multiple fields]
If the symmetry involves more than one field, then $j^{\mu}(x)$ is
really the sum of these sort of terms (one term for each field).
\end{remark}

\N{Noether Charge}
We can express the conservation law by sating that the quantity
\begin{equation}
Q := \int_{\RR^{3}}j^{0}\,\D^{3}x
\end{equation}
is a constant in time. This $Q$ is sometimes referred to as the
``Noether Charge''\index{Noether!Charge} in the literature.

\begin{example}[Canonical Stress--Energy]\label{ex:classical-field-theory:noether:canonical-stress-energy}
Consider the situation when the symmetry is an infinitesimal translation
in spacetime,
\begin{equation}
x^{\mu}\to x^{\mu} + a^{\mu},
\end{equation}
and the field transforms as\footnote{Remember, the scalar field
transforms under a symmetry $g$ like
$g\cdot\varphi(x)=\varphi(g^{-1}\cdot x)$.}:
\begin{equation}
\varphi(x)\to\varphi(x-a)=\varphi(x)-a^{\mu}\partial_{\mu}\varphi(x).
\end{equation}
The Lagrangian must transform like a scalar,
\begin{equation}
\mathcal{L}\to\mathcal{L}-a^{\mu}\partial_{\mu}\mathcal{L}
=\mathcal{L} - \partial_{\mu}(a^{\mu}\mathcal{L}).
\end{equation}
Then $\Delta\mathcal{L}(x)=- \partial_{\mu}(a^{\mu}\mathcal{L})$
allowing us to identify $\mathcal{J}^{\mu}(x)=-a^{\mu}\mathcal{L}(x)$.
The conserved Noether current is then
\begin{equation}
\begin{split}
j^{\mu}(x) &= \frac{\partial\mathcal{L}(x)}{\partial(\partial_{\mu}\varphi(x))}(-a^{\mu}\partial_{\mu}\varphi(x))
- (-a^{\mu}\mathcal{L}(x))\\
&=a_{\nu}\canonicalStressEnergy^{\mu\nu}.
\end{split}
\end{equation}
This $\canonicalStressEnergy^{\mu\nu}$ is the
\define*{Canonical Stress-Energy Tensor}\index{Stress-energy tensor!Canonical}, not to be confused 
with the stress-energy tensor appearing in the Einstein field equations.
More generally, if we have a collection of fields $\varphi_{a}$ ($a=1,\dots,n$)
we have (implicitly summing over $a$),
\begin{subequations}
\begin{align}
\canonicalStressEnergy^{\mu\nu} &:= -\frac{\partial\mathcal{L}}{\partial(\partial_{\mu}\varphi_{a})}\partial^{\nu}\varphi_{a}+\eta^{\mu\nu}\mathcal{L},\\
\intertext{or, written as a rank-2 covariant tensor,}
\canonicalStressEnergy_{\mu\nu} &:= -\frac{\partial\mathcal{L}}{\partial(\partial^{\mu}\varphi_{a})}\partial_{\nu}\varphi_{a}+\eta_{\mu\nu}\mathcal{L}.
\end{align}
\end{subequations}
\end{example}

\begin{exercise}
Prove $\partial_{\mu}{\canonicalStressEnergy^{\mu}}_{\nu}=0$ for each $\nu$.
\end{exercise}

\begin{exercise}
Prove or find a counter-example: $\canonicalStressEnergy_{\mu\nu}=\eta_{\mu\rho}{\canonicalStressEnergy^{\rho}}_{\nu}$
is not symmetric. When will it be symmetric?
\end{exercise}

\begin{exercise}
What are the 4 Noether charges for the canonical stress-energy tensor?
How do we interpret them physically?
\end{exercise}

\begin{exercise}
For the Scalar Field's Lagrangian from Eq~\eqref{eq:classical-field-theory:scalar-field:lagrangian},
compute the canonical stress-energy tensor. What happens for arbitrary
potential terms $V(\varphi)$?
\end{exercise}

\begin{exercise}
The stress-energy tensor appearing in Einstein's field equation is
obtained by (\S21.3 in Misner, Thorne, Wheeler~\cite{Misner:1973prb}):
\begin{equation}
\mbox{(RHS)}_{\mu\nu} = -2\frac{\partial\mathcal{L}_{\text{matter}}}{\partial g^{\mu\nu}}
+ g_{\mu\nu}\mathcal{L}_{\text{matter}}.
\end{equation}
In flat spacetime (i.e., when $g_{\mu\nu}=\eta_{\mu\nu}$), when does
this differ from the canonical stress-energy tensor? When do they coincide?
\end{exercise}

\N{References}
Noether's original paper~\cite{Noether1918:iv} may be worth reading,
though the terminology of group theory may be archaic compared to
today's vocabulary.
The discussion of Noether's theorems is largely inspired from Chapter 3
section 5 of Kiefer~\cite{Kiefer:2007ria},
though we also rely on Peskin and Schroeder~\cite{Peskin:1995ev} for the
discussion of Noether's first theorem and the canonical stress-energy
tensor. See also Chapter 22 of Srednicki~\cite{Srednicki:2007qs}, but
care must be taken: there is a sign error in his derivation of the
canonical stress-energy tensor (which is corrected in our derivation).
Sundermeyer~\cite{Sundermeyer:1982gv} is quite explicit in the
adjustments necessary for Noether's theorem to work with fields.
\section{Electromagnetism}

\M
We have reviewed in section~\ref{section:relativity:electromagnetism}
the covariant formalism of electromagnetism. From the perspective of
classical field theory, we now know the ``correct'' way to think of
things is that the field quantity of interest is the 4-potential
$A^{\mu}$ (\S\ref{chunk:relativity:electromagnetism:four-potential}).

\N{Lagrangian Density}
We recover Maxwell's equations using the Lagrangian
\begin{equation}
\mathcal{L} = \frac{-1}{4}F^{\alpha\beta}F_{\alpha\beta} = \frac{-1}{4}\eta^{\alpha\mu}\eta^{\beta\nu}F_{\mu\nu}F_{\alpha\beta}.
\end{equation}
If we work in curved spacetime, we need to multiply by the determinant
of the metric tensor $\sqrt{-\det(g_{\mu\nu})}$, but we will ignore this
factor.

\N{Equations of Motion}
We can now determine the equations of motion for the Lagrangian density.
The Euler--Lagrange equations take the form
\begin{equation}
\partial_{\mu}\frac{\partial\mathcal{L}}{\partial(\partial_{\mu}A_{\nu})}-\frac{\partial\mathcal{L}}{\partial A_{\nu}}=0.
\end{equation}
These will turn out to be:
\begin{equation}
\boxed{\partial_{\mu}F^{\mu\nu} = 0.}
\end{equation}
This is the source-free Maxwell's equations.

\begin{proof}[Proof (slick)]
We can compute the variation of the action directly, ignoring boundary terms,
as
\begin{calculation}
\variation\mathcal{L}
\step{unfolding the definition of the Lagrangian density}
\variation\left(\frac{-1}{4}\eta^{\alpha\mu}\eta^{\beta\nu}F_{\mu\nu}F_{\alpha\beta}\right)
\step{product rule and index gymnastics}
\frac{-1}{4}\eta^{\alpha\mu}\eta^{\beta\nu}F_{\mu\nu}(2\,\variation F_{\alpha\beta})
\step{unfolding the definition of field-strength tensor}
\frac{-1}{2}\eta^{\alpha\mu}\eta^{\beta\nu}F_{\mu\nu}(\variation\partial_{\beta}A_{\alpha}-\variation\partial_{\beta}A_{\alpha})
\step{index gymnastics, antisymmetry of field-strength tensor}
-\eta^{\alpha\mu}\eta^{\beta\nu}F_{\mu\nu}\variation(\partial_{\alpha}A_{\beta})
\step{integration by parts}
\eta^{\alpha\mu}\eta^{\beta\nu}(\partial_{\alpha}F_{\mu\nu})\variation A_{\beta}
+\mbox{(boundary terms)}.
\end{calculation}
This vanishes when
\begin{equation}
\partial_{\mu}F^{\mu\nu} = 0,
\end{equation}
and this is the result from the Euler--Lagrange equations of motion.
\end{proof}

\M
We can unfold the result of the Euler--Lagrange equations for
electromagnetism, and find
\begin{equation}
\partial_{\mu}F^{\mu\nu}=0\iff g^{\alpha\gamma}\partial_{\gamma}F_{\alpha\beta}=0.
\end{equation}
Then
\begin{calculation}
  g^{\alpha\gamma}\partial_{\gamma}F_{\alpha\beta}
\step{unfold definition of field-strength tensor}
g^{\alpha\gamma}\partial_{\gamma}(\partial_{\alpha}A_{\beta}-\partial_{\beta}A_{\alpha})
\step{distributivity, index gymnastics}
\partial^{\alpha}\partial_{\alpha}A_{\beta}-\partial^{\alpha}\partial_{\beta}A_{\alpha}
\step{equations of motion}
0.
\end{calculation}
When we impose the gauge condition $\partial^{\alpha}A_{\alpha}=0$,
we recover the familiar Maxwell equations as a wave equation
$\Box A_{\beta}=0$.

\N{Lagrangian coupled to matter}
We can write down the Lagrangian density for electromagnetism coupled to
some charged matter, recovering the Maxwell equations with some source
as in
Eq~\eqref{eq:relativity:electromagnetism:maxwell-for-electric-field}:
\begin{equation}\label{eq:classical-field-theory:electromagnetism:lagrangian}
\mathcal{L} = \frac{-1}{4}F^{\alpha\beta}F_{\alpha\beta}-4\pi J^{\mu}A_{\mu}.
\end{equation}

\begin{exercise}
Recall (\S\ref{ex:classical-field-theory:noether:canonical-stress-energy})
the notion of the canonical stress--energy tensor. Calculate
$\canonicalStressEnergy_{\mu\nu}$ for the Lagrangian in Eq~\eqref{eq:classical-field-theory:electromagnetism:lagrangian}.

[Hint: your answer should \emph{not} be symmetric --- that is, 
$\canonicalStressEnergy_{\mu\nu}\neq\canonicalStressEnergy_{\nu\mu}$.]
\end{exercise}

\N{Heuristic regarding interactions}\index{Heuristics}
If we want to describe interactions between two fields, or a field and
some matter, then we need to add a term to our Lagrangian of the form:
\begin{equation}
\mathcal{L}_{\text{interaction}}\sim\begin{pmatrix}\mbox{coupling}\\
\mbox{constant}
\end{pmatrix}\begin{pmatrix}\mbox{field}\\
\mbox{quantity}
\end{pmatrix}\begin{pmatrix}\mbox{matter}\\
\mbox{terms}
\end{pmatrix}.
\end{equation}
This is added to the potential term in the Lagrangian.

\subsection{Hamiltonian Formalism}

\M We will work through the calculations of the Symplectic two-form as a
series of exercises, and show it is degenerate. This is a consequence of
gauge symmetries. Then we will work through the Hamiltonian formalism
with its phase space parametrized by initial conditions.

\begin{exercise}
Prove $\displaystyle\variation L = \int\bigl(\variation A_{\beta}\partial_{\alpha}F^{\alpha\beta}+\partial_{0}(-F^{0\beta}\,\variation A_{\beta})\bigr)\,\D^{3}x$.
\end{exercise}

\begin{exercise}
  Prove the Symplectic potential is
  \[ \Theta(\variation A) = \int F^{\beta0}\,\variation A_{\beta}\,\D^{3}x =\int F^{i0}\,\variation A_{i}\,\D^{3}x\]
\end{exercise}

\begin{exercise}
Suppose $A^{\alpha}$ is a solution to the Maxwell's equations. Determine
what the linearized Maxwell equations are for tangent ``vectors''
$\variation A^{\beta}$ with base ``point'' $A^{\alpha}$.
\end{exercise}

\begin{exercise}[Symplectic form]
We will compute the Symplectic two-form for electromagnetism, and verify
it is degenerate when one of the fields is pure gauge.
\begin{enumerate}
\item Show the [naive] Symplectic two-form for electromagnetism $\Omega=\D\Theta$
is
\[\Omega(\variation_{1}A,\variation_{2}A) = \int\bigl((\partial_{t}\variation_{1}A^{i}-\partial^{i}\variation_{1}A_{t})\variation_{2}A_{i}-(\partial_{t}\variation_{2}A^{i}-\partial^{i}\variation_{2}A_{t})\variation_{1}A_{i}\bigr)\,\D^{3}x.\]
\item Consider $\variation_{2}A_{t}=\partial_{t}\Lambda$ and
  $\variation_{2}A_{i}=\partial_{i}\Lambda$ where $\Lambda$ is any
  function with compact support. Show
\[\int(\partial_{t}\variation_{1}A^{i}-\partial^{i}\variation_{1}A_{t})\variation_{2}A_{i}\,\D^{3}x=-\int\Lambda\partial_{i}(\partial_{t}\variation_{1}A^{i}-\partial^{i}\variation_{1}A_{t})\,\D^{3}x.\]
\item Show (if you haven't already) the linearized Maxwell's equations
  includes $\partial^{i}(\partial_{t}\variation A_{i}-\partial_{i}\variation A_{t})=0$.
\item Comparing the last two steps in this exercise, show the integral
  from step 2 vanishes, and this implies $\Omega(\variation_{1}A,\variation_{2}A) =0$
 (i.e., $\Omega$ is degenerate).
\end{enumerate}
\end{exercise}

\N{Initial Data}
Assuming we have picked some time-slicing and we have some initial data
\begin{equation}
\phi:=-A_{t}(\vec{x},t=0),\quad\mbox{and}\quad Q_{i}:=A_{i}(\vec{x},t=0),
\end{equation}
we can find the canonically conjugate momentum to $Q_{i}$ as,
\begin{subequations}
\begin{align}
P^{i} &= \frac{\partial\mathcal{L}}{\partial(\partial_{t}A_{i})}\\
&=\partial_{t}A_{i}-\partial_{i}A_{t}\\
&=\partial_{t}Q_{i}+\partial_{i}\phi.
\end{align}
\end{subequations}

\begin{exercise}
Verify that $\displaystyle\frac{\partial\mathcal{L}}{\partial(\partial_{t}A_{t})}=0$,
and therefore the canonically conjugate momentum for $\phi$ vanishes.
\end{exercise}

\begin{exercise}
Rewrite the Lagrangian as a functional of $Q_{i}$, $P^{i}$, and
$\phi$. Try to write it in ``canonical form'', i.e., as
\[ L[\phi, Q_{i}, P^{i}] = \int \bigl(P^{i}\partial_{t}Q^{i}-\mbox{(something)}\bigr)\,\D^{3}x.\]
[Hint: integration by parts and the divergence theorem are your friends.]
\end{exercise}

\M
The Lagrangian which you ought to obtain from the previous exercises
should be:
\begin{equation}\label{eq:classical-field-theory:electromagnetism:canonical-formalism:lagrangian}
L =
\int\left(P^{i}\partial_{t}Q_{i}-\left[\frac{1}{2}(P_{i}P^{i}+\frac{F_{ij}F^{ij}}{2}) + \phi\,\partial_{i}P^{i}\right]\right)\D^{3}x.
\end{equation}
Observe then that the equations of motion for $\phi$ are precisely
Gauss's Law:
\begin{equation}
\partial_{i}P^{i} = 0.
\end{equation}
We interpret $\phi$ as a Lagrange multiplier.

\begin{exercise}
Using the Lagrangian you should have computed (or lifted from the
previous chunk), prove the Euler--Lagrange equations:
\begin{equation}
\frac{\variation L}{\variation P^{i}}-\frac{\D}{\D t}\frac{\variation L}{\variation\partial_{t}P^{i}}=0,\quad
\frac{\variation L}{\variation Q^{i}}-\frac{\D}{\D t}\frac{\variation L}{\variation\partial_{t}Q^{i}}=0,\quad
\frac{\variation L}{\variation\phi}=0,
\end{equation}
are equivalent to the Maxwell equations.
\end{exercise}

\N{Hamiltonian}
We find the Hamiltonian functional by inspection of
Eq~\eqref{eq:classical-field-theory:electromagnetism:canonical-formalism:lagrangian}
to be:
\begin{equation}
H = \int\left(\frac{1}{2}(P_{i}P^{i}+\frac{F_{ij}F^{ij}}{2}) + \phi\,\partial_{i}P^{i}\right)\D^{3}x.
\end{equation}
The first two terms coincide with our expectations, but the last term
may be surprising.

\begin{ddanger}
When we have a constrained Hamiltonian system, we add the first-class
constraints to the Hamiltonian. This is precisely what's going on with
the Hamiltonian functional as we've written it down. This is studied
thoroughly in Henneaux and Teitelboim~\cite{Henneaux:1992ig}.
\end{ddanger}

\begin{exercise}\index{Hamilton's equations}
The reader can verify Hamilton's equations,
\begin{subequations}
\begin{align}
\partial_{t}Q_{i} &= \frac{\delta H}{\delta P^{i}} = P_{i} - \partial_{i}\phi\\
\partial_{t}P_{i} &= -\frac{\delta H}{\delta P^{i}} = \partial_{i}F^{ij}.
\end{align}
\end{subequations}
\end{exercise}

\begin{exercise}\index{Poisson bracket}
Using the Poisson brackets,
\begin{equation}
\PB{M}{N} = \int\left(\frac{\delta M}{\delta Q_{i}(\vec{x}')}\frac{\delta N}{\delta P^{i}(\vec{x}')}
-\frac{\delta N}{\delta Q_{i}(\vec{x}')}\frac{\delta M}{\delta P^{i}(\vec{x}')}\right)\D^{3}x',
\end{equation}
show the quantity
\begin{equation}
G = - \int\Lambda(\vec{x})\partial_{i}P^{i}\,\D^{3}x
\end{equation}
is the generating function for gauge transformations
\begin{equation}
\variation Q_{i}=\PB{Q_{i}}{G}=\partial_{i}\Lambda,\quad
\variation P^{i}=\PB{P^{i}}{G}=0.
\end{equation}
\end{exercise}

\subsection{Scalar Electrodynamics}

\M
One of the first models we study in quantum field theory is something
called ``scalar electrodynamics''. This is obtained by taking a complex
Scalar field and coupling it to Electromagnetism. Let us review the
pertinent aspects of the complex Scalar field, then let us try to couple
it to electromagnetism.

\subsubsection{Complex Scalar Field}

\M We take 2 real-valued scalar fields
$\varphi_{1}$ and $\varphi_{2}$, then form the complex scalar
field\footnote{This is an abuse of notation, similar to using $z$ and
$\bar{z}$ in complex analysis as the independent coordinates of the
Complex plane.} 
\begin{equation}
\varphi(x) = \frac{\varphi_{1}(x)+\I\varphi_{2}(x)}{\sqrt{2}},\quad\mbox{and}\quad
\varphi^{*}(x) = \frac{\varphi_{1}(x)-\I\varphi_{2}(x)}{\sqrt{2}}.
\end{equation}
Then we couple the complex scalar field to electromagnetism.

\N{Complex Scalar Field}\index{Scalar Field!Complex}
The complex scalar field (also called the \emph{charged Klein--Gordon field})
may be viewed as a mapping
\begin{equation}
\varphi\colon\RR^{3,1}\to\CC.
\end{equation}
The Lagrangian for the complex Scalar field is:
\begin{equation}
\mathcal{L}_{cs} = -(\eta^{\alpha\beta}\partial_{\alpha}\varphi\partial_{\beta}\varphi^{*}+\mu^{2}|\varphi|^{2}).
\end{equation}

\begin{exercise}
Show the Euler--Lagrange equations give you the equations of motion
\begin{subequations}
\begin{align}
\frac{\variation S}{\variation\varphi}=0 &\implies (\partial^{\alpha}\partial_{\alpha}+\mu^{2})\varphi^{*}=0,
\intertext{and}  
\frac{\variation S}{\variation\varphi^{*}}=0 &\implies (\partial^{\alpha}\partial_{\alpha}+\mu^{2})\varphi=0.
\end{align}
\end{subequations}
\end{exercise}

\N{Symmetries of Complex Scalar Field}
We can use Noether's theorem to find that the complex scalar field
admits a continuous symmetry:
\begin{equation}
\varphi_{\lambda}=\E^{\I\lambda}\varphi,\quad
\varphi_{\lambda}^{*}=\E^{-\I\lambda}\varphi^{*},
\end{equation}
where $\lambda\in\RR$ is arbitrary. Since $\lambda$ is a constant, we
see the kinetic term of the Lagrangian is invariant under this
transformation. Similarly, we see $|\varphi|^{2}=\varphi^{*}\varphi$ is
invariant under this transformation.

\begin{remark}\index{Symmetry!Global}\index{Symmetry!Local}
This is a ``global $\U(1)$'' symmetry. It's ``global'' because the
parameter $\lambda$ is a real number independent of spacetime. (If
$\lambda$ were a function of spacetime $\lambda=\lambda(\vec{x},t)$,
then we would call it a ``local'' symmetry.) It's a $\U(1)$ symmetry
because $\E^{\I\lambda}\in\U(1)$.

Older literature use the term ``gauge transformation of the first kind''\index{Gauge!Transformation!Of the first kind}
instead of ``global symmetry transformation''
\end{remark}

\begin{exercise}
Use Noether's theorem to prove this is a continuous symmetry of the
complex scalar field. Then determine the conserved current $j^{\beta}$
for this symmetry.
\end{exercise}
\begin{exercise}
Prove the Noether charge for the complex Scalar field is
\begin{equation}
Q = \I\int_{V}(\varphi^{*}\partial_{t}\varphi-\varphi\partial_{t}\varphi^{*})\,\D^{3}x.
\end{equation}
\end{exercise}

\N{Sigma Models}\index{Sigma model@$\sigma$ Model}
There is a way to generalize this construction from 2 real Scalar fields
to $N$ real Scalar fields. This is a family of models called
\define{Sigma models} where the scalar fields are components of a smooth
function $\sigma\colon\RR^{3,1}\to\mathcal{M}$ where $\mathcal{M}$ is
usually a Lie group. (There is no significance to the choice of $\sigma$
for scalar fields, and Sigma models refer to this historic artifact of
arbitrary notation.) Then the Lagrangian density for the massless case
is:
\begin{equation}
\mathcal{L} = \frac{1}{2}\sum^{N}_{A,B=1}g_{AB}(\sigma)\partial^{\mu}\sigma^{A}\partial_{\mu}\sigma^{B},
\end{equation}
where $G_{AB}(\sigma)$ is the metric tensor on the field space $\mathcal{M}$,
and $\partial_{\mu}$ are the derivatives on the underlying spacetime
manifold $\RR^{3,1}$. When we include some self-interaction terms, we
obtain a \emph{Nonlinear $\sigma$ Model}.

When $\mathcal{M}=\CC^{2}$, for example, we can show the $\sigma$ model
enjoys an $\SU(2)$ symmetry. Similarly, for $\mathcal{M}=\CC^{n}$, the
$\sigma$ model enjoys an $\SU(n)$ symmetry. These models are useful as
``prolegomenon'' to Yang--Mills theory for the Standard Model.

\begin{remark}
Sigma models were first introduced in \S\S5--6 of Gell-Mann and Levy~\cite{Gell-Mann:1960mvl}.
Initially, $\sigma$ was ``just another scalar field'' in that paper. Later
physicists adopted $\sigma$ as we have introduced it: as a familar of
scalar fields.
\end{remark}

\subsubsection{Charged Scalar Field coupled to Electromagnetism}

\M
Now we can couple the complex Scalar field to Electromagnetism.
The basic idea is we will form the Lagrangian density for scalar
electrodynamics by adding the Lagrangian density for the complex Scalar
field to the Lagrangian density for the Electromagnetic field, plus the
4-current coupling the charged Scalar field to the Electromagnetic field:
\begin{equation}
\begin{split}
\mathcal{L}_{sED}&=\mathcal{L}_{cs}+\mathcal{L}_{EM}+\mathcal{L}_{int}\\
&=-(\eta^{\alpha\beta}\partial_{\alpha}\varphi\partial_{\beta}\varphi^{*}+\mu^{2}|\varphi|^{2})
-\frac{1}{4}F^{\alpha\beta}F_{\alpha\beta}
+4\pi j^{\alpha}A_{\alpha}.
\end{split}
\end{equation}
We just need to determine $j^{\alpha}$ in terms of the complex Scalar
field $\varphi$.

We know from Noether's theorem there is a conserved current for the
$\U(1)$ Symmetry for the complex Scalar field,
\begin{equation}
j^{\alpha} = -\I\eta^{\alpha\beta}(\varphi^{*}\partial_{\beta}\varphi - \varphi\partial_{\beta}\varphi^{*}).
\end{equation}
There is some slight difficulties with the Lagrangian as we have written
it: it is no longer gauge invariant under $A^{\mu}\to
A^{\mu}+\partial^{\mu}\Lambda$.

\N{Minimal coupling}
The ``physically correct way'' to get a gauge-invariant Lagrangian which
still gives the $j^{\alpha}A_{\alpha}$ coupling is rather unintuitive:
we use a different differential operator than $\partial_{\mu}$ in the
charged Scalar field's Lagrangian density.

This is the so-called \define{Minimal Coupling}, where we replace
\begin{subequations}
\begin{equation}
\partial_{\alpha}\varphi\to D_{\alpha}\varphi := (\partial_{\alpha}+\I qA_{\alpha})\varphi,
\end{equation}
and
\begin{equation}
\partial_{\alpha}\varphi^{*}\to D_{\alpha}\varphi^{*} := (\partial_{\alpha}-\I qA_{\alpha})\varphi^{*}.
\end{equation}
\end{subequations}
Here $q$ is a parameter reflecting the coupling strength between the
charged scalar field $\varphi$ and the Electromagnetic field. It's an
example of a \emph{coupling constant}.\index{Coupling constant}
Then we modify the complex Scalar field's Lagrangian density to use
these gauge covariant derivatives,
\begin{equation}\label{eq:classical-field-theory:electromagnetism:lagrangian-density-for-complex-scalar-field-using-gauge-covariant-derivatives}
\mathcal{L}_{cs} = -\eta^{\alpha\beta}D_{\alpha}\varphi^{*}D_{\beta}\varphi-\mu^{2}|\varphi|^{2}.
\end{equation}
This Lagrangian density yields field equations which are the usual wave
equations plus some modifications involving the electromagnetic
potential.

\begin{exercise}
Compute the Euler--Lagrange equations for $\varphi$ and $\varphi^{*}$ using the Lagrangian density from Eq~\eqref{eq:classical-field-theory:electromagnetism:lagrangian-density-for-complex-scalar-field-using-gauge-covariant-derivatives}.
\end{exercise}

\M
Now observe, under a gauge transformation of the electromagnetic
4-potential
\begin{subequations}
\begin{equation}
A_{\alpha}\to A_{\alpha}+\partial_{\alpha}\Lambda,
\end{equation}
for the gauge covariant derivatives of the complex Scalar field to
remain invariant under these gauge transformations, we need:
\begin{align}
  \varphi&\to\E^{-\I q\Lambda}\varphi,\\
  \intertext{and}
  \varphi^{*}&\to\E^{\I q\Lambda}\varphi^{*}.
\end{align}
\end{subequations}
The reader can verify that the gauge covariant derivatives of the
complex Scalar field then transform as
\begin{subequations}
\begin{align}
D_{\alpha}\varphi &\to\E^{-\I q\Lambda}D_{\alpha}\varphi,\\
D_{\alpha}\varphi^{*} &\to\E^{\I q\Lambda}D_{\alpha}\varphi^{*}.
\end{align}
\end{subequations}
We can see that the kinetic term for the modified complex Scalar
Lagrangian density remains invariant under these transformations.

\M
Since the electromagnetic interaction with the complex Scalar fields are
``swept into'' the gauge covariant derivatives, we can write the
Lagrangian density for the scalar electrodynamic theory as:
\begin{subequations}
\begin{equation}
\mathcal{L}_{sED} = \frac{-1}{4}F^{\alpha\beta}F_{\alpha\beta} - \eta^{\alpha\beta}D_{\alpha}\varphi^{*}D_{\beta}\varphi-\mu^{2}|\varphi|^{2}.
\end{equation}
When we expand the gauge covariant derivatives in this Lagrangian
density, we have:
\begin{equation}
\mathcal{L}_{sED} = \frac{-1}{4}F^{\alpha\beta}F_{\alpha\beta} - \eta^{\alpha\beta}\partial_{\alpha}\varphi^{*}\partial_{\beta}\varphi-\mu^{2}|\varphi|^{2}
+\I q A^{\alpha}(\varphi^{*}\partial_{\alpha}\varphi-\varphi\partial_{\alpha}\varphi^{*}+\I q A_{\alpha}|\varphi|^{2}).
\end{equation}
\end{subequations}
The Euler--Lagrange equations for the Electromagnetic 4-potential are
\begin{equation}
\partial_{\beta}F^{\alpha\beta}=-4\pi J^{\alpha},
\end{equation}
where the current is defined using the gauge covariant derivatives as
\begin{equation}\label{eq:classical-field-theory:sed:charged-current}
J^{\alpha} = -\frac{\I q}{4\pi}(\varphi^{*}D^{\alpha}\varphi-\varphi D^{\alpha}\varphi^{*}).
\end{equation}
This is rather magical, but we could derive the same results using
Noether's first theorem for fields.

\M
We should mention that physicists look at
Eq~\eqref{eq:classical-field-theory:sed:charged-current} and interpret
it as telling us the electromagnetic charge for the complex scalar field
cannot ``exist alone'' in the Scalar field. In an interacting system,
the division between ``source fields'' and ``fields mediating interactions''
is rather artificial and arbitrary. This is physically reasonable (even
if a little surprising). Mathematically this feature emerges from
demanding gauge invariance.

The complex Scalar field is no longer uniquely defined in scalar
electrodynamics: it is subject to a gauge transformation, just like the
electromagnetic 4-potential.

When we have such an interaction, if we want to compute (say) the
electromagnetic field contained in a region $V$, we need a solution
$(A,\varphi)$ of the coupled Maxwell--Scalar equations, then substitute
it into:
\begin{equation}
Q = \frac{1}{4\pi}\int_{V}\I q(\varphi^{*}D_{0}\varphi-\varphi D_{0}\varphi^{*})\,\D^{3}x.
\end{equation}
This charge is conserved and gauge invariant.

\begin{exercise}
Prove the total electric charge $Q$ is conserved \emph{and} gauge invariant.
\end{exercise}

\begin{exercise}
Suppose we had a Lagrangian for complex scalar fields coupled to
electromagnetism of the form
\begin{equation}
\mathcal{L}=\mathcal{L}_{EM}-\frac{1}{2}[\varphi g^{\mu}D_{\mu}\varphi^{*}-\varphi^{*}g^{\mu}D_{\mu}\varphi]-\mu^{2}|\varphi|^{2},
\end{equation}
where $g^{\mu}$ is ``some [constant] vector'', and $D_{\mu}$ is the
gauge covariant derivative.
\begin{enumerate}
\item How must $\varphi$ and $\varphi^{*}$ transform under gauge
  transformations $A_{\alpha}\to A_{\alpha}+\partial_{\alpha}\Lambda$?
\item How do the gauge covariant derivatives $D_{\alpha}\varphi$ and
  $D_{\alpha}\varphi^{*}$ transform under gauge transformations?
\end{enumerate}
\end{exercise}

\subsection{Chern--Simons Theory}

\M
In $2+1$ dimensions (instead of $3+1$ dimensions), we have a particular
action which plays an important role in physics called the Chern--Simons
action named after its discoverer Albert Schwarz:
\begin{equation}
\begin{split}
  \action_{CS}[A] &= \frac{k}{4\pi}\int\epsilon^{\mu\nu\rho}A_{\mu}\partial_{\nu}A_{\rho}\,\D^{3}x\\
&\mbox{``=''}\; \frac{k}{4\pi}\int(A\times\nabla)^{\rho} A_{\rho}\,\D^{3}x
\end{split}
\end{equation}
where $A_{\mu}$ is a ``4''-potential for electromagnetism. Later we will
generalize Electromagnetism to Yang--Mills theory, and the Chern--Simons
theory will play an important role in something called the Quantum Hall
effect. It also describes quantum gravity in $2+1$ dimensions.

For a Yang--Mills field, however, we also have a term that looks like
$A^{3}$ in the action. Such a term vanishes for commutative gauge groups
like $\U(1)$ (i.e., like for Electromagnetism).

\begin{exercise}
From demanding stationary action $\variation\action_{CS}[A]/\variation A^{\mu}(x)=0$,
determine the equations of motion for Chern--Simons theory for the
electromagnetic field.
\end{exercise}
\section{*Path Independence}

\M Teitelboim showed in his PhD thesis and several follow-up articles~\cite{Hojman:1976vp,Teitelboim:1980hs} how
we can use the canonical formalism coupled to the principle of ``path
independence'' to derive General Relativity, Yang--Mills gauge theory,
among other things. We also saw in passing
(\S\ref{chunk:rqm:poincare-algebra:elementary-particles-irreps}) the
only fields possible are Scalar fields, Vector fields, and (rank-2
symmetric) Tensor fields.

\N{Foliating Spacetime}
We foliate spacetime by space-like hypersurfaces $\Sigma_{t}$ indexed by
time $t$. We have the metric $h_{ij}$ on $\Sigma_{t}$ where $i$,
$j=1,2,3$ are spatial indices.

We stipulate that $h_{ij}$ has its canonically conjugate momenta density
$p^{ij}$. 

\N{Evolution}
If we want to consider the time evolution of a function $F(h_{ij}(x), p^{k\ell})$
from one spatial hypersurface $\Sigma_{t}$ to another $\Sigma_{t+\delta t}$
be described using generators $\mathcal{H}_{i}(x)$ and $\mathcal{H}_{\bot}(x)$
and the Poisson bracket
\begin{equation}
\begin{split}
\dot{F}(h_{ij}(x),p^{k\ell}(x))
&=\int\bigl(\PB{F}{\mathcal{H}_{\bot}(x')}N(x') + \PB{F}{\mathcal{H}_{i}(x')}N^{i}(x')\bigr)\,\D^{3}x'\\
&=\int\PB{F}{\mathcal{H}_{\mu}(x')}N^{\mu}(x')\,\D^{3}x'.
\end{split}
\end{equation}
If we want to consider spatial diffeomorphisms $F\to F + \variation F$
along the same hypersurface $\Sigma_{t}$, then we use:
\begin{equation}
\variation F = -\int\PB{F}{\mathcal{H}_{i}(x')}\,\variation N^{i}(x')\,\D^{3}x'.
\end{equation}


\N{Fundamental Poisson Brackets}
We generate the transformations desired by the Poisson brackets:
\begin{subequations}
\begin{align}
\PB{\mathcal{H}_{\bot}(x)}{\mathcal{H}_{\bot}(x')} &= [h^{ij}(x)\mathcal{H}_{j}(x) + h^{ij}(x')\mathcal{H}_{j}(x')]\partial_{i}\delta(x,x')\\
\PB{\mathcal{H}_{i}(x)}{\mathcal{H}_{\bot}(x')} &= \mathcal{H}_{\bot}(x)\partial_{i}\delta(x,x')\\
\PB{\mathcal{H}_{i}(x)}{\mathcal{H}_{j}(x')} &= \mathcal{H}_{i}(x')\partial_{j}\delta(x,x')+\mathcal{H}_{j}(x)\partial_{i}\delta(x,x').
\end{align}
\end{subequations}
The ``only'' generator of ``real'' dynamical interest is $\mathcal{H}_{\bot}$.
The three $\mathcal{H}_{i}$ generate spatial diffeomorphisms, i.e.,
displacements that lie on the same spatial hypersurface which amount to
a change of spatial coordinates.

\N{Gravitating and Matter Generators}
We separate the generators $\mathcal{H}_{\bot}$ and $\mathcal{H}_{i}$
into a sum of the ``gravitational part'' and the ``matter part''. We
would have
\begin{equation}
\mathcal{H}_{\bot} = \mathcal{H}_{\bot}^{\text{grav}}[h_{ij},p^{k\ell}]
+ \mathcal{H}_{\bot}^{\text{mat}}[h_{ij},p^{k\ell},\mbox{matter canonical variables}].
\end{equation}
If $\mathcal{H}_{\bot}^{\text{mat}}$ did not depend on gravitational
variables (the spatial metric and its conjugate momenta), then the
Poisson bracket
$\PB{\mathcal{H}_{\bot}^{\text{mat}}}{\mathcal{H}_{\bot}^{\text{mat}}}$
cannot satisfy the desired relation (it'd be independent of $h_{ij}$,
but it must involve $h^{ij}$).

\M
The general form of $\mathcal{H}_{i}$ meanwhile may be constrained by
the following desired requirements:
\begin{enumerate}
\item It must be linear in the momenta in order to generate
  transformations of the fields under coordinate transformations (and
  not mix fields and momenta);
\item It should contain the momenta only up to the first spatial
  derivatives because it should generate first-order derivatives in the
  fields (think ``Taylor expansion to first order'').
\end{enumerate}
Therefore we expect
\begin{equation}
\mathcal{H}_{i}={b_{i}}^{jB}(\phi^{C})\partial_{b}p^{B} + {a_{a}}^{B}(\phi^{C})p_{B},
\end{equation}
where $\phi^{A}$ is now a symbolic notation for \emph{all} fields
\emph{including gravity}, and $p_{A}$ denotes the corresponding momenta.

\N{Heuristic for $\mathcal{H}_{\bot}$}
We stipulate that $\mathcal{H}_{\bot}$ is a quadratic polynomial in the
momenta,
\begin{equation}
\mathcal{H}_{\bot}\sim M_{AB}(\phi^{C})p^{A}p^{B} + N_{A}(\phi^{C})p^{A} + V(\phi^{C}),
\end{equation}
then we just need to determine the coefficients $M_{AB}$, $N_{A}$, $V$.
The calculations for the various fields appears in Section 6 of
Teitelboim~\cite{Teitelboim:1980hs}. 

\N{Spatial Diffeomorphism Requirements}
We demand the existence of \emph{ultralocal solutions}, in the sense
that a deformation localized at a point $\vec{x}_{0}$ must change the
field only at $\vec{x}_{0}$, then we find
\begin{equation}
b^{ijB}=b^{jiB}.
\end{equation}
So under an infinitesimal coordinate transformation
\begin{equation}
x'^{i} = x^{i} - \variation N^{i}(x),
\end{equation}
we will demand a field quantity $\phi$ transforms as
\begin{equation}
\LieD_{\variation\vec{N}}\phi =
-\int\PB{\phi}{\mathcal{H}_{i}(x')}\,\variation N^{i}(x')\,\D^{3}x',
\end{equation}
where $\LieD$ denotes the Lie derivative.\index{Lie derivative}
This will then give us a way to find $\mathcal{H}_{i}(x')$.

\N{Spatial Diffeomorphism Invariance of Scalar Fields}
For Scalar fields $\phi(x)$, it must transform as
\begin{equation}
\variation\phi(x):=\phi'(x)-\phi(x)\weakEq\partial_{i}\phi\,\variation N^{i}=\LieD_{\variation\vec{N}}\phi.
\end{equation}
This is generated by
\begin{equation}\label{eq:classical-field-theory:henneaux:scalar-field-momenta-constraints}
\mathcal{H}_{i} = p_{\phi}\partial_{i}\phi,
\end{equation}
where $p_{\phi}$ is the conjugate momenta density for $\phi$.

\begin{exercise}
Show that Eq~\eqref{eq:classical-field-theory:henneaux:scalar-field-momenta-constraints}
implies $b^{ij}=0=b^{ji}$. Therefore it satisfies ultralocality.
\end{exercise}

\begin{exercise}
Verify Eq~\eqref{eq:classical-field-theory:henneaux:scalar-field-momenta-constraints}
generates the correct transformations for $\phi$.
\end{exercise}

\N{For Vector Fields}
The vector field $A_{i}(x)$ transforms as
\begin{equation}
\variation A_{i} = (\partial_{j}A_{i})\,\variation
N^{j}+A_{j}\partial_{i}(\variation N^{j}) = (\LieD_{\variation\vec{N}}A)_{i},
\end{equation}
which is generated by
\begin{equation}\label{eq:classical-field-theory:henneaux:vector-field-momenta-constraints}
\mathcal{H}_{i} = -A_{i}\partial_{j}p^{j} + (\partial_{i}A_{j}-\partial_{j}A_{i})p^{j}.
\end{equation}

\begin{exercise}
Verify Eq~\eqref{eq:classical-field-theory:henneaux:vector-field-momenta-constraints}
generates $\variation A_{i}$.
\end{exercise}

\begin{exercise}[DO THIS]
Show that Eq~\eqref{eq:classical-field-theory:henneaux:vector-field-momenta-constraints}
implies ${{b_{i}}^{j}}_{C}=-A_{i}{\delta^{j}}_{C}$ and therefore the
condition for ultralocality is not fulfilled for vector fields.
\end{exercise}

\N{Covariant Rank-2 Tensor}
For a covariant rank-2 tensor (not necessarily symmetric) $t_{ij}$, we
have
\begin{equation}
\variation t_{ij} = (\partial_{k}t_{ij})\variation N^{k}
+ t_{ik}\partial_{j}(\variation N^{k})
+ t_{jk}\partial_{i}(\variation N^{k})
= (\LieD_{\variation\vec{N}}t)_{ij}.
\end{equation}
This is generated by
\begin{equation}
\mathcal{H}_{i} = p^{k\ell}\partial_{i}t_{k\ell}-\partial_{k}(t_{ij}p^{kj})-\partial_{j}(t_{ki}p^{kj}).
\end{equation}
In order for ultralocality to hold, we must have
\begin{equation}
t_{ij} = f(x)h_{ij}
\end{equation}
where $f(x)$ is an arbitrary function. That is to say, the tensor field
must be proportional to the metric tensor. We can interpret this as
another piece of evidence suggesting the ``only'' spin-2 field allowed
is the gravitational field.\index{Spin!Uniqueness of (---)-2}

\begin{exercise}[Fermions]\index{Lie derivative!Of spinor field}
  Can we do a similar analysis for a Dirac spinor field $\psi$ in curved
  spacetime? (I honestly do not know!)
  
  There are various generalizations of the Lie derivative to the
  spin-$1/2$ field. Godina and Matteucci~\cite{Godina:2005mt} reviews the various
  generalizations of the Lie derivative for spinors.  Specifically:
\begin{enumerate}
\item Find the ``right'' generalization of the Lie derivative to spinor
  fields. There are many possibilities! For one single example,
  Choquet-Bruhat and DeWitt-Morette~\cite{Choquet-Bruhat:2000amp2} propose
  \[\LieD_{X}\psi = X^{j}\partial_{i}\psi - \frac{1}{8}(\partial_{i}X_{j}-\partial_{j}X_{i})\gamma^{i}\gamma^{j}\psi = \variation\psi,\]
  where $\psi$ is a 4-component Dirac spinor field,
  $\gamma^{i}$, $\gamma^{j}$ are the Dirac matrices; is this the
  ``right'' generalization for our purposes?
\item Since we're dealing with
  fermionic fields, how must we modify the spatial diffeomorphism
  condition
  $\variation\psi\sim-\int\PB{\psi}{\mathcal{H}_{i}(y)}\,\variation N^{i}(y)\,\D^{3}y$?
  Does it suffice to use the Poisson \emph{super}-bracket?
\item What $\mathcal{H}_{i}$ generates this? 
\item Is this $\mathcal{H}_{i}$ ultralocal?
\end{enumerate}
\end{exercise}

\N{Gauge Theory: Recovering Ultralocality for Vector Fields}
We see that $\mathcal{H}_{i}$ is not ultralocal. It's caused by the
presence of the $A_{i}\partial^{j}p_{j}$ term. So it's tempting to make
the replacement
\begin{equation}
\mathcal{H}_{i}\to\widetilde{\mathcal{H}}_{i}
:=\mathcal{H}_{i} + A_{i}\partial_{j}p^{j}
=(\partial_{i}A_{j}-\partial_{j}A_{i})p^{j}.
\end{equation}
We see then that the Poisson bracket of this generator with itself is:
\begin{equation}\label{eq:classical-field-theory:henneaux-pi:modified-vector-PB}
\PB{\widetilde{\mathcal{H}}_{i}(x)}{\widetilde{\mathcal{H}}_{j}(y)}
= \widetilde{\mathcal{H}}_{j}(x)\partial_{i}\delta(x,y)
+ \widetilde{\mathcal{H}}_{i}(y)\partial_{j}\delta(x,y)
- F_{ij}(x)\partial_{k}p^{k}(x)\delta(x,y),
\end{equation}
where $F_{ij}=\partial_{i}A_{j}-\partial_{j}A_{i}$. This new term
appearing in the right-hand side of
Eq~\eqref{eq:classical-field-theory:henneaux-pi:modified-vector-PB} is
harmless \emph{only} if it generates physically irrelevant
transformations (``gauge transformations''). There are two ways this
could happen: either $F_{ij}$ is constrained to vanish (which is too
strong, this would imply $A_{i}=\partial_{i}\varphi$), or we demand
$\partial_{i}p^{i}\weakEq0$ is a constraint. We therefore introduce the
constraint
\begin{equation}
\mathcal{G}(x) := \frac{-1}{e}\partial_{i}p^{i}(x) =
\frac{-1}{e}\partial_{i}E^{i}(x) = \frac{-1}{e}\nabla\cdot\vec{E}(x),
\end{equation}
where $e$ is the electric charge, and the momentum is just $\vec{E}$ the
electric field. This constraint $\mathcal{G}\weakEq0$ is precisely
\emph{Gauss's Law}.

\begin{exercise}
Prove $\widetilde{H}_{i}$ is gauge invariant. Hint: it suffices to prove
the electric field is gauge invariant, and $F_{ij}$ is gauge invariant.
\end{exercise}

\begin{exercise}
Show Gauss's Law generates the gauge transformations
\begin{subequations}
\begin{align}
\variation A_{i}(x) &= \int\PB{A_{i}(x)}{\mathcal{G}(y)}\,\xi(y)\,\D^{3}y=\frac{1}{e}\partial_{i}\xi(x)\\
\variation p^{i}(x) &= \int\PB{p^{i}(x)}{\mathcal{G}(y)}\,\xi(y)\,\D^{3}y=0.
\end{align}
\end{subequations}
\end{exercise}

\M
Therefore we conclude that $A_{i}(x)$ transforms under
$\widetilde{H}_{j}$ as a vector modulo a gauge transfromation.

\N{Generalization to Yang--Mills}
When we replace $A_{i}(x)$ with several fields $A_{i}^{a}(x)$ for $a=1$,
\dots, $N$. We assume that:
\begin{enumerate} 
\item $A_{i}^{a}(x)$ does not mix with its momenta $p^{i}_{a}(x)$ under
  a gauge transformation,
\item the momenta should transform homogenously, and
\item the gauge constraint (the non-Abelian version of Gauss's Law) is local.
\end{enumerate}
This forces us to,
\begin{equation}
\mathcal{G}_{i} = \frac{-1}{f}\partial_{i}{p_{a}}^{i} + {C_{ab}}^{c}A^{a}_{i}p^{i}_{c}\weakEq0,
\end{equation}
where $f$ and ${C_{ab}}^{c}$ are constants. If we demand the commutator
of two gauge transformations yields another gauge transformation, then
we find ${C_{ab}}^{c}$ are the structure constants of a Lie algebra. We
have:
\begin{equation}
\PB{\mathcal{G}_{a}(x)}{\mathcal{G}_{b}(y)}={C_{ab}}^{c}\mathcal{G}_{c}(x)\delta(x,y),
\end{equation}
which precisely characterizes a Yang--Mills theory.

\M
If we split up $\mathcal{H}_{\bot}$ as a sum
\begin{equation}
\mathcal{H}_{\bot}=\mathcal{H}_{\bot}^{\text{grav}}+\mathcal{H}_{\bot}^{\text{YM}},
\end{equation}
and demanding the Yang--Mills part is independent of the gravitational
momenta (so $\mathcal{H}_{\bot}^{\text{YM}}$ depends only ultralocally
on the metric), then we find
\begin{equation}
\mathcal{H}_{\bot}^{\text{YM}} = \frac{1}{2\sqrt{h}}(h_{ij}\gamma^{ab}p^{i}_{a}p^{j}_{b}+h^{ij}\gamma_{ab}B^{a}_{i}B^{b}_{j}),
\end{equation}
where $\gamma_{ab}={C_{ac}}^{d}{C_{bd}}^{c}$ is the ``group metric''
($\gamma^{ab}$ is its inverse), and
$B^{a}_{i}=\frac{1}{2}\epsilon_{ijk}F^{ajk}$ are the non-Abelian
``magnetic fields''. The non-Abelian field strength is given by:
\begin{equation}
F^{a}_{ij}=\partial_{i}A^{a}_{j} - \partial_{j}A^{a}_{i}
+f{C^{a}}_{bc}A^{b}_{i}A^{c}_{j}.
\end{equation}
The Hamiltonian corresponds to the action,
\begin{equation}
\action_{\text{YM}} = \frac{-1}{4}\int\gamma_{ab}F^{a}_{\mu\nu}F^{b\mu\nu}\sqrt{-g}\,\D^{4}x
\end{equation}
where $\sqrt{-g}$ is the determinant of the metric of spacetime.

\M
So the principle of path independent taken together with the demand that
$\mathcal{H}_{\bot}$ be ultralocal in the momenta necessarily leads to
the concept of gauge theories.


\chapter{Outline of Quantum Field Theory}

\M
The big idea is that we want to describe scattering using quantum
calculations, involving different types of particles and different types
of interactions [fields], using different ``mathematical toolkits'' and
pictures [sum over histories, functional Schr\"{o}dinger, etc.].
Consequently any thorough textbook would walk through 9 calculations:
for each of the three toolkits [Heisenberg/interaction, path integral,
functional Schr\"{o}dinger], we compute the propagator and scattering
of the three particles [scalar, ``spinor'' (spin-$1/2$), and vector/gauge].


\section{Scattering}

\N{Scattering}
The basic idea is we have $2\to2$ scattering\footnote{Decay may be
interpreted as $1\to n$, and other situations may be described
analogously.}  (i.e., we collide two particles towards each other, and
then two particles emerge from the ``collision'') and we try to measure
something in the laboratory. Usually this is the scattering angle
$\theta$, which is related to the cross-sectional area $\sigma$ by some
equation of the form
\begin{equation}
\frac{\D\sigma}{\D\cos\theta}\sim f(\cos\theta).
\end{equation}
Quantum theory permits us to write an equation relating $\D\sigma$ to
entries of the $S$-matrix. The problem for the theorist is to compute
these $S$-matrix components.

\M
The LSZ formula relates $S$-matrix theory to quantum field theory.
Specifically, the components of the $S$-matrix
\begin{equation}
S_{f,i} = \langle f|S|i\rangle
\end{equation}
may be related to the asymptotic free field via the LSZ formula.

\N{Adiabatic ``Theorem''}
The vacuum state used in the LSZ formula is $|\Omega\rangle$, the vacuum
state for the interacting theory --- compared to the $|0\rangle$ vacuum
state for the free theory. 

The assumption is that $|\Omega\rangle$ is a perturbation of
$|0\rangle$, so they can be related. This is carried out in Peskin and
Schroeder (chapter 4, section 2; see esp.\ Eq~(4.27)).

\N{Feynman Diagrams}
We expand the LSZ formula perturbatively, and organize the terms using
Feynman diagrams describing different interactions.

% https://physics.stackexchange.com/a/510046
% https://physics.stackexchange.com/a/15164
\N{Propagator}
For a particle with spin-$J$ and mass $m$, the propagator schematically
look like\footnote{The denominator depends on the channel the scattering
takes place, I think it could be $t-m^{2}$ in the $t$-channel.}
\begin{equation}
A_{J}(s,t) \sim \frac{s^{J}}{s-m^{2}}.
\end{equation}
This can be derived from representation theory, since any spin-$J$ field
may be formed from tensoring $2J$ copies of a spinor
\begin{equation}
\phi^{\alpha_{1}\cdots\alpha_{2J}}\sim\psi^{\alpha_{1}}\otimes\cdots\otimes\psi^{\alpha_{2J}}
\end{equation}
in the sense that the representation $J$ is a subspace of the
representation $(1/2)^{\otimes 2J}$:
\begin{equation}
\bigotimes^{2j} \frac12=j\oplus (j-1)\oplus\cdots
\end{equation}
from the usual rules of addition of angular momenta. Then the propagator
reads:
\begin{equation}
\langle\phi^{\alpha_{1}\cdots\alpha_{2J}}\phi^{\alpha_{1}'\cdots\alpha_{2J}'}\rangle(p)
=\frac{p^{\alpha_{1}\alpha_{1}'}\cdots p^{\alpha_{2J}\alpha_{2J}'}}{p^{2}-m^{2}}
+\mbox{permutations} + \mbox{subleading}.
\end{equation}
Here $p^{\alpha\alpha'}:=p^{\mu}\sigma^{\alpha\alpha'}_{\mu}$ gives us
the terms in the numerator. The ``permutations'' term(s) are all the
ways to pair up the indices, and ``subleading'' refers to terms where
the Lorentz indices are provided by $\delta^{\alpha\alpha'}$ instead of
momenta.
This gives us:
\begin{equation}
\Delta\sim\frac{p^{2J}}{p^{2} - m^{2}} = \frac{s^{J}}{s - m^{2}}.
\end{equation}

\begin{remark}
For $J=0,\frac{1}{2}$ the propagators shrink at large momentum, for
$J=1$ the scattering amplitudes are constant in some directions, and for
$J>1$ they grow.

For $J=\frac{3}{2}$, we have the Rarita--Schwinger field propagator, and
it grows like $\sqrt{s}$ at large energies. This leads to unphysical
growth unless the field is coupled to a conserved current. The conserved
current is precisely the Supersymmetry current (thanks to the
Haag--Lopuszanski--Sohnius theorem). Therefore the number of gravitinos
must be (less than or equal to) the number of supercharges.

For $J=2$, by similar reasoning the field is coupled to a conserved
current, and the only one available is the stress-energy tensor (and
possibly the angular momentum tensor $S_{\mu\nu\sigma}$ which requires
some careful analysis). This gives us the graviton and general relativity.
\end{remark}

\N{References, Recommended Reading}
Thus far, a lot has to be discussed and derived. Folland~\cite{Folland:2008zz}
spends the bulk of his book discussing carefully ``what physicists mean''
when they're setting equations for scattering, LSZ reduction, and so on.
The tradeoff for this precision and care is, well, a lack of scope.

Hatfield~\cite{Hatfield:1992rz} expedites the discussion to several
chapters, showing the calculations which physicists typically abbreviate
and sweep under the rug. If you wish to imagine what a physicist would
say when administered truth syrum, Hatfield is your book.

Ticciati~\cite{Ticciati:1999qp}, like Folland, takes greater care in
discussion of the calculations involved. Unlike Folland, Ticciati
discusses gauge symmetry in greater detail. The first five (or ten)
chapters are dedicated to the framework used to setup Feynman diagrams
and perturbative calculations. The next seven chapters are dedicated to
the Standard Model and quantizing gauge systems. The last four chapters
discuss renormalization.

\section{Symmetries}

\M
Symmetries play an important role in quantum field theory. Arguably,
quantum field theory amounts to studying representations of Lie groups
and Lie algebras.

Broadly speaking, physicists divide symmetries into two classes:
spacetime symmetries and internal symmetries.

\N{Spacetime Symmetries}
We expect observable quantities to be relativistic, i.e., invariant
under Lorentz boosts, rotations, spatial translations, and time
translations [relabeling time, not ``time travel'']. That is to say,
physics should be invariant under the Poincar\'e group.

\M
Physicsts define an ``elementary particle'' to be an irreducible
representation of the Poincar\'e group. So there's some importance here.

\N{Internal Symmetries}
The other symmetries physicists consider: transform fields among
themselves. These come in two flavours: global/rigid symmetries, and
local/gauge symmetries. A rigid symmetry does not depend on spacetime,
for example, if $\varphi(x)$ is a complex scalar field, then
\begin{equation}
\varphi(x)\to\E^{\I\alpha}\varphi(x),\quad\mbox{and}\quad
\varphi^{*}(x)\to\E^{-\I\alpha}\varphi^{*}(x)
\end{equation}
for some fixed constant $\alpha\in\RR$. Since $\alpha$ is a constant,
this is a rigid symmetry.

\M
When we have $\alpha$ be a real-valued function, then we have a gauge
symmetry. This usually involved modifying the notion of a derivative to
use a gauge covariant derivative, to kill off derivatives of $\alpha$
which occur in the kinetic term when substituting the transformed fields
back into the Lagrangian.

These $\alpha$ are then generalized to coefficients to Lie algebra
generators, particles are identified with weight vectors, quantum
numbers are identified with weights, and so on. Slansky~\cite{Slansky:1981yr}
is a fun read, as well as Baez and Huerta~\cite{Baez:2009dj}.

\M
Mathematicians curious about dabbling in quantum field theory are
usually interested in the internal symmetries of fields. 

\textsc{Caution:} Physicists are extraordinarily sloppy in their
language and confuse ``Lie groups'' with ``Lie algebras''. Almost none
of them know what they're talking about, so be careful.

\N{``Emergent'' symmetries}
At low energies, a quantum field is governed by the values of a
relatively small number of ``relevant'' or ``marginal'' couplings. These
correspond to relevant and marginal operators that (typically) involve
only a small number of powers of the fields (or their derivatives). It's
often the case these few relevant and marginal operators are invariant
under a wider range of field transformations (rather than the generic,
irrelevant operator would be). The effects of irrelevant operators are
strongly suppressed at low energies, making it appear as though the
theory has a larger symmetry group.\footnote{I learned this from David
Skinner's lecture notes on advanced quantum field theory; see, e.g.,
\url{http://www.damtp.cam.ac.uk/user/dbs26/AQFT/chap6.pdf}}

This then connects a notion of ``emergent symmetry'' to the
renormalization group flow and in particular ``low energy effective
field theory''. I have not seen this discussed in many places, but it's
worth noting.

\begin{theorem}[Wigner]
Let $H$ be a Hilbert space describing our given quantum system.
We can describe any state $\Psi\in H$ modulo multiplying by some nonzero
complex number $\lambda\in\CC\setminus\{0\}$ --- that is, we identify
$\Psi\sim\lambda\Psi$. So really we work with the projective Hilbert
space $\PP H=H/\sim$.

Any symmetry of our quantum system is represented by a symmetry
transformation
\begin{equation}
T\colon\PP H\to\PP H,
\end{equation}
which corresponds to a unitary or anti-unitary transformation $U\colon H\to H$
compatible with $T$.
\end{theorem}

\begin{remark}
The only places I have seen this discussed in adequate detail is 
Bargmann~\cite{Bargmann:1964zj}, Weinberg~\cite{Weinberg:1995mt},
Freed~\cite{Freed:2011aa}.
\end{remark}

\section{Particles and Fields}

\N{Deriving the Scalar Field}
We derive the scalar field by considering point masses (of identical
mass $m$) connected by an array of identical springs in each
dimension. When we take the ``spacing goes to zero'' limit, we obtain a
continuum expression which corresponds to the Lagrangian density for the
scalar field. This is the intuition for what a field looks like. When we
``quantize'' the field, we use the quantum harmonic oscillator, and
obtain the Klein--Gordon [free scalar] field.

This is cute, but usually particles in quantum field theory can be
neatly derived from studying irreducible representations of the Poincar\'{e} Group.

\N{Lorentz Group and Algebra}\marginpar{In $-+++$ signature}
Consider proper orthochronous Lorentz transformations
$\Lambda\in\ISO(3, 1)\subset\O(3, 1)$ such that $\det(\Lambda)=+1$ and
${\Lambda^{0}}_{0}=+1$. Then we can write any element of this group as
\begin{equation}
{\Lambda^{\mu}}_{\nu} = [\exp\left(\frac{-\I}{2}\omega_{\kappa\lambda}M^{\kappa\lambda}\right)]{{}^{\mu}}_{\nu}
\end{equation}
where $\omega_{\kappa\lambda}=-\omega_{\lambda\kappa}$ are ``rotation
angles'' (real constants parametrizing the symmetry) and
$M^{\kappa\lambda}$ is an indexed family of matrices (i.e., fix a value
of $\kappa$ and $\lambda$, and you get a $4\times4$ matrix). These $M^{\kappa\lambda}$ are
generators of the Lie algebra for the Lorentz group. Explicitly
\begin{equation}
(M^{\kappa\lambda})_{\mu\nu} = \I(\delta^{\kappa}_{\mu}\delta^{\lambda}_{\nu}-\delta^{\kappa}_{\nu}\delta^{\lambda}_{\mu})
\end{equation}
Now the trick is that we can write the generators of the Lorentz Lie
algebra using
\begin{subequations}
\begin{align}
L^{i} &= \frac{1}{2}\epsilon^{ijk}M_{jk}\\
\intertext{for spatial rotations, and}
K^{i} &= M^{0i}\\
\intertext{for Lorentz boosts. We define}
\vec{J}_{\pm} &= \frac{1}{2}(\vec{L}\pm\I\vec{K}).
\end{align}
\end{subequations}
The reader may verify the commutation relations become
\begin{equation}
[J^{i}_{\pm}, J^{j}_{\pm}] = \I\epsilon^{ijk}J^{k}_{\pm}.
\end{equation}
But now look, this is precisely two copies of $\su(2)$ (more precisely,
it is $\sl(2,\CC)$).

The punchline, however, is: \textit{Each irreducible representation of $\so(3,1)$
is characterized by a pair of half-integers $(j_{+}, j_{-})$.} We can
interpret these irreducible representations as particles, summarized by
the handy-dandy table:

\begin{center}
\begin{tabular}{c|c|c}
  $(j_{+}, j_{-})$ & Name of Field & Dimension of Rep \\\hline
  $(0, 0)$ &	Scalar  &	1\\
$(1/2, 0)$ & 	Left-handed Weyl Spinor &	2\\
$(0, 1/2)$ &	Right-handed Weyl Spinor &	2\\
$(1, 0)$ &	(Imaginary) Self-dual 2-form &	3\\
$(0, 1)$ &	(Imaginary) Anti-self-dual 2-form &	3\\
$(1/2, 1/2)$ &	Vector (gauge field) &	4\\
$(1/2, 1)$ & 	Left-Handed Rarita-Schwinger field &	6\\
$(1, 1/2)$ &	Right-Handed Rarita-Schwinger field &	6\\
$(1, 1)$ &	Graviton (spin-2 field) &	9
\end{tabular}
\end{center}

\M
We study the scalar, the Dirac spinor $(1/2, 0)\oplus(0, 1/2)$, and
Vector fields specifically, since these are the necessary ingredients
for the Standard Model (and they are renormalizable fields).

\section{Wightman Axioms}

\M
We can formalize [canonical] quantum field theory using about a half
dozen axioms. That is to say, we can formalize \emph{one particular}
picture of quantum field theory using a handful of axioms.
These are the Wightman axioms, which is studied in great detail in
Streater and Wightman's book~\cite{Streater:1989vi}.

\begin{remark}
Peter Lowdon's slides ``A (brief) Introduction to Non-perturbative
Quantum Field Theory'' (6th IDPASC/LIP PhD Students Workshop) summarizes
the axioms as follows. Victor Kac's \textit{Vertex Algebras for Beginners}
has similar axioms, with some supermathematical generalizations to allow
easier discussion of fermions. \textbf{But I think} some signs may be
messed up, because I'm using the opposite metric-signature convention
than these authors.
\end{remark}

\begin{axiom}[Hilbert space structure]
The states of the theory are rays in a Hilbert space $\mathcal{H}$ which
possesses a continuous unitary representation $U(a,\alpha)$ of the
Poincar\'e spinor group $\overline{\mathscr{P}^{\uparrow}_{+}}$.
\end{axiom}

\begin{axiom}[Spectral condition]
The spectrum of the energy-momentum operator $P^{\mu}$ is confined to
the closed forward lightcone (\S\ref{defn:relativity:light-cone})
$\overline{V}^{+}=\{p^{\mu}|p^{2}\leq0,p^{0}\geq0\}$
where $U(a,1)=\exp(\I P^{\mu}a_{\mu})$.
\end{axiom}

\begin{remark}
This means the energy is bounded from below. Morally, this means ``The
theory is stable''.
\end{remark}

\begin{axiom}[Uniqueness of the vacuum]
There exists a unit state vector $|0\rangle$ called the vacuum state
which is a unique translationally-invariant state in $\mathcal{H}$.
\end{axiom}

Morally: ``The vacuum is unique and looks the same to all observers''.

\begin{axiom}[Field operators]
The theory consists of fields $\varphi^{(\kappa)}(x)$ (of type $\kappa$)
which have components $\varphi^{(\kappa)}_{\ell}(x)$ that are
operator-valued tempered distributions in $\mathcal{H}$, and the vacuum
state $|0\rangle$ is a cyclic vector for the fields.
\end{axiom}

Morally: ``Quantum fields $\varphi$ are operator-valued distributions''.
This means in particular:
\begin{enumerate}
\item Quantum fields cannot be evaluated at a single point (e.g., the
  Diract delta function $\delta(x)$ at $x=0$);
\item We need to ``average them out'' over some region of spacetime $\Sigma$.
\end{enumerate}
The physical justification for this is the Heisenberg uncertainty principle.

\begin{axiom}[Relativistic Covariance]
The fields $\varphi^{(\kappa)}_{\ell}(x)$ transform covariantly under
the action of $\overline{\mathscr{P}^{\uparrow}_{+}}$:
\begin{equation}
U(a,\alpha)\varphi^{(\kappa)}_{\ell}(x)U(a,\alpha)^{-1} = S_{ij}^{(\kappa)}(\alpha^{-1})\varphi_{j}^{(\kappa)}(\Lambda(\alpha)x+a),
\end{equation}
where $S(\alpha)$ is a finite-dimensional matrix representation of the
Lorentz spinor group $\overline{\mathscr{L}^{\uparrow}_{+}}$, and
$\Lambda(\alpha)$ is the Lorentz transformation corresponding to
$\alpha\in\overline{\mathscr{L}^{\uparrow}_{+}}$
\end{axiom}

\begin{axiom}[Microcausality]
If the support of the test functions $f$ and $g$ of the fields
$\varphi^{(\kappa)}_{\ell}$ and $\varphi^{(\kappa')}_{m}$ (respectively)
are space-like separated, then its graded commutator:
\begin{equation}
[\varphi^{(\kappa)}_{\ell}(f), \varphi^{(\kappa')}_{m}(g)]_{\pm} = 0
\end{equation}
when applied to any state in $\mathcal{H}$ for any fields
$\varphi^{(\kappa)}_{\ell}$ and $\varphi^{(\kappa')}_{m}$.
\end{axiom}

\N{Consequences}
We have a number of consequence from these Wightman axioms. Roughly, the
outline for results for scalar fields (and other simple fields) are as
follows:
\begin{enumerate}
\item The correlation functions $\langle0|\varphi^{(\kappa_{1})}_{\ell_{1}}(x_{1}) \cdots\varphi^{(\kappa_{n})}_{\ell_{n}}(x_{n})|0\rangle$
are distributions.
\item A quantum field theory can be fully reconstructed from knowledge
  of all of the correlation functions.
\item We can connect Minkowski and Euclidean field theories (taking the
  Wick rotation $t\mapsto\I\tau$)
\item Spin-statistics theorem: bosonic fields have commutators,
  fermionic fields have anticommutators.
\item CPT is a symmetry of any theory, even though how the operators
  ($C$, $P$, $T$) are implemented differently depending on the
  conventions adopted.
\end{enumerate}

\N{Gauss Law Constraint}\label{chunk:outline:pseudo-wightman-axioms}
For theories of physical interest (that is, gauge theories) significant
complications arise with the Wightman axioms. For QCD, these
complications underpin the \emph{nonperturbative} structure of
correlation functions.

Specifically, we have a ``Local Gauss Law'' for Yang--Mills theory
\begin{equation}
\partial^{\nu}F^{I}_{\mu\nu} = J^{I}_{\mu}.
\end{equation}
We have two ways to impose this constraint, which faces the following
trade-off:
\begin{enumerate}
\item Preserve positivity and lose locality (see: Coulomb gauge in QED); or
\item Preserve locality and lose positivity (see: Landau gauge in QCD).
\end{enumerate}
The second option, to recover locality, we modify the constraint
equation to be imposed for physical states:
\begin{equation}
\langle\mbox{phys}|\partial^{\nu}F^{I}_{\mu\nu} - J^{I}_{\mu}|\mbox{phys}\rangle=0,
\end{equation}
which either gives us:
\begin{enumerate}
\item The Gupta--Bleuler condition $\partial^{\mu}A_{\mu}^{(+)}|\mbox{phys}\rangle=0$,
or
\item BRST condition $Q_{B}|\mbox{phys}\rangle=0$.
\end{enumerate}
This necessarily introduces both \emph{zero-norm} and
\emph{negative-norm states}, which as discussed in
Section~\ref{sec:qm:basic-rules} corresponds to states with zero and
negative probabilities. We need to modify the axioms, resulting in
what some have called the ``Pseudo-Wightman approach''. For more about
this, see Bogolubov and friends~\cite{Bogolyubov:1990kw}.

\section{Yang--Mills Theory}

\N{Global and Local Symmetries}
Textbooks usually begin by studying ``global symmetries'', which do not
depend on spacetime coordinates. For example, if we have $N$ real scalar
fields, then we may put them into a column vector, and rotate by some
orthogonal $N\times N$ matrix. This works because the kinetic and
potential terms of the Lagrangian involve the norm squared of these
$N$-vectors, which are invariant under such rotations.

Physicists then ``gauge'' these symmetries and make them ``local''. But
then the kinetic terms will end up with derivatives of the rotation
matrix. These are then ``gauged away'' by changing the differential
operator.

\N{Yang--Mills Theory}
Another way to approach this is to start with electromagnetism, which
involves the electromagnetic 4-potential $A_{\mu}$. Then we consider
some Lie algebra $\mathfrak{g}$ and work with Lie algebra-valued
4-potentials $A_{\mu}^{I}T_{I}$ where $T_{I}$ are the generators of the
Lie algebra. We compute the field tensor:
\begin{equation}
F_{\mu\nu}^{I} = \partial_{\mu}A^{I}_{\nu} - \partial_{\nu}A^{I}_{\mu}
+g{f^{I}}_{JK}A^{J}_{\mu}A^{K}_{\nu}
\end{equation}
where we use the Lie bracket to determine the structure constants
${f^{I}}_{JK}$ by:
\begin{equation}
[T_{J}, T_{K}] = \I {f^{I}}_{JK}T_{I}.
\end{equation}
We usually work with $\su(n)$ as our Lie algebra, since $\su(3)$
describes the strong force, and $\su(2)\oplus\mathfrak{u}(1)$ describes
the electroweak forces.

\textsc{Cautionary Note}: a lot of books get confused over indices of
the Lie algebra, and use a bizarre Euclidean summation convention for
Lie algebra indices (but Einstein summation convention for spacetime
indices). Weinberg carefully works through the correct summation
conventions in his book \textit{The Quantum Theory of Fields}~\cite[\S15.1]{Weinberg:1996kr}
(see also~\cite[\S2.2]{Weinberg:1995mt} for discussion of symmetries and
conventions).
%(volume II, \S15.1; see also volume I, \S2.2).

\N{Equations of Motion}
Recall Maxwell's equations,
\begin{subequations}
\begin{align}
\partial^{i}E_{i} &= \rho\\
\partial^{i}B_{i} &= 0\\
-\partial_{t}E^{i} &= j^{i} -\epsilon^{ijk}\partial_{j}B_{k}\\
\partial_{t}B^{i} &= -\epsilon^{ijk}\partial_{j}E_{k}.
\end{align}
\end{subequations}
Yang--Mills equations of motion resemble this:
\begin{subequations}
\begin{align}
\partial^{i}E_{i} + [A^{i},E_{i}] &= \rho\\
\partial^{i}B_{i} + [A^{i},B_{i}] &= 0\\
-\partial_{t}E^{i} &= j^{i} -\epsilon^{ijk}(\partial_{j}B_{k} + [B_{j},E_{k}])\\
\partial_{t}B^{i} &= -\epsilon^{ijk}(\partial_{j}E_{k} + [A_{j},E_{k}]).
\end{align}
\end{subequations}
where we have the Yang--Mills ``electric'' and ``magnetic'' fields be
defined by (in the temporal gauge $A_{0}=0$):
\begin{equation}
E_{i} = -\partial_{t}A_{i}^{I}T_{I},\quad\mbox{and}\quad
B^{i} = \epsilon^{ijk}(\partial_{j}A_{k}^{I}T_{I} - \partial_{k}A_{j}^{I}T_{I}
+ [A_{j}^{J}T_{J}, A_{k}^{K}T_{K}]).
\end{equation}

\begin{remark}
See also Sanchez-Monroy and Quimbay~\cite{Sanchez-Monroy:2006sie} for
working out the Yang--Mills equations of motion for $\SU(3)$.
\end{remark}

\M
Geometrically, what's happening is we're working with connections on the
adjoint bundle over spacetime, and physicists call the connections
``gauge fields'' (the associated curvature is the ``field strength tensor'').
When $G$ is our gauge group, we have the principal $G$-bundle $P\to M$
over spacetime $M$, and the associated adjoint bundle $\mathrm{Ad}(P)\to M$
with its fibre being isomorphic to the vector space underlying
$\mathfrak{g}=\mathrm{Lie}(G)$ the Lie algebra of $G$.

There are as many gauge bosons as there are basis vectors for $\mathfrak{g}=\mathrm{Lie}(G)$.

For more on the mathematics related to the geometry needed for
Yang--Mills theory, see Hamilton~\cite{Hamilton:2017gbn}.

\N{``Charge''}
We can talk meaningfully about the analogous quantities to ``electric
charge'' in Yang--Mills theory. When the gauge group $G$ is non-Abelian
and $\mathrm{Lie}(G)$ is semisimple,
the charge is quantized (i.e., not continuous).\footnote{Weinberg
mentions this in passing in his book on quantum field theory. See, e.g.,~\cite[esp.~\S3.3,\S23.3]{Weinberg:1995mt}.}

Physicists call the ``charge'' different terms for different gauge
groups. Within QCD, they are called ``color''; for the weak force, they
are ``flavors''.

\N{Problems with Massive Yang--Mills}
If we try to add a nonzero mass to a non-Abelian Yang--Mills theory,
then we sacrifice either renormalizability or unitarity; Delbourgo and
friends argued this first~\cite{Delbourgo:1987np}.
Ellwanger and Wschebor~\cite{Ellwanger:2002sj} constructed a small
counter-example in $\su(2)$ by modifying BRST variations, working in a
particular gauge.

If we give up unitarity, we basically give up probabilities adding up to
$100\%$. On the other hand, nonrenormalizable fields have ``runaway
self-interactions'' which lead to infinities.

It's also worth mentioning that the mass term is not gauge-invariant,
which causes its own special Hell.

\N{BRST Symmetry}
When quantizing the Standard Model (really, Yang--Mills theory) we run
into problems: intuitively, gauge orbits describe the same physical
state, and we want to work with the quotient of the phase space modulo
gauge symmetries. The Faddeev--Poppov method implements this at the
quantum level, but this is generalized by using a BRST symmetry instead
of the usual gauge symmetry. This trick extends the phase space, embedding it as
the even part of a supermanifold, then we replace the gauge symmetry
with a rigid fermionic symmetry. This is the heart of BRST construction
and handles technical difficulties when quantizing non-abelian Yang--Mills.
This is how Henneaux and Teitelboim~\cite[\S8.1]{Henneaux:1992ig}
describe the BRST procedure.

Further, BRST quantization handles the negative norm and zero norm
states (\S\ref{chunk:outline:pseudo-wightman-axioms}).

\section{Standard Model}

\M
Despite its name, the ``Standard Model'' is neither standard nor a
model. We can decompose it along pseudo-historical lines as consisting
of electroweak force (with gauge group $\SU(2)\times\U(1)$)
and strong force (with gauge group $\SU(3)$). Then the Standard Model
is just a $\SU(3)\times\SU(2)\times\U(1)$ Yang--Mills coupled
to Dirac spinors and a Higgs boson.

\subsection{Strong Force}


\M
Perturbative QCD studies high energy interactions, because the coupling
$\alpha_{s}$ is quite small (which allows for perturbative calculations).
This is what would be found in, e.g., Peskin and Schroeder.

Then there is low energy QCD, which is studied using nonperturbative
methods (like lattice field theory).

The difficulty is in connecting these two domains together.

\N{Confinement} For quarks in the strong force, they experience a
phenomenon called ``confinement'' which we define as stating \emph{there
is a force between quarks which do not decrease with distance.} See
Esprieu~\cite{Espriu:1994br}, especially \S7. As a consequence, we will
not be able to find an isolated quark.

\N{Open problem: analytical description of confinement}
We lack an adequate analytical definition or criteria for ``confinement''.
This is a major open problem. One of the earliest criteria was Wilson~\cite{Wilson:1974sk}
and his so-called ``area law''\index{Area Law} for the Wilson loop along a
rectangle with edges $R_{1}$ and $R_{2}$, we expect:
\begin{equation}
\ln\langle W(R_{1}, R_{2})\rangle\sim R_{1}R_{2}.
\end{equation}
When this is true, the field theory is believed to enjoy confinement.

\begin{remark}[Folklore]
We have some results, e.g., 't Hooft~\cite{tHooft:1977nqb} proved that
mass gap plus unbroken center symmetry implies confinement.
``Unbroken center symmetry'' is not a rigorously defined notion.
Chatterjee~\cite{Chatterjee:2020nrl} attempted to make this rigorous.

Consequently, it is viewed that if the gauge group has a trivial center,
then confinement is impossible. I'm not so convinced by this, because 't
Hooft gives \emph{one possible route} to confinement, but this is not a
proof that it is \emph{the only possible route}.
\end{remark}

\N{Strong $\mathtt{G}_{2}$ Force}
The idea of using $\mathtt{G}_{2}$ as the gauge group for quarks and the
strong force was first explored in 1962 by Behrends and
friends~\cite{Behrends:1962zz}, then rediscovered [seemingly]
independently and explored in great detail by Holland and
friends~\cite{Holland:2003jy}. It turns out to be very useful as a
``test suite'' for lattice field theory
software~\cite{Ilgenfritz:2012wg,Maas:2012ts,Pepe:2006er,Wellegehausen:2011sc}.
There is some nuance around the deconfinement phase transition in
$\mathtt{G}_{2}$ strong force~\cite{Pepe:2005sz,Pepe:2006er}: when at
high temperatures, quarks become ``deconfined'' (due to asymptotic
freedom) leading to quark--gluon plasma.

For a review of the phenomenology surrounding quark--gluon plasma, see
Pasechnik and \v{S}umbera~\cite{Pasechnik:2016wkt} who note, ``The study
of ultra-relativistic heavy-ion collisions appears so far to be our only
way of studying the phase transitions in non-Abelian gauge theories
(most likely taken place in the early universe) under laboratory
conditions.''

It appears that $\mathtt{G}_{2}$ cannot explain deconfinement phase
transition, which we witness in laboratory experiments. Bruno and
friends~\cite{Bruno:2014rxa} have argued that confinement can be
reproduced with $\mathtt{G}_{2}$ Yang--Mills, pointing out there is no
adequate definition for these notions, and qualitative behaviour can be
captured. (Bruno and friends also thoroughly review the literature of
numerical and analytical studies of $\mathtt{G}_{2}$ Yang--Mills theory.)

\begin{remark}[$\mathtt{G}_{2}$ and Octonions]
For more about $\mathtt{G}_{2}$ as the automorphism group for the Octonions $\mathtt{G}_{2}\iso\aut(\OO)$, see
\S4.1 in Baez~\cite{Baez:2001dm}.
\end{remark}

\subsection{As an Effective Field Theory}

\M
We can treat the Standard Model as a low energy approximation to some
``real'' field. This is handled using the toolkit of ``effective field
theory''.
For a review of this (particularly applied to the Standard Model), see
Brivio and Trott~\cite{Brivio:2017vri}.

\N{TODO List}
Stuff I should write about:
\begin{enumerate}
\item Lattice QCD~\cite{Lepage:1998dt}
\item Elitzur's theorem~\cite{Elitzur:1975im} (``No such thing as spontaneous symmetry breaking of a local gauge symmetry.'')
\item The Fradkin--Shenker--Osterwalder--Seiler (FSOS) Theorem (``There is no transition in coupling-constant space which isolates the Higgs phase from a confinement-like phase.'')
\item Haag's Theorem~\cite{Haag:1992hx}: an interacting theory and a
  free theory are not unitarily equivalent
\item Fr\"{o}lich--Morchio--Strocchi mechanism (\arXiv{2305.01960})
  which resolves the paradox that perturbation theory applied to a free
  theory can approximate results for interacting theories.
\item Kinoshita--Lee--Nauenberg (KLN) theorem (``perturbatively the standard model as a whole is infrared (IR) finite; i.e., the infrared divergences coming from loop integrals are canceled by IR divergences coming from phase space integrals.'')
\item Factorization theorems (for hadron collider processes) \arXiv{hep-ph/0409313}
\item Block--Nordsieck theorem in QED: Infrared (\textbf{soft}) divergences
  cancel out after summation over all degenerate \textbf{final} states
  compatible with experimental detection. Does not work in non-Abelian
  gauge theories (see {\tt\doi{10.1016/0550-3213(81)90554-X}}).
\item Coleman--Norton theorem (from {\tt\doi{10.1007/BF02750472}})
\item Multiplets are irreducible representations (or subrepresentations)
  for a Lie algebra/group.
\item Technicolor (i.e., composite Higgs); for some reviews, see
  \arXiv{hep-ph/9401324} and \arXiv{1104.1255}.
\item Functional derivatives --- physicists use the notation $\delta F[\varphi]/\delta\varphi(\vec{x})$
  for the Gateaux derivative $\lim_{\varepsilon\to 0}(F[\varphi+\varepsilon\delta_{\vec{x}}]-F[\varphi])/\varepsilon$
  using the Dirac delta function $\delta_{\vec{x}}$.
\end{enumerate}

% Zeidler, \textit{Quantum Field Theory III}, \S3.14.3 discusses highest
% weight vectors and elementary particle physics.

% https://www2.pd.infn.it/ecfa/6_p_loopv_d_comelli.PDF

% Greiner's "Quantum Mechanics: Symmetries" appears to be quite good

% https://arxiv.org/abs/2212.08470

% Segal's notes on QFT http://web.archive.org/web/20000901075112/http://www.cgtp.duke.edu/ITP99/segal/

\chapter{Grand Unified Theories}

\M
Graham Ross's \textit{Grand Unified Theories} is one of the only books
on the subject; it was originally written in 1985, and much has changed
since then.

There is also Rabindra Mohapatra's \textit{Unification and Supersymmetry: The Frontiers of Quark-Lepton Physics},
which is more recent.

Baez and Heurta~\cite{Baez:2009dj} review of the algebra behind grand
unified theories.

\appendix
\chapter{Mathematical Results}

\begin{theorem}[Gaussian Integral]
$\displaystyle\int^{\infty}_{-\infty}\E^{-x^{2}}\,\D x = \sqrt{\pi}$.
\end{theorem}

\begin{proof}
  Let
  \begin{equation}
I = \int^{\infty}_{-\infty}\E^{-x^{2}}\,\D x.
  \end{equation}
  Then
\begin{calculation}
  I^{2}
  \step{unfold}
\int^{\infty}_{-\infty} \int^{\infty}_{-\infty}\E^{-x^{2}-y^{2}}\,\D x\,\D y
\step{change to polar coordinates}
\int^{\infty}_{0}\int^{2\pi}_{0}\E^{-r^{2}}r\,\D\theta\,\D r
\step{since integrand is constant with respect to $\theta$}
2\pi\int^{\infty}_{0}\E^{-r^{2}}r\,\D r
\step{change coordinates with $u=r^{2}$, $\D u = 2r\,\D r$}
\pi\int^{r=\infty}_{r=0}\E^{-u}\,\D u
\step{simple integration}
\left.-\pi\E^{-r^{2}}\right|^{r=\infty}_{r=0}
\step{substitution}
-\pi\E^{-\infty}+\pi\E^{-0} = \pi.
\end{calculation}
Therefore $I=\pm\sqrt{\pi}$. Since the integrand is always positive, we
conclude $I=\sqrt{\pi}$ as desired.
\end{proof}

\begin{corollary}\label{cor:math:general-gaussian-integral-in-one-dim}
Let $a,b\in\CC$. Then $\displaystyle\int^{\infty}_{-\infty}\E^{-a(x-b)^{2}}\,\D x = \sqrt{\frac{\pi}{a}}$
\end{corollary}

\begin{thebibliography}{99}
\footnotesize%  \small

\bibitem{adams1996:ex}
J.F.\ Adams,
\textit{Lectures on Exceptional Lie Groups}.
University of Chicago Press, 1996.

%\cite{Baez:2001dm}
\bibitem{Baez:2001dm}
John C.~Baez,
``The Octonions''.
\journal{Bull.Am.Math.Soc.} \textbf{39} (2002) 145--205
[erratum: \journal{Bull.Am.Math.Soc.} \textbf{42} (2005) 213]
{\tt\doi{10.1090/S0273-0979-01-00934-X}}
[\arXiv{math/0105155} [math.RA]].
%396 citations counted in INSPIRE as of 30 May 2023

%\cite{Ekins:1975yu}
\bibitem{Ekins:1975yu}
J.~M.~Ekins and J.~F.~Cornwell,
``Semisimple Real Subalgebras of Noncompact Semisimple Real Lie Algebras. 5.,''
\journal{Rept.Math.Phys.} \textbf{7} (1975) 167--203
{\tt\doi{10.1016/0034-4877(75)90026-9}}
%4 citations counted in INSPIRE as of 02 Jun 2023


%\cite{Figueroa-OFarrill:2007jcv}
\bibitem{Figueroa-OFarrill:2007jcv}
Jos\'e Figueroa-O'Farrill,
``A Geometric construction of the exceptional Lie algebras $F_{4}$ and $E_{8}$''.
\journal{Commun.Math.Phys.} \textbf{283} (2008) 663--674
{\tt\doi{10.1007/s00220-008-0581-7}}
[\arXiv{0706.2829} [math.DG]].
%14 citations counted in INSPIRE as of 02 Jun 2023

%\cite{Garling:2011zz}
\bibitem{Garling:2011zz}
D.~J.~H.~Garling,
\textit{Clifford Algebras: An Introduction}.
Cambridge University Press, 2011.
%4 citations counted in INSPIRE as of 02 Jun 2023

%\cite{Gogberashvili:2019ojg}
\bibitem{Gogberashvili:2019ojg}
Merab Gogberashvili and Alexandre Gurchumelia,
``Geometry of the Non-Compact $G(2)$''.
\journal{J.Geom.Phys.} \textbf{144} (2019) 308--313
{\tt\doi{10.1016/j.geomphys.2019.06.015}}
[\arXiv{1903.04888} [physics.gen-ph]].
%3 citations counted in INSPIRE as of 02 Jun 2023

%\cite{Gunaydin:2001bt}
\bibitem{Gunaydin:2001bt}
M.~Gunaydin, K.~Koepsell and H.~Nicolai,
``The Minimal unitary representation of $\mathtt{E}_{8(8)}$''.
\journal{Adv.Theor.Math.Phys.} \textbf{5} (2002) 923--946
{\tt\doi{10.4310/ATMP.2001.v5.n5.a3}}
[\arXiv{hep-th/0109005} [hep-th]].
%59 citations counted in INSPIRE as of 02 Jun 2023

\bibitem{1212.3182}
Aaron Wangberg, Tevian Dray,
``$E_{6}$, the Group: The structure of $\SL(3,\OO)$''.
\arXiv{1212.3182}

%\cite{Zhang:2011ym}
\bibitem{Zhang:2011ym}
R.B.~Zhang,
``Serre presentations of Lie superalgebras''.
[\arXiv{1101.3114} [math.RT]].
%4 citations counted in INSPIRE as of 30 May 2023

\end{thebibliography}

% Tits construction of the exceptional simple Lie algebras
% https://arxiv.org/abs/0907.3789



\end{document}
