\section{Functional Derivatives}\label{section:math:functional-derivatives}

\M
The intuition physicists use is to work directly analogous to discrete
indices. Just as we have
\begin{subequations}
\begin{equation}
\frac{\partial}{\partial x_{i}}x_{j} = \delta_{ij}
\end{equation}
we expect functional derivatives to obey similarly (for functions of
$n\in\NN$ variables),
\begin{equation}
\frac{\delta}{\delta f(x)}f(y)=\delta^{(n)}(x-y).
\end{equation}
\end{subequations}
Likewise, if we had a dot product
\begin{subequations}
\begin{equation}
x\cdot k = \sum_{j}x_{j}k_{j},
\end{equation}
this would correspond to the integral
\begin{equation}
\int f(x)k(y)\,\D^{n}y.
\end{equation}
Differentiation of the dot product
\begin{equation}
\frac{\partial}{\partial x_{i}}\sum_{j}x_{j}k_{j} = k_{i},
\end{equation}
corresponds to the analogous result,
\begin{equation}
\frac{\delta}{\delta f(x)}\int f(y)g(y)\,\D^{n}y = g(x).
\end{equation}
\end{subequations}
When we try to take the functional derivative with respect to $f(x)$ of
an integral involving the derivative of $f(x)$, we integrate by parts
before doing anything (and discard the boundary terms).

Physicists assume the functional derivative obeys the chain rule as well
as the Leibniz product rule. This turns out to be related to the
Fr\'echet derivative.

\begin{definition}
Let $F[\phi]$ be a functional, which is a mapping from a normed linear
space of functions [Banach space] $M = \{\phi(x)\mid x\in\RR\}$ to the
field of real or complex functions, $F\colon M\to\RR$ or $F\colon M\to\CC$.
Then we define the \define{Fr\'echet derivative} of $F$ in the direction
of $\lambda\in M$ to be the linear mapping $L\colon M\to\FF$ (where
$\FF=\RR$ if $F$ is real-valued, and $\FF=\CC$ if $F$ is complex-valued)
such that
\begin{equation}
F[\phi + \varepsilon\lambda]-F[\phi]=\varepsilon L[\phi,\lambda]
+o(\varepsilon).
\end{equation}
Physicists often write $\delta\phi$ instead of $\lambda$, and $\delta F$
instead of $L$.
\end{definition}

\N{Notation}
We write $\delta_{y}(x)=\delta(y-x)$.

\N{Physicist's Functional Derivative}
If we take $\delta\phi(x)=\varepsilon\delta_{y}(x)$, then
\begin{calculation}
\delta F[\phi]
\step{definition of Fr\'echet derivative}
F[\phi+\varepsilon\delta_{y}] - F[\phi]
\step{in analogy to differential form}
\int\frac{\delta F[\phi]}{\delta\phi(x)}\varepsilon\delta(y-x)\,\D x.
\end{calculation}
In the limit of vanishing $\varepsilon$, we obtain what physicists call
the functional derivative:
\begin{equation}
\boxed{\frac{\delta F[\phi]}{\delta\phi(y)}=\lim_{\varepsilon\to0}\frac{F[\phi+\varepsilon\delta_{y}]-F[\phi]}{\varepsilon}.}
\end{equation}
We call this the \define{Physicist's Functional Derivative} and it is a
generalized function.

%% \N{Caution: Variational Derivative}
%% There is also a different derivative used by physicists called the
%% ``variational derivative''. Confusingly, it uses almost identical
%% syntax:
%% \begin{subequations}
%% \begin{equation}
%% \mbox{Variational Derivative:}\qquad\frac{\variation F[\phi]}{\variation\phi}
%% \end{equation}
%% \begin{equation}
%% \mbox{Functional Derivative:}\qquad\frac{\delta F[\phi]}{\delta\phi(x)}.
%% \end{equation}
%% Notice the different ``denominators'': functional derivatives explicitly
%% ``include'' the point $x$.
%% The two derivatives are related by:
%% \begin{equation}
%% \frac{\delta F[\phi]}{\delta\phi(x)}
%% =\frac{\variation F[\phi]}{\variation\phi}(x).
%% \end{equation}
%% \end{subequations}
%% For this reason, the principle of stationary action may be expressed by
%% a functional derivative of the action.

\begin{example}\label{ex:math:functional-derivative:delta-f-over-delta-f-is-delta-fun}
  Let
  \begin{equation}
F[\phi] = \int \phi(y')\delta(y'-x)\,\D y' = \phi(x).
  \end{equation}
  Then
  \begin{equation}
\frac{\delta\phi(x)}{\delta\phi(y)}=\frac{\delta F[\phi]}{\delta\phi(y)}=\delta(y-x).
  \end{equation}
  The proof is by direct calculation:
\begin{calculation}
\frac{\delta F[\phi]}{\delta\phi(y)}
\step{definition of physicist's functional derivative}
\lim_{\varepsilon\to0}\frac{F[\phi + \varepsilon\delta_{y}]-F[\phi]}{\varepsilon}
\step{unfolding definition of $F[-]$}
\lim_{\varepsilon\to0}\frac{\int\bigl(\phi(y')+\varepsilon\delta(y-y')\bigr)\delta(y'-x)\,\D y' -\int \phi(y')\delta(y'-x)\,\D y'}{\varepsilon}
\step{arithmetic in the numerator}
\lim_{\varepsilon\to0}\frac{\varepsilon\int\delta(y-y')\delta(y'-x)\,\D y'}{\varepsilon}
\step{algebra, taking the limit}
\int\delta(y-y')\delta(y'-x)\,\D y'
\step{defining property of the delta function}
\delta(y-x).
\end{calculation}
\end{example}

\begin{example}
Let $\alpha\in\CC$ be nonzero $\alpha\neq0$.
Then
\begin{equation}
\frac{\delta\phi(x)^{\alpha}}{\delta\phi(y)}=\delta(x-y)\alpha\phi(x)^{\alpha-1}.
\end{equation}
\begin{proof}
  We begin by defining the functional
  \begin{equation}
F[\phi] = \int\phi(y')^{\alpha}\delta(y'-x)\,\D y'.
  \end{equation}
  Then by direct calculation, we find
\begin{calculation}
\frac{\delta F[\phi]}{\delta\phi(y)}
\step{definition of physicist's functional derivative}
\lim_{\varepsilon\to0}\frac{F[\phi + \varepsilon\delta_{y}]-F[\phi]}{\varepsilon}
\step{unfolding definition of $F[-]$}
\lim_{\varepsilon\to0}\frac{\int\bigl(\phi(y')+\varepsilon\delta(y-y')\bigr)^{\alpha}\delta(y'-x)\,\D y' -\int \phi(y')^{\alpha}\delta(y'-x)\,\D y'}{\varepsilon}
\step{Taylor expand to first order in $\varepsilon$ in the first term}
\lim_{\varepsilon\to0}\frac{\int\bigl(\phi(y')^{\alpha}+\alpha\varepsilon\delta(y-y')\phi(y')^{\alpha-1}+\bigOh(\varepsilon^{2})\bigr)\delta(y'-x)\,\D y' -\int \phi(y')^{\alpha}\delta(y'-x)\,\D y'}{\varepsilon}
\step{arithmetic in the numerator, factoring out a power of $\varepsilon$}
\lim_{\varepsilon\to0}\frac{\varepsilon\int\bigl(\delta(y-y')\alpha\phi(y')^{\alpha-1}+\bigOh(\varepsilon)\bigr)\delta(y'-x)\,\D y'}{\varepsilon}
\step{algebra, taking the limit}
\int\delta(y-y')\alpha\phi(y')^{\alpha-1}\delta(y'-x)\,\D y'
\step{defining property of the delta function}
\delta(y-x)\alpha\phi(y)^{\alpha-1}.\qedhere
\end{calculation}
\end{proof}
\end{example}

\begin{theorem}
Let $f\colon\RR\to\RR$ be a sufficiently smooth function, $F[-]$ be a
functional defined by
\begin{subequations}
\begin{equation}
F[\phi] = \int f\bigl(\phi(x)\bigr)\,\D x.
\end{equation}
Then
\begin{equation}
\frac{\delta F[\phi]}{\delta\phi(y)} = f'\bigl(\phi(y)\bigr),
\end{equation}
where $f'$ is the first derivative of $f$.
\end{subequations}
\end{theorem}

\begin{proof}
We see that, Taylor expanding $F[\phi+\varepsilon\delta_{y}]$ to first
order in $\varepsilon$,
\begin{equation}
F[\phi+\varepsilon\delta_{y}]=\int\left(f\bigl(\phi(x)\bigr)+\varepsilon\delta(x-y)f'\bigl(\phi(x)\bigr)+\bigOh(\varepsilon^{2})\right)\D x.
\end{equation}
Then
\begin{equation}
F[\phi+\varepsilon\delta_{y}]-F[\phi]=\int\left(\varepsilon\delta(x-y)f'\bigl(\phi(x)\bigr)+\bigOh(\varepsilon^{2})\right)\D x.
\end{equation}
Dividing both sides by $\varepsilon$ gives us
\begin{equation}
\frac{F[\phi+\varepsilon\delta_{y}]-F[\phi]}{\varepsilon}=\int\left(\delta(x-y)f'\bigl(\phi(x)\bigr)+\bigOh(\varepsilon)\right)\D x.
\end{equation}
Taking the $\varepsilon\to0$ limit gives us the physicist's functional derivative
\begin{equation}
\frac{\delta F[\phi]}{\delta\phi(y)}
=\int\left(\delta(x-y)f'\bigl(\phi(x)\bigr)\right)\D x.
\end{equation}
Then invoking the defining property of the delta function produces the result.
\end{proof}

\begin{corollary}
Suppose $F$ is a functional that eats in $N$ functions $\varphi^{A}$ on
$n$-dimensional space, suppose $f$ is a sufficiently smooth function
(real-valued or complex-valued) such that we have
\begin{equation}
F[\varphi^{A}] = \int f\bigl(\varphi^{A}(x')\bigr)\,\D^{n}x'
 = \int f\bigl(\varphi^{1}(x'), \varphi^{2}(x'),\dots,\varphi^{N}(x')\bigr)\,\D^{n}x'.
\end{equation}
Then
\begin{subequations}
\begin{equation}
\frac{\delta F[\varphi^{A}]}{\delta\varphi^{B}(x)} =
\left.\frac{\partial f(\varphi^{A})}{\partial\varphi^{B}}\right|_{\varphi^{A}=\varphi^{A}(x)},
\end{equation}
and
\begin{equation}
\delta F[\varphi^{A}] = \int\left(\sum_{B}\left.\frac{\partial f(\varphi^{A})}{\partial\varphi^{B}}\right|_{\varphi^{A}=\varphi^{A}(x')}\delta\varphi^{B}(x')\right)\D^{n}x'.
\end{equation}
\end{subequations}
\end{corollary}

\begin{theorem}[Linearity]
Let $F[-]$, $G[-]$ be functionals, $c_{1}$, $c_{2}$ be (real or complex)
constants, then
\begin{equation}
\frac{\delta(c_{1}F[\phi]+c_{2}G[\phi])}{\delta\phi(x)}
=c_{1}\frac{\delta F[\phi]}{\delta\phi(x)}
+c_{2}\frac{\delta G[\phi]}{\delta\phi(x)}.
\end{equation}
\end{theorem}

\begin{theorem}[Leibniz ``product'' rule]
Let $F[\phi]=G[\phi]H[\phi]$ be the product of two functionals.
Then
\begin{equation}
\frac{\delta F[\phi]}{\delta\phi(x)}
=\frac{\delta G[\phi]}{\delta\phi(x)}
H[\phi] + G[\phi]\frac{\delta H[\phi]}{\delta\phi(x)}.
\end{equation}
\end{theorem}

\begin{theorem}
[Function chain rule]
Let $g$ be a sufficiently smooth function, $F[-]$ be a functional.
Then
\begin{equation}
\frac{\delta F\bigl[g(\phi)\bigr]}{\delta\phi(y)}
=\frac{\delta F\bigl[g(\phi)\bigr]}{\delta g\bigl(\phi(y)\bigr)}g'\bigl(\phi(y)\bigr).
\end{equation}
\end{theorem}

\begin{remark}
The chain rule for functional differentiation holds, but it requires
$F\colon U\to Y$ to be differentiable at $\phi\in U$, and $G\colon Y\to W$
is differentiable at $\psi=F[\phi]$, then $G\circ F$ is differentiable
at $\phi$. But since $Y$ is usually taken to be the complex or real
numbers, this amounts to making $G$ ``just'' a function $G[-]=g(-)$, and we are
considering $(G\circ F)[\phi] = g\bigl(F[\phi]\bigr)$. The chain rule
would be
\begin{equation}
\frac{\delta g\bigl(F[\phi]\bigr)}{\delta\phi(x)} = g'\bigl(F[\phi]\bigr)
\frac{\delta F[\phi]}{\delta\phi(x)},
\end{equation}
where $g'$ is the derivative of $g$.
\end{remark}

\N{Functional Taylor Series}
We can perform something analogous to the Taylor series expansion for a
functional using functional derivatives, writing
\begin{equation}
F[\phi + \psi] =\sum^{\infty}_{k=0}\frac{1}{k!}\int\frac{\delta^{k}F[\phi]}{\delta\phi(x_{1})\cdots\delta\phi(x_{k})}\psi(x_{1})\cdots\psi(x_{k})\,\D^{n}x_{1}\cdots\D^{n}x_{k}.
\end{equation}
This will prove important when working with the effective action of a
quantum theory.

\N{References}
Peskin and Schroeder~\cite{Peskin:1995ev}, in chapter 9 section 2, give
the heuristic presented initially as the defining axioms for the
functional derivative. Greiner~\cite{Greiner:1996zu}, section 2.3, discusses
functional derivatives in greater detail but is sloppy with his
discussion of the chain-rule.
