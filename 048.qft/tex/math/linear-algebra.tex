\section{Linear Algebra}

\begin{definition}
Let $\mathcal{H}$ be a Hilbert space and
Let $L\colon\mathcal{H}\to\mathcal{H}$ be a linear operator.
Then the \define{Adjoint} of $L$ is a linear operator denoted
$L^{\dagger}\colon\mathcal{H}\to\mathcal{H}$ defined by, for any
$\vec{u}$, $\vec{v}\in\mathcal{H}$ we have
\begin{equation}
\langle L\vec{u},\vec{v}\rangle = \langle \vec{u},L^{\dagger}\vec{v}\rangle.
\end{equation}
\end{definition}

\begin{definition}
Let $\mathcal{H}$ be a Hilbert space and
Let $L\colon\mathcal{H}\to\mathcal{H}$ be a linear operator.
We call $L$ \define{Self-Adjoint} if
\begin{equation}
L^{\dagger} = L.
\end{equation}
\end{definition}


\begin{theorem}
Let $\mathcal{H}$ be a finite-dimensional Hilbert space and
$L\colon \mathcal{H}\to\mathcal{H}$ be a self-adjoint operator. Then:
\begin{enumerate}
\item The eigenvalues of $L$ are all real;
\item The eigenvectors of $L$ form a complete set;
\item The normalized eigenvectors of $L$ are orthonormal.
\end{enumerate}
\end{theorem}

\begin{proof}[Proof (Eigenvalues of $L$ are real)]
Let $\vec{u}\in\mathcal{H}$ be an eigenvector for $L$ with eigenvalue
$\lambda\in\CC\setminus\{0\}$. Then we have
\begin{equation}
  \begin{array}{ccc}
    \langle L\vec{u},\vec{u}\rangle & = & \langle \vec{u},L\vec{u}\rangle\\
= &  & =\\
\lambda\langle\vec{u},\vec{u}\rangle & & \overline{\lambda}\langle\vec{u},\vec{u}\rangle
  \end{array}
\end{equation}
since $L$ is self-adjoint, $L^{\dagger}=L$ which gives us the top line,
and then linearity in the first slot (and anti-linearity in the second
slot) of the inner product gives us the second line. This implies
\begin{equation}
\lambda=\overline{\lambda},
\end{equation}
hence $\lambda\in\RR$.
\end{proof}

\begin{proof}[Proof (Eigenvectors of $L$ are orthogonal)]
Let $\vec{u}_{1}, \vec{u}_{2}\in\mathcal{H}$ be eigenvectors of $L$ with
distinct eigenvalues $\lambda_{1}$, $\lambda_{2}$ respectively.
Then
\begin{equation}
  \begin{array}{ccc}
    \langle L\vec{u}_{1},\vec{u}_{2}\rangle & = & \langle \vec{u}_{1},L\vec{u}_{2}\rangle\\
= &  & =\\
\lambda_{1}\langle\vec{u}_{1},\vec{u}_{2}\rangle & & \overline{\lambda}_{2}\langle\vec{u}_{1},\vec{u}_{2}\rangle.
  \end{array}
\end{equation}
Since the eigenvalues are real, $\lambda_{2}=\overline{\lambda}_{2}$,
and since $\lambda_{1}\neq\lambda_{2}$, we must have
\begin{equation}
\langle\vec{u}_{1},\vec{u}_{2}\rangle=0.
\end{equation}
Further, if the eigenvectors $\vec{u}_{i}$ are unit vectors, then they
form an orthonormal basis.
\end{proof}

\section{Unitary Operators}

\begin{definition}
Let $\mathcal{H}$ be a Hilbert space and
let $L\colon\mathcal{H}\to\mathcal{H}$ be a linear operator.
We call $L$ \define{Unitary} if
\begin{enumerate}
\item it is an isometry: $L^{\dagger}L=\id$
\item it is a co-isometry: $LL^{\dagger}=\id$
\end{enumerate}
\end{definition}

\begin{theorem}
Let $\mathcal{H}$ be a Hilbert space and
let $U\colon\mathcal{H}\to\mathcal{H}$ be a unitary operator.
Then $U$ is a bijection and $U^{-1}=U^{\dagger}$.
\end{theorem}