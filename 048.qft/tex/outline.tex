\chapter{Outline of Quantum Field Theory}

\M
The big idea is that we want to describe scattering using quantum
calculations, involving different types of particles and different types
of interactions [fields], using different ``mathematical toolkits'' and
pictures [sum over histories, functional Schr\"{o}dinger, etc.].
Consequently any thorough textbook would walk through 9 calculations:
for each of the three toolkits [Heisenberg/interaction, path integral,
functional Schr\"{o}dinger], we compute the propagator and scattering
of the three particles [scalar, ``spinor'' (spin-$1/2$), and vector/gauge].


\section{Scattering}

\N{Scattering}
The basic idea is we have $2\to2$ scattering\footnote{Decay may be
interpreted as $1\to n$, and other situations may be described
analogously.}  (i.e., we collide two particles towards each other, and
then two particles emerge from the ``collision'') and we try to measure
something in the laboratory. Usually this is the scattering angle
$\theta$, which is related to the cross-sectional area $\sigma$ by some
equation of the form
\begin{equation}
\frac{\D\sigma}{\D\cos\theta}\sim f(\cos\theta).
\end{equation}
Quantum theory permits us to write an equation relating $\D\sigma$ to
entries of the $S$-matrix. The problem for the theorist is to compute
these $S$-matrix components.

\M
The LSZ formula relates $S$-matrix theory to quantum field theory.
Specifically, the components of the $S$-matrix
\begin{equation}
S_{f,i} = \langle f|S|i\rangle
\end{equation}
may be related to the asymptotic free field via the LSZ formula.

\N{Adiabatic ``Theorem''}
The vacuum state used in the LSZ formula is $|\Omega\rangle$, the vacuum
state for the interacting theory --- compared to the $|0\rangle$ vacuum
state for the free theory. 

The assumption is that $|\Omega\rangle$ is a perturbation of
$|0\rangle$, so they can be related. This is carried out in Peskin and
Schroeder (chapter 4, section 2; see esp.\ Eq~(4.27)).

\N{Feynman Diagrams}
We expand the LSZ formula perturbatively, and organize the terms using
Feynman diagrams describing different interactions.

% https://physics.stackexchange.com/a/510046
% https://physics.stackexchange.com/a/15164
\N{Propagator}
For a particle with spin-$J$ and mass $m$, the propagator schematically
look like\footnote{The denominator depends on the channel the scattering
takes place, I think it could be $t-m^{2}$ in the $t$-channel.}
\begin{equation}
A_{J}(s,t) \sim \frac{s^{J}}{s-m^{2}}.
\end{equation}
This can be derived from representation theory, since any spin-$J$ field
may be formed from tensoring $2J$ copies of a spinor
\begin{equation}
\phi^{\alpha_{1}\cdots\alpha_{2J}}\sim\psi^{\alpha_{1}}\otimes\cdots\otimes\psi^{\alpha_{2J}}
\end{equation}
in the sense that the representation $J$ is a subspace of the
representation $(1/2)^{\otimes 2J}$:
\begin{equation}
\bigotimes^{2j} \frac12=j\oplus (j-1)\oplus\cdots
\end{equation}
from the usual rules of addition of angular momenta. Then the propagator
reads:
\begin{equation}
\langle\phi^{\alpha_{1}\cdots\alpha_{2J}}\phi^{\alpha_{1}'\cdots\alpha_{2J}'}\rangle(p)
=\frac{p^{\alpha_{1}\alpha_{1}'}\cdots p^{\alpha_{2J}\alpha_{2J}'}}{p^{2}-m^{2}}
+\mbox{permutations} + \mbox{subleading}.
\end{equation}
Here $p^{\alpha\alpha'}:=p^{\mu}\sigma^{\alpha\alpha'}_{\mu}$ gives us
the terms in the numerator. The ``permutations'' term(s) are all the
ways to pair up the indices, and ``subleading'' refers to terms where
the Lorentz indices are provided by $\delta^{\alpha\alpha'}$ instead of
momenta.
This gives us:
\begin{equation}
\Delta\sim\frac{p^{2J}}{p^{2} - m^{2}} = \frac{s^{J}}{s - m^{2}}.
\end{equation}

\begin{remark}
For $J=0,\frac{1}{2}$ the propagators shrink at large momentum, for
$J=1$ the scattering amplitudes are constant in some directions, and for
$J>1$ they grow.

For $J=\frac{3}{2}$, we have the Rarita--Schwinger field propagator, and
it grows like $\sqrt{s}$ at large energies. This leads to unphysical
growth unless the field is coupled to a conserved current. The conserved
current is precisely the Supersymmetry current (thanks to the
Haag--Lopuszanski--Sohnius theorem). Therefore the number of gravitinos
must be (less than or equal to) the number of supercharges.

For $J=2$, by similar reasoning the field is coupled to a conserved
current, and the only one available is the stress-energy tensor (and
possibly the angular momentum tensor $S_{\mu\nu\sigma}$ which requires
some careful analysis). This gives us the graviton and general relativity.
\end{remark}

\N{References, Recommended Reading}
Thus far, a lot has to be discussed and derived. Folland~\cite{Folland:2008zz}
spends the bulk of his book discussing carefully ``what physicists mean''
when they're setting equations for scattering, LSZ reduction, and so on.
The tradeoff for this precision and care is, well, a lack of scope.

Hatfield~\cite{Hatfield:1992rz} expedites the discussion to several
chapters, showing the calculations which physicists typically abbreviate
and sweep under the rug. If you wish to imagine what a physicist would
say when administered truth syrum, Hatfield is your book.

Ticciati~\cite{Ticciati:1999qp}, like Folland, takes greater care in
discussion of the calculations involved. Unlike Folland, Ticciati
discusses gauge symmetry in greater detail. The first five (or ten)
chapters are dedicated to the framework used to setup Feynman diagrams
and perturbative calculations. The next seven chapters are dedicated to
the Standard Model and quantizing gauge systems. The last four chapters
discuss renormalization.

\section{Symmetries}

\M
Symmetries play an important role in quantum field theory. Arguably,
quantum field theory amounts to studying representations of Lie groups
and Lie algebras.

Broadly speaking, physicists divide symmetries into two classes:
spacetime symmetries and internal symmetries.

\N{Spacetime Symmetries}
We expect observable quantities to be relativistic, i.e., invariant
under Lorentz boosts, rotations, spatial translations, and time
translations [relabeling time, not ``time travel'']. That is to say,
physics should be invariant under the Poincar\'e group.

\M
Physicsts define an ``elementary particle'' to be an irreducible
representation of the Poincar\'e group. So there's some importance here.

\N{Internal Symmetries}
The other symmetries physicists consider: transform fields among
themselves. These come in two flavours: global/rigid symmetries, and
local/gauge symmetries. A rigid symmetry does not depend on spacetime,
for example, if $\varphi(x)$ is a complex scalar field, then
\begin{equation}
\varphi(x)\to\E^{\I\alpha}\varphi(x),\quad\mbox{and}\quad
\varphi^{*}(x)\to\E^{-\I\alpha}\varphi^{*}(x)
\end{equation}
for some fixed constant $\alpha\in\RR$. Since $\alpha$ is a constant,
this is a rigid symmetry.

\M
When we have $\alpha$ be a real-valued function, then we have a gauge
symmetry. This usually involved modifying the notion of a derivative to
use a gauge covariant derivative, to kill off derivatives of $\alpha$
which occur in the kinetic term when substituting the transformed fields
back into the Lagrangian.

These $\alpha$ are then generalized to coefficients to Lie algebra
generators, particles are identified with weight vectors, quantum
numbers are identified with weights, and so on. Slansky~\cite{Slansky:1981yr}
is a fun read, as well as Baez and Huerta~\cite{Baez:2009dj}.

\M
Mathematicians curious about dabbling in quantum field theory are
usually interested in the internal symmetries of fields. 

\textsc{Caution:} Physicists are extraordinarily sloppy in their
language and confuse ``Lie groups'' with ``Lie algebras''. Almost none
of them know what they're talking about, so be careful.

\N{``Emergent'' symmetries}
At low energies, a quantum field is governed by the values of a
relatively small number of ``relevant'' or ``marginal'' couplings. These
correspond to relevant and marginal operators that (typically) involve
only a small number of powers of the fields (or their derivatives). It's
often the case these few relevant and marginal operators are invariant
under a wider range of field transformations (rather than the generic,
irrelevant operator would be). The effects of irrelevant operators are
strongly suppressed at low energies, making it appear as though the
theory has a larger symmetry group.\footnote{I learned this from David
Skinner's lecture notes on advanced quantum field theory; see, e.g.,
\url{http://www.damtp.cam.ac.uk/user/dbs26/AQFT/chap6.pdf}}

This then connects a notion of ``emergent symmetry'' to the
renormalization group flow and in particular ``low energy effective
field theory''. I have not seen this discussed in many places, but it's
worth noting.

\begin{theorem}[Wigner]
Let $H$ be a Hilbert space describing our given quantum system.
We can describe any state $\Psi\in H$ modulo multiplying by some nonzero
complex number $\lambda\in\CC\setminus\{0\}$ --- that is, we identify
$\Psi\sim\lambda\Psi$. So really we work with the projective Hilbert
space $\PP H=H/\sim$.

Any symmetry of our quantum system is represented by a symmetry
transformation
\begin{equation}
T\colon\PP H\to\PP H,
\end{equation}
which corresponds to a unitary or anti-unitary transformation $U\colon H\to H$
compatible with $T$.
\end{theorem}

\begin{remark}
The only places I have seen this discussed in adequate detail is 
Bargmann~\cite{Bargmann:1964zj}, Weinberg~\cite{Weinberg:1995mt},
Freed~\cite{Freed:2011aa}.
\end{remark}

\section{Particles and Fields}

\N{Deriving the Scalar Field}
We derive the scalar field by considering point masses (of identical
mass $m$) connected by an array of identical springs in each
dimension. When we take the ``spacing goes to zero'' limit, we obtain a
continuum expression which corresponds to the Lagrangian density for the
scalar field. This is the intuition for what a field looks like. When we
``quantize'' the field, we use the quantum harmonic oscillator, and
obtain the Klein--Gordon [free scalar] field.

This is cute, but usually particles in quantum field theory can be
neatly derived from studying irreducible representations of the Poincar\'{e} Group.

\N{Lorentz Group and Algebra}\marginpar{In $-+++$ signature}
Consider proper orthochronous Lorentz transformations
$\Lambda\in\ISO(3, 1)\subset\O(3, 1)$ such that $\det(\Lambda)=+1$ and
${\Lambda^{0}}_{0}=+1$. Then we can write any element of this group as
\begin{equation}
{\Lambda^{\mu}}_{\nu} = [\exp\left(\frac{-\I}{2}\omega_{\kappa\lambda}M^{\kappa\lambda}\right)]{{}^{\mu}}_{\nu}
\end{equation}
where $\omega_{\kappa\lambda}=-\omega_{\lambda\kappa}$ are ``rotation
angles'' (real constants parametrizing the symmetry) and
$M^{\kappa\lambda}$ is an indexed family of matrices (i.e., fix a value
of $\kappa$ and $\lambda$, and you get a $4\times4$ matrix). These $M^{\kappa\lambda}$ are
generators of the Lie algebra for the Lorentz group. Explicitly
\begin{equation}
(M^{\kappa\lambda})_{\mu\nu} = \I(\delta^{\kappa}_{\mu}\delta^{\lambda}_{\nu}-\delta^{\kappa}_{\nu}\delta^{\lambda}_{\mu})
\end{equation}
Now the trick is that we can write the generators of the Lorentz Lie
algebra using
\begin{subequations}
\begin{align}
L^{i} &= \frac{1}{2}\epsilon^{ijk}M_{jk}\\
\intertext{for spatial rotations, and}
K^{i} &= M^{0i}\\
\intertext{for Lorentz boosts. We define}
\vec{J}_{\pm} &= \frac{1}{2}(\vec{L}\pm\I\vec{K}).
\end{align}
\end{subequations}
The reader may verify the commutation relations become
\begin{equation}
[J^{i}_{\pm}, J^{j}_{\pm}] = \I\epsilon^{ijk}J^{k}_{\pm}.
\end{equation}
But now look, this is precisely two copies of $\su(2)$ (more precisely,
it is $\sl(2,\CC)$).

The punchline, however, is: \textit{Each irreducible representation of $\so(3,1)$
is characterized by a pair of half-integers $(j_{+}, j_{-})$.} We can
interpret these irreducible representations as particles, summarized by
the handy-dandy table:

\begin{center}
\begin{tabular}{c|c|c}
  $(j_{+}, j_{-})$ & Name of Field & Dimension of Rep \\\hline
  $(0, 0)$ &	Scalar  &	1\\
$(1/2, 0)$ & 	Left-handed Weyl Spinor &	2\\
$(0, 1/2)$ &	Right-handed Weyl Spinor &	2\\
$(1, 0)$ &	(Imaginary) Self-dual 2-form &	3\\
$(0, 1)$ &	(Imaginary) Anti-self-dual 2-form &	3\\
$(1/2, 1/2)$ &	Vector (gauge field) &	4\\
$(1/2, 1)$ & 	Left-Handed Rarita-Schwinger field &	6\\
$(1, 1/2)$ &	Right-Handed Rarita-Schwinger field &	6\\
$(1, 1)$ &	Graviton (spin-2 field) &	9
\end{tabular}
\end{center}

\M
We study the scalar, the Dirac spinor $(1/2, 0)\oplus(0, 1/2)$, and
Vector fields specifically, since these are the necessary ingredients
for the Standard Model (and they are renormalizable fields).

\section{Wightman Axioms}

\M
We can formalize [canonical] quantum field theory using about a half
dozen axioms. That is to say, we can formalize \emph{one particular}
picture of quantum field theory using a handful of axioms.
These are the Wightman axioms, which is studied in great detail in
Streater and Wightman's book~\cite{Streater:1989vi}.

\begin{remark}
Peter Lowdon's slides ``A (brief) Introduction to Non-perturbative
Quantum Field Theory'' (6th IDPASC/LIP PhD Students Workshop) summarizes
the axioms as follows. Victor Kac's \textit{Vertex Algebras for Beginners}
has similar axioms, with some supermathematical generalizations to allow
easier discussion of fermions. \textbf{But I think} some signs may be
messed up, because I'm using the opposite metric-signature convention
than these authors.
\end{remark}

\begin{axiom}[Hilbert space structure]
The states of the theory are rays in a Hilbert space $\mathcal{H}$ which
possesses a continuous unitary representation $U(a,\alpha)$ of the
Poincar\'e spinor group $\overline{\mathscr{P}^{\uparrow}_{+}}$.
\end{axiom}

\begin{axiom}[Spectral condition]
The spectrum of the energy-momentum operator $P^{\mu}$ is confined to
the closed forward lightcone (\S\ref{defn:relativity:light-cone})
$\overline{V}^{+}=\{p^{\mu}|p^{2}\leq0,p^{0}\geq0\}$
where $U(a,1)=\exp(\I P^{\mu}a_{\mu})$.
\end{axiom}

\begin{remark}
This means the energy is bounded from below. Morally, this means ``The
theory is stable''.
\end{remark}

\begin{axiom}[Uniqueness of the vacuum]
There exists a unit state vector $|0\rangle$ called the vacuum state
which is a unique translationally-invariant state in $\mathcal{H}$.
\end{axiom}

Morally: ``The vacuum is unique and looks the same to all observers''.

\begin{axiom}[Field operators]
The theory consists of fields $\varphi^{(\kappa)}(x)$ (of type $\kappa$)
which have components $\varphi^{(\kappa)}_{\ell}(x)$ that are
operator-valued tempered distributions in $\mathcal{H}$, and the vacuum
state $|0\rangle$ is a cyclic vector for the fields.
\end{axiom}

Morally: ``Quantum fields $\varphi$ are operator-valued distributions''.
This means in particular:
\begin{enumerate}
\item Quantum fields cannot be evaluated at a single point (e.g., the
  Diract delta function $\delta(x)$ at $x=0$);
\item We need to ``average them out'' over some region of spacetime $\Sigma$.
\end{enumerate}
The physical justification for this is the Heisenberg uncertainty principle.

\begin{axiom}[Relativistic Covariance]
The fields $\varphi^{(\kappa)}_{\ell}(x)$ transform covariantly under
the action of $\overline{\mathscr{P}^{\uparrow}_{+}}$:
\begin{equation}
U(a,\alpha)\varphi^{(\kappa)}_{\ell}(x)U(a,\alpha)^{-1} = S_{ij}^{(\kappa)}(\alpha^{-1})\varphi_{j}^{(\kappa)}(\Lambda(\alpha)x+a),
\end{equation}
where $S(\alpha)$ is a finite-dimensional matrix representation of the
Lorentz spinor group $\overline{\mathscr{L}^{\uparrow}_{+}}$, and
$\Lambda(\alpha)$ is the Lorentz transformation corresponding to
$\alpha\in\overline{\mathscr{L}^{\uparrow}_{+}}$
\end{axiom}

\begin{axiom}[Microcausality]
If the support of the test functions $f$ and $g$ of the fields
$\varphi^{(\kappa)}_{\ell}$ and $\varphi^{(\kappa')}_{m}$ (respectively)
are space-like separated, then its graded commutator:
\begin{equation}
[\varphi^{(\kappa)}_{\ell}(f), \varphi^{(\kappa')}_{m}(g)]_{\pm} = 0
\end{equation}
when applied to any state in $\mathcal{H}$ for any fields
$\varphi^{(\kappa)}_{\ell}$ and $\varphi^{(\kappa')}_{m}$.
\end{axiom}

\N{Consequences}
We have a number of consequence from these Wightman axioms. Roughly, the
outline for results for scalar fields (and other simple fields) are as
follows:
\begin{enumerate}
\item The correlation functions $\langle0|\varphi^{(\kappa_{1})}_{\ell_{1}}(x_{1}) \cdots\varphi^{(\kappa_{n})}_{\ell_{n}}(x_{n})|0\rangle$
are distributions.
\item A quantum field theory can be fully reconstructed from knowledge
  of all of the correlation functions.
\item We can connect Minkowski and Euclidean field theories (taking the
  Wick rotation $t\mapsto\I\tau$)
\item Spin-statistics theorem: bosonic fields have commutators,
  fermionic fields have anticommutators.
\item CPT is a symmetry of any theory, even though how the operators
  ($C$, $P$, $T$) are implemented differently depending on the
  conventions adopted.
\end{enumerate}

\N{Gauss Law Constraint}\label{chunk:outline:pseudo-wightman-axioms}
For theories of physical interest (that is, gauge theories) significant
complications arise with the Wightman axioms. For QCD, these
complications underpin the \emph{nonperturbative} structure of
correlation functions.

Specifically, we have a ``Local Gauss Law'' for Yang--Mills theory
\begin{equation}
\partial^{\nu}F^{I}_{\mu\nu} = J^{I}_{\mu}.
\end{equation}
We have two ways to impose this constraint, which faces the following
trade-off:
\begin{enumerate}
\item Preserve positivity and lose locality (see: Coulomb gauge in QED); or
\item Preserve locality and lose positivity (see: Landau gauge in QCD).
\end{enumerate}
The second option, to recover locality, we modify the constraint
equation to be imposed for physical states:
\begin{equation}
\langle\mbox{phys}|\partial^{\nu}F^{I}_{\mu\nu} - J^{I}_{\mu}|\mbox{phys}\rangle=0,
\end{equation}
which either gives us:
\begin{enumerate}
\item The Gupta--Bleuler condition $\partial^{\mu}A_{\mu}^{(+)}|\mbox{phys}\rangle=0$,
or
\item BRST condition $Q_{B}|\mbox{phys}\rangle=0$.
\end{enumerate}
This necessarily introduces both \emph{zero-norm} and
\emph{negative-norm states}, which as discussed in
Section~\ref{sec:qm:basic-rules} corresponds to states with zero and
negative probabilities. We need to modify the axioms, resulting in
what some have called the ``Pseudo-Wightman approach''. For more about
this, see Bogolubov and friends~\cite{Bogolyubov:1990kw}.

\section{Yang--Mills Theory}

\N{Global and Local Symmetries}
Textbooks usually begin by studying ``global symmetries'', which do not
depend on spacetime coordinates. For example, if we have $N$ real scalar
fields, then we may put them into a column vector, and rotate by some
orthogonal $N\times N$ matrix. This works because the kinetic and
potential terms of the Lagrangian involve the norm squared of these
$N$-vectors, which are invariant under such rotations.

Physicists then ``gauge'' these symmetries and make them ``local''. But
then the kinetic terms will end up with derivatives of the rotation
matrix. These are then ``gauged away'' by changing the differential
operator.

\N{Yang--Mills Theory}
Another way to approach this is to start with electromagnetism, which
involves the electromagnetic 4-potential $A_{\mu}$. Then we consider
some Lie algebra $\mathfrak{g}$ and work with Lie algebra-valued
4-potentials $A_{\mu}^{I}T_{I}$ where $T_{I}$ are the generators of the
Lie algebra. We compute the field tensor:
\begin{equation}
F_{\mu\nu}^{I} = \partial_{\mu}A^{I}_{\nu} - \partial_{\nu}A^{I}_{\mu}
+g{f^{I}}_{JK}A^{J}_{\mu}A^{K}_{\nu}
\end{equation}
where we use the Lie bracket to determine the structure constants
${f^{I}}_{JK}$ by:
\begin{equation}
[T_{J}, T_{K}] = \I {f^{I}}_{JK}T_{I}.
\end{equation}
We usually work with $\su(n)$ as our Lie algebra, since $\su(3)$
describes the strong force, and $\su(2)\oplus\mathfrak{u}(1)$ describes
the electroweak forces.

\textsc{Cautionary Note}: a lot of books get confused over indices of
the Lie algebra, and use a bizarre Euclidean summation convention for
Lie algebra indices (but Einstein summation convention for spacetime
indices). Weinberg carefully works through the correct summation
conventions in his book \textit{The Quantum Theory of Fields}~\cite[\S15.1]{Weinberg:1996kr}
(see also~\cite[\S2.2]{Weinberg:1995mt} for discussion of symmetries and
conventions).
%(volume II, \S15.1; see also volume I, \S2.2).

\N{Equations of Motion}
Recall Maxwell's equations,
\begin{subequations}
\begin{align}
\partial^{i}E_{i} &= \rho\\
\partial^{i}B_{i} &= 0\\
-\partial_{t}E^{i} &= j^{i} -\epsilon^{ijk}\partial_{j}B_{k}\\
\partial_{t}B^{i} &= -\epsilon^{ijk}\partial_{j}E_{k}.
\end{align}
\end{subequations}
Yang--Mills equations of motion resemble this:
\begin{subequations}
\begin{align}
\partial^{i}E_{i} + [A^{i},E_{i}] &= \rho\\
\partial^{i}B_{i} + [A^{i},B_{i}] &= 0\\
-\partial_{t}E^{i} &= j^{i} -\epsilon^{ijk}(\partial_{j}B_{k} + [B_{j},E_{k}])\\
\partial_{t}B^{i} &= -\epsilon^{ijk}(\partial_{j}E_{k} + [A_{j},E_{k}]).
\end{align}
\end{subequations}
where we have the Yang--Mills ``electric'' and ``magnetic'' fields be
defined by (in the temporal gauge $A_{0}=0$):
\begin{equation}
E_{i} = -\partial_{t}A_{i}^{I}T_{I},\quad\mbox{and}\quad
B^{i} = \epsilon^{ijk}(\partial_{j}A_{k}^{I}T_{I} - \partial_{k}A_{j}^{I}T_{I}
+ [A_{j}^{J}T_{J}, A_{k}^{K}T_{K}]).
\end{equation}

\begin{remark}
See also Sanchez-Monroy and Quimbay~\cite{Sanchez-Monroy:2006sie} for
working out the Yang--Mills equations of motion for $\SU(3)$.
\end{remark}

\M
Geometrically, what's happening is we're working with connections on the
adjoint bundle over spacetime, and physicists call the connections
``gauge fields'' (the associated curvature is the ``field strength tensor'').
When $G$ is our gauge group, we have the principal $G$-bundle $P\to M$
over spacetime $M$, and the associated adjoint bundle $\mathrm{Ad}(P)\to M$
with its fibre being isomorphic to the vector space underlying
$\mathfrak{g}=\mathrm{Lie}(G)$ the Lie algebra of $G$.

There are as many gauge bosons as there are basis vectors for $\mathfrak{g}=\mathrm{Lie}(G)$.

For more on the mathematics related to the geometry needed for
Yang--Mills theory, see Hamilton~\cite{Hamilton:2017gbn}.

\N{``Charge''}
We can talk meaningfully about the analogous quantities to ``electric
charge'' in Yang--Mills theory. When the gauge group $G$ is non-Abelian
and $\mathrm{Lie}(G)$ is semisimple,
the charge is quantized (i.e., not continuous).\footnote{Weinberg
mentions this in passing in his book on quantum field theory. See, e.g.,~\cite[esp.~\S3.3,\S23.3]{Weinberg:1995mt}.}

Physicists call the ``charge'' different terms for different gauge
groups. Within QCD, they are called ``color''; for the weak force, they
are ``flavors''.

\N{Problems with Massive Yang--Mills}
If we try to add a nonzero mass to a non-Abelian Yang--Mills theory,
then we sacrifice either renormalizability or unitarity; Delbourgo and
friends argued this first~\cite{Delbourgo:1987np}.
Ellwanger and Wschebor~\cite{Ellwanger:2002sj} constructed a small
counter-example in $\su(2)$ by modifying BRST variations, working in a
particular gauge.

If we give up unitarity, we basically give up probabilities adding up to
$100\%$. On the other hand, nonrenormalizable fields have ``runaway
self-interactions'' which lead to infinities.

It's also worth mentioning that the mass term is not gauge-invariant,
which causes its own special Hell.

\N{BRST Symmetry}
When quantizing the Standard Model (really, Yang--Mills theory) we run
into problems: intuitively, gauge orbits describe the same physical
state, and we want to work with the quotient of the phase space modulo
gauge symmetries. The Faddeev--Poppov method implements this at the
quantum level, but this is generalized by using a BRST symmetry instead
of the usual gauge symmetry. This trick extends the phase space, embedding it as
the even part of a supermanifold, then we replace the gauge symmetry
with a rigid fermionic symmetry. This is the heart of BRST construction
and handles technical difficulties when quantizing non-abelian Yang--Mills.
This is how Henneaux and Teitelboim~\cite[\S8.1]{Henneaux:1992ig}
describe the BRST procedure.

Further, BRST quantization handles the negative norm and zero norm
states (\S\ref{chunk:outline:pseudo-wightman-axioms}).

\section{Standard Model}

\M
Despite its name, the ``Standard Model'' is neither standard nor a
model. We can decompose it along pseudo-historical lines as consisting
of electroweak force (with gauge group $\SU(2)\times\U(1)$)
and strong force (with gauge group $\SU(3)$). Then the Standard Model
is just a $\SU(3)\times\SU(2)\times\U(1)$ Yang--Mills coupled
to Dirac spinors and a Higgs boson.

\subsection{Strong Force}


\M
Perturbative QCD studies high energy interactions, because the coupling
$\alpha_{s}$ is quite small (which allows for perturbative calculations).
This is what would be found in, e.g., Peskin and Schroeder.

Then there is low energy QCD, which is studied using nonperturbative
methods (like lattice field theory).

The difficulty is in connecting these two domains together.

\N{Confinement} For quarks in the strong force, they experience a
phenomenon called ``confinement'' which we define as stating \emph{there
is a force between quarks which do not decrease with distance.} See
Esprieu~\cite{Espriu:1994br}, especially \S7. As a consequence, we will
not be able to find an isolated quark.

\N{Open problem: analytical description of confinement}
We lack an adequate analytical definition or criteria for ``confinement''.
This is a major open problem. One of the earliest criteria was Wilson~\cite{Wilson:1974sk}
and his so-called ``area law''\index{Area Law} for the Wilson loop along a
rectangle with edges $R_{1}$ and $R_{2}$, we expect:
\begin{equation}
\ln\langle W(R_{1}, R_{2})\rangle\sim R_{1}R_{2}.
\end{equation}
When this is true, the field theory is believed to enjoy confinement.

\begin{remark}[Folklore]
We have some results, e.g., 't Hooft~\cite{tHooft:1977nqb} proved that
mass gap plus unbroken center symmetry implies confinement.
``Unbroken center symmetry'' is not a rigorously defined notion.
Chatterjee~\cite{Chatterjee:2020nrl} attempted to make this rigorous.

Consequently, it is viewed that if the gauge group has a trivial center,
then confinement is impossible. I'm not so convinced by this, because 't
Hooft gives \emph{one possible route} to confinement, but this is not a
proof that it is \emph{the only possible route}.
\end{remark}

\N{Strong $\mathtt{G}_{2}$ Force}
The idea of using $\mathtt{G}_{2}$ as the gauge group for quarks and the
strong force was first explored in 1962 by Behrends and
friends~\cite{Behrends:1962zz}, then rediscovered [seemingly]
independently and explored in great detail by Holland and
friends~\cite{Holland:2003jy}. It turns out to be very useful as a
``test suite'' for lattice field theory
software~\cite{Ilgenfritz:2012wg,Maas:2012ts,Pepe:2006er,Wellegehausen:2011sc}.
There is some nuance around the deconfinement phase transition in
$\mathtt{G}_{2}$ strong force~\cite{Pepe:2005sz,Pepe:2006er}: when at
high temperatures, quarks become ``deconfined'' (due to asymptotic
freedom) leading to quark--gluon plasma.

For a review of the phenomenology surrounding quark--gluon plasma, see
Pasechnik and \v{S}umbera~\cite{Pasechnik:2016wkt} who note, ``The study
of ultra-relativistic heavy-ion collisions appears so far to be our only
way of studying the phase transitions in non-Abelian gauge theories
(most likely taken place in the early universe) under laboratory
conditions.''

It appears that $\mathtt{G}_{2}$ cannot explain deconfinement phase
transition, which we witness in laboratory experiments. Bruno and
friends~\cite{Bruno:2014rxa} have argued that confinement can be
reproduced with $\mathtt{G}_{2}$ Yang--Mills, pointing out there is no
adequate definition for these notions, and qualitative behaviour can be
captured. (Bruno and friends also thoroughly review the literature of
numerical and analytical studies of $\mathtt{G}_{2}$ Yang--Mills theory.)

\begin{remark}[$\mathtt{G}_{2}$ and Octonions]
For more about $\mathtt{G}_{2}$ as the automorphism group for the Octonions $\mathtt{G}_{2}\iso\aut(\OO)$, see
\S4.1 in Baez~\cite{Baez:2001dm}.
\end{remark}

\subsection{As an Effective Field Theory}

\M
We can treat the Standard Model as a low energy approximation to some
``real'' field. This is handled using the toolkit of ``effective field
theory''.
For a review of this (particularly applied to the Standard Model), see
Brivio and Trott~\cite{Brivio:2017vri}.

\N{TODO List}
Stuff I should write about:
\begin{enumerate}
\item Lattice QCD~\cite{Lepage:1998dt}
\item Elitzur's theorem~\cite{Elitzur:1975im} (``No such thing as spontaneous symmetry breaking of a local gauge symmetry.'')
\item The Fradkin--Shenker--Osterwalder--Seiler (FSOS) Theorem (``There is no transition in coupling-constant space which isolates the Higgs phase from a confinement-like phase.'')
\item Haag's Theorem~\cite{Haag:1992hx}: an interacting theory and a
  free theory are not unitarily equivalent
\item Fr\"{o}lich--Morchio--Strocchi mechanism (\arXiv{2305.01960})
  which resolves the paradox that perturbation theory applied to a free
  theory can approximate results for interacting theories.
\item Kinoshita--Lee--Nauenberg (KLN) theorem (``perturbatively the standard model as a whole is infrared (IR) finite; i.e., the infrared divergences coming from loop integrals are canceled by IR divergences coming from phase space integrals.'')
\item Factorization theorems (for hadron collider processes) \arXiv{hep-ph/0409313}
\item Block--Nordsieck theorem in QED: Infrared (\textbf{soft}) divergences
  cancel out after summation over all degenerate \textbf{final} states
  compatible with experimental detection. Does not work in non-Abelian
  gauge theories (see {\tt\doi{10.1016/0550-3213(81)90554-X}}).
\item Coleman--Norton theorem (from {\tt\doi{10.1007/BF02750472}})
\item Multiplets are irreducible representations (or subrepresentations)
  for a Lie algebra/group.
\item Technicolor (i.e., composite Higgs); for some reviews, see
  \arXiv{hep-ph/9401324} and \arXiv{1104.1255}.
\item Functional derivatives --- physicists use the notation $\delta F[\varphi]/\delta\varphi(\vec{x})$
  for the Gateaux derivative $\lim_{\varepsilon\to 0}(F[\varphi+\varepsilon\delta_{\vec{x}}]-F[\varphi])/\varepsilon$
  using the Dirac delta function $\delta_{\vec{x}}$.
\end{enumerate}

% Zeidler, \textit{Quantum Field Theory III}, \S3.14.3 discusses highest
% weight vectors and elementary particle physics.

% https://www2.pd.infn.it/ecfa/6_p_loopv_d_comelli.PDF

% Greiner's "Quantum Mechanics: Symmetries" appears to be quite good

% https://arxiv.org/abs/2212.08470

% Segal's notes on QFT http://web.archive.org/web/20000901075112/http://www.cgtp.duke.edu/ITP99/segal/
