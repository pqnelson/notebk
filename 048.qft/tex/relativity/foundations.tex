\section{Foundations}

\begin{axiom}[Principle of Relativity]
The laws of physics are identical in all inertial frames. That is to
say, the outcome of any physical experiment is the same when performed
with identical initial conditions relative to any inertial frame.
\end{axiom}

\begin{axiom}
There exists an inertial reference frame in which light signals in
vacuum always travel rectilinearly at constant speed $c$, in all
directions, independently of the motion of the source.
\end{axiom}

\begin{remark}
These are the axioms as formulated in Rinder~\cite[\S4]{Rindler:1991sr}.
Traditionally, introductory texts to special relativity use some version
of these axioms, then ``derive'' results in special relativity.

A more intuitive approach uses $K$-calculus, as first formulated by
Bondi, using spacetime diagrams.
\end{remark}

\M
The best approach to discussing special relativity is the geometric
approach, using spacetime diagrams. However, the point of this chapter
is not to teach special relativity, but to review the Lorentz group as
encoding Lorentz invariance.

The geometric approach uses a 4-dimensional affine space describing
physical space-time, consisting of ``events'' [points]. We do the usual
trick setting up a reference frame as consisting of a point (or trajectory)
which we think of as the ``origin'', then take 4 other distinct points
to construct unit vectors (by taking the displacement from the
``origin'' to each of these points). This gives us a coordinate system.

We can describe time intervals in difference reference frames by
``shooting off'' one photon at the start of the interval (and then
another at the end of the interval). This translates to an interval
$k\,\Delta t$ in the other reference frame, where $k$ is a factor to be
determined. This is done in D'Inverno's \textit{Introducing Einstein's Relativity},
chapters 2 and 3.

\N{On ``Paradoxes''}
One last word, a lot of ``paradoxes'' encountered in physics (like the
``Twin paradox'' in special relativity) are not actually paradoxes: they
are just physical situations where our intuitions fail to describe the
phenomenon properly.

\endinput