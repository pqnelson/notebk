\section{Electromagnetism}\label{section:relativity:electromagnetism}

\N{Maxwell's Equations}\index{Maxwell!Equations}
Maxwell's equations which we learn in Physics 101 are, in Gaussian
units, for statics:
\begin{subequations}
\begin{align}
\nabla\cdot\vec{E} &= 4\pi\rho_{e}\\
\nabla\cdot\vec{B} &= 0\\
\intertext{and for dynamics:}
-\frac{\partial\vec{E}}{\partial t} + c\nabla\times\vec{B} &= 4\pi\vec{j}\\
\frac{\partial\vec{B}}{\partial t} + c\nabla\times\vec{E} &= 0.
\end{align}
\end{subequations}
Here $\rho_{e}$ is the electric charge density, and $\vec{j}$ is
the electric current density.
The exact form of these equations depend on the units used. In SI units,
the coefficients need to be altered:
\begin{subequations}
\begin{align}
\nabla\cdot\vec{E} &= \frac{\rho_{e}}{\varepsilon_{0}}\\
\nabla\cdot\vec{B} &= 0
\end{align}
\end{subequations}
\begin{subequations}
\begin{align}
-\mu_{0}\varepsilon_{0}\frac{\partial\vec{E}}{\partial t} + \nabla\times\vec{B} &= \mu_{0}\vec{j}\\
\frac{\partial\vec{B}}{\partial t} + \nabla\times\vec{E} &= 0,
\end{align}
\end{subequations}
where $\varepsilon_{0}$ is the permittivity of free space, $\mu_{0}$ is
the permeability of free space, and $c=1/\sqrt{\varepsilon_{0}\mu_{0}}$
is the speed of light in a vacuum.

\N{Potentials}
It is common to introduce the electric potential $\varphi(\vec{x})$
and magnetic potential $\vec{A}(\vec{x}, t)$. Then the electric field is
taken to be:
\begin{equation}
\vec{E} = -\vec{\nabla}\varphi - \frac{\partial\vec{A}}{\partial t}.
\end{equation}
We call $\varphi$ the \define{Electric Potential} and $\vec{A}$ the
\define{Magnetic Potential}.
The magnetic field is,
\begin{equation}
\vec{B} = \nabla\times\vec{A}.
\end{equation}

\begin{remark}
Any time-dependence the electric field enjoys may be traced back to the
magnetic potential contribution.
\end{remark}

\N{Gauge Invariance}\index{Gauge!Invariance}
These potentials are not unique. We could, for any function
$\lambda(\vec{x},t)$, consider the potentials
\begin{subequations}
\begin{align}
\varphi' &= \varphi - \frac{\partial\lambda}{\partial t}\\
\vec{A}' &= \vec{A} + \vec{\nabla}\lambda.
\end{align}
\end{subequations}
This arbitrariness is an example of a gauge symmetry, a redundancy
describing the same physical conditions. We can eliminate this
redundancy (a process called \define{Gauge-Fixing}) by adding an
additional condition.

\N{Lorenz Gauge}\index{Gauge!Condition!Lorenz}
We impose the condition
\begin{equation}
\vec{\nabla}\cdot\vec{A} + \frac{1}{c}\frac{\partial\varphi}{\partial t}=0.
\end{equation}
This is the \emph{Lorenz Gauge} (note the lack of ``t'', because it is
named after the Dutch physicist Ludvig Lorenz, not to be confused with
the other Dutch physicist Hendrik Lorentz [of ``Lorentz transformation''
fame]). Maxwell's equations
simplify to 
\begin{subequations}
\begin{align}
\nabla^{2}\varphi - \frac{1}{c^{2}}\frac{\partial^{2}\varphi}{\partial t^{2}}
&= -\frac{\rho_{e}}{\varepsilon_{0}}\\
\nabla^{2}\vec{A} - \frac{1}{c^{2}}\frac{\partial^{2}\vec{A}}{\partial t^{2}}
&= -\mu_{0}\vec{j}.
\end{align}
\end{subequations}
This makes manifest the wave structure of electric and magnetic fields,
since these are wave equations.

\begin{remark}
There are other gauge choices, and the exact form of the equations
depend on the gauge choice.
\end{remark}

\N{Four-Potential}\label{chunk:relativity:electromagnetism:four-potential}\index{4-Potential}\index{$A^{\mu}$}
The first step towards describing electromagnetism in special relativity
is to use four-vectors. The potentials together form a four-vector
called the \define{Four-Potential} $A^{\mu} = (\varphi, \vec{A})$.
Conventions vary, some authors take $A^{t}=c^{-1}\varphi$. This implies
\begin{equation}
A_{\mu} = (-\varphi, \vec{A}).
\end{equation}

\begin{remark}
We can assemble a one-form called the Potential one-form
$A = \eta_{\mu\nu}A^{\mu}\,\D x^{\nu}$.
\end{remark}

\N{Partial Derivatives}\label{chunk:relativity:partial-derivatives}
We have, in our conventions,\index{$\partial^{\mu}$}
\begin{equation}
\partial^{\mu} := (-c^{-1}\partial_{t}, \vec{\nabla}).
\end{equation}
Then, in Cartesian coordinates,\index{$\partial_{\mu}$}
\begin{equation}
\partial_{\nu} = \eta_{\mu\nu}\partial^{\mu} = (c^{-1}\partial_{t}, \vec{\nabla}).
\end{equation}

\N{Field-Strength Tensor}\index{$F_{\mu\nu}$}\index{Field-Strength tensor}
We can form the field-strength tensor from the four-potential as
\begin{subequations}
\begin{equation}
F^{\mu\nu} := \partial^{\mu}A^{\nu} - \partial^{\nu}A^{\mu},
\end{equation}
or with indices downstairs
\begin{equation}
F_{\mu\nu} := \partial_{\mu}A_{\nu} - \partial_{\nu}A_{\mu}.
\end{equation}
\end{subequations}
The field strength tensor's components are then
\begin{subequations}
\begin{align}
F_{0j} &= \partial_{0}A_{j} - \partial_{j}A_{0}\\
&= \partial_{t}A_{j} - \partial_{j}(-\varphi)\\
&= -E_{j},
\end{align}
\end{subequations}
and
\begin{equation}
F_{ij} = \partial_{i}A_{j} - \partial_{j}A_{i} = \begin{pmatrix}
 0     &  B_{z} & -B_{y}\\
-B_{z} &  0     & B_{x}\\
 B_{y} & -B_{x} & 0
\end{pmatrix}
\end{equation}
(where $j$ indexes the columns, $i$ the rows). Then
\begin{equation}
F_{\mu\nu} = \begin{pmatrix}F_{00} & F_{0j}\\
F_{i0} & F_{ij}
\end{pmatrix} = \begin{pmatrix}
 0       & -E_{x}/c & -E_{y}/c & -E_{z}/c\\
 E_{x}/c &  0       &  B_{z}   & -B_{y}\\
 E_{y}/c & -B_{z}   &  0       &  B_{x}\\
 E_{z}/c &  B_{y}   & -B_{x}   &  0
\end{pmatrix}.
\end{equation}

\begin{exercise}
Compute the components of $F^{\mu\nu}=\eta^{\alpha\mu}\eta^{\beta\nu}F_{\alpha\beta}$.
\end{exercise}

\begin{exercise}
Expression $F_{\mu\nu}F^{\mu\nu}$ in terms of $\vec{E}$ and $\vec{B}$.
\end{exercise}

\N{Recovering Maxwell's Equations}\label{chunk:relativity:electromagnetism:recovering-maxwell-equations}
We see that
\begin{calculation}
  \partial^{\nu}F_{0\nu}
\step{since $F_{00}=0$}
  \partial^{j}F_{0j}
\step{since $F_{0j}=-E_{j}$}
  -\partial^{j}E_{j}
\step{by Gauss's Law}
  -4\pi\rho_{e}.
\end{calculation}
This gives us the first of Maxwell's equations.
We also see
\begin{calculation}
  \partial^{\nu}F_{i\nu}
\step{breaking $\mu$ up into $(0,j$)}
  \partial^{0}F_{i,t} + \partial^{j}F_{i,j}
\step{since $F_{i,t} = E_{i}$ and $F_{i,j} = \epsilon_{ijk}B^{k}$}
  \partial^{0}E_{i} + \partial^{j}\epsilon_{ijk}B^{k}
\step{since $\partial^{0}=-c^{-1}\partial_{t}$ and
    $\partial^{j}\epsilon_{ijk}B^{k} = (\nabla\times\vec{B})_{i}$}
  \frac{-1}{c}\partial_{t}E_{i} + \partial^{j}\epsilon_{ijk}B^{k}
\step{using Electrodynamic equation of motion}
  4\pi J_{i}.
\end{calculation}
This motivates a 4-vector for the current density called the \define{Four-Current}:
\begin{equation}
J_{\mu} = (-\rho_{e}, \vec{j}).
\end{equation}
Therefore two of Maxwell's equations for the electric field (both
electrostatics and electrodynamics) are encoded by
\begin{equation}\label{eq:relativity:electromagnetism:maxwell-for-electric-field}
\boxed{\partial^{\nu}F_{\mu\nu} = 4\pi J_{\mu}.}
\end{equation}
We just need to express the remaining equations using the field strength
tensor.

\N{Equations for Magnetism}
Magnetodynamics may be written as
\begin{equation}
\partial_{t}\epsilon_{ijk}B^{k} + \partial_{i}E_{j} - \partial_{j}E_{i}
= 0
\end{equation}
for \emph{any} $i$, $j$, $k=1,2,3$. Then
\begin{calculation}
0
\step{Magnetodynamics equation}
\partial_{t}\epsilon_{ijk}B^{k} + \partial_{i}E_{j} - \partial_{j}E_{i}
\step{since $F_{ij} = \epsilon_{ijk}B^{k}$}
\partial_{t}F_{ij} + \partial_{i}E_{j} - \partial_{j}E_{i}
\step{since $E_{j}=F_{j0}$ and $-E_{i}=F_{0i}$}
\partial_{t}F_{ij} + \partial_{i}F_{j0} + \partial_{j}F_{0i}.
\end{calculation}
This suggests more generally (replacing the spatial indices with
4-indices and $t$ with another distinct index)
\begin{equation}\label{eq:relativity:electromagnetism:guessed-magnetic-equations}
0 \stackrel{???}{=} \partial_{\alpha}F_{\beta\gamma} + \partial_{\beta}F_{\gamma\alpha} + \partial_{\gamma}F_{\alpha\beta}.
\end{equation}
For $\alpha=0$, $\beta=i$, $\gamma=j$, this recovers Magnetodynamics.
Does this recover the magnetostatic Maxwell equation?

If any two indices are repeated (say $\alpha=\beta$), then $F_{\mu\nu}$
being antisymmetric implies Eq~\eqref{eq:relativity:electromagnetism:guessed-magnetic-equations}
holds trivially. Similarly, if $\alpha=\beta=\gamma$, then the equation
holds. Therefore, the only remaining cases are spatial indices. We see
\begin{calculation}
\partial_{x}F_{yz} + \partial_{y}F_{zx} + \partial_{z}F_{xy}
\step{since $F_{yz}=B_{x}$, $F_{zx}=B_{y}$, $F_{xy}=B_{z}$}
\partial_{x}B_{x} + \partial_{y}B_{y} + \partial_{z}B_{z}
\step{this is the divergence of the magnetic field}
\nabla\cdot\vec{B}
\step{by magnetostatics}
0.
\end{calculation}
Permuting the indices will just multiply through by $-1$, which doesn't
change the result. Therefore we conclude
\begin{equation}
\partial_{i}F_{jk} + \partial_{j}F_{ki} + \partial_{k}F_{ij} = 0,
\end{equation}
and moreover this encodes the magnetostatic Maxwell equation. Combining
things together, we find the magnetic Maxwell equations are:
\begin{equation}\label{eq:relativity:magnetic-eom}
\boxed{\partial_{\alpha}F_{\beta\gamma} + \partial_{\beta}F_{\gamma\alpha} + \partial_{\gamma}F_{\alpha\beta}
= 0.}
\end{equation}

\begin{exercise}
Prove Eq~\eqref{eq:relativity:magnetic-eom} holds identically for $F_{\mu\nu}=\partial_{\mu}A_{\nu}-\partial_{\nu}A_{\mu}$,
and therefore we don't need to worry about it when writing down a
Lagrangian for the electromagnetic field.
\end{exercise}

\subsection{Lagrangian for the Electromagnetic Field}

\begin{exercise}
Prove $\displaystyle\frac{\partial \left(\partial_{\mu} A_{\nu}\right)}{\partial\left(\partial_{\rho} A_{\sigma}\right)}= \delta_{\mu}^{\rho} \delta_{\nu}^{\sigma}$.
\end{exercise}

\begin{exercise}
Compute $\displaystyle\frac{\partial(F_{\alpha\beta}F^{\alpha\beta})}{\partial(\partial_{\mu}A_{\nu})}$.
Remember: $F^{\alpha\beta} = \eta^{\alpha\rho}\eta^{\beta\sigma}F_{\rho\sigma}$.
\end{exercise}

\begin{exercise}
Is it true or not that 
$\displaystyle\frac{\partial(F_{\alpha\beta}F^{\alpha\beta})}{\partial(A_{\nu})}=0$?
\end{exercise}

\M
Combining the results from these exercises, you can show the Lagrangian
density for Electromagnetism is
\begin{equation}\label{eq:relativity:lagrangian-for-em}
\mathcal{L} = \frac{-1}{4}F_{\mu\nu}F^{\mu\nu} + A_{\mu}J^{\mu}.
\end{equation}
The Euler--Lagrange equations for this would be (summing over $\beta$):
\begin{equation}
\partial_{\beta}\left[\frac{\partial\mathcal{L}}{\partial(\partial_{\beta}A_{\alpha})}\right]
-\frac{\partial\mathcal{L}}{\partial A_{\alpha}} = 0.
\end{equation}
This should recover $\partial_{\alpha}F^{\alpha\beta}=J^{\beta}$.

\begin{exercise}
Prove the Lagrangian density in Eq~\eqref{eq:relativity:lagrangian-for-em}
is Lorentz invariant.
\end{exercise}

\begin{exercise}
\begin{enumerate}
\item Find the canonically conjugate momenta density $\pi_{i} = \partial\mathcal{L}/\partial(\partial_{t}A^{i})$.
\item Rewrite $\mathcal{L}$ using $A^{i}$, $\pi_{j}$
\item Compute the Hamiltonian density $\mathcal{H} = \pi_{i}\partial_{t}A^{i}-\mathcal{L}$.
\end{enumerate}
\noindent \textsc{Hint}: you should have a constrained system, since the
Gauss law does not involve time derivatives. (Therefore we should expect
to find a constraint equivalent to the Gauss law.)
\end{exercise}

\endinput

