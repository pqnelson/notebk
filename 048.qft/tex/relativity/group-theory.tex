\section{Group Theoretic Description}\label{section:relativity:group-theoretic-description}

\begin{definition}\index{Group!Lorentz}
The \define{(Homogeneous) Lorentz Group} for $\RR^{n,1}$ is the group
generated by spatial rotations and Lorentz boosts --- that is, the indefinite
orthogonal group $\O(n,1)$. A generic element of the Lorentz group is
called a \define{Lorentz Transformation}.
\end{definition}

\begin{exercise}
Prove that this actually is a group. That is, the composition of Lorentz
transformation are a Lorentz transformation; inverses of Lorentz
transformations are Lorentz transformations.
\end{exercise}

\M
There are 4 connected components to the Lorentz group $\O(3,1)$ which
are related by parity operator $P$ and time reversal operator $T$, which
are represented by the matrices (acting on the ``obvious'' fundamental
representation as):
\begin{subequations}
\begin{align}
P &= \diag(1, -1, -1, -1)\\
T &= \diag(-1, 1, 1, 1).
\end{align}
\end{subequations}

\begin{definition}\index{Group!Poincar\'e}\index{Poincar\'e!Group}\index{Group!Lorentz!Inhomogeneous}\label{defn:relativity:poincare-group}
The \define{Poincar\'e Group} for $\RR^{n,1}$ is the group generated by
translations in spacetime [by a constant displacement], spatial
rotations, and Lorentz boosts; it is given by the group
$G = \RR^{n,1}\rtimes\O(n,1)$.
\end{definition}

\begin{remark}
Weinberg~\cite{Weinberg:1995mt} calls the Poincar\'e Group the \emph{Inhomogeneous Lorentz Group}.
\end{remark}

\begin{remark}\index{Poincar\'e!Spin group}\index{Group!Poincar\'e!Spin}
We actually want to use the ``universal covering group'' of the
Poincar\'e Group, which is known in the literature as the ``Poincar\'e Spin
Group''.
\end{remark}

\M
We can always consider how an element of $g\in\O(3,1)$ will act on any
4-position $x\in\RR^{3,1}$ ``in the obvious way'' as a Lorentz
transformation. We will write this as $g\cdot x$. Specifically this acts
by means of a $4\times4$ matrix ${\Lambda^{\alpha'}}_{\mu}$ sending
$x^{\mu}\to x^{\alpha'} = f^{\alpha'}(x^{\mu}) = (\Lambda x)^{\alpha'}$.
Explicitly, using Einstein summation convention,
\begin{equation*}
x^{\alpha'} = {\Lambda^{\alpha'}}_{\mu}x^{\mu}.
\end{equation*}

\N{Inverse Transformation}
We can compose Lorentz transformations to find the inverse
transformation ${\Lambda^{\nu}}_{\alpha'}$ which satisfies
\begin{equation}
{\Lambda^{\nu}}_{\alpha'}{\Lambda^{\alpha'}}_{\mu}={\delta^{\nu}}_{\mu}.
\end{equation}
For covariant vectors, we need to use
% We then have
${[\transpose{(\Lambda^{-1})}]^{\mu}}_{\alpha'} = {\Lambda_{\alpha'}}^{\mu}$
be the transpose of the inverse Lorentz transformation for $\Lambda$.
This is because covariant vectors transform under the dual
representation. 


\M
The Lorentz group acts on a scalar field $\varphi(x)$ by
\begin{equation}
g\cdot\varphi(x) = \varphi\bigl(g^{-1}\cdot x\bigr).
\end{equation}
This is because a scalar field is ``just a function'', and this is how
groups act on functions.

\N{Action on Vector Fields}
For a vector field with components $A^{\mu}(x)$, a Lorentz
transformation $g\in\O(3,1)$ acts on this in a ``mixed'' way. We have
the matrix ${\Lambda^{\alpha'}}_{\mu}$ describe the action $(g\cdot x)^{\alpha'} = {\Lambda^{\alpha'}}_{\nu}x^{\nu}$,
which is the change of coordinates $x^{\mu}\to x^{\alpha'}=f^{\alpha'}(x^{\mu})$,
so therefore a contravariant vector field transforms as:
\begin{equation}
[(g\cdot \vec{A})(x)]^{\alpha'} = {\Lambda^{\alpha'}}_{\nu}A^{\nu}(g^{-1}\cdot x).
\end{equation}
It acts on the vector as a whole as we would expect, and on each
component as a function.

For a covariant vector $A_{\mu}(x)$, we see it transforms as
\begin{equation}
[(g\cdot \vec{A})(x)]_{\alpha'} = {\Lambda_{\alpha'}}^{\mu}A_{\mu}(g^{-1}\cdot x)
\end{equation}
where ${\Lambda_{\alpha'}}^{\mu}$ is the matrix inverse of the Lorentz
transformation. It satisfies
\begin{equation}
{\Lambda_{\nu}}^{\alpha'}{\Lambda^{\nu}}_{\beta'} = {\delta^{\alpha'}}_{\beta'}.
\end{equation}

\N{Action on Tensor Fields}
More generally, for a rank $n$ tensor with components $[\mathsf{T}(x)]^{\mu_{1}\cdots\mu_{n}}=T^{\mu_{1}\cdots\mu_{n}}(x)$,
the element $g\in\O(3,1)$ acts as
\begin{equation}
[(g\cdot \tens{T})(x)]^{\alpha_{1}'\cdots\alpha_{n}'}
= {\Lambda^{\alpha_{1}'}}_{\nu_{1}}(\cdots){\Lambda^{\alpha_{n}'}}_{\nu_{n}}
T^{\nu_{1}\cdots\nu_{n}}(g^{-1}\cdot x).
\end{equation}
For covariant indices, we multiply by matrix inverses of $\Lambda$.
These actions are all basic representation theory.

\N{Lorentz Invariance}
We call a quantity $Q(x)$ \define{Lorentz Invariant} if for each $g\in\O(3,1)$
we have
\begin{equation}
g\cdot Q(x) = Q(x).
\end{equation}
For example $\D s^{2}$ is Lorentz invariant.

\endinput