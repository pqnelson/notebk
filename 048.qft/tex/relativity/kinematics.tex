\section{Kinematics}

\N{Four-Vectors} Special relativity differs from Newtonian physics by
working with 4-vectors. The usual vectors we encountered in physics
$\vec{x}$ are 3-vectors in physical space. Now we will use 4-component
vectors of the form $(t, \vec{x})$ or more generally using indices
$(x^{0}, x^{1}, x^{2}, x^{3})$. We can write this using basis vectors
\begin{subequations}
\begin{align}
  A &= (A^{0}, A^{1}, A^{2}, A^{3})\\
&= A^{0}\vec{e}_{0} + A^{1}\vec{e}_{1} + A^{2}\vec{e}_{2} + A^{3}\vec{e}_{3}\\
&= A^{0}\vec{e}_{0} + A^{i}\vec{e}_{i}\\
&= A^{\mu}\vec{e}_{\mu}
\end{align}
\end{subequations}
where $\vec{e}_{\mu}$ are basis vectors, we implicitly sum over repeated
indices, $i = 1,2,3$, and $\mu=0,1,2,3$.

The zeroth component is the time component, the remaining components are
the spatial components.

\begin{remark}
When the components of the four-vector are upstairs $A^{\mu}$,
these are \define{Contravariant} vectors.
\end{remark}

\N{Covectors}
We can also consider ``covectors'' or ``dual vectors'' (which eat in a
4-vector and produce a [real] number). These have components with
downstairs indices $B_{\mu}\vec{f}^{\mu}$ where $\vec{f}^{\mu}$ are the
co-basis covectors.

\M
Normally we can transform the components of a four-vector
$\vec{A}=A^{\mu}\vec{e}_{\mu}$ into the components of a covector by
using the metric $g_{\mu\nu}$ in the usual way:
\begin{equation}
A_{\nu}\vec{f}^{\nu} = (g_{\mu\nu}A^{\mu})\vec{f}^{\nu}.
\end{equation}
We can do the same, but for special relativity the metric is denoted
$\eta_{\mu\nu}$.\marginpar{\footnotesize Minkowski metric $\eta_{\mu\nu}$}
We call this the \define{Minkowski metric} and in Cartesian coordinates
has components
\begin{equation}
\eta_{\mu\nu} = \begin{pmatrix}-1 & 0 & 0 & 0\\
0 & 1 & 0 & 0\\
0 & 0 & 1 & 0\\
0 & 0 & 0 & 1
\end{pmatrix}.
\end{equation}
This is the so-called \define{East-Coast Convention} (particle
physicists multiply this by $-1$ and that's the \emph{West-coast convention}).

\N{Magnitude of Four-Vectors}
When $a^{\mu}$ is any four-vector, its magnitude is the scalar quantity
given by:
\begin{equation}
\|a^{\mu}\|^{2} := \eta_{\mu\nu}a^{\mu}a^{\nu}.
\end{equation}

\begin{definition}
The \define{Four-Position} is the four-vector with components
\begin{equation*}
x^{\mu} = (ct, x^{1}, x^{2}, x^{3}).
\end{equation*}
In Cartesian coordinates, $x^{\mu} = (ct, x, y, z)$.
\end{definition}

\N{Events}
Events in spacetime are described using 4-position vectors. This is an
idealization that events have no ``duration''. We could handle events
with some duration by having one 4-position for the ``start'' and
another 4-position for the ``end''.

\begin{definition}
When we have two 4-position vectors $\vec{A}_{1}=(ct_{1},\vec{r}_{1})$
and $\vec{A}_{2}=(ct_{2},\vec{r}_{2})$, we define the
\define{Displacement Four-Vector} as the four-vector
\begin{equation*}
\Delta\vec{A} = (c\,\Delta t,\Delta\vec{r}) = \vec{A}_{2} - \vec{A}_{1}.
\end{equation*}
For an \define{Infinitesimal Displacement Four-Vector} (or
\emph{Differential Four-Position}), we write
$\D\vec{A}$.
\end{definition}

\M
Suppose we have two events in spacetime separated by an infinitesimal
displacement 4-vector $\D x^{\mu}$. Special relativity demands the
infinitesimal interval,
\begin{equation}
(\D s)^{2} := \eta_{\mu\nu}\,\D x^{\mu}\,\D x^{\nu} = -c^{2}(\D t)^{2} + (\D x)^{2} + (\D y)^{2} + (\D z)^{2},
\end{equation}
must be the same for all inertial observers. Here we use summation
conventions where, when we have an index downstairs and upstairs, we sum
over it (so in our equation, we sum over $\mu$ and $\nu$).

\begin{definition} Let $a^{\mu}$ be a 4-vector.
  \begin{enumerate}
  \item If $\eta_{\mu\nu}a^{\mu}a^{\nu} < 0$, then we call $a^{\mu}$ \define{Time-like}.
  \item If $\eta_{\mu\nu}a^{\mu}a^{\nu} = 0$, then we call $a^{\mu}$ \define{Light-like}.
  \item If $\eta_{\mu\nu}a^{\mu}a^{\nu} > 0$, then we call $a^{\mu}$ \define{Space-like}.
  \end{enumerate}
\end{definition}

\begin{remark}
To see the motivation for these definitions, consider the trajectory of
a photon moving along the $x$-axis in Cartesian coordinates. It would be
$\vec{\gamma}(t)=(ct,ct,0,0)$ and its displacement from the origin would
be always zero for any $t\in\RR$. Therefore, the displacement would be
light-like.

For time-like vectors, consider the displacement 4-vector
$\vec{\alpha}(t)=(ct,0,0,0)$ which stays at the spatial origin. We see
its magnitude is $-c^{2}t^{2}<0$.

For space-like vectors, consider the displacement 4-vector
$\vec{\beta}(t)=(0,ct,0,0)$. Its magnitude is $c^{2}t^{2}>0$.
\end{remark}

\begin{definition}[Minkowski spacetime]
We write $\RR^{3,1}$ for \define{Minkowski Spacetime}, i.e., $\RR^{4}$
equipped with the Minkowski metric $\eta$.
\end{definition}

\begin{remark}
If we were using $+---$ signature conventions, Minkowski space would be
$\RR^{1,3}$. 
\end{remark}

\subsection{Invariance of Differential Displacement}

\M We have the obvious spatial rotations and spatial translations leave
the differential displacement $\D s^{2}$ invariant, since this is a
carry-over from Euclidean geometry. The more interesting case is the
``rotation'' of time and space. We will restrict focus to $\RR^{1,1}$
for simplicity, but the reasoning generalizes to $\RR^{n,1}$ by
composing with rotations to make the ``direction of motion'' a spatial
direction.

%% \M Let us restrict focus to $\RR^{1,1}$ for simplicity. Suppose we
%% change coordinates
%% \begin{equation}
%% \begin{pmatrix}c\,\D t'\\\D x'
%% \end{pmatrix}
%% = \begin{pmatrix}a & b\\k & d
%% \end{pmatrix}
%% \begin{pmatrix}c\,\D t\\\D x
%% \end{pmatrix}.
%% \end{equation}
%% Then writing $(\D s)^{2}$ in the new coordinates:
%% \begin{calculation}
%%   (\D s)^{2}
%% \step{by definition}
%%   -c^{2}(\D t')^{2} + (\D x)^{2}
%% \step{plugging in the values of $\D t'$ and $\D x'$}
%%   -c^{2}(a\,\D t + b\,\D x)^{2} + (k\,\D t + d\,\D x)^{2}
%% \step{expanding terms}
%%   -c^{2}[a^{2}\,(\D t)^{2} + 2ab\,\D t\D x + b^{2}\,(\D x)^{2}]
%%   +[k^{2}\,(\D t)^{2} + 2kd\,\D t\D x + d^{2}\,(\D x)^{2}]
%% \step{collecting terms}
%%   -(c^{2}a^{2} - k^{2})(\D t)^{2} - 2(c^{2}ab - kd)\,\D t\D x + (-c^{2}b^{2}+d^{2})(\D x)^{2}.
%% \end{calculation}
%% Comparing to $(\D s)^{2} = -(\D t)^{2} + (\D x)^{2}$, we have the
%% conditions:
%% \begin{subequations}
%% \begin{align}
%% (c^{2}a^{2} - k^{2}) &= 1\\
%% (c^{2}ab - kd) &= 0\\
%% (-c^{2}b^{2}+d^{2}) &= 1.
%% \end{align}
%% \end{subequations}

%% \M
%% Specifically, when changing from one inertial frame (at rest) to another
%% (moving at speed $v$ in the $x$-direction relative to the first frame),
%% we expect
%% \begin{equation}
%% x' = x - vt.
%% \end{equation}
%% There may be a relativistic correction, a factor $\alpha(v)$ such that
%% for $v\ll c$ we have $\alpha(v)\approx 1$. Then we use
%% \begin{equation}
%% x' = \alpha(v)(x - vt).
%% \end{equation}
%% This gives us
%% \begin{equation}
%% \begin{pmatrix}c\,\D t'\\\D x'
%% \end{pmatrix}
%% = \begin{pmatrix}a & b\\-\alpha v & \alpha
%% \end{pmatrix}
%% \begin{pmatrix}c\,\D t\\\D x
%% \end{pmatrix}.
%% \end{equation}
%% This lets us substitute $k=-\alpha(v)v/c$ and $d=\alpha(v)$. This gives us
%% the system of equations
%% \begin{subequations}
%% \begin{align}
%% (c^{2}a^{2} - v^{2}\alpha^{2}) &= c^{2}\\
%% (c^{2}ab + \alpha^{2}v) &= 0\\
%% (-c^{2}b^{2}+\alpha^{2}) &= 1.
%% \end{align}
%% \end{subequations}
%% This is 3 equations in 3 unknowns, which has a solution.
%% \begin{subequations}
%% \begin{align}
%% \alpha(v) &= \frac{-v/c}{\sqrt{1 - v^{2}/c^{2}}}\\
%% \intertext{and}
%% a = b &= \frac{1}{\sqrt{1 - v^{2}/c^{2}}}.
%% \end{align}
%% \end{subequations}

\N{Problem statement}
Given an inertial observer at rest, and another observer moving with
constant speed $v$, how do the coordinate change from the observer at
rest to the moving observer?

\N{Desiderata}
We want, for $v^{2}\ll c^{2}$, to recover Galilean transformations. That
is, $x' \approx x - vt$ (where primed coordinates are those relative to
the moving observer).

\M Let me try this again. We consider an inertial reference frame moving
with velocity $v$ in the $x$-direction relative to an inertial reference
frame at rest. Then the new coordinates are
\begin{subequations}
\begin{align}
ct' &= a_{11}ct + a_{12}x\\
x' &= a_{22}(x - vt).
\end{align}
\end{subequations}
This reduces to the familiar Galilean transformation for $v\ll c$, which
we demand $a_{22}\approx 1$. Now, for an infinitesimal displacement, its
components are
\begin{subequations}
\begin{align}
c\,\D t' &= a_{11}c\,\D t + a_{12}\,\D x\\
\D x' &= a_{22}(\D x - v\,\D t).
\end{align}
\end{subequations}
Demanding the invariance of the infinitesimal displacement is the same
as the condition:
\begin{equation}
-c^{2}(\D t')^{2} + (\D x')^{2} =-c^{2}(\D t)^{2} + (\D x)^{2}.
\end{equation}
Expanding the primed quantities gives us
\begin{equation}
-(a_{11}^{2}c^{2} - a_{22}^{2}v^{2})(\D t)^{2}
+ 2 (a_{22}^{2}v - a_{11}a_{12}c)\,\D x\,\D t
+ (a_{22}^{2} - a_{12}^{2})(\D x)^{2} =-c^{2}(\D t)^{2} + (\D x)^{2}.
\end{equation}
This gives us the system of 3 equations in 3 unknowns:
\begin{subequations}
\begin{align}
-(a_{11}^{2}c^{2} - a_{22}^{2}v^{2}) &= -c^{2}\\
2 (a_{22}^{2}v - a_{11}a_{12}c) &= 0\\
(a_{22}^{2} - a_{12}^{2}) &= 1.
\end{align}
\end{subequations}
We can solve this system:
\begin{subequations}
\begin{align}
a_{11} = a_{22} &= \frac{1}{\sqrt{1 - v^{2}/c^{2}}}\\
a_{12} &= \frac{-v/c}{\sqrt{1 - v^{2}/c^{2}}}.
\end{align}
\end{subequations}
The standard notation is
\begin{equation}
\gamma(v) = \frac{1}{\sqrt{1 - v^{2}/c^{2}}}
\end{equation}
and
\begin{equation}
\beta(v) = \frac{v}{c}.
\end{equation}
Then our transformation is
\begin{equation}
\begin{pmatrix}ct'\\ x'
\end{pmatrix}
=\begin{pmatrix}\gamma & -\gamma\beta\\
-\gamma\beta & \gamma
\end{pmatrix}
\begin{pmatrix}ct\\ x
\end{pmatrix}.
\end{equation}
This transformation is called a \define{Lorentz Boost}.

\M
Observe when $\beta=1/2$ that
\begin{equation}
\gamma = \frac{2}{\sqrt{3}}\approx 1.1547.
\end{equation}
In other words, for an inertial observer moving half the speed of light,
there is a deviation approximately 15\% from non-relativistic values.

At the Large Hadron Collider, protons are accelerated to speeds of about
$99.9999991\%$ the speed of light --- that is, $1-\beta\approx9\times10^{-9}$.
This gives us $\gamma\approx7453$.

\begin{definition}
We call a velocity \define{Ultra-Relativistic} when $1-\beta\ll1$.
\end{definition}

\begin{remark}
For ultra-relativistic velocities, we see that $1+\beta\approx2$, and
therefore
\begin{equation}
\gamma = \frac{1}{\sqrt{(1 + \beta)(1 - \beta)}}
\approx\frac{1}{\sqrt{2(1 - \beta)}}.
\end{equation}
Simple algebra gives us
\begin{equation}
1 - \beta\approx\frac{1}{2\gamma^{2}}.
\end{equation}
\end{remark}

\begin{exercise}
What is the error in the approximation
$1-\beta\approx(2\gamma^{2})^{-1}$ for $\beta=9/10$? For $\gamma=2$? For $\beta=99/100$?
\end{exercise}

\begin{exercise}
For $0\leq\beta\leq0.9$, find a linear approximation $L(\beta)$ for
$\gamma(\beta)$ such that $L(0)=\gamma(0)$ and
$L(0.9)=\gamma(0.9)$. What is the $L^{2}$-norm of the residual for this approximation?
Is there a better linear approximation?
\end{exercise}

\subsection{Light Cones and Causality}

\begin{definition}\label{defn:relativity:light-cone}
Let $a^{\mu}$ be any event. We define the \define{Lightcone} for $a^{\mu}$
to be the set of events light-like separated from $a^{\mu}$,
\begin{equation}
\mathscr{C} = \{\,b^{\mu}\in\RR^{3,1} \mid b^{\mu}~\mbox{is light-like separated from}~a^{\mu}\,\}.
\end{equation}
We can separate the light-cone in two: events in the future and events
in the past, giving us the \define{Future Light Cone}
\begin{equation}
\mathscr{C}^{+} = \{\,b^{\mu}\in\mathscr{C} \mid b^{0} > a^{0}\,\},
\end{equation}
and the \define{Past Light Cone} consisting of events preceding $a^{\mu}$,
\begin{equation}
\mathscr{C}^{-} = \{\,b^{\mu}\in\mathscr{C} \mid b^{0} < a^{0}\,\}.
\end{equation}
We can take the \define{Closed Light Cone} to consist of all light-like
separated events \emph{and} all time-like separated events, and denote
it by
\begin{equation}
\overline{\mathscr{C}} = \{\,b^{\mu}\in\RR^{3,1} \mid \eta_{\mu\nu}(b^{\mu}-a^{\mu})(b^{\nu}-a^{\nu})\leq0\,\}.
\end{equation}
\end{definition}

\M
The only possible causal influences for an event $a^{\mu}$
\emph{must lie within the past causal light cone} for the event, since
nothing can travel faster than light. For this reason, we call any
four-vector $b^{\mu}$ \define{Causal-like} if it is not space-like
separated from $a^{\mu}$, i.e., if the displacement $b^{\mu} - a^{\mu}$
four-vector is either time-like or light-like.

\subsection{World Lines and Trajectories}

\N{World lines}
A curve $\gamma$ in spacetime is called a \define{World Line}
if its tangent vector is future time-like at each point along the
curve. More generally, we could weaken the condition, and allow tangent
vectors to be causal-like.

Particles move along world lines in special relativity.

\M
If we have a curve $\gamma$ in spacetime such that its tangent vector is
space-like at each point along the curve, we call the curve
\define{Space-like}.

If we have a curve in spacetime such that its tangent vector is
ligh-like at each point along the curve, we call the curve
\define{Light-like}. 


\N{Proper Time}\label{chunk:relativity:proper-time}
The time between two events along a world line, according to the
observer moving along the trajectory, is precisely the
\define{Proper Time} and denoted $\tau$. For an infinitesimal
displacement along the trajectory, the infinitesimal change in proper
time is given by
\begin{equation}
c^{2}\,(\D\tau)^{2} = -(\D s)^{2}.
\end{equation}
The proper time interval along a trajectory is given by the integral
\begin{equation}
\Delta\tau = \int_{\gamma}\D\tau =
\int_{\gamma}\frac{\sqrt{-\eta_{\mu\nu}\,\D x^{\mu}\,\D x^{\nu}}}{c}.
\end{equation}

\N{Parametrizing World Lines}
We parametrize a world line by its proper time, and write it as
$x^{\mu}(\tau)$. This gives us a smooth family of four-positions for a
physical body.

Care must be taken when working with world lines for photons (or other
bodies moving at the speed of light), since $\D s = 0$ and therefore
$\D\tau=0$. We need to work with an affine parameter $\lambda$, but
physicists usually just reason along the lines of
\begin{equation}
(\D s)^{2} = 0 = -c^{2}\,(\D t)^{2} + (\D\vec{r})^{2}\implies c^{2}(\D t)^{2}=(\D\vec{r})^{2}.
\end{equation}
This is ``morally right'' but mathematically wrong.

\begin{definition}
Let $x^{\mu}(\tau)$ be a world line. We define its \define{Four-Velocity}
as the four-vector
\begin{equation*}
U^{\mu}(\tau) = \frac{\D x^{\mu}(\tau)}{\D\tau}.
\end{equation*}
\end{definition}

\begin{remark}
Whenever we are tempted to take the time derivative of a quantity, we
really want to take the derivative with respect to \emph{proper time}.
\end{remark}

\N{Lorentz Factor}
We have the familiar Newtonian 3-velocity given by
\begin{equation}
v^{i} = \frac{\D x^{i}}{\D t}.
\end{equation}
We can relate the coordinate time $x^{0}=ct$ with proper time $\tau$ by
the \define{Lorentz Factor} (which is a function of the Newtonian 3-velocity),
\begin{equation}
  \begin{split}
\gamma(\vec{v}) &= \frac{\D t}{\D\tau} = \left(1 - \frac{\eta_{ij}v^{i}v^{j}}{c^{2}}\right)^{-1/2}\\
&=\frac{c}{\sqrt{-\eta_{\mu\nu}U^{\mu}U^{\nu}}}.
  \end{split}
\end{equation}
This allows us to relate the familiar Newtonian 3-velocity
to the four-velocity by
\begin{equation}
U^{\mu} = (c, \gamma(\vec{v})\vec{v}).
\end{equation}

\N{Four-Acceleration}
We can define the four-acceleration analogous to how we defined the
four-velocity as
\begin{equation}
A^{\mu} := \frac{\D U^{\mu}}{\D\tau}.
\end{equation}

\begin{definition}
The \define{Instantaneous Rest Frame} is the frame in which the 3-speed vanishes $v=0$.
\end{definition}

\M
In the instantaneous rest frame we have $v=0$ so $\gamma=1$. The
four-velocity has components $V^{\mu}=(c,\vec{0})$ and the
four-acceleration has components $A^{\mu}=(0,\vec{a})$. Then:
\begin{enumerate}
\item $\eta_{\mu\nu}A^{\mu}A^{\nu}=a^{2}$ where $a$ is the acceleration
  measured in the instantaneous rest frame (that is, the acceleration
  felt by the body);
\item $\eta_{\mu\nu}U^{\mu}U^{\nu}=c^{2}$;
\item $\eta_{\mu\nu}A^{\mu}U^{\nu}=0$ (which can be obtained by
  differentiating the previous result with respect to $\tau$).
\end{enumerate}

\begin{exercise}
Let $v$ be the Newtonian speed of an object, $v = |\D\vec{x}/\D t|$.
Prove
\begin{equation*}
\frac{\D\gamma}{\D t} = \frac{\gamma^{3}v}{c^{2}}\frac{\D v}{\D t}.
\end{equation*}
\end{exercise}

\begin{exercise}
Let $X^{\mu}$ be the position 4-vector, $t$ be coordinate time. Prove
the 4-acceleration may be written as:
\begin{equation*}
A^{\mu} = \gamma\frac{\D}{\D t}\left(\gamma\frac{\D}{\D t}X^{\mu}\right)
\end{equation*}
\end{exercise}

\N{Constant Acceleration}
Consider a particle moving along the $x$-axis with constant acceleration
$a$ as measured in its rest frame at each point.

Let $U^{\mu}=(c\dot{t},\dot{x},0,0)$ where $\dot{x}=\D x/\D\tau$ and
$\dot{t}=\D t/\D\tau$. Then $A^{\mu}=(c\ddot{t},\ddot{x},0,0)$. We have
\begin{subequations}
\begin{align}
-c^{2} &= \eta_{\mu\nu}U^{\mu}U^{\nu} = -c^{2}\dot{t}^{\,2} + \dot{x}^{2}\\
\intertext{and}
a^{2} &= \eta_{\mu\nu}A^{\mu}A^{\nu} = -c^{2}\ddot{t}^{\,2} + \ddot{x}^{2}.
\end{align}
\end{subequations}
Differentiating the first equation gives us $c^{2}\dot{t}\ddot{t}=\dot{x}\ddot{x}$,
then square both sides
$c^{4}\dot{t}^{\,2}\ddot{t}^{\,2}=\dot{x}^{2}\ddot{x}^{2}$, and then
we can eliminate $\dot{x}^{2}=c^{2}\dot{t}^{2}-c^{2}$ and
$\ddot{x}^{\,2}=a^{2}+c^{2}\ddot{t}^{\,2}$ to give us
\begin{equation}
c^{4}\dot{t}^{\,2}\ddot{t}^{\,2}=(c^{2}\dot{t}^{\,2}-c^{2})(a^{2}+c^{2}\ddot{t}^{\,2}).
\end{equation}
We subtract $c^{2}\ddot{t}^{\,2}(\dot{t}^{\,2}-1)$ from both sides to
obtain
\begin{equation}
c^{2}\ddot{t}^{\,2} = a^{2}(\dot{t}^{\,2} - 1)\implies\ddot{t}=\frac{a}{c}\sqrt{\dot{t}^{\,2} - 1}.
\end{equation}
We can solve this differential equation with the initial condition
$t(\tau=0)=0$ and $x(\tau=0)=0$, first integrating to find
\begin{equation}
\dot{t} = \cosh(a\tau/c),
\end{equation}
then integrating again to find
\begin{equation}
t = \frac{c}{a}\sinh(a\tau/c).
\end{equation}
Then from $\dot{x}^{2} = c^{2}(\dot{t}^{\,2}-1)$ we find $\dot{x} = c\sqrt{\cosh^{2}(a\tau/c)-1}=c\sinh(a\tau/c)$
which can be integrated to give us
\begin{equation}
x = \frac{c^{2}}{a}\cosh(a\tau/c).
\end{equation}

\begin{remark}
  Observe that the magnitude of the spatial components for the
  4-velocity (the ``Newtonian speed'') is:
  \begin{equation}
v(\tau) = \frac{\D x}{\D t} = \frac{\dot{x}}{\dot{t}} = c\tanh(a\tau/c)\leq c.
  \end{equation}
\end{remark}

\N{Example}
How do we interpret this? Well, if a space ship leaves a planet at
constant acceleration, and the planet is at rest in the $(x,t)$
coordinates at $x=c^{2}/a$ at $t=0$, then afer some proper time $\tau$
has elapsed the space ship will be located at $x=(c^{2}/a)\cosh(a\tau/c)$.
The time elapsed relative to a clock on the planet will read $t=(c/a)\sinh(a\tau/c)$
where $\tau$ is the time elapsed relative to a clock on the space ship.
For $a\approx g$ and $\tau\approx 10~\mbox{years}$, then $(c/a)=1.03092~\mbox{year}\approx 1~\mbox{year}$
and
\begin{equation}
t\approx\frac{1}{2}\exp(10)~\mbox{years}\approx 11,000~\mbox{years}.
\end{equation}
Observe that
\begin{equation}
x\approx\frac{1}{2}\exp(10)~\mbox{light years}\approx11,000~\mbox{light years}.
\end{equation}
Thus, although only 10 years elapsed on the space ship moving at 1~g,
relative to clocks on the planet which launched the space ship it would
appear to have been 11,000 years. In Newtonian mechanics, the space ship
would be about $50$ light years away (since $a\approx c~\mbox{year}^{-1}$).

\begin{remark}
This may seem counter-intuitive (and it is), but we should emphasize it
is difficult to have a space ship move at a constant 1~g acceleration.
Indeed, \emph{constant acceleration} is the bit belonging to science
fiction (at the moment).
\end{remark}

\begin{exercise}
In Robert Rath's \textit{The Infinite and the Divine}, a super-advanced
species of aliens (who existed hundreds of millions of years ago)
transferred their consciousness into robots. This problem concerns one
of their spaceship's trajectory using their advanced technology.

We are told the initial velocity of the spaceship is $1000$ leagues per
hour, but after a decade [literally 10 years] the spaceship reaches a
billion [$1.00\times10^{9}$] leagues per hour. Note: 1 league per hour
is approximately 1.54333 meters per second.
\begin{enumerate}
\item Assume constant acceleration. Find the constant acceleration,
  assuming the universe is Newtonian, and forces acting on the spaceship
  are negligible.
\item Using the solution to the previous problem, how far did the
  spaceship travel? The story suggests it is the distance between
  several stars. If there is 5 light years between stars in the galaxy,
  is this reasonable? If not, how fast would the spaceship have to
  travel?
\item If we allow for special relativity to apply, then time dilation
  presumably occurs. Suppose the ten years in our problem refers to the
  proper time as measured by the ship's chronometer [clock]. Does this
  affect the distance the ship travels? If so, how far does this ship
  travel?
\end{enumerate}
\end{exercise}

\N{References}
For accelerated observers, Chapter 6 of Misner, Thorne, and
Wheeler~\cite{Misner:1973prb} is quite good.

\endinput