\section{Dynamics}

\M
Often the dynamics of particles are omitted in discussions of special
relativity, because things getting complicated conceptually.

\begin{definition}
Let $x^{\mu}(\tau)$ be the world line for a massive body.
Then the \define{Rest Mass} (or \emph{invariant mass}) for the body is
the mass $m_{0}$ as measured in its reference frame.
\end{definition}

\begin{remark}
Some authors use a notion of \emph{relativistic mass}
$m = \gamma(\vec{v})m_{0}$, which is mildly controversial. 
\end{remark}

\N{Four-Momentum}\label{chunk:relativity:four-momentum}
For a massive particle of rest mass $m_{0}$, its \define{Four-Momentum}
$\vec{P}$ is defined as the product of the rest mass and its
four-velocity, i.e.,
\begin{equation}
\vec{P} := m_{0}\vec{U}.
\end{equation}
We can write out its explicit components
\begin{equation}
\vec{P} = m_{0}\gamma(\vec{v})(c, \vec{v}) = (E/c, \vec{p}).
\end{equation}
Here the total energy of the moving particle is given by
\begin{equation}
E = \gamma(\vec{v})m_{0}c^{2},
\end{equation}
and the total (relativistic) 3-momentum is
\begin{equation}
\vec{p} = m_{0}\gamma(\vec{v})\vec{v}.
\end{equation}

\N{Energy--Momentum Relation}\label{chunk:relativity:mass-shell-relation}
We have the energy--momentum relation (or \define{Mass--Shell Relation})
be
\begin{equation}
E^{2} = c^{2}\vec{p}\cdot\vec{p} + \bigl(m_{0}c^{2}\bigr)^{2}.
\end{equation}
Equivalently, we have this relation describe the magnitude of the
four-momentum as a constant:
\begin{equation}
\eta_{\mu\nu}p^{\mu}p^{\nu} = -m_{0}^{2}c^{2}.
\end{equation}

\begin{ddanger}
In special relativity, we can meaningfully talk about the Center-of-Mass
reference frame for a system of particles. This is especially useful for
scattering problems. However, if we tried to carry this notion over to
General Relativity, then we run into problems because it is a nonlocal
concept. 
\end{ddanger}

\begin{exercise}
From the mass-shell relation and $\vec{P} = (E/c, \vec{p})$, deduce
$E = \gamma(\vec{v})m_{0}c^{2}$ and 
$\vec{p}\cdot\vec{p}=\pm(\gamma(\vec{v})^{2}-1)m_{0}^{2}c^{2}$.
Determine the correct sign in that second relation.
\end{exercise}

\begin{exercise}
Suppose we have two massive particles with four-momenta $\vec{P}_{1}$
and $\vec{P}_{2}$ and relative speed $v$. Determine
$\vec{P}_{1}\cdot\vec{P}_{2}=\eta_{\mu\nu}P^{\mu}_{(1)}P^{\nu}_{(2)}$ in
terms of their rest mass $m_{01}$ and $m_{02}$ and relative speed $v$.
Hint: if $\vec{P}_{1}=\vec{P}_{2}$ and $v=0$, then you should recover
the mass--shell relation.
\end{exercise}

\N{Four-Force}
We can define the four-force as the four-vector
\begin{equation}
\vec{F} = \frac{\D\vec{P}}{\D\tau}.
\end{equation}
As an immediate consequence of the mass--shell relation, we find
\begin{equation}
\vec{F}\cdot\vec{P} = \eta_{\mu\nu}F^{\mu}P^{\nu} = 0.
\end{equation}

\N{Lagrangian for Point-Particle}\label{chunk:relativity:lagrangian-for-point-particle}
We can write the Lagrangian for a massive point-particle with rest mass
$m_{0}$ as
\begin{equation}
L = -cm_{0}\sqrt{-\eta_{\mu\nu}\dot{x}^{\mu}\dot{x}^{\nu}} - V
\end{equation}
where $\dot{x}^{\mu} = \D x^{\mu}/\D\tau = U^{\mu}$, and $V$ is the potential
energy term. The action is then 
\begin{equation}
I = \int L\,\D\tau.
\end{equation}
Varying the action with respect to $\delta x^{\mu}$ then gives us the
equations of motion.

For massless particles, care must be taken with the parametrization, as
well as using its four-momentum to write
\begin{equation}
L = -c\sqrt{-\eta_{\mu\nu}P^{\mu}P^{\nu}} - V.
\end{equation}
We can use the relation (which holds for both massive and massless particles):
\begin{equation}
\frac{\D x^{\mu}}{\D t} = \frac{P^{\mu}}{P^{0}}.
\end{equation}

\begin{danger}
This is the correct Lagrangian to work with, especially if we want to
quantize it. There is some subtlety with it, which we should confess
openly: it is a constrained system. To see this, compute the Hamiltonian
for a free massive relativistic particle. It will vanish. This is
because time is a coordinate (proper time is a parameter), and its
conjugate momentum is ``the Hamiltonian''. So we end up with a
constraint. 
\end{danger}

\begin{ddanger}
Some authors insist that canonical mechanics for special relativistic
systems ``breaks Lorentz invariance'', which is not really true. If
you've picked $\mu=0$ to be the time component for four-vectors, then
you've also ``broken Lorentz invariance'' just as much as canonical
mechanics has. We can describe a Lorentz boost as a canonical
transformation (which preserves Lorentz invariance as much as anything
else). This is just offered as a lazy and sloppy justification for using
the path integral formalism, which makes no coherent sense.
\end{ddanger}

\begin{exercise}\label{xca:relativity:canonical-analysis-of-free-particle}
Work out the canonical analysis for the Lagrangian of a point-particle
in special relativity. Recall, the steps are:
\begin{enumerate}
\item Find the conjugate momenta $p_{\mu}$ for $\dot{x}^{\mu}=\D x^{\mu}/\D\tau$
\item Find any primary constraints
\item Rewrite the Lagrangian in terms of $x^{\mu}$ and $p_{\mu}$
\item Write the Hamiltonian $H = p_{\mu}\dot{x}^{\mu}-L$.
\item Find if there are any first-class constraints.
\item Write down the extended Hamiltonian (or total Hamiltonian).
\end{enumerate}
\end{exercise}

\endinput