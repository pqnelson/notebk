\chapter{Mathematical Results}

\section{Integrals}

\begin{theorem}[Gaussian Integral]
$\displaystyle\int^{\infty}_{-\infty}\E^{-x^{2}}\,\D x = \sqrt{\pi}$.
\end{theorem}

\begin{proof}
  Let
  \begin{equation}
I = \int^{\infty}_{-\infty}\E^{-x^{2}}\,\D x.
  \end{equation}
  Then
\begin{calculation}
  I^{2}
  \step{unfold}
\int^{\infty}_{-\infty} \int^{\infty}_{-\infty}\E^{-x^{2}-y^{2}}\,\D x\,\D y
\step{change to polar coordinates}
\int^{\infty}_{0}\int^{2\pi}_{0}\E^{-r^{2}}r\,\D\theta\,\D r
\step{since integrand is constant with respect to $\theta$}
2\pi\int^{\infty}_{0}\E^{-r^{2}}r\,\D r
\step{change coordinates with $u=r^{2}$, $\D u = 2r\,\D r$}
\pi\int^{r=\infty}_{r=0}\E^{-u}\,\D u
\step{simple integration}
\left.-\pi\E^{-r^{2}}\right|^{r=\infty}_{r=0}
\step{substitution}
-\pi\E^{-\infty}+\pi\E^{-0} = \pi.
\end{calculation}
Therefore $I=\pm\sqrt{\pi}$. Since the integrand is always positive, we
conclude $I=\sqrt{\pi}$ as desired.
\end{proof}

\begin{corollary}\label{cor:math:general-gaussian-integral-in-one-dim}
Let $a\in\RR$ and $b\in\CC$. Then $\displaystyle\int^{\infty}_{-\infty}\E^{-a(x-b)^{2}}\,\D x = \sqrt{\frac{\pi}{a}}$
\end{corollary}

\begin{remark}
We can let $a\in\CC$ provided its real part is positive (so $\Re(-ax^{2})\leq0$).
\end{remark}

\begin{corollary}
Let $a\in\RR$ be positive $a>0$, and $b$, $c\in\CC$. Then $\displaystyle\int^{\infty}_{-\infty}\E^{-ax^{2}+bx+c} \,\D x = \sqrt{\frac{\pi}{a}}\exp\left(\frac{b^{2}}{4a}+c\right)$.
\end{corollary}

\begin{corollary}
Let $a\in\RR$ be positive $a>0$, $n\in\NN$. Then
\begin{equation*}
\int^{\infty}_{-\infty}x^{2n}\E^{-ax^{2}/2}\,\D x = \sqrt{\frac{2\pi}{a}}\frac{(2n-1)!!}{a^{n}}.
\end{equation*}
The odd moments all vanish.
\end{corollary}

\begin{proof}
  By induction on $n\in\NN$. We see the base case $n=1$ is obtained from:
  \begin{calculation}
    \int^{\infty}_{-\infty}x^{2}\E^{-ax^{2}/2}\,\D x
\step{differentiating under the integral sign}
    -2\frac{\D}{\D a}\int^{\infty}_{-\infty}\E^{-ax^{2}/2}\,\D x
\step{using the Gaussian integral result}
    -2\frac{\D}{\D a}\left(\frac{2\pi}{a}\right)^{1/2}
\step{power rule for derivatives}
    -2\frac{-1}{2}\frac{1}{a}\left(\frac{2\pi}{a}\right)^{1/2}
\step{algebra}
    \frac{1}{a}\left(\frac{2\pi}{a}\right)^{1/2}.
  \end{calculation}
  This establishes the result.

  The inductive hypothesis is that, for arbitrary $n\in\NN$, 
\begin{equation}
\int^{\infty}_{-\infty}x^{2n}\E^{-ax^{2}/2}\,\D x = \sqrt{\frac{2\pi}{a}}\frac{(2n-1)!!}{a^{n}}.
\end{equation}
Now we prove the $n+1$ case
\begin{calculation}
\int^{\infty}_{-\infty}x^{2(n+1)}\E^{-ax^{2}/2}\,\D x
\step{rewrite using $x^{2(n+1)}=x^{2n}x^{2}$}
\int^{\infty}_{-\infty}x^{2}x^{2n}\E^{-ax^{2}/2}\,\D x
\step{using differentiation under the integral sign}
-2\frac{\D}{\D a}\int^{\infty}_{-\infty}x^{2n}\E^{-ax^{2}/2}\,\D x
\step{using the inductive hypothesis}
-2\frac{\D}{\D a}\left(\sqrt{\frac{2\pi}{a}}\frac{(2n-1)!!}{a^{n}}\right)
\step{using the derivative of $a^{-(2n+1)/2}$}
-2\frac{-(2n+1)}{2}\frac{1}{a}\left(\sqrt{\frac{2\pi}{a}}\frac{(2n-1)!!}{a^{n}}\right)
\step{since $(2n+1)!! = (2n + 1)\cdot (2n-1)!!$}
-2\frac{-1}{2}\frac{1}{a}\left(\sqrt{\frac{2\pi}{a}}\frac{(2n+1)!!}{a^{n}}\right)
\step{simple algebra}
\left(\sqrt{\frac{2\pi}{a}}\frac{(2n+1)!!}{a^{n+1}}\right).
\end{calculation}
Hence the result.
\end{proof}

\begin{theorem}
Let $n\in\NN$, let $A$ be a symmetric positive-definite $n\times n$ matrix.
Then
\begin{equation*}
\int_{\RR^{n}}\exp\left(\frac{-1}{2}\sum^{n}_{i,j=1}A_{i,j}x_{i}x_{j}\right)\D^{n}x
=\sqrt{\frac{(2\pi)^{n}}{\det(A)}}.
\end{equation*}
\end{theorem}

\begin{theorem}
Let $n\in\NN$, let $A$ be a symmetric positive-definite $n\times n$ matrix,
let $\vec{b}$ be an $n$-vector (of constants).
Then
\begin{equation*}
\int_{\RR^{n}}\exp\left(\frac{-1}{2}\sum^{n}_{i,j=1}A_{i,j}x_{i}x_{j}+\sum^{n}_{i}b_{i}x_{i}\right)\D^{n}x
=\sqrt{\frac{(2\pi)^{n}}{\det(A)}}\exp\left(\frac{1}{2}\transpose{\vec{b}}A^{-1}\vec{b}\right).
\end{equation*}
\end{theorem}

\section{Linear Algebra}

\begin{theorem}
Let $\mathcal{H}$ be a Hilbert space and $L\colon \mathcal{H}\to\mathcal{H}$
be a self-adjoint operator. Then:
\begin{enumerate}
\item The eigenvalues of $L$ are all real;
\item The eigenvectors of $L$ form a complete set;
\item The normalized eigenvectors of $L$ are orthonormal.
\end{enumerate}
\end{theorem}

\section{Dirac Delta Function}

\begin{theorem}
For any non-zero scalar $\alpha\neq0$, we have
\begin{equation}
\delta(\alpha x) = \frac{\delta(x)}{|\alpha|}.
\end{equation}
\end{theorem}

\begin{proof}
By direct calculation,
\begin{calculation}
\int^{\infty}_{-\infty}\delta(\alpha x)\,\D x
\step{change of variables, $u=\alpha x$, $\D u=\alpha\,\D x$}
\int^{\infty}_{-\infty}\delta(u)\frac{\D u}{|\alpha|}
\step{defining property of delta function}
\frac{1}{|\alpha|}.\qedhere
\end{calculation}
\end{proof}

\begin{lemma}
Let $g$ be continuously differentiable, assume $g'(x)$ is nowhere zero,
and there is only one simple root to $g$ at $g(x_{0})=0$. Then
\begin{equation}
\delta(g(x)) = \frac{\delta(x-x_{0})}{|g'(x_{0})|}.
\end{equation}
\end{lemma}

\begin{theorem}\label{thm:math:delta-function-of-polynomials}
Let $g$ be continuously differentiable and $g'(x)$ is nowhere zero.
Then
\begin{equation}
\delta(g(x)) = \sum_{\{x_{0}\mid g(x_{0})=0\}}\frac{1}{|g'(x_{0})|}\delta(x-x_{0}),
\end{equation}
where the sum extends to all distinct (non-repeated) simple roots.
\end{theorem}

\begin{corollary}\label{cor:math:dirac-delta-function-of-quadratic-polynomial}
  Let $\alpha$ be a nonzero scalar. Then we have
  \begin{equation}
\delta(x^{2} - \alpha^{2}) = \frac{1}{2|\alpha|}\left[\delta(x+\alpha)+\delta(x-\alpha)\right].
  \end{equation}
\end{corollary}