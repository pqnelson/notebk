\chapter*{Preface}
\addcontentsline{toc}{chapter}{Preface}

\epigraph{In learning the sciences\\
  examples are of more use than precepts.}%
{Isaac Newton, \textit{Arithmetica Universalis} (1707)}

\M
This is the book I wish I had when learning quantum field theory.
It's probably not a good introduction to the subject, since the
conventions are idiosyncratic and subject material oriented towards
quantum gravity (rather than particle physics).

My interests have shifted over time, and I add material which piques my
interest at the time of writing.

\M
For mathematicians, the thing to bear in mind is Newton's words:
examples are more important than theorems. There are three families of
examples (scalar fields, spin-$1/2$ fields, and vector [gauge] fields).

\N{``Suitably Nice''}
There are a number of instances where I will describe a mathematical
gadget as ``suitably nice''. By this, I mean it is sufficiently smooth,
or decays sufficiently fast, or some other property. When the property
is relevant, I will say ``suppse $\langle$gadget$\rangle$ is suitably smooth''
or something similar. Mathematicians will find these mentions useful
reminders, physicists will find this an eccentricity.

\N{Conventions}
Since I'm working with an eye towards quantum gravity, I use the
relativist's metric signature $-+++$. This differs from almost every
other QFT book (I stand in the company of Weinberg and Srednicki), and
it simplifies the representation theory of Spin groups. Most physicists
end up multiplying the generators of the Spin group by $\I=\sqrt{-1}$,
only to avoid mathematical difficulties in the $+---$ metric
signature. But this is not really well motivated from the mathematical
perspective, and causes needless complication. I prefer being an honest
broker.

\N{Unit System}
Most QFT texts also work in the geometrized units $\hbar=c=1$ which
means $c=1~L/T$ and $\hbar=1~ML^{2}/T$ in where $L$ is the unit of
length, $T$ is the unit of time, and $M$ is the unit of mass. This is
used to simplify formulas, since $c=1$ means $L=T$, and then $\hbar=1$
means $M=L^{-1}$. In particular, $c=1$ means $c$ is dimensionless,
$\hbar=1$ means $\hbar$ is dimensionless.

I try to be an honest broker about things, and preserve the constants.

\N{Starred sections}
Starred sections are optional ``fluff'' or background, which is
interesting but tangential. They may be skipped on first reading.

\N{Exercises}
The exercises range from straightforward ``Check the dimensions of
[expression]'' to calculations. Sometimes research will be involved.
Other times they task the reader to explore certain ``edge cases''
which would be too much of a diversion from the main narration.

Sometimes I will give an ``oriented set of exercises'' which walk
through some discussion, but leaves the calculations to the
reader. Usually this breaks down the discussion into an enumerated list
of points to make, each involving some calculation depending on the
previous points. These exercises enrich the soul and the reader's
understanding, but may be skipped.

\N{About ``Gauge Theory''}
Physicists tend to use the term ``gauge theory'' as synonymous with
Yang--Mills theory, which is like saying a chef only makes sandwiches.
Gauge theory is more general than \emph{just} Yang--Mills, and I'm an
honest broker when it comes to the topic. This will involve constraints
and so on. I defer to Henneaux and Teitelboim~\cite{Henneaux:1992ig}
among others on this topic.

For gauge theory, like all of physics, examples are as important as
theorems. 
