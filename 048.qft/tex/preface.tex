\chapter*{Preface}
\addcontentsline{toc}{chapter}{Preface}

\M
This is the book I wish I had when learning quantum field theory.
It's probably not a good introduction to the subject, since the
conventions are idiosyncratic and subject material oriented towards
quantum gravity (rather than particle physics).

My interests have shifted over time, and I add material which piques my
interest at the time of writing.

\N{Conventions}
Since I'm working with an eye towards quantum gravity, I use the
relativist's metric signature $-+++$. This differs from almost every
other QFT book (I stand in the company of Weinberg and Srednicki), and
it simplifies the representation theory of Spin groups. Most physicists
end up multiplying the generators of the Spin group by $\I=\sqrt{-1}$,
only to avoid mathematical difficulties in the $+---$ metric
signature. But this is not really well motivated from the mathematical
perspective, and causes needless complication. I prefer being an honest
broker.

\N{Unit System}
Most QFT texts also work in the geometrized units $\hbar=c=1$ which
means $c=1~L/T$ and $\hbar=1~ML^{2}/T$ in where $L$ is the unit of
length, $T$ is the unit of time, and $M$ is the unit of mass. This is
used to simplify formulas, since $c=1$ means $L=T$, and then $\hbar=1$
means $M=L^{-1}$. In particular, $c=1$ means $c$ is dimensionless,
$\hbar=1$ means $\hbar$ is dimensionless.

I try to be an honest broker about things, and preserve the constants.

\N{Starred sections}
Starred sections are optional ``fluff'' or background, which is
interesting but tangential. They may be skipped on first reading.
