\section{Scalar Field}

\M
We have the Klein--Gordon field\index{Klein--Gordon!Field}
be described by the Lagrangian in
Eq~\eqref{eq:classical-field-theory:linear-chain:continuum-limit:mass-term:lagrangian}.
More generally, since we could add an arbitrary potential term
$V(\varphi)$, the Lagrangian looks like:
\begin{equation}\label{eq:classical-field-theory:scalar-field:lagrangian}
\begin{split}
  \mathcal{L} &= -\frac{1}{2}\partial^{\mu}\varphi\partial_{\mu}\varphi
-\frac{1}{2}\frac{m^{2}c^{2}}{\hbar^{2}}\varphi^{2} - V(\varphi)\\
&=\frac{1}{2}c^{-2}(\partial_{t}\varphi)^{2}-\frac{1}{2}(\nabla\varphi)^{2}
-\frac{1}{2}\frac{m^{2}c^{2}}{\hbar^{2}}\varphi^{2} - V(\varphi).
\end{split}
\end{equation}
Our physical intuition should be that of a linear chain of
point-particles connected by massless identical springs, as we have
analyzed in the previous section.

\begin{exercise}
Can we interpret the $\frac{1}{2}(\nabla\varphi)^{2}$ term in the
Lagrangian as a contribution to the potential? If so, what is its
physical interpretation?
\end{exercise}

\begin{definition}\index{Klein--Gordon!Field}\index{Field!Scalar!Free}\index{Scalar Field!Free}
When $V=0$ in the Lagrangian, we call the type of field a
\define{Klein--Gordon Field} (or \emph{Free Scalar Field}).
\end{definition}

\begin{remark}
We could have $V(\varphi) = a + b\varphi + c\varphi^{2}$ and reabsorb
$a$, $b$, $c$ into the mass and elsewhere in the Lagrangian, producing
an equivalent free scalar field. And physicists will be a little sloppy
in their language, referring to such Lagrangians with
nonzero-but-quadratic potentials as ``free''.
\end{remark}

\begin{definition}
When $V(\varphi)\neq0$ (and specifically $V(\varphi)$ is not a constant,
or a linear or quadratic polynomial), we say we have a
\define{Self-Interacting Scalar Field}.
\end{definition}

\begin{definition}
When we have a linear $\varphi$ term in the Lagrangian's potential term
(something like $\sigma\varphi$), we refer to it as an
\define{External Source}.
\end{definition}

\begin{remark}
This terminology may seem bizarre at first, but it generalizes the
4-current which we used to recover Maxwell's equations (\S\ref{chunk:relativity:electromagnetism:recovering-maxwell-equations}).
The 4-current captured our intuition of coupling electromagnetism to
matter (``charged particles''), and the external source couples the
field to ``generic matter''.

We will also find, however, that it's useful to stick in an external
source into the Lagrangian for mathematical purposes. It's a
mathematical trick where we will differentiate with respect to external
sources to compute moments of integrals, then set the external source to
zero. 
\end{remark}

\N{Goals}
We will study the \emph{free} scalar field in this section. 

\N{Variational Analysis}
We consider a region $\mathcal{R}\subset\RR^{3,1}$, usually taken to be
$\RR^{3}\times[t_{1},t_{2}]$ for some $t_{1}<t_{2}$. Now we will
consider the action
\begin{equation}
\action[\varphi] = \int_{\mathcal{R}}\left(\frac{1}{2}c^{-2}(\partial_{t}\varphi)^{2}-\frac{1}{2}(\nabla\varphi)^{2}
-\frac{1}{2}\frac{m^{2}c^{2}}{\hbar^{2}}\varphi^{2}\right)\D^{3}\vec{x}\,\D t.
\end{equation}
As usual, we will try to find the critical points of the action, and
argue these are the physical solutions to the equations of motion.

Now, we take a variation of this action with respect to $\varphi$. This
is done by writing
\begin{equation}
\varphi_{\lambda} = \varphi + \lambda\psi,
\end{equation}
where $\lambda$ is a real parameter, $\psi|_{\partial\mathcal{R}}=0$ is
an (otherwise) arbitrary function. Physicists write
\begin{equation}
\delta\varphi = \lambda\psi,
\end{equation}
and pretend $\lambda$ is an infinitesimal quantity $\lambda^{2}\ll1$.
Then expanding the integrand in the action to first-order in $\lambda\psi$,
we demand the coefficient to $\lambda\psi$ vanish. We find,
\begin{calculation}
\action[\varphi_{\lambda}]
\step{plugging in the definition of the action}
\int_{\mathcal{R}}\left(\frac{1}{2}c^{-2}(\partial_{t}\varphi_{\lambda})^{2}-\frac{1}{2}(\nabla\varphi_{\lambda})^{2}
-\frac{1}{2}\frac{m^{2}c^{2}}{\hbar^{2}}\varphi_{\lambda}^{2}\right)\D^{3}\vec{x}\,\D t
\step{unfold the definition of $\varphi_{\lambda}$}
\int_{\mathcal{R}}\left(\frac{1}{2}c^{-2}(\partial_{t}[\varphi + \lambda\psi])^{2}-\frac{1}{2}(\nabla[\varphi + \lambda\psi])^{2}
-\frac{1}{2}\frac{m^{2}c^{2}}{\hbar^{2}}[\varphi + \lambda\psi]^{2}\right)\D^{3}\vec{x}\,\D t
%% \step{expand}
%% \int_{\mathcal{R}}\left(\frac{1}{2}c^{-2}
%% (\partial_{t}\varphi)^{2}+c^{-2}\partial_{t}\varphi\partial_{t}\psi+\frac{1}{2}c^{-2}(\partial_{t}\psi)^{2}
%% -\frac{1}{2}(\nabla\varphi)^{2}
%% -(\nabla\varphi)\cdot(\nabla(\lambda\psi))
%% -\frac{1}{2}(\nabla(\lambda\psi))\cdot(\nabla(\lambda\psi))
%% -\frac{1}{2}\frac{m^{2}c^{2}}{\hbar^{2}}[\varphi^{2} + 2\lambda\psi\varphi + \lambda^{2}\psi^{2}]\right)\D^{3}\vec{x}\,\D t
\step{expanding and collecting coefficients of $\lambda$}
\action[\varphi] + \int_{\mathcal{R}}\left(
c^{-2}\partial_{t}\varphi\partial_{t}(\lambda\psi)
-(\nabla\varphi)\cdot(\nabla(\lambda\psi))
-\frac{m^{2}c^{2}}{\hbar^{2}}\varphi\cdot(\lambda\psi)\right)\D^{3}\vec{x}\,\D t
+\action[\lambda\psi].
\end{calculation}
We write the first variation of the action as:
\begin{equation}
\delta\action[\varphi_{\lambda}] = \int_{\mathcal{R}}\left(
c^{-2}\partial_{t}\varphi\partial_{t}(\lambda\psi)
-(\nabla\varphi)\cdot(\nabla(\lambda\psi))
-\frac{m^{2}c^{2}}{\hbar^{2}}\varphi\cdot\lambda\psi\right)\D^{3}\vec{x}\,\D t.
\end{equation}

\N{Initial conditions}
We need to specify the initial and final configuration for the scalar
field, so for $\mathcal{R}=\mathcal{R}_{3}\times[t_{1},t_{2}]\subset\RR^{3,1}$,
we need functions
\begin{equation}
\varphi_{1},\varphi_{2}\colon\mathcal{R}_{3}\to\RR,
\end{equation}
such that
\begin{equation}
\varphi|_{\mathcal{R}_{3}\times\{t_{1}\}}=\varphi_{1},\quad\mbox{and}\quad
\varphi|_{\mathcal{R}_{3}\times\{t_{2}\}}=\varphi_{2}.
\end{equation}

\M
We can integrate $\delta\action[\varphi_{\lambda}]$ by parts (with
respect to time) to get
\begin{equation}
\delta\action[\varphi_{\lambda}] = \int_{\mathcal{R}}\left(
-c^{-2}(\lambda\psi)\partial_{t}^{2}\varphi
-(\nabla\varphi)\cdot(\nabla(\lambda\psi))
-\frac{m^{2}c^{2}}{\hbar^{2}}\varphi\cdot\lambda\psi\right)\D^{3}\vec{x}\,\D t.
\end{equation}
The boundary terms from this integration-by-parts vanishes since
$\lambda\psi|^{t_{2}}_{t_{1}}=0$.

\M
When we have $\mathcal{R}=\RR^{3}\times[t_{1},t_{2}]$, we treat
$\RR^{3}$ as a sphere with radius $r\to\infty$. Doing so allows us to
use the divergence theorem to further rewrite the first variation of the
action as, when $\vec{n}$ is the outward-pointing unit normal vector,
\begin{equation}
\begin{split}
\delta\action[\varphi_{\lambda}] &= \int_{\mathcal{R}}\left(
-c^{-2}(\lambda\psi)\partial_{t}^{2}\varphi
+(\lambda\psi)\nabla^{2}\varphi
-\frac{m^{2}c^{2}}{\hbar^{2}}\varphi\cdot\lambda\psi\right)\D^{3}\vec{x}\,\D t\\
&\qquad-\int^{t_{2}}_{t_{1}}\int_{r\to\infty}(\lambda\psi)\vec{n}\cdot(\nabla\varphi)%
\,\D A\,\D t,
\end{split}
\end{equation}
where $\D A$ is the differential area for the sphere. (In $n+1$
dimensions, $\D A\sim r^{n-2}\,\D r$.)
We have implicitly used the fact that
\begin{equation}
(\nabla\varphi)\cdot\nabla(\lambda\psi)=\nabla\cdot((\nabla\varphi)\lambda\psi)
-(\lambda\psi)\nabla^{2}\varphi,
\end{equation}
before using the divergence theorem.

\N{Assumption on growth of field}
We need to assume that, at a fixed time $t$, the scalar field $\varphi$
(and therefore both $\varphi_{\lambda}$ and $\lambda\psi$) fall off as
$r\to\infty$ sufficiently fast so as to make the boundary term in the
first variation of the action $\delta\action$ vanishes. One possibility
is to work with fields with compact support.

The usual assumption physicists make is that, since the area eleement
$\D A$ grows like $r^{2}$, the integrand must fall faster than $1/r^{2}$
as $r\to\infty$ for the boundary term to vanish.

More generally, in $n+1$ dimensional spacetime, $\D A$ grows like $r^{n-1}$,
requiring the integrand to fall faster than $1/r^{n-1}$ as $r\to\infty$
for the boundary terms to vanish.

\M
Now we see that the first variation of the action is just
\begin{subequations}
\begin{equation}
\delta\action[\varphi] = \int_{\mathcal{R}}\left(
-c^{-2}(\lambda\psi)\partial_{t}^{2}\varphi
+(\lambda\psi)\nabla^{2}\varphi
-\frac{m^{2}c^{2}}{\hbar^{2}}\varphi\cdot\lambda\psi\right)\D^{3}\vec{x}\,\D t,
\end{equation}
or factoring out the $\delta\varphi=\lambda\psi$,
\begin{equation}
\delta\action[\varphi] = \int_{\mathcal{R}}\left(
-c^{-2}\partial_{t}^{2}\varphi
+\nabla^{2}\varphi
-\frac{m^{2}c^{2}}{\hbar^{2}}\varphi\right)\delta\varphi\,\D^{3}\vec{x}\,\D t.
\end{equation}
\end{subequations}
Now we need to find the critical points of the action, which is to say,
we demand $\delta\action[\varphi]=0$. This gives us a rather tricky
differential-integral equation, but fortunately the fundamental lemma of
variational calculus\marginnote{TODO: cite fundamental lemma of variational calculus} says the condition is exactly demanding the
integrand's coefficient of $\delta\varphi$ vanishes, i.e.,
\begin{equation}
\boxed{-c^{-2}\partial_{t}^{2}\varphi
+\nabla^{2}\varphi
-\frac{m^{2}c^{2}}{\hbar^{2}}\varphi=0.}
\end{equation}
Solving this differential equation gives us the critical point for the
action.

Usually we expedite this whole process of variational analysis, and just
use the Euler--Lagrange equations.

\begin{exercise}
Re-perform this analysis with an arbitrary potential contribution
$V(\varphi)$ and observe how the equations of motion change (i.e., what
extra term will be added to the equations of motion, something involving
$V'(\varphi)$ we expect).
\end{exercise}

\N{Equations of Motion}
We can now derive the equations of motion using the Euler--Lagrange
equations, which are:
\begin{equation}
c^{-2}\partial_{t}^{2}\varphi - \nabla^{2}\varphi + \frac{m^{2}c^{2}}{\hbar^{2}}\varphi+V'(\varphi)=0.
\end{equation}
Also note that $V(\varphi)$ is usually some polynomial in $\varphi$, and
we can discard the terms lower than quadratic order (and absorb the
quadratic term into the mass term).

\begin{proof}
  We have
  \begin{subequations}
  \begin{equation}
\frac{\partial\mathcal{L}}{\partial(\partial_{\mu}\varphi)}
=-\partial^{\mu}\varphi,
  \end{equation}
  so the ``acceleration'' part of the equations of motion:
  \begin{equation}
\partial_{\mu}\frac{\partial\mathcal{L}}{\partial(\partial_{\mu}\varphi)}
=-\partial_{\mu}\partial^{\mu}\varphi.
  \end{equation}
  Then the ``force'' part of the equations of motion
\begin{equation}
\frac{\partial\mathcal{L}}{\partial\varphi} = -\frac{m^{2}c^{2}}{\hbar^{2}}\varphi
- V'(\varphi).
\end{equation}
Taken altogether, the equations of motion read:
\begin{equation}
-\partial_{\mu}\partial^{\mu}\varphi = -\frac{m^{2}c^{2}}{\hbar^{2}}\varphi
- V'(\varphi).
\end{equation}
  \end{subequations}
  Some gentle messaging yields the result.
\end{proof}

\N{Solutions}
We can solve the equations of motion for the \emph{free} scalar field by
taking its Fourier transform. We have
\begin{equation}
\varphi(t,\vec{x}) = \iint\E^{-\I(\omega t-\vec{k}\cdot\vec{x})}\widetilde{\varphi}(\omega,\vec{k})\frac{\D^{3}\vec{k}}{(2\pi\hbar)^{3}}\frac{\D\omega}{2\pi\hbar}
\end{equation}
where $\widetilde{\varphi}$ is the Fourier transform, $\omega=E/\hbar$,
and $\vec{k}=\vec{p}/\hbar$.

\begin{exercise}
Explicitly work this out. I mean, it's \emph{obvious}, but you should
check that it's \emph{true}.
\end{exercise}

\begin{exercise}
Let us consider the Fourier transform in the spatial variables only for
the scalar field,
\begin{equation}
\widetilde{\varphi}(\vec{k},t) = \int\E^{\I(\vec{k}\cdot\vec{x})}\varphi(\vec{x},t)\D^{3}\vec{x}.
\end{equation}
Prove or find a counter-example: the complex conjugate of the
\emph{spatially} Fourier
transformed scalar field $\widetilde{\varphi}(\vec{k},t)^{*}$
satisfies $\widetilde{\varphi}(\vec{k},t)^{*}=\widetilde{\varphi}(-\vec{k},t)$.

Also: do we need to include a factor of $(2\pi)^{-3/2}$ in the spatial Fourier
transform, or will things work out fine as we presented it? 
\end{exercise}

\subsection{Aside: Variational Calculus}

\M
We will typically use some heuristics in physics when doing variational
calculus, rather than working with the level of rigour anyone would hope
for. Given some functional of the form
\begin{equation}
F[\varphi] = \int_{\mathcal{R}}f(\varphi(x), \partial_{\mu}\varphi(x))\,\D^{4}x,
\end{equation}
we consider its first variation with respect to $\varphi$ by pretending:
\begin{enumerate}
\item $\delta$ is a linear operator obeying the Leibniz product rule (i.e., it
is a derivation), and
\item variations commute with differentiation (e.g., $\delta(\partial_{\mu}\varphi)=\partial_{\mu}(\delta\varphi)$).
\end{enumerate}
So to be clear, in the integrand, the $\delta$ operator acts like:
\begin{equation}
\delta f(\varphi(x)) 
= \frac{\partial f(\varphi)}{\partial\varphi}\delta\varphi.
\end{equation}
When there are also derivatives of $\varphi$, we have
\begin{subequations}
\begin{align}
\delta f(\varphi(x),\partial_{\mu}\varphi) 
&= \frac{\partial f(\varphi,\partial_{\mu}\varphi)}{\partial\varphi}\delta\varphi
+ \frac{\partial f(\varphi,\partial_{\mu}\varphi)}{\partial(\partial_{\mu}\varphi)}\delta(\partial_{\mu}\varphi)\\
\intertext{or, commuting variation with partial derivatives in the second term,}
\delta f(\varphi(x),\partial_{\mu}\varphi) &= \frac{\partial f(\varphi,\partial_{\mu}\varphi)}{\partial\varphi}\delta\varphi
+ \frac{\partial f(\varphi,\partial_{\mu}\varphi)}{\partial(\partial_{\mu}\varphi)}\partial_{\mu}(\delta\varphi)
\end{align}
\end{subequations}
When computing these partial derivatives, we pretend
\begin{equation}
\frac{\partial(\partial_{\mu}\varphi)}{\partial\varphi}
=0,\quad\mbox{and}\quad
\frac{\partial\varphi}{\partial(\partial_{\mu}\varphi)}=0.
\end{equation}

Then we have
\begin{equation}
\delta F[\varphi] = \int\left(\frac{\partial f}{\partial\varphi}\delta\varphi
+\frac{\partial f}{\partial(\partial_{\mu}\varphi)}\delta(\partial_{\mu}\varphi)\right)\,\D^{4}x.
\end{equation}
We try to integrate by parts to rewrite the integrand as
\begin{equation}
\delta F[\varphi] = \begin{pmatrix}\mbox{boundary}\\\mbox{terms}
\end{pmatrix}
+ \int\mbox{(something)}\delta\varphi\,\D^{4}x,
\end{equation}
then using the fundamental lemma of
variational calculus to set $\mbox{(something)}=0$. For most problems in
physics, we discard the boundary terms.\footnote{General Relativity is a
notable example where boundary terms are important.} This will give us a
differential equation whose solution is a critical point of the
functional.

\M
For multiple fields $\varphi_{a}$ with $a=1,\dots,N$, we need to take
the variation with respect to each of these fields. The first variation
would require summing over $a$, giving us $N$ coefficients of
$\delta\varphi_{a}$ (one for each $a$). We need each coefficient to
separately vanish, giving us a system of $N$ equations.

\M
These heuristics aren't ``wrong'', they just sweep details under the
rug. Specifically we would have
$\varphi_{a}^{(\lambda)}=\varphi_{a}+\lambda_{a}\psi_{a}$ where
$\psi_{a}$ are arbitrary functions which vanish at the initial and final
time slices. Then for any functional $F[\varphi_{a}]$ we find its first
variation as
\begin{equation}
\left.\frac{\D}{\D\lambda}F[\varphi^{(\lambda)}_{a}]\right|_{\lambda=0}=\delta
F[\varphi_{a},\psi_{a}]=\int\sum_{a}\frac{\delta F}{\delta\varphi_{a}}\psi_{a}\,\D^{4}x,
\end{equation}
where we just abuse notation left and right, using
$\delta F/\delta\varphi_{a}$ for the integrand, and sometimes writing
$\delta\varphi_{a}$ instead of $\psi_{a}$, and so on.

\begin{ddanger}
The notation used here is horribly sloppy in the physics literature, and
we have chosen to be consistent with that literature as much as possible.
The notation for a functional derivative coincides with a variational
derivative, which is horribly unfortunate. Some physicists try to relate
the two, but it's not quite the same thing.
\end{ddanger}

\N{Conditions for a minimum}
We recall from calculus in a single variable that not all critical
points of a function are minima (sometimes they are maxima, or
inflection points, or something else). This should caution us from being
too optimistic about critical points of functionals: the critical points
are \emph{candidates} for the minima, but we must check they are minima.

In calculus, this involves examining the sign of
$f''(x_{\text{crit}})$. When $f''(x_{\text{crit}})>0$, we have a
[possibly local] minimum.

For functionals, we need to examine the second variation $\delta^{2}F$
at the critical point. Specifically, for variation $\varphi_{\text{crit}}+\lambda\psi$,
we need
\begin{equation}
\delta^{2}F[\varphi_{\text{crit}}+\psi]\geq k\|\psi\|^{2}
\end{equation}
for all $\psi$ and some constant $k>0$. This is completely analogous to
the situation in calculus. Here we have (Taylor expanding in $\lambda\psi$, integrating by parts,
so the integrand is a Taylor polynomial in $\lambda\psi$):
\begin{equation}
F[\varphi+\lambda\psi] = F[\varphi] + \int\left(\frac{\delta F}{\delta\varphi}\lambda\psi
+\frac{\delta^{2} F}{\delta\varphi^{2}}\lambda^{2}\psi^{2} + \bigOh(\lambda^{3}\psi^{3})\right)\D^{n}x.
\end{equation}
Then we have 
\begin{equation}
\delta^{2}F[\varphi+\lambda\psi] = \int\frac{\delta^{2} F}{\delta\varphi^{2}}\lambda^{2}\psi^{2}\,\D^{n}x.
\end{equation}
This is the \define{Second Variation} of $F$.

However, this is all rather tedious, and in practice physicists drop the
``Mission Accomplished'' banner upon finding critical points for
functionals.

\begin{ddanger}
Not all functionals have a second variation. We need it to be twice
differentiable. For the functionals we care about in physics, which is
just the integral of a Lagrangian, we can compute the second variation.
\end{ddanger}

\N{References}
The quartic self-interacting scalar field (i.e., with
$V(\varphi)=\lambda\varphi^{4}/4!$ where $\lambda$ is the coupling
constant) has a few known exact solutions, which are presented in
Frasca~\cite{Frasca:2009bc}. These are quite tricky, since it requires
knowledge of Jacobi elliptical functions.