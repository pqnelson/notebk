\section{*Path Independence}

\M Teitelboim showed in his PhD thesis and several follow-up articles~\cite{Hojman:1976vp,Teitelboim:1980hs} how
we can use the canonical formalism coupled to the principle of ``path
independence'' to derive General Relativity, Yang--Mills gauge theory,
among other things. We also saw in passing
(\S\ref{chunk:rqm:poincare-algebra:elementary-particles-irreps}) the
only fields possible are Scalar fields, Vector fields, and (rank-2
symmetric) Tensor fields.

\N{Foliating Spacetime}
We foliate spacetime by space-like hypersurfaces $\Sigma_{t}$ indexed by
time $t$. We have the metric $h_{ij}$ on $\Sigma_{t}$ where $i$,
$j=1,2,3$ are spatial indices.

We stipulate that $h_{ij}$ has its canonically conjugate momenta density
$p^{ij}$. 

\N{Evolution}
If we want to consider the time evolution of a function $F(h_{ij}(x), p^{k\ell})$
from one spatial hypersurface $\Sigma_{t}$ to another $\Sigma_{t+\delta t}$
be described using generators $\mathcal{H}_{i}(x)$ and $\mathcal{H}_{\bot}(x)$
and the Poisson bracket
\begin{equation}
\begin{split}
\dot{F}(h_{ij}(x),p^{k\ell}(x))
&=\int\bigl(\PB{F}{\mathcal{H}_{\bot}(x')}N(x') + \PB{F}{\mathcal{H}_{i}(x')}N^{i}(x')\bigr)\,\D^{3}x'\\
&=\int\PB{F}{\mathcal{H}_{\mu}(x')}N^{\mu}(x')\,\D^{3}x'.
\end{split}
\end{equation}
If we want to consider spatial diffeomorphisms $F\to F + \variation F$
along the same hypersurface $\Sigma_{t}$, then we use:
\begin{equation}
\variation F = -\int\PB{F}{\mathcal{H}_{i}(x')}\,\variation N^{i}(x')\,\D^{3}x'.
\end{equation}


\N{Fundamental Poisson Brackets}
We generate the transformations desired by the Poisson brackets:
\begin{subequations}
\begin{align}
\PB{\mathcal{H}_{\bot}(x)}{\mathcal{H}_{\bot}(x')} &= [h^{ij}(x)\mathcal{H}_{j}(x) + h^{ij}(x')\mathcal{H}_{j}(x')]\partial_{i}\delta(x,x')\\
\PB{\mathcal{H}_{i}(x)}{\mathcal{H}_{\bot}(x')} &= \mathcal{H}_{\bot}(x)\partial_{i}\delta(x,x')\\
\PB{\mathcal{H}_{i}(x)}{\mathcal{H}_{j}(x')} &= \mathcal{H}_{i}(x')\partial_{j}\delta(x,x')+\mathcal{H}_{j}(x)\partial_{i}\delta(x,x').
\end{align}
\end{subequations}
The ``only'' generator of ``real'' dynamical interest is $\mathcal{H}_{\bot}$.
The three $\mathcal{H}_{i}$ generate spatial diffeomorphisms, i.e.,
displacements that lie on the same spatial hypersurface which amount to
a change of spatial coordinates.

\N{Gravitating and Matter Generators}
We separate the generators $\mathcal{H}_{\bot}$ and $\mathcal{H}_{i}$
into a sum of the ``gravitational part'' and the ``matter part''. We
would have
\begin{equation}
\mathcal{H}_{\bot} = \mathcal{H}_{\bot}^{\text{grav}}[h_{ij},p^{k\ell}]
+ \mathcal{H}_{\bot}^{\text{mat}}[h_{ij},p^{k\ell},\mbox{matter canonical variables}].
\end{equation}
If $\mathcal{H}_{\bot}^{\text{mat}}$ did not depend on gravitational
variables (the spatial metric and its conjugate momenta), then the
Poisson bracket
$\PB{\mathcal{H}_{\bot}^{\text{mat}}}{\mathcal{H}_{\bot}^{\text{mat}}}$
cannot satisfy the desired relation (it'd be independent of $h_{ij}$,
but it must involve $h^{ij}$).

\M
The general form of $\mathcal{H}_{i}$ meanwhile may be constrained by
the following desired requirements:
\begin{enumerate}
\item It must be linear in the momenta in order to generate
  transformations of the fields under coordinate transformations (and
  not mix fields and momenta);
\item It should contain the momenta only up to the first spatial
  derivatives because it should generate first-order derivatives in the
  fields (think ``Taylor expansion to first order'').
\end{enumerate}
Therefore we expect
\begin{equation}
\mathcal{H}_{i}={b_{i}}^{jB}(\phi^{C})\partial_{b}p^{B} + {a_{a}}^{B}(\phi^{C})p_{B},
\end{equation}
where $\phi^{A}$ is now a symbolic notation for \emph{all} fields
\emph{including gravity}, and $p_{A}$ denotes the corresponding momenta.

\N{Heuristic for $\mathcal{H}_{\bot}$}
We stipulate that $\mathcal{H}_{\bot}$ is a quadratic polynomial in the
momenta,
\begin{equation}
\mathcal{H}_{\bot}\sim M_{AB}(\phi^{C})p^{A}p^{B} + N_{A}(\phi^{C})p^{A} + V(\phi^{C}),
\end{equation}
then we just need to determine the coefficients $M_{AB}$, $N_{A}$, $V$.
The calculations for the various fields appears in Section 6 of
Teitelboim~\cite{Teitelboim:1980hs}. 

\N{Spatial Diffeomorphism Requirements}
We demand the existence of \emph{ultralocal solutions}, in the sense
that a deformation localized at a point $\vec{x}_{0}$ must change the
field only at $\vec{x}_{0}$, then we find
\begin{equation}
b^{ijB}=b^{jiB}.
\end{equation}
So under an infinitesimal coordinate transformation
\begin{equation}
x'^{i} = x^{i} - \variation N^{i}(x),
\end{equation}
we will demand a field quantity $\phi$ transforms as
\begin{equation}
\LieD_{\variation\vec{N}}\phi =
-\int\PB{\phi}{\mathcal{H}_{i}(x')}\,\variation N^{i}(x')\,\D^{3}x',
\end{equation}
where $\LieD$ denotes the Lie derivative.\index{Lie derivative}
This will then give us a way to find $\mathcal{H}_{i}(x')$.

\N{Spatial Diffeomorphism Invariance of Scalar Fields}
For Scalar fields $\phi(x)$, it must transform as
\begin{equation}
\variation\phi(x):=\phi'(x)-\phi(x)\weakEq\partial_{i}\phi\,\variation N^{i}=\LieD_{\variation\vec{N}}\phi.
\end{equation}
This is generated by
\begin{equation}\label{eq:classical-field-theory:henneaux:scalar-field-momenta-constraints}
\mathcal{H}_{i} = p_{\phi}\partial_{i}\phi,
\end{equation}
where $p_{\phi}$ is the conjugate momenta density for $\phi$.

\begin{exercise}
Show that Eq~\eqref{eq:classical-field-theory:henneaux:scalar-field-momenta-constraints}
implies $b^{ij}=0=b^{ji}$. Therefore it satisfies ultralocality.
\end{exercise}

\begin{exercise}
Verify Eq~\eqref{eq:classical-field-theory:henneaux:scalar-field-momenta-constraints}
generates the correct transformations for $\phi$.
\end{exercise}

\N{For Vector Fields}
The vector field $A_{i}(x)$ transforms as
\begin{equation}
\variation A_{i} = (\partial_{j}A_{i})\,\variation
N^{j}+A_{j}\partial_{i}(\variation N^{j}) = (\LieD_{\variation\vec{N}}A)_{i},
\end{equation}
which is generated by
\begin{equation}\label{eq:classical-field-theory:henneaux:vector-field-momenta-constraints}
\mathcal{H}_{i} = -A_{i}\partial_{j}p^{j} + (\partial_{i}A_{j}-\partial_{j}A_{i})p^{j}.
\end{equation}

\begin{exercise}
Verify Eq~\eqref{eq:classical-field-theory:henneaux:vector-field-momenta-constraints}
generates $\variation A_{i}$.
\end{exercise}

\begin{exercise}[DO THIS]
Show that Eq~\eqref{eq:classical-field-theory:henneaux:vector-field-momenta-constraints}
implies ${{b_{i}}^{j}}_{C}=-A_{i}{\delta^{j}}_{C}$ and therefore the
condition for ultralocality is not fulfilled for vector fields.
\end{exercise}

\N{Covariant Rank-2 Tensor}
For a covariant rank-2 tensor (not necessarily symmetric) $t_{ij}$, we
have
\begin{equation}
\variation t_{ij} = (\partial_{k}t_{ij})\variation N^{k}
+ t_{ik}\partial_{j}(\variation N^{k})
+ t_{jk}\partial_{i}(\variation N^{k})
= (\LieD_{\variation\vec{N}}t)_{ij}.
\end{equation}
This is generated by
\begin{equation}
\mathcal{H}_{i} = p^{k\ell}\partial_{i}t_{k\ell}-\partial_{k}(t_{ij}p^{kj})-\partial_{j}(t_{ki}p^{kj}).
\end{equation}
In order for ultralocality to hold, we must have
\begin{equation}
t_{ij} = f(x)h_{ij}
\end{equation}
where $f(x)$ is an arbitrary function. That is to say, the tensor field
must be proportional to the metric tensor. We can interpret this as
another piece of evidence suggesting the ``only'' spin-2 field allowed
is the gravitational field.\index{Spin!Uniqueness of (---)-2}

\begin{exercise}[Fermions]\index{Lie derivative!Of spinor field}
  Can we do a similar analysis for a Dirac spinor field $\psi$ in curved
  spacetime? (I honestly do not know!)
  
  There are various generalizations of the Lie derivative to the
  spin-$1/2$ field. Godina and Matteucci~\cite{Godina:2005mt} reviews the various
  generalizations of the Lie derivative for spinors.  Specifically:
\begin{enumerate}
\item Find the ``right'' generalization of the Lie derivative to spinor
  fields. There are many possibilities! For one single example,
  Choquet-Bruhat and DeWitt-Morette~\cite{Choquet-Bruhat:2000amp2} propose
  \[\LieD_{X}\psi = X^{j}\partial_{i}\psi - \frac{1}{8}(\partial_{i}X_{j}-\partial_{j}X_{i})\gamma^{i}\gamma^{j}\psi = \variation\psi,\]
  where $\psi$ is a 4-component Dirac spinor field,
  $\gamma^{i}$, $\gamma^{j}$ are the Dirac matrices; is this the
  ``right'' generalization for our purposes?
\item Since we're dealing with
  fermionic fields, how must we modify the spatial diffeomorphism
  condition
  $\variation\psi\sim-\int\PB{\psi}{\mathcal{H}_{i}(y)}\,\variation N^{i}(y)\,\D^{3}y$?
  Does it suffice to use the Poisson \emph{super}-bracket?
\item What $\mathcal{H}_{i}$ generates this? 
\item Is this $\mathcal{H}_{i}$ ultralocal?
\end{enumerate}
\end{exercise}

\N{Gauge Theory: Recovering Ultralocality for Vector Fields}
We see that $\mathcal{H}_{i}$ is not ultralocal. It's caused by the
presence of the $A_{i}\partial^{j}p_{j}$ term. So it's tempting to make
the replacement
\begin{equation}
\mathcal{H}_{i}\to\widetilde{\mathcal{H}}_{i}
:=\mathcal{H}_{i} + A_{i}\partial_{j}p^{j}
=(\partial_{i}A_{j}-\partial_{j}A_{i})p^{j}.
\end{equation}
We see then that the Poisson bracket of this generator with itself is:
\begin{equation}\label{eq:classical-field-theory:henneaux-pi:modified-vector-PB}
\PB{\widetilde{\mathcal{H}}_{i}(x)}{\widetilde{\mathcal{H}}_{j}(y)}
= \widetilde{\mathcal{H}}_{j}(x)\partial_{i}\delta(x,y)
+ \widetilde{\mathcal{H}}_{i}(y)\partial_{j}\delta(x,y)
- F_{ij}(x)\partial_{k}p^{k}(x)\delta(x,y),
\end{equation}
where $F_{ij}=\partial_{i}A_{j}-\partial_{j}A_{i}$. This new term
appearing in the right-hand side of
Eq~\eqref{eq:classical-field-theory:henneaux-pi:modified-vector-PB} is
harmless \emph{only} if it generates physically irrelevant
transformations (``gauge transformations''). There are two ways this
could happen: either $F_{ij}$ is constrained to vanish (which is too
strong, this would imply $A_{i}=\partial_{i}\varphi$), or we demand
$\partial_{i}p^{i}\weakEq0$ is a constraint. We therefore introduce the
constraint
\begin{equation}
\mathcal{G}(x) := \frac{-1}{e}\partial_{i}p^{i}(x) =
\frac{-1}{e}\partial_{i}E^{i}(x) = \frac{-1}{e}\nabla\cdot\vec{E}(x),
\end{equation}
where $e$ is the electric charge, and the momentum is just $\vec{E}$ the
electric field. This constraint $\mathcal{G}\weakEq0$ is precisely
\emph{Gauss's Law}.

\begin{exercise}
Prove $\widetilde{H}_{i}$ is gauge invariant. Hint: it suffices to prove
the electric field is gauge invariant, and $F_{ij}$ is gauge invariant.
\end{exercise}

\begin{exercise}
Show Gauss's Law generates the gauge transformations
\begin{subequations}
\begin{align}
\variation A_{i}(x) &= \int\PB{A_{i}(x)}{\mathcal{G}(y)}\,\xi(y)\,\D^{3}y=\frac{1}{e}\partial_{i}\xi(x)\\
\variation p^{i}(x) &= \int\PB{p^{i}(x)}{\mathcal{G}(y)}\,\xi(y)\,\D^{3}y=0.
\end{align}
\end{subequations}
\end{exercise}

\M
Therefore we conclude that $A_{i}(x)$ transforms under
$\widetilde{H}_{j}$ as a vector modulo a gauge transfromation.

\N{Generalization to Yang--Mills}
When we replace $A_{i}(x)$ with several fields $A_{i}^{a}(x)$ for $a=1$,
\dots, $N$. We assume that:
\begin{enumerate} 
\item $A_{i}^{a}(x)$ does not mix with its momenta $p^{i}_{a}(x)$ under
  a gauge transformation,
\item the momenta should transform homogenously, and
\item the gauge constraint (the non-Abelian version of Gauss's Law) is local.
\end{enumerate}
This forces us to,
\begin{equation}
\mathcal{G}_{i} = \frac{-1}{f}\partial_{i}{p_{a}}^{i} + {C_{ab}}^{c}A^{a}_{i}p^{i}_{c}\weakEq0,
\end{equation}
where $f$ and ${C_{ab}}^{c}$ are constants. If we demand the commutator
of two gauge transformations yields another gauge transformation, then
we find ${C_{ab}}^{c}$ are the structure constants of a Lie algebra. We
have:
\begin{equation}
\PB{\mathcal{G}_{a}(x)}{\mathcal{G}_{b}(y)}={C_{ab}}^{c}\mathcal{G}_{c}(x)\delta(x,y),
\end{equation}
which precisely characterizes a Yang--Mills theory.

\M
If we split up $\mathcal{H}_{\bot}$ as a sum
\begin{equation}
\mathcal{H}_{\bot}=\mathcal{H}_{\bot}^{\text{grav}}+\mathcal{H}_{\bot}^{\text{YM}},
\end{equation}
and demanding the Yang--Mills part is independent of the gravitational
momenta (so $\mathcal{H}_{\bot}^{\text{YM}}$ depends only ultralocally
on the metric), then we find
\begin{equation}
\mathcal{H}_{\bot}^{\text{YM}} = \frac{1}{2\sqrt{h}}(h_{ij}\gamma^{ab}p^{i}_{a}p^{j}_{b}+h^{ij}\gamma_{ab}B^{a}_{i}B^{b}_{j}),
\end{equation}
where $\gamma_{ab}={C_{ac}}^{d}{C_{bd}}^{c}$ is the ``group metric''
($\gamma^{ab}$ is its inverse), and
$B^{a}_{i}=\frac{1}{2}\epsilon_{ijk}F^{ajk}$ are the non-Abelian
``magnetic fields''. The non-Abelian field strength is given by:
\begin{equation}
F^{a}_{ij}=\partial_{i}A^{a}_{j} - \partial_{j}A^{a}_{i}
+f{C^{a}}_{bc}A^{b}_{i}A^{c}_{j}.
\end{equation}
The Hamiltonian corresponds to the action,
\begin{equation}
\action_{\text{YM}} = \frac{-1}{4}\int\gamma_{ab}F^{a}_{\mu\nu}F^{b\mu\nu}\sqrt{-g}\,\D^{4}x
\end{equation}
where $\sqrt{-g}$ is the determinant of the metric of spacetime.

\M
So the principle of path independent taken together with the demand that
$\mathcal{H}_{\bot}$ be ultralocal in the momenta necessarily leads to
the concept of gauge theories.
