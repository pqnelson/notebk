\section{Linear Chain}

\N{Problem Statement}
Consider $N\in\NN$ identical point-masses (each with rest mass $m$)
each of which are connected to two neighboring point masses by identical
massless springs with spring constant $k$ and equilibrium length $a$.
The point-masses then form a line segment.

We will take $N\to\infty$ limit in such a way that at any point-mass
there are infinitely many point-masses in either direction.
This specifically is to let us ignore the boundary conditions.

What are the equations of motion for a point-mass in this chain? What is
the Lagrangian for this system?

Take the continuum limit where $a\to0$ while $a^{2}k/m$ is held
constant. What happens to the equations of motion and the Lagrangian?

\begin{exercise}
What dimensions does $a^{2}k/m$ have?
\end{exercise}

\N{Coordinates}
Since we have the point-masses form a one-dimensional system, we will
write $x_{j}$ for the position of the point-mass with $j\in\ZZ$.

\N{Free Body Diagram}
Suppose we examine the free-body diagram for the point-mass. The only
forces acting on a point-mass $x_{j}$ are the spring forces:
\begin{center}
\includegraphics{img/img.0}
\end{center}

\N{Equations of Motion}
Then we see the force acting on $x_{j}$ is
\begin{equation}
\begin{split}
  F_{j} &= -k(x_{j}-x_{j-1}-a) + k(x_{j+1}-x_{j}-a)\\
  &= k(x_{j+1}-2x_{j}+x_{j}).
\end{split}
\end{equation}
Using Newton's second Law,
\begin{equation}
m\ddot{x}_{j} = F_{j} = k(x_{j+1}-2x_{j}+x_{j}).
\end{equation}
We will rearrange this to:
\begin{equation}\label{eq:classical-field-theory:linear-chain:newton-eom}
\ddot{x}_{j} = \frac{k}{m}(x_{j+1}-2x_{j}+x_{j}).
\end{equation}


\N{Lagrangian}
We can then write the Lagrangian for this system,
\begin{equation}
L = \sum_{j\in\ZZ}\frac{m}{2}\dot{x}^{2}_{j} - \frac{k}{2}(x_{j+1}-x_{j})^{2}.
\end{equation}
Since the sum is over all integers, the forces acting on $x_{j}$ come
from the $j-1$ term and the $j$ term.

\N{Continuum Limit}
Now care must be taken, because as $a\to 0$ the index $j$ labeling
particles will become a real number indicating the position of the
particle. To avoid ambiguity, we will write $q_{j}(t)$ for the position
of particle $j$.

We observe as $a\to0$, we have $q_{j}(t)\to q(x,t)$
\begin{equation}
\frac{x_{j+1}-2x_{j}+x_{j-1}}{a^{2}}\xrightarrow{a\to0}\frac{\partial^{2}}{\partial x^{2}}q(x,t).
\end{equation}
Then the continuum limit of the equations of motion,
Eq~\eqref{eq:classical-field-theory:linear-chain:newton-eom}, (first
dividing through by $m$) is:
\begin{equation}
\ddot{x}_{j}\xrightarrow{a\to0}\frac{\partial^{2}}{\partial t^{2}}q(x,t),
\quad\mbox{and}\quad\frac{ka^{2}}{m}\frac{x_{j+1}-2x_{j}+x_{j-1}}{a^{2}}
\xrightarrow{a\to0}v^{2}
\frac{\partial^{2}}{\partial x^{2}}q(x,t).
\end{equation}
Then equating both limits gives us:
\begin{equation}
\frac{\partial^{2}}{\partial t^{2}}q(x,t) = v^{2}
\frac{\partial^{2}}{\partial x^{2}}q(x,t).
\end{equation}
This is precisely the wave equation for an elastic string.
Here $v=\sqrt{a^{2}k/m}$ is the velocity of propagation.

\begin{exercise}
We have been working with one spatial dimension, assuming it is $\RR$.
Suppose space is a circle $S^{1}$ and our linear chain forms a
ring. Perform the continuum limit analysis for this situation.
\end{exercise}

\begin{exercise}
If we took space to be a closed interval $[a,b]$ instead of $\RR$,
then what boundary conditions do we need to impose for things to work
out in the continuum limit?
\end{exercise}

\N{Canonical Analysis}
We can perform the Legendre transform of the Lagrangian, first finding
the conjugate momenta
\begin{equation}
p_{j} = \frac{\partial L}{\partial\dot{q}_{j}} = m\dot{q}_{j}.
\end{equation}
Then
\begin{equation}
H = \sum_{j\in\ZZ}\frac{p_{j}^{2}}{2m} + k(q_{j+1}-q_{j})^{2}.
\end{equation}
