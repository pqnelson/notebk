\section{Linear Chain}

\N{Problem Statement}
Consider $N\in\NN$ identical point-masses (each with rest mass $m$)
each of which are connected to two neighboring point masses by identical
massless springs with spring constant $k$ and equilibrium length $a$.
The point-masses then form a line segment.

We will take $N\to\infty$ limit in such a way that at any point-mass
there are infinitely many point-masses in either direction.
This specifically is to let us ignore the boundary conditions.

What are the equations of motion for a point-mass in this chain? What is
the Lagrangian for this system?

Take the continuum limit where $a\to0$ while $a^{2}k/m$ is held
constant. What happens to the equations of motion and the Lagrangian?

\begin{exercise}
What dimensions does $a^{2}k/m$ have?
\end{exercise}

\N{Assumptions}
We assume that the point-masses all have identical masses, otherwise the
chain is not homogeneous. We similarly assume the springs connecting
them are all identical and massless, otherwise it's not
homogeneous. When we extend the chain from $n=1$ spatial dimensions to
$n>1$ dimensions, we lose isotropy (i.e., there are ``preferred directions''
and space is not rotationally invariant).
These are important when we take the continuum $a\to0$ limit, so we
recover a Poincar\'e invariant system (that is, the physical system is
relativistic). 

\N{Coordinates}
Since we have the point-masses form a one-dimensional system, we will
write $x_{j}$ for the position of the point-mass with $j\in\ZZ$.

\N{Free Body Diagram}
Suppose we examine the free-body diagram for the point-mass. The only
forces acting on a point-mass $x_{j}$ are the spring forces:
\begin{center}
\includegraphics{img/img.0}
\end{center}

\N{Equations of Motion}
Then we see the force acting on $x_{j}$ is
\begin{equation}
\begin{split}
  F_{j} &= -k(x_{j}-x_{j-1}-a) + k(x_{j+1}-x_{j}-a)\\
  &= k(x_{j+1}-2x_{j}+x_{j}).
\end{split}
\end{equation}
Using Newton's second Law,
\begin{equation}
m\ddot{x}_{j} = F_{j} = k(x_{j+1}-2x_{j}+x_{j}).
\end{equation}
We will rearrange this to:
\begin{equation}\label{eq:classical-field-theory:linear-chain:newton-eom}
\ddot{x}_{j} = \frac{k}{m}(x_{j+1}-2x_{j}+x_{j}).
\end{equation}


\N{Lagrangian}
We can then write the Lagrangian for this system,
\begin{equation}\label{eq:classical-field-theory:linear-chain:lagrangian-with-point-masses}
L = \sum_{j\in\ZZ}\frac{m}{2}\dot{x}^{2}_{j} - \frac{k}{2}(x_{j+1}-x_{j})^{2}.
\end{equation}
Since the sum is over all integers, the forces acting on $x_{j}$ come
from the $j-1$ term and the $j$ term.

\N{Continuum Limit}
Now care must be taken, because as $a\to 0$ the index $j$ labeling
particles will become a real number indicating the position of the
particle. To avoid ambiguity, we will write $q_{j}(t)$ for the position
of particle $j$.

We observe as $a\to0$, we have $q_{j}(t)\to q(x,t)$
\begin{equation}
\frac{x_{j+1}-2x_{j}+x_{j-1}}{a^{2}}\xrightarrow{a\to0}\frac{\partial^{2}}{\partial x^{2}}q(x,t).
\end{equation}
Then the continuum limit of the equations of motion,
Eq~\eqref{eq:classical-field-theory:linear-chain:newton-eom}, (first
dividing through by $m$) is:
\begin{equation}
\ddot{x}_{j}\xrightarrow{a\to0}\frac{\partial^{2}}{\partial t^{2}}q(x,t),
\quad\mbox{and}\quad\frac{ka^{2}}{m}\frac{x_{j+1}-2x_{j}+x_{j-1}}{a^{2}}
\xrightarrow{a\to0}v^{2}
\frac{\partial^{2}}{\partial x^{2}}q(x,t).
\end{equation}
Then equating both limits gives us:
\begin{equation}
\frac{\partial^{2}}{\partial t^{2}}q(x,t) = v^{2}
\frac{\partial^{2}}{\partial x^{2}}q(x,t).
\end{equation}
This is precisely the wave equation for an elastic string.
Here
\begin{equation}\label{eq:classical-field-theory:linear-chain:speed-of-propagation}
v^{2} = \frac{a^{2}k}{m}
\end{equation}
is the velocity of propagation.

\begin{exercise}
We have been working with one spatial dimension, assuming it is $\RR$.
Suppose space is a circle $S^{1}$ and our linear chain forms a
ring. Perform the continuum limit analysis for this situation.
\end{exercise}

\begin{exercise}
If we took space to be a closed interval $[a,b]$ instead of $\RR$,
then what boundary conditions do we need to impose for things to work
out in the continuum limit?
\end{exercise}

\subsection{Canonical Analysis}

\N{Periodic Boundary Conditions}
First, we will use periodic boundary conditions. This means that there
will be $N\in\NN$ point masses such that
\begin{equation}
q_{N+1}(t) = q_{1}(t).
\end{equation}
Then we just change the sum of the Lagrangian to be $j=1,\dots,N$.

\N{Legendre Transform}
We can perform the Legendre transform of the Lagrangian, first finding
the conjugate momenta
\begin{equation}\label{eq:classical-field-theory:linear-chain:momentum}
p_{j} = \frac{\partial L}{\partial\dot{q}_{j}} = m\dot{q}_{j}.
\end{equation}
Then
\begin{equation}
H = \sum^{N}_{j=1}\frac{p_{j}^{2}}{2m} + \frac{k}{2}(q_{j+1}-q_{j})^{2}.
\end{equation}
For us to extract more information, we will do a series expansion for
$q_{j}(t)$ in powers of some nice family of basis functions, then
determine $p_{j}(t)$ as a sum of these basis functions, and finally the
Hamiltonian will be determined. This constitutes a long detour, but
reflects a lot of the typical calculations we encounter in field theory. 

\N{Normal Coordinates}
In classical mechanics, the dynamics of a set of coupled oscillators is
best studied by using \emph{normal coordinates}. The position
coordinates $q_{j}(t)$ may be expanded with respect to a set of linearly
independent ``basis functions'' $u_{j}^{k}$,
\begin{equation}\label{eq:classical-field-theory:linear-chain:discrete-fourier-decomposition}
q_{j}(t) = \sum_{k}a_{k}(t)u^{k}_{j},
\end{equation}
where the index $k$ counts the members of the basis set. The natural
(and convenient) choice for basis is the harmonic functions
\begin{equation}\label{eq:classical-field-theory:linear-chain:basis-harmonic-functions}
u_{j}^{k} = \frac{1}{\sqrt{N}}\exp(\I kaj).
\end{equation}
We now recognize the decomposition Eq~\eqref{eq:classical-field-theory:linear-chain:discrete-fourier-decomposition}
as a discrete Fourier decomposition.

\begin{remark}
The choice of $u_{j}^{k}$ makes it ``almost'' orthonormal. Its $L^{2}$-norm is
\begin{equation}
\|u_{j}^{k}\|_{L^{2}}^{2}=\frac{1}{N}\neq1,
\end{equation}
which will impact calculations later on. Physicists sweep this fact
under the rug, hoping no one will notice, and later give it physical
significance when certain divergences appear in the continuum limit. But
this is just an artifact of a particular choice of basis functions which
are not ``unit vectors'' with respect to the $L^{2}$-norm.
\end{remark}

\begin{ddanger}
Physicists will work with a different inner product, one with respect to
the $k$ indices rather than the one mathematicians would expect. This is
especially confusing, because it's never announced (much less justified).
The reader should be forewarned about this trickiness.
\end{ddanger}

\begin{exercise}
What are the dimensions of the basis functions $u^{k}_{j}$? What are the
dimensions for the coefficients $a_{k}(t)$?
\end{exercise}

\M The index $k$ has the dimension of inverse length, and physically
corresponds to the \emph{wave number} of the ``plane wave'' $u^{k}_{j}$.
The periodic boundary conditions restricts the range of $k$ since
\begin{equation}
u^{k}_{N+j}=u^{k}_{j},
\end{equation}
which implies
\begin{equation}
\exp(\I kaN) = 1.
\end{equation}
Therefore
\begin{equation}\label{eq:classical-field-theory:linear-chain:wave-length-range}
k = \frac{2\pi}{N a}\ell,
\end{equation}
where $\ell$ is an integer. 
Periodicity then restricts the range of values for $\ell$ to lie in the 
range
\begin{equation}
\frac{-N}{2}<\ell\leq\frac{N}{2}.
\end{equation}
Larger $\ell$ ``cycles back''.

\M
The dimension of the basis $\{u^{k}_{j}\}$ coincides the number of
degrees of freedom $N$ of the system of oscillators.

\N{Orthonormality and Completeness of Basis}
We can see that the basis are orthonormal:
\begin{equation}
\sum^{N}_{j=1}\overline{u^{k'}_{j}}u^{k}_{j}=\delta_{k,k'}.
\end{equation}
Further, the basis is complete:
\begin{equation}\label{eq:classical-field-theory:linear-chain:canonical-analysis:completeness-of-basis-functions}
\sum_{k}\overline{u^{k}_{j'}}u^{k}_{j}=\delta_{j',j}.
\end{equation}

\begin{proof}[Proof (Orthonormality)]
  Orthonormality follows from
  \begin{equation}
\sum^{N}_{j=1}\E^{2\pi\I(\ell-\ell')j/N}=N\delta_{\ell,\ell'}.
  \end{equation}
  When $\ell=\ell'$, the left-hand side reduces to $\sum^{N}_{j=1}1=N$,
  and the result follows.

  We just need to prove the result for $\ell\neq\ell'$. We see
\begin{calculation}
\sum^{N}_{j=1}\E^{2\pi\I(\ell-\ell')j/N}
\step{law of powers}
\sum^{N}_{j=1}(\E^{2\pi\I(\ell-\ell')/N})^{j}
\step{finite geometric sum}
\frac{1 - \E^{2\pi\I(\ell-\ell')N/N}}{1 - \E^{2\pi\I(\ell-\ell')/N}}
\step{algebra}
\frac{1 - \E^{2\pi\I(\ell-\ell')}}{1 - \E^{2\pi\I(\ell-\ell')/N}}.
\end{calculation}
We need to prove this vanishes. This requires us to prove (i) the
denominator is nonzero, and (ii) the numerator vanishes. These claims
are proven thus:
\begin{enumerate}
\item Since $\ell-\ell'\neq0$ and $|\ell-\ell'|<N$, it follows that
  $(\ell-\ell')/N\notin\ZZ$. Therefore the denominator cannot be zero.
\item Since $\ell$, $\ell'\in\NN$, it follows that $\ell-\ell'\in\ZZ$
  and therefore $\exp(2\pi\I(\ell-\ell'))=1$.  It then follows the
  numerator vanishes.
\end{enumerate}
Orthonormality follows immediately.
\end{proof}

\begin{proof}[Proof (Completeness)]
The completeness property follows in a similar manner, by direct
calculation
\begin{calculation}
\sum_{k}\overline{u^{k}_{j'}}u^{k}_{j}
\step{unfolding the definition of $u^{k}_{j}$ from Eq~\eqref{eq:classical-field-theory:linear-chain:basis-harmonic-functions}}
\sum_{k}\left(\frac{1}{\sqrt{N}}\E^{-\I kaj'}\right)\left(\frac{1}{\sqrt{N}}\E^{\I kaj}\right)
\step{algebra}
\frac{1}{N}\sum_{k}\E^{\I ka(j-j')}
\step{since $k=\ell 2\pi/(N a)$ by Eq~\eqref{eq:classical-field-theory:linear-chain:wave-length-range}}
\frac{1}{N}\sum^{N/2}_{\ell=-(N/2)+1}\E^{\I 2\pi(j-j')\ell/N}
\step{reindexing $1\leq\ell\leq N$}
\frac{1}{N}\E^{-\I\pi(j-j')}\sum^{N}_{\ell=1}\E^{\I 2\pi(j-j')\ell/N}
\step{since $\sum^{N}_{\ell=1}\E^{\I 2\pi(j-j')\ell/N}=N\delta_{j,j'}$}
\frac{1}{N}\E^{-\I\pi(j-j')}(N\delta_{j,j'})
\step{algebra}
\E^{-\I\pi(j-j')}\delta_{j,j'}
\step{since $\E^{-\I\pi(j-j')}$ contributes to nonzero term when $j=j'$}
\E^{-\I\pi(j-j)}\delta_{j,j'}
\step{since $j-j=0$ and $\E^{0}=1$}
\delta_{j,j'}.\qedhere
\end{calculation}
\end{proof}

\N{Basis Under Conjugation}
It follows from these calculations that these basis functions under
complex conjugation behave like:
\begin{equation}
\overline{u^{k}_{j}} = u^{-k}_{j}.
\end{equation}

\N{Coefficients Reality Conditions}
The coefficients $a_{k}(t)$ in the expansion in Eq~\eqref{eq:classical-field-theory:linear-chain:discrete-fourier-decomposition}
must have some reality conditions so $\overline{q_{j}}=q_{j}$ the
positions are real numbers. This means,
\begin{equation}
\overline{a_{k}(t)} = a_{-k}(t).
\end{equation}

\M
The equations of motion, using our expansion in Eq~\eqref{eq:classical-field-theory:linear-chain:discrete-fourier-decomposition},
gives us differential equations in terms of the coefficients
$a_{k}(t)$:\footnote{This requires expanding both sides of the equations
of motion, multiplying both sides by $\overline{u^{k}_{j}}$ and then
summing over $j$. The orthonormality condition is invoked and the
defining property of the Kronecker delta is used, which produces the equation.}
\begin{equation}
\ddot{a}_{k}(t) = \frac{k_{s}}{m}\sum_{k'}a_{k'}(t)\sum_{j}\overline{u^{k}_{j}}
\bigl(u^{k'}_{j+1} -2u^{k'}_{j} + u^{k'}_{j-1}\bigr).
\end{equation}
Here we write $k_{s}$ to stress we work with the spring constant, not an
index. 
Observe the basis functions at different $j$ differ only by a phase
factor
\begin{equation}
u^{k}_{j\pm1}=\E^{\pm\I ka}u^{k}_{j}.
\end{equation}
We use completeness of the basis functions from Eq~\eqref{eq:classical-field-theory:linear-chain:canonical-analysis:completeness-of-basis-functions}
to simplify the sum over $k'$
\begin{calculation}
  \sum_{k'}a_{k}(t)\sum_{j}\overline{u^{k}_{j}}
\bigl(u^{k'}_{j+1} -2u^{k}_{j} + u^{k'}_{j-1}\bigr)
\step{using $u^{k'}_{j\pm1}=\E^{\pm\I k'a}u^{k'}_{j}$}
  \sum_{k'}a_{k}(t)\sum_{j}\overline{u^{k}_{j}}
\bigl(\E^{\I k'a}u^{k'}_{j} -2u^{k}_{j} + \E^{-\I k'a}u^{k'}_{j}\bigr)
\step{distributivity}
  \sum_{k'}a_{k}(t)\sum_{j}%\overline{u^{k}_{j}}
\bigl(\E^{\I k'a}\overline{u^{k}_{j}}u^{k'}_{j} -2\overline{u^{k}_{j}}u^{k}_{j} + \E^{-\I k'a}\overline{u^{k}_{j}}u^{k'}_{j}\bigr)
\step{summing over $j$, orthonormality}
\sum_{k'}a_{k'}%\overline{u^{k}_{j}}
\bigl(\E^{\I k'a}\delta_{k,k'} -2\delta_{k,k'} + \delta_{k,k'}\E^{-\I k'a}\bigr)
\step{summing over $k'$}
a_{k}\bigl(\E^{\I ka} -2 + \E^{-\I ka}\bigr)
\end{calculation}
This simplifies the equations of motion to
\begin{equation}
\ddot{a}_{k}(t) = \frac{k_{s}}{m}\bigl(\E^{\I ka}+\E^{-\I ka}-2\bigr)a_{k}(t).
\end{equation}
This is the differential equation of a harmonic oscillator with
frequency
\begin{equation}
\omega_{k} = \sqrt{2\frac{k_{s}}{m}(1 - \cos(ka))}.
\end{equation}
So the equations of motion is $\ddot{a}_{k}(t)=-\omega_{k}^{2}a_{k}(t)$.

Now after a long detour, we announce our grand achievement: we have
changed the equations of motion to describe a system of \emph{uncoupled}
simple harmonic oscillators.

\begin{exercise}
What are the dimensions to $\omega_{k}$? They \emph{should} be in units
of frequency (``per unit time''). If the dimensions for $\omega_{k}$
don't match our expectations, what went wrong? How can we remedy the
situation?
\end{exercise}

\N{General Solution}
The solution to the equations of motion may be written down as
\begin{equation}
a_{k}(t) = b_{k}\E^{-\I\omega_{k}t} + \overline{b_{-k}}\E^{+\I\omega_{k}t}.
\end{equation}
The general expansion for the coordinates
\begin{equation}
q_{j}(t) = \sum_{k}(b_{k}\E^{-\I\omega_{k}t} + \overline{b_{-k}}\E^{+\I\omega_{k}t})u^{k}_{j}.
\end{equation}
We can use distributivity, substitute $k\to-k$ in the second term, and
we obtain
\begin{equation}
  \begin{split}
q_{j}(t) &= \sum_{k}(b_{k}\E^{-\I\omega_{k}t}u^{k}_{j} +\overline{b_{k}}\E^{+\I\omega_{k}t}\overline{u^{k}_{j}})\\
&=\frac{1}{\sqrt{N}}\sum_{k}\bigl(b_{k}\E^{-\I(\omega_{k}t-kan)}
+\overline{b_{k}}\E^{+\I(\omega_{k}t-kan)}\bigr).
  \end{split}
\end{equation}
Observe, since the second term is the complex conjugate of the first
term, the sum produces a real number.

\N{Momentum}
The conjugate momentum, from Eq~\eqref{eq:classical-field-theory:linear-chain:momentum},
may be written down as a sum of basis functions
\begin{equation}
p_{j}(t) = m\sum_{k}(-\I\omega_{k})\bigl(b_{k}\E^{-\I\omega_{k}t}u^{k}_{j} - \overline{b_{k}}\E^{+\I\omega_{k}t}\overline{u^{k}_{j}}\bigr)
\end{equation}

\begin{exercise}
Verify the Hamiltonian satisfies
\begin{equation}
H = \sum_{k}2m\omega_{k}^{2}\overline{b_{k}}b_{k}.
\end{equation}
Observe the Hamiltonian is a constant with respect to time.
[Hint: this is just a straightforward calculation of the kinetic term
  using the series expansion of momenta, and the potential term using
  the series expansion of position, and adding them together. A trig
  identity may be needed.]
\end{exercise}

\N{Poisson Bracket}
The Poisson brackets between the normal coordinates $b_{k}$ and
$\overline{b_{k'}}$ take the simple form
\begin{subequations}
\begin{equation}
\{b_{k},\overline{b_{k'}}\}_{PB} = \frac{-\I}{2m\omega_{k}}\delta_{k,k'},
\end{equation}
\begin{equation}
\{b_{k},b_{k'}\}_{PB} = \{\overline{b_{k}},\overline{b_{k'}}\}_{PB} = 0.
\end{equation}
\end{subequations}

\begin{exercise}
Verify this is, indeed, the Poisson bracket relations for the normal
coordinates. [Hint: using the series expansion for $q_{j}(t)$ and
  $p_{j}(t)$, write $b_{k}\sim\sum_{j} (??) [q_{j}(t)\pm(\omega_{k}m)^{-1}p_{j}(t)]$,
where $(??)$ is some factor to be determined,
then use the ``vanilla'' Poisson brackets.]
\end{exercise}

\N{References}
Greiner~\cite{Greiner:1996zu} discusses the linear chain in quite a bit
of detail, both classically and quantum mechanically. Care must be
taken, he is a bit slick with explanations at times, and there are a few
typos in his equations (which do not affect the results but will confuse
the reader trying to reproduce the equations).

\subsection{Continuum Limit}

\M Let us consider the continuum limit for the Lagrangian and
Hamiltonian expressions.
We can cheat and realize the answer we're looking for should produce the
equations of motion
\begin{equation}
\partial^{\mu}\partial_{\mu}q(\vec{x},t)=0.
\end{equation}
Then write down a Lagrangian which produces this equation of motion.
But let's use that as a last resort.

\M
We can take the Lagrangian from Eq~\eqref{eq:classical-field-theory:linear-chain:lagrangian-with-point-masses},
\begin{equation*}
L = \sum_{j\in\ZZ}\frac{m}{2}\dot{x}^{2}_{j} - \frac{k}{2}(x_{j+1}-x_{j})^{2}.
\end{equation*}
The continuum limit will keep $k^{2}a$ constant, equal to $v^{2}m$.
We have
\begin{equation}
\frac{k}{2}\frac{a^{2}}{a^{2}}(x_{j+1}-x_{j})^{2}
=\frac{a^{2}k}{2}\left(\frac{x_{j+1}-x_{j}}{a}\right)^{2}\xrightarrow{a\to0}
\frac{v^{2}m}{2}\left(\frac{\partial v(x,t)}{\partial x}\right)^{2}.
\end{equation}
The heuristic is that $\Delta x_{j}\approx a$.

\N{Continuum Limit}
As $a\to0$, we replace in one-dimension:
\begin{equation}
\sum_{j}(\dots)\to\int(\dots)\frac{\D x}{a}.
\end{equation}
In $n$ spatial dimensions
\begin{equation}
\sum_{j_{1},\dots,j_{n}}(\dots)\to\int(\dots)\frac{\D^{n} x}{a^{n}}.
\end{equation}


\N{Kinetic Term}
The kinetic energy term
\begin{equation}
K = \sum_{j}\frac{1}{2}m\dot{q}^{2}_{j}.
\end{equation}
The continuum limit requires introducing the ``mass density''\marginnote{$\rho=m/a$} $\rho=m/a$
so
\begin{equation}
K = \sum_{j}\frac{1}{2}a\rho\dot{q}^{2}_{j},
\end{equation}
then the continuum limit becomes
\begin{equation}
\sum_{j}\frac{1}{2}a\rho\dot{q}^{2}_{j}\xrightarrow{a\to0}\int\rho\dot{q}(x,t)^{2}\,\D x.
\end{equation}
Hence the continuum limit for the kinetic term is:
\begin{equation}
\boxed{K = \int\rho\dot{q}(x,t)^{2}\,\D x.}
\end{equation}

\N{Potential Term}
We now have the potential term
\begin{equation}
V = \sum_{j}\frac{k}{2}(q_{j+1}-q_{j})^{2}.
\end{equation}
Observe that the continuum limit of the potential contribution summand is:
\begin{equation}
\frac{a^{2}}{a^{2}}\frac{m}{m}k(q_{j+1}-q_{j})^{2}
=\frac{a^{2}k}{m}m\left(\frac{q_{j+1}-q_{j}}{a}\right)^{2}
\xrightarrow{a\to0}
v^{2}m\left(\frac{\partial q(x,t)}{\partial x}\right)^{2}.
\end{equation}
Then
\begin{equation}
\sum_{j}\frac{k}{2}(q_{j+1}-q_{j})^{2}\xrightarrow{a\to0}\int \frac{v^{2}\rho}{2}
\left(\frac{\partial q(x,t)}{\partial x}\right)^{2}\,\D x.
\end{equation}
That is to say,
\begin{equation}
\boxed{V = \int \frac{v^{2}\rho}{2}\left(\frac{\partial q(x,t)}{\partial x}\right)^{2}\,\D x.}
\end{equation}

\M
Combining results, we have the continuum limit for the Lagrangian give
us
\begin{equation}
L = \int\rho\left[\frac{1}{2}\dot{q}^{2}-\frac{1}{2}v^{2}\left(\frac{\partial q(x,t)}{\partial x}\right)^{2}\right]\D x.
\end{equation}

\N{Re-Scaling Position}
We want to re-scale $q(x,t)$ by some physical constants
$q(x,t)\mapsto\varphi(x,t)$ so its physical dimension works out and our
Lagrangian is
\begin{equation}
  \begin{split}
L &= \int \rho v^{2}\partial^{\mu}q\partial_{\mu}q\,\D^{n}x\\
&= \int \partial^{\mu}\varphi\partial_{\mu}\varphi\,\D^{n}x.
  \end{split}
\end{equation}
Then
\begin{equation}
\mathsf{M}\mathsf{L}^{2}\mathsf{T}^{-2} = [L] = [\int \partial^{\mu}\varphi\partial_{\mu}\varphi\,\D^{n}x],
\end{equation}
and so in $n$ spatial dimensions, since $[\D^{n}x]=\mathsf{L}^{n}$,
\begin{equation}
[\partial^{\mu}\varphi\partial_{\mu}\varphi] = \mathsf{M}\mathsf{L}^{2-n}\mathsf{T}^{-2}.
\end{equation}
Then
\begin{equation}
[\varphi^{2}] = \mathsf{M}\mathsf{L}^{4-n}\mathsf{T}^{-2},
\end{equation}
and in particular
\begin{equation}
[\varphi] = \mathsf{M}^{1/2}\mathsf{L}^{2-(n/2)}\mathsf{T}^{-1}.
\end{equation}
Observe further
\begin{equation}
[\rho]=\mathsf{M}^{1}\mathsf{L}^{-n},\quad\mbox{and}\quad
[v]=\mathsf{L}^{1}\mathsf{T}^{-1}.
\end{equation}
Dimensional analysis suggests our new variable should
be:\marginnote{Change of variables $q\to\varphi$}
\begin{equation}\label{eq:classical-field-theory:linear-chain:change-of-variables-to-varphi}
\boxed{\varphi(x,t) := \sqrt{\rho}vq(x,t).}
\end{equation}
The Lagrangian under change of variable $q\to\varphi$ is then
\begin{equation}
L = \int\left(\frac{1}{v^{2}}\left(\frac{\partial\varphi(x,t)}{\partial t}\right)^{2}
-\left(\frac{\partial\varphi(x,t)}{\partial x}\right)^{2}\right)\D x.
\end{equation}

\begin{exercise}
Re-perform this analysis in $n=2$ spatial dimensions and verify
everything works out all the same. Then try working it out in $n=3$
spatial dimensions. 
\end{exercise}

\N{Field quantity}
The change of variables in
Eq~\eqref{eq:classical-field-theory:linear-chain:change-of-variables-to-varphi}
carries over to $n$ spatial dimensions as
\begin{equation}
\varphi(\vec{x},t) := \sqrt{\rho}vq(\vec{x},t).
\end{equation}
This describes a so-called scalar field quantity $\varphi$.

\N{Speed of Propagation}
Following special relativity, we usually set the speed of propagation Eq~\eqref{eq:classical-field-theory:linear-chain:speed-of-propagation}
to the speed of light, so
\begin{equation}
v^{2} = \frac{a^{2}k}{m} = c^{2}.
\end{equation}
This gives us $c^{2}m=a^{2}k$.

\N{Lagrangian Density}
It is useful to focus on the integrand of the Lagrangian, rather than
the entire Lagrangian itself. The integrand is then called the
\define{Lagrangian Density}, and written as $\mathcal{L}$. For us,
\begin{equation}
\mathcal{L} = -\frac{1}{2}\partial_{\mu}\varphi\partial^{\mu}\varphi.
\end{equation}
This is with Einstein summation convention. Note the negative sign, this
is because
\begin{equation}
\partial_{\mu}\varphi\partial^{\mu}\varphi = -c^{-2}(\partial_{t}\varphi)^{2}
+(\nabla\varphi)\cdot(\nabla\varphi).
\end{equation}
This differs from the Lagrangian we derived before by precisely an
overall sign.

\N{Mass Term}
We can add a mass term to the Lagrangian
\begin{equation}
\mathcal{L}=-\frac{1}{2}\partial_{\mu}\varphi\partial^{\mu}\varphi-\frac{1}{2}\frac{m^{2}c^{2}}{\hbar^{2}}\varphi^{2}.
\end{equation}
We may sometimes use the notation
\begin{equation}
\mu = \frac{mc}{\hbar},
\end{equation}
which simplifies the mass term to $-\frac{1}{2}\mu^{2}\varphi^{2}$.

\begin{exercise}
If we return to the linear chain's Lagrangian, what term must be added
to the potential energy $V_{\text{mass}}$ which will produce the mass term when
we take the continuum limit? That is, such that $V_{\text{mass}}\xrightarrow{a\to0}-\frac{1}{2}\mu^{2}\varphi^{2}$? How should we interpret this term
in the linear chain with [finite or countably many] point-masses?
\end{exercise}

\begin{exercise}
What is the dimension of $\mu$? 
If we worked in $n$ spatial dimensions, does the units of $\mu$ change?
\end{exercise}

\begin{exercise}
Suppose we had $n=2$ spatial dimensions and the point-masses were
constant along the $x$-direction but varied continuously along the
$y$-direction. Prove this is not Lorentz-invariant.
\end{exercise}

\begin{exercise}
Suppose we had $n=2$ spatial dimensions and the spring constants were
identical along the $x$-direction but varied continuously along the
$y$-direction. Prove this is not Lorentz-invariant.
\end{exercise}


\begin{exercise}
Consider $n=2$ dimensions. Suppose we created a network with identical
point-masses and identical [massless] springs using equilateral
triangles instead of squares. Does this affect the continuum limit?
What about other tilings of the plane?

Is this still true in $n=3$? [Hint: the regular tetrahedron will not
fill $\RR^{3}$, you need a different polyhedron.]

If this is not true, how do we interpret this result?
\end{exercise}

\N{Cautionary Remark}\index{Field (Physics)!Section of Bundle}
It is tempting to think of the scalar field $\varphi(\vec{x},t)$ as
``just'' a smooth function $\varphi\colon\RR^{3,1}\to\RR$, and in flat
spacetime we can make this identification. But it is misleading, because
it is a partial truth. In reality we have a [real] line bundle over
spacetime $\mathcal{M}=\RR^{3,1}$, and scalar fields are sections of
this line bundle.

From this perspective, we can change the manifold describing spacetime
to an arbitrary Lorentzian manifold $\mathcal{M}$, then a field is just
a section of a suitable bundle over spacetime $\mathcal{M}$.
This is the correct way to view things, but horribly abstract to
physicists (their eyes just glaze over when making these distinctions).

For technical reasons, we want a locally trivializable fibre bundle over
spacetime, which we will discuss later on.
