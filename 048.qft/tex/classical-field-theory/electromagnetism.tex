\section{Electromagnetism}

\M
We have reviewed in section~\ref{section:relativity:electromagnetism}
the covariant formalism of electromagnetism. From the perspective of
classical field theory, we now know the ``correct'' way to think of
things is that the field quantity of interest is the 4-potential
$A^{\mu}$ (\S\ref{chunk:relativity:electromagnetism:four-potential}).

\N{Lagrangian Density}
We recover Maxwell's equations using the Lagrangian
\begin{equation}
\mathcal{L} = \frac{-1}{4}F^{\alpha\beta}F_{\alpha\beta} = \frac{-1}{4}\eta^{\alpha\mu}\eta^{\beta\nu}F_{\mu\nu}F_{\alpha\beta}.
\end{equation}
If we work in curved spacetime, we need to multiply by the determinant
of the metric tensor $\sqrt{-\det(g_{\mu\nu})}$, but we will ignore this
factor.

\N{Equations of Motion}
We can now determine the equations of motion for the Lagrangian density.
The Euler--Lagrange equations take the form
\begin{equation}
\partial_{\mu}\frac{\partial\mathcal{L}}{\partial(\partial_{\mu}A_{\nu})}-\frac{\partial\mathcal{L}}{\partial A_{\nu}}=0.
\end{equation}
These will turn out to be:
\begin{equation}
\boxed{\partial_{\mu}F^{\mu\nu} = 0.}
\end{equation}
This is the source-free Maxwell's equations.

\begin{proof}[Proof (slick)]
We can compute the variation of the action directly, ignoring boundary terms,
as
\begin{calculation}
\variation\mathcal{L}
\step{unfolding the definition of the Lagrangian density}
\variation\left(\frac{-1}{4}\eta^{\alpha\mu}\eta^{\beta\nu}F_{\mu\nu}F_{\alpha\beta}\right)
\step{product rule and index gymnastics}
\frac{-1}{4}\eta^{\alpha\mu}\eta^{\beta\nu}F_{\mu\nu}(2\,\variation F_{\alpha\beta})
\step{unfolding the definition of field-strength tensor}
\frac{-1}{2}\eta^{\alpha\mu}\eta^{\beta\nu}F_{\mu\nu}(\variation\partial_{\beta}A_{\alpha}-\variation\partial_{\beta}A_{\alpha})
\step{index gymnastics, antisymmetry of field-strength tensor}
-\eta^{\alpha\mu}\eta^{\beta\nu}F_{\mu\nu}\variation(\partial_{\alpha}A_{\beta})
\step{integration by parts}
\eta^{\alpha\mu}\eta^{\beta\nu}(\partial_{\alpha}F_{\mu\nu})\variation A_{\beta}
+\mbox{(boundary terms)}.
\end{calculation}
This vanishes when
\begin{equation}
\partial_{\mu}F^{\mu\nu} = 0,
\end{equation}
and this is the result from the Euler--Lagrange equations of motion.
\end{proof}

\M
We can unfold the result of the Euler--Lagrange equations for
electromagnetism, and find
\begin{equation}
\partial_{\mu}F^{\mu\nu}=0\iff g^{\alpha\gamma}\partial_{\gamma}F_{\alpha\beta}=0.
\end{equation}
Then
\begin{calculation}
  g^{\alpha\gamma}\partial_{\gamma}F_{\alpha\beta}
\step{unfold definition of field-strength tensor}
g^{\alpha\gamma}\partial_{\gamma}(\partial_{\alpha}A_{\beta}-\partial_{\beta}A_{\alpha})
\step{distributivity, index gymnastics}
\partial^{\alpha}\partial_{\alpha}A_{\beta}-\partial^{\alpha}\partial_{\beta}A_{\alpha}
\step{equations of motion}
0.
\end{calculation}
When we impose the gauge condition $\partial^{\alpha}A_{\alpha}=0$,
we recover the familiar Maxwell equations as a wave equation
$\Box A_{\beta}=0$.

\N{Lagrangian coupled to matter}
We can write down the Lagrangian density for electromagnetism coupled to
some charged matter, recovering the Maxwell equations with some source
as in
Eq~\eqref{eq:relativity:electromagnetism:maxwell-for-electric-field}:
\begin{equation}\label{eq:classical-field-theory:electromagnetism:lagrangian}
\mathcal{L} = \frac{-1}{4}F^{\alpha\beta}F_{\alpha\beta}-4\pi J^{\mu}A_{\mu}.
\end{equation}

\begin{exercise}
Recall (\S\ref{ex:classical-field-theory:noether:canonical-stress-energy})
the notion of the canonical stress--energy tensor. Calculate
$\canonicalStressEnergy_{\mu\nu}$ for the Lagrangian in Eq~\eqref{eq:classical-field-theory:electromagnetism:lagrangian}.

[Hint: your answer should \emph{not} be symmetric --- that is, 
$\canonicalStressEnergy_{\mu\nu}\neq\canonicalStressEnergy_{\nu\mu}$.]
\end{exercise}

\N{Heuristic regarding interactions}\index{Heuristics}
If we want to describe interactions between two fields, or a field and
some matter, then we need to add a term to our Lagrangian of the form:
\begin{equation}
\mathcal{L}_{\text{interaction}}\sim\begin{pmatrix}\mbox{coupling}\\
\mbox{constant}
\end{pmatrix}\begin{pmatrix}\mbox{field}\\
\mbox{quantity}
\end{pmatrix}\begin{pmatrix}\mbox{matter}\\
\mbox{terms}
\end{pmatrix}.
\end{equation}
This is added to the potential term in the Lagrangian.

\subsection{Hamiltonian Formalism}

\M We will work through the calculations of the Symplectic two-form as a
series of exercises, and show it is degenerate. This is a consequence of
gauge symmetries. Then we will work through the Hamiltonian formalism
with its phase space parametrized by initial conditions.

\begin{exercise}
Prove $\displaystyle\variation L = \int\bigl(\variation A_{\beta}\partial_{\alpha}F^{\alpha\beta}+\partial_{0}(-F^{0\beta}\,\variation A_{\beta})\bigr)\,\D^{3}x$.
\end{exercise}

\begin{exercise}
  Prove the Symplectic potential is
  \[ \Theta(\variation A) = \int F^{\beta0}\,\variation A_{\beta}\,\D^{3}x =\int F^{i0}\,\variation A_{i}\,\D^{3}x\]
\end{exercise}

\begin{exercise}
Suppose $A^{\alpha}$ is a solution to the Maxwell's equations. Determine
what the linearized Maxwell equations are for tangent ``vectors''
$\variation A^{\beta}$ with base ``point'' $A^{\alpha}$.
\end{exercise}

\begin{exercise}[Symplectic form]
We will compute the Symplectic two-form for electromagnetism, and verify
it is degenerate when one of the fields is pure gauge.
\begin{enumerate}
\item Show the [naive] Symplectic two-form for electromagnetism $\Omega=\D\Theta$
is
\[\Omega(\variation_{1}A,\variation_{2}A) = \int\bigl((\partial_{t}\variation_{1}A^{i}-\partial^{i}\variation_{1}A_{t})\variation_{2}A_{i}-(\partial_{t}\variation_{2}A^{i}-\partial^{i}\variation_{2}A_{t})\variation_{1}A_{i}\bigr)\,\D^{3}x.\]
\item Consider $\variation_{2}A_{t}=\partial_{t}\Lambda$ and
  $\variation_{2}A_{i}=\partial_{i}\Lambda$ where $\Lambda$ is any
  function with compact support. Show
\[\int(\partial_{t}\variation_{1}A^{i}-\partial^{i}\variation_{1}A_{t})\variation_{2}A_{i}\,\D^{3}x=-\int\Lambda\partial_{i}(\partial_{t}\variation_{1}A^{i}-\partial^{i}\variation_{1}A_{t})\,\D^{3}x.\]
\item Show (if you haven't already) the linearized Maxwell's equations
  includes $\partial^{i}(\partial_{t}\variation A_{i}-\partial_{i}\variation A_{t})=0$.
\item Comparing the last two steps in this exercise, show the integral
  from step 2 vanishes, and this implies $\Omega(\variation_{1}A,\variation_{2}A) =0$
 (i.e., $\Omega$ is degenerate).
\end{enumerate}
\end{exercise}

\N{Initial Data}
Assuming we have picked some time-slicing and we have some initial data
\begin{equation}
\phi:=-A_{t}(\vec{x},t=0),\quad\mbox{and}\quad Q_{i}:=A_{i}(\vec{x},t=0),
\end{equation}
we can find the canonically conjugate momentum to $Q_{i}$ as,
\begin{subequations}
\begin{align}
P^{i} &= \frac{\partial\mathcal{L}}{\partial(\partial_{t}A_{i})}\\
&=\partial_{t}A_{i}-\partial_{i}A_{t}\\
&=\partial_{t}Q_{i}+\partial_{i}\phi.
\end{align}
\end{subequations}

\begin{exercise}
Verify that $\displaystyle\frac{\partial\mathcal{L}}{\partial(\partial_{t}A_{t})}=0$,
and therefore the canonically conjugate momentum for $\phi$ vanishes.
\end{exercise}

\begin{exercise}
Rewrite the Lagrangian as a functional of $Q_{i}$, $P^{i}$, and
$\phi$. Try to write it in ``canonical form'', i.e., as
\[ L[\phi, Q_{i}, P^{i}] = \int \bigl(P^{i}\partial_{t}Q^{i}-\mbox{(something)}\bigr)\,\D^{3}x.\]
[Hint: integration by parts and the divergence theorem are your friends.]
\end{exercise}

\M
The Lagrangian which you ought to obtain from the previous exercises
should be:
\begin{equation}\label{eq:classical-field-theory:electromagnetism:canonical-formalism:lagrangian}
L =
\int\left(P^{i}\partial_{t}Q_{i}-\left[\frac{1}{2}(P_{i}P^{i}+\frac{F_{ij}F^{ij}}{2}) + \phi\,\partial_{i}P^{i}\right]\right)\D^{3}x.
\end{equation}
Observe then that the equations of motion for $\phi$ are precisely
Gauss's Law:
\begin{equation}
\partial_{i}P^{i} = 0.
\end{equation}
We interpret $\phi$ as a Lagrange multiplier.

\begin{exercise}
Using the Lagrangian you should have computed (or lifted from the
previous chunk), prove the Euler--Lagrange equations:
\begin{equation}
\frac{\variation L}{\variation P^{i}}-\frac{\D}{\D t}\frac{\variation L}{\variation\partial_{t}P^{i}}=0,\quad
\frac{\variation L}{\variation Q^{i}}-\frac{\D}{\D t}\frac{\variation L}{\variation\partial_{t}Q^{i}}=0,\quad
\frac{\variation L}{\variation\phi}=0,
\end{equation}
are equivalent to the Maxwell equations.
\end{exercise}

\N{Hamiltonian}
We find the Hamiltonian functional by inspection of
Eq~\eqref{eq:classical-field-theory:electromagnetism:canonical-formalism:lagrangian}
to be:
\begin{equation}
H = \int\left(\frac{1}{2}(P_{i}P^{i}+\frac{F_{ij}F^{ij}}{2}) + \phi\,\partial_{i}P^{i}\right)\D^{3}x.
\end{equation}
The first two terms coincide with our expectations, but the last term
may be surprising.

\begin{ddanger}
When we have a constrained Hamiltonian system, we add the first-class
constraints to the Hamiltonian. This is precisely what's going on with
the Hamiltonian functional as we've written it down. This is studied
thoroughly in Henneaux and Teitelboim~\cite{Henneaux:1992ig}.
\end{ddanger}

\begin{exercise}\index{Hamilton's equations}
The reader can verify Hamilton's equations,
\begin{subequations}
\begin{align}
\partial_{t}Q_{i} &= \frac{\delta H}{\delta P^{i}} = P_{i} - \partial_{i}\phi\\
\partial_{t}P_{i} &= -\frac{\delta H}{\delta P^{i}} = \partial_{i}F^{ij}.
\end{align}
\end{subequations}
\end{exercise}

\begin{exercise}\index{Poisson bracket}
Using the Poisson brackets,
\begin{equation}
\PB{M}{N} = \int\left(\frac{\delta M}{\delta Q_{i}(\vec{x}')}\frac{\delta N}{\delta P^{i}(\vec{x}')}
-\frac{\delta N}{\delta Q_{i}(\vec{x}')}\frac{\delta M}{\delta P^{i}(\vec{x}')}\right)\D^{3}x',
\end{equation}
show the quantity
\begin{equation}
G = - \int\Lambda(\vec{x})\partial_{i}P^{i}\,\D^{3}x
\end{equation}
is the generating function for gauge transformations
\begin{equation}
\variation Q_{i}=\PB{Q_{i}}{G}=\partial_{i}\Lambda,\quad
\variation P^{i}=\PB{P^{i}}{G}=0.
\end{equation}
\end{exercise}

\subsection{Scalar Electrodynamics}

\M
One of the first models we study in quantum field theory is something
called ``scalar electrodynamics''. This is obtained by taking a complex
Scalar field and coupling it to Electromagnetism. Let us review the
pertinent aspects of the complex Scalar field, then let us try to couple
it to electromagnetism.

\subsubsection{Complex Scalar Field}

\M We take 2 real-valued scalar fields
$\varphi_{1}$ and $\varphi_{2}$, then form the complex scalar
field\footnote{This is an abuse of notation, similar to using $z$ and
$\bar{z}$ in complex analysis as the independent coordinates of the
Complex plane.} 
\begin{equation}
\varphi(x) = \frac{\varphi_{1}(x)+\I\varphi_{2}(x)}{\sqrt{2}},\quad\mbox{and}\quad
\varphi^{*}(x) = \frac{\varphi_{1}(x)-\I\varphi_{2}(x)}{\sqrt{2}}.
\end{equation}
Then we couple the complex scalar field to electromagnetism.

\N{Complex Scalar Field}\index{Scalar Field!Complex}
The complex scalar field (also called the \emph{charged Klein--Gordon field})
may be viewed as a mapping
\begin{equation}
\varphi\colon\RR^{3,1}\to\CC.
\end{equation}
The Lagrangian for the complex Scalar field is:
\begin{equation}
\mathcal{L}_{cs} = -(\eta^{\alpha\beta}\partial_{\alpha}\varphi\partial_{\beta}\varphi^{*}+\mu^{2}|\varphi|^{2}).
\end{equation}

\begin{exercise}
Show the Euler--Lagrange equations give you the equations of motion
\begin{subequations}
\begin{align}
\frac{\variation S}{\variation\varphi}=0 &\implies (\partial^{\alpha}\partial_{\alpha}+\mu^{2})\varphi^{*}=0,
\intertext{and}  
\frac{\variation S}{\variation\varphi^{*}}=0 &\implies (\partial^{\alpha}\partial_{\alpha}+\mu^{2})\varphi=0.
\end{align}
\end{subequations}
\end{exercise}

\N{Symmetries of Complex Scalar Field}
We can use Noether's theorem to find that the complex scalar field
admits a continuous symmetry:
\begin{equation}
\varphi_{\lambda}=\E^{\I\lambda}\varphi,\quad
\varphi_{\lambda}^{*}=\E^{-\I\lambda}\varphi^{*},
\end{equation}
where $\lambda\in\RR$ is arbitrary. Since $\lambda$ is a constant, we
see the kinetic term of the Lagrangian is invariant under this
transformation. Similarly, we see $|\varphi|^{2}=\varphi^{*}\varphi$ is
invariant under this transformation.

\begin{remark}\index{Symmetry!Global}\index{Symmetry!Local}
This is a ``global $\U(1)$'' symmetry. It's ``global'' because the
parameter $\lambda$ is a real number independent of spacetime. (If
$\lambda$ were a function of spacetime $\lambda=\lambda(\vec{x},t)$,
then we would call it a ``local'' symmetry.) It's a $\U(1)$ symmetry
because $\E^{\I\lambda}\in\U(1)$.

Older literature use the term ``gauge transformation of the first kind''\index{Gauge!Transformation!Of the first kind}
instead of ``global symmetry transformation''
\end{remark}

\begin{exercise}
Use Noether's theorem to prove this is a continuous symmetry of the
complex scalar field. Then determine the conserved current $j^{\beta}$
for this symmetry.
\end{exercise}
\begin{exercise}
Prove the Noether charge for the complex Scalar field is
\begin{equation}
Q = \I\int_{V}(\varphi^{*}\partial_{t}\varphi-\varphi\partial_{t}\varphi^{*})\,\D^{3}x.
\end{equation}
\end{exercise}

\N{Sigma Models}\index{Sigma model@$\sigma$ Model}
There is a way to generalize this construction from 2 real Scalar fields
to $N$ real Scalar fields. This is a family of models called
\define{Sigma models} where the scalar fields are components of a smooth
function $\sigma\colon\RR^{3,1}\to\mathcal{M}$ where $\mathcal{M}$ is
usually a Lie group. (There is no significance to the choice of $\sigma$
for scalar fields, and Sigma models refer to this historic artifact of
arbitrary notation.) Then the Lagrangian density for the massless case
is:
\begin{equation}
\mathcal{L} = \frac{1}{2}\sum^{N}_{A,B=1}g_{AB}(\sigma)\partial^{\mu}\sigma^{A}\partial_{\mu}\sigma^{B},
\end{equation}
where $G_{AB}(\sigma)$ is the metric tensor on the field space $\mathcal{M}$,
and $\partial_{\mu}$ are the derivatives on the underlying spacetime
manifold $\RR^{3,1}$. When we include some self-interaction terms, we
obtain a \emph{Nonlinear $\sigma$ Model}.

When $\mathcal{M}=\CC^{2}$, for example, we can show the $\sigma$ model
enjoys an $\SU(2)$ symmetry. Similarly, for $\mathcal{M}=\CC^{n}$, the
$\sigma$ model enjoys an $\SU(n)$ symmetry. These models are useful as
``prolegomenon'' to Yang--Mills theory for the Standard Model.

\begin{remark}
Sigma models were first introduced in \S\S5--6 of Gell-Mann and Levy~\cite{Gell-Mann:1960mvl}.
Initially, $\sigma$ was ``just another scalar field'' in that paper. Later
physicists adopted $\sigma$ as we have introduced it: as a familar of
scalar fields.
\end{remark}

\subsubsection{Charged Scalar Field coupled to Electromagnetism}

\M
Now we can couple the complex Scalar field to Electromagnetism.
The basic idea is we will form the Lagrangian density for scalar
electrodynamics by adding the Lagrangian density for the complex Scalar
field to the Lagrangian density for the Electromagnetic field, plus the
4-current coupling the charged Scalar field to the Electromagnetic field:
\begin{equation}
\begin{split}
\mathcal{L}_{sED}&=\mathcal{L}_{cs}+\mathcal{L}_{EM}+\mathcal{L}_{int}\\
&=-(\eta^{\alpha\beta}\partial_{\alpha}\varphi\partial_{\beta}\varphi^{*}+\mu^{2}|\varphi|^{2})
-\frac{1}{4}F^{\alpha\beta}F_{\alpha\beta}
+4\pi j^{\alpha}A_{\alpha}.
\end{split}
\end{equation}
We just need to determine $j^{\alpha}$ in terms of the complex Scalar
field $\varphi$.

We know from Noether's theorem there is a conserved current for the
$\U(1)$ Symmetry for the complex Scalar field,
\begin{equation}
j^{\alpha} = -\I\eta^{\alpha\beta}(\varphi^{*}\partial_{\beta}\varphi - \varphi\partial_{\beta}\varphi^{*}).
\end{equation}
There is some slight difficulties with the Lagrangian as we have written
it: it is no longer gauge invariant under $A^{\mu}\to
A^{\mu}+\partial^{\mu}\Lambda$.

\N{Minimal coupling}
The ``physically correct way'' to get a gauge-invariant Lagrangian which
still gives the $j^{\alpha}A_{\alpha}$ coupling is rather unintuitive:
we use a different differential operator than $\partial_{\mu}$ in the
charged Scalar field's Lagrangian density.

This is the so-called \define{Minimal Coupling}, where we replace
\begin{subequations}
\begin{equation}
\partial_{\alpha}\varphi\to D_{\alpha}\varphi := (\partial_{\alpha}+\I qA_{\alpha})\varphi,
\end{equation}
and
\begin{equation}
\partial_{\alpha}\varphi^{*}\to D_{\alpha}\varphi^{*} := (\partial_{\alpha}-\I qA_{\alpha})\varphi^{*}.
\end{equation}
\end{subequations}
Here $q$ is a parameter reflecting the coupling strength between the
charged scalar field $\varphi$ and the Electromagnetic field. It's an
example of a \emph{coupling constant}.\index{Coupling constant}
Then we modify the complex Scalar field's Lagrangian density to use
these gauge covariant derivatives,
\begin{equation}\label{eq:classical-field-theory:electromagnetism:lagrangian-density-for-complex-scalar-field-using-gauge-covariant-derivatives}
\mathcal{L}_{cs} = -\eta^{\alpha\beta}D_{\alpha}\varphi^{*}D_{\beta}\varphi-\mu^{2}|\varphi|^{2}.
\end{equation}
This Lagrangian density yields field equations which are the usual wave
equations plus some modifications involving the electromagnetic
potential.

\begin{exercise}
Compute the Euler--Lagrange equations for $\varphi$ and $\varphi^{*}$ using the Lagrangian density from Eq~\eqref{eq:classical-field-theory:electromagnetism:lagrangian-density-for-complex-scalar-field-using-gauge-covariant-derivatives}.
\end{exercise}

\M
Now observe, under a gauge transformation of the electromagnetic
4-potential
\begin{subequations}
\begin{equation}
A_{\alpha}\to A_{\alpha}+\partial_{\alpha}\Lambda,
\end{equation}
for the gauge covariant derivatives of the complex Scalar field to
remain invariant under these gauge transformations, we need:
\begin{align}
  \varphi&\to\E^{-\I q\Lambda}\varphi,\\
  \intertext{and}
  \varphi^{*}&\to\E^{\I q\Lambda}\varphi^{*}.
\end{align}
\end{subequations}
The reader can verify that the gauge covariant derivatives of the
complex Scalar field then transform as
\begin{subequations}
\begin{align}
D_{\alpha}\varphi &\to\E^{-\I q\Lambda}D_{\alpha}\varphi,\\
D_{\alpha}\varphi^{*} &\to\E^{\I q\Lambda}D_{\alpha}\varphi^{*}.
\end{align}
\end{subequations}
We can see that the kinetic term for the modified complex Scalar
Lagrangian density remains invariant under these transformations.

\M
Since the electromagnetic interaction with the complex Scalar fields are
``swept into'' the gauge covariant derivatives, we can write the
Lagrangian density for the scalar electrodynamic theory as:
\begin{subequations}
\begin{equation}
\mathcal{L}_{sED} = \frac{-1}{4}F^{\alpha\beta}F_{\alpha\beta} - \eta^{\alpha\beta}D_{\alpha}\varphi^{*}D_{\beta}\varphi-\mu^{2}|\varphi|^{2}.
\end{equation}
When we expand the gauge covariant derivatives in this Lagrangian
density, we have:
\begin{equation}
\mathcal{L}_{sED} = \frac{-1}{4}F^{\alpha\beta}F_{\alpha\beta} - \eta^{\alpha\beta}\partial_{\alpha}\varphi^{*}\partial_{\beta}\varphi-\mu^{2}|\varphi|^{2}
+\I q A^{\alpha}(\varphi^{*}\partial_{\alpha}\varphi-\varphi\partial_{\alpha}\varphi^{*}+\I q A_{\alpha}|\varphi|^{2}).
\end{equation}
\end{subequations}
The Euler--Lagrange equations for the Electromagnetic 4-potential are
\begin{equation}
\partial_{\beta}F^{\alpha\beta}=-4\pi J^{\alpha},
\end{equation}
where the current is defined using the gauge covariant derivatives as
\begin{equation}\label{eq:classical-field-theory:sed:charged-current}
J^{\alpha} = -\frac{\I q}{4\pi}(\varphi^{*}D^{\alpha}\varphi-\varphi D^{\alpha}\varphi^{*}).
\end{equation}
This is rather magical, but we could derive the same results using
Noether's first theorem for fields.

\M
We should mention that physicists look at
Eq~\eqref{eq:classical-field-theory:sed:charged-current} and interpret
it as telling us the electromagnetic charge for the complex scalar field
cannot ``exist alone'' in the Scalar field. In an interacting system,
the division between ``source fields'' and ``fields mediating interactions''
is rather artificial and arbitrary. This is physically reasonable (even
if a little surprising). Mathematically this feature emerges from
demanding gauge invariance.

The complex Scalar field is no longer uniquely defined in scalar
electrodynamics: it is subject to a gauge transformation, just like the
electromagnetic 4-potential.

When we have such an interaction, if we want to compute (say) the
electromagnetic field contained in a region $V$, we need a solution
$(A,\varphi)$ of the coupled Maxwell--Scalar equations, then substitute
it into:
\begin{equation}
Q = \frac{1}{4\pi}\int_{V}\I q(\varphi^{*}D_{0}\varphi-\varphi D_{0}\varphi^{*})\,\D^{3}x.
\end{equation}
This charge is conserved and gauge invariant.

\begin{exercise}
Prove the total electric charge $Q$ is conserved \emph{and} gauge invariant.
\end{exercise}

\begin{exercise}
Suppose we had a Lagrangian for complex scalar fields coupled to
electromagnetism of the form
\begin{equation}
\mathcal{L}=\mathcal{L}_{EM}-\frac{1}{2}[\varphi g^{\mu}D_{\mu}\varphi^{*}-\varphi^{*}g^{\mu}D_{\mu}\varphi]-\mu^{2}|\varphi|^{2},
\end{equation}
where $g^{\mu}$ is ``some [constant] vector'', and $D_{\mu}$ is the
gauge covariant derivative.
\begin{enumerate}
\item How must $\varphi$ and $\varphi^{*}$ transform under gauge
  transformations $A_{\alpha}\to A_{\alpha}+\partial_{\alpha}\Lambda$?
\item How do the gauge covariant derivatives $D_{\alpha}\varphi$ and
  $D_{\alpha}\varphi^{*}$ transform under gauge transformations?
\end{enumerate}
\end{exercise}

\subsection{Chern--Simons Theory}

\M
In $2+1$ dimensions (instead of $3+1$ dimensions), we have a particular
action which plays an important role in physics called the Chern--Simons
action named after its discoverer Albert Schwarz:
\begin{equation}
\begin{split}
  \action_{CS}[A] &= \frac{k}{4\pi}\int\epsilon^{\mu\nu\rho}A_{\mu}\partial_{\nu}A_{\rho}\,\D^{3}x\\
&\mbox{``=''}\; \frac{k}{4\pi}\int(A\times\nabla)^{\rho} A_{\rho}\,\D^{3}x
\end{split}
\end{equation}
where $A_{\mu}$ is a ``4''-potential for electromagnetism. Later we will
generalize Electromagnetism to Yang--Mills theory, and the Chern--Simons
theory will play an important role in something called the Quantum Hall
effect. It also describes quantum gravity in $2+1$ dimensions.

For a Yang--Mills field, however, we also have a term that looks like
$A^{3}$ in the action. Such a term vanishes for commutative gauge groups
like $\U(1)$ (i.e., like for Electromagnetism).

\begin{exercise}
From demanding stationary action $\variation\action_{CS}[A]/\variation A^{\mu}(x)=0$,
determine the equations of motion for Chern--Simons theory for the
electromagnetic field.
\end{exercise}

\subsection{*Photon Mass}

\M
We could ostensibly add a mass term to the Maxwell Lagrangian density,
giving us the Proca Lagrangian density:\index{Proca!Lagrangian}
\begin{equation}
\mathcal{L}_{\text{Proca}} = \frac{-1}{4}F^{\alpha\beta}F_{\alpha\beta}-\frac{1}{2}\frac{m_{\gamma}^{2}c^{2}}{\hbar^{2}}A_{\alpha}A^{\alpha},
\end{equation}
where $m_{\gamma}$ is the nonzero photon mass. It's useful to introduce
\begin{equation}
\mu = \frac{m_{\gamma}c}{\hbar},
\end{equation}
to simplify expressions (as we did in the case of the massive Scalar
field). We could relax things further and allow $A^{\mu}$ to be
complex-valued, in which case we have 
\begin{equation}
\mathcal{L}_{\CC\text{Proca}} = \frac{-1}{4}\overline{F^{\alpha\beta}}F_{\alpha\beta}-\frac{1}{2}\frac{m_{\gamma}^{2}c^{2}}{\hbar^{2}}\overline{A_{\alpha}}A^{\alpha},
\end{equation}
where $\overline{X}$ is the complex conjugate of $X$.

\begin{exercise}
Show, for nonzero mass, the Proca equations differ from the Maxwell
equations in electrostatics, with its Gauss Law being $\nabla\cdot\vec{E}=4\pi\rho-\mu^{2}V$,
and its electrodynamics equation, $\partial_{t}\vec{E}-\nabla\times\vec{B}=-4\pi\vec{j}+\mu^{2}\vec{A}$.
Double check the signs and coefficients.
\end{exercise}

\begin{exercise}
If $\mu\neq0$, is the Lagrangian density still invariant under gauge
transformations $A^{\alpha}\to A^{\alpha}\pm\partial^{\alpha}\Lambda$?
\end{exercise}

\N{Experimental Bounds}\index{Mass!Photon!Bounds}\index{Photon!Mass!Bounds}
Currently, the best experimental bounds on the photon mass $m_{\gamma}$,
as reported by Goldhaber and Nieto~\cite[\S III.B.2]{Goldhaber:2008xy}, is:
\begin{equation}
m_{\gamma}\lesssim10^{-18}~\mathrm{eV}.
\end{equation}
This is determined using the solar wind magnetic field and a generous
upper bound for the $\mu^{2}A^{2}$ contribution to energy calculations.

\begin{remark}
We could use Galactic scale data to improve the bounds to
$m_{\gamma}\lesssim10^{-26}~\mathrm{eV}$. This is based on a review from
2008, doubtless the bounds have improved since then.
\end{remark}
