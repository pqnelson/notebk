\section{Electromagnetism}

\M
We have reviewed in section~\ref{section:relativity:electromagnetism}
the covariant formalism of electromagnetism. From the perspective of
classical field theory, we now know the ``correct'' way to think of
things is that the field quantity of interest is the 4-potential
$A^{\mu}$ (\S\ref{chunk:relativity:electromagnetism:four-potential}).

\N{Lagrangian Density}
We recover Maxwell's equations using the Lagrangian
\begin{equation}
\mathcal{L} = \frac{-1}{4}F^{\alpha\beta}F_{\alpha\beta} = \frac{-1}{4}\eta^{\alpha\mu}\eta^{\beta\nu}F_{\mu\nu}F_{\alpha\beta}.
\end{equation}
If we work in curved spacetime, we need to multiply by the determinant
of the metric tensor $\sqrt{-\det(g_{\mu\nu})}$, but we will ignore this
factor.

\N{Equations of Motion}
We can now determine the equations of motion for the Lagrangian density.
The Euler--Lagrange equations take the form
\begin{equation}
\partial_{\mu}\frac{\partial\mathcal{L}}{\partial(\partial_{\mu}A_{\nu})}-\frac{\partial\mathcal{L}}{\partial A_{\nu}}=0.
\end{equation}
These will turn out to be:
\begin{equation}
\boxed{\partial_{\mu}F^{\mu\nu} = 0.}
\end{equation}
This is the source-free Maxwell's equations.

\begin{proof}[Proof (slick)]
We can compute the variation of the action directly, ignoring boundary terms,
as
\begin{calculation}
\variation\mathcal{L}
\step{unfolding the definition of the Lagrangian density}
\variation\left(\frac{-1}{4}\eta^{\alpha\mu}\eta^{\beta\nu}F_{\mu\nu}F_{\alpha\beta}\right)
\step{product rule and index gymnastics}
\frac{-1}{4}\eta^{\alpha\mu}\eta^{\beta\nu}F_{\mu\nu}(2\,\variation F_{\alpha\beta})
\step{unfolding the definition of field-strength tensor}
\frac{-1}{2}\eta^{\alpha\mu}\eta^{\beta\nu}F_{\mu\nu}(\variation\partial_{\beta}A_{\alpha}-\variation\partial_{\beta}A_{\alpha})
\step{index gymnastics, antisymmetry of field-strength tensor}
-\eta^{\alpha\mu}\eta^{\beta\nu}F_{\mu\nu}\variation(\partial_{\alpha}A_{\beta})
\step{integration by parts}
\eta^{\alpha\mu}\eta^{\beta\nu}(\partial_{\alpha}F_{\mu\nu})\variation A_{\beta}
+\mbox{(boundary terms)}.
\end{calculation}
This vanishes when
\begin{equation}
\partial_{\mu}F^{\mu\nu} = 0,
\end{equation}
and this is the result from the Euler--Lagrange equations of motion.
\end{proof}

\M
We can unfold the result of the Euler--Lagrange equations for
electromagnetism, and find
\begin{equation}
\partial_{\mu}F^{\mu\nu}=0\iff g^{\alpha\gamma}\partial_{\gamma}F_{\alpha\beta}=0.
\end{equation}
Then
\begin{calculation}
  g^{\alpha\gamma}\partial_{\gamma}F_{\alpha\beta}
\step{unfold definition of field-strength tensor}
g^{\alpha\gamma}\partial_{\gamma}(\partial_{\alpha}A_{\beta}-\partial_{\beta}A_{\alpha})
\step{distributivity, index gymnastics}
\partial^{\alpha}\partial_{\alpha}A_{\beta}-\partial^{\alpha}\partial_{\beta}A_{\alpha}
\step{equations of motion}
0.
\end{calculation}
When we impose the gauge condition $\partial^{\alpha}A_{\alpha}=0$,
we recover the familiar Maxwell equations as a wave equation
$\Box A_{\beta}=0$.

\N{Lagrangian coupled to matter}
We can write down the Lagrangian density for electromagnetism coupled to
some charged matter, recovering the Maxwell equations with some source
as in
Eq~\eqref{eq:relativity:electromagnetism:maxwell-for-electric-field}:
\begin{equation}\label{eq:classical-field-theory:electromagnetism:lagrangian}
\mathcal{L} = \frac{-1}{4}F^{\alpha\beta}F_{\alpha\beta}-4\pi J^{\mu}A_{\mu}.
\end{equation}

\begin{exercise}
Recall (\S\ref{ex:classical-field-theory:noether:canonical-stress-energy})
the notion of the canonical stress--energy tensor. Calculate
$\canonicalStressEnergy_{\mu\nu}$ for the Lagrangian in Eq~\eqref{eq:classical-field-theory:electromagnetism:lagrangian}.

[Hint: your answer should \emph{not} be symmetric --- that is, 
$\canonicalStressEnergy_{\mu\nu}\neq\canonicalStressEnergy_{\nu\mu}$.]
\end{exercise}

\N{Heuristic regarding interactions}
If we want to describe interactions between two fields, or a field and
some matter, then we need to add a term to our Lagrangian of the form:
\begin{equation}
\mathcal{L}_{\text{interaction}}\sim\begin{pmatrix}\mbox{coupling}\\
\mbox{constant}
\end{pmatrix}\begin{pmatrix}\mbox{field}\\
\mbox{quantity}
\end{pmatrix}\begin{pmatrix}\mbox{matter}\\
\mbox{terms}
\end{pmatrix}.
\end{equation}
This is added to the potential term in the Lagrangian.