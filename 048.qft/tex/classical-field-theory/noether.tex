\section{Noether's Theorem}

\subsection{For Mechanics}

\M
Consider a mechanical system consisting of $N$ particles with positions
$q^{i}(t)$ for $i=1,\dots,N$. We describe it by its action
\begin{equation}
\action[q^{i}(t)] = \int^{t_{2}}_{t_{1}}L(q^{i},\dot{q}^{i})\,\D t.
\end{equation}
Now we suppose the dynamics is invariant under
\begin{equation}
q^{i}(t)\to q^{i}(t)+\delta q^{i}(t)
\end{equation}
where
\begin{equation}\label{eq:classical-field-theory:noether:epsilon-variation}
\delta q^{i}(t) = \epsilon^{a}(t)F^{i}_{a}(q,\dot{q}) + \dot{\epsilon}^{a}
G^{i}_{a}(q,\dot{q}),
\end{equation}
the $\epsilon^{a}$ are arbitrary functions of time (except possibly at
the endpoints) and $a=1,\dots,n$.

\M The action varies like:
\begin{equation}
\delta\action = \int^{t_{2}}_{t_{1}}L_{i}\,\delta q^{i}\,\D t
  + \left.\frac{\partial L}{\partial\dot{q}^{i}}\delta q^{i}\right|^{t_{2}}_{t_{1}},
\end{equation}
where the ``Euler'' derivatives are
\begin{equation}
L_{i} := \frac{\partial L}{\partial q^{i}}
- \frac{\D}{\D t}\frac{\partial L}{\partial\dot{q}^{i}}.
\end{equation}
The equations of motion are satisfied if and only if $L_{i}=0$ (these
are the Euler--Lagrange equations).

\N{Noether's Second Theorem}
Now, the variation of the action using
Eq~\eqref{eq:classical-field-theory:noether:epsilon-variation}
gives us
\begin{calculation}
\delta\action
\step{definition of variation of action}
\int^{t_{2}}_{t_{1}}L_{i}\,\delta q^{i}\,\D t
  + \left.\frac{\partial L}{\partial\dot{q}^{i}}\delta q^{i}\right|^{t_{2}}_{t_{1}}
\step{unfolding $\delta q^{i}$}
\int^{t_{2}}_{t_{1}}L_{i}\bigl(\epsilon^{a}(t)F^{i}_{a}(q,\dot{q}) + \dot{\epsilon}^{a} G^{i}_{a}(q,\dot{q})\bigr)\,\D t
  + \left.\frac{\partial L}{\partial\dot{q}^{i}}\bigl(\epsilon^{a}(t)F^{i}_{a}(q,\dot{q}) + \dot{\epsilon}^{a} G^{i}_{a}(q,\dot{q})\bigr)\right|^{t_{2}}_{t_{1}}
\step{integrate by parts to eliminate $\dot\epsilon^{a}$}
\int^{t_{2}}_{t_{1}}\epsilon^{a}(t)\left(L_{i}F^{i}_{a}(q,\dot{q})
- \frac{\D}{\D t}\bigl(L_{i} G^{i}_{a}(q,\dot{q})\bigr)\right)\D t
+ \left[\epsilon^{a}(t)L_{i}G^{i}_{a}
+ \frac{\partial L}{\partial\dot{q}^{i}}\bigl(\epsilon^{a}(t)F^{i}_{a}(q,\dot{q}) + \dot{\epsilon}^{a} G^{i}_{a}(q,\dot{q})\bigr)\right]^{t_{2}}_{t_{1}}
\step{associativity applied to boundary terms}
\int^{t_{2}}_{t_{1}}\epsilon^{a}(t)\left(L_{i}F^{i}_{a}(q,\dot{q})
- \frac{\D}{\D t}\bigl(L_{i} G^{i}_{a}(q,\dot{q})\bigr)\right)\D t
+ \left[\epsilon^{a}(t)\left(L_{i}G^{i}_{a}
+\frac{\partial L}{\partial\dot{q}^{i}}F^{i}_{a}\right)
+ \dot{\epsilon}^{a}\frac{\partial L}{\partial\dot{q}^{i}}G^{i}_{a}(q,\dot{q})\right]^{t_{2}}_{t_{1}}
\end{calculation}
As usual, we ignore boundary terms, then invariance of the action (up to
boundary terms) demands that:
\begin{equation}
\boxed{L_{i}F^{i}_{a}(q,\dot{q}) - \frac{\D}{\D t}\bigl(L_{i} G^{i}_{a}(q,\dot{q})\bigr) = 0.}
\end{equation}
These $n$ are identities are known as \define{Noether's Second Theorem}\index{Noether!Second theorem}
(or the \emph{generalized Bianchi identities}\index{Bianchi identity!Generalized}).

\N{Example, Noether's First Theorem}\index{Noether!First theorem}
Consider the special case when
\begin{equation}
\epsilon^{a}(t) = \epsilon^{a} = \mbox{constant}.
\end{equation}
Then
\begin{equation}
\delta q^{i} = \epsilon^{a}F^{i}_{a}.
\end{equation}
We then have the variation of the action, under this variation,
\begin{equation}
\delta\action = \int^{t_{2}}_{t_{1}}\epsilon^{a}F^{i}_{a}L_{i}\,\D t + \left.\vphantom{\frac{1}{1}}\epsilon^{a}F^{i}_{a}p_{i}\right|^{t_{2}}_{t_{1}},
\end{equation}
where we introduce the canonical momentum $p_{i}$. Define
\begin{equation}
C_{a} := F^{i}_{a}p_{i},
\end{equation}
then invariance of the action \emph{including the boundary terms}
leads to the condition
\begin{equation}
\left.\vphantom{\frac{1}{1}}F^{i}_{a}p_{i}-C_{a}\right|_{t_{2}}
=
\left.\vphantom{\frac{1}{1}}F^{i}_{a}p_{i}-C_{a}\right|_{t_{1}}.
\end{equation}
An invariance under a group with a finite number of parameters thus
leads to \emph{conservation laws}. This is Noether's first theorem.

\begin{remark}
Noether ends the first section of her paper by stating (as translated by
Traver), ``With these supplementary remarks, Theorem I comprises all
theorems on first integrals known to mechanics etc., while Theorem II
may be described as the utmost possible generalization of the `general
theory of relativity' in group theory.''
\end{remark}

\begin{remark}
More generally, when the general invariance under a group is
parametrized by functions of time, this will lead to \emph{constraints}
instead of conservation laws. This is what leads to the study of
constrained Hamiltonian systems and underpins the canonical formalism of
gauge theory.
\end{remark}

\subsection{For Fields}

\M
The results still hold, we just assume invariance under
\begin{subequations}
\begin{equation}
\varphi^{A}(x)\to\varphi^{A}+\delta\varphi^{A}(x)
\end{equation}
with
\begin{equation}
\delta\varphi^{A}(x) = \epsilon^{A}(x)F^{A}_{a} + (\partial_{\mu}\epsilon^{a}(x))G^{A\mu}_{a},
\end{equation}
\end{subequations}
and we now have arbitrary functions $\epsilon^{A}(x)$ of spacetime (not
just time).

\M
The more common presentation of Noether's first theorem begins by
supposing we have some infinitesimal symmetry transformation of the field
\begin{equation}
\varphi(x)\to\varphi'(x)=\varphi(x) + \alpha\,\Delta\varphi(x),
\end{equation}
where $\alpha$ is an infinitesimal quantity and $\Delta\varphi(x)$ is
the change in the field. We will hope the Lagrangian density transforms
like
\begin{equation}
\mathcal{L}(x)\to\mathcal{L}(x) + \alpha\partial_{\mu}\mathcal{J}^{\mu}(x)
\end{equation}
where $\mathcal{J}^{\mu}$ is ``something''. This is the same as adding
some boundary contribution to the action, which will not affect the
equations of motion.

Now, if we plug in the infinitesimally transformed field into the
Lagrangian
\begin{equation}
\mathcal{L}(\varphi + \alpha\,\Delta\varphi) = \mathcal{L} + \alpha\,\Delta\mathcal{L},
\end{equation}
where $\alpha\,\Delta\mathcal{L}$ may be found by Taylor expanding to
first-order:
\begin{calculation}
\alpha\,\Delta\mathcal{L}
\step{Taylor expansion to first order}
\frac{\partial\mathcal{L}}{\partial\varphi}(\alpha\,\Delta\varphi)
+\frac{\partial\mathcal{L}}{\partial(\partial_{\mu}\varphi)}\partial_{\mu}(\alpha\,\Delta\varphi)
\step{since $A\partial_{\mu}B=\partial_{\mu}(AB)-B\partial_{\mu}A$}
\frac{\partial\mathcal{L}}{\partial\varphi}(\alpha\,\Delta\varphi)
+\partial_{\mu}\left(\frac{\partial\mathcal{L}}{\partial(\partial_{\mu}\varphi)}\alpha\,\Delta\varphi\right)
-\left(\partial_{\mu}\frac{\partial\mathcal{L}}{\partial(\partial_{\mu}\varphi)}\right)(\alpha\,\Delta\varphi)
\step{collecting terms, factoring out $\alpha$}
\alpha\partial_{\mu}\left(\frac{\partial\mathcal{L}}{\partial(\partial_{\mu}\varphi)}\Delta\varphi\right)
+\alpha\left[\frac{\partial\mathcal{L}}{\partial\varphi}
-\partial_{\mu}\left(\frac{\partial\mathcal{L}}{\partial(\partial_{\mu}\varphi)}\right)\right]\Delta\varphi
\step{when the Euler--Lagrange equations hold}
\alpha\partial_{\mu}\left(\frac{\partial\mathcal{L}}{\partial(\partial_{\mu}\varphi)}\Delta\varphi\right)
+\alpha\cdot0\cdot\Delta\varphi
\step{multiplying the second term by zero makes it vanish}
\alpha\partial_{\mu}\left(\frac{\partial\mathcal{L}}{\partial(\partial_{\mu}\varphi)}\Delta\varphi\right).
\end{calculation}
This is precisely the sort of thing we're looking for: we want
$\alpha\,\Delta\mathcal{L}=\alpha\partial_{\mu}\mathcal{J}^{\mu}$. We
then define the \define{Noether Current}\index{Noether!Current},
\begin{equation}
j^{\mu}(x) := \frac{\partial\mathcal{L}}{\partial(\partial_{\mu}\varphi)}\Delta\varphi-\mathcal{J}^{\mu}(x),
\end{equation}
and our demands for symmetry invariance amounts to the conservation of
the Noether current:
\begin{equation}
\partial_{\mu}j^{\mu}(x) = 0.
\end{equation}
We see that this means
\begin{equation}
\partial_{\mu}j^{\mu}(x) = 0\iff\partial_{\mu}\left(\frac{\partial\mathcal{L}}{\partial(\partial_{\mu}\varphi)}\Delta\varphi\right)=\partial_{\mu}\mathcal{J}^{\mu},
\end{equation}
and therefore the two terms are interchangeable, allowing us to recover
our desired symmetry.

\begin{remark}[Multiple fields]
If the symmetry involves more than one field, then $j^{\mu}(x)$ is
really the sum of these sort of terms (one term for each field).
\end{remark}

\N{Noether Charge}
We can express the conservation law by sating that the quantity
\begin{equation}
Q := \int_{\RR^{3}}j^{0}\,\D^{3}x
\end{equation}
is a constant in time. This $Q$ is sometimes referred to as the
``Noether Charge''\index{Noether!Charge} in the literature.

\begin{example}[Canonical Stress--Energy]\label{ex:classical-field-theory:noether:canonical-stress-energy}
Consider the situation when the symmetry is an infinitesimal translation
in spacetime,
\begin{equation}
x^{\mu}\to x^{\mu} + a^{\mu},
\end{equation}
and the field transforms as\footnote{Remember, the scalar field
transforms under a symmetry $g$ like
$g\cdot\varphi(x)=\varphi(g^{-1}\cdot x)$.}:
\begin{equation}
\varphi(x)\to\varphi(x-a)=\varphi(x)-a^{\mu}\partial_{\mu}\varphi(x).
\end{equation}
The Lagrangian must transform like a scalar,
\begin{equation}
\mathcal{L}\to\mathcal{L}-a^{\mu}\partial_{\mu}\mathcal{L}
=\mathcal{L} - \partial_{\mu}(a^{\mu}\mathcal{L}).
\end{equation}
Then $\Delta\mathcal{L}(x)=- \partial_{\mu}(a^{\mu}\mathcal{L})$
allowing us to identify $\mathcal{J}^{\mu}(x)=-a^{\mu}\mathcal{L}(x)$.
The conserved Noether current is then
\begin{equation}
\begin{split}
j^{\mu}(x) &= \frac{\partial\mathcal{L}(x)}{\partial(\partial_{\mu}\varphi(x))}(-a^{\mu}\partial_{\mu}\varphi(x))
- (-a^{\mu}\mathcal{L}(x))\\
&=a_{\nu}\canonicalStressEnergy^{\mu\nu}.
\end{split}
\end{equation}
This $\canonicalStressEnergy^{\mu\nu}$ is the
\define{Canonical Stress-Energy Tensor}, not to be confused 
with the stress-energy tensor appearing in the Einstein field equations.
More generally, if we have a collection of fields $\varphi_{a}$ ($a=1,\dots,n$)
we have (implicitly summing over $a$),
\begin{subequations}
\begin{align}
\canonicalStressEnergy^{\mu\nu} &:= -\frac{\partial\mathcal{L}}{\partial(\partial_{\mu}\varphi_{a})}\partial^{\nu}\varphi_{a}+\eta^{\mu\nu}\mathcal{L},\\
\intertext{or, written as a rank-2 covariant tensor,}
\canonicalStressEnergy_{\mu\nu} &:= -\frac{\partial\mathcal{L}}{\partial(\partial^{\mu}\varphi_{a})}\partial_{\nu}\varphi_{a}+\eta_{\mu\nu}\mathcal{L}.
\end{align}
\end{subequations}
\end{example}

\begin{exercise}
Prove $\partial_{\mu}{\canonicalStressEnergy^{\mu}}_{\nu}=0$ for each $\nu$.
\end{exercise}

\begin{exercise}
Prove or find a counter-example: $\canonicalStressEnergy_{\mu\nu}=\eta_{\mu\rho}{\canonicalStressEnergy^{\rho}}_{\nu}$
is not symmetric. When will it be symmetric?
\end{exercise}

\begin{exercise}
What are the 4 Noether charges for the canonical stress-energy tensor?
How do we interpret them physically?
\end{exercise}

\begin{exercise}
For the Scalar Field's Lagrangian from Eq~\eqref{eq:classical-field-theory:scalar-field:lagrangian},
compute the canonical stress-energy tensor. What happens for arbitrary
potential terms $V(\varphi)$?
\end{exercise}

\begin{exercise}
The stress-energy tensor appearing in Einstein's field equation is
obtained by (\S21.3 in Misner, Thorne, Wheeler~\cite{Misner:1973prb}):
\begin{equation}
\mbox{(RHS)}_{\mu\nu} = -2\frac{\partial\mathcal{L}_{\text{matter}}}{\partial g^{\mu\nu}}
+ g_{\mu\nu}\mathcal{L}_{\text{matter}}.
\end{equation}
In flat spacetime (i.e., when $g_{\mu\nu}=\eta_{\mu\nu}$), when does
this differ from the canonical stress-energy tensor? When do they coincide?
\end{exercise}

\N{References}
Noether's original paper~\cite{Noether1918:iv} may be worth reading,
though the terminology of group theory may be archaic compared to
today's vocabulary.
The discussion of Noether's theorems is largely inspired from Chapter 3
section 5 of Kiefer~\cite{Kiefer:2007ria},
though we also rely on Peskin and Schroeder~\cite{Peskin:1995ev} for the
discussion of Noether's first theorem and the canonical stress-energy
tensor. See also Chapter 22 of Srednicki~\cite{Srednicki:2007qs}, but
care must be taken: there is a sign error in his derivation of the
canonical stress-energy tensor (which is corrected in our derivation).
Sundermeyer~\cite{Sundermeyer:1982gv} is quite explicit in the
adjustments necessary for Noether's theorem to work with fields.