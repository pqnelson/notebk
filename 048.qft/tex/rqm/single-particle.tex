\section{One Particle Systems}

\M
The first problem we will try to answer is: What is the relativistic
version of the Schrodinger equation?

\N{Four-Momentum Operator}
The momentum four-vector (\S\ref{chunk:relativity:four-momentum}) is
turned into an operator $\widehat{p}_{\mu}$. Recall
(\S\ref{chunk:relativity:partial-derivatives}) 
\begin{subequations}
\begin{align}
\partial_{\mu} &= \frac{\partial}{\partial x^{\mu}} = \left(\frac{+1}{c}\frac{\partial}{\partial
t},\vec{\nabla}\right)\\
\intertext{and}
\partial^{\mu} &= \frac{\partial}{\partial x_{\mu}} = \left(\frac{-1}{c}\frac{\partial}{\partial
t},\vec{\nabla}\right).
\end{align}
\end{subequations}
We have $\widehat{H}=\I\hbar\partial_{t}$ and $\widehat{\vec{p}}=-\I\hbar\vec{\nabla}$.
Then
\begin{equation}
\widehat{p}_{\mu}=-\I\hbar\partial_{\mu} = (-c^{-1}\widehat{H},\widehat{\vec{p}}),
\end{equation}
and
\begin{equation}
\widehat{p}^{\mu} = (c^{-1}\widehat{H},\widehat{\vec{p}}).
\end{equation}

\N{Equations of Motion}
There are a variety of ways to derive the equations of motion. We can
observe
\begin{equation}
c^{2}E^{2} = \vec{p}\cdot\vec{p} + m_{0}^{2}c^{2},
\end{equation}
then
\begin{equation}
cE = \sqrt{\vec{p}\cdot\vec{p} + m_{0}^{2}c^{2}}.
\end{equation}
If we tried ``putting on hats'' now (turning $E\mapsto\widehat{H}$ and
$\vec{p}\mapsto\widehat{\vec{p}}$), then we end up with a nonlocal equation
and we'd be forced to use pseudodifferential operators to handle the
squareroot.

Instead, we Taylor expand in powers of $\vec{p}\cdot\vec{p}$ to get
\begin{equation}
cE = m_{0}c^{2} + \frac{1}{2m}\vec{p}\cdot\vec{p}+\cdots.
\end{equation}
If we quantize this equation, then we end up with the Schrodinger
equation. Therefore, the argument goes, the Schrodinger equation is the
low velocity limit to:
\begin{subequations}
\begin{equation}
(\widehat{p}^{\mu}\widehat{p}_{\mu}+m_{0}^{2}c^{2})\mid\psi\rangle=0.
\end{equation}
In position-space, this is the Klein--Gordon equation:\index{Klein--Gordon!Equation}\marginnote{Klein--Gordon equation}
\begin{equation}
\left(\hbar^{2}\frac{\partial^{2}}{\partial t^{2}}
-\hbar^{2}c^{2}\nabla^{2}+m_{0}^{2}c^{4}\right)\psi(\vec{x},t)=0.
\end{equation}
\end{subequations}
Thus we conclude this must be the equations of motion.

\begin{remark}
There is another argument using the Lagrangian for a relativistic
point-particle (\S\ref{chunk:relativity:lagrangian-for-point-particle}),
which is a constrained system. We then Dirac quantize it in Section~\ref{section:rqm:constrained-quantization-of-free-particle}.
\end{remark}

\N{Energy Eigenvalues}
We can write the Klein--Gordon equation applied to momentum eigenstates
(recalling $\vec{p}=\hbar\vec{k}$) as:
\begin{equation}
(c^{-2}\widehat{H}^{2}-\widehat{p}_{i}\widehat{p}^{i})\mid\vec{k}\rangle=m_{0}^{2}c^{2}\mid\vec{k}\rangle
\end{equation}
Since these are momentum eigenstates, we have:
\begin{equation}
\vec{p}_{i}\mid\vec{k}\rangle=\hbar k_{i}\mid\vec{k}\rangle,
\end{equation}
and then
\begin{equation}
  (c^{-2}\widehat{H}^{2}-\widehat{p}_{i}\widehat{p}^{i})\mid\vec{k}\rangle=
  (c^{-2}\widehat{H}^{2}-\hbar^{2} k_{i}k^{i})\mid\vec{k}\rangle.
\end{equation}
Simple algebra rearranges this to
\begin{equation}
\widehat{H}^{2}\mid\vec{k}\rangle =(c^{2}\hbar^{2} k_{i}k^{i}+m_{0}^{2}c^{4})\mid\vec{k}\rangle.
\end{equation}
Thus we conclude the eigenvalues for the Hamiltonian are:
\begin{subequations}
\begin{equation}
\boxed{\widehat{H}\mid\vec{k}\rangle =(c^{2}\hbar^{2} k_{i}k^{i}+m_{0}^{2}c^{4})^{1/2}\mid\vec{k}\rangle.}
\end{equation}
We will use the eigenvalues for the Hamiltonian operator frequently
enough that it merits its own abbreviation as:\marginnote{Hamiltonian Eigenvalues $\omega(\vec{k})$}
\begin{equation}
\boxed{\omega(\vec{k}) := \bigl(c^{2} k_{i}k^{i}+\hbar^{-2}m_{0}^{2}c^{4}\bigr)^{1/2}.}
\end{equation}
\end{subequations}
Observe then that $\hbar\omega(\vec{k})$ has units of energy, and
$\omega(\vec{k})$ has units of frequency.

\subsection{Quantum Lorentz Transformations}

\M
We have Lorentz transformations describe symmetries of our quantum
system, and therefore by Wigner's theorem must be described by unitary
operators. This is a unitary representation of the Poincar\'e group
$U(\Lambda,a)$ where $a\in\RR^{3,1}$ is a space-time translation and
$\Lambda\in\O(3,1)$ is a Lorentz transformation. This representation
satisfies
\begin{equation}
U(\Lambda_{2},a_{2})U(\Lambda_{1},a_{1}) =
U(\Lambda_{2}\Lambda_{1},\Lambda_{2}a_{1}+a_{2}).
\end{equation}

\N{Lorentz Transformations}
If we have $|k\rangle$ be a Lorentz-invariant eigenfunction for the 4-momentum
operator, then $U(\Lambda,0)$ acts on it by
\begin{equation}
U(\Lambda,0)\mid k\rangle = |\Lambda k\rangle.
\end{equation}
We will prove this in the next subsection.

\N{Space-time translation operator}.
Unitary time-evolution operator is given by
\begin{equation}
  |\psi(t)\rangle = \E^{-\I t\widehat{H}/\hbar}\mid\psi(0)\rangle.
\end{equation}
We also have unitary spatial-translation operators sending
$\vec{x}\mapsto\vec{x}+\vec{a}$ given by
$\exp(\I\vec{a}\cdot\widehat{\vec{p}}/\hbar)$. Then translation by a
4-vector $a^{\mu}$ is given by the unitary operator
\begin{equation}
\boxed{U(0,a^{\mu}) = \exp(\I a_{\mu}\widehat{p}^{\mu}/\hbar).}
\end{equation}

\begin{exercise}
Verify $a_{\mu}\widehat{p}^{\mu}=-a^{0}\widehat{H}+\vec{a}\cdot\widehat{\vec{p}}$.
\end{exercise}

\subsection{Lorentz Invariant Measure in Phase Space}

\M
The energy of a particle with 4-momentum $p^{\mu}=(E/c,\vec{p})$ has
non-negative energy
\begin{equation}
p^{0}\geq0,
\end{equation}
as well as obeying the mass--shell condition (\S\ref{chunk:relativity:mass-shell-relation})
\begin{equation}
p^{\mu}p_{\mu}+m_{0}^{2}c^{2}=0.
\end{equation}
Therefore the possible 4-momentum vectors lie on a hyperbolic sheet in
momentum-space, the so-called \define{Mass Hyperboloid}. We need an
integration measure on this hyperboloid if we want to do Lorentz
invariant computations.

Let $\heavisideStep(t)$ be the Heaviside step function
\begin{equation}
\heavisideStep(t):=\begin{cases}0 & \mbox{if }t<0\\
1 & \mbox{if }t\geq0.
\end{cases}
\end{equation}
We define an integration measure $\D\lambda(k)$ on the positive
hyperboloid as follows:
\begin{equation}\label{eq:rqm:single-particle:lorentz-invariant-phase-space-measure}
\D\lambda(k) := \delta(k^{2}+\hbar^{-2}m_{0}^{2}c^{2})\heavisideStep(k^{0})\,\D^{4}k.
\end{equation}
The Lebesgue measure $\D^{4}k$ is Lorentz invariant due to Lorentz
transformations having unit determinant. Here
$k^{2}+\hbar^{-2}m_{0}^{2}c^{2}$ is Lorentz 
invariant, the $\delta$ function is Lorentz invariant, and
$\heavisideStep(k^{0})$ is Lorentz invariant. Therefore $\D\lambda(k)$
is Lorentz invariant.

\begin{exercise}
Prove $\heavisideStep(k^{0})$ is Lorentz invariant.
\end{exercise}

\M
We observe
\begin{subequations}
\begin{align}
\hbar^{2}k_{\mu}k^{\mu} + m_{0}^{2}c^{2}
&=(-\hbar^{2}(k^{0})^{2}+\hbar^{2}\|\vec{k}\|^{2}) + m_{0}^{2}c^{2}\\
&=-\hbar^{2}(k^{0})^{2} + (\hbar^{2}\|\vec{k}\|^{2} + m_{0}^{2}c^{2})\\
&=-\hbar^{2}(k^{0})^{2} + \hbar^{2}\omega(\vec{k})^{2}\\
&= -(\hbar^{2}(k^{0})^{2} - \hbar^{2}\omega(\vec{k})^{2})\\
%&=-\hbar^{2}(k^{0})^{2} + c^{2}\omega(k)^{2} = -(\hbar^{2}(k^{0})^{2} - c^{2}\omega(k)^{2})\\
&= - (\hbar k^{0} + \hbar\omega(\vec{k}))(\hbar k^{0} - \hbar\omega(\vec{k})).
\end{align}
\end{subequations}
Therefore, dividing through by $\hbar^{2}$,
\begin{equation}
k^{\mu}k_{\mu} + \hbar^{-2}m_{0}^{2}c^{2} = -(k^{0} + \omega(\vec{k}))(k^{0} - \omega(\vec{k}))
\end{equation}
We have the Dirac delta function identity from Corollary~\ref{cor:math:dirac-delta-function-of-quadratic-polynomial}
to write
\begin{calculation}
\delta(k_{\mu}k^{\mu} + \hbar^{-2}m_{0}^{2}c^{2})\heavisideStep(k^{0})
\step{from previous calculations}
\delta\bigl(- (k^{0} + \omega(\vec{k}))(k^{0} - \omega(\vec{k}))\bigr)\heavisideStep(k^{0})
\step{since $\delta$-function is even}
\delta\bigl((k^{0} + \omega(\vec{k}))(k^{0} - \omega(\vec{k}))\bigr)\heavisideStep(k^{0})
\step{using Corollary~\ref{cor:math:dirac-delta-function-of-quadratic-polynomial}}
\frac{1}{2\omega(k)}\left[\delta(k^{0}-\omega(\vec{k})) + \delta(k^{0}+\omega(\vec{k}))\right]\heavisideStep(k^{0})
\step{since $\delta(k^{0}+\omega(\vec{k}))\heavisideStep(k^{0})=\heavisideStep(-\omega(\vec{k}))=0$}
\frac{1}{2\omega(\vec{k})}\delta(k^{0}-\omega(\vec{k}))\heavisideStep(k^{0}).
\end{calculation}
Thus
\begin{equation}\label{eq:rqm:single-particle:delta-identity-applied-to-dlips}
\boxed{\delta(k_{\mu}k^{\mu} + \hbar^{-2}m_{0}^{2}c^{2})\heavisideStep(k^{0})
=\frac{1}{2\omega(\vec{k})}\delta(k^{0}-\omega(\vec{k}))\heavisideStep(k^{0}).}
\end{equation}

\M
We can integrate a test function $f(k)$ against $\D\lambda(k)$,
\begin{calculation}
\int f(k)\,\D\lambda(k)
\step{using the definition of $\D\lambda(k)$ from Eq~\eqref{eq:rqm:single-particle:lorentz-invariant-phase-space-measure}}
\int f(k)\delta(k^{2}+\hbar^{-2}m_{0}^{2}c^{2})\heavisideStep(k^{0})\,\D^{4}k
\step{using Eq~\eqref{eq:rqm:single-particle:delta-identity-applied-to-dlips}}
\int f(k^{0},\vec{k})\left(\frac{1}{2\omega(\vec{k})}\delta(k^{0}-\omega(\vec{k}))\heavisideStep(k^{0})\,\D^{3}\vec{k}\,\D k^{0}\right)
\step{integrating over $k^{0}$ with a delta function}
\int f(\omega(\vec{k}),\vec{k})\frac{1}{2\omega(k)}\,\D^{3}\vec{k}.
\end{calculation}
Therefore we conclude, once the dust has settled,
\begin{subequations}
\begin{equation}
\boxed{\D\lambda(\vec{k}) = \frac{\D^{3}\vec{k}}{2\omega(\vec{k})}}
\end{equation}
and
\begin{equation}
\boxed{k^{0} = \omega(\vec{k}).}
\end{equation}
\end{subequations}

\N{Normalized Momentum Eigenfunctions}
We define Lorentz-normalized kets $|k\rangle$ by
\begin{equation}
|k\rangle = \bigl(2\omega(\vec{k})\bigr)^{1/2}(2\pi)^{3/2}\mid\vec{k}\rangle,
\end{equation}
with $k^{0}=\omega(\vec{k})$. Then the new normalization condition for
momentum eigenvectors is:
\begin{equation}
\langle k\mid k'\rangle=2\omega(\vec{k})(2\pi)^{3}\delta^{(3)}(\vec{k}-\vec{k}').
\end{equation}
The resolution of the identity based on the Lorentz invariant measure is
\begin{equation}
\id = \int|k\rangle\langle k|\frac{\D^{3}\vec{k}}{(2\pi)^{3}2\omega(\vec{k})}.
\end{equation}

\begin{theorem}
If we define the unitary representation of Lorentz transformations
$U(\Lambda,0)$ acting on the 4-momentum eigenket $|k\rangle$ as
\begin{equation}
U(\Lambda,0)\mid k\rangle=|\Lambda k\rangle,
\end{equation}
then $U(\Lambda,0)$ really is a unitary representation of the Lorentz group.
\end{theorem}

\begin{proof}
The multiplication property $U(\Lambda,0)U(\Lambda',0)=U(\Lambda\Lambda',0)$
follows immediately
\begin{calculation}
U(\Lambda,0)U(\Lambda',0)\mid k\rangle
\step{applying the inner most action}
U(\Lambda,0)\mid \Lambda'k\rangle
\step{applying the action}
\mid \Lambda\Lambda'k\rangle
\step{associativity}
\mid (\Lambda\Lambda')k\rangle
\step{from the group action}
U(\Lambda\Lambda',0)\mid k\rangle.
\end{calculation}
This suffices to prove we have a representation, we just need to prove
it is unitary.

We can prove unitary by direct calculation,
\begin{calculation}
U(\Lambda,0)U(\Lambda,0)^{\dagger}
\step{using the resolution of the identity}
\int U(\Lambda,0)\mid k\rangle\langle k\mid U(\Lambda,0)^{\dagger}\frac{\D^{3}\vec{k}}{(2\pi)^{3}2\omega(\vec{k})}
\step{by the group action}
\int |\Lambda k\rangle\langle\Lambda k|\frac{\D^{3}\vec{k}}{(2\pi)^{3}2\omega(\vec{k})}
\step{by Lorentz invariance of the measure}
\int |\Lambda k\rangle\langle\Lambda k|\frac{\D^{3}(\Lambda \vec{k})}{(2\pi)^{3}2\omega(\Lambda \vec{k})}
\step{resolution of the identity}
\id.
\end{calculation}
Hence $U(-,0)$ gives us a unitary representation of the Lorentz group. 
\end{proof}

\begin{exercise}
Now prove that $U(\Lambda,a)$ really is a unitary representation for the
Poincar\'e group. There are two claims you need to prove: (1) it's a
representation, and (2) it's outputs are all unitary operators.
\end{exercise}

\subsection{Problems with Position, Faster-than-light}

\N{Problem 1}
From first principles, we should expect the position operator to not be
well-defined. Why? Well consider the Compton wavelength $\lambda$ for
the particle of rest mass $m_{0}$. If $\Delta x\ll\lambda$, then
$\Delta p\gg m_{0}c$. This means the momentum would be so large it could
create more particles of mass $m_{0}$. Since we do not witness this, we
are forced to believe that $\Delta x$ cannot be less than the Compton wavelength.

\N{Problem 2: Faster-than-light propagation}
Suppose we have a relativistic position operator
$\widehat{\vec{x}}$. Let $|\vec{x}\rangle$ be eigenvector for 
$\widehat{\vec{x}}$. Then we pick the normalization of these kets such
that
\begin{equation}\label{eq:rqm:single-particle:problems:dot-product}
\langle\vec{x}\mid\vec{p}\rangle = \exp(\I\vec{x}\cdot\vec{p}/\hbar).
\end{equation}
Consider a state $|\psi_{0}\rangle$ initially localized at the origin,
we will want to know how it evolves over time. So
\begin{subequations}
\begin{equation}
\psi(\vec{x},t=0) := \psi_{0}(\vec{x}) = (2\pi)^{3}\delta^{(3)}(\vec{x}),
\end{equation}
then its Fourier transform is:
\begin{equation}
\widetilde{\psi}_{0}(\vec{k}) = 1,
\end{equation}
and
\begin{equation}
|\psi_{0}\rangle = \int|\vec{k}\rangle\,\D^{3}\vec{k}.
\end{equation}
\end{subequations}
The evolution of this state is given by
\begin{calculation}
\psi(\vec{x},t)
\step{using position-space coordinate expansion}
\langle\vec{x}\mid\E^{-\I\widehat{H}t/\hbar}\mid\psi_{0}\rangle
\step{inverse Fourier transform it}
\int\langle\vec{x}\mid\E^{-\I\widehat{H}t/\hbar}\mid\vec{k}\rangle\,\D^{3}\vec{k}
\step{using Eigenvalues of Hamiltonian operator}
\int\langle\vec{x}\mid\E^{-\I\omega(\vec{k})t}\mid\vec{k}\rangle\,\D^{3}\vec{k}
\step{linearity}
\int \E^{-\I\omega(\vec{k})t}\langle\vec{x}\mid\vec{k}\rangle\,\D^{3}\vec{k}
\step{using $\vec{p}=\hbar\vec{k}$ and Eq~\eqref{eq:rqm:single-particle:problems:dot-product}}
\int \E^{-\I\omega(\vec{k})t}\E^{\I\vec{k}\cdot\vec{x}}\,\D^{3}\vec{k}.
\end{calculation}
We can estimate $\psi(\vec{x},t)$ outside the light-cone, when
$|\vec{x}|>ct$. Working in spherical coordinates, we have
\begin{calculation}
\iiint \E^{-\I\omega(\vec{k})t}\E^{\I\vec{k}\cdot\vec{x}}\,\D^{3}\vec{k}
\step{expanding in spherical coordinates}
\int^{\infty}_{0}\int^{2\pi}_{0}\int^{1}_{-1}k^{2}\E^{-\I t\sqrt{k^{2} + \hbar^{-2}m_{0}^{2}c^{2}}}\E^{\I
kr\cos(\theta)}\,\D(\cos(\theta))\,\D\phi\D k
\step{doing the $\theta$ integral}
\int^{\infty}_{0}\int^{2\pi}_{0}\left.\frac{k^{2}\E^{-\I t\sqrt{k^{2} + \hbar^{-2}m_{0}^{2}c^{2}}}\E^{\I
kr\cos(\theta)}}{\I k r}\right|^{\cos\theta=1}_{\cos\theta=-1}\,\D\phi\D k
\step{evaluating the endpoints of the integral}
\int^{\infty}_{0}\int^{2\pi}_{0}k^{2}\E^{-\I t\sqrt{k^{2} + \hbar^{-2}m_{0}^{2}c^{2}}}\frac{\E^{\I
kr}-\E^{-\I kr}}{\I k r}\,\D\phi\D k
\step{simplify the integrand}
\frac{1}{\I r}\int^{\infty}_{0}\int^{2\pi}_{0}k\E^{-\I t\sqrt{k^{2} + \hbar^{-2}m_{0}^{2}c^{2}}}(\E^{\I
kr}-\E^{-\I kr})\,\D\phi\D k
\step{since the integrand is independent of $\phi$}
\frac{2\pi}{\I r}\int^{\infty}_{0}k\E^{-\I t\sqrt{k^{2} + \hbar^{-2}m_{0}^{2}c^{2}}}(\E^{\I
kr}-\E^{-\I kr})\,\D k
\step{simple calculus}
\frac{2\pi}{\I r}\int^{\infty}_{-\infty}k\E^{-\I t\sqrt{k^{2} + \hbar^{-2}m_{0}^{2}c^{2}}}\E^{\I
kr}\,\D k
\end{calculation}
If we set $m_{0}=0$, then
\begin{equation}
\int k\E^{\I k(r - ct)}\D k = \E^{\I k(r - ct)}\left(\frac{1}{(r - ct)^2}
- \frac{\I k}{r - ct}\right)\sim\I\E^{\I k(r-ct)}k
\end{equation}
which does not converge as $k\to\pm\infty$. Including the mass term does
not improve the situation.

To be clear, this means the probability amplitude for a relativistic
particle to appear outside the future lightcone of its initial position
is nonzero, which implies faster-than-light phenomena. This is exactly
what we want to avoid.

\begin{remark}
Penrose argued that, really, the initial condition $|\psi_{0}\rangle$
being localized at a point is unphysical. The quantum picture to have in
mind is the light cone should be ``quantized'', including the vertex ---
so its vertex is ``fuzzy'' rather than localized at a point. This is
what led to Twistor theory.
\end{remark}

\N{References}
See \S2.5 of Weinberg~\cite{Weinberg:1995mt},
Chapter 1 in Ticciati~\cite{Ticciati:1999qp}. For another computation of
relativistic position being outside the light cone, see Chapter 1 of
Peskin and Schroeder~\cite{Peskin:1995ev}.
