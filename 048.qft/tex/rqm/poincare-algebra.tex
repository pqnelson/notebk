\section{Poincar\'e Algebra}

\N{Big Idea}
We study Lie groups by studying their Lie algebras (``infinitesimal
symmetries near the identity''), and we want to find finite-dimensional
unitary representations for the Poincar\'e group. It suffices for our
purposes to study the Lie algebra for the Lorentz group (which can be
extended to the Lie algebra for the Poincar\'e group by adjoining
generators of translations --- i.e., the momentum and Hamiltonian
operators). We want to find irreducible representations, since these
correspond to ``elementary'' quantum systems.

Wigner forwarded the interpretation that irreducible representations for
the ``Poincar\'e group'' should be viewed as elementary
particles.\index{Elementary Particle!As Irreducible Representation}

\subsection{Lorentz Algebra}

\begin{definition}
The subgroup of \define{Proper Orthochronous Lorentz Transformations}
denoted either as $\ISO(3,1)$ or $\SO^{+}(3,1)$, consists of Lorentz
transformations $\Lambda\in\O(3,1)$ such that
\begin{enumerate}
\item it is proper (``preserves orientation''): $\det(\Lambda)=+1$, and
\item it is orthochronous: ${\Lambda^{0}}_{0}\geq+1$.
\end{enumerate}
\end{definition}
\begin{exercise}
Prove the identity transformation is a proper orthochronous Lorentz
transformation. 
\end{exercise}

\N{Non-examples}
There are two critical example:
\begin{enumerate}
\item The transformation of time reversal $T=\diag(-1,1,1,1)$ is not an
  orthochronous Lorentz transformation (but it is a Lorentz transformation).
\item Parity reversal $P=\diag(1,-1,-1,-1)$ is a Lorentz transformation
  but it is not proper.
\end{enumerate}
Composing these two operators does not give us a proper orthochronous
Lorentz transformation. But in four dimensions, the Lorentz group
$\O(3,1)$ has four connected components, and applying $P$, $T$, $PT$ to
the identity component gives us the other three. Therefore, we can work
with an arbitrary proper orthochronous Lorentz transformation
$\Lambda\in\ISO(3,1)$ then $P\Lambda$, $T\Lambda$, $PT\Lambda$ gives us
arbitrary Lorentz transformations \emph{in four dimensions}.

\N{Obtaining the Lorentz Lie Algebra}
It's a common trick to parametrize a proper orthochronous Lorentz
transformation $\Lambda\in\ISO(3,1)$ using 
``angles'' [scalars], just as we would with rotations, as
\begin{equation}
{\Lambda^{\mu}}_{\nu} = [\exp\left(\frac{-\I}{2}\omega_{\kappa\lambda}M^{\kappa\lambda}\right)]{{}^{\mu}}_{\nu}
\end{equation}
where $\omega_{\kappa\lambda}=-\omega_{\lambda\kappa}$ are ``rotation
angles'' (real constants parametrizing the symmetry) and
$M^{\kappa\lambda}$ is an indexed family of matrices (i.e., fix a value
of $\kappa$ and $\lambda$, and you get a $4\times4$ matrix). These $M^{\kappa\lambda}$ are
generators of the Lie algebra for the Lorentz group. Explicitly
\begin{equation}
(M^{\kappa\lambda})_{\mu\nu} = \I(\delta^{\kappa}_{\mu}\delta^{\lambda}_{\nu}-\delta^{\kappa}_{\nu}\delta^{\lambda}_{\mu})
\end{equation}
Now the trick is that we can write the generators of the Lorentz Lie
algebra using
\begin{subequations}\label{eq:rqm:poincare-algebra:rotations-and-boosts-generators}
\begin{align}
L^{i} &= \frac{1}{2}\epsilon^{ijk}M_{jk}\\
\intertext{for spatial rotations, and}
K^{i} &= M^{0i}%\\
%% \intertext{for Lorentz boosts. We define}
%% \vec{J}_{\pm} &= \frac{1}{2}(\vec{L}\pm\I\vec{K}).
\end{align}
\end{subequations}

\N{Commutation Relations}\label{chunk:rqm:poincare-algebra:lorentz-algebra:commutation-relations}
We can obtain the commutation relations by considering an arbitrary
Lorentz transformation $\Lambda\in\ISO(3,1)$ and an infinitesimal
Lorentz transformation $1+\delta\omega$, then evaluating to first order
in $\delta\omega$ gives us:
\begin{subequations}
\begin{align}
  U(\Lambda,0)^{-1}U(1 + \delta\omega,0)U(\Lambda,0)
  &= 1+U(\Lambda,0)^{-1}M^{\mu\nu}U(\Lambda,0)\delta\omega_{\mu\nu}\\
  &= 1+{\Lambda^{\mu}}_{\rho}{\Lambda^{\nu}}_{\sigma}M^{\rho\sigma}\delta\omega_{\mu\nu}.
\end{align}
\end{subequations}
Since $\delta\omega_{\mu\nu}$ is arbitrary (but antisymmetric), we have
\begin{equation}
U(\Lambda,0)^{-1}M^{\mu\nu}U(\Lambda,0)
= {\Lambda^{\mu}}_{\rho}{\Lambda^{\nu}}_{\sigma}M^{\rho\sigma}.
\end{equation}
Now if we let $\Lambda$ be an infinitesimal Lorentz transformation, we
arrive at
\begin{equation}
[M^{\mu\nu}, M^{\rho\sigma}]
= (\eta^{\mu\sigma}M^{\nu\rho} - \eta^{\nu\sigma}M^{\mu\rho})
- (\eta^{\mu\rho}M^{\nu\sigma} - \eta^{\nu\rho}M^{\mu\sigma}).
\end{equation}
This gives us the commutation relations in terms of rotations $L^{i}$
and Lorentz boosts $K^{i}$ as
\begin{subequations}
\begin{align}
[L^{i}, L^{j}] &= \I{\epsilon^{ij}}_{k}J^{k}\\
[L^{i}, K^{j}] &= \I{\epsilon^{ij}}_{k}K^{k}\\
[K^{i}, K^{j}] &= -\I{\epsilon^{ij}}_{k}J^{k}.
\end{align}
\end{subequations}

\begin{exercise}
Double check the calculations are correct. There may be a sign error.
\end{exercise}

\N{Complexified Lorentz Lie Algebra}
We are really interested in the irreducible complex linear
representations of the complexification $\iso(3,1)_{\CC}$ of the Lie
algebra $\iso(3,1)$ of the Lorentz group. This is because the Lorentz
group is non-compact (which makes it impossible to obtain a
finite-dimensional Unitary representation), so we change the goal to
find holomorphic representations of the complexified Lie Algebra for the
Lorentz group. This is precisely Weyl's Unitarian trick.

We define
\begin{equation}
\vec{J}_{\pm} = \frac{1}{2}(\vec{L}\pm\I\vec{K}).
\end{equation}
The reader may verify the commutation relations become
\begin{equation}
[J^{i}_{\pm}, J^{j}_{\pm}] = \I\epsilon^{ijk}J^{k}_{\pm}.
\end{equation}
But now look, this is precisely two copies of the complexified Lie
algebra $\su(2)_{\CC}$ (each copy is isomorphic to $\sl(2,\CC)$)
parametrized by the ``$\pm$'' subscripts.

Thus by the Unitarian trick, we can study representations of $\iso(3,1)$
by means of representations for $\su(2)_{\CC}$. Specifically
highest-weight representations of $\su(2)_{\CC}$.

\M
The punchline:
\textit{Each irreducible representation of $\iso(3,1)$
is characterized by a pair of half-integers $(j_{+}, j_{-})$.} We can
interpret these irreducible representations as particles, summarized by
the handy-dandy table:

\begin{center}
\begin{tabular}{c|c|c}
  $(j_{+}, j_{-})$ & Name of Field & Dimension of Rep \\\hline
  $(0, 0)$ &	Scalar  &	1\\
$(1/2, 0)$ & 	Left-handed Weyl Spinor &	2\\
$(0, 1/2)$ &	Right-handed Weyl Spinor &	2\\
$(1, 0)$ &	(Imaginary) Self-dual 2-form &	3\\
$(0, 1)$ &	(Imaginary) Anti-self-dual 2-form &	3\\
$(1/2, 1/2)$ &	Vector (gauge field) &	4\\
$(1/2, 1)$ & 	Left-Handed Rarita-Schwinger field &	6\\
$(1, 1/2)$ &	Right-Handed Rarita-Schwinger field &	6\\
$(1, 1)$ &	Graviton (spin-2 field) &	9
\end{tabular}
\end{center}

Quantum Field Theory studies merely three such representations:
\begin{itemize}
\item $(0,0)$ the scalar particle,
\item $(1/2,0)\oplus(0,1/2)$ Dirac fermionic particle, and
\item $(1/2, 1/2)$ vector (gauge Boson) particles.
\end{itemize}

\N{Infinite-Dimensional Unitary Representations}
Finite-dimensional representations have Lorentz transformations act on
finite-dimensional constant vectors. If we want to have Lorentz
transformations act on a multi-component field $\Phi_{a}(t,\vec{x})$,
then we need a Unitary infinite-dimensional representation for the
Lorentz group (or Poincar\'e group), something like:
\begin{equation}
\Phi_{a}\xrightarrow{\Lambda}\sum_{b}M_{ab}(\Lambda)\Phi_{b}
\end{equation}
where $M(\Lambda)$ treats the components of the field as components of a
constant vector. This is close but incomplete, we can also transform the
spacetime position:
\begin{equation}
\Phi_{a}(x)\xrightarrow{\Lambda}\sum_{b}M_{ab}(\Lambda)\Phi_{b}(\Lambda^{-1}x).
\end{equation}
As a ``smoke check'', we see that a scalar field is just a function on
spacetime $\psi(x)$, and a representation of a group $G$ on function space
gives
\begin{equation}
\forall g\in G, \rho(g)\psi(x)=\psi(g^{-1}\cdot x)
\end{equation}
where there's an ``obvious'' representation of $G$ on the domain of $\psi$.
And behold: the Poincar\'e group acts in an obvious way on space-time!

% https://sharif.edu/~sadooghi/QFT-I-96-97-2/LorentzPoincareMaciejko.pdf

\subsection{Poincar\'e Algebra}

\M We can adjoin the space-time translation generators to the algebra of
Lorentz transformations. We see $H$ generates time translations and
$P^{j}$ generates spatial translations.

\M
We know
\begin{subequations}\label{eq:rqm:poincare-algebra:conjugation-of-elements}
\begin{align}
U(\Lambda,a)M^{\alpha\beta}U^{-1}(\Lambda,a) 
&= {\Lambda_{\mu}}^{\alpha}{\Lambda_{\nu}}^{\beta}(M^{\mu\nu} - a^{\mu}P^{\nu} + a^{\nu}P^{\mu})\\
U(\Lambda,a)P^{\alpha}U^{-1}(\Lambda,a) &= {\Lambda_{\beta}}^{\alpha}P^{\beta}.
\end{align}
\end{subequations}

\N{Determining Commutation Relations}
We now consider infinitesimal group elements
${\Lambda_{\mu}}^{\nu}={\delta_{\mu}}^{\nu}+{\omega_{\mu}}^{\nu}$ and
$a^{\mu}=\epsilon^{\mu}$ in Eq~\eqref{eq:rqm:poincare-algebra:conjugation-of-elements}.
This gives us
\begin{subequations}
\begin{align}
\I[\frac{1}{2}\omega_{\mu\nu}M^{\mu\nu}-\epsilon_{\mu}P^{\mu},M^{\rho\sigma}]
&={\omega_{\mu}}^{\rho}M^{\mu\sigma} + {\omega_{\nu}}^{\sigma}M^{\rho\nu}
-\epsilon^{\rho}P^{\sigma} + \epsilon^{\sigma}P^{\rho}\\
\I[\frac{1}{2}\omega_{\mu\nu}M^{\mu\nu}-\epsilon_{\mu}P^{\mu},P^{\rho}]
&={\omega_{\mu}}^{\rho}P^{\mu}.
\end{align}
\end{subequations}
Equating coefficients of $\omega_{\mu\nu}$ and $\epsilon_{\mu}$ on both
sides of these equations gives us the commutation rules
\begin{subequations}
\begin{align}
\I[M^{\mu\nu},M^{\rho\sigma}] &= \eta^{\nu\rho}M^{\mu\sigma}
- \eta^{\mu\rho}M^{\nu\sigma} - \eta^{\sigma\mu}M^{\rho\nu} + \eta^{\sigma\tau}M^{\rho\mu}\\
\I[P^{\mu},M^{\rho\sigma}] &= \eta^{\mu\rho}P^{\sigma} - \eta^{\mu\sigma}P^{\rho}\\
[P^{\mu},P^{\nu}] &= 0.
\end{align}
\end{subequations}
This is the Lie algebra for the Poincar\'e group.

\M
We can rewrite the commutation relations using the generators for
rotations $L^{i}$ and Lorentz boosts $K^{i}$ from
Eq~\eqref{eq:rqm:poincare-algebra:rotations-and-boosts-generators}.
We recall
(\S\ref{chunk:rqm:poincare-algebra:lorentz-algebra:commutation-relations})
the commutation rules for the Lorentz Lie algebra with these rules:
\begin{subequations}
\begin{align}
[J^{i}, J^{j}] &= \I{\epsilon^{ij}}_{k}J^{k}\\
[J^{i}, K^{j}] &= \I{\epsilon^{ij}}_{k}K^{k}\\
[K^{i}, K^{j}] &= -\I{\epsilon^{ij}}_{k}J^{k}
\end{align}
\end{subequations}
Now we add the generators $P^{j}$ for spatial translations and $H$ for
temporal translations.
\begin{subequations}
\begin{align}
[J^{i}, P^{j}] &= \I{\epsilon^{ij}}_{k}P^{k}\\
[K^{i}, P^{j}] &= \I\delta^{ij}H\\
[J^{i}, H] &= [P^{i}, H] = [H, H] = 0\\
[K^{i}, H] &= \I P^{i},
\end{align}
\end{subequations}
where the indices $i$, $j$, $k=1,2,3$.

\begin{ddanger}
For our purposes in quantum field theory, we will use the universal
covering group for the Poincar\'e group. The universal covering group
for the Lorentz group $\ISO(3,1)$ is the indefinite Spin group $\Spin(3,1)$.
Since the Poincar\'e group is the semidirect product of translations
$\RR^{3,1}$ with the Lorentz group $\ISO(3,1)$, the universal covering
is the semidirect product with $\Spin(3,1)$. This is referred to as the
\define{Poincar\'e Spin Group} or \emph{Poincar\'e Spinor Group} in the
literature, though physicists seldom care about such sordid details.
\end{ddanger}

\N{References}
A detailed discussion of the Poincar\'e and Lorentz groups (and their
Lie algebras) may be found in Chapter 2 of
Weinberg~\cite{Weinberg:1995mt}.
An ``executive summary'' may be found in Srednicki~\cite{Srednicki:2007qs}.

