\section{*Quantization of Free Relativistic Particle}\label{section:rqm:constrained-quantization-of-free-particle}

\N{Warning/disclaimer/spoiler alert}
The reader should first solve
Exercise~\ref{xca:relativity:canonical-analysis-of-free-particle} before
continuing with this section. You have been warned.

\M It turns out quantizing the free relativistic particle will produce
the Klein--Gordon equation, but this requires care because the free
relativistic particle \emph{is} a constrained system. We will work
through the constraint analysis, and perform some rudimentary Dirac
quantization of the system. This is just another route to the same
destination.

\N{Lagrangian}
The Lagrangian we will be working with is:
\begin{equation}
L = -m_{0}c\sqrt{-\eta_{\alpha\beta}\frac{\D x^{\alpha}}{\D\tau}\frac{\D x^{\beta}}{\D\tau}}.
\end{equation}

\begin{remark}
This carries over to General Relativity, if we replace the Minkowski
metric with the spacetime metric $\eta_{\alpha\beta}\mapsto g_{\alpha\beta}$.
The Lagrangian describes geodesics.
\end{remark}

\N{Parameter Time}
To be fully general, we can replace the proper time parametrization
$x^{\mu}(\tau)$ with any other parametrization, but will keep the
notation $\tau$ for the parameter of the trajectory. We refer to $\tau$
as ``parameter time''.

If we really need to use ``proper time'' in an equation, we will use $s/c$.
But I may slip up and refer to $\tau$ as proper time (because old habits
die hard).

\N{Conjugate Momenta}
We find
\begin{calculation}
  p_{\mu}
  \step{definition of conjugate momenta}
  \frac{\partial L}{\partial\dot{x}^{\mu}(\tau)}
  \step{since $L=-m_{0}c\sqrt{-\eta_{\alpha\beta}\dot{x}^{\alpha}\dot{x}^{\beta}}$}
\frac{-m_{0}c}{2}\frac{(-\eta_{\alpha\beta})(\delta^{\alpha}_{\mu}\dot{x}^{\beta}+\dot{x}^{\alpha}\delta^{\beta}_{\mu})}{\sqrt{-\eta_{\alpha\beta}\dot{x}^{\alpha}\dot{x}^{\beta}}}
\step{index gymnastics and pulling the factor of $-1$ to the front}
\frac{m_{0}c}{2}\frac{(\eta_{\mu\beta}\dot{x}^{\beta}+\dot{x}^{\alpha}\eta_{\alpha\mu})}{\sqrt{-\eta_{\alpha\beta}\dot{x}^{\alpha}\dot{x}^{\beta}}}
\step{more index gymnastics}
\frac{m_{0}c}{2}\frac{(\dot{x}_{\mu}+\dot{x}_{\mu})}{\sqrt{-\eta_{\alpha\beta}\dot{x}^{\alpha}\dot{x}^{\beta}}}
\step{arithmetic}
m_{0}c\frac{\dot{x}_{\mu}}{\sqrt{-\eta_{\alpha\beta}\dot{x}^{\alpha}\dot{x}^{\beta}}}.
\end{calculation}
Thus
\begin{equation}
\boxed{p_{\mu} = m_{0}c\frac{\dot{x}_{\mu}}{\sqrt{-\eta_{\alpha\beta}\dot{x}^{\alpha}\dot{x}^{\beta}}}.}
\end{equation}

\begin{exercise}
Prove $p_{\mu}p^{\mu}+m_{0}^{2}c^{2}=0$.
\end{exercise}

\begin{exercise}
Prove $\sqrt{\vec{p}\cdot\vec{p}+m_{0}^{2}c^{2}} = m_{0}c\dot{x}^{0}/\sqrt{-\dot{x}^{2}}$.
And therefore
\begin{equation}
\sqrt{-\dot{x}^{2}} = \frac{m_{0}c\dot{x}^{0}}{\sqrt{\vec{p}\cdot\vec{p}+m_{0}^{2}c^{2}}}.
\end{equation}
\end{exercise}

\begin{exercise}[IMPORTANT: DO THIS!!]\label{xca:rqm:quantize-free-particle:spatial-part-of-kinetic-term}
Prove
\begin{subequations}
  \begin{align}
    \sqrt{\vec{p}\cdot\vec{p}+m_{0}^{2}c^{2}} &= \frac{m_{0}c\dot{x}^{0}}{\sqrt{-\dot{x}^{2}}}\\
    \vec{p}\cdot\dot{\vec{x}} &= \frac{\vec{p}\cdot\vec{p}\dot{x}^{0}}{\sqrt{\vec{p}^{2}+c^{2}m_{0}^{2}}}
  \end{align}
and so
\begin{equation}
\sqrt{-\eta_{\mu\nu}\dot{x}^{\mu}\dot{x}^{\nu}}=\sqrt{-\dot{x}^{2}} = \frac{m_{0}c\dot{x}^{0}}{\sqrt{\vec{p}\cdot\vec{p}+m_{0}^{2}c^{2}}}.
\end{equation}
\end{subequations}
[Hint: use solution to the previous exercise, and
  $\dot{\vec{x}}=(m_{0}c/m_{0}c)(\sqrt{-\dot{x}^{2}}/\sqrt{-\dot{x}^{2}})\dot{\vec{x}}$
and the definition of the conjugate momenta.]
\end{exercise}

\N{Canonical Hamiltonian}
I will write
$\dot{x}^{2}=\eta_{\alpha\beta}\dot{x}^{\alpha}\dot{x}^{\beta}$.
Then the naive Legendre transform of the Lagrangian is
\begin{calculation}
H_{\text{can}}
\step{Legendre transform}
p_{\mu}\dot{x}^{\mu} - L
\step{plugging in $L=-m_{0}c\sqrt{-\dot{x}^{2}}$}
p_{\mu}\dot{x}^{\mu} + m_{0}c\sqrt{-\dot{x}^{2}}
\step{plugging in $p_{\mu}=m_{0}c\dot{x}_{\mu}/\sqrt{-\dot{x}^{2}}$}
m_{0}c\frac{\dot{x}_{\mu}}{\sqrt{-\dot{x}^{2}}}\dot{x}^{\mu} + m_{0}c\sqrt{-\dot{x}^{2}}
\step{since $\dot{x}_{\mu}\dot{x}^{\mu}=(-1)(-\dot{x}^{2})=(-1)(\sqrt{-\dot{x}^{2}})^{2}$}
m_{0}c\frac{(-1)(\sqrt{-\dot{x}^{2}})^{2}}{\sqrt{-\dot{x}^{2}}}\dot{x}^{\mu} + m_{0}c\sqrt{-\dot{x}^{2}}
\step{algebra}
-m_{0}c\sqrt{-\dot{x}^{2}}+m_{0}c\sqrt{-\dot{x}^{2}}
\step{more algebra}
0.
\end{calculation}
Therefore
\begin{equation}
\boxed{H_{\text{can}} = 0.}
\end{equation}
This is a constraint.

A more careful analysis shows that
\begin{calculation}
H_{\text{can}}
\step{Legendre transform with spatial vectors identifier}
-p^{0}\dot{x}^{0} + \vec{p}\cdot\dot{\vec{x}} - L
\step{using Exercise~\ref{xca:rqm:quantize-free-particle:spatial-part-of-kinetic-term}}
-p^{0}\dot{x}^{0} + \frac{\vec{p}\cdot\vec{p}\dot{x}^{0}}{\sqrt{\vec{p}\cdot\vec{p}+m_{0}^{2}c^{2}}}
- L
\step{unfolding the definition of $L$}
-p^{0}\dot{x}^{0} + \frac{\vec{p}\cdot\vec{p}\dot{x}^{0}}{\sqrt{\vec{p}\cdot\vec{p}+m_{0}^{2}c^{2}}}
+ m_{0}c\sqrt{(\dot{x}^{0})^{2}- \dot{\vec{x}}\cdot\dot{\vec{x}}}
\step{using Exercise~\ref{xca:rqm:quantize-free-particle:spatial-part-of-kinetic-term}}
-p^{0}\dot{x}^{0} + \frac{\vec{p}\cdot\vec{p}\dot{x}^{0}}{\sqrt{\vec{p}\cdot\vec{p}+m_{0}^{2}c^{2}}}
+ \frac{m_{0}^{2}c^{2}\dot{x}^{0}}{\sqrt{\vec{p}\cdot\vec{p}+m_{0}^{2}c^{2}}}
\step{collecting terms}
-p^{0}\dot{x}^{0} + \frac{\vec{p}\cdot\vec{p}\dot{x}^{0} + m_{0}^{2}c^{2}\dot{x}^{0}}{\sqrt{\vec{p}\cdot\vec{p}+m_{0}^{2}c^{2}}}
\step{factoring out $\dot{x}^{0}$}
\dot{x}^{0}\left(-p^{0} + \frac{\vec{p}\cdot\vec{p} + m_{0}^{2}c^{2}}{\sqrt{\vec{p}\cdot\vec{p}+m_{0}^{2}c^{2}}}\right)
\step{algebra}
\dot{x}^{0}\left(-p^{0} + \sqrt{\vec{p}\cdot\vec{p}+c^{2}m_{0}^{2}}\right)
\step[\approx]{imposed as a constraint}
0.
\end{calculation}
Returning to the mass--shell condition
$p_{\mu}p^{\mu}+m_{0}^{2}c^{2}\approx0$, the Hamiltonian tells us to use
the positive branch for the squareroot.

\N{Super-Hamiltonian}
Continuing along with our analysis, we can write down the ``super
Hamiltonian'' for the relativistic free particle as
\begin{equation}
H_{S} = \eta^{\mu\nu}p_{\mu}p_{\nu} + m_{0}^{2}c^{2}\approx 0,
\end{equation}
and the action becomes
\begin{equation}
S = \int (p_{\mu}\dot{x}^{\mu} - NH_{S})\,\D\tau,
\end{equation}
where $N$ is the Lagrange multiplier associated with the [Hamiltonian]
constraint.

\M
We can determine the meaning of the Lagrange multiplier by using
Hamilton's equation of motion for $x^{0}$ to give us
\begin{equation}
\dot{x}^{0} = \frac{\partial(NH_{S})}{\partial p_{0}} = -2Np_{0}.
\end{equation}
We then have
\begin{calculation}
N 
\step{from the previous equation}
\frac{-1}{2}\frac{\dot{x}^{0}}{p_{0}}
\step{using $-p_{0}=p^{0}=\sqrt{\vec{p}\cdot\vec{p}+c^{2}m_{0}^{2}}$}
\frac{1}{2}\frac{\dot{x}^{0}}{\sqrt{\vec{p}\cdot\vec{p}+c^{2}m_{0}^{2}}}
\step{multiply top and bottom by $m_{0}c$}
\frac{1}{2m_{0}c}\frac{m_{0}c\dot{x}^{0}}{\sqrt{\vec{p}\cdot\vec{p}+c^{2}m_{0}^{2}}}
\step{using Exercise~\ref{xca:rqm:quantize-free-particle:spatial-part-of-kinetic-term}}
\frac{1}{2m_{0}c}\sqrt{-\dot{x}^{2}}
\step{since $\sqrt{-\dot{x}^{2}}=\D s/\D\tau$}
\frac{1}{2m_{0}c}\frac{\D s}{\D\tau}.
\end{calculation}
This has the interpretation of being directly proportional to the rate
of change of proper time $s/c$ with respect to the parameter time
$\tau$.

\begin{exercise}
Prove:
\begin{equation}
N = \frac{\dot{x}}{2m_{0}c\gamma},
\end{equation}
where $\gamma$ is the standard Lorentz factor.
\end{exercise}

\N{Quantizing}
We can then quantize to obtain the constraint operator
\begin{equation}
  \begin{split}
\widehat{H}_{S} &= \eta^{\mu\nu}\widehat{p}_{\mu}\widehat{p}_{\nu}+m_{0}^{2}c^{2}\\
&= (-\hbar^{2}\Box + m_{0}^{2}c^{2}).
  \end{split}
\end{equation}
Here $\Box = -c^{-2}\partial_{t}^{2} + \nabla^{2}$ is the D'Alembertian
operator.
The physical states then live in the kernel of $\widehat{H}_{S}$, i.e.,
satisfy the constraint equation:
\begin{equation}
\widehat{H}_{S}\mid\mbox{phys}\rangle = 
(-\hbar^{2}\Box + m_{0}^{2}c^{2})\mid\mbox{phys}\rangle = 0.
\end{equation}
The astute reader will realize this is the Klein--Gordon equation for a
scalar field.

Also observe that the [classical] parameter time $\tau$ drops out of the
quantum equations of motion because ``trajectories'' do not exist in the
quantum setting.

\N{Reparametrization Invariance}
The system we have just analyzed is invariant under reparametrization
$\tau\mapsto\alpha\tau+\beta$ for some positive $\alpha>0$ and
real $\beta\in\RR$.
This has been analyzed in the literature on quantum gravity, see, e.g.,
Teitelboim~\cite{Teitelboim:1981ua}.
