\chapter{Classical Fields}

\M
The general idea is that we will review/introduce classical fields, then
appeal to an heuristic quantization procedure to obtain quantum fields.
The motivation is a careful derivation of the linear chain, where we
have an infinite number of identical point-masses connected by identical
massless springs. Taking the continuum limit gives us the scalar field.

Most fields can intuitively be imagined as a ``tuple of scalar fields''.
For example, electromagnetism has its field $A^{\mu}$ be a 4-tuple of
scalar fields which are related in some manner (e.g., enjoying Lorentz
invariance and satisfies the equations of motion). This is a lie, but a
comforting lie that physicists tell themselves.

\M
There is another, less discussed, derivation of classical fields from
demanding certain ``background independence'' properties. We can derive
gauge theory in this manner, too. Teitelboim~\cite{Teitelboim:1980hs}
wrote a review of this approach, and it appears to be folklore among
quantum gravity researchers (especially those working in the canonical
approaches). This requires accepting a few innocent axioms and working
within the Hamiltonian framework.

%\includegraphics{img/img.0}
\section{Linear Chain}

\N{Problem Statement}
Consider $N\in\NN$ identical point-masses (each with rest mass $m$)
each of which are connected to two neighboring point masses by identical
massless springs with spring constant $k$ and equilibrium length $a$.
The point-masses then form a line segment.

We will take $N\to\infty$ limit in such a way that at any point-mass
there are infinitely many point-masses in either direction.
This specifically is to let us ignore the boundary conditions.

What are the equations of motion for a point-mass in this chain? What is
the Lagrangian for this system?

Take the continuum limit where $a\to0$ while $a^{2}k/m$ is held
constant. What happens to the equations of motion and the Lagrangian?

\begin{exercise}
What dimensions does $a^{2}k/m$ have?
\end{exercise}

\N{Coordinates}
Since we have the point-masses form a one-dimensional system, we will
write $x_{j}$ for the position of the point-mass with $j\in\ZZ$.

\N{Free Body Diagram}
Suppose we examine the free-body diagram for the point-mass. The only
forces acting on a point-mass $x_{j}$ are the spring forces:
\begin{center}
\includegraphics{img/img.0}
\end{center}

\N{Equations of Motion}
Then we see the force acting on $x_{j}$ is
\begin{equation}
\begin{split}
  F_{j} &= -k(x_{j}-x_{j-1}-a) + k(x_{j+1}-x_{j}-a)\\
  &= k(x_{j+1}-2x_{j}+x_{j}).
\end{split}
\end{equation}
Using Newton's second Law,
\begin{equation}
m\ddot{x}_{j} = F_{j} = k(x_{j+1}-2x_{j}+x_{j}).
\end{equation}
We will rearrange this to:
\begin{equation}\label{eq:classical-field-theory:linear-chain:newton-eom}
\ddot{x}_{j} = \frac{k}{m}(x_{j+1}-2x_{j}+x_{j}).
\end{equation}


\N{Lagrangian}
We can then write the Lagrangian for this system,
\begin{equation}
L = \sum_{j\in\ZZ}\frac{m}{2}\dot{x}^{2}_{j} - \frac{k}{2}(x_{j+1}-x_{j})^{2}.
\end{equation}
Since the sum is over all integers, the forces acting on $x_{j}$ come
from the $j-1$ term and the $j$ term.

\N{Continuum Limit}
Now care must be taken, because as $a\to 0$ the index $j$ labeling
particles will become a real number indicating the position of the
particle. To avoid ambiguity, we will write $q_{j}(t)$ for the position
of particle $j$.

We observe as $a\to0$, we have $q_{j}(t)\to q(x,t)$
\begin{equation}
\frac{x_{j+1}-2x_{j}+x_{j-1}}{a^{2}}\xrightarrow{a\to0}\frac{\partial^{2}}{\partial x^{2}}q(x,t).
\end{equation}
Then the continuum limit of the equations of motion,
Eq~\eqref{eq:classical-field-theory:linear-chain:newton-eom}, (first
dividing through by $m$) is:
\begin{equation}
\ddot{x}_{j}\xrightarrow{a\to0}\frac{\partial^{2}}{\partial t^{2}}q(x,t),
\quad\mbox{and}\quad\frac{ka^{2}}{m}\frac{x_{j+1}-2x_{j}+x_{j-1}}{a^{2}}
\xrightarrow{a\to0}v^{2}
\frac{\partial^{2}}{\partial x^{2}}q(x,t).
\end{equation}
Then equating both limits gives us:
\begin{equation}
\frac{\partial^{2}}{\partial t^{2}}q(x,t) = v^{2}
\frac{\partial^{2}}{\partial x^{2}}q(x,t).
\end{equation}
This is precisely the wave equation for an elastic string.
Here $v=\sqrt{a^{2}k/m}$ is the velocity of propagation.

\begin{exercise}
We have been working with one spatial dimension, assuming it is $\RR$.
Suppose space is a circle $S^{1}$ and our linear chain forms a
ring. Perform the continuum limit analysis for this situation.
\end{exercise}

\begin{exercise}
If we took space to be a closed interval $[a,b]$ instead of $\RR$,
then what boundary conditions do we need to impose for things to work
out in the continuum limit?
\end{exercise}

\N{Canonical Analysis}
We can perform the Legendre transform of the Lagrangian, first finding
the conjugate momenta
\begin{equation}
p_{j} = \frac{\partial L}{\partial\dot{q}_{j}} = m\dot{q}_{j}.
\end{equation}
Then
\begin{equation}
H = \sum_{j\in\ZZ}\frac{p_{j}^{2}}{2m} + k(q_{j+1}-q_{j})^{2}.
\end{equation}

\section{Scalar Field}

\M
We have the Klein--Gordon field\index{Klein--Gordon!Field}
be described by the Lagrangian in
Eq~\eqref{eq:classical-field-theory:linear-chain:continuum-limit:mass-term:lagrangian}.
More generally, since we could add an arbitrary potential term
$V(\varphi)$, the Lagrangian looks like:
\begin{equation}\label{eq:classical-field-theory:scalar-field:lagrangian}
\begin{split}
  \mathcal{L} &= -\frac{1}{2}\partial^{\mu}\varphi\partial_{\mu}\varphi
-\frac{1}{2}\frac{m^{2}c^{2}}{\hbar^{2}}\varphi^{2} - V(\varphi)\\
&=\frac{1}{2}c^{-2}(\partial_{t}\varphi)^{2}-\frac{1}{2}(\nabla\varphi)^{2}
-\frac{1}{2}\frac{m^{2}c^{2}}{\hbar^{2}}\varphi^{2} - V(\varphi).
\end{split}
\end{equation}
Our physical intuition should be that of a linear chain of
point-particles connected by massless identical springs, as we have
analyzed in the previous section.

\begin{exercise}
Can we interpret the $\frac{1}{2}(\nabla\varphi)^{2}$ term in the
Lagrangian as a contribution to the potential? If so, what is its
physical interpretation?
\end{exercise}

\begin{definition}\index{Klein--Gordon!Field}\index{Field!Scalar!Free}\index{Scalar Field!Free}
When $V=0$ in the Lagrangian, we call the type of field a
\define{Klein--Gordon Field} (or \emph{Free Scalar Field}).
\end{definition}

\begin{remark}
We could have $V(\varphi) = a + b\varphi + c\varphi^{2}$ and reabsorb
$a$, $b$, $c$ into the mass and elsewhere in the Lagrangian, producing
an equivalent free scalar field. And physicists will be a little sloppy
in their language, referring to such Lagrangians with
nonzero-but-quadratic potentials as ``free''.
\end{remark}

\begin{definition}
When $V(\varphi)\neq0$ (and specifically $V(\varphi)$ is not a constant,
or a linear or quadratic polynomial), we say we have a
\define{Self-Interacting Scalar Field}.
\end{definition}

\begin{definition}
When we have a linear $\varphi$ term in the Lagrangian's potential term
(something like $\sigma\varphi$), we refer to it as an
\define{External Source}.
\end{definition}

\begin{remark}
This terminology may seem bizarre at first, but it generalizes the
4-current which we used to recover Maxwell's equations (\S\ref{chunk:relativity:electromagnetism:recovering-maxwell-equations}).
The 4-current captured our intuition of coupling electromagnetism to
matter (``charged particles''), and the external source couples the
field to ``generic matter''.

We will also find, however, that it's useful to stick in an external
source into the Lagrangian for mathematical purposes. It's a
mathematical trick where we will differentiate with respect to external
sources to compute moments of integrals, then set the external source to
zero. 
\end{remark}

\N{Goals}
We will study the \emph{free} scalar field in this section. 

\N{Variational Analysis}
We consider a region $\mathcal{R}\subset\RR^{3,1}$, usually taken to be
$\RR^{3}\times[t_{1},t_{2}]$ for some $t_{1}<t_{2}$. Now we will
consider the action
\begin{equation}
\action[\varphi] = \int_{\mathcal{R}}\left(\frac{1}{2}c^{-2}(\partial_{t}\varphi)^{2}-\frac{1}{2}(\nabla\varphi)^{2}
-\frac{1}{2}\frac{m^{2}c^{2}}{\hbar^{2}}\varphi^{2}\right)\D^{3}\vec{x}\,\D t.
\end{equation}
As usual, we will try to find the critical points of the action, and
argue these are the physical solutions to the equations of motion.

Now, we take a variation of this action with respect to $\varphi$. This
is done by writing
\begin{equation}
\varphi_{\lambda} = \varphi + \lambda\psi,
\end{equation}
where $\lambda$ is a real parameter, $\psi|_{\partial\mathcal{R}}=0$ is
an (otherwise) arbitrary function. Physicists write
\begin{equation}
\delta\varphi = \lambda\psi,
\end{equation}
and pretend $\lambda$ is an infinitesimal quantity $\lambda^{2}\ll1$.
Then expanding the integrand in the action to first-order in $\lambda\psi$,
we demand the coefficient to $\lambda\psi$ vanish. We find,
\begin{calculation}
\action[\varphi_{\lambda}]
\step{plugging in the definition of the action}
\int_{\mathcal{R}}\left(\frac{1}{2}c^{-2}(\partial_{t}\varphi_{\lambda})^{2}-\frac{1}{2}(\nabla\varphi_{\lambda})^{2}
-\frac{1}{2}\frac{m^{2}c^{2}}{\hbar^{2}}\varphi_{\lambda}^{2}\right)\D^{3}\vec{x}\,\D t
\step{unfold the definition of $\varphi_{\lambda}$}
\int_{\mathcal{R}}\left(\frac{1}{2}c^{-2}(\partial_{t}[\varphi + \lambda\psi])^{2}-\frac{1}{2}(\nabla[\varphi + \lambda\psi])^{2}
-\frac{1}{2}\frac{m^{2}c^{2}}{\hbar^{2}}[\varphi + \lambda\psi]^{2}\right)\D^{3}\vec{x}\,\D t
%% \step{expand}
%% \int_{\mathcal{R}}\left(\frac{1}{2}c^{-2}
%% (\partial_{t}\varphi)^{2}+c^{-2}\partial_{t}\varphi\partial_{t}\psi+\frac{1}{2}c^{-2}(\partial_{t}\psi)^{2}
%% -\frac{1}{2}(\nabla\varphi)^{2}
%% -(\nabla\varphi)\cdot(\nabla(\lambda\psi))
%% -\frac{1}{2}(\nabla(\lambda\psi))\cdot(\nabla(\lambda\psi))
%% -\frac{1}{2}\frac{m^{2}c^{2}}{\hbar^{2}}[\varphi^{2} + 2\lambda\psi\varphi + \lambda^{2}\psi^{2}]\right)\D^{3}\vec{x}\,\D t
\step{expanding and collecting coefficients of $\lambda$}
\action[\varphi] + \int_{\mathcal{R}}\left(
c^{-2}\partial_{t}\varphi\partial_{t}(\lambda\psi)
-(\nabla\varphi)\cdot(\nabla(\lambda\psi))
-\frac{m^{2}c^{2}}{\hbar^{2}}\varphi\cdot(\lambda\psi)\right)\D^{3}\vec{x}\,\D t
+\action[\lambda\psi].
\end{calculation}
We write the first variation of the action as:
\begin{equation}
\delta\action[\varphi_{\lambda}] = \int_{\mathcal{R}}\left(
c^{-2}\partial_{t}\varphi\partial_{t}(\lambda\psi)
-(\nabla\varphi)\cdot(\nabla(\lambda\psi))
-\frac{m^{2}c^{2}}{\hbar^{2}}\varphi\cdot\lambda\psi\right)\D^{3}\vec{x}\,\D t.
\end{equation}

\N{Initial conditions}
We need to specify the initial and final configuration for the scalar
field, so for $\mathcal{R}=\mathcal{R}_{3}\times[t_{1},t_{2}]\subset\RR^{3,1}$,
we need functions
\begin{equation}
\varphi_{1},\varphi_{2}\colon\mathcal{R}_{3}\to\RR,
\end{equation}
such that
\begin{equation}
\varphi|_{\mathcal{R}_{3}\times\{t_{1}\}}=\varphi_{1},\quad\mbox{and}\quad
\varphi|_{\mathcal{R}_{3}\times\{t_{2}\}}=\varphi_{2}.
\end{equation}

\M
We can integrate $\delta\action[\varphi_{\lambda}]$ by parts (with
respect to time) to get
\begin{equation}
\delta\action[\varphi_{\lambda}] = \int_{\mathcal{R}}\left(
-c^{-2}(\lambda\psi)\partial_{t}^{2}\varphi
-(\nabla\varphi)\cdot(\nabla(\lambda\psi))
-\frac{m^{2}c^{2}}{\hbar^{2}}\varphi\cdot\lambda\psi\right)\D^{3}\vec{x}\,\D t.
\end{equation}
The boundary terms from this integration-by-parts vanishes since
$\lambda\psi|^{t_{2}}_{t_{1}}=0$.

\M
When we have $\mathcal{R}=\RR^{3}\times[t_{1},t_{2}]$, we treat
$\RR^{3}$ as a sphere with radius $r\to\infty$. Doing so allows us to
use the divergence theorem to further rewrite the first variation of the
action as, when $\vec{n}$ is the outward-pointing unit normal vector,
\begin{equation}
\begin{split}
\delta\action[\varphi_{\lambda}] &= \int_{\mathcal{R}}\left(
-c^{-2}(\lambda\psi)\partial_{t}^{2}\varphi
+(\lambda\psi)\nabla^{2}\varphi
-\frac{m^{2}c^{2}}{\hbar^{2}}\varphi\cdot\lambda\psi\right)\D^{3}\vec{x}\,\D t\\
&\qquad-\int^{t_{2}}_{t_{1}}\int_{r\to\infty}(\lambda\psi)\vec{n}\cdot(\nabla\varphi)%
\,\D A\,\D t,
\end{split}
\end{equation}
where $\D A$ is the differential area for the sphere. (In $n+1$
dimensions, $\D A\sim r^{n-2}\,\D r$.)
We have implicitly used the fact that
\begin{equation}
(\nabla\varphi)\cdot\nabla(\lambda\psi)=\nabla\cdot((\nabla\varphi)\lambda\psi)
-(\lambda\psi)\nabla^{2}\varphi,
\end{equation}
before using the divergence theorem.

\N{Assumption on growth of field}
We need to assume that, at a fixed time $t$, the scalar field $\varphi$
(and therefore both $\varphi_{\lambda}$ and $\lambda\psi$) fall off as
$r\to\infty$ sufficiently fast so as to make the boundary term in the
first variation of the action $\delta\action$ vanishes. One possibility
is to work with fields with compact support.

The usual assumption physicists make is that, since the area eleement
$\D A$ grows like $r^{2}$, the integrand must fall faster than $1/r^{2}$
as $r\to\infty$ for the boundary term to vanish.

More generally, in $n+1$ dimensional spacetime, $\D A$ grows like $r^{n-1}$,
requiring the integrand to fall faster than $1/r^{n-1}$ as $r\to\infty$
for the boundary terms to vanish.

\M
Now we see that the first variation of the action is just
\begin{subequations}
\begin{equation}
\delta\action[\varphi] = \int_{\mathcal{R}}\left(
-c^{-2}(\lambda\psi)\partial_{t}^{2}\varphi
+(\lambda\psi)\nabla^{2}\varphi
-\frac{m^{2}c^{2}}{\hbar^{2}}\varphi\cdot\lambda\psi\right)\D^{3}\vec{x}\,\D t,
\end{equation}
or factoring out the $\delta\varphi=\lambda\psi$,
\begin{equation}
\delta\action[\varphi] = \int_{\mathcal{R}}\left(
-c^{-2}\partial_{t}^{2}\varphi
+\nabla^{2}\varphi
-\frac{m^{2}c^{2}}{\hbar^{2}}\varphi\right)\delta\varphi\,\D^{3}\vec{x}\,\D t.
\end{equation}
\end{subequations}
Now we need to find the critical points of the action, which is to say,
we demand $\delta\action[\varphi]=0$. This gives us a rather tricky
differential-integral equation, but fortunately the fundamental lemma of
variational calculus\marginnote{TODO: cite fundamental lemma of variational calculus} says the condition is exactly demanding the
integrand's coefficient of $\delta\varphi$ vanishes, i.e.,
\begin{equation}
\boxed{-c^{-2}\partial_{t}^{2}\varphi
+\nabla^{2}\varphi
-\frac{m^{2}c^{2}}{\hbar^{2}}\varphi=0.}
\end{equation}
Solving this differential equation gives us the critical point for the
action.

Usually we expedite this whole process of variational analysis, and just
use the Euler--Lagrange equations.

\begin{exercise}
Re-perform this analysis with an arbitrary potential contribution
$V(\varphi)$ and observe how the equations of motion change (i.e., what
extra term will be added to the equations of motion, something involving
$V'(\varphi)$ we expect).
\end{exercise}

\N{Equations of Motion}
We can now derive the equations of motion using the Euler--Lagrange
equations, which are:
\begin{equation}
c^{-2}\partial_{t}^{2}\varphi - \nabla^{2}\varphi + \frac{m^{2}c^{2}}{\hbar^{2}}\varphi+V'(\varphi)=0.
\end{equation}
Also note that $V(\varphi)$ is usually some polynomial in $\varphi$, and
we can discard the terms lower than quadratic order (and absorb the
quadratic term into the mass term).

\begin{proof}
  We have
  \begin{subequations}
  \begin{equation}
\frac{\partial\mathcal{L}}{\partial(\partial_{\mu}\varphi)}
=-\partial^{\mu}\varphi,
  \end{equation}
  so the ``acceleration'' part of the equations of motion:
  \begin{equation}
\partial_{\mu}\frac{\partial\mathcal{L}}{\partial(\partial_{\mu}\varphi)}
=-\partial_{\mu}\partial^{\mu}\varphi.
  \end{equation}
  Then the ``force'' part of the equations of motion
\begin{equation}
\frac{\partial\mathcal{L}}{\partial\varphi} = -\frac{m^{2}c^{2}}{\hbar^{2}}\varphi
- V'(\varphi).
\end{equation}
Taken altogether, the equations of motion read:
\begin{equation}
-\partial_{\mu}\partial^{\mu}\varphi = -\frac{m^{2}c^{2}}{\hbar^{2}}\varphi
- V'(\varphi).
\end{equation}
  \end{subequations}
  Some gentle messaging yields the result.
\end{proof}

\N{Solutions}
We can solve the equations of motion for the \emph{free} scalar field by
taking its Fourier transform. We have
\begin{equation}
\varphi(t,\vec{x}) = \iint\E^{-\I(\omega t-\vec{k}\cdot\vec{x})}\widetilde{\varphi}(\omega,\vec{k})\frac{\D^{3}\vec{k}}{(2\pi\hbar)^{3}}\frac{\D\omega}{2\pi\hbar}
\end{equation}
where $\widetilde{\varphi}$ is the Fourier transform, $\omega=E/\hbar$,
and $\vec{k}=\vec{p}/\hbar$.

\begin{exercise}
Explicitly work this out. I mean, it's \emph{obvious}, but you should
check that it's \emph{true}.
\end{exercise}

\begin{exercise}
Let us consider the Fourier transform in the spatial variables only for
the scalar field,
\begin{equation}
\widetilde{\varphi}(\vec{k},t) = \int\E^{\I(\vec{k}\cdot\vec{x})}\varphi(\vec{x},t)\D^{3}\vec{x}.
\end{equation}
Prove or find a counter-example: the complex conjugate of the
\emph{spatially} Fourier
transformed scalar field $\widetilde{\varphi}(\vec{k},t)^{*}$
satisfies $\widetilde{\varphi}(\vec{k},t)^{*}=\widetilde{\varphi}(-\vec{k},t)$.

Also: do we need to include a factor of $(2\pi)^{-3/2}$ in the spatial Fourier
transform, or will things work out fine as we presented it? 
\end{exercise}

\subsection{Aside: Variational Calculus}

\M
We will typically use some heuristics in physics when doing variational
calculus, rather than working with the level of rigour anyone would hope
for. Given some functional of the form
\begin{equation}
F[\varphi] = \int_{\mathcal{R}}f(\varphi(x), \partial_{\mu}\varphi(x))\,\D^{4}x,
\end{equation}
we consider its first variation with respect to $\varphi$ by pretending:
\begin{enumerate}
\item $\delta$ is a linear operator obeying the Leibniz product rule (i.e., it
is a derivation), and
\item variations commute with differentiation (e.g., $\delta(\partial_{\mu}\varphi)=\partial_{\mu}(\delta\varphi)$).
\end{enumerate}
So to be clear, in the integrand, the $\delta$ operator acts like:
\begin{equation}
\delta f(\varphi(x)) 
= \frac{\partial f(\varphi)}{\partial\varphi}\delta\varphi.
\end{equation}
When there are also derivatives of $\varphi$, we have
\begin{subequations}
\begin{align}
\delta f(\varphi(x),\partial_{\mu}\varphi) 
&= \frac{\partial f(\varphi,\partial_{\mu}\varphi)}{\partial\varphi}\delta\varphi
+ \frac{\partial f(\varphi,\partial_{\mu}\varphi)}{\partial(\partial_{\mu}\varphi)}\delta(\partial_{\mu}\varphi)\\
\intertext{or, commuting variation with partial derivatives in the second term,}
\delta f(\varphi(x),\partial_{\mu}\varphi) &= \frac{\partial f(\varphi,\partial_{\mu}\varphi)}{\partial\varphi}\delta\varphi
+ \frac{\partial f(\varphi,\partial_{\mu}\varphi)}{\partial(\partial_{\mu}\varphi)}\partial_{\mu}(\delta\varphi)
\end{align}
\end{subequations}
When computing these partial derivatives, we pretend
\begin{equation}
\frac{\partial(\partial_{\mu}\varphi)}{\partial\varphi}
=0,\quad\mbox{and}\quad
\frac{\partial\varphi}{\partial(\partial_{\mu}\varphi)}=0.
\end{equation}

Then we have
\begin{equation}
\delta F[\varphi] = \int\left(\frac{\partial f}{\partial\varphi}\delta\varphi
+\frac{\partial f}{\partial(\partial_{\mu}\varphi)}\delta(\partial_{\mu}\varphi)\right)\,\D^{4}x.
\end{equation}
We try to integrate by parts to rewrite the integrand as
\begin{equation}
\delta F[\varphi] = \begin{pmatrix}\mbox{boundary}\\\mbox{terms}
\end{pmatrix}
+ \int\mbox{(something)}\delta\varphi\,\D^{4}x,
\end{equation}
then using the fundamental lemma of
variational calculus to set $\mbox{(something)}=0$. For most problems in
physics, we discard the boundary terms.\footnote{General Relativity is a
notable example where boundary terms are important.} This will give us a
differential equation whose solution is a critical point of the
functional.

\M
For multiple fields $\varphi_{a}$ with $a=1,\dots,N$, we need to take
the variation with respect to each of these fields. The first variation
would require summing over $a$, giving us $N$ coefficients of
$\delta\varphi_{a}$ (one for each $a$). We need each coefficient to
separately vanish, giving us a system of $N$ equations.

\M
These heuristics aren't ``wrong'', they just sweep details under the
rug. Specifically we would have
$\varphi_{a}^{(\lambda)}=\varphi_{a}+\lambda_{a}\psi_{a}$ where
$\psi_{a}$ are arbitrary functions which vanish at the initial and final
time slices. Then for any functional $F[\varphi_{a}]$ we find its first
variation as
\begin{equation}
\left.\frac{\D}{\D\lambda}F[\varphi^{(\lambda)}_{a}]\right|_{\lambda=0}=\delta
F[\varphi_{a},\psi_{a}]=\int\sum_{a}\frac{\delta F}{\delta\varphi_{a}}\psi_{a}\,\D^{4}x,
\end{equation}
where we just abuse notation left and right, using
$\delta F/\delta\varphi_{a}$ for the integrand, and sometimes writing
$\delta\varphi_{a}$ instead of $\psi_{a}$, and so on.

\begin{ddanger}
The notation used here is horribly sloppy in the physics literature, and
we have chosen to be consistent with that literature as much as possible.
The notation for a functional derivative coincides with a variational
derivative, which is horribly unfortunate. Some physicists try to relate
the two, but it's not quite the same thing.
\end{ddanger}

\N{Conditions for a minimum}
We recall from calculus in a single variable that not all critical
points of a function are minima (sometimes they are maxima, or
inflection points, or something else). This should caution us from being
too optimistic about critical points of functionals: the critical points
are \emph{candidates} for the minima, but we must check they are minima.

In calculus, this involves examining the sign of
$f''(x_{\text{crit}})$. When $f''(x_{\text{crit}})>0$, we have a
[possibly local] minimum.

For functionals, we need to examine the second variation $\delta^{2}F$
at the critical point. Specifically, for variation $\varphi_{\text{crit}}+\lambda\psi$,
we need
\begin{equation}
\delta^{2}F[\varphi_{\text{crit}}+\psi]\geq k\|\psi\|^{2}
\end{equation}
for all $\psi$ and some constant $k>0$. This is completely analogous to
the situation in calculus. Here we have (Taylor expanding in $\lambda\psi$, integrating by parts,
so the integrand is a Taylor polynomial in $\lambda\psi$):
\begin{equation}
F[\varphi+\lambda\psi] = F[\varphi] + \int\left(\frac{\delta F}{\delta\varphi}\lambda\psi
+\frac{\delta^{2} F}{\delta\varphi^{2}}\lambda^{2}\psi^{2} + \bigOh(\lambda^{3}\psi^{3})\right)\D^{n}x.
\end{equation}
Then we have 
\begin{equation}
\delta^{2}F[\varphi+\lambda\psi] = \int\frac{\delta^{2} F}{\delta\varphi^{2}}\lambda^{2}\psi^{2}\,\D^{n}x.
\end{equation}
This is the \define{Second Variation} of $F$.

However, this is all rather tedious, and in practice physicists drop the
``Mission Accomplished'' banner upon finding critical points for
functionals.

\begin{ddanger}
Not all functionals have a second variation. We need it to be twice
differentiable. For the functionals we care about in physics, which is
just the integral of a Lagrangian, we can compute the second variation.
\end{ddanger}

\N{References}
The quartic self-interacting scalar field (i.e., with
$V(\varphi)=\lambda\varphi^{4}/4!$ where $\lambda$ is the coupling
constant) has a few known exact solutions, which are presented in
Frasca~\cite{Frasca:2009bc}. These are quite tricky, since it requires
knowledge of Jacobi elliptical functions.
\section{Noether's Theorem}

\subsection{For Mechanics}

\M
Consider a mechanical system consisting of $N$ particles with positions
$q^{i}(t)$ for $i=1,\dots,N$. We describe it by its action
\begin{equation}
\action[q^{i}(t)] = \int^{t_{2}}_{t_{1}}L(q^{i},\dot{q}^{i})\,\D t.
\end{equation}
Now we suppose the dynamics is invariant under
\begin{equation}
q^{i}(t)\to q^{i}(t)+\delta q^{i}(t)
\end{equation}
where
\begin{equation}\label{eq:classical-field-theory:noether:epsilon-variation}
\delta q^{i}(t) = \epsilon^{a}(t)F^{i}_{a}(q,\dot{q}) + \dot{\epsilon}^{a}
G^{i}_{a}(q,\dot{q}),
\end{equation}
the $\epsilon^{a}$ are arbitrary functions of time (except possibly at
the endpoints) and $a=1,\dots,n$.

\M The action varies like:
\begin{equation}
\delta\action = \int^{t_{2}}_{t_{1}}L_{i}\,\delta q^{i}\,\D t
  + \left.\frac{\partial L}{\partial\dot{q}^{i}}\delta q^{i}\right|^{t_{2}}_{t_{1}},
\end{equation}
where the ``Euler'' derivatives are
\begin{equation}
L_{i} := \frac{\partial L}{\partial q^{i}}
- \frac{\D}{\D t}\frac{\partial L}{\partial\dot{q}^{i}}.
\end{equation}
The equations of motion are satisfied if and only if $L_{i}=0$ (these
are the Euler--Lagrange equations).

\N{Noether's Second Theorem}
Now, the variation of the action using
Eq~\eqref{eq:classical-field-theory:noether:epsilon-variation}
gives us
\begin{calculation}
\delta\action
\step{definition of variation of action}
\int^{t_{2}}_{t_{1}}L_{i}\,\delta q^{i}\,\D t
  + \left.\frac{\partial L}{\partial\dot{q}^{i}}\delta q^{i}\right|^{t_{2}}_{t_{1}}
\step{unfolding $\delta q^{i}$}
\int^{t_{2}}_{t_{1}}L_{i}\bigl(\epsilon^{a}(t)F^{i}_{a}(q,\dot{q}) + \dot{\epsilon}^{a} G^{i}_{a}(q,\dot{q})\bigr)\,\D t
  + \left.\frac{\partial L}{\partial\dot{q}^{i}}\bigl(\epsilon^{a}(t)F^{i}_{a}(q,\dot{q}) + \dot{\epsilon}^{a} G^{i}_{a}(q,\dot{q})\bigr)\right|^{t_{2}}_{t_{1}}
\step{integrate by parts to eliminate $\dot\epsilon^{a}$}
\int^{t_{2}}_{t_{1}}\epsilon^{a}(t)\left(L_{i}F^{i}_{a}(q,\dot{q})
- \frac{\D}{\D t}\bigl(L_{i} G^{i}_{a}(q,\dot{q})\bigr)\right)\D t
+ \left[\epsilon^{a}(t)L_{i}G^{i}_{a}
+ \frac{\partial L}{\partial\dot{q}^{i}}\bigl(\epsilon^{a}(t)F^{i}_{a}(q,\dot{q}) + \dot{\epsilon}^{a} G^{i}_{a}(q,\dot{q})\bigr)\right]^{t_{2}}_{t_{1}}
\step{associativity applied to boundary terms}
\int^{t_{2}}_{t_{1}}\epsilon^{a}(t)\left(L_{i}F^{i}_{a}(q,\dot{q})
- \frac{\D}{\D t}\bigl(L_{i} G^{i}_{a}(q,\dot{q})\bigr)\right)\D t
+ \left[\epsilon^{a}(t)\left(L_{i}G^{i}_{a}
+\frac{\partial L}{\partial\dot{q}^{i}}F^{i}_{a}\right)
+ \dot{\epsilon}^{a}\frac{\partial L}{\partial\dot{q}^{i}}G^{i}_{a}(q,\dot{q})\right]^{t_{2}}_{t_{1}}
\end{calculation}
As usual, we ignore boundary terms, then invariance of the action (up to
boundary terms) demands that:
\begin{equation}
\boxed{L_{i}F^{i}_{a}(q,\dot{q}) - \frac{\D}{\D t}\bigl(L_{i} G^{i}_{a}(q,\dot{q})\bigr) = 0.}
\end{equation}
These $n$ are identities are known as \define{Noether's Second Theorem}\index{Noether!Second theorem}
(or the \emph{generalized Bianchi identities}\index{Bianchi identity!Generalized}).

\N{Example, Noether's First Theorem}\index{Noether!First theorem}
Consider the special case when
\begin{equation}
\epsilon^{a}(t) = \epsilon^{a} = \mbox{constant}.
\end{equation}
Then
\begin{equation}
\delta q^{i} = \epsilon^{a}F^{i}_{a}.
\end{equation}
We then have the variation of the action, under this variation,
\begin{equation}
\delta\action = \int^{t_{2}}_{t_{1}}\epsilon^{a}F^{i}_{a}L_{i}\,\D t + \left.\vphantom{\frac{1}{1}}\epsilon^{a}F^{i}_{a}p_{i}\right|^{t_{2}}_{t_{1}},
\end{equation}
where we introduce the canonical momentum $p_{i}$. Define
\begin{equation}
C_{a} := F^{i}_{a}p_{i},
\end{equation}
then invariance of the action \emph{including the boundary terms}
leads to the condition
\begin{equation}
\left.\vphantom{\frac{1}{1}}F^{i}_{a}p_{i}-C_{a}\right|_{t_{2}}
=
\left.\vphantom{\frac{1}{1}}F^{i}_{a}p_{i}-C_{a}\right|_{t_{1}}.
\end{equation}
An invariance under a group with a finite number of parameters thus
leads to \emph{conservation laws}. This is Noether's first theorem.

\begin{remark}
Noether ends the first section of her paper by stating (as translated by
Traver), ``With these supplementary remarks, Theorem I comprises all
theorems on first integrals known to mechanics etc., while Theorem II
may be described as the utmost possible generalization of the `general
theory of relativity' in group theory.''
\end{remark}

\begin{remark}
More generally, when the general invariance under a group is
parametrized by functions of time, this will lead to \emph{constraints}
instead of conservation laws. This is what leads to the study of
constrained Hamiltonian systems and underpins the canonical formalism of
gauge theory.
\end{remark}

\subsection{For Fields}

\M
The results still hold, we just assume invariance under
\begin{subequations}
\begin{equation}
\varphi^{A}(x)\to\varphi^{A}+\delta\varphi^{A}(x)
\end{equation}
with
\begin{equation}
\delta\varphi^{A}(x) = \epsilon^{A}(x)F^{A}_{a} + (\partial_{\mu}\epsilon^{a}(x))G^{A\mu}_{a},
\end{equation}
\end{subequations}
and we now have arbitrary functions $\epsilon^{A}(x)$ of spacetime (not
just time).

\M
The more common presentation of Noether's first theorem begins by
supposing we have some infinitesimal symmetry transformation of the field
\begin{equation}
\varphi(x)\to\varphi'(x)=\varphi(x) + \alpha\,\Delta\varphi(x),
\end{equation}
where $\alpha$ is an infinitesimal quantity and $\Delta\varphi(x)$ is
the change in the field. We will hope the Lagrangian density transforms
like
\begin{equation}
\mathcal{L}(x)\to\mathcal{L}(x) + \alpha\partial_{\mu}\mathcal{J}^{\mu}(x)
\end{equation}
where $\mathcal{J}^{\mu}$ is ``something''. This is the same as adding
some boundary contribution to the action, which will not affect the
equations of motion.

Now, if we plug in the infinitesimally transformed field into the
Lagrangian
\begin{equation}
\mathcal{L}(\varphi + \alpha\,\Delta\varphi) = \mathcal{L} + \alpha\,\Delta\mathcal{L},
\end{equation}
where $\alpha\,\Delta\mathcal{L}$ may be found by Taylor expanding to
first-order:
\begin{calculation}
\alpha\,\Delta\mathcal{L}
\step{Taylor expansion to first order}
\frac{\partial\mathcal{L}}{\partial\varphi}(\alpha\,\Delta\varphi)
+\frac{\partial\mathcal{L}}{\partial(\partial_{\mu}\varphi)}\partial_{\mu}(\alpha\,\Delta\varphi)
\step{since $A\partial_{\mu}B=\partial_{\mu}(AB)-B\partial_{\mu}A$}
\frac{\partial\mathcal{L}}{\partial\varphi}(\alpha\,\Delta\varphi)
+\partial_{\mu}\left(\frac{\partial\mathcal{L}}{\partial(\partial_{\mu}\varphi)}\alpha\,\Delta\varphi\right)
-\left(\partial_{\mu}\frac{\partial\mathcal{L}}{\partial(\partial_{\mu}\varphi)}\right)(\alpha\,\Delta\varphi)
\step{collecting terms, factoring out $\alpha$}
\alpha\partial_{\mu}\left(\frac{\partial\mathcal{L}}{\partial(\partial_{\mu}\varphi)}\Delta\varphi\right)
+\alpha\left[\frac{\partial\mathcal{L}}{\partial\varphi}
-\partial_{\mu}\left(\frac{\partial\mathcal{L}}{\partial(\partial_{\mu}\varphi)}\right)\right]\Delta\varphi
\step{when the Euler--Lagrange equations hold}
\alpha\partial_{\mu}\left(\frac{\partial\mathcal{L}}{\partial(\partial_{\mu}\varphi)}\Delta\varphi\right)
+\alpha\cdot0\cdot\Delta\varphi
\step{multiplying the second term by zero makes it vanish}
\alpha\partial_{\mu}\left(\frac{\partial\mathcal{L}}{\partial(\partial_{\mu}\varphi)}\Delta\varphi\right).
\end{calculation}
This is precisely the sort of thing we're looking for: we want
$\alpha\,\Delta\mathcal{L}=\alpha\partial_{\mu}\mathcal{J}^{\mu}$. We
then define the \define{Noether Current}\index{Noether!Current},
\begin{equation}
j^{\mu}(x) := \frac{\partial\mathcal{L}}{\partial(\partial_{\mu}\varphi)}\Delta\varphi-\mathcal{J}^{\mu}(x),
\end{equation}
and our demands for symmetry invariance amounts to the conservation of
the Noether current:
\begin{equation}
\partial_{\mu}j^{\mu}(x) = 0.
\end{equation}
We see that this means
\begin{equation}
\partial_{\mu}j^{\mu}(x) = 0\iff\partial_{\mu}\left(\frac{\partial\mathcal{L}}{\partial(\partial_{\mu}\varphi)}\Delta\varphi\right)=\partial_{\mu}\mathcal{J}^{\mu},
\end{equation}
and therefore the two terms are interchangeable, allowing us to recover
our desired symmetry.

\begin{remark}[Multiple fields]
If the symmetry involves more than one field, then $j^{\mu}(x)$ is
really the sum of these sort of terms (one term for each field).
\end{remark}

\N{Noether Charge}
We can express the conservation law by sating that the quantity
\begin{equation}
Q := \int_{\RR^{3}}j^{0}\,\D^{3}x
\end{equation}
is a constant in time. This $Q$ is sometimes referred to as the
``Noether Charge''\index{Noether!Charge} in the literature.

\begin{example}[Canonical Stress--Energy]\label{ex:classical-field-theory:noether:canonical-stress-energy}
Consider the situation when the symmetry is an infinitesimal translation
in spacetime,
\begin{equation}
x^{\mu}\to x^{\mu} + a^{\mu},
\end{equation}
and the field transforms as\footnote{Remember, the scalar field
transforms under a symmetry $g$ like
$g\cdot\varphi(x)=\varphi(g^{-1}\cdot x)$.}:
\begin{equation}
\varphi(x)\to\varphi(x-a)=\varphi(x)-a^{\mu}\partial_{\mu}\varphi(x).
\end{equation}
The Lagrangian must transform like a scalar,
\begin{equation}
\mathcal{L}\to\mathcal{L}-a^{\mu}\partial_{\mu}\mathcal{L}
=\mathcal{L} - \partial_{\mu}(a^{\mu}\mathcal{L}).
\end{equation}
Then $\Delta\mathcal{L}(x)=- \partial_{\mu}(a^{\mu}\mathcal{L})$
allowing us to identify $\mathcal{J}^{\mu}(x)=-a^{\mu}\mathcal{L}(x)$.
The conserved Noether current is then
\begin{equation}
\begin{split}
j^{\mu}(x) &= \frac{\partial\mathcal{L}(x)}{\partial(\partial_{\mu}\varphi(x))}(-a^{\mu}\partial_{\mu}\varphi(x))
- (-a^{\mu}\mathcal{L}(x))\\
&=a_{\nu}\canonicalStressEnergy^{\mu\nu}.
\end{split}
\end{equation}
This $\canonicalStressEnergy^{\mu\nu}$ is the
\define{Canonical Stress-Energy Tensor}, not to be confused 
with the stress-energy tensor appearing in the Einstein field equations.
More generally, if we have a collection of fields $\varphi_{a}$ ($a=1,\dots,n$)
we have (implicitly summing over $a$),
\begin{subequations}
\begin{align}
\canonicalStressEnergy^{\mu\nu} &:= -\frac{\partial\mathcal{L}}{\partial(\partial_{\mu}\varphi_{a})}\partial^{\nu}\varphi_{a}+\eta^{\mu\nu}\mathcal{L},\\
\intertext{or, written as a rank-2 covariant tensor,}
\canonicalStressEnergy_{\mu\nu} &:= -\frac{\partial\mathcal{L}}{\partial(\partial^{\mu}\varphi_{a})}\partial_{\nu}\varphi_{a}+\eta_{\mu\nu}\mathcal{L}.
\end{align}
\end{subequations}
\end{example}

\begin{exercise}
Prove $\partial_{\mu}{\canonicalStressEnergy^{\mu}}_{\nu}=0$ for each $\nu$.
\end{exercise}

\begin{exercise}
Prove or find a counter-example: $\canonicalStressEnergy_{\mu\nu}=\eta_{\mu\rho}{\canonicalStressEnergy^{\rho}}_{\nu}$
is not symmetric. When will it be symmetric?
\end{exercise}

\begin{exercise}
What are the 4 Noether charges for the canonical stress-energy tensor?
How do we interpret them physically?
\end{exercise}

\begin{exercise}
For the Scalar Field's Lagrangian from Eq~\eqref{eq:classical-field-theory:scalar-field:lagrangian},
compute the canonical stress-energy tensor. What happens for arbitrary
potential terms $V(\varphi)$?
\end{exercise}

\begin{exercise}
The stress-energy tensor appearing in Einstein's field equation is
obtained by (\S21.3 in Misner, Thorne, Wheeler~\cite{Misner:1973prb}):
\begin{equation}
\mbox{(RHS)}_{\mu\nu} = -2\frac{\partial\mathcal{L}_{\text{matter}}}{\partial g^{\mu\nu}}
+ g_{\mu\nu}\mathcal{L}_{\text{matter}}.
\end{equation}
In flat spacetime (i.e., when $g_{\mu\nu}=\eta_{\mu\nu}$), when does
this differ from the canonical stress-energy tensor? When do they coincide?
\end{exercise}

\N{References}
Noether's original paper~\cite{Noether1918:iv} may be worth reading,
though the terminology of group theory may be archaic compared to
today's vocabulary.
The discussion of Noether's theorems is largely inspired from Chapter 3
section 5 of Kiefer~\cite{Kiefer:2007ria},
though we also rely on Peskin and Schroeder~\cite{Peskin:1995ev} for the
discussion of Noether's first theorem and the canonical stress-energy
tensor. See also Chapter 22 of Srednicki~\cite{Srednicki:2007qs}, but
care must be taken: there is a sign error in his derivation of the
canonical stress-energy tensor (which is corrected in our derivation).
Sundermeyer~\cite{Sundermeyer:1982gv} is quite explicit in the
adjustments necessary for Noether's theorem to work with fields.
\section{Electromagnetism}

\M
We have reviewed in section~\ref{section:relativity:electromagnetism}
the covariant formalism of electromagnetism. From the perspective of
classical field theory, we now know the ``correct'' way to think of
things is that the field quantity of interest is the 4-potential
$A^{\mu}$ (\S\ref{chunk:relativity:electromagnetism:four-potential}).

\N{Lagrangian Density}
We recover Maxwell's equations using the Lagrangian
\begin{equation}
\mathcal{L} = \frac{-1}{4}F^{\alpha\beta}F_{\alpha\beta} = \frac{-1}{4}\eta^{\alpha\mu}\eta^{\beta\nu}F_{\mu\nu}F_{\alpha\beta}.
\end{equation}
If we work in curved spacetime, we need to multiply by the determinant
of the metric tensor $\sqrt{-\det(g_{\mu\nu})}$, but we will ignore this
factor.

\N{Equations of Motion}
We can now determine the equations of motion for the Lagrangian density.
The Euler--Lagrange equations take the form
\begin{equation}
\partial_{\mu}\frac{\partial\mathcal{L}}{\partial(\partial_{\mu}A_{\nu})}-\frac{\partial\mathcal{L}}{\partial A_{\nu}}=0.
\end{equation}
These will turn out to be:
\begin{equation}
\boxed{\partial_{\mu}F^{\mu\nu} = 0.}
\end{equation}
This is the source-free Maxwell's equations.

\begin{proof}[Proof (slick)]
We can compute the variation of the action directly, ignoring boundary terms,
as
\begin{calculation}
\variation\mathcal{L}
\step{unfolding the definition of the Lagrangian density}
\variation\left(\frac{-1}{4}\eta^{\alpha\mu}\eta^{\beta\nu}F_{\mu\nu}F_{\alpha\beta}\right)
\step{product rule and index gymnastics}
\frac{-1}{4}\eta^{\alpha\mu}\eta^{\beta\nu}F_{\mu\nu}(2\,\variation F_{\alpha\beta})
\step{unfolding the definition of field-strength tensor}
\frac{-1}{2}\eta^{\alpha\mu}\eta^{\beta\nu}F_{\mu\nu}(\variation\partial_{\beta}A_{\alpha}-\variation\partial_{\beta}A_{\alpha})
\step{index gymnastics, antisymmetry of field-strength tensor}
-\eta^{\alpha\mu}\eta^{\beta\nu}F_{\mu\nu}\variation(\partial_{\alpha}A_{\beta})
\step{integration by parts}
\eta^{\alpha\mu}\eta^{\beta\nu}(\partial_{\alpha}F_{\mu\nu})\variation A_{\beta}
+\mbox{(boundary terms)}.
\end{calculation}
This vanishes when
\begin{equation}
\partial_{\mu}F^{\mu\nu} = 0,
\end{equation}
and this is the result from the Euler--Lagrange equations of motion.
\end{proof}

\M
We can unfold the result of the Euler--Lagrange equations for
electromagnetism, and find
\begin{equation}
\partial_{\mu}F^{\mu\nu}=0\iff g^{\alpha\gamma}\partial_{\gamma}F_{\alpha\beta}=0.
\end{equation}
Then
\begin{calculation}
  g^{\alpha\gamma}\partial_{\gamma}F_{\alpha\beta}
\step{unfold definition of field-strength tensor}
g^{\alpha\gamma}\partial_{\gamma}(\partial_{\alpha}A_{\beta}-\partial_{\beta}A_{\alpha})
\step{distributivity, index gymnastics}
\partial^{\alpha}\partial_{\alpha}A_{\beta}-\partial^{\alpha}\partial_{\beta}A_{\alpha}
\step{equations of motion}
0.
\end{calculation}
When we impose the gauge condition $\partial^{\alpha}A_{\alpha}=0$,
we recover the familiar Maxwell equations as a wave equation
$\Box A_{\beta}=0$.

\N{Lagrangian coupled to matter}
We can write down the Lagrangian density for electromagnetism coupled to
some charged matter, recovering the Maxwell equations with some source
as in
Eq~\eqref{eq:relativity:electromagnetism:maxwell-for-electric-field}:
\begin{equation}\label{eq:classical-field-theory:electromagnetism:lagrangian}
\mathcal{L} = \frac{-1}{4}F^{\alpha\beta}F_{\alpha\beta}-4\pi J^{\mu}A_{\mu}.
\end{equation}

\begin{exercise}
Recall (\S\ref{ex:classical-field-theory:noether:canonical-stress-energy})
the notion of the canonical stress--energy tensor. Calculate
$\canonicalStressEnergy_{\mu\nu}$ for the Lagrangian in Eq~\eqref{eq:classical-field-theory:electromagnetism:lagrangian}.

[Hint: your answer should \emph{not} be symmetric --- that is, 
$\canonicalStressEnergy_{\mu\nu}\neq\canonicalStressEnergy_{\nu\mu}$.]
\end{exercise}

\N{Heuristic regarding interactions}\index{Heuristics}
If we want to describe interactions between two fields, or a field and
some matter, then we need to add a term to our Lagrangian of the form:
\begin{equation}
\mathcal{L}_{\text{interaction}}\sim\begin{pmatrix}\mbox{coupling}\\
\mbox{constant}
\end{pmatrix}\begin{pmatrix}\mbox{field}\\
\mbox{quantity}
\end{pmatrix}\begin{pmatrix}\mbox{matter}\\
\mbox{terms}
\end{pmatrix}.
\end{equation}
This is added to the potential term in the Lagrangian.

\subsection{Hamiltonian Formalism}

\M We will work through the calculations of the Symplectic two-form as a
series of exercises, and show it is degenerate. This is a consequence of
gauge symmetries. Then we will work through the Hamiltonian formalism
with its phase space parametrized by initial conditions.

\begin{exercise}
Prove $\displaystyle\variation L = \int\bigl(\variation A_{\beta}\partial_{\alpha}F^{\alpha\beta}+\partial_{0}(-F^{0\beta}\,\variation A_{\beta})\bigr)\,\D^{3}x$.
\end{exercise}

\begin{exercise}
  Prove the Symplectic potential is
  \[ \Theta(\variation A) = \int F^{\beta0}\,\variation A_{\beta}\,\D^{3}x =\int F^{i0}\,\variation A_{i}\,\D^{3}x\]
\end{exercise}

\begin{exercise}
Suppose $A^{\alpha}$ is a solution to the Maxwell's equations. Determine
what the linearized Maxwell equations are for tangent ``vectors''
$\variation A^{\beta}$ with base ``point'' $A^{\alpha}$.
\end{exercise}

\begin{exercise}[Symplectic form]
We will compute the Symplectic two-form for electromagnetism, and verify
it is degenerate when one of the fields is pure gauge.
\begin{enumerate}
\item Show the [naive] Symplectic two-form for electromagnetism $\Omega=\D\Theta$
is
\[\Omega(\variation_{1}A,\variation_{2}A) = \int\bigl((\partial_{t}\variation_{1}A^{i}-\partial^{i}\variation_{1}A_{t})\variation_{2}A_{i}-(\partial_{t}\variation_{2}A^{i}-\partial^{i}\variation_{2}A_{t})\variation_{1}A_{i}\bigr)\,\D^{3}x.\]
\item Consider $\variation_{2}A_{t}=\partial_{t}\Lambda$ and
  $\variation_{2}A_{i}=\partial_{i}\Lambda$ where $\Lambda$ is any
  function with compact support. Show
\[\int(\partial_{t}\variation_{1}A^{i}-\partial^{i}\variation_{1}A_{t})\variation_{2}A_{i}\,\D^{3}x=-\int\Lambda\partial_{i}(\partial_{t}\variation_{1}A^{i}-\partial^{i}\variation_{1}A_{t})\,\D^{3}x.\]
\item Show (if you haven't already) the linearized Maxwell's equations
  includes $\partial^{i}(\partial_{t}\variation A_{i}-\partial_{i}\variation A_{t})=0$.
\item Comparing the last two steps in this exercise, show the integral
  from step 2 vanishes, and this implies $\Omega(\variation_{1}A,\variation_{2}A) =0$
 (i.e., $\Omega$ is degenerate).
\end{enumerate}
\end{exercise}

\N{Initial Data}
Assuming we have picked some time-slicing and we have some initial data
\begin{equation}
\phi:=-A_{t}(\vec{x},t=0),\quad\mbox{and}\quad Q_{i}:=A_{i}(\vec{x},t=0),
\end{equation}
we can find the canonically conjugate momentum to $Q_{i}$ as,
\begin{subequations}
\begin{align}
P^{i} &= \frac{\partial\mathcal{L}}{\partial(\partial_{t}A_{i})}\\
&=\partial_{t}A_{i}-\partial_{i}A_{t}\\
&=\partial_{t}Q_{i}+\partial_{i}\phi.
\end{align}
\end{subequations}

\begin{exercise}
Verify that $\displaystyle\frac{\partial\mathcal{L}}{\partial(\partial_{t}A_{t})}=0$,
and therefore the canonically conjugate momentum for $\phi$ vanishes.
\end{exercise}

\begin{exercise}
Rewrite the Lagrangian as a functional of $Q_{i}$, $P^{i}$, and
$\phi$. Try to write it in ``canonical form'', i.e., as
\[ L[\phi, Q_{i}, P^{i}] = \int \bigl(P^{i}\partial_{t}Q^{i}-\mbox{(something)}\bigr)\,\D^{3}x.\]
[Hint: integration by parts and the divergence theorem are your friends.]
\end{exercise}

\M
The Lagrangian which you ought to obtain from the previous exercises
should be:
\begin{equation}\label{eq:classical-field-theory:electromagnetism:canonical-formalism:lagrangian}
L =
\int\left(P^{i}\partial_{t}Q_{i}-\left[\frac{1}{2}(P_{i}P^{i}+\frac{F_{ij}F^{ij}}{2}) + \phi\,\partial_{i}P^{i}\right]\right)\D^{3}x.
\end{equation}
Observe then that the equations of motion for $\phi$ are precisely
Gauss's Law:
\begin{equation}
\partial_{i}P^{i} = 0.
\end{equation}
We interpret $\phi$ as a Lagrange multiplier.

\begin{exercise}
Using the Lagrangian you should have computed (or lifted from the
previous chunk), prove the Euler--Lagrange equations:
\begin{equation}
\frac{\variation L}{\variation P^{i}}-\frac{\D}{\D t}\frac{\variation L}{\variation\partial_{t}P^{i}}=0,\quad
\frac{\variation L}{\variation Q^{i}}-\frac{\D}{\D t}\frac{\variation L}{\variation\partial_{t}Q^{i}}=0,\quad
\frac{\variation L}{\variation\phi}=0,
\end{equation}
are equivalent to the Maxwell equations.
\end{exercise}

\N{Hamiltonian}
We find the Hamiltonian functional by inspection of
Eq~\eqref{eq:classical-field-theory:electromagnetism:canonical-formalism:lagrangian}
to be:
\begin{equation}
H = \int\left(\frac{1}{2}(P_{i}P^{i}+\frac{F_{ij}F^{ij}}{2}) + \phi\,\partial_{i}P^{i}\right)\D^{3}x.
\end{equation}
The first two terms coincide with our expectations, but the last term
may be surprising.

\begin{ddanger}
When we have a constrained Hamiltonian system, we add the first-class
constraints to the Hamiltonian. This is precisely what's going on with
the Hamiltonian functional as we've written it down. This is studied
thoroughly in Henneaux and Teitelboim~\cite{Henneaux:1992ig}.
\end{ddanger}

\begin{exercise}\index{Hamilton's equations}
The reader can verify Hamilton's equations,
\begin{subequations}
\begin{align}
\partial_{t}Q_{i} &= \frac{\delta H}{\delta P^{i}} = P_{i} - \partial_{i}\phi\\
\partial_{t}P_{i} &= -\frac{\delta H}{\delta P^{i}} = \partial_{i}F^{ij}.
\end{align}
\end{subequations}
\end{exercise}

\begin{exercise}\index{Poisson bracket}
Using the Poisson brackets,
\begin{equation}
\PB{M}{N} = \int\left(\frac{\delta M}{\delta Q_{i}(\vec{x}')}\frac{\delta N}{\delta P^{i}(\vec{x}')}
-\frac{\delta N}{\delta Q_{i}(\vec{x}')}\frac{\delta M}{\delta P^{i}(\vec{x}')}\right)\D^{3}x',
\end{equation}
show the quantity
\begin{equation}
G = - \int\Lambda(\vec{x})\partial_{i}P^{i}\,\D^{3}x
\end{equation}
is the generating function for gauge transformations
\begin{equation}
\variation Q_{i}=\PB{Q_{i}}{G}=\partial_{i}\Lambda,\quad
\variation P^{i}=\PB{P^{i}}{G}=0.
\end{equation}
\end{exercise}

\subsection{Scalar Electrodynamics}

\M
One of the first models we study in quantum field theory is something
called ``scalar electrodynamics''. This is obtained by taking a complex
Scalar field and coupling it to Electromagnetism. Let us review the
pertinent aspects of the complex Scalar field, then let us try to couple
it to electromagnetism.

\subsubsection{Complex Scalar Field}

\M We take 2 real-valued scalar fields
$\varphi_{1}$ and $\varphi_{2}$, then form the complex scalar
field\footnote{This is an abuse of notation, similar to using $z$ and
$\bar{z}$ in complex analysis as the independent coordinates of the
Complex plane.} 
\begin{equation}
\varphi(x) = \frac{\varphi_{1}(x)+\I\varphi_{2}(x)}{\sqrt{2}},\quad\mbox{and}\quad
\varphi^{*}(x) = \frac{\varphi_{1}(x)-\I\varphi_{2}(x)}{\sqrt{2}}.
\end{equation}
Then we couple the complex scalar field to electromagnetism.

\N{Complex Scalar Field}\index{Scalar Field!Complex}
The complex scalar field (also called the \emph{charged Klein--Gordon field})
may be viewed as a mapping
\begin{equation}
\varphi\colon\RR^{3,1}\to\CC.
\end{equation}
The Lagrangian for the complex Scalar field is:
\begin{equation}
\mathcal{L}_{cs} = -(\eta^{\alpha\beta}\partial_{\alpha}\varphi\partial_{\beta}\varphi^{*}+\mu^{2}|\varphi|^{2}).
\end{equation}

\begin{exercise}
Show the Euler--Lagrange equations give you the equations of motion
\begin{subequations}
\begin{align}
\frac{\variation S}{\variation\varphi}=0 &\implies (\partial^{\alpha}\partial_{\alpha}+\mu^{2})\varphi^{*}=0,
\intertext{and}  
\frac{\variation S}{\variation\varphi^{*}}=0 &\implies (\partial^{\alpha}\partial_{\alpha}+\mu^{2})\varphi=0.
\end{align}
\end{subequations}
\end{exercise}

\N{Symmetries of Complex Scalar Field}
We can use Noether's theorem to find that the complex scalar field
admits a continuous symmetry:
\begin{equation}
\varphi_{\lambda}=\E^{\I\lambda}\varphi,\quad
\varphi_{\lambda}^{*}=\E^{-\I\lambda}\varphi^{*},
\end{equation}
where $\lambda\in\RR$ is arbitrary. Since $\lambda$ is a constant, we
see the kinetic term of the Lagrangian is invariant under this
transformation. Similarly, we see $|\varphi|^{2}=\varphi^{*}\varphi$ is
invariant under this transformation.

\begin{remark}\index{Symmetry!Global}\index{Symmetry!Local}
This is a ``global $\U(1)$'' symmetry. It's ``global'' because the
parameter $\lambda$ is a real number independent of spacetime. (If
$\lambda$ were a function of spacetime $\lambda=\lambda(\vec{x},t)$,
then we would call it a ``local'' symmetry.) It's a $\U(1)$ symmetry
because $\E^{\I\lambda}\in\U(1)$.

Older literature use the term ``gauge transformation of the first kind''\index{Gauge!Transformation!Of the first kind}
instead of ``global symmetry transformation''
\end{remark}

\begin{exercise}
Use Noether's theorem to prove this is a continuous symmetry of the
complex scalar field. Then determine the conserved current $j^{\beta}$
for this symmetry.
\end{exercise}
\begin{exercise}
Prove the Noether charge for the complex Scalar field is
\begin{equation}
Q = \I\int_{V}(\varphi^{*}\partial_{t}\varphi-\varphi\partial_{t}\varphi^{*})\,\D^{3}x.
\end{equation}
\end{exercise}

\N{Sigma Models}\index{Sigma model@$\sigma$ Model}
There is a way to generalize this construction from 2 real Scalar fields
to $N$ real Scalar fields. This is a family of models called
\define{Sigma models} where the scalar fields are components of a smooth
function $\sigma\colon\RR^{3,1}\to\mathcal{M}$ where $\mathcal{M}$ is
usually a Lie group. (There is no significance to the choice of $\sigma$
for scalar fields, and Sigma models refer to this historic artifact of
arbitrary notation.) Then the Lagrangian density for the massless case
is:
\begin{equation}
\mathcal{L} = \frac{1}{2}\sum^{N}_{A,B=1}g_{AB}(\sigma)\partial^{\mu}\sigma^{A}\partial_{\mu}\sigma^{B},
\end{equation}
where $G_{AB}(\sigma)$ is the metric tensor on the field space $\mathcal{M}$,
and $\partial_{\mu}$ are the derivatives on the underlying spacetime
manifold $\RR^{3,1}$. When we include some self-interaction terms, we
obtain a \emph{Nonlinear $\sigma$ Model}.

When $\mathcal{M}=\CC^{2}$, for example, we can show the $\sigma$ model
enjoys an $\SU(2)$ symmetry. Similarly, for $\mathcal{M}=\CC^{n}$, the
$\sigma$ model enjoys an $\SU(n)$ symmetry. These models are useful as
``prolegomenon'' to Yang--Mills theory for the Standard Model.

\begin{remark}
Sigma models were first introduced in \S\S5--6 of Gell-Mann and Levy~\cite{Gell-Mann:1960mvl}.
Initially, $\sigma$ was ``just another scalar field'' in that paper. Later
physicists adopted $\sigma$ as we have introduced it: as a familar of
scalar fields.
\end{remark}

\subsubsection{Charged Scalar Field coupled to Electromagnetism}

\M
Now we can couple the complex Scalar field to Electromagnetism.
The basic idea is we will form the Lagrangian density for scalar
electrodynamics by adding the Lagrangian density for the complex Scalar
field to the Lagrangian density for the Electromagnetic field, plus the
4-current coupling the charged Scalar field to the Electromagnetic field:
\begin{equation}
\begin{split}
\mathcal{L}_{sED}&=\mathcal{L}_{cs}+\mathcal{L}_{EM}+\mathcal{L}_{int}\\
&=-(\eta^{\alpha\beta}\partial_{\alpha}\varphi\partial_{\beta}\varphi^{*}+\mu^{2}|\varphi|^{2})
-\frac{1}{4}F^{\alpha\beta}F_{\alpha\beta}
+4\pi j^{\alpha}A_{\alpha}.
\end{split}
\end{equation}
We just need to determine $j^{\alpha}$ in terms of the complex Scalar
field $\varphi$.

We know from Noether's theorem there is a conserved current for the
$\U(1)$ Symmetry for the complex Scalar field,
\begin{equation}
j^{\alpha} = -\I\eta^{\alpha\beta}(\varphi^{*}\partial_{\beta}\varphi - \varphi\partial_{\beta}\varphi^{*}).
\end{equation}
There is some slight difficulties with the Lagrangian as we have written
it: it is no longer gauge invariant under $A^{\mu}\to
A^{\mu}+\partial^{\mu}\Lambda$.

\N{Minimal coupling}
The ``physically correct way'' to get a gauge-invariant Lagrangian which
still gives the $j^{\alpha}A_{\alpha}$ coupling is rather unintuitive:
we use a different differential operator than $\partial_{\mu}$ in the
charged Scalar field's Lagrangian density.

This is the so-called \define{Minimal Coupling}, where we replace
\begin{subequations}
\begin{equation}
\partial_{\alpha}\varphi\to D_{\alpha}\varphi := (\partial_{\alpha}+\I qA_{\alpha})\varphi,
\end{equation}
and
\begin{equation}
\partial_{\alpha}\varphi^{*}\to D_{\alpha}\varphi^{*} := (\partial_{\alpha}-\I qA_{\alpha})\varphi^{*}.
\end{equation}
\end{subequations}
Here $q$ is a parameter reflecting the coupling strength between the
charged scalar field $\varphi$ and the Electromagnetic field. It's an
example of a \emph{coupling constant}.\index{Coupling constant}
Then we modify the complex Scalar field's Lagrangian density to use
these gauge covariant derivatives,
\begin{equation}\label{eq:classical-field-theory:electromagnetism:lagrangian-density-for-complex-scalar-field-using-gauge-covariant-derivatives}
\mathcal{L}_{cs} = -\eta^{\alpha\beta}D_{\alpha}\varphi^{*}D_{\beta}\varphi-\mu^{2}|\varphi|^{2}.
\end{equation}
This Lagrangian density yields field equations which are the usual wave
equations plus some modifications involving the electromagnetic
potential.

\begin{exercise}
Compute the Euler--Lagrange equations for $\varphi$ and $\varphi^{*}$ using the Lagrangian density from Eq~\eqref{eq:classical-field-theory:electromagnetism:lagrangian-density-for-complex-scalar-field-using-gauge-covariant-derivatives}.
\end{exercise}

\M
Now observe, under a gauge transformation of the electromagnetic
4-potential
\begin{subequations}
\begin{equation}
A_{\alpha}\to A_{\alpha}+\partial_{\alpha}\Lambda,
\end{equation}
for the gauge covariant derivatives of the complex Scalar field to
remain invariant under these gauge transformations, we need:
\begin{align}
  \varphi&\to\E^{-\I q\Lambda}\varphi,\\
  \intertext{and}
  \varphi^{*}&\to\E^{\I q\Lambda}\varphi^{*}.
\end{align}
\end{subequations}
The reader can verify that the gauge covariant derivatives of the
complex Scalar field then transform as
\begin{subequations}
\begin{align}
D_{\alpha}\varphi &\to\E^{-\I q\Lambda}D_{\alpha}\varphi,\\
D_{\alpha}\varphi^{*} &\to\E^{\I q\Lambda}D_{\alpha}\varphi^{*}.
\end{align}
\end{subequations}
We can see that the kinetic term for the modified complex Scalar
Lagrangian density remains invariant under these transformations.

\M
Since the electromagnetic interaction with the complex Scalar fields are
``swept into'' the gauge covariant derivatives, we can write the
Lagrangian density for the scalar electrodynamic theory as:
\begin{subequations}
\begin{equation}
\mathcal{L}_{sED} = \frac{-1}{4}F^{\alpha\beta}F_{\alpha\beta} - \eta^{\alpha\beta}D_{\alpha}\varphi^{*}D_{\beta}\varphi-\mu^{2}|\varphi|^{2}.
\end{equation}
When we expand the gauge covariant derivatives in this Lagrangian
density, we have:
\begin{equation}
\mathcal{L}_{sED} = \frac{-1}{4}F^{\alpha\beta}F_{\alpha\beta} - \eta^{\alpha\beta}\partial_{\alpha}\varphi^{*}\partial_{\beta}\varphi-\mu^{2}|\varphi|^{2}
+\I q A^{\alpha}(\varphi^{*}\partial_{\alpha}\varphi-\varphi\partial_{\alpha}\varphi^{*}+\I q A_{\alpha}|\varphi|^{2}).
\end{equation}
\end{subequations}
The Euler--Lagrange equations for the Electromagnetic 4-potential are
\begin{equation}
\partial_{\beta}F^{\alpha\beta}=-4\pi J^{\alpha},
\end{equation}
where the current is defined using the gauge covariant derivatives as
\begin{equation}\label{eq:classical-field-theory:sed:charged-current}
J^{\alpha} = -\frac{\I q}{4\pi}(\varphi^{*}D^{\alpha}\varphi-\varphi D^{\alpha}\varphi^{*}).
\end{equation}
This is rather magical, but we could derive the same results using
Noether's first theorem for fields.

\M
We should mention that physicists look at
Eq~\eqref{eq:classical-field-theory:sed:charged-current} and interpret
it as telling us the electromagnetic charge for the complex scalar field
cannot ``exist alone'' in the Scalar field. In an interacting system,
the division between ``source fields'' and ``fields mediating interactions''
is rather artificial and arbitrary. This is physically reasonable (even
if a little surprising). Mathematically this feature emerges from
demanding gauge invariance.

The complex Scalar field is no longer uniquely defined in scalar
electrodynamics: it is subject to a gauge transformation, just like the
electromagnetic 4-potential.

When we have such an interaction, if we want to compute (say) the
electromagnetic field contained in a region $V$, we need a solution
$(A,\varphi)$ of the coupled Maxwell--Scalar equations, then substitute
it into:
\begin{equation}
Q = \frac{1}{4\pi}\int_{V}\I q(\varphi^{*}D_{0}\varphi-\varphi D_{0}\varphi^{*})\,\D^{3}x.
\end{equation}
This charge is conserved and gauge invariant.

\begin{exercise}
Prove the total electric charge $Q$ is conserved \emph{and} gauge invariant.
\end{exercise}

\begin{exercise}
Suppose we had a Lagrangian for complex scalar fields coupled to
electromagnetism of the form
\begin{equation}
\mathcal{L}=\mathcal{L}_{EM}-\frac{1}{2}[\varphi g^{\mu}D_{\mu}\varphi^{*}-\varphi^{*}g^{\mu}D_{\mu}\varphi]-\mu^{2}|\varphi|^{2},
\end{equation}
where $g^{\mu}$ is ``some [constant] vector'', and $D_{\mu}$ is the
gauge covariant derivative.
\begin{enumerate}
\item How must $\varphi$ and $\varphi^{*}$ transform under gauge
  transformations $A_{\alpha}\to A_{\alpha}+\partial_{\alpha}\Lambda$?
\item How do the gauge covariant derivatives $D_{\alpha}\varphi$ and
  $D_{\alpha}\varphi^{*}$ transform under gauge transformations?
\end{enumerate}
\end{exercise}

\subsection{Chern--Simons Theory}

\M
In $2+1$ dimensions (instead of $3+1$ dimensions), we have a particular
action which plays an important role in physics called the Chern--Simons
action named after its discoverer Albert Schwarz:
\begin{equation}
\begin{split}
  \action_{CS}[A] &= \frac{k}{4\pi}\int\epsilon^{\mu\nu\rho}A_{\mu}\partial_{\nu}A_{\rho}\,\D^{3}x\\
&\mbox{``=''}\; \frac{k}{4\pi}\int(A\times\nabla)^{\rho} A_{\rho}\,\D^{3}x
\end{split}
\end{equation}
where $A_{\mu}$ is a ``4''-potential for electromagnetism. Later we will
generalize Electromagnetism to Yang--Mills theory, and the Chern--Simons
theory will play an important role in something called the Quantum Hall
effect. It also describes quantum gravity in $2+1$ dimensions.

For a Yang--Mills field, however, we also have a term that looks like
$A^{3}$ in the action. Such a term vanishes for commutative gauge groups
like $\U(1)$ (i.e., like for Electromagnetism).

\begin{exercise}
From demanding stationary action $\variation\action_{CS}[A]/\variation A^{\mu}(x)=0$,
determine the equations of motion for Chern--Simons theory for the
electromagnetic field.
\end{exercise}
\section{*Path Independence}

\M Teitelboim showed in his PhD thesis and several follow-up articles~\cite{Hojman:1976vp,Teitelboim:1980hs} how
we can use the canonical formalism coupled to the principle of ``path
independence'' to derive General Relativity, Yang--Mills gauge theory,
among other things. We also saw in passing
(\S\ref{chunk:rqm:poincare-algebra:elementary-particles-irreps}) the
only fields possible are Scalar fields, Vector fields, and (rank-2
symmetric) Tensor fields.

\N{Foliating Spacetime}
We foliate spacetime by space-like hypersurfaces $\Sigma_{t}$ indexed by
time $t$. We have the metric $h_{ij}$ on $\Sigma_{t}$ where $i$,
$j=1,2,3$ are spatial indices.

We stipulate that $h_{ij}$ has its canonically conjugate momenta density
$p^{ij}$. 

\N{Evolution}
If we want to consider the time evolution of a function $F(h_{ij}(x), p^{k\ell})$
from one spatial hypersurface $\Sigma_{t}$ to another $\Sigma_{t+\delta t}$
be described using generators $\mathcal{H}_{i}(x)$ and $\mathcal{H}_{\bot}(x)$
and the Poisson bracket
\begin{equation}
\begin{split}
\dot{F}(h_{ij}(x),p^{k\ell}(x))
&=\int\bigl(\PB{F}{\mathcal{H}_{\bot}(x')}N(x') + \PB{F}{\mathcal{H}_{i}(x')}N^{i}(x')\bigr)\,\D^{3}x'\\
&=\int\PB{F}{\mathcal{H}_{\mu}(x')}N^{\mu}(x')\,\D^{3}x'.
\end{split}
\end{equation}
If we want to consider spatial diffeomorphisms $F\to F + \variation F$
along the same hypersurface $\Sigma_{t}$, then we use:
\begin{equation}
\variation F = -\int\PB{F}{\mathcal{H}_{i}(x')}\,\variation N^{i}(x')\,\D^{3}x'.
\end{equation}


\N{Fundamental Poisson Brackets}
We generate the transformations desired by the Poisson brackets:
\begin{subequations}
\begin{align}
\PB{\mathcal{H}_{\bot}(x)}{\mathcal{H}_{\bot}(x')} &= [h^{ij}(x)\mathcal{H}_{j}(x) + h^{ij}(x')\mathcal{H}_{j}(x')]\partial_{i}\delta(x,x')\\
\PB{\mathcal{H}_{i}(x)}{\mathcal{H}_{\bot}(x')} &= \mathcal{H}_{\bot}(x)\partial_{i}\delta(x,x')\\
\PB{\mathcal{H}_{i}(x)}{\mathcal{H}_{j}(x')} &= \mathcal{H}_{i}(x')\partial_{j}\delta(x,x')+\mathcal{H}_{j}(x)\partial_{i}\delta(x,x').
\end{align}
\end{subequations}
The ``only'' generator of ``real'' dynamical interest is $\mathcal{H}_{\bot}$.
The three $\mathcal{H}_{i}$ generate spatial diffeomorphisms, i.e.,
displacements that lie on the same spatial hypersurface which amount to
a change of spatial coordinates.

\N{Gravitating and Matter Generators}
We separate the generators $\mathcal{H}_{\bot}$ and $\mathcal{H}_{i}$
into a sum of the ``gravitational part'' and the ``matter part''. We
would have
\begin{equation}
\mathcal{H}_{\bot} = \mathcal{H}_{\bot}^{\text{grav}}[h_{ij},p^{k\ell}]
+ \mathcal{H}_{\bot}^{\text{mat}}[h_{ij},p^{k\ell},\mbox{matter canonical variables}].
\end{equation}
If $\mathcal{H}_{\bot}^{\text{mat}}$ did not depend on gravitational
variables (the spatial metric and its conjugate momenta), then the
Poisson bracket
$\PB{\mathcal{H}_{\bot}^{\text{mat}}}{\mathcal{H}_{\bot}^{\text{mat}}}$
cannot satisfy the desired relation (it'd be independent of $h_{ij}$,
but it must involve $h^{ij}$).

\M
The general form of $\mathcal{H}_{i}$ meanwhile may be constrained by
the following desired requirements:
\begin{enumerate}
\item It must be linear in the momenta in order to generate
  transformations of the fields under coordinate transformations (and
  not mix fields and momenta);
\item It should contain the momenta only up to the first spatial
  derivatives because it should generate first-order derivatives in the
  fields (think ``Taylor expansion to first order'').
\end{enumerate}
Therefore we expect
\begin{equation}
\mathcal{H}_{i}={b_{i}}^{jB}(\phi^{C})\partial_{b}p^{B} + {a_{a}}^{B}(\phi^{C})p_{B},
\end{equation}
where $\phi^{A}$ is now a symbolic notation for \emph{all} fields
\emph{including gravity}, and $p_{A}$ denotes the corresponding momenta.

\N{Heuristic for $\mathcal{H}_{\bot}$}
We stipulate that $\mathcal{H}_{\bot}$ is a quadratic polynomial in the
momenta,
\begin{equation}
\mathcal{H}_{\bot}\sim M_{AB}(\phi^{C})p^{A}p^{B} + N_{A}(\phi^{C})p^{A} + V(\phi^{C}),
\end{equation}
then we just need to determine the coefficients $M_{AB}$, $N_{A}$, $V$.
The calculations for the various fields appears in Section 6 of
Teitelboim~\cite{Teitelboim:1980hs}. 

\N{Spatial Diffeomorphism Requirements}
We demand the existence of \emph{ultralocal solutions}, in the sense
that a deformation localized at a point $\vec{x}_{0}$ must change the
field only at $\vec{x}_{0}$, then we find
\begin{equation}
b^{ijB}=b^{jiB}.
\end{equation}
So under an infinitesimal coordinate transformation
\begin{equation}
x'^{i} = x^{i} - \variation N^{i}(x),
\end{equation}
we will demand a field quantity $\phi$ transforms as
\begin{equation}
\LieD_{\variation\vec{N}}\phi =
-\int\PB{\phi}{\mathcal{H}_{i}(x')}\,\variation N^{i}(x')\,\D^{3}x',
\end{equation}
where $\LieD$ denotes the Lie derivative.\index{Lie derivative}
This will then give us a way to find $\mathcal{H}_{i}(x')$.

\N{Spatial Diffeomorphism Invariance of Scalar Fields}
For Scalar fields $\phi(x)$, it must transform as
\begin{equation}
\variation\phi(x):=\phi'(x)-\phi(x)\weakEq\partial_{i}\phi\,\variation N^{i}=\LieD_{\variation\vec{N}}\phi.
\end{equation}
This is generated by
\begin{equation}\label{eq:classical-field-theory:henneaux:scalar-field-momenta-constraints}
\mathcal{H}_{i} = p_{\phi}\partial_{i}\phi,
\end{equation}
where $p_{\phi}$ is the conjugate momenta density for $\phi$.

\begin{exercise}
Show that Eq~\eqref{eq:classical-field-theory:henneaux:scalar-field-momenta-constraints}
implies $b^{ij}=0=b^{ji}$. Therefore it satisfies ultralocality.
\end{exercise}

\begin{exercise}
Verify Eq~\eqref{eq:classical-field-theory:henneaux:scalar-field-momenta-constraints}
generates the correct transformations for $\phi$.
\end{exercise}

\N{For Vector Fields}
The vector field $A_{i}(x)$ transforms as
\begin{equation}
\variation A_{i} = (\partial_{j}A_{i})\,\variation
N^{j}+A_{j}\partial_{i}(\variation N^{j}) = (\LieD_{\variation\vec{N}}A)_{i},
\end{equation}
which is generated by
\begin{equation}\label{eq:classical-field-theory:henneaux:vector-field-momenta-constraints}
\mathcal{H}_{i} = -A_{i}\partial_{j}p^{j} + (\partial_{i}A_{j}-\partial_{j}A_{i})p^{j}.
\end{equation}

\begin{exercise}
Verify Eq~\eqref{eq:classical-field-theory:henneaux:vector-field-momenta-constraints}
generates $\variation A_{i}$.
\end{exercise}

\begin{exercise}[DO THIS]
Show that Eq~\eqref{eq:classical-field-theory:henneaux:vector-field-momenta-constraints}
implies ${{b_{i}}^{j}}_{C}=-A_{i}{\delta^{j}}_{C}$ and therefore the
condition for ultralocality is not fulfilled for vector fields.
\end{exercise}

\N{Covariant Rank-2 Tensor}
For a covariant rank-2 tensor (not necessarily symmetric) $t_{ij}$, we
have
\begin{equation}
\variation t_{ij} = (\partial_{k}t_{ij})\variation N^{k}
+ t_{ik}\partial_{j}(\variation N^{k})
+ t_{jk}\partial_{i}(\variation N^{k})
= (\LieD_{\variation\vec{N}}t)_{ij}.
\end{equation}
This is generated by
\begin{equation}
\mathcal{H}_{i} = p^{k\ell}\partial_{i}t_{k\ell}-\partial_{k}(t_{ij}p^{kj})-\partial_{j}(t_{ki}p^{kj}).
\end{equation}
In order for ultralocality to hold, we must have
\begin{equation}
t_{ij} = f(x)h_{ij}
\end{equation}
where $f(x)$ is an arbitrary function. That is to say, the tensor field
must be proportional to the metric tensor. We can interpret this as
another piece of evidence suggesting the ``only'' spin-2 field allowed
is the gravitational field.\index{Spin!Uniqueness of (---)-2}

\begin{exercise}[Fermions]\index{Lie derivative!Of spinor field}
  Can we do a similar analysis for a Dirac spinor field $\psi$ in curved
  spacetime? (I honestly do not know!)
  
  There are various generalizations of the Lie derivative to the
  spin-$1/2$ field. Godina and Matteucci~\cite{Godina:2005mt} reviews the various
  generalizations of the Lie derivative for spinors.  Specifically:
\begin{enumerate}
\item Find the ``right'' generalization of the Lie derivative to spinor
  fields. There are many possibilities! For one single example,
  Choquet-Bruhat and DeWitt-Morette~\cite{Choquet-Bruhat:2000amp2} propose
  \[\LieD_{X}\psi = X^{j}\partial_{i}\psi - \frac{1}{8}(\partial_{i}X_{j}-\partial_{j}X_{i})\gamma^{i}\gamma^{j}\psi = \variation\psi,\]
  where $\psi$ is a 4-component Dirac spinor field,
  $\gamma^{i}$, $\gamma^{j}$ are the Dirac matrices; is this the
  ``right'' generalization for our purposes?
\item Since we're dealing with
  fermionic fields, how must we modify the spatial diffeomorphism
  condition
  $\variation\psi\sim-\int\PB{\psi}{\mathcal{H}_{i}(y)}\,\variation N^{i}(y)\,\D^{3}y$?
  Does it suffice to use the Poisson \emph{super}-bracket?
\item What $\mathcal{H}_{i}$ generates this? 
\item Is this $\mathcal{H}_{i}$ ultralocal?
\end{enumerate}
\end{exercise}

\N{Gauge Theory: Recovering Ultralocality for Vector Fields}
We see that $\mathcal{H}_{i}$ is not ultralocal. It's caused by the
presence of the $A_{i}\partial^{j}p_{j}$ term. So it's tempting to make
the replacement
\begin{equation}
\mathcal{H}_{i}\to\widetilde{\mathcal{H}}_{i}
:=\mathcal{H}_{i} + A_{i}\partial_{j}p^{j}
=(\partial_{i}A_{j}-\partial_{j}A_{i})p^{j}.
\end{equation}
We see then that the Poisson bracket of this generator with itself is:
\begin{equation}\label{eq:classical-field-theory:henneaux-pi:modified-vector-PB}
\PB{\widetilde{\mathcal{H}}_{i}(x)}{\widetilde{\mathcal{H}}_{j}(y)}
= \widetilde{\mathcal{H}}_{j}(x)\partial_{i}\delta(x,y)
+ \widetilde{\mathcal{H}}_{i}(y)\partial_{j}\delta(x,y)
- F_{ij}(x)\partial_{k}p^{k}(x)\delta(x,y),
\end{equation}
where $F_{ij}=\partial_{i}A_{j}-\partial_{j}A_{i}$. This new term
appearing in the right-hand side of
Eq~\eqref{eq:classical-field-theory:henneaux-pi:modified-vector-PB} is
harmless \emph{only} if it generates physically irrelevant
transformations (``gauge transformations''). There are two ways this
could happen: either $F_{ij}$ is constrained to vanish (which is too
strong, this would imply $A_{i}=\partial_{i}\varphi$), or we demand
$\partial_{i}p^{i}\weakEq0$ is a constraint. We therefore introduce the
constraint
\begin{equation}
\mathcal{G}(x) := \frac{-1}{e}\partial_{i}p^{i}(x) =
\frac{-1}{e}\partial_{i}E^{i}(x) = \frac{-1}{e}\nabla\cdot\vec{E}(x),
\end{equation}
where $e$ is the electric charge, and the momentum is just $\vec{E}$ the
electric field. This constraint $\mathcal{G}\weakEq0$ is precisely
\emph{Gauss's Law}.

\begin{exercise}
Prove $\widetilde{H}_{i}$ is gauge invariant. Hint: it suffices to prove
the electric field is gauge invariant, and $F_{ij}$ is gauge invariant.
\end{exercise}

\begin{exercise}
Show Gauss's Law generates the gauge transformations
\begin{subequations}
\begin{align}
\variation A_{i}(x) &= \int\PB{A_{i}(x)}{\mathcal{G}(y)}\,\xi(y)\,\D^{3}y=\frac{1}{e}\partial_{i}\xi(x)\\
\variation p^{i}(x) &= \int\PB{p^{i}(x)}{\mathcal{G}(y)}\,\xi(y)\,\D^{3}y=0.
\end{align}
\end{subequations}
\end{exercise}

\M
Therefore we conclude that $A_{i}(x)$ transforms under
$\widetilde{H}_{j}$ as a vector modulo a gauge transfromation.

\N{Generalization to Yang--Mills}
When we replace $A_{i}(x)$ with several fields $A_{i}^{a}(x)$ for $a=1$,
\dots, $N$. We assume that:
\begin{enumerate} 
\item $A_{i}^{a}(x)$ does not mix with its momenta $p^{i}_{a}(x)$ under
  a gauge transformation,
\item the momenta should transform homogenously, and
\item the gauge constraint (the non-Abelian version of Gauss's Law) is local.
\end{enumerate}
This forces us to,
\begin{equation}
\mathcal{G}_{i} = \frac{-1}{f}\partial_{i}{p_{a}}^{i} + {C_{ab}}^{c}A^{a}_{i}p^{i}_{c}\weakEq0,
\end{equation}
where $f$ and ${C_{ab}}^{c}$ are constants. If we demand the commutator
of two gauge transformations yields another gauge transformation, then
we find ${C_{ab}}^{c}$ are the structure constants of a Lie algebra. We
have:
\begin{equation}
\PB{\mathcal{G}_{a}(x)}{\mathcal{G}_{b}(y)}={C_{ab}}^{c}\mathcal{G}_{c}(x)\delta(x,y),
\end{equation}
which precisely characterizes a Yang--Mills theory.

\M
If we split up $\mathcal{H}_{\bot}$ as a sum
\begin{equation}
\mathcal{H}_{\bot}=\mathcal{H}_{\bot}^{\text{grav}}+\mathcal{H}_{\bot}^{\text{YM}},
\end{equation}
and demanding the Yang--Mills part is independent of the gravitational
momenta (so $\mathcal{H}_{\bot}^{\text{YM}}$ depends only ultralocally
on the metric), then we find
\begin{equation}
\mathcal{H}_{\bot}^{\text{YM}} = \frac{1}{2\sqrt{h}}(h_{ij}\gamma^{ab}p^{i}_{a}p^{j}_{b}+h^{ij}\gamma_{ab}B^{a}_{i}B^{b}_{j}),
\end{equation}
where $\gamma_{ab}={C_{ac}}^{d}{C_{bd}}^{c}$ is the ``group metric''
($\gamma^{ab}$ is its inverse), and
$B^{a}_{i}=\frac{1}{2}\epsilon_{ijk}F^{ajk}$ are the non-Abelian
``magnetic fields''. The non-Abelian field strength is given by:
\begin{equation}
F^{a}_{ij}=\partial_{i}A^{a}_{j} - \partial_{j}A^{a}_{i}
+f{C^{a}}_{bc}A^{b}_{i}A^{c}_{j}.
\end{equation}
The Hamiltonian corresponds to the action,
\begin{equation}
\action_{\text{YM}} = \frac{-1}{4}\int\gamma_{ab}F^{a}_{\mu\nu}F^{b\mu\nu}\sqrt{-g}\,\D^{4}x
\end{equation}
where $\sqrt{-g}$ is the determinant of the metric of spacetime.

\M
So the principle of path independent taken together with the demand that
$\mathcal{H}_{\bot}$ be ultralocal in the momenta necessarily leads to
the concept of gauge theories.

