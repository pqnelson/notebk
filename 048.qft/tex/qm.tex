\chapter{Quantum Mechanics}

\section{Basic Rules}\label{sec:qm:basic-rules}

\M We will review the basic rules for describing a quantum system using
a wave function. We will revise the rules later, for greater generality,
but we begin with relevant versions for physical systems.

\N{Position--Momentum Commutator Relation}
We have
\begin{equation}\label{eq:position-momentum-uncertainty-commutator}
[\widehat{x},\widehat{p}] = \I\hbar.
\end{equation}


\subsection{Wave Functions}

\M Following Isham~\cite{Isham:1995lq}, we summarize the rules for
interpreting wave functions.

\N{First rule}
The quantum state of a point particle moving in one-dimension is
represented by a complex-valued wave function $\psi(x)$ which is
normalized to one:
\begin{equation}
\int^{\infty}_{-\infty}|\psi(x)|^{2}\,\D x = 1.
\end{equation}
We will say that ``$\psi$ is normalized to unity'' when this condition holds.

\N{Superposition principle}
We can superimpose any two wave functions $\psi_{1}(x)$ and
$\psi_{2}(x)$ with arbitrary complex coordinates $\alpha_{1}$, $\alpha_{2}$.
That is to say, we get a new wave function
$\alpha_{1}\psi_{1}(x) + \alpha_{2}\psi_{2}(x)$, provided that
$\alpha_{1}$ and $\alpha_{2}$ are chosen to make the new wave function
normalized to unity.

\N{Second rule}
Any physical quantity which we can measure (i.e., any observable) is
represented by a linear differential operator that acts on wave functions
and is self-adjoint.

\begin{remark}[Eigenvalues of self-adjoint operators]
``Self-adjoint'' operators have real eigenvalues. We measure
eigenvalues. If we are measuring a complex-valued quantity, then we
need an anti-self-adjoint operator.
\end{remark}

\begin{remark}[Completeness of eigenfunctions for self-adjoint operators]
The eigenfunctions for a self-adjoint operator are ``complete'' in the
sense that they form a basis. Further, normalized eigenfunctions satisfy
the orthogonality condition
\begin{equation}
\int^{\infty}_{-\infty}f_{m}^{*}(x)f_{n}(x)\,\D x = \delta_{m,n}
\end{equation}
where $\delta_{m,n}=1$ if $m=n$ and $\delta_{m,n}=0$ otherwise.
\end{remark}

\N{Third rule}
The only possible result of measuring an observable $A$ is one of the
eigenvalues of the self-adjoint operator $\widehat{A}$ which represents
it.

\M
For nondegenerate operators (i.e., any two eigenfunctions of
$\widehat{A}$ with the same eigenvalue are proportional to each other)
and with a discrete set of eigenvalues $a_{1}$, $a_{2}$, \dots
corresponding to eigenfunctions $f_{1}$, $f_{2}$, \dots; if the state is
$\psi(x)$, then the probability that a measurement of $A$ will yield a
particular eigenvalue $a_{n}$ is
\begin{equation}
\Pr(A = a_{n}; \psi) = |\psi_{n}|^{2},
\end{equation}
assuming the wave function is written as a linear combination of the
eigenfunctions,
\begin{equation}
\psi(x) = \sum_{n}\psi_{n}f_{n}(x)
\end{equation}
and the $\psi_{n}\in\CC$ are constants. We also assume $\psi(x)$ is
normalized to unity.

\begin{remark}
As a consequence of this, a wave function being normalized to unity is
the same as the ``sum of all probabilities equal 1''.
\end{remark}

\N{Fourth Rule}
The state function evolved in time according to the time-dependent
Schr\"{o}dinger equation
\begin{equation}
\I\hbar\frac{\partial\psi(x,t)}{\partial t} = \widehat{H}\psi(x,t),
\end{equation}
where the Hamiltonian operator $\widehat{H}$ is obtained from the
classical enerygy expression
\begin{equation}
H = \frac{p^{2}}{2m} + V(x),
\end{equation}
replacing the momentum $p$ and position $x$ by their corresponding
operators $\widehat{p}$, $\widehat{x}$.

\section{Quantum Harmonic Oscillator}

\N{Problem} Given the simple Harmonic oscillator's Hamiltonian
\begin{equation}\label{eq:quantum-harmonic-oscillator:defn:hamiltonian}
\widehat{H} = \frac{1}{2m}\left(\widehat{p}^{2} + (m\omega \widehat{x})^{2}\right),
\end{equation}
determine the eigenstates for the Schrodinger equation.

\M The trick is to use the identity
\begin{equation}
a^{2} + b^{2} = (a + \I b)(a -\I b)
\end{equation}
to introduce the operators:
\begin{equation}\label{eq:quantum-harmonic-oscillator:defn:ladder-operators}
a_{\pm} := \frac{1}{\sqrt{2\hbar m\omega}}\left(m\omega \widehat{x} \mp\I\widehat{p}\right).
\end{equation}
We will use this to rewrite the Hamiltonian operator.

\begin{lemma}\label{lemma:quantum-harmonic-oscillator:ladder-operator-hamiltonian-relation}
We find $a_{+}a_{-} = (\widehat{H}/\hbar\omega) - 1/2$.
\end{lemma}
\begin{proof}[Proof (lengthy calculation)]
  By direct computation
  \begin{calculation}
    a_{+}a_{-}
\step{by Eq~\eqref{eq:quantum-harmonic-oscillator:defn:ladder-operators}}
    \frac{1}{2\hbar m\omega}\left(m\omega \widehat{x}-\I\widehat{p}\right)\left(m\omega \widehat{x}+\I\widehat{p}\right)
\step{distributivity}
    \frac{1}{2\hbar m\omega}\left((m\omega \widehat{x})^{2}-\I\widehat{p}m\omega \widehat{x}+m\omega \widehat{x}\I\widehat{p}-\I^{2}\widehat{p}^{2}\right)
\step{linearity of operators}
    \frac{1}{2\hbar m\omega}\left((m\omega \widehat{x})^{2}-\I m\omega \widehat{p}\widehat{x}+\I m\omega \widehat{x}\widehat{p}+\widehat{p}^{2}\right)
\step{explicitly insert a commutator}
    \frac{1}{2\hbar m\omega}\left((m\omega \widehat{x})^{2}+\I m\omega[\widehat{x},\widehat{p}]+\widehat{p}^{2}\right)
\step{commutator relation Eq~\eqref{eq:position-momentum-uncertainty-commutator}}
    \frac{1}{2\hbar m\omega}\left((m\omega \widehat{x})^{2}+\I m\omega (\I\hbar)+\widehat{p}^{2}\right)
\step{associativity of addition}
    \frac{1}{2\hbar m\omega}\left((m\omega \widehat{x})^{2}+\widehat{p}^{2}\right)
+\frac{\I m\omega (\I\hbar)}{2\hbar m\omega}
\step{folding the definition of the Hamiltonian Eq~\eqref{eq:quantum-harmonic-oscillator:defn:hamiltonian}}
    \frac{1}{\hbar\omega}\widehat{H} +\frac{\I m\omega (\I\hbar)}{2\hbar m\omega}
\step{algebra}
    \frac{1}{\hbar\omega}\widehat{H} -\frac{1}{2}\qedhere
  \end{calculation}
\end{proof}

\begin{lemma}\label{lemma:quantum-harmonic-oscillator:ladder-operator-hamiltonian-relation2}
We find $a_{-}a_{+} = (\widehat{H}/\hbar\omega) + 1/2$.
\end{lemma}

\begin{proof}
Our calculation is exactly the same, except the sign of the commutator
$[\widehat{x},\widehat{p}]$ changes in our consideration, which is the
source of the $1/2$ contribution.
\end{proof}

\N{Commutator of Ladder Operators}
We have $[a_{-},a_{+}]=1$.

\begin{proof}
  We have, by our previous two lemmas,
  \begin{equation}
a_{-}a_{+}-a_{+}a_{-} = \left(\frac{\widehat{H}}{\hbar\omega}+\frac{1}{2}\right)-\left(\frac{\widehat{H}}{\hbar\omega}-\frac{1}{2}\right).
  \end{equation}
Hence the result.
\end{proof}

\N{Ladder Operators Acting on Eigenstates}
Suppose $\psi(x)$ is an eigenfunction for the Hamiltonian operator, with
energy eigenvalue $E$. We claim $a_{+}\psi(x)$ is an eigenfunction with
energy eigenvalue $E+\hbar\omega$.

\begin{proof} By direct computation,
\begin{calculation}
  \widehat{H}(a_{+}\psi(x))
\step{using $\widehat{H} = \hbar\omega(a_{+}a_{-}+1/2)$}
  \hbar\omega\left(a_{+}a_{-}+\frac{1}{2}\right)(a_{+}\psi(x))
\step{move $a_{+}$ inside the parentheses using right distributivity}
  \hbar\omega\left(a_{+}a_{-}a_{+}+\frac{1}{2}a_{+}\right)(\psi(x))
\step{associativity}
  \hbar\omega\left(a_{+}(a_{-}a_{+})+\frac{1}{2}a_{+}\right)(\psi(x))
\step{using Lemma~\ref{lemma:quantum-harmonic-oscillator:ladder-operator-hamiltonian-relation2}}
  \hbar\omega\left(a_{+}\left(\frac{\widehat{H}}{\hbar\omega}+\frac{1}{2}\right)+\frac{1}{2}a_{+}\right)(\psi(x))
\step{algebra}
  (a_{+}\widehat{H}+a_{+}\hbar\omega)\psi(x)
\step{pull out a factor of $a_{+}$ using left distributivity}
  a_{+}(\widehat{H} + \hbar\omega)\psi(x)
\step{since $\psi$ is an energy eigenstate}
  a_{+}(E + \hbar\omega)\psi(x)
\step{commutativity of numbers and operators}
  (E + \hbar\omega)(a_{+}\psi(x))
\end{calculation}
Hence $a_{+}\psi(x)$ is an energy eigenstate with eigenvalue $E + \hbar\omega$.
\end{proof}

\M
Similarly, for any energy eigenstate $\psi(x)$ with eigenvalue $E$, we
have $a_{-}\psi(x)$ is another energy eigenstate with eigenvalue
$E-\hbar\omega$.

\begin{proof} By direct computation,
\begin{calculation}
  \widehat{H}(a_{-}\psi(x))
\step{using $\widehat{H} = \hbar\omega(a_{-}a_{+}-1/2)$}
  \hbar\omega\left(a_{-}a_{+}-\frac{1}{2}\right)(a_{-}\psi(x))
\step{move $a_{-}$ inside using right distributivity}
  \hbar\omega\left(a_{-}a_{+}a_{-}-\frac{1}{2}a_{-}\right)(\psi(x))
\step{associativity}
  \hbar\omega\left(a_{-}(a_{+}a_{-})-\frac{1}{2}a_{+}\right)(\psi(x))
\step{using Lemma~\ref{lemma:quantum-harmonic-oscillator:ladder-operator-hamiltonian-relation}}
  \hbar\omega\left(a_{-}\left(\frac{\widehat{H}}{\hbar\omega}-\frac{1}{2}\right)-\frac{1}{2}a_{-}\right)(\psi(x))
\step{algebra}
  (a_{-}\widehat{H}-a_{-}\hbar\omega)\psi(x)
\step{pull out $a_{-}$ using left distributivity}
  a_{-}(\widehat{H} - \hbar\omega)\psi(x)
\step{since $\psi$ is an energy eigenstate}
  a_{-}(E - \hbar\omega)\psi(x)
\step{commutativity of numbers and operators}
  (E - \hbar\omega)(a_{-}\psi(x))
\end{calculation}
Hence $a_{-}\psi(x)$ is an energy eigenstate with eigenvalue $E - \hbar\omega$.
\end{proof}

\N{Ground State}
We can't keep decreasing the energy eigenstate forever, there must be
some groundstate $\psi_{0}(x)$ such that
\begin{equation}
a_{-}\psi_{0}(x) =0.
\end{equation}
Solve this differential equation for $\psi_{0}(x)$.

\begin{proof}[Solution]
We recall the definition of the ladder operator $a_{-}$ from Eq~\eqref{eq:quantum-harmonic-oscillator:defn:ladder-operators}
\begin{equation}
\frac{1}{\sqrt{2\hbar m\omega}}\left(m\omega \widehat{x}+\I\widehat{p}\right)\psi_{0}(x)=0.
\end{equation}
Doing some algebra, this becomes
\begin{equation}
(m\omega\widehat{x} + \I\widehat{p})\psi_{0}(x) = 0,
\end{equation}
which in position-space becomes
\begin{equation}
\left(m\omega x + \I(-\I)\hbar\frac{\partial}{\partial x}\right)\psi_{0}(x)=0.
\end{equation}
This becomes
\begin{equation}
\psi'_{0}(x) = -\frac{m\omega}{\hbar}x\psi_{0}(x).
\end{equation}
Divide both sides by $\psi_{0}$, then integrating with respect to $x$
gives
\begin{equation}
\int\frac{\D\psi_{0}}{\psi_{0}} = -\frac{m\omega}{\hbar}\int x\,\D x.
\end{equation}
Hence
\begin{equation}
\ln(\psi_{0}(x)) = -\frac{m\omega}{\hbar}\frac{x^{2}}{2} + c_{0},
\end{equation}
where $c_{0}$ is some integration constant.
Then we find
\begin{equation}
\psi_{0}(x) = C\exp\left(-\frac{m\omega}{\hbar}\frac{x^{2}}{2}\right).
\end{equation}
We can determine $C$ by demanding $\psi_{0}(x)$ be normalizable. This
gives us
\begin{equation}
C^{2}\int\exp\left(-\frac{m\omega}{\hbar}x^{2}\right)\,\D x = 1.
\end{equation}
This is a Gaussian integral (\S\ref{cor:math:general-gaussian-integral-in-one-dim}), which means
\begin{equation}
C^{2}\sqrt{\frac{\pi\hbar}{m\omega}}=1\implies C=\left(\frac{m\omega}{\pi\hbar}\right)^{1/4}.
\end{equation}
Thus
\begin{equation}
\boxed{\psi_{0}(x) = \left(\frac{m\omega}{\pi\hbar}\right)^{1/4}\exp\left(-\frac{m\omega}{\hbar}\frac{x^{2}}{2}\right).}
\end{equation}
Up to phase factor (i.e., a factor of $\E^{\I c_{1}}$ for any
arbitrary constant $c_{1}\in\RR$).
\end{proof}

\N{Other Eigenstates}
We can apply the ladder operator $a_{+}$ finitely many times to get
other eigenstates $\psi_{n}(x)\sim(a_{+})^{n}\psi_{0}(x)$. More
precisely,
\begin{equation}
\psi_{n}(x) = C_{n}(a_{+})^{n}\psi_{0}(x)
\end{equation}
where $C_{n}$ is some normalization constant.

