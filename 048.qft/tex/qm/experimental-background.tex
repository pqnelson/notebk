\section[*Experimental Background]{*Experimental Background\footnote{This section can be skipped on first reading, it just summarizes the
experimental problems which classical physics could not answer (which
forced physicists to invent quantum mechanics).}}

\M Around the turn of the $20^{\text{th}}$ century, physics faced some
puzzling experimental results which formed the motivation for deriving
quantum mechanics. Let us review superficially what was known at the
time, then in each subsection we shall consider an experiment, why it
produced unexplainable results, and how physicists resolved things.

\N{Status of Light}\label{chunk:qm:experimental-background:status-of-light}
We should stress that at the time, physicists believed light propagated
as a wave. Thomas Young first tested this in 1800 with his famous
double-slit experiment, where light propagated through two slits and
these two light sources appeared to ``interfer'' with each other. This
has a description using geometric optics and waves, which undergraduates
learn routinely --- see, e.g., Young and Freedman's
\textit{University Physics}.

\N{Atomic Theory}\label{chunk:qm:experimental-background:atomic-theory}
Around the turn of the $20^{\text{th}}$ century, physicists began to
recognize that matter can be described as consisting of
atoms. Rutherford supervised Geiger and Marsden's 1909 experiment, then
later in 1911 explained that the experiment implied an atom consists of
a nucleus (which contains most of the mass and is positively charged)
and electrons which orbit the nucleus, just as planets orbit a star. We
stress that this interprets an electron as a particle.

Millikan's oil drop experiment (conducted in 1908, reported in 1913)
determined the electric charge for the elementary electric charge
$e\approx1.602\times10^{-19}~\mathrm{C}$.  Each electron has a charge of
$-e$, and the nucleus has a charge of $+ne$ where $n$ is the atomic
number.

\subsection{Black Body Radiation}

\begin{definition}
A \define{Black Body} is an idealized object which absorbs all electromagnetic
radiation which hits it.
\end{definition}

\begin{remark}
Just because a black body \emph{absorbs} all radiation which hits
\emph{does not mean} a black body cannot emit any radiation. A black
body \emph{can} and \emph{does} emit radiation.
\end{remark}

\begin{remark}
It turns out that ``thermal radiation'' is precisely ``electromagnetic radiation'',
so I may use the terms interchangeably. But at the end of the day,
they're just photons.
\end{remark}

\N{Problem Statement}
Suppose we make a hollow cube out of a black body. Then the problem we
want to answer: describe the distribution of the frequencies of the
[emitted] radiation inside the cube.

\N{Classical Expectations}
Inside the body, the radiation will constantly be absorbed and re-emitted,
eventually reaching thermal equilibrium.
At each frequency, the absorption and emission of radiation will be
perfectly balanced.

We can invoke the ``Equipartition Theorem'' (of classical statistical
mechanics) which states the energy in any given mode of electromagnetic
radiation should be exponentially distributed with an average value
equal to $k_{B}T$ where $T$ is the temperature in Kelvin and $k_{B}$ is
the Boltzmann constant.

\N{Ultraviolet Catastrophe}
The difficulty is that the average amount of energy is the same for
every mode\footnote{Physicists use the term ``mode'' and ``frequency''
interchangeably.}. So when we add up the energy for each mode (for which
there are infinitely many), we get an infinite amount of energy for the
radiation in our black body oven. This is referred to as the
\define{Ultraviolet Catastrophe}, since the infinity comes from the
ultraviolet [high-frequency] end of the spectrum.

This contradicts our observation of a finite amount of energy inside a
black body oven.

\N{Planck's Solution}
In 1900, Max Planck offered an alternative prediction for this problem. The
key step is that Planck postulated energy in the electromagnetic field
at a frequency $\omega$ is ``quantized'', meaning it comes in an integer
multiple of a certain basic unit $\hbar\omega$ where $\hbar$ is a
constant we now recognize as Planck's constant.\marginpar{{\footnotesize Planck's Constant $\hbar$}}

Planck postulated the energy is then exponentially distributed over
integer multiples of $\hbar\omega$. At low frequencies, this will
coincide with classical statistical mechanics. But at high frequencies
(where $\hbar\omega$ is comparable to $k_{B}T$), Planck's theory
predicts a rapid fall-off of the average energy.

\begin{exercise}
Let $c>0$ be real. Prove $\displaystyle\sum^{\infty}_{n=0}n\E^{-cn}=\frac{\E^{c}}{(1-\E^{-c})^{2}}$.
\end{exercise}

\begin{exercise}
In Planck's model, the energies for electromagnetic radiation in a
frequency $\omega$ (in units of $[\mbox{time}]^{-1}$) is
distributed randomly over all numbers $n\hbar\omega$ for $n=0,1,2,\dots$.
We postulate the probability of finding energy $n\hbar\omega$ is
\begin{equation}
\Pr(E = n\hbar\omega) = \frac{1}{Z}\E^{\beta n\hbar\omega}
\end{equation}
where $\beta = 1/(k_{B}T)$ and $Z$ is a normalization constant. Then the
expected value for energy is
\begin{equation}
\langle E\rangle = \sum^{\infty}_{n=0}(n\hbar\omega)\Pr(E=n\hbar\omega)= \frac{1}{Z}\sum^{\infty}_{n=0}(n\hbar\omega)\E^{-\beta n\hbar\omega}.
\end{equation}
Now you come in:
\begin{enumerate}
\item Find $Z$ by demanding $\sum_{n=0}^{\infty}\Pr(E = n\hbar\omega)=1$.
\item Prove $\displaystyle\langle E\rangle = \frac{\hbar\omega}{\E^{\beta\hbar\omega}-1}$.
\item Show $\langle E\rangle$ behaves like $1/\beta = k_{B}T$ for small $\omega$,
but $\langle E\rangle$ decays exponentially as $\omega\to\infty$.
\end{enumerate}
\end{exercise}

\subsection{Photoelectric Effect}

\N{Problem Statement}
Suppose we have a metal surface. We can observe as electromagnetic
radiation (``incident light'') hits the surface, electrons will be emitted from the surface.

Einstein found as we increase the \emph{intensity} of the incident
light, the \emph{number} of electrons increases but the \emph{energy} of
each electron does not change. This is bizarre for a number of reasons.

If electromagnetic radiation were waves, then low-frequency light at
high intensity  should ``build up'' the energy necessary to produce
electrons, but we do not observe this\dots which suggests
electromagnetic radiation (``light'') is not a wave but consists of
particles. More precisely, light comes in discrete energy packages which
Einstein called \define*{Photons}\index{Photon!Photoelectric effect}.

\N{Einstein's Solution}
In 1905, Einstein proposed that the electromagnetic radiation travels in
discrete energy packets called photons. For a frequency $\omega$, the
packet has energy $\hbar\omega$. (At the time, Einstein did not realize
$\hbar$ was Planck's constant, and used some constant of proportionality.)

The highest kinetic energy $K_{\text{max}}$ for the electron removed
from their atomic bindings by the absorption of a photon with energy
$\hbar\omega$ is then
\begin{equation}
K_{\text{max}} = \hbar\omega - W,
\end{equation}
where $W$ is the minimum energy required to remove an electron from the
metal surface. The literature refers to $W$ as the ``work function'' of
the surface. If we write the work function as:
\begin{equation}
W = \hbar\omega_{0},
\end{equation}
then the kinetic energy upper bound is
\begin{equation}
K_{\text{max}} = \hbar(\omega - \omega_{0}).
\end{equation}
Obviously emission can only occur when $K_{\text{max}}$ is positive,
requiring $\omega>\omega_{0}$.

\N{Quantum?}
Observe this solution requires $\hbar$ (a quantum quantity) and
classical physics fails to predict observed phenomena. More explicitly
the wave description of light fails to predict what we observe,
suggesting light consists of discrete packets (``particles'').

\subsection{Double-Slit Experiment}

\N{Problem statement}
We have assumed (\S\ref{chunk:qm:experimental-background:status-of-light})
light is a wave, since that appears to be supported by Young's
Double-Slit experiment. But if Einstein is correct and light propagated
in discrete packets, then how can we explain the double-slit phenomenon?

\N{Solutions}
J.J.~Thomson\footnote{J.J.~Thomson, ``On the ionization of Gases by
Ultra- Violet Light and on the evidence as to the structure of light
afforded by its Electrical Effect''. \journal{Prof.Cam.Phil.Soc.}
\volume{14} (1907) 417--424.} sketched out an experimental test in 1907, 
suggesting the results observed in the double-slit experiment could be
explained by the photons somehow interacting with each other.
Geoffrey Taylor\footnote{G.I.~Taylor, ``Interference Fringes with Feeble
Light''. \journal{Prof.Cam.Phil.Soc.} \volume{15} (1909) 114--115.}
first performed a low-intensity double-slit experiment
in 1909 by reducing the level of incident light (to the point where the
experiment took roughly 2000 hours to form an interference pattern ---
or 83 days and 12 hours). Taylor observed interference occurring still
at such low intensities of light.

Taylor's experimental results are interpreted as interference remains
even when photons are widely separated from each other (which is
weird!), but the photons are not interferring \emph{with each other}.
Instead, as Dirac writes in his book \textit{Principles of Quantum Mechanics},
``Each photon interfers only with itself.''

\subsection{Hydrogen Spectrum}

\N{Experiment}
If we pass electricity through a tube of hydrogen gas, then light will
be emitted. We can pass that light through a prism and four visible
``bands'' of light will form (red, cyan, blue, violet). This indicates
light is not a continuous spectrum but consists of discrete
frequencies. From the perspective of photons, this suggests only certain
energy packets are emitted. But this is surprising from a classical
perspective where light is a wave, and has no classical explanation.

\N{(Phenomenological) Explanation}
The Hydrogen atom consists of one proton and one electron. As we pass
electricity through the Hydrogen gas, the orbital electron ``gets
excited'' and moves farther away from the nucleus. Eventually the
electron will return to lower states, emitting a photon in the
process. This emitted photon takes on certain discrete
frequencies. Johannes Rydberg concluded in 1888 based on empirical data
that the energies of the emitted photon satisfy
\begin{equation}
E_{n} = -\frac{R}{n^{2}}
\end{equation}
where $n\in\NN$ and $R$ is the Rydberg constant
\begin{equation}
R = \frac{m_{e}Q^{4}}{2\hbar^{2}}.
\end{equation}
Here $m_{e}$ is the mass of the electron, $Q=-e$ is the charge of the
electron.
Note that we should use the reduced mass $\mu=m_{e}m_{p}/(m_{e}+m_{p})$
where $m_{p}$ is the mass of the proton,
but since $m_{p}\gg m_{e}$ ($m_{p}\sim 2\times10^{3}m_{e}$), we find
$\mu\approx m_{e}$ (the error would be something like $10^{-3}m_{e}$).

The frequencies for the emitted photon are then of the form
\begin{equation}
\omega = \frac{1}{\hbar}(E_{n} - E_{m})
\end{equation}
for some $n>m$, which agrees with observation.

But this lacks theoretical basis, and is rather unsatisfactory.

\N{Bohr Model}
Niels Bohr was as unsatisfied as I feel, and sought a theoretical
explanation for the Hydrogen spectrum. We can recall uniform circular
motion in the plane is described by the trajectory
\begin{equation}
(x(t),y(t)) = (r\cos(\omega t), r\sin(\omega t)).
\end{equation}
Its acceleration is obtained by taking the second derivative with
respect to time,
\begin{equation}
\vec{a}(t) = (-\omega^{2}r\cos(\omega t), -\omega^{2}r\sin(\omega t)).
\end{equation}
Observe the magnitude of the acceleration vector is $\omega^{2}r$. If
the only force acting on the electron (considered as a point-particle in
uniform circular motion) is Coulomb's law describing electromagnetism,
\begin{equation}
F = \frac{Q^{2}}{r^{2}}
\end{equation}
(up to some proportionality constant depending on our system of units),
then Newton's second Law gives us the equation of motion for our electron:
\begin{equation}
m_{e}\omega^{2}r = \frac{Q^{2}}{r^{2}}.
\end{equation}
Now we find the frequency $\omega$ by simple algebra
\begin{equation}\label{eq:qm:experimental-background:bohr-omega}
\omega = \sqrt{\frac{Q^{2}}{m_{e}r^{3}}}.
\end{equation}
Happy?

Well, we can do a few more things. We know the magnitude of velocity
$|\vec{v}|=\omega r$, multiplying by mass gives us momentum
$p=m_{e}\omega r$. When we plug in
Eq~\eqref{eq:qm:experimental-background:bohr-omega},
\begin{equation}\label{eq:qm:experimental-background:bohr:momenta}
p = \sqrt{\frac{m_{e}Q^{2}}{r}}.
\end{equation}
We can find the angular momentum, since its magnitude $J=pr$, giving us
\begin{equation}
J = \sqrt{m_{e}rQ^{2}}.
\end{equation}
So far, we haven't introduced anything new.

\N{Quantization Condition}
Bohr now introduces a quantization condition, namely that angular
momentum is an integer multiple of $\hbar$:
\begin{equation}
J = n\hbar = \sqrt{m_{e}rQ^{2}}.
\end{equation}
Solving for $r$ gives us
\begin{equation}
r_{n} = \frac{n^{2}\hbar^{2}}{m_{e}Q^{2}}.
\end{equation}
If we compute the energy for the electron with this radius, then we
recover observed Hydrogen spectrum.

\begin{remark}
What Bohr actually did was a bit more complicated, but if we were more
faithful to Bohr, we would make the big picture rather opaque.
\end{remark}

\begin{exercise}
Recall kinetic energy is $K=\frac{1}{2}m_{e}\vec{v}\cdot\vec{v}$ and the
potential energy for the electron would be $V = -Q^{2}/r$. Compute the
total energy $E = K + V$ with $r_{n}$ and $v=\omega r_{n}$. [Hint: you
  should recover $E_{n}=-R/n^{2}$.]
\end{exercise}

\N{De Broglie Condition}
Louis de Broglie [pronounced ``Broy-Lee''] proposed in 1924 that we
should interpret the Bohr quantization condition of the angular momentum
as a condition on the wave. That is, we should expect
\begin{equation}
2\pi r = n\lambda_{B}
\end{equation}
where $\lambda_{B}$ is the de Broglie wavelength (in units of length), so
the angular momentum 
\begin{equation}
J = rp = n\hbar
\end{equation}
is quantized when we have
\begin{equation}
\lambda_{B} = \frac{h}{p}.
\end{equation}
We write $p = h/\lambda_{B}$ (where $h=2\pi\hbar$) but it is more useful
to introduce a quantity $k=\lambda_{B}/2\pi$ satisfying
\begin{equation}
p = \hbar k.
\end{equation}
For vector quantities,
\begin{equation}
\vec{p} = \hbar\vec{k}.
\end{equation}
We call $\vec{k}$ the \define{Angular Wave Vector} (but the terminology
may vary depending on the reference) and it has units of
$[\mbox{length}]^{-1}$. We can work backwards, starting with this
condition, and derive Bohr's work. 

Specifically we have
\begin{calculation}
2\pi r
\step{de Broglie condition that an orbit is an integer number of periods}
n \frac{2\pi}{k}
\step{since $k=p/\hbar$}
n \frac{2\pi}{p/\hbar}
\step{using Eq~\eqref{eq:qm:experimental-background:bohr:momenta}}
n 2\pi\hbar\sqrt{\frac{r}{m_{e}Q^{2}}}.
\end{calculation}
We can solve this for $r$
\begin{equation}
  \sqrt{r} = \frac{n\hbar}{\sqrt{m_{e}Q^{2}}}\implies
r = \frac{n^{2}\hbar^{2}}{m_{e}Q^{2}}.
\end{equation}
This is precisely the result Bohr obtained.

\begin{remark}
Compare the de Broglie relations to Exercise~\ref{xca:qm:free-particle:de-broglie-relations}.
\end{remark}

\begin{remark}
Just to review, we have seen two \emph{different} theoretical
derivations for the energy spectrum of the Hydrogen atom. They are
equivalent, but de Broglie's relation $\vec{p}=\hbar\vec{k}$ turns out
to be an important relation which will play a critical role in quantum
theory. 
\end{remark}

\subsection{Electron Wave-Like Behaviour}

\M
There are a number of experiments which test the wave-like behaviour of
electrons, I will give two. They are rather ``hairy'' and technical, so
I'll skip the usual format and jump to the punch lines.

\N{Low Energy Electron Diffraction}
Around 1926,
Clinton Davisson and Lester Germer shot a beam of low energy electrons
at a thin sheet of nickel. There was an error in the experimental setup
and the nickel sheet heated up to extraordinary temperatures, forming a
different crystal structure than expected, which changed the scattering
behaviour of the electrons. The intensity of the backscattered electrons
had an angular dependence which suggested there was some interference
pattern which X-ray waves experience when scattering off metal sheets
(which Bragg determined a couple decades earlier).

But this only makes sense if electrons behaved like waves following the
de Broglie relations. This was the earliest strong evidence supporting
de Broglie's conjecture.

\N{Hitachi Double-Slit Experiment}
Akira Tonomura led a team of experimentalists at Hitachi in 1989
performing the double-slit experiment with electrons. The setup had a
screen which marked the point where an individual electron particle hits
it. After running $N$ electrons through this (for various $N$ ranging
from 160 to thousands), a clear interference pattern emerges which
resembles the double-slit experiment for light. This corroborates the
hypothesis that electrons have a wave-like behaviour, despite being
particles. 
