\section{Bra-Ket Formalism}

\M
Quantum mechanics boils down to linear algebra over complex numbers,
with some ``physics fluff'' motivation.

\N{Inner Product}
When we have a finite-dimensional vector space $\mathcal{H}\iso\CC^{n}$
we have the Hermitian inner product
\begin{equation}
H(\vec{u}, \vec{v}) = \sum^{n}_{j=1}u_{j}\overline{v_{j}}
\end{equation}
where most authors write $\langle\vec{u},\vec{v}\rangle$ but this will
be confusing for our purposes.

Quantum physics calls vectors \define{Kets} and write them as
$\mid u\rangle$ instead of $\vec{u}$. The covectors are called \define{Bras}
and written $\langle v\mid$ which stands for $H(-,\vec{v})$. Taken
together, we have the inner product formed from a bra-ket
\begin{equation}
\langle v\mid u\rangle = H(\vec{u},\vec{v}) = \sum^{n}_{j=1}u_{j}\overline{v_{j}}.
\end{equation}
As what happens so often in physics and mathematics, the conventions
taken are opposite of each other. Physicists read bra-kets from
right-to-left, Mathematicians work from left-to-right.

\begin{example}
The space $L^{2}(\RR)$ of twice integrable functions on the real line
form an infinite-dimensional complex vector space. For any $f,g\in L^{2}(\RR)$,
we have
\begin{equation}
\langle g\mid f\rangle = \int^{\infty}_{-\infty}f(x)\overline{g(x)}\,\D x.
\end{equation}
We could generalize this to $L^{2}(\RR^{n})$.
\end{example}

\N{Problem}
For $L^{2}(\RR)$, we have the position operator $\widehat{x}$ and the
momentum operator $\widehat{p}$. In position-space, what are their
eigenfunctions?

Well, the position eigenfunction satisfies
\begin{equation}
\widehat{x}u_{x}(x') = xu_{x}(x').
\end{equation}
This has as its solution
\begin{equation}
u_{x}(x') = \delta(x-x').
\end{equation}
But this is not an element of $L^{2}(\RR)$.

The momentum eigenfunction satisfies
\begin{equation}
\widehat{p}v_{p}(x) = pv_{p}(x).
\end{equation}
This is a differential equation
\begin{equation}
\frac{\hbar}{\I}\frac{\partial}{\partial x}v_{p}(x) = pv_{p}(x).
\end{equation}
Its solution is something like:
\begin{equation}
v_{p}(x) = C\exp(I px/\hbar),
\end{equation}
where $C$ is some complex constant. But this is not square-integrable,
i.e., $v_{p}(x)\notin L^{2}(\RR)$.

In particular, this means there is no ket $|x\rangle$ which is an
eigenvector for the position-operator, nor is there a ket $|p\rangle$
which is an eigenvector for the momentum-operator. This is bad.

\N{Domain of Position, Momentum Operators}
Consider now the set
\begin{equation}
\mathcal{D}(\widehat{x}) = \{f\in L^{2}(\RR)\mid H(\widehat{x}f,\widehat{x}f)<\infty\}.
\end{equation}
This describes the domain of the position operator, i.e., the elements
$f\in L^{2}(\RR)$ such that $\widehat{x}f$ also belongs in $L^{2}(\RR)$.

Care must be taken: $\mathcal{D}(\widehat{x})$ is a proper subset of
$L^{2}(\RR)$. For a counter-example, $g(x)=1/(x+\I)$ belongs to
$L^{2}(\RR)$ but does not live in $\mathcal{D}(\widehat{x})$.

Furthermore $\widehat{x}$ acting on
$\mathcal{D}(\widehat{x})$ does not produce a subset of
$\mathcal{D}(\widehat{x})$ --- i.e., $\widehat{x}$ does not restrict to
a linear operator on the subspace $\mathcal{D}(\widehat{x})$.
For example, $h(x)=1/(1+x^{2})$ belongs to $\mathcal{D}(\widehat{x})$,
but $\widehat{x}h(x)$ does not.

Similar reasoning holds for the domain of the momentum operator.

But worse, we could consider any operator of the form $A^{m}B^{n}$ where
$A$, $B\in\{\widehat{x},\widehat{p},\widehat{H}\}$ and $m$, $n\in\NN_{0}$.
The domains for these operators are all different, but overlap. We want
to take the physically relevant subspace
\begin{equation}
\Phi = \bigcap^{\infty}_{\substack{m,n=0\\ A,B=\widehat{x},\widehat{p},\widehat{H}}}\mathcal{D}(A^{m}B^{n}).
\end{equation}
This is the maximally invariant subspace for the algebra of operators
generated by $\{\widehat{x},\widehat{p},\widehat{H}\}$. In particular,
we see
\begin{equation}
\widehat{x}\Phi\subset\Phi,\quad\mbox{and}\quad
\widehat{p}\Phi\subset\Phi,\quad\mbox{and}\quad
\widehat{H}\Phi\subset\Phi.
\end{equation}
That is, the physically interesting operators restrict to operators on
this subspace.

Observe the eigenfunctions (``eigen-kets'') for position and momentum operators live in
$\Phi^{\times}$, the space of antilinear\footnote{Recall, antilinear is
like linear except $\varphi(zf)=\overline{z}\varphi(f)$ for any
$z\in\CC$. So antilinear is ``conjugate homogeneous'' and additive.} functionals over $\Phi$. Here
we interpret $\Phi$ is the space of test
functions\index{Functions!Test}\index{Test Functions}, and
$\Phi^{\times}$ is the space of distributions.\index{Distribution}
Dually, the ``eigen-bras'' for these operators live in $\Phi'$ the dual
space of $\Phi$ containing the linear functionals over $\Phi$.

\N{Rigged Hilbert Spaces}
The triplet $\Phi\subset\mathcal{H}\subset\Phi'$ is called a
\define{Rigged Hilbert Space}. The adjective ``rigged'' should be in the
sense of a sailing ship is ``fully-rigged'' when it has at least three
masts and each mast has three segments (and not
in the sense that a Casino is ``rigged'' to always have its customers
lose).

For more about the exact details of Rigged Hilbert Spaces in quantum
mechanics, see R.~de la Madrid~\cite{Madrid:2005rs}.

Most physicists have a devil-may-care attitude towards these details,
but it is worth mentioning because when we generalize quantum
theory\dots we end up with spaces of test functions and whatnot, and
they appear as artificial curios without some context.

\N{Abusing Notation}
If we have a [rigged] Hilbert space $\mathcal{H}$ and some self-adjoint
operator $\widehat{A}$ acting on it, then we will write an eigenvector
for it using its eigenvalue,
\begin{equation}
\widehat{A}\mid a\rangle = a\mid a\rangle.
\end{equation}

\M
We can express a ket $|\psi\rangle$ in position-space using the basis
$|x\rangle$, the coefficients are precisely the function:
\begin{equation}
\psi(x) = \langle x\mid\psi\rangle.
\end{equation}
Then we reconstruct the ket as:
\begin{equation}
|\psi\rangle = \int | x\rangle\langle x\mid\psi\rangle\,\D x.
\end{equation}
We could do likewise in momentum-space.

\N{Fourier Transform}
The reader can verify that
\begin{equation}
\langle x\mid p\rangle = C\exp(\I px/\hbar)
\end{equation}
where $C$ is some complex constant. We will pick $C=1/\sqrt{2\pi\hbar}$,
because then
\begin{equation}
\widetilde{\psi}(p) = \int\langle p\mid x\rangle\langle x\mid\psi\rangle\,\D x
\end{equation}
is precisely the Fourier transform, and
\begin{equation}
\psi(x) = \int\langle x\mid p\rangle\langle p\mid\psi\rangle\,\D p
\end{equation}
is the inverse Fourier transform.

\begin{remark}
We can use the de Broglie relation to write $|p\rangle=\hbar|k\rangle$.
Then $\langle x\mid k\rangle = \exp(\I kx)/\sqrt{2\pi}$.
\end{remark}

\N{Resolution of the Identity}
We effectively have the identity operator satisfy the relation:
\begin{equation}
\id = \int |x\rangle\langle x|\,\D x = \int |p\rangle\langle p|\,\D p.
\end{equation}
This is the generalization of the relation in finite-dimensional vector
spaces, when $|e_{j}\rangle$ form an orthonormal basis,
\begin{equation}
\id = \sum^{n}_{j=1}|e_{j}\rangle\langle e_{j}|.
\end{equation}
If we were working with, say, Qubits (or whatever), then we'd use this
latter relation.

\M
We also have the orthogonality relations
\begin{equation}
\langle x\mid x'\rangle = \delta(x-x'),\quad\mbox{and}\quad
\langle p\mid p'\rangle = \delta(p - p').
\end{equation}
These are completely consistent with the Fourier analysis from the
previous chunks.

\begin{exercise}
Let $|\psi\rangle$ be a ket in our Hilbert space (in particular, it is
normalizable). Determine what the projection operator onto $|\psi\rangle$
is in terms of $|\psi\rangle$ and $\langle\psi|$.
\end{exercise}
