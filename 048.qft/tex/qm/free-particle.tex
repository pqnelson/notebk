\section{Quantum Free Particle}

\N{Problem} Given the free particle Hamiltonian for one spatial dimension,
\begin{equation}
\widehat{H} = \frac{1}{2m}\widehat{p}^{2},
\end{equation}
where $m$ is the particle mass, determine the eigenstates for the
Schrodinger equation.

\M
We effectively have from Schrodinger's equation
\begin{equation}
\I\hbar\frac{\partial}{\partial t}\psi(x,t) =\widehat{H}\psi(x,t)
\end{equation}
a second-order linear partial differential equation.

\N{Separation of Variables}
We can write
\begin{equation}
\psi(x,t) = \E^{Et/(\I\hbar)}\varphi(x)
\end{equation}
which makes the left-hand side of the Schrodinger equation become
\begin{equation}
\begin{split}
\I\hbar\frac{\partial}{\partial t}\psi(x,t) &= \I\hbar\frac{E\E^{Et/(\I\hbar)}}{\I\hbar}\varphi(x)\\
&= E \E^{Et/(\I\hbar)}\varphi(x).
\end{split}
\end{equation}
The right-hand side becomes
\begin{equation}
\widehat{H}\psi(x,t) = \E^{Et/(\I\hbar)}\widehat{H}\varphi(x).
\end{equation}
Equating these two terms transforms Schrodinger's equation into
\begin{equation}
\widehat{H}\varphi(x) = E\varphi(x).
\end{equation}

\M
In position-space, the momentum operator acts as
\begin{equation}
  \widehat{p}\varphi(x) = -\I\hbar\frac{\partial}{\partial x}\varphi(x).
\end{equation}
Thus the Hamiltonian operator acts as
\begin{equation}
\widehat{H}\varphi(x) = \frac{-\hbar^{2}}{2m}\frac{\partial^{2}}{\partial x^{2}}\varphi(x).
\end{equation}
The Schrodinger equation is then
\begin{equation}
\frac{\partial^{2}}{\partial x^{2}}\varphi(x) = \frac{-2Em}{\hbar^{2}}\varphi(x).
\end{equation}
Let us take
\begin{equation}
k^{2} = \frac{2Em}{\hbar^{2}}.
\end{equation}
Then the solution to Schrodinger's equation is,
\begin{equation}
\varphi(x) = c_{1}\E^{\I kx} + c_{2}\E^{-\I kx},
\end{equation}
where $c_{1}$, $c_{2}\in\CC$ are constants to be determined by initial conditions.

\begin{exercise}\label{xca:qm:free-particle:dimension-of-k}
What is the dimension of $k$? What is the dimension of $\hbar k^{2}/m$?
\end{exercise}
\begin{exercise}
Write $E$ in terms of $k$.
\end{exercise}

\N{Interpreting Terms}
The full solution is then
\begin{equation}
\psi(t,x) \stackrel{???}{=} c_{1}\E^{\I k(x + [\hbar k/2m]t)} + c_{2} \E^{-\I k(x - [\hbar k/2m]t)}.
\end{equation}
Recall that, for fixed speed $v$, the quantity $x + vt$ describes a
particle moving in the $(-x)$-direction at constant speed
$v$. Similarly, $x-vt$ describes a particle moving in the
$(+x)$-direction at constant speed.

We know that $\hbar^{2}k/2m$ is constant. But we never specified the
sign of $k$ --- it could be positive or negative.

If $k>0$, the $c_{1}$ term describes a ``left moving'' wave, and the
$c_{2}$ term describes a ``right moving'' wave.
If $k<0$, then the roles are reversed.

In fact, we could take advantage of this ambiguity of sign and note when
$k<0$ in the $c_{2}$ term we recover the $c_{1}$ term, allowing us to
write
\begin{equation}
\psi_{k}(t,x) := c_{1}\E^{\I k(x + [\hbar k/2m]t)}.
\end{equation}

\N{Normalizability}
We should check $\psi_{k}(t,x)$ is normalizable, so we can interpret it
as a physical state:
\begin{equation}
\int^{\infty}_{-\infty}\psi_{k}(t,x)^{*}\psi_{k}(t,x)\,\D x =
\int^{\infty}_{-\infty}|c_{1}|^{2}\,\D x = +\infty.
\end{equation}
Well, that's\dots problematic.
The standard textbook argues we cannot interpret these stationary
solutions to the free particle Schrodinger equation as ``physical states'',
but they are useful mathematical results which we will use in
calculations later on.

\begin{definition}
In $n$-spatial dimensions let $\vec{k}$ be a constant $n$-vector, and
$\omega$ be a constant of dimensions $[\mbox{length}]^{-1}$, $[\mbox{time}]^{-1}$
respectively. We define a \define{Wave Packet} to be the [dimensionless] function
\begin{equation}
\psi(\vec{x}, t) = C\exp\bigl(\I(\vec{k}\cdot\vec{x} - \omega t)\bigr).
\end{equation}
The constant coefficient $C$ is usually taken to be $(2\pi)^{-(n+1)/2}$
or $1$.
\end{definition}

\begin{example}
  The free quantum particle's wave function is a wave packet, with
  $\omega = \hbar k^{2}/(2m)$.
\end{example}

\begin{exercise}\label{xca:qm:free-particle:de-broglie-relations}
Let $\psi(\vec{x},t)$ be a wave packet with parameters $\vec{k}$,
$\omega$. Prove the de Broglie relations: 
\begin{enumerate}
\item $\I\hbar\partial_{t}\psi(\vec{x},t) = \hbar\omega\psi(\vec{x},t)$,
  that is, $E=\hbar\omega$.
\item $\widehat{\vec{p}}\psi(\vec{x},t) = \hbar\vec{k}\psi(\vec{x},t)$,
  that is, $\vec{p}=\hbar\vec{k}$.
\end{enumerate}
\end{exercise}

\N{Fourier Transform}
When we have a function in position-space $f(\vec{x},t)$, we can perform
the Fourier transform into momentum-space (taking $\omega=0$) by
integrating against the wave-packet
\begin{equation}
\widetilde{f}(\vec{k}, t) = \frac{1}{(2\pi)^{n/2}}\int f(\vec{x},t)\exp\bigl(\I(\vec{k}\cdot\vec{x})\bigr)\,\D^{n}x.
\end{equation}
We can inverse Fourier-transform by integrating against the
wave-packet's complex conjugate,
\begin{equation}
f(\vec{x},t) = \frac{1}{(2\pi)^{n/2}}\int \widetilde{f}(\vec{k}, t)\exp\bigl(-\I(\vec{k}\cdot\vec{x})\bigr)\,\D^{n}k.
\end{equation}
This is why wave packets are useful mathematical gadgets, despite being
``unphysical'' (in some super-strict sense of the term).

\begin{exercise}
From your solution to Exercise~\ref{xca:qm:free-particle:dimension-of-k},
double check that $\vec{k}\cdot\vec{x}$ is dimensionless.
\end{exercise}
