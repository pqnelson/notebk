\section{Constraints}

\M
Recall we can write down the Euler--Lagrange equations for a mechanical
system of point particles as:
\begin{equation}
\frac{\partial L}{\partial q^{i}} - \frac{\D}{\D t}\frac{\partial L}{\partial\dot{q}^{i}}=0,
\end{equation}
where $i=1,\dots,N$.
We can expand the total time derivative as
\begin{equation}
\frac{\partial L}{\partial q^{i}}
- \left(\ddot{q}^{j}\frac{\partial}{\partial\dot{q}^{j}}\frac{\partial L}{\partial\dot{q}^{i}}
+\dot{q}^{j}\frac{\partial}{\partial q^{j}}\frac{\partial L}{\partial\dot{q}^{i}}\right)=0.
\end{equation}
Therefore we can write this as:
\begin{equation}
\ddot{q}^{j}\frac{\partial}{\partial\dot{q}^{j}}\frac{\partial L}{\partial\dot{q}^{i}}
=\frac{\partial L}{\partial q^{i}}
-\dot{q}^{j}\frac{\partial}{\partial q^{j}}\frac{\partial L}{\partial\dot{q}^{i}}.
\end{equation}
If the matrix $\partial^{2}L/\partial\dot{q}^{i}\partial\dot{q}^{j}$ is
invertible, then we can uniquely determine the accelerations as a
function of positions and velocities.

\M When this matrix is singular
(i.e., non-invertible), what happens? Well, recall the definition of the
canonically conjugate momenta:
\begin{equation}
p_{i} = \frac{\partial L}{\partial\dot{q}^{i}}.
\end{equation}
Its Jacobian matrix is
\begin{equation}
\frac{\partial p_{i}}{\partial\dot{q}^{j}} = \frac{\partial^{2} L}{\partial\dot{q}^{j}\partial\dot{q}^{i}}.
\end{equation}
When our desired matrix is not invertible, then it's logically
equivalent to having the Jacobian for the canonically conjugate momenta
being non-invertible. This means the momenta are not all independent of
each other; instead we have some relations
\begin{equation}
\phi_{m}(q,p)=0
\end{equation}
for $m=1,\dots,M$.

\begin{remark}
We will assume that the $\phi_{m}$ are independent of each other.
\end{remark}

\begin{definition}
The conditions $\phi_{m}(q,p)=0$ obtained from the definition of the
conjugate momenta are called \define{Primary Constraints}.\index{Constraint!Primary|textbf}\index{Primary constraint|see{Constraint}}
The submanifold defined by these equations in the phase space is called
the \define{Primary Constraint Surface}.
\end{definition}

\begin{remark}
If we permute/reindex the velocities so the first $N-M$ canonically
conjugate momenta \emph{are} independent of each other, then the primary
constraints will take the form:
\[\phi_{m}=p_{N-M+m}-f_{m}(q^{i},p_{j'}),\]
where $j'=1,\dots,N-M$ ranges over the independent momenta.
\end{remark}

\begin{remark}
The word ``primary'' here refers to the fact that we obtained them from
trying to obtain the canonically conjugate momenta. Later we will find
``secondary'' constraints by demanding, roughly, $\PB{\phi_{m}}{H}=0$ on
the constraint surface. This invokes the equations of motion on the
primary constraints, which gives us ``secondary'' constraints if the
Poisson bracket is nonzero.
\end{remark}

\N{Canonical Hamiltonian}
The next step in the Hamiltonian analysis is to determine the
Hamiltonian by means of the Legendre transform:
\begin{equation}
H = \dot{q}^{i}p_{i} - L.
\end{equation}

\begin{lemma}
The canonical Hamiltonian $H$ has velocities $\dot{q}^{i}$ enter through
the momenta $p_{i}(q,\dot{q})$ alone.
\end{lemma}

\begin{proof}
We consider the change $\variation H$ induced by arbitrary independent
variations of the positions and velocities
\begin{calculation}
\variation H
\step{linearity of $\variation$}
\variation(\dot{q}^{i}p_{i}) - \variation L
\step{product rule}
\dot{q}^{i}\,\variation p_{i} + p_{i}\,\variation\dot{q}^{i} - \variation L
\step{expanding $\variation L$ using partial derivatives}
\dot{q}^{i}\,\variation p_{i} + p_{i}\,\variation\dot{q}^{i} 
- \variation\dot{q}^{i}\frac{\partial L}{\partial\dot{q}^{i}}
- \variation q^{i}\frac{\partial L}{\partial q^{i}}
\step{collecting coefficients of $\variation\dot{q}^{i}$}
\dot{q}^{i}\,\variation p_{i} + \left(p_{i} 
- \frac{\partial L}{\partial\dot{q}^{i}}\right)\variation\dot{q}^{i}
- \variation q^{i}\frac{\partial L}{\partial q^{i}}
\step{using the definition of the conjugate momentum}
\dot{q}^{i}\,\variation p_{i} + 
- \frac{\partial L}{\partial q^{i}}\variation q^{i}.
\end{calculation}
Hence
\begin{equation}
\variation H = 
\dot{q}^{i}\,\variation p_{i} + 
- \frac{\partial L}{\partial q^{i}}\variation q^{i},
\end{equation}
establishing the result, since $\variation p_{i}$ is not an independent
variation but really is a combination of the $\variation q$'s and
$\variation\dot{q}$'s (and therefore only through $\variation p_{i}$ can
$\variation\dot{q}$ can enter into $\variation H$). 
\end{proof}

\N{Problem: Non-Uniqueness of Hamiltonian}
We should observe that the momenta are not all independent of each
other, so the variations $\variation p_{i}$ are not all independent of
each other. Instead the $\variation p_{i}$ are restricted to preserve
the primary constraints $\phi_{m} = 0$.

We are forced to conclude the canonical Hamiltonian is well-defined only
on the submanifold defined by the primary constraints, but can be
extended arbitrarily off that manifold.

It follows that the formalism should remain unchanged by the replacement
\begin{equation}\label{eq:constrained-dynamics:constraints:hamiltonian-remains-unchanged-by-replacement}
H\to H + c^{m}(q,p)\phi_{m}
\end{equation}
for arbitrary functions $c^{m}$.

\begin{theorem}[{Henneaux and Teitelboim~\cite[Th1.2]{Henneaux:1992ig}}]
If $\lambda_{i}\,\variation q^{i} + \mu^{i}\,\variation p_{i}=0$ for
arbitrary variations $\variation q^{i}$, $\variation p_{i}$ tangent to
the constraint surface, then we have
\begin{subequations}
\begin{align}
\lambda_{i} &= u^{m}\frac{\partial\phi_{m}}{\partial q^{i}}\\
\mu^{i} &= u^{m}\frac{\partial\phi_{m}}{\partial p_{i}}
\end{align}
\end{subequations}
for some $u^{m}$. The equalities here are equalities on the constraint surface.
\end{theorem}

\M
Working on Eq~\eqref{eq:constrained-dynamics:constraints:hamiltonian-remains-unchanged-by-replacement},
we can rewrite it as:
\begin{equation}
\left(\frac{\partial H}{\partial q^{i}} + \frac{\partial L}{\partial q^{i}}\right)\variation q^{i}
+\left(\frac{\partial H}{\partial p_{i}} -\dot{q}^{i}\right)\variation p_{i}=0,
\end{equation}
from which we now apply our preceding theorem to
find:\marginnote{Introduce $u^{m}$}
\begin{subequations}
\begin{align}
\dot{q}^{i} &= \frac{\partial H}{\partial p_{i}} + u^{m}\frac{\partial\phi_{m}}{\partial p_{i}}\\
\left.-\frac{\partial L}{\partial q^{i}}\right|_{\dot{q}}
&=\left.\frac{\partial H}{\partial q^{i}}\right|_{p}+u^{m}\frac{\partial\phi_{m}}{\partial q^{i}}.
\end{align}
\end{subequations}
The first of these equations allows us to recover the velocities
$\dot{q}^{i}$ from knowing (1) the momenta $p_{i}$ (obeying the primary
constraints $\phi_{m}=0$) and (2) the extra parameters $u^{m}$. These
extra parameters can be thought of as coordinates on the surface of the
inverse images of a given $p_{i}$.

\N{Determining $u$'s}.
If the constraints are independent, then the vectors
$\partial\phi_{m}/\partial p_{i}$ are also independent on the constraint
surface $\phi_{m}=0$ because of the regularity conditions. Hence no two
different sets of the $u$'s can yield the same velocities. This means
that the $u$'s can be expressed, in principle, as functions of the
coordinates and velocities by solving the equations
\begin{equation}
\dot{q}^{i}
= \frac{\partial H}{\partial p_{i}}\bigl(q,p(q,\dot{q})\bigr)
+ u^{m}(q,\dot{q})\frac{\partial\phi_{m}}{\partial p_{i}}\bigl(q,p(q,\dot{q})\bigr).
\end{equation}

% Exercise 1.1(a)

\N{Equations of Motion using Poisson Bracket}\label{chunk:constrained-hamiltonian:constraint:initial-form-of-equations-of-motion-using-poisson-brackets}
We have, for any phase-space function $F(q,p)$, its time derivative:
\begin{equation}
\dot{F} = \PB{F}{H} + u^{m}\PB{F}{\phi_{m}}.
\end{equation}
This is the usual Poisson bracket.

\N{Consistency of Constraints}
We will want the primary constraints be preserved in time
$\dot{\phi}_{m}=0$, which can be expressed using the Poisson brackets as
\begin{equation}
\PB{\phi_{m}}{H} + u^{m'}\PB{\phi_{m}}{\phi_{m'}}=0.
\end{equation}
There are two possible consequences to this condition:
\begin{enumerate}
\item it reduces to a relation independent of the $u$'s (thus involving
  only the $q$'s and $p$'s), or
\item it may impose a restriction on the $u$'s.
\end{enumerate}
The first case, if the relation is independent of the primary
constraints, then we have a \define{Secondary Constraint}.

\M
If there is a secondary constraint, call it $C(q,p)=0$ for illustration,
then we must impose a new consistency condition
\begin{equation}
\PB{C}{H} + u^{m}\PB{C}{\phi_{m}}=0.
\end{equation}
As before, we must check whether this implies a new secondary constraint
or whether it restricts the $u$'s. After this process is finished, we
are left with a number of secondary constraints, which we denote
by\marginnote{Secondary constraints $\phi_{k}$}
\begin{equation}
\phi_{k}=0
\end{equation}
for $k=M+1$, \dots, $M+K$ where $K$ is the total number of secondary
constraints. 

We then have all the constraints (primary and secondary) written as
\begin{equation}
\phi_{j} = 0,
\end{equation}
for $j=1$, \dots, $J=M+K$.

\begin{remark}
Some authors are painfully pedantic, calling ``secondary constraints''
only those constraints emerging from the consistency of ``primary constraints''.
The constraints emerging from the consistency of primary and secondary
constraints are dubbed ``tertiary constraints''. Presumably demanding
consistency of the primary, secondary, and tertiary constraints gives us
quaternary constraints. We then end up with quinary, senary, septenary,
octonary, nonary, denary constraints (and hopefully no more).

However, as we will see soon, this division into primary, secondary,
etc., constraints is not physically fundamental. There is another way to
classify constraints due to Dirac.
\end{remark}

\begin{definition}
Let $F$, $G$ be two phase space functions. Then we say $F$ and $G$ are
\define{Weakly Equal} and write $F\weakEq G$ if and only if they
coincide on the constraint manifold $\phi_{j}=0$. We refer to the symbol
``$\weakEq$'' as the \emph{weak equality symbol}.

If $F$, $G$ coincide \emph{everywhere} in the phase space, then we say
$F$ and $G$ are \define{Strongly Equal} and indicate this as $F = G$
with the usual equality symbol.
\end{definition}

\begin{theorem}[{Henneaux and Teitelboim~\cite[Th1.2]{Henneaux:1992ig}}]
If $F$ and $G$ are two phase space functions, then
\begin{equation}
F\weakEq G\iff F-G = c^{j}(q,p)\phi_{j}.
\end{equation}
\end{theorem}

\N{Re-phrasing Consistency Condition: Restricting $u^{m}$}
We have a complete set of constraints $\phi_{j}\weakEq0$, and we want to
study the restrictions on the Lagrange multipliers $u^{m}$. These
restrictions are
\begin{equation}
\PB{\phi_{j}}{H} + u^{m}\PB{\phi_{j}}{\phi_{m}}\weakEq0,
\end{equation}
where we sum $m$ from $1$ to $M$, and $j$ is an index taking any value
from $1$ to $J$. This describes a system of $J$ linear equations in the
$M\leq J$ unknowns $u^{m}$ with coefficients which are functions of the
$q$'s and $p$'s. These equations should have solutions, otherwise the
Lagrangian description would be inconsistent.

The general solution to our system of linear equations is of the form
\begin{equation}
u^{m} = U^{m} + V^{m}
\end{equation}
where $U^{m}$ is a particular solution of the inhomogeneous system of
equations, and $V^{m}$ is the most general solution to the associated
homogeneous system
\begin{equation}
V^{m}\PB{\phi_{j}}{\phi_{m}}\weakEq0.
\end{equation}
The most general $V^{m}$ is a linear combination of linearly independent
solutions ${V_{a}}^{m}$ for $a=1$, \dots, $A$. The number $A$ of
independent solutions is the same for all $q$, $p$ on the constraint
surface, because we assumed the rank of $\PB{\phi_{j}}{\phi_{m}}$ is
constant on the constraint surface.
Then we can write
\begin{equation}
u^{m}\weakEq U^{m} + v^{a}{V_{a}}^{m},
\end{equation}
where the coefficients $v^{a}$ \emph{are totally arbitrary}.

Hence we have separated $u^{m}$ into the part $U^{m}$ determined by the
consistency conditions and separately the totally arbitrary part.

\begin{definition}
We define the \define{First-Class Hamiltonian} to be the function
\begin{equation}
H' := H + U^{m}\phi_{m}.
\end{equation}
\end{definition}

\N{Equations of Motion}
We can revisit the equations of motion (\S\ref{chunk:constrained-hamiltonian:constraint:initial-form-of-equations-of-motion-using-poisson-brackets})
to write the time derivative of any phase space function $F$ as
\begin{equation}
\dot{F}\weakEq\PB{F}{H' + v^{a}\phi_{a}}
\end{equation}
where $\phi_{a} = {V_{a}}^{m}\phi_{m}$.

To see this, we have used
\begin{equation}
\PB{F}{U^{m}\phi_{m}} = U^{m}\PB{F}{\phi_{m}}+\PB{F}{U^{m}}\phi_{m}
\weakEq U^{m}\PB{F}{\phi_{m}},
\end{equation}
and similar manipulations for $\PB{F}{{V_{a}}^{m}\phi_{m}}\weakEq{V_{a}}^{m}\PB{F}{\phi_{m}}$.

\begin{definition}
We define the \define{Total Hamiltonian} $H_{T} = H' + v^{a}\phi_{a} = H + u^{m}\phi_{m}$.
\end{definition}

\begin{remark}
The total Hamiltonian recovers the Lagrangian dynamics, but in terms of
canonical variables.
\end{remark}

\M
We can then describe the equations of motion using the total Hamiltonian
as
\begin{equation}
\dot{F}\weakEq\PB{F}{H_{T}}.
\end{equation}
These equations are equivalent, by construction, to the original
Lagrangian equations of motion.

\begin{definition}[{Henneaux and Teitelboim~\cite[\S1.1.10]{Henneaux:1992ig}}]
We call a phase-space function $F$ \define{First-Class} if its Poisson
bracket with every constraint weakly
vanishes: $\PB{F}{\phi_{j}}\weakEq0$ for every $j=1,\dots,J$.
Otherwise, if $F$ is not first-class, we call it \define{Second-Class}.
\end{definition}

\M Observe that the total Hamiltonian is the sum of the first-class
Hamiltonian $H'$ and the first-class primary constraints multiplied by
arbitrary coefficients. This splitting of $H_{T}$ into $H'$ and
$v^{a}\phi_{a}$ is not unique because $U^{m}$ appearing in this
decomposition can be \emph{any solution} to a (possibly underdetermined)
system of linear equations.

\begin{theorem}
If $F$ and $G$ are first-class functions on the phase space, then their
Poisson bracket $\PB{F}{G}$ is also a first-class function.
\end{theorem}

\begin{proof}[Proof sketch]
The proof is a straightforward calculation, using
$\PB{F}{\phi_{j}}={f_{j}}^{j'}\phi_{j'}$
and
$\PB{G}{\phi_{j}}={g_{j}}^{j'}\phi_{j'}$, then using the Jacobi identity
to write $\PB{\PB{F}{G}}{\phi_{j}}=\PB{F}{\PB{G}{\phi_{j}}}-\PB{G}{\PB{F}{\phi_{j}}}$.
This will turn out to be a linear combination of constraints, which
proves the theorem.
\end{proof}

\N{Gauge Transformations}
We can transform a function $F$ using an arbitrary variation of the
arbitrary functions $\variation v^{a}=\overline{v}^{a}-v^{a}$ as
\begin{equation}
\variation F = \variation v^{a}\PB{F}{\phi_{a}}.
\end{equation}
These are precisely gauge transformations, i.e., transformations which
do not change the physical state of the system.
