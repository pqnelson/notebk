\chapter{Special Relativity}

\M
There's a heuristic taught to graduate students:
\begin{equation}
\begin{pmatrix}\mbox{Quantum}\\
\mbox{Field}\\
\mbox{Theory}
\end{pmatrix}
=
\begin{pmatrix}\mbox{Special}\\
\mbox{Relativity}
\end{pmatrix}
+\begin{pmatrix}\mbox{Quantum}\\
\mbox{Mechanics}
\end{pmatrix}.
\end{equation}
This is partially true, but we should review special relativity (if only
to specify our conventions).

Historically, 
\begin{equation}
\begin{pmatrix}\mbox{Special}\\
\mbox{Relativity}
\end{pmatrix}
=
\begin{pmatrix}\mbox{Newtonian}\\
\mbox{Mechanics}
\end{pmatrix}
+
(\mbox{Electromagnetism}).
\end{equation}
This is because electromagnetism determines that the speed of light is
constant in a vacuum, and Newtonian mechanics assumes Galilean
relativity. But what happens if an inertial observer is moving at the
speed of light? Addition of velocities would suggest that bodies could
move faster than light\footnote{Einstein's thought experiment: suppose
you were riding on a bicycle going $0.999c$, and then you turn on your
headlights. What speed will the photons emitted from your headlights
travel?}, which violates results from electromagnetism.



\section{Foundations}

\begin{axiom}[Principle of Relativity]
The laws of physics are identical in all inertial frames. That is to
say, the outcome of any physical experiment is the same when performed
with identical initial conditions relative to any inertial frame.
\end{axiom}

\begin{axiom}
There exists an inertial reference frame in which light signals in
vacuum always travel rectilinearly at constant speed $c$, in all
directions, independently of the motion of the source.
\end{axiom}

\begin{remark}
These are the axioms as formulated in Rinder~\cite[\S4]{Rindler:1991sr}.
Traditionally, introductory texts to special relativity use some version
of these axioms, then ``derive'' results in special relativity.

A more intuitive approach uses $K$-calculus, as first formulated by
Bondi, using spacetime diagrams.
\end{remark}

\M
The best approach to discussing special relativity is the geometric
approach, using spacetime diagrams. However, the point of this chapter
is not to teach special relativity, but to review the Lorentz group as
encoding Lorentz invariance.

The geometric approach uses a 4-dimensional affine space describing
physical space-time, consisting of ``events'' [points]. We do the usual
trick setting up a reference frame as consisting of a point (or trajectory)
which we think of as the ``origin'', then take 4 other distinct points
to construct unit vectors (by taking the displacement from the
``origin'' to each of these points). This gives us a coordinate system.

We can describe time intervals in difference reference frames by
``shooting off'' one photon at the start of the interval (and then
another at the end of the interval). This translates to an interval
$k\,\Delta t$ in the other reference frame, where $k$ is a factor to be
determined. This is done in D'Inverno's \textit{Introducing Einstein's Relativity},
chapters 2 and 3.

\N{On ``Paradoxes''}
One last word, a lot of ``paradoxes'' encountered in physics (like the
``Twin paradox'' in special relativity) are not actually paradoxes: they
are just physical situations where our intuitions fail to describe the
phenomenon properly.

\endinput
\section{Kinematics}

\N{Four-Vectors} Special relativity differs from Newtonian physics by
working with 4-vectors. The usual vectors we encountered in physics
$\vec{x}$ are 3-vectors in physical space. Now we will use 4-component
vectors of the form $(t, \vec{x})$ or more generally using indices
$(x^{0}, x^{1}, x^{2}, x^{3})$. We can write this using basis vectors
\begin{subequations}
\begin{align}
  A &= (A^{0}, A^{1}, A^{2}, A^{3})\\
&= A^{0}\vec{e}_{0} + A^{1}\vec{e}_{1} + A^{2}\vec{e}_{2} + A^{3}\vec{e}_{3}\\
&= A^{0}\vec{e}_{0} + A^{i}\vec{e}_{i}\\
&= A^{\mu}\vec{e}_{\mu}
\end{align}
\end{subequations}
where $\vec{e}_{\mu}$ are basis vectors, we implicitly sum over repeated
indices, $i = 1,2,3$, and $\mu=0,1,2,3$.

The zeroth component is the time component, the remaining components are
the spatial components.

\begin{remark}
When the components of the four-vector are upstairs $A^{\mu}$,
these are \define{Contravariant} vectors.
\end{remark}

\N{Covectors}
We can also consider ``covectors'' or ``dual vectors'' (which eat in a
4-vector and produce a [real] number). These have components with
downstairs indices $B_{\mu}\vec{f}^{\mu}$ where $\vec{f}^{\mu}$ are the
co-basis covectors.

\M
Normally we can transform the components of a four-vector
$\vec{A}=A^{\mu}\vec{e}_{\mu}$ into the components of a covector by
using the metric $g_{\mu\nu}$ in the usual way:
\begin{equation}
A_{\nu}\vec{f}^{\nu} = (g_{\mu\nu}A^{\mu})\vec{f}^{\nu}.
\end{equation}
We can do the same, but for special relativity the metric is denoted
$\eta_{\mu\nu}$.\marginpar{\footnotesize Minkowski metric $\eta_{\mu\nu}$}
We call this the \define{Minkowski metric} and in Cartesian coordinates
has components
\begin{equation}
\eta_{\mu\nu} = \begin{pmatrix}-1 & 0 & 0 & 0\\
0 & 1 & 0 & 0\\
0 & 0 & 1 & 0\\
0 & 0 & 0 & 1
\end{pmatrix}.
\end{equation}
This is the so-called \define{East-Coast Convention} (particle
physicists multiply this by $-1$ and that's the \emph{West-coast convention}).

\N{Magnitude of Four-Vectors}
When $a^{\mu}$ is any four-vector, its magnitude is the scalar quantity
given by:
\begin{equation}
\|a^{\mu}\|^{2} := \eta_{\mu\nu}a^{\mu}a^{\nu}.
\end{equation}

\begin{definition}
The \define{Four-Position} is the four-vector with components
\begin{equation*}
x^{\mu} = (ct, x^{1}, x^{2}, x^{3}).
\end{equation*}
In Cartesian coordinates, $x^{\mu} = (ct, x, y, z)$.
\end{definition}

\N{Events}
Events in spacetime are described using 4-position vectors. This is an
idealization that events have no ``duration''. We could handle events
with some duration by having one 4-position for the ``start'' and
another 4-position for the ``end''.

\begin{definition}
When we have two 4-position vectors $\vec{A}_{1}=(ct_{1},\vec{r}_{1})$
and $\vec{A}_{2}=(ct_{2},\vec{r}_{2})$, we define the
\define{Displacement Four-Vector} as the four-vector
\begin{equation*}
\Delta\vec{A} = (c\,\Delta t,\Delta\vec{r}) = \vec{A}_{2} - \vec{A}_{1}.
\end{equation*}
For an \define{Infinitesimal Displacement Four-Vector} (or
\emph{Differential Four-Position}), we write
$\D\vec{A}$.
\end{definition}

\M
Suppose we have two events in spacetime separated by an infinitesimal
displacement 4-vector $\D x^{\mu}$. Special relativity demands the
infinitesimal interval,
\begin{equation}
(\D s)^{2} := \eta_{\mu\nu}\,\D x^{\mu}\,\D x^{\nu} = -c^{2}(\D t)^{2} + (\D x)^{2} + (\D y)^{2} + (\D z)^{2},
\end{equation}
must be the same for all inertial observers. Here we use summation
conventions where, when we have an index downstairs and upstairs, we sum
over it (so in our equation, we sum over $\mu$ and $\nu$).

\begin{definition} Let $a^{\mu}$ be a 4-vector.
  \begin{enumerate}
  \item If $\eta_{\mu\nu}a^{\mu}a^{\nu} < 0$, then we call $a^{\mu}$ \define{Time-like}.
  \item If $\eta_{\mu\nu}a^{\mu}a^{\nu} = 0$, then we call $a^{\mu}$ \define{Light-like}.
  \item If $\eta_{\mu\nu}a^{\mu}a^{\nu} > 0$, then we call $a^{\mu}$ \define{Space-like}.
  \end{enumerate}
\end{definition}

\begin{remark}
To see the motivation for these definitions, consider the trajectory of
a photon moving along the $x$-axis in Cartesian coordinates. It would be
$\vec{\gamma}(t)=(ct,ct,0,0)$ and its displacement from the origin would
be always zero for any $t\in\RR$. Therefore, the displacement would be
light-like.

For time-like vectors, consider the displacement 4-vector
$\vec{\alpha}(t)=(ct,0,0,0)$ which stays at the spatial origin. We see
its magnitude is $-c^{2}t^{2}<0$.

For space-like vectors, consider the displacement 4-vector
$\vec{\beta}(t)=(0,ct,0,0)$. Its magnitude is $c^{2}t^{2}>0$.
\end{remark}

\begin{definition}[Minkowski spacetime]
We write $\RR^{3,1}$ for \define{Minkowski Spacetime}, i.e., $\RR^{4}$
equipped with the Minkowski metric $\eta$.
\end{definition}

\begin{remark}
If we were using $+---$ signature conventions, Minkowski space would be
$\RR^{1,3}$. 
\end{remark}

\subsection{Invariance of Differential Displacement}

\M We have the obvious spatial rotations and spatial translations leave
the differential displacement $\D s^{2}$ invariant, since this is a
carry-over from Euclidean geometry. The more interesting case is the
``rotation'' of time and space. We will restrict focus to $\RR^{1,1}$
for simplicity, but the reasoning generalizes to $\RR^{n,1}$ by
composing with rotations to make the ``direction of motion'' a spatial
direction.

%% \M Let us restrict focus to $\RR^{1,1}$ for simplicity. Suppose we
%% change coordinates
%% \begin{equation}
%% \begin{pmatrix}c\,\D t'\\\D x'
%% \end{pmatrix}
%% = \begin{pmatrix}a & b\\k & d
%% \end{pmatrix}
%% \begin{pmatrix}c\,\D t\\\D x
%% \end{pmatrix}.
%% \end{equation}
%% Then writing $(\D s)^{2}$ in the new coordinates:
%% \begin{calculation}
%%   (\D s)^{2}
%% \step{by definition}
%%   -c^{2}(\D t')^{2} + (\D x)^{2}
%% \step{plugging in the values of $\D t'$ and $\D x'$}
%%   -c^{2}(a\,\D t + b\,\D x)^{2} + (k\,\D t + d\,\D x)^{2}
%% \step{expanding terms}
%%   -c^{2}[a^{2}\,(\D t)^{2} + 2ab\,\D t\D x + b^{2}\,(\D x)^{2}]
%%   +[k^{2}\,(\D t)^{2} + 2kd\,\D t\D x + d^{2}\,(\D x)^{2}]
%% \step{collecting terms}
%%   -(c^{2}a^{2} - k^{2})(\D t)^{2} - 2(c^{2}ab - kd)\,\D t\D x + (-c^{2}b^{2}+d^{2})(\D x)^{2}.
%% \end{calculation}
%% Comparing to $(\D s)^{2} = -(\D t)^{2} + (\D x)^{2}$, we have the
%% conditions:
%% \begin{subequations}
%% \begin{align}
%% (c^{2}a^{2} - k^{2}) &= 1\\
%% (c^{2}ab - kd) &= 0\\
%% (-c^{2}b^{2}+d^{2}) &= 1.
%% \end{align}
%% \end{subequations}

%% \M
%% Specifically, when changing from one inertial frame (at rest) to another
%% (moving at speed $v$ in the $x$-direction relative to the first frame),
%% we expect
%% \begin{equation}
%% x' = x - vt.
%% \end{equation}
%% There may be a relativistic correction, a factor $\alpha(v)$ such that
%% for $v\ll c$ we have $\alpha(v)\approx 1$. Then we use
%% \begin{equation}
%% x' = \alpha(v)(x - vt).
%% \end{equation}
%% This gives us
%% \begin{equation}
%% \begin{pmatrix}c\,\D t'\\\D x'
%% \end{pmatrix}
%% = \begin{pmatrix}a & b\\-\alpha v & \alpha
%% \end{pmatrix}
%% \begin{pmatrix}c\,\D t\\\D x
%% \end{pmatrix}.
%% \end{equation}
%% This lets us substitute $k=-\alpha(v)v/c$ and $d=\alpha(v)$. This gives us
%% the system of equations
%% \begin{subequations}
%% \begin{align}
%% (c^{2}a^{2} - v^{2}\alpha^{2}) &= c^{2}\\
%% (c^{2}ab + \alpha^{2}v) &= 0\\
%% (-c^{2}b^{2}+\alpha^{2}) &= 1.
%% \end{align}
%% \end{subequations}
%% This is 3 equations in 3 unknowns, which has a solution.
%% \begin{subequations}
%% \begin{align}
%% \alpha(v) &= \frac{-v/c}{\sqrt{1 - v^{2}/c^{2}}}\\
%% \intertext{and}
%% a = b &= \frac{1}{\sqrt{1 - v^{2}/c^{2}}}.
%% \end{align}
%% \end{subequations}

\N{Problem statement}
Given an inertial observer at rest, and another observer moving with
constant speed $v$, how do the coordinate change from the observer at
rest to the moving observer?

\N{Desiderata}
We want, for $v^{2}\ll c^{2}$, to recover Galilean transformations. That
is, $x' \approx x - vt$ (where primed coordinates are those relative to
the moving observer).

\M Let me try this again. We consider an inertial reference frame moving
with velocity $v$ in the $x$-direction relative to an inertial reference
frame at rest. Then the new coordinates are
\begin{subequations}
\begin{align}
ct' &= a_{11}ct + a_{12}x\\
x' &= a_{22}(x - vt).
\end{align}
\end{subequations}
This reduces to the familiar Galilean transformation for $v\ll c$, which
we demand $a_{22}\approx 1$. Now, for an infinitesimal displacement, its
components are
\begin{subequations}
\begin{align}
c\,\D t' &= a_{11}c\,\D t + a_{12}\,\D x\\
\D x' &= a_{22}(\D x - v\,\D t).
\end{align}
\end{subequations}
Demanding the invariance of the infinitesimal displacement is the same
as the condition:
\begin{equation}
-c^{2}(\D t')^{2} + (\D x')^{2} =-c^{2}(\D t)^{2} + (\D x)^{2}.
\end{equation}
Expanding the primed quantities gives us
\begin{equation}
-(a_{11}^{2}c^{2} - a_{22}^{2}v^{2})(\D t)^{2}
+ 2 (a_{22}^{2}v - a_{11}a_{12}c)\,\D x\,\D t
+ (a_{22}^{2} - a_{12}^{2})(\D x)^{2} =-c^{2}(\D t)^{2} + (\D x)^{2}.
\end{equation}
This gives us the system of 3 equations in 3 unknowns:
\begin{subequations}
\begin{align}
-(a_{11}^{2}c^{2} - a_{22}^{2}v^{2}) &= -c^{2}\\
2 (a_{22}^{2}v - a_{11}a_{12}c) &= 0\\
(a_{22}^{2} - a_{12}^{2}) &= 1.
\end{align}
\end{subequations}
We can solve this system:
\begin{subequations}
\begin{align}
a_{11} = a_{22} &= \frac{1}{\sqrt{1 - v^{2}/c^{2}}}\\
a_{12} &= \frac{-v/c}{\sqrt{1 - v^{2}/c^{2}}}.
\end{align}
\end{subequations}
The standard notation is
\begin{equation}
\gamma(v) = \frac{1}{\sqrt{1 - v^{2}/c^{2}}}
\end{equation}
and
\begin{equation}
\beta(v) = \frac{v}{c}.
\end{equation}
Then our transformation is
\begin{equation}
\begin{pmatrix}ct'\\ x'
\end{pmatrix}
=\begin{pmatrix}\gamma & -\gamma\beta\\
-\gamma\beta & \gamma
\end{pmatrix}
\begin{pmatrix}ct\\ x
\end{pmatrix}.
\end{equation}
This transformation is called a \define{Lorentz Boost}.

\M
Observe when $\beta=1/2$ that
\begin{equation}
\gamma = \frac{2}{\sqrt{3}}\approx 1.1547.
\end{equation}
In other words, for an inertial observer moving half the speed of light,
there is a deviation approximately 15\% from non-relativistic values.

At the Large Hadron Collider, protons are accelerated to speeds of about
$99.9999991\%$ the speed of light --- that is, $1-\beta\approx9\times10^{-9}$.
This gives us $\gamma\approx7453$.

\begin{definition}
We call a velocity \define{Ultra-Relativistic} when $1-\beta\ll1$.
\end{definition}

\begin{remark}
For ultra-relativistic velocities, we see that $1+\beta\approx2$, and
therefore
\begin{equation}
\gamma = \frac{1}{\sqrt{(1 + \beta)(1 - \beta)}}
\approx\frac{1}{\sqrt{2(1 - \beta)}}.
\end{equation}
Simple algebra gives us
\begin{equation}
1 - \beta\approx\frac{1}{2\gamma^{2}}.
\end{equation}
\end{remark}

\begin{exercise}
What is the error in the approximation
$1-\beta\approx(2\gamma^{2})^{-1}$ for $\beta=9/10$? For $\gamma=2$? For $\beta=99/100$?
\end{exercise}

\begin{exercise}
For $0\leq\beta\leq0.9$, find a linear approximation $L(\beta)$ for
$\gamma(\beta)$ such that $L(0)=\gamma(0)$ and
$L(0.9)=\gamma(0.9)$. What is the $L^{2}$-norm of the residual for this approximation?
Is there a better linear approximation?
\end{exercise}

\subsection{Light Cones and Causality}

\begin{definition}\label{defn:relativity:light-cone}
Let $a^{\mu}$ be any event. We define the \define{Lightcone} for $a^{\mu}$
to be the set of events light-like separated from $a^{\mu}$,
\begin{equation}
\mathscr{C} = \{\,b^{\mu}\in\RR^{3,1} \mid b^{\mu}~\mbox{is light-like separated from}~a^{\mu}\,\}.
\end{equation}
We can separate the light-cone in two: events in the future and events
in the past, giving us the \define{Future Light Cone}
\begin{equation}
\mathscr{C}^{+} = \{\,b^{\mu}\in\mathscr{C} \mid b^{0} > a^{0}\,\},
\end{equation}
and the \define{Past Light Cone} consisting of events preceding $a^{\mu}$,
\begin{equation}
\mathscr{C}^{-} = \{\,b^{\mu}\in\mathscr{C} \mid b^{0} < a^{0}\,\}.
\end{equation}
We can take the \define{Closed Light Cone} to consist of all light-like
separated events \emph{and} all time-like separated events, and denote
it by
\begin{equation}
\overline{\mathscr{C}} = \{\,b^{\mu}\in\RR^{3,1} \mid \eta_{\mu\nu}(b^{\mu}-a^{\mu})(b^{\nu}-a^{\nu})\leq0\,\}.
\end{equation}
\end{definition}

\M
The only possible causal influences for an event $a^{\mu}$
\emph{must lie within the past causal light cone} for the event, since
nothing can travel faster than light. For this reason, we call any
four-vector $b^{\mu}$ \define{Causal-like} if it is not space-like
separated from $a^{\mu}$, i.e., if the displacement $b^{\mu} - a^{\mu}$
four-vector is either time-like or light-like.

\subsection{World Lines and Trajectories}

\N{World lines}
A curve $\gamma$ in spacetime is called a \define{World Line}
if its tangent vector is future time-like at each point along the
curve. More generally, we could weaken the condition, and allow tangent
vectors to be causal-like.

Particles move along world lines in special relativity.

\M
If we have a curve $\gamma$ in spacetime such that its tangent vector is
space-like at each point along the curve, we call the curve
\define{Space-like}.

If we have a curve in spacetime such that its tangent vector is
ligh-like at each point along the curve, we call the curve
\define{Light-like}. 


\N{Proper Time}\label{chunk:relativity:proper-time}
The time between two events along a world line, according to the
observer moving along the trajectory, is precisely the
\define{Proper Time} and denoted $\tau$. For an infinitesimal
displacement along the trajectory, the infinitesimal change in proper
time is given by
\begin{equation}
c^{2}\,(\D\tau)^{2} = -(\D s)^{2}.
\end{equation}
The proper time interval along a trajectory is given by the integral
\begin{equation}
\Delta\tau = \int_{\gamma}\D\tau =
\int_{\gamma}\frac{\sqrt{-\eta_{\mu\nu}\,\D x^{\mu}\,\D x^{\nu}}}{c}.
\end{equation}

\N{Parametrizing World Lines}
We parametrize a world line by its proper time, and write it as
$x^{\mu}(\tau)$. This gives us a smooth family of four-positions for a
physical body.

Care must be taken when working with world lines for photons (or other
bodies moving at the speed of light), since $\D s = 0$ and therefore
$\D\tau=0$. We need to work with an affine parameter $\lambda$, but
physicists usually just reason along the lines of
\begin{equation}
(\D s)^{2} = 0 = -c^{2}\,(\D t)^{2} + (\D\vec{r})^{2}\implies c^{2}(\D t)^{2}=(\D\vec{r})^{2}.
\end{equation}
This is ``morally right'' but mathematically wrong.

\begin{definition}
Let $x^{\mu}(\tau)$ be a world line. We define its \define{Four-Velocity}
as the four-vector
\begin{equation*}
U^{\mu}(\tau) = \frac{\D x^{\mu}(\tau)}{\D\tau}.
\end{equation*}
\end{definition}

\begin{remark}
Whenever we are tempted to take the time derivative of a quantity, we
really want to take the derivative with respect to \emph{proper time}.
\end{remark}

\N{Lorentz Factor}
We have the familiar Newtonian 3-velocity given by
\begin{equation}
v^{i} = \frac{\D x^{i}}{\D t}.
\end{equation}
We can relate the coordinate time $x^{0}=ct$ with proper time $\tau$ by
the \define{Lorentz Factor} (which is a function of the Newtonian 3-velocity),
\begin{equation}
  \begin{split}
\gamma(\vec{v}) &= \frac{\D t}{\D\tau} = \left(1 - \frac{\eta_{ij}v^{i}v^{j}}{c^{2}}\right)^{-1/2}\\
&=\frac{c}{\sqrt{-\eta_{\mu\nu}U^{\mu}U^{\nu}}}.
  \end{split}
\end{equation}
This allows us to relate the familiar Newtonian 3-velocity
to the four-velocity by
\begin{equation}
U^{\mu} = (c, \gamma(\vec{v})\vec{v}).
\end{equation}

\N{Four-Acceleration}
We can define the four-acceleration analogous to how we defined the
four-velocity as
\begin{equation}
A^{\mu} := \frac{\D U^{\mu}}{\D\tau}.
\end{equation}

\begin{definition}
The \define{Instantaneous Rest Frame} is the frame in which the 3-speed vanishes $v=0$.
\end{definition}

\M
In the instantaneous rest frame we have $v=0$ so $\gamma=1$. The
four-velocity has components $V^{\mu}=(c,\vec{0})$ and the
four-acceleration has components $A^{\mu}=(0,\vec{a})$. Then:
\begin{enumerate}
\item $\eta_{\mu\nu}A^{\mu}A^{\nu}=a^{2}$ where $a$ is the acceleration
  measured in the instantaneous rest frame (that is, the acceleration
  felt by the body);
\item $\eta_{\mu\nu}U^{\mu}U^{\nu}=c^{2}$;
\item $\eta_{\mu\nu}A^{\mu}U^{\nu}=0$ (which can be obtained by
  differentiating the previous result with respect to $\tau$).
\end{enumerate}

\begin{exercise}
Let $v$ be the Newtonian speed of an object, $v = |\D\vec{x}/\D t|$.
Prove
\begin{equation*}
\frac{\D\gamma}{\D t} = \frac{\gamma^{3}v}{c^{2}}\frac{\D v}{\D t}.
\end{equation*}
\end{exercise}

\begin{exercise}
Let $X^{\mu}$ be the position 4-vector, $t$ be coordinate time. Prove
the 4-acceleration may be written as:
\begin{equation*}
A^{\mu} = \gamma\frac{\D}{\D t}\left(\gamma\frac{\D}{\D t}X^{\mu}\right)
\end{equation*}
\end{exercise}

\N{Constant Acceleration}
Consider a particle moving along the $x$-axis with constant acceleration
$a$ as measured in its rest frame at each point.

Let $U^{\mu}=(c\dot{t},\dot{x},0,0)$ where $\dot{x}=\D x/\D\tau$ and
$\dot{t}=\D t/\D\tau$. Then $A^{\mu}=(c\ddot{t},\ddot{x},0,0)$. We have
\begin{subequations}
\begin{align}
-c^{2} &= \eta_{\mu\nu}U^{\mu}U^{\nu} = -c^{2}\dot{t}^{\,2} + \dot{x}^{2}\\
\intertext{and}
a^{2} &= \eta_{\mu\nu}A^{\mu}A^{\nu} = -c^{2}\ddot{t}^{\,2} + \ddot{x}^{2}.
\end{align}
\end{subequations}
Differentiating the first equation gives us $c^{2}\dot{t}\ddot{t}=\dot{x}\ddot{x}$,
then square both sides
$c^{4}\dot{t}^{\,2}\ddot{t}^{\,2}=\dot{x}^{2}\ddot{x}^{2}$, and then
we can eliminate $\dot{x}^{2}=c^{2}\dot{t}^{2}-c^{2}$ and
$\ddot{x}^{\,2}=a^{2}+c^{2}\ddot{t}^{\,2}$ to give us
\begin{equation}
c^{4}\dot{t}^{\,2}\ddot{t}^{\,2}=(c^{2}\dot{t}^{\,2}-c^{2})(a^{2}+c^{2}\ddot{t}^{\,2}).
\end{equation}
We subtract $c^{2}\ddot{t}^{\,2}(\dot{t}^{\,2}-1)$ from both sides to
obtain
\begin{equation}
c^{2}\ddot{t}^{\,2} = a^{2}(\dot{t}^{\,2} - 1)\implies\ddot{t}=\frac{a}{c}\sqrt{\dot{t}^{\,2} - 1}.
\end{equation}
We can solve this differential equation with the initial condition
$t(\tau=0)=0$ and $x(\tau=0)=0$, first integrating to find
\begin{equation}
\dot{t} = \cosh(a\tau/c),
\end{equation}
then integrating again to find
\begin{equation}
t = \frac{c}{a}\sinh(a\tau/c).
\end{equation}
Then from $\dot{x}^{2} = c^{2}(\dot{t}^{\,2}-1)$ we find $\dot{x} = c\sqrt{\cosh^{2}(a\tau/c)-1}=c\sinh(a\tau/c)$
which can be integrated to give us
\begin{equation}
x = \frac{c^{2}}{a}\cosh(a\tau/c).
\end{equation}

\begin{remark}
  Observe that the magnitude of the spatial components for the
  4-velocity (the ``Newtonian speed'') is:
  \begin{equation}
v(\tau) = \frac{\D x}{\D t} = \frac{\dot{x}}{\dot{t}} = c\tanh(a\tau/c)\leq c.
  \end{equation}
\end{remark}

\N{Example}
How do we interpret this? Well, if a space ship leaves a planet at
constant acceleration, and the planet is at rest in the $(x,t)$
coordinates at $x=c^{2}/a$ at $t=0$, then afer some proper time $\tau$
has elapsed the space ship will be located at $x=(c^{2}/a)\cosh(a\tau/c)$.
The time elapsed relative to a clock on the planet will read $t=(c/a)\sinh(a\tau/c)$
where $\tau$ is the time elapsed relative to a clock on the space ship.
For $a\approx g$ and $\tau\approx 10~\mbox{years}$, then $(c/a)=1.03092~\mbox{year}\approx 1~\mbox{year}$
and
\begin{equation}
t\approx\frac{1}{2}\exp(10)~\mbox{years}\approx 11,000~\mbox{years}.
\end{equation}
Observe that
\begin{equation}
x\approx\frac{1}{2}\exp(10)~\mbox{light years}\approx11,000~\mbox{light years}.
\end{equation}
Thus, although only 10 years elapsed on the space ship moving at 1~g,
relative to clocks on the planet which launched the space ship it would
appear to have been 11,000 years. In Newtonian mechanics, the space ship
would be about $50$ light years away (since $a\approx c~\mbox{year}^{-1}$).

\begin{remark}
This may seem counter-intuitive (and it is), but we should emphasize it
is difficult to have a space ship move at a constant 1~g acceleration.
Indeed, \emph{constant acceleration} is the bit belonging to science
fiction (at the moment).
\end{remark}

\begin{exercise}
In Robert Rath's \textit{The Infinite and the Divine}, a super-advanced
species of aliens (who existed hundreds of millions of years ago)
transferred their consciousness into robots. This problem concerns one
of their spaceship's trajectory using their advanced technology.

We are told the initial velocity of the spaceship is $1000$ leagues per
hour, but after a decade [literally 10 years] the spaceship reaches a
billion [$1.00\times10^{9}$] leagues per hour. Note: 1 league per hour
is approximately 1.54333 meters per second.
\begin{enumerate}
\item Assume constant acceleration. Find the constant acceleration,
  assuming the universe is Newtonian, and forces acting on the spaceship
  are negligible.
\item Using the solution to the previous problem, how far did the
  spaceship travel? The story suggests it is the distance between
  several stars. If there is 5 light years between stars in the galaxy,
  is this reasonable? If not, how fast would the spaceship have to
  travel?
\item If we allow for special relativity to apply, then time dilation
  presumably occurs. Suppose the ten years in our problem refers to the
  proper time as measured by the ship's chronometer [clock]. Does this
  affect the distance the ship travels? If so, how far does this ship
  travel?
\end{enumerate}
\end{exercise}

\N{References}
For accelerated observers, Chapter 6 of Misner, Thorne, and
Wheeler~\cite{Misner:1973prb} is quite good.

\endinput
\section{Dynamics}

\M
Often the dynamics of particles are omitted in discussions of special
relativity, because things getting complicated conceptually.

\begin{definition}
Let $x^{\mu}(\tau)$ be the world line for a massive body.
Then the \define{Rest Mass} (or \emph{invariant mass}) for the body is
the mass $m_{0}$ as measured in its reference frame.
\end{definition}

\begin{remark}
Some authors use a notion of \emph{relativistic mass}
$m = \gamma(\vec{v})m_{0}$, which is mildly controversial. 
\end{remark}

\N{Four-Momentum}\label{chunk:relativity:four-momentum}
For a massive particle of rest mass $m_{0}$, its \define{Four-Momentum}
$\vec{P}$ is defined as the product of the rest mass and its
four-velocity, i.e.,
\begin{equation}
\vec{P} := m_{0}\vec{U}.
\end{equation}
We can write out its explicit components
\begin{equation}
\vec{P} = m_{0}\gamma(\vec{v})(c, \vec{v}) = (E/c, \vec{p}).
\end{equation}
Here the total energy of the moving particle is given by
\begin{equation}
E = \gamma(\vec{v})m_{0}c^{2},
\end{equation}
and the total (relativistic) 3-momentum is
\begin{equation}
\vec{p} = m_{0}\gamma(\vec{v})\vec{v}.
\end{equation}

\N{Energy--Momentum Relation}\label{chunk:relativity:mass-shell-relation}
We have the energy--momentum relation (or \define{Mass--Shell Relation})
be
\begin{equation}
E^{2} = c^{2}\vec{p}\cdot\vec{p} + \bigl(m_{0}c^{2}\bigr)^{2}.
\end{equation}
Equivalently, we have this relation describe the magnitude of the
four-momentum as a constant:
\begin{equation}
\eta_{\mu\nu}p^{\mu}p^{\nu} = -m_{0}^{2}c^{2}.
\end{equation}

\begin{ddanger}
In special relativity, we can meaningfully talk about the Center-of-Mass
reference frame for a system of particles. This is especially useful for
scattering problems. However, if we tried to carry this notion over to
General Relativity, then we run into problems because it is a nonlocal
concept. 
\end{ddanger}

\begin{exercise}
From the mass-shell relation and $\vec{P} = (E/c, \vec{p})$, deduce
$E = \gamma(\vec{v})m_{0}c^{2}$ and 
$\vec{p}\cdot\vec{p}=\pm(\gamma(\vec{v})^{2}-1)m_{0}^{2}c^{2}$.
Determine the correct sign in that second relation.
\end{exercise}

\begin{exercise}
Suppose we have two massive particles with four-momenta $\vec{P}_{1}$
and $\vec{P}_{2}$ and relative speed $v$. Determine
$\vec{P}_{1}\cdot\vec{P}_{2}=\eta_{\mu\nu}P^{\mu}_{(1)}P^{\nu}_{(2)}$ in
terms of their rest mass $m_{01}$ and $m_{02}$ and relative speed $v$.
Hint: if $\vec{P}_{1}=\vec{P}_{2}$ and $v=0$, then you should recover
the mass--shell relation.
\end{exercise}

\N{Four-Force}
We can define the four-force as the four-vector
\begin{equation}
\vec{F} = \frac{\D\vec{P}}{\D\tau}.
\end{equation}
As an immediate consequence of the mass--shell relation, we find
\begin{equation}
\vec{F}\cdot\vec{P} = \eta_{\mu\nu}F^{\mu}P^{\nu} = 0.
\end{equation}

\N{Lagrangian for Point-Particle}\label{chunk:relativity:lagrangian-for-point-particle}
We can write the Lagrangian for a massive point-particle with rest mass
$m_{0}$ as
\begin{equation}
L = -cm_{0}\sqrt{-\eta_{\mu\nu}\dot{x}^{\mu}\dot{x}^{\nu}} - V
\end{equation}
where $\dot{x}^{\mu} = \D x^{\mu}/\D\tau = U^{\mu}$, and $V$ is the potential
energy term. The action is then 
\begin{equation}
I = \int L\,\D\tau.
\end{equation}
Varying the action with respect to $\delta x^{\mu}$ then gives us the
equations of motion.

For massless particles, care must be taken with the parametrization, as
well as using its four-momentum to write
\begin{equation}
L = -c\sqrt{-\eta_{\mu\nu}P^{\mu}P^{\nu}} - V.
\end{equation}
We can use the relation (which holds for both massive and massless particles):
\begin{equation}
\frac{\D x^{\mu}}{\D t} = \frac{P^{\mu}}{P^{0}}.
\end{equation}

\begin{danger}
This is the correct Lagrangian to work with, especially if we want to
quantize it. There is some subtlety with it, which we should confess
openly: it is a constrained system. To see this, compute the Hamiltonian
for a free massive relativistic particle. It will vanish. This is
because time is a coordinate (proper time is a parameter), and its
conjugate momentum is ``the Hamiltonian''. So we end up with a
constraint. 
\end{danger}

\begin{ddanger}
Some authors insist that canonical mechanics for special relativistic
systems ``breaks Lorentz invariance'', which is not really true. If
you've picked $\mu=0$ to be the time component for four-vectors, then
you've also ``broken Lorentz invariance'' just as much as canonical
mechanics has. We can describe a Lorentz boost as a canonical
transformation (which preserves Lorentz invariance as much as anything
else). This is just offered as a lazy and sloppy justification for using
the path integral formalism, which makes no coherent sense.
\end{ddanger}

\begin{exercise}\label{xca:relativity:canonical-analysis-of-free-particle}
Work out the canonical analysis for the Lagrangian of a point-particle
in special relativity. Recall, the steps are:
\begin{enumerate}
\item Find the conjugate momenta $p_{\mu}$ for $\dot{x}^{\mu}=\D x^{\mu}/\D\tau$
\item Find any primary constraints
\item Rewrite the Lagrangian in terms of $x^{\mu}$ and $p_{\mu}$
\item Write the Hamiltonian $H = p_{\mu}\dot{x}^{\mu}-L$.
\item Find if there are any first-class constraints.
\item Write down the extended Hamiltonian (or total Hamiltonian).
\end{enumerate}
\end{exercise}

\endinput
\section{Group Theoretic Description}\label{section:relativity:group-theoretic-description}

\begin{definition}\index{Group!Lorentz}
The \define*{(Homogeneous) Lorentz Group}\index{Group!Lorentz!Homogeneous} for $\RR^{n,1}$ is the group
generated by spatial rotations and Lorentz boosts --- that is, the indefinite
orthogonal group $\O(n,1)$. A generic element of the Lorentz group is
called a \define*{Lorentz Transformation}.\index{Lorentz transformation}
\end{definition}

\begin{exercise}
Prove that this actually is a group. That is, the composition of Lorentz
transformation are a Lorentz transformation; inverses of Lorentz
transformations are Lorentz transformations.
\end{exercise}

\M
There are 4 connected components to the Lorentz group $\O(3,1)$ which
are related by parity operator $P$ and time reversal operator $T$, which
are represented by the matrices (acting on the ``obvious'' fundamental
representation as):
\begin{subequations}
\begin{align}
P &= \diag(1, -1, -1, -1)\\
T &= \diag(-1, 1, 1, 1).
\end{align}
\end{subequations}

\begin{definition}\index{Group!Poincar\'e}\index{Poincar\'e!Group}\index{Group!Lorentz!Inhomogeneous}\label{defn:relativity:poincare-group}
The \define*{Poincar\'e Group} for $\RR^{n,1}$ is the group generated by
translations in spacetime [by a constant displacement], spatial
rotations, and Lorentz boosts; it is given by the group
$G = \RR^{n,1}\rtimes\O(n,1)$.
\end{definition}

\begin{remark}
Weinberg~\cite{Weinberg:1995mt} calls the Poincar\'e Group the \emph{Inhomogeneous Lorentz Group}.
\end{remark}

\begin{remark}\index{Poincar\'e!Spin group}\index{Group!Poincar\'e!Spin}
We actually want to use the ``universal covering group'' of the
Poincar\'e Group, which is known in the literature as the ``Poincar\'e Spin
Group''.
\end{remark}

\M
We can always consider how an element of $g\in\O(3,1)$ will act on any
4-position $x\in\RR^{3,1}$ ``in the obvious way'' as a Lorentz
transformation. We will write this as $g\cdot x$. Specifically this acts
by means of a $4\times4$ matrix ${\Lambda^{\alpha'}}_{\mu}$ sending
$x^{\mu}\to x^{\alpha'} = f^{\alpha'}(x^{\mu}) = (\Lambda x)^{\alpha'}$.
Explicitly, using Einstein summation convention,
\begin{equation*}
x^{\alpha'} = {\Lambda^{\alpha'}}_{\mu}x^{\mu}.
\end{equation*}

\N{Inverse Transformation}
We can compose Lorentz transformations to find the inverse
transformation ${\Lambda^{\nu}}_{\alpha'}$ which satisfies
\begin{equation}
{\Lambda^{\nu}}_{\alpha'}{\Lambda^{\alpha'}}_{\mu}={\delta^{\nu}}_{\mu}.
\end{equation}
For covariant vectors, we need to use
% We then have
${[\transpose{(\Lambda^{-1})}]^{\mu}}_{\alpha'} = {\Lambda_{\alpha'}}^{\mu}$
be the transpose of the inverse Lorentz transformation for $\Lambda$.
This is because covariant vectors transform under the dual
representation. 


\M
The Lorentz group acts on a scalar field $\varphi(x)$ by
\begin{equation}
g\cdot\varphi(x) = \varphi\bigl(g^{-1}\cdot x\bigr).
\end{equation}
This is because a scalar field is ``just a function'', and this is how
groups act on functions.

\N{Action on Vector Fields}
For a vector field with components $A^{\mu}(x)$, a Lorentz
transformation $g\in\O(3,1)$ acts on this in a ``mixed'' way. We have
the matrix ${\Lambda^{\alpha'}}_{\mu}$ describe the action $(g\cdot x)^{\alpha'} = {\Lambda^{\alpha'}}_{\nu}x^{\nu}$,
which is the change of coordinates $x^{\mu}\to x^{\alpha'}=f^{\alpha'}(x^{\mu})$,
so therefore a contravariant vector field transforms as:
\begin{equation}
[(g\cdot \vec{A})(x)]^{\alpha'} = {\Lambda^{\alpha'}}_{\nu}A^{\nu}(g^{-1}\cdot x).
\end{equation}
It acts on the vector as a whole as we would expect, and on each
component as a function.

For a covariant vector $A_{\mu}(x)$, we see it transforms as
\begin{equation}
[(g\cdot \vec{A})(x)]_{\alpha'} = {\Lambda_{\alpha'}}^{\mu}A_{\mu}(g^{-1}\cdot x)
\end{equation}
where ${\Lambda_{\alpha'}}^{\mu}$ is the matrix inverse of the Lorentz
transformation. It satisfies
\begin{equation}
{\Lambda_{\nu}}^{\alpha'}{\Lambda^{\nu}}_{\beta'} = {\delta^{\alpha'}}_{\beta'}.
\end{equation}

\N{Action on Tensor Fields}
More generally, for a rank $n$ tensor with components $[\mathsf{T}(x)]^{\mu_{1}\cdots\mu_{n}}=T^{\mu_{1}\cdots\mu_{n}}(x)$,
the element $g\in\O(3,1)$ acts as
\begin{equation}
[(g\cdot \tens{T})(x)]^{\alpha_{1}'\cdots\alpha_{n}'}
= {\Lambda^{\alpha_{1}'}}_{\nu_{1}}(\cdots){\Lambda^{\alpha_{n}'}}_{\nu_{n}}
T^{\nu_{1}\cdots\nu_{n}}(g^{-1}\cdot x).
\end{equation}
For covariant indices, we multiply by matrix inverses of $\Lambda$.
These actions are all basic representation theory.

\N{Lorentz Invariance}
We call a quantity $Q(x)$ \define{Lorentz Invariant} if for each $g\in\O(3,1)$
we have
\begin{equation}
g\cdot Q(x) = Q(x).
\end{equation}
For example $\D s^{2}$ is Lorentz invariant.

\endinput
\section{Electromagnetism}

\M
We have reviewed in section~\ref{section:relativity:electromagnetism}
the covariant formalism of electromagnetism. From the perspective of
classical field theory, we now know the ``correct'' way to think of
things is that the field quantity of interest is the 4-potential
$A^{\mu}$ (\S\ref{chunk:relativity:electromagnetism:four-potential}).

\N{Lagrangian Density}
We recover Maxwell's equations using the Lagrangian
\begin{equation}
\mathcal{L} = \frac{-1}{4}F^{\alpha\beta}F_{\alpha\beta} = \frac{-1}{4}\eta^{\alpha\mu}\eta^{\beta\nu}F_{\mu\nu}F_{\alpha\beta}.
\end{equation}
If we work in curved spacetime, we need to multiply by the determinant
of the metric tensor $\sqrt{-\det(g_{\mu\nu})}$, but we will ignore this
factor.

\N{Equations of Motion}
We can now determine the equations of motion for the Lagrangian density.
The Euler--Lagrange equations take the form
\begin{equation}
\partial_{\mu}\frac{\partial\mathcal{L}}{\partial(\partial_{\mu}A_{\nu})}-\frac{\partial\mathcal{L}}{\partial A_{\nu}}=0.
\end{equation}
These will turn out to be:
\begin{equation}
\boxed{\partial_{\mu}F^{\mu\nu} = 0.}
\end{equation}
This is the source-free Maxwell's equations.

\begin{proof}[Proof (slick)]
We can compute the variation of the action directly, ignoring boundary terms,
as
\begin{calculation}
\variation\mathcal{L}
\step{unfolding the definition of the Lagrangian density}
\variation\left(\frac{-1}{4}\eta^{\alpha\mu}\eta^{\beta\nu}F_{\mu\nu}F_{\alpha\beta}\right)
\step{product rule and index gymnastics}
\frac{-1}{4}\eta^{\alpha\mu}\eta^{\beta\nu}F_{\mu\nu}(2\,\variation F_{\alpha\beta})
\step{unfolding the definition of field-strength tensor}
\frac{-1}{2}\eta^{\alpha\mu}\eta^{\beta\nu}F_{\mu\nu}(\variation\partial_{\beta}A_{\alpha}-\variation\partial_{\beta}A_{\alpha})
\step{index gymnastics, antisymmetry of field-strength tensor}
-\eta^{\alpha\mu}\eta^{\beta\nu}F_{\mu\nu}\variation(\partial_{\alpha}A_{\beta})
\step{integration by parts}
\eta^{\alpha\mu}\eta^{\beta\nu}(\partial_{\alpha}F_{\mu\nu})\variation A_{\beta}
+\mbox{(boundary terms)}.
\end{calculation}
This vanishes when
\begin{equation}
\partial_{\mu}F^{\mu\nu} = 0,
\end{equation}
and this is the result from the Euler--Lagrange equations of motion.
\end{proof}

\M
We can unfold the result of the Euler--Lagrange equations for
electromagnetism, and find
\begin{equation}
\partial_{\mu}F^{\mu\nu}=0\iff g^{\alpha\gamma}\partial_{\gamma}F_{\alpha\beta}=0.
\end{equation}
Then
\begin{calculation}
  g^{\alpha\gamma}\partial_{\gamma}F_{\alpha\beta}
\step{unfold definition of field-strength tensor}
g^{\alpha\gamma}\partial_{\gamma}(\partial_{\alpha}A_{\beta}-\partial_{\beta}A_{\alpha})
\step{distributivity, index gymnastics}
\partial^{\alpha}\partial_{\alpha}A_{\beta}-\partial^{\alpha}\partial_{\beta}A_{\alpha}
\step{equations of motion}
0.
\end{calculation}
When we impose the gauge condition $\partial^{\alpha}A_{\alpha}=0$,
we recover the familiar Maxwell equations as a wave equation
$\Box A_{\beta}=0$.

\N{Lagrangian coupled to matter}
We can write down the Lagrangian density for electromagnetism coupled to
some charged matter, recovering the Maxwell equations with some source
as in
Eq~\eqref{eq:relativity:electromagnetism:maxwell-for-electric-field}:
\begin{equation}\label{eq:classical-field-theory:electromagnetism:lagrangian}
\mathcal{L} = \frac{-1}{4}F^{\alpha\beta}F_{\alpha\beta}-4\pi J^{\mu}A_{\mu}.
\end{equation}

\begin{exercise}
Recall (\S\ref{ex:classical-field-theory:noether:canonical-stress-energy})
the notion of the canonical stress--energy tensor. Calculate
$\canonicalStressEnergy_{\mu\nu}$ for the Lagrangian in Eq~\eqref{eq:classical-field-theory:electromagnetism:lagrangian}.

[Hint: your answer should \emph{not} be symmetric --- that is, 
$\canonicalStressEnergy_{\mu\nu}\neq\canonicalStressEnergy_{\nu\mu}$.]
\end{exercise}

\N{Heuristic regarding interactions}\index{Heuristics}
If we want to describe interactions between two fields, or a field and
some matter, then we need to add a term to our Lagrangian of the form:
\begin{equation}
\mathcal{L}_{\text{interaction}}\sim\begin{pmatrix}\mbox{coupling}\\
\mbox{constant}
\end{pmatrix}\begin{pmatrix}\mbox{field}\\
\mbox{quantity}
\end{pmatrix}\begin{pmatrix}\mbox{matter}\\
\mbox{terms}
\end{pmatrix}.
\end{equation}
This is added to the potential term in the Lagrangian.

\subsection{Hamiltonian Formalism}

\M We will work through the calculations of the Symplectic two-form as a
series of exercises, and show it is degenerate. This is a consequence of
gauge symmetries. Then we will work through the Hamiltonian formalism
with its phase space parametrized by initial conditions.

\begin{exercise}
Prove $\displaystyle\variation L = \int\bigl(\variation A_{\beta}\partial_{\alpha}F^{\alpha\beta}+\partial_{0}(-F^{0\beta}\,\variation A_{\beta})\bigr)\,\D^{3}x$.
\end{exercise}

\begin{exercise}
  Prove the Symplectic potential is
  \[ \Theta(\variation A) = \int F^{\beta0}\,\variation A_{\beta}\,\D^{3}x =\int F^{i0}\,\variation A_{i}\,\D^{3}x\]
\end{exercise}

\begin{exercise}
Suppose $A^{\alpha}$ is a solution to the Maxwell's equations. Determine
what the linearized Maxwell equations are for tangent ``vectors''
$\variation A^{\beta}$ with base ``point'' $A^{\alpha}$.
\end{exercise}

\begin{exercise}[Symplectic form]
We will compute the Symplectic two-form for electromagnetism, and verify
it is degenerate when one of the fields is pure gauge.
\begin{enumerate}
\item Show the [naive] Symplectic two-form for electromagnetism $\Omega=\D\Theta$
is
\[\Omega(\variation_{1}A,\variation_{2}A) = \int\bigl((\partial_{t}\variation_{1}A^{i}-\partial^{i}\variation_{1}A_{t})\variation_{2}A_{i}-(\partial_{t}\variation_{2}A^{i}-\partial^{i}\variation_{2}A_{t})\variation_{1}A_{i}\bigr)\,\D^{3}x.\]
\item Consider $\variation_{2}A_{t}=\partial_{t}\Lambda$ and
  $\variation_{2}A_{i}=\partial_{i}\Lambda$ where $\Lambda$ is any
  function with compact support. Show
\[\int(\partial_{t}\variation_{1}A^{i}-\partial^{i}\variation_{1}A_{t})\variation_{2}A_{i}\,\D^{3}x=-\int\Lambda\partial_{i}(\partial_{t}\variation_{1}A^{i}-\partial^{i}\variation_{1}A_{t})\,\D^{3}x.\]
\item Show (if you haven't already) the linearized Maxwell's equations
  includes $\partial^{i}(\partial_{t}\variation A_{i}-\partial_{i}\variation A_{t})=0$.
\item Comparing the last two steps in this exercise, show the integral
  from step 2 vanishes, and this implies $\Omega(\variation_{1}A,\variation_{2}A) =0$
 (i.e., $\Omega$ is degenerate).
\end{enumerate}
\end{exercise}

\N{Initial Data}
Assuming we have picked some time-slicing and we have some initial data
\begin{equation}
\phi:=-A_{t}(\vec{x},t=0),\quad\mbox{and}\quad Q_{i}:=A_{i}(\vec{x},t=0),
\end{equation}
we can find the canonically conjugate momentum to $Q_{i}$ as,
\begin{subequations}
\begin{align}
P^{i} &= \frac{\partial\mathcal{L}}{\partial(\partial_{t}A_{i})}\\
&=\partial_{t}A_{i}-\partial_{i}A_{t}\\
&=\partial_{t}Q_{i}+\partial_{i}\phi.
\end{align}
\end{subequations}

\begin{exercise}
Verify that $\displaystyle\frac{\partial\mathcal{L}}{\partial(\partial_{t}A_{t})}=0$,
and therefore the canonically conjugate momentum for $\phi$ vanishes.
\end{exercise}

\begin{exercise}
Rewrite the Lagrangian as a functional of $Q_{i}$, $P^{i}$, and
$\phi$. Try to write it in ``canonical form'', i.e., as
\[ L[\phi, Q_{i}, P^{i}] = \int \bigl(P^{i}\partial_{t}Q^{i}-\mbox{(something)}\bigr)\,\D^{3}x.\]
[Hint: integration by parts and the divergence theorem are your friends.]
\end{exercise}

\M
The Lagrangian which you ought to obtain from the previous exercises
should be:
\begin{equation}\label{eq:classical-field-theory:electromagnetism:canonical-formalism:lagrangian}
L =
\int\left(P^{i}\partial_{t}Q_{i}-\left[\frac{1}{2}(P_{i}P^{i}+\frac{F_{ij}F^{ij}}{2}) + \phi\,\partial_{i}P^{i}\right]\right)\D^{3}x.
\end{equation}
Observe then that the equations of motion for $\phi$ are precisely
Gauss's Law:
\begin{equation}
\partial_{i}P^{i} = 0.
\end{equation}
We interpret $\phi$ as a Lagrange multiplier.

\begin{exercise}
Using the Lagrangian you should have computed (or lifted from the
previous chunk), prove the Euler--Lagrange equations:
\begin{equation}
\frac{\variation L}{\variation P^{i}}-\frac{\D}{\D t}\frac{\variation L}{\variation\partial_{t}P^{i}}=0,\quad
\frac{\variation L}{\variation Q^{i}}-\frac{\D}{\D t}\frac{\variation L}{\variation\partial_{t}Q^{i}}=0,\quad
\frac{\variation L}{\variation\phi}=0,
\end{equation}
are equivalent to the Maxwell equations.
\end{exercise}

\N{Hamiltonian}
We find the Hamiltonian functional by inspection of
Eq~\eqref{eq:classical-field-theory:electromagnetism:canonical-formalism:lagrangian}
to be:
\begin{equation}
H = \int\left(\frac{1}{2}(P_{i}P^{i}+\frac{F_{ij}F^{ij}}{2}) + \phi\,\partial_{i}P^{i}\right)\D^{3}x.
\end{equation}
The first two terms coincide with our expectations, but the last term
may be surprising.

\begin{ddanger}
When we have a constrained Hamiltonian system, we add the first-class
constraints to the Hamiltonian. This is precisely what's going on with
the Hamiltonian functional as we've written it down. This is studied
thoroughly in Henneaux and Teitelboim~\cite{Henneaux:1992ig}.
\end{ddanger}

\begin{exercise}\index{Hamilton's equations}
The reader can verify Hamilton's equations,
\begin{subequations}
\begin{align}
\partial_{t}Q_{i} &= \frac{\delta H}{\delta P^{i}} = P_{i} - \partial_{i}\phi\\
\partial_{t}P_{i} &= -\frac{\delta H}{\delta P^{i}} = \partial_{i}F^{ij}.
\end{align}
\end{subequations}
\end{exercise}

\begin{exercise}\index{Poisson bracket}
Using the Poisson brackets,
\begin{equation}
\PB{M}{N} = \int\left(\frac{\delta M}{\delta Q_{i}(\vec{x}')}\frac{\delta N}{\delta P^{i}(\vec{x}')}
-\frac{\delta N}{\delta Q_{i}(\vec{x}')}\frac{\delta M}{\delta P^{i}(\vec{x}')}\right)\D^{3}x',
\end{equation}
show the quantity
\begin{equation}
G = - \int\Lambda(\vec{x})\partial_{i}P^{i}\,\D^{3}x
\end{equation}
is the generating function for gauge transformations
\begin{equation}
\variation Q_{i}=\PB{Q_{i}}{G}=\partial_{i}\Lambda,\quad
\variation P^{i}=\PB{P^{i}}{G}=0.
\end{equation}
\end{exercise}

\subsection{Scalar Electrodynamics}

\M
One of the first models we study in quantum field theory is something
called ``scalar electrodynamics''. This is obtained by taking a complex
Scalar field and coupling it to Electromagnetism. Let us review the
pertinent aspects of the complex Scalar field, then let us try to couple
it to electromagnetism.

\subsubsection{Complex Scalar Field}

\M We take 2 real-valued scalar fields
$\varphi_{1}$ and $\varphi_{2}$, then form the complex scalar
field\footnote{This is an abuse of notation, similar to using $z$ and
$\bar{z}$ in complex analysis as the independent coordinates of the
Complex plane.} 
\begin{equation}
\varphi(x) = \frac{\varphi_{1}(x)+\I\varphi_{2}(x)}{\sqrt{2}},\quad\mbox{and}\quad
\varphi^{*}(x) = \frac{\varphi_{1}(x)-\I\varphi_{2}(x)}{\sqrt{2}}.
\end{equation}
Then we couple the complex scalar field to electromagnetism.

\N{Complex Scalar Field}\index{Scalar Field!Complex}
The complex scalar field (also called the \emph{charged Klein--Gordon field})
may be viewed as a mapping
\begin{equation}
\varphi\colon\RR^{3,1}\to\CC.
\end{equation}
The Lagrangian for the complex Scalar field is:
\begin{equation}
\mathcal{L}_{cs} = -(\eta^{\alpha\beta}\partial_{\alpha}\varphi\partial_{\beta}\varphi^{*}+\mu^{2}|\varphi|^{2}).
\end{equation}

\begin{exercise}
Show the Euler--Lagrange equations give you the equations of motion
\begin{subequations}
\begin{align}
\frac{\variation S}{\variation\varphi}=0 &\implies (\partial^{\alpha}\partial_{\alpha}+\mu^{2})\varphi^{*}=0,
\intertext{and}  
\frac{\variation S}{\variation\varphi^{*}}=0 &\implies (\partial^{\alpha}\partial_{\alpha}+\mu^{2})\varphi=0.
\end{align}
\end{subequations}
\end{exercise}

\N{Symmetries of Complex Scalar Field}
We can use Noether's theorem to find that the complex scalar field
admits a continuous symmetry:
\begin{equation}
\varphi_{\lambda}=\E^{\I\lambda}\varphi,\quad
\varphi_{\lambda}^{*}=\E^{-\I\lambda}\varphi^{*},
\end{equation}
where $\lambda\in\RR$ is arbitrary. Since $\lambda$ is a constant, we
see the kinetic term of the Lagrangian is invariant under this
transformation. Similarly, we see $|\varphi|^{2}=\varphi^{*}\varphi$ is
invariant under this transformation.

\begin{remark}\index{Symmetry!Global}\index{Symmetry!Local}
This is a ``global $\U(1)$'' symmetry. It's ``global'' because the
parameter $\lambda$ is a real number independent of spacetime. (If
$\lambda$ were a function of spacetime $\lambda=\lambda(\vec{x},t)$,
then we would call it a ``local'' symmetry.) It's a $\U(1)$ symmetry
because $\E^{\I\lambda}\in\U(1)$.

Older literature use the term ``gauge transformation of the first kind''\index{Gauge!Transformation!Of the first kind}
instead of ``global symmetry transformation''
\end{remark}

\begin{exercise}
Use Noether's theorem to prove this is a continuous symmetry of the
complex scalar field. Then determine the conserved current $j^{\beta}$
for this symmetry.
\end{exercise}
\begin{exercise}
Prove the Noether charge for the complex Scalar field is
\begin{equation}
Q = \I\int_{V}(\varphi^{*}\partial_{t}\varphi-\varphi\partial_{t}\varphi^{*})\,\D^{3}x.
\end{equation}
\end{exercise}

\N{Sigma Models}\index{Sigma model@$\sigma$ Model}
There is a way to generalize this construction from 2 real Scalar fields
to $N$ real Scalar fields. This is a family of models called
\define{Sigma models} where the scalar fields are components of a smooth
function $\sigma\colon\RR^{3,1}\to\mathcal{M}$ where $\mathcal{M}$ is
usually a Lie group. (There is no significance to the choice of $\sigma$
for scalar fields, and Sigma models refer to this historic artifact of
arbitrary notation.) Then the Lagrangian density for the massless case
is:
\begin{equation}
\mathcal{L} = \frac{1}{2}\sum^{N}_{A,B=1}g_{AB}(\sigma)\partial^{\mu}\sigma^{A}\partial_{\mu}\sigma^{B},
\end{equation}
where $G_{AB}(\sigma)$ is the metric tensor on the field space $\mathcal{M}$,
and $\partial_{\mu}$ are the derivatives on the underlying spacetime
manifold $\RR^{3,1}$. When we include some self-interaction terms, we
obtain a \emph{Nonlinear $\sigma$ Model}.

When $\mathcal{M}=\CC^{2}$, for example, we can show the $\sigma$ model
enjoys an $\SU(2)$ symmetry. Similarly, for $\mathcal{M}=\CC^{n}$, the
$\sigma$ model enjoys an $\SU(n)$ symmetry. These models are useful as
``prolegomenon'' to Yang--Mills theory for the Standard Model.

\begin{remark}
Sigma models were first introduced in \S\S5--6 of Gell-Mann and Levy~\cite{Gell-Mann:1960mvl}.
Initially, $\sigma$ was ``just another scalar field'' in that paper. Later
physicists adopted $\sigma$ as we have introduced it: as a familar of
scalar fields.
\end{remark}

\subsubsection{Charged Scalar Field coupled to Electromagnetism}

\M
Now we can couple the complex Scalar field to Electromagnetism.
The basic idea is we will form the Lagrangian density for scalar
electrodynamics by adding the Lagrangian density for the complex Scalar
field to the Lagrangian density for the Electromagnetic field, plus the
4-current coupling the charged Scalar field to the Electromagnetic field:
\begin{equation}
\begin{split}
\mathcal{L}_{sED}&=\mathcal{L}_{cs}+\mathcal{L}_{EM}+\mathcal{L}_{int}\\
&=-(\eta^{\alpha\beta}\partial_{\alpha}\varphi\partial_{\beta}\varphi^{*}+\mu^{2}|\varphi|^{2})
-\frac{1}{4}F^{\alpha\beta}F_{\alpha\beta}
+4\pi j^{\alpha}A_{\alpha}.
\end{split}
\end{equation}
We just need to determine $j^{\alpha}$ in terms of the complex Scalar
field $\varphi$.

We know from Noether's theorem there is a conserved current for the
$\U(1)$ Symmetry for the complex Scalar field,
\begin{equation}
j^{\alpha} = -\I\eta^{\alpha\beta}(\varphi^{*}\partial_{\beta}\varphi - \varphi\partial_{\beta}\varphi^{*}).
\end{equation}
There is some slight difficulties with the Lagrangian as we have written
it: it is no longer gauge invariant under $A^{\mu}\to
A^{\mu}+\partial^{\mu}\Lambda$.

\N{Minimal coupling}
The ``physically correct way'' to get a gauge-invariant Lagrangian which
still gives the $j^{\alpha}A_{\alpha}$ coupling is rather unintuitive:
we use a different differential operator than $\partial_{\mu}$ in the
charged Scalar field's Lagrangian density.

This is the so-called \define{Minimal Coupling}, where we replace
\begin{subequations}
\begin{equation}
\partial_{\alpha}\varphi\to D_{\alpha}\varphi := (\partial_{\alpha}+\I qA_{\alpha})\varphi,
\end{equation}
and
\begin{equation}
\partial_{\alpha}\varphi^{*}\to D_{\alpha}\varphi^{*} := (\partial_{\alpha}-\I qA_{\alpha})\varphi^{*}.
\end{equation}
\end{subequations}
Here $q$ is a parameter reflecting the coupling strength between the
charged scalar field $\varphi$ and the Electromagnetic field. It's an
example of a \emph{coupling constant}.\index{Coupling constant}
Then we modify the complex Scalar field's Lagrangian density to use
these gauge covariant derivatives,
\begin{equation}\label{eq:classical-field-theory:electromagnetism:lagrangian-density-for-complex-scalar-field-using-gauge-covariant-derivatives}
\mathcal{L}_{cs} = -\eta^{\alpha\beta}D_{\alpha}\varphi^{*}D_{\beta}\varphi-\mu^{2}|\varphi|^{2}.
\end{equation}
This Lagrangian density yields field equations which are the usual wave
equations plus some modifications involving the electromagnetic
potential.

\begin{exercise}
Compute the Euler--Lagrange equations for $\varphi$ and $\varphi^{*}$ using the Lagrangian density from Eq~\eqref{eq:classical-field-theory:electromagnetism:lagrangian-density-for-complex-scalar-field-using-gauge-covariant-derivatives}.
\end{exercise}

\M
Now observe, under a gauge transformation of the electromagnetic
4-potential
\begin{subequations}
\begin{equation}
A_{\alpha}\to A_{\alpha}+\partial_{\alpha}\Lambda,
\end{equation}
for the gauge covariant derivatives of the complex Scalar field to
remain invariant under these gauge transformations, we need:
\begin{align}
  \varphi&\to\E^{-\I q\Lambda}\varphi,\\
  \intertext{and}
  \varphi^{*}&\to\E^{\I q\Lambda}\varphi^{*}.
\end{align}
\end{subequations}
The reader can verify that the gauge covariant derivatives of the
complex Scalar field then transform as
\begin{subequations}
\begin{align}
D_{\alpha}\varphi &\to\E^{-\I q\Lambda}D_{\alpha}\varphi,\\
D_{\alpha}\varphi^{*} &\to\E^{\I q\Lambda}D_{\alpha}\varphi^{*}.
\end{align}
\end{subequations}
We can see that the kinetic term for the modified complex Scalar
Lagrangian density remains invariant under these transformations.

\M
Since the electromagnetic interaction with the complex Scalar fields are
``swept into'' the gauge covariant derivatives, we can write the
Lagrangian density for the scalar electrodynamic theory as:
\begin{subequations}
\begin{equation}
\mathcal{L}_{sED} = \frac{-1}{4}F^{\alpha\beta}F_{\alpha\beta} - \eta^{\alpha\beta}D_{\alpha}\varphi^{*}D_{\beta}\varphi-\mu^{2}|\varphi|^{2}.
\end{equation}
When we expand the gauge covariant derivatives in this Lagrangian
density, we have:
\begin{equation}
\mathcal{L}_{sED} = \frac{-1}{4}F^{\alpha\beta}F_{\alpha\beta} - \eta^{\alpha\beta}\partial_{\alpha}\varphi^{*}\partial_{\beta}\varphi-\mu^{2}|\varphi|^{2}
+\I q A^{\alpha}(\varphi^{*}\partial_{\alpha}\varphi-\varphi\partial_{\alpha}\varphi^{*}+\I q A_{\alpha}|\varphi|^{2}).
\end{equation}
\end{subequations}
The Euler--Lagrange equations for the Electromagnetic 4-potential are
\begin{equation}
\partial_{\beta}F^{\alpha\beta}=-4\pi J^{\alpha},
\end{equation}
where the current is defined using the gauge covariant derivatives as
\begin{equation}\label{eq:classical-field-theory:sed:charged-current}
J^{\alpha} = -\frac{\I q}{4\pi}(\varphi^{*}D^{\alpha}\varphi-\varphi D^{\alpha}\varphi^{*}).
\end{equation}
This is rather magical, but we could derive the same results using
Noether's first theorem for fields.

\M
We should mention that physicists look at
Eq~\eqref{eq:classical-field-theory:sed:charged-current} and interpret
it as telling us the electromagnetic charge for the complex scalar field
cannot ``exist alone'' in the Scalar field. In an interacting system,
the division between ``source fields'' and ``fields mediating interactions''
is rather artificial and arbitrary. This is physically reasonable (even
if a little surprising). Mathematically this feature emerges from
demanding gauge invariance.

The complex Scalar field is no longer uniquely defined in scalar
electrodynamics: it is subject to a gauge transformation, just like the
electromagnetic 4-potential.

When we have such an interaction, if we want to compute (say) the
electromagnetic field contained in a region $V$, we need a solution
$(A,\varphi)$ of the coupled Maxwell--Scalar equations, then substitute
it into:
\begin{equation}
Q = \frac{1}{4\pi}\int_{V}\I q(\varphi^{*}D_{0}\varphi-\varphi D_{0}\varphi^{*})\,\D^{3}x.
\end{equation}
This charge is conserved and gauge invariant.

\begin{exercise}
Prove the total electric charge $Q$ is conserved \emph{and} gauge invariant.
\end{exercise}

\begin{exercise}
Suppose we had a Lagrangian for complex scalar fields coupled to
electromagnetism of the form
\begin{equation}
\mathcal{L}=\mathcal{L}_{EM}-\frac{1}{2}[\varphi g^{\mu}D_{\mu}\varphi^{*}-\varphi^{*}g^{\mu}D_{\mu}\varphi]-\mu^{2}|\varphi|^{2},
\end{equation}
where $g^{\mu}$ is ``some [constant] vector'', and $D_{\mu}$ is the
gauge covariant derivative.
\begin{enumerate}
\item How must $\varphi$ and $\varphi^{*}$ transform under gauge
  transformations $A_{\alpha}\to A_{\alpha}+\partial_{\alpha}\Lambda$?
\item How do the gauge covariant derivatives $D_{\alpha}\varphi$ and
  $D_{\alpha}\varphi^{*}$ transform under gauge transformations?
\end{enumerate}
\end{exercise}

\subsection{Chern--Simons Theory}

\M
In $2+1$ dimensions (instead of $3+1$ dimensions), we have a particular
action which plays an important role in physics called the Chern--Simons
action named after its discoverer Albert Schwarz:
\begin{equation}
\begin{split}
  \action_{CS}[A] &= \frac{k}{4\pi}\int\epsilon^{\mu\nu\rho}A_{\mu}\partial_{\nu}A_{\rho}\,\D^{3}x\\
&\mbox{``=''}\; \frac{k}{4\pi}\int(A\times\nabla)^{\rho} A_{\rho}\,\D^{3}x
\end{split}
\end{equation}
where $A_{\mu}$ is a ``4''-potential for electromagnetism. Later we will
generalize Electromagnetism to Yang--Mills theory, and the Chern--Simons
theory will play an important role in something called the Quantum Hall
effect. It also describes quantum gravity in $2+1$ dimensions.

For a Yang--Mills field, however, we also have a term that looks like
$A^{3}$ in the action. Such a term vanishes for commutative gauge groups
like $\U(1)$ (i.e., like for Electromagnetism).

\begin{exercise}
From demanding stationary action $\variation\action_{CS}[A]/\variation A^{\mu}(x)=0$,
determine the equations of motion for Chern--Simons theory for the
electromagnetic field.
\end{exercise}
% https://itp-www-beisert.ethz.ch/hs10/ppp1/PPP1_2.pdf
\section{Scattering}

\M
The basic setup is that we have $n\in\NN$ incoming bodies which
``collide'', and then $n'\in\NN$ outgoing bodies emerge from the
interaction --- this is referred to as ``$n\to n'$ Scattering''.
Usually $n=n'=2$.

\begin{remark}
In particle physics, it becomes conventional to write down the particle
types as a sum. For example, Bhabha scattering between a position
$\positron$ and an electron $\electron$ (they repel each other, no other
outgoing particles are created) is denoted either
$\positron+\electron\to\positron+\electron$ or more concisely as $\positron\electron\to\positron\electron$.
\end{remark}

\N{Conservation of Four-Momentum}
In \emph{all scattering situations}, the sum of four-momenta of all the bodies
is equal to a constant. Specifically, we would have the sum of all
4-momenta \emph{before the collision} equal the sum of all 4-momenta
\emph{after the collision},
\begin{equation}
\sum_{a}\vec{P}^{\text{(pre)}}_{a}
=\sum_{a'}\vec{P}^{\text{(post)}}_{a'},
\end{equation}
where $a=1,\dots,n$ and $a'=1,\dots,n'$.

\subsection{Particle Decay}

\N{Decay}
When we have $n=1$ and $n'>1$, this describes a particle decaying into
several particles. Schematically, we draw this like:
\begin{center}
  \includegraphics{img/scattering.0}
\end{center}

For example, in the rest frame, the decaying particle's 4-momentum is
given by
\begin{equation}
\vec{P} = \bigl(m_{0}\gamma(\vec{v})c, 0, 0, 0\bigr).
\end{equation}
The decay time (``lifetime'') is
\begin{equation}
(\Delta\tau)^{2} = \Delta t^{2}(1 - \vec{v}^{2}/c^{2})
\end{equation}
where $\Delta t$ is the lifetime relative to the laboratory frame:
\begin{equation}
\Delta t = \gamma\,\Delta\tau.
\end{equation}
Since $\gamma>1$, we see $\Delta t>\Delta\tau$.

\begin{example}
Pions (which consist of a quark and antiquark, one of them is up [or
  anti-up] the other is down [or anti-down]) are unstable and decay.
The positively charged pion $\pi^{+}$ (consisting of an up quark and
anti-down quark) has a mean lifetime when decaying into a Muon and
neutrino of:
\begin{subequations}
\begin{equation}
\Delta\tau_{\pi^{+}\to\mu^{+}\nu_{\mu}}\approx 2.6033\times10^{-8}~\mathrm{s}.
\end{equation}
We know experimentally $m_{\pi^{\pm}}c^{2}\approx 140~\mathrm{MeV}$ is the
rest mass for the charged pion. If (relative to the lab frame) the
Pion's energy is $E_{\pi}=20\times10^{3}~\mathrm{MeV}$, then
\begin{equation}
\gamma = \frac{E_{\pi}}{m_{\pi}}\approx 142.8571\approx 143,
\end{equation}
and (relative to the lab frame) the 3-velocity of the Pion has a magnitude
of approximately
\begin{equation}
\|\vec{v}\| = \frac{\sqrt{\gamma^{2}-1}}{\gamma}\approx\frac{12\sqrt{142}}{143}c\approx0.999975c.
\end{equation}
So relative to the lab frame, the particle lifetime is approximately
$143\,\Delta\tau$ due to time-dilation. 
\end{subequations}
\end{example}

\N{Energy of Resulting Particles from Decay}
For $1\to2$ decay, we have the following conditions:
\begin{enumerate}
\item conservation of 4-momenta $p=p_{1}+p_{2}$ (which gives us 4 equations)
\item the mass-shell conditions $p_{(0)}^{2}=-m_{0}^{2}c^{2}$,
  $p_{(1)}^{2}=-m_{1}^{2}c^{2}$, $p_{(2)}^{2}=-m_{2}^{2}c^{2}$ where
  $p_{(0)}=(m_{0}c,\vec{0})$,
  $p_{(1)}=(\gamma(\vec{v}_{1})m_{1}c,\vec{p}_{1})$,
  $p_{(2)}=(\gamma(\vec{v}_{2})m_{2}c,\vec{p}_{2})$.
\end{enumerate}
From these conditions we can compute
$\eta_{\mu\nu}p^{\mu}_{(0)}p^{\nu}_{(i)}=m_{0}E_{(i)}$. Then we can
write
\begin{calculation}
E_{i}
\step{divide through by $m_{0}$}
\frac{1}{m_{0}}\eta_{\mu\nu}p^{\mu}_{(0)}p^{\nu}_{(i)}
\step{conservation of 4-momenta gives $p^{\mu}_{(0)}=p^{\mu}_{(1)}+p^{\mu}_{(2)}$}
\frac{1}{m_{0}}\eta_{\mu\nu}(p^{\mu}_{(1)}p^{\nu}_{(i)} + p^{\mu}_{(2)}p^{\nu}_{(i)}).
\end{calculation}
We combine this relation for $E_{i}$ with
\begin{calculation}
  \eta_{\mu\nu}p^{\mu}_{(1)}p^{\nu}_{(2)}
\step{definition of Minkowski-space dot product}
p_{1}\cdot p_{2}
\step{algebraic identity}
\frac{1}{2}[(p_{1}+p_{2})^{2}-p_{1}^{2}-p_{2}^{2}]
\step{mass-shell condition}
\frac{1}{2}[m_{0}^{2} - m_{1}^{2} - m_{2}^{2}]c^{2}.
\end{calculation}
This gives us
\begin{subequations}
\begin{align}
E_{1} &= \frac{1}{m_{0}}(p_{(1)}^{2} + p_{(1)}\cdot p_{(2)}) = \frac{1}{2m_{0}}[m_{0}^{2} + m_{1}^{2} - m_{2}^{2}]c^{2}\\
E_{2} &= \frac{1}{2m_{0}}[m_{0}^{2} - m_{1}^{2} + m_{2}^{2}]c^{2}.
\end{align}
\end{subequations}

\begin{exercise}
Since $E=\sqrt{m_{0}^{2}c^{4}-c^{2}\|\vec{p}\|^{2}}$, and the spatial
components of the 4-momenta satisfy $\vec{p}_{(1)}+\vec{p}_{(2)}=\vec{0}$,
the show the absolute value of the three momenta satisfy:
\begin{enumerate}
\item $\|\vec{p}_{(1)}\|^{2} = \|\vec{p}_{(2)}\|^{2}$,
\item $\|\vec{p}_{(1)}\|^{2} = \displaystyle\frac{1}{4m_{0}^{2}}\left(m_{0}^{4}-2m_{0}^{2}(m_{1}^{2}+m_{2}^{2})+(m_{1}^{2}-m_{2}^{2})^{2}\right)$
\end{enumerate}
This means, for $1\to2$ particle decay, only the direction of the
spatial components of momenta for the resulting particles needs to be determined.
\end{exercise}

\subsection{$2\to 2$ Scattering}

\M
The other family of scattering events witnessed in nature (and in
particle colliders) broadly is $2\to2$ scattering. We have particle $i$
before the collision, and $i'$ after the collision. Schematically, we
draw this as (with time moving ``from left to right''):
\begin{center}
  \includegraphics{img/scattering.1}
\end{center}
We have the mass-shell condition give us 4 constraints
\begin{equation}
p_{i}^{2} = -m_{i}^{2}c^{2}
\end{equation}
for $i=1,\dots,4$ and the conservation of momentum gives another 4
equations
\begin{equation}
p^{\mu}_{1} + p^{\mu}_{2} = p^{\mu}_{3} + p^{\mu}_{4}.
\end{equation}
We have, so far, made no assumptions concerning the collision being
inelastic or elastic.

\N{Elastic Scattering Motivates new Parameters}
When $m_{1}=m_{3}$ and $m_{2}=m_{4}$, elastic scattering simplifies and
we can express everything in terms of the masses and the 6 invariants
$p_{i}\cdot p_{j}$ for $i\neq j$. \emph{We will not assume elastic scattering},
but the invariants are sufficiently useful that we can parametrize them
in terms of 3 variables and the rest masses.

\N{Mandelstam Variables}
We have three kinematic variables which help parametrize the scattering
process, called the \emph{Mandelstam variables}. They are, in our metric
signature conventions (and writing it using primed versions of the
momenta to stress their physical interpretation),
\begin{subequations}
\begin{align}
s &:= -(p_{1} + p_{2})^{2}c^{2}\\
t &:= -(p_{1} - p_{3})^{2}c^{2} = -(p_{1} - p_{1}')^{2}c^{2}\\
u &:= -(p_{1} - p_{4})^{2}c^{2} = -(p_{1} - p_{2}')^{2}c^{2}.
\end{align}
\end{subequations}
We interpret these quantities as:
\begin{enumerate}
\item $s$ is the square of the center-of-mass energy, and
\item $t$ is the square of the 4-momentum transfer.
\end{enumerate}

\begin{exercise}
Prove $s+t+u=\sum_{i}m_{i}^{2}c^{4}$. Does this require assuming the
scattering is elastic?
\end{exercise}

\begin{exercise}
  Prove:
  \begin{enumerate}
  \item $2p_{1}\cdot p_{2} = c^{-2}s - m_{1}^{2}c^{2} - m_{2}^{2}c^{2}$
  \item $2p_{1}\cdot p_{3} = m_{1}^{2}c^{2} + m_{3}^{2}c^{2} - c^{-2}t$
  \item $2p_{1}\cdot p_{4} = m_{1}^{2}c^{2} + m_{4}^{2}c^{2} - c^{-2}u$.
  \end{enumerate}
\end{exercise}

\begin{definition}
The \define{Center-of-Mass Frame} is the reference frame defined by
\begin{equation}
\vec{p}_{1}+\vec{p}_{2} = 0 = \vec{p}_{3}+\vec{p}_{4}.
\end{equation}
This coincides with the usual definition in the nonrelativistic limit.
\end{definition}

\begin{definition}\label{defn:relativity:scattering:kallen-function}
We define the \define{K\"{a}llen function} (or \emph{Triangle Function}):
\begin{subequations}
\begin{align}
\lambda(a,b,c) &= a^{2}+b^{2}+c^{2}-2ab-2ac-2bc\\
&=\left[a-(\sqrt{b}+\sqrt{c})^{2}\right]\left[a-(\sqrt{b}-\sqrt{c})^{2}\right]\\
&=a^{2}-2a(b+c) + (b-c)^{2}.
\end{align}
\end{subequations}
\end{definition}

\begin{exercise}
Prove $\lambda(a,b,c)$ is symmetric under swapping any of the arguments.
\end{exercise}
\begin{exercise}
When $a\gg b$ and $a\gg c$, prove $\lambda(a,b,c)\to a^{2}$.
\end{exercise}

\begin{exercise}
Let $a$, $b$, $c$ be the lengths of the three sides of an arbitrary
triangle. Prove the area of the triangle $A$ satisfies
$4A=\sqrt{-\lambda(a^{2},b^{2},c^{2})}$. [Hint: think about Heron's formula.]
\end{exercise}


\M
In the center-of-mass frame, $2\to2$ scattering looks like:
\begin{center}
\includegraphics{img/scattering.2}
\end{center}
There is also, in this frame, an angle $\Theta$ relating the incoming
particle $i$ to its outgoing trajectory $i\to i'$ is related by an angle
$\Theta$. How do we determine an expression for $\Theta$ in terms
of\dots well, anything?

We know the dot product is precisely the magnitudes of the vectors
multiplied by the cosine of the angle between them. Therefore, we expect
\begin{equation}
\vec{p}\cdot\vec{p}' = \|\vec{p}\|\cdot\|\vec{p}'\|\cos(\Theta),
\end{equation}
just by definition of the (spatial) dot product. This means
\textbf{in the center of mass frame}:
\begin{equation}
\eta_{\mu\nu}(p_{1}^{\mu}p_{3}^{\nu})_{CM} = c^{-2}(E_{1}E_{3})_{CM} - 
\|\vec{p}_{1}^{(CM)}\|\cdot\|\vec{p}_{3}^{(CM)}\|\cos(\Theta).
\end{equation}
We can also use the fact that
\begin{equation}
  t = -c^{2}(p_{1} - p_{3})^{2}
  = c^{4}m_{1}^{2} + c^{4}m_{3}^{2} - 2\eta_{\mu\nu}p_{1}^{\mu}p_{3}^{\nu},
\end{equation}
and other relations of the Mandelstam variables, to express
$\cos(\Theta)$ as a function of $s$, $t$, and the squared masses:
\begin{equation}\label{eq:relativity:scattering:t-angle}
\cos(\Theta) = \frac{s(t - u) + (m_{1}^{2} - m_{2}^{2})(m_{3}^{2} - m_{4}^{2})c^{8}}{\sqrt{\lambda(s,m_{1}^{2}c^{4},m_{2}^{2}c^{4})}\sqrt{\lambda(s,m_{3}^{2}c^{4},m_{4}^{2}c^{4})}}
\end{equation}
where we used the K\"{a}llen function (\S\ref{defn:relativity:scattering:kallen-function}).
\emph{We have not assumed the scattering is elastic or inelastic}.

\begin{exercise}
Derive Eq~\eqref{eq:relativity:scattering:t-angle}.
\end{exercise}

\N{Channels}
We have the usual $s$-channel describing scattering of the form $1+2\to 3+4$.
However, we could swap $2$ and $3$ (interpreting them as
anti-particles), interpreting the process as
$1+\overline{3}\to\overline{2}+4$. This describes the \define{$t$-Channel}.
Here $t$ is interpreted as the square of the center-of-mass energy, and
$s$ is the momentum transfer squared.

If we swapped $2$ with $4$, giving us a scattering process
$1+\overline{4}\to3+\overline{2}$,
then we have a scattering in the \define{$u$-Channel}. The $u$-Channel
is just the $t$-Channel with the roles of particles 3 and 4 swapped.
Here, in the $u$-Channel, the $u$ variables then equals to square of the
center-of-mass energy.

\begin{remark}
When $(1,2)$ are identical particles, or when $(3,4)$ are identical
particles, the $t$-channel scattering is indistinguishable from
$u$-channel scattering. When $1=2$, we'd have $1+1\to3+4$
$t$-channel correspond to $1+\overline{3}\to\overline{1}+4$,
but its $u$-channel corresponds to $1+\overline{4}\to\overline{1}+3$.
\end{remark}

\N{References}
See Rindler~\cite{Rindler:1991sr}. For the particulars of relativistic
scattering, Hagedorn~\cite{Hagedorn:1963hdh}.

\endinput

\section*{References}
% \N{References}
For the uninitiated, Taylor and Wheeler~\cite{Taylor:1992sp} is a great
introduction. Rindler~\cite{Rindler:1991sr} is a good review.
Misner, Thorne, and Wheeler's \textit{Gravitation}~\cite{Misner:1973prb} (ch.2) discusses
special relativity and we rely on its index notation conventions, and
Chapter 6 discusses accelerated observers in Special Relativity.
We also have double checked calculations against Srednicki~\cite{Srednicki:2007qs}.

There are a lot of subtleties to the Hamiltonian analysis of
electromagnetism and, more generally, any theory with symmetries. The
best review of the topic is the first 5 or so chapters of Henneaux and Teitelboim~\cite{Henneaux:1992ig}.

\N{TODO: Index Correct?}
I should double check the indices are correct for the Lorentz
transformation ${\Lambda^{\mu}}_{\nu}$ --- or is this the inverse of the
Lorentz transformation? After some investigation, we should have
$(\Lambda x)^{\mu} = {\Lambda^{\mu}}_{\nu}x^{\nu}$.
Rows are indexed by covariant indices, columns are indexed by
contravariant indices. MTW \S2.9 uses the notation
$x^{\alpha'} = {\Lambda^{\alpha'}}_{\mu}x^{\mu}$ and the inverse
transformation by $x^{\mu} = {\Lambda^{\mu}}_{\alpha'}x^{\alpha'}$.
Therefore composing these guys gives us
${\Lambda^{\mu}}_{\alpha'}{\Lambda^{\alpha'}}_{\nu}={\delta^{\mu}}_{\nu}$
and ${\Lambda^{\alpha'}}_{\mu}{\Lambda^{\mu}}_{\beta'} = {\delta^{\alpha'}}_{\beta'}$.

\N{TODO: Scattering}
It is probably good to discuss $2\to2$ scattering in special relativity,
since that's the basis of a lot of particle physics experiments.
