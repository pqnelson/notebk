\chapter{Special Relativity}

\M
There's a heuristic taught to graduate students:
\begin{equation}
\begin{pmatrix}\mbox{Quantum}\\
\mbox{Field}\\
\mbox{Theory}
\end{pmatrix}
=
\begin{pmatrix}\mbox{Special}\\
\mbox{Relativity}
\end{pmatrix}
+\begin{pmatrix}\mbox{Quantum}\\
\mbox{Mechanics}
\end{pmatrix}.
\end{equation}
This is partially true, but we should review special relativity (if only
to specify our conventions).

Historically, 
\begin{equation}
\begin{pmatrix}\mbox{Special}\\
\mbox{Relativity}
\end{pmatrix}
=
\begin{pmatrix}\mbox{Newtonian}\\
\mbox{Mechanics}
\end{pmatrix}
+
(\mbox{Electromagnetism}).
\end{equation}
This is because electromagnetism determines that the speed of light is
constant in a vacuum, and Newtonian mechanics assumes Galilean
relativity. But what happens if an inertial observer is moving at the
speed of light? Addition of velocities would suggest that bodies could
move faster than light\footnote{Einstein's thought experiment: suppose
you were riding on a bicycle going $0.999c$, and then you turn on your
headlights. What speed will the photons emitted from your headlights
travel?}, which violates results from electromagnetism.

\section{Foundations}

\begin{axiom}[Principle of Relativity]
The laws of physics are identical in all inertial frames. That is to
say, the outcome of any physical experiment is the same when performed
with identical initial conditions relative to any inertial frame.
\end{axiom}

\begin{axiom}
There exists an inertial reference frame in which light signals in
vacuum always travel rectilinearly at constant speed $c$, in all
directions, independently of the motion of the source.
\end{axiom}

\begin{remark}
These are the axioms as formulated in Rinder~\cite[\S4]{Rindler:1991sr}.
Traditionally, introductory texts to special relativity use some version
of these axioms, then ``derive'' results in special relativity.

A more intuitive approach uses $K$-calculus, as first formulated by
Bondi, using spacetime diagrams.
\end{remark}

\M
The best approach to discussing special relativity is the geometric
approach, using spacetime diagrams. However, the point of this chapter
is not to teach special relativity, but to review the Lorentz group as
encoding Lorentz invariance.

The geometric approach uses a 4-dimensional affine space describing
physical space-time, consisting of ``events'' [points]. We do the usual
trick setting up a reference frame as consisting of a point (or trajectory)
which we think of as the ``origin'', then take 4 other distinct points
to construct unit vectors (by taking the displacement from the
``origin'' to each of these points). This gives us a coordinate system.

We can describe time intervals in difference reference frames by
``shooting off'' one photon at the start of the interval (and then
another at the end of the interval). This translates to an interval
$k\,\Delta t$ in the other reference frame, where $k$ is a factor to be
determined. This is done in D'Inverno's \textit{Introducing Einstein's Relativity},
chapters 2 and 3.

\N{On ``Paradoxes''}
One last word, a lot of ``paradoxes'' encountered in physics (like the
``Twin paradox'' in special relativity) are not actually paradoxes: they
are just physical situations where our intuitions fail to describe the
phenomenon properly.

\section{Kinematics}

\N{Four-Vectors} Special relativity differs from Newtonian physics by
working with 4-vectors. The usual vectors we encountered in physics
$\vec{x}$ are 3-vectors in physical space. Now we will use 4-component
vectors of the form $(t, \vec{x})$ or more generally using indices
$(x^{0}, x^{1}, x^{2}, x^{3})$. We can write this using basis vectors
\begin{subequations}
\begin{align}
  A &= (A^{0}, A^{1}, A^{2}, A^{3})\\
&= A^{0}\vec{e}_{0} + A^{1}\vec{e}_{1} + A^{2}\vec{e}_{2} + A^{3}\vec{e}_{3}\\
&= A^{0}\vec{e}_{0} + A^{i}\vec{e}_{i}\\
&= A^{\mu}\vec{e}_{\mu}
\end{align}
\end{subequations}
where $\vec{e}_{\mu}$ are basis vectors, we implicitly sum over repeated
indices, $i = 1,2,3$, and $\mu=0,1,2,3$.

The zeroth component is the time component, the remaining components are
the spatial components.

\begin{remark}
When the components of the four-vector are upstairs $A^{\mu}$,
these are \define{Contravariant} vectors.
\end{remark}

\N{Covectors}
We can also consider ``covectors'' or ``dual vectors'' (which eat in a
4-vector and produce a [real] number). These have components with
downstairs indices $B_{\mu}\vec{f}^{\mu}$ where $\vec{f}^{\mu}$ are the
co-basis covectors.

\M
Normally we can transform the components of a four-vector
$\vec{A}=A^{\mu}\vec{e}_{\mu}$ into the components of a covector by
using the metric $g_{\mu\nu}$ in the usual way:
\begin{equation}
A_{\nu}\vec{f}^{\nu} = (g_{\mu\nu}A^{\mu})\vec{f}^{\nu}.
\end{equation}
We can do the same, but for special relativity the metric is denoted
$\eta_{\mu\nu}$.\marginpar{\footnotesize Minkowski metric $\eta_{\mu\nu}$}
We call this the \define{Minkowski metric} and in Cartesian coordinates
has components
\begin{equation}
\eta_{\mu\nu} = \begin{pmatrix}-1 & 0 & 0 & 0\\
0 & 1 & 0 & 0\\
0 & 0 & 1 & 0\\
0 & 0 & 0 & 1
\end{pmatrix}.
\end{equation}
This is the so-called \define{East-Coast Convention} (particle
physicists multiply this by $-1$ and that's the \emph{West-coast convention}).

\N{Magnitude of Four-Vectors}
When $a^{\mu}$ is any four-vector, its magnitude is the scalar quantity
given by:
\begin{equation}
\|a^{\mu}\|^{2} := \eta_{\mu\nu}a^{\mu}a^{\nu}.
\end{equation}

\begin{definition}
The \define{Four-Position} is the four-vector with components
\begin{equation*}
x^{\mu} = (ct, x^{1}, x^{2}, x^{3}).
\end{equation*}
In Cartesian coordinates, $x^{\mu} = (ct, x, y, z)$.
\end{definition}

\N{Events}
Events in spacetime are described using 4-position vectors. This is an
idealization that events have no ``duration''. We could handle events
with some duration by having one 4-position for the ``start'' and
another 4-position for the ``end''.

\begin{definition}
When we have two 4-position vectors $\vec{A}_{1}=(ct_{1},\vec{r}_{1})$
and $\vec{A}_{2}=(ct_{2},\vec{r}_{2})$, we define the
\define{Displacement Four-Vector} as the four-vector
\begin{equation*}
\Delta\vec{A} = (c\,\Delta t,\Delta\vec{r}) = \vec{A}_{2} - \vec{A}_{1}.
\end{equation*}
For an \define{Infinitesimal Displacement Four-Vector} (or
\emph{Differential Four-Position}), we write
$\D\vec{A}$.
\end{definition}

\M
Suppose we have two events in spacetime separated by an infinitesimal
displacement 4-vector $\D x^{\mu}$. Special relativity demands the
infinitesimal interval,
\begin{equation}
(\D s)^{2} := \eta_{\mu\nu}\,\D x^{\mu}\,\D x^{\nu} = -c^{2}(\D t)^{2} + (\D x)^{2} + (\D y)^{2} + (\D z)^{2},
\end{equation}
must be the same for all inertial observers. Here we use summation
conventions where, when we have an index downstairs and upstairs, we sum
over it (so in our equation, we sum over $\mu$ and $\nu$).

\begin{definition} Let $a^{\mu}$ be a 4-vector.
  \begin{enumerate}
  \item If $\eta_{\mu\nu}a^{\mu}a^{\nu} < 0$, then we call $a^{\mu}$ \define{Time-like}.
  \item If $\eta_{\mu\nu}a^{\mu}a^{\nu} = 0$, then we call $a^{\mu}$ \define{Light-like}.
  \item If $\eta_{\mu\nu}a^{\mu}a^{\nu} > 0$, then we call $a^{\mu}$ \define{Space-like}.
  \end{enumerate}
\end{definition}

\begin{remark}
To see the motivation for these definitions, consider the trajectory of
a photon moving along the $x$-axis in Cartesian coordinates. It would be
$\vec{\gamma}(t)=(ct,ct,0,0)$ and its displacement from the origin would
be always zero for any $t\in\RR$. Therefore, the displacement would be
light-like.

For time-like vectors, consider the displacement 4-vector
$\vec{\alpha}(t)=(ct,0,0,0)$ which stays at the spatial origin. We see
its magnitude is $-c^{2}t^{2}<0$.

For space-like vectors, consider the displacement 4-vector
$\vec{\beta}(t)=(0,ct,0,0)$. Its magnitude is $c^{2}t^{2}>0$.
\end{remark}

\begin{definition}[Minkowski spacetime]
We write $\RR^{3,1}$ for \define{Minkowski Spacetime}, i.e., $\RR^{4}$
equipped with the Minkowski metric $\eta$.
\end{definition}

\begin{remark}
If we were using $+---$ signature conventions, Minkowski space would be
$\RR^{1,3}$. 
\end{remark}

\subsection{Invariance of Differential Displacement}

\M We have the obvious spatial rotations and spatial translations leave
the differential displacement $\D s^{2}$ invariant, since this is a
carry-over from Euclidean geometry. The more interesting case is the
``rotation'' of time and space. We will restrict focus to $\RR^{1,1}$
for simplicity, but the reasoning generalizes to $\RR^{n,1}$ by
composing with rotations to make the ``direction of motion'' a spatial
direction.

%% \M Let us restrict focus to $\RR^{1,1}$ for simplicity. Suppose we
%% change coordinates
%% \begin{equation}
%% \begin{pmatrix}c\,\D t'\\\D x'
%% \end{pmatrix}
%% = \begin{pmatrix}a & b\\k & d
%% \end{pmatrix}
%% \begin{pmatrix}c\,\D t\\\D x
%% \end{pmatrix}.
%% \end{equation}
%% Then writing $(\D s)^{2}$ in the new coordinates:
%% \begin{calculation}
%%   (\D s)^{2}
%% \step{by definition}
%%   -c^{2}(\D t')^{2} + (\D x)^{2}
%% \step{plugging in the values of $\D t'$ and $\D x'$}
%%   -c^{2}(a\,\D t + b\,\D x)^{2} + (k\,\D t + d\,\D x)^{2}
%% \step{expanding terms}
%%   -c^{2}[a^{2}\,(\D t)^{2} + 2ab\,\D t\D x + b^{2}\,(\D x)^{2}]
%%   +[k^{2}\,(\D t)^{2} + 2kd\,\D t\D x + d^{2}\,(\D x)^{2}]
%% \step{collecting terms}
%%   -(c^{2}a^{2} - k^{2})(\D t)^{2} - 2(c^{2}ab - kd)\,\D t\D x + (-c^{2}b^{2}+d^{2})(\D x)^{2}.
%% \end{calculation}
%% Comparing to $(\D s)^{2} = -(\D t)^{2} + (\D x)^{2}$, we have the
%% conditions:
%% \begin{subequations}
%% \begin{align}
%% (c^{2}a^{2} - k^{2}) &= 1\\
%% (c^{2}ab - kd) &= 0\\
%% (-c^{2}b^{2}+d^{2}) &= 1.
%% \end{align}
%% \end{subequations}

%% \M
%% Specifically, when changing from one inertial frame (at rest) to another
%% (moving at speed $v$ in the $x$-direction relative to the first frame),
%% we expect
%% \begin{equation}
%% x' = x - vt.
%% \end{equation}
%% There may be a relativistic correction, a factor $\alpha(v)$ such that
%% for $v\ll c$ we have $\alpha(v)\approx 1$. Then we use
%% \begin{equation}
%% x' = \alpha(v)(x - vt).
%% \end{equation}
%% This gives us
%% \begin{equation}
%% \begin{pmatrix}c\,\D t'\\\D x'
%% \end{pmatrix}
%% = \begin{pmatrix}a & b\\-\alpha v & \alpha
%% \end{pmatrix}
%% \begin{pmatrix}c\,\D t\\\D x
%% \end{pmatrix}.
%% \end{equation}
%% This lets us substitute $k=-\alpha(v)v/c$ and $d=\alpha(v)$. This gives us
%% the system of equations
%% \begin{subequations}
%% \begin{align}
%% (c^{2}a^{2} - v^{2}\alpha^{2}) &= c^{2}\\
%% (c^{2}ab + \alpha^{2}v) &= 0\\
%% (-c^{2}b^{2}+\alpha^{2}) &= 1.
%% \end{align}
%% \end{subequations}
%% This is 3 equations in 3 unknowns, which has a solution.
%% \begin{subequations}
%% \begin{align}
%% \alpha(v) &= \frac{-v/c}{\sqrt{1 - v^{2}/c^{2}}}\\
%% \intertext{and}
%% a = b &= \frac{1}{\sqrt{1 - v^{2}/c^{2}}}.
%% \end{align}
%% \end{subequations}

\N{Problem statement}
Given an inertial observer at rest, and another observer moving with
constant speed $v$, how do the coordinate change from the observer at
rest to the moving observer?

\N{Desiderata}
We want, for $v^{2}\ll c^{2}$, to recover Galilean transformations. That
is, $x' \approx x - vt$ (where primed coordinates are those relative to
the moving observer).

\M Let me try this again. We consider an inertial reference frame moving
with velocity $v$ in the $x$-direction relative to an inertial reference
frame at rest. Then the new coordinates are
\begin{subequations}
\begin{align}
ct' &= a_{11}ct + a_{12}x\\
x' &= a_{22}(x - vt).
\end{align}
\end{subequations}
This reduces to the familiar Galilean transformation for $v\ll c$, which
we demand $a_{22}\approx 1$. Now, for an infinitesimal displacement, its
components are
\begin{subequations}
\begin{align}
c\,\D t' &= a_{11}c\,\D t + a_{12}\,\D x\\
\D x' &= a_{22}(\D x - v\,\D t).
\end{align}
\end{subequations}
Demanding the invariance of the infinitesimal displacement is the same
as the condition:
\begin{equation}
-c^{2}(\D t')^{2} + (\D x')^{2} =-c^{2}(\D t)^{2} + (\D x)^{2}.
\end{equation}
Expanding the primed quantities gives us
\begin{equation}
-(a_{11}^{2}c^{2} - a_{22}^{2}v^{2})(\D t)^{2}
+ 2 (a_{22}^{2}v - a_{11}a_{12}c)\,\D x\,\D t
+ (a_{22}^{2} - a_{12}^{2})(\D x)^{2} =-c^{2}(\D t)^{2} + (\D x)^{2}.
\end{equation}
This gives us the system of 3 equations in 3 unknowns:
\begin{subequations}
\begin{align}
-(a_{11}^{2}c^{2} - a_{22}^{2}v^{2}) &= -c^{2}\\
2 (a_{22}^{2}v - a_{11}a_{12}c) &= 0\\
(a_{22}^{2} - a_{12}^{2}) &= 1.
\end{align}
\end{subequations}
We can solve this system:
\begin{subequations}
\begin{align}
a_{11} = a_{22} &= \frac{1}{\sqrt{1 - v^{2}/c^{2}}}\\
a_{12} &= \frac{-v/c}{\sqrt{1 - v^{2}/c^{2}}}.
\end{align}
\end{subequations}
The standard notation is
\begin{equation}
\gamma(v) = \frac{1}{\sqrt{1 - v^{2}/c^{2}}}
\end{equation}
and
\begin{equation}
\beta(v) = \frac{v}{c}.
\end{equation}
Then our transformation is
\begin{equation}
\begin{pmatrix}ct'\\ x'
\end{pmatrix}
=\begin{pmatrix}\gamma & -\gamma\beta\\
-\gamma\beta & \gamma
\end{pmatrix}
\begin{pmatrix}ct\\ x
\end{pmatrix}.
\end{equation}
This transformation is called a \define{Lorentz Boost}.

\M
Observe when $\beta=1/2$ that
\begin{equation}
\gamma = \frac{2}{\sqrt{3}}\approx 1.1547.
\end{equation}
In other words, for an inertial observer moving half the speed of light,
there is a deviation approximately 15\% from non-relativistic values.

At the Large Hadron Collider, protons are accelerated to speeds of about
$99.9999991\%$ the speed of light --- that is, $1-\beta\approx9\times10^{-9}$.
This gives us $\gamma\approx7453$.

\begin{definition}
We call a velocity \define{Ultra-Relativistic} when $1-\beta\ll1$.
\end{definition}

\begin{remark}
For ultra-relativistic velocities, we see that $1+\beta\approx2$, and
therefore
\begin{equation}
\gamma = \frac{1}{\sqrt{(1 + \beta)(1 - \beta)}}
\approx\frac{1}{\sqrt{2(1 - \beta)}}.
\end{equation}
Simple algebra gives us
\begin{equation}
1 - \beta\approx\frac{1}{2\gamma^{2}}.
\end{equation}
\end{remark}

\begin{exercise}
What is the error in the approximation
$1-\beta\approx(2\gamma^{2})^{-1}$ for $\beta=9/10$? For $\gamma=2$? For $\beta=99/100$?
\end{exercise}

\begin{exercise}
For $0\leq\beta\leq0.9$, find a linear approximation $L(\beta)$ for
$\gamma(\beta)$ such that $L(0)=\gamma(0)$ and
$L(0.9)=\gamma(0.9)$. What is the $L^{2}$-norm of the residual for this approximation?
Is there a better linear approximation?
\end{exercise}

\subsection{Light Cones and Causality}

\begin{definition}\label{defn:relativity:light-cone}
Let $a^{\mu}$ be any event. We define the \define{Lightcone} for $a^{\mu}$
to be the set of events light-like separated from $a^{\mu}$,
\begin{equation}
\mathscr{C} = \{\,b^{\mu}\in\RR^{3,1} \mid b^{\mu}~\mbox{is light-like separated from}~a^{\mu}\,\}.
\end{equation}
We can separate the light-cone in two: events in the future and events
in the past, giving us the \define{Future Light Cone}
\begin{equation}
\mathscr{C}^{+} = \{\,b^{\mu}\in\mathscr{C} \mid b^{0} > a^{0}\,\},
\end{equation}
and the \define{Past Light Cone} consisting of events preceding $a^{\mu}$,
\begin{equation}
\mathscr{C}^{-} = \{\,b^{\mu}\in\mathscr{C} \mid b^{0} < a^{0}\,\}.
\end{equation}
We can take the \define{Closed Light Cone} to consist of all light-like
separated events \emph{and} all time-like separated events, and denote
it by
\begin{equation}
\overline{\mathscr{C}} = \{\,b^{\mu}\in\RR^{3,1} \mid \eta_{\mu\nu}(b^{\mu}-a^{\mu})(b^{\nu}-a^{\nu})\leq0\,\}.
\end{equation}
\end{definition}

\M
The only possible causal influences for an event $a^{\mu}$
\emph{must lie within the past causal light cone} for the event, since
nothing can travel faster than light. For this reason, we call any
four-vector $b^{\mu}$ \define{Causal-like} if it is not space-like
separated from $a^{\mu}$, i.e., if the displacement $b^{\mu} - a^{\mu}$
four-vector is either time-like or light-like.

\subsection{World Lines and Trajectories}

\N{World lines}
A curve $\gamma$ in spacetime is called a \define{World Line}
if its tangent vector is future time-like at each point along the
curve. More generally, we could weaken the condition, and allow tangent
vectors to be causal-like.

Particles move along world lines in special relativity.

\M
If we have a curve $\gamma$ in spacetime such that its tangent vector is
space-like at each point along the curve, we call the curve
\define{Space-like}.

If we have a curve in spacetime such that its tangent vector is
ligh-like at each point along the curve, we call the curve
\define{Light-like}. 


\N{Proper Time}
The time between two events along a world line, according to the
observer moving along the trajectory, is precisely the
\define{Proper Time} and denoted $\tau$. For an infinitesimal
displacement along the trajectory, the infinitesimal change in proper
time is given by
\begin{equation}
c^{2}\,(\D\tau)^{2} = -(\D s)^{2}.
\end{equation}
The proper time interval along a trajectory is given by the integral
\begin{equation}
\Delta\tau = \int_{\gamma}\D\tau =
\int_{\gamma}\frac{\sqrt{-\eta_{\mu\nu}\,\D x^{\mu}\,\D x^{\nu}}}{c}.
\end{equation}

\N{Parametrizing World Lines}
We parametrize a world line by its proper time, and write it as
$x^{\mu}(\tau)$. This gives us a smooth family of four-positions for a
physical body.

\begin{definition}
Let $x^{\mu}(\tau)$ be a world line. We define its \define{Four-Velocity}
as the four-vector
\begin{equation*}
U^{\mu}(\tau) = \frac{\D x^{\mu}(\tau)}{\D\tau}.
\end{equation*}
\end{definition}

\begin{remark}
Whenever we are tempted to take the time derivative of a quantity, we
really want to take the derivative with respect to \emph{proper time}.
\end{remark}

\N{Lorentz Factor}
We have the familiar Newtonian 3-velocity given by
\begin{equation}
v^{i} = \frac{\D x^{i}}{\D t}.
\end{equation}
We can relate the coordinate time $x^{0}=ct$ with proper time $\tau$ by
the \define{Lorentz Factor} (which is a function of the Newtonian 3-velocity),
\begin{equation}
  \begin{split}
\gamma(\vec{v}) &= \frac{\D t}{\D\tau} = \left(1 - \frac{\eta_{ij}v^{i}v^{j}}{c^{2}}\right)^{-1/2}\\
&=\frac{c}{\sqrt{-\eta_{\mu\nu}U^{\mu}U^{\nu}}}.
  \end{split}
\end{equation}
This allows us to relate the familiar Newtonian 3-velocity
to the four-velocity by
\begin{equation}
U^{\mu} = (c, \gamma(\vec{v})\vec{v}).
\end{equation}

\begin{exercise}
In Robert Rath's \textit{The Infinite and the Divine}, a super-advanced
species of aliens (who existed hundreds of millions of years ago)
transferred their consciousness into robots. This problem concerns one
of their spaceship's trajectory using their advanced technology.

We are told the initial velocity of the spaceship is $1000$ leagues per
hour, but after a decade [literally 10 years] the spaceship reaches a
billion [$1.00\times10^{9}$] leagues per hour. Note: 1 league per hour
is approximately 1.54333 meters per second.
\begin{enumerate}
\item Assume constant acceleration. Find the constant acceleration,
  assuming the universe is Newtonian, and forces acting on the spaceship
  are negligible.
\item Using the solution to the previous problem, how far did the
  spaceship travel? The story suggests it is the distance between
  several stars. If there is 5 light years between stars in the galaxy,
  is this reasonable? If not, how fast would the spaceship have to
  travel?
\item If we allow for special relativity to apply, then time dilation
  presumably occurs. Suppose the ten years in our problem refers to the
  proper time as measured by the ship's chronometer [clock]. Does this
  affect the distance the ship travels? If so, how far does this ship
  travel?
\end{enumerate}
\end{exercise}

\section{Dynamics}

\M
Often the dynamics of particles are omitted in discussions of special
relativity, because things getting complicated conceptually.

\begin{definition}
Let $x^{\mu}(\tau)$ be the world line for a massive body.
Then the \define{Rest Mass} (or \emph{invariant mass}) for the body is
the mass $m_{0}$ as measured in its reference frame.
\end{definition}

\begin{remark}
Some authors use a notion of \emph{relativistic mass}
$m = \gamma(\vec{v})m_{0}$, which is mildly controversial. 
\end{remark}

\N{Four-Momentum}
For a massive particle of rest mass $m_{0}$, its \define{Four-Momentum}
$\vec{P}$ is defined as the product of the rest mass and its
four-velocity, i.e.,
\begin{equation}
\vec{P} := m_{0}\vec{U}.
\end{equation}
We can write out its explicit components
\begin{equation}
\vec{P} = m_{0}\gamma(\vec{v})(c, \vec{v}) = (E/c, \vec{p}).
\end{equation}
Here the total energy of the moving particle is given by
\begin{equation}
E = \gamma(\vec{v})m_{0}c^{2},
\end{equation}
and the total (relativistic) 3-momentum is
\begin{equation}
\vec{p} = m_{0}\gamma(\vec{v})\vec{v}.
\end{equation}

\N{Energy--Momentum Relation}
We have the energy--momentum relation (or \define{Mass--Shell Relation})
be
\begin{equation}
E^{2} = c^{2}\vec{p}\cdot\vec{p} + \bigl(m_{0}c^{2}\bigr)^{2}.
\end{equation}
Equivalently, we have this relation describe the magnitude of the
four-momentum as a constant:
\begin{equation}
\eta_{\mu\nu}p^{\mu}p^{\nu} = -m_{0}^{2}c^{2}.
\end{equation}

\begin{ddanger}
In special relativity, we can meaningfully talk about the Center-of-Mass
reference frame for a system of particles. This is especially useful for
scattering problems. However, if we tried to carry this notion over to
General Relativity, then we run into problems because it is a nonlocal
concept. 
\end{ddanger}

\begin{exercise}
From the mass-shell relation and $\vec{P} = (E/c, \vec{p})$, deduce
$E = \gamma(\vec{v})m_{0}c^{2}$ and 
$\vec{p}\cdot\vec{p}=\pm(\gamma(\vec{v})^{2}-1)m_{0}^{2}c^{2}$.
Determine the correct sign in that second relation.
\end{exercise}

\begin{exercise}
Suppose we have two massive particles with four-momenta $\vec{P}_{1}$
and $\vec{P}_{2}$ and relative speed $v$. Determine
$\vec{P}_{1}\cdot\vec{P}_{2}=\eta_{\mu\nu}P^{\mu}_{(1)}P^{\nu}_{(2)}$ in
terms of their rest mass $m_{01}$ and $m_{02}$ and relative speed $v$.
Hint: if $\vec{P}_{1}=\vec{P}_{2}$ and $v=0$, then you should recover
the mass--shell relation.
\end{exercise}

\N{Four-Force}
We can define the four-force as the four-vector
\begin{equation}
\vec{F} = \frac{\D\vec{P}}{\D\tau}.
\end{equation}
As an immediate consequence of the mass--shell relation, we find
\begin{equation}
\vec{F}\cdot\vec{P} = \eta_{\mu\nu}F^{\mu}P^{\nu} = 0.
\end{equation}

\N{Lagrangian for Point-Particle}
We can write the Lagrangian for a massive point-particle with rest mass
$m_{0}$ as
\begin{equation}
L = cm_{0}\sqrt{-\eta_{\mu\nu}\dot{x}^{\mu}\dot{x}^{\nu}} - V
\end{equation}
where $\dot{x}^{\mu} = \D x^{\mu}/\D\tau = U^{\mu}$, and $V$ is the potential
energy term. The action is then 
\begin{equation}
I = \int L\,\D\tau.
\end{equation}
Varying the action with respect to $\delta x^{\mu}$ then gives us the
equations of motion.

For massless particles, care must be taken with the parametrization, as
well as using its four-momentum to write
\begin{equation}
L = c\sqrt{-\eta_{\mu\nu}P^{\mu}P^{\nu}} - V.
\end{equation}
We can use the relation (which holds for both massive and massless particles):
\begin{equation}
\frac{\D x^{\mu}}{\D t} = \frac{P^{\mu}}{P^{0}}.
\end{equation}

\begin{danger}
This is the correct Lagrangian to work with, especially if we want to
quantize it. There is some subtlety with it, which we should confess
openly: it is a constrained system. To see this, compute the Hamiltonian
for a free massive relativistic particle. It will vanish. This is
because time is a coordinate (proper time is a parameter), and its
conjugate momentum is ``the Hamiltonian''. So we end up with a
constraint. 
\end{danger}

\begin{ddanger}
Some authors insist that canonical mechanics for special relativistic
systems ``breaks Lorentz invariance'', which is not really true. If
you've picked $\mu=0$ to be the time component for four-vectors, then
you've also ``broken Lorentz invariance'' just as much as canonical
mechanics has. We can describe a Lorentz boost as a canonical
transformation (which preserves Lorentz invariance as much as anything
else). This is just offered as a lazy and sloppy justification for using
the path integral formalism, which makes no coherent sense.
\end{ddanger}

\section{Group Theoretic Description}

\begin{definition}
The \define{(Homogeneous) Lorentz Group} for $\RR^{n,1}$ is the group
generated by spatial rotations and Lorentz boosts --- that is, the indefinite
orthogonal group $\O(n,1)$. A generic element of the Lorentz group is
called a \define{Lorentz Transformation}.
\end{definition}

\begin{exercise}
Prove that this actually is a group. That is, the composition of Lorentz
transformation are a Lorentz transformation; inverses of Lorentz
transformations are Lorentz transformations.
\end{exercise}

\M
There are 4 connected components to the Lorentz group $\O(3,1)$ which
are related by parity operator $P$ and time reversal operator $T$, which
are represented by the matrices (acting on the ``obvious'' fundamental
representation as):
\begin{subequations}
\begin{align}
P &= \diag(1, -1, -1, -1)\\
T &= \diag(-1, 1, 1, 1).
\end{align}
\end{subequations}

\begin{definition}
The \define{Poncar\'e Group} for $\RR^{n,1}$ is the group generated by
translations in spacetime [by a constant displacement], spatial
rotations, and Lorentz boosts; it is given by the group
$G = \RR^{n,1}\rtimes\O(n,1)$.
\end{definition}

\begin{remark}
Weinberg calls the Poncar\'e Group the \emph{Inhomogeneous Lorentz Group}.
\end{remark}

\M
We can always consider how an element of $g\in\O(3,1)$ will act on any
4-position $x\in\RR^{3,1}$ ``in the obvious way'' as a Lorentz
transformation. We will write this as $g\cdot x$. Specifically this acts
by means of a $4\times4$ matrix ${\Lambda^{\alpha'}}_{\mu}$ sending
$x^{\mu}\to x^{\alpha'} = f^{\alpha'}(x^{\mu}) = (\Lambda x)^{\alpha'}$.
Explicitly, using Einstein summation convention,
\begin{equation*}
x^{\alpha'} = {\Lambda^{\alpha'}}_{\mu}x^{\mu}.
\end{equation*}

\N{Inverse Transformation}
We can compose Lorentz transformations to find the inverse
transformation ${\Lambda^{\nu}}_{\alpha'}$ which satisfies
\begin{equation}
{\Lambda^{\nu}}_{\alpha'}{\Lambda^{\alpha'}}_{\mu}={\delta^{\nu}}_{\mu}.
\end{equation}
For covariant vectors, we need to use
% We then have
${[\transpose{(\Lambda^{-1})}]^{\mu}}_{\alpha'} = {\Lambda_{\alpha'}}^{\mu}$
be the transpose of the inverse Lorentz transformation for $\Lambda$.
This is because covariant vectors transform under the dual
representation. 


\M
The Lorentz group acts on a scalar field $\varphi(x)$ by
\begin{equation}
g\cdot\varphi(x) = \varphi\bigl(g^{-1}\cdot x\bigr).
\end{equation}
This is because a scalar field is ``just a function'', and this is how
groups act on functions.

\N{Action on Vector Fields}
For a vector field with components $A^{\mu}(x)$, a Lorentz
transformation $g\in\O(3,1)$ acts on this in a ``mixed'' way. We have
the matrix ${\Lambda^{\alpha'}}_{\mu}$ describe the action $(g\cdot x)^{\alpha'} = {\Lambda^{\alpha'}}_{\nu}x^{\nu}$,
which is the change of coordinates $x^{\mu}\to x^{\alpha'}=f^{\alpha'}(x^{\mu})$,
so therefore a contravariant vector field transforms as:
\begin{equation}
[(g\cdot \vec{A})(x)]^{\alpha'} = {\Lambda^{\alpha'}}_{\nu}A^{\nu}(g^{-1}\cdot x).
\end{equation}
It acts on the vector as a whole as we would expect, and on each
component as a function.

For a covariant vector $A_{\mu}(x)$, we see it transforms as
\begin{equation}
[(g\cdot \vec{A})(x)]_{\alpha'} = {\Lambda_{\alpha'}}^{\mu}A_{\mu}(g^{-1}\cdot x)
\end{equation}
where ${\Lambda_{\alpha'}}^{\mu}$ is the matrix inverse of the Lorentz
transformation. It satisfies
\begin{equation}
{\Lambda_{\nu}}^{\alpha'}{\Lambda^{\nu}}_{\beta'} = {\delta^{\alpha'}}_{\beta'}.
\end{equation}

\N{Action on Tensor Fields}
More generally, for a rank $n$ tensor with components $[\mathsf{T}(x)]^{\mu_{1}\cdots\mu_{n}}=T^{\mu_{1}\cdots\mu_{n}}(x)$,
the element $g\in\O(3,1)$ acts as
\begin{equation}
[(g\cdot \tens{T})(x)]^{\alpha_{1}'\cdots\alpha_{n}'}
= {\Lambda^{\alpha_{1}'}}_{\nu_{1}}(\cdots){\Lambda^{\alpha_{n}'}}_{\nu_{n}}
T^{\nu_{1}\cdots\nu_{n}}(g^{-1}\cdot x).
\end{equation}
For covariant indices, we multiply by matrix inverses of $\Lambda$.
These actions are all basic representation theory.

\N{Lorentz Invariance}
We call a quantity $Q(x)$ \define{Lorentz Invariant} if for each $g\in\O(3,1)$
we have
\begin{equation}
g\cdot Q(x) = Q(x).
\end{equation}
For example $\D s^{2}$ is Lorentz invariant.

\section{Electromagnetism}

\N{Maxwell's Equations}
Maxwell's equations which we learn in Physics 101 are, in Gaussian
units, for statics:
\begin{subequations}
\begin{align}
\nabla\cdot\vec{E} &= 4\pi\rho_{e}\\
\nabla\cdot\vec{B} &= 0\\
\intertext{and for dynamics:}
-\frac{\partial\vec{E}}{\partial t} + c\nabla\times\vec{B} &= 4\pi\vec{j}\\
\frac{\partial\vec{B}}{\partial t} + c\nabla\times\vec{E} &= 0.
\end{align}
\end{subequations}
Here $\rho_{e}$ is the electric charge density, and $\vec{j}$ is
the electric current density.
The exact form of these equations depend on the units used. In SI units,
the coefficients need to be altered:
\begin{subequations}
\begin{align}
\nabla\cdot\vec{E} &= \frac{\rho_{e}}{\varepsilon_{0}}\\
\nabla\cdot\vec{B} &= 0
\end{align}
\end{subequations}
\begin{subequations}
\begin{align}
-\mu_{0}\varepsilon_{0}\frac{\partial\vec{E}}{\partial t} + \nabla\times\vec{B} &= \mu_{0}\vec{j}\\
\frac{\partial\vec{B}}{\partial t} + \nabla\times\vec{E} &= 0,
\end{align}
\end{subequations}
where $\varepsilon_{0}$ is the permittivity of free space, $\mu_{0}$ is
the permeability of free space, and $c=1/\sqrt{\varepsilon_{0}\mu_{0}}$
is the speed of light in a vacuum.

\N{Potentials}
It is common to introduce the electric potential $\varphi(\vec{x})$
and magnetic potential $\vec{A}(\vec{x}, t)$. Then the electric field is
taken to be:
\begin{equation}
\vec{E} = -\vec{\nabla}\varphi - \frac{\partial\vec{A}}{\partial t}.
\end{equation}
We call $\varphi$ the \define{Electric Potential} and $\vec{A}$ the
\define{Magnetic Potential}.
The magnetic field is,
\begin{equation}
\vec{B} = \nabla\times\vec{A}.
\end{equation}

\begin{remark}
Any time-dependence the electric field enjoys may be traced back to the
magnetic potential contribution.
\end{remark}

\N{Gauge Invariance}
These potentials are not unique. We could, for any function
$\lambda(\vec{x},t)$, consider the potentials
\begin{subequations}
\begin{align}
\varphi' &= \varphi - \frac{\partial\lambda}{\partial t}\\
\vec{A}' &= \vec{A} + \vec{\nabla}\lambda.
\end{align}
\end{subequations}
This arbitrariness is an example of a gauge symmetry, a redundancy
describing the same physical conditions. We can eliminate this
redundancy (a process called \define{Gauge-Fixing}) by adding an
additional condition.

\N{Lorenz Gauge}
We impose the condition
\begin{equation}
\vec{\nabla}\cdot\vec{A} + \frac{1}{c}\frac{\partial\varphi}{\partial t}=0.
\end{equation}
This is the \emph{Lorenz Gauge} (note the lack of ``t'', because it is
named after the Dutch physicist Ludvig Lorenz, not to be confused with
the other Dutch physicist Hendrik Lorentz [of ``Lorentz transformation''
fame]). Maxwell's equations
simplify to 
\begin{subequations}
\begin{align}
\nabla^{2}\varphi - \frac{1}{c^{2}}\frac{\partial^{2}\varphi}{\partial t^{2}}
&= -\frac{\rho_{e}}{\varepsilon_{0}}\\
\nabla^{2}\vec{A} - \frac{1}{c^{2}}\frac{\partial^{2}\vec{A}}{\partial t^{2}}
&= -\mu_{0}\vec{j}.
\end{align}
\end{subequations}
This makes manifest the wave structure of electric and magnetic fields,
since these are wave equations.

\begin{remark}
There are other gauge choices, and the exact form of the equations
depend on the gauge choice.
\end{remark}

\N{Four-Potential}
The first step towards describing electromagnetism in special relativity
is to use four-vectors. The potentials together form a four-vector
called the \define{Four-Potential} $A^{\mu} = (\varphi, \vec{A})$.
Conventions vary, some authors take $A^{t}=c^{-1}\varphi$. This implies
\begin{equation}
A_{\mu} = (-\varphi, \vec{A}).
\end{equation}

\begin{remark}
We can assemble a one-form called the Potential one-form
$A = \eta_{\mu\nu}A^{\mu}\,\D x^{\nu}$.
\end{remark}

\N{Partial Derivatives}
We have, in our conventions,
\begin{equation}
\partial^{\mu} := (-c^{-1}\partial_{t}, \vec{\nabla}).
\end{equation}
Then, in Cartesian coordinates,
\begin{equation}
\partial_{\nu} = \eta_{\mu\nu}\partial^{\mu} = (c^{-1}\partial_{t}, \vec{\nabla}).
\end{equation}

\N{Field-Strength Tensor}
We can form the field-strength tensor from the four-potential as
\begin{subequations}
\begin{equation}
F^{\mu\nu} := \partial^{\mu}A^{\nu} - \partial^{\nu}A^{\mu},
\end{equation}
or with indices downstairs
\begin{equation}
F_{\mu\nu} := \partial_{\mu}A_{\nu} - \partial_{\nu}A_{\mu}.
\end{equation}
\end{subequations}
The field strength tensor's components are then
\begin{subequations}
\begin{align}
F_{0j} &= \partial_{0}A_{j} - \partial_{j}A_{0}\\
&= \partial_{t}A_{j} - \partial_{j}(-\varphi)\\
&= -E_{j},
\end{align}
\end{subequations}
and
\begin{equation}
F_{ij} = \partial_{i}A_{j} - \partial_{j}A_{i} = \begin{pmatrix}
 0     &  B_{z} & -B_{y}\\
-B_{z} &  0     & B_{x}\\
 B_{y} & -B_{x} & 0
\end{pmatrix}
\end{equation}
(where $j$ indexes the columns, $i$ the rows). Then
\begin{equation}
F_{\mu\nu} = \begin{pmatrix}F_{00} & F_{0j}\\
F_{i0} & F_{ij}
\end{pmatrix} = \begin{pmatrix}
 0       & -E_{x}/c & -E_{y}/c & -E_{z}/c\\
 E_{x}/c &  0       &  B_{z}   & -B_{y}\\
 E_{y}/c & -B_{z}   &  0       &  B_{x}\\
 E_{z}/c &  B_{y}   & -B_{x}   &  0
\end{pmatrix}.
\end{equation}

\begin{exercise}
Compute the components of $F^{\mu\nu}=\eta^{\alpha\mu}\eta^{\beta\nu}F_{\alpha\beta}$.
\end{exercise}

\begin{exercise}
Expression $F_{\mu\nu}F^{\mu\nu}$ in terms of $\vec{E}$ and $\vec{B}$.
\end{exercise}

\N{Recovering Maxwell's Equations}
We see that
\begin{calculation}
  \partial^{\nu}F_{0\nu}
\step{since $F_{00}=0$}
  \partial^{j}F_{0j}
\step{since $F_{0j}=-E_{j}$}
  -\partial^{j}E_{j}
\step{by Gauss's Law}
  -4\pi\rho_{e}.
\end{calculation}
This gives us the first of Maxwell's equations.
We also see
\begin{calculation}
  \partial^{\nu}F_{i\nu}
\step{breaking $\mu$ up into $(0,j$)}
  \partial^{0}F_{i,t} + \partial^{j}F_{i,j}
\step{since $F_{i,t} = E_{i}$ and $F_{i,j} = \epsilon_{ijk}B^{k}$}
  \partial^{0}E_{i} + \partial^{j}\epsilon_{ijk}B^{k}
\step{since $\partial^{0}=-c^{-1}\partial_{t}$ and
    $\partial^{j}\epsilon_{ijk}B^{k} = (\nabla\times\vec{B})_{i}$}
  \frac{-1}{c}\partial_{t}E_{i} + \partial^{j}\epsilon_{ijk}B^{k}
\step{using Electrodynamic equation of motion}
  4\pi J_{i}.
\end{calculation}
This motivates a 4-vector for the current density called the \define{Four-Current}:
\begin{equation}
J_{\mu} = (-\rho_{e}, \vec{j}).
\end{equation}
Therefore two of Maxwell's equations for the electric field (both
electrostatics and electrodynamics) are encoded by
\begin{equation}
\boxed{\partial^{\nu}F_{\mu\nu} = 4\pi J_{\mu}.}
\end{equation}
We just need to express the remaining equations using the field strength
tensor.

\N{Equations for Magnetism}
Magnetodynamics may be written as
\begin{equation}
\partial_{t}\epsilon_{ijk}B^{k} + \partial_{i}E_{j} - \partial_{j}E_{i}
= 0
\end{equation}
for \emph{any} $i$, $j$, $k=1,2,3$. Then
\begin{calculation}
0
\step{Magnetodynamics equation}
\partial_{t}\epsilon_{ijk}B^{k} + \partial_{i}E_{j} - \partial_{j}E_{i}
\step{since $F_{ij} = \epsilon_{ijk}B^{k}$}
\partial_{t}F_{ij} + \partial_{i}E_{j} - \partial_{j}E_{i}
\step{since $E_{j}=F_{j0}$ and $-E_{i}=F_{0i}$}
\partial_{t}F_{ij} + \partial_{i}F_{j0} + \partial_{j}F_{0i}.
\end{calculation}
This suggests more generally (replacing the spatial indices with
4-indices and $t$ with another distinct index)
\begin{equation}\label{eq:relativity:electromagnetism:guessed-magnetic-equations}
0 \stackrel{???}{=} \partial_{\alpha}F_{\beta\gamma} + \partial_{\beta}F_{\gamma\alpha} + \partial_{\gamma}F_{\alpha\beta}.
\end{equation}
For $\alpha=0$, $\beta=i$, $\gamma=j$, this recovers Magnetodynamics.
Does this recover the magnetostatic Maxwell equation?

If any two indices are repeated (say $\alpha=\beta$), then $F_{\mu\nu}$
being antisymmetric implies Eq~\eqref{eq:relativity:electromagnetism:guessed-magnetic-equations}
holds trivially. Similarly, if $\alpha=\beta=\gamma$, then the equation
holds. Therefore, the only remaining cases are spatial indices. We see
\begin{calculation}
\partial_{x}F_{yz} + \partial_{y}F_{zx} + \partial_{z}F_{xy}
\step{since $F_{yz}=B_{x}$, $F_{zx}=B_{y}$, $F_{xy}=B_{z}$}
\partial_{x}B_{x} + \partial_{y}B_{y} + \partial_{z}B_{z}
\step{this is the divergence of the magnetic field}
\nabla\cdot\vec{B}
\step{by magnetostatics}
0.
\end{calculation}
Permuting the indices will just multiply through by $-1$, which doesn't
change the result. Therefore we conclude
\begin{equation}
\partial_{i}F_{jk} + \partial_{j}F_{ki} + \partial_{k}F_{ij} = 0,
\end{equation}
and moreover this encodes the magnetostatic Maxwell equation. Combining
things together, we find the magnetic Maxwell equations are:
\begin{equation}\label{eq:relativity:magnetic-eom}
\boxed{\partial_{\alpha}F_{\beta\gamma} + \partial_{\beta}F_{\gamma\alpha} + \partial_{\gamma}F_{\alpha\beta}
= 0.}
\end{equation}

\begin{exercise}
Prove Eq~\eqref{eq:relativity:magnetic-eom} holds identically for $F_{\mu\nu}=\partial_{\mu}A_{\nu}-\partial_{\nu}A_{\mu}$,
and therefore we don't need to worry about it when writing down a
Lagrangian for the electromagnetic field.
\end{exercise}

\subsection{Lagrangian for the Electromagnetic Field}

\begin{exercise}
Prove $\displaystyle\frac{\partial \left(\partial_{\mu} A_{\nu}\right)}{\partial\left(\partial_{\rho} A_{\sigma}\right)}= \delta_{\mu}^{\rho} \delta_{\nu}^{\sigma}$.
\end{exercise}

\begin{exercise}
Compute $\displaystyle\frac{\partial(F_{\alpha\beta}F^{\alpha\beta})}{\partial(\partial_{\mu}A_{\nu})}$.
Remember: $F^{\alpha\beta} = \eta^{\alpha\rho}\eta^{\beta\sigma}F_{\rho\sigma}$.
\end{exercise}

\begin{exercise}
Is it true or not that 
$\displaystyle\frac{\partial(F_{\alpha\beta}F^{\alpha\beta})}{\partial(A_{\nu})}=0$?
\end{exercise}

\M
Combining the results from these exercises, you can show the Lagrangian
density for Electromagnetism is
\begin{equation}\label{eq:relativity:lagrangian-for-em}
\mathcal{L} = \frac{-1}{4}F_{\mu\nu}F^{\mu\nu} + A_{\mu}J^{\mu}.
\end{equation}
The Euler--Lagrange equations for this would be (summing over $\beta$):
\begin{equation}
\partial_{\beta}\left[\frac{\partial\mathcal{L}}{\partial(\partial_{\beta}A_{\alpha})}\right]
-\frac{\partial\mathcal{L}}{\partial A_{\alpha}} = 0.
\end{equation}
This should recover $\partial_{\alpha}F^{\alpha\beta}=J^{\beta}$.

\begin{exercise}
Prove the Lagrangian density in Eq~\eqref{eq:relativity:lagrangian-for-em}
is Lorentz invariant.
\end{exercise}

\begin{exercise}
\begin{enumerate}
\item Find the canonically conjugate momenta density $\pi_{i} = \partial\mathcal{L}/\partial(\partial_{t}A^{i})$.
\item Rewrite $\mathcal{L}$ using $A^{i}$, $\pi_{j}$
\item Compute the Hamiltonian density $\mathcal{H} = \pi_{i}\partial_{t}A^{i}-\mathcal{L}$.
\end{enumerate}
\noindent \textsc{Hint}: you should have a constrained system, since the
Gauss law does not involve time derivatives. (Therefore we should expect
to find a constraint equivalent to the Gauss law.)
\end{exercise}


\N{References}
For the uninitiated, Taylor and Wheeler~\cite{Taylor:1992sp} is a great
introduction. Rindler~\cite{Rindler:1991sr} is a good review.
Misner, Thorne, and Wheeler's \textit{Gravitation} (ch.2) discusses
special relativity and we rely on its index notation conventions.
We also have double checked calculations against Srednicki~\cite{Srednicki:2007qs}.

There are a lot of subtleties to the Hamiltonian analysis of
electromagnetism and, more generally, any theory with symmetries. The
best review of the topic is the first 5 or so chapters of Henneaux and Teitelboim~\cite{Henneaux:1992ig}.

\N{TODO: Index Correct?}
I should double check the indices are correct for the Lorentz
transformation ${\Lambda^{\mu}}_{\nu}$ --- or is this the inverse of the
Lorentz transformation? After some investigation, we should have
$(\Lambda x)^{\mu} = {\Lambda^{\mu}}_{\nu}x^{\nu}$.
Rows are indexed by covariant indices, columns are indexed by
contravariant indices. MTW \S2.9 uses the notation
$x^{\alpha'} = {\Lambda^{\alpha'}}_{\mu}x^{\mu}$ and the inverse
transformation by $x^{\mu} = {\Lambda^{\mu}}_{\alpha'}x^{\alpha'}$.
Therefore composing these guys gives us
${\Lambda^{\mu}}_{\alpha'}{\Lambda^{\alpha'}}_{\nu}={\delta^{\mu}}_{\nu}$
and ${\Lambda^{\alpha'}}_{\mu}{\Lambda^{\mu}}_{\beta'} = {\delta^{\alpha'}}_{\beta'}$.

\N{TODO: Scattering}
It is probably good to discuss $2\to2$ scattering in special relativity,
since that's the basis of a lot of particle physics experiments.
