%%
%% net.tex
%% 
%% Made by Alex Nelson
%% Login   <alex@tomato>
%% 
%% Started on  Sat May 16 15:29:19 2009 Alex Nelson
%% Last update Sat May 16 15:29:19 2009 Alex Nelson
%%

\begin{figure}
\includegraphics{img.1}
\caption{A Simple Spin Network.}\label{fig:img1}
\end{figure}

Consider the simplest spin network, we need two nodes and three
edges (if we had two nodes and two edges, it'd be an
identity). This simplest nontrivial spin network is shown in
figure \eqref{fig:img1}. We don't want to have spin 0 for an
edge, if we did the propagator would be the identity. For each of
these lines (which we labeled $c_1$, $c_2$, and $c_3$ for
reference) we have the Wilson line 
\begin{equation}
\mathcal{U}=\mathcal{P}\exp\left[-\int A\right]
\end{equation}
We have the Wilson lines for each of these edges:
$\mathcal{U}^{m_{1}}_{1,n_{1}}$,
$\mathcal{U}^{m_{2}}_{2,n_{2}}$,
$\mathcal{U}^{m_{3}}_{3,n_{3}}$. The $\mathcal{U}$'s tell us
how a spin-half state rotates in the spin space. We have
$m_{1}=-1/2$, $1/2$ for spin up and spin down (respectively). We
see $m_{2}$ is also spin half, but $m_{3}$ is spin 1 with
possible values of -1, 0, 1. We can now use Clebsch-Gordon
Coefficients
\begin{equation}
\< j\,m|j_{1}\,m_{1},\; j_{2}\,m_{2}\>
\end{equation}
and find
\begin{equation}
\sum_{\substack{ m_1, m_2, m_3\\
n_1, n_2, n_3}} \mathcal{U}^{m_{1}}_{1,n_{1}}\mathcal{U}^{m_{2}}_{2,n_{2}}\mathcal{U}^{m_{3}}_{3,n_{3}}\<1\,m_{3}|\frac{1}{2}\,m_{1},\;\frac{1}{2}\,m_{2}\>\<1\,n_{3}|\frac{1}{2}\,n_{1},\;\frac{1}{2}\,n_{2}\>
\end{equation}
which is a function of $A$.

\begin{figure}
\includegraphics{img.2}
\caption{A spin network edge piercing a ``small'' surface.}\label{fig:img2}
\end{figure}
Now, we can take a surface $\Sigma$ and ask ``What is the area of
this surface?'' Suppose we have some spin network that ``goes
through'' our surface $\Sigma$. We won't consider any edge of the
spin networking ``grazing'' the surface (that it, just touching
it in one spot) or lying on the surface, we will only suppose
that the edges pierce the surface. We will only really consider a
simple example choosing a surface where $x^3=0$. The area of the
surface would be classically 
\begin{equation}
A = \int_{\Sigma}\sqrt{{}^{(2)}g}
\end{equation}
where ${}^{(2)}g$ is the determinant of the metric induced on the
surface. We see that
\begin{equation}
{}^{(2)}g = g_{11}g_{22}-g_{12}^{2} =
gg^{33}=\widetilde{E}^{3}_{\hat{I}}\widetilde{E}^{3\hat{I}}.
\end{equation}
So the area is
\begin{equation}
A = \int_{\Sigma} \sqrt{\widetilde{E}^{3}_{\hat{I}}\widetilde{E}^{3\hat{I}}}.
\end{equation}
Consider a more general surface with intrinsic coordinates
$\sigma^1$, $\sigma^2$. Then in this general setting, we have the
correspondence
\begin{equation}
\widetilde{E}^{3}_{\hat{I}}\to\varepsilon_{ijk}\frac{\partial
  x^{i}}{\partial \sigma^{1}}\frac{\partial x^{j}}{\partial
  \sigma^{2}}\widetilde{E}^{k}_{\hat{I}}.
\end{equation}
Lets define
\begin{equation}
\widetilde{E}_{\hat{I}}:=\int_{\text{small
    region}}d\sigma^{1}d\sigma^{2}\varepsilon_{ijk}\frac{\partial
  x^i}{\partial \sigma^1}\frac{\partial x^j}{\partial
  \sigma^2}\widetilde{E}^{k}_{\hat{I}}
\end{equation}
In the classical arena, the notion of a ``small region'' is ill
defined. However, in the quantum world, it's just a region where
one edge of a spin network pierces it. We can turn this into an
operator
\begin{equation}
\widetilde{E}_{\hat{I}}:=\int_{\text{small
    region}}d\sigma^{1}d\sigma^{2}\varepsilon_{ijk}\frac{\partial
  x^i}{\partial \sigma^1}\frac{\partial x^j}{\partial
  \sigma^2}\frac{\delta}{\delta A^{\hat{I}}_{k}}
\end{equation}
We need to consider 
\begin{equation}
\frac{\delta}{\delta A^{\hat{I}}_{k}}\mathcal{U}=???
\end{equation}
So we remember
\begin{equation}
\mathcal{U} = \mathcal{P}\exp\left[-\int A^{\hat{I}}_{i}\tau_{\hat{I}}\frac{dx^i}{ds}ds\right]
\end{equation}
If we didn't have the path ordering, this would be trivial, but
we need to be more careful since things don't commute. Consider
the path as shown in figure \eqref{fig:img3}. We have
\begin{subequations}
\begin{align}
\mathcal{U}(s_2,s_1) &= \mathcal{U}(s_2,s)\mathcal{U}(s,s_1)\\
&= \mathcal{U}(s_2,s+\varepsilon)\mathcal{U}(s+\varepsilon,s-\varepsilon)\mathcal{U}(s-\varepsilon,s_1)
\end{align}
\end{subequations}
If $\varepsilon$ is ``small enough'', we have
\begin{equation}
\frac{\delta}{\delta A^{\hat{I}}_{i}(s)}\mathcal{U}(s_2,s_1) =
\mathcal{U}(s_2,s)\left(-\tau_{\hat{I}}\frac{dx^i}{ds}\right)\mathcal{U}(s,s_1)
\end{equation}
More generally we can write
\begin{equation}
\frac{\delta}{\delta A^{\hat{I}}_{i}(x)}\mathcal{U}(s_2,s_1) =
\int ds \delta^{(3)}(C(s),x)\mathcal{U}(s_2,s)\left(-\tau_{\hat{I}}\frac{dx^i}{ds}\right)\mathcal{U}(s,s_1)
\end{equation}
where $C(s)$ is our path parametrized by $s$, we see this is 0 if
$x$ is not on the path. We see now that
\begin{equation}
E_{\hat{I}}\mathcal{U}(s_2,s_1) = 8\pi G\gamma \int d\sigma^1 d\sigma^2 ds \varepsilon_{ijk}\frac{\partial
  x^i}{\partial \sigma^1}\frac{\partial x^j}{\partial
  \sigma^2}\frac{\partial x^k}{\partial s}\delta^{(3)}(C(s),x)\mathcal{U}(s_2,s)\left(-\tau_{\hat{I}}\frac{dx^i}{ds}\right)\mathcal{U}(s,s_1)
\end{equation}
We also see that
\begin{equation}
\int d\sigma^1 d\sigma^2 ds \varepsilon_{ijk}\frac{\partial
  x^i}{\partial \sigma^1}\frac{\partial x^j}{\partial
  \sigma^2}\frac{\partial x^k}{\partial s}\delta^{(3)}(C(s),x)
\end{equation}
is called the ``oriented circle number'' (it's $\pm1$ if $C(s)$
intersects $\Sigma$, 0 otherwise).
\begin{figure}
\includegraphics{img.3}
\caption{An example path.}\label{fig:img3}
\end{figure}

The moral of the story is that the oriented intersection number
$I(C,\Sigma)$ is used to find $E_{\hat{I}}\mathcal{U}(s_2,s_1) =
k I(C,\Sigma)\mathcal{U}(s_2,s)\tau_{\hat{I}}\mathcal{U}(s,s_1)$
where $s$ is the point of intersection, and $k=8\pi\gamma G$.

Lets consider 
\begin{equation}
E_{\hat{I}}E^{\hat{I}}\mathcal{U}(s_2,s_1)=(8\pi\gamma G)^{2} \mathcal{U}(s_2,s)\tau_{\hat{I}}\tau^{\hat{I}}\mathcal{U}(s,s_1)
\end{equation}
We see that for SU(2), $\tau_{\hat{I}}\tau^{\hat{I}}$ is the
quadratic Casimir (it's not too surprising, it's kind of like
$J^2$ from quantum mechanics). So we can plug in $j(j+1)$ instead
and we end up with
\begin{equation}
E_{\hat{I}}E^{\hat{I}}\mathcal{U}(s_2,s_1)=(8\pi\gamma G)^{2}j(j+1) \mathcal{U}(s_2,s_1).
\end{equation}
This is assuming there is an intersection, of course. We can
write a spin network state $|s\rangle$ so
\begin{equation}
E_{\hat{I}}E^{\hat{I}}|s\rangle = \sum_{\text{intersections}}(8\pi\gamma G)^{2}j(j+1)|s\rangle
\end{equation}
We can now define the area operator
\begin{equation}
\widehat{A} = \sum_{\text{small regions of
  }\Sigma}\sqrt{E_{\hat{I}}E^{\hat{I}}}
\end{equation}
Classically we had
\begin{equation}
A = \int (E^{3}_{\widehat{I}}E^{3\widehat{I}})^{1/2}
\end{equation}
and an integral is nothing more than a continuous sum, so we see
that this is a sensible definition in the classical limit. Given
this area operator, we see that when it acts on a spin network
that
\begin{equation}
\widehat{A}_{\Sigma}|s\rangle =
\sum_{\text{intersections}}8\pi\gamma G\sqrt{j(j+1)}|s\rangle
\end{equation}
The spectrum of the area is discrete ($j$ comes in half integer
values). The spacing between the high $j$'s is ``smaller'' than
the spacing between the low $j$'s. There are some attempts at a
number theoretic explanation.

The volume operator can also be defined similarly using the
product of 3 $\widetilde{E}$'s instead of 2, but its construction
is horrible. The area operator has contributions from edges, but
the volume operator has contributions from vertices ``with enough
edges'' (4 edges at a node should be viewed as dual to a
tetrahedron, 3 edges has area but no volume). At this point, the
volume spectrum is not well understood.

If we have a cubic network, if we have too many edges the metric
``looks flat''. There are angle operators with fairly odd
properties. One fairly old but beautiful reference is~\cite{Rovelli:1998gg}.
