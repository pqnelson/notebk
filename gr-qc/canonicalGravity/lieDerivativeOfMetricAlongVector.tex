%%
%% lieDerivativeOfMetricAlongVector.tex
%% 
%% Made by Alex Nelson
%% Login   <alex@tomato>
%% 
%% Started on  Fri Jun  5 13:49:18 2009 Alex Nelson
%% Last update Fri Jun  5 13:49:18 2009 Alex Nelson
%%
The Lie derivative of the metric along a vector $\xi^{a}$ is
\begin{equation}\label{eq:lieDerivativeOfMetric}
\mathscr{L}_{\xi}g_{ab} =
g_{ac}\partial_{b}\xi^{c} + 
g_{bc}\partial_{a}\xi^{c} +
\xi^{c}\partial_{c}g_{ab}.
\end{equation}
Observe that
\begin{equation}\label{eq:covariantDerivativeContraction}
g_{bc}\nabla_{a}\xi^{c} = g_{bc}(\partial_a\xi^c + \Gamma^{c}_{ad}\xi^{d})
\end{equation}
where $\Gamma$ is the Christoffel symbol, $\nabla$ is the
covariant derivative. We specifically find
\begin{equation}\label{eq:firstManipulation}
g_{bc}\Gamma^{c}_{ad}\xi^{d} = \Gamma_{bad}\xi^{d}.
\end{equation}
For the affine connection, we have
\begin{equation}\label{eq:affineConnectionConditions}
\partial_{c}g_{ab} = \Gamma_{acb} + \Gamma_{bca} = 0.
\end{equation}
So we plug this into eq \eqref{eq:lieDerivativeOfMetric} to find
\begin{equation}\label{eq:lieDerivativeMutatisMutandi}
\mathscr{L}_{\xi}g_{ab} =
g_{ac}\partial_{b}\xi^{c} + 
g_{bc}\partial_{a}\xi^{c} +
\xi^{c}\left(\Gamma_{acb} + \Gamma_{bca}\right)
\end{equation}
By the properties of the Christoffel symbol, specifically
\begin{equation}%\label{eq:}
\Gamma_{cab} = \Gamma_{cba}
\end{equation}
we can rewrite eq \eqref{eq:covariantDerivativeContraction}
as
\begin{equation}%\label{eq:}
g_{bc}\nabla_{a}\xi^{c} = g_{bc}\partial_a\xi^c + \Gamma_{bad}\xi^{d}.
\end{equation}
Now observe the Lie derivative of the metric along our vector
$\xi^{a}$ can be grouped in terms
\begin{equation}%\label{eq:}
\mathscr{L}_{\xi}g_{ab} =
(g_{ac}\partial_{b}\xi^{c} + \Gamma_{abc}\xi^{c}) + 
(g_{bc}\partial_{a}\xi^{c} + \Gamma_{bac}\xi^{c})
\end{equation}
since the $c$ index is summed over, it's a dummy index. We can
rewrite this in more familiar terms
\begin{equation}%\label{eq:}
\mathscr{L}_{\xi}g_{ab} =
(g_{ac}\partial_{b}\xi^{c} + \Gamma_{abd}\xi^{d}) + 
(g_{bc}\partial_{a}\xi^{c} + \Gamma_{bad}\xi^{d})
\end{equation}
thus
\begin{equation}%\label{eq:}
\mathscr{L}_{\xi}g_{ab} = g_{ac}\nabla_{b}\xi^{c} + g_{bc}\nabla_{a}\xi^{c}.
\end{equation}
Since the connection is metric compatible, we can ``bring the
metric inside the derivative''
\begin{equation*}%\label{eq:}
g_{ab}\nabla_{c}(\cdots) \to \nabla_{c}(g_{ab}\cdots)
\end{equation*}
since $\nabla g_{ab} = 0$. Now we can rewrite our Lie derivative
as
\begin{equation}%\label{eq:}
\mathscr{L}_{\xi}g_{ab} = \nabla_{b}\xi_{a} + \nabla_{a}\xi_{b}.
\end{equation}
This is precisely the Killing equation.
