%%
%% canonicalAction.tex
%% 
%% Made by Alex Nelson
%% Login   <alex@tomato>
%% 
%% Started on  Fri Jun  5 12:08:09 2009 Alex Nelson
%% Last update Fri Jun  5 12:08:09 2009 Alex Nelson
%%
Now, we can write the action in terms of our new canonical
variables as
\begin{equation}%\label{eq:}
I = \int dt\; d^{3}x \left[\pi^{ij}\dot{q}_{ij} - N\mathscr{H} - N_{i}\mathscr{H}^{i}\right]
\end{equation}
where
\begin{equation}\label{eq:admHamiltonian}
\mathscr{H} = \frac{16\pi
  G_{N}}{\sqrt{q}}\left(\pi^{ij}\pi_{ij}-\pi^{2}\right) -
\frac{\sqrt{q}}{16\pi G_{N}}\sqrt{q}{}^{(3)}R
\end{equation}
and
\begin{equation}%\label{eq:}
\mathscr{H}^{i} = -2 D_{j}\pi^{ij}.
\end{equation}
Eq \eqref{eq:admHamiltonian} is a Hamiltonian for General
Relativity based off of a certain set of variables -- the metric
for a spatial hypersurface as the position variable, and its time
derivative as the canonically conjugate momenta.

We now find the dynamics in the usual way, by using the Poisson
bracket. We see that
\begin{equation}%\label{eq:}
\{ q_{ij}(x), \pi^{kl}(x') \} = \frac{1}{2}\left(\delta^{k}_{i}\delta^{l}_{j}+\delta^{k}_{j}\delta^{l}_{i}\right)\tilde{\delta}^{(3)}(x-x')
\end{equation}
where $\widetilde{\delta}^{(3)}$ is the densitized delta
function, i.e. the delta function such that
\begin{equation}%\label{eq:}
\int\tilde{\delta}^{(3)}(x)d^{3}x = 1
\end{equation}
so we won't need $\sqrt{q}$. Now, this is a completely
constrained system, with the momentum constraints generating
spatial change of coordinates. Consider the Poisson bracket of
the momentum constraints with the spatial metric:
\begin{subequations}
\begin{align}
\left\{\int\xi^{i}\mathscr{H}_{i}(x)d^{3}x,\; q_{kl}(x')\right\} &=  
\left\{-2\int\xi^{i}D^{j}\pi_{ij}(x)d^{3}x,\; q_{kl}(x')\right\}\\
&= \left\{\int(D_{i}\xi_{j}+D_{j}\xi_{i})\pi^{ij}(x)d^{3}x,\; q_{kl}(x')\right\}\\
&= -(D_{k}\xi_{l}+D_{l}\xi_{k})\\
&= -\mathscr{L}_{\xi}q_{kl}
\end{align}
\end{subequations}
where $\mathscr{L}_{\xi}$ is the Lie derivative. This means that
$\mathscr{H}_{i}$ are the generators of spatial coordinate
transformations. The Poisson bracket for the momentum constraints
and the $\pi^{ij}$ are a bit more complicated.

We are working on spatial hypersurfaces, so the question of \emph{what
``$\mathscr{H}$ generates time translations'' means} needs to be
investigated. Again, the easy Poisson bracket to consider is with
the spatial metric. Technically, these are ``spatial
deformations'' yielded by such a bracket.
