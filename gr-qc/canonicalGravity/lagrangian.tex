%%
%% lagrangian.tex
%% 
%% Made by Alex Nelson
%% Login   <alex@tomato>
%% 
%% Started on  Fri Jun  5 12:03:19 2009 Alex Nelson
%% Last update Fri Jun  5 12:03:19 2009 Alex Nelson
%%
Recall that for General Relativity, the Einstein Hilbert action
is
\begin{equation}%\label{eq:}
I = \frac{1}{16\pi G_{N}}\int d^{4}x\sqrt{|g|}R
\end{equation}
where $G_N$ is Newton's constant, $g$ is the determinant of the
metric, and $R$ is the Ricci scalar. We demand the action
vanishes upon variation of the metric
\begin{equation}%\label{eq:}
g_{\mu\nu}\to g_{\mu\nu}+\delta g_{\mu\nu}
\end{equation}
and we obtain Einstein's field equations this way. We use a very
specific variation in the metric, namely a diffeomorphism
invariant variation
\begin{equation}%\label{eq:}
\delta g_{\mu\nu} = \nabla_{\nu}\xi_{\mu}+\nabla_{\mu}\xi_{\nu}
\end{equation}
which is precisely the killing equation for diffeomorphism
invariance.

Naively, one could suppose that we can do the canonical formalism
in a covariant way by sleight of hand. We just pretend the
canonical position is the 4 metric $g_{\mu\nu}$ and find its time
derivative, then apply Dirac's constrained dynamics scheme. This
was the approach of Pirant, Schild, and Skinner~\cite{Pirant:1952zz}. Their
analysis was not complete, as Kiriushcheva et al noted~\cite{Kiriushcheva:2008fn}
the time development of secondary constraints were not
considered. Neither was the closure of the Dirac procedure
demonstrated. Dirac attacked this problem~\cite{Dirac:1958sc} and
came to the conclusion that four dimensions should be split into
three plus one dimensions.

