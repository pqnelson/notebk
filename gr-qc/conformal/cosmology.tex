%%
%% cosmology.tex
%% 
%% Made by Alex Nelson
%% Login   <alex@tomato>
%% 
%% Started on  Sun Dec  7 19:30:38 2008 Alex Nelson
%% Last update Wed Dec 10 02:01:26 2008 Alex Nelson
%%

We begin by thinking about breaking symmetry differently (read:
in the naive way) by considering\footnote{Note that this is for
  De-Sitter spacetime, to make this anti-de-Sitter spacetime we
  need to change the sign of the $\phi^{4}$ term. This has been
  calculated in \cite{Edery:2006hg}.} the Lagrangian of matter
  conformally coupled to gravity~\cite{Mannheim:1999nc,Mannheim:2007ki} 
\begin{equation}\label{symmetryBreakingAction}
I_{M} = -\int
d^{4}x\sqrt{-g}\left[\underbrace{\frac{1}{2}\nabla^{\mu}\phi\nabla_{\mu}\phi -
  \frac{1}{12}\phi^{2}R + \lambda \phi^{4}}_{\text{scalar}} +
  \underbrace{i\bar{\psi}\gamma^{\mu}(x)[\partial_{\mu} +
    \Gamma_{\mu}(x)]\psi}_{\text{fermion}} - \underbrace{g\phi\bar{\psi}\psi}_{\text{interaction}}\right]
\end{equation}
where $\Gamma_{\mu}(x)$ is the fermion spin connection, $\lambda$ and
$g$ are the dimensionless coupling constants, $\phi(x)$ is the (symmetry
breaking) scalar field and $\psi$ is a fermionic field. 

We will demonstrate that the scalar spontaneously breaks
symmetry. Observe that the potential term for the scalar field is 
\begin{equation}
V(\phi) = \frac{\phi^{2}R}{12} - \lambda\phi^{4}
\end{equation}
we take its derivative 
\begin{equation}
V'(\phi) = \frac{\phi R}{6} - 4\lambda\phi^3
\end{equation}
then set it to zero and solve for $\phi$. The resulting value is
\begin{equation}
v = \pm\sqrt{\frac{R}{24\lambda}}
\end{equation}
then we plug it back into the potential to find
\begin{equation}
V\left(\pm\sqrt{\frac{R}{24\lambda}}\right)
= \left(\frac{R}{24\lambda}\right)\frac{R}{12} - \lambda\left(\frac{R}{24\lambda}\right)^{2} = \frac{R^2}{576\lambda}
\end{equation}
which is nonzero, which implies that symmetry is spontaneously
broken. Note that if we included the fermion-scalar interaction term,
the results would not have changed as it would have been equivalent to
adding a term linear in $\phi$ into the potential (for explicit
calculations refer to appendix B). (Observe the dependence on $R$ is
directly proportional too.)

When the scalar field $\phi(x)$ in $I_M$ obtains a nonzero mass (which
we are free to rotate to some ``spacetime constant'' $\phi_{0}$ due to
conformal invariance), the fermion then obeys the curved space Dirac
equations
\begin{equation}
i\hbar\overline{\psi}\gamma^{\mu}(x)(\partial_{\mu}
+ \Gamma_{\mu}(x))\psi = \hbar g\phi_{0}\psi
\end{equation}
and acquires a mass $\hbar g\phi_{0}$. The scalar field's equation of
motion is
\begin{equation}
\nabla_{\mu}\nabla^{\mu} \phi + \frac{\phi R}{6} - 4\lambda \phi^{3} +
g\bar{\psi}\psi = 0.
\end{equation}

The corresponding stress-energy tensor to \eqref{symmetryBreakingAction} is
\begin{eqnarray}
T^{\mu\nu} &=& \hbar\Big[i\bar{\psi}\gamma^\mu(\partial^\nu
+ \Gamma^\nu)\psi + \frac{2}{3}\nabla^{\mu}\phi\nabla^{\nu}\phi
- \frac{g^{\mu\nu}}{6}\nabla^{\alpha}\phi\nabla_{\alpha}\phi
-\frac{\phi\nabla^{\mu}\nabla_{\nu}\phi}{3} \nonumber\\
& & + \frac{g^{\mu\nu}\phi\nabla^{\alpha}\nabla_{\alpha}\phi}{3}
- \frac{\phi^2}{6}(R^{\mu\nu} -\frac{g^{\mu\nu}}{2}R) - g^{\mu\nu}\lambda\phi^{4}\Big]
\end{eqnarray}
which can be rewritten grouping terms in a more elegant manner. If we
think of 
\begin{equation}
\rho u^\mu u^\nu = i\hbar\bar{\psi}\gamma^\mu(\partial^\nu
+ \Gamma^\nu)\psi + \frac{\hbar}{2}\nabla^{\mu}\phi\nabla^{\nu}\phi
\end{equation}
where $\rho$ is the ``pressure'' of an ideal fluid, $u^\mu$ is thought of
as the worldline (so it satisfies $g_{\mu\nu}u^\mu u^\nu=-1$), and
\begin{equation}
pu^\mu u^\nu = \frac{-\hbar}{3}\phi\nabla^{\mu}\nabla^{\nu}\phi + \frac{\hbar}{6}\nabla^{\mu}\phi\nabla^{\nu}\phi
\end{equation}
where $p$ is the ``pressure'' of an ideal fluid, then the stress
energy tensor may be written as
\begin{equation}\label{stressEnergyTensorConformalCosmology}
T_{\mu\nu} = (\rho + p)U_{\mu}U_{\nu} + pg_{\mu\nu} -
\frac{1}{6}\phi_{0}^{2}\left(R_{\mu\nu} -
\frac{\hbar}{2}g_{\mu\nu}R\right) - g_{\mu\nu}\hbar\lambda \phi_{0}^{4}.
\end{equation}
This is the right hand side of our fourth order field equations.

By working in an isotropic and homogeneous geometry, the left hand side of
\eqref{stressEnergyTensorConformalCosmology} necessarily
vanishes, giving us the equation
\begin{equation}
\frac{1}{6}\phi^{2}\left(R_{\mu\nu} -
\frac{1}{2}g_{\mu\nu}R\right) = (\rho + p)U_{\mu}U_{\nu} +
pg_{\mu\nu}  - g_{\mu\nu}\lambda \phi^{4}
\end{equation}
Thus conformal cosmology looks like the standard cosmology with a
perfect matter fluid and a nonzero cosmological constant with the
\marginpar{\footnotesize{Replace Newton's $G_{N}$ with an effective
one}}important exception that Newton's constant has been replaced by
an ``effective'' constant of the form 
\begin{equation}
G_\text{eff} = \frac{-3}{4\pi \phi_{0}^{2}}.
\end{equation}
\emph{This is not Newton's constant as Cavendish measured, but instead
a term which we identify to be analagous to the Newton gravitational
constant!} Observe that as we change scales, $G_\text{eff}$ changes in inverse
proportion. 

We can also identify the\marginpar{\footnotesize{Cosmological Constant Emerges}} $\Lambda = \lambda \phi_{0}^{4}$ term as being a
cosmological constant. Note that this term really is effectively a
cosmological constant since it is a homogeneous and isotropic global
scalar field. Observe that this cosmological constant
is \begin{inparaenum}
\item scale dependent (that is, quartic in $\phi$),
\item always positive (that is, we have de Sitter spacetime, so
gravity is repulsive \emph{but at this scale}).
\end{inparaenum}
The notion that the cosmological constant is scale dependent is novel,
but the important change is that the sign explains how gravity is
repulsive instead of attractive. Due to the sign, there is no initial
singularity in this model. Instead the universe expands from a finite
minimum radius, and is not subject to the same problems that one
encounters in the standard cosmological model.

Despite finding gravity being globally repulsive, it is locally
attractive. This reconciles the use of the fourth order Poisson
equation \eqref{fourthOrderPoisson} which merely adds an extra term linear in $r$
(radial distance) to the gravitational potential that would be
negligible at \emph{local} scales. It turns out that Mannheim et
al~\cite{Mannheim:1996jt} demonstrate the empirical strength of such a
proposition at the galactic level, but that is beyond the scope of
this article to review it too.
