%%
%% intro.tex
%% 
%% Made by Alex Nelson
%% Login   <alex@tomato>
%% 
%% Started on  Wed Jun  3 16:30:09 2009 Alex Nelson
%% Last update Wed Jun  3 16:30:09 2009 Alex Nelson
%%
Recall for classical mechanics, the Harmonic oscillator potential
is
\begin{equation}%\label{eq:}
V(x) = \frac{1}{2}kx^2 = \frac{1}{2}m\omega^{2}x^{2}
\end{equation}
where $\omega$ is the angular velocity. We plug this into
Schrodinger's equation
\begin{equation}%\label{eq:}
\left[\frac{-\hbar^2}{2m}\frac{\partial^{2}}{\partial x^{2}}+\frac{1}{2}kx^{2}\right]|\psi\>=E|\psi\>.
\end{equation}
How to solve this? Well, there are two ways: the smart way and
the stupid way. We'll do it the smart way.

We introduce a change of variable\footnote{Note that this is done
  \emph{classically}, that is \emph{before} quantization. After
  quantization changing coordinates is always a fuzzy
  subject. These two steps are done tacitly in most derivations,
  but it should be known in the back of one's mind what's going on.}:
\begin{equation}%\label{eq:}
Q = \sqrt{\frac{m\omega}{2\hbar}},\qquad
\widehat{P}=\frac{\partial}{\partial Q}=\sqrt{\frac{-1}{2m\omega\hbar}}\widehat{p}.
\end{equation}
Note these are dimensionless and simplify computations
significantly. We can factor the Hamiltonian, since in these new
variables we have
\begin{equation}%\label{eq:}
\hbar\omega\left[Q^{2}-\frac{\partial^2}{\partial Q^2}\right] = \widehat{H}.
\end{equation}
We want to use the coordinates
\begin{equation}%\label{eq:}
a=\left[Q+\frac{\partial}{\partial Q}\right],\qquad a^{\dag}=\left[Q-\frac{\partial}{\partial Q}\right]
\end{equation}
which we call ``\define{annihilation and creation operators}''
respectively.

Observe that
\begin{subequations}
\begin{align}
2a^{\dag}a &= \left[Q+\frac{\partial}{\partial
    Q}\right]\left[Q-\frac{\partial}{\partial Q}\right]\\
&= Q^2 + \frac{\partial}{\partial Q} Q -
Q\frac{\partial}{\partial Q} - \frac{\partial^{2}}{\partial
  Q^{2}}\\
&= Q^{2}-\frac{\partial^2}{\partial Q^2} + [\widehat{P},Q].
\end{align}
\end{subequations}
All computation is by definition and substitution. Nothing too
fancy so far. We can now write the Hamiltonian operator as
\begin{equation}%\label{eq:}
\widehat{H} = \hbar\omega\left(a^{\dag}a+\frac{1}{2}[Q,\widehat{P}]\right)
\end{equation}
Observe that
\begin{equation}%\label{eq:}
[Q,\widehat{P}] = 1
\end{equation}
since $Q$ and $\widehat{P}$ are the dimensionless counterparts to
$x$ and $\widehat{p}$ which implies we set $\hbar\to1$. We
end up with the form of the Hamiltonian operator
\begin{equation}%\label{eq:}
\widehat{H} = \hbar\omega\left(a^{\dag}a+\frac{1}{2}\right).
\end{equation}
It would then be logical to investigate how these creation and
annihilation operators behave.
