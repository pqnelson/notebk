\section{Differential Geometry of Curves in $\RR^{3}$ (or $\RR^{n}$)}

\M
The idea is that we have introduced the basic gadgetry of differential
geometry, but in the setting of $\RR^{n}$. Now we will consider curves
in $\RR^{n}$, and use the gadgetry we've introduced to study properties
of curves (for example, how to vector fields on a curve, and what do
they tell us). This appears in classical mechanics (especially
Lagrangian and Hamiltonian mechanics).

\begin{definition}
An \define{Unparametrized Curve} in $\RR^{3}$ is a ``one-dimensional
subset'' of points.
\end{definition}

\begin{example}
The doodle below is a closed unparametrized curve --- ``closed'' meaning
it forms a ``loop'' (eventually):
\begin{center}
  \includegraphics{img/img.0}
\end{center}
\end{example}

\begin{example}
Here is a happy open unparametrized curve --- ``open'' meaning it is
``not closed'':
\begin{center}
  \includegraphics{img/img.1}
\end{center}
\end{example}

\begin{definition}
Let $I=(a,b)$ be an open interval (it is possible $a=-\infty$, or
$b=+\infty$, or both). We define a \define{(Parametrized) Curve} in
$\RR^{3}$ to be a smooth function $\alpha\colon I\to\RR^{3}$
\end{definition}

\begin{remark}
Henceforth, we will reserve the term ``curve'' for an unparametrized
curve, and ``path'' for a parametrized curve.
\begin{itemize}
\item ``Path'' = ``Parametrized Curve''
\item ``Curve'' = ``Unparametrized Curve''.
\end{itemize}
In fact, almost always we care about curves, so unless otherwise stated,
all curves are parametrized.
\end{remark}

\begin{example}
Consider the \define{Elliptic Helix}, a parametrized curve $\alpha(t)=(a\cos(t),b\sin(t),ct)$ where
$a,b,c\in\RR$ are positive constants, and $t\in[0,+\infty)$ looks like:
\begin{center}
  \includegraphics{img/img.3}
\end{center}
We could consider a different parametrization of the same curve, like
$\beta(t)=(a\cos(3t),b\sin(3t),3ct)=\alpha(3t)$. How do we know this is
the same curve? Well, one way is to establish a bijection of points
$\beta(t/3)=\alpha(t)$ for all $t\in[0,\infty)$.
\end{example}

\begin{definition}
Let $\alpha\colon I\to\RR^{3}$ be a path with components
$(\alpha_{1},\alpha_{2},\alpha_{3})$. The \define{Velocity} of $\alpha$
at time $t\in I$ is the tangent vector
\begin{subequations}
  \begin{equation}
\alpha'(t) = \left(\frac{\D\alpha(t)}{\D t}\right)_{\alpha(t)}\in\T_{\alpha(t)}\RR^{3}
\end{equation}
at base point $\alpha(t)$, whose components are
\begin{equation}
\alpha'(t)=\left(\frac{\D\alpha_{1}(t)}{\D t},
\frac{\D\alpha_{2}(t)}{\D t},
\frac{\D\alpha_{3}(t)}{\D t}\right).
\end{equation}
\end{subequations}
\end{definition}

\begin{remark}
The velocity defines a sort of vector field, but defined only on the
curve and not all of $\RR^{3}$.
\end{remark}

\begin{example}
  For the elliptical helix,
  $\alpha(t)=(a\cos(t),b\sin(t),ct)$ for $t\in\RR$,
  we have its velocity be
  \begin{equation}
\alpha'(t) = (-a\sin(t),b\cos(t),c)_{\alpha(t)}.
  \end{equation}
\end{example}


\begin{example}
  Let $\vec{p},\vec{v}\in\RR^{3}$ be constants.
  Consider the curve $\beta(t)=\vec{p}+t\vec{v}$. This is the straight
  line with initial position $\beta(0)=\vec{p}$ and initial velocity $\beta'(0)=\vec{v}_{\vec{p}}$.
  We've used this before when we've worked with the directional
  derivative of $f\colon\RR^{3}\to\RR$,
  \begin{equation}
\vec{v}_{\vec{p}}[f] = \left.\frac{\D}{\D t}f(\vec{p}+t\vec{v})\right|_{t=0}.
  \end{equation}
  In fact, we could use \emph{any} curve $\alpha$ with
  $\alpha(0)=\vec{p}$ and $\alpha'(0) = \vec{v}_{\vec{p}}$ to define $\vec{v}_{\vec{p}}[f]$.
\end{example}

\begin{theorem}
Let $\vec{v}_{\vec{p}}\in\T_{\vec{p}}\RR^{3}$, $I$ be an interval
containing zero, and let $\alpha\colon I\to\RR^{3}$ be such that
$\alpha(0)=\vec{p}$ and $\alpha'(0) = \vec{v}_{\vec{p}}$. Then
\begin{equation}
\vec{v}_{\vec{p}}[f] = \left.\frac{\D}{\D t}f(\alpha(t))\right|_{t=0}.
\end{equation}
\end{theorem}

\begin{proof}
We know $\alpha'(0) = \vec{v}_{\vec{p}}$. So, let us calculation
\begin{calculation}
  \vec{v}_{\vec{p}}[f]
\step{definition of directional derivative}
  \left.\frac{\D}{\D t}f(\vec{p}+t\vec{v})\right|_{t=0}
\step{since $\alpha(t)=\vec{p}+t\vec{v}$}
  \left.\frac{\D}{\D t}f(\alpha(t))\right|_{t=0}
\step{chain rule}
  \sum_{j}\left.\frac{\partial f}{\partial x_{j}}(\alpha(t))\frac{\D\alpha^{j}(t)}{\D t}\right|_{t=0}
\step{since $\D\alpha^{j}(0)/\D t=v^{j}$, $\alpha(0)=\vec{p}$}
  \sum_{j}\frac{\partial f}{\partial x_{j}}(\vec{p})v^{j}
\end{calculation}
On the other hand, repeating the last three steps with
$$\left.\frac{\D}{\D t}f(\alpha(t))\right|_{t=0}$$
gives the same result since the only facts used were $\alpha(0)=\vec{p}$, $\alpha'(0)=\vec{v}_{\vec{p}}$.
\end{proof}

\begin{corollary}
For any curve $\alpha$ and smooth function $f\colon\RR^{3}\to\RR$, we
have
\begin{equation}
\alpha'(t)[f] = \left.\frac{\D}{\D s}f(\alpha(s))\right|_{s=t}.
\end{equation}
That is to say, the directional derivative of $f$ with respect to the
velocity vector field is the rate of change of $f$ as we move along the
curve $\alpha$.
\end{corollary}

\subsection{Metric, Distances, Angles}

\M
In linear algebra, the key geometric tool is the concept of the inner
product (``dot product''). Any vector space with an inner product
automatically gets notations of:
\begin{itemize}
\item \textbf{magnitude} of vectors $\|\vec{v}\| = \sqrt{\langle\vec{v},\vec{v}\rangle}$,
and
\item \textbf{angle} between vectors $\cos(\theta) = \langle\vec{v},\vec{w}\rangle/(\|\vec{v}\|\cdot\|\vec{w}\|)$.
\end{itemize}
In differential geometry, we have not just \emph{one} vector space, but
a vector space \emph{at each point} (e.g., for each $\vec{p}\in\RR^{3}$,
we have $\T_{\vec{p}}\RR^{3}$). We can, in principle, put a different
inner product at each of these tangent spaces. In other words, at each
point we may have a different notion of magnitude and angle.

\begin{definition}
An assignment of an inner product to each $\T_{\vec{p}}\RR^{3}$ (varying
smoothly with $\vec{p}\in\RR^{3}$) is called a \define{Riemannian metric}.
\end{definition}

\begin{example}
For now, we will be sticking with the usual Riemannian metric on
$\RR^{3}$ given by
\begin{equation}
\underbrace{\langle\vec{v}_{\vec{p}},\vec{w}_{\vec{p}}\rangle_{\vec{p}}}_{\text{inner product on }\T_{\vec{p}}\RR^{3}}=
\overbrace{\vec{v}\cdot\vec{w}}^{\text{usual dot product}}
\end{equation}
\end{example}

\begin{example}
  Suppose we have two curves $\alpha\colon I\to \RR^{n}$ and
  $\beta\colon J\to\RR^{n}$ which intersects at a point
  \begin{equation}
\vec{p} = \alpha(s_{0}) = \beta(t_{0}).
  \end{equation}
\begin{center}
\includegraphics{img/img.4}
\end{center}
We can define the angle between $\alpha$ and $\beta$ at $\vec{p}$ by
\begin{equation}
  \cos(\theta) = \frac{\langle\alpha'(s_{0}),\beta'(t_{0})\rangle}{\|\alpha'(s_{0})\|\cdot\|\beta'(t_{0})\|}.
\end{equation}
This breaks down if $\|\beta'(t_{0})\|=0$ or $\|\alpha'(s_{0})\|=0$.
For this reason, we usually work with \define{Regular Curves} which always have
nonzero velocity.
\end{example}

\begin{definition}
We call a curve $\alpha\colon I\to\RR^{n}$ \define{Regular} if its
velocity is never zero $\alpha'(t)\neq 0$ for all $t\in I$.
\end{definition}

\begin{example}
  An example of a singular curve would be one with a cusp, for example,
  $x^{2}-y^{5}=0$

\begin{center}
  \includegraphics{img/img.5}
\end{center}
\end{example}

\N{Measuring Distance}
The metric also allows us to measure distances along regular curves. The
distance is found by integrating the \define{Speed} or magnitude of
velocity. Let $\alpha\colon(a,b)\to\RR^{n}$ be a regular curve, then
\begin{equation}
\int^{b}_{a}\|\alpha'(t)\|\,\D t.
\end{equation}
A key fact in this for this to be well-defined, we need to check it's
independent of (regular) parametrization. That is to say, suppose we
have
\begin{equation}
t = t(s)
\end{equation}
where $s$ is another parameter, and define a \define{Reparametrization}
of the curve by
\begin{equation}
\beta(s) = \alpha(t(s))
\end{equation}
where
\begin{equation}
\frac{\D t(s)}{\D s}>0
\end{equation}
for all $s\in J$, so $\beta$ is regular. Suppose $J=(a',b')$.
Then we have to integrate from $a'$ to $b'$ of the magnitude of the
velocity of $\beta$ and demand they be equal
\begin{equation}
\int^{b'}_{a'}\|\beta'(s)\|\,\D s = \int^{b}_{a}\|\alpha'(t)\|\,\D t.
\end{equation}
We see, actually, we could just unfold the definition of $\beta$ and use
the chain rule,
\begin{equation}
\int^{b'}_{a'}\|\beta'(s)\|\,\D s = \int^{b'}_{a'}\|\alpha'(t(s))\frac{\D
  t(s)}{\D s}\|\,\D s.
\end{equation}
Then using substitution rule for calculus on the right-hand side
\begin{equation}
\int^{b'}_{a'}\|\alpha'(t(s))\frac{\D t(s)}{\D s}\|\,\D s
 = \int^{b}_{a}\|\alpha'(t)\|\,\D t.
\end{equation}
We have a particularly useful parametrization:

\begin{proposition}
  Any regular curve $\alpha$ has a reparametrization $\beta$ such that
  \begin{equation}
\|\beta'(s)\|=1
  \end{equation}
  for all $s$, called the \define{Unit Speed Reparametrization}.
\end{proposition}

\begin{proof}
Let $\alpha\colon I\to\RR^{3}$ be our given curve. Suppose $I=(a,b)$.
Define the distance function
\begin{equation}
s(t) = \int^{t}_{a}\|\alpha'(t)\|\,\D t.
\end{equation}
Since $\|\alpha'(t)\|\neq0$ (and speed is never negative, it follows the
speed is always positive), hence $s$ is strictly increasing. In
particular, this means
\begin{equation}
\frac{\D s}{\D t}>0.
\end{equation}
So far, so good.

This also means $s(t)$ has an inverse. So we can write $t$ as a function
of s,
\begin{equation}
t = t(s).
\end{equation}
Then we can reparametrize by letting
\begin{equation}
\beta(s) = \alpha(t(s)),
\end{equation}
and
\begin{calculation}
  \|\beta'(s)\|
\step{chain rule}
  \displaystyle\left\|\alpha'(t(s))\frac{\D t(s)}{\D s}\right\|
\step{since $\D t(s)/\D s>0$ always}
  \displaystyle\left\|\alpha'(t(s))\right\|\frac{\D t(s)}{\D s}
\step{differentiating under the integral sign}
  \displaystyle\frac{\D s}{\D t}\frac{\D t}{\D s}
\step{basic calculus}
  1.
\end{calculation}
Hence $\beta$ has unit-speed.
\end{proof}

\begin{remark}
For any regular curve, we get a vector field on the curve called the
\define{Unit Tangent Field} defined by taking the velocity of a unit
speed reparametrization.
\end{remark}

\subsection{Frenet Frame}
% HW 3:
% 1.4 #2
% 2.2 #3,4,5
% 2.3 #1,5,6

\M
The basic idea is to study a curve by using a different frame at each
point, suitably chosen at each point. Towards that end, we should
probably make rigorous what we mean by a ``vector field along a curve''
and whatnot.

\begin{definition}
Let $\alpha\colon I\to\RR^{3}$ be a (regular) curve.
A \define{Vector Field on $\alpha$} $Y$ is an assignment of a tangent
vector $Y(t)\in\T_{\alpha(t)}\RR^{3}$ for each $t\in I$.

An \define{(Orthonormal) Frame Field on a curve} $\alpha$ is a triple of
vector fields such that for each $t\in I$ they restrict to an
(orthonormal) basis of $\T_{\alpha(t)}\RR^{3}$.

We assume, without loss of generality, that frame fields are orthonormal
unless explicitly stated otherwise.
\end{definition}

\M
For now, we restrict attention to unit speed curves.

We have constructed on natural unit vector field on $\beta$:
\begin{equation}
T(s) = \beta'(s)\in\T_{\beta(s)}\RR^{3}.
\end{equation}
We call $T(S)$ the \define{Unit Tangent}.
To get a frame field, we need two more vector fields on $\beta$.

First note,
\begin{subequations}
  \begin{equation}
T(s)\cdot T(s)=1,
  \end{equation}
  because $T(s)$ is a unit tangent vector. It follows then that
  \begin{align}
    \frac{\D}{\D s}(T(s)\cdot T(s))
    &= \frac{\D T(s)}{\D s}\cdot T(s) +
    T(s)\cdot\frac{\D T(s)}{\D s}\\
    &= 0,
  \end{align}
  and in particular
  \begin{equation}
\boxed{T(s)\cdot T'(s)=0.}
  \end{equation}
\end{subequations}
If $T'(s)\neq 0$, then define the \define{Principal Normal} field on
$\beta$ as
\begin{equation}
N(s) := \frac{T'(s)}{\|T'(s)\|}.
\end{equation}
We see (since the right-hand side is a vector divided by its norm)
$N(s)$ is a unit vector field, i.e., for any $s$ we have
\begin{equation}
\|N(s)\|=1,
\end{equation}
and it is orthogonal to $T(s)$ by construction.

Now, to obtain one last unit vector field on $\beta$, we can just take
the cross-product of $T$ and $N$ to obtain the \define{Binormal Field}
on $\beta$ as
\begin{equation}
B(s) := T(s)\times N(s).
\end{equation}

\begin{definition}
Let $\beta$ be a unit speed curve, then we define the \define{Curvature}
of $\beta$ as
\begin{equation}
\kappa(s) := \|T'(s)\|.
\end{equation}
\end{definition}

\begin{remark}
If the curvature of $\beta$ is zero, we get nonunique frames from the
construction we have just sketched.
\end{remark}


\begin{definition}
Along a regular curve $\beta$, we define the \define{Frenet Frame}
to consist of $T(s) := \beta'(s)$, $N(s) := T'(s)/\|T'(s)\|$, and
$B(s) := T(s)\times N(s)$.
\end{definition}

\M
For any regular curve with unit-speed parametrization $\beta$,
we know that $T'(s)$ is proportional to $N(s)$. We claim that $B'(s)$ is
also proportional to $N(s)$. How can we see this? We will prove $B'(s)$
is orthogonal to both $B(s)$ and $T(s)$, which means it's either zero or
directly proportional to the remaining orthonormal unit vector $N(s)$.

First, observe that $B'(s)$ is orthogonal to $B(s)$. How? We see from it
being a unit vector,
\begin{subequations}
\begin{equation}
B(s)\cdot B(s) =1,
\end{equation}
taking the derivative with respect to $s$ of both sides,
\begin{equation}
\frac{\D}{\D s}(B(s)\cdot B(s)) = 0,
\end{equation}
the left-hand side expands according to the product rule as
\begin{equation}
\frac{\D}{\D s}(B(s)\cdot B(s)) = \frac{\D B(s)}{\D s}\cdot B(s) +
B(s)\cdot\frac{\D B(s)}{\D s}.
\end{equation}
Thanks to commutativity of the dot product, the right hand side
simplifies to:
\begin{equation}
\frac{\D}{\D s}(B(s)\cdot B(s)) = 2\frac{\D B(s)}{\D s}\cdot B(s).
\end{equation}
Thus we find
\begin{equation}
2\frac{\D B(s)}{\D s}\cdot B(s) = 0\implies\frac{\D B(s)}{\D s}\cdot B(s)=0.
\end{equation}
\end{subequations}
Thus we conclude $B'(s)$ is orthogonal to $B(s)$.

Our second step is to show $B'(s)$ is orthogonal to $T(s)$.
We have
\begin{subequations}
\begin{equation}
\frac{\D}{\D s}(B\cdot T) = B'(s)\cdot T(s) + B(s)\cdot T'(s).
\end{equation}
Now since
\begin{equation}
T'(s) = \kappa(s)N(s),
\end{equation}
we find
\begin{equation}
B'(s)\cdot T(s) + B(s)\cdot T'(s) = B'(s)\cdot T(s) + B(s)\cdot(\kappa(s)N(s)).
\end{equation}
But since
\begin{equation}
B(s)\cdot(\kappa(s)N(s)) = \kappa(s)(B(s)\cdot N(s)) = \kappa(s)(0) = 0,
\end{equation}
we find
\begin{equation}
B'(s)\cdot T(s) + B(s)\cdot T'(s) = B'(s)\cdot T(s).
\end{equation}
But remember, $B(s)\cdot T(s) = (T(s)\times N(s))\cdot T(s)$ using the
definition of the binormal field, and recalling the basic property of
the cross product (it produces a vector orthogonal to its factors), we
find
\begin{equation}
B(s)\cdot T(s) = 0.
\end{equation}
Therefore its derivative with respect to $s$ vanishes, and we find
\begin{equation}
B'(s)\cdot T(s) = 0.
\end{equation}
\end{subequations}
Hence $B'(s)$ is orthogonal to $T(s)$.

Since $B'(s)$ is orthogonal to both $T(s)$ and $B(s)$, we conclude it
must be directly proportional to $N(s)$:
\begin{equation}
B'(s) = -\tau(s)N(s)
\end{equation}
where ``$-\tau(s)$'' is the constant of proportionality. We call
$\tau(s)$ the \define{Torsion} of the curve $\beta$.

\begin{theorem}[The Frenet Formulas]
Given a unit speed curve $\beta$, whose curvature $\kappa$ is
nonvanishing and whose torsion is $\tau$, we have the \define{Frenet Formulas}:
\begin{subequations}
  \begin{alignat}{2}
    T'(s) &=                & \kappa(s)N(s) & \\
    N'(s) &= -\kappa(s)T(s) &               & +\tau(s)B(s)\\
    B'(s) &=                & -\tau(s)N(s), &
  \end{alignat}
\end{subequations}
or, using matrix multiplication,
\begin{equation}
\begin{bmatrix}T'(s)\\ N'(s)\\ B'(s)\end{bmatrix}
=
\begin{bmatrix}
  0 & \kappa(s) & 0\\
  -\kappa(s) & 0 & \tau(s)\\
  0 & -\tau(s) & 0
\end{bmatrix}
\begin{bmatrix}T(s)\\ N(s)\\ B(s)\end{bmatrix}.
\end{equation}
\end{theorem}

\begin{proof}
We know the first and third formulas already, so we just need to prove
the second formula. Since the Frenet field forms an orthonormal basis,
we know
\begin{equation}
N'(s) = a(s)T(s) + b(s)N(s) + c(s)B(s).
\end{equation}
We just need to determine the coefficients. We take the inner product of
both sides with $T$ and $B$.

We know that
\begin{subequations}
\begin{equation}
N'(s)\cdot T(s) = a(s).
\end{equation}
We also know $N(s)$ and $T(s)$ are orthonormal vector, in particular,
\begin{equation}
N(s)\cdot T(s) = 0.
\end{equation}
But taking the derivative of both sides, we find
\begin{equation}
\frac{\D}{\D s}(N(s)\cdot T(s)) = N'(s)\cdot T(s) + N(s)\cdot T'(s).
\end{equation}
We can use the Frenet formula $T'(s) = \kappa(s)N(s)$ to rewrite the
right-hand side
\begin{align}
N'(s)\cdot T(s) + N(s)\cdot T'(s) &= N'(s)\cdot T(s) + N(s)\cdot(\kappa(s)N(s)\\
&= N'(s)\cdot T(s) + \kappa(s),
\end{align}
and further, since $N'(s)\cdot T(s)=a(s)$, we conclude
\begin{align}
N'(s)\cdot T(s) + N(s)\cdot T'(s) &= N'(s)\cdot T(s) + \kappa(s)\nonumber\\
&= a(s) + \kappa(s).
\end{align}
Returning to our original statement, we find
\begin{equation}
\frac{\D}{\D s}(N(s)\cdot T(s)) = a(s) + \kappa(s) = 0.
\end{equation}
Hence, in particular
\begin{equation}
\boxed{a(s) = -\kappa(s).}
\end{equation}
\end{subequations}

We know $b(s)=0$ since
\begin{subequations}
\begin{equation}
N(s)\cdot N(s)=1\implies\frac{\D}{\D s}(N(s)\cdot N(s)) = 0.
\end{equation}
This gives us $2b(s)=0$, which implies
\begin{equation}
\boxed{b(s)=0.}
\end{equation}
\end{subequations}

Now for the last coefficient. Very similar to the first, since
$N(s)\cdot B(s)=0$ (thanks to their being orthonormal vectors) we find
their derivative with respect to $s$ is zero. (We will use the fact that
$N(s)\cdot B'(s)=0$.)
But we find
\begin{subequations}
\begin{align}
\frac{\D}{\D s}(N(s)\cdot B(s)) &= N'(s)\cdot B(s) + N(s)\cdot B'(s)\\
&= N'(s)\cdot B(s)\\
&=\tau(s)
\end{align}
hence
\begin{equation}
\boxed{c(s) = \tau(s).}
\end{equation}
\end{subequations}
This proves the remaining Frenet formula.
\end{proof}

\begin{proposition}
A unit speed curve is a straight line if and only if its curvature
vanishes $\kappa(s)=0$.
\end{proposition}

Our proof will consist of two direct proofs, one in the forward
direction (straight line implies zero curvature), the other in the
backwards direction (zero curvature implies straight line).

\begin{proof}
  Let $\beta(s)$ be a unit speed curve.

  $(\implies)$ Assume $\beta$ is a straight line, then
  \begin{equation}
\beta(s) = \vec{p} + s\vec{v}
  \end{equation}
  for some $\vec{p}$ and (unit) vector $\vec{v}$. So
  \begin{subequations}
    \begin{align}
      \kappa(s) &= \|T'(s)\|\\
      &= \|\beta''(s)\|\\
      &= \|0\| = 0.
    \end{align}
  \end{subequations}
  Hence straight lines have zero curvature.

  $(\impliedby)$ Conversely, assume for all $s$ we have $\kappa(s)=0$.
  Write out the components of the curve as
\begin{subequations}
  \begin{equation}
\beta(s) = (\beta_{1}(s), \beta_{2}(s), \beta_{3}(s)).
  \end{equation}
  We find
  \begin{equation}
\beta''(s) = (\beta_{1}''(s), \beta_{2}''(s), \beta_{3}''(s)).
  \end{equation}
  For the curvature to be zero everywhere, we need the second derivative
  of each component to vanish, i.e., for each $s$ we have
  \begin{equation}
\beta_{j}''(s) = 0
  \end{equation}
  for $j=1,2,3$. We integrate this equation twice to find
  \begin{equation}
\beta_{j}(s) = p_{j} + sv_{j}
  \end{equation}
  for some constants of integration $v_{j}$, $p_{j}$. This means
  \begin{equation}
\beta(s) = \vec{p} + s\vec{v},
  \end{equation}
\end{subequations}
  i.e., that $\beta$ is a straight line.
\end{proof}

\begin{proposition}
A unit speed curve $\beta\colon I\to\RR^{3}$ with positive curvature
$\kappa(s)>0$ has
vanishing torsion if and only if $\beta$ is a \define{Plane Curve}
(i.e., its image lies in some plane in $\RR^{3}$).
\end{proposition}

\begin{proof}
$(\implies)$ Assume $\beta$ is a plane curve, i.e., it lies in the plane
through $\vec{p}\in\RR^{3}$ with unit normal vector $\vec{n}$.
For all $s$, we have
\begin{subequations}
\begin{equation}
\vec{n}\cdot(\beta(s)-\vec{p})=0.
\end{equation}
By differentiating with respect to $s$ (and remembering $\vec{p}$ and
$\vec{n}$ are constants), we find
\begin{equation}
\vec{n}\cdot\beta'(s) = 0 \implies \vec{n}\cdot T(s)=0.
\end{equation}
Differentiating once more, we find
\begin{equation}
\vec{n}\cdot\beta''(s) = 0\implies \vec{n}\cdot N(s)=0.
\end{equation}
But we assumed $\vec{n}$ is a unit vector (in particular, it's nonzero)
and the Frenet frame is a collection of orthonormal vectors. This forces
us to conclude,
\begin{equation}
\vec{n} = \pm B(s).
\end{equation}
In particular, $B(s)$ is constant. So
\begin{equation}
B'(s) = 0,
\end{equation}
and by the Frenet formulas
\begin{equation}
B'(s) = -\tau(s)N(s) = 0.
\end{equation}
Since $N(s)\neq0$ we conclude,
\begin{equation}
\boxed{\tau(s) = 0.}
\end{equation}
\end{subequations}
This is the first half of the proof.

$(\impliedby)$ Suppose $\tau(s)=0$. Then by the Frenet formulas, $B'(s)=0$.
In particular, $B(s)$ is a constant unit vector. Pick some $s_{0}\in I$,
and define
\begin{equation}
f(s) = (\beta(s)-\beta(s_{0}))\cdot B.
\end{equation}
We want to prove $f(s)=0$ for all $s\in I$.

The first step is to consider its derivative, and since $B$ is constant
(with respect to $s$), we find:
\begin{equation}
f'(s) = (\beta'(s) - 0)\cdot B = T(s)\cdot B.
\end{equation}
But we know $T(s)\cdot B=0$ since they are orthonormal vectors. Hence we
conclude
\begin{equation}
f'(s) = 0.
\end{equation}
This just tells us that $f(s)$ is a constant.

But we also know that
\begin{equation}
f(s_{0}) = 0.
\end{equation}
Hence we conclude, for all $s\in I$, that:
\begin{equation}
f(s) = 0.
\end{equation}
This is precisely the description of a plane,
$\{\vec{x}\in\RR^{3}\mid (\vec{x}-\vec{p})\cdot B=0\}$.
\end{proof}


\begin{example}[Frenet field for circular helix]
  Let
  \begin{equation}
\alpha(t) = (a\cos(t),a\sin(t),bt)
  \end{equation}
  where $a>0$, $b>0$ are constants. Find the Frenet field for this
  curve.

  First, we need to verify the curve is a unit-speed curve (and, if not,
  find its unit-speed reparamtrization). We find
\begin{subequations}
\begin{align}
s(t) &= \int^{t}_{0} \|\alpha'(u)\|\,\D u\\
&= \int^{t}_{0} \sqrt{a^{2} + b^{2}}\,\D u\\
&= t\sqrt{a^{2} + b^{2}}.
\end{align}
Then we can invert this to find $t$ as a function of $s$:
\begin{equation}
t(s) = \frac{s}{\sqrt{a^{2}+b^{2}}}.
\end{equation}
For simplicity, we will define
\begin{equation}
c = \sqrt{a^{2}+b^{2}},
\end{equation}
so
\begin{equation}
t(s) = s/c.
\end{equation}
\end{subequations}
Then we find the unit-speed parametrization,
\begin{equation}
\boxed{\beta(s) = \alpha(t(s)) = (a \cos(s/c), a\sin(s/c), bs/c).}
\end{equation}
OK, the zeroeth step is complete.

Now we may use the Frenet formulas to get the curvature and torsion. The
unit tangent vector is
\begin{equation}
T(s) := \beta'(s) = \frac{1}{c} (-a\sin(s/c), a\cos(s/c),b).
\end{equation}
Now we find its derivative
\begin{equation}
T'(s) = \frac{-a}{c^{2}}(\cos(s/c),\sin(s/c),0),
\end{equation}
and the curvature is,
\begin{equation}
\kappa(s) := \|T(s)\| = \frac{a}{c^{2}}.
\end{equation}
That is to say, the curvature is constant.

The next Frenet formulas give us
\begin{equation}
N(s) := \frac{T'(s)}{\kappa(s)} = (-\cos(s/c),-\sin(s,c),0).
\end{equation}
The binormal is obtained from the cross-product,
\begin{equation}
B(s) := N(s)\times T(s) = \frac{1}{c}(b\sin(s/c),-b\cos(s/c),a).
\end{equation}
The derivative of $B(s)$ with respect to $s$ gives us
\begin{equation}
B'(s) = \frac{b}{c^{2}}(\cos(s/c),\sin(s/c),0),
\end{equation}
but we know using the Frenet formula $B'(s) = -\tau(s)N(s)$, and
inspection of terms forces us to conclude,
\begin{equation}
\tau(s) = \frac{-b}{c^{2}}.
\end{equation}
The torsion is also constant!

In fact, any curve with constant [nonzero] curvature and nonzero torsion
is either a helix ($\tau\neq0$) or a degenerate helix (a.k.a., a circle).
\end{example}

\subsection{Frenet Approximation at a Point}

\M
Consider the unit-speed curve $\beta$. Expand this in a Taylor series,
e.g., about zero. We need to know several values:
\begin{subequations}
  \begin{align}
    \beta'(0) &:= T(0) = T_{0}\\
    \beta''(0) &:=\kappa(0)N(0) = \kappa_{0}N_{0}\\
    \beta'''(0) &= (\kappa N)'(0) = \kappa'(0)N_{0} + \kappa_{0}(-\kappa_{0}T_{0}+\tau_{0}B_{0}).
  \end{align}
\end{subequations}
Then the Taylor series expansion yields
\begin{equation}
  \beta(s)\approx\beta_{0} + s\beta'(0) + \frac{s^{2}}{2!}\beta''(0) + \frac{s^{3}}{3!}\beta'''(0).
  %T_{0} + \frac{s^{2}}{2}\kappa_{0}N_{0} + \frac{s^{3}}{3!}\kappa_{0}\tau_{0}B_{0}.
\end{equation}
We find
\begin{subequations}
\begin{align}
\beta(s)&\approx\beta_{0} + s\beta'(0) + \frac{s^{2}}{2!}\beta''(0) + \frac{s^{3}}{3!}\beta'''(0)\\
&=\beta_{0} + sT_{0} + \frac{s^{2}}{2}(\kappa(0)N(0) = \kappa_{0}N_{0}) + \frac{s^{3}}{3!}(\kappa'(0)N_{0} + \kappa_{0}(-\kappa_{0}T_{0}+\tau_{0}B_{0}))\\
&=\beta_{0} +\left(s - \frac{\kappa_{0}^{2}s^{3}}{3!}\right)T_{0}
+\left(\frac{\kappa_{0}}{2}s^{2} + \kappa'(0)\frac{s^{3}}{3!}\right)N_{0}
+\left(\tau_{0}\kappa_{0}\frac{s^{3}}{3!}\right)B_{0}.
\end{align}
\end{subequations}

\M
We can truncate this series to be scalar multiples of the Frenet vectors
at $\beta(0)$, with the following interpretation:
\begin{equation}
  \begin{split}
  \beta(s)&\approx
  \overbrace{\beta_{0} +T_{0} s
+{\color{DarkRed}N_{0}s^{2}\frac{\kappa_{0}}{2}}}^{\begin{pmatrix}\mbox{Tangent Parabola}\\
\mbox{in Osculating Plane}
\end{pmatrix}}
+{\color{DarkGreen}B_{0}\tau_{0}\kappa_{0}\frac{s^{3}}{3!}}\\
&\approx\begin{pmatrix}
  \mbox{Linear}\\
  \mbox{Approximation}
\end{pmatrix}
+{\color{DarkRed}\begin{pmatrix}
    \mbox{How Fast the Curve}\\
    \mbox{Deviates From}\\
    \mbox{Tangent Line}
\end{pmatrix}}
+{\color{DarkGreen}\begin{pmatrix}\mbox{How fast the curve}\\
  \mbox{moves out of}\\
  \mbox{the osculating plane}
\end{pmatrix}}
  \end{split}
\end{equation}

\subsection{Frenet Data for Arbitrary Curves}

\M
We defined the Frenet field for unit-speed curves. But we want the
Frenet data $T$, $N$, $B$, $\tau$, $\kappa$ to be independent of our
choice of parametrization. The trick: given some regular curve
$\alpha\colon I\to\RR^{3}$, let $\bar{\alpha}\colon \bar{I}\to\RR^{3}$
be its unit-speed parametrization. 
\begin{center}
  \includegraphics{img/img.6}
\end{center}
We \emph{define} the Frenet data of $\alpha$ to be the Frenet data of
$\bar{\alpha}$. The Frenet data of $\alpha$ are unbarred quantities, the
Frenet data of $\bar{\alpha}$ are barred quantities. We define them by:
\begin{subequations}
  \begin{align}
    T(t) &= \bar{T}(s) = \bar{T}(s(t))\\
    N(t) &= \bar{N}(s)\\
    B(t) &= \bar{B}(s)\\
    \kappa(t) &= \bar{\kappa}(s)\\
    \tau(t) &= \bar{\tau}(s)
  \end{align}
\end{subequations}
In principle we can find $\bar{\alpha}$ and compute its Frenet data, but
in practice this is usually impossible (analytically) because we must do
two things:
\begin{enumerate}
\item calculate $\displaystyle s(t)=\int^{t}_{t_{0}}\|\alpha'(u)\|\,\D u$,
\item invert this to get $t = t(s)$.
\end{enumerate}
Both are hard, sometimes impossible. We need a better way to calculate
these things directly.

\N{Unit Tangent}
We find by direct calculation
\begin{subequations}
  \begin{align}
    T(t) &= \bar{T}(s) = \bar{\alpha}'(s)\\
    &= \frac{\D}{\D s}\alpha(t(s))\\
    &= \alpha'(t)\frac{\D t}{\D s}.
  \end{align}
\end{subequations}
But
\begin{equation}
\frac{\D s}{\D t}=\|\alpha'(t)\|,
\end{equation}
so
\begin{equation}
T(t) = \frac{\alpha'(t)}{\|\alpha'(t)\|}.
\end{equation}

\N{Curvature}
We compute directly,
\begin{subequations}
  \begin{align}
    \kappa(t) &=\bar{\kappa}(s) :=\left\|\frac{\D}{\D s}\bar{T}(s)\right\|\\
    &=\left\|\frac{\D}{\D s}T(t)\right\|\\
    &=\left\|T'(t)\frac{\D t}{\D s}\right\|\\
    &=\frac{\|T'(t)\|}{\|\alpha'(t)\|}
  \end{align}
\end{subequations}

\N{Principal Normal}
We find, letting $v(t)=\|\alpha'(t)\|$,
\begin{equation}
N(t) = \bar{N}(s) = \frac{\bar{T}'(s)}{\bar{\kappa}(s)} = \frac{T'(t)/v(t)}{\kappa(t)}
=\frac{T'(t)}{\|T'(t)\|}.
\end{equation}
This matches intuition of $N(t)\propto T'(t)$.

\N{Binormal}
Again, direct computation,
\begin{equation}
B(t) = \bar{B}(s) = \bar{T}(s)\times\bar{N}(s) = T(t)\times N(t).
\end{equation}
This matches intuition of the binormal as cross-product of $T$ and $N$.

\N{Torsion}
We know
\begin{equation}
\bar{B}'(s) = -\bar{\tau}(s)\bar{N}(s)
\end{equation}
and so, letting $v(t)=\|\alpha'(t)\|$,
\begin{equation}
\frac{\D}{\D s}B(t) = -\tau(t)v(t)N(t).
\end{equation}
To summarize our results, we have this handy theorem:

\begin{theorem}
If $\alpha\colon I\to\RR^{33}$ is a regular curve with positive
curvature, then up to some factor $v(t)=\|\alpha'(t)\|$ we have,
\begin{subequations}
  \begin{alignat}{2}
T'(t) &=                    & \kappa(t)v(t)N(t) & \\
N'(t) &= -\kappa(t)v(t)T(t) &                   & + \tau(t)v(t)B(t)\\
B'(t) &=                    & -\tau(t)v(t)N(t)  &
  \end{alignat}
\end{subequations}
Or, using matrices,
\begin{equation}
\begin{bmatrix}T'(t)\\ N'(t)\\ B'(t)
\end{bmatrix}
=\begin{bmatrix}0 & \kappa(t)v(t) & 0\\
-\kappa(t)v(t) & 0 & \tau(t)v(t)\\
0 & -\tau(t)v(t) & 0
\end{bmatrix}
\begin{bmatrix}T(t)\\ N(t)\\ B(t)
\end{bmatrix}.
\end{equation}
\end{theorem}

\M
Any vector field $Y$ on a regular curve $\alpha$ could be written as a
linear combination of Frenet vector fields:
\begin{equation}
Y(t) = f(t)T(t) + g(t)N(t) + h(t)B(t).
\end{equation}
We find its derivative:
\begin{subequations}
\begin{align}
  Y'(t) &= f'T + fT' + g'N + gN' + h'B + hB'\\
  &= f'T + f\kappa vN + g'N + g(t)(-\kappa vT + \tau v B) + h'B +
  h(t)(-\tau vN)\\
  &= (f' - \kappa v g)T + (f\kappa v + g' - \tau v h)N + (g\tau v + h')B.
\end{align}
\end{subequations}
If $Y$ has Frenet components $(f,g,h)$, then the Frenet components of
$Y'(t)$ are
\begin{equation}
\begin{bmatrix}f'(t)\\g'(t)\\h'(t)
\end{bmatrix}
+
\begin{bmatrix}0 & -\kappa(t)v(t) & 0\\
\kappa(t)v(t) & 0 & -\tau(t)v(t)\\
0 & \tau(t)v(t) & 0
\end{bmatrix}
\begin{bmatrix}f(t)\\ g(t)\\ h(t)
\end{bmatrix} = Y'(t).
\end{equation}
Note: we use the \emph{transpose} of the matrix from the Frenet
formulas.

\begin{remark}
We should observe, when the frame we're working with is not the natural
frame field, then we cannot write the derivative of a vector as just the
derivatives of the components. We just saw this won't work with the
Frenet frame. There was a correction term. This idea underlies the idea
of covariant derivatives.
\end{remark}

\subsection{Covariant Differentiation}

\M
We want to differentiate vector fields. Suppose we are given some vector
field $W\in\Vect(\RR^{n})$ and a tangent vector
$\vec{v}_{\vec{p}}\in\T_{\vec{p}}\RR^{n}$. There is one obvious way to
differentiate $W$ in the direction of $\vec{v}_{\vec{p}}$: consider
expressing $W$ using coordinates relative to the natural frame field,
\begin{equation}
W = \sum_{j}w^{j}U_{j},
\end{equation}
then we just consider
\begin{equation}
\vec{v}_{\vec{p}}[W] = \sum_{j}\vec{v}_{\vec{p}}[w^{j}]U_{j}(\vec{p}).
\end{equation}
Why not?

If we did this using a vector field $V\in\Vect(\RR^{n})$ at each point
$\vec{p}\in\RR^{n}$, with $\vec{v}_{\vec{p}}=V(\vec{p})$, then we get
\begin{equation}
\begin{split}
  \sum_{j}V(\vec{p})[w^{j}]U_{j}(\vec{p})&=\left(\sum_{j}V[w^{j}]U_{j}\right)(\vec{p})\\
&=\nabla_{V}W\in\Vect(\RR^{n}).
\end{split}
\end{equation}
So we get a vector field which is the natural covariant derivative of
$W$ with respect to the vector field $V$. 

\begin{definition}
Let $V,W\in\Vect(\RR^{n})$. We define the \define{Natural Covariant Derivative}
of $W$ with respect to $V$ is the vector field $\nabla_{V}W$ defined by
coordinates relative to the natural frame field,
\begin{equation}
\nabla_{V}W = \sum_{j}V[w^{j}]U_{j},
\end{equation}
where $W=\sum_{j}w^{j}U_{j}$ are the coordinates of $W$ relative to the
natural frame field $U_{j}$.
\end{definition}

\begin{remark}
This is really dependent on the natural frame field, but we would like a
notion of covariant differentiation \emph{independent} of the choice of
frame field.
\end{remark}

\begin{example}
Let $V=\sum_{i}v^{i}U_{i}$ and $W=\sum_{j}w^{j}U_{j}$ be the vector
fields expressed in coordinates relative to the natural frame field
$U_{j}$. Then
\begin{subequations}
  \begin{align}
    \nabla_{V}W &= \sum_{j}V[w^{j}]U_{j}\\
    &= \sum_{j}\left(\sum_{i}v^{i}U_{i}[w^{j}]\right)U_{j}\\
    &= \sum_{j}\left(\sum_{i}v^{i}\frac{\partial w^{j}}{\partial x_{i}}\right)U_{j}\\
 &= \sum_{j}\underbrace{\sum_{i}v^{i}\frac{\partial w^{j}}{\partial
        x_{i}}}_{\text{coordinates of }\nabla_{V}W}U_{j}
  \end{align}
\end{subequations}
  \end{example}

\begin{theorem}[Essential properties of the covariant derivative]
\begin{enumerate}
\item $\nabla_{V}(aY + bZ) = a\nabla_{V}Y + b\nabla_{V}Z$
\item $\nabla_{fV + gW}Z = f\nabla_{V}Z + g\nabla_{W}Z$
\item $\nabla_{V}(fZ) = V[f]Z + f\nabla_{V}Z$
\item Metric compatibility: $\nabla_{V}(Y\cdot Z) = V[Y\cdot Z] = (\nabla_{V}Y)\cdot Z + Y\cdot(\nabla_{V}Z)$.
\end{enumerate}
\end{theorem}

\begin{remark}
More generally, any operation $\bar\nabla$ taking two vector fields
$V\times W\to Z$ and produces a third, which satisfies the first three
properties is called a derivative operation. The fourth property is most
geometric, as it deals with angles.
\end{remark}

% HW 4
% 2.4 # 1,3
% 2.5 # 2(a,b,f), 3, 5a

\N{Connection Forms}
Suppose $E_{i}$ are some orthonormal frame field on $\RR^{n}$, and we
express $W\in\Vect(\RR^{n})$ in coordinates relative to $E_{i}$:
\begin{equation}
W = \sum_{i}w^{i}E_{i}.
\end{equation}
Let $V\in\Vect(\RR^{n})$. Then we want to find the coordinates of
$\nabla_{V}W$ relative to $E_{i}$. We know, using linearity and the
Leibniz property,
\begin{equation}
\nabla_{V}W = \sum_{i}\nabla_{V}(w^{i}E_{i}) = \sum_{i}V[w^{i}]E_{i} + w^{i}\nabla_{V}E_{i}.
\end{equation}
Now we just need to express $\nabla_{V}E_{i}$ in coordinates relative to
the frame field $E_{i}$. We expect
\begin{equation}
\nabla_{V}E_{i} = \sum_{j}c_{ij}E_{j},
\end{equation}
and hope the coefficients $c_{ij}$ somehow depend on $V$. The usual
notation is for these coefficients to be written $\omega_{ij}$, and we
would have
\begin{equation}
\nabla_{V}E_{i} = \sum_{j}\omega_{ij}[V]E_{j}.
\end{equation}
We can get the components by applying $\langle-,E_{k}\rangle$ to both
sides
\begin{subequations}
\begin{align}
\langle\nabla_{V}E_{i},E_{k}\rangle
&=\langle\sum_{j}\omega_{ij}[V]E_{j},E_{k}\rangle\\
&=\sum_{j}\omega_{ij}[V]\langle E_{j},E_{k}\rangle\\
&=\sum_{j}\omega_{ij}[V]\delta_{j,k} = \omega_{ik}[V].
\end{align}
\end{subequations}

\M
We can use metric compatibility to find
\begin{subequations}
\begin{align}
  V[\langle E_{i},E_{j}\rangle] &= \langle\nabla_{V}E_{i},E_{j}\rangle
  + \langle E_{i},\nabla_{V}E_{j}\rangle\\
  &=\omega_{ij}[V] + \omega_{ji}[V]\\
  &= V[\delta_{ij}] = 0.
\end{align}
\end{subequations}
Hence in particular, we find the coefficients are antisymmetric,
\begin{equation}
\boxed{\omega_{ij}[V] =-\omega_{ji}[V].}
\end{equation}
The diagonal components would satisfy $\omega_{ii}[V]=-\omega_{ii}[V]$,
which could only happen if $\omega_{ii}[V]=0$.

When we look back on our system of equations, we find there are only
$n(n-1)/2$ independent components (thanks to antisymmetry).

\M
For the case when $n=3$, and we work in $\RR^{3}$, we only needto know
$\omega_{12}[V]$, $\omega_{13}[V]$, and $\omega_{23}[V]$ for all
$V\in\Vect(\RR^{3})$. In fact $\omega_{ij}[V]$ depends linearly on
$V$. Consider for arbitrary $f,g\in C^{\infty}(\RR^{3})$ and
$V,W\in\Vect(\RR^{3})$,
\begin{subequations}
  \begin{align}
    \omega_{ij}[fV + gW] &= \langle\nabla_{fV + gW}E_{i},E_{j}\rangle\\
    &= \langle f\nabla_{V}E_{i} + g\nabla_{W}E_{i},E_{j}\rangle\\
    &= f\omega_{ij}[V] + g\omega_{ij}[W].
  \end{align}
\end{subequations}
These coefficients $\omega_{ij}$ take a vector field and produce
functions. They are one-forms called \define{Connection Forms}.

\subsection{Worked Example}

\M This is a long example. Consider working in spherical coordinates in
$\RR^{3}$. We want to find the spherical frame field on $\RR^{3}$ (or
$\RR^{3}$ minus the $z$ axis). We say the order of the basis will be
$(\rho,\theta,\lambda)$, as doodled below.
\begin{center}
  \includegraphics{img/img.7}
\end{center}
We use the standard Riemannian metric, and the frame field vectors have
the following interpretations:
\begin{itemize}
\item $E_{1}$ is ``up''
\item $E_{2}$ points ``East''
\item $E_{3}$ points ``North''.
\end{itemize}
Recall, the coordinates are given by
\begin{subequations}
  \begin{align}
x &=\rho\cos(\lambda)\cos(\theta)\\
y &=\rho\cos(\lambda)\sin(\theta)\\
z &= \rho\sin(\lambda)
  \end{align}
\end{subequations}

\M
How do we find $E_{1}$? Well, we find a curve passing through $(x,y,z)$
that specifically changes $\rho\to\rho+t$, then its initial velocity
unit vector gives us the coefficients for $E_{1}$ relative to the
natural frame field. So
\begin{equation}
\alpha(t) = ((\rho+t)\cos(\lambda)\cos(\theta), (\rho+t)\cos(\lambda)\sin(\theta),(\rho+t)\sin(\lambda).
\end{equation}
Then we find
\begin{equation}
\alpha'(t) = (\cos(\lambda)\cos(\theta),\cos(\lambda)\sin(\theta),\sin(\lambda)).
\end{equation}
Hence
\begin{equation}
  \begin{split}
    E_{1} &= \alpha'(0)\cdot(U_{1},U_{2},U_{3})\\
    &= \cos(\lambda)\cos(\theta)U_{1} + \cos(\lambda)\sin(\theta)_{2} + \sin(\lambda)U_{3}.
  \end{split}
\end{equation}

\M
For $E_{2}$, a similar calculation with the curve
\begin{equation}
\alpha(t) = (\rho\cos(\lambda)\cos(\theta+t), \rho\cos(\lambda)\sin(\theta+t),\rho\sin(\lambda)).
\end{equation}
We find its velocity,
\begin{equation}
\alpha'(t) = (-\rho\cos(\lambda)\sin(\theta+t),\rho\cos(\lambda)\cos(\theta+t),0).
\end{equation}
We find its unit vector, since its length is
\begin{equation}
\|\alpha'(t)\| =
\rho\cos(\lambda)\implies\frac{\alpha'(t)}{\|\alpha'(t)\|} = (-\sin(\theta),\cos(\theta),0).
\end{equation}
Its unit vector gives us the coordinates for $E_{2}$ relative to the
natural frame field,
\begin{equation}
E_{2} = -\sin(\theta)U_{1} + \cos(\theta)U_{2}
\end{equation}

\M
The last frame field we need the curve
\begin{equation}
\alpha(t) = (\rho\cos(\lambda+t)\cos(\theta), \rho\cos(\lambda+t)\sin(\theta),\rho\sin(\lambda+t)).
\end{equation}
The velocity vector for this curve,
\begin{equation}
\alpha'(t) = (-\rho\cos(\theta)\sin(\lambda+t), -\rho\sin(\theta)\sin(\lambda+t),
\rho\cos(\lambda+t)).
\end{equation}
Its unit vector
\begin{equation}
\frac{\alpha'(t)}{\|\alpha'(t)\|} = (-\cos(\theta)\sin(\lambda+t), -\sin(\theta)\sin(\lambda+t),
\cos(\lambda+t)).
\end{equation}
Hence we find
\begin{equation}
E_{3} = -\sin(\lambda)\cos(\theta)U_{1} - \sin(\lambda)\sin(\theta)U_{3}+\cos(\lambda)U_{3}.
\end{equation}

\N{Connection Coefficients}
Now that we have obtained our spherical frame field, we can compute the
connection coefficients $\omega_{ij}[V] = \nabla_{V}E_{i}\cdot E_{j}$.