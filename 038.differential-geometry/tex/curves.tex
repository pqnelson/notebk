\section{Differential Geometry of Curves in $\RR^{3}$ (or $\RR^{n}$)}

\M
The idea is that we have introduced the basic gadgetry of differential
geometry, but in the setting of $\RR^{n}$. Now we will consider curves
in $\RR^{n}$, and use the gadgetry we've introduced to study properties
of curves (for example, how to vector fields on a curve, and what do
they tell us). This appears in classical mechanics (especially
Lagrangian and Hamiltonian mechanics).

\begin{definition}
An \define{Unparametrized Curve} in $\RR^{3}$ is a ``one-dimensional
subset'' of points.
\end{definition}

\begin{example}
The doodle below is a closed unparametrized curve --- ``closed'' meaning
it forms a ``loop'' (eventually):
\begin{center}
  \includegraphics{img/img.0}
\end{center}
\end{example}

\begin{example}
Here is a happy open unparametrized curve --- ``open'' meaning it is
``not closed'':
\begin{center}
  \includegraphics{img/img.1}
\end{center}
\end{example}

\begin{definition}
Let $I=(a,b)$ be an open interval (it is possible $a=-\infty$, or
$b=+\infty$, or both). We define a \define{(Parametrized) Curve} in
$\RR^{3}$ to be a smooth function $\alpha\colon I\to\RR^{3}$
\end{definition}

\begin{remark}
Henceforth, we will reserve the term ``curve'' for an unparametrized
curve, and ``path'' for a parametrized curve.
\begin{itemize}
\item ``Path'' = ``Parametrized Curve''
\item ``Curve'' = ``Unparametrized Curve''.
\end{itemize}
In fact, almost always we care about curves, so unless otherwise stated,
all curves are parametrized.
\end{remark}

\begin{example}
Consider the \define{Elliptic Helix}, a parametrized curve $\alpha(t)=(a\cos(t),b\sin(t),ct)$ where
$a,b,c\in\RR$ are positive constants, and $t\in[0,+\infty)$ looks like:
\begin{center}
  \includegraphics{img/img.3}
\end{center}
We could consider a different parametrization of the same curve, like
$\beta(t)=(a\cos(3t),b\sin(3t),3ct)=\alpha(3t)$. How do we know this is
the same curve? Well, one way is to establish a bijection of points
$\beta(t/3)=\alpha(t)$ for all $t\in[0,\infty)$.
\end{example}

\begin{definition}
Let $\alpha\colon I\to\RR^{3}$ be a path with components
$(\alpha_{1},\alpha_{2},\alpha_{3})$. The \define{Velocity} of $\alpha$
at time $t\in I$ is the tangent vector
\begin{subequations}
  \begin{equation}
\alpha'(t) = \left(\frac{\D\alpha(t)}{\D t}\right)_{\alpha(t)}\in\T_{\alpha(t)}\RR^{3}
\end{equation}
at base point $\alpha(t)$, whose components are
\begin{equation}
\alpha'(t)=\left(\frac{\D\alpha_{1}(t)}{\D t},
\frac{\D\alpha_{2}(t)}{\D t},
\frac{\D\alpha_{3}(t)}{\D t}\right).
\end{equation}
\end{subequations}
\end{definition}

\begin{remark}
The velocity defines a sort of vector field, but defined only on the
curve and not all of $\RR^{3}$.
\end{remark}

\begin{example}
  For the elliptical helix,
  $\alpha(t)=(a\cos(t),b\sin(t),ct)$ for $t\in\RR$,
  we have its velocity be
  \begin{equation}
\alpha'(t) = (-a\sin(t),b\cos(t),c)_{\alpha(t)}.
  \end{equation}
\end{example}


\begin{example}
  Let $\vec{p},\vec{v}\in\RR^{3}$ be constants.
  Consider the curve $\beta(t)=\vec{p}+t\vec{v}$. This is the straight
  line with initial position $\beta(0)=\vec{p}$ and initial velocity $\beta'(0)=\vec{v}_{\vec{p}}$.
  We've used this before when we've worked with the directional
  derivative of $f\colon\RR^{3}\to\RR$,
  \begin{equation}
\vec{v}_{\vec{p}}[f] = \left.\frac{\D}{\D t}f(\vec{p}+t\vec{v})\right|_{t=0}.
  \end{equation}
  In fact, we could use \emph{any} curve $\alpha$ with
  $\alpha(0)=\vec{p}$ and $\alpha'(0) = \vec{v}_{\vec{p}}$ to define $\vec{v}_{\vec{p}}[f]$.
\end{example}

\begin{theorem}
Let $\vec{v}_{\vec{p}}\in\T_{\vec{p}}\RR^{3}$, $I$ be an interval
containing zero, and let $\alpha\colon I\to\RR^{3}$ be such that
$\alpha(0)=\vec{p}$ and $\alpha'(0) = \vec{v}_{\vec{p}}$. Then
\begin{equation}
\vec{v}_{\vec{p}}[f] = \left.\frac{\D}{\D t}f(\alpha(t))\right|_{t=0}.
\end{equation}
\end{theorem}

\begin{proof}
We know $\alpha'(0) = \vec{v}_{\vec{p}}$. So, let us calculation
\begin{calculation}
  \vec{v}_{\vec{p}}[f]
\step{definition of directional derivative}
  \left.\frac{\D}{\D t}f(\vec{p}+t\vec{v})\right|_{t=0}
\step{since $\alpha(t)=\vec{p}+t\vec{v}$}
  \left.\frac{\D}{\D t}f(\alpha(t))\right|_{t=0}
\step{chain rule}
  \sum_{j}\left.\frac{\partial f}{\partial x_{j}}(\alpha(t))\frac{\D\alpha^{j}(t)}{\D t}\right|_{t=0}
\step{since $\D\alpha^{j}(0)/\D t=v^{j}$, $\alpha(0)=\vec{p}$}
  \sum_{j}\frac{\partial f}{\partial x_{j}}(\vec{p})v^{j}
\end{calculation}
On the other hand, repeating the last three steps with
$$\left.\frac{\D}{\D t}f(\alpha(t))\right|_{t=0}$$
gives the same result since the only facts used were $\alpha(0)=\vec{p}$, $\alpha'(0)=\vec{v}_{\vec{p}}$.
\end{proof}

\begin{corollary}
For any curve $\alpha$ and smooth function $f\colon\RR^{3}\to\RR$, we
have
\begin{equation}
\alpha'(t)[f] = \left.\frac{\D}{\D s}f(\alpha(s))\right|_{s=t}.
\end{equation}
That is to say, the directional derivative of $f$ with respect to the
velocity vector field is the rate of change of $f$ as we move along the
curve $\alpha$.
\end{corollary}

\M
In linear algebra, the key geometric tool is the concept of the inner
product (``dot product''). Any vector space with an inner product
automatically gets notations of:
\begin{itemize}
\item \textbf{magnitude} of vectors $\|\vec{v}\| = \sqrt{\langle\vec{v},\vec{v}\rangle}$,
and
\item \textbf{angle} between vectors $\cos(\theta) = \langle\vec{v},\vec{w}\rangle/(\|\vec{v}\|\cdot\|\vec{w}\|)$.
\end{itemize}
In differential geometry, we have not just \emph{one} vector space, but
a vector space \emph{at each point} (e.g., for each $\vec{p}\in\RR^{3}$,
we have $\T_{\vec{p}}\RR^{3}$). We can, in principle, put a different
inner product at each of these tangent spaces. In other words, at each
point we may have a different notion of magnitude and angle.

\begin{definition}
An assignment of an inner product to each $\T_{\vec{p}}\RR^{3}$ (varying
smoothly with $\vec{p}\in\RR^{3}$) is called a \define{Riemannian metric}.
\end{definition}

\begin{example}
For now, we will be sticking with the usual Riemannian metric on
$\RR^{3}$ given by
\begin{equation}
\underbrace{\langle\vec{v}_{\vec{p}},\vec{w}_{\vec{p}}\rangle_{\vec{p}}}_{\text{inner product on }\T_{\vec{p}}\RR^{3}}=
\overbrace{\vec{v}\cdot\vec{w}}^{\text{usual dot product}}
\end{equation}
\end{example}