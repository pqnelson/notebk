\section{Surfaces}

\M What is a surface? Intuitively, it's an ``$\RR^{2}$'' subset of
$\RR^{3}$. What does that even mean?

A set should be two-dimensional if it can be built out of pieces that
look like open sets of $\RR^{2}$, i.e., with two-dimensional patches of
``fabric'' we may ``sew'' together. Conceptually captured by this picture:
\begin{center}
  \includegraphics{img/surfaces.0}
\end{center}
The map like the one above gives ``coordinates'' to each point on a
surface. So a map like this is called a \emph{coordinate
patch}.\footnote{This will be made precise shortly, but caution should
be given: the literature is inconsistent on which way the arrow
goes. Some authors prefer taking the green patch of the surface, and
mapping it to some subset of $\RR^{2}$. It is a matter of convention,
and either choice is perfectly acceptable.}

\N{Possible Problems With Our Definition} We should pause a moment and
ponder if our notion of a surface is really well-defined, or if there
are some problems with it. The main issues we should think about:
\begin{enumerate}
\item The coordinates could be ``degenerate'', meaning that different
  values of $(u,v)$ correspond to the same point on the surface.

  \textsc{Solution:} We demand the coordinate patch be
  injective\footnote{Recall, a function $f\colon X\to Y$ is injective
  means for every $x_{1}$, $x_{2}\in X$ we have $f(x_{1})=f(x_{2})$ implies $x_{1}=x_{2}$.} to
  avoid this problem.
\item Even if $\chart{x}$ is injective, it could behave badly in other
  ways and not define a smooth surface. For example, the following
  ``surfaces'' are too ``pointy'' to be smooth:
  \begin{center}
    \includegraphics{img/surfaces.1} \includegraphics{img/surfaces.2}
  \end{center}

  \textsc{Solution:} Require that $\chart{x}$ be \emph{regular}.
\end{enumerate}

\begin{definition}
Given a map $F\colon \RR^{m}\to\RR^{n}$ (suppose $m\leq n$), its
\define{Tangent Map} at $\vec{p}\in\RR^{m}$
\begin{equation*}
F_{*\vec{p}}\colon\T_{\vec{p}}\RR^{m}\to\T_{F(\vec{p})}\RR^{n}
\end{equation*}
is defined as follows: given any $\vec{v}_{\vec{p}}\in\T_{\vec{p}}\RR^{m}$,
pick some curve $\alpha\colon I\to\RR^{m}$ such that $\alpha(0)=\vec{p}$
and $\alpha'(0) = \vec{v}_{\vec{p}}$. Then define
\begin{equation}
F_{*\vec{p}}(\vec{v}_{\vec{p}}) = \left.\frac{\D}{\D t}F\bigl(\alpha(t)\bigr)\right|_{t=0}.
\end{equation}
\begin{center}
  \includegraphics{img/surfaces.3}
\end{center}
\end{definition}

\begin{remark}
We should intuitively think of $F_{*\vec{p}}$ as ``the best linear
approximation to $F$ at $\vec{p}$''.
\end{remark}

\begin{remark}
This definition does not depend on choice of the curve $\alpha$.
\end{remark}

\begin{definition}
A map $F\colon\RR^{m}\to\RR^{n}$ is \define{Regular} if for every
$\vec{p}\in\RR^{m}$ we have $F_{*\vec{p}}$ be injective.
\end{definition}

\begin{remark}
%The intuition is that regular maps act like infinitesimals on tangent vectors.
This is a good definition, because if $\alpha$ is a regular curve, and
$F$ is a regular map, then the composition $F\circ\alpha$ is a regular
curve. Composing regular stuff together gives us something regular.
\end{remark}

\begin{definition}
A \define{(Coordinate) Chart} in $\RR^{3}$ is an injective regular map
$\chart{x}\colon D\to\RR^{3}$ where $D\subset\RR^{2}$ is some open
subset called the \define{Patch}.

Further, we call a chart \define{Proper} if $\chart{x}^{-1}\colon\chart{x}(D)\to\RR^{2}$
is continuous.

We may abuse language, and refer to the $D$ as the patch, and
$\chart{x}$ as the chart or parametrization. Technically, the local
coordinates refer to the components of the vector-valued function
$\chart{x}^{-1}\colon\chart{x}(D)\to D$ mapping a patch of our surface
to Euclidean space (the ``space of parameters'').
\end{definition}

\begin{remark}[Abuse of language]
Again, just to reiterate, people mix up what they're referring to when
using the terms ``chart'' and ``patch''. Undoubtedly \emph{we} will
too. 
\end{remark}

\begin{remark}
We must stress the importance of a patch $\chart{x}\colon D\to\RR^{3}$
being regular, which means for any $(u,v)\in D$, the map
\begin{equation}
\chart{x}_{*}\colon\T_{(u,v)}\RR^{2}\to\T_{\chart{x}(u,v)}\RR^{3}
\end{equation}
is injective.
\end{remark}

\begin{remark}
``Proper'' patches convey topological information. 
\end{remark}

\begin{remark}
The image of any coordinate patch gives an example of a surface.
\end{remark}

\M
Most surfaces cannot be covered by one coordinate patch. The famous
example: any coordinates on a sphere is degenerate around the
poles. Consequently, we need to use a set of patches to define a
surface.

\begin{definition}
Given a subset $M\subset\RR^{3}$ and a point $\vec{p}\in M$, a
\define{Neighborhood} of $\vec{p}$ is a set consisting of all points in
$M$ whose Euclidean distance to $\vec{p}$ is less than $\varepsilon$,
for some $\varepsilon>0$ [fixed for the neighborhood].
\end{definition}

\begin{definition}
A \define{Surface} in $\RR^{3}$ is a subset $M\subset\RR^{3}$ such that
for each point $\vec{p}\in M$ there exists a neighborhood $N$
of $\vec{p}$ in $M$ and a proper patch $\chart{x}\colon D\to\RR^{3}$
such that $N\subset\chart{x}(D)\subset M$.

\begin{center}
  \includegraphics{img/surfaces.4}
\end{center}
\end{definition}

\medbreak
\begin{remark}
We don't want self-intersecting surfaces, we want to avoid the following
doodle:
\begin{center}
\includegraphics{img/surfaces.5}
\end{center}
This is because there's no way to have a neighborhood ``near the
intersection''. It would locally look like:
\begin{center}
\includegraphics{img/surfaces.6}
\end{center}
Why is this a problem?\footnote{It's not Hausdorff, that's the problem.} This is a neighborhood of some point on the
intersection, say $N(\vec{p})$. We would like to find a chart
$\chart{x}\colon D\to\RR^{3}$ such that $N(\vec{p})\subset\chart{x}(D)$.
But this is impossible, because $\chart{x}(D)$ couldn't contain an
intersection (thanks to topology).
\end{remark}

\N{Determining if a Patch is Regular}
How do we even determine if a patch is regular, anyways?
Well, if $F\colon\RR^{m}\to\RR^{n}$ were regular at $\vec{p}\in\RR^{m}$,
then $F_{*\vec{p}}$ is injective. We know from linear algebra this means
the dimension of the image equals the dimension of the domain, i.e.,
\begin{equation}
\dim(\T_{F(\vec{p})}\RR^{n})=\dim(\T_{\vec{p}}\RR^{m}).
\end{equation}
This is equivalent to saying that the rank of $F_{*\vec{p}}$ is of
maximal rank for every $\vec{p}\in\RR^{m}$.
In other words, we know a patch $\chart{x}\colon D\to\RR^{3}$ is regular
if for each $\vec{p}\in D$ we have $\chart{x}_{*\vec{p}}$ be of maximal
rank. Our strategy for checking this will be to find some frame field
$\vec{e}_{1}$, $\vec{e}_{2}$ defined on $D$ and some frame field
$E_{1}$, $E_{2}$, $E_{3}$ on $\chart{x}(D)\subset\RR^{3}$. Then we will
express $\chart{x}_{*}$ as a $2\times3$ matrix, and we could use row
reduction to find the rank.

What we do is we consider the following diagram:
\begin{center}
  \includegraphics{img/surfaces.7}
\end{center}
Consider a curve $\alpha\colon I\to\RR^{2}$ such that
$\alpha(0)=\vec{p}$ and $\alpha'(0)=\vec{e}_{1}$ --- i.e., the curve
points in the $u$-direction. We compute
$\chart{x}_{*}(\vec{e}_{1})$ to get the components in the first column of
the matrix representing $\chart{x}_{*}$, and we find another curve
pointing in the $\vec{e}_{2}$ (i.e., in the $v$) direction to find the
second column of the matrix of $\chart{x}_{*}$.

We find,
\begin{subequations}
  \begin{align}
    \chart{x}_{*}(\vec{e}_{1}) &= \left.\frac{\D}{\D t}\chart{x}\bigl(\alpha(t)\bigr)\right|_{t=0}\\
&= \left.\left(\frac{\partial x^{1}}{\partial\alpha^{1}}\frac{\D\alpha^{1}}{\D t} +
\frac{\partial x^{1}}{\partial\alpha^{2}}\frac{\D\alpha^{2}}{\D t},
\frac{\partial x^{2}}{\partial\alpha^{1}}\frac{\D\alpha^{1}}{\D t} +
\frac{\partial x^{2}}{\partial\alpha^{2}}\frac{\D\alpha^{2}}{\D t},
\frac{\partial x^{3}}{\partial\alpha^{1}}\frac{\D\alpha^{1}}{\D t} +
\frac{\partial x^{3}}{\partial\alpha^{2}}\frac{\D\alpha^{2}}{\D t}\right)
\right|_{t=0}\\
&=\left(\frac{\partial x^{1}}{\partial u},
\frac{\partial x^{2}}{\partial u},
\frac{\partial x^{3}}{\partial u}\right)=\sum^{3}_{j=1}\frac{\partial x^{j}}{\partial u}E_{j}=:\chart{x}_{u}.
  \end{align}
\end{subequations}
Similarly, we find
\begin{equation}
\chart{x}_{*}(\vec{e}_{2})=\left(\frac{\partial x^{1}}{\partial v},
\frac{\partial x^{2}}{\partial v},
\frac{\partial x^{3}}{\partial v}\right)=\sum^{3}_{j=1}\frac{\partial x^{j}}{\partial v}E_{j}=: \chart{x}_{v}.
\end{equation}
We call the quantities $x_{u}$ and $x_{v}$ \define{Partial Velocities}.
Hence, if
$\vec{w}_{\vec{p}}=(w^{1},w^{2})_{\vec{p}}\in\T_{\vec{p}}\RR^{2}$, then
\begin{equation}
\chart{x}_{*}\begin{pmatrix}w^{1}\\w^{2}
\end{pmatrix}=\begin{pmatrix}
\displaystyle\frac{\partial x^{1}}{\partial u} &\displaystyle\frac{\partial x^{1}}{\partial v}\\
\displaystyle\frac{\partial x^{2}}{\partial u} &\displaystyle\frac{\partial x^{2}}{\partial v}\\
\displaystyle\frac{\partial x^{3}}{\partial u} &\displaystyle\frac{\partial x^{3}}{\partial v}
\end{pmatrix}\begin{pmatrix}w^{1}\\w^{2}
\end{pmatrix}.
\end{equation}
Now that we have expressed $\vec{x}_{*}$ as a matrix, we just need to
check there are at least 2 linearly independent rows, which can be done
by row reduction. Enough humourless logic, let us look at some examples.

\begin{example}
Consider the unit sphere $S^{2}=\{(x,y,z)\in\RR^{3}\mid x^{2}+y^{2}+z^{2}=1\}$
in three-dimensions. We have a path formed by ``bending'' the 
open unit disc $D^{2}=\{(x,y)\in\RR^{2}\mid x^{2}+y^{2}<1\}$.
More explicitly,
\begin{subequations}
\begin{equation}
\chart{x}\colon D^{2}\to\RR^{3}
\end{equation}
defined by
\begin{equation}
\chart{x}(u,v) = (u, v, \sqrt{1 - u^{2} - v^{2}}).
\end{equation}
\end{subequations}
This is just one possible patch, we could consider another by taking the
third component to be $-\sqrt{1-u^{2}-v^{2}}$, and we can consider
others by swapping the third component with either the first or second
components.

Now, our patch is clearly injective. If you do not believe it, then just
examine $\chart{x}(u_{1},v_{1})=\chart{x}(u_{2},v_{2})$; the first two
components reads $u_{1}=u_{2}$ and $v_{1}=v_{2}$. It follows that
$(u_{1},v_{1})=(u_{2},v_{2})$ and moreover $\chart{x}$ is injective.

But is our patch \emph{regular}? We can find the matrix of
$\chart{x}_{*}$ relative to the canonical frame fields, which reads
\begin{equation}
\chart{x}_{*} = \begin{bmatrix}1 & 0\\0 & 1\\\mbox{stuff}_{1} & \mbox{stuff}_{2}
\end{bmatrix}.
\end{equation}
Since the top $2\times 2$ submatrix is the identity matrix, it follows
that $\chart{x}_{*}$ has rank 2. Hence our patch is regular.
\end{example}

\begin{remark}
This example is a special case of a more general fact: if we have a
smooth function $f\colon D\to\RR$, and we consider its graph
$\Gamma(f)=\{(x,y,f(x,y))\in\RR^{3}\mid(x,y)\in D\}$ (or more generally,
for any $D\subset\RR^{n}$, we have
$\Gamma(f)=\{(\vec{x},f(\vec{x}))\in\RR^{n+1}\mid\vec{x}\in D\}$), then
this graph is a patch of a surface.

Algebraic geometry generalizes this further, by studying the zero sets
of functions $\{\vec{x}\in\RR^{n}\mid f(\vec{x})=0\}$. These generalize
the notion of surfaces. A lot of differential geometry is generalized in
this manner, it's very deep and profound.
\end{remark}

\begin{example}[Surface of revolution]
%% Suppose we have a regular curve $\alpha\colon I\to\RR^{2}$.
%% We can embed it in three-dimensions by mapping it into the $xy$-plane:
%% \begin{equation}
%% t\mapsto(\alpha^{1}(t),\alpha^{2}(t),0).
%% \end{equation}
Undergraduates are taught in integral calculus of a single variable
about a surface of revolution by taking a curve $y=f(x)$, then sweeping
it out around the $x$-axis, in the sense that
\begin{equation}
y^{2}+z^{2}=f(x)^{2}.
\end{equation}
This yields a parametrization in terms of $x$ and $\theta$. Our patch
would be
\begin{equation}
\chart{x}(x,\theta) = \left(x, f(x)\cos(\theta),f(x)\sin(\theta)\right).
\end{equation}
If $f$ is a regular curve, then we have a regular surface.
\end{example}

\N{What Patches Give Us}
The basic idea for patches is that they let us transfer data on the
surface $M$ to data (of various kinds) on the domain $D\subset\RR^{2}$
where we know how to do calculus. What kinds of things do patches give
us?
\begin{enumerate}
\item \textsc{``Local coordinates'' on $M$.} Grid lines in $D$ are paths like
  $\alpha(t)=(u_{0},v_{0}+t)$ which pass through the point
  $(u_{0},v_{0})$. These induce grid lines on $M$ by
  $\chart{x}\circ\alpha(t)=\chart(u_{0},v_{0}+t)$. (Although these
  describe grid lines of constant $u_{0}$, we can form grid lines of
  constant $v_{0}$ by examining $\chart(u_{0}+t,v_{0})$ for example.)
  \begin{center}
    \includegraphics{img/surfaces.8}
  \end{center}
\item \textsc{Convenient ways to get tangent vectors on $M$.} We have
  regularity map basis vectors (frame fields) to basis vectors (frame
  fields) by $\chart{x}_{*\vec{p}}\colon\T_{\vec{p}}\RR^{2}\to\T_{\chart{x}(\vec{p})}M$.
  Regularity guarantees we get a whole tangent plane, not just a line.
  \begin{center}
    \includegraphics{img/surfaces.9}
  \end{center}
\item \textsc{``Local'' frame fields on $M$.} This is given to us by the
  partial velocities $\chart{x}_{u}(u,v)$ and $\chart{x}_{v}(u,v)$.
  \begin{center}
    \includegraphics{img/surfaces.10}
  \end{center}
\item \textsc{Convenient ways to compute ``the normal vector'' to a surface.}
  Since we have found $\chart{x}_{u}(u,v)$ and $\chart{x}_{v}(u,v)$ are
  frame fields for the tangent vectors on the surface, we can consider
  their cross product
  $\vec{n}=\chart{x}_{u}(u,v)\times\chart{x}_{v}(u,v)$ which is normal
  to the surface.
  We can do this globally only for ``orientable'' surfaces (e.g., not
  for the M\"{o}bius strip).
  \begin{center}
    \includegraphics{img/surfaces.11}
  \end{center}
\end{enumerate}

\subsection*{Exercises}
\addcontentsline{toc}{subsection}{Exercises}

%% This roughly is before \S2.2 in Arcade's transcription of the notes.

%% At this point, homework 5 would be due [May 7, 2008].
%% It consisted of the following
%% exercises from O'Neill's \emph{Elementary Differential Geometry}
%% (Revised second ed.),
%% \begin{itemize}
%% \item 2.7 \# 1, 4, 5
%% \item 2.8 \# 2, 4
%% \end{itemize}

Here are some review questions, to make sure you don't forget too
quickly what we learned from section 2. (This is an experiment, let me
know if you hate this technique. It probably won't happen again in these
notes, though.)

\begin{enumerate}
  % 2.7 #4 
\item Recall Parabolic coordinates --- we have $0\leq u<\infty$,
$0\leq v<\infty$, and $0\leq\varphi<2\pi$, and the Cartesian coordinates
  are parametrized as
  \begin{subequations}
    \begin{align}
      x &= uv\cos(\varphi)\\
      y &= uv\sin(\varphi)\\
      z &= \frac{1}{2}(u^{2}-v^{2}).
    \end{align}
  \end{subequations}
  \begin{enumerate}
  \item Compute the Parabolic frame field $E_{1}$, $E_{2}$, $E_{3}$
  \item Compute the connection forms for the parabolic frame field.
  \end{enumerate}
% 2.7 # 5
\item Let $E_{1}$, $E_{2}$, $E_{3}$ constitute a frame field and
  $W=\sum_{j}f_{j}E_{j}$. Let $V$ be an arbitrary vector field.
  Prove or find a counter-example: the covariant derivative satisfies,
\begin{equation}
\nabla_{V}W = \sum_{j}\left(V[f_{j}] + \sum_{i}f_{i}\omega_{ij}[V]\right)E_{j}.
\end{equation}
\item Check the structure equations for the parabolic frame field.
% 2.8 # 4, frame fields on \RR^2. Prove there is an angle \theta such that
  % E_{1} = \cos(\theta)U_{1} + \sin(\theta)U_{2}
  % E_{2} = -\sin(\theta)U_{1} + \cos(\theta)U_{2}
  % (a) Express the connection form and dual 1-forms in terms of
  % $\theta$ and the natural coordinates $x$ and $y$
  % (b) What are the structural equations in this case? Check the
  % results of (a) satisfy them.
\end{enumerate}

% calculus on a surface
\subsection{Calculus on a Surface}

OK, we're on the home stretch now. We've generalized calculus in
$\RR^{n}$ using the machinery of tangent vectors and differential forms,
talked about curves and surfaces. Now our goal is to figure out how to
do calculus on surfaces. Ready? Let's go!

\M
Our goal is to completely generalize what we know about calculus on
$\RR^{2}$ to any surface.

This means we need to define: tangent vectors, vector fields, frame
fields, one-forms, differential forms, smooth functions, covariant
derivatives, etc., \emph{on a surface}.\footnote{The generalization to arbitrary
manifolds will be simple.} We'll use this to study various surfaces and
properties they have.

\emph{What's most fundamental in mathematics is making the correct definitions.}

\begin{figure}[h]
\centering
  \includegraphics{img/surfaces.12}
\caption{Intuition of a function $f\colon M\to\RR^{n}$ being smooth}\label{fig:surfaces:smooth-function-from-surface}
\end{figure}

\begin{definition}\label{defn:surfaces:smooth-function-from-surface}
Let $M\subset\RR^{3}$ be a surface, $f\colon M\to\RR^{n}$ be some
function on the surface.
We say that $f$ is \define{Smooth} if, for every patch $\chart{x}\colon D\to M$
(where $D\subset\RR^{2}$ is open),
\begin{equation}
f\circ\chart{x}\colon D\to\RR^{n}
\end{equation}
is smooth in the usual sense, i.e., $f\circ\chart{x}\in C^{\infty}(D)$.
This is schematically doodled in Figure~\ref{fig:surfaces:smooth-function-from-surface}.
\end{definition}

\begin{figure}[h]
\centering
  \includegraphics{img/surfaces.13}
\caption{Intuition of a function $f\colon \RR^{n}\to M$ being smooth}\label{fig:surfaces:smooth-function-to-surface}
\end{figure}

\begin{definition}\label{defn:surfaces:smooth-function-to-surface}
Let $M\subset\RR^{3}$ be a surface, $f\colon\RR^{n}\to M$ be a function
to the surface. We call $f$ \define{Differentiable} if, for every patch
$\chart{x}\colon D\to M$, the map
\begin{equation}
\chart{x}^{-1}\circ f\colon\mathcal{E}\to D,
\end{equation}
where $\mathcal{E}=f^{-1}(\chart{x}(D))\subset\RR^{n}$ is the preimage
of the patch under $f$, is a smooth ($C^{\infty}$) function. Note: we \emph{do
not} require $f(\RR^{n})=M$.

The intuition of this definition is doodled in Figure~\ref{fig:surfaces:smooth-function-to-surface}.
\end{definition}

\begin{remark}
This requires a bit of explanation. Consider the preimage of
$\chart{x}(D)$ under $f$, i.e., the set of points $\vec{x}\in\RR^{n}$
which $f$ maps into the image of the chart $\mathbf{x}(D)$; call this
set $\mathcal{E} = \{\vec{x}\in\RR^{n}\mid f(\vec{x})\in\chart(D)\}$.
We want this to be an open set for topological reasons (this makes $f$
continuous, a necessary prerequisite for derivatives). Now we could
consider $f|_{\mathcal{E}}\colon\mathcal{E}\to M$ by restricting $f$.
We know its image will be within the image of the chart, so we
then take the preimage of $f(\mathcal{E})\subset M$ under the chart
$\chart{x}$ to produce the mapping
$\chart{x}^{-1}\colon f|_{\mathcal{E}}(\mathcal{E})\to D$.
But this is the same as considering the composition
$\chart(x)^{-1}\circ f|_{\mathcal{E}}\colon\mathcal{E}\to D$.
The restriction of $f$ to $\mathcal{E}$ has been purely a crutch, the
preimage of $\chart{x}$ will restrict the composite function for us. So
we arrive at our definition.
\end{remark}

\begin{remark}
As a quick check, we could consider $M=\RR^{3}$ with $\chart{x}=\id$
being the identity function. Then $f\colon\RR^{n}\to\RR^{3}$ being
differentiable is the same as $f\in C^{\infty}(\RR^{n},\RR^{3})$. This
is good! Our definition of differentiable functions to surfaces
coincides with our pre-existing definition of differentiable
multivariate vector-valued functions.
\end{remark}

\begin{example}
Let $M\subset\RR^{3}$ be a surface, and consider a [smooth] path
$\alpha\colon I\to\RR^{3}$
such that the curve lies on the surface $\alpha(I)\subset M$.
For any patch $\chart{x}\colon D\to M$, we could consider 
the interval $J=\alpha^{-1}(\chart{x}(D))$ given by the preimage of the
curve which lies in $\chart{x}(D)$. The situation is as doodled below:
\begin{center}
  \includegraphics{img/surfaces.14}
\end{center}
Proving $\alpha$ is smooth on $M$ amounts to proving, for every patch
$\chart{x}\colon D\to M$ such that $\chart{x}(D)$ contains some part of
the path, the restriction $\alpha|_{J}$ is smooth in the preimage of the
chart in the familiar way. 
\end{example}

\N{Smooth Functions Between Surfaces}
Suppose now we have two surfaces $M_{1}$ and $M_{2}$. We can construct a
notion of a smooth function $f\colon M_{1}\to M_{2}$ between these
surfaces. The solution is to cheat.

Given arbitrary patches $\chart{x}\colon D\to M_{1}$ on $M_{1}$
and $\chart{y}\colon E\to M_{2}$ on $M_{2}$, we have the situation as
doodled below:
\begin{center}
  \includegraphics{img/surfaces.15}
\end{center}
Now we have to make sense of $f$. We first take the preimage of
$\chart{y}(E)$ under $f$, which may or may not intersect $\chart{x}(D)$
on $M_{1}$. If it doesn't, then we're in the trivial situation, and
everything works out fine. So let's examine the exciting case where
$f^{-1}\left(\chart{y}(E)\right)\cap\chart{x}(D)\neq\emptyset$. This
gives us the etched region doodled below:
\begin{center}
  \includegraphics{img/surfaces.16}
\end{center}
We can pull back $f^{-1}\left(\chart{y}(E)\right)$ to the patch $D$
using the preimage of $\chart{x}$, which produces the following
situation (with the hatched region indicating the $\chart{x}^{-1}\circ f^{-1}$
preimage):
\begin{center}
  \includegraphics{img/surfaces.17}
\end{center}
It looks like we're making this more complicated, doesn't it? There is
one thing we have not yet exploited: we can move \emph{forward} as well
as backward. If we start with $\chart{x}|_{f^{-1}(E)}^{-1}(D)$ the
portion of the patch which, when charted onto $M_{1}$ will be mapped by
$f$ to part of $\chart{y}(E)$, then move forward along these lines, we
end up with a subset
\begin{equation}
(f\circ\chart{x})\left(\chart{x}|_{f^{-1}(E)}^{-1}(D)\right)\subset\chart{y}(E).
\end{equation}
We can take its preimage under $\chart{y}$ to get a subset in
$E\subset\RR^{2}$. This gives us a mapping, however, from $D$ to $E$:
\begin{equation}
\chart{y}^{-1}\circ f\circ\chart{x}\colon D\to E,
\end{equation}
which is a function where we can sensibly discuss smoothness and
derivatives. In pictures, we get the situation as follows:
\begin{center}
  \includegraphics{img/surfaces.18}
\end{center}
The induced function is drawn with a dashed arrow, and it is the one
\emph{we know} how to determine if it's smooth or not (because it's a
function of an open subset in $\RR^{2}$ to an open subset in $\RR^{2}$).
And if we do this for every possible pair of patches on $M_{1}$ and
$M_{2}$, we end up verifying $f$ is smooth and differentiable.

More precisely, we have $f\circ\chart{x}$ be smooth function to $M_{2}$
in the sense of Definition~\ref{defn:surfaces:smooth-function-to-surface}
We also have $\chart{y}^{-1}\circ f$ be a smooth function on $M_{1}$, in
the sense of Definition~\ref{defn:surfaces:smooth-function-from-surface}.
Since this is done for every possible charts on $M_{1}$ and $M_{2}$, we
conclude that $f\colon M_{1}\to M_{2}$ is smooth.


\subsection*{Exercises}
\addcontentsline{toc}{subsection}{Exercises}

%% This roughly is before \S2.2 in Arcade's transcription of the notes.

%% At this point, homework 6 would be due [May 26(?), 2008].
%% It consisted of the following
%% exercises from O'Neill's \emph{Elementary Differential Geometry}
%% (Revised second ed.),
%% \begin{itemize}
%% \item 4.2 \# 2
%% \item 4.3 \# 2, 4
%% \end{itemize}

\begin{enumerate}
\item Partial velocities $\chart{x}_{u}$, $\chart{x}_{v}$ are defined for
  an arbitrary mapping $\chart{x}\colon D\subset\RR^{2}\to\RR^{3}$, so
  we can consider the [real-valued] functions
  \begin{equation}
E=\chart{x}_{u}\cdot\chart{x}_{u},\quad
F=\chart{x}_{u}\cdot\chart{x}_{v},\quad
G=\chart{x}_{v}\cdot\chart{x}_{v}
  \end{equation}
  on $D$.
\begin{enumerate}
\item Prove $\|\chart{x}_{u}\times\chart{x}_{v}\|^{2} = EG-F^{2}$.
\item Prove $\chart{x}$ is regular if and only if $EG-F^{2}$ is never zero.
\end{enumerate}
% 4.3 # 2, 4  
\end{enumerate}

\begin{framed}
\begin{quotation}
  \begin{center}
    {\large\bfseries Homework: Stereographic Projection}\medbreak

    \textbf{Mathematics 116 --- Differential Geometry}

    Spring 2008

    Derek Wise
  \end{center}
  
Stereographic projection gives a nice coordinate patch on the unit
sphere $x^{2}+y^{2}+z^{2}=1$. It is defined by
\begin{equation*}
\vec{x}\colon\RR^{2}\to\RR^{3}
\end{equation*}
where $\vec{x}(u,v)$ is defined to be the unique point in $\RR^{3}$ that
lies both on the unit sphere and the ray from $(0,0,1)$ through $(u,v,0)$.

\begin{enumerate}
\item Derive an explicit formula for $\vec{x}(u,v)$. [Hint: use a
  parameterization of the line, and solve for the time $t$ when it
  passes through the unit sphere.]
\item Find the matrix of the tangent map $\vec{x}_{*}$, relative to the
  natural frame fields on $\RR^{2}$ and $\RR^{3}$.
\item Prove that $\vec{x}$ is a patch.
\item Show that $\vec{x}$ is \emph{conformal}, meaning that it preserves
  angles. That is, given a pair of tangent vectors $w_{p}$, $z_{p}$ at
  the same point in $\RR^{2}$, show that the angle between them (defined
  by the dot product in $\RR^{2}$) is the same as the angle between
  $\vec{x}_{*}(w_{p})$ and $\vec{x}_{*}(z_{p})$ (defined by the dot
  product in $\RR^{3}$).
\end{enumerate}
\end{quotation}
\end{framed}

\vfill\eject

\subsection{Vectors on Surfaces}

\M We require $\chart{x}\colon D\to\RR^{3}$ be smooth ($C^{\infty}$) and
$\chart{x}^{-1}\colon D\gets\chart{x}(D)$ be continuous. We require an
additional property for the inverse to be differentiable.

\begin{definition}
Let $M$ be a surface, let $\vec{p}\in M$ be some point.
We define the \define{Tangent Space} to $\vec{p}$ in $M$, denoted
$\T_{\vec{p}}M$, is the set of all vectors
$\vec{v}_{\vec{p}}\in\T_{\vec{p}}\RR^{3}$ such that
$\vec{v}_{\vec{p}}=\alpha'(0)$ for some curve on the surface $\alpha\colon I\to M$.
\end{definition}

\N{Base points are important}
In $\RR^{2}$, we often just ignored the point of tangency and pretended
that $\T_{\vec{p}}\RR^{2}$ and $\T_{\vec{q}}\RR^{2}$ are the same just
by sliding $\vec{p}$ to $\vec{q}$. But for general surfaces (of which
$\RR^{2}$ is just the most boring example), we cannot do this. There is
no way to slide $\T_{\vec{p}}M$ to $\T_{\vec{q}}M$ without embedding $M$
into $\RR^{2}$.
\begin{center}
  \includegraphics{img/surfaces.19}
\end{center}

\begin{definition}
A \define{Vector Field} $V$ on a surface $M$ is an assignment to each
point $\vec{p}\in M$ a vector $V(\vec{p})\in\T_{\vec{p}}M$.
\end{definition}

\M
Originally, we defined the derivative of $f$ in the direction of some tangent
vector $\vec{v}_{\vec{p}}$ as
\begin{equation}
\vec{v}_{\vec{p}}[f] = \left.\frac{\D}{\D t}f(\vec{p}+t\vec{v})\right|_{t=0}.
\end{equation}
This captures the information of how much $f$ changes in the direction
of $\vec{v}$ (at base-point $\vec{p}$). There's a problem generalizing
this to a surface: it doesn't work if $f$ is defined only on $M$. Why
not? Well, the line $\vec{p}+t\vec{v}$ will leave the surface, and $f$
is undefined off the surface, so we're out of luck.

But later we proved, if $\alpha\colon I\to\RR^{n}$ passes through
$\vec{p}=\alpha(0)$ and it has velocity $\vec{v}_{\vec{p}}=\alpha'(0)$
there, then we could define the directional derivative as:
\begin{equation}
\vec{v}_{\vec{p}}[f] = \left.\frac{\D}{\D t}f(\alpha(t))\right|_{t=0}.
\end{equation}
So if $\alpha\colon I\to M$ has initial position $\alpha(0)=\vec{p}$ and
initial velocity $\alpha'(0)=\vec{v}_{\vec{p}}\in\T_{\vec{p}}M$, and
if $f\colon M\to\RR$ is smooth, then we can define the directional
derivative of a function on our surface by
\begin{equation}
\vec{v}_{\vec{p}}[f] = \left.\frac{\D}{\D t}f(\alpha(t))\right|_{t=0}.
\end{equation}
This is independent of the choice of such $\alpha$.

If we do this at every point, we can differentiate functions with
respect to vector fields using
\begin{equation}
V[f](\vec{p}) = V(\vec{p})[f].
\end{equation}
This works out perfectly.

\begin{remark}[Boring]
This should be boring, because we defined things in a clever way.
Generalizations follow easily once we have the right definitions.
So if you find this boring, good: it means you have a grasp of the
concepts underlying the definitions.
\end{remark}

\subsection{Differential Forms on Surfaces}

\M A one-form $\phi$ on $M$ assigns to each point $\vec{p}\in M$ a
covector on $\T_{\vec{p}}M$, i.e., a linear function
\begin{equation}
\phi_{\vec{p}}\colon \T_{\vec{p}}M\to\RR.
\end{equation}
The most important examples: let $f\colon M\to\RR$ be a smooth function,
then $\D f$ is a one-form given by
\begin{equation}
\D f[\vec{v}_{\vec{p}}] = \vec{v}_{\vec{p}}[f],
\end{equation}
for every $\vec{v}_{\vec{p}}\in\T_{\vec{p}}M$.
But let us see what the zoo of differential forms becomes on a surface.

\N{Zero-Forms}The 0-forms are just smooth functions on $M$, i.e.,
functions like $\phi\colon M\to\RR$ such thatfor every patch
$\chart{x}\colon D\to M$, the function $\phi\colon\chart{x}\colon D\to\RR$
is smooth.

\N{One-Forms}
The 1-forms are defined just like in Euclidean space. A
\define{One-Form} $\phi$ is a linear map at each point $\vec{p}\in M$
taking tangent vectors to real numbers
\begin{equation}
\phi_{\vec{p}}\colon\T_{\vec{p}}M\to\RR
\end{equation}
in a linear way, and taking vector fields to functions
\begin{equation}
\phi\colon\Vect(M)\to C^{\infty}(M).
\end{equation}
Given a zero-form $f$, the differential $\D f$ is the one-form given by
\begin{equation}
\D f[V] = V[f],
\end{equation}
for any vector field $V\in\Vect(M)$.

\N{Two-Forms} Now we have something slightly different. But it tells us
what 2-forms \emph{do}. A 2-form $\eta$ on $M$ is a map at each point
$\vec{p}\in M$ that takes an \emph{ordered pair} of tangent vectors and
gives a number, that is to say,
\begin{equation}
\eta_{\vec{p}}\colon\T_{\vec{p}}M\times\T_{\vec{p}}M\to\RR
\end{equation}
such that
\begin{enumerate}
\item Antisymmetry: $\eta(\vec{v},\vec{w}) = -\eta(\vec{w},\vec{v})$
\item Linearity in first slow: for any $a,b\in\RR$,
  $\eta(a\vec{u}+b\vec{v},\vec{w}) = a\eta(\vec{u},\vec{w})+b\eta(\vec{v},\vec{w})$.
\end{enumerate}
It's easy to prove from these two properties that a 2-form is also
linear in the second slot; that is to say, it's \emph{bilinear}.

If we use $\eta$ at every point, we get a mapping
\begin{equation}
\eta\colon\Vect(M)\times\Vect(M)\to C^{\infty}(M).
\end{equation}

\N{But\dots the wedge product?}
Earlier we defined 2-forms in terms of the formal wedge product. Let us now
endeavour to produce a definition of the wedge product for differential
forms on a surface which is consistent with how we defined 2-forms.

Let $\phi$, $\psi$ be two 1-forms on $M$. We want to make a 2-form out
of them, and call it $\phi\wedge\psi$ (and make it a mapping
$\Vect(M)\times\Vect(M)\to C^{\infty}(M)$). The most obvious thing we
could try is,
\begin{equation}
(\phi\wedge\psi)(V,W) = \phi(V)\psi(W).
\end{equation}
Does it work? No, not by a long shot, since
\begin{equation}
(\phi\wedge\psi)(V,W)=\phi(V)\psi(W)\neq-\phi(W)\psi(V)\mbox{ in general}.
\end{equation}
Let us try
\begin{equation}
(\phi\wedge\psi)(V,W)\stackrel{???}{=}\phi(V)\psi(W)-\phi(W)\psi(V).
\end{equation}
Does it work?

We can see it is antisymmetric, since
\begin{subequations}
  \begin{align}
    (\phi\wedge\psi)(V,W)=\phi(V)\psi(W)-\phi(W)\psi(V)\\
    &=-\phi(W)\psi(V)-(-\phi(V)\psi(W))\\
    &=-(\phi\wedge\psi)(W,V),
  \end{align}
\end{subequations}
which is a relief. So this is possibly a good definition.

Now the real moment of truth: is it linear in the
first slot? We have something stronger than \emph{mere} linearity, it's
linear with respect to arbitrary smooth function $f,g\in C^{\infty}(M)$,
we have
\begin{equation}
(\phi\wedge\psi)(fU + gV,W) = f(\phi\wedge\psi)(U,W) + g(\phi\wedge\psi)(V,W).
\end{equation}
This is awesome!

And what's really cute: we had an axiom
(\S\ref{sec:introduction:axioms-of-wedge-product}) that the formal wedge
product is anticommutative on 1-forms. We see that
\begin{subequations}
  \begin{align}
    (\phi\wedge\psi)(V,W)
    &=\phi(V)\psi(W) -\phi(W)\psi(V)\\
    &=-(-\phi(V)\psi(W)+\phi(W)\psi(V))\\
    &=-(-\psi(W)\phi(V)+\psi(V)\phi(W))\\
    &=-(\psi\wedge\phi)(V,W).
  \end{align}
\end{subequations}
In fact we have
\begin{equation}
\begin{array}{ccc}
(\phi\wedge\psi)(V,W) &=& -(\phi\wedge\psi)(W,V)\\
\rotatebox[origin=c]{-90}{$=$} & & \rotatebox[origin=c]{-90}{$=$}\\
-(\psi\wedge\phi)(V,W) &=& (\psi\wedge\phi)(W,V).
\end{array}
\end{equation}
It's consistent!

\begin{remark}
We see the 3-form a 2-dimensional surface $M\subset\RR^{3}$ is zero.
\end{remark}

\N{Corollary: Nilpotence}
The reader can verify that, for any one-form $\phi$, we have
$\phi\wedge\phi=0$. We proved this formally, as a consequence of
antisymmetry, but the reader may verify this is true for our concrete
realization of the wedge product.

\M
Let us consider a 2-form $\eta$ and suppose $\vec{e}_{1}$,
$\vec{e}_{2}\in\T_{\vec{p}}M$ is a basis. We will do some multilinear
algebra: once we know how $\eta$ acts on all possible linear
combinations of our basis vectors $\vec{e}_{1}$ and $\vec{e}_{2}$,
then we will know how it acts on any arbitrary vector in $\T_{\vec{p}}M$.

Consider
\begin{subequations}
\begin{align}
\eta(\alpha\vec{e}_{1}+\beta\vec{e}_{2},
\gamma\vec{e}_{1}+\delta\vec{e}_{2})
&=\alpha\eta(\vec{e}_{1}, \gamma\vec{e}_{1}+\delta\vec{e}_{2})
+\beta\eta(\vec{e}_{2}, \gamma\vec{e}_{1}+\delta\vec{e}_{2})\\
&=\alpha(\gamma\eta(\vec{e}_{1}, \vec{e}_{1}) +\delta\eta(\vec{e}_{1}, \vec{e}_{2}))
+\beta(\gamma\eta(\vec{e}_{2}, \vec{e}_{1}) +\delta\eta(\vec{e}_{2}, \vec{e}_{2}))\\
&=\alpha\gamma\eta(\vec{e}_{1}, \vec{e}_{1})
+\alpha\delta\eta(\vec{e}_{1}, \vec{e}_{2})
+\beta\gamma\eta(\vec{e}_{2}, \vec{e}_{1})
+\beta\delta\eta(\vec{e}_{2}, \vec{e}_{2})\\
&=0 +\alpha\delta\eta(\vec{e}_{1}, \vec{e}_{2})
-\beta\gamma\eta(\vec{e}_{1}, \vec{e}_{2})
+ 0\\
&= (\alpha\delta-\beta\gamma)\eta(\vec{e}_{1}, \vec{e}_{2})\\
&=\det\begin{pmatrix}\alpha & \beta\\
\gamma & \delta
\end{pmatrix}\eta(\vec{e}_{1}, \vec{e}_{2}).
\end{align}
\end{subequations}
We only need to compute $\eta(\vec{e}_{1}, \vec{e}_{2})$ once, and then
computing $\eta(\vec{v},\vec{w})$ amounts to computing a determinant.



\begin{comment}
  \subsection*{Exercises}
\addcontentsline{toc}{subsection}{Exercises}

%% June 4, 2008 due date.
%% At this point, homework 7 would be due [June 4, 2008].
%% It consisted of the following
%% exercises from O'Neill's \emph{Elementary Differential Geometry}
%% (Revised second ed.),
%% \begin{itemize}
%% \item 4.4 \# 3
%% \item 5.1 \# 1, 4, 5, 6
%% \item 5.2 \# 4 a,c
%% \item 5.3 \# 4a
%% \end{itemize}

\begin{enumerate}
\item Let $f\colon\Sigma\to\RR$ be a real-valued function on the surface
  $\Sigma$, let $g\colon\RR\to\RR$ be some smooth function.
  \begin{enumerate}
  \item Prove or find a counter-example: for any
    $\vec{v}_{\vec{p}}\in\T_{\vec{p}}\Sigma$, we have $\vec{v}_{\vec{p}}[g\circ f] = g'(f(\vec{p}))\vec{v}_{\vec{p}}[f]$
  \item Deduce $\D(g\circ f) = (g'\circ f)\,\D f$.
  \end{enumerate}
% 5.1 # 4: describe the Gauss map for a variety of surfaces (the plane,
% sphere, cylinder, cone)
\item Let $T^{2}$ be the familiar torus.
  \begin{enumerate}
  \item What are the image curves under
    the Gauss map for meridians and parallels of $T^{2}$?
  \item Are there any points $\vec{q}\in G(T^{2})$ for which exactly two
    distinct points on the torus $\vec{p}_{1}\in T^{2}$ and
    $\vec{p}_{2}\in T^{2}$ are mapped to $G(\vec{p}_{1})=G(\vec{p}_{2})=\vec{q}$?
  \end{enumerate}
\item Consider the surface $\Sigma$ defined by $z=xy$.
  \begin{enumerate}
  \item What are the image curves under
    the Gauss map for $x=\mbox{constant}$ on $\Sigma$?
  \item Are there any points $\vec{q}\in G(\Sigma)$ for which exactly two
    distinct points on the surface $\vec{p}_{1}\in\Sigma$ and
    $\vec{p}_{2}\in\Sigma$ (so $\vec{p}_{1}\neq\vec{p}_{2}$) which are mapped to $G(\vec{p}_{1})=G(\vec{p}_{2})=\vec{q}$?
  \end{enumerate}
\item For each of the following surfaces, find the quadratic
  approximation near the origin:
  \begin{enumerate}
  \item $z = x^{2}+y^{2}$
  \item $z = x^{2}-y^{2}$
  \item $x^{2}+y^{2}-z^{2}=0$.
  \item $z = (2x + 3y)^{5}$.
  \end{enumerate}
\item Recall from linear algebra, if $A$ is any $n\times n$ matrix, the
  \define{Characteristic Polynomial} of $A$ is the polynomial in
  $\lambda$ defined by
  \begin{equation*}
p(\lambda) = \det(A - \lambda\cdot I_{n}),
  \end{equation*}
  where $I_{n}$ is the $n\times n$ identity matrix. \textbf{Compute} the
  characteristic polynomial for the shape operator.
\end{enumerate}
\end{comment}