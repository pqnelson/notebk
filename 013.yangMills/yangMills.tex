%%
%% yangMills.tex
%% 
%% Made by Alex Nelson
%% Login   <alex@tomato>
%% 
%% Started on  Fri Sep 18 11:33:33 2009 Alex Nelson
%% Last update Fri Sep 18 11:33:33 2009 Alex Nelson
%%
%% TODO:
% * need to reperform the Poisson Bracket of the Gauss Constraint with the Hamiltonian
% * need to start expanding some more on the general Yang-Mills situation
\documentclass{article}
\usepackage{notebk,exercises}


\newcommand{\ssn}[1]{\medbreak\refstepcounter{subsection}\addcontentsline{toc}{subsection}{\thesubsection~#1}\noindent\ignorespaces\textbf{\thesubsection~#1.}~}

\hypersetup{pdftitle={Notes on the Yang Mills Theory},
            pdfauthor={Alex Nelson},
            pdfcreator={Alex Nelson},
            pdfsubject={File Systems},
            unicode={false}}
\title{Notes on the Yang Mills Theory}
\date{September 18, 2009}
\begin{document}
\maketitle
\begin{abstract}
We review classical electromagnetism in relativity. We consider
generalizing the notion to ``general gauge groups'' which is
precisely Yang-Mills theory. We will consider everything in the
classical domain, quantization is not discussed.
\end{abstract}
\tableofcontents

\section{From Maxwell to Tensors}
\section{Maxwell's Equations in a Nutshell}

Recall in classical electromagnetism we have it summed in Maxwell's equations~\cite{jackson}. In the presence of a charge density $\rho(\vec{x},t)$ and a 
current density $\vec{j}(\vec{x},t)$, the electric and magnetic fields $\vec{E}$
and $\vec{B}$ satisfy the equations
\begin{subequations}\label{maxwellsEquations}
\begin{align}
\nabla\cdot\vec{E}&=\rho\label{gaussLaw}\\
\nabla\times\vec{B}&=\frac{1}{c}\vec{j} + \frac{1}{c}\frac{\partial\vec{E}}{\partial t}\label{AmpereLaw}\\
\nabla\cdot\vec{B} &= 0\label{GaussLawMagnet}\\
\nabla\times\vec{E} &= -\frac{1}{c}\frac{\partial\vec{B}}{\partial t} \label{FaradayLaw}
\end{align}
\end{subequations}
where cgs units are used.

In the second pair of equations (Eqs \ref{GaussLawMagnet} and \ref{FaradayLaw})
follows the existence of scalar and vector potentials $\phi(\vec{x},t)$ and
$\vec{A}(\vec{x},t)$ defined by
\begin{equation}
\vec{B} = \nabla\times\vec{A},\quad\vec{E}=-\nabla\phi - \frac{1}{c}\frac{\partial\vec{A}}{\partial t}.
\end{equation}
However, this does not determine the system uniquely, since for an \emph{arbitrary}
function $f(\vec{x},t)$ the transformation
\begin{equation}\label{emGaugeTransformation}
\phi\to\phi'=\phi + \frac{1}{c}\frac{\partial f}{\partial t},\quad\vec{A}\to\vec{A}' = \vec{A} - \nabla f
\end{equation}
leaves the fields $\vec{E}$ and $\vec{B}$ unaltered. The transformation (\ref{emGaugeTransformation})
is known as a gauge transformation of the second kind\footnote{I.e. it is described
mathematically in differential geometry as a connection form.}. Since all 
observable quantities can be expressed in terms of 
$\vec{E}$ and $\vec{B}$, it is a fundamental requirement of any theory
formulated in terms of potentials that is gauge; i.e. the predictions for the 
observable quantities are invariant under such gauge transformations.

When we express Maxwell's equations in terms of potentials, the second pair
are automatically satisfied. The first pair (\ref{gaussLaw} and \ref{AmpereLaw}) become
\begin{subequations}
\begin{align}
-\nabla^2\phi - \frac{1}{c}\frac{\partial}{\partial t}(\nabla\cdot\vec{A}) = \Box\phi - \frac{1}{c}\frac{\partial}{\partial t}\left(\frac{1}{c}\frac{\partial\phi}{\partial t} + \nabla\cdot\vec{A}\right) = \rho\\
\Box\vec{A} + \nabla\left(\frac{1}{c}\frac{\partial\phi}{\partial t} + \nabla\cdot\vec{A}\right) = \frac{1}{c}\vec{j}
\end{align}
\end{subequations}
where
\begin{equation}
\Box \equiv \frac{1}{c^2}\frac{\partial^2}{\partial t^2} - \nabla^2
\end{equation}
is called the ``D'Alembertian''.

We can now consider the so-called ``free field'' case. That is, we have no
charge or current so $\rho=0$ and $\vec{j}=0$. We can choose a gauge for the
system such that
\begin{equation}\label{radiationGauge}
\nabla\cdot\vec{A} = 0.
\end{equation}
The condition (\ref{radiationGauge}) defines the \textbf{Coulomb or radiation gauge}. A vector field with vanishing divergence (ie satisfying Eq (\ref{radiationGauge})) is called a ``transverse field'' since for a wave
\begin{equation}
\vec{A}(\vec{x},t) = \vec{A}_0 \exp(i(\vec{k}\cdot\vec{x}-\omega t))
\end{equation}
gives
\begin{equation}
\vec{k}\cdot\vec{A} = 0,
\end{equation}
or in other words $\vec{A}$ is perpendicular to the direction of propagation 
$\vec{k}$ of the wave. In the Coulomb gauge, the vector potential is a transverse
vector.

\pagebreak
\section{General Gauge Group: Introducing the Yang-Mills Theory}
\section{Feynman Rules in a Nutshell with a Toy Model}


We will work in a toy model\footnote{It is commonly referred to as the $\phi^4$ 
model in the literature.} with massive spinless particles (so we won't have to 
worry about spin). This is the easiest nontrivial example of the use of Feynman 
diagrams. The basic ritual of Feynman diagrams is outlined thus:
\begin{enumerate}
\item{(Notation)} Label the incoming and outgoing four-momenta $p_1$, $p_2$,
$\ldots$, $p_n$. Label the internal momenta $q_1$, $q_2$, $\ldots$. Put an
arrow on each line, keeping track of the ``positive'' direction (antiparticles
move ``backward'' in time).

\item{(Coupling Constant)} At each vertex, write a factor of
\begin{equation*}
-ig
\end{equation*}
where $g$ is called the ``\textbf{coupling constant}''; it specifies the
strength of the interaction. In our toy model, $g$ will have dimensions of
momentum, but in the real world it is dimensionless.

\item{(Propagator)} For each internal line, write a factor
\begin{equation*}
\frac{i}{q_{j}^{2} - m_{j}^2c^2}
\end{equation*}
where $q_j$ is the four-momentum of the line ($q_j^2=q_{j}^{\mu}q_{j\mu}$; i.e.
$j$ is just a label keeping track of which internal line we are dealing with)
and $m_j$ is the mass of the particle the line describes. (Note that for
virtual particles, we don't have the $E^2 - \vec{p}\cdot\vec{p}=m^2c^2$ relation
that's for external legs only!)

\item{(Conservation of Momentum)} For each vertex, write a delta function of
the form
\begin{equation*}
(2\pi)^4\delta^{(4)}(k_1+k_2+k_3)
\end{equation*}
where the $k$'s are the three four-momenta coming \emph{into} the vertex (if
the arrow leads outward, then $k$ is \emph{minus} the four-momentum of that
line). This factor imposes conservation of energy and momentum at each vertex,
since the delta function is zero unless the sum of the incoming momenta equals
the sum of the outgoing momenta.

\item{(Integration over Internal Momenta)} For each internal line, write down
a factor
\begin{equation*}
\frac{1}{(2\pi)^4}d^{4}q_{j}
\end{equation*}
and integrate over all internal momenta.

\item{(Cancel the Delta Function)} The result will include a delta function
\begin{equation*}
(2\pi)^4\delta^{(4)}(p_1 + p_2 + \cdots - p_n)
\end{equation*}
enforcing overall conservation of energy and momentum. Erase this factor, and
what remains is $i\mathcal{M}$ that is $-i$ times the contribution to the 
amplitude from this process.
\end{enumerate}

What we do with these rules is we form an integrand by multiplying everything
together, so at the end it should look something like this:
\begin{equation}
i\mathcal{M} \textrm{ ``='' } \begin{pmatrix}$coupling$\\
$constants$
\end{pmatrix}
\int
\begin{pmatrix}
$propagators$
\end{pmatrix}
\begin{pmatrix}
$delta$\\ $functions$
\end{pmatrix}d
\begin{pmatrix}
$internal$\\ $lines$
\end{pmatrix}
\end{equation}

\pagebreak
\section{Answers to Selected Exercises}
\dumpanswers
\pagebreak
\nocite{*}
\bibliographystyle{utcaps}\addcontentsline{toc}{section}{References}
\bibliography{yangMills}
\end{document}
