%%
%% ladder.tex
%% 
%% Made by Alex Nelson
%% Login   <alex@tomato>
%% 
%% Started on  Sun Jun  7 14:38:41 2009 Alex Nelson
%% Last update Sun Jun  7 14:38:41 2009 Alex Nelson
%%
\documentclass[draft]{amsart}
\usepackage{url}
\usepackage{manfnt}
\usepackage{amsthm}
\usepackage{amsmath}
\usepackage{amsthm}
\usepackage{amssymb}
\usepackage{amsfonts}
\usepackage{amscd}
\usepackage{graphicx}
\usepackage{mathrsfs}

\numberwithin{equation}{section}

\theoremstyle{definition}
\newtheorem{defn}{Definition}
\newtheorem{thm}{Theorem}
\newtheorem{rmk}{Remark}
\newtheorem{lem}{Lemma}
\newtheorem{cor}{Corollary}
\newtheorem{ex}{Example}
\newtheorem{prop}{Proposition}
\newtheorem{sch}{Scholium}
\newtheorem{axm}{Axiom}
\newtheorem*{prob}{Problem}

\def\re{\operatorname{Re}}
\def\tr{\operatorname{Tr}}
\def\<{\langle}
\def\>{\rangle}

%%
% This macro header is what controls the ``dangerous bend''
% paragraph
%%
\def\rd{\noindent\begingroup\hangindent=3pc\hangafter=-2\def\par{\endgraf\endgroup}\hbox
  to0pt{\hskip-\hangindent\dbend\hfill}\ignorespaces}
%%
% This command allows you to write stuff in small font size and
% use the
% bourbaki ``dangerous bend'' so it's great when you want to
% ramble on 
% about some extra stuff!
%%
\newcommand{\danger}[1] {\rd{\small {#1}}}

%%
% This macro header is what controls the ``dangerous bend''
% paragraph
%%
\def\ddbend{\dbend\kern1pt\dbend}

\def\rdd{\noindent\begingroup\hangindent=4pc\hangafter=-2\def\par{\endgraf\endgroup}\hbox
  to0pt{\hskip-\hangindent\ddbend\hfill}\ignorespaces}

\newcommand{\ddanger}[1] {\rdd{\small {#1}}}

\newcommand{\define}[1] {\textbf{#1}\index{#1}}
\title{Notes on Ladder Operators}
\date{June 07, 2009}
\email{pqnelson@gmail.com}
\author{Alex Nelson}
\begin{document}
\begin{abstract}
We review ladder operators.
\end{abstract}
\maketitle
\section{Commutation Relations}
%%
%% commutationRelations.tex
%% 
%% Made by Alex Nelson
%% Login   <alex@tomato>
%% 
%% Started on  Sun Jun  7 14:39:34 2009 Alex Nelson
%% Last update Sun Jun  7 14:39:34 2009 Alex Nelson
%%
We have the operators $\widehat{X}$ and $\widehat{N}$ such that
\begin{equation}%\label{eq:}
[\widehat{N},\widehat{X}]=c\widehat{X}
\end{equation}
where $c$ is a scalar value. We have the eigenstates of
$\widehat{N}$ such that
\begin{equation}%\label{eq:}
\widehat{N}|n\> = n|n\>
\end{equation}
where we have abused notation letting $|n\>$ be a vector and $n$
be a scalar (the eigenvalue of the $|n\>$ vector). We see that
\begin{subequations}
\begin{align}
\widehat{N}\widehat{X}|n\> &= \left(\widehat{X}\widehat{N}+[\widehat{N},\widehat{X}]\right)|n\>\\
&=\left(\widehat{X}\widehat{N}+c\widehat{X}\right)|n\>\\
&=\widehat{X}\widehat{N}|n\>+c\widehat{X}|n\>\\
&=\widehat{X}n|n\>+c\widehat{X}|n\>\\
&=(n+c)\widehat{X}|n\>.
\end{align}
\end{subequations}
This means that if $|n\>$ is an eigenstate of $\widehat{N}$ with
eigenvalue $n$, then $(\widehat{X}|n\>)$ is an eigenstate of
$\widehat{N}$ with eigenvalue $n+c$. If $c>0$ then $n+c>n$, so
the operator $\widehat{X}$ is called the ``\define{Creation Operator}''. 

If $\widehat{N}$ is a physical observable, it's necessarily self
adjoint. This implies that $c$ is real, since $n$ is real and
$n+c$ is an eigenvalue of a self-adjoint operator. We should
remember from linear algebra the eigenvalues of a self-adjoint
operator is always real. The Hermitian adjoint of $\widehat{X}$
satisfies
\begin{equation}%\label{eq:}
[\widehat{N},\widehat{X}^{\dag}]=-c\widehat{X}^{\dag}
\end{equation}
We call $\widehat{X}^\dag$ an ``\define{Annihilation Operator}''
if $\widehat{X}$ is a creation operator.

\begin{prop}
Let $\widehat{A}$, $\widehat{B}$, $\widehat{C}$ be
operators. Then
\begin{equation}%\label{eq:}
[\widehat{A},\widehat{B}\widehat{C}]=[\widehat{A},\widehat{B}]\widehat{C}+\widehat{B}[\widehat{A},\widehat{C}].
\end{equation}
\end{prop}
\begin{proof}
We see by direct computation
\begin{subequations}
\begin{align}
[\widehat{A},\widehat{B}\widehat{C}]&=\widehat{A}\widehat{B}\widehat{C}-\widehat{B}\widehat{C}\widehat{A}\\
\widehat{B}[\widehat{A},\widehat{C}]&=\widehat{B}(\widehat{A}\widehat{C}-\widehat{C}\widehat{A})
\label{eq:eqTwo}\\
[\widehat{A},\widehat{B}]\widehat{C}&=(\widehat{A}\widehat{B}-\widehat{B}\widehat{A})\widehat{C}\label{eq:eqThree}
\end{align}
\end{subequations}
so when we add equation \eqref{eq:eqTwo} to equation \eqref{eq:eqThree} we get
\begin{equation}%\label{eq:}
\widehat{A}\widehat{B}\widehat{C}-\widehat{B}\widehat{C}\widehat{A}
= [\widehat{A},\widehat{B}\widehat{C}]
\end{equation}
as desired.
\end{proof}
\begin{prop}\label{prop:vanishingCommutators}
Let $\widehat{A}$, $\widehat{B}$ be operators. If their
commutator vanishes
\begin{equation}%\label{eq:}
[\widehat{A},\widehat{B}] = 0
\end{equation}
then the two operators are equal up to a constant.
\end{prop}
\begin{proof}
Trivial.
\end{proof}
\begin{prop}
Given these two propositions and the commutations relations, we
can find that
\begin{equation}%\label{eq:}
[\widehat{N},\widehat{X}\widehat{X}^\dag]=0.
\end{equation}
Or equivalently
\begin{equation}%\label{eq:}
\widehat{N}=\widehat{X}^{\dag}\widehat{X}
\end{equation}
up to some constant.
\end{prop}
\begin{proof}
How to prove this? By direct computation
\begin{subequations}
\begin{align}
[\widehat{N},\widehat{X}^\dag\widehat{X}]&=[\widehat{N},\widehat{X}^\dag]\widehat{X}+\widehat{X}^{\dag}[\widehat{N},\widehat{X}]\\
&=(-c\widehat{X}^\dag)\widehat{X}+\widehat{X}^\dag(c\widehat{X})\\
&=(c-c)\widehat{X}^\dag\widehat{X}\\
&=0.
\end{align}
\end{subequations}
This means, up to some constant and ordering,
\begin{equation}%\label{eq:}
\widehat{N}=\widehat{X}^\dag\widehat{X}
\end{equation}
just as desired.
\end{proof}


\nocite{*}
\bibliographystyle{utcaps}
\bibliography{ladder}
\end{document}
