\subsection{Electron-Muon Scattering}

We draw the diagram (note the use of $\mu$ and $\nu$ at the vertices, which are 
used to sum over in the integral):

\strut

\begin{center}
\begin{fmffile}{QEDexOneImg1}
  \begin{fmfgraph*}(50,25) \fmfpen{0.2mm}
    \fmfset{arrow_len}{3mm}\fmfset{arrow_ang}{10}
    \fmfleft{i1,o1} % change i2->o1 
    \fmfright{i2,o2} % change o1->i2
    \fmflabel{$p_{1},s_{1}$}{i1}
    \fmflabel{$p_{3},s_{3}$}{o1} %
    \fmflabel{$p_{2},s_{2}$}{i2} %
    \fmflabel{$p_{4},s_{4}$}{o2}
    \fmflabel{$\mu$}{v1}
    \fmflabel{$\nu$}{v2}
    \fmf{fermion}{i1,v1,o1} %
    \fmf{dbl_plain_arrow}{i2,v2,o2} %
    \fmf{boson,label=$q$}{v1,v2}
  \end{fmfgraph*}
\end{fmffile}
\end{center}
\strut

We will now evaluate it in a haphazard manner. Observe how it is
done when spinors are in the game.

\textbf{Step One:} We will evaluate the part emboldened in Red
first.


\strut
\begin{center}
\begin{fmffile}{QEDexOneImg2}
  \begin{fmfgraph*}(50,25)  \fmfpen{0.2mm}
    \fmfset{arrow_len}{3mm}\fmfset{arrow_ang}{10}
    \fmfleft{i1,o1} % change i2->o1 
    \fmfright{i2,o2} % change o1->i2
    \fmflabel{$p_{1},s_{1}$}{i1}
    \fmflabel{$p_{3},s_{3}$}{o1} %
    \fmflabel{$p_{2},s_{2}$}{i2} %
    \fmflabel{$p_{4},s_{4}$}{o2}
    \fmflabel{$\mu$}{v1}
    \fmflabel{$\nu$}{v2}
    \fmf{fermion,fore=red}{i1,v1,o1} %
    \fmf{dbl_plain_arrow}{i2,v2,o2} %
    \fmf{boson,label=$q$}{v1,v2}
  \end{fmfgraph*}
\end{fmffile}
\end{center}
\strut

We will now analyze it in careful detail so we will ``pull it
out'' and ``dissect'' it carefully.

We evaluate it in the following manner: since we write quantum mechanics like
we write chinese (from right to left), we begin with 
\begin{equation}
\Diagram{\vertexlabel^{p_3,s_3}\\
fd \\
& g\vertexlabel_{\mu,q}\\
\vertexlabel_{p_1,s_1} {\color{red}fu}\\
} = u(s_1,p_1),
\qquad
\Diagram{\vertexlabel^{p_3,s_3}\\
fd \\
& {\color{red}g}\vertexlabel_{\mu,q}\\
\vertexlabel_{p_1,s_1} fu\\
} = (ig_{e}\gamma^{\mu})u(s_1,p_1)
\end{equation}
We have one last step to do
\begin{equation}
\Diagram{\vertexlabel^{p_3,s_3}\\
{\color{red}fd} \\
  & g\vertexlabel_{\mu,q} \\
\vertexlabel_{p_1,s_1} fu\\
} = \bar{u}(s_3,p_3)(ig_{e}\gamma^{\mu})u(s_1,p_1)
\end{equation}
So this contributes
\begin{equation}\label{exOneElectronTerm}
(\bar{u}(s_3,p_3))(ig_{e}\gamma^\mu)(u(s_1, p_1))
\end{equation}
to the integrand. Our integrand is going to take the form
\begin{equation}
[(\bar{u}(s_3,p_3))(ig_{e}\gamma^\mu)(u(s_1, p_1))]\begin{pmatrix} $photon$\\$propagator$\end{pmatrix}\begin{pmatrix}$muon$\\$terms$\end{pmatrix} \begin{pmatrix}$conservation$\ $of$\\
$momentum$\ $delta$\\
$functions$\end{pmatrix}
\end{equation}
We will now move on to step two.

\textbf{Step Two:} We will consider the photon propagator, which corresponds to
the red line in the following diagram


\strut
\begin{center}
\begin{fmffile}{QEDexOneImg3}
  \begin{fmfgraph*}(50,25)  \fmfpen{0.2mm}
    \fmfset{arrow_len}{3mm}\fmfset{arrow_ang}{10}
    \fmfleft{i1,o1} % change i2->o1 
    \fmfright{i2,o2} % change o1->i2
    \fmflabel{$p_{1},s_{1}$}{i1}
    \fmflabel{$p_{3},s_{3}$}{o1} %
    \fmflabel{$p_{2},s_{2}$}{i2} %
    \fmflabel{$p_{4},s_{4}$}{o2}
    \fmflabel{$\mu$}{v1}
    \fmflabel{$\nu$}{v2}
    \fmf{fermion}{i1,v1,o1} %
    \fmf{dbl_plain_arrow}{i2,v2,o2} %
    \fmf{boson,label=$q$,fore=red}{v1,v2}
  \end{fmfgraph*}
\end{fmffile}
\end{center}
\strut


This corresponds to the term
\begin{equation}
\frac{-ig_{\mu\nu}}{q^2}
\end{equation}
giving our integrand to be
\begin{equation}
[(\bar{u}(s_3,p_3))(ig_{e}\gamma^\mu)(u(s_1, p_1))]\frac{-ig_{\mu\nu}}{q^2}\begin{pmatrix}$muon$\\$terms$\end{pmatrix} \begin{pmatrix}$conservation$\ $of$\\
$momentum$\ $delta$\\
$functions$\end{pmatrix}.
\end{equation}

\textbf{Step Three and Four:} Moving right along to the Muon terms, we have exactly a term
analagous to Eq (\ref{exOneElectronTerm}). Muons are fermions with spin 1/2,
with the same electric charge as an electron. So translating this into Feynman
diagram terms, it is translated in the exact same fashion we translated the
electron terms. So we have a contribution of
\begin{equation}
(\bar{u}(s_4,p_4))(ig_{e}\gamma^\nu)(u(s_2, p_2)).
\end{equation}
Our integrand now becomes
\begin{equation}
[(\bar{u}(s_3,p_3))(ig_{e}\gamma^\mu)(u(s_1, p_1))]\frac{-ig_{\mu\nu}}{q^2}[(\bar{u}(s_4,p_4))(ig_{e}\gamma^\nu)(u(s_2, p_2))] \begin{pmatrix}$conservation$\ $of$\\
$momentum$\ $delta$\\
$functions$\end{pmatrix}.
\end{equation}

\textbf{Step Five:} We kind of ``fudged up'' steps 1 through 4 because they are
so interconnected it is hard to seperate them out from each other. We are now
safely onto step 5 of the Feynman rules of QED: conservation of momentum! We
have two places to do this (at the $\mu$ and $\nu$ vertices). We have for $\mu$
(chosen randomly) the input momentum in red and output momentum in blue:


\strut
\begin{center}
\begin{fmffile}{QEDexOneImg4}
  \begin{fmfgraph*}(50,25)  \fmfpen{0.2mm}
    \fmfset{arrow_len}{3mm}\fmfset{arrow_ang}{10}
    \fmfleft{i1,o1} % change i2->o1 
    \fmfright{i2,o2} % change o1->i2
    \fmflabel{$p_{1},s_{1}$}{i1}
    \fmflabel{$p_{3},s_{3}$}{o1} %
    \fmflabel{$p_{2},s_{2}$}{i2} %
    \fmflabel{$p_{4},s_{4}$}{o2}
    \fmflabel{$\mu$}{v1}
    \fmflabel{$\nu$}{v2}
    \fmf{fermion,fore=red}{i1,v1} %
    \fmf{fermion,fore=blue}{v1,o1}
    \fmf{dbl_plain_arrow}{i2,v2,o2} %
    \fmf{boson,label=$q$,fore=blue}{v1,v2}
  \end{fmfgraph*}
\end{fmffile}
\end{center}
\strut


This corresponds to the conservation of momentum
\begin{equation}
p_1 = p_3 + q \quad\Rightarrow\quad p_1 - p_3 - q = 0
\end{equation}
which gives us the delta function
\begin{equation}\label{exOneTermToTakeAdvantageOf}
(2\pi)^{4}\delta^{(4)}(p_1 - p_3 - q).
\end{equation}
We have another conservation of momentum point, which is at the
vertex $\nu$:



\strut
\begin{center}
\begin{fmffile}{QEDexOneImg5}
  \begin{fmfgraph*}(50,25)  \fmfpen{0.2mm}
    \fmfset{arrow_len}{3mm}\fmfset{arrow_ang}{10}
    \fmfleft{i1,o1} % change i2->o1 
    \fmfright{i2,o2} % change o1->i2
    \fmflabel{$p_{1},s_{1}$}{i1}
    \fmflabel{$p_{3},s_{3}$}{o1} %
    \fmflabel{$p_{2},s_{2}$}{i2} %
    \fmflabel{$p_{4},s_{4}$}{o2}
    \fmflabel{$\mu$}{v1}
    \fmflabel{$\nu$}{v2}
    \fmf{fermion}{i1,v1} %
    \fmf{fermion}{v1,o1}
    \fmf{dbl_plain_arrow,fore=red}{i2,v2} %
    \fmf{dbl_plain_arrow,fore=blue}{v2,o2} %
    \fmf{boson,label=$q$,fore=red}{v1,v2}
  \end{fmfgraph*}
\end{fmffile}
\end{center}
\strut


Which corresponds to a conservation of momentum 
\begin{equation}
p_2 + q = p_4\quad\Rightarrow\quad p_2 + q - p_4 = 0
\end{equation}
and thus contributes the delta function term
\begin{equation}
(2\pi)^4 \delta^{(4)}(p_2 + q - p_4)
\end{equation}
rendering our integrand to be
\begin{eqnarray}
\quad&&[(\bar{u}(s_3,p_3))(ig_{e}\gamma^\mu)(u(s_1, p_1))]\frac{-ig_{\mu\nu}}{q^2}[(\bar{u}(s_4,p_4))(ig_{e}\gamma^\nu)(u(s_2, p_2))] (2\pi)^{8}\nonumber\\
& &\times \delta^{(4)}(p_1 - p_3 - q)\delta^{(4)}(p_2 + q - p_4) d^4q. \nonumber
\end{eqnarray}

\textbf{Step Six:} We integrate over the internal momenta (in our case the photon's momentum), so we have the integral expression:
\begin{eqnarray}
i\mathcal{M} &\textrm{``=''}& (2\pi)^{4} \int [(\bar{u}(s_3,p_3))(ig_{e}\gamma^\mu)(u(s_1, p_1))]\frac{-ig_{\mu\nu}}{q^2}[(\bar{u}(s_4,p_4))(ig_{e}\gamma^\nu)(u(s_2, p_2))] \nonumber\\
& &\times \delta^{(4)}(p_1 - p_3 - q)\delta^{(4)}(p_2 + q - p_4) d^4q. \nonumber
\end{eqnarray}
Observe this is harder than it looks because we are taking the trace of gamma
matrices! That is the whole point of having the metric tensor $g^{\mu\nu}$ here.
So it is a bit tricky to compute...

We will integrate over $q$ and take advantage of the delta function term (\ref{exOneTermToTakeAdvantageOf})
to make the switch
\begin{equation*}
q\to p_1 - p_3
\end{equation*}
giving us the result from the integral
\begin{equation}\label{exOneIntegralResult}
(2\pi)^{4} \frac{ig_{e}^2}{(p_1 - p_3)^2} [(\bar{u}(s_3,p_3))(ig_{e}\gamma^\mu)(u(s_1, p_1))][(\bar{u}(s_4,p_4))(ig_{e}\gamma_\mu)(u(s_2, p_2))]\delta^{(4)}(p_2 + p_1 - p_3 - p_4).
\end{equation}

\textbf{Step Seven:} We simply set Eq (\ref{exOneIntegralResult}) to be equal
to $-i\mathcal{M}\delta^{(4)}(p_2 + p_1 - p_3 - p_4)$, and we solve to find
\begin{equation}
\mathcal{M} = \frac{-g_{e}^2}{(p_1 - p_3)^2} [(\bar{u}(s_3,p_3))(ig_{e}\gamma^\mu)(u(s_1, p_1))][(\bar{u}(s_4,p_4))(ig_{e}\gamma_\mu)(u(s_2, p_2))]
\end{equation}
is the probability amplitude. In spite of this nightmarish appearence, with 
four spinors and eight $\gamma$ matrices, this is still just a number. We can
figure it out when the spins are specified.
