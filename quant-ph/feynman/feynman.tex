\documentclass{amsart}
\usepackage{feyn}
\usepackage{fly}
\usepackage{color}
\usepackage{feynmp}

\def\slashchar#1{\setbox0=\hbox{$#1$}           % set a box for #1
   \dimen0=\wd0                                 % and get its size
   \setbox1=\hbox{/} \dimen1=\wd1               % get size of /
   \ifdim\dimen0>\dimen1                        % #1 is bigger
      \rlap{\hbox to \dimen0{\hfil/\hfil}}      % so center / in box
      #1                                        % and print #1
   \else                                        % / is bigger
      \rlap{\hbox to \dimen1{\hfil$#1$\hfil}}   % so center #1
      /                                         % and print /
   \fi}                                         % 


\pagestyle{plain} 


\title{Notes on Feynman Diagrams}
\date{August 04, 2008}
\begin{document}\setlength{\unitlength}{1mm}
\begin{abstract}
We introduce in a pedagogical manner how to compute probability amplitudes
from Feynman diagrams, starting with $\phi^4$ model. We introduce the notion of
renormalization in this model at the one-loop level. Then we review the Dirac
equation, and introduce QED. We then perform several example calculations in
QED. The appendices gives a survey of Gamma matrices and use of Feynman diagrams
in computing decay rates.
\end{abstract}
\maketitle\footnotetext{Last Updated: \today}


%\tableofcontents
\section{Feynman Rules in a Nutshell with a Toy Model}


We will work in a toy model\footnote{It is commonly referred to as the $\phi^4$ 
model in the literature.} with massive spinless particles (so we won't have to 
worry about spin). This is the easiest nontrivial example of the use of Feynman 
diagrams. The basic ritual of Feynman diagrams is outlined thus:
\begin{enumerate}
\item{(Notation)} Label the incoming and outgoing four-momenta $p_1$, $p_2$,
$\ldots$, $p_n$. Label the internal momenta $q_1$, $q_2$, $\ldots$. Put an
arrow on each line, keeping track of the ``positive'' direction (antiparticles
move ``backward'' in time).

\item{(Coupling Constant)} At each vertex, write a factor of
\begin{equation*}
-ig
\end{equation*}
where $g$ is called the ``\textbf{coupling constant}''; it specifies the
strength of the interaction. In our toy model, $g$ will have dimensions of
momentum, but in the real world it is dimensionless.

\item{(Propagator)} For each internal line, write a factor
\begin{equation*}
\frac{i}{q_{j}^{2} - m_{j}^2c^2}
\end{equation*}
where $q_j$ is the four-momentum of the line ($q_j^2=q_{j}^{\mu}q_{j\mu}$; i.e.
$j$ is just a label keeping track of which internal line we are dealing with)
and $m_j$ is the mass of the particle the line describes. (Note that for
virtual particles, we don't have the $E^2 - \vec{p}\cdot\vec{p}=m^2c^2$ relation
that's for external legs only!)

\item{(Conservation of Momentum)} For each vertex, write a delta function of
the form
\begin{equation*}
(2\pi)^4\delta^{(4)}(k_1+k_2+k_3)
\end{equation*}
where the $k$'s are the three four-momenta coming \emph{into} the vertex (if
the arrow leads outward, then $k$ is \emph{minus} the four-momentum of that
line). This factor imposes conservation of energy and momentum at each vertex,
since the delta function is zero unless the sum of the incoming momenta equals
the sum of the outgoing momenta.

\item{(Integration over Internal Momenta)} For each internal line, write down
a factor
\begin{equation*}
\frac{1}{(2\pi)^4}d^{4}q_{j}
\end{equation*}
and integrate over all internal momenta.

\item{(Cancel the Delta Function)} The result will include a delta function
\begin{equation*}
(2\pi)^4\delta^{(4)}(p_1 + p_2 + \cdots - p_n)
\end{equation*}
enforcing overall conservation of energy and momentum. Erase this factor, and
what remains is $i\mathcal{M}$ that is $-i$ times the contribution to the 
amplitude from this process.
\end{enumerate}

What we do with these rules is we form an integrand by multiplying everything
together, so at the end it should look something like this:
\begin{equation}
i\mathcal{M} \textrm{ ``='' } \begin{pmatrix}$coupling$\\
$constants$
\end{pmatrix}
\int
\begin{pmatrix}
$propagators$
\end{pmatrix}
\begin{pmatrix}
$delta$\\ $functions$
\end{pmatrix}d
\begin{pmatrix}
$internal$\\ $lines$
\end{pmatrix}
\end{equation}
 % done
\subsection{Example}

We will consider the process $A+A\to B+B$, which is represented by the following
Feynman diagram (note that the x axis is the spatial dimension, the y axis is
the time dimension):

\strut

\begin{center}
\begin{fmffile}{exOneImg1}
  \begin{fmfgraph*}(50,25)  \fmfpen{0.1mm}
%     \\fmfset{arrow_len}{4mm}\\fmfset{arrow_ang}{10}
    \fmfleft{i1,o1} % change i2->o1 
    \fmfright{i2,o2} % change o1->i2
    \fmflabel{$A,p_{1}$}{i1}
    \fmflabel{$B,p_{3}$}{o1} %
    \fmflabel{$A,p_{2}$}{i2} %
    \fmflabel{$B,p_{3}$}{o2}
    \fmf{plain}{i1,v1,o1} %
    \fmf{plain}{i2,v2,o2} %
    \fmf{plain,label=$C,,q$}{v1,v2}
  \end{fmfgraph*}
\end{fmffile}
\end{center}
\strut
\textbf{Step One:} We drew it careful about notation (note the internal momentum line $q$ and the external lines $p_j$). 



\textbf{Step Two:} We have to worry about the vertices, at each one we have to
award a term of 
\begin{equation*}
-ig
\end{equation*}
So here is the Feynman diagram with the vertices enlarged in red:

\strut

\begin{center}
\begin{fmffile}{exOneImg2}
  \begin{fmfgraph*}(50,25) \fmfpen{0.1mm}
%     \\fmfset{arrow_len}{4mm}\\fmfset{arrow_ang}{10}
    \fmfleft{i1,o1} % change i2->o1 
    \fmfright{i2,o2} % change o1->i2
    \fmflabel{$A,p_{1}$}{i1}
    \fmflabel{$B,p_{3}$}{o1} %
    \fmflabel{$A,p_{2}$}{i2} %
    \fmflabel{$B,p_{3}$}{o2}
    \fmf{plain}{i1,v1,o1} %
    \fmf{plain}{i2,v2,o2} %
    \fmf{plain,label=$C,,q$}{v1,v2}    
    \fmfv{decor.shape=circle,decor.filled=full,decor.size=2thick,fore=red}{v1,v2}
  \end{fmfgraph*}
\end{fmffile}\\
\end{center}

\strut


We see that there are two vertices, one where $A$ emits
$C$ and becomes $B$:

\strut


\begin{center}
\strut
\begin{fmffile}{exOneImg3}
  \begin{fmfgraph*}(50,25) \fmfpen{0.1mm}
%     \\fmfset{arrow_len}{4mm}\\fmfset{arrow_ang}{10}
    \fmfleft{i1,o1} % change i2->o1 
    \fmfright{i2,o2} % change o1->i2
    \fmflabel{$A,p_{1}$}{i1}
    \fmflabel{$B,p_{3}$}{o1} %
    \fmflabel{$A,p_{2}$}{i2} %
    \fmflabel{$B,p_{3}$}{o2}
    \fmf{plain}{i1,v1,o1} %
    \fmf{plain}{i2,v2,o2} %
    \fmf{plain,label=$C,,q$}{v1,v2}    
    \fmfv{decor.shape=circle,decor.filled=full,decor.size=2thick,fore=red}{v1}
  \end{fmfgraph*}
\end{fmffile}
\strut
\end{center}
\strut
\\
 and the other where $A$ receives $C$ and turns into $B$:
\strut
\\
\begin{center}
\strut
\begin{fmffile}{exOneImg4}
  \begin{fmfgraph*}(50,25) \fmfpen{0.1mm}
%     \\fmfset{arrow_len}{4mm}\\fmfset{arrow_ang}{10}
    \fmfleft{i1,o1} % change i2->o1 
    \fmfright{i2,o2} % change o1->i2
    \fmflabel{$A,p_{1}$}{i1}
    \fmflabel{$B,p_{3}$}{o1} %
    \fmflabel{$A,p_{2}$}{i2} %
    \fmflabel{$B,p_{3}$}{o2}
    \fmf{plain}{i1,v1,o1} %
    \fmf{plain}{i2,v2,o2} %
    \fmf{plain,label=$C,,q$}{v1,v2}    
    \fmfv{decor.shape=circle,decor.filled=full,decor.size=2thick,fore=red}{v2}
  \end{fmfgraph*}
\end{fmffile}
\end{center}
\strut


By our rules, this means we get two factors of 
\begin{equation*}
-ig.
\end{equation*}
That is to say, our integrand is thus
\begin{equation}
(-ig)^2
\end{equation}
and we will add even more to it!

\textbf{Step Three:} (Let $m_{C}$ be the mass of a $C$ particle.) We also need a propagator for the internal line; we have below the Feynman diagram with the 
internal line in red:
\strut

\begin{center}
\begin{fmffile}{exOneImg5}
  \begin{fmfgraph*}(50,25) \fmfpen{0.1mm}
%     \\fmfset{arrow_len}{4mm}\\fmfset{arrow_ang}{10}
    \fmfleft{i1,o1} % change i2->o1 
    \fmfright{i2,o2} % change o1->i2
    \fmflabel{$A,p_{1}$}{i1}
    \fmflabel{$B,p_{3}$}{o1} %
    \fmflabel{$A,p_{2}$}{i2} %
    \fmflabel{$B,p_{3}$}{o2}
    \fmf{plain}{i1,v1,o1} %
    \fmf{plain}{i2,v2,o2} %
    \fmf{plain,label=$C,,q$,fore=red}{v1,v2}    

  \end{fmfgraph*}
\end{fmffile}
\end{center}
\strut

This means we have a factor of
\begin{equation*}
\frac{i}{q^2 - m^{2}_{C}c^2}.
\end{equation*}
We multiply this into our integrand which becomes
\begin{equation}
(-ig)^2\frac{i}{q^2 - m^{2}_{C}c^2}.
\end{equation}


\textbf{Step Four:} Now conservation of momentum demands two delta functions;
we see that the momentum has to be conserved at the vertices, so we have two
diagrams in color this time. At one vertex, we have the input momentum (the red line be)
equal to the sum of the output momentum (blue lines):

\strut

\begin{center}
\begin{fmffile}{exOneImg6}
  \begin{fmfgraph*}(50,25) \fmfpen{0.1mm}
%     \\fmfset{arrow_len}{4mm}\\fmfset{arrow_ang}{10}
    \fmfleft{i1,o1} % change i2->o1 
    \fmfright{i2,o2} % change o1->i2
    \fmflabel{$A,p_{1}$}{i1}
    \fmflabel{$B,p_{3}$}{o1} %
    \fmflabel{$A,p_{2}$}{i2} %
    \fmflabel{$B,p_{3}$}{o2}
    \fmf{plain,fore=red}{i1,v1} %
    \fmf{plain,fore=blue}{v1,o1}
    \fmf{plain}{i2,v2,o2} %
    \fmf{plain,label=$C,,q$,fore=blue}{v1,v2}    

  \end{fmfgraph*}
\end{fmffile}
\end{center}
\strut

This means we have the conservation of momentum:
\begin{equation}
p_1 = p_3 + q.
\end{equation}
This corresponds to the delta function of
\begin{equation*}
(2\pi)^4\delta^{(4)}(p_1 - p_3 - q)
\end{equation*}
(i.e. the left vertex has momentum conserved). We multiply this into the integrand,
which becomes
\begin{equation}
(2\pi)^4(-ig)^2\frac{i}{q^2 - m^{2}_{C}c^2}\delta^{(4)}(p_1 - p_3 - q).
\end{equation}
We have another vertex too,
which we have the ``input momenta lines'' in red summed to have the same momentum 
as the ``output momenta lines'' in blue:

\strut

\begin{center}
\begin{fmffile}{exOneImg7}
  \begin{fmfgraph*}(50,25) \fmfpen{0.1mm}
%     \\fmfset{arrow_len}{4mm}\\fmfset{arrow_ang}{10}
    \fmfleft{i1,o1} % change i2->o1 
    \fmfright{i2,o2} % change o1->i2
    \fmflabel{$A,p_{1}$}{i1}
    \fmflabel{$B,p_{3}$}{o1} %
    \fmflabel{$A,p_{2}$}{i2} %
    \fmflabel{$B,p_{3}$}{o2}
    \fmf{plain,fore=red}{i2,v2} %
    \fmf{plain,fore=blue}{v2,o2}
    \fmf{plain}{i1,v1,o1} %
    \fmf{plain,label=$C,,q$,fore=red}{v1,v2}    

  \end{fmfgraph*}
\end{fmffile}
\end{center}
\strut

This corresponds to the conservation of momentum
\begin{equation}
p_2 + q = p_4\Rightarrow p_2 + q - p_4 = 0
\end{equation}
and this corresponds to a delta function of
\begin{equation*}
(2\pi)^4\delta^{(4)}(p_{2}+q-p_{4})
\end{equation*}
(i.e. the right vertex has momentum conserved). The integrand becomes
\begin{equation}
(2\pi)^8(-ig)^2\frac{i}{q^2 - m^{2}_{C}c^2}\delta^{(4)}(p_1 - p_3 - q)\delta^{(4)}(p_{2}+q-p_{4}).
\end{equation}

\textbf{Step Five:} We integrate over the internal lines, luckily we only have one! We have one integration thus one term
\begin{equation*}
\frac{1}{(2\pi)^4}d^{4}q.
\end{equation*}
Combining rules 1 through 5 gives us the final expression
\begin{equation}
-i(2\pi)^4g^2\int\frac{1}{q^2-m_{C}^2c^2}\delta^{(4)}(p_1 - p_3 - q)\delta^{(4)}(p_{2}+q-p_{4}).
\end{equation}
The second delta function serves to pick out the value of everything else at the
point $q=p_4-p_2$, so we have
\begin{equation}
-ig^2\frac{1}{(p_4-p_2)^2-m_{C}^2c^2}(2\pi)^4\delta^{(4)}(p_1+p_2-p_3-p_4).
\end{equation}
And we have one last delta function which tells us that we conserved the
overall energy and momentum. We erase it by rule 6 and we get the amplitude for
this particular process to be
\begin{equation}
\mathcal{M} = \frac{g^2}{(p_4-p_2)^2 - m_{C}^2c^2}.
\end{equation}
This particular process is called a ``\textbf{tree diagram}'' because we
do not have any internal loops. Lets consider such an example next.
 % done
\subsection{A Slightly More Complicated Example}

This example will teach you that any idiot can complicate a simple scheme, consider
the following diagram:

\strut

\begin{center}
\begin{fmffile}{exTwoImg1}
  \begin{fmfgraph*}(70,25)  \fmfpen{0.1mm}

    \fmfleft{i1,o1} 
    \fmfright{i2,o2} 
    \fmflabel{$A,p_{1}$}{i1}
    \fmflabel{$B,p_{3}$}{o1} %
    \fmflabel{$A,p_{2}$}{i2} %
    \fmflabel{$B,p_{3}$}{o2}
    \fmf{plain}{i1,v1,o1} %
    \fmf{plain}{i2,v4,o2} %
    \fmf{plain,label=$C,,q_1$}{v1,v2}
    \fmf{plain,left,label=$B,,q_3$,tension=0.5}{v2,v3}
    \fmf{plain,left,label=$A,,q_2$,tension=0.5}{v3,v2}
    \fmf{plain,label=$C,,q_4$}{v3,v4}
  \end{fmfgraph*}
\end{fmffile}
\end{center}
\strut

\textbf{Step One:} We drew it carefully and made special care of the notation used.

\textbf{Step Two:} We need to take note of how many vertices we have, so we have
them enlarged in red to see how many:

\strut

\begin{center}
\begin{fmffile}{exTwoImg2}
  \begin{fmfgraph*}(70,25)  \fmfpen{0.1mm}

    \fmfleft{i1,o1} 
    \fmfright{i2,o2} 
    \fmflabel{$A,p_{1}$}{i1}
    \fmflabel{$B,p_{3}$}{o1} %
    \fmflabel{$A,p_{2}$}{i2} %
    \fmflabel{$B,p_{3}$}{o2}
    \fmf{plain}{i1,v1,o1} %
    \fmf{plain}{i2,v4,o2} %
    \fmf{plain,label=$C,,q_1$}{v1,v2}
    \fmf{plain,left,label=$B,,q_3$,tension=0.5}{v2,v3}
    \fmf{plain,left,label=$A,,q_2$,tension=0.5}{v3,v2}
    \fmf{plain,label=$C,,q_4$}{v3,v4}
    \fmfv{decor.shape=circle,decor.filled=full,decor.size=2thick,fore=red}{v1,v2,v3,v4}
  \end{fmfgraph*}
\end{fmffile}
\end{center}
\strut


we have 4 vertex terms, that is we have
\begin{equation*}
(-ig)^4
\end{equation*}
in the integrand so far.

\textbf{Step Three:} We need to take care of the internal lines now, so we will
see which lines those are exactly:


\strut

\begin{center}
\begin{fmffile}{exTwoImg3}
  \begin{fmfgraph*}(70,25)  \fmfpen{0.1mm}

    \fmfleft{i1,o1} 
    \fmfright{i2,o2} 
    \fmflabel{$A,p_{1}$}{i1}
    \fmflabel{$B,p_{3}$}{o1} %
    \fmflabel{$A,p_{2}$}{i2} %
    \fmflabel{$B,p_{3}$}{o2}
    \fmf{plain}{i1,v1,o1} %
    \fmf{plain}{i2,v4,o2} %
    \fmf{plain,label=$C,,q_1$,fore=red}{v1,v2}
    \fmf{plain,left,label=$B,,q_3$,tension=0.5,fore=red}{v2,v3}
    \fmf{plain,left,label=$A,,q_2$,tension=0.5,fore=red}{v3,v2}
    \fmf{plain,label=$C,,q_4$,fore=red}{v3,v4}
  \end{fmfgraph*}
\end{fmffile}
\end{center}
\strut

Since we are being pedagogical, we will go through one by one and indicate which
line we are dealing with and what we evaluate it to be. We will begin in any
old arbitrary manner we please with this particular example, \textbf{it won't
be that way in general!} We will first consider:

\strut
\begin{center}
\begin{fmffile}{exTwoImg4}
  \begin{fmfgraph*}(70,25)  \fmfpen{0.1mm}

    \fmfleft{i1,o1} 
    \fmfright{i2,o2} 
    \fmflabel{$A,p_{1}$}{i1}
    \fmflabel{$B,p_{3}$}{o1} %
    \fmflabel{$A,p_{2}$}{i2} %
    \fmflabel{$B,p_{3}$}{o2}
    \fmf{plain}{i1,v1,o1} %
    \fmf{plain}{i2,v4,o2} %
    \fmf{plain,label=$C,,q_1$,fore=red}{v1,v2}
    \fmf{plain,left,label=$B,,q_3$,tension=0.5}{v2,v3}
    \fmf{plain,left,label=$A,,q_2$,tension=0.5}{v3,v2}
    \fmf{plain,label=$C,,q_4$}{v3,v4}
  \end{fmfgraph*}
\end{fmffile}
\end{center}
\strut


We evaluate this to be the propagator described by
\begin{equation*}
\frac{i}{q_{1}^2 - m_{C}^2c^2}
\end{equation*}
so we multiply it into the integrand. The integrand is then
\begin{equation}
(-ig)^4 \frac{i}{q_{1}^2 - m_{C}^2c^2}.
\end{equation}
We continue on and we see the term given by the internal line in
red


\strut
\begin{center}
\begin{fmffile}{exTwoImg5}
  \begin{fmfgraph*}(70,25)  \fmfpen{0.1mm}

    \fmfleft{i1,o1} 
    \fmfright{i2,o2} 
    \fmflabel{$A,p_{1}$}{i1}
    \fmflabel{$B,p_{3}$}{o1} %
    \fmflabel{$A,p_{2}$}{i2} %
    \fmflabel{$B,p_{3}$}{o2}
    \fmf{plain}{i1,v1,o1} %
    \fmf{plain}{i2,v4,o2} %
    \fmf{plain,label=$C,,q_1$}{v1,v2}
    \fmf{plain,left,label=$B,,q_3$,tension=0.5}{v2,v3}
    \fmf{plain,left,label=$A,,q_2$,tension=0.5,fore=red}{v3,v2}
    \fmf{plain,label=$C,,q_4$}{v3,v4}
  \end{fmfgraph*}
\end{fmffile}
\end{center}
\strut

This corresponds to the propagator described by
\begin{equation*}
\frac{i}{q_{2}^2 - m_{A}^2c^2}.
\end{equation*}
We multiply this into our integrand now and we get
\begin{equation}
(-ig)^4 \frac{i}{q_{1}^2 - m_{C}^2c^2}\frac{i}{q_{2}^2 - m_{A}^2c^2} = -g^4\frac{1}{q_{1}^2 - m_{C}^2c^2}\frac{1}{q_{2}^2 - m_{A}^2c^2}.
\end{equation}
Similarly we can do likewise for the other part of the loop in
red:

\strut
\begin{center}
\begin{fmffile}{exTwoImg6}
  \begin{fmfgraph*}(70,25)  \fmfpen{0.1mm}

    \fmfleft{i1,o1} 
    \fmfright{i2,o2} 
    \fmflabel{$A,p_{1}$}{i1}
    \fmflabel{$B,p_{3}$}{o1} %
    \fmflabel{$A,p_{2}$}{i2} %
    \fmflabel{$B,p_{3}$}{o2}
    \fmf{plain}{i1,v1,o1} %
    \fmf{plain}{i2,v4,o2} %
    \fmf{plain,label=$C,,q_1$}{v1,v2}
    \fmf{plain,left,label=$B,,q_3$,tension=0.5,fore=red}{v2,v3}
    \fmf{plain,left,label=$A,,q_2$,tension=0.5}{v3,v2}
    \fmf{plain,label=$C,,q_4$}{v3,v4}
  \end{fmfgraph*}
\end{fmffile}
\end{center}
\strut

This corresponds to the propagator
\begin{equation*}
\frac{i}{q_{3}^2 - m_{B}^2c^2}
\end{equation*}
and we just multiply it into the integrand, which becomes
\begin{equation}
-g^4\frac{1}{q_{1}^2 - m_{C}^2c^2}\frac{1}{q_{2}^2 - m_{A}^2c^2}\frac{i}{q_{3}^2 - m_{B}^2c^2}.
\end{equation}
We have one last internal line left! We highlight it in red:


\strut
\begin{center}
\begin{fmffile}{exTwoImg7}
  \begin{fmfgraph*}(70,25)  \fmfpen{0.1mm}

    \fmfleft{i1,o1} 
    \fmfright{i2,o2} 
    \fmflabel{$A,p_{1}$}{i1}
    \fmflabel{$B,p_{3}$}{o1} %
    \fmflabel{$A,p_{2}$}{i2} %
    \fmflabel{$B,p_{3}$}{o2}
    \fmf{plain}{i1,v1,o1} %
    \fmf{plain}{i2,v4,o2} %
    \fmf{plain,label=$C,,q_1$}{v1,v2}
    \fmf{plain,left,label=$B,,q_3$,tension=0.5}{v2,v3}
    \fmf{plain,left,label=$A,,q_2$,tension=0.5}{v3,v2}
    \fmf{plain,label=$C,,q_4$,fore=red}{v3,v4}
  \end{fmfgraph*}
\end{fmffile}
\end{center}
\strut

This corresponds to the propagator
\begin{equation*}
\frac{i}{q_{4}^2-m_{C}^2c^2}
\end{equation*}
and we multiply it into the integrand, which becomes
\begin{equation}
-g^4\frac{1}{q_{1}^2 - m_{C}^2c^2}\frac{1}{q_{2}^2 - m_{A}^2c^2}\frac{i}{q_{3}^2 - m_{B}^2c^2}\frac{i}{q_{4}^2-m_{C}^2c^2} = g^4\frac{1}{q_{1}^2 - m_{C}^2c^2}\frac{1}{q_{2}^2 - m_{A}^2c^2}\frac{1}{q_{3}^2 - m_{B}^2c^2}\frac{1}{q_{4}^2-m_{C}^2c^2}.
\end{equation}

\textbf{Step Four:} We need to enfore the conservation of momentum, so what do
we do? We simply go back to our graph and go one vertex at a time and enforce
conservation of momentum. At the first vertex, the input momentum is in red
and the output momentum is in blue:


\strut
\begin{center}
\begin{fmffile}{exTwoImg8}
  \begin{fmfgraph*}(70,25)  \fmfpen{0.1mm}

    \fmfleft{i1,o1} 
    \fmfright{i2,o2} 
    \fmflabel{$A,p_{1}$}{i1}
    \fmflabel{$B,p_{3}$}{o1} %
    \fmflabel{$A,p_{2}$}{i2} %
    \fmflabel{$B,p_{3}$}{o2}
    \fmf{plain,fore=red}{i1,v1} %
    \fmf{plain,fore=blue}{v1,o1}
    \fmf{plain}{i2,v4,o2} %
    \fmf{plain,label=$C,,q_1$,fore=blue}{v1,v2}
    \fmf{plain,left,label=$B,,q_3$,tension=0.5}{v2,v3}
    \fmf{plain,left,label=$A,,q_2$,tension=0.5}{v3,v2}
    \fmf{plain,label=$C,,q_4$}{v3,v4}
  \end{fmfgraph*}
\end{fmffile}
\end{center}
\strut

So we want to have momentum here conserved, i.e.
\begin{equation}
p_1 = p_3 + q_1
\end{equation}
so we multiply the integrand by the term
\begin{equation*}
(2\pi)^{4}\delta^{(4)}(p_{1}-p_{3}-q_{1}).
\end{equation*}
Our integrand, which is ever expanding, is then
\begin{equation}
g^4\frac{1}{q_{1}^2 - m_{C}^2c^2}\frac{1}{q_{2}^2 - m_{A}^2c^2}\frac{1}{q_{3}^2 - m_{B}^2c^2}\frac{1}{q_{4}^2-m_{C}^2c^2}(2\pi)^{4}\delta^{(4)}(p_{1}-p_{3}-q_{1}).
\end{equation}
There are three other vertices which we must impose conservation laws on, so we
will move right along to the next vertex; again the input momentum is in red,
and the output momentum is in blue:


\strut
\begin{center}
\begin{fmffile}{exTwoImg9}
  \begin{fmfgraph*}(70,25)  \fmfpen{0.1mm}

    \fmfleft{i1,o1} 
    \fmfright{i2,o2} 
    \fmflabel{$A,p_{1}$}{i1}
    \fmflabel{$B,p_{3}$}{o1} %
    \fmflabel{$A,p_{2}$}{i2} %
    \fmflabel{$B,p_{3}$}{o2}
    \fmf{plain}{i1,v1} %
    \fmf{plain}{v1,o1}
    \fmf{plain}{i2,v4,o2} %
    \fmf{plain,label=$C,,q_1$,fore=red}{v1,v2}
    \fmf{plain,left,label=$B,,q_3$,tension=0.5,fore=blue}{v2,v3}
    \fmf{plain,left,label=$A,,q_2$,tension=0.5,fore=blue}{v3,v2}
    \fmf{plain,label=$C,,q_4$}{v3,v4}
  \end{fmfgraph*}
\end{fmffile}
\end{center}
\strut

This corresponds to a conservation of momentum of 
\begin{equation}
q_1 \approx q_2 + q_3
\end{equation}
which corresponds to the dirac delta function term of
\begin{equation*}
(2\pi)^4\delta^{(4)}(q_1 - q_2 - q_3).
\end{equation*}
Multiplying this into our integrand, we get
\begin{equation}
(2\pi)^8g^4\frac{1}{q_{1}^2 - m_{C}^2c^2}\frac{1}{q_{2}^2 - m_{A}^2c^2}\frac{1}{q_{3}^2 - m_{B}^2c^2}\frac{1}{q_{4}^2-m_{C}^2c^2}\delta^{(4)}(p_{1}-p_{3}-q_{1})\delta^{(4)}(q_1 - q_2 - q_3).
\end{equation}
Two vertices down, two to go! We simply move right along to find the next
conservation of momentum to be at the next vertex. The input momentums are in
red, and the output momentum is in blue:


\strut
\begin{center}
\begin{fmffile}{exTwoImg10}
  \begin{fmfgraph*}(70,25)  \fmfpen{0.1mm}

    \fmfleft{i1,o1} 
    \fmfright{i2,o2} 
    \fmflabel{$A,p_{1}$}{i1}
    \fmflabel{$B,p_{3}$}{o1} %
    \fmflabel{$A,p_{2}$}{i2} %
    \fmflabel{$B,p_{3}$}{o2}
    \fmf{plain}{i1,v1} %
    \fmf{plain}{v1,o1}
    \fmf{plain}{i2,v4,o2} %
    \fmf{plain,label=$C,,q_1$}{v1,v2}
    \fmf{plain,left,label=$B,,q_3$,tension=0.5,fore=red}{v2,v3}
    \fmf{plain,left,label=$A,,q_2$,tension=0.5,fore=red}{v3,v2}
    \fmf{plain,label=$C,,q_4$,fore=blue}{v3,v4}
  \end{fmfgraph*}
\end{fmffile}
\end{center}
\strut

This corresponds to the conservation
\begin{equation}
q_2 + q_3 \approx q_4
\end{equation}
which means we have a delta function of the form
\begin{equation*}
(2\pi)^4\delta^{(4)}(q_2 + q_3 - q_4).
\end{equation*}
Our integrand becomes
\begin{equation}
(2\pi)^{12}g^4\frac{1}{q_{1}^2 - m_{C}^2c^2}\frac{1}{q_{2}^2 - m_{A}^2c^2}\frac{1}{q_{3}^2 - m_{B}^2c^2}\frac{1}{q_{4}^2-m_{C}^2c^2}\delta^{(4)}(p_{1}-p_{3}-q_{1})\delta^{(4)}(q_1 - q_2 - q_3)\delta^{(4)}(q_2 + q_3 - q_4).
\end{equation}
\snote{As we can see, this is getting really really messy! God help us when
we try to feebly evaluate this beast!} Thank god only one vertex left! This
is the last term to add prior to integration. The input momentums are in
red and the output momentum is in blue:


\strut
\begin{center}
\begin{fmffile}{exTwoImg10}
  \begin{fmfgraph*}(70,25)  \fmfpen{0.1mm}

    \fmfleft{i1,o1} 
    \fmfright{i2,o2} 
    \fmflabel{$A,p_{1}$}{i1}
    \fmflabel{$B,p_{3}$}{o1} %
    \fmflabel{$A,p_{2}$}{i2} %
    \fmflabel{$B,p_{3}$}{o2}
    \fmf{plain}{i1,v1} %
    \fmf{plain}{v1,o1}
    \fmf{plain,fore=red}{i2,v4} %
    \fmf{plain,fore=blue}{v4,o2}
    \fmf{plain,label=$C,,q_1$}{v1,v2}
    \fmf{plain,left,label=$B,,q_3$,tension=0.5}{v2,v3}
    \fmf{plain,left,label=$A,,q_2$,tension=0.5}{v3,v2}
    \fmf{plain,label=$C,,q_4$,fore=red}{v3,v4}
  \end{fmfgraph*}
\end{fmffile}
\end{center}
\strut


This has the conservation of
\begin{equation}
q_4 + p_2\approx p_4
\end{equation}
which takes the delta form of
\begin{equation}
(2\pi)^4\delta^{(4)}(q_4 + p_2 - p_4)
\end{equation}
and our integrand finally becomes
\begin{equation}
(2\pi)^{16}g^4\frac{\delta^{(4)}(p_{1}-p_{3}-q_{1})}{q_{1}^2 - m_{C}^2c^2}\frac{\delta^{(4)}(q_1 - q_2 - q_3)}{q_{2}^2 - m_{A}^2c^2}\frac{\delta^{(4)}(q_2 + q_3 - q_4)}{q_{3}^2 - m_{B}^2c^2}\frac{\delta^{(4)}(q_4 + p_2 - p_4)}{q_{4}^2-m_{C}^2c^2} 
\end{equation}
At this point, if I were a professor, I would say ``This is trivial...bye!'' But
I am no professor!

\textbf{Step Five:} We then integrate over the internal lines. This is the fun 
part, like when the dentist says he needs to give you a root canal and he's
all outta Novocaine! We integrate over $q_1$, $q_2$, $q_3$, $q_4$. We notice
that the factors of $2\pi$ completely cancel out which is nice, so all we have
is
\begin{equation}
g^4\int\frac{\delta^{(4)}(p_{1}-p_{3}-q_{1})}{q_{1}^2 - m_{C}^2c^2}\frac{\delta^{(4)}(q_1 - q_2 - q_3)}{q_{2}^2 - m_{A}^2c^2}\frac{\delta^{(4)}(q_2 + q_3 - q_4)}{q_{3}^2 - m_{B}^2c^2}\frac{\delta^{(4)}(q_4 + p_2 - p_4)}{q_{4}^2-m_{C}^2c^2}d^{4}q_{1}d^{4}q_{2}d^{4}q_{3}d^{4}q_{4}.
\end{equation}
We will take this slow and step by step, we see that in the first delta function
we have the replacement of $q_1$ by $p_1-p_3$ which is nice! We make this move:
\begin{equation}
\frac{g^4}{(p_1-p_3)^2 - m_{C}^2c^2}\int\frac{\delta^{(4)}(p_1-p_3 - q_2 - q_3)}{q_{2}^2 - m_{A}^2c^2}\frac{\delta^{(4)}(q_2 + q_3 - q_4)}{q_{3}^2 - m_{B}^2c^2}\frac{\delta^{(4)}(q_4 + p_2 - p_4)}{q_{4}^2-m_{C}^2c^2}d^{4}q_{2}d^{4}q_{3}d^{4}q_{4}.
\end{equation}
Similarly, we find the our last delta function allows us to make the switcheroo
of $q_4$ for $p_2 - p_4$, so we make it so:
\begin{equation}
\frac{g^4}{(p_1-p_3)^2 - m_{C}^2c^2}\frac{1}{(p_2 - p_4)^2-m_{C}^2c^2}\int\frac{\delta^{(4)}(p_1-p_3 - q_2 - q_3)}{q_{2}^2 - m_{A}^2c^2}\frac{\delta^{(4)}(q_2 + q_3 - p_2 - p_4)}{q_{3}^2 - m_{B}^2c^2}d^{4}q_{2}d^{4}q_{3}.
\end{equation}
The term $\delta^{(4)}(p_1-p_3 - q_2 - q_3)$ tells us $q_2$ is replaced by
$p_1 - p_3 - q_3$ so our last delta function (after monkeying around with
integration) becomes
\begin{equation*}
\delta^{(4)}(p_1 + p_2 - p_3 - p_4).
\end{equation*}
We are left with
\begin{eqnarray}
\frac{g^4}{(p_1-p_3)^2 - m_{C}^2c^2}\frac{1}{(p_2 - p_4)^2-m_{C}^2c^2} \nonumber \\
\quad\times\int
\frac{1}{(p_1 - p_3 - q_3)^2 - m_{A}^2c^2}
\frac{\delta^{(4)}(p_1 - p_3 + p_2 - p_4)}{q_{3}^2 - m_{B}^2c^2}d^{4}q_{3}.
\end{eqnarray}
So we can skip ahead to rule 6 and assert: our contribution to the probability
amplitude from this diagram is
\begin{eqnarray}\label{divergentSonOfAGun}
\mathcal{M} &=& i\left(\frac{g}{2\pi}\right)^4\frac{1}{[(p_1-p_3)^2 - m_{C}^2c^2][(p_2 - p_4)^2-m_{C}^2c^2]} \nonumber\\
&&\times \int\frac{1}{[(p_1 - p_3 - q_3)^2 - m_{A}^2c^2][q_{3}^2 - m_{B}^2c^2]}d^{4}q_{3}.
\end{eqnarray}
 % done
\input{renormalization} % done
\section*{Before Going to QED...}

Before we can go ahead to Quantum Electrodynamics, we need to first introduce
(or in some cases, review) the Dirac equation. We will proceed to do that now...
For a more thorough treatment, see Dyson~\cite{Dyson:2006cp}. Note for the most
part, the inspiration of this entire article can be found in Griffiths~\cite{griffiths}.
It is a good introductory text on general particle physics too.

\section{Klein-Gordon Review}

Recall that the Schrodinger equation for the free nonrelativistic particle is
\begin{equation}
\frac{-\hbar^2}{2m}\nabla^2 \ket{\psi} = -i\hbar\partial_{t}\ket{\psi}
\end{equation}
which corresponds to a sort of quantized Newton's second law
\begin{equation}
\frac{p^2}{2m} \approx E.
\end{equation}
However, in special relativity we have the mass shell constraint
\begin{equation}\label{massShell}
p^\mu p_\mu = E^2 - \bold{p}\cdot\bold{p} = m^2
\end{equation}
(when $c=1$) using Einstein summation convention. If we naively quantize this, we end up with
\begin{equation}\label{KGeqn}
\partial_{\mu}\partial^{\mu} - m^2 \ket{\psi} = 0
\end{equation}
by moving the mass term onto the left hand side. This is the Klein-Gordon equation, it is plagued by problems such as negative probabilities, etc.


%%%%%%%%%%%%%%%%%%%%%%%%%%%%%%%%%%%%%%%%%%%%%%%%%%%%%%%%%%%%%%%%%%%%%%%%%%

\section{Dirac takes it up a notch...bam!}

Naively, we want something simpler than this. We can rewrite Eq (\ref{massShell}) to be
\begin{equation}
E^2 = \bold{p}\cdot\bold{p} + m^2\Rightarrow E = \sqrt{\bold{p}\cdot\bold{p} + m^2}
\end{equation}
then quantize it to be
\begin{equation}
i\hbar\frac{\partial}{\partial t}\ket{\psi} = \sqrt{-\hbar^2\nabla^2 + m^2}\ket{\psi}.
\end{equation}
We end up being forced to use pseudo-differential operators, unfortunately, and it turns out that this results in nonlocality\footnote{In general, whenever there is a squareroot quantity, there is nonlocality.}. For further details see Laemmerzahl~\cite{pseudodifferentialKG}.

The approach Dirac takes is basically taking the squareroot of the operator, but he does it with class. He uses a clifford algebra with generators $\gamma^\mu$\footnote{If the reader is unfamiliar with the Gamma Matrices, see the appendix A and/or CORE~\cite{Borodulin:1995xd}.} such that the squareroot of the Klein Gordon equation breaks into two equations:
\begin{equation}\label{diracEqn}
(i\hbar\gamma^\mu\partial_\mu - m)\psi(x) = 0
\end{equation}
where $\psi(x)$ is a spinor wave function with 4 components. The adjoint field $\bar{\psi}(x)$ is defined by 
\begin{equation}
\bar{\psi}(x) = \psi^{\dag}(x)\gamma^0
\end{equation}
and satisfies the \emph{adjoint} Dirac equation
\begin{equation}\label{adjointEqn}
\bar{\psi}(x)(i\hbar\gamma^\mu\partial_\mu + m) = 0.
\end{equation}
It is to be understood here the differential operator $\partial^\mu$ acts on the left. Observe that when we multiply the two operators together we get
\begin{equation}
(i\hbar\gamma^\mu\partial_\mu + m)(i\hbar\gamma^\mu\partial_\mu - m) = -\hbar^{2}\left(\gamma^\mu\right)^2\partial_{\mu}\partial^{\mu} - m^2 = \hat{p}^\mu\hat{p}_\mu - m^2 
\end{equation}
which is \emph{precisely} the Klein-Gordon operator (\ref{KGeqn})! We should be content now with the connection back to what we already know. \notetoself{And an added advantage is that the Dirac equation is a first order partial differential equation, whereas the Klein-Gordon equation is a second order one!}

%%%%%%%%%%%%%%%%%%%%%%%%%%%%%%%%%%%%%%%%%%%%%%%%%%%%%%%%%%%%%%%%%%%%%%%%%%

\subsection{A Somewhat Rigorous Derivation of the Dirac Equation}

We want the squareroot of the wave operator thus
\begin{equation}
\nabla^2 - \frac{1}{c^2}\frac{\partial^2}{\partial t^2} = (A \partial_x + B \partial_y + C \partial_z + \frac{i}{c}D \partial_t)(A \partial_x + B \partial_y + C \partial_z + \frac{i}{c}D \partial_t)
\end{equation}
We see multiplying out the right hand side all the cross-terms must vanish. To have this we want
\begin{equation}
AB + BA = 0, 
\end{equation}
and so on for all cross-term coefficients, with the property that
\begin{equation}
A^2 = B^2 = C^2 = D^2 = 1.
\end{equation}
Dirac had previously worked out rigorous results with Heisenberg's matrix mechanics, and concluded that these conditions could be met if $A, B,\ldots$ were \emph{matrices} which has the implication that the wave function has \emph{multiple components}. 

In the mean time, Pauli had been working on quantum mechanics as well. Pauli had a model with two-component wave functions that was involved in a phenomenological theory of spin. At this point in time, spin was not well understood. 

Given the factorization of these matrices, one can now write down immediately an equation
\begin{equation}
(A\partial_x + B\partial_y + C\partial_z + \frac{i}{c}D\partial_t)\psi = \kappa\psi
\end{equation}
with $\kappa$ to be determined. Applying the same operation on either side yields
\begin{equation}
(\nabla^2 - \frac{1}{c^2}\partial_{t}^2)\psi = \kappa^2\psi.
\end{equation}
If one take $\kappa = mc/\hbar$ we find that all the components of the wave function \emph{individually} satisfy the mass-shell relation (\ref{massShell}). Thus we have a first order differential equation in both space and time described by
\begin{equation}
(A\partial_x + B\partial_y + C\partial_z + \frac{i}{c}D\partial_t - \frac{mc}{\hbar})\psi = 0
\end{equation}
where $(A,B,C)=i\beta\alpha_k$ and $D=\beta$, which is precisely the Dirac equation for a spin-1/2 particle of rest mass $m$.


%%%%%%%%%%%%%%%%%%%%%%%%%%%%%%%%%%%%%%%%%%%%%%%%%%%%%%%%%%%%%%%%%%%%%%%%%%

\subsection{A Comparison to the Pauli Theory}

The necessity of introducing half-integer spin goes back experimentally to the results of the Stern-Gerlach experiment. \notetoself{A beam of atoms is run through a strong inhomogeneous magnetic field, which then splits into N parts depending on the intrinsic angular momentum of the atoms. It was found that for silver atoms, the beam was split in two - the ground state therefore could not be integral, because even if the intrinsic angular momentum of the atoms were as small as possible, 1, the beam would be split into 3 parts, corresponding to atoms with $L_z = -1, 0, +1$. The conclusion is that silver atoms have net intrinsic angular momentum of 1/2.} Pauli set up a model which explained the splitting by introducing a two-component wave-function and a corresponding correction term in the Hamiltonian(representing a semiclassical coupling of this wave function to an applied magnetic field) as
\begin{equation}
H = \frac{1}{2m}(\sigma^{I}_{i}(p^{i} - \frac{e}{c}A^{i})\sigma_{Ij}(p^{j} - \frac{e}{c}A^{j})) + e A^0 \qquad (i,j,I=1,2,3).
\end{equation}
We have here $A^\mu$ is the magnetic potential, and the van Warden symbols $\sigma^{J}_{j}$ which translates a vector into the Pauli matrix basis (if one is unfamiliar with Pauli matrices, see \S\ref{Representations of the Gamma Matrices}), $e$ is the electric charge of the particle (here $e=-e_0$ for the electron), and $m$ is the mass of the particle. Now we have just described the Hamiltonian of our system by a 2 by 2 matrix. The Schrodinger equation based on it
\begin{equation}
H \phi = i\hbar \frac{\partial\phi}{\partial t}
\end{equation}
must use a two-component wave function. (If you're like me you're too lazy to flip to the appendix, so I'll reproduce some of it here)\marginpar{SU(2) is the set of all 2 by 2 matrices that is self-adjoint and has a determinant of 1} Pauli used the SU(2) matrices
\begin{equation}\label{Pauli}
\sigma_k = \begin{bmatrix} 0 & 1 \\ 1 & 0 \end{bmatrix},\begin{bmatrix} 0 & -i \\ i & 0 \end{bmatrix},\begin{bmatrix} 1 & 0 \\ 0 & -1 \end{bmatrix}
\end{equation}
due to \emph{phenomenological reasons} (explaining the Gerlach experiment). Dirac now has a \emph{theoretical argument} that implies spin is a \emph{consequence} of introducing special relativity into quantum theory.

The Pauli matrices share the same properties as the Dirac matrices -- they are all self-adjoint, when squared are equal to the identity, and they anticommute. We can now use the Pauli matrices Eq (\ref{Pauli}) to describe a representation of the Dirac matrices:
\begin{equation}
\alpha_k = \begin{bmatrix} 0 & \sigma_k \\ \sigma_k & 0 \end{bmatrix}\qquad \beta = \begin{bmatrix} 1_2 & 0 \\ 0 & -1_2 \end{bmatrix}.
\end{equation}
We now may write the Dirac equation as an equation coupling two-component spinors:
\begin{equation}
\begin{bmatrix} mc^2 & c\sigma\cdot p \\ c\sigma\cdot p & -mc^2 \end{bmatrix} \begin{bmatrix} \phi_+ \\ \phi_- \end{bmatrix} = i\hbar\frac{\partial}{\partial t}\begin{bmatrix} \phi_+ \\ \phi_- \end{bmatrix}.
\end{equation}
Observe that we have on the diagonal the rest mass. If we bring the particle to rest, we have
\begin{equation}
i\hbar\frac{\partial}{\partial t}\begin{pmatrix} \phi_+ \\ \phi_- \end{pmatrix} = \begin{pmatrix} mc^2 & 0 \\ 0 & -mc^2 \end{pmatrix} \begin{pmatrix} \phi_+ \\ \phi_- \end{pmatrix}.
\end{equation}
The equations for the individual two-spinors are now decoupled, and we see that the ``spin-up'' and ``spin-down'' (or ``right-handed'' and ``left-handed'', ``positive frequency'' and ``negative frequency'' respectively) are individual eigenfunctions \snote{Eigenspinors?} of the energy with eigenvalues equal to $\pm$ the rest energy. The appearence of \emph{negative} energy should not be alarming, it is completely consistent with relativity.

\textbf{Note} that this seperation is in the rest frame and \textbf{is not an invariant statement} -- the bottom component does not generally represent antimatter. The \emph{entire} four-component spinor represents an \emph{irreducible whole} -- in general states will have an admixture of positive \emph{and} negative energy components.

\subsection{Covariant Form and Relativistic Invariance}

The explicity covariant form of the Dirac Equation is (using Einstein summation convention)
\begin{equation}
i\hbar\gamma^\mu\partial_\mu\psi - mc\psi = 0,
\end{equation}
where $\gamma^\mu$ are the Dirac gamma matrices. We have
\begin{equation}
\gamma^0 = \beta \qquad \gamma^k = \gamma^0\alpha_k.
\end{equation}
See the appendix for more details on this representation.

The Dirac equation may be interpreted as an eigenvalue expression, where the rest mass is proportional to an eigenvalue of the 4-momentum operator, the proportion being $c$:
\begin{equation}
\hat{P}\psi = mc\psi.
\end{equation}
In practice we often work in units where we set $\hbar$ and $c$ to be 1. The equation is multiplied by $-i$ and takes the form
\begin{equation}
\left(\gamma^\mu\partial_\mu + im \right)\psi = 0.
\end{equation}
We may employ the Feynman slash notation to simplify this to
\begin{equation}
(\slashed{\partial} + im)\psi = 0. 
\end{equation}
For any two representations of the Dirac Gamma matrices, they are related by a unitary transformation. Likewise, the solutions in the two representations are related by the same way.

%%%%%%%%%%%%%%%%%%%%%%%%%%%%%%%%%%%%%%%%%%%%%%%%%%%%%%%%%%%%%%%%%%%%%%%%%

\subsection{Conservation Laws and Canonical Structure}

Recall the Dirac equation and its adjoint version, Eqns (\ref{diracEqn}) and (\ref{adjointEqn}). We notice from the definition of the adjoint
\begin{equation*}
\bar{\psi} = \psi^\dag\gamma^0
\end{equation*}
that
\begin{equation}
\left(\gamma^\mu\right)^\dag\gamma^0 = \gamma^0\gamma^\mu
\end{equation}
we can obtain the Hermitian conjugate of the Dirac equation and multiplying from the right by $\gamma^0$  we get its adjoint version:
\begin{equation*}
\bar{\psi}(\gamma^\mu\overleftarrow{\partial}_\mu - im) = 0
\end{equation*} 
where $\overleftarrow{\partial}_\mu$ acts on the left. When we multiply the Dirac equation by $\bar{\psi}$ from the left
\begin{equation}
\bar{\psi}(\gamma^\mu\overrightarrow{\partial}_\mu + im)\psi = 0
\end{equation}
(where $\overrightarrow{\partial}_\mu$ acts on the right) and multiply the adjoint equation by $\psi$ on the right
\begin{equation}
\bar{\psi}(\gamma^\mu\overleftarrow{\partial}_\mu - im)\psi = 0
\end{equation}
then add the two together we get\marginpar{Conservation Law of Dirac Current}
\begin{equation}
\bar{\psi}(\gamma^\mu\overrightarrow{\partial}_\mu + im)\psi + \bar{\psi}(\gamma^\mu\overleftarrow{\partial}_\mu - im)\psi = \partial(\bar{\psi}\gamma^\mu\psi) = \partial_\mu J^\mu 0
\end{equation}
(where $J^\mu$ is the Dirac Current) which is the law of conservation of the Dirac current in covariant form. We see the huge advantage this has over the Klein-Gordon equation: this has conserved probability current desnity as required by relativistic invariance...only now its temporal component is \emph{positive definite}:
\begin{equation}
J^0 = \bar{\psi}\gamma^0\psi = \bar{\psi}\psi.
\end{equation}
From this we can find a conserved charge
\begin{equation}
Q = q\int \psi^\dag(\bold{x})\psi(\bold{x})d^3x
\end{equation}
where $q$ is to be thought of as ``charge''.

We can now see that the Dirac equation (and its adjoint) are the Euler-Lagrange equations of motion of the four dimensional invariant action
\begin{equation}
S = \int \mathcal{L}d^4x
\end{equation}
where the Dirac Lagrangian density $\mathcal{L}$ is given by
\begin{equation}
\mathcal{L} = c\bar{\psi}(x)\Big[ i\hbar\gamma^\mu\partial_\mu - mc \Big]\psi(x)
\end{equation}
and for purposes of variation, $\psi$ and $\bar{\psi}$ are considered to be independent fields. Relativistic invariance follows from the variational principle.

\marginpar{Canonical Structure, Hamiltonian and Momentum operators}We can find the canonically conjugate momenta to the fields $\psi$ and $\bar{\psi}$:
\begin{equation}
\pi(x) = \frac{\partial\mathcal{L}}{\partial\dot{\psi}} = i\hbar\psi^\dag\qquad\bar{\pi}(x) = \frac{\partial\mathcal{L}}{\partial\dot{\bar{\psi}}} = 0.
\end{equation}
We can find the Hamiltonian of the Dirac field
\begin{equation}\label{Hamiltonian}
H = \int d^{3}x( \pi_{\alpha}(x)\dot{\psi}^{\alpha}(x) - \mathcal{L} ) = \int d^{3}x \bar{\psi}(x)[-i\hbar c\gamma^{j}\partial_{j} + mc]\psi(x).
\end{equation}
Similarly, the momentum of some field $\phi$ to be given by
\begin{equation*}
cP^\alpha \equiv \int d^3x\mathcal{T}^{0\alpha} = \int d^{3}x\left[c\pi_r(x)\frac{\partial\phi_r(x)}{\partial x_{\alpha}} - \mathcal{L}\eta^{0\alpha}\right]
\end{equation*}
where $\mathcal{T}^{\alpha\beta}$ is the stress-energy density tensor of the field $\phi$. Recall that we define the stress-energy density tensor by the equation
\begin{equation}
\mathcal{T}^{\alpha\beta}\equiv \frac{\partial\mathcal{L}}{\partial\phi_{r,\alpha}}\frac{\partial\phi_r}{\partial x_\beta} - \mathcal{L}\eta^{\alpha\beta}.
\end{equation}
Using this, we can find the momentum of the Dirac Field to be
\begin{equation}
\bold{P} = -i\hbar\int d^3x\psi^\dag(x)\nabla\psi(x).
\end{equation}
Of course, the Hamiltonian given by (\ref{Hamiltonian}) could have been discovered by finding the Hamiltonian density applied to the current case.

We\marginpar{Angular Momentum} can similarly find the angular momentum of the Dirac Field by simply following the scheme of finding the angular momentum for a general field. That is, an infinitesmal transformation of the coordinates
\begin{equation}
x_\alpha\to x'_\alpha\equiv x_\alpha + \delta x_\alpha = x_\alpha + \varepsilon_{\alpha\beta}x^\beta + \delta_\alpha
\end{equation}
(where $\delta_\alpha$ is an infinitesmal displacement and $\varepsilon_{\alpha\beta}$ is an infinitesmal antisymmetric tensor to ensure invariance of $x_\alpha x^\alpha$ under homogeneous Lorentz transformations, i.e. ones with $\delta_\alpha=0$) induces an infinitesmal transformation of the field $\phi$:
\begin{equation}
\phi_r(x)\to\phi'_r(x') = \phi_r(x) + \frac{1}{2}\varepsilon_{\alpha\beta}S^{\alpha\beta}_{rs}\phi_{s}(x).
\end{equation}
Here the coefficients $S^{\alpha\beta}_{rs}$ are antisymmetric in $\alpha$ and $\beta$, like $\varepsilon_{\alpha\beta}$, and are determined by the transformation properties of the fields.

For a rotation (i.e. $\delta_\alpha=0$) we have the continuity equation
\begin{equation}
\frac{\partial\mathcal{M}^{\alpha\beta\gamma}}{\partial x^{\alpha}} = 0
\end{equation}
where
\begin{equation}
\mathcal{M}^{\alpha\beta\gamma}\equiv\frac{\partial\mathcal{L}}{\partial\phi_{r,\alpha}}S^{\beta\gamma}_{rs}\phi_{s}(x) + [x^{\beta}\mathcal{T}^{\alpha\gamma} - x^{\gamma}\mathcal{T}^{\alpha\beta}],
\end{equation}
(note that $\mathcal{M}^{\alpha\beta\gamma}=-\mathcal{M}^{\alpha\gamma\beta}$) and the six conserved quantities are\marginpar{We interpret $M^{\alpha\beta}$ as angular momentum}
\begin{eqnarray}
cM^{\alpha\beta} &=& \int d^3x \mathcal{M}^{0\alpha\beta} \nonumber\\
&=& \int d^3x \Big( [x^{\beta}\mathcal{T}^{0\alpha} - x^{\beta}\mathcal{T}^{0\alpha}] + c\pi_r(x)S^{\alpha\beta}_{rs}\phi_{s}(x) \Big).\label{angMom}
\end{eqnarray}
We have stated that $\mathcal{T}^{0i}/c$ is the momentum density of the field, so we interpret the square brackets of Eq (\ref{angMom}) as the orbital momentum, and the last term as the intrinsic spin angular momentum.

We can apply similar technqiues to the Dirac field. The transformation of the Dirac field under an infinitesmal Lorentz transformation is given by
\begin{equation}
\psi_{\alpha}\to\psi'_{\alpha}(x') = \psi_{\alpha}(x) - \frac{i}{4}\epsilon_{\mu\nu}\sigma^{\mu\nu}_{\alpha\beta}\psi_{\beta}(x),
\end{equation}
where summation over $\mu,\nu=0,\ldots,3$ and $\beta=1,\ldots,4$ is implied, and where $\sigma^{\mu\nu}_{\alpha\beta}$ is the $(\alpha,\beta)$ matrix element of the $4\times 4$ matrix
\begin{equation}
\sigma^{\mu\nu}\equiv\frac{i}{2}[\gamma^{\mu},\gamma^{\nu}].
\end{equation}
 We can now ``plug and chug'' to find the angular momentum of the Dirac field
\begin{equation}\label{diracAngMom}
\bold{M} = \int d^{3}x \psi^{\dag}(x)[\bold{x}\wedge(-i\hbar\nabla)]\psi(x) + \int d^{3}x\psi^{\dag}\left(\frac{\hbar}{2}\bold{\sigma}\right)\psi(x)
\end{equation}
where
\begin{equation}
\bold{\sigma} = (\sigma^{23},\sigma^{31},\sigma^{12})
\end{equation}
are 4$\times$4 matrices generalizing Pauli matrices. We also observe that Eq (\ref{diracAngMom}) represent the orbital and spin angular momentum of particles of spin 1/2.

\subsection{Solutions to the Dirac Equation}

The easiest approach to find solutions to the Dirac equation is to insist that
the solution is independent of spatial position:
\begin{equation}
\frac{\partial\psi}{\partial x} = \frac{\partial\psi}{\partial y} = \frac{\partial\psi}{\partial z} = 0.
\end{equation}
This really describes a particle with zero momentum, since the momentum operator
is $i\hbar\partial_\mu$ and all the spatial eigenvalues vanish. The Dirac 
equation simplifies to
\begin{equation}
\frac{i\hbar}{c}\gamma^0\frac{\partial\psi}{\partial t} - mc\psi = 0
\end{equation}
or equivalently
\begin{equation}
\begin{bmatrix}
1 & 0 \\
0 & -1
\end{bmatrix}
\begin{bmatrix}
\partial\psi_A/\partial t\\
\partial\psi_B/\partial t
\end{bmatrix}
= -i\frac{mc^2}{\hbar}
\begin{bmatrix}
\psi_A \\
\psi_B
\end{bmatrix}
\end{equation}
where
\begin{equation}
\psi_A = \begin{bmatrix}
\psi_1\\
\psi_2
\end{bmatrix}
\end{equation}
carries the upper two components and 
\begin{equation}
\psi_B = \begin{bmatrix}
\psi_3\\
\psi_4
\end{bmatrix}
\end{equation}
carries the lower two components. Thus
\begin{equation}
\frac{\partial\psi_A}{\partial t} = -i\left(\frac{mc^2}{\hbar}\right)\psi_A,\quad -\frac{\partial\psi_B}{\partial t} = -i\left(\frac{mc^2}{\hbar}\right)\psi_B
\end{equation}
and the solutions are
\begin{equation}
\psi_A(t) = \exp[-i(mc^2/\hbar)t]\psi_A(0),\quad\psi_B(t)=\exp[i(mc^2/\hbar)t]\psi_B(0).
\end{equation}
We should know that in Quatum mechanics, the term
\begin{equation}
\exp(-iEt/\hbar)
\end{equation}
is the characteristic for time dependence of a quantum state with energy $E$. It
follows that at rest with $\bold{p}=0$, the energy of the particle is $E=mc^2$.
So $\psi_A$ is what we expect.

What about $\psi_B$? It has negative energy! What the heck?! This is a famous
disaster, and Dirac's response was like the Hindenberg of physics. He suggested
something called the Hole theory, we will not discuss it here. 

We interpret these ``negative'' energy particles as \emph{antiparticles} with
\emph{positive} energy. So for us in our Dirac equation, $\psi_B$ describes
positrons (or antielectrons if one prefers to be outlandish) and $\psi_A$ 
describes electrons. Each of them is a 2 component spinor (a 2 column vector).
This is ideal as such a mathematical object describes a spin 1/2 particle. So,
to sum up, we have 2 particles that are each 2 solutions for a grand total of
4 independent solutions with momentum $\bold{p}=0$:
\begin{equation}
\psi^{(1)} = \exp(i(mc^2/\hbar)t)\begin{bmatrix}
1\\
0\\
0\\
0
\end{bmatrix}\quad\psi^{(2)} = \exp(i(mc^2/\hbar)t)\begin{bmatrix}
0\\
1\\
0\\
0
\end{bmatrix}
\end{equation}
\begin{equation}
\psi^{(3)} = \exp(-i(mc^2/\hbar)t)\begin{bmatrix}
0\\
0\\
1\\
0
\end{bmatrix}\quad\psi^{(4)} = \exp(-i(mc^2/\hbar)t)\begin{bmatrix}
0\\
0\\
0\\
1
\end{bmatrix}
\end{equation}
describing (respectively) an electron with spin up, an electron with spin
down, a positron with spin up and an electron with spin down.

So to look at this from the perspective of solving differential equations, we
have a solution to the homogeneous equation and we will use the method of variation
of parameters to get solutions to the Dirac equation. What does this mean? Well,
it means we are looking for ``plane wave solutions'' that look like
\begin{equation}
\psi(\bold{r},t) = ae^{-i(Et-\bold{p}\cdot\bold{r})/\hbar}u(E,\bold{p})
\end{equation}
where $a$ is a normalization constant (so probabilities add up to 1). We want to
solve for $u(E,\bold{p})=u(p)$ (we will use $p=(E/c,\bold{p})$ which is a 4 vector, and
similarly $x=(ct,\bold{x}$), which is a mathematical object called a ``bispinor''.
We don't want any old bispinor, we want one that will solve Dirac's equation!
We have $x$ dependence only in the exponent, so we find
\begin{equation}
\partial_\mu\psi = \frac{-i}{\hbar}p_\mu a e^{-(i/hbar)x^\mu p_\mu}u
\end{equation}
By plugging this into Dirac's equation, we get
\begin{equation}
\gamma^\mu p_\mu a e^{-(i/\hbar)x\cdot p}u - mcae^{-(i/\hbar)x\cdot p}u = 0
\end{equation}
or if we want a neater and cleaner way to present it
\begin{equation}\label{momentumSpaceDiracEquation}
(\gamma^\mu p_\mu - mc)u = 0.
\end{equation}
This is the ``momentum space Dirac equation'' (which we get by taking the
Fourier Transform of the Dirac equation we all know and love). Notice this is
purely algebraic, no derivatives! That's the beauty of Fourier transforms in
solving differential equations! If $u$ satisfies (\ref{momentumSpaceDiracEquation})
then $\psi$ satisfies the Dirac equation.

Now to \emph{prove} this (because an assertion is always meaningless without a
rigorous proof -- take note of this social ``scientists'') we need to use a lot
of gamma matrix manipulations. Remember all representations are ``equivalent''
in the sense that they are related by unitary transformations. First we have
\begin{equation}
\gamma^\mu p_\mu = \gamma^0 p^0 - \bold{\gamma}\cdot\bold{p} = \frac{E}{c}\begin{bmatrix}
1 & 0 \\
0 & -1
\end{bmatrix}
- \bold{p}\cdot\begin{bmatrix}
0 & \sigma \\
-\sigma & 0 
\end{bmatrix}
=
\begin{bmatrix}
E/c & -\bold{p}\cdot\bold{\sigma} \\
\bold{p}\cdot\bold{\sigma} & -E/c
\end{bmatrix}
\end{equation}
SO it follows that
\begin{eqnarray*}
(\gamma^\mu p_\mu - mc)u &=& \begin{bmatrix}
\left(\frac{E}{c}-mc\right) & -\bold{p}\cdot\sigma \\
\bold{p}\cdot\sigma & \left(\frac{-E}{c}-mc\right)
\end{bmatrix}
\begin{bmatrix}
u_A \\
u_B
\end{bmatrix} \\
&=& \begin{bmatrix}
\left(\frac{E}{c}-mc\right)u_A & -\bold{p}\cdot\sigma u_B\\
\bold{p}\cdot\sigma u_A & \left(\frac{-E}{c}-mc\right) u_B
\end{bmatrix}
\end{eqnarray*}
where the subscript $A$ is for the upper two components and the $B$ stands for
the lower two. In order to satisfy the momentum space Dirac equation, we
must have
\begin{equation}\label{stepTowardsSolutions}
u_A = \frac{c}{E - mc^2}(\bold{p}\cdot\sigma)u_B,\quad u_B = \frac{c}{E + mc^2}(\bold{p}\cdot\sigma) u_A
\end{equation}
We substitute the second into the first to give us
\begin{equation}
u_A =\frac{c^2}{E^2 - m^2c^4}(\bold{p}\cdot\sigma)^2 u_A
\end{equation}
Observe
\begin{eqnarray*}
\bold{p}\cdot\sigma &=& p_x\begin{bmatrix}
0 & 1\\
1 & 0 
\end{bmatrix}
+ p_{y}\begin{bmatrix}
0 & -i\\
i & 0
\end{bmatrix}
+ p_{z}\begin{bmatrix}
1 & 0 \\
0 & -1
\end{bmatrix} \\
&=& \begin{bmatrix}
p_{z} & (p_{x} - ip_{y}) \\
(p_{x} + ip_{y}) & -p_{z}
\end{bmatrix}
\end{eqnarray*}
We find then by matrix multiplication (we will not calculate this out with every
detail, but we will show the result):
\begin{equation}
(\bold{p}\cdot\sigma)^2 = \begin{bmatrix} p_{z}^2 + (p_x - ip_{y})(p_{x} + ip_{y}) & p_{z}(p_{x} - ip_{y}) - p_{z}(p_{x} - ip_{y}) \\
p_{z}(p_{x} + ip_{y}) - p_{z}(p_{x} + ip_{y}) & (p_{x} + ip_{y})(p_{x} - ip_{y}) + p_{z}^2
\end{bmatrix} = \bold{p}^2 I
\end{equation}
where $I$ is the 2 by 2 identity matrix. We see then that by plugging this into our equation for $u_A$
\begin{equation}
u_A = \frac{\bold{p}^2c^2}{E^2 - m^2c^4}u_A
\end{equation}
which can be rearranged to be
\begin{eqnarray*}
(E^2 - m^2c^4)u_A &=& \bold{p}^2c^2 u_A \\
\Rightarrow (E^2 - \bold{p}^2c^2)u_A &=& m^2c^4 u_A
\end{eqnarray*}
and thus
\begin{equation}
E^2 - \bold{p}^2c^2 = m^2c^4
\end{equation}
which is the famous Einstein equation we all know and love. This tells us that
in order to satisfy the Dirac equation, we have to obey the mass shell constraint.
This admits two solutions for $E$:
\begin{equation}
E = \pm\sqrt{m^2 c^4 + \bold{p}^2c^2}
\end{equation}
where the positive root is associated with particle states, and the negative
root with antiparticle states.

Using Eq (\ref{stepTowardsSolutions}), it is straightforward to calculate out
the solutions to the Dirac equation (ignoring normalization constants):
\begin{equation*}
\mbox{Pick } u_A = \begin{bmatrix} 1\\0\end{bmatrix}\quad\mbox{then } 
u_B = \frac{c}{E + mc^2}(\bold{p}\cdot\sigma)\begin{bmatrix}1\\0\end{bmatrix} 
= \frac{c}{E + mc^2}\begin{bmatrix} p_{z}\\ p_{x}+ip_{y}\end{bmatrix}
\end{equation*}
\begin{equation*}
\mbox{Pick } u_A = \begin{bmatrix} 0\\1\end{bmatrix}\quad\mbox{then } 
u_B = \frac{c}{E + mc^2}(\bold{p}\cdot\sigma)\begin{bmatrix}0\\1\end{bmatrix} 
= \frac{c}{E + mc^2}\begin{bmatrix} p_{x}-ip_{y}\\ -p_{z}\end{bmatrix}
\end{equation*}
\begin{equation*}
\mbox{Pick } u_B = \begin{bmatrix} 1\\0\end{bmatrix}\quad\mbox{then } 
u_A = \frac{c}{E - mc^2}(\bold{p}\cdot\sigma)\begin{bmatrix}1\\0\end{bmatrix} 
= \frac{c}{E - mc^2}\begin{bmatrix} p_{z}\\ p_{x}+ip_{y}\end{bmatrix}
\end{equation*}
\begin{equation*}
\mbox{Pick } u_B = \begin{bmatrix} 0\\1\end{bmatrix}\quad\mbox{then } 
u_A = \frac{c}{E - mc^2}(\bold{p}\cdot\sigma)\begin{bmatrix}0\\1\end{bmatrix} 
= \frac{c}{E - mc^2}\begin{bmatrix} p_{x}-ip_{y}\\-p_{z}\end{bmatrix}
\end{equation*}
For the first two of these, we must use the positive energy otherwise we have
division by zero, and if you divide by zero you go to hell. For the same reason,
the energy in the latter two are negative. It is convenient to ``normalize'' these
spinors in such a way that
\begin{equation}
u^\dag u = 2|E|/c
\end{equation}
where the dagger indicates the transpose conjugate (``Hermitian conjugate'') is
used:
\begin{equation*}
u = \begin{bmatrix}a\\b\\c\\d\end{bmatrix}\Rightarrow\quad u^\dag = (a^*, b^*, c^*, d^*)
\end{equation*}
so that
\begin{equation}
u^\dag u = |a|^2 + |b|^2 + |c|^2 + |d|^2.
\end{equation}
So we find that the four solutions are:
\begin{equation}
u^{(1)} = N\begin{bmatrix}1\\0\\\frac{\displaystyle cp_{z}}{\displaystyle E+mc^2}\\ \frac{\displaystyle c(p_{x}+ip_{y})}{\displaystyle E+mc^2}\end{bmatrix}
\end{equation}
\begin{equation}
u^{(2)} = N\begin{bmatrix}0\\1\\ \frac{\displaystyle c(p_{x}-ip_{y})}{\displaystyle E+mc^2}\\\frac{\displaystyle -cp_{z}}{\displaystyle E+mc^2}\end{bmatrix}
\end{equation}
with $E = +\sqrt{m^2c^4 + \bold{p}^2c^2}$
\begin{equation}
u^{(3)} = N\begin{bmatrix}\frac{\displaystyle cp_{z}}{\displaystyle E-mc^2}\\ \frac{\displaystyle c(p_{x}+ip_{y})}{\displaystyle E-mc^2}\\1\\0\end{bmatrix}
\end{equation}
\begin{equation}
u^{(4)} = N\begin{bmatrix} \frac{\displaystyle c(p_{x}-ip_{y})}{\displaystyle E-mc^2}\\\frac{\displaystyle c(-p_{z})}{\displaystyle E-mc^2}\\ 0\\1\end{bmatrix}
\end{equation}
with $E = -\sqrt{m^2c^4 + \bold{p}^2c^2}$, and the normalization constant is
\begin{equation}
N = \sqrt{(|E|+mc^2)/c}.
\end{equation}
Now we are really tempted to say that $u^{(1)}$ is an electron with spin up,
and $u^{(2)}$ is an electron with spin down, and so on, but this is not quite so.
For Dirac Particles, the spin matrices are
\begin{equation}
S = \frac{\hbar}{2}\Sigma\quad\mbox{with }\Sigma\equiv\begin{bmatrix}\sigma & 0\\0 & \sigma\end{bmatrix}
\end{equation}
and it's easy to check that $u^{(1)}$ is \emph{not} an eigenstate of $\Sigma$.
However, if we orient the $z$ axis so it points along the direction of motion
(in which case $p_x = p_y = 0$) then $u^{(1)}$, $u^{(2)}$, $u^{(3)}$, and $u^{(4)}$
are eigenspinors of $S_z$; $u^{(1)}$ and $u^{(3)}$ are spin up, and
$u^{(2)}$ and $u^{(4)}$ are spin down\footnote{It is actually mathematically
impossible to construct spinors that satisfies the momentum Dirac equation and
are simultaneously eigenspinors of $S_z$ (except for the special case $\bold{p}$ = $p_z\hat{z}$).
The reason is that $S$ by itself is \emph{not a conserved quantity.} Only the
\emph{total} angular momentum $L+S$ is conserved. It is possible to construct
eigenspinors of \emph{helicity}, $\Sigma\cdot\hat{p}$ (there's no \emph{orbital}
angular momentum about the direction of motion), but these are cumbersome and in
practice we like to work with the spinors we have constructed, even though it is
difficult to have a physical intuition to what they mean. In the end, all that
really matters is that we have a complete set of solutions.}

Now we have to discuss the importance of $E$ and $\bold{p}$, which are mathematical
parameters which correspond physically to energy and momentum. At least, this is
true for the electron states $u^{(1)}$ and $u^{(2)}$; but in $u^{(3)}$ and 
$u^{(4)}$ the $E<0$...so it \emph{cannot} represent positron energy. All free
particles -- electrons and positrons alike -- carry \emph{positive} energy.
The ``negative-energy'' solutions must be reinterpreted as \emph{positive}
energy \emph{antiparticle} states. To express these solutions in terms of 
the \emph{physical} energy and momentum of the positron, we flip the signs of
$E$ and $\bold{p}$:
\begin{equation}
\psi(\bold{r},t) = ae^{i/\hbar(Et - \bold{p}\cdot\bold{r})}u(-E,-\bold{p})
\end{equation}
for solutions (3) and (4) of course. These are the same solutions, we just have
changed the signs of two parameters so it is physically appealing. It is 
customary to use $v$ for positron states, expressed in terms of the physical
energy and momentum:
\begin{equation}
v^{(1)}(E,\bold{p}) = u^{(4)}(-E,-\bold{p}) = N\begin{bmatrix} 
\frac{\displaystyle c(p_x - ip_y)}{\displaystyle E + mc^2}\\
\frac{\displaystyle c(-p_z)}{\displaystyle E + mc^2}\\
0\\
1\end{bmatrix}
\end{equation}
\begin{equation}
v^{(2)}(E,\bold{p}) = u^{(4)}(-E,-\bold{p}) = N\begin{bmatrix} 
\frac{\displaystyle c(p_z)}{\displaystyle E + mc^2}\\
\frac{\displaystyle c(p_x + ip_y)}{\displaystyle E + mc^2}\\
1\\
0\end{bmatrix}
\end{equation}
(with $E=\sqrt{m^2c^4 + \bold{p}^2c^2}$).

So we will no longer be working with $u^{(3)}$ and $u^{(4)}$; instead, the
set of solutions we will be working with are $u^{(1)}$, $u^{(2)}$ (representing
the two spin states of an electron with energy $E$ and momentum $\bold{p}$),
and $v^{(1)}$, $v^{(2)}$ (representing the two spin states of a positron with
energy $E$ and momentum $\bold{p}$). Notice that whereas the $u$'s satisfy the
momentum space Dirac equation in the form
\begin{equation}
(\gamma^\mu p_\mu - mc)u = 0
\end{equation}
the $v$'s obey the equation with the sign of $p_\mu$ reversed:
\begin{equation}
(\gamma^\mu p_\mu + mc)v = 0.
\end{equation}
Sure this is interesting, but it's only the special case of plane waves. Why
bother? Well, they are of interest because they describe particles with 
specified energies and momenta, and in a typical experiment that's what we
control and measure.

 % done
\section{Some Notes on Spinor Technology}

It was mentioned that the Dirac spinor does not transform as a four-vector when
one changes from one inertial reference frame to another. So how exactly do they
transform? Well, it's quite a bit of work to do, but we will simply quote the
result. If we go to a system moving with speed $v$ in the $x$ direction, the
transformation rule is
\begin{equation}
\psi\to\psi' = S\psi
\end{equation}
where $S$ is the $4\times 4$ matrix
\begin{equation}
S = a_+ + a_- \gamma^0\gamma^1 = \begin{bmatrix} a_+ & a_-\sigma_1 \\
a_-\sigma_1 & a_+\end{bmatrix}
\end{equation}
with
\begin{equation}
a_{\pm} = \pm\sqrt{(\gamma\pm 1)/2}
\end{equation}
and $\gamma = 1/\sqrt{1-v^2/c^2}$ is the Lorentz factor as usual.

Suppose we want to construct a scalar quantity out of a spinor $\psi$ (we can
do this with vectors, it's just the dot product). It would be reasonable to
follow suite with the dot product and try the following:
\begin{equation}
\psi^\dag\psi = \begin{bmatrix}\psi^{*}_1 & \psi^{*}_2 & \psi^{*}_3 & \psi^{*}_4\end{bmatrix}
\begin{bmatrix}
\psi_1\\
\psi_2\\
\psi_3\\
\psi_4
\end{bmatrix} = |\psi_1|^2 + |\psi_2|^2 + |\psi_3|^2 + |\psi_4|^2.
\end{equation}
Unfortunately this doesn't quite work as well as we would like. We can illustrate
this by transforming coordinates:
\begin{equation}
(\psi^\dag\psi)' = (\psi')^\dag\psi' = \psi^\dag S^\dag S\psi\ne (\psi^\dag\psi)
\end{equation}
In fact
\begin{equation}
S^\dag S = S^2 = \gamma \begin{bmatrix} 1 & -v\sigma_1/c\\
-v\sigma_1/c & 1\end{bmatrix} \ne 1.
\end{equation}
Of course we shouldn't expect this to be invariant, with 4-vectors we have
(if we are particle physicists) the time component squared minus the sum of the
space components squared. We see now that we can introduce a notion of \emph{adjointness}, 
that is an adjoint spinor:
\begin{equation}
\bar{\psi} \equiv \psi^\dag\gamma^0 = \begin{bmatrix}\psi^*_1 & \psi^*_2 & -\psi^*_3 & -\psi^*_4\end{bmatrix}
\end{equation}
We can see that
\begin{equation}
\bar{\psi}\psi = \psi^\dag\gamma^0\psi = |\psi_1|^2 + |\psi_2|^2 - |\psi_3|^2 - |\psi_4|^2
\end{equation}
is a relativistic invariant. Why? Well, $S^\dag\gamma^0 S=\gamma^0$ so we avoid
the problems from our first attempt.
 % done
\subsection*{Please Take Note!}

We will be covering everything relevant here, and when the time comes we will
be performing in excrutiating detail every Feynman diagram of significance in
QED.

%%
%% em.tex
%% 
%% Made by Alex Nelson
%% Login   <alex@tomato>
%% 
%% Started on  Fri Sep 18 12:06:10 2009 Alex Nelson
%% Last update Fri Sep 18 12:06:10 2009 Alex Nelson
%%
%% TODO: the canonical formalism,

We begin our analysis with classical electromagnetism. Consider
Maxwell's equations
\begin{subequations}\label{eq:maxwell}
\begin{align}
\nabla\cdot\vec{E}&=\frac{\rho}{\varepsilon_{0}}\label{eq:gauss}\\
\nabla\cdot\vec{B}&=0\label{eq:gaussMag}\\
\nabla\times\vec{E}&=-\frac{\partial\vec{B}}{\partial t}\label{eq:faraday}\\
\nabla\times\vec{B}&=\mu_{0}\vec{J}+\mu_{0}\varepsilon_{0}\frac{\partial\vec{E}}{\partial t}\label{eq:ampere}
\end{align}
\end{subequations}
where $\vec{E}$ is the electric field vector, $\vec{B}$ is the
magnetic field, $\vec{J}$ is the total current density,
$\varepsilon_{0}$ is the permittivity of free space, and
$\mu_{0}$ is the permeability of free space.

From eq \eqref{eq:gaussMag} we can deduce that
\begin{equation}%\label{eq:}
\vec{B} = \nabla\times\vec{A}
\end{equation}
since we necessarily have the identity
\begin{equation}%\label{eq:}
\nabla\cdot\nabla\times\vec{A} = 0
\end{equation}
for any $\vec{A}$. We call $\vec{A}$ the (vector) magnetic
potential. We plug this into eq \eqref{eq:faraday} to find
\begin{subequations}
\begin{align}
\nabla\times\vec{E} = -\frac{\partial\vec{B}}{\partial t}\\
\nabla\times\vec{E}+\frac{\partial}{\partial t}\left(\nabla\times\vec{A}\right) = 0\label{eq:potential:stepTwo}\\
\nabla\times\vec{E}+\nabla\times\left(\frac{\partial}{\partial t}\vec{A}\right) = 0\label{eq:potential:stepThree}\\
\nabla\times\left(\vec{E}+\frac{\partial\vec{A}}{\partial t}\right) = 0.\label{eq:potential:stepFour}
\end{align}
\end{subequations}
We justify the step from \eqref{eq:potential:stepTwo} to
\eqref{eq:potential:stepThree} by assuming that $\vec{A}$ is at least twice
differentiable, which allows us to change the order of
differentiation. By Hemholtz theorem, we see that the general
form of the solution to \eqref{eq:potential:stepFour} is 
\begin{equation}%\label{eq:}
\vec{E} + \frac{\partial \vec{A}}{\partial t} = -\nabla \varphi
\end{equation} 
where $\varphi$ is the (scalar) electric potential. This implies then that
\begin{equation}%\label{eq:}
\vec{E} = -\frac{\partial \vec{A}}{\partial t} -\nabla \varphi
\end{equation}
which allows us to write the electric and magnetic fields in
terms of electric and magnetic potentials.

\ssn{Index Gymnastics} We are going to
introduce a slightly foreign way to express the same results. We
are going to use tensors, which behave in a particular way when
we change coordinates. We have vectors, which are essentially a
``list'' of components. We will start by writing 4-vectors with
indices. When familiarity with indices in Linear Algebra is
established, we'll move on to matrices and then general
tensors. Exercises are provided in this section to establish
familiarity with index gymnastics.

Indices can be a bit daunting at first, especially since everyone
uses different conventions. We will try to use the conventions
established in Misner, Thorne, and
Wheeler~\cite{MisnerThorneWheeler}. Namely, lowercase Latin indices refer
to spatial components of a four-vector, and lowercase Greek
indices refer to the spacetime components of a four-vector. When
we have multiple indices, that's a generalization of a matrix. A
quantity with two indices is precisely a matrix.

A four-vector can be written as
\begin{equation}%\label{eq:}
x^{\mu}(\tau) = \left(x^{0}(\tau),x^{i}(\tau)\right)= (ct,\vec{x}),\qquad(i=1,2,3)
\end{equation}
where $\tau$ is the proper time, $c$ is the speed of light, $t$
is the \emph{time coordinate}, $x^{0}=ct$ is the time component,
$x^{i}$ is the spatial components. We can similarly write the
electric potential and vector potential as a single four-vector:
\begin{equation}%\label{eq:}
A^{\mu} = (A^{0},A^{i}) = (\varphi,\vec{A}).
\end{equation}
We can write the four-gradient as
\begin{equation}%\label{eq:}
\partial_{\mu} = (\partial_{0},\partial_{i}),\qquad(i=1,2,3)
\end{equation}
where $\partial_{\mu}=\partial/\partial x^{\mu}$ is a partial
derivative.

Now, we can write indices ``upstairs'' (e.g. $x^{\alpha}$) or
``downstairs'' (e.g. $\partial_{\beta}$). What's the difference?
The difference is when we change coordinates. We see by the chain
rule that
\begin{subequations}
\begin{align}
\partial_{\alpha'} &= \frac{\partial}{\partial x^{\alpha'}}\\
&= \frac{\partial x^{\alpha}}{\partial
  x^{\alpha'}}\frac{\partial}{\partial x^{\alpha}}
\end{align}
\end{subequations}
where summation is implied over $\alpha$. Note that the new
coordinates $x^{\alpha'}$ have derivatives that transform
``ethically'' as
\begin{equation}%\label{eq:}
\begin{pmatrix}
$derivatives$\ $with$\\
$respect$\ $to$\ $new$\\
$coordinates$
\end{pmatrix} =
\begin{pmatrix}
$derivatives$\ $of$\ $old$\\
$coordinates$\ $with$\\
$respect$\ $to$\ $new$\\
$coordinates$
\end{pmatrix}
\begin{pmatrix}
$derivatives$\ $with$\\
$respect$\ $to$\ $old$\\
$coordinates$
\end{pmatrix}
\end{equation}
This is all by the chain rule. Whenever we multiply by a factor
of ``$d$ (old coordinates)/ $d$ (new coordinates)'' we call these
type of indices something special: \textbf{covariant
  indices}. They transform precisely by multiplying by $d$(old)/$d$(new).

On the other hand we have indices ``upstairs''. These transform
by the complete opposite way, that is 
\begin{equation}%\label{eq:}
A^{\alpha'}=\frac{\partial y^{\alpha'}}{\partial x^{\alpha}}A^{\alpha}
\end{equation}
where we have changed coordinates $x^{\alpha}\to
y^{\alpha'}$. These indices upstairs always transform in this
manner, and are called \textbf{contravariant indices.}

We can consider matrices using this abstract index notation. For
instance, the inner produce between two vectors $x^{\alpha}$ and
$y^{\beta}$ can be written using a matrix $\eta_{\alpha\beta}$ as
\begin{equation}%\label{eq:}
\<x,y\> = \sum_{\alpha=0}^{4}\sum_{\beta=0}^{4}
\eta_{\alpha\beta}x^{\alpha}y^{\beta} = \eta_{\alpha\beta}x^{\alpha}y^{\beta}
\end{equation}
where we have used the Einstein summation formula, i.e. we sum
over $\alpha$ and we sum over $\beta$. We can write the Lorentz
transformation as a change of coordinates (i.e. a Jacobian):
\begin{equation}%\label{eq:}
\frac{\partial x^{\alpha'}}{\partial x^{\beta}} = {\Lambda^{\alpha'}}_{\beta}
\end{equation}
where $x^{\alpha'}$ is the new set of coordinates, and
$x^{\beta}$ is the old coordinates. To change the electromagnetic
four-potential we use the matrix ${\Lambda^{\alpha'}}_{\beta}$ to
translate expressions in the ``old coordinates'' into
corresponding expressions in the ``new coordinates''. That is
\begin{equation}%\label{eq:}
A^{\alpha'} = \sum_{\beta} {\Lambda^{\alpha'}}_{\beta}A^{\beta} = {\Lambda^{\alpha'}}_{\beta}A^{\beta}
\end{equation}
where $A^{\beta}$ is the four-potential in the old coordinates,
and $A^{\alpha'}$ is the four-potential in the new coordinates.

One important matrix that must be noted is the Kronecker
delta\marginpar{Kronecker Delta} ${\delta_{\alpha}}^{\beta}$. It
is precisely the identity matrix (i.e. the diagonal components
are 1, all others are zero). More formally we can write it as
\begin{equation}%\label{eq:}
{\delta_{\alpha}}^{\beta} = \begin{cases} 1 & \alpha=\beta\\
0 & \text{otherwise.}\end{cases}
\end{equation}
Why is this important? Well, we use it all the time in math. In
special relativity, however, we use something slightly
different. We use the Minkowski metric $\eta^{\alpha\beta}$. In
the Cartesian coordinates it looks like
\begin{equation}%\label{eq:}
\eta^{\alpha\beta} = \begin{pmatrix} -1 & 0 & 0 & 0\\
0 & 1 & 0 & 0\\
0 & 0 & 1 & 0\\
0 & 0 & 0 & 1
\end{pmatrix}
\end{equation}
where we have chosen this signature to be in agreement with
relativists (we could have multiplied this matrix by -1, to be in
agreement with particle physicists; but the author is partial to
Misner, Thorne and Wheeler's conventions, so we use their
signature). We ``raise'' and ``lower'' indices with the Minkowski
metric. That is
\begin{equation}%\label{eq:}
x_{\beta} = \eta_{\alpha\beta}x^{\alpha}
\end{equation}
and
\begin{equation}%\label{eq:}
S^{\mu} = \eta^{\mu\nu}S_{\nu}.
\end{equation}
When we take the inner produce, as noted, we contract the indices
using the Minkowski metric. If this were flat Euclidean space
(i.e. nonrelativistic physics), we could use the Kronecker delta
for our metric when using Cartesian coordinates.

If we take this tensor point of view, how can we express Maxwell's
equations? We use a mathematical tool called the field strength
tensor. It's a matrix with components described by
\begin{equation}%\label{eq:}
F_{\alpha\beta} = \partial_{\alpha}A_{\beta}-\partial_{\beta}A_{\alpha} 
\end{equation}
where $A_{\mu}=\eta_{\mu\nu}A^{\nu}$. How is this useful? It's a
completely abstract piece of gibberish. 

Observation: the field strength tensor is antisymmetric. That is,
if we consider $F_{\alpha\beta}$ we know its value is
\begin{subequations}
\begin{align}
F_{\beta\alpha} &= \partial_{\beta}A_{\alpha} -
\partial_{\alpha}A_{\beta}\\
&= -\Big(-\partial_{\beta}A_{\alpha} +
\partial_{\alpha}A_{\beta}\Big)\\
&= -F_{\alpha\beta}
\end{align}
\end{subequations}
so in particular when $\alpha=\beta$, we find that
$F_{\alpha\alpha}=0$. This eliminates 4 components. We have 16-4=12
left. But we just deduced that there are duplicate copies, so we
have 12/2=6 independent components.

So lets consider the components of this field strength tensor. We
see that the temporal components of the field strength tensor are
\begin{equation}%\label{eq:}
F_{0j} = \partial_{0}A_{j} - \partial_{j}A_{0}
\end{equation}
or in other words we have written this as
\begin{equation}%\label{eq:}
F_{0j} =  \frac{\partial\vec{A}}{\partial t} - \nabla\varphi .
\end{equation}
We previously deduced that this corresponds to the electric field
vector $\vec{E}$. Similarly, for the other three components we
find that
\begin{subequations}
\begin{align}
F_{12} &= \partial_{1}A_{2} - \partial_{2}A_{1}\\
F_{23} &= \partial_{2}A_{3} - \partial_{3}A_{2}\\
F_{31} &= \partial_{3}A_{1} - \partial_{1}A_{3}
\end{align}
\end{subequations}
which amounts to (up to some sign)
$\nabla\times\vec{A}=\vec{B}$. The other three (spatial) components of the
field strength tensor corresponds to the magnetic field.

Observe that its components can be written as
\begin{equation}%\label{eq:}
F_{\mu\nu} = \begin{bmatrix} 
  0    & -E_x/c & -E_y/c & -E_z/c \\
 E_x/c & 0      & B_z    & -B_y   \\ 
 E_y/c & -B_z   & 0      & B_x    \\ 
 E_z/c & B_y & -B_x & 0 \end{bmatrix}
\end{equation}
where we have used $(-+++)$ for the metric signature, and $E_{i}$
are the components of the electric field vector, $B_{j}$ are the
components of the magnetic field vector.

\begin{exercise}
Let ${\delta_{a}}^{b}$ be the Kronecker delta in $n$
dimensions. Find ${\delta_{a}}^{a}$.
\end{exercise}
\answer{We find that \begin{equation}{\delta_{a}}^{a} =
    \sum_{a=1}^{n}{\delta_{a}}^{a}=\sum^{n}_{a=1}1=n\end{equation}
    by simply using the Einstein summation convention.}
\begin{exercise}\label{exercise:two}
Suppose that $S_{ab}$ is symmetric (that is $S_{ab}=S_{ba}$) and
$A^{ab}$ is anyisymmetric (that is $A^{ab}=-A^{ba}$). Show that $S_{ab}A^{ab}=0$.
\end{exercise}
\answer{We find that we can write the matrix $A^{ab}$ as
\begin{equation}
A^{ab} = \frac{1}{2}(A^{ab}-A^{ba})
\end{equation}
so in particular, we can plug this into the question and find
\begin{subequations}
\begin{align}
S_{ab}A^{ab} &= \frac{1}{2}\sum_{a,b}S_{ab}(A^{ab}-A^{ba})\\
 &= \frac{1}{2}\sum_{a,b}(S_{ab}A^{ab}-S_{ab}A^{ba})\\
 &=\frac{1}{2}\sum_{a,b}(S_{ab}A^{ab}-S_{ba}A^{ba})\\
 &=\frac{1}{2}\sum_{a,b}(S_{ab}A^{ab}-S_{ab}A^{ab})\\
 &= 0.
\end{align}
\end{subequations}
We justify the second to last step by noting we are summing over
\emph{dummy indices}, so we can relabel them how we want.}
\begin{exercise}
For $A^{ab}$ (as we considered in exercise \ref{exercise:two})
and for an arbitrary tensor $T_{ab}$, show that
\begin{equation}
A^{ab}T_{ab} = \frac{1}{2}A^{ab}(T_{ab}-T_{ba}).
\end{equation}
(Hint: one way is to observe we can write any tensor as the sum
of a symmetric tensor and an antisymmetric tensor.)
\end{exercise}
\answer{We find by definition that
\begin{subequations}
\begin{align}
A^{ab}T_{ab} &= \frac{1}{2}(A^{ab}-A^{ba})T_{ab}\\
&=\frac{1}{2}(A^{ab}T_{ab}-A^{ba}T_{ab})\\
&=\frac{1}{2}(A^{ab}T_{ab}-A^{ab}T_{ba})\\
&=\frac{1}{2}A^{ab}(T_{ab}-T_{ba})
\end{align}
\end{subequations}
where we once again use the fact that we can switch dummy indices
when we contract over them. Alternatively, we can observe that we
can write
\begin{equation}%\label{eq:}
T_{ab} = \frac{1}{2}(T_{ab}+T_{ba})+\frac{1}{2}(T_{ab}-T_{ba})
\end{equation}
that is, any arbitrary tensor can be written as the sum of a
symmetric tensor (first term) and an antisymmetric tensor (second
term). Upon this realization, we find that the first term
contracted with $A^{ab}$ is --- by the previous exercise! ---
precisely zero. We are left with
\begin{equation}%\label{eq:}
A^{ab}T_{ab} = 0 + \frac{1}{2}A^{ab}(T_{ab}-T_{ba})
\end{equation}
precisely as desired.}
\begin{exercise}
Show that under a change of coordinates, the Kronecker delta
${\delta_{\beta}}^{\alpha}$ transforms as a tensor.
\end{exercise}
\answer{We transform coordinates by $x^{\alpha}\mapsto
  y^{\alpha'}$ and then we find
\begin{subequations}
\begin{align}
{\delta_{\beta}}^{\alpha} \mapsto& \left[\frac{\partial x^{\beta}}{\partial
  y^{\beta'}}\frac{\partial y^{\alpha'}}{\partial x^{\alpha}}\right]{\delta_{\beta}}^{\alpha}\\
&= \frac{\partial x^{\alpha}}{\partial
  y^{\beta'}}\frac{\partial y^{\alpha'}}{\partial x^{\alpha}}\\
&= \delta^{\alpha}_{\beta'}{\delta_{\alpha}}^{\alpha'}\\
&= {\delta_{\beta'}}^{\alpha'}
\end{align}
\end{subequations}
precisely as desired.}
\begin{exercise}
Show that the derivative $\partial_{a}v^{b}$ of the components of
a vector \emph{does not} transform as a tensor under coordinate
changes.
\end{exercise}
\answer{We find that the partial derivatives transform
  covariantly
\begin{equation}%\label{eq:}
\partial_{a'} = \frac{\partial x^{a}}{\partial y^{a'}}\partial_{a}
\end{equation}
and the components of a vector transform contravariantly
\begin{equation}%\label{eq:}
v^{b'} = \frac{\partial y^{b'}}{\partial x^{b}}v^{b}.
\end{equation}
When we plug these into our expected equation we find
\begin{subequations}
\begin{align}
\partial_{a'}v^{b'} &= \left(\frac{\partial x^{a}}{\partial y^{a'}}\partial_{a}\right)\left(\frac{\partial y^{b'}}{\partial x^{b}}v^{b}\right)\\
&= \frac{\partial x^{a}}{\partial y^{a'}}\left(v^{b}\partial_{a}\frac{\partial y^{b'}}{\partial x^{b}}
+ \frac{\partial y^{b'}}{\partial x^{b}}\partial_{a}v^{b}\right)
\end{align}
\end{subequations}
which we see has some extra term involving the derivatives of the
Jacobian matrix.}
\begin{exercise}
Does the antisymmetrized derivative
$\partial_{a}v_{b}-\partial_{b}v_{a}$ of the components of a
one-form (covariant vector) transform as a tensor under
coordinate changes?
\end{exercise}
\answer{We see from the previous exercise that
\begin{equation}%\label{eq:}
\partial_{a'}v^{b'} = \frac{\partial x^{a}}{\partial y^{a'}}\left(v^{b}\partial_{a}\frac{\partial y^{b'}}{\partial x^{b}}
+ \frac{\partial y^{b'}}{\partial x^{b}}\partial_{a}v^{b}\right)
\end{equation}
so we find that
\begin{equation}%\label{eq:}
\partial_{a'}v_{b'} = \frac{\partial x^{a}}{\partial y^{a'}}\left(v_{b}\partial_{a}\frac{\partial x^{b}}{\partial y^{b'}}
+ \frac{\partial x^{b}}{\partial y^{b'}}\partial_{a}v_{b}\right)
\end{equation}
Then by antisymmetrization we find that
\begin{subequations}
\begin{align}
\partial_{a'}v_{b'}-\partial_{b'}v_{a'} &= \frac{\partial x^{a}}{\partial y^{a'}}\left(v_{b}\partial_{a}\frac{\partial x^{b}}{\partial y^{b'}}
+ \frac{\partial x^{b}}{\partial y^{b'}}\partial_{a}v_{b}\right) \nonumber\\
& - \frac{\partial x^{b}}{\partial
  y^{b'}}\left(v_{a}\partial_{b}\frac{\partial x^{a}}{\partial y^{a'}}+\frac{\partial x^{a}}{\partial y^{a'}}\partial_{b}v_{a}\right)
\end{align}
\end{subequations}
We see that by index gymnastics the expression
\begin{equation}%\label{eq:}
\frac{\partial x^{a}}{\partial y^{a'}}v_{b}\partial_{a}\frac{\partial x^{b}}{\partial y^{b'}}-\frac{\partial x^{b}}{\partial
  y^{b'}} v_{a}\partial_{b}\frac{\partial x^{a}}{\partial y^{a'}}
= 0
\end{equation}
identically, since we are subtracting out one thing from
itself. (Again, justified by switching dummy indices.) This then
implies we get the expression
\begin{equation}%\label{eq:}
\partial_{a'}v_{b'}-\partial_{b'}v_{a'} = \frac{\partial
  x^{a}}{\partial y^{a'}}\frac{\partial x^{b}}{\partial y^{b'}}\partial_{a}v_{b}
 - \frac{\partial x^{b}}{\partial y^{b'}}\frac{\partial x^{a}}{\partial y^{a'}}\partial_{b}v_{a}
\end{equation}
which transforms precisely as a covariant tensor with two indices.}

\ssn{Recovering Maxwell's Equations} We
can recover Maxwell's equations by, well, several techniques. One
is to consider the symmetries of the field strength tensor and
then show mathemagically\footnote{Bear in mind ``mathemagically''
  here refers to the fact that this ``miraculously'' works out,
  or at least to the unsuspecting observer it appears
  as\ldots{magic}.} this is equivalent to Maxwell's
equations. Conversely, we can begin with Maxwell's equations and
deduce their form using the Field Strength tensor. We'll take
this approach, i.e. beginning with Maxwell's equations and
demonstrate what their form would be using the Field Strength
tensor.

Lets first analyze electrostatics and electrodynamics. We see that Gauss' Law is
\begin{equation}%\label{eq:}
\nabla\cdot\vec{E} = \partial_{i}E^{i} = \partial_{i}F^{0i} = \rho
\end{equation}
where we set $\varepsilon_{0}=1$ since it's just a constant (and
doesn't really affect the underlying physics here). Without loss
of generality since $F^{00}=0$ we can rewrite this as
\begin{equation}%\label{eq:}
\partial_{\mu}F^{0\mu} = \rho.
\end{equation}
This is precisely the Gauss' Law for relativistic
electromagnetism. From Ampere's Law, we can find
\begin{equation}%\label{eq:}
(\nabla\times\vec{B})^{k} = \partial_{m}F^{km}
\end{equation}
where we have ``forced'' the index $k$ on the left hand side,
i.e. we specify the components of the resulting vector by
$k$. The rest of Ampere's Law is defined as
\begin{equation}%\label{eq:}
\mu_{0}J^{k}+\underbracket[0.5pt]{\mu_{0}\varepsilon_{0}\partial_{0}F^{0k}}_{=\mu_{0}\varepsilon_{0}\partial\vec{E}/\partial{t}}
\end{equation}
Thus putting it all together, we get
\begin{equation}%\label{eq:}
\partial_{m}F^{km} = 4\pi J^{k} + \partial_{0}F^{0k}
\end{equation}
or by rearranging terms we have thus
\begin{equation}%\label{eq:}
\partial_{0}F^{k0}+\partial_{m}F^{km} = \partial_{\mu}F^{k\mu} = 4\pi J^{k}.
\end{equation}
We can summarize\marginpar{Electrodynamics and electrostatics in terms of the field strength tensor}
the laws of electrostatics and electrodynamics in terms of the
field strength tensor thus
\begin{equation}%\label{eq:}
\partial_{\beta}F^{\alpha\beta}=4\pi J^{\alpha}
\end{equation}
where we introduce the four-current
\begin{equation}%\label{eq:}
J^{\alpha} = \left(\frac{\rho}{4\pi},J^{k}\right).
\end{equation}
We thus have half of Maxwell's equations contained in one.

What about magnetism? We have magnetostatics and
magnetodynamics. Magnetostatics is Gauss' Law for Magnetism:
\begin{equation}%\label{eq:}
\nabla\cdot\vec{B} = 0.
\end{equation}
First lets consider the magnetic field vector in terms of the
field strength tensor. We know that
\begin{equation}%\label{eq:}
B^{k} = (A_{3,2}-A_{2,3}, A_{1,3}-A_{3,1},A_{2,1}-A_{1,2}) = (F_{23},F_{31},f_{12})
\end{equation}
where we use shorthand $x_{j,k} = \partial_{k}x_{j}$. We have
thus\marginpar{Gauss Law for Magnetism}
\begin{equation}%\label{eq:}
\partial_{k}B^{k}=\partial_{1}F_{23}+\partial_{2}F_{31}+\partial_{3}F_{12}=0.
\end{equation}
What about magnetodynamics? We consider the Faraday's Law:
\begin{equation}%\label{eq:}
\nabla\times\vec{E}=-\frac{\partial\vec{B}}{\partial t}
\end{equation}
or equivalently we can write
\begin{equation}%\label{eq:}
-\partial_{0}B_{j}=\partial_{0}(F_{32},F_{13},F_{21})
\end{equation}
where we used the antisymmetry of the Field Strenght tensor to
absorb the sign. We can write the curl of the electric field by
considering it componentwise
\begin{subequations}
\begin{align}
(\nabla\times\vec{E})^{1} &= \partial_{2}E_{3}-\partial_{3}E_{2}\\
(\nabla\times\vec{E})^{2} &= \partial_{3}E_{1}-\partial_{1}E_{3}\\
(\nabla\times\vec{E})^{3} &= \partial_{1}E_{2}-\partial_{2}E_{1}
\end{align}
\end{subequations}
So inspecting this componentwise we get that
\begin{subequations}
\begin{align}
\partial_{2}F_{03}+\partial_{3}F_{20}+\partial_{0}F_{23}=0\\
\partial_{3}F_{01}+\partial_{1}F_{30}+\partial_{0}F_{31}=0\\
\partial_{1}F_{02}+\partial_{2}F_{10}+\partial_{0}F_{12}=0.
\end{align}
\end{subequations}
Combining everything together, we find that we can rewrite these
two equations as one single equation:
\begin{equation}%\label{eq:}
\partial_{\alpha}F_{\beta\gamma}+\partial_{\beta}F_{\gamma\alpha}+\partial_{\gamma}F_{\alpha\beta}=0
\end{equation}
which encodes the laws of magnetism.

Thus Maxwell's equations can be written by two important
equations:
\begin{equation}
\boxed{\begin{array}{c}\displaystyle\partial_{\beta}F^{\alpha\beta}=4\pi J^{\alpha}\\
\displaystyle\partial_{\alpha}F_{\beta\gamma}+\partial_{\beta}F_{\gamma\alpha}+\partial_{\gamma}F_{\alpha\beta}=0\end{array}}
\end{equation}

\ssn{Gauge Freedom} We have just reduced Maxwell's equations from
4 to 2 in a rather beautiful way. Now the question is: are the
four-potential vectors unique? That is, if we have some solution
$A^{\mu}$, will everyone agree on its value?

Suppose we had some other four vector, call it
\begin{equation}%\label{eq:}
\bar{A}^{\mu} = A^{\mu} + a^{\mu}
\end{equation}
where $a^{\mu}$ is some four-vector. If we can show that
$a^{\mu}=0$ identically, then the solution is unique. That is,
everyone agrees that $A^{\mu}=\bar{A}^{\mu}$. So no matter what
reference frame you are in, you see the same thing.

Consider the simple situation, when $a^{\mu}=\partial^{\mu}a$ is
just the gradient of ``some scalar function''. The field strength
tensor would read
\begin{subequations}
\begin{align}
\bar{F}_{\mu\nu}&=\partial_{\nu}\bar{A}_{\mu}-\partial_{\mu}\bar{A}_{\nu}\\
&=\partial_{\nu}(A_{\mu}+\partial_{\mu}a)-\partial_{\mu}(A_{\nu}+\partial_{\nu}a)\\
&=\partial_{\nu}A_{\mu}-\partial_{\mu}A_{\nu}+(\partial_{\nu}\partial_{\mu}-\partial_{\mu}\partial_{\nu})a\\
&=F_{\mu\nu}+0
\end{align}
\end{subequations}
which is not good! Two clearly distinct four-potentials give the
same answer! They only differ by some term
$\partial^{\mu}a$. This is scary, because we have no
uniqueness. What to do?

What's happening here mathematically speaking is we have too many
degrees of freedom. That is, we have to basically say ``Given two
different potentials $A^{\mu}_{(1)}$ and $A^{\mu}_{(2)}$, they're
really equivalent if we can write
$A^{\mu}_{(1)}=A^{\mu}_{(2)}+\partial^{\mu}f$ for some function
$f$''. Or we can look at it slightly differently saying we can
``change coordinates (potentials)'' by a transformation
\begin{equation}%\label{eq:}
A^{\mu}\to\bar{A}^{\mu}=A^{\mu}+\partial^{\mu}f
\end{equation}
where $f$ is ``some scalar function''. This sort of
transformation is known as a \textbf{gauge
  transformation}. Formally speaking, the component
$\partial^{\mu}f$ of the transformation should be seen as an
element of the space of one forms
\begin{equation}%\label{eq:}
dx^{\mu}\partial_{\mu}f\in\Omega^{1}\left(M,\mathfrak{u}(1)\right)
\end{equation}
where $M$ is spacetime, $\mathfrak{u}(1)$ is the Lie Algebra of $U(1)$ (the unitary
one-by-one matrices), $\Omega^{1}(M,\mathfrak{u}(1))$ is the
space of one forms that ``eat in'' points of spacetime $M$ and
``spits out'' values in the Lie algebra $\mathfrak{u}(1)$. This
should really be viewed as a one form $df$ times an element of
the Lie algebra $\mathfrak{u}(1)$. For more on Lie Algebras, see
the author's notes~\cite{notesOnLieAlgebras}.

Now, to avoid this problem, we can do several things depending on
the formalism. Usually, with a few exceptions, in the canonical
formalism (i.e. the Hamiltonian field theory formalism) we use
first class constraints for gauge theories. These will be
reviewed in a forthcoming note by the
author~\cite{constraints}. The other approach is to ``fix the
gauge''. We usually do this by demanding some additional
equation holds, among many of the choices the favorites are
\begin{subequations}
\begin{align}
\partial_{\mu}A^{\mu} = 0 &&& \text{Lorenz Gauge}\\
\partial_{i}A^{i} = 0 &&& \text{Coulomb Gauge}\\
x^{\mu}A_{\mu} = 0 &&& \text{Fock-Schwinger Gauge}\\
\varphi = 0 &&& \text{Weyl Gauge}\\
x^{i}A_{i} = 0 &&& \text{Multipolar Gauge}
\end{align}
\end{subequations}
and so on.

\ssn{Action Principle, Lagrangian Formalism} We will now shift our focus to derive the Maxwell
equations from an action principle. This will focus on the
Lagrangian formalism, we will turn our attention to the
Hamiltonian formalism next. We should remember when working with
field theory we work with \emph{densities}. That is, instead of
working with a Lagrangian we work with a \emph{Lagrangian density}.
The reader is invited to review classical field theory in the
author's notes~\cite{hamiltonianFieldTheory}. We will use the
convention from Misner, Thorne, and
Wheeler~\cite{MisnerThorneWheeler} and write the action as
\begin{equation}%\label{eq:}
I = \int \frac{-1}{16\pi}F_{\mu\nu}F^{\mu\nu}d^{4}x
\end{equation}
where $I$ is the action (this convention is chosen to be
consistent with the rest of my other notes). This is the
Lagrangian density for the ``free electromagnetic field''.

The only problem that may come to mind immediately, if one is
unfamiliar with classical field theory, is ``What plays the role
of `position' and `velocity' here?'' For the Lagrangian
formalism, we work with $A^{\mu}$ as the positions, and
$\partial_{\nu}A^{\mu}$ as the velocities. We first compute
\begin{subequations}
\begin{align}
F^{\mu\nu}F_{\mu\nu} &= F^{\mu\nu}(\partial_{\nu}A_{\mu}-\partial_{\mu}A_{\nu})\\
&=F^{\mu\nu}\partial_{\nu}A_{\mu}-F^{\mu\nu}\partial_{\mu}A_{\nu}\\
&=F^{\mu\nu}\partial_{\nu}A_{\mu}-F^{\nu\mu}\partial_{\nu}A_{\mu}\\
&=F^{\mu\nu}\partial_{\nu}A_{\mu}-(-F^{\mu\nu})\partial_{\nu}A_{\mu}\\
&=2F^{\mu\nu}\partial_{\nu}A_{\mu}
\end{align}
\end{subequations}
so we plug this into the action to find
\begin{subequations}
\begin{align}
I &= \frac{-1}{16\pi}\int F_{\mu\nu}F^{\mu\nu}d^{4}x\\
&= \frac{-1}{16\pi}\int2F^{\mu\nu}\partial_{\nu}A_{\mu}d^{4}x
\end{align}
\end{subequations}
where we used integration by parts to get to the third step, and
one of Maxwell's equations to get the last step. We then find
that
\begin{equation}%\label{eq:}
\frac{\delta I[A^{\mu}(x')]}{\delta A^{\nu}(x)} =
\frac{1}{8\pi}{\delta^{\mu}}_{\nu}\delta^{(4)}(x'-x)\partial_{\nu}F^{\mu\nu}(x') = \frac{1}{8\pi}\partial_{\nu}F^{\mu\nu}(x).
\end{equation}
and the variation of the boundary terms vanish (i.e. are zero).
Since there is no source (this is the free electromagnetic field,
after all) this is set to zero. In other words, the current is
zero for the free electromagnetic field, as desired (and
expected!). If we had any sources, we would have to add a term
into the Lagrangian density accounting for them. Such a term
resembles
\begin{equation}%\label{eq:}
\frac{1}{2}A_{\alpha}J^{\alpha} = \mathcal{L}_{int}
\end{equation}
for the ``interaction Lagrangian density''
$\mathcal{L}_{int}$. We would get the variation of the action
giving us
\begin{equation}%\label{eq:}
\frac{\delta I[A^{\mu}(x')]}{\delta A^{\nu}(x)} = 0\quad\Rightarrow\quad
\frac{1}{8\pi}\partial_{\nu}F^{\mu\nu}(x) - \frac{1}{2}J^{\mu} = 0.
\end{equation}
This is merely the law of electricity.

And what of magnetism? Well, it follows trivially from an
identity the field strength tensor obeys called the ``Jacobi
Identity''. That is, the field strength tensor always obeys
\begin{equation}%\label{eq:}
\partial_{\alpha}F_{\beta\gamma}+
\partial_{\beta}F_{\gamma\alpha}+
\partial_{\gamma}F_{\alpha\beta}=0.
\end{equation}
This is an identity, it's \emph{identically zero!}

\ssn{Action Principle, Hamiltonian Formalism.} Let's consider the
slightly less intuitive Hamiltonian formalism. It is a little
more involved, but it is mathematically richer (at least in the
author's humble opinion). We will try to show calculations as
explicitly as possible, but this is done only to the best of the
author's abilities.

The first step is to first find the canonically conjugate momenta:
\begin{equation}%\label{eq:}
\Pi^{j} = \frac{\partial\mathcal{L}}{\partial(\partial_{t}A_{j})}.
\end{equation}
When working with the free electric field, we've got
\begin{subequations}
\begin{align}
\Pi^{j} &=\frac{-1}{16\pi} \frac{\partial}{\partial(\partial_{t}A_{j})}\left(F_{\mu\nu}F^{\mu\nu}\right)\\
&= \frac{-2}{16\pi}F^{\mu\nu}\frac{\partial F_{\mu\nu}}{\partial(\partial_{t}A_{j})}\\
&=
\frac{-2}{16\pi}F^{\mu\nu}({\delta_{\mu}}^{t}{\delta_{\nu}}^{j} - {\delta_{\mu}}^{j}{\delta_{\nu}}^{t})\\
&= \frac{-1}{8\pi}(F^{tj}-F^{jt}) = \frac{1}{4\pi}E^{j}
\end{align}
\end{subequations}
In other words, the electric field\footnote{Here sadly the jargon
  becomes confusing. The ``field'' is the potential --- usually
  called the ``gauge field''. The ``conjugate momenta'' is not
  usually referred to as the field.} is the ``conjugate momenta'' to
the ``position'' potential.

We construct the Hamiltonian density\marginpar{Hamiltonian density} by the Legendre transform:
\begin{equation}%\label{eq:}
\mathcal{H} = E^{j}\partial_{t}A_{j} - \mathcal{L}.
\end{equation}
To get this expression in terms of the field $A_{j}$ and its
conjugate momenta $E^{j}$ we need to first reconsider the
Lagrangian density expression. We find the expression for the
Lagrangian density for the free electric field reads
\begin{equation}%\label{eq:}
16\pi\mathcal{L} = F_{0\nu}F^{0\nu} + F_{i\nu}F^{i\nu} =
F_{0\nu}F^{0\nu} + F_{i0}F^{i0} + F_{ij}F^{ij} = 2E^{j}E_{j}-2B^{j}B_{j}
\end{equation}
We see that we can write the electric field part as the conjugate
momenta, and the magnetic field part as the derivatives of the
position variable. We find that we can write
\begin{equation}%\label{eq:}
B_{i} = \frac{1}{2}\varepsilon_{ijk}F^{jk}
\end{equation}
where $\varepsilon_{ijk}$ is the Levi-Civita symbol, thus
\begin{equation}%\label{eq:}
B_{i}B^{i} = \frac{1}{4}\left(\varepsilon_{ijk}F^{jk}\right)\left(\varepsilon^{imn}F_{mn}\right)
\end{equation}
We use the contracted epsilon identity for the Levi-Civita symbol
\begin{equation}%\label{eq:}
 \varepsilon_{ijk}\varepsilon^{imn} = {\delta_{j}}^{m}{\delta_{k}}^{n} - {\delta_{j}}^{n}{\delta_{k}}^{m}.
\end{equation}
Thus
\begin{subequations}
\begin{align}
B_{i}B^{i} &= \frac{1}{4}({\delta_{j}}^{m}{\delta_{k}}^{n} - {\delta_{j}}^{n}{\delta_{k}}^{m})F^{jk}F_{mn}\\
&= \frac{1}{4}(F^{mn}-F^{nm})F_{mn}\\
&= \frac{1}{2}F^{mn}F_{mn}
\end{align}
\end{subequations}
We thus verify what has been done so far.

We can write the magnetic field using the Levi-Civita symbol as
\begin{equation}%\label{eq:}
B_{i} = \varepsilon_{ijk}\partial^{j}A^{k}
\end{equation}
thus the Lagrangian density becomes
\begin{equation}%\label{eq:}
8\pi\mathcal{L} = E^{j}E_{j}-B^{j}B_{j} 
\end{equation}
We plug this back into the Hamiltonian density
\begin{equation}%\label{eq:}
\mathcal{H} = \Pi^{j}\partial_{t}A_{j}-\frac{1}{8\pi}\left[E^{j}E_{j}-B^{j}B_{j} \right]
\end{equation}
We observe that
\begin{equation}%\label{eq:}
%\partial_{t}A_{j} = 0
\partial_{t}A_{j} = E_{j} + \partial_{j}A_{t}
\end{equation}
thus
\begin{subequations}
\begin{align}
\mathcal{H} &= \frac{1}{8\pi}\left[E^{j}E_{j}+B^{j}B_{j}\right] + \Pi^{j}\partial_{j}A_{t}\\
&= \frac{1}{8\pi}\left[E^{j}E_{j}+B^{j}B_{j}\right] +
\partial_{j}(\Pi^{j}A_{t}) - A_{t}\partial_{j}\Pi^{j}
\end{align}
\end{subequations}
%\begin{equation}%\label{eq:}
%\mathcal{H} = \frac{1}{8\pi}\left[E^{j}E_{j}+B^{j}B_{j}\right] + \Pi^{j}\partial_{j}A_{t}
%\end{equation}
is the expression for the Hamiltonian (energy) density. However,
the $\partial_{j}(\Pi A_{t})$ term is a boundary term, which
doesn't affect the principle of stationary action, so we can drop
it. We are left with
\begin{equation}%\label{eq:}
\mathcal{H} = \frac{1}{8\pi}\left[E^{j}E_{j}+B^{j}B_{j}\right] - A_{t}\partial_{j}\Pi^{j}
\end{equation}
as our Hamiltonian density.

We should consider how to interpret this
expression\marginpar{Gauss' Law is a constraint}
$A_{t}\partial_{j}\Pi^{j}$. We see that this is roughly equal to
\begin{equation}%\label{eq:}
A_{t}\partial_{j}\Pi^{j} \approx A_{t}\partial_{j}E^{j}
\end{equation}
up to some factor of $4\pi$. We are working with the free
electromagnetic field, so this should be zero. \emph{But that's
  only because the field is sourceless!} Lets add some source
term and see how the expression changes. To do this we take the
interaction Lagrangian density
\begin{equation}%\label{eq:}
\mathcal{L}_{int} = \frac{1}{2}A_{\mu}J^{\mu}
\end{equation}
and we find that it has no conjugate momenta. So we end up with
\begin{equation}%\label{eq:}
\mathcal{H} = \frac{1}{8\pi}\left[E^{j}E_{j}+B^{j}B_{j}\right] - A_{t}(\partial_{j}\Pi^{j}+\frac{1}{2}J^{t})-\frac{1}{2}A_{j}J^{j}
\end{equation}
Our equation becomes --- magically enough! --- merely Gauss' Law
\begin{equation}%\label{eq:}
A_{t}(\partial_{j}\Pi^{j}+\frac{1}{2}J^{t})\approx 0
\end{equation}
Here we use the ``weak equality'' $\approx$ to indicate this is a
constraint. Now, we should ask if it is first class (i.e. it
commutes with the Hamiltonian in the Poisson bracket) or if it is
second class (it doesn't commute).

\marginpar{Poisson Bracket} We will now start
considering the Poisson bracket of the terms involving $A_{j}$,
$\Pi^{k}$, $B_{l}$, $\mathcal{H}$. (Indeed, finding the
constraint which generates gauge transformations \emph{requires}
us to be comfortable with the Poisson bracket; we're looking for
something that commutes with the Hamiltonian in the Poisson
bracket, but is nonzero.) The Poisson bracket, for our situation,
should resemble
\begin{equation}%\label{eq:}
\{A,B\}\eqdef\int\left(\frac{\delta A}{\delta
  A^{j}(\bar{x})}\frac{\delta
  B}{\delta\Pi_{j}(\bar{x})}-\frac{\delta B}{\delta A^{j}(\bar{x})}\frac{\delta A}{\delta\Pi_{j}(\bar{x})}\right)d^{3}\bar{x}.
\end{equation}
Here the $\bar{x}$ is used to indicate that we are working with
the spatial components of the four-vectors, and not the entire
four-vector. We are using ``equal-time commutators''. That is,
the integral is taken over a ``slice'' of spacetime with fixed
time, i.e. a ``constant time slice''.

We find, first of all, that the commutation relations hold. That
is
\begin{equation}%\label{eq:}
\{A_{j}(x),A_{k}(x')\}=\{\Pi^{j}(x),\Pi^{k}(x')\}=0
\end{equation}
and
\begin{equation}%\label{eq:}
\{A_{j}(x),\Pi^{k}(x')\} = {\delta^{k}}_{j}\delta^{(3)}(x-x')
\end{equation}
where $\delta^{(3)}(x-x')$ is the Dirac delta for spatial
vectors, ${\delta^{k}}_{j}$ is our favorite Kronecker delta matrix.

\begin{comment}
Now there are 5 interesting combinations we can look at, but we
don't want to look at combinations involving the Hamiltonian
right away. That leaves merely 2 combinations. One is the Poisson
bracket of $A_{j}$ with $B_{l}$, the other is $\Pi^{k}$ with $B_{l}$.
We see that 
\begin{subequations}
\begin{align}
\{A_{m}(x),B_{i}(x')\} &=
\varepsilon_{ijk}\{A_{m}(x),\partial^{j}A^{k}(x')\}\\
&= \varepsilon_{ijk}\int\left({\delta_{m}}^{n}\delta^{(3)}(y-x)\frac{\delta\partial^{j}A^{k}(x')}{\delta\Pi^{n}(y)}\right)d^{3}x\\
&= \varepsilon_{ijk}\int\left({\delta_{m}}^{n}\delta^{(3)}(y-x)\partial^{j}\frac{\delta
  A^{k}(x')}{\delta\Pi^{n}(y)}\right)d^{3}x\\
&= 0.
\end{align}
\end{subequations}
Well\ldots that was anticlimactic.

With regards to the other combination, we find that we end up
with
\begin{subequations}
\begin{align}
\{B_{i}(x),\Pi^{l}(x')\} &=
\varepsilon_{ijk}\int\left({\delta_{n}}^{l}\delta^{(3)}(y-x')\frac{\delta\partial^{j}A^{k}(x)}{\delta A_{n}(y)}\right)d^{3}y.
\end{align}
\end{subequations}
To solve this tricky situation, we let
\begin{equation}%\label{eq:}
F[A^{k}(x)] = \int \partial^{j}A^{k}(x) d^{3}x.
\end{equation}
Then we use the definition of the functional derivative to find
\begin{subequations}
\begin{align}
\lim_{\varepsilon\to0}\frac{1}{\varepsilon}\left(F[A^{k}(x)+\varepsilon\delta^{(3)}(x-y)]-F[A^{k}(x)]\right)
&= \lim_{\varepsilon\to0}\frac{1}{\varepsilon}\int \partial^{j}\left(A^{k}(x)+\varepsilon\delta^{(3)}(x-y)-A^{k}(x)\right)d^{3}x\\
&= \lim_{\varepsilon\to0}\int \partial^{j}\left(\delta^{(3)}(x-y)\right)d^{3}x\\
&= \partial^{j}\int\delta^{(3)}(x-y)d^{3}y\\
&= -\delta^{(3)}(y-x).
\end{align}
\end{subequations}
We plug this into our Poisson bracket equation to find that
\begin{equation}%\label{eq:}
\{B_{i}(x),\Pi^{l}(x')\} \sim {\delta_{n}}^{l}{\delta^{kn}}\varepsilon_{ijk}\delta^{(3)}(x-x')
= \delta^{kl}\varepsilon_{ijk}\delta^{(3)}(x-x')
\end{equation}
where we have some fudge factors due to the derivative of the
Dirac delta function.
\end{comment}

Now, just for the sake of curiosity, what happens to the Gauss
law (which we determined was a constraint) in the Poisson bracket
with the Hamiltonian? Note the difference between the Hamiltonian
and the Hamiltonian density, we can write the Hamiltonian $H$ in
terms of the density $\mathcal{H}$ by the following:
\begin{equation}%\label{eq:}
H = \int \mathcal{H} d^{3}x
\end{equation}
where we integrate over a constant time slice. Let
\begin{equation}%\label{eq:}
C(x) = (\partial_{j}\Pi^{j}(x)+\frac{1}{2}J^{t}(x))
\end{equation}
denote our constraint. (We can deduce that, since constraints are
``enforced'' by a Lagrange multiplier, $A_{t}$ must be the
Lagrange multiplier to $C$.) We see that
\begin{subequations}
\begin{align}
\left\{\int C(x')d^{3}x',~H\right\} &= \partial_{t}\int C(x')d^{3}x'\\
&= \int\left(\frac{\delta\int C(x')d^{3}x'}{\delta
  A_{j}(y)}\frac{\delta H}{\delta \Pi^{j}(y)} -\frac{\delta H}{\delta
  A_{j}(y)}\frac{\delta \int C(x')d^{3}x'}{\delta \Pi^{j}(y)} \right) d^{3}y \\
&= \int\left(\frac{\delta C}{\delta
  A_{j}(y)}\partial_{t}A_{j} +\partial_{t}\Pi^{j}
\frac{\delta \int C(x')d^{3}x'}{\delta \Pi^{j}(y)} \right) d^{3}y\\
&= \int\left(\partial_{t}\Pi^{j}\frac{\delta \int C(x')d^{3}x'}{\delta \Pi^{j}(y)}\right) d^{3}y\\
&= 0.
\end{align}
\end{subequations}
This should be intuitively appealing (we don't expect Gauss' Law
to change as time goes on). This means that the constraint is
first class, and generates a gauge symmetry. By shear
coincidence, we noted that there is a gauge symmetry in our
$A^{\mu}$ variables!

The natural question is ``Will \marginpar{Gauss' Law generates gauge symmetries?} Gauss' Law (as a constraint)
generate the gauge symmetries of our field theory?'' We should
investigate this possibility. We want to show that
\begin{subequations}
\begin{align}
\left\{\int \Lambda(x)C(x)d^{3}x,~ A_{i}(x')\right\} &= \partial_{i}\Lambda(x')\\
\left\{\int \Lambda(x)C(x)d^{3}x,~ E^{j}(x')\right\} &= 0
\end{align}
\end{subequations}
are both true for some arbitrary function $\Lambda(x)$. This
corresponds to the gauge transformation described by
\begin{equation}%\label{eq:}
A_{i}(x)\mapsto A_{i'}(x) = A_{i}(x) + \partial_{i}\Lambda(x).
\end{equation}
For simplicity we will consider the free field case, i.e. when
there are no sources. Having made this simplifying condition, our
constraint becomes ``merely'' $C(x) = \partial_{j}\Pi^{j}$. The
second condition is easy to show, we see that by direct
computation
\begin{subequations}
\begin{align}
\left\{\int \Lambda(x)\partial_{k}\Pi^{k}(x)d^{3}x,~ \Pi^{j}(x')\right\}
&=\left\{-\int \Pi^{k}(x)\partial_{k}\Lambda(x)d^{3}x,~ \Pi^{j}(x')\right\}\\
&=-\int\left(\underbracket[0.5pt]{\frac{\delta\int \Pi^{k}(x)\partial_{k}\Lambda(x)d^{3}x}{\delta
  A_{l}(y)}}_{=0}\frac{\delta \Pi^{j}(x')}{\delta
  \Pi^{l}(y)}\right.\nonumber\\
&\phantom{-\int}\left. -\underbracket[0.5pt]{\frac{\delta \Pi^{j}(x')}{\delta
  A_{l}(y)}}_{=0}\frac{\delta \int
  \Pi^{k}(x)\partial_{k}\Lambda(x)d^{3}x}{\delta \Pi^{l}(y)}
\right) d^{3}y \\
&= -\int\left(0\right)d^{3}y = 0.
\end{align}
\end{subequations}
The first step, which may be subtle, is through integration by
parts. We end up with a constant plus the term in the first slot
of the Poisson bracket. The Poisson bracket of a constant with
the canonical momenta (or position) vanishes identically (similar
to how the derivative of a constant is zero identically). The
last step was by virtue of the fact that our constraint is
independent of $A_{j}$.

So we just have to consider the case when we have the Poisson
bracket of our constraint (multiplied by some function) together
with our ``gauge field'' $A_{j}$. This is also fairly easy (in
the source-free case):
\begin{subequations}
\begin{align}
\left\{\int \Lambda(x)\partial_{i}\Pi^{i}(x)d^{3}x,~ A_{j}(x')\right\}
&=-\left\{\int \Pi^{i}(x)\partial_{i}\Lambda(x)d^{3}x,~ A_{j}(x')\right\}
\\
&= -\int\left(\frac{\delta\int
  \Pi^{i}(x)\partial_{i}\Lambda(x)d^{3}x}{\delta~A_{k}(y)}\frac{\delta~A_{j}(x')}{\delta\Pi^{k}(y)}\right.\\\nonumber
&\phantom{-\int\left(\frac{}{}\right.~}\left.-\frac{\delta~A_{j}(x')}{\delta~A_{k}(y)}\frac{\delta\int \Pi^{i}(x)\partial_{i}\Lambda(x)d^{3}x}{\delta\Pi^{k}(y)} \right)d^{3}y\\
&=-\int\left(0-\frac{\delta~A_{j}(x')}{\delta~A_{k}(y)}\frac{\delta\int \Pi^{i}(x)\partial_{i}\Lambda(x)d^{3}x}{\delta\Pi^{k}(y)} \right)d^{3}y\\
&=\int\left(\delta^{(3)}(x'-y){\delta^{k}}_{j}{\delta_{k}}^{i}\partial_{i}\Lambda(y)\right)d^{3}y\\
&=\partial_{j}\Lambda(x').
\end{align}
\end{subequations}
This is precisely what we wanted to show.

\begin{exercise}
Carefully double check these computations.
\end{exercise}
\begin{exercise}
Reperform these calculations when one source is present. Then
consider the case when multiple sources are present.
\end{exercise}



\section{A Comically Brief Review}
The following is a table summarizing the properties of the solutions of the 
Dirac equation.

\begin{tabular}{|p{3cm}|c|c|}
\hline
Property & Electrons & Positrons\\ \hline
Spinor components & $\psi(x)=au^{(s)}(p)\exp[-(i/\hbar)p\cdot x]$ & $\psi(x)=av^{(s)}(p)\exp[-(i/\hbar)p\cdot x]$\\ \hline
Momentum Space Dirac Equation & $(\gamma^\mu p_\mu - mc)u = 0$ & $(\gamma^\mu p_\mu + mc)v = 0$ \\ \hline
Adjoint Dirac Equation & $\bar{u}(\gamma^\mu p_\mu - mc) = 0$ & $\bar{v}(\gamma^\mu p_\mu + mc) = 0$ \\ \hline
Orthogonality & $\bar{u}^{(1)}u^{(2)} = 0$ & $\bar{v}^{(1)}v^{(2)} = 0$ \\ \hline
Normalization & $\bar{u}u = 2mc$ & $\bar{v}v = -2mc$ \\ \hline
Complete & $\sum_{s} u^{(s)}\bar{u}^{(s)} = (\gamma^\mu p_\mu + mc)$ & $\sum_{s} v^{(s)}\bar{v}^{(s)} = (\gamma^\mu p_\mu - mc)$ \\ \hline
\end{tabular}

A free photon, on the other hand, of momentum $p = (E/c, \bold{p})$ with
$E = |\bold{p}|c$ is represented by the wave function
\begin{equation}
A^{\mu}(x) = ae^{-(i/\hbar)p\cdot x}\epsilon^\mu_{(s)}
\end{equation}
where $\epsilon^\mu$ is a spin dependent vector, $s=1,2$ for the two polarizations
(``spin states'') of the photon. The polarization vectors $\epsilon^\mu_{(s)}$
satisfy the \emph{momentum space Lorentz condition:}
\begin{equation}
\epsilon^\mu p_\mu = 0.
\end{equation}
 They are orthogonal in the sense that 
\begin{equation}
\epsilon_{\mu(1)}^{*}\epsilon^{\mu}_{(2)} = 0.
\end{equation}
They are further normalized
\begin{equation}
\epsilon^{*}_{\mu}\epsilon^{\mu} = 1.
\end{equation}
In the Coulomb gauge
\begin{equation}
\epsilon^0=0,\quad \epsilon\cdot p = 0
\end{equation}
and the polarization three-vectors obey the completeness relation
\begin{equation}
\sum_{s=1,2} (\epsilon_{(s)})_i (\epsilon^{*}_{(s)})_j = \delta_{ij} - \hat{p}_{i}\hat{p}_{j}.
\end{equation}

\section{Quantum Electrodynamics}
\subsection{The Rules to the Game}

So we want to calculate out the probability amplitude $\mathcal{M}$ associated
with a particular Feynman diagram, we proceed as follows:
\begin{enumerate}
\item{(Notation)} We must be more careful here! We label the incoming and 
outgoing four-momenta $p_1$, $p_2$, \ldots, $p_n$ and the corresponding spins
$s_1$, $s_2$, \ldots, $s_n$. We label the inernal four-momenta $q_1$, $q_2$, \ldots.
Assign arrows to the lines as follows: the arrows on \emph{external} fermion lines 
indicates whether it is an electron or positron (if the arrow points forward in 
time, it is an electron; backwards in time it is a positron); arrows on
\emph{internal} fermion lines are assigned so that the ``direction of the flow''
through the diagram is preserved (i.e. every vertex must have at least one arrow
entering and one arrow leaving). The arrows on photon lines (which is optional,
since arrows are used to indicate whether the particle is an antiparticle or not;
bosons are their own antipartners) point ``forward'' in time.

\item{(External Lines)} External lines contribute factors as follows:
\begin{equation*}
\mbox{Electrons: } \begin{cases} \mbox{Incoming: } u\\
\mbox{Outgoing: }\bar{u} \end{cases}
\end{equation*}
\begin{equation*}
\mbox{Positrons: } \begin{cases} \mbox{Incoming: } \bar{v}\\
\mbox{Outgoing: }v\end{cases}
\end{equation*}
\begin{equation*}
\mbox{Photons: } \begin{cases} \mbox{Incoming: }\epsilon^\mu \\
\mbox{Outgoing: }(\epsilon^\mu)^* \end{cases}
\end{equation*}

\item{(Vertex Factors)} Each vertex contributes a factor
\begin{equation}
ig_{e}\gamma^\mu
\end{equation}
The dimensionless coupling constant $g_e$ is related to the charge of the
positron $g_e$ = $e\sqrt{4\pi/\hbar c}$ = $\sqrt{4\pi\alpha_E}$\footnote{Here
$\alpha_E$ is the coupling constant of the electromagnetic force. In \emph{general},
the QED coupling is $-q\sqrt{4\pi/\hbar c}$ where $q$ is the charge of the \emph{particle}
(as opposed to antiparticle). For electrons $q=-e$, for an up quark $q=(2/3)e$.}
\item{(Propagators)} Each internal line contributes a factor as follows
\begin{equation}
\mbox{Electrons and Positrons: }\frac{i\gamma^\mu q_\mu + mc}{q^2 - m^2c^2}
\end{equation}
\begin{equation}
\mbox{Photons: }\frac{-ig_{\mu\nu}}{q^2}
\end{equation}
\item{(Conservation of Energy and Momentum)} For each vertex, write a delta 
function of the form
\begin{equation}
(2\pi)^4\delta^{(4)}(k_1 + k_2 + k_3)
\end{equation}
This enforces the conservation of momentum at the vertex.
\item{(Integrate Over Internal Momenta)} For each internal momentum $q$, write
a factor
\begin{equation}
\frac{d^4 q}{(2\pi)^4}
\end{equation}
and integrate.
\item{(Cancel the Delta Function)} The result will include a factor
\begin{equation}
(2\pi)^4\delta^{(4)}(p_1 + p_2 + \cdots - p_n)
\end{equation}
which corresponds to the overall energy-momentum conservation. Cancel this factor,
and we get $-i\mathcal{M}$.
\item{(Antisymmetrization)} Include a minus sign between diagrams that differ
only in the interchange of two incoming (or outgoing) electrons (or positrons),
or of an incoming electron with an outgoing positron (or vice versa).
\end{enumerate}

\section{Elastic Processes}

An elastic (relativistic) process is one where kinetic energy, rest energy, and
mass are all conserved. We will explore such examples in QED.

\subsection{Electron-Muon Scattering}

We draw the diagram (note the use of $\mu$ and $\nu$ at the vertices, which are 
used to sum over in the integral):

\strut

\begin{center}
\begin{fmffile}{QEDexOneImg1}
  \begin{fmfgraph*}(50,25) \fmfpen{0.2mm}
    \fmfset{arrow_len}{3mm}\fmfset{arrow_ang}{10}
    \fmfleft{i1,o1} % change i2->o1 
    \fmfright{i2,o2} % change o1->i2
    \fmflabel{$p_{1},s_{1}$}{i1}
    \fmflabel{$p_{3},s_{3}$}{o1} %
    \fmflabel{$p_{2},s_{2}$}{i2} %
    \fmflabel{$p_{4},s_{4}$}{o2}
    \fmflabel{$\mu$}{v1}
    \fmflabel{$\nu$}{v2}
    \fmf{fermion}{i1,v1,o1} %
    \fmf{dbl_plain_arrow}{i2,v2,o2} %
    \fmf{boson,label=$q$}{v1,v2}
  \end{fmfgraph*}
\end{fmffile}
\end{center}
\strut

We will now evaluate it in a haphazard manner. Observe how it is
done when spinors are in the game.

\textbf{Step One:} We will evaluate the part emboldened in Red
first.


\strut
\begin{center}
\begin{fmffile}{QEDexOneImg2}
  \begin{fmfgraph*}(50,25)  \fmfpen{0.2mm}
    \fmfset{arrow_len}{3mm}\fmfset{arrow_ang}{10}
    \fmfleft{i1,o1} % change i2->o1 
    \fmfright{i2,o2} % change o1->i2
    \fmflabel{$p_{1},s_{1}$}{i1}
    \fmflabel{$p_{3},s_{3}$}{o1} %
    \fmflabel{$p_{2},s_{2}$}{i2} %
    \fmflabel{$p_{4},s_{4}$}{o2}
    \fmflabel{$\mu$}{v1}
    \fmflabel{$\nu$}{v2}
    \fmf{fermion,fore=red}{i1,v1,o1} %
    \fmf{dbl_plain_arrow}{i2,v2,o2} %
    \fmf{boson,label=$q$}{v1,v2}
  \end{fmfgraph*}
\end{fmffile}
\end{center}
\strut

We will now analyze it in careful detail so we will ``pull it
out'' and ``dissect'' it carefully.

We evaluate it in the following manner: since we write quantum mechanics like
we write chinese (from right to left), we begin with 
\begin{equation}
\Diagram{\vertexlabel^{p_3,s_3}\\
fd \\
& g\vertexlabel_{\mu,q}\\
\vertexlabel_{p_1,s_1} {\color{red}fu}\\
} = u(s_1,p_1),
\qquad
\Diagram{\vertexlabel^{p_3,s_3}\\
fd \\
& {\color{red}g}\vertexlabel_{\mu,q}\\
\vertexlabel_{p_1,s_1} fu\\
} = (ig_{e}\gamma^{\mu})u(s_1,p_1)
\end{equation}
We have one last step to do
\begin{equation}
\Diagram{\vertexlabel^{p_3,s_3}\\
{\color{red}fd} \\
  & g\vertexlabel_{\mu,q} \\
\vertexlabel_{p_1,s_1} fu\\
} = \bar{u}(s_3,p_3)(ig_{e}\gamma^{\mu})u(s_1,p_1)
\end{equation}
So this contributes
\begin{equation}\label{exOneElectronTerm}
(\bar{u}(s_3,p_3))(ig_{e}\gamma^\mu)(u(s_1, p_1))
\end{equation}
to the integrand. Our integrand is going to take the form
\begin{equation}
[(\bar{u}(s_3,p_3))(ig_{e}\gamma^\mu)(u(s_1, p_1))]\begin{pmatrix} $photon$\\$propagator$\end{pmatrix}\begin{pmatrix}$muon$\\$terms$\end{pmatrix} \begin{pmatrix}$conservation$\ $of$\\
$momentum$\ $delta$\\
$functions$\end{pmatrix}
\end{equation}
We will now move on to step two.

\textbf{Step Two:} We will consider the photon propagator, which corresponds to
the red line in the following diagram


\strut
\begin{center}
\begin{fmffile}{QEDexOneImg3}
  \begin{fmfgraph*}(50,25)  \fmfpen{0.2mm}
    \fmfset{arrow_len}{3mm}\fmfset{arrow_ang}{10}
    \fmfleft{i1,o1} % change i2->o1 
    \fmfright{i2,o2} % change o1->i2
    \fmflabel{$p_{1},s_{1}$}{i1}
    \fmflabel{$p_{3},s_{3}$}{o1} %
    \fmflabel{$p_{2},s_{2}$}{i2} %
    \fmflabel{$p_{4},s_{4}$}{o2}
    \fmflabel{$\mu$}{v1}
    \fmflabel{$\nu$}{v2}
    \fmf{fermion}{i1,v1,o1} %
    \fmf{dbl_plain_arrow}{i2,v2,o2} %
    \fmf{boson,label=$q$,fore=red}{v1,v2}
  \end{fmfgraph*}
\end{fmffile}
\end{center}
\strut


This corresponds to the term
\begin{equation}
\frac{-ig_{\mu\nu}}{q^2}
\end{equation}
giving our integrand to be
\begin{equation}
[(\bar{u}(s_3,p_3))(ig_{e}\gamma^\mu)(u(s_1, p_1))]\frac{-ig_{\mu\nu}}{q^2}\begin{pmatrix}$muon$\\$terms$\end{pmatrix} \begin{pmatrix}$conservation$\ $of$\\
$momentum$\ $delta$\\
$functions$\end{pmatrix}.
\end{equation}

\textbf{Step Three and Four:} Moving right along to the Muon terms, we have exactly a term
analagous to Eq (\ref{exOneElectronTerm}). Muons are fermions with spin 1/2,
with the same electric charge as an electron. So translating this into Feynman
diagram terms, it is translated in the exact same fashion we translated the
electron terms. So we have a contribution of
\begin{equation}
(\bar{u}(s_4,p_4))(ig_{e}\gamma^\nu)(u(s_2, p_2)).
\end{equation}
Our integrand now becomes
\begin{equation}
[(\bar{u}(s_3,p_3))(ig_{e}\gamma^\mu)(u(s_1, p_1))]\frac{-ig_{\mu\nu}}{q^2}[(\bar{u}(s_4,p_4))(ig_{e}\gamma^\nu)(u(s_2, p_2))] \begin{pmatrix}$conservation$\ $of$\\
$momentum$\ $delta$\\
$functions$\end{pmatrix}.
\end{equation}

\textbf{Step Five:} We kind of ``fudged up'' steps 1 through 4 because they are
so interconnected it is hard to seperate them out from each other. We are now
safely onto step 5 of the Feynman rules of QED: conservation of momentum! We
have two places to do this (at the $\mu$ and $\nu$ vertices). We have for $\mu$
(chosen randomly) the input momentum in red and output momentum in blue:


\strut
\begin{center}
\begin{fmffile}{QEDexOneImg4}
  \begin{fmfgraph*}(50,25)  \fmfpen{0.2mm}
    \fmfset{arrow_len}{3mm}\fmfset{arrow_ang}{10}
    \fmfleft{i1,o1} % change i2->o1 
    \fmfright{i2,o2} % change o1->i2
    \fmflabel{$p_{1},s_{1}$}{i1}
    \fmflabel{$p_{3},s_{3}$}{o1} %
    \fmflabel{$p_{2},s_{2}$}{i2} %
    \fmflabel{$p_{4},s_{4}$}{o2}
    \fmflabel{$\mu$}{v1}
    \fmflabel{$\nu$}{v2}
    \fmf{fermion,fore=red}{i1,v1} %
    \fmf{fermion,fore=blue}{v1,o1}
    \fmf{dbl_plain_arrow}{i2,v2,o2} %
    \fmf{boson,label=$q$,fore=blue}{v1,v2}
  \end{fmfgraph*}
\end{fmffile}
\end{center}
\strut


This corresponds to the conservation of momentum
\begin{equation}
p_1 = p_3 + q \quad\Rightarrow\quad p_1 - p_3 - q = 0
\end{equation}
which gives us the delta function
\begin{equation}\label{exOneTermToTakeAdvantageOf}
(2\pi)^{4}\delta^{(4)}(p_1 - p_3 - q).
\end{equation}
We have another conservation of momentum point, which is at the
vertex $\nu$:



\strut
\begin{center}
\begin{fmffile}{QEDexOneImg5}
  \begin{fmfgraph*}(50,25)  \fmfpen{0.2mm}
    \fmfset{arrow_len}{3mm}\fmfset{arrow_ang}{10}
    \fmfleft{i1,o1} % change i2->o1 
    \fmfright{i2,o2} % change o1->i2
    \fmflabel{$p_{1},s_{1}$}{i1}
    \fmflabel{$p_{3},s_{3}$}{o1} %
    \fmflabel{$p_{2},s_{2}$}{i2} %
    \fmflabel{$p_{4},s_{4}$}{o2}
    \fmflabel{$\mu$}{v1}
    \fmflabel{$\nu$}{v2}
    \fmf{fermion}{i1,v1} %
    \fmf{fermion}{v1,o1}
    \fmf{dbl_plain_arrow,fore=red}{i2,v2} %
    \fmf{dbl_plain_arrow,fore=blue}{v2,o2} %
    \fmf{boson,label=$q$,fore=red}{v1,v2}
  \end{fmfgraph*}
\end{fmffile}
\end{center}
\strut


Which corresponds to a conservation of momentum 
\begin{equation}
p_2 + q = p_4\quad\Rightarrow\quad p_2 + q - p_4 = 0
\end{equation}
and thus contributes the delta function term
\begin{equation}
(2\pi)^4 \delta^{(4)}(p_2 + q - p_4)
\end{equation}
rendering our integrand to be
\begin{eqnarray}
\quad&&[(\bar{u}(s_3,p_3))(ig_{e}\gamma^\mu)(u(s_1, p_1))]\frac{-ig_{\mu\nu}}{q^2}[(\bar{u}(s_4,p_4))(ig_{e}\gamma^\nu)(u(s_2, p_2))] (2\pi)^{8}\nonumber\\
& &\times \delta^{(4)}(p_1 - p_3 - q)\delta^{(4)}(p_2 + q - p_4) d^4q. \nonumber
\end{eqnarray}

\textbf{Step Six:} We integrate over the internal momenta (in our case the photon's momentum), so we have the integral expression:
\begin{eqnarray}
i\mathcal{M} &\textrm{``=''}& (2\pi)^{4} \int [(\bar{u}(s_3,p_3))(ig_{e}\gamma^\mu)(u(s_1, p_1))]\frac{-ig_{\mu\nu}}{q^2}[(\bar{u}(s_4,p_4))(ig_{e}\gamma^\nu)(u(s_2, p_2))] \nonumber\\
& &\times \delta^{(4)}(p_1 - p_3 - q)\delta^{(4)}(p_2 + q - p_4) d^4q. \nonumber
\end{eqnarray}
Observe this is harder than it looks because we are taking the trace of gamma
matrices! That is the whole point of having the metric tensor $g^{\mu\nu}$ here.
So it is a bit tricky to compute...

We will integrate over $q$ and take advantage of the delta function term (\ref{exOneTermToTakeAdvantageOf})
to make the switch
\begin{equation*}
q\to p_1 - p_3
\end{equation*}
giving us the result from the integral
\begin{equation}\label{exOneIntegralResult}
(2\pi)^{4} \frac{ig_{e}^2}{(p_1 - p_3)^2} [(\bar{u}(s_3,p_3))(ig_{e}\gamma^\mu)(u(s_1, p_1))][(\bar{u}(s_4,p_4))(ig_{e}\gamma_\mu)(u(s_2, p_2))]\delta^{(4)}(p_2 + p_1 - p_3 - p_4).
\end{equation}

\textbf{Step Seven:} We simply set Eq (\ref{exOneIntegralResult}) to be equal
to $-i\mathcal{M}\delta^{(4)}(p_2 + p_1 - p_3 - p_4)$, and we solve to find
\begin{equation}
\mathcal{M} = \frac{-g_{e}^2}{(p_1 - p_3)^2} [(\bar{u}(s_3,p_3))(ig_{e}\gamma^\mu)(u(s_1, p_1))][(\bar{u}(s_4,p_4))(ig_{e}\gamma_\mu)(u(s_2, p_2))]
\end{equation}
is the probability amplitude. In spite of this nightmarish appearence, with 
four spinors and eight $\gamma$ matrices, this is still just a number. We can
figure it out when the spins are specified.

\subsection{Moller Scattering}

Moller scattering is the scattering of electrons
\begin{equation}
e^{-} + e^- \to e^- + e^-.
\end{equation}
We have two diagrams to consider this time! In fact, from here on out, we will
always have two diagrams to consider (the exception being one third order example,
which is the most important third order example because it explains the 
anamolous magnetic moment of an electron -- we'll burn that bridge when we get
to it).

\textbf{Step One:} The first diagram to consider is the
following:



\strut
\begin{center}
\begin{fmffile}{mollerImg1}
  \begin{fmfgraph*}(50,25)  \fmfpen{0.2mm}
    \fmfset{arrow_len}{3mm}\fmfset{arrow_ang}{10}
    \fmfleft{i1,o1} % change i2->o1 
    \fmfright{i2,o2} % change o1->i2
    \fmflabel{$p_{1},s_{1}$}{i1}
    \fmflabel{$p_{3},s_{3}$}{o1} %
    \fmflabel{$p_{2},s_{2}$}{i2} %
    \fmflabel{$p_{4},s_{4}$}{o2}
    \fmflabel{$\mu$}{v1}
    \fmflabel{$\nu$}{v2}
    \fmf{fermion}{i1,v1} %
    \fmf{fermion}{v1,o1}
    \fmf{fermion}{i2,v2} %
    \fmf{fermion}{v2,o2} %
    \fmf{boson,label=$q$}{v1,v2}
  \end{fmfgraph*}
\end{fmffile}
\end{center}
\strut


This is precisely the electron-muon diagram with the exception that the muon
has been replaced by an electron. Thus we will simply use the \textbf{exact
same steps} we did in the first example; we will copy/paste the results here.

The integrand should take the form
\begin{eqnarray}
\quad&&[(\bar{u}(s_3,p_3))(ig_{e}\gamma^\mu)(u(s_1, p_1))]\frac{-ig_{\mu\nu}}{q^2}[(\bar{u}(s_4,p_4))(ig_{e}\gamma^\nu)(u(s_2, p_2))] (2\pi)^{8}\nonumber\\
& &\times \delta^{(4)}(p_1 - p_3 - q)\delta^{(4)}(p_2 + q - p_4) d^4q. \nonumber
\end{eqnarray}
This has the contribution to the total probability amplitude that this process
will happen of
\begin{equation}
\mathcal{M}_1 = \frac{-g_{e}^2}{(p_1 - p_3)^2} [(\bar{u}(s_3,p_3))(ig_{e}\gamma^\mu)(u(s_1, p_1))][(\bar{u}(s_4,p_4))(ig_{e}\gamma_\mu)(u(s_2, p_2))]
\end{equation}
We will add it to the probability amplitude from the other graph to get the total
probability amplitude of the process happening.

The second diagram is odd:

\strut
\begin{center}
\begin{fmffile}{mollerImg2}
  \begin{fmfgraph*}(25,50)  \fmfpen{0.2mm}
    \fmfset{arrow_len}{3mm}\fmfset{arrow_ang}{10}
    \fmfleft{i2,o2}
    \fmfright{i1,o1}
    \fmf{fermion}{i1,v1}
    \fmf{phantom}{v1,o1} % Invisible rubber band
    \fmf{fermion}{i2,v2}
    \fmf{phantom}{v2,o2} % also invisible rubber band
    \fmf{photon,label=$q$}{v1,v2}
    % These are visible, but have no tension.
    \fmf{fermion,tension=0}{v1,o2}
    \fmf{fermion,tension=0}{v2,o1}
    \fmfdot{v1,v2}
    \fmflabel{$p_2,s_2$}{i1}
    \fmflabel{$p_1,s_1$}{i2}
    \fmflabel{$p_3,s_3$}{o1}
    \fmflabel{$p_4,s_4$}{o2}
    \fmflabel{$\mu$}{v1}
    \fmflabel{$\nu$}{v2}
  \end{fmfgraph*}
\end{fmffile}
\end{center}
\strut


We make the switch of $(s_3,p_3)\iff (s_4,p_4)$ for this diagram, and low and
behold we have a rule that takes care of this!

\textbf{Step Eight:} (Yes we are hopping right along!) We have by the eighth rule
a change in signs. So the probability amplitude from this second diagram is
(when we make the switches of $p_3\mapsto p_4$, $p_4\mapsto p_3$, $s_3\mapsto s_4$, $s_4\mapsto s_3$)
\begin{equation}
\mathcal{M}_2 = \frac{g_{e}^2}{(p_1 - p_4)^2} [(\bar{u}(s_4,p_4))(ig_{e}\gamma^\mu)(u(s_1, p_1))][(\bar{u}(s_3,p_3))(ig_{e}\gamma_\mu)(u(s_2, p_2))]
\end{equation}
So the total probability amplitude is then
\begin{eqnarray*}
\mathcal{M} &=& \frac{-g_{e}^2}{(p_1 - p_3)^2} [(\bar{u}(s_3,p_3))(ig_{e}\gamma^\mu)(u(s_1, p_1))][(\bar{u}(s_4,p_4))(ig_{e}\gamma_\mu)(u(s_2, p_2))] \\
&&+ \frac{g_{e}^2}{(p_1 - p_4)^2} [(\bar{u}(s_4,p_4))(ig_{e}\gamma^\mu)(u(s_1, p_1))][(\bar{u}(s_3,p_3))(ig_{e}\gamma_\mu)(u(s_2, p_2))].
\end{eqnarray*}


%
% \section{Inelastic Processes}
% An inelastic (relativistic) process is one where kinetic energy, rest energy,
% or mass are not conserved. We will explore such examples in QED.

% % Anamolous magnetic moment for the electron
% \input{thirdOrderEx}
%
 % done
%%
%% conclusion.tex
%% 
%% Made by Alex Nelson
%% Login   <alex@tomato>
%% 
%% Started on  Sat Jul 25 14:56:27 2009 Alex Nelson
%% Last update Sat Jul 25 14:56:27 2009 Alex Nelson
%%

We've reviewed some notions from quantum mechanics, such as the
Rigged Hilbert Space and using unitary operators for
observables. When using representation theory, we need a unitary
representation of a group for use in quantum theory.

We've introduced various aspects of making quantum mechanics
relativistic. The main approach is to take advantage of the fact
that special relativity is basically ``just'' the Poincar\'e
group. We then proceeded to find a unitary representation of the
Lorentz group and the group of spacetime translations, then
combined them in a suitably nice way.

We've considered the situation of making the position operator
relativistic, and concluded after a few naive attempts that it
wouldn't work. 

The interested reader is free to peruse the resources cited in
the bibliography for further reading (specifically, the notion of
measurement relative to an observer is tackled beautifully in
Gambini and Porto~\cite{Gambini:2001pq}).

\appendix
\section{Gamma Matrices}

For an extensive reference, see~\cite{Borodulin:1995xd}. The defining property for the gamma matrices is that they form a Clifford algebra with the anticommutation relations
\begin{equation}\label{anticommutator}
\{\gamma^{\mu},\gamma^{\nu}\} = \gamma^{\mu}\gamma^{\nu} + \gamma^{\nu}\gamma^{\mu} = 2\eta^{\mu\nu}I
\end{equation}
where $\eta^{\mu\nu}$ is the Minkowski metric with signature (+---) and $I$ is the unit (identity) matrix.

We can also define covariant gamma matrices by
\begin{equation}
\gamma_\mu = \eta_{\mu \nu} \gamma^\nu = \left(\gamma^0, -\gamma^1, -\gamma^2, -\gamma^3 \right)
\end{equation}
where Einstein summation is used.

\begin{rmk}
We may define a fifth element of our Clifford algebra,
\begin{equation}
 \gamma^5 := i\gamma^0\gamma^1\gamma^2\gamma^3
\end{equation}
or equivalently
\begin{equation}
 \gamma^5 = \frac{i}{4!} \epsilon_{\mu \nu \alpha \beta} \gamma^{\mu} \gamma^{\nu} \gamma^{\alpha} \gamma^{\beta} 
\end{equation}
which is true due to the anticommutation relations (\ref{anticommutator}). It has the following properties:
\begin{enumerate}
\item{(Hermitian)} $(\gamma^5)^\dagger = \gamma^5 \,$
\item{(Eigenvalues are $\pm1$)} $\left\{ \gamma^5,\gamma^\mu \right\} =\gamma^5 \gamma^\mu + \gamma^\mu \gamma^5 = 0 \,$
\item{(Anticommutes with other 4 generators)} $\left\{ \gamma^5,\gamma^\mu \right\} =\gamma^5 \gamma^\mu + \gamma^\mu \gamma^5 = 0 \,$
\end{enumerate}
\end{rmk}

\begin{rmk}
We can project a Dirac field onto its left-handed and right-handed components by
\begin{equation}
\psi_L= \frac{1-\gamma^5}{2}\psi, \qquad\psi_R= \frac{1+\gamma^5}{2}\psi 
\end{equation}
which is often useful when dealing with chirality in a quantum mechanical setting.
\end{rmk}

We should think of the tuple $\gamma^{\mu} = \left(\gamma^0, \gamma^1, \gamma^2, -\gamma^3 \right) = \gamma^{0}e^0 + \gamma^1e^1 + \gamma^2e^2 + \gamma^3e^3$ sort of as a 4-vector (where $e^\mu$ is the basis vectors). But this is misleading! We should view the $\gamma^\mu$ as a mapping operator that ``eats up'' a 4-vector $a^\mu$ and ``spits out'' the corresponding vector in the Clifford representation.

Such a result would be represented by the \textbf{Feynman Slash}
\begin{equation}
\slashed{a} := \gamma^\mu a_\mu. 
\end{equation}
It should be noted that this beast, $\slashed{a}$ \!\!, ``lives'' in the Clifford space so any changes to the basis vectors are irrelevant.

A quick review of some of the properties of the Dirac Gamma matrices!

\begin{property}{(Normalisation)}
Due to the anticommutation relations (\ref{anticommutator}), we can show
\begin{equation}
\left(\gamma^0\right)^\dag = \gamma^0\qquad\textrm{and }\left(\gamma^0\right)^2 = I
\end{equation}
and for the other gamma matrices (for $k=1,2,3$) we have
\begin{equation}
\left(\gamma^k\right)^\dag = -\gamma^k\qquad\textrm{and }\left(\gamma^k\right)^2 = -I
\end{equation}
which results in a generalized relationship which encapsulates all this information:
\begin{equation}
\left( \gamma^\mu \right)^\dagger = \gamma^0 \gamma^\mu \gamma^0.
\end{equation}
\end{property}

\begin{rmk}
These relationships described below, and the property described above, are in the (+---) signature; if we used the (-+++) signature, things would be different.
\end{rmk}

We also have a list of identities the Gamma matrices obey:
\begin{enumerate}
\item $\displaystyle\gamma^\mu\gamma_\mu = 4I$,
\item $\displaystyle\gamma^\mu\gamma^\nu\gamma_\mu=-2\gamma^\nu$,
\item $\displaystyle\gamma^\mu\gamma^\nu\gamma^\rho\gamma_\mu=4\eta^{\nu\rho} I$,
\item $\displaystyle\gamma^\mu\gamma^\nu\gamma^\rho\gamma^\sigma\gamma_\mu=-2\gamma^\sigma\gamma^\rho\gamma^\nu$.
\end{enumerate}
Similarly, there are 5 trace identities the Gamma matrices obey
\begin{enumerate}
\item The trace of the product of an odd number of $\gamma$ is 0,
\item $\operatorname{tr} (\gamma^\mu\gamma^\nu) = 4\eta^{\mu\nu}$,
\item $\operatorname{tr}(\gamma^\mu\gamma^\nu\gamma^\rho\gamma^\sigma)=4(\eta^{\mu\nu}\eta^{\rho\sigma}-\eta^{\mu\rho}\eta^{\nu\sigma}+\eta^{\mu\sigma}\eta^{\nu\rho})$,
\item $\operatorname{tr}(\gamma^5)=\operatorname{tr} (\gamma^\mu\gamma^\nu\gamma^5) = 0$,
\item $\operatorname{tr} (\gamma^\mu\gamma^\nu\gamma^\rho\gamma^\sigma\gamma^5) = -4i\epsilon^{\mu\nu\rho\sigma}$.
\end{enumerate}

%%%%%%%%%%%%%%%%%%%%%%%%%%%%%%%%%%%%%%%%%%%%%%%%%%%%%%%%%%%%%%%%%%%%%%%%

\subsection{Representations of the Gamma Matrices}\label{Representations of the Gamma Matrices}

We can represent the gamma matrices in various different ways that satisfy the anticommutation relations and all the above identities and properties. First recall the Pauli matrices, as they will prove useful in our discussion:
\begin{equation}
\sigma_1 = \sigma_x = \begin{bmatrix} 0&1\\ 1&0 \end{bmatrix}
\end{equation}
\begin{equation}
\sigma_2 = \sigma_y = \begin{bmatrix} 0&-i\\ i&0 \end{bmatrix}
\end{equation}
\begin{equation}
\sigma_3 = \sigma_z = \begin{bmatrix} 1&0\\ 0&-1 \end{bmatrix}.
\end{equation}
We will let the 2 by 2 identity be denoted by $I_{2}$ in this section.

One representation is the \textbf{Dirac Basis}
\begin{equation}
\gamma^0 = \begin{bmatrix} I & 0 \\ 0 & -I \end{bmatrix},\quad \gamma^i = \begin{bmatrix} 0 & \sigma^i \\ -\sigma^i & 0 \end{bmatrix},\quad \gamma^5 = \begin{bmatrix} 0 & I \\ I & 0 \end{bmatrix}.
\end{equation}

Another common one used is the \textbf{Weyl (chiral) basis} which basically changes the ``temporal'' gamma matrix while leaving the others the same. This causes the $\gamma^5$ quantity to change too. We can succinctly describe it as:
\begin{equation}
\gamma^0 = \begin{bmatrix} 0 & I \\ I & 0 \end{bmatrix},\quad \gamma^i = \begin{bmatrix} 0 & \sigma^i \\ -\sigma^i & 0 \end{bmatrix},\quad \gamma^5 = \begin{bmatrix} -I & 0 \\ 0 & I \end{bmatrix}.
\end{equation}
This has the advantage that the chiral projections are merely
\begin{equation}
\psi_L=\begin{bmatrix} I & 0 \\0 & 0 \end{bmatrix}\psi,\quad \psi_R=\begin{bmatrix} 0 & 0 \\0 & I \end{bmatrix}\psi.
\end{equation}
By slightly abusing notation, we can identify
\begin{equation}
\psi=\begin{bmatrix} \psi_L \\ \psi_R \end{bmatrix},
\end{equation}
where $\psi_L$ and $\psi_R$ are left-handed and right-handed two-component Weyl spinors.

The third, and for our investigations last, basis is the Majorana basis, in which all the Dirac matrices are imaginary. We can write them as
\begin{equation*}
\gamma^0 = \begin{bmatrix} 0 & \sigma^2 \\ \sigma^2 & 0 \end{bmatrix}, \quad \gamma^1 = \begin{bmatrix} i\sigma^3 & 0 \\ 0 & i\sigma^3 \end{bmatrix}
\end{equation*}
\begin{equation}
\gamma^2 = \begin{bmatrix} 0 & -\sigma^2 \\ \sigma^2 & 0 \end{bmatrix}, \quad \gamma^3 = \begin{bmatrix} -i\sigma^1 & 0 \\ 0 & -i\sigma^1 \end{bmatrix}, \quad \gamma^5 = \begin{bmatrix} \sigma^2 & 0 \\ 0 & -\sigma^2 \end{bmatrix}.
\end{equation}

\subsection{Euclidean Representation}

Oftentimes in path integral approaches, we can Wick Rotate from Minkowski to Euclidean spacetime by making time imaginary\footnote{Not as in ``eleventeen is an imaginary number'' but as in $\sqrt{-5}$ is an imaginary number.}. We then are forced to work with Euclidean gamma matrices. There are two major representations in the Euclidean framework for them.

The first is the chiral representation, defined by
\begin{equation}
\gamma^{1,2,3} = \begin{bmatrix} 0 & -i \sigma^{1,2,3} \\ i \sigma^{1,2,3} & 0 \end{bmatrix}, \quad \gamma^4=\begin{bmatrix} 0 & I \\ I & 0 \end{bmatrix}.
\end{equation}
This is different from the Minkowski set by the relation
\begin{equation}
\gamma^{5} = \gamma^1\gamma^2\gamma^3\gamma^4 = \gamma^{5+}.
\end{equation}
So in a Chiral basis we have
\begin{equation}
\gamma^5 = \begin{bmatrix} I & 0 \\ 0 & -I \end{bmatrix} .
\end{equation}

The other form is the nonrelativistic form, which is succinctly described by
\begin{equation}
\gamma^{1,2,3} = \begin{bmatrix} 0 & -i \sigma^{1,2,3} \\ i \sigma^{1,2,3} & 0 \end{bmatrix}, \quad \gamma^4=\begin{bmatrix} I & 0 \\ 0 & -I \end{bmatrix}, \quad  \gamma^5=\begin{bmatrix} 0 & -I \\ -I & 0 \end{bmatrix}.
\end{equation}


\section{Decay Rates and Feynman Diagrams}

Remember if we have some collection of particles (e.g. Muons) and they decay,
the decay rate $\Gamma$ (the probability per unit time that any given muon will
disintegrate) satisfies a particular relation. If $N(t)$ is the number of particles
at time $t$, the infinitesmal change in $N$ from $t$ to $t+dt$ is
\begin{equation}
dN = -\Gamma N(t)dt
\end{equation}
which tells us the number is decreasing when we move forward in time. It follows
that
\begin{eqnarray*}
\frac{1}{N}dN &=& -\Gamma dt\\
\int\frac{1}{N}dN &=& -\int\Gamma dt\\
\ln(N(t)) &=& -\Gamma t + C \\
N(t) &=& \exp(-\Gamma t)\exp(C) \\
&=& N(0)\exp(-\Gamma t)
\end{eqnarray*}
where $N(0)$ is the initial number of particles, and $C$ is the constant of
integration. 

The \textbf{mean lifetime} of the particle is simply the reciprocal of the 
decay rate
\begin{equation}
\tau = \frac{1}{\Gamma}.
\end{equation}
If there are several different ways for the particle decay, each with different
decay rates, the total decay rate is given by the sum of the individual rates:
\begin{equation}
\Gamma_\textrm{tot} = \sum_{j=1}^{n} \Gamma_{j}
\end{equation}
and the mean lifetime is the reciprocal of this quantity
\begin{equation}
\tau = \frac{1}{\Gamma_\textrm{tot}}.
\end{equation}

\subsection{Fermi's Golden Rule}

Suppose we have one particle that decays into several others
\begin{equation}
1\to 2+3+\cdots+n
\end{equation}
If $\mathcal{M}$ is the total probability amplitude from the various Feynman
diagram representations of this process, then the infinitesmal decay rate is
given by
\begin{equation}
d\Gamma = |\mathcal{M}|^2 \frac{S}{2\hbar m_1}\left[\left(\frac{cd^3p_2}{(2\pi)^32E_2}\right)\left(\frac{cd^3p_3}{(2\pi)^32E_3}\right) \cdots \left(\frac{cd^3p_n}{(2\pi)^32E_n}\right) \right]\times (2\pi)^4 \delta^{(4)}(p_1 - (p_2 + p_3 + \cdots + p_n))
\end{equation}

\nocite{thaller1992de}\nocite{mandlShaw}\nocite{peskinSchroeder}
\bibliographystyle{utphys}
\bibliography{feynman}
\end{document}
