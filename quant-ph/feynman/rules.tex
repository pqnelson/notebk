\section{Feynman Rules in a Nutshell with a Toy Model}


We will work in a toy model\footnote{It is commonly referred to as the $\phi^4$ 
model in the literature.} with massive spinless particles (so we won't have to 
worry about spin). This is the easiest nontrivial example of the use of Feynman 
diagrams. The basic ritual of Feynman diagrams is outlined thus:
\begin{enumerate}
\item{(Notation)} Label the incoming and outgoing four-momenta $p_1$, $p_2$,
$\ldots$, $p_n$. Label the internal momenta $q_1$, $q_2$, $\ldots$. Put an
arrow on each line, keeping track of the ``positive'' direction (antiparticles
move ``backward'' in time).

\item{(Coupling Constant)} At each vertex, write a factor of
\begin{equation*}
-ig
\end{equation*}
where $g$ is called the ``\textbf{coupling constant}''; it specifies the
strength of the interaction. In our toy model, $g$ will have dimensions of
momentum, but in the real world it is dimensionless.

\item{(Propagator)} For each internal line, write a factor
\begin{equation*}
\frac{i}{q_{j}^{2} - m_{j}^2c^2}
\end{equation*}
where $q_j$ is the four-momentum of the line ($q_j^2=q_{j}^{\mu}q_{j\mu}$; i.e.
$j$ is just a label keeping track of which internal line we are dealing with)
and $m_j$ is the mass of the particle the line describes. (Note that for
virtual particles, we don't have the $E^2 - \vec{p}\cdot\vec{p}=m^2c^2$ relation
that's for external legs only!)

\item{(Conservation of Momentum)} For each vertex, write a delta function of
the form
\begin{equation*}
(2\pi)^4\delta^{(4)}(k_1+k_2+k_3)
\end{equation*}
where the $k$'s are the three four-momenta coming \emph{into} the vertex (if
the arrow leads outward, then $k$ is \emph{minus} the four-momentum of that
line). This factor imposes conservation of energy and momentum at each vertex,
since the delta function is zero unless the sum of the incoming momenta equals
the sum of the outgoing momenta.

\item{(Integration over Internal Momenta)} For each internal line, write down
a factor
\begin{equation*}
\frac{1}{(2\pi)^4}d^{4}q_{j}
\end{equation*}
and integrate over all internal momenta.

\item{(Cancel the Delta Function)} The result will include a delta function
\begin{equation*}
(2\pi)^4\delta^{(4)}(p_1 + p_2 + \cdots - p_n)
\end{equation*}
enforcing overall conservation of energy and momentum. Erase this factor, and
what remains is $i\mathcal{M}$ that is $-i$ times the contribution to the 
amplitude from this process.
\end{enumerate}

What we do with these rules is we form an integrand by multiplying everything
together, so at the end it should look something like this:
\begin{equation}
i\mathcal{M} \textrm{ ``='' } \begin{pmatrix}$coupling$\\
$constants$
\end{pmatrix}
\int
\begin{pmatrix}
$propagators$
\end{pmatrix}
\begin{pmatrix}
$delta$\\ $functions$
\end{pmatrix}d
\begin{pmatrix}
$internal$\\ $lines$
\end{pmatrix}
\end{equation}
