%%
%% compositionOne.tex
%% 
%% Made by Alex Nelson
%% Login   <alex@black-cherry>
%% 
%% Started on  Wed Aug 26 11:30:52 2009 Alex Nelson
%% Last update Wed Aug 26 11:30:52 2009 Alex Nelson
%%

\begin{figure}[ht]
\includegraphics{img/img.0}
\caption{Composition of trousers with upside down trousers}\label{fig:img0}
\end{figure}

Consider the composition as doodled in figure \ref{fig:img0}. We
can consider the time evolution operation described by
$Z(m)\circ{Z(\Delta)}$, or --- read from right to left --- ``the
time evolution operator for the rightside up trousers followed by
the time evolution operator for the upside down trousers''. We
see that this behavior is determined by
\begin{equation}%\label{eq:}
\<\phi,Z(m)\circ{Z(\Delta)}\psi\> = \int (\cdots) \mathcal{D}A
\end{equation}
where $\phi,\psi\in L^{2}(U(1))$ describes the initial and final
state vectors, respectively. We recall that
\begin{equation}%\label{eq:}
Z(\Delta):e^{ikA}\mapsto e^{-k^{2}e^{2}V/2}e^{ikA_{1}}e^{ikA_{2}}
\end{equation}
and
\begin{equation}%\label{eq:}
Z(m):e^{ikA_{1}}e^{ikA_{2}}\mapsto \delta_{k_{1},k}\delta_{k_{2},k}e^{-k^{2}e^{2}V/2}e^{ikA}
\end{equation}
so it would logically seem that
\begin{subequations}
\begin{align}
[Z(m)\circ Z(\Delta)](e^{ikA}) &= Z(m)\left(e^{-k^{2}e^{2}V/2}e^{ikA_{1}}e^{ikA_{2}}\right)\\
&= e^{-k^{2}e^{2}V/2}Z(m)\left(e^{ikA_{1}}e^{ikA_{2}}\right)\\
&= e^{-k^{2}e^{2}V/2}\delta_{k_{1},k}\delta_{k_{2},k}e^{-k^{2}e^{2}V/2}e^{ikA}\\
&= e^{-k^{2}e^{2}V}e^{ikA}\delta_{k_{1},k}\delta_{k_{2},k} \label{eq:compositionOne}
\end{align}
\end{subequations}
where we justify the second step by linearity.

Compare this to the time evolution operator described by the
cylinder with the same amount of volume, i.e. $2V$. We see that
\begin{equation}%\label{eq:}
[Z(C)]\left(e^{ikA}\right) = e^{-k^{2}e^{2}(2V)/2}e^{ikA} = e^{-k^{2}e^{2}V}e^{ikA}
\end{equation}
which looks surprisingly familiar (hint hint). In fact the only
difference between \eqref{eq:compositionOne} and our cylinder is
a factor of $\delta_{k_{1},k}\delta_{k_{2},k}$; should this be
interpreted as a residue of topological origin? Some sign that
there was a ``hole''? The answer is, unsurprisingly, no. It
wouldn't be testable. Its contributions would be either 1 or 0,
and in the case of the former it matches exactly with the
cylinder. In the case of the latter, we wouldn't know anything
happened. So it is as though that factor is 1. 

\begin{figure}[ht]
\includegraphics{img/img.1}
\caption{A cup followed by a cap}\label{fig:img1}
\end{figure}

We can also compare this to the cobordism described by figure
\ref{fig:img1}. We consider how the operator
$Z(\iota)\circ{}Z(\varepsilon)$ acts on a basis state vector
$\exp(ik'A)$. We see that
\begin{subequations}
\begin{align}
\<e^{ikA_{2}},Z(\iota)\circ{}Z(\varepsilon)e^{ik'A_{1}}\> &= \int e^{-ikA_{2}}e^{ik'A_{1}}\sum_{n}e^{-n^{2}e^{2}(2V)/2}e^{in(A_{2}-A_{1})}\mathcal{D}A\\
&= \int^{2\pi}_{0} e^{-ikA_{2}}\sum_{n}e^{-n^{2}e^{2}(2V)/2}e^{inA_{2}}\int^{2\pi}_{0}e^{i(k'-n)A_{2}}\frac{dA_{1}}{2\pi}\frac{dA_{2}}{2\pi}\\
&= \int^{2\pi}_{0} e^{-ikA_{2}}\sum_{n}e^{-n^{2}e^{2}(2V)/2}e^{inA_{2}}\delta_{n,k'}\frac{dA_{2}}{2\pi}\\
&= \int^{2\pi}_{0} e^{-ikA_{2}}e^{-(k')^{2}e^{2}(2V)/2}e^{ik'A_{2}}\frac{dA_{2}}{2\pi}
\end{align}
\end{subequations}
which implies that
\begin{equation}%\label{eq:}
Z(\iota)\circ{}Z(\varepsilon):\exp(ik'A)\mapsto e^{-(k')^{2}e^{2}V}\exp(ik'A).
\end{equation}
