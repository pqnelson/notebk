%%
%% functionalQFT.tex
%% 
%% Made by Alex Nelson
%% Login   <alex@tomato>
%% 
%% Started on  Mon Jul 27 16:55:06 2009 Alex Nelson
%% Last update Mon Jul 27 16:55:06 2009 Alex Nelson
%%
\documentclass{amsart}
\usepackage{url}
\usepackage{manfnt}
\usepackage{amsthm}
\usepackage{amsmath}
\usepackage{amsthm}
\usepackage{amssymb}
\usepackage{amsfonts}
\usepackage{amscd}
\usepackage{graphicx}
\usepackage{mathrsfs}
\usepackage{hyperref}
\hypersetup{
    colorlinks,%
    citecolor=black,%
    filecolor=black,%
    linkcolor=black,%
    urlcolor=black
}
\usepackage{comment}
\usepackage[colorinlistoftodos, shadow]{todonotes}
\usepackage{pdfcomment}
\usepackage{feyn}
\usepackage{color}
\usepackage{feynmp}
\usepackage{brackets}
\numberwithin{equation}{section}

\theoremstyle{definition}
\newtheorem{defn}{Definition}
\newtheorem{thm}{Theorem}
\newtheorem{rmk}{Remark}
\newtheorem{lem}{Lemma}
\newtheorem{cor}{Corollary}
\newtheorem{ex}{Example}
\newtheorem{nonex}[ex]{NON-Example}
\newtheorem{prop}{Proposition}
\newtheorem{sch}{Scholium}
\newtheorem{axm}{Axiom}
\newtheorem*{prob}{Problem}
%\setcounter{tocdepth}{4}
\def\re{\operatorname{Re}}
\def\im{\operatorname{Im}}
\def\tr{\operatorname{Tr}}
\def\<{\langle}
\def\>{\rangle}

%%
% This macro header is what controls the ``dangerous bend''
% paragraph
%%
\def\rd{\noindent\begingroup\hangindent=3pc\hangafter=-2\def\par{\endgraf\endgroup}\hbox
  to0pt{\hskip-\hangindent\dbend\hfill}\ignorespaces}
%%
% This command allows you to write stuff in small font size and
% use the
% bourbaki ``dangerous bend'' so it's great when you want to
% ramble on 
% about some extra stuff!
%%
\newcommand{\danger}[1] {\rd{\small {#1}}}

%%
% This macro header is what controls the ``dangerous bend''
% paragraph
%%
\def\ddbend{\dbend\kern1pt\dbend}

\def\rdd{\noindent\begingroup\hangindent=4pc\hangafter=-2\def\par{\endgraf\endgroup}\hbox
  to0pt{\hskip-\hangindent\ddbend\hfill}\ignorespaces}

\newcommand{\ddanger}[1] {\rdd{\small {#1}}}
\newcommand{\sgn}{\operatorname{{\mathrm sgn}}}
\newcommand{\define}[1] {\textbf{#1}\index{#1}}
\title{Funcational Methods in Quantum Field Theory}
\date{July 27, 2009}
\email{pqnelson@gmail.com}
\author{Alex Nelson}
\begin{document}
\begin{fmffile}{img/feynmanDiags}
\maketitle
\tableofcontents

\section{Path Integrals In Quantum Mechanics}
%%
%% pathIntegralQM.tex
%% 
%% Made by Alex Nelson
%% Login   <alex@tomato>
%% 
%% Started on  Mon Jul 27 17:17:38 2009 Alex Nelson
%% Last update Mon Jul 27 17:17:38 2009 Alex Nelson
%%

\subsection{Definition via Unitary Time Evolution}
Consider a nonrelativistic particle in one dimension. The
Hamiltonian for this system is generically given as
\begin{equation}%\label{eq:}
H = \frac{p^{2}}{2m} + V(x).
\end{equation}
Suppose we want to consider the probability amplitude for this
particle moving from $x_0$ to $x_1$ in some time interval
$T$. We'll denote such an amplitude $U(x_0,x_1;T)$. It is the
position representation of the Schrodinger time-evolution
operator. In the canonical Hamiltonian formalism, $U$ is computed
by the expression
\begin{equation}\label{eq:canonicalUnitaryTimeEvolution}
U(x_0,x_1;T) = \<x_1|e^{-i\widehat{H}T/\hbar}|x_0\>.
\end{equation}
The right hand side is read from right to left, like Chinese, and
the is interpreted as ``The quantum process moving from some
initial state $|x_0\>$ to some final state $\<x_1|$ by means of
the physical process described by the operator
$\exp(-i\hat{H}T/\hbar)$''. In the path integral formalism,
however, our expression for $U$ is remarkably different. We'll
try to give a motivation of the path integral expression of $U$
and then prove it is equal to \eqref{eq:canonicalUnitaryTimeEvolution}.

The fundamental principle that we will be using in the derivation
of the path integral formalism is the superposition
principle. What do we mean by this? Well, we take the observation
summed up thus:
\begin{description}
\item[Superposition Principle] when a process can take place in
  more than one way, its total probability amplitude is the
  ``coherent sum'' of the amplitudes of each possible way.
\end{description}
A simple, nontrivial example is the double slit experiment. The
amplitude for an electron to arrive at the detector from the
source is the sum of the amplitudes of the two paths drawn in
figure \eqref{fig:doubleSlit}. As the two paths differ in length,
there is necessarily interference.


\begin{figure}[t]
\includegraphics{img/img.1}
\caption{The double slit experiment.}\label{fig:doubleSlit}
\end{figure}

For a general quantum system, we might try to write the total
amplitude for traveling from $x_a$ to $x_b$ as:
\begin{equation}\label{eq:defnOfPathIntegral}
U(x_a,x_b;T) = \sum_{\substack{
\text{all paths}\\
\text{from}\\
\text{$x_a$ to $x_b$}
}} e^{i\cdot(\text{phase})}
= \int\mathcal{D}x(t) e^{i\cdot(\text{phase})}.
\end{equation}
Now the physicist's explanation is the following: to be
democratic, we have each path be written as a pure phase, so
there is no interference and thus no path is ``more important
than others''. There is probably some kernel of truth to this,
but it's unrigorous --- there's little sign that \emph{a priori}
nature is either fair or democratic. A probably more realistic
explanation is that any two distinct paths are ``orthogonal'' in
some sense, so they are each different ``ways'' from the initial
state to the final state. Note that the symbol $\int\mathcal{D}x(t)$
should be intuitively interpreted as ``sum over all paths''. For
our example, all paths that start at $x_a$ and ends at
$x_b$. This sum (in our situation) is a continuous one, i.e. an
integral.

Recall that a ``\emph{functional}'' is a mathematical gadget that
``eats in'' a function and ``spits out'' a number. The integrand
in \eqref{eq:defnOfPathIntegral} is a functional, since it ``eats
in'' a path $x(t)$ and it ``spits out'' a complex number. The
argument of a functional will be written in square brackets
e.g. $F[x(t)]$, it usually is convention to do so. And just as a
function $y(t)$ can be integrated over a set of points, a
functional can be integrated  over a set of functions. The
measure of such a ``functional integral'' is notationally written
with a prefix $\mathcal{D}$. We can also differentiate a
functional with respect to its argument, and this ``functional
derivative'' is denoted by $\delta F/\delta x(t)$. 

\subsection{Functional Derivatives} 

Lets take some time to develop these notions a bit
further. We'll set up the analogous situation in vector
calculus. Consider the following differential situations
\begin{equation}%\label{eq:}
\frac{\partial}{\partial x_{i}}x_{j}=\delta_{ij},\quad\text{or}\quad\frac{\partial}{\partial x_{i}}\sum_{j}x_{j}k_{j} = k_{i}.
\end{equation}
When we think of a vector as a function from a discrete (usually
finite) set of ordinal numbers, we see that the indices are sort
of analogous to variables. Now consider the functional $F[x]$
where $x$ is either a single variable or a vector of variables,
it doesn't matter (the generalization to many variables is
straightforward). We expect by analogy that (in four dimensions)
\begin{equation}%\label{eq:}
\frac{\delta}{\delta f(x)}f(y) = \delta^{(4)}(x-y),\quad\text{or}\quad\frac{\delta}{\delta f(x)}\int f(y)\phi(y)d^{4}y = \phi(x).
\end{equation}
This gives us some intuitive understanding of the functional
derivative. The notion of a functional integral is questionable
in terms of mathematical rigor, but a good reference on the
subject is DeWitt-Morette et al.~\cite{cartier2006functional}

Now we have the bothersome problem that should be plaguing us: in
\eqref{eq:defnOfPathIntegral} what the devil is that ``phase''
term?! In answering this question, we should bear in mind we want
some limit which will reproduce classical mechanics. We thus
expect that only one path --- the classical one --- contributes
to the total amplitude. We may hope to evaluate the functional
integral by method of stationary phase, identifying the classical
path $x_{cl}(t)$ by the stationary condition
\begin{equation}%\label{eq:}
\left.\frac{\delta}{\delta x(t)}\left(\operatorname{phase}[x(t)]\right)\right|_{x_{cl}}=0.
\end{equation}
But the classical path is the one that satisfies the principle of
least action
\begin{equation}%\label{eq:}
\left.\frac{\delta}{\delta x(t)}\left(S[x(t)]\right)\right|_{x_{cl}} = 0,
\end{equation}
where $S=\int Ldt$ is the classical action. It is very tempting
to say that the phase is just $S$ up to some constant. Since the
stationary-phase approximation should be valid in the classical
limit --- i.e. when $S\gg\hbar$ --- we need to demand that the
phase is $S/\hbar$...or at least that's what we will use. We end
up with the conclusion that unitary time evolution in the path
integral formalism should thus be
\begin{equation}%\label{eq:}
U(x_a,x_b;T) = \left\<x_b\left|e^{i\widehat{H}t/\hbar}\right|x_a\right\> = \int \mathcal{D}x(t)e^{iS[x(t)]/\hbar}
\end{equation}
based off of our hand-wavy reasoning involving orthogonal
democratic paths and the superposition principle.

\subsection{Connecting Back to Canonical Formalism}

Now, we should be worried that this hand-waviness is complete hog
wash. We need to check somehow that we're on the right track, and
the usual way we do this is by recovering something from the
usual theory (i.e. the Hamiltonian formalism of quantum
mechanics). Consider our favorite double slit experiment as in
figure \ref{fig:doubleSlit}. The action for both paths would be
$(1/2)mv^{2}t$, the kinetic energy multiplied by time. For path
1, once it gets to the slit, it has to travel a distance $D$ in
time $t$, so it has a velocity 
\begin{equation}%\label{eq:}
v_{1} \approx \frac{D}{t}
\end{equation}
so the phase would thus be
\begin{equation}%\label{eq:}
S\approx\frac{mD^{2}}{2\hbar{t}}.
\end{equation}
For path 2, it travels a distance $d+D$ from the slit in time
$t$, so it has velocity
\begin{equation}%\label{eq:}
v_{2}\approx\frac{D+d}{t},
\end{equation}
and similarly the phase would be
\begin{equation}%\label{eq:}
S\approx\frac{m(D+d)^{2}}{2\hbar{t}}.
\end{equation}
We may assume for argument's sake that $d\ll D$ so that
\begin{equation}%\label{eq:}
(D+d)^{2} = D^{2}+2dD+\mathcal{O}(d^{2})\approx D^{2}+2dD
\end{equation}
and moreover $v_{1}\approx v_{2}$. But in our approximation, the
excess phase for path 2 is ``merely''
\begin{equation}%\label{eq:}
\frac{mDd}{\hbar{t}}=\left(\frac{mD}{t}\right)\frac{d}{\hbar}\approx\frac{pd}{\hbar}
\end{equation}
where $p$ is momentum. This is exactly what we should expect from
the de Broglie wave relations 
\begin{equation}%\label{eq:}
p = \frac{h}{\lambda}
\end{equation}
which is a good sign that we're on to something!


\begin{figure}[t]
\includegraphics{img/img.2}
\caption{The discretized approximation of the path integral}\label{fig:pathIntegralApprox}
\end{figure}


\subsection{Limit of Discretized Scheme.}
Recall for the usual integral, when it is first introduced in
college, we partition the domain of integration thus breaking the
integral up into a discrete sum. To further investigate the
functional integral, we'll do a similar discretization of the
time interval $[0,T]$. We actually did this albeit it sloppily,
we split it in two and demanded that there are only two possible
physical processes (through slit 1 or slit 2). By generalizing to
finitely many possible time steps and continuously many
positions, we expect to obtain more information of how the
functional integral works. We will divide the time interval into
$N$ intervals of duration $\varepsilon$, and approximate the
trajectory $x(t)$ by a sequence of straight lines (one line that
starts and ends at each time slice). Upon discretization, the
action becomes
\begin{equation}%\label{eq:}
S=\int^{T}_{0}\left(\frac{m}{2}\dot{x}^{2}-V(x)\right)dt\mapsto\sum_{k}\left[\frac{m}{2}\frac{(x_{k+1}-x_{k})^{2}}{\varepsilon^{2}}-V\left(\frac{x_{k+1}+x_{k}}{2}\right)\right]\varepsilon
\end{equation}
We then define the path integral by taking the limit
$\varepsilon\to0$ of
\begin{equation}\label{eq:limitDefinitionFunctionalIntegral}
\int\mathcal{D}x(t)\equiv\frac{1}{C(\varepsilon)}\int\frac{dx_{1}}{C(\varepsilon)}\int\frac{dx_{2}}{C(\varepsilon)}(\cdots)\int\frac{dx_{N-1}}{C(\varepsilon)}=\frac{1}{C(\varepsilon)}\prod_{k}\int\frac{dx_{k}}{C(\varepsilon)}
\end{equation}
where $C(\varepsilon)$ is some constant we'll worry about
later. We see this discretization depicted in figure
\ref{fig:pathIntegralApprox}. This scheme of setting up a
discretized partition then taking the continuum limit is used a
lot in functional quantization.



\subsection{Equivalence of Definitions}
Now the astute reader should be asking themselves ``Wait, we have
defined the function integral twice, what's up with that?'' This
is true, we first defined it in \eqref{eq:defnOfPathIntegral} as
\begin{equation*}
U(x_a,x_b;T) = \sum_{\substack{
\text{all paths}\\
\text{from}\\
\text{$x_a$ to $x_b$}
}} e^{i\cdot(\text{phase})}
= \int\mathcal{D}x(t) e^{i\cdot(\text{phase})}.
\end{equation*}
and again in eq
\eqref{eq:limitDefinitionFunctionalIntegral}. Using the second
definition, we will prove the validity of the first. Here's the
sketch of the proof: we'll show that both are obtained by
integrating the same differential equation with the same initial
condition. How we'll do this is by considering an individual
subinterval in our partition of time as we take the
$\varepsilon\to0$ limit. 

To derive the differential equation in question, we'll start by
considering the last time slice in our discrete approximation to
the path integral. According to both of our definitions, we
expect to have
\begin{equation}\label{eq:lastTimeSlice}
\begin{split}
U(x_a,x_b;T) = \int^{\infty}_{-\infty}\frac{dx'}{C(\varepsilon)}
\exp\left(\frac{i}{\hbar}\frac{m(x_b-x')^{2}}{2\varepsilon}-\frac{i\varepsilon}{\hbar}
V\left(\frac{x_{b}+x'}{2}\right)\right)\\
\times{U(x_a,x';T-\varepsilon)}
\end{split}
\end{equation}
The integral over $x'$ is just the contribution from the last
time slice, while the exponential is from the $e^{iS/\hbar}$ of
that slice. The contributions from the prior slices are contained
in $U(x_a,x';T-\varepsilon)$.

Now, we take the $\varepsilon\to0$ limit, and as we do so the
rapid oscillations of the first term in the exponential
constrains $x'$ to be ``very close'' to $x_b$. We can expand eq
\eqref{eq:lastTimeSlice} to be
\begin{equation}\label{eq:expansionOfLastTimeSliceAsEpsilonToZero}
\begin{split}
U(x_a,x_b;t) =
\int^{\infty}_{-\infty}\frac{dx'}{C(\varepsilon)}\exp\left(\frac{i}{\hbar}\frac{m(x_b-x')^{2}}{2\varepsilon}\right)[1-\frac{i\varepsilon}{\hbar}V(x_b)+\cdots]\\
\times [1  + (x'- x_b)\frac{\partial}{\partial x_{b}} + \frac{1}{2}(x'- x_b)^{2}\frac{\partial^{2}}{\partial x_{b}^{2}}+\cdots]U(x_a,x_b;T-\varepsilon) 
\end{split}
\end{equation}
Observe that we make use of Taylor expanding
$U(x_a,x_b;T-\varepsilon)$ on the right hand side, since
$x_b-x'\ll1$. We then look up a few Gaussian integrals
\begin{equation}%\label{eq:}
\int e^{-b\xi^{2}}d\xi = \sqrt{\frac{\pi}{b}},\quad\int\xi
e^{-b\xi^{2}}d\xi=0,\quad\int \xi^{2} e^{-b\xi^{2}}d\xi = \frac{1}{2b}\sqrt{\frac{\pi}{b}}
\end{equation}
and then apply these to our time slice to find
\begin{equation}%\label{eq:}
\begin{split}
U(x_a,x_b;T) =
\left(\frac{1}{C(\varepsilon)}\sqrt{\frac{2\pi\hbar\varepsilon}{-im}}\right)\left[1
  - \frac{i\varepsilon}{\hbar}V(x_b) +
  \frac{i\varepsilon\hbar}{2m}\frac{\partial^{2}}{\partial
    x_{b}^{2}} + \mathcal{O}(\varepsilon^2)\right]\\
\times U(x_a,x_b;T-\varepsilon).
\end{split}
\end{equation}
Observe that the parenthetic term causes problems as we take the
limit $\varepsilon\to0$. We demand that
\begin{equation}%\label{eq:}
C(\varepsilon) = \sqrt{\frac{2\pi\hbar\varepsilon}{-im}}
\end{equation}
so the parenthetic term becomes unity, and the rest of the
expression is well defined. If we now compare terms of order
$\varepsilon$ and multiply both sides by $i\hbar$ we find
\begin{subequations}
\begin{align}
i\hbar\frac{\partial}{\partial T}U(x_a,x_b;T)
&= \left[\frac{-\hbar^{2}}{2m}\frac{\partial^{2}}{\partial
    x_{b}^{2}}+V(x_{b})\right]U(x_{a},x_{b};T)\\
&= \hat{H}U(x_{a},x_{b};T).
\end{align}
\end{subequations}
This is the Schrodinger equation! This is great news, we have
another connection to canonical Quantum Mechanics. Observe that
$U$ as defined in eq \eqref{eq:canonicalUnitaryTimeEvolution}
satisfies the same Schrodinger equation.

As $T\to0$ the left hand side of \eqref{eq:defnOfPathIntegral}
tends to $\delta(x_a-x_b)$. Compare this to how our other
definition behaves in this limit, specifically in the case of one
time slice:
\begin{equation}%\label{eq:}
\frac{1}{C(\varepsilon)}\exp\left[\frac{i}{\hbar}\frac{m(x_b-x_a)^{2}}{2\varepsilon}+\mathcal{O}(\varepsilon)\right]
\end{equation}
This is just the peaked exponential of eq \eqref{eq:expansionOfLastTimeSliceAsEpsilonToZero}
\begin{equation*}%\label{eq:}
\begin{split}
U(x_a,x_b;t) =
\int^{\infty}_{-\infty}\frac{dx'}{C(\varepsilon)}\exp\left(\frac{i}{\hbar}\frac{m(x_b-x')^{2}}{2\varepsilon}\right)[1-\frac{i\varepsilon}{\hbar}V(x_b)+\cdots]\\
\times [1  + (x'- x_b)\frac{\partial}{\partial x_{b}} + \frac{1}{2}(x'- x_b)^{2}\frac{\partial^{2}}{\partial x_{b}^{2}}+\cdots]U(x_a,x_b;T-\varepsilon) 
\end{split}
\end{equation*}
and it also tends to $\delta(x_a-x_b)$ as $\varepsilon\to0$. Thus
we have both of our definitions of the path integral satisfy the
same differential equation with the same initial condition, which
necessarily implies they are the same. Moreover, we have shown
that the canonical expression with the Hamiltonian yields the
same results, so we have an explicit connection from the path
integral formalism back to the canonical formalism.

\subsection{Generalization to Multiple Dimensions} We have
considered thus far working in one dimension, but that was purely
for simplicity (the physics shouldn't change if we work in one or
one million dimensions; the math is just simpler in one). We are
now going to generalize the path integral to arbitrarily many
coordinates and momenta. We have $q^i$ for the position, and
$p_i$ for the momentum. We denote the Hamiltonian by $H(q,p).$
Note that when we don't use indices like this, it's to indicate
that we are working with all of the position or momenta
(depending on the context). The transition amplitude in the
canonical formalism should be
\begin{equation}%\label{eq:}
U(q_a,q_b;T) = \<q_b|e^{-i\widehat{H}T}|q_a\>.
\end{equation}
\textbf{Note:} we have been remarkably careful about explicitly
stating $\hbar$, we will now use the standard lazy convention of
setting $\hbar=1$.

We perform the same song and dance, beginning with a
discretization in time, specifically $N$ slices of duration
$\varepsilon$. We can write
\begin{equation}%\label{eq:}
e^{-i\widehat{H}T} = e^{-i\sum\varepsilon\widehat{H}} =
\underbrace{e^{-i\varepsilon\widehat{H}}e^{-i\varepsilon\widehat{H}}\cdots
  e^{-i\varepsilon\widehat{H}}}_{\text{($N$ times)}}
\end{equation}
The trick is to insert a complete set of intermediate states
between each of these factors. Here ``complete'' in the sense
that
\begin{equation}%\label{eq:}
\mathbf{1} = \left(\prod_{j}\int dq^{j}_{k}\right)|q_{k}\>\<q_{k}|.
\end{equation}
Note that subscripts refer to the time slice, and superscripts
refer to the components of the vector.
Inserting these factors for $k=1,\ldots,N-1$, we are left with a
product of factors of the form 
\begin{equation}\label{eq:factorsOfMatrixForHamiltonian}
\<q_{k+1}|e^{-i\widehat{H}\varepsilon}|q_{k}\>\xrightarrow[\varepsilon\to0]{}\<q_{k+1}|1-i\varepsilon\hat{H}+\mathcal{O}(\varepsilon^2)|q_{k}\>.
\end{equation}
To express the first and last factors as this form, we define
$q_0=q_a$ and $q_N=q_b$.

Now we will investigate $\hat{H}$. Specifically, we are concerned
with what terms it contains. The simplest Hamiltonian to consider
is when it is a function of position only, not of momenta. The
matrix element of such a term would be
\begin{equation}%\label{eq:}
\<q_{k+1}|f(q)|q_{k}\> = f(q_{k})\prod_{j}\delta(q^{j}_{k}-q^{j}_{k+1}).
\end{equation}
It'd be convenient to write this as
\begin{equation}%\label{eq:}
\<q_{k+1}|f(q)|q_{k}\> = f\left(\frac{q_{k}+q_{k+1}}{2}\right)\left(\prod_{j}\int\frac{dp^{j}_{k}}{2\pi}\right)\exp[i\sum_{j}p^{j}_{k}(q^{j}_{k+1}-q^{j}_{k})],
\end{equation}
for reasons that will hopefully be obvious soon.

We will, for the sake of symmetry, consider a Hamiltonian which
is purely a function of momenta. We introduce a complete set of
momentum eigenstates to obtain
\begin{equation}%\label{eq:}
\<q_{k+1}|f(p)|q_{k}\> = \left(\prod_{j}\int\frac{dp^{j}_{k}}{2\pi}\right)f(p_{k})\exp[i\sum_{j}p^{j}_{k}(q^{j}_{k+1}-q^{j}_{k})]
\end{equation}
So if $\hat{H}$ contains only terms of the form $f(q)$ and
$f(p)$, its matrix element can be written
\begin{equation}\label{eq:matrixElementForHamiltonianGuessOne}
\begin{split}
&\<q_{k+1}|\hat{H}(q,p)|q_{k}\> = \\ &
\left(\prod_{j}\int\frac{dp^{j}_{k}}{2\pi}\right)\hat{H}
\left(\frac{q_{k+1}+q_{k}}{2},p_{k}\right)
\exp[i\sum_{j}p^{j}_{k}(q^{j}_{k+1}-q^{j}_{k})].
\end{split}
\end{equation}

In general, eq \eqref{eq:matrixElementForHamiltonianGuessOne} is
wrong. We usually have mixed terms of order
$\mathcal{O}(q^{\alpha}p^{\beta})$ in the Hamiltonian. So why
bother considering Hamiltonians of such a form? Because we can
pick a sufficiently nice ordering that makes it true in
general. For example, this combination
\begin{equation}%\label{eq:}
\<q_{k+1}|\frac{1}{4}(q^{2}p^{2}+2qp^{2}q+p^{2}q^{2})|q_{k}\>=\left(\frac{q_{k+1}+q-{k}}{2}\right)^{2}\<q_{k+1}|p^{2}|q^{k}\>
\end{equation}
works out as desired. This is due to the $q$'s appearing
symmetrically on the left hand side and the right hand side in
just the right way. When this happens, we say that the
Hamiltonian is ``\emph{Weyl ordered}''\index{Weyl Ordering}. Any
Hamiltonian can be Weyl ordered by commuting $p$'s and $q$'s; in
general this procedure will introduce extra terms which need to
be put on the right hand side of eq \eqref{eq:matrixElementForHamiltonianGuessOne}.
For more on Weyl ordering, see Ticciatic~\cite{Ticciati:1999qp}
(specifically pages 335 et seq.).

Supposing that from now on (unless explicitly states otherwise)
$\hat{H}$ is Weyl ordered, our typical matrix element from \eqref{eq:factorsOfMatrixForHamiltonian}
can be expressed as
\begin{equation}%\label{eq:}
\<q_{k+1}|e^{-i\varepsilon\widehat{H}}|q_{k}\> = \left(\prod_{j}\int\frac{dp^{j}_{k}}{2\pi}\right)\exp\left[-i\varepsilon\hat{H}\left(\frac{q_{k+1}+q_{k}}{2},p_{k}\right)\right]\exp[i\sum_{j}p^{j}_{k}(q^{j}_{k+1}-q^{j}_{k})].
\end{equation}
(We use the well known that that as $\epsilon\to0$,
$\epsilon\ll1$.) To obtain $U(q_a,q_b;T)$ we multiply $N$ such
factors together (one for each $k$) and integrate on our
intermediate coordinates $q_{k}$:
\begin{equation}%\label{eq:}
\begin{split}
U(q_0,q_N;T) &= \left(\prod_{j,k}\int
dq^{j}_{k}\int\frac{dp^{j}_{k}}{2\pi}\right)\\
&\times\exp\left[i\sum_{k}\left(\sum_{j}p^{j}_{k}(q^{j}_{k+1}-q^{j}_{k})-\varepsilon\hat{H}(\frac{q_{k+1}+q_{k}}{2},p_{k})\right)\right].
\end{split}
\end{equation}
There is one momentum integral for each $k$ from 0 to $N-1$, and
one coordinate integral for each $k$ from 1 to $N-1$.

This expression, we deduce, is the discretized form of
\begin{equation}\label{eq:functionalProductManyDimensions}
U(q_a,q_b;T) = \left(\prod_{j}\int\mathcal{D}q(t)\mathcal{D}p(t)\right)\exp\left[i\int^{T}_{0}(\dot{q}^{i}p_{i}-H(q,p))dt\right],
\end{equation}
where the functions $q(t)$ are constrained at the endpoints, but
the functions $p(t)$ are not. Note that we don't have any whacky
constants in the integration measure $\mathcal{D}q$, unlike the
situation in one dimension. The functional measure in eq
\eqref{eq:functionalProductManyDimensions} is just the product of
the standard integral over the phase space
\begin{equation}%\label{eq:}
\prod_{j}\int\frac{dq^{j}dp_{j}}{2\pi}
\end{equation}
at each point in time. We can conclude that eq
\eqref{eq:functionalProductManyDimensions} is the most general
formula for computing transition amplitudes via functional
integration for practical purposes.

%\subsection


\section{Path Integral Quantization of Harmonic Oscillator}
%%
%% functionalHarmonicOscillator.tex
%% 
%% Made by Alex Nelson
%% Login   <alex@tomato>
%% 
%% Started on  Sat Aug 15 12:02:21 2009 Alex Nelson
%% Last update Sat Aug 15 12:02:21 2009 Alex Nelson
%%

The functional quantization of the Harmonic oscillator provides a
powerful example when generalizing to fields, so it is worth
while to first study it in detail. Recall the Hamiltonian of the
Harmonic oscillator is
\begin{equation}%\label{eq:}
H = \frac{p^{2}}{2m} + \frac{m}{2}\omega^{2}q^{2}
\end{equation}
In the presence of an external force, this yeilds the path
integral expression
\begin{equation}%\label{eq:}
\<0|0\>_{f} = \int\mathcal{D}p\mathcal{D}q\exp\left[i\int^{\infty}_{-\infty}(p\dot{q}-(1-i\varepsilon)H+fq)dt\right]
\end{equation}
where $\varepsilon$ is an ``infinitesimal'' quantity. We see by
inspection that multiplication of $H$ by $(1-i\varepsilon)$ is
completely equivalent as \emph{BOTH}
\begin{subequations}
\begin{align}
m^{-1}&\mapsto(1-i\varepsilon)m^{-1}\\
\frac{m}{2}\omega^{2}q^{2}&\mapsto\frac{(1-i\varepsilon)m}{2}\omega^{2}q^{2}.
\end{align}
\end{subequations}
So this means that
\begin{equation}%\label{eq:}
\frac{p^{2}}{2m}=\frac{m}{2}\dot{q}^{2}\mapsto \frac{(1+i\varepsilon)m}{2}\dot{q}^{2}
\end{equation}
thus transforming the path integral expression to be
\begin{equation}\label{eq:pathIntegralExpressionForHarmonicOscillator}
\<0|0\>_{f} =  \int\mathcal{D}q\exp\left[i\int^{\infty}_{-\infty}(\frac{m}{2}(1-i\varepsilon)\dot{q}^{2}-(1-i\varepsilon)\frac{m}{2}\omega^{2}q^{2}+fq)dt\right].
\end{equation}
We will henceforth set $m=1$.

We find the Fourier transformed quantities
\begin{equation}%\label{eq:}
\widetilde{q}(E) = \int^{\infty}_{-\infty}q(t)e^{iEt}dt,\qquad q(t)=\int^{\infty}_{-\infty}\widetilde{q}(E)e^{-iEt}\frac{dE}{2\pi}
\end{equation}
Thus we can compute
\begin{subequations}
\begin{align}
q(t)^{2} &= \left(\int^{\infty}_{-\infty}\widetilde{q}(E)e^{iEt}\frac{dE}{2\pi}\right)\left(\int^{\infty}_{-\infty}\widetilde{q}(E')e^{iE't}\frac{dE'}{2\pi}\right)\\
&= \int^{\infty}_{-\infty}\int^{\infty}_{-\infty} e^{-i(E+E')t}\widetilde{q}(E)\widetilde{q}(E')\frac{dE}{2\pi}\frac{dE'}{2\pi}
\end{align}
\end{subequations}
and
\begin{equation}%\label{eq:}
\dot{q}(t)^{2} = \int^{\infty}_{-\infty} -EE'e^{-i(E+E')t}\widetilde{q}(E)\widetilde{q}(E')\frac{dE}{2\pi}\frac{dE'}{2\pi}.
\end{equation}
We also compute
\begin{subequations}
\begin{align}
f(t)q(t) &= \frac{1}{2}(f(t)q(t)+q(t)f(t))\\
&= \frac{1}{2}\int\left(\widetilde{f}(E)\widetilde{q}(E')+\widetilde{f}(E')\widetilde{q}(E)\right)e^{-i(E+E')t}\frac{dE}{2\pi}\frac{dE'}{2\pi}.
\end{align}
\end{subequations}
We can plug these into eq \eqref{eq:pathIntegralExpressionForHarmonicOscillator}
to find
\begin{subequations}
\begin{alignat}{3}
\<0|0\>_{f} &=&
\int\mathcal{D}q\exp\Big[
\int[\left(\frac{(1+i\varepsilon)m}{2}(-EE')-\frac{(1-i\varepsilon)m\omega^{2}}{2}\right)\widetilde{q}(E)\widetilde{q}(E')\nonumber\\
& &+ \frac{1}{2}\left(\widetilde{f}(E)\widetilde{q}(E')+\widetilde{f}(E')\widetilde{q}(E)\right)]\underbrace{e^{-i(E+E')t}dt}_{=\delta(E+E)}\frac{dE}{2\pi}\frac{dE'}{2\pi}
\Big]\\
&=&
\int\mathcal{D}q\exp\Big[
\int[\left(\frac{(1+i\varepsilon)m}{2}(-EE')-\frac{(1-i\varepsilon)m\omega^{2}}{2}\right)\widetilde{q}(E)\widetilde{q}(E')\nonumber\\
& &+ \frac{1}{2}\left(\widetilde{f}(E)\widetilde{q}(E')+\widetilde{f}(E')\widetilde{q}(E)\right)]\delta(E+E')\frac{dE}{2\pi}\frac{dE'}{2\pi}
\Big]\\
&=&
\int\mathcal{D}q\exp\left[
\int[\left(\frac{(1+i\varepsilon)m}{2}(E^{2})-\frac{(1-i\varepsilon)m\omega^{2}}{2}\right)\widetilde{q}(E)^{2}
  + \frac{1}{2}\widetilde{f}(E)\widetilde{q}(E)]\frac{dE}{2\pi}
\right]
\end{alignat}
\end{subequations}
This simplifies things slightly.

Lets make a change of variables:
\begin{equation}%\label{eq:}
\widetilde{x}(E) = \widetilde{q}(E) + \frac{\widetilde{f}(E)}{E^{2}-\omega^{2}+i\varepsilon}
\end{equation}
which allows us to write the action as
\begin{equation}%\label{eq:}
S = \frac{1}{2}\int^{\infty}_{-\infty}\left[\widetilde{x}(E)(E^{2}-\omega^{2}+i\varepsilon)\widetilde{x}(-E)-\frac{\widetilde{f}(E)\widetilde{f}(-E)}{E^{2}-\omega^{2}+i\varepsilon}\right]\frac{dE}{2\pi}
\end{equation}
But how does this change the measure? Well, according to
mathematica, if we suppose that the expression for the external
force looks like
\begin{equation}%\label{eq:}
\widetilde{f}(E) = const
\end{equation}
then the measure changes by 
\begin{equation}%\label{eq:}
dx = \left(1+\frac{d|q|}{dq}e^{\sqrt{-\sgn(k)} k |q|} \sqrt{-\frac{pi}{2k^{2}}}\right)dq
\end{equation}
which is just a shift by a constant really. So we find that
\begin{equation}%\label{eq:}
\mathcal{D}x = \mathcal{D}q.
\end{equation}
We see that this makes the path integral expression
\begin{equation}%\label{eq:}
\<0|0\>_{f} =
\underbracket[0.25pt]{\exp\left[\frac{-i}{2}\int^{\infty}_{-\infty}\frac{\widetilde{f}(E)\widetilde{f}(-E)}{E^{2}-\omega^{2}+i\varepsilon}\right]}_{\text{dependent on $\widetilde{f}(E)$, external force}}
\underbracket[0.25pt]{\int\mathcal{D}x\exp\left[\frac{i}{2}\int^{\infty}_{-\infty}\widetilde{x}(E)(E^{2}-\omega^{2}+i\varepsilon)\widetilde{x}(-E)\frac{dE}{2\pi}\right]}_{\text{independent of external force}}
\end{equation}
So here's the trick: we have two factors, one dependent of the
external force, the other independent of the external force. So
in a vacuum (i.e. a free Harmonic oscillator) one has $f=0$. This
occurs for $\<0|0\>_{f}$. But a Harmonic oscillator that is not
acted on by an external force remains in its ground state
$\<0|0\>_{f}=1$. Thus we find
\begin{equation}%\label{eq:}
\<0|0\>_{f} = \exp\left[\frac{-i}{2}\int^{\infty}_{-\infty}\frac{\widetilde{f}(E)\widetilde{f}(-E)}{E^{2}-\omega^{2}+i\varepsilon}\right].
\end{equation}
We want a more intuitive interpretation of this, so it requires
performing the Fourier transform to the time domain, yielding
\begin{equation}%\label{eq:}
\<0|0\>_{f} = \exp\left[\frac{i}{2}\int f(t)G(t-t')f(t')dt~dt'\right]
\end{equation}
where we have introduced the shorthand notation
\begin{equation}%\label{eq:}
G(t-t') = \int\frac{e^{-iE(t-t')}}{E^{2}-\omega^{2}+i\varepsilon}\frac{dE}{2\pi}
\end{equation}
which is the Green's function for the Harmonic oscillator
equation of motion. That is, it satisfies
\begin{equation}%\label{eq:}
\left(\frac{d^{2}}{dt^{2}}+\omega^{2}\right)G(t-t') = \delta(t-t'),
\end{equation}
which can be computed and verified by direct substitution, then
taking the $\varepsilon\to0$ limit. We can also use the method of
residues to find
\begin{equation}%\label{eq:}
G(t-t')=2\pi
i\lim_{E\to\omega}(E-\omega)\frac{1}{2\pi}\frac{e^{-iE(t-t')}}{(E+\omega)(E-\omega)}
= \frac{i}{2\omega}\exp\big[-i\omega|t-t'|\big].
\end{equation}
The trick with plugging this into the Harmonic oscillator
equation of motion is that the absolute value, when
differentiated, yields the Step function...which when
differentiated yields a delta function.


\section{Functional Quantization of Scalar Fields}
%%
%% functionalScalar.tex
%% 
%% Made by Alex Nelson
%% Login   <alex@tomato>
%% 
%% Started on  Sat Aug 15 13:06:34 2009 Alex Nelson
%% Last update Sat Aug 15 13:06:34 2009 Alex Nelson
%%

We would like to generalize our functional methods to field
theory by generalizing the parametrization from a 1 dimensional
parameter (time $t$) to a 4 dimensional parametrization
$x^{\mu}$. So this means that
\begin{align*}
t&\to x^{\mu}\\
\frac{d}{dt}&\to \partial_{\mu}\\
q(t)&\to \varphi(x^{\mu})
\end{align*}
In fact we can set up a dictionary for our generalization
\begin{center}
\begin{tabular}{|c|c|}\hline
mechanics          & field theory\\\hline
$q(t)$             & $\varphi(\bar{x},t)$\qquad{classical field}\\
$\widehat{Q}$      & $\varphi(\bar{x},t)$\qquad{operator field}\\
$f(t)$             & $J(\bar{x},t)$\qquad{classical source}\\\hline
\end{tabular}
\end{center}
We can write the Hamiltonian density for the free scalar field as
\begin{equation}%\label{eq:}
\mathcal{H}_{0} = 
\underbracket[0.25pt]{\frac{1}{2}\Pi^{2} +
\frac{1}{2}(\nabla\varphi)^{2}}_{\mathclap{\substack{=\partial^{\mu}\varphi\partial_{\mu}\varphi\\
=\text{kinetic energy}}}}+
\underbracket[0.25pt]{\frac{1}{2}m^{2}\varphi^{2}}_{\mathclap{\text{potential}}}
\end{equation}
The trick we've employed so many times before using
infinitesimals is done precisely the same way using the
Hamiltonian density. That is,
$\mathcal{H}_{0}\mapsto(1-i\varepsilon)\mathcal{H}_{0}$. For
simplicity, we tacitly assume we always mean
$(1-i\varepsilon)m^{2}$ whenever we write $m^{2}$, it's
completely equivalent to
$\mathcal{H}_{0}\mapsto(1-i\varepsilon)\mathcal{H}_{0}$. 

Consider the functional integral
\begin{equation}%\label{eq:}
Z[J] = \<0|0\>_{J} = \int\mathcal{D}\varphi\exp\left[i\int[\mathcal{L}_{0}+J\varphi]d^{4}x\right]
\end{equation}
where
\begin{equation}%\label{eq:}
\mathcal{L}_{0} =
\frac{1}{2}\partial^{\mu}\varphi\partial_{\mu}\varphi - \frac{1}{2}m^{2}\varphi^{2}
\end{equation}
is the Lagrangian density and
\begin{equation}%\label{eq:}
\mathcal{D}\varphi\propto
\prod_{\mathclap{\substack{
\text{space-}\\
\text{time}\\
\text{points $x$}}}}
d\varphi(x)
\end{equation}
is the functional measure.

\subsubsection{Remark on Peskin and Schroeder's ``Derivation''.}
In Chapter 9 of Peskin and Schroeder~\cite{Peskin:1995ev} there
is a ``derivation'' of the functional quantization of the scalar
field using the limit of a discrete lattice. There is some
mathematically dubious aspects of this derivation, for one they
use a Fourier series when they should be using a discrete Fourier
transform. What difference does this make? Well, the intuition of
the discrete Fourier transform is that the field exists only on
the nodes of the lattice; the intuition of a Fourier series is
that it exists everywhere continuously in a finite universe. The
latter is not in spirit with taking the continuum limit! In fact,
this beautifully motivates us to read the paper by Wise on chain field
theory~\cite{Wise:2005mp}, which resolves this problem by a
topological approach which yields the correct continuum limit by
taking advantage of recent work on discrete differential geometry.

\subsection{Slick Method of Computing Correlation Functions}
%%
%% functionalScalarFieldSlick.tex
%% 
%% Made by Alex Nelson
%% Login   <alex@tomato>
%% 
%% Started on  Sat Aug  1 12:23:40 2009 Alex Nelson
%% Last update Sat Aug  1 12:23:40 2009 Alex Nelson
%%
\subsubsection{Outline}
As promised, we'll introduce a slicker way to compute Feynman
rules using functional derivatives. It's a lot more
mathematically rigorous (and simpler) than the discretization
scheme. The method uses a mathematical gadget which generalizes
the notion of a generating function --- the generating
\emph{functional}. Recall the generating function is used to
compute constants and other useful numbers by taking its
$n^{\text{th}}$ derivative and evaluating it at 0. We
\emph{functionally} do the same thing, take the functional
derivative of the generating functional and evaluate it at 0.


\subsubsection{Properties of the Functional Derivative}
First lets try to review the properties of the functional
$\delta/\delta J(x)$. The functional derivative obeys the basic
property (in four dimensions)
\begin{equation}%\label{eq:}
\frac{\delta}{\delta J(x)}J(y)=\delta^{(4)}(x-y),
\quad\text{or}\quad
\frac{\delta}{\delta J(x)}\int J(y)\phi(y)d^{4}y = \phi(x).
\end{equation}
This can be viewed as a continuous generalization of the vector
calculus derivative 
\begin{equation}%\label{eq:}
\frac{\partial}{\partial x_i}x_j = \delta_{ij}
\quad\text{or}\quad
\frac{\partial}{\partial x_i}\sum_{j}x_{j}k_{j}=k_{i}.
\end{equation}
To take the functional derivatives of more complicated
situations, we use the basic properties of the chain rule and the
product rule. \textbf{Warning:} we \textbf{assume without proof}
that these properties hold, we'll not divulge into the proof
here. So we have situations like the following:
\begin{equation}%\label{eq:}
\frac{\delta}{\delta J(x)}\exp\left[i\int J(y)\phi(y)d^{4}y\right]
=i\phi(x)\exp\left[i\int J(y)\phi(y)d^{4}y\right].
\end{equation}
When the functional depends on the derivative of $J$ we integrate
by parts --- and for all practical purposes, we always can
integrate by parts in quantum field theory --- then apply the
functional derivative as follows:
\begin{equation}%\label{eq:}
\frac{\delta}{\delta J(x)}\int V^{\mu}(y)\partial_{\mu}J(y)d^{4}y
=
\frac{\delta}{\delta J(x)}\left(
\operatorname{bdry terms} + 
\int J(y)\partial_{\mu}V^{\mu}(y)d^{4}y
\right)
= - \partial_{\mu}V^{\mu}(x).
\end{equation}
This concludes our review of basic properties that we'll use
later on.

\subsubsection{Generating Functional}

As alluded to earlier, the basic object of interest is the
generating functional of correlation functions. We denote this
object of interest by $Z[J]$. In a scalar field theory, it's
defined as
\begin{equation}\label{eq:scalarFieldTheoryGeneratingFunctional}
Z[J] \stackrel{\text{def}}{=}\int\mathcal{D}\phi\exp\left[
i\int[\mathcal{L}+J(x)\phi(x)]d^{4}x
\right].
\end{equation}
This is a functional integral over $\phi$. We've merely added to
$\mathcal{L}$ in the exponent an extra term $J(x)\phi(x)$, which
we usually refer to as the ``\emph{source term}''.

\subsubsection{Derivation of Correlation Function}
Now we use it to compute the generating functions for the
Klein-Gordon field (the free scalar field). For example the
two-point function is
\begin{equation}%\label{eq:}
\<0|T\{\phi(x_1)\phi(x_2)\}|0\> = \left.\frac{1}{Z_0}
\left(-i\frac{\delta}{\delta J(x_{1})}\right)
\left(-i\frac{\delta}{\delta J(x_{2})}\right)
Z[J]\right|_{J=0}
\end{equation}
where $Z_0=Z[0]$. Each functional derivative brings down a factor
of $\phi$ in the numerator of $Z[J]$; setting $J=0$ we recover
our desired expression. In more explicit detail, we can compute
\begin{subequations}
\begin{align}
\frac{\delta}{\delta J(x_2)}Z[J] &= \frac{\delta}{\delta J(x_2)}
\int\mathcal{D}\phi\exp\left[
i\int[\mathcal{L}+J(x)\phi(x)]d^{4}x
\right] \\
&= \int\mathcal{D}\phi\frac{\delta}{\delta J(x_2)}
\exp\left[
i\int[\mathcal{L}+J(x)\phi(x)]d^{4}x
\right] \\
&= \int\mathcal{D}\phi i\phi(x_2)
\exp\left[
i\int[\mathcal{L}+J(x)\phi(x)]d^{4}x
\right]
\end{align}
\end{subequations}
This is the effect of one functional derivative, we multiply by
$-i$ to finish one functional operation. We need to do another to
get the expression
\begin{equation}%\label{eq:}
\left(-i\frac{\delta}{\delta J(x_{1})}\right)
\left(-i\frac{\delta}{\delta J(x_{2})}\right)
Z[J]
= \int\mathcal{D}\phi \phi(x_1)\phi(x_2)
\exp\left[
i\int[\mathcal{L}+J(x)\phi(x)]d^{4}x
\right].
\end{equation}
To get the final expression, we need to divide by $Z_0$ and
evaluate at $J=0$ to get
\begin{equation*}%\label{eq:}
\left.\frac{1}{Z_0}
\left(-i\frac{\delta}{\delta J(x_{1})}\right)
\left(-i\frac{\delta}{\delta J(x_{2})}\right)
Z[J]\right|_{J=0} =\displaystyle{
\frac{\displaystyle\int\mathcal{D}\phi\; \phi(x_1)\phi(x_2)\exp\left[i\int[\mathcal{L}]d^{4}x\right]}
{\displaystyle\int\mathcal{D}\phi\exp\left[i\int\mathcal{L}d^{4}x\right]}}
\end{equation*}
which is precisely what is expected.

\subsubsection{Slicker Way to Compute Correlation Functions}
We've seen that eq
\eqref{eq:scalarFieldTheoryGeneratingFunctional} recovers the
expected expression for two-point functions. It's pretty nifty
for us since the free scalar field can be written fairly
easily. It's explicitly written after integrating by parts (on
the first term)
\begin{equation}\label{eq:slickPartialIntegration}
\int[\mathcal{L}_{0}(\phi)+J\phi]d^{4}x 
= 
\int[\frac{1}{2}\phi(-\partial^{2}-m^{2}+i\varepsilon)\phi+J\phi]d^{4}x.
\end{equation}
The factor of $i\varepsilon$ is to guarantee convergence. We
complete the square by introducing a shifted scalar field
\begin{equation}%\label{eq:}
\phi'(x)\stackrel{\text{def}}{=}\phi(x)-i\int D_{F}(x-y)J(y)d^{4}y
\end{equation}
where $D_{F}(x-y)$ is the Feynman propagator --- i.e. the Green's
function of the Klein-Gordon operator, we find that our original
expression \eqref{eq:slickPartialIntegration} becomes
\begin{equation}%\label{eq:}
\begin{split}
\int[&\mathcal{L}_{0}(\phi)+J\phi]d^{4}x =\\
&\int[\frac{1}{2}\phi'(-\partial^{2}-m^{2}+i\varepsilon)\phi']d^{4}x
-\int\frac{1}{2}J(x)\left[-iD_{F}(x-y)\right]J(y)d^{4}y.
\end{split}
\end{equation}
More symbolically, we could rewrite the change of variables as
\begin{equation}%\label{eq:}
\phi'\stackrel{\text{def}}{=}\phi+(-\partial^{2}-m^{2}+i\varepsilon)^{-1}J,
\end{equation}
and the result becomes
\begin{equation}%\label{eq:}
\int[\mathcal{L}_0 + J\phi]d^{4}x = 
\int\left[\frac{1}{2}\phi'(-\partial^{2}-m^{2}+i\varepsilon)\phi'-\frac{1}{2}J(-\partial^{2}-m^{2}+i\varepsilon)^{-1}J\right]d^{4}x.
\end{equation}
This looks nasty, but we are not done yet. We have a few tricks
left.

When we change variables from $\phi$ to $\phi'$, it's just a
shift, so the Jacobian in the functional integral definition of
$Z[J]$ is the identity. The result is
\begin{equation}%\label{eq:}
\underbrace{\int\mathcal{D}\phi'
\exp\left[i\int\mathcal{L}_{0}(\phi')d^{4}x\right]}_{Z_{0}}
\underbrace{\exp\left[-i\int\frac{1}{2}J(x)[-iD_{F}(x-y)J(y)]d^{4}xd^{4}y\right]}_{\text{independent of }\phi'}
\end{equation}
As noted, the second integral is independent of $\phi'$ and the
first is precisely $Z_{0}$. The generating functional of the free
scalar field is thus
\begin{equation}\label{eq:generatingFunctionPostManip}
Z[J] = Z_{0}\exp\left[\frac{-1}{2}\int J(x)D_{F}(x-y)J(y)d^{4}xd^{4}y\right]
\end{equation}
Lets use \eqref{eq:scalarFieldTheoryGeneratingFunctional} and
\eqref{eq:generatingFunctionPostManip} to compute some
correlation functions. The two-point function is by definition
\begin{equation}%\label{eq:}
\<0|T\{\phi(x_1)\phi(x_2)\}|0\>
=\left.
\frac{-\delta}{\delta J(x_{1})}
\frac{\delta}{\delta J(x_{2})}
\exp\left[\frac{-1}{2}\int J(x)D_{F}(x-y)J(y)d^{4}xd^{4}y\right]\right|_{J=0}
\end{equation}
We can evaluate one functional derivative to find
\begin{equation}%\label{eq:}
\<0|T\{\phi(x_1)\phi(x_2)\}|0\>
=\left.
\frac{-\delta}{\delta J(x_{1})}
\left[\frac{-1}{2}\int D_{F}(x_{2}-y)J(y)d^{4}y 
-\frac{1}{2}\int J(x)D_{F}(x-x_{2})d^{4}x
\right]
\frac{Z[J]}{Z_{0}}\right|_{J=0}
\end{equation}
\begin{comment}
We can see that the negatives cancel out nicely, we end up with 2
terms that effectively look like
\begin{equation*}%\label{eq:}
\frac{-\delta}{\delta J(x_{1})}
\left[\frac{-1}{2}\int D_{F}(x_{2}-y)J(y)d^{4}y 
-\frac{1}{2}\int J(x)D_{F}(x-x_{2})d^{4}x
\right]
\propto
\frac{\delta}{\delta J(x_1)}\int D_{F}(x_2-y)J(y)d^{4}y
\end{equation*}
\end{comment}
We can see that this is just what happens after evaluating one
functional derivative of the definition of the two-point
function, which allows us to conclude that
\begin{equation}%\label{eq:}
\<0|T\{\phi(x_1)\phi(x_2)\}|0\> = D_{F}(x_1-x_2).
\end{equation}
This is good because it connects back to what we should already know.

\subsubsection{Example: Four Point Correlation Functions}

This is a rather space-consuming computation, so we need to
introduce some abbreviations. Namely we'll use the conventions
that arguments of functions are subscripts: $\phi_1 = \phi(x_1)$,
$J_x=J(x)$, $D_{x4}=D_{F}(x-x_{4})$, etc. Repeated subscripts
will be integrated over implicitly (like a continuous Einstein's
summation convention). The four-point function is thus
\begin{subequations}
\begin{align}
\<0|T\{\phi_1\phi_2\phi_3\phi_4\}|0\> 
&=\left. \frac{\delta}{\delta J_{1}}
\frac{\delta}{\delta J_{2}}
\frac{\delta}{\delta J_{3}}
[-J_{x}D_{x4}]e^{-\frac{1}{2}J_{x}D_{xy}J_{y}}\right|_{J=0}\\
&=\left. \frac{\delta}{\delta J_{1}}
\frac{\delta}{\delta J_{2}}
[-D_{34}+J_{x}D_{x4}J_{y}D_{y3}]e^{-\frac{1}{2}J_{x}D_{xy}J_{y}}\right|_{J=0}\\
=&\left. \frac{\delta}{\delta J_{1}}
[D_{34}J_{x}D_{x2}+D_{24}J_{y}D_{y3}+J_{x}D_{x4}J_{y}D_{23}]e^{-\frac{1}{2}J_{x}D_{xy}J_{y}}\right|_{J=0}\\
&=D_{34}D_{12}+D_{24}D_{13}+D_{14}D_{23}
\end{align}
\end{subequations}
which is precisely what we expect by Wick's theorem.

\subsubsection{The Beauty of the Generating Functional}

The beauty of the situation is that these calculations are
completely independent of whether things are free or
interacting. The catch is the $Z[J=0]$ factor is not trivial in
the interacting situation. It gives us the sum of the vacuum
diagrams.


%%%
%% quantizationScalarField.tex
%% 
%% Made by Alex Nelson
%% Login   <alex@tomato>
%% 
%% Started on  Tue Jul 28 12:07:19 2009 Alex Nelson
%% Last update Tue Jul 28 12:07:19 2009 Alex Nelson
%%

We will apply what we've learned about functional techniques of
quantization to the scalar field. The real aim here is to
demonstrate how to derive Feynman rules.

\subsection{Generalizing from Particles to Fields}

We recklessly claimed that eq
\eqref{eq:functionalProductManyDimensions} holds for any quantum
system (if not, we're claiming it now). So it should hold for a
quantum field theory. In our situation --- the beloved 
scalar field --- the position coordinates $q^i$ are replaced by
the scalar field amplitudes $\phi(\bar{x})$, and the Hamiltonian
is
\begin{equation}%\label{eq:}
H = \int[\frac{1}{2}\pi^{2}+\frac{1}{2}(\nabla\phi)^2+V(\phi)]d^{3}x.
\end{equation}
Thus making the suitable changes, our formula becomes
\begin{equation}%\label{eq:}
\begin{split}
\<\phi_{b}(\bar{x})|&e^{-iHT}|\phi_{a}(\bar{x})\> = \\&\int\mathcal{D}\phi\mathcal{D}\pi\exp\left[i\int^{T}_{0}[\pi\dot{\phi}-\frac{1}{2}\pi^{2}-\frac{1}{2}(\nabla\phi)^{2}-V(\phi)]d^{4}x\right]
\end{split}
\end{equation}
where the functions $\phi(x)$ over which we integrate are
constrained to the specific configurations $\phi_a(\bar{x})$ at
$x^0=0$ and $\phi_b(\bar{x})$ at $x^0=T$. We observe that the
exponential is quadratic in $\pi$ (the canonically conjugate
momenta to the field $\phi$), so we can complete the square and
evaluate to find
\begin{equation}\label{eq:pathIntegralForScalarFieldInGeneral}
\<\phi_b(\bar{x})|e^{-iHT}|\phi_{a}(\bar{x})\> = \int\mathcal{D}\phi\exp\left[i\int^{T}_{0}\mathcal{L}d^{4}x\right],
\end{equation}
where $\mathcal{L} = \frac{1}{2}(\partial_{\mu}\phi)^{2}-V(\phi)$
is the Lagrangian density. The masure $\mathcal{D}\phi$ here
again involves odd constants.

The time integral in the exponent of eq \eqref{eq:pathIntegralForScalarFieldInGeneral}
goes from 0 to $T$ as determined by our choice of what transition
function to computes is, in all other respects, ``manifestly
Lorentz invariant'' \textbf{we claim without proof.} Any other
symmetries that the Lagrangian has are also explicitly preserved
by the functional integral (again, claimed without
proof). Symmetries take an increasingly important role in quantum
field theory, so we take the following (at first seemingly rash)
step: \emph{completely abandon Hamiltonian dynamics} and claim
that eq \eqref{eq:pathIntegralForScalarFieldInGeneral}
\emph{DEFINES} the Hamiltonian dynamics. Any such formula
corresponds to ``some'' Hamiltonian; and we can alsways
differentiate with respect to $T$ to derive the Schrodinger
equation to figure out which, as we did in the previous
section. We can thus consider the Lagrangian $\mathcal{L}$ as the
most fundamental specification of a quantum field
theory\footnote{Although recently there has been some ``blasphemous'' dabbling in working \emph{sans} Lagrangian or Hamiltonian, see Kochan~\cite{Kochan:2008}.}
We'll focus on working with the functional integral to compute
from $\mathcal{L}$ directly, no dependence on $H$ at all.

\subsection{Correlation Functions} In Quantum Field theories, we
usually work with vacuum states denoted as $|0\>$ or sometimes
$|\Omega\>$. There is a special function of interest which can be
interpreted as the amplitude for proagation of a particle or
excitation from $y$ to $x$, we call them ``\emph{Correlation Functions}''.
Here we will often denote them by
\begin{equation}%\label{eq:}
\<\Omega|T\{\phi(x)\phi(y)\}|\Omega\>
\end{equation}
where $T\{-\}$ is the time ordering operator\footnote{The
  interested reader is invited to read Ticciati page
  83~\cite{Ticciati:1999qp}.}, we can remember what it does by
the mnemonic ``Later goes to the left'':
\begin{equation}%\label{eq:}
T\{\phi(x)\phi(y)\} = \begin{cases}
\phi(x)\phi(y)&\text{if }x^0>y^0\\
\phi(y)\phi(x)&\text{if }y^0>x^0.
\end{cases}
\end{equation}
This (the correlation function) is the standard tool we use in
Feynman diagrams, which is used everywhere in quantum field
theory.

The first step therefore towards making direct use of functional
integrals is to find a functional formula for computing
correlation functions. To deduce what it should be consider the
object
\begin{equation}\label{eq:functionalIntegralCorrelationFunction}
\int\mathcal{D}\phi(x)\phi(x_1)\phi(x_2)\exp\left[i\int^{T}_{-T}\mathcal{L}(\phi)d^{4}x\right],
\end{equation}
where the boundary conditions on the path integral are
$\phi(-T,\bar{x})=\phi_{a}(\bar{x})$ and
$\phi(T,\bar{x})=\phi_{b}(\bar{x})$ for some (specified)
$\phi_a$, $\phi_b$. We want to relate this to the two-point
correlation function
\begin{equation}%\label{eq:}
\<\Omega|T\{\phi_{H}(x_1)\phi_{H}(x_2)\}|\Omega\>
\end{equation}
where $\phi_H$ denotes the operator in the Heisenberg picture
(contrasted with $\phi_S$ the operator in the Schrodinger
picture). 

First thing to do is to break up the functional integral in eq
\eqref{eq:functionalIntegralCorrelationFunction} as follows:
\begin{equation}%\label{eq:}
\int\mathcal{D}\phi(x) =
\int\mathcal{D}\phi_{1}(\bar{x})\int\mathcal{D}\phi_{2}(\bar{x})\int_{\substack{
\phi(x^{0}_{1},\bar{x})=\phi_{1}(\bar{x})\\
\phi(x^{0}_{2},\bar{x})=\phi_{2}(\bar{x})
}}\mathcal{D}\phi(x)
\end{equation}
The main functional integral $\int\mathcal{D}\phi(x)$ is now
constrained at times \todo[color=red!40]{\textbf{To Do:}\\ \small{Elaborate on these intermediate states}}$x^{0}_{1}$ and $x^{0}_{2}$ (in addition to
the endpoints $-T$, $T$), but we must integrate seperately over
the intermediate configurations $\phi_1(\bar{x})$,
$\phi_{2}(\bar{x})$. The amplitude is now broken in three, each
being a simple transition amplitude according to eq
\eqref{eq:pathIntegralForScalarFieldInGeneral}. The times
$x^{0}_{1}$, $x^{0}_{2}$ automatically fall into order; e.g. if
$x^{0}_{1}<x^{0}_{2}$ then eq
\eqref{eq:functionalIntegralCorrelationFunction} becomes
\begin{equation}%\label{eq:}
\int\mathcal{D}\phi_{1}(\bar{x})
\int\mathcal{D}\phi_{2}(\bar{x})
\phi_{1}(\bar{x})\phi_{2}(\bar{x})
\<\phi_{b}|e^{-iH(T-x^{0}_{2})}|\phi_2\>
\<\phi_2|e^{-iH(x^{0}_{2}-x^{0}_{1})}|\phi_{1}\>
\<\phi_{1}|e^{-iH(x^{0}_{1}-T)}|\phi_{a}\>.
\end{equation}
We can turn the field $\phi_{1}(\bar{x})$ into a Schrodinger
operator by
\begin{equation}%\label{eq:}
\phi_{S}(\bar{x}_{1})|\phi_{1}\> =
\phi_{1}(\bar{x}_{1})|\phi_{1}\>.
\end{equation}
The completeness relations (which appears to be effectively just
a generalization of the resolution of the identity) demands
\begin{equation}%\label{eq:}
\int\mathcal{D}\phi_{1}|\phi_{1}\>\<\phi_{1}| = \mathbf{1}
\end{equation}
which allows us to eliminate the intermediate state
$|\phi_1\>$. A similar argument holds for eliminating the
intermediate state $|\phi_{2}\>$, which allows us to obtain the
final expression
\begin{equation}%\label{eq:}
\<\phi_b|e^{-iH(T-x^{0}_{2})}\phi_{S}(\bar{x}_2)e^{-iH(x^{0}_{2}-x^{0}_{1})}\phi_{S}(\bar{x}_{1})e^{-iH(x^{0}_{1}+T)}|\phi_{a}\>.
\end{equation}
We can now note most of the exponential factors cancel out to
yield Heisenberg operators. In the case when
$x^{0}_{1}>x^{0}_{2}$, the order of $x_1$ and $X_2$ would be
interchanged! We conclude that eq
\eqref{eq:functionalIntegralCorrelationFunction} is equal to
\begin{equation}%\label{eq:}
\<\phi_b|e^{-iHT}T\{\phi_{H}(x_1)\phi_{H}(x_2)\}e^{-iHT}|\phi_a\>.
\end{equation}
This expression is remarkably close to being correct.

We need to consider the limit as $T\to\infty(1-i\varepsilon)$ for
``small $\varepsilon$''. This trick picks out the vacuum state
$|\Omega\>$ from $|\phi_a\>$ and $|\phi_b\>$ (provided that we
have some overlap between the specified $|\phi\>$ and
$|\Omega\>$, which we assume). For example, if we write
$|\phi_a\>$ as a linear combination of eigenstates $|n\>$ of $H$
we then have
\begin{equation}%\label{eq:}
e^{-iHT}|\phi_{a}\> = \sum_{n}e^{-iE_{n}T}|n\>\<n|\phi_a\>\xrightarrow[T\to\infty(1-i\varepsilon)]{}\<\Omega|\phi_a\>e^{-iE_{0}\cdot\infty(1-i\varepsilon)}|\Omega\>.
\end{equation}
Here this should be taken with a grain of salt, since we have ---
in our usual hand wavy manner --- treated infinity ``as if'' it
were a number. Now explicitly what we do is the following
\begin{equation}%\label{eq:}
\left(\<\phi_{b}|e^{-iHT}\right)T\{\phi(x_1)\phi(x_2)\}\left(e^{-iHT}|\phi_{a}\>\right)\xrightarrow[T\to\infty(1-i\varepsilon)]{}\<\phi_b|\Omega\>\<\Omega|\phi_a\>>e^{-iE_{0}\cdot\infty(1-i\varepsilon)}e^{iE_{0}\cdot\infty(1-i\varepsilon)}\<\Omega|T\{\phi(x_1)\phi(x_2)\}|\Omega\>
\end{equation}
We obtain awkward phase and overlap factors, but
don't worry: this is actually a good thing! These factors cancel
out if we divide them out. That is, we obtain the formula
\begin{equation}\label{eq:functionalIntegralTwoPointCorrelationFunction}
\begin{split}
\<\Omega|&T\{\phi_{H}(x_1)\phi_{H}(x_2)\}|\Omega\> =\\& \lim_{T\to\infty(1-i\varepsilon)}\left(\frac{\displaystyle\int\mathcal{D}\phi(x)\phi(x_1)\phi(x_2)\exp[i\int^{T}_{-T}\mathcal{L}d^{4}x]}{\displaystyle\int\mathcal{D}\phi(x)\exp[i\int^{T}_{-T}\mathcal{L}d^{4}x]}\right)
\end{split}
\end{equation}
This is our desired formula, we have the two-point correlation
function purely from functional integration. We can ask ``What
about higher-point correlation functions?'' The trick is simple,
the numerator simply increases the number of $\phi(x_j)$ factors.

\subsection*{Outline of the Derivation of Feynman Rules}

So here's the general algorithm we'll use when deriving the
Feynman rules: first we are going to define the measure
$\mathcal{D}\phi$ by discretizing spacetime with a cubic lattice;
then we will switch via a Fourier \emph{series} (yes, series, due
to discretization) to change the field from $\phi(x_i)$ to
$\phi(k^{\mu}_{n})$. This simplifies the evaluation of the
discretized functional integral. We make use of analogies made
with discrete vector calculus when considering a generalization
of Gaussian integrals. This results in introducing the notion of
a functional determinant. Although we will try to avoid the
notion of the functional determinant for the most part (since it
turns out it's canceled out in practice), the interested reader
can refer to appendix \ref{appendix:functionDet}.

The approach of implementing a discretized lattice then taking
the continuum limit is clumsy, bulky, and inelegant, but it gives
a good intuition about what's really going on mathematically. But
that isn't a good reason to continue to do long, bulky
calculations. We are going to consider a slicker way to compute
the functional integral using a sort of generalization of the
notion of a generating function. We'll consider this derivation
of Feynman rules in its own subsection.

\subsection{Feynman Rules}

Our next task, to accomplish our dream of deriving Feynman rules
in a functional manner, is to compute various correlation
functions direclty from the RHS of eq \eqref{eq:functionalIntegralTwoPointCorrelationFunction}.
In other words, we will use eq
\eqref{eq:functionalIntegralTwoPointCorrelationFunction} to
derive the Feynman rules for scalar field theory. We'll start
slow, with the two-point function in the free Klein-Gordon
theory, then proceed to generalize to higher
correlations. Finally we'll consider the $\phi^{4}$ theory.

\subsubsection{Free Scalar Field} Consider a noninteracting real
scalar field
\begin{equation}\label{eq:actionFreeScalarField}
S_{0} = \int\mathcal{L}_{0}d^{4}x = \int[\frac{1}{2}(\partial_{\mu}\phi)^2-\frac{1}{2}m^{2}\phi^{2}]d^{4}x.
\end{equation}
Since $\mathcal{L}_{0}$ is quadratic in $\phi$, the functional
integrals in eq
\eqref{eq:functionalIntegralTwoPointCorrelationFunction} take the
form of generalized Gaussian integrals. We will, for this
situation, be able to calculate them exactly.

\subsubsection{Discretization with Cubic Lattice} We want to
define the integral $\mathcal{D}\phi$ over field
configurations. The way we defined it for quantum mechanics was
to consider the functional integral as the limit of a large
number number of integrals. We take the number of integrals to go
to infinity. But we were always working with the discrete setting
(i.e. finitely many integrals) with a discretized path, i.e. the
replacement 
\begin{equation}%\label{eq:}
x(t)\longmapsto x_{k}
\end{equation}
where $k$ indicates the time slice. We perform a similar ritual
and replace the field operators
\begin{equation}%\label{eq:}
\phi(x)\longmapsto\phi(x_k)
\end{equation}
where $x_k$ is some point on the lattice\footnote{It should be
  noted that the lattice shouldn't be pictured as a sort of
  collection of wires, where we can consider values ``on the
  wire'' and anything ``off the wire'' is zero, ignored, or
  undefined. The intuition should be we have a discrete set of
  points, and we specify the connectedness of these points by
  some graph, or network, or something similar.}. Since we have
assumed the lattice is cubic, the spacing is even in every
dimension. For brevity, we will call the spacing
$\varepsilon$. We will also let the four-dimensional volume of
the spacetime region be denoted by $L^4$, and we will define
\begin{equation}%\label{eq:}
\mathcal{D}\phi = \prod_{j}d\phi(x_j)
\end{equation}
up to some constant.

\subsubsection{Fourier Expansion} The field values $\phi(x_j)$
can be represented by a ``discrete'' Fourier series (instead of a
``continuous'' Fourier transform):
\begin{equation}\label{eq:fourierSeriesField}
\phi(x_j) = \frac{1}{V}\sum_{n}e^{-ik_{n}\cdot{x_{j}}}\tilde{\phi}(k_n)
\end{equation}
where $k^{\mu}_{n}=2\pi{n^{\mu}}/L$, with
$n^{\mu}\in\mathbb{Z}^{3,1}$, $|k^{\mu}|<\pi/\varepsilon$, and
$V=L^{4}$. The Fourier coefficients $\tilde{\phi}$ are
complex. However, $\phi$ is real, which implies the coefficients
are constrained by
\begin{equation}%\label{eq:}
\tilde{\phi}^{*}(k) = \phi(-k).
\end{equation}
We will consider the real and imaginary parts of $\tilde{\phi}(k_{n})$
(with $k_n>0$) as independent variables. The change of variables
from the $\phi(x_j)$ to the new $\tilde{\phi}(k_n)$ is a unitary
transformation, enabling us to rewrite the integral as
\begin{equation}%\label{eq:}
\int\mathcal{D}\phi(x) = \prod_{k^{0}_{n}>0}\int
d\re(\tilde{\phi}(k_{n}))
d\im(\tilde{\phi}(k_{n})).
\end{equation}
Later we will take the limit as $L\to\infty$,
$\varepsilon\to0$. The effect of the limit is to recover the
continuum, it converts discrete sums over $k_n$ to continuous
integrals over k:
\begin{equation}%\label{eq:}
\frac{1}{V}\sum_{n}\longmapsto\int\frac{d^{4}k}{(2\pi)^4}.
\end{equation}
In our discussion, this will produce Feynman perturbative theory;
but we will not eliminate the ultraviolet and infrared
divergences of Feynman diagrams. (Renormalization and
Regularization is beyond the scope of this note.)

\subsubsection{Action under Change of Variables} Having now
defined the functional measure, somewhat simplified by our
Fourier analysis, we can now compute the functional integral over
$\phi$ (or more precisely $\tilde{\phi}$). The action for the
free real scalar field \eqref{eq:actionFreeScalarField} can be
rewritten in terms of the Fourier coefficients as
\begin{subequations}
\begin{align}
\int[\frac{1}{2}(\partial_{\mu}\phi)^{2}-\frac{1}{2}m^{2}\phi^{2}]d^{4}x 
&=
\frac{-1}{V}\sum_{n}\frac{1}{2}(m^{2}-k_{n}^{2})|\tilde{\phi}(k_{n})|^{2}\\
&= 
\frac{-1}{V}\sum_{n}\frac{1}{2}(m^{2}-k_{n}^{2})[\re(\tilde{\phi}_{n})^{2}+\im(\tilde{\phi}_{n})^{2}]
\end{align}
\end{subequations}
where we have introduced
$\tilde{\phi}_{n}=\widetilde{\phi}(k_{n})$. The quantity
\begin{equation}%\label{eq:}
(m^{2}-k_{n}^{2}) = (m^{2}+\|\bar{k}_{n}\|^{2}-(k_{n}^{0})^{2})
\end{equation}
is positive provided that $k^{0}_{n}$ is ``not too large''. For
the rest of our calculations, we will assume that this quantity
is positive (i.e. we will evaluate it by analytic continuation
from the region where $\|\bar{k}_{n}\|>k^{0}_{n}$).

\subsubsection{The Gaussian Connection} The denominator of the
two-point correlation function can be written as a product of
Gaussian integrals:
\begin{subequations}
\begin{align}
\int\mathcal{D}\phi e^{iS_{0}} &=
\left(\prod_{k_{n}^{0}>0}\int d\re(\tilde{\phi}(k_{n}))
d\im(\tilde{\phi}(k_{n}))\right)\nonumber\\
\;\;&\times\exp\left[\frac{-i}{V}\sum_{k^{0}_{n}>0}(m^{2}-k_{n}^{2})\|\tilde{\phi}_{n}\|^{2}\right].\\
&= \prod_{k^{0}_{n}>0}\left(\int d\tilde{\phi}_{n} \exp\left[\frac{-i}{V}(m^{2}-k_{n}^{2})\|\tilde{\phi}_{n}\|^{2}\right]\right)
\end{align}
\end{subequations}
We can easily rewrite this by expanding the product, grouping the
real and imaginary terms, yielding the expression
\begin{equation}
\begin{split}
\int\mathcal{D}\phi e^{iS_{0}} =&  \prod_{k^{0}_{n}>0} 
\left(\int d\re(\tilde{\phi}(k_{n}))\exp\left[\frac{-i}{V}(m^{2}-k_{n}^{2})\re(\tilde{\phi}_{n})^{2}\right]\right)\\
&\times
\left(\int d\im(\tilde{\phi}(k_{n}))\exp\left[\frac{-i}{V}(m^{2}-k_{n}^{2})\im(\tilde{\phi}_{n})^{2}\right]\right).
\end{split}
\end{equation}
We can evaluate these integrals by using the usual Gaussian
tricks which yields
\begin{subequations}\label{eq:functionalIntegralAsProductOfGaussians}
\begin{align}
\int\mathcal{D}\phi e^{iS_{0}} =& \prod_{k^{0}_{n}>0}\sqrt{\frac{-i\pi{V}}{m^{2}-k_{n}^{2}}}\sqrt{\frac{-i\pi{V}}{m^{2}-k_{n}^{2}}}\\
=& \prod_{\text{all } k_{n}}\sqrt{\frac{-i\pi{V}}{m^{2}-k_{n}^{2}}}.
\end{align}
\end{subequations}
We used the Gaussian integration formula, but this really ought
to be justified since functional integrals are a foreign beast.

\subsubsection{Functional Determinant Detrimant.}
Recall that before we had to rotate the contour of integration to
$t\to t(1-i\varepsilon)$ to guarantee convergence for the time
integral. This means we should really have to change $k^0\to
k^{0}(1+i\varepsilon)$ in eq \eqref{eq:fourierSeriesField} and
everything that follows from it. In particular, this means that
\begin{equation}%\label{eq:}
(k_{n}^{2}-m^{2})\to(k_{n}^{2}-m^{2}+i\varepsilon).
\end{equation}
The $i\varepsilon$ term gives the necessary convergence factor
for the Gaussian integrals. It also defines the direction of the
analytic continuation that might be necessary to define the
squareroots in eq \eqref{eq:functionalIntegralAsProductOfGaussians}.

Consider an analogous situation with the general Gaussian
integral
\begin{equation}%\label{eq:}
\left(\prod_{k}\int d\xi_{k}\right)\exp[-\xi_{i}B^{ij}\xi_{j}]
\end{equation}
where $B$ is a symmetric matrix with eigenvalues $b_i$. To
evaluate this integral, let $\xi_{i} = {A_{i}}^{j}x_{j}$ where
${A_{i}}^{j}$ is an orthogonal matrix of eigenvectors that
diagonalizes $B$. Changing variables from $\xi_i$ to $x_i$ we
have
\begin{subequations}
\begin{align}
\left(\prod_{k}\int d\xi_{k}\right)\exp[-\xi_{i}B^{ij}\xi_{j}] &= \left(\prod_{k}\int dx_{k}\right)\exp[-\sum_{j}b_{j}x_{j}^{2}]\\
&= \prod_{k}\left(\int \exp[-b_{k}x_{k}^{2}]dx_{k}\right)\\
&= \prod_{k}\sqrt{\pi/b_{k}}\\
&= \operatorname{const}/\sqrt{\det(B)}.
\end{align}
\end{subequations}
The astute reader should note that the numerator diverges, it is
a constant that is of order $\mathcal{O}(k^{n/2})$ and we take
the limit as $n\to\infty$. We typically overlook this
inconvenient fact. Now we should apply this to our problem at hand.

The analogy is made clearer if we first perofrm an integration by
parts to write the Klein Gordon action as
\begin{equation}%\label{eq:}
S_{0}=\frac{1}{2}\int\phi(-\partial^{2}-m^{2})\phi d^{4}x +
\text{boundary terms}.
\end{equation}
Thus the matrix $B$ corresponds to the operator
$(m^2+\partial^2)$ and we can formally write our result as
\begin{equation}%\label{eq:}
\int\mathcal{D}\phi e^{iS_{0}} = (\text{const})[\det(m^2+\partial^2)]^{-1/2}.
\end{equation}
The object on the right hand side is called a ``\emph{functional
  determinant}'', to see one way to calculate it refer to
appendix \ref{appendix:functionDet}. It can, at first, look quite
ill defined; but rest assured, the functional determinant usually
cancels out.

\subsubsection{Two Point Function} 

Now consider the numerator of the two point function as defined
by eq
\eqref{eq:functionalIntegralTwoPointCorrelationFunction}. We need
to Fourier-expand the two extra factors of $\phi$:
\begin{equation}%\label{eq:}
\phi(x_1)\phi(x_2) = \left(\frac{1}{V}\sum_{j}e^{-ik_{j}\cdot{x_1}}\tilde{\phi}_{j}\right)\left(\frac{1}{V}\sum_{l}e^{-ik_{l}\cdot{x_2}}\tilde{\phi}_{l}\right)
\end{equation}
We can plug this in to find the numerator:
\begin{equation}%\label{eq:}
\begin{split}
&\left(\prod_{k^{0}_{n}>0}\int
d\re(\tilde{\phi}_{n})d\im(\tilde{\phi}_{n})\right)\left(\frac{1}{V}\sum_{j}e^{-ik_{j}\cdot{x_1}}\tilde{\phi}_{j}\right)\\
&\times\left(\frac{1}{V}\sum_{l}e^{-ik_{l}\cdot{x_2}}\tilde{\phi}_{l}\right)\exp\left[\frac{-i}{V}\sum_{k^{-}_{n}>0}(m^{2}-k_{n}^{2})\|\tilde{\phi}_{n}\|^{2}\right]\\
&= \frac{1}{V^{2}}\sum_{m,l}e^{-i(k_{m}\cdot x_{1}+k_{l}\cdot
  x_{2})}\\
&\times\prod_{k^{0}_{n}>0}\left(\int
d\re(\tilde{\phi}_{n})d\im(\tilde{\phi}_{n}) (\tilde{\phi}_{l}\tilde{\phi}_{m})\exp\left[\frac{-i}{V}(m^{2}-k_{n}^{2})\|\tilde{\phi}_{n}\|^{2}\right]\right)
\end{split}
\end{equation}
\todo[color=red!40]{\textbf{To Do:}\\ \small{I am skeptical of these calculations, sometime when I have
  the leisure of time to perform the calculations explicitly I
  would love to make certain this reasoning is absolutely correct.}}Now, most of the time the integrand will be odd. There is an
exception when $k_m = \pm k_l$. If $k_m = +k_l$ then the term
involving $\re(\tilde{\phi}_m)^2$ is nonzero but it's exactly
canceled by the term involving $\im(\tilde{\phi}_m)^2$. If
$k_m=-k_l$, we take advantage of the relation
$\tilde{\phi}^{*}(k)=\tilde{\phi}(-k)$. In this situation, the
two terms add. When $k^{0}_{m}<0$, we obtain the following
expression for the numerator:
\begin{equation}%\label{eq:}
\frac{1}{V^{2}}\sum_{m}e^{-ik_{m}\cdot(x_{1}+x_{2})}\left(\prod_{k^{0}_{n}}\frac{-i\pi{V}}{m^{2}-k_{n}^{2}}\right)\frac{-iV}{m^{2}-k_{m}^{2}+i\varepsilon}.
\end{equation}
Note the parenthetic term cancels with the denominator exactly,
leaving us the Feynman propagator when we take the continuum limit.

\todo[color=red!40]{\textbf{To Do:}\\ \small{Type up the
    four-point correlation function calculation, and the
    $\phi^{4}$ model in the discretization scheme.}}

\subsection{Generating Functional Tricks}
%%
%% functionalScalarFieldSlick.tex
%% 
%% Made by Alex Nelson
%% Login   <alex@tomato>
%% 
%% Started on  Sat Aug  1 12:23:40 2009 Alex Nelson
%% Last update Sat Aug  1 12:23:40 2009 Alex Nelson
%%
\subsubsection{Outline}
As promised, we'll introduce a slicker way to compute Feynman
rules using functional derivatives. It's a lot more
mathematically rigorous (and simpler) than the discretization
scheme. The method uses a mathematical gadget which generalizes
the notion of a generating function --- the generating
\emph{functional}. Recall the generating function is used to
compute constants and other useful numbers by taking its
$n^{\text{th}}$ derivative and evaluating it at 0. We
\emph{functionally} do the same thing, take the functional
derivative of the generating functional and evaluate it at 0.


\subsubsection{Properties of the Functional Derivative}
First lets try to review the properties of the functional
$\delta/\delta J(x)$. The functional derivative obeys the basic
property (in four dimensions)
\begin{equation}%\label{eq:}
\frac{\delta}{\delta J(x)}J(y)=\delta^{(4)}(x-y),
\quad\text{or}\quad
\frac{\delta}{\delta J(x)}\int J(y)\phi(y)d^{4}y = \phi(x).
\end{equation}
This can be viewed as a continuous generalization of the vector
calculus derivative 
\begin{equation}%\label{eq:}
\frac{\partial}{\partial x_i}x_j = \delta_{ij}
\quad\text{or}\quad
\frac{\partial}{\partial x_i}\sum_{j}x_{j}k_{j}=k_{i}.
\end{equation}
To take the functional derivatives of more complicated
situations, we use the basic properties of the chain rule and the
product rule. \textbf{Warning:} we \textbf{assume without proof}
that these properties hold, we'll not divulge into the proof
here. So we have situations like the following:
\begin{equation}%\label{eq:}
\frac{\delta}{\delta J(x)}\exp\left[i\int J(y)\phi(y)d^{4}y\right]
=i\phi(x)\exp\left[i\int J(y)\phi(y)d^{4}y\right].
\end{equation}
When the functional depends on the derivative of $J$ we integrate
by parts --- and for all practical purposes, we always can
integrate by parts in quantum field theory --- then apply the
functional derivative as follows:
\begin{equation}%\label{eq:}
\frac{\delta}{\delta J(x)}\int V^{\mu}(y)\partial_{\mu}J(y)d^{4}y
=
\frac{\delta}{\delta J(x)}\left(
\operatorname{bdry terms} + 
\int J(y)\partial_{\mu}V^{\mu}(y)d^{4}y
\right)
= - \partial_{\mu}V^{\mu}(x).
\end{equation}
This concludes our review of basic properties that we'll use
later on.

\subsubsection{Generating Functional}

As alluded to earlier, the basic object of interest is the
generating functional of correlation functions. We denote this
object of interest by $Z[J]$. In a scalar field theory, it's
defined as
\begin{equation}\label{eq:scalarFieldTheoryGeneratingFunctional}
Z[J] \stackrel{\text{def}}{=}\int\mathcal{D}\phi\exp\left[
i\int[\mathcal{L}+J(x)\phi(x)]d^{4}x
\right].
\end{equation}
This is a functional integral over $\phi$. We've merely added to
$\mathcal{L}$ in the exponent an extra term $J(x)\phi(x)$, which
we usually refer to as the ``\emph{source term}''.

\subsubsection{Derivation of Correlation Function}
Now we use it to compute the generating functions for the
Klein-Gordon field (the free scalar field). For example the
two-point function is
\begin{equation}%\label{eq:}
\<0|T\{\phi(x_1)\phi(x_2)\}|0\> = \left.\frac{1}{Z_0}
\left(-i\frac{\delta}{\delta J(x_{1})}\right)
\left(-i\frac{\delta}{\delta J(x_{2})}\right)
Z[J]\right|_{J=0}
\end{equation}
where $Z_0=Z[0]$. Each functional derivative brings down a factor
of $\phi$ in the numerator of $Z[J]$; setting $J=0$ we recover
our desired expression. In more explicit detail, we can compute
\begin{subequations}
\begin{align}
\frac{\delta}{\delta J(x_2)}Z[J] &= \frac{\delta}{\delta J(x_2)}
\int\mathcal{D}\phi\exp\left[
i\int[\mathcal{L}+J(x)\phi(x)]d^{4}x
\right] \\
&= \int\mathcal{D}\phi\frac{\delta}{\delta J(x_2)}
\exp\left[
i\int[\mathcal{L}+J(x)\phi(x)]d^{4}x
\right] \\
&= \int\mathcal{D}\phi i\phi(x_2)
\exp\left[
i\int[\mathcal{L}+J(x)\phi(x)]d^{4}x
\right]
\end{align}
\end{subequations}
This is the effect of one functional derivative, we multiply by
$-i$ to finish one functional operation. We need to do another to
get the expression
\begin{equation}%\label{eq:}
\left(-i\frac{\delta}{\delta J(x_{1})}\right)
\left(-i\frac{\delta}{\delta J(x_{2})}\right)
Z[J]
= \int\mathcal{D}\phi \phi(x_1)\phi(x_2)
\exp\left[
i\int[\mathcal{L}+J(x)\phi(x)]d^{4}x
\right].
\end{equation}
To get the final expression, we need to divide by $Z_0$ and
evaluate at $J=0$ to get
\begin{equation*}%\label{eq:}
\left.\frac{1}{Z_0}
\left(-i\frac{\delta}{\delta J(x_{1})}\right)
\left(-i\frac{\delta}{\delta J(x_{2})}\right)
Z[J]\right|_{J=0} =\displaystyle{
\frac{\displaystyle\int\mathcal{D}\phi\; \phi(x_1)\phi(x_2)\exp\left[i\int[\mathcal{L}]d^{4}x\right]}
{\displaystyle\int\mathcal{D}\phi\exp\left[i\int\mathcal{L}d^{4}x\right]}}
\end{equation*}
which is precisely what is expected.

\subsubsection{Slicker Way to Compute Correlation Functions}
We've seen that eq
\eqref{eq:scalarFieldTheoryGeneratingFunctional} recovers the
expected expression for two-point functions. It's pretty nifty
for us since the free scalar field can be written fairly
easily. It's explicitly written after integrating by parts (on
the first term)
\begin{equation}\label{eq:slickPartialIntegration}
\int[\mathcal{L}_{0}(\phi)+J\phi]d^{4}x 
= 
\int[\frac{1}{2}\phi(-\partial^{2}-m^{2}+i\varepsilon)\phi+J\phi]d^{4}x.
\end{equation}
The factor of $i\varepsilon$ is to guarantee convergence. We
complete the square by introducing a shifted scalar field
\begin{equation}%\label{eq:}
\phi'(x)\stackrel{\text{def}}{=}\phi(x)-i\int D_{F}(x-y)J(y)d^{4}y
\end{equation}
where $D_{F}(x-y)$ is the Feynman propagator --- i.e. the Green's
function of the Klein-Gordon operator, we find that our original
expression \eqref{eq:slickPartialIntegration} becomes
\begin{equation}%\label{eq:}
\begin{split}
\int[&\mathcal{L}_{0}(\phi)+J\phi]d^{4}x =\\
&\int[\frac{1}{2}\phi'(-\partial^{2}-m^{2}+i\varepsilon)\phi']d^{4}x
-\int\frac{1}{2}J(x)\left[-iD_{F}(x-y)\right]J(y)d^{4}y.
\end{split}
\end{equation}
More symbolically, we could rewrite the change of variables as
\begin{equation}%\label{eq:}
\phi'\stackrel{\text{def}}{=}\phi+(-\partial^{2}-m^{2}+i\varepsilon)^{-1}J,
\end{equation}
and the result becomes
\begin{equation}%\label{eq:}
\int[\mathcal{L}_0 + J\phi]d^{4}x = 
\int\left[\frac{1}{2}\phi'(-\partial^{2}-m^{2}+i\varepsilon)\phi'-\frac{1}{2}J(-\partial^{2}-m^{2}+i\varepsilon)^{-1}J\right]d^{4}x.
\end{equation}
This looks nasty, but we are not done yet. We have a few tricks
left.

When we change variables from $\phi$ to $\phi'$, it's just a
shift, so the Jacobian in the functional integral definition of
$Z[J]$ is the identity. The result is
\begin{equation}%\label{eq:}
\underbrace{\int\mathcal{D}\phi'
\exp\left[i\int\mathcal{L}_{0}(\phi')d^{4}x\right]}_{Z_{0}}
\underbrace{\exp\left[-i\int\frac{1}{2}J(x)[-iD_{F}(x-y)J(y)]d^{4}xd^{4}y\right]}_{\text{independent of }\phi'}
\end{equation}
As noted, the second integral is independent of $\phi'$ and the
first is precisely $Z_{0}$. The generating functional of the free
scalar field is thus
\begin{equation}\label{eq:generatingFunctionPostManip}
Z[J] = Z_{0}\exp\left[\frac{-1}{2}\int J(x)D_{F}(x-y)J(y)d^{4}xd^{4}y\right]
\end{equation}
Lets use \eqref{eq:scalarFieldTheoryGeneratingFunctional} and
\eqref{eq:generatingFunctionPostManip} to compute some
correlation functions. The two-point function is by definition
\begin{equation}%\label{eq:}
\<0|T\{\phi(x_1)\phi(x_2)\}|0\>
=\left.
\frac{-\delta}{\delta J(x_{1})}
\frac{\delta}{\delta J(x_{2})}
\exp\left[\frac{-1}{2}\int J(x)D_{F}(x-y)J(y)d^{4}xd^{4}y\right]\right|_{J=0}
\end{equation}
We can evaluate one functional derivative to find
\begin{equation}%\label{eq:}
\<0|T\{\phi(x_1)\phi(x_2)\}|0\>
=\left.
\frac{-\delta}{\delta J(x_{1})}
\left[\frac{-1}{2}\int D_{F}(x_{2}-y)J(y)d^{4}y 
-\frac{1}{2}\int J(x)D_{F}(x-x_{2})d^{4}x
\right]
\frac{Z[J]}{Z_{0}}\right|_{J=0}
\end{equation}
\begin{comment}
We can see that the negatives cancel out nicely, we end up with 2
terms that effectively look like
\begin{equation*}%\label{eq:}
\frac{-\delta}{\delta J(x_{1})}
\left[\frac{-1}{2}\int D_{F}(x_{2}-y)J(y)d^{4}y 
-\frac{1}{2}\int J(x)D_{F}(x-x_{2})d^{4}x
\right]
\propto
\frac{\delta}{\delta J(x_1)}\int D_{F}(x_2-y)J(y)d^{4}y
\end{equation*}
\end{comment}
We can see that this is just what happens after evaluating one
functional derivative of the definition of the two-point
function, which allows us to conclude that
\begin{equation}%\label{eq:}
\<0|T\{\phi(x_1)\phi(x_2)\}|0\> = D_{F}(x_1-x_2).
\end{equation}
This is good because it connects back to what we should already know.

\subsubsection{Example: Four Point Correlation Functions}

This is a rather space-consuming computation, so we need to
introduce some abbreviations. Namely we'll use the conventions
that arguments of functions are subscripts: $\phi_1 = \phi(x_1)$,
$J_x=J(x)$, $D_{x4}=D_{F}(x-x_{4})$, etc. Repeated subscripts
will be integrated over implicitly (like a continuous Einstein's
summation convention). The four-point function is thus
\begin{subequations}
\begin{align}
\<0|T\{\phi_1\phi_2\phi_3\phi_4\}|0\> 
&=\left. \frac{\delta}{\delta J_{1}}
\frac{\delta}{\delta J_{2}}
\frac{\delta}{\delta J_{3}}
[-J_{x}D_{x4}]e^{-\frac{1}{2}J_{x}D_{xy}J_{y}}\right|_{J=0}\\
&=\left. \frac{\delta}{\delta J_{1}}
\frac{\delta}{\delta J_{2}}
[-D_{34}+J_{x}D_{x4}J_{y}D_{y3}]e^{-\frac{1}{2}J_{x}D_{xy}J_{y}}\right|_{J=0}\\
=&\left. \frac{\delta}{\delta J_{1}}
[D_{34}J_{x}D_{x2}+D_{24}J_{y}D_{y3}+J_{x}D_{x4}J_{y}D_{23}]e^{-\frac{1}{2}J_{x}D_{xy}J_{y}}\right|_{J=0}\\
&=D_{34}D_{12}+D_{24}D_{13}+D_{14}D_{23}
\end{align}
\end{subequations}
which is precisely what we expect by Wick's theorem.

\subsubsection{The Beauty of the Generating Functional}

The beauty of the situation is that these calculations are
completely independent of whether things are free or
interacting. The catch is the $Z[J=0]$ factor is not trivial in
the interacting situation. It gives us the sum of the vacuum
diagrams.



\section{Functional Quantization of Electromagnetic Field}
%%
%% quantizationEM.tex
%% 
%% Made by Alex Nelson
%% Login   <alex@tomato>
%% 
%% Started on  Sat Aug  1 13:45:29 2009 Alex Nelson
%% Last update Sat Aug  1 13:45:29 2009 Alex Nelson
%%

We assert the Feynman rule for the photon propagator to be
\begin{equation}%\label{eq:}
\parbox{20mm}{
\begin{fmfgraph*}(50,25)\fmfpen{0.2mm}
    \fmfleft{v1} \fmfright{v2}
    \fmf{photon,label=$\scriptstyle{k}\to$}{v1,v2}
\end{fmfgraph*}}
 = \frac{-ig_{\mu\nu}}{k^{2} + i\varepsilon}.
\end{equation}
Now that we have the basic tools of functional quantization, lets
try to prove it.

Consider the functional integral 
\begin{equation}%\label{eq:}
\int \mathcal{D}A\;e^{iS[A]}
\end{equation}
where $S[A]$ is the action for the free electromagnetic
field. The functional integral is taken over the four components,
i.e. we have
$\mathcal{D}A=\mathcal{D}A^{0}\mathcal{D}A^{1}\mathcal{D}A^{2}\mathcal{D}A^{3}$. We
integrate by parts and Fourier expand to find the action to be
\begin{subequations}
\begin{align}
S &= \int \left[\frac{-1}{4}F_{\mu\nu}F^{\mu\nu}\right]d^{4}x\\
&= \frac{1}{2}\int A_{\mu}(x)\left[\partial^{2}g^{\mu\nu}-\partial^{\mu}\partial^{\nu}\right]A_{\nu}(x)d^{4}x\\
&= \frac{-1}{2}\int\tilde{A}_{\mu}(k)\left[-k^{2}g^{\mu\nu}+k^{\mu}k^{\nu}\right]\tilde{A}_{\nu}(-k)\frac{d^{4}k}{(2\pi)^{4}}.
\end{align}
\end{subequations}
Observe that if $\tilde{A}_{\mu}(k)=k_{\mu}\alpha(k)$ for some
scalar function $\alpha$, the integrand becomes
\begin{subequations}
\begin{align}
\tilde{A}_{\mu}(k)\left[-k^{2}g^{\mu\nu}+k^{\mu}k^{\nu}\right]\tilde{A}_{\nu}(-k)
&=
\alpha(k)k_{\mu}\left[-k^{2}g^{\mu\nu}+k^{\mu}k^{\nu}\right](-k_{\nu})(\alpha(-k))\\
&=\alpha(k)\alpha(-k)\left[k_{\mu}k_{\nu}k^{2}g^{\mu\nu}-k_{\mu}k_{\nu}k^{\mu}k^{\nu}\right]\\
&=\alpha(k)\alpha(-k)[0] = 0.
\end{align}
\end{subequations}
This integrand vanishes for \textbf{any} choice of scalar
function $\alpha$. In this situation, the integrand of the
functional integral $\int \mathcal{D}A e^{iS[A]}$ is 1, which
more importantly implies it is a badly divergent functional
integral. We deduce that
\begin{equation}\label{eq:photonPropagatorSingularity}
\begin{split}
&(\partial^{2}g_{\mu\nu}-\partial_{\mu}\partial_{\nu})D^{\nu\rho}_{F}(x-y)=i{\delta_{\mu}}^{\rho}\delta^{(4)}(x-y)\\
&\text{or}\quad(-k^{2}g_{\mu\nu}+k_{\mu}k_{\nu})\tilde{D}^{\nu\rho}_{F}(k)=i{\delta_{\mu}}^{\rho}
\end{split}
\end{equation}
(which importantly defines the Feynman propagator
$D^{\mu\rho}_{F}$) has no solution. This shouldn't surprise
anyone since $(-k^{2}g_{\mu\nu}+k_{\mu}k_{\nu})$ has a singularity.

The real problem child here is gauge invariance. Recall that
$F_{\mu\nu}$ (and thus $\mathcal{L}$) is invariant under a
general U(1) gauge transformation of the form
\begin{equation}%\label{eq:}
A_{\mu}(x)\to A_{\mu}(x)+\frac{1}{e}\partial_{\mu}\alpha(x)
\end{equation}
where $\alpha(x)$ is any scalar function. This transformation
means that potentials of the form
$A_{\mu}(x)=\frac{1}{e}\partial_{\mu}\alpha(x)$ is gauge
equivalent to 0. The functional integral is badly defined because
we are redundantly integrating over a continuous infginity of
physically equivalent field configurations. That is, we haven't
gauge-fixed the action yet! We need to count each physically
interesting state once.

We can accomplish this gauge fixing by a method due to Faddeev
and Popov (for the original paper,
see~\cite{Faddeev:1967fc}). Let $G(A)$ be a sort of generalized
``indicating functional'' which is zero for a certain gauge
condition, e.g. for the Lorenz gauge we have 
\begin{equation}%\label{eq:}
G(A)=\partial_{\mu}A^{\mu}(x).
\end{equation} 
We can constrain the functional integral to be when $G(A)=0$ by
using a delta function $\delta\left(G(A)\right)$. Geometrically
we could have the intuition that we assign a delta function at
each point $x$. To do this legally we insert 1 in the functional
integral of the form
\begin{equation}\label{eq:insertOne}
1 = \int \mathcal{D}\alpha(x)
\delta\left(G(A^{\alpha})\right)\det\left(\frac{\delta
  G(A^{\alpha})}{\delta \alpha}\right)
\end{equation}
where we have $A^{\alpha}_{\mu}(x) = A_{\mu}(x) +
\frac{1}{e}\partial_{\mu}\alpha(x)$. We see that this condition
\eqref{eq:insertOne} generalizes the identity
\begin{equation}%\label{eq:}
1 = \left(\prod_{j}\int d a_{j}\right)
\delta^{(n)}(\bar{g}(\bar{a}))\det(\partial g_{j}/\partial a_{k})
\end{equation}
for discrete $n$-dimensional vectors. In the Lorenz gauge we have 
\begin{equation}%\label{eq:}
G(A^{\alpha}) = \partial^{\mu}A_{\mu} + \frac{1}{e}\partial^{2}\alpha
\end{equation}
so the functional determinant in our situation (due to our case
being the vacuum) is precisely $\det(\partial^{2}/e)$. For our
situation the only  thing that matters is that the functional
determinant is independent of $A$, i.e. that it's like a constant
term in our functional integral.

After inserting \eqref{eq:insertOne} the functional integral
becomes
\begin{equation}%\label{eq:}
\det\left(\frac{\delta G(A^{\alpha})}{\delta \alpha}\right)\int
\mathcal{D}\alpha\int\mathcal{D}A e^{iS[A]}\delta\left(G(A^{\alpha})\right)
\end{equation}
Now we will change the variable of integration from $A$ to
$A^{\alpha}$. This is a simple shift so
$\mathcal{D}A=\mathcal{D}A^{\alpha}$, and by gauge invariance
$S[A]=S[A^{\alpha}]$. We then obtain
\begin{equation}%\label{eq:}
\int\mathcal{D}A e^{iS[A]} = \det\left(\frac{\delta G(A^{\alpha})}{\delta \alpha}\right)\int
\mathcal{D}\alpha\int\mathcal{D}A^{\alpha} e^{iS[A^{\alpha}]}\delta\left(G(A^{\alpha})\right).
\end{equation}
The functional integral over $A$ is now restricted by the delta
function to physically distinct inequivalent states, as
desired. The infinity comes from a divergent integral over
$\alpha(x)$ which simply gives an infinite multiplicative factor.

To go any further in our investigation, we have to gauge fix
$G(A)$. We choose the general class of functions
\begin{equation}%\label{eq:}
G(A) = \partial^{\mu}A_{\mu}(x) - \omega(x)
\end{equation}
where $\omega(x)$ is any scalar function. Setting this $G(A)$
zero occurs as a generalization of the Lorenz gauge. The
functional determinant is the same as in the Lorenz gauge
\begin{equation}%\label{eq:}
\det\left(\frac{\delta G(A^{\alpha})}{\delta\alpha}\right)=\det(\partial^{2}/e).
\end{equation}
Thus the functional integral becomes:
\begin{equation}%\label{eq:}
\int\mathcal{D}A e^{iS[A]} =
\det(\partial^{2}/e)\left(\int\mathcal{D}\alpha\right)\left(\int
\mathcal{D}A^{\alpha} e^{iS[A^{\alpha}]}\delta\left(\partial^{\mu}A^{\alpha}_{\mu}(x)-\omega(x)\right)\right).
\end{equation}
This equality holds for any $\omega(x)$, so it will hold if we
replace the right hand side with any properly normalized linear
combination involving different functions $\omega(x)$. Our final
trick will be integrating over all $\omega(x)$ with a Gaussian
weighting function centered at $\omega=0$. Our expression will
become
\begin{equation*}%\label{eq:}
\begin{split}
&N(\xi)\int\mathcal{D}\omega\;
\exp\left[-i\int\frac{\omega^{2}}{2\xi}d^{4}x\right]\det(\frac{\partial^{2}}{e})\int\mathcal{D}\alpha\int\mathcal{D}A^{\alpha}\;e^{iS[A^{\alpha}]}\delta(\partial^{\mu}A_{\mu}^{\alpha}(x)-\omega(x))
=\\
&N(\xi)\det(\partial^{2}/e)\int\mathcal{D}\alpha\int\mathcal{D}A^{\alpha}\;e^{iS[A^{\alpha}]}\exp\left[-i\int\frac{(\partial^{\mu}A^{\alpha}_{\mu}(x))^{2}}{2\xi}d^{4}x\right]
\end{split}
\end{equation*}
where $N(\xi)$ is a normalization constant and we have used the
delta function to perform the integral over $\omega$. We can
choose $\xi$ to be any finite constant. What we have really done
is we've effectively added a new term
$-(\partial^{\mu}A_{\mu}^{\alpha})^{2}/(2\xi)$ to the Lagrangian.

What we have done thus far in our functional quantization of the
electromagnetic field is we have worked with the denominator of
our formula for the correlation functions:
\begin{equation}%\label{eq:}
\<\Omega|T\{\mathcal{O}(A)\}|\Omega\>=\lim_{T\to\infty(1-i\varepsilon)}\frac{\displaystyle\int\mathcal{D}A\;\mathcal{O}(A)\exp\left[-i\int^{T}_{-T}\mathcal{L}d^{4}x\right]}{\displaystyle\int\mathcal{D}A\;\exp\left[-i\int^{T}_{-T}\mathcal{L}d^{4}x\right]}.
\end{equation}
We can do the same manipulations for the numerator, provided that
the operator $\mathcal{O}(A)$ is gauge invariant. (If it isn't,
the change of variables trick $A\to A^{\alpha}$ won't work.)
Assuming that the operator $\mathcal{O}(A)$ is in fact gauge
invariant, we find that its correlation function becomes
\begin{equation}%\label{eq:}
\begin{split}
&\<\Omega|T\{\mathcal{O}(A)\}|\Omega\>\\
&\quad = \lim_{T\to\infty(1-i\varepsilon)}
\frac{\displaystyle\int\mathcal{D}A\;\mathcal{O}(A)
\exp\left[-i\int^{T}_{-T}\left(\mathcal{L}-\frac{1}{2\xi}(\partial^{\mu}A_{\mu})^{2}\right)d^{4}x\right]}
{\displaystyle\int\mathcal{D}A\;\exp\left[-i\int^{T}_{-T}\left(\mathcal{L}-\frac{1}{2\xi}(\partial^{\mu}A_{\mu})^{2}\right)d^{4}x\right]}.
\end{split}
\end{equation}
Mathemagically the awkward constants we had previously are all
canceled out. The only trace left behind of our meddling is the
extra $\xi$-term that is added to the action.

We concluded that eq \eqref{eq:photonPropagatorSingularity}
implies that it is not sensible to obtain a photon propagator
from the action $S[A]$. With our new $\xi$-term, however, that
equation becomes
\begin{equation}%\label{eq:}
(-k^{2}g_{\mu\nu}+(1-\frac{1}{\xi})k_{\mu}k_{\nu})\tilde{D}^{\nu\rho}_{F}(k)
  = i {\delta_{\mu}}^{\rho},
\end{equation}
which has the solution
\begin{equation}\label{eq:feynmanPropagatorFromFPprocedure}
\tilde{D}^{\mu\nu}_{F}(k) = \frac{-i}{k^{2}+i\varepsilon}\left(g^{\mu\nu}-(1-\xi)\frac{k^{\mu}k^{\nu}}{k^{2}}\right).
\end{equation}
This is our desired expression for the photon propagator. The
$i\varepsilon$ term in the denominator arises exactly in the same
way as in the Klein-Gordon free field case.

In practice, one usually chooses a specific value of $\xi$ to
actually perform computations. Two choices that are often
convenient are
\begin{align*}
\xi=0&\qquad\text{Landau Gauge}\\
\xi=1&\qquad\text{Feynman Gauge}
\end{align*}
In our notes on the subject of Feynman diagrams, we have used the
Feynman gauge.

The Faddeev-Popov procedure guarantees the value of any
correlation function of gauge invariant operators computed from
Feynman diagrams \emph{will be independent of the value of $\xi$ used
in the calculation} (provided the value of $\xi$ is used
consistently; we can't do half of one computation using the
Landau gauge, then finish up the rest in the Feynman gauge
because it makes things nicer!). It's easy to show that in QED
this assertion is true (I suspect due to the fact that it is an
Abelian gauge group). Note in eq \eqref{eq:feynmanPropagatorFromFPprocedure}
that $\xi$ multiplies a term in the photon propagator
proportional to $k^{\mu}k^{\nu}$. According to the Ward-Takahashi
identity, the replacement in a Green's function of any photon
propagator by $k^{\mu}k^{\nu}$ yields zero, except for terms
involving external off-shell fermions. These terms are equal and
opposite for particle and anti-particle, and vanish when the
fermions are grouped into gauge-invariant combinations.

Now, QED isn't just photon propagators. We have
something else we need to compute: S-matrix elements from
correlation functions of non-gauge-invariant operators $\psi(x)$,
$\overline{\psi}(x)$ and $A_{\mu}(x)$. We will assert that the
S-matrix elements are correctly computed by this procedure. The
S-matrix is really defined between asymptotic states, we can
compute S-matrix elements in a formalism where the coupling
constant is ``turned off'' some ``infinitely long time ago'' in
the past and far in the future. In the zero coupling limit, there
is a clean seperation between gauge-invariant and gauge-variant
states. On the one hand, single-particle states containing one
electron, one positron, or one transversely polzarized photon are
gauge-invariant; while, on the other, states with timelike and
longitudinal photon polarizations transform under gauge
motions. We can thus define a gauge-invariant S-matrix in the
following way: Let $S_{FP}$ be the S-matrix between general
asymptotic states, computed from the Faddeev-Popov procedure. The
matrix is unitary but not gauge invariant (proof?). Let $P_{0}$
be a projection onto the subspace of the space of asymptotic
states in which all particles are either electrons, positrons, or
transverse photons. Let
\begin{equation}\label{eq:SMatrixOnGaugeInvariantStates}
S = P_{0} S_{FP} P_{0}.
\end{equation}
This S-matrix is gauge-invariant by construction, \emph{it is projected
onto gauge-invariant states.} Now it is not obvious that it is
unitary. 

In a handwavy way, we'll summarize the reasoning that this matrix
is in fact unitary. Any matrix element
$\mathcal{M}^{\mu}\epsilon^{*}_{\mu}$ for photon emission
satisfies
\begin{equation}%\label{eq:}
\sum_{i=1,2}\epsilon^{*}_{i\mu}\epsilon_{i\nu}\mathcal{M}^{\mu}\mathcal{M}^{*\nu}=(-g_{\mu\nu})\mathcal{M}^{\mu}\mathcal{M}^{*\nu},
\end{equation}
where the sum on the left hand side runs over transverse
polarizations only. (The same argument applies if
$\mathcal{M}^{\mu}$ and $\mathcal{M}^{*\nu}$ are distinct
amplitudes, as long as they satisfy the Ward identity.) This is
exactly the information we need to see that
\begin{equation}%\label{eq:}
SS^{\dag}=P_{0}S_{FP}P_{0}S_{FP}^{\dag}P_{0}=P_{0}S_{FP}S_{FP}^{\dag}P_{0}.
\end{equation}
Now, we can use the unitarity of $S_{FP}$ to see that $S$ is
unitary, $SS^{\dag}=1$ on the subspace of gauge-invariant
states. It is easy to check explicitly that the formula \eqref{eq:SMatrixOnGaugeInvariantStates}
for the S-matrix is independent of $\xi$: \emph{the Ward identity
implies that any QED matrix element with all external fermions
on-shell is unchanged if we add to the photon propagator
$D^{\mu\nu}(q)$ any term proportional to $q^{\mu}$}.






\section{Functional Quantization of Spinor Fields}
%%
%% quantizationSpinor.tex
%% 
%% Made by Alex Nelson
%% Login   <alex@tomato>
%% 
%% Started on  Sat Aug 15 11:54:58 2009 Alex Nelson
%% Last update Sat Aug 15 11:54:58 2009 Alex Nelson
%%

The functional methods we have considered so far allows us to
compute correlation functions of fields obeying the canonical
commutation relations. To generalize to spinor fields (i.e. ones
obeying canonical anticommutation relations) we must do something
different: we must represent even the classical fields by
anticommuting (``fermionic'') numbers. Lets first review some of
the properties of these fermionic numbers.

\subsection{Fermionic Numbers}

We will define ``Fermionic'' numbers (or \emph{Grassmann numbers}) 
by giving algebraic properties for their manipulation. 

The basic feature for Grassmannians is that they
\emph{anticommute}. For any two such numbers $\theta,\eta$,
\begin{equation}%\label{eq:}
\theta\eta=-\eta\theta.
\end{equation}
In particular this implies for $\eta=\theta$ that
\begin{equation}%\label{eq:}
\theta^{2} = 0.
\end{equation}
A product of two Grassmannians $(\theta\eta)$ commutes with other
Grassmannians. We really have a so-called $\mathbb{Z}$-grading,
and the commuting terms are the ``even'' terms, the anticommuting
ones are the ``odd'' terms (corresponding to bosonic and
fermionic notions, respectively).

We can define integrals of Grassmannians. A function $f(\theta)$
of one Grassmannian is really of the form
\begin{equation}%\label{eq:}
f(\theta)=f(0)+f'(0)\theta
\end{equation}
by Taylor expansion. This presupposes we have some notion of
``differentiation'' of Grassmannians. We really do, it's simply
\begin{equation}%\label{eq:}
\frac{d}{d\theta}(\eta\theta)=\frac{d}{d\theta}(-\theta\eta)=-\eta.
\end{equation}
Integration is ``dually''
\begin{equation}%\label{eq:}
\int f(\theta)d\theta = \int \Big(f(0)+f'(0)\theta\Big)d\theta = f'(0)
\end{equation}
which is curious, as
\begin{equation}%\label{eq:}
\frac{d}{d\theta}f(\theta) =
\frac{d}{d\theta}\Big(f(0)+f'(0)\theta\Big) = f'(0).
\end{equation}
In this situation, as in complex analysis, we can write the
derivative in terms of the integral. We adopt the convention
\begin{equation}%\label{eq:}
\int d\theta\int d\eta ~\eta\theta = +1
\end{equation}
performing the innermost integral first, as is usual in calculus.

When we have the Dirac field, we need to introduce complex
Grassmannians. We think of them as sort of like operators, so
complex conjugation ``acts like'' Hermitian conjugation:
\begin{equation}%\label{eq:}
(\theta\eta)^{*}\stackrel{\text{def}}{=}\eta^{*}\theta^{*}=-\theta^{*}\eta^{*}.
\end{equation}
To integrate over complex Grassmannians, lets define
\begin{equation}%\label{eq:}
\theta=\frac{\theta_{1}+i\theta_{2}}{\sqrt{2}},\qquad\theta^{*}=\frac{\theta_{1}-i\theta_{2}}{\sqrt{2}}.
\end{equation}
We can treat $\theta,\theta^{*}$ as independent Grassmann
numbers, and use the convention $\int d\theta^{*}d\theta
~(\theta\theta^{*})=1$.

We can consider a Grassmann Gaussian
\begin{equation}%\label{eq:}
\int d\theta^{*}d\theta e^{-\theta^{*}b\theta} = \int
d\theta^{*}d\theta(1-\theta^{*}b\theta)=\int d\theta^{*}d\theta
-(-\theta\theta^{*})b = b.
\end{equation}
If $\theta$ were some ordinary, everyday complex number, this
integral would instead be $2\pi/b$. The factor of $2\pi$ is more
or less unimportant, the main difference is the fact that one is
the inverse of the other (more or less). However, observe that by
adding another factor of $\theta\theta^{*}$, we find
\begin{equation}%\label{eq:}
\int d\theta^{*}d\theta ~\theta\theta^{*}e^{-\theta^{*}b\theta} = 1.
\end{equation}
So now, the extra factor of $\theta\theta^{*}$ ``cancels'' the
$b$ factor, which should look familiar from our regular,
old-fashioned Gaussian integrals.

To perform higher dimensional Grassmann Gaussian integrals, we
first have to observe the behavior of unitary transformations
acting on Grassmannians. Consider $n$ complex Grassmannians
$\theta_{i}$, and a unitary matrix $U={U_{j}}^{i}$. Let $\theta_{i}'={U_{i}}^{j}\theta_{j}$. Then
\begin{subequations}
\begin{align}
\prod_{i}{U_{i}}^{j}\theta_{j} &= \frac{1}{n!}\epsilon^{ij\ldots~l}\theta_{i}'\theta_{j}'(\cdots)\theta_{l}'\\
&= \frac{1}{n!}\epsilon^{ij(\cdots)l}({U_{i}}^{i'}\theta_{i'})({U_{j}}^{j'}\theta_{j'})(\cdots)({U_{l}}^{l'}\theta_{l'})\\
&= \frac{1}{n!}\epsilon^{ij(\cdots)l}{U_{i}}^{i'}{U_{j}}^{j'}(\cdots){U_{l}}^{l'}\epsilon_{i'j'(\cdots)l'}\left(\prod_{i}\theta_{i}\right)\\
&= \det(U)\left(\prod_{i}\theta_{i}\right).
\end{align}
\end{subequations}
In general, we find that
\begin{equation}%\label{eq:}
\left(\prod_{i}\int d\theta^{*}_{i}d\theta_{i}\right)f(\theta),
\end{equation}
the only term that survives is proportional to $(\prod
\theta_{i})(\prod \theta^{*}_{i})$. If we replace $\theta$ by
$U\theta$, we end up with an extra factor of
$\det(U)\det(U)^{*}=1$, which doesn't change anything.

We can evaluate a general Grassmann Gaussian involving a
Hermitian matrix $B$ with eigenvalues $b_{i}$:
\begin{equation}%\label{eq:}
\left(\prod_{i}\int d\theta^{*}_{i}d\theta_{i}\right)
e^{-\theta^{*}_{i}B^{ij}\theta_{j}} = \left(\prod_{i}\int
d\theta^{*}_{i}d\theta_{i}\right)e^{-\Sigma_{i}\theta^{*}_{i}b_{i}\theta_{i}}
= \prod_{i}b_{i} = \det(B).
\end{equation}
Similarly we can see that
\begin{equation}%\label{eq:}
\left(\prod_{i}\int d\theta^{*}_{i}d\theta_{i}\right)\theta_{k}\theta^{*}_{l}e^{-\theta^{*}_{i}B^{ij}\theta_{j}}=\Big(\det(B)\Big)(B^{-1})_{kl}
\end{equation}
where $(B^{-1})_{kl}$ is a component from the inverse for $B$. As
a general rule of thumb, Gaussian Grassmann integrals behave
similar to Gaussian integrals, with the exception that the
determinant is in the numerator (as opposed to the denominator).

\subsection{Dirac Propagator}

A Grassmann \emph{field} is a function of spacetime whose values
are anticommuting numbers. More precisely we can define a
Grassmann field $\psi(x)$ in terms of any set of orthonormal
basis functions:
\begin{equation}%\label{eq:}
\psi(x) = \sum_{i} \psi_{i}\phi_{i}(x)
\end{equation}
where the basis functions $\phi_{i}(x)$ are ordinary functions,
while the coefficients $\psi_{i}$ are Grassmann numbers. To
describe the Dirac field, we take the $\phi_{i}$ to be a basis of
four-component spinors.

Surprisingly enough, we have enough tools to start evaluating
functional integrals (and thus correlations functions) involving
fermions. Consider, for example, the two-point function given by
\begin{equation}%\label{eq:}
\<0|T\{\psi(x_{1})\bar{\psi}(x_{2})\}|0\> = \frac{\int\mathcal{D}\bar{\psi}\mathcal{D}\psi\exp\left[i\int\bar{\psi}(i\slashed{\partial}-m)\psi~d^{4}x\right]\psi(x_{1})\bar{\psi}(x_{2})}{\int\mathcal{D}\bar{\psi}\mathcal{D}\psi\exp[i\int\bar{\psi}(i\slashed{\partial}-m)\psi~d^{4}x]}
\end{equation}
Note that $\mathcal{D}\bar{\psi}$ is unitarily equivalent to
$\mathcal{D}\psi^{*}$. 


\clearpage
\appendix
\section{Notes on the Functional Determinant}\label{appendix:functionDet}
%%
%% appendixAFunctionalDet.tex
%% 
%% Made by Alex Nelson
%% Login   <alex@tomato>
%% 
%% Started on  Tue Jul 28 12:49:49 2009 Alex Nelson
%% Last update Tue Jul 28 12:49:49 2009 Alex Nelson
%%
\subsection{Introduction}

We review the zeta regularisation approach to calculating the functional
determinant of differential operators. First we review a few
properties of the finite dimensional determinant. These include
the `exp(tr(A)) = det(exp(A))' identity, and the usefulness of
eigenvalues when calculating determinants. 

We move on to the general case of infinite dimensions. We
generalize the determinant ``definition'' as the infinite product
series, and use the aforementioned identity to use the Riemann
zeta function in calculating determinants of differential
operators.

Then we compute a few examples from various fields. First, the
example of the heat equation. Then we compute the determinant of
the Hamiltonian for a quantum Harmonic oscillator. 

The interested reader can refer to
Elizalde~\cite{Elizalde:1999zy} for a similar review with more
references. 

\subsection{Review of Finite Dimensional Determinants}

Recall that given a (square) $n\times n$ matrix $A$, we can write it as
\begin{equation}
A = P^{-1}DP
\end{equation}
where $D$ is a diagonal matrix with entries being eigenvalues,
and $P$ is an orthogonal matrix whose columns are the
corresponding eigenvectors. We find its determinant to be:
\begin{equation}
\begin{split}
\det{(A)}  =& \det{(P^{-1}DP)}\\
 =&  \det{(P^{-1})}\det{(D)}\det{(P)}\\
 =&  \det{(D)}\\
 =&  \prod_{j=1}^{n} \lambda_{j}
\label{eq:det}
\end{split}
\end{equation}
where $\lambda_{j}$ is the $j^{\text{th}}$ eigenvalue of $A$. 

There is one identity that would be nice to cover before we move
on (because I am most familiar with the so-called
`zeta-regularisation' scheme, this is a necessary identity). If
we represent $A$ with the diagonalized matrix $D$, we find that
\begin{equation}
\text{trace}(\ln(D)) = \ln(\lambda_{1}) + (\ldots) +\ln(\lambda_{n})
\end{equation}
and we find
\begin{equation}
\begin{split}
\exp{(\text{trace}(\ln(D)))}  = &  \exp{(\ln(\lambda_{1}) + \ln(\ldots) + \ln(\lambda_{n}))}\\
 = & \lambda_{1}(\ldots)\lambda_{n} \\
 \stackrel{\text{def}}{=} & \det{(A)}.
\end{split}
\label{eq:id}
\end{equation}
One may see this and say ``Huh, that's neat but seemingly useless'' and you would be right! Well, so far you would be, but you will see that it's useful!

\subsection{Zeta-Regularisation Scheme}

Recall from differential equations, that we may treat a derivative as an ``infinite dimensional'' linear operator (and recall from linear algebra all linear operators may be represented as a square matrix!). So we may apply Eq. (\ref{eq:det}) to a differential operator in general. The first thought that comes to my mind is ``You're crazy, man! That would diverge like...a...divergent product series!'' That is true, if we naively write
\begin{equation}
\det{(\mathcal{D})} = \prod^{\infty}_{n=1}\lambda_{n}.
\end{equation}
(Where $\mathcal{D}$ is a differential operator). But as we have just seen, we may use the identity derived in Eq. (\ref{eq:id}) to find:
\begin{equation}
\exp{(\text{trace}(\ln(\mathcal{D})))} := \det{(\mathcal{D})}
\end{equation}
so the question is to find an equation that gives us
\begin{equation}
(\text{trace}(\ln(\mathcal{D}))) = \sum^{\infty}_{n=0}\ln{(\lambda_{n})}.
\end{equation}

Let us now introduce the Riemann zeta function
\begin{equation}
\zeta_{R}(s) = \sum^{\infty}_{n=1}n^{-s} \mbox{  with  } \text{Re}(s)>1.
\end{equation}
It is something well studied in number theory, modern analysis, and a number of other subjects. Note that
\begin{equation}
\zeta^{\prime}_{R}(s) = -\sum^{\infty}_{n=1}\frac{\ln{(n)}}{n^{s}}
\end{equation}
and what physicists like to do is then set $s=0$, so we find
\begin{equation}
\zeta^{\prime}_{R}(0) = -\sum^{\infty}_{n=1}\ln{(n)}.
\end{equation}
This is remarkably similar to what we're looking for!

What we do is we take the so-called ``zeta-trace'' of our differential operator $\mathcal{D}$:
\begin{equation}
\zeta_{D}(s) =\sum^{\infty}_{n=1}\frac{1}{\lambda_{n}^{s}}.
\end{equation}
Then we just take its derivative and set $s=0$. Usually we'd want to write this in terms of the Riemann zeta function, since we know a lot more about the Riemann zeta function in general (such as what its value of its derivative at 0 is!). 

\subsection{Example: Heat Equation}

Recall the infamous heat equation
\begin{equation}
\frac{\partial^{2}}{\partial x^{2}}U(x,t) = \frac{\partial}{\partial t}U(x,t)
\end{equation}
with the boundary conditions
\begin{equation}
\begin{split}
U(x,0)=&f(x)\quad\forall x\in[0,L]\\
U(0,t)=&U(L,t)=0\quad\forall t>0
\end{split}
\end{equation}
By the famous calculation, we seperate the variables and we are know dealing with a sort of eigenvalue problem
\begin{equation}
\frac{\partial^{2}}{\partial x^{2}}u(x) = -\lambda u(x)
\end{equation}
which tells us that
\begin{equation}
\begin{split}
u(x)  = & B\sin{(\sqrt{\lambda}x)}  \\
\lambda_{n}  = & (n\pi/L)^2 \quad \forall n\in\mathbb{N}
\end{split}
\end{equation}
So lets figure out its determinant!

Well, we first note that the zeta trace of the differential operator is
\begin{equation}
\zeta_{A}(s) = \sum^{\infty}_{n=1} \frac{L^{2s}}{(n\pi)^{2s}}
\label{eq:detA}
\end{equation}
We can factor this out to be
\begin{equation}
\zeta_{A}(s) = \frac{L^{2s}}{\pi^{2s}}\sum^{\infty}_{n=1} n^{-2s}
\end{equation}
which is tricky to work with, but we'd really like to write it in terms of the Riemann zeta function! We find:
\begin{equation}
\zeta_{A}(s) = \left(\frac{L}{\pi}\right)^{2s}\zeta_{R}(2s)
\end{equation}
and thus
\begin{equation}
\zeta_{A}^{\prime}(s) = -2\ln(L/\pi)\left(\frac{L}{\pi}\right)^{2s}\zeta_{R}(2s) + 2\left(\frac{L}{\pi}\right)^{2s}\zeta_{R}^{\prime}(2s).
\end{equation}
We find
\begin{equation}
\zeta_{A}^{\prime}(0) = -2\ln(L/\pi)\zeta_{R}(0) + 2\zeta_{R}^{\prime}(0).
\end{equation}
Because I don't know the Riemann Zeta Function like the back of my hand (unfortunately), I look up its values\footnote{From \url{http://mathworld.wolfram.com/RiemannZetaFunction.html}}:
\begin{equation}
\begin{split}
\zeta_{R}(0)  = & \frac{-1}{2}\\
\zeta_{R}^{\prime}(0)  = & -\frac{1}{2}\ln(2\pi)
\end{split}
\end{equation}
So we find then
\begin{equation}
\zeta_{A}^{\prime}(0) = \ln(L/\pi) - \ln(2\pi).
\end{equation}
and then, by defintion, the determinant of our operator $A$ is
\begin{equation}
\det{(A)} := \exp{(-\zeta_{A}^{\prime}(0))} = \frac{2\pi^2}{L}.
\end{equation}
This concludes this section.

\subsection{Example: Quantum Harmonic Oscillator}

In the one dimensional case, a particle of mass $m$ has the potential
\begin{equation}
V(x) = \frac{1}{2}m\omega^2x^2
\end{equation}
where $m\omega^2=k$ is called the ``spring stiffness coefficient'' or sometimes the ``force constant'', and $\omega$ is the circular frequency. We find the Hamiltonian to be
\begin{equation}
H = \frac{p^2}{2m} + \frac{1}{2}m\omega^2x^2
\end{equation}
and so the Schrodinger equation is
\begin{equation}
\frac{-\hbar^2}{2m}\frac{d^2}{dx^2} + \frac{1}{2}m\omega^2x^2 |\psi\rangle = E |\psi\rangle.
\end{equation}
We will not show all the work, but through solving this differential equation, the eigenvalues for the Hamiltonian are
\begin{equation}
E_{n} = \hbar\omega(n + \frac{1}{2}) \mbox{    with    }\forall n\in\mathbb{N}
\end{equation}
which will be used in our computation of the determinant of the Hamiltonian.

We find that the zeta trace of the Hamiltonian is thus
\begin{equation}
\zeta_{H}(s) = \sum^{\infty}_{n=1}E_{n}^{-s} = (\hbar\omega)^{-s}\sum^{\infty}_{n=1}(n+\frac{1}{2})^{-s}
\end{equation}
and we find this to be
\begin{equation}
\zeta_{H}(s) = (\hbar\omega)^{-s}\sum^{\infty}_{n=1}(n+\frac{1}{2})^{-s} = (\hbar\omega/2)^{-s}\sum^{\infty}_{n=1}(2n+1)^{-s}.
\end{equation}
We can refer to the Hurwitz zeta-function
\begin{equation}
\zeta(s,q) = \sum^{\infty}_{k=0}(k+q)^{-s}
\end{equation}
and rewrite our zeta trace for the Hamiltonian to be
\begin{equation}
\zeta_{H}(s) = (\hbar\omega/2)^{-s}\left(\zeta(s,3)-2\zeta(s,1)+2^{-s}\right).
\end{equation}
Again, referring to a table\footnote{More precisely, Eqs (9) and (16) of \url{http://mathworld.wolfram.com/HurwitzZetaFunction.html}} we find
\begin{equation}
\begin{array}{c}
\zeta(0,a)=\frac{1}{2}-a\\
\frac{d}{ds}\zeta(0,a) = \ln(\Gamma(a))-\frac{1}{2}\ln(2\pi)
\end{array}
\end{equation}
So we have
\begin{equation}
\begin{split}
\zeta_{H}^{\prime}(0) & =  -\ln(\hbar\omega/2)\left(\zeta(0,3)-2\zeta(0,1)+1\right) + \left(\ln(6)+\frac{1}{2}\ln(2\pi)-\ln(2)\right) \\
& =  -\ln(\hbar\omega/2)\left(\frac{-1}{2}\right) + \left(\ln(6)+\frac{1}{2}\ln(2\pi)-\ln(2)\right) \\
& =  \ln(\sqrt{\hbar\omega/2}) + \ln(3) + \ln(\sqrt{2\pi})
\end{split}
\end{equation}
We can now compute
\begin{equation}
\det{(\hat{H})} =  \exp{(-\zeta_{H}^{\prime}(0))} =  \frac{1}{3}\sqrt{\frac{1}{2\pi\hbar\omega}}
\end{equation}
which is the determinant of the above differential operator.

\section{Glossary}
%%
%% glossary.tex
%% 
%% Made by Alex Nelson
%% Login   <alex@tomato>
%% 
%% Started on  Sat Aug 15 13:42:56 2009 Alex Nelson
%% Last update Sat Aug 15 13:42:56 2009 Alex Nelson
%%
A small glossary of various terms used in quantum field theory.
\begin{description}
\item[Abelian Group] when the group is commutative, we call it
  ``Abelian''.
\item[BRST Method] a generalization of the Faddeev-Popov
  procedure for non-Abelian gauge symmetries, it involves Ghosts.
\item[Correlation Function] the two-point correlation function
  corresponds to the amplitude for propagation of a
  particle or excitation between $y$ and $x$; Wick's theorem
  tells us we can express higher order correlation functions in
  terms of two-point correlation functions.
\item[Faddeev-Popov Procedure] a method to avoid redundant infinities in
  the functional integral when we have (Abelian) gauge symmetries.
\item[Gauge Symmetry] a change in coordinates which leaves the
  equations of motion invariant.
\item[Ghosts] a sort of ``particle'' which serves as negative
  degrees of freedom to cancel the effects of the unphysical
  timelike and longitudinal polarization states of gauge bosons.
\item[LSZ Reduction Formula] the multiparticle generalization of
  the Lorentz-invariant formula $\<k|k'\>=(2\pi)^{3}2\omega\delta^{(3)}(\bar{k}-\bar{k}')$.
\item[Non-Abelian Group] when the group's multiplication
  operation is not commutative, it is Non-Abelian. Contrast to an
  Abelian Group.
\end{description}

\nocite{*}
\bibliographystyle{utcaps}
\bibliography{functionalQFT}
\end{fmffile}
\end{document}
