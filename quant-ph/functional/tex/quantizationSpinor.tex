%%
%% quantizationSpinor.tex
%% 
%% Made by Alex Nelson
%% Login   <alex@tomato>
%% 
%% Started on  Sat Aug 15 11:54:58 2009 Alex Nelson
%% Last update Sat Aug 15 11:54:58 2009 Alex Nelson
%%

The functional methods we have considered so far allows us to
compute correlation functions of fields obeying the canonical
commutation relations. To generalize to spinor fields (i.e. ones
obeying canonical anticommutation relations) we must do something
different: we must represent even the classical fields by
anticommuting (``fermionic'') numbers. Lets first review some of
the properties of these fermionic numbers.

\subsection{Fermionic Numbers}

We will define ``Fermionic'' numbers (or \emph{Grassmann numbers}) 
by giving algebraic properties for their manipulation. 

The basic feature for Grassmannians is that they
\emph{anticommute}. For any two such numbers $\theta,\eta$,
\begin{equation}%\label{eq:}
\theta\eta=-\eta\theta.
\end{equation}
In particular this implies for $\eta=\theta$ that
\begin{equation}%\label{eq:}
\theta^{2} = 0.
\end{equation}
A product of two Grassmannians $(\theta\eta)$ commutes with other
Grassmannians. We really have a so-called $\mathbb{Z}$-grading,
and the commuting terms are the ``even'' terms, the anticommuting
ones are the ``odd'' terms (corresponding to bosonic and
fermionic notions, respectively).

We can define integrals of Grassmannians. A function $f(\theta)$
of one Grassmannian is really of the form
\begin{equation}%\label{eq:}
f(\theta)=f(0)+f'(0)\theta
\end{equation}
by Taylor expansion. This presupposes we have some notion of
``differentiation'' of Grassmannians. We really do, it's simply
\begin{equation}%\label{eq:}
\frac{d}{d\theta}(\eta\theta)=\frac{d}{d\theta}(-\theta\eta)=-\eta.
\end{equation}
Integration is ``dually''
\begin{equation}%\label{eq:}
\int f(\theta)d\theta = \int \Big(f(0)+f'(0)\theta\Big)d\theta = f'(0)
\end{equation}
which is curious, as
\begin{equation}%\label{eq:}
\frac{d}{d\theta}f(\theta) =
\frac{d}{d\theta}\Big(f(0)+f'(0)\theta\Big) = f'(0).
\end{equation}
In this situation, as in complex analysis, we can write the
derivative in terms of the integral. We adopt the convention
\begin{equation}%\label{eq:}
\int d\theta\int d\eta ~\eta\theta = +1
\end{equation}
performing the innermost integral first, as is usual in calculus.

When we have the Dirac field, we need to introduce complex
Grassmannians. We think of them as sort of like operators, so
complex conjugation ``acts like'' Hermitian conjugation:
\begin{equation}%\label{eq:}
(\theta\eta)^{*}\stackrel{\text{def}}{=}\eta^{*}\theta^{*}=-\theta^{*}\eta^{*}.
\end{equation}
To integrate over complex Grassmannians, lets define
\begin{equation}%\label{eq:}
\theta=\frac{\theta_{1}+i\theta_{2}}{\sqrt{2}},\qquad\theta^{*}=\frac{\theta_{1}-i\theta_{2}}{\sqrt{2}}.
\end{equation}
We can treat $\theta,\theta^{*}$ as independent Grassmann
numbers, and use the convention $\int d\theta^{*}d\theta
~(\theta\theta^{*})=1$.

We can consider a Grassmann Gaussian
\begin{equation}%\label{eq:}
\int d\theta^{*}d\theta e^{-\theta^{*}b\theta} = \int
d\theta^{*}d\theta(1-\theta^{*}b\theta)=\int d\theta^{*}d\theta
-(-\theta\theta^{*})b = b.
\end{equation}
If $\theta$ were some ordinary, everyday complex number, this
integral would instead be $2\pi/b$. The factor of $2\pi$ is more
or less unimportant, the main difference is the fact that one is
the inverse of the other (more or less). However, observe that by
adding another factor of $\theta\theta^{*}$, we find
\begin{equation}%\label{eq:}
\int d\theta^{*}d\theta ~\theta\theta^{*}e^{-\theta^{*}b\theta} = 1.
\end{equation}
So now, the extra factor of $\theta\theta^{*}$ ``cancels'' the
$b$ factor, which should look familiar from our regular,
old-fashioned Gaussian integrals.

To perform higher dimensional Grassmann Gaussian integrals, we
first have to observe the behavior of unitary transformations
acting on Grassmannians. Consider $n$ complex Grassmannians
$\theta_{i}$, and a unitary matrix $U={U_{j}}^{i}$. Let $\theta_{i}'={U_{i}}^{j}\theta_{j}$. Then
\begin{subequations}
\begin{align}
\prod_{i}{U_{i}}^{j}\theta_{j} &= \frac{1}{n!}\epsilon^{ij\ldots~l}\theta_{i}'\theta_{j}'(\cdots)\theta_{l}'\\
&= \frac{1}{n!}\epsilon^{ij(\cdots)l}({U_{i}}^{i'}\theta_{i'})({U_{j}}^{j'}\theta_{j'})(\cdots)({U_{l}}^{l'}\theta_{l'})\\
&= \frac{1}{n!}\epsilon^{ij(\cdots)l}{U_{i}}^{i'}{U_{j}}^{j'}(\cdots){U_{l}}^{l'}\epsilon_{i'j'(\cdots)l'}\left(\prod_{i}\theta_{i}\right)\\
&= \det(U)\left(\prod_{i}\theta_{i}\right).
\end{align}
\end{subequations}
In general, we find that
\begin{equation}%\label{eq:}
\left(\prod_{i}\int d\theta^{*}_{i}d\theta_{i}\right)f(\theta),
\end{equation}
the only term that survives is proportional to $(\prod
\theta_{i})(\prod \theta^{*}_{i})$. If we replace $\theta$ by
$U\theta$, we end up with an extra factor of
$\det(U)\det(U)^{*}=1$, which doesn't change anything.

We can evaluate a general Grassmann Gaussian involving a
Hermitian matrix $B$ with eigenvalues $b_{i}$:
\begin{equation}%\label{eq:}
\left(\prod_{i}\int d\theta^{*}_{i}d\theta_{i}\right)
e^{-\theta^{*}_{i}B^{ij}\theta_{j}} = \left(\prod_{i}\int
d\theta^{*}_{i}d\theta_{i}\right)e^{-\Sigma_{i}\theta^{*}_{i}b_{i}\theta_{i}}
= \prod_{i}b_{i} = \det(B).
\end{equation}
Similarly we can see that
\begin{equation}%\label{eq:}
\left(\prod_{i}\int d\theta^{*}_{i}d\theta_{i}\right)\theta_{k}\theta^{*}_{l}e^{-\theta^{*}_{i}B^{ij}\theta_{j}}=\Big(\det(B)\Big)(B^{-1})_{kl}
\end{equation}
where $(B^{-1})_{kl}$ is a component from the inverse for $B$. As
a general rule of thumb, Gaussian Grassmann integrals behave
similar to Gaussian integrals, with the exception that the
determinant is in the numerator (as opposed to the denominator).

\subsection{Dirac Propagator}

A Grassmann \emph{field} is a function of spacetime whose values
are anticommuting numbers. More precisely we can define a
Grassmann field $\psi(x)$ in terms of any set of orthonormal
basis functions:
\begin{equation}%\label{eq:}
\psi(x) = \sum_{i} \psi_{i}\phi_{i}(x)
\end{equation}
where the basis functions $\phi_{i}(x)$ are ordinary functions,
while the coefficients $\psi_{i}$ are Grassmann numbers. To
describe the Dirac field, we take the $\phi_{i}$ to be a basis of
four-component spinors.

Surprisingly enough, we have enough tools to start evaluating
functional integrals (and thus correlations functions) involving
fermions. Consider, for example, the two-point function given by
\begin{equation}%\label{eq:}
\<0|T\{\psi(x_{1})\bar{\psi}(x_{2})\}|0\> = \frac{\int\mathcal{D}\bar{\psi}\mathcal{D}\psi\exp\left[i\int\bar{\psi}(i\slashed{\partial}-m)\psi~d^{4}x\right]\psi(x_{1})\bar{\psi}(x_{2})}{\int\mathcal{D}\bar{\psi}\mathcal{D}\psi\exp[i\int\bar{\psi}(i\slashed{\partial}-m)\psi~d^{4}x]}
\end{equation}
Note that $\mathcal{D}\bar{\psi}$ is unitarily equivalent to
$\mathcal{D}\psi^{*}$. 
