%%
%% angular.tex
%% 
%% Made by Alex Nelson
%% Login   <alex@tomato>
%% 
%% Started on  Mon Mar 30 11:07:25 2009 Alex Nelson
%% Last update Mon Mar 30 11:07:25 2009 Alex Nelson
%%

In classical mechanics, we described the angular momentum of a
body by 
\begin{equation}
\vec{x}\times\vec{p}=\vec{L}.
\end{equation}
In quantum mechanics, we do the same thing more or less, but we
need to make the replacement
\begin{equation}
p\to\widehat{p}=\frac{\hbar}{i}\frac{\partial}{\partial x}
\end{equation}
and so on. We will recklessly switch notations at random between
using hats to denote operators and not using hats. It is
understood, unless otherwise specified, we will work henceforth
with operators.

Here it is convenient to note that we can write the cross product
in a slick way using matrix multiplication. Observe
\begin{equation}
\vec{a}\times\vec{b} = \begin{bmatrix}0 & a_3 & -a_2\\
-a_3 & 0 & a_1\\
a_2 & -a_1 &
0\end{bmatrix}\begin{bmatrix}b_1\\b_2\\b_3\end{bmatrix}.
\end{equation}
Also observe that
\begin{equation}
\begin{bmatrix}0 & a_3 & -a_2\\
-a_3 & 0 & a_1\\
a_2 & -a_1 &
0\end{bmatrix}^2 = - \begin{bmatrix} a_{2}^{2}+a_{3}^{2} & -a_1a_2 & -a_1a_3\\
-a_1a_2 & a_{1}^{2}+a_{3}^{2} & -a_{2}a_{3}\\
-a_{1}a_{3} & -a_{2}a_{3} & a_{1}^{2}+a_{2}^{2}
0\end{bmatrix}.
\end{equation}
If we perform that computations, we find that (using, apologies
to the physicists struggling with it, abstract index notation)
\begin{subequations}
\begin{align}
L^{2} &= \vec{L}\cdot\vec{L}\\
&=
\sum_{i,j,k,m,n}\epsilon^{ijk}\epsilon^{imn}x^{j}p_{k}x^{m}p_{n}\\
&= \sum_{i,j,k,m,n}(\delta^{jm}\delta^{kn}-\delta^{jn}\delta^{km})x^{j}p_{k}x^{m}p_{n}\\
&= \sum_{i,j,k,m,n}x^{j}p_{k}x^{j}p_{k} - x^{j}p_{k}x^{k}p_{j}.
\end{align}
\end{subequations}
We can using the commutation relations
\begin{equation}
[\widehat{x}^{i},\widehat{p}_{j}]=i{\delta^{i}}_{j}
\end{equation}
to reorder terms, for example
\begin{subequations}
\begin{align}
p_{k}x_{j} &= x_{j}p_{k}-i\delta_{kj}\\
x_{j}p_{k} &= p_{k}x_{j}+i\delta_{kj}.
\end{align}
\end{subequations}
We can rewrite the angular momentum squared as
\begin{subequations}
\begin{align}
L^2 &= x^{j}(x{j}p_{k} -
i{\delta^{j}}_{k})p_{k}-(p_{k}x^{j}+i{\delta^{j}}_{k})x_{k}p_{j}\\
&=
x^{j}x_{j}p^{k}p_{k}-ix^{j}p_{j}-p_{k}x^{k}x^{j}p_{j}-ix^{j}p_{j}\\\
&=x^{j}x_{j}p^{k}p_{k}-2ix^{j}p_{j}-(x^{k}p_{k}-i{\delta^{k}}_{k})x^{j}p_{j}.
\end{align}
\end{subequations}
