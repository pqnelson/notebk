%%
%% threeDCFT.tex
%% 
%% Made by Alex Nelson
%% Login   <alex@tomato>
%% 
%% Started on  Sat Aug 29 12:38:56 2009 Alex Nelson
%% Last update Sat Aug 29 12:38:56 2009 Alex Nelson
%%
\documentclass[10pt,oneside]{article}
\usepackage{fly}
\usepackage{brackets}
\usepackage{danger}
\usepackage{float}
%\def\comment#1{}
\title{Calculations in Three Dimensional Chain Field Theory}
\date{August 29, 2009}
%\author{Alex Nelson\\{\tt Email: \href{mailto:pqnelson@gmail.com}{pqnelson@gmail.com}}}
\begin{document}
\maketitle

\section{Notational Warning!}

We will use slightly odd notation for oriented $p+1$ cells
$X_{p+1}$. We will denote it and its orientation in one fell
swoop by
\begin{equation}%\label{eq:}
X_{p+1} = [X_{p} \quad X^{\prime}_{p}]
\end{equation}
as being oriented \emph{from} $X_{p}$ directed \emph{towards} $X^{\prime}_{p}$.
Also we'll use the notation
\begin{equation}%\label{eq:}
e^{-1}_{n}=-e_{n}
\end{equation}
i.e. inverses as such for edges, vertices, faces, $p$-cells, are
actually additive inverses. We can easily deduce how the boundary
operator behaves, since the orientation and boundaries are clearly noted.

\section{The 3-Torus}
%%
%% torus.tex
%% 
%% Made by Alex Nelson
%% Login   <alex@tomato>
%% 
%% Started on  Sat Aug 29 12:56:09 2009 Alex Nelson
%% Last update Sat Aug 29 12:56:09 2009 Alex Nelson
%%
\begin{figure}[ht]
\includegraphics{img/img.0}
\caption{Identifying the top of the cube with the bottom, we are
  then left to identify one pair of sides of the cube as the
  same, and the remaining pair of sides as the same.}\label{fig:img0}
\end{figure}

Consider the Torus. We will construct it by considering a cube,
which has 6 faces, 12 edges, and 8 vertices. We will identify
opposite pairs of faces as being ``the same'' (glued
together). We will consider what this looks like as far as the
vertices, edges, and faces are concerned.

If we start by examining the vertices, we will cheat and identify
the top face with the bottom face. We are left with 4 distinct
vertices, as seen in figure \ref{fig:img0}. We then identify one
pair of opposite faces as the same, then the other pair of
opposite faces as the same. This is precisely what we do in
figure \ref{fig:img0}. We see that there is thus only one vertex
for the 3-torus.

Now, with regard to the edges, this is a bit trickier. Lets begin
with something easier: faces. We identify opposite pairs of faces
as being ``the same''. There are 6 faces, thus 3 such pairs. We
have 3 faces in the 3-torus (if one is unsatisfied with this
quick construction, one can pull out a regular six sided die and
observe that the each opposite face sums to 7; there are 3
distinct ways to add positive integers up to be 7, precisely 1+6,
2+5, and 3+4).

We have 3 faces, 1 vertex, and now the hard part: edges. We see
in figure \ref{fig:img1} that the distinct faces are colorized,
with the duplicate faces removed. We only, therefore, have to
worry about the edges which are drawn in black. This means that
there are 3 distinct edges. The others are duplicates.

\begin{figure}[ht]
\includegraphics{img/img.1}
\caption{The distinct faces for the 3-torus are colorized, with the duplicate faces removed.}\label{fig:img1}
\end{figure}

We can now start considering the chain complex for the
3-torus. We see that there are 1 vertex, 3 edges, 3 faces, and 1
3-cell. This implies the chain complex should look like
\begin{equation}%\label{eq:}
\mathbb{Z}\xleftarrow{\partial_{0}}\mathbb{Z}^{3}\xleftarrow{\partial_{1}}\mathbb{Z}^{3}\xleftarrow{\partial_{2}}\mathbb{Z}.
\end{equation}
We would like to consider what exactly the boundary maps
$\partial$ are as matrices. We need more information, we need to
pick some orientations for the faces, edges, and the
3-cell. %We would also like to calculate that
% $\partial^{2}=0$ to make certain we're on the right track!
We will use the orientation indicated in figure \ref{fig:img1}. 

We will consider the operator
$\partial_{0}:\mathbb{Z}^{3}\to\mathbb{Z}$. Since there is only
one vertex, that means all edges have \emph{the same} source and
target. This means that the operator $\partial_{0}(e)=t(e)-s(e)$
would be zero, since $t(e)=s(e)$.

If we consider $\partial_{1}:\mathbb{Z}^{3}\to\mathbb{Z}^{3}$
which takes faces to edges, we have something more interesting.
If we let
\begin{equation}%\label{eq:}
af_{1}+bf_{2}+cf_{3} = \begin{bmatrix}a\\b\\c\end{bmatrix},\quad\text{and}\quad
ae_{1}+be_{2}+ce_{3} = \begin{bmatrix}a\\b\\c\end{bmatrix}
\end{equation}
(where $a,b,c\in\mathbb{Z}$), we can consider the $\partial_{1}$
boundary operator as a square matrix. Consider face $f_{1}$ as
drawn in figure \ref{fig:img2}. Since there's only one distinct
vertex, we see travelling along the path in red gets us back to
where we started. This means that
\begin{equation}%\label{eq:}
\partial_{1}\left(\begin{bmatrix}1\\0\\0\end{bmatrix}\right) = \begin{bmatrix}1\\1\\0\end{bmatrix}
\end{equation}
since both $e_{1}$, $e_{2}$ are moving ``along'' the orientation
chosen for $f_{1}$. Referring to figure \ref{fig:img1}, we see
that the other faces are similar. This allows us to explicitly compute
\begin{equation}%\label{eq:}
\partial_{1}\left(\begin{bmatrix}0\\1\\0\end{bmatrix}\right) = \begin{bmatrix}1\\0\\1\end{bmatrix}\quad\text{and}\quad
\partial_{1}\left(\begin{bmatrix}0\\0\\1\end{bmatrix}\right) = \begin{bmatrix}0\\1\\1\end{bmatrix}.
\end{equation}
But this \emph{completely determines} what $\partial_{1}$ must
be, explicitly as a linear operator it is ``merely''
\begin{equation}%\label{eq:}
\partial_{1} = \begin{bmatrix}1 & 1 & 0\\
1 & 0 & 1\\
0 & 1 & 1\end{bmatrix}
\end{equation}
which is a curious operator in its own right. It's also a good
hint that our value for $\partial_{0}$ is correct, since
$\det(\partial_{1})=-2\neq0$ implies $\partial_{1}$ has a trivial
kernel. The only way $\partial_{0}\circ\partial_{1}=0$ is iff
$\partial_{0}=[0,0,0]$.

\begin{figure}[t]
\includegraphics{img/img.2}
\caption{The face $f_{1}$ of the 3-torus, boundary edges $e_{1}$
  and $e_{2}$ are drawn in. The path we use for the chain complex
  is drawn in red.}\label{fig:img2}
\end{figure}

We now are stuck with our last operator $\partial_{2}$. We can
cheat, and use the same argument for $\partial_{0}=[0,0,0]$
to deduce that
\begin{equation}%\label{eq:}
\partial_{2} = \begin{bmatrix}0\\0\\0\end{bmatrix}.
\end{equation}
I have a feeling the preferred route would be to prove that it
must be this via picking a preferred orientation, etc.

\section{The 3-Sphere}
%%
%% sphere.tex
%% 
%% Made by Alex Nelson
%% Login   <alex@tomato>
%% 
%% Started on  Sat Aug 29 16:03:43 2009 Alex Nelson
%% Last update Sat Aug 29 16:03:43 2009 Alex Nelson
%%

Observe the inductive procedure to constructing an $n$-sphere. We
start with $B^{0}$, the 0-ball, which is just a humble point. We
have $S^{0}$, the 0-sphere, which consists of 2 points. The
1-ball is a line segment between 2 points. Now here is the first
inductive case study: the 1-sphere is a circle. It consists of 2
$B^{1}$ with their boundaries glued together: 
\begin{figure}[h]
\begin{center}
\includegraphics{img/img.3}
\end{center}
\end{figure}
The region whose boundary is $S^{1}$ is precisely $B^{2}$. We
should consider the chain complex describing $S^{1}$ before
moving on. We have already drawn in the orientation of the edges
(and labeled them too!), so we need to pick an orientation of the
face itself. We see first of all that there are 2 edges and 2
vertices. This allows us to set up the chain complex as
\begin{equation}%\label{eq:}
\mathbb{Z}^{2}\xleftarrow{\partial_{0}}\mathbb{Z}^{2}
\end{equation}
We have picked an orientation for the edges. We see that
\begin{equation}%\label{eq:}
ae_{1}+be_{2}=\begin{bmatrix}a\\b\end{bmatrix}\quad\text{and}\quad 
av_{1}+bv_{2}=\begin{bmatrix}a\\b\end{bmatrix}
\end{equation}
We can describe the boundary operator by
\begin{equation}%\label{eq:}
\partial_{0}\left(\begin{bmatrix}1\\0\end{bmatrix}\right)=\begin{bmatrix}-1\\1\end{bmatrix}\quad\text{and}\quad
\partial_{0}\left(\begin{bmatrix}0\\1\end{bmatrix}\right)=\begin{bmatrix}1\\-1\end{bmatrix}
\end{equation}
which means that
\begin{equation}%\label{eq:}
\partial_{0}:\begin{bmatrix}a\\b\end{bmatrix}\mapsto\begin{bmatrix}b-a\\a-b\end{bmatrix}
\end{equation}
describes the behavior of the boundary operator.

We can now similarly construct $S^{2}$. We have 2 $B^{2}$ and glue their
boundaries together, or diagrammaticaly:
\begin{figure}[H]
\begin{center}
\includegraphics{img/img.4}
\end{center}
\end{figure}
\noindent We should consider now the chain complex that describes
$S^{3}$. Observe that the two vertices on each of the boundaries
of the $B^{2}$ are identified to be the same. We are thus left
with two vertices for $S^{3}$ (as doodled on the right hand side
of our figure). The two edges for the boundary of $B^{2}$ are
likewise identified to be the same. We have, however, two
\emph{distinct faces} for $S^{3}$ which corresponds to the two
distinct $B^{2}$. This produces the following chain
\begin{equation}%\label{eq:}
\mathbb{Z}^{2}\xleftarrow{\partial_{0}}\mathbb{Z}^{2}\xleftarrow{\partial_{1}}\mathbb{Z}^{2}
\end{equation}
now we are left to deduce the boundary operators. This demands a
choice of orientation of edges, and faces. We have picked an
orientation for the edges as drawn in our figure, let $e_{1}$
begin at $v_{1}$ and end at $v_{2}$; $e_{2}$ begin at $v_{2}$ and
end at $v_{1}$. We then let
\begin{equation}%\label{eq:}
ae_{1}+be_{2}=\begin{bmatrix}a\\b\end{bmatrix}\quad\text{and}\quad 
av_{1}+bv_{2}=\begin{bmatrix}a\\b\end{bmatrix}
\end{equation}
and then deduce that
\begin{equation}%\label{eq:}
\partial_{0}\left(\begin{bmatrix}1\\0\end{bmatrix}\right)=\begin{bmatrix}-1\\1\end{bmatrix}\quad\text{and}\quad
\partial_{0}\left(\begin{bmatrix}0\\1\end{bmatrix}\right)=\begin{bmatrix}-1\\1\end{bmatrix}
\end{equation}
which implies that
\begin{equation}%\label{eq:}
\partial_{0}:\begin{bmatrix}a\\b\end{bmatrix}\mapsto\begin{bmatrix}-a-b\\a+b\end{bmatrix}
\end{equation}
determines one of the boundary operators completely. To determine
the other, we can cheat and figure out what square matrix $A$ is such
that
\begin{equation}%\label{eq:}
\begin{bmatrix} -1 & -1\\1 & 1\end{bmatrix}A = 0
\end{equation}
so the desired relation $\partial^{2}=0$ holds. This would
determine $A$ up to an arbitrary sign (which is determined by the
orientation of the faces). We see that
\begin{equation}%\label{eq:}
A = \begin{bmatrix}a & b\\
-a & -b\end{bmatrix}
\end{equation}
is the structure of $\partial_{1}$. This in fact implies
$(\partial_{0})^{2}=0$. We are forced to pick orientations for
the actual open balls $B^{2}$, but open doing so we see that it's
simply one $B^{2}$ minus the other, implying that (up to a sign
determined by orientation!) $a=b=\pm1$.

The inductive ritual should be clear: we take two copies of
$B^{n}$ and glue the boundaries, the $S^{n-1}$, together to
create a $S^{n}$. More formally, we present it as a theorem.

\begin{thm}%\label{thm:}
Consider the sphere $S^{n}$. Then the chain complex describing it
is
\begin{equation}%\label{eq:}
\mathbb{Z}^{2} \xleftarrow{\partial_{0}}\mathbb{Z}^{2}\xleftarrow{\partial_{1}}\cdots\xleftarrow{\partial_{n-1}}\mathbb{Z}^{2}
\end{equation}
where $\partial_{i}$ are the boundary operators.
\end{thm}
\begin{sketch}
\noindent\textbf{Base Case (n=1):} We have already done this, two $B^{1}$
(line segments) glued together at their boundary $S^{0}$ (two
points) results in the chain complex
\begin{equation}%\label{eq:}
\mathbb{Z}^{2}\xleftarrow{\partial_{0}}\mathbb{Z}^{2}
\end{equation}
where
\begin{equation}%\label{eq:}
\partial_{0} = \begin{bmatrix}-1 & -1\\1 & 1\end{bmatrix}
\end{equation}
up to a sign determined by orientation.

\noindent\textbf{Inductive Hypothesis:} assume this works for arbitrary
$n$. That is, we have
\begin{equation}%\label{eq:}
\mathbb{Z}^{2}\xleftarrow{\partial_{0}}\cdots\xleftarrow{\partial_{n-1}}\mathbb{Z}^{2}
\end{equation}
for our chain complex.

\noindent\textbf{Inductive Case (n+1):} We have the boundaries of the
$B^{n+1}$, which takes care of the first $n-1$ boundary operators
and the first $n$ $\mathbb{Z}^{2}$ components of the chain. We
are still left with two copies of the \emph{interior} of
$B^{n+1}$. This constitutes an additional $\mathbb{Z}^{2}$
component in the chain diagram, and implies that the chain
complex looks like
\begin{equation}%\label{eq:}
\underbracket[0.25pt]{\left(\mathbb{Z}^{2}\xleftarrow{\partial_{0}}\cdots\xleftarrow{\partial_{n-1}}\mathbb{Z}^{2}\right)}_{\text{inductive hypothesis}}\xleftarrow{\partial_{n}}\mathbb{Z}^{2}
\end{equation}
where $\partial$ is the boundary operator. This concludes the
inductive case, and the proof. 
\end{sketch}
For our intentions however, we need to consider
two $B^{3}$, and glue their boundaries together! We see the
scheme presented mildly formally, the boundaries of the two
$B^{3}$ provide the vertices, edges, and faces in the chain
complex:
\begin{equation}%\label{eq:}
\underbracket[0.25pt]{\left(\mathbb{Z}^{2}\xleftarrow{\partial_{0}}\mathbb{Z}^{2}\xleftarrow{\partial_{1}}\mathbb{Z}^{2}\right)}_{\text{from $\partial{B^{3}}$}}\xleftarrow{\partial}\cdots
\end{equation}
and the rest of the chain complex comes from the interior of the
open balls. We have two distinct interiors, so that contributes
$\mathbb{Z}^{2}$ and nothing more. This implies our chain complex
becomes
\begin{equation}%\label{eq:}
\underbracket[0.25pt]{\left(\mathbb{Z}^{2}\xleftarrow{\partial_{0}}\mathbb{Z}^{2}\xleftarrow{\partial_{1}}\mathbb{Z}^{2}\right)}_{\text{from $\partial{B^{3}}$}}\xleftarrow{\partial_{2}}\mathbb{Z}^{2}
\end{equation}
where $\partial_{i}$ are the boundary operators (as usual).


\section{Chain Field Theoretic Calculations}
%%
%% calc.tex
%% 
%% Made by Alex Nelson
%% Login   <alex@black-cherry>
%% 
%% Started on  Wed Jul  4 12:55:27 2012 Alex Nelson
%% Last update Wed Jul  4 12:55:43 2012 Alex Nelson
%%

\section{Alternating Series Tests}
%%
%% altSeriesTest.tex
%% 
%% Made by Alex Nelson
%% Login   <alex@black-cherry>
%% 
%% Started on  Sat Jun 16 17:09:03 2012 Alex Nelson
%% Last update Sun Jun 17 14:45:03 2012 Alex Nelson
%%
%\documentclass{article}
%\usepackage{blog}
%\begin{document}
\N{Proposition} Let
\begin{equation}
\sum^{\infty}_{n=1}(-1)^{n+1}a_{n} =
a_{1}-a_{2}+a_{3}-a_{4}+\dots
\end{equation}
be a given series where $a_{n}>0$. Then the series converges if
\begin{enumerate}
\item $a_{n}\geq a_{n+1}$ for each $n$;
\item $\displaystyle\lim_{n\to\infty}a_{n}=0$.
\end{enumerate}
\begin{proof}
We have in the sequence of partial sums
\begin{equation}
S_{2n}=\underbrace{(a_{1}-a_{2})}_{\geq0}
+\underbrace{(a_{3}-a_{4})}_{\geq0}+\dots+\underbrace{(a_{2n-1}-a_{2n})}_{\geq0}\geq0
\end{equation}
so
\begin{equation}
0\leq S_{2}\leq S_{4}\leq \dots\leq S_{2n}\leq\dots
\end{equation}
But we also have
\begin{equation}
\begin{aligned}
S_{2n} &= a_{1} - (a_{2}-a_{3})-(a_{4}-a_{5})-(\dots)-a_{2n}\\
&\leq a_{1}.
\end{aligned}
\end{equation}
So
\begin{equation}
\lim_{n\to\infty}S_{2n}=L\leq a_{1}
\end{equation}
and
\begin{equation}
\begin{aligned}
\lim_{n\to\infty}S_{2n+1} &=\lim_{n\to\infty}(S_{2n}+a_{2n+1})\\
&=\left(\lim_{n\to\infty}S_{2n}\right)+\left(\lim_{n\to\infty}a_{2n+1}\right)\\
&=L+0=L.
\end{aligned}
\end{equation}
Conclusion: $\displaystyle\sum^{\infty}_{n=1}(-1)^{n+1}a_{n}=L$.
\end{proof}
\begin{example}
Consider the series
\begin{equation}
\sum^{\infty}_{n=1}\frac{(-1)^{n+1}}{n}
\end{equation}
We see that $1/n\geq1/(n+1)$ for each $n$, and 
\begin{equation}
\lim_{n\to\infty}\frac{1}{n}=0.
\end{equation}
Thus the Alternating Series Test implies the alternating Harmonic
series converges.
\end{example}
\begin{example}
Consider the series
\begin{equation}
\sum^{\infty}_{n=1}\frac{(-1)^{n+1}n}{(n+1)(n+2)}
\end{equation}
We see
\begin{equation}
\frac{n}{(n+1)(n+2)}\geq\frac{n+1}{(n+2)(n+3)};
\end{equation}
why? Well, multiply both sides by $(n+2)$ and we get
\begin{equation}
\frac{n}{n+1}\geq \frac{n+1}{n+3}
\end{equation}
Cross multiplication gives us
\begin{equation}
n(n+3)\geq (n+1)^{2} \iff n^{2}+3n \geq n^{2}+2n+1.
\end{equation}
Subtracting $n^{2}+2n$ from both sides gives us
\begin{equation}
n\geq1.
\end{equation}
So our series satisfies the first condition for the alternating
series test.

We also see that
\begin{equation}
\frac{n}{(n+1)(n+2)}\approx\frac{1}{n}
\end{equation}
for ``large $n$''. So we see 
\begin{equation}
\lim_{n\to\infty}\frac{n}{(n+1)(n+2)}=0.
\end{equation}
Thus our series satisfies the criteria for the alternating series
test, which implies convergence.
\end{example}

\N{Definitions} A series $\sum a_{n}$ is \textbf{``Absolutely Convergent''}
if $\sum|a_{n}|$ converges.

On the other hand, if $\sum|a_{n}|$ is divergent, then we call
the series $\sum a_{n}$ \textbf{``Conditionally Convergent''}.

\begin{example}
We see that $\sum(-1)^{n}/n$ is conditionally convergent, since
$\sum 1/n$ diverges.
\end{example}
\begin{example}
The series $\sum (-1)^{n}n^{-3/2}$ is absolutely convergent since
the integral test tells us $\sum n^{-3/2}$ converges.
\end{example}

\medbreak\noindent\emph{Question:} let $\sum a_{n}$ be a
convergent series, and $a_{n}\geq0$ for each $n$. Does the series
$\sum (-1)^{n}a_{n}$ converge? 

Stop and think before continuing!

\N{Absolute Convergence Test} 
If $\sum |a_{n}|$ conveges, then $\sum a_{n}$ converges.

\begin{proof}
We see first that 
\begin{equation}
-|a_{n}|\leq a_{n}\leq|a_{n}|\quad\mbox{for each }n
\end{equation}
So what? Well, we see that
\begin{equation}
0\leq a_{n}+|a_{n}|\leq2|a_{n}|\quad\mbox{for each }n
\end{equation}
Since $\sum|a_{n}|$ converges, we see $\sum2|a_{n}|$ converges
too. But by the comparison test, we see
\begin{equation}
\sum^{\infty}_{n=1}a_{n}+|a_{n}|
\end{equation}
converges. So what? We haven't proven $\sum a_{n}$ converges,
have we? Consider the following trick
\begin{equation}
\sum a_{n} = \underbrace{\sum
  a_{n}+|a_{n}|}_{\text{converges}}-\underbrace{\sum
  |a_{n}|}_{\text{converges}}.
\end{equation}
Therefore the series $\sum a_{n}$ converges.
\end{proof}
\begin{example}
Consider the series
\begin{equation}\label{eq:altSeriesTest:ex4:eq1}
\sum^{\infty}_{n=1}\frac{\cos(n)}{n^{3/2}}.
\end{equation}
What to do? We know
\begin{equation}
\left|\frac{\cos(n)}{n^{3/2}}\right|\leq\frac{1}{n^{3/2}}.
\end{equation}
So what? We know $\sum n^{-3/2}$ converges by the integral
test. Then
\begin{equation}
\sum^{\infty}_{n=1}\left|\frac{\cos(n)}{n^{3/2}}\right|
\end{equation}
converges by comparison. By the absolute convergence test, we know
our series in Equation \eqref{eq:altSeriesTest:ex4:eq1} converges.
\end{example}


%\end{document}



\section{Other Alternating Series Tests}
%%
%% altRatioTest.tex
%% 
%% Made by Alex Nelson
%% Login   <alex@black-cherry>
%% 
%% Started on  Sun Jun 17 14:33:38 2012 Alex Nelson
%% Last update Sun Jun 17 15:36:44 2012 Alex Nelson
%%
%% \documentclass{article}
%% \pdfinfo{ /CreationDate (20120617143338)}
%% \usepackage{blog}
%% \begin{document}
\M
We have another couple of methods testing if an alternating
series converges. It's worth knowing as many different ways as
possible, because sometimes one doesn't work well (or at all).

We will consider a couple tests. For each test we provide a
proof that it works, and a couple examples.

\N{Alternating Ratio Test}
Consider the series 
\begin{equation}
\sum^{\infty}_{n=0}(-1)^{n}a_{n}
\end{equation}
Let
\begin{equation}
L = \lim_{n\to\infty}\frac{a_{n+1}}{a_{n}}.
\end{equation}
\begin{enumerate}
\item If $L<1$, then the series $\sum (-1)^{n}a_{n}$ conveges
  absolutely.
\item If $L>1$ (or $L=\infty$), then the series
  $\sum (-1)^{n}a_{n}$ diverges.
\end{enumerate}

\begin{proof}[Proof of Convergence]
We will prove if $L<1$, then the series
\begin{equation}
\sum^{\infty}_{n=0}a_{n}
\end{equation}
converges. The absolute convergence test implies the alternating
series will converge. What to do?

We first consider some $\varepsilon$ satisfying
$L<\varepsilon<1$. (Can we do this? Sure, pick $(L+1)/2$, and
we're good!) Since we suppose $L<1$, then there exists an $N$
such that
\begin{equation}
\left|\frac{a_{n+1}}{a_{n}}\right|<r\quad\mbox{for any }n\geq N.
\end{equation}
We have
\begin{equation}
\begin{aligned}
a_{N+1}&<r a_{N}\\
a_{N+2}&<r a_{N+1}<r^{2}a_{N}\\
a_{N+3}&<r^{3}a_{N}\\
a_{N+k}&<r^{k}a_{N}
\end{aligned}
\end{equation}
So we form a geometric series
\begin{equation}
\sum^{\infty}_{k=0}a_{N}r^{k} = \frac{a_{N}}{1-r}
\end{equation}
which bounds the ``most'' of our series
\begin{equation}
0\leq\sum^{\infty}_{k=0}a_{N+k}\leq\sum^{\infty}_{k=0}a_{N}r^{k}
\end{equation}
The comparison test tells us that ``most'' of our series
converges. But what about our whole series? We write it as
\begin{equation}
\sum^{\infty}_{n=0}a_{n} = \underbrace{\sum^{N-1}_{n=0}a_{n}}_{\text{finite}}+\underbrace{\sum^{\infty}_{k=0}a_{N+k}}_{\text{converges}}
\end{equation}
so it converges.
\end{proof}

\begin{proof}[Proof of Divergence]
We assume that $L>1$. The ratio $a_{n+1}/a_{n}$ will eventually
be greater than 1, too. So there exists an $N$ such that
\begin{equation}
\left|\frac{a_{n+1}}{a_{n}}\right|>1\quad\mbox{for any }n\geq N.
\end{equation}
So we see $|a_{n+1}|>|a_{n}|$ whenever $n\geq N$, thus
\begin{equation}
\lim_{n\to\infty}a_{n}\not=0.
\end{equation}
Thus it's impossible for the series $\sum a_{n}$ to converge!
\end{proof}


\begin{remark}
We only really proved that the series doesn't converge
\emph{absolutely}. If we pick some $r$ between $1<r<L$, then
there exists an $N$ such that
\begin{equation}
\left|\frac{a_{n+1}}{a_{n}}\right|>r\quad\mbox{for any }n\geq N.
\end{equation}
So we have
\begin{equation}
a_{N+k} > r^{k}a_{N}.
\end{equation}
We have our series
\begin{equation}
\begin{aligned}
\sum^{\infty}_{k=1}(-r)^{k}a_{N}
&= a_{N} \sum^{\infty}_{k=1}(r^{2k}-r^{2k-1})\\
&= a_{N} (r-1)\sum^{\infty}_{k=1}r^{2k}.
\end{aligned}
\end{equation}
We see since $r>1$ that the series
\begin{equation}
\sum^{\infty}_{k=1}r^{2k}\quad\mbox{diverges}
\end{equation}
Thus
\begin{equation}
\sum^{\infty}_{k=1}(-r)^{k}a_{N}\quad\mbox{diverges}
\end{equation}
We see then that the series
\begin{equation}
\sum^{\infty}_{k=1}a_{N+k}\geq\sum^{\infty}_{k=1}(-r)^{k}a_{N}
\end{equation}
diverges by the comparison test.
\end{remark}


\N{The Root Test}
Consider the series 
\begin{equation}
\sum^{\infty}_{n=1}(-1)^{n}a_{n}
\end{equation}
Let
\begin{equation}
L = \lim_{n\to\infty}\sqrt[n]{a_{n}}
\end{equation}
\begin{enumerate}
\item If $L<1$, then the series converges absolutely;
\item If $L>1$ (or $L=\infty$), then the series diverges.
\end{enumerate}

\begin{proof}
It's similar to the ratio test, for the convergent case we pick
some $r$ satisfying $L<r<1$. Then we have some $N$ satisfying
\begin{equation}
\sqrt[n]{a_{n}}<r\quad\mbox{for any }n\geq N.
\end{equation}
which implies
\begin{equation}
a_{n}<r^{n}.
\end{equation}
Thus we have
\begin{equation}
\sum^{\infty}_{k=1}a_{N+k}<\sum^{\infty}_{k=1}r^{N+k}<\sum^{\infty}_{k=0}r^{k}=\frac{1}{1-r}
\end{equation}
which by the  comparison test implies convergence. The proof for
divergence is similar.
\end{proof}
\begin{remark}
When $L=1$, the ratio test doesn't say anything about convergence
or divergence for the series.
\end{remark}

%\end{document}

\section{Power Series}
%%
%% powerSeries.tex
%% 
%% Made by Alex Nelson
%% Login   <alex@black-cherry>
%% 
%% Started on  Sat Jun 16 17:37:15 2012 Alex Nelson
%% Last update Sun Jun 17 15:45:52 2012 Alex Nelson
%%
%\documentclass{article}
%\usepackage{blog}
%\begin{document}

\N{Definitions}
Let $x$ be a variable. a series of the form
\begin{equation}
\sum^{\infty}_{n=0}a_{n}x^{n}=a_{0}+a_{1}x+a_{2}x^{2}+\dots+a_{n}x^{n}+\dots
\end{equation}
is called a \textbf{``Power Series about $x=0$''}.

A series of the form
\begin{equation}
\sum^{\infty}_{n=0}b_{n}(x-a)^{n}=b_{0}+b_{1}(x-a)+b_{2}(x-a)^{2}+\dots+b_{n}(x-a)^{n}+\dots
\end{equation}
is called a \textbf{``Power Series about $x=a$''}.

\begin{example}
Consider the series
\begin{equation}
f(x)=\sum^{\infty}_{n=0}\frac{n+1}{2^{n}}x^{n}.
\end{equation}
For what values of $x$ does this series converge? When will the
series diverge?\more{}

\emph{Solution.} Using the absolute ratio test, we see
\begin{equation}
\begin{aligned}
\lim_{n\to\infty}\left|\frac{\displaystyle\frac{n+2}{2^{n+1}}x^{n+1}}{\displaystyle\frac{n+1}{2^{n}}x^{n}}\right|
&=\lim_{n\to\infty}\frac{n+2}{n+1}\cdot\frac{2^{n}}{2^{n+1}}\cdot|x|\\
&=(1)\cdot\left(\frac{1}{2}\right)\cdot|x|\\
&=\frac{1}{2}|x|
\end{aligned}
\end{equation}
For convergence, we need
\begin{equation}
\frac{1}{2}|x|<1\quad\Longrightarrow\quad|x|<2.
\end{equation}
So divergence would be when
\begin{equation}
|x|>2.
\end{equation}
We need to check the $|x|=2$ case. Observe, for $x=2$, we have
\begin{equation}
f(2)=\sum^{\infty}_{n=0}\frac{n+1}{2^{n}}2^{n}=\sum^{\infty}_{n=0}n+1
\end{equation}
which diverges. And at $x=-2$ we have
\begin{equation}
f(-2)=\sum^{\infty}_{n=0}\frac{n+1}{2^{n}}(-2)^{n}=\sum^{\infty}_{n=0}(-1)^{n}(n+1)
\end{equation}
which still diverges!
\end{example}

\N{Theorem}
Let $c\not=0$ and 
\begin{equation}
f(x)=\sum^{\infty}_{n=0}a_{n}x^{n}
\end{equation}
converge at $x=c$. Then $f(x)$ converges absolutely for
$|x|<|c|$.

\begin{proof}
Let $x$ be any value such that $|x|<|c|$. Since $f(c)$ converges,
there exists an $M$ such that
\begin{equation}
|a_{n}c^{n}|\leq M\quad\mbox{for any }n.
\end{equation}
Well, we see
\begin{equation}
|x/c|\leq 1
\end{equation}
so
\begin{equation}
a_{n}x^{n}=a_{n}c^{n}\left(\frac{x}{c}\right)^{n}
\end{equation}
and moreover
\begin{equation}
|a_{n}x^{n}| = |a_{n}c^{n}|\cdot\left|\frac{x}{c}\right|^{n}\leq M\left|\frac{x}{c}\right|^{n}
\end{equation}
But observe the series
\begin{equation}
g(x)=\sum^{\infty}_{n=0}M\cdot\left|\frac{x}{c}\right|^{n}
\end{equation}
is a geometric series which converges since $|x/c|\leq1$.

Therefre the series $\sum|a_{n}x^{n}|$ converges by comparison,
and the absolute convergence test tells us $\sum a_{n}x^{n}$
converges. Therefore $\sum a_{n}x^{n}$ is absolutely convergent.
\end{proof}


\begin{example}
Find the values of $x$ for which
\begin{equation}
\sum^{\infty}_{n=1}\frac{(x+7)^{n}}{\sqrt{n}}
\end{equation}
converge.

\emph{Solution}: Using the absolute ratio test, we find
\begin{equation}
\begin{aligned}
\lim_{n\to\infty}\left|\frac{\left(\displaystyle\frac{(x+7)^{n+1}}{\sqrt{n+1}}\right)}{\left(\displaystyle\frac{(x+7)^{n}}{\sqrt{n}}\right)}\right|
&=\lim_{n\to\infty}\frac{\sqrt{n}}{\sqrt{n+1}}|x+7|\\
&=|x+7|.
\end{aligned}
\end{equation}
We get convergence for $|x+7|<1$. So
\begin{equation}
-1<x+7<1\quad\Longrightarrow\quad-8<x<-6.
\end{equation}
We have to check the boundary cases.

When $x=-6$, we have
\begin{equation}
\sum^{\infty}\frac{(7-6)^{n}}{\sqrt{n}}=\sum^{\infty}_{n=1}\frac{1}{\sqrt{n}}
\end{equation}
which diverges by the integral test (or the comparison test with
the Harmonic series). For $x=-8$ we have our series become
\begin{equation}
\sum^{\infty}_{n=1}\frac{(-1)^{n}}{\sqrt{n}}
\end{equation}
which converges by the alternating series test.
\end{example}


%\end{document}

\section{Taylor Series}
%%
%% taylorSeries.tex
%% 
%% Made by Alex Nelson
%% Login   <alex@black-cherry>
%% 
%% Started on  Sun Jun 17 15:47:04 2012 Alex Nelson
%% Last update Mon Jun 18 10:34:00 2012 Alex Nelson
%%
\M
Consider the function
\begin{equation}
f(x)  = \sum^{\infty}_{n=0}b_{n}(x-a)^{n} = b_{0} + b_{1}(x-a) + \dots.
\end{equation}
We suppose it is smooth (i.e., has infinitely many
derivatives). We see
\begin{equation}
\begin{aligned}
f'(x) &= \sum^{\infty}_{n=1}b_{n}n(x-a)^{n-1}\\
f'(a) &= b_{1}
\end{aligned}
\end{equation}
and
\begin{equation}
\begin{aligned}
f''(x) &= \sum^{\infty}_{n=2}b_{n}n(n-1)\cdot(x-a)^{n-2}\\
f''(a) &= 2b_{2}.
\end{aligned}
\end{equation}
We also have
\begin{equation}
\begin{aligned}
f'''(x) &= \sum^{\infty}_{n=3}b_{n}n(n-1)(n-2)\cdot(x-a)^{n-3}\\
f'''(a) &= 3\cdot2\cdot1 b_{3} = 6b_{3}.
\end{aligned}
\end{equation}
The general case appears to be
\begin{equation}
\left.\frac{\D^{n}}{\D x^{n}}f(x)\right|_{x=a}=n! b_{n}
\end{equation}
Thus the coefficients are
\begin{equation}
b_{n} = \left.\frac{1}{n!}\frac{\D^{n}}{\D x^{n}}f(x)\right|_{x=a}
\end{equation}
and we can reconstruct the function $f(x)$.

\N{Definition}
Let $f(x)$ be any function. The \textbf{``Taylor Series about $x=a$''}
is a series
\begin{equation}
\sum^{\infty}_{n=0}f^{(n)}(a)\cdot(x-a)^{n}
\end{equation}
When $a=0$, it's called a \emph{MacLaurin Series}.

\begin{remark}
If we have the MacLaurin series for a function, and if the series
converges for any value of $x$, then we can use the MacLaurin
series as a synonym for the original function. That's the
usefulness of MacLaurin series.
\end{remark}
\begin{remark}
The Taylor series helps us compute $f(x+h)$ when $h\ll x$ and
when $f(x)$ is known. For example, $\sqrt{1.001}$ can be computed
using the Taylor series of $\sqrt{1+x}$ about $x=1$.
\end{remark}

\begin{example}
Consider the function $\exp(x)$. We see that
\begin{equation}
\frac{\D\exp(x)}{\D x}=\exp(x).
\end{equation}
Thus the MacLaurin series for the exponential function is
\begin{equation}
\exp(x)=\sum^{\infty}_{n=0}\frac{x^{n}}{n!}
\end{equation}
Where does this series converge?

Using the absolute ratio test, we find
\begin{equation}
\lim_{n\to\infty}\frac{x^{n+1}/(n+1)!}{x^{n}/n!} =
\lim_{n\to\infty}\frac{x}{n+1}=0.
\end{equation}
So the series converges \emph{for any value of $x$}.
\end{example}
\begin{example}
Consider the sine function $\sin(x)$. What is its MacLaurin
series? Writing
\begin{equation}
\sum^{\infty}_{n=0}c_{n}x^{n}=\sin(x)
\end{equation}
we have
\begin{equation}
\begin{aligned}
c_{0}&=\sin(0)=0\\
c_{1}&=\cos(0)=1\\
c_{2}&=\frac{-\sin(0)}{2!}=0\\
c_{3}&=\frac{-\cos(0)}{3!}=-1/3!
\end{aligned}
\end{equation}
The coefficients are sort of ``periodic'' in the sense that only
the odd ones remain, and their sign alternates. We have
\begin{equation}
c_{n} = c_{2k-1} = \frac{(-1)^{n}}{n!} = \frac{(-1)^{k+1}}{(2k-1)!}
\end{equation}
thus the MacLaurin series is
\begin{equation}
\sin(x)=\sum^{\infty}_{n=1}\frac{(-1)^{n+1}x^{n}}{(2n-1)!}.
\end{equation}
Where does this converge? 

Using the ratio test, we have
\begin{equation}
\lim_{n\to\infty}\left|\frac{\left(\displaystyle\frac{x^{2n+3}}{(2n+3)!}\right)}{\left(\displaystyle\frac{x^{2n+1}}{(2n+1)!}\right)}\right|
=\lim_{n\to\infty}\frac{x^{2}}{(2n+1)(2n+2)} =  0.
\end{equation}
Thus it converges for any value of $x$.
\end{example}


\begin{example}

Lets find the MacLaurin series for $\cos(x)$. We see that
\begin{equation}
\cos(x)=\sum^{\infty}_{n=0}b_{n}x^{n},
\end{equation}
we need to find the $b_{n}$. We see that
\begin{equation}
\begin{aligned}
b_{0} &= \cos(0) = 1\\
b_{1} &= -\sin(0) = 0\\
b_{2} &= \frac{-1}{2!}\cos(0) = -1/2\\
b_{3} &= \frac{1}{3!}\sin(0) = 0\\
b_{2k} &= \frac{(-1)^{k}}{(2k)!}
\end{aligned}
\end{equation}
So we have the MacLaurin series be
\begin{equation}
\cos(x) = \sum^{\infty}_{n=0}\frac{(-1)^{n}}{(2n)!}x^{2n}.
\end{equation}
Where will it converge? Using the absolute ratio test, we have
\begin{equation}
\begin{aligned}
\lim_{n\to\infty}\left|\frac{\left(\displaystyle\frac{(-1)^{n+1}x^{2n+2}}{(2n+2)!}\right)}{\left(\displaystyle\frac{(-1)^{n}x^{2n}}{(2n)!}\right)}\right|
&=\lim_{n\to\infty}\frac{x^{2}}{(2n+1)(2n+2)}\\
&=0.
\end{aligned}
\end{equation}
Thus the MacLaurin series for $\cos(x)$ converges for any value
of $x$.
\end{example}



\N{Euler's Formula}
Recall Euler's formula states
\begin{equation}
\exp(\I\theta)=\cos(\theta)+\I\sin(\theta)
\end{equation}
and we have the MacLaurin  series for $\exp(x)$. Setting
$x=\I\theta$, we find
\begin{equation}
\begin{aligned}
\exp(\I\theta) &= \sum^{\infty}_{n=0}\frac{(\I\theta)^{n}}{n!}\\
&=1+\I\theta-\frac{\theta^{2}}{2!}-\frac{\I\theta^{3}}{3!}+\frac{\theta^{4}}{4!}+\dots
\end{aligned}
\end{equation}
Gathering the real and imaginary parts together we get
\begin{equation}\label{eq:taylorSeries:expITheta:comparison}
\exp(\I\theta) =
\left(\sum^{\infty}_{n=0}\frac{(-1)^{n}x^{2n}}{(2n)!}\right)
+\I\left(\sum^{\infty}_{n=0}\frac{(-1)^{n}x^{2n+1}}{(2n+1)!}\right)
\end{equation}
But look: the imaginary part is precisely the MacLaurin series
for $\sin(x)$! And the real part is the MacLaurin series for
$\cos(x)$, too! So what did we do? We just derived Euler's
formula. 

\N{Taylor Series Makes Approximations}
Consider the function
\begin{equation}
f(x)=\sqrt{x}.
\end{equation}
Its Taylor series about $x=1$ is \emph{the same} as the MacLaurin
series for the function
\begin{equation}
g(h)=\sqrt{1+h}.
\end{equation}
Just write $x=1+h$. For small $0<h\ll1$, we can use the first
couple of terms in the Taylor series as an approximate value. So
what's the first 6 nonzero terms in the Taylor series? We see
\begin{subequations}
\begin{equation}
f(x)=x^{1/2}\quad\implies\quad f(1)=1
\end{equation}
describes the constant term. The linear term has coefficient
\begin{equation}
f'(x)=\frac{x^{-1/2}}{2}\quad\implies\quad f'(1)=\frac{1}{2}.
\end{equation}
The quadratic term
\begin{equation}
f''(x)=\frac{-x^{-3/2}}{2^{2}}\quad\implies\quad f''(1)=-1/2^{2}.
\end{equation}
Observe how the exponent behaves when differentiating:
$(1/2)-n$. The numerator is always odd, the denominator doesn't
change. So the $n^{th}$ derivative would be
\begin{equation}
f^{(n)}(x)=\frac{1\cdot3\cdot5\cdot(\dots)\cdot(2n-1)}{2^{n}}(-1)^{n-1}x^{(1-2n)/2}
\end{equation}
This gives us our coefficients! We then have
\begin{equation}
c_{n} = \frac{(-1)^{n-1}(2n)!}{(n!2^{n})^{2}}
\end{equation}
for $n>0$. Observe the numerator appears odd, but we can justify
it thus:
\begin{equation}
\begin{aligned}
\frac{(2n)!}{n!2^{n}}
&= \frac{1\cdot2\cdot3\cdot(\dots)\cdot(2n)}{(1\cdot2\cdot(\dots)\cdot n)2^{n}}\\
&= \frac{\bigl(1\cdot3\cdot(\dots)\cdot(2n-1)\bigr)\bigl(2\cdot4\cdot(\dots)\cdot 2n\bigr)}{(2\cdot4\cdot(\dots)\cdot2n)}\\
&= 1\cdot3\cdot(\dots)\cdot(2n-1)\frac{\bigl(2\cdot4\cdot(\dots)\cdot 2n\bigr)}{(2\cdot4\cdot(\dots)\cdot2n)}\\
&=
1\cdot3\cdot(\dots)\cdot(2n-1)\left(\vphantom{\frac{a}{a}}1\right)\\
&= 1\cdot3\cdot(\dots)\cdot(2n-1)
\end{aligned}
\end{equation}
\end{subequations}
At any rate, this gives us a polynomial expression for $f(1+h)$
as
\begin{equation}
f(1+h) = \left(1+\sum^{6}_{n=1}\frac{(-1)^{n-1}(2n)!}{(n!2^{n})^{2}}h^{n}\right)
+ R_{6}(h)
\end{equation}
where $R_{6}(h)$ is the ``error'' term. What does the error term
tell us? Precisely how good or bad our polynomial approximates
$f(1+h)$. That is to say, how $(1+c_{1}h+\dots+c_{6}h^{6})$
approximates $f(1+h)$.

\emph{Question}: how do we find the error term?

\subsection{Taylor Polynomials}
%%
%% taylorErrorPt1.tex
%% 
%% Made by Alex Nelson
%% Login   <alex@black-cherry>
%% 
%% Started on  Mon Jun 18 10:34:28 2012 Alex Nelson
%% Last update Mon Jun 18 10:51:51 2012 Alex Nelson
%%

\M
So last time we constructed a polynomial using the first $n$
terms in the Taylor series. This ``Taylor polynomial''
approximated our function, and we want to know \emph{how well does it approximate?}

We will derive the Taylor series differently. Our derivation will
give us a natural error term. \more{}

\N{Set Up} Lets consider $f(x)$ which is sufficiently nice around
$x=0$ (i.e., it has enough derivatives at $x=0$). The fundamental
theorem of calculus tells us that
\begin{equation}\label{eq:taylorErr1:line1}
f(x) = f(0) + \int^{x}_{0}f'(t)\,\D t.
\end{equation}
Wonderful.

\N{Linear Approximation}
We now use integration by parts for the integral in Equation \eqref{eq:taylorErr1:line1}:
\begin{equation}
\begin{aligned}
\int^{x}_{0}f'(t)\,\D t &=
\left.tf'(t)\right|^{x}_{0}-\int^{x}_{0}tf''(t)\,\D t\\
&= xf'(x)-0f'(0) - \int^{x}_{0}tf''(t)\,\D t\\
&= xf'(x) - \int^{x}_{0}tf''(t)\,\D t.
\end{aligned}
\end{equation}
Now we can plug in Equation \eqref{eq:taylorErr1:line1} for the
first term in the right hand side:
\begin{equation}
\begin{aligned}
xf'(x) - \int^{x}_{0}tf''(t)\,\D t &=
x\left(f'(0)+\int^{x}_{0}f'(t)\,\D t\right) - \int^{x}_{0}tf''(t)\,\D t\\
&= xf'(0) - \int^{x}_{0}(t-x)f''(t)\,\D t.
\end{aligned}
\end{equation}
We plug this back into Equation \eqref{eq:taylorErr1:line1}
\begin{equation}
\begin{aligned}
f(x) &= f(0) + \int^{x}_{0}f'(t)\,\D t\\
&= f(0) +\left[ xf'(0) - \int^{x}_{0}(t-x)f''(t)\,\D t\right].
\end{aligned}
\end{equation}
Observe the first two terms are precisely the linear
approximation to $f(x)$. What's the third term? \emph{The error term!}
But we're not done yet!

\N{Inductive Procedure}
We can iterate a procedure to get a better and better
approximation. Performing integration by parts on
\begin{equation}
\int^{x}_{0}(t-x)f''(t)\,\D t
\end{equation}
will give us the next term in our approximation plus an
integral. The integral gives us the error for using a quadratic
approximation. 

Lets do it! We use integration by parts 
\begin{equation}
\begin{aligned}
\int^{x}_{0}(t-x)f''(t)\,\D t &=
\left.\frac{(t-x)^{2}}{2!}f''(t)\right|^{x}_{0}-\int^{x}_{0}\frac{(t-x)^{2}}{2!}f'''(t)\,\D t\\
&=\frac{-(-x)^{2}}{2!}f''(0)-\int^{x}_{0}\frac{(t-x)^{2}}{2!}f'''(t)\,\D t
\end{aligned}
\end{equation}
So our approximation becomes
\begin{equation}
\begin{aligned}
f(x) &= f(0) +  xf'(0) - \left[\int^{x}_{0}(t-x)f''(t)\,\D  t\right]\\
&=f(0) + xf'(0) - \left[\frac{-(-x)^{2}}{2!}f''(0)-\int^{x}_{0}\frac{(t-x)^{2}}{2!}f'''(t)\,\D t\right]\\
&=f(0) + xf'(0) + \frac{x^{2}}{2!}f''(0)+\left[\int^{x}_{0}\frac{(t-x)^{2}}{2!}f'''(t)\,\D t\right]
\end{aligned}
\end{equation}
The brackted term in the final line is the error term for using
the quadratic approximation. But look: the quadratic
approximation corresponds to the first 3 terms of the Taylor series!
When we do this iterative procedure on the error term,
integrating by parts will produce a higher order
approximation. And for free, we get the error term telling us how
good our approximation is!

\section{Geometry of Space}
%%
%% threeD.tex
%% 
%% Made by Alex Nelson
%% Login   <alex@black-cherry>
%% 
%% Started on  Wed Jun 20 13:38:30 2012 Alex Nelson
%% Last update Thu Jun 21 17:40:32 2012 Alex Nelson
%%

\N{The 3 Dimensional Coordinate System}
The $x$, $y$, $z$ axes are perpendicular to each other. We will
doodle three dimensions as follows:
\begin{center}
\includegraphics{img/threeD.0}
\end{center}
The formula for distance between two points $(x_{0}, y_{0},
z_{0})$ and $(x_{1},y_{1},z_{1})$ would be
\begin{equation}
s = \sqrt{(x_{1}-x_{0})^{2}+(y_{1}-y_{0})^{2}+(z_{1}-z_{0})^{2}}.
\end{equation}
A sphere with its center at $(a,b,c)$ would be the points which
are a distance $r$ away from the center. So we would have
\begin{equation}
S^{2} = \{\,(x,y,z) : \sqrt{(x-a)^{2}+(y-b)^{2}+(z-c)^{2}}=r\,\}
\end{equation}
Usually the formula is given as
\begin{equation}
(x-a)^{2}+(y-b)^{2}+(z-c)^{2}=r^{2}
\end{equation}
describes a sphere.

\N{Vectors}
A \textbf{``Vector''} is a directed line segment.

We know a directed line segment has both length (magnitude) and
direction, so any two directed line segments with the same length
\emph{and} direction represent the same vector.

Vectors are ``transportable'' in the sense that we may translate
their base point. We will represent the length of a vector
$\vec{u}$ as $\|\vec{u}\|$ or $|\vec{u}|$.

The notation for a vector would be $\langle x,y\rangle$ (in two
dimensions) or $\langle x,y,z\rangle$ (in three dimensions). The
vector from $(0,0,0)$ to $(x,y,z)$ is given as $\langle x,y,z\rangle$.
For $P=(-1,4,7)$ and $Q=(2,5,3)$, then the vector from $P$ to $Q$
is denoted $\overrightarrow{PQ}$.

We can add vectors graphically:
\begin{center}
\includegraphics{img/threeD.1}
\end{center}
Subtraction would amount to $\vec{A}-\vec{B}=\vec{A}+(-\vec{B})$,
and graphically this is:
\begin{center}
\includegraphics{img/threeD.2}
\end{center}
Algebraically, if
\begin{equation}
\vec{A}=\langle x_{a},y_{a},z_{a}\rangle,\quad
\vec{B}=\langle x_{b},y_{b},z_{b}\rangle
\end{equation}
then
\begin{equation}
\begin{aligned}
\vec{A}+\vec{B} &= \langle x_{a}+x_{b}, y_{a}+y_{b},
z_{a}+z_{b}\rangle\\
\vec{A}-\vec{B} &= \langle x_{a}-x_{b}, y_{a}-y_{b},
z_{a}-z_{b}\rangle
\end{aligned}
\end{equation}
This describes vector addition and subtraction on the components.

Note in two-dimensional space, the vectors
\begin{equation}
\widehat{\textbf{\i}}=\langle 1,0\rangle
\quad\mbox{and}\quad
\widehat{\textbf{\j}}=\langle 0,1\rangle
\end{equation}
are unit vectors (i.e., vectors whose length is $1$). They are
also called \emph{basis vectors} since any other vector $\vec{v}$
in two-dimensions can be written as
\begin{equation}
\begin{aligned}
\vec{v}&=\langle v_{x}, v_{y}\rangle\\
&=\langle v_{x},0\rangle + \langle0,v_{y}\rangle\\
&=v_{x}\langle1,0\rangle + v_{y}\langle0,1\rangle\\
&=v_{x}\widehat{\textbf{\i}}+v_{y}\widehat{\textbf{\j}}
\end{aligned}
\end{equation}
where $\langle v_{x}$ and $v_{y}$ are called the vector's 
\emph{components}. Note that the components of the vector depends on a
choice of coordinates (i.e., a choice of basis vectors).

The last notion we will discuss: given any vector $\vec{v}$ which
is nonzero, then we can construct the unit vector
\begin{equation}
\widehat{v} = \frac{\vec{v}}{\|\vec{v}\|}
\end{equation}
which has magnitude 1. We use hats to indicate unit vectors, and
arrows for arbitrary vectors.

\N{Caution:} Everything stated about vectors is a
half-truth. Really, these are ``tangent vectors'' which has a
base point and a vector part (i.e., where we stick the line
segment, and the directed line segment itself). We can only
add/subtract two tangent vectors if they have the same base point. 
But since we work in Euclidean space (which is flat), we can
transport vectors without a problem. This is a very special
situation! 

Since this is never mentioned, often students become confused
when they finish vector calculus and begin studying linear
algebra. Linear algebra fixes a base point, and considers the
collection of all vectors sharing the same base point. This is
the honest definition of a vector. 

\begin{remark}
We will also use the phrase ``three-space'' instead of
``three-dimensional space'', and ``two-space'' replacing
``two-dimensional space''. In general $n$-space is
$n$-dimensional Euclidean space.
\end{remark}

\subsection{Dot Products}
\N{Definition}
Given two vectors
\begin{equation}
\vec{u}=u_{1}\widehat{\textbf{\i}}+u_{2}\widehat{\textbf{\j}}+u_{3}\widehat{\textbf{k}},\quad\mbox{and}\quad
\vec{v}=v_{1}\widehat{\textbf{\i}}+v_{2}\widehat{\textbf{\j}}+v_{3}\widehat{\textbf{k}}
\end{equation}
their \textbf{``Dot Product''} is the number
\begin{equation}
\begin{aligned}
\vec{u}\cdot\vec{v} &= \sum_{i} u_{i}v_{i}\\
&= u_{1}v_{1}+u_{2}v_{2}+u_{3}v_{3}
\end{aligned}
\end{equation}
Let $\vec{u}$ and $\vec{v}$ be given vectors in three-space. 

\N{Angles}
How do we find the angle $\theta$ between the two vectors?
\begin{center}
\includegraphics{img/threeD.3}
\end{center}
We find that
\begin{equation}
\theta = \arccos\left(\frac{\|\vec{v}\|}{\|\vec{u}\|}\right)
\end{equation}
How is this? Well, we should recall the law of cosines
\begin{equation}
\|\vec{w}\|^{2}=\|\vec{u}\|^{2}+\|\vec{v}\|^{2}-2\|\vec{u}\|\cdot\|\vec{v}\|\cos(\theta)
\end{equation}
which can be written as
\begin{equation}
\begin{aligned}
\vec{w}\cdot\vec{w} &=
(u_{1}-v_{1})^{2}+(u_{2}-v_{2})^{2}+(u_{3}-v_{3})^{2}\\
&= \|\vec{u}\|^{2}-2(\vec{u}\cdot\vec{v})+\|\vec{v}\|^{2}
\end{aligned}
\end{equation}
Setting equals to equals gives us
\begin{equation}
-2(\vec{u}\cdot\vec{v})=-2\|\vec{u}\|\cdot\|\vec{v}\|\cos(\theta)
\end{equation}
and thus
\begin{equation}
\frac{\vec{u}\cdot\vec{v}}{\|\vec{u}\|\cdot\|\vec{v}\|}=\cos(\theta)
\end{equation}
Taking the arc cosine of both sides yields
\begin{equation}
\arccos\left(\frac{\vec{u}\cdot\vec{v}}{\|\vec{u}\|\cdot\|\vec{v}\|}\right)=\theta.
\end{equation}
A useful formula worth remembering 
\begin{equation}
(\vec{u}\cdot\vec{v})=\|\vec{u}\|\cdot\|\vec{v}\|\cos(\theta)
\end{equation}
\begin{example}
Let $\vec{u}=\langle3,-1,4\rangle$ and
$\vec{v}=\langle1,5,-2\rangle$. What's the angle between them?

\emph{Solution}: We first find
\begin{equation}
\begin{aligned}
\vec{u}\cdot\vec{v} &= (3\cdot1)+(-1\cdot5)+(4\cdot-2)\\
&=3-5-8=-10.
\end{aligned}
\end{equation}
We then compute
\begin{equation}
\|\vec{u}\|=\sqrt{9+1+16}=\sqrt{26}
\end{equation}
and
\begin{equation}
\|\vec{v}\|=\sqrt{1+25+4}=\sqrt{30}.
\end{equation}
Thus the angle between $\vec{u}$ and $\vec{v}$ is
\begin{equation}
\theta=\arccos\left(\frac{-10}{\sqrt{26}\sqrt{30}}\right)\approx1.937
\end{equation}
(radians).
\end{example}

\subsection{Orthogonality}

\M
Vectors are perpendicular or \textbf{``Orthogonal''} if
$\vec{u}\cdot\vec{v}=0$. Sometimes this is denoted $\vec{u}\bot\vec{v}$.

\begin{example}
Consider
\begin{equation}
\vec{u}=\langle6,-3,8\rangle,\quad\mbox{and}\quad
\vec{v}=\langle-2,4,3\rangle.
\end{equation}
We see
\begin{equation}
\vec{u}\cdot\vec{v}=-12-12+24=0
\end{equation}
which implies $\vec{u}$ and $\vec{v}$ are orthogonal.
\end{example}

\M
Consider the following diagram
\begin{center}
\includegraphics{img/threeD.4}
\end{center}
We're given vectors $\vec{u}=\overrightarrow{PQ}$ and
$\vec{v}=\overrightarrow{PS}$ in 3-space. Notice that if the
angle between the vectors $\theta$ is acute, as doodled, then
$\overrightarrow{PR}$ is the projection of $\vec{u}$ onto
$\vec{v}$. 

However, if $\theta$ is obtuse, we doodle the situation thus:
\begin{center}
\includegraphics{img/threeD.5}
\end{center}
Observe the projection of $\vec{u}$ onto $\vec{v}$ will not fall
on $\vec{v}$. The projection of $\vec{u}$ onto $\vec{v}$ is
syntactically 
\begin{equation}
\proj_{\vec{v}}\vec{u}
\end{equation}
The natural question: \emph{what is the formula for projecting
$\vec{u}$ onto $\vec{v}$?} We have
\begin{equation}
\|\overrightarrow{PR}\|=\begin{cases}\|\vec{u}\|\cos(\theta)
&\mbox{for $\theta$ acute}\\
-\|\vec{u}\|\cos(\theta)&\mbox{for $\theta$ obtuse}
\end{cases}
\end{equation}
The direction of $\overrightarrow{PR}$ depends on whether
$\theta$ is acute or obtuse; we have its unit vector be
\begin{equation}
\widehat{PR}=\begin{cases}\widehat{v}&\mbox{for $\theta$ acute}\\
-\widehat{v}&\mbox{for $\theta$ obtuse}
\end{cases}
\end{equation}
But now look, for both obtuse and acute $\theta$ we have
\begin{equation}
\begin{aligned}
\overrightarrow{PR}
&=\|\vec{u}\|\cos(\theta)\frac{\vec{v}}{\|\vec{v}\|}\\
&=\frac{\|\vec{u}\|\|\vec{v}\|\cos(\theta)}{\|\vec{v}\|}\frac{\vec{v}}{\|\vec{v}\|}\\
&=\frac{\vec{u}\cdot\vec{v}}{\|\vec{v}\|}\frac{\vec{v}}{\|\vec{v}\|}
\end{aligned}
\end{equation}
This describes the projection of $\vec{u}$ onto $\vec{v}$
for \emph{any} $\theta$:
\begin{equation}
\proj_{\vec{v}}\vec{u}=\frac{\vec{u}\cdot\vec{v}}{\|\vec{v}\|}\frac{\vec{v}}{\|\vec{v}\|}
\end{equation}
Notice that its magnitude is $\vec{u}\cdot\vec{v}/\|\vec{v}\|$. 

\M
Consider the same situation again. We have a vector
$\vec{w}=\overrightarrow{RQ}$ as doodled
\begin{center}
\includegraphics{img/threeD.6}
\end{center}
This vector $\vec{w}$ is orthogonal to the projection of
$\vec{u}$ onto $\vec{v}$. For this reason, we write
\begin{equation}
\vec{w} = \mathop{\mathrm{orth}}\nolimits_{\vec{v}}\vec{u}
\end{equation}
What is it? Well, using basic vector arithmetic, we find
\begin{equation}
\vec{w} = \vec{u} - \proj_{\vec{v}}\vec{u}.
\end{equation}
We will conclude our discussion of vectors here, but continue
next time.

\subsection{More Vector Fun}
%%
%% moreVectors.tex
%% 
%% Made by Alex Nelson
%% Login   <alex@black-cherry>
%% 
%% Started on  Thu Jun 21 15:16:44 2012 Alex Nelson
%% Last update Thu Jun 21 17:28:46 2012 Alex Nelson
%%

\M
Last time we ended with discussing how to project a vector onto
another. So if we consider projecting $\vec{u}$ onto $\vec{v}$,
we can write this as
\begin{equation}
\proj_{\vec{v}}\vec{u} = \vec{u}_{\|}
\end{equation}
Observe, we have another vector constructed
\begin{equation}
\vec{u}_{\bot} = \vec{u}-\vec{u}_{\|}
\end{equation}
which is orthogonal to $\vec{v}$. We have a closed form
expression for projection, namely
\begin{equation}
\proj_{\vec{v}}\vec{u} = (\vec{u}\cdot\widehat{v})\widehat{v}
\end{equation}
where $\widehat{v}=\vec{v}/\|\vec{v}\|$ is a unit vector. But do
we have a closed form expression for $\vec{u}_{\bot}$?\more

\N{Cross-Product}
The cross product of $\vec{u}$ and $\vec{v}$ is 
\begin{equation}
\vec{u}\times\vec{v} = \begin{vmatrix}
\widehat{\textbf{\i}} & \widehat{\textbf{\j}} & \widehat{\mathbf{k}}\\
       u_{1}          &            u_{2}      & u_{3}\\
       v_{1}          &            v_{2}      & v_{3}
\end{vmatrix} =
\bigl(\|\vec{u}\|\|\vec{v}\|\sin(\theta)\bigr)\widehat{u}
\end{equation}
Note we are using notation from linear algebra writing, recursively,
\begin{equation}
\begin{aligned}
\det(A) &= \begin{vmatrix} a_{11} & a_{12} & a_{13}\\
a_{21} & a_{22} & a_{23}\\
a_{31} & a_{32} & a_{33}
\end{vmatrix} \\
&= a_{11}\begin{vmatrix} a_{22} & a_{23} \\ a_{32} &
  a_{33}
\end{vmatrix}
- a_{12} \begin{vmatrix} a_{21} & a_{23}\\ a_{31} & a_{33}
\end{vmatrix}
+a_{13}\begin{vmatrix} a_{21} & a_{22}\\ a_{31} & a{32}
\end{vmatrix}
\end{aligned}
\end{equation}
where
\begin{equation}
\begin{vmatrix} a & b\\ c & d
\end{vmatrix} = ad-bc.
\end{equation}
\begin{remark}
Observe this implies
$\widehat{\textbf{\i}}\times\widehat{\textbf{\j}}=\widehat{\textbf{k}}$, 
$\widehat{\textbf{\j}}\times\widehat{\textbf{k}}=\widehat{\textbf{\i}}$,
and 
$\widehat{\textbf{k}}\times\widehat{\textbf{\i}}=\widehat{\textbf{\j}}$.
\end{remark}
\begin{remark}
The cross-product takes two vectors, and \emph{produces a third vector}.
It \emph{does not} produce a scalar (a number, unlike the dot product).
\end{remark}
\emph{Pop quiz}: let $\vec{u}$ and $\vec{v}$ be vectors. Is $\vec{u}\times\vec{v}=\vec{v}\times\vec{u}$?

\begin{example}
Consider $\vec{u}=\langle2,1,-3\rangle$ and
$\vec{v}=\langle1,-2,1\rangle$. What is $\vec{u}\times\vec{v}$?

\emph{Solution}: we find
\begin{equation}
\begin{aligned}
\vec{u}\times\vec{v} &= \begin{vmatrix}
\widehat{\textbf{\i}} & \widehat{\textbf{\j}} &\widehat{\textbf{k}}\\
2 & 1 & -3\\
1 & -2 & 1
\end{vmatrix}\\
&= (1\cdot1-(-3)\cdot(-2))\widehat{\textbf{\i}} 
- (2\cdot1-(-3)\cdot1)\widehat{\textbf{\j}} 
+(2\cdot(-2)-1\cdot1)\widehat{\textbf{k}}\\
&=(1-6)\widehat{\textbf{\i}} 
- (2+3)\widehat{\textbf{\j}} 
+(-4-1)\widehat{\textbf{k}}\\
&=\langle-5,5,-5\rangle
\end{aligned}
\end{equation}
Another approach would have been to write
\begin{equation}
\vec{u}\times\vec{v} 
= (2\widehat{\textbf{\i}} + \widehat{\textbf{\j}} -3\widehat{\textbf{k}})
\times(\widehat{\textbf{\i}} -2 \widehat{\textbf{\j}} +\widehat{\textbf{k}})
\end{equation}
and used the cross-product's anticommutativity to do the calculations.
\end{example}

\N{Parallelogram Area}
Consider three distinct points $P$, $Q$, and $R$. We can
construct a parallelograph, as in the following diagram:
\begin{center}
\includegraphics{img/moreVectors.0}
\end{center}
We see that
$\overrightarrow{PQ}\times\overrightarrow{PR}=\vec{N}$, then the
area of the parallelogram is $\|\vec{N}\|$.

\N{Parallepiped Volume}
If we work in 3-space, and we have a six-sided region whose sides
are each parallelograms, we call this region a
parallepiped. Observe that we only need 3 vectors to specify the
vertices: $\vec{u}$, $\vec{v}$, and $\vec{w}$. Then we consider
$\vec{u}+\vec{v}$, $\vec{u}+\vec{w}$, $\vec{v}+\vec{w}$, and
$\vec{u}+\vec{v}+\vec{w}$ for the remaining vertices. What is the
volume of this region?

Lets draw a diagram:
\begin{center}
\includegraphics{img/moreVectors.1}
\end{center}
Lets first consider the face described by $\overrightarrow{PQ}=\vec{u}$ and
$\overrightarrow{PR}=\vec{v}$. We see the parallepiped may be considered as a
``stack'' of such faces, whose height is given by the third
vector $\vec{w}$. Then we see the area of the face is shaded in
the diagram, and algebraically it's given by
$\vec{u}\times\vec{v}$, and this produces a vector whose
magnitude is the area of the face. When we ``dot'' this with
$\proj_{\vec{u}\times\vec{v}}\vec{w}$, it's intuitively taking the product of the ``area of
aa face'' ($\vec{u}\times\vec{v}$) and the ``height of the
parallepiped'' ($\proj_{\vec{u}\times\vec{v}}\vec{w}$) producing the volume 
\begin{equation}
\mbox{volume } = (\vec{u}\times\vec{v})\cdot\vec{w}.
\end{equation}
Note this can be negative, and this just tells us information
regarding the parallepiped's \emph{orientation}.


\subsection{Line Constructions}
%%
%% lines.tex
%% 
%% Made by Alex Nelson
%% Login   <alex@black-cherry>
%% 
%% Started on  Thu Jun 21 17:27:08 2012 Alex Nelson
%% Last update Thu Jun 21 17:28:45 2012 Alex Nelson
%%
\N{Constructing Lines}
Suppose we have two points
\begin{equation}
A = (4,2,-1)\quad\mbox{and}\quad
B = (3,5,7).
\end{equation}
We want to find a line $\ell$ passing through these points. What to do?

First we form the vector
\begin{equation}
\begin{aligned}
\vec{v}=\overrightarrow{AB}
&= (3-4)\widehat{\textbf{\i}}+(5-2)\widehat{\textbf{\j}}+(7+1)\widehat{\textbf{k}}\\
&=-\widehat{\textbf{\i}}+3\widehat{\textbf{\j}}+8\widehat{\textbf{k}}
\end{aligned}
\end{equation}
This vector is parallel to $\ell$; the numbers given by this
vector's components (i.e., -1, 3, 8) are called
the \textbf{``Direction Numbers''} of $\ell$.  

In general, we have two distinct points $P_{0}=(x_0,y_0,z_0)$ and
$P=(x,y,z)$ on the line $\ell$, then we construct the vector
\begin{equation}
\overrightarrow{P_{0}P} = (x-x_{0})\widehat{\textbf{\i}}+
(y-y_{0})\widehat{\textbf{\j}}+(z-z_{0})\widehat{\textbf{k}}
\end{equation}
and this is equal to some scalar multiple of $\vec{v}$ (i.e.,
it's a dilation of the vector).  We write
\begin{equation}
(x-x_{0})\widehat{\textbf{\i}}+
(y-y_{0})\widehat{\textbf{\j}}+(z-z_{0})\widehat{\textbf{k}}
=t\vec{v}
\end{equation}
which lets us write
\begin{equation}
\left.\begin{array}{rl}
x-x_{0} &=tv_{1}\\
y-y_{0} &=tv_{2}\\
z-z_{0} &=tv_{3}
\end{array}
\right\}\implies
\left\{\begin{array}{rl}
x &=x_{0}+tv_{1}\\
y &=y_{0}+tv_{2}\\
z &=z_{0}+tv_{3}
\end{array}\right.
\end{equation}
This is the parametric equations of $\ell$. So returning to our
example, we have
\begin{equation}
\begin{aligned}
x &= 4 - t\\
y &= 2 + 3t\\
z &= -1 + 8t
\end{aligned}
\end{equation}
where the constant terms are precisely the values of the
components of $A$, and the coefficients of $t$ are the components
of the vector $\overrightarrow{AB}$.

\N{Distance From a Point to a Line}
What's the distance from any point $S$ in 3-space to a given line
$\ell$? 

We pick a point $P$ on $\ell$ and form a vector
$\overrightarrow{PS}$. The distance from $S$ to $\ell$ can be
given as $d$, as in the following diagram:
\begin{center}
\includegraphics{img/moreVectors.2}
\end{center}
We see $d=\|\overrightarrow{PS}\|\sin(\theta)$. If $\vec{v}$ is a
vector parallel to $\ell$, then we have
\begin{equation}
\|\overrightarrow{PS}\times\widehat{v}\|=\|\overrightarrow{PS}\|\sin(\theta)
\end{equation}
This is all abstract, lets consider an example.

\begin{example}
Find the distance from $S=(2,1,3)$ to the line given by
\begin{equation}
\begin{aligned}
x &= -1+t\\
y &= 2+t\\
z &= 1+2t
\end{aligned}
\end{equation}
First we pick the point when $t=0$, we call it
\begin{equation}
P = (-1,2,1).
\end{equation}
Observe
\begin{equation}
\overrightarrow{PS} =
3\widehat{\textbf{\i}}-\widehat{\textbf{\j}}+2\widehat{\textbf{k}}. 
\end{equation}
Now we need to find a vector parallel to the line. What to do?
Construct a vector by considering the point when $t=1$, which
would be $P_{1}=(0,3,3)$. Thus
\begin{equation}
\begin{aligned}
\vec{v}
&=\overrightarrow{PP_{1}}\\
&=(-1-0)\widehat{\textbf{\i}}+(2-3)\widehat{\textbf{\j}}+(1-3)\widehat{\textbf{k}}\\
&=-\widehat{\textbf{\i}}-\widehat{\textbf{\j}}-2\widehat{\textbf{k}}
\end{aligned}
\end{equation}
Its unit vector is
\begin{equation}
\begin{aligned}
\widehat{v} &= \vec{v}/\|\vec{v}\|\\
&= \vec{v}/\sqrt{1+1+4}\\ 
&= \vec{v}/\sqrt{6}
\end{aligned}
\end{equation}
We have
\begin{equation}
\begin{aligned}
\overrightarrow{PS}\times\widehat{v}
&= \frac{1}{\sqrt{6}}\begin{vmatrix}
\widehat{\textbf{\i}} & \widehat{\textbf{\j}} & \widehat{\textbf{k}}\\
3 & -1 & 2\\
-1 & -1 & -2
\end{vmatrix}\\
&= \frac{4\widehat{\textbf{\i}}+4\widehat{\textbf{\j}}-4\widehat{\textbf{k}}}{\sqrt{6}}
\end{aligned}
\end{equation}
This has its magnitude be
\begin{equation}
\|\overrightarrow{PS}\times\widehat{v}\|
= \frac{4\sqrt{3}}{\sqrt{6}}=2\sqrt{2},
\end{equation}
which describes the distance between our point $S$ and the given line.
\end{example}


\subsection{Constructing Planes}
%%
%% planes.tex
%% 
%% Made by Alex Nelson
%% Login   <alex@black-cherry>
%% 
%% Started on  Thu Jun 21 17:24:56 2012 Alex Nelson
%% Last update Thu Jun 21 17:25:10 2012 Alex Nelson
%%

\N{Determining a Plane From a Point and Normal Vector}
Given a vector
\begin{equation}
\vec{N} = A\widehat{\textbf{\i}}+b\widehat{\textbf{\j}}+C\widehat{\textbf{k}}
\end{equation}
and a point $P_{0}=(x_{0},y_{0},z_{0})$, there exists a unique
plane which is perpindicular to $\vec{N}$ and contains $P_{0}$. 

How? Well, let $P$ be an arbitrary point on the plane. Then the
vector $\overrightarrow{P_{0}P}$ would be parallel to the
plane. Being parallel to the plane implies its orthogonal to
$\vec{N}$:
\begin{equation}
\overrightarrow{P_{0}P}\cdot\vec{N}=0.
\end{equation}
This gives us an equation
\begin{equation}
\begin{aligned}
\overrightarrow{P_{0}P}\cdot\vec{N}&=
\bigl(
(x-x_{0})\widehat{\textbf{\i}}+(y-y_{0})\widehat{\textbf{\j}}+(z-z_{0})\widehat{\textbf{k}}
\bigr)\cdot(A\widehat{\textbf{\i}}+b\widehat{\textbf{\j}}-C\widehat{\textbf{k}})
\\
&=A(x-x_{0})+B(y-y_{0})+C(z-z_{0})=0.
\end{aligned}
\end{equation}
So any point $(x,y,z)$ lies on the plane if it satisfies this
equation. 

\N{Determine a Plane from Three Points}
Given three points $A$, $B$, $C$, find a plane containing these
points. 

We construct the vectors $\overrightarrow{AB}$ and
$\overrightarrow{AC}$. Take the cross product, which produces the
normal vector
\begin{equation}
\vec{N} = \overrightarrow{AB}\times\overrightarrow{AC}.
\end{equation}
If we write $A=(x_{0},y_{0},z_{0})$ and $\vec{N}=\langle
N_{1},N_{2},N_{3}\rangle$, then
\begin{equation}
N_{1}(x-x_{0})+N_{2}(y-y_{0})+N_{3}(z-z_{0}) = 0
\end{equation}
describes the plane. (It follows from our last construction of
the plane.)

\section{Curves}
\subsection{Curves, Classical Kinematics}
%%
%% curves.tex
%% 
%% Made by Alex Nelson
%% Login   <alex@black-cherry>
%% 
%% Started on  Fri Jun 29 11:50:28 2012 Alex Nelson
%% Last update Sat Jun 30 20:32:51 2012 Alex Nelson
%%
\N{Curves}
We are interested in describing the motion of my car. Well,
\emph{everyone} is interested in the motion of my car. How can we
describe it mathematically? 

First we approximate the car as a point. The point-like car moves
in time, so the value of its components are functions of
time. More precisely, the position of my car is 
\begin{equation}
\vec{r}(t) = \langle f(t),g(t),h(t)\rangle =
f(t)\widehat{\textbf{\i}} +
g(t)\widehat{\textbf{\j}} +
h(t)\widehat{\textbf{k}}
\end{equation}
where the functions $f(t)$, $g(t)$, and $h(t)$ are sometimes
called \emph{component functions}. Another way to think about
this is writing
\begin{equation}
\vec{r}\colon[0,1]\to\RR^{3}
\end{equation}
where $0\leq t\leq1$. 

Classical mechanics studies such curves under various
circumstances. We will discuss some notions of kinematics, and
study what it means to differentiate curves.

\begin{example}
Consider a point traveling in circular motion in the
$xy$-plane. What does this look like? 

Well, it's a paramteric curve, using trigonometric functions we write
\begin{equation}
\vec{r}(t) = \cos(t)\widehat{\textbf{\i}}+\sin(t)\widehat{\textbf{\j}}.
\end{equation}
This descrivbes an anti-clockwise circular motion with radius 1,
lying in the $xy$-plane.
\end{example}

\begin{example}
Suppose a particle travels along a parabolic curve, what does the
curve look like? We can write it explicitly as
\begin{equation}
\vec{r}(t) =
t\,\widehat{\textbf{\i}}+(t^{2}-1)\widehat{\textbf{\j}}
\end{equation}
This is precisely aa parabola.
\end{example}

\N{Calculus with Vector-Valued Functions}
We should recall the construction of the tangent line to a curve
$y=f(x)$ at a point $(x_{0},f(x_{0})=y_{0})$ had us write
\begin{equation}
t(h) = y_{0} + f'(x_{0}) \cdot h.
\end{equation}
When we consider the situation when we work with $\vec{r}(t)$
instead of a function $f(x)$. We have
$\vec{r}_{0}=\vec{r}(t_{0})$ be the base point for the tangent to
the curve, then we have
\begin{equation}
\vec{T}(h) = \vec{r}_{0} + \vec{r}'(t_{0})\cdot h
\end{equation}
The problem: what exactly is $\vec{r}'(t_{0})$?

\M
We can let $\vec{r}\colon(0,1)\to\RR^{3}$ (or more generally the
codomain can be $\RR^{n}$ for any positive integer $n\in\NN$). We
have
\begin{equation}
\frac{\D\vec{r}(t)}{\D t}=\vec{v}(t) = \lim_{\Delta t\to0}\frac{\vec{r}(t+\Delta t)-\vec{r}(t)}{\Delta t}
\end{equation}
describe the rate of change of the position vector $\vec{r}(t)$
with respect to time. What does this look like? Well, writiing out
\begin{equation}
\vec{r}(t) = \langle f(t),g(t),h(t)\rangle =
f(t)\widehat{\textbf{\i}} +
g(t)\widehat{\textbf{\j}} +
h(t)\widehat{\textbf{k}}
\end{equation}
we have
\begin{equation}
\frac{\D\vec{r}(t)}{\D t} = \langle f'(t), g'(t), h'(t)\rangle =
f'(t)\widehat{\textbf{\i}} +
g'(t)\widehat{\textbf{\j}} +
h'(t)\widehat{\textbf{k}}
\end{equation}
where primes denote differentiation with respect to time.

\begin{remark}
We can keep iterating this procedure to obtain higher order
derivatives of a curve.
\end{remark}

\N{Kinematics}
We have $\vec{r}(t)$ describe the position of a particle. The
velocity of the particle is a vector-valued function
\begin{equation}
\begin{aligned}
\vec{v}(t)
&=\lim_{\Delta t\to 0}\frac{\vec{r}(t+\Delta
  t)-\vec{r}(t)}{\Delta t}\\
&=\frac{\D\vec{r}(t)}{\D t}
\end{aligned}
\end{equation}
However, we also can consider the \emph{speed} or the magnitude
of the velocity
\begin{equation}
\|\vec{v}(t)\|=\frac{\D s}{\D t} = \begin{pmatrix}
\mbox{rate of change of distance}\\
\mbox{with respect to time}
\end{pmatrix}
\end{equation}
Observe the speed is a scalar quantity: it's just some function
of time. The velocity is a vector-valued function of time. 

We have one last kinematical quantity to consider: the
acceleration. This is just the rate of change of velocity with
respect to time:
\begin{equation}
\vec{a}(t) = \frac{\D\vec{v}(t)}{\D t} =
\frac{\D^{2}\vec{r}(t)}{\D t^{2}}
\end{equation}
Observe we can reconstruct the position from the velocity by
considering
\begin{equation}
\vec{r}(t) = \vec{r}(t_{0}) +
\int^{t}_{t_{0}}\frac{\D\vec{r}(\tau)}{\D\tau}\,\D\tau
\end{equation}
which when we consider $\vec{r}(t)=\langle x(t),y(t),z(t)\rangle$
we have the integral evaluated ``component-wise'':
\begin{equation}
\vec{r}(t) = \vec{r}(t_{0}) +
\left\langle \int^{t}_{t_{0}}\frac{\D x(\tau)}{\D\tau}\,\D\tau,
\int^{t}_{t_{0}}\frac{\D y(\tau)}{\D\tau}\,\D\tau,
\int^{t}_{t_{0}}\frac{\D z(\tau)}{\D\tau}\,\D\tau\right\rangle.
\end{equation}
We can similarly reconstruct velocity from acceleration.

\begin{example}
Consider the curve describing circular motion
\begin{equation}
\vec{r}(t) = \cos(t)\widehat{\textbf{\i}}
+\sin(t)\widehat{\textbf{\j}}
\end{equation}
What is its velocity vector, acceleration vector, and speed?

\emph{Solution}: We find its velocity
\begin{equation}
\begin{aligned}
\vec{v}(t) &= \frac{\D\vec{r}(t)}{\D t}\\
&=-\sin(t)(t)\widehat{\textbf{\i}}
+\cos(t)\widehat{\textbf{\j}}
\end{aligned}
\end{equation}
From this we can compute its speed as
\begin{equation}
\begin{aligned}
\|\vec{v}(t)\| &= \sqrt{\vec{v}(t)\cdot\vec{v}(t)}\\
&=\sqrt{\sin^{2}(t)+\cos^{2}(t)} = 1.
\end{aligned}
\end{equation}
The acceleration is precisely the derivative of the velocity
vector
\begin{equation}
\frac{\D\vec{v}(t)}{\D t}
= -\cos(t)\widehat{\textbf{\i}}-\sin(t)\widehat{\textbf{\j}}
\end{equation}
That concludes our example.
\end{example}

\begin{exercise}
Find the velocity vector, speed, and acceleration of the
parabolic curve $\vec{r}(t) =
t\,\widehat{\textbf{\i}}+(t^{2}-1)\widehat{\textbf{\j}}$
\end{exercise}

\begin{exercise}
Let $\vec{u}(t)$ and $\vec{v}(t)$ be differentiable vector-valued
functions of time. Prove or find a counter-example that
\begin{equation}
\frac{\D}{\D t}\bigl[\vec{u}(t)\times\vec{v}(t)\bigr]=
\frac{\D\vec{u}(t)}{\D t}\times\vec{v}(t) +
\vec{u}(t)\times\frac{\D\vec{v}(t)}{\D t}.
\end{equation}
\end{exercise}
\begin{exercise}
Calculate $\displaystyle\frac{\D}{\D t}\bigl[\vec{a}(t)\cdot\bigl(\vec{b}(t)\times\vec{c}(t)\bigr)\bigr]$
\end{exercise}


\section{Surfaces}
%%
%% surfaces.tex
%% 
%% Made by Alex Nelson
%% Login   <alex@black-cherry>
%% 
%% Started on  Fri Jun 22 13:32:11 2012 Alex Nelson
%% Last update Thu Jun 28 12:18:49 2012 Alex Nelson
%%
\M
Let $w=f(x_{1},x_{2},\dots,x_{n})$ where $x_{1}$, $x_{2}$, \dots,
$x_{n}$ are all independent variables. Then the
\textbf{``Domain''} of $f$ is the set of $n$-tuples
$\RR^{n}$. Note that an ordered pair is 
\begin{equation}
(x,y)=\mbox{2-tuple}.
\end{equation}
The set of corresponding values is the \textbf{``Range''} (or
\emph{Codomain}) of the function. So we have
\begin{equation}
%% \begin{aligned}
%% f\colon &\RR^{n}&\to&\RR\\
%% &\mbox{(Domain)}&\to&\mbox{(Range)}
%% \end{aligned}
f\colon\underbrace{\RR^{n}}_{\text{domain}}\to\underbrace{\RR}_{\text{range}}
\end{equation}
Sometimes we write $\mathrm{dom}(f)$ for the domain of $f$, and
$\mathrm{ran}(f)$ or $\mathrm{cod}(f)$ for the range (or
codomain) of $f$. We \emph{DO NOT} write $f(\RR^{2})$ for the
range, because this is the collection of all points mapped by
$f$. 

\begin{example}
Consider $f\colon\RR^{2}\to\RR$ defined by
\begin{equation}
f(x,y) = \sqrt{x^{2}+y^{2}}
\end{equation}
Notice that $f(x,y)\geq0$ for any $x,y\in\RR$. So
\begin{equation}
f(\RR^{2})=\{ u\in\RR : u\geq0\}\not=\RR.
\end{equation}
This is not the codomain! It's contained in the codomain,
though. The image is \emph{always} a subset of the codomain.
\end{example}
\more

\begin{example}
Let $g(x,y)=\ln(x^{2}-y)$. For $g$ to be defined, we need
\begin{equation}
x^{2}-y>0\quad\mbox{or}\quad x^{2}>y.
\end{equation}
The boundary of the domain is $x^{2}=y$, which we can doodle:
\begin{center}
  \includegraphics{img/surfaces.0}
\end{center}
Since the boundary is not in the domain, then the domain of $g$
is open.
\end{example}

\begin{example}
Consider 
\begin{equation}
h(x,y)=\sqrt{25-x^{2}-y^{2}}.
\end{equation}
The domain is the set of $(x,y)$ such that
\begin{equation}
x^{2}+y^{2}\leq25.
\end{equation}
We can doodle this:
\begin{center}
\includegraphics{img/surfaces.1}
\end{center}
Observe that the boundary is the circle; the disc is the circle
and everything enclosed in it.
\end{example}

\M
So we have just discussed domains and codomains, but we have not
discussed the graph of the function $z=f(x,y)$. What would this
look like? Well, when we plot $y=g(x)$, it's on the plane
$\RR^{2}$. So plotting $z=f(x,y)$ would be on the 3-space
$\RR^{3}$. Lets start considering examples of what this looks
like. 

\begin{example}
Consider the graph given by
\begin{equation}
z = f(x,y) = 9 - x^{2}-y^{2}.
\end{equation}
How can we draw this? Well, the first trick is to draw when $x=0$
and $y=0$:
\begin{equation}
\begin{aligned}
z &= f(0,y) = 9-y^{2}\\
z &= f(x,0) = 9-x^{2}
\end{aligned}
\end{equation}
These are parabolas in the $yz$ and $xz$ planes,
respectively. Now we can start drawing \textbf{``Level Curves''},
i.e., curves where we fix $z$ to be some constant. For example,
when $z=0$, we have a circle
\begin{equation}
x^{2}+y^{2}=9
\end{equation}
Observe then that $z$ controls the radius of the circles: for
nonzero $z$, we have
\begin{equation}
x^{2}+y^{2}=9-z.
\end{equation}
The left hand side must be non-negative, and can be zero only
when $z=9$. So we get a surface that looks like
\begin{center}
 \includegraphics{img/paraboloid.0}
\end{center}
We call this surface a \textbf{``Paraboloid''} as we have
parabolas along the $x$- and $y$-axes.

This is the general scheme for picturing a surface: draw level
curves $f(x,y)=c$ for some constant $c$, which produces the level
curve for $f$ corresponding to $c$.
\end{example}

\begin{example}
Consider $x^{2}+y^{2}-z^{2}=0$. What does this surface look like?

\emph{Solution}: First we observe the curves along the $x$-axis,
i.e., when $y=0$ is $x^{2}=z^{2}$. Similarly when $x=0$ we have
$y^{2}=z^{2}$. What sort of curves are these? Well, $(\pm t,0,t)$
and $(0,\pm t,t)$ are the curves, for $t\in\RR$. 

Also note that $x^{2}+y^{2}$ describes a circle, whose radius
happens to be $z^{2}$. This tells us the level curves are simply
circles. So we have a cone: 
\begin{center}
  \includegraphics{img/surfaces.2}  
\end{center}
This surface is precisely a \textbf{``Cone''}.
\end{example}
\begin{example}
What if we deform the previous example, writing
\begin{equation}
x^{2}+y^{2}-z^{2}=1.
\end{equation}
What surface does this describe?

\emph{Solution}: Well, we see that the radius of the level curves
deform $z^{2}\to z^{2}+1$. So the resulting surface looks like a
cone, but along the $x$-axis we don't have a straight-line: we
have a hyperbola $x^{2}-z^{2}=1$. Similarly along the $y$-axis we
have another hyperbola $y^{2}-z^{2}=1$. Thus our surface is
doodled as:
\begin{center}
\includegraphics{img/surfaces.3}
\end{center}
This ``deformed cone'' is called a \textbf{``One Sheeted Hyperboloid''}.
\end{example}
\begin{example}
Another variation, consider the surface
\begin{equation}
x^{2}+y^{2}-z^{2}=-1
\end{equation}
What does it look like?

\emph{Solution}: Well, we see that $x^{2}+y^{2}\geq0$ always,
whereas $z^{2}-1\geq0$ only when $|z|\geq1$. So we have two
``surfaces'' or \textbf{``Sheets''} here. The level curves are
again circles, and this enables us to doodle the graph:
\begin{center}
\includegraphics{img/surfaces.4}
\end{center}
Along the $x$-axis and $y$-axis, we have hyperbolas. For this
reason, we call the surface a \textbf{``Two-Sheeted Hyperbaloid''}.
\end{example}

\begin{example}
Suppose we have a surface given by $z=x^{2}-y^{2}$. What does it
look like?

\emph{Solution}: Observe the level curves for $z>0$ gives us
hyperbolas, and for $z<0$ we again have hyperbolas (the same as
before but reflected about the line $x=y$). For $z=0$ we have a
cone. 

When we take $y=\pm1,0$ we see $z=x^{2}-C$ for some constant
$C$. These curves look like smiles. For $x=\pm1,0$, we have
$z=C-y^{2}$ for some constant $C$. These curves look like
frowns. 

This surface is a saddle, and we can doodle it as
\begin{center}
\includegraphics{img/surfaces.5}
\end{center}
Since the curves when we hold $z$ constant form hyperbolas, and
the curves when we hold $x$ or $y$ constaant are parabolas,
people sometimes call this saddle a \textbf{``Hyperbolic Paraboloid''}.
\end{example}

\subsection{Limits for Multivariable Functions, Partial Derivatives}
%%
%% multivarContinuity.tex
%% 
%% Made by Alex Nelson
%% Login   <alex@black-cherry>
%% 
%% Started on  Wed Jun 27 11:16:54 2012 Alex Nelson
%% Last update Thu Jun 28 12:47:00 2012 Alex Nelson
%%
\M
We considered differentiating and integrating functions of a
single-variable. How? We began with the notion of a limit, and
then considered the derivative. If we have a, e.g., polynomial
\begin{equation}
p(x,y) = x^{3}+x^{2}y+xy^{2}+y^{3}
\end{equation}
we see
\begin{equation}
p(x+\Delta x,y) = x^{3}+x^{2}y+xy^{2}+y^{3}+\bigl(3x^{2}+2xy+y\bigr)\Delta x+
\mathcal{O}(\Delta x^{2})
\end{equation}
Again we stop and reflect: this treats $y$ as if it were
constant. So the derivative formed by
\begin{equation}
\lim_{\Delta x\to0}\frac{p(x+\Delta x,y)-p(x,y)}{\Delta x}=3x^{2}+2xy+y
\end{equation}
are ``incomplete'' or \textbf{partial}. There is some subtlety
here due to using multiple variables, and we have to discuss the
problems of limits first.\more

\N{Definition}
The function $z=f(x,y)$ is \textbf{``Continuous''} at
$(x_{0},y_{0})$ if

(i) $f(x_{0},y_{0})$ is defined and finite;

(ii) $\displaystyle\lim_{(x,y)\to(x_{0},y_{0})}f(x,y)=f(x_{0},y_{0})$ is
defined;

(iii) $\displaystyle\lim_{(x,y)\to(x_{0},y_{0})}f(x,y)$ is defined (and finite).

\noindent{Note:} this can be determined by picking any curve
$\gamma\colon[0,1]\to\RR^{2}$ which satisfies
\begin{equation}
\gamma(t_{0}) = (x_{0},y_{0})
\end{equation}
for some $0\leq t_{0}\leq 1$, then taking 
\begin{equation}
\lim_{t\to t_{0}}f\bigl(\gamma(t)\bigr) = \lim_{(x,y)\to(x_{0},y_{0})}f(x,y).
\end{equation}
The subtletly here lies with $\gamma$ being \emph{arbitrary}. If
two different curves produce two different results, the limit
\emph{does not exist}. Lets consider some examples and non-examples.

\begin{example}[Limit Exists]
Find
\begin{equation}
\lim_{(x,y)\to(2,4)}\frac{y+4}{x^{2}y-xy+4x^{2}-4x}.
\end{equation}

\emph{Solution}: for this, we can simply plug in the values
\begin{equation}
\begin{aligned}
\lim_{(x,y)\to(2,4)}\frac{y+4}{x^{2}y-xy+4x^{2}-4x}
&=\frac{(4)+4}{(2)^{2}(4)-(2)(4)+4(2^{2})-4(2)}\\
&=\frac{8}{16-8+16-8}=\frac{1}{2}.
\end{aligned}
\end{equation}
This is because the function is sufficiently nice.
\end{example}

\begin{example}[Limit Doesn't Exist]
What is
\begin{equation}
\lim_{(x,y)\to(0,0)}\frac{x^{4}}{x^{4}+y^{2}}=?
\end{equation}

\emph{Solution}: Lets first approach it along the $x$-axis,
i.e. first setting $y=0$. We find
\begin{equation}
\lim_{(x,y)\to(0,0)}\frac{x^{4}}{x^{4}+y^{2}}=\lim_{x\to0}\frac{x^{4}}{x^{4}}=1.
\end{equation}
Now lets approach it on the $y$-axis, i.e. first setting
$x=0$. We see
\begin{equation}
\lim_{(x,y)\to(0,0)}\frac{x^{4}}{x^{4}+y^{2}}=\lim_{y\to0}\frac{0}{0+y^{2}}=0.
\end{equation}
Still, approaching along the curve $y=x^{2}$ we see
\begin{equation}
\lim_{(x,y)\to(0,0)}\frac{x^{4}}{x^{4}+y^{2}}=\lim_{x\to0}\frac{x^{4}}{x^{4}+x^{4}}=\frac{1}{2}.
\end{equation}
But we have a problem: this implies $0=1/2=1$. This cannot be! So
the limit \emph{cannot exist!} Very sad.
\end{example}


\N{Definition} Let $z=f(x,y)$ be defined on a region $R$ in the
$xy$-plane, and let $(x_{0},y_{0})$ be an inerior point of $R$,
we just don't want a boundary point!

If
\begin{equation}
\lim_{\Delta x\to0}\frac{f(x_{0}+\Delta
  x,y_{0})-f(x_{0},y_{0})}{\Delta x}
\end{equation}
exists, then it is called the \textbf{``Partial Derivative''} of
$z=f(x,y)$ at $(x_{0},y_{0})$ with respect to $x$. It is denoted
\begin{equation}
\frac{\partial}{\partial x}f = \frac{\partial}{\partial x}z
=f_{x} = \partial_{x}f = \partial_{x}z
\end{equation}
evaluated at $(x_{0},y_{0})$. NB: the subscripts in the
$\partial_{x}$ indicate what variable we are taking the partial
derivative of, i.e., it's shorthand for
$\partial_{x}=\partial/\partial x$.

Under similar conditions,
\begin{equation}
\lim_{\Delta y\to0}\frac{f(x_{0},y_{0}+\Delta y)-f(x_{0},y_{0})}{\Delta y}
\end{equation}
is the partial derivative of $z=f(x,y)$ with respect to $y$ at
$(x_{0},y_{0})$. We denote this by
\begin{equation}
\frac{\partial f}{\partial y}=\frac{\partial z}{\partial
  y}=\partial_{y}f = \partial_{y}z
\end{equation}
among a myriad of different conventions.

\M Higher order partial derivatives are done by taking it one at
a time. So if 
\begin{equation}
z=\E^{xy}
\end{equation}
for example, we have
\begin{equation}
\partial_{y}z=x\E^{xy}
\end{equation}
and taking its derivative again yields
\begin{equation}
\begin{aligned}
\partial_{y}^{2}z &= \partial_{y}\left(x\E^{xy}\right)\\
&=x\partial_{y}(\E^{xy})
\end{aligned}
\end{equation}
Note we factor $x$ out in front of the partial derivative with
respect to $y$ because $x$ is constant with respect to $y$. So we
then obtain
\begin{equation}
\partial_{y}^{2}z = x^{2}\E^{xy}.
\end{equation}
We take partial derivatives one at a time, from right to left:
\begin{equation}
\partial_{x}\partial_{y}z
= \partial_{x}\bigl(\partial_{y}z\bigr).
\end{equation}
\emph{Question}: do partial derivatives commute? I.e., is
$\partial_{x}\partial_{y}=\partial_{y}\partial_{x}$ always?
Lets first consider an example calculation before considering an answer.

\begin{example}
Consider the function $u=x^{2}-y^{2}$. Find
$\partial_{x}^{2}u+\partial_{y}^{2}u$.

\emph{Solution}: We find that
\begin{equation}
\partial_{x}u = 2x\implies \partial_{x}^{2}u = 2.
\end{equation}
Similarly, we find
\begin{equation}
\partial_{y}u=-2y\implies \partial_{y}^{2}y=-2.
\end{equation}
Thus we conclude
\begin{equation}
\partial_{x}^{2}u+\partial_{y}^{2}u=2-2=0.
\end{equation}
\end{example}

\N{Do Partial Derivatives Commute?}
Answer: not always. The conditions are fairly weak: if
$\partial_{x}z$, $\partial_{y}z$, $\partial_{x}\partial_{y}z$ and
$\partial_{y}\partial_{x}z$ are continuous throughout their
respective domains, then 
\begin{equation}
\partial_{x}\partial_{y}z = \partial_{y}\partial_{x}z.
\end{equation}

\begin{exercise}
Let $u(x,t) = f(x+vt) + g(x-vt)$ where $v\not=0$ is some
constant. Prove
\begin{equation}
\partial_{t}^{2}u(x,t)=v^{2}\partial_{x}^{2}u(x,t).
\end{equation}
\end{exercise}
\begin{exercise}
Let $f(x,y)=\ln(x^{2}+y^{2})$. What is
$\partial_{x}^{2}f(x,y)+\partial_{y}^{2}f(x,y)$? 
\end{exercise}
\begin{exercise}
Consider $g(x,y)=1/\sqrt{x^{2}+y^{2}}$. What is
$\partial_{x}^{2}g(x,y)$? What is $\partial_{y}^{2}g(x,y)$? Is
$\partial_{x}\partial_{y}g(x,y)=\partial_{y}\partial_{x}g(x,y)$? 
\end{exercise}
\begin{exercise}
Let $f(x,y)=3x^{2}+4y^{3}++x^{2}y^{3}+\sin(xy)$. What is
$\partial_{x}f$? What is $\partial_{y}f$?
\end{exercise}
\begin{exercise}
Let $z=\arctan(x^{2}\E^{2y})$. What is $\partial_{x}z$? What is
$\partial_{y}z$? 
\end{exercise}

\section{Partial Derivatives}
\subsection{Chain Rule for Partial Derivatives}
%%
%% pdChain.tex
%% 
%% Made by Alex Nelson
%% Login   <alex@black-cherry>
%% 
%% Started on  Thu Jun 28 15:42:27 2012 Alex Nelson
%% Last update Tue Jul  3 10:54:12 2012 Alex Nelson
%%

\N{Problem}
Consider a function $f(x,y)$ where we parametrize 
\begin{equation}
x=x(t,u),\quad\mbox{and}\quad y=y(t,u).
\end{equation}
If $t\to t+\Delta t$, how does $f\to f+\Delta f$ change?

\M
We first note
\begin{equation}
f(x+\Delta x,y) = f(x,y)+\Delta x\partial_{x}f(x,y)+\bigO(\Delta
x^{2}).
\end{equation}
Similarly 
\begin{equation}
f(x,y+\Delta y)= f(x,y)+\Delta y\partial_{y}f(x,y)+\bigO(\Delta
y^{2}).
\end{equation}
Thus we find
\begin{equation}
\begin{aligned}
f(x+\Delta x,y+\Delta y)
&=f(x,y+\Delta y)+\Delta x\partial_{x}f(x,y + \Delta y)+\bigO(\Delta
x^{2})\\
&=\bigl(f(x,y) + \Delta y\partial_{y}f(x,y) + \bigO(\Delta
y^{2})\bigr)\\
&\quad+ \Delta x\partial_{x}\bigl(f(x,y) + \Delta y\partial_{y}f(x,y) + \bigO(\Delta
y^{2})\bigr)\\
&\quad+\bigO(\Delta x^{2})\\
&= f(x,y) + \Delta y\partial_{y}f(x,y) + \Delta
x\partial_{x}f(x,y)\\
&\quad + \bigO(\Delta x\Delta y)%\partial_{x}\partial_{y}f(x,y)
+\bigO(\Delta x^{2})+\bigO(\Delta y^{2}).
\end{aligned}
\end{equation}
But specifically, we are interested in
\begin{equation}
\Delta x = \Delta t\partial_{t}x(t,u)+\bigO(\Delta t^{2})
\end{equation}
and
\begin{equation}
\Delta y = \Delta t\partial_{t}y(t,u)+\bigO(\Delta t^{2})
\end{equation}
Plugging this in allows us to write
\begin{equation}
\Delta f=\Delta y\partial_{y}f(x,y) + \Delta
x\partial_{x}f(x,y) + \bigO(\Delta x\Delta y)%\partial_{x}\partial_{y}f(x,y)
+\bigO(\Delta x^{2})+\bigO(\Delta y^{2})
\end{equation}
as
\begin{equation}
\Delta f = \bigl(\Delta t\partial_{t}y\bigr)\bigl(\partial_{y}f\bigr) +
\bigl(\Delta t\partial_{t}x\bigr)\bigl(\partial_{x}f\bigr) +
\bigO(\Delta t^{2})
\end{equation}
Observe under our substitution, we have the $\bigO(\Delta
x\Delta y)$ and other big O terms be gathered into the
$\bigO(\Delta t^{2})$ term.

So what? Observe
\begin{equation}
\begin{aligned}
\frac{\partial f}{\partial t} &= \lim_{\Delta t\to 0}\frac{f\bigl(x(t+\Delta t,u),y(t+\Delta t)\bigr)-f\bigl(x(t,u),y(t,u)\bigr)}{\Delta t} \\
&=\frac{\partial f}{\partial x}\frac{\partial x}{\partial t}+
\frac{\partial f}{\partial y}\frac{\partial y}{\partial t}.
\end{aligned}
\end{equation}
This is precisely the chain rule. Similarly, we find
\begin{equation}
\frac{\partial f}{\partial u}
=\frac{\partial f}{\partial x}\frac{\partial x}{\partial u}+
\frac{\partial f}{\partial y}\frac{\partial y}{\partial u}.
\end{equation}

\N{Implicit Differentiation Revisited}
Recall implicit differentiation required us to find $\D y/\D x$
from some complicated expression like
\begin{equation}
\E^{xy}+4y^{2}+\tan(x+y)=0
\end{equation}
What to do? First we write
\begin{equation}
z = F(x,y) = \E^{xy}+4y^{2}+\tan(x+y)=0.
\end{equation}
Next we say $y=y(x)$. So we find
\begin{equation}
\begin{aligned}
\frac{\D z}{\D x}
&= \frac{\partial F(x,y)}{\partial x}\frac{\D x}{\D x}+\frac{\partial F(x,y)}{\partial y}\frac{\D y}{\D x}\\
&=0
\end{aligned}
\end{equation}
where we set the derivative of $F$ to be zero since it's equal to the
derivative of zero. We can then write (taking $\D x/\D x=1$)
\begin{equation}
-\frac{\partial F(x,y)}{\partial x}=\frac{\partial
  F(x,y)}{\partial y}\frac{\D y}{\D x}
\end{equation}
and divide both sides by $\partial_{y}F$ to get
\begin{equation}
\frac{-\partial_{x}F}{\partial_{y}F} = \frac{\D y}{\D x}.
\end{equation}
But this is precisely what implicit differentiation gives us!

\N{Warning for Physicists} 
Physicists often use partial derivative notation slightly
differently. If $q(t)$ is the position of a particle, and $p(t)$
is its momentum, physicists consider arbitrary functions of the
form
\begin{equation}
f=f(q,p,t)
\end{equation}
and write
\begin{equation}
\frac{\partial f}{\partial t} = \lim_{\Delta t\to0}
\frac{f\bigl(q(t),p(t),t+\Delta t\bigr)-f\bigl(q(t),p(t),t\bigr)}{\Delta t}.
\end{equation}
This is strictly speaking not quite true. The error committed
lies in treating $q$ and $p$ as functions of time: really they
are variables whom we are trying to express as functions of
time. 

\begin{exercise}
Let $w=\sqrt{x}+y^{2}/z$ where $x=\exp(2t)$, $y=t^{3}+4t$, and
$z=t^{2}-t$. Find $\D w/\D t$.
\end{exercise}
\begin{exercise}
Let $z=\cos(xy)+y\sin(x)$ where $x=v^{2}+u$ and $y=u-v$. Find
$\partial_{u} z$ and $\partial_{v}z$.
\end{exercise}

\subsection{Directional Derivatives}
%%
%% directionalDerivatives.tex
%% 
%% Made by Alex Nelson
%% Login   <alex@black-cherry>
%% 
%% Started on  Fri Jun 29 12:30:18 2012 Alex Nelson
%% Last update Tue Jul  3 11:02:55 2012 Alex Nelson
%%
\M
Suppose we have a scalar function of several variables
\begin{equation}
f\colon\RR^{3}\to\RR
\end{equation}
Let $\widehat{u}$ be some unit vector. How does $f$ change in the
$\widehat{u}$ direction?

We can consider this quantity as a function
\begin{equation}
g(\vec{x}) =
\lim_{h\to0}\frac{f(\vec{x}+h\widehat{u})-f(\vec{x})}{h}
\end{equation}
What does this look like?

\M Lets restrict our attention to the
smallest non-boring case: $f\colon\RR^2\to\RR$. Then we write
$\widehat{u} = \langle p,q\rangle$. We have
\begin{equation}
f(\vec{x}+h\widehat{u}) = f(x+hp,y+hq).
\end{equation}
Expanding this to first order in $h$ lets us write
\begin{equation}
\begin{aligned}
f(x+hp,y+hq) &= f(x,y+hq) +
hp\partial_{x}f(x,y+hq)+\bigO(h^{2})\\
&= \bigl(f(x,y)+hq\partial_{y}f(x,y)+\bigO(h^{2})\bigr)\\
&\quad+hp\partial_{x}\bigl(f(x,y)+hq\partial_{y}f(x,y)+\bigO(h^{2})\bigr)\\
&\quad+\bigO(h^{2})\\
&= f(x,y) + h\bigl(q\partial_{y}f(x,y) +
p\partial_{x}f(x,y)\bigr) +\bigO(h^{2}).
\end{aligned}
\end{equation}
But what does this look like? It's simply
\begin{equation}
f(\vec{x}+h\widehat{u}) = f(\vec{x}) +
h\widehat{u}\cdot\langle\partial_{x},\partial_{y}\rangle
f(\vec{x}) + \bigO(h^{2}).
\end{equation}

\begin{example}
What is the derivative of $f(x,y)=x/y$ in the direction of
$\vec{v}=\langle 1,3\rangle$ at $(5,3)$?

\emph{Solution}: We find the directional derivative is
\begin{equation}
v_{1}\partial_{x}f + v_{2}\partial_{y}f
\end{equation}
We compute
\begin{equation}
\partial_{x}f = 1/y
\end{equation}
and
\begin{equation}
\partial_{y}f = -x/y^{2}
\end{equation}
Thus we have
\begin{equation}
v_{1}\partial_{x}f + v_{2}\partial_{y}f = v_{1}(1/y) + v_{2}(-x/y^{2}).
\end{equation}
We plug in $v_{1}=1$, $v_{2}=3$
\begin{equation}
v_{1}\partial_{x}f + v_{2}\partial_{y}f = (1/y) + 3(-x/y^{2}).
\end{equation}
Then we evaluate $(x,y)=(5,3)$ to get
\begin{equation}
v_{1}\partial_{x}f + v_{2}\partial_{y}f = (1/3) + 3(-5/3^{2})
=-4/3.
\end{equation}
This gives us the directional derivative of $f$.
\end{example}


\M We denote
\begin{equation}
\vec{\nabla} = \langle\partial_{1},\dots,\partial_{n}\rangle
\end{equation}
and call it the \textbf{``Gradient''}. Note we will write
$\nabla$ interchangeably with the vector arrow $\vec{\nabla}$,
and they mean the same thing. The vector arrow doesn't add
anything semantically, it's just different syntax.

The directional derivative is then
\begin{equation}
\widehat{u}\cdot\vec{\nabla}f(\vec{x}) = \widehat{u}\cdot\langle
\partial_{1}f,\dots,\partial_{n}f\rangle.
\end{equation}
Note that the gradient acting on a scalar function produces a
vector-valued function of several variables, but we can also take
the dot product of the gradient with such a monstrosity.

\N{Question:} What is a vector-valued function of several variables?

For us, in practice, we think of this as a \emph{Vector Field}: a
``function'' which assigns to each point a vector. Each
vector-component is a function, usually smooth (i.e., infinitely
differentiable). 

(Again, just as we warned the reader with vectors, this too is a
lie. A vector field is a bit more than just a function
$\RR^{n}\to\RR^{n}$, it's a more complicated beast which is
studied further in differential geometry.)

\N{Meaning of Gradient}
Consider a family of level curves $f(\vec{x})=c$. The gradient
points towards the direction of increasing $c$. How can we see
this? Well, consider the function
\begin{equation}
f(x,y)=x^{2}-y^{2}.
\end{equation}
We see its gradient is
\begin{equation}
\vec{\nabla}f(x,y) = \langle 2x, -2y\rangle.
\end{equation}
Lets draw a few level-curves and see what the vectors point to:
\begin{center}
\includegraphics{img/gradient.0}
\end{center}
We see the vectors point towards $(x,y)\to(\pm\infty,0)$. 

\begin{exercise}
Consider the function $f\colon\RR^{2}\to\RR$ defined by
$f(x,y)=\ln(x^{2}+y^{2})$. What is its gradient?
\end{exercise}
\begin{exercise}
Let $g\colon\RR^{3}\to\RR$ be defined by $g(x,y,z) =
(x^{2}+y^{2}+z^{2})^{-1/2}$. Find its gradient.
\end{exercise}
\begin{exercise}
Let $f(x,y,z)=x/(y+z)$. Find its derivative in the direction
$\vec{u}=\langle 1,1,1\rangle$ at the point $(1,6,2)$. 
\end{exercise}
\begin{exercise}
Let $f(x,y)=x\exp(-y) + 3y$. Find its gradient.
\end{exercise}

\subsection{Extrema}
%%
%% extrema.tex
%% 
%% Made by Alex Nelson
%% Login   <alex@black-cherry>
%% 
%% Started on  Fri Jun 29 13:09:57 2012 Alex Nelson
%% Last update Fri Jun 29 13:43:18 2012 Alex Nelson
%%

\M
Remember for a curve $y=f(x)$, we have maxima and minima occur
whenever
\begin{equation}
f'(x_{0}) = 0
\end{equation}
What's the multivariable analog to this notion? \more

\N{Definition}
If $\vec{\nabla}f(\vec{x}_{0})=\vec{0}$, we say $\vec{x}_{0}$ is
a \textbf{``Critical Point''} of $f$.

\begin{example}
Consider $f(x,y) = x^{2}+y^{2} - 2x - 8y$. What are its critical
points?

\emph{Solution}: We find its gradient first
\begin{equation}
\vec{\nabla}f = \langle 2x - 2, 2y - 8\rangle
\end{equation}
Next we need to set each component to vanish. This implies
\begin{equation}
\vec{x}_{0} = \langle 1, 4\rangle
\end{equation}
is the only critical point.
\end{example}

\N{Problem:} How do we determine if a critical point describes a
maxima or minima?

Lets consider the critical point $\vec{x}_{0}$ for $f$. We Taylor
expand $f$ to second order about $\vec{x}_{0}$ writing
\begin{equation}
f(\vec{x}_{0}+\vec{h}) \approx f(\vec{x}_{0}) +
\vec{h}\cdot\underbrace{\vec{\nabla}f(\vec{x}_{0})}_{=0} + \frac{1}{2} \vec{h}\cdot\mathrm{Hess}(f)\vec{h}
\end{equation}
where we use the matrix
\begin{equation}
\mathrm{Hess}(f) = \begin{bmatrix}
\partial_{1}^{2} f & \dots & \partial_{1}\partial_{n}f \\
 \vdots  & \ddots & \vdots \\
\partial_{n}\partial_{1} f & \dots & \partial_{n}^{2}f
\end{bmatrix}
\end{equation}
Each row is precisely $\vec{\nabla}\partial_{j}f$, and each
column likewise is $\partial_{i}\vec{\nabla}f$; the intuition is
$\mathrm{Hess}(f) \approx \vec{\nabla}^{2}f$. 

Now, since we are Taylor expanding about a critical point, our
approximation becomes
\begin{equation}
f(\vec{x}_{0}+\vec{h}) \approx f(\vec{x}_{0}) +
\frac{1}{2} \vec{h}\cdot\mathrm{Hess}(f)\vec{h}.
\end{equation}
We can consider the behaviour of $f$ near $\vec{x}_{0}$ by
studying the properties of $\mathrm{Hess}(f)$. Specifically, the
signs of the eigenvalues tells us whether the critical point is a
local maxima (all eigenvalues are positive) or a minima (all are
negative) or some saddle point (mixture having both positive and
negative eigenvalues). If there exists at least one eigenvalue
that vanishes, this test is inconclusive.

\N{Parting Thoughts:} What if we want to optimize a function
constrained to live on a surface? 

\subsection{Lagrange Multipliers}
%%
%% lagrangeMultiplier.tex
%% 
%% Made by Alex Nelson
%% Login   <alex@black-cherry>
%% 
%% Started on  Fri Jun 29 13:44:22 2012 Alex Nelson
%% Last update Tue Jul  3 10:48:45 2012 Alex Nelson
%%

\M
So, we considered finding extrema for some function
$f\colon\RR^{n}\to\RR$, but what if we constrain our focus to
some surface $g\colon\RR^{n}\to\RR$? For example, the unit circle
would have
\begin{equation}
g(x,y) = x^{2}+y^{2} - 1=0
\end{equation}
How do we find extrema for $f(x,y)=xy$ on the unit circle?\more

\M What can we do? First we can consider the level curves
$f(x,y)=c$. These are precisely the curves $y=c/x$. The gradient
vector for $f$ is precisely
\begin{equation}
\nabla f=\langle y,x\rangle.
\end{equation}
This points in the direction of increasing values of $f$. 

\M
We want to consider the situation when $\nabla f=\lambda\nabla
g$, i.e., when the gradient of $f$ is precisely a scaled tangent
of $g$. Why? Because we want the gradient of $f$ to point in the
direction of a tangent to our surface. We draw the circle, the
gradient vector $\nabla f$ in red, and $\nabla g$ in
blue. Remember, the red vectors point in the direction of
increasing values of $f$, and we restrict our movement along the circle:
\begin{center}
  \includegraphics{img/lagrangeMultiplier.0}
\end{center}
Observe when the red and blue vectors are perpendicular,
$f=0$. But when they overlap as a purple vector or point in
completely opposite direction, what happens?

This happens when
\begin{equation}
\langle y,x\rangle = \lambda\langle 2x,2y\rangle
\end{equation}
or equivalently
\begin{equation}
y=2\lambda x,\quad\mbox{and}\quad
x=2\lambda y.
\end{equation}
Solving for $2\lambda$, we find
\begin{equation}
2\lambda = \frac{y}{x} = \frac{x}{y}
\end{equation}
which implies
\begin{equation}
x^{2}=y^{2}.
\end{equation}
But we're not quite done!

\M
We must remain on the circle, so we also must demand that
$x^{2}+y^{2}=1$. This equivalently implies
\begin{equation}
x^{2}=\frac{1}{2}\implies x=\pm\frac{\sqrt{2}}{2}.
\end{equation}

\M
Is this optimal? Lets try approaching the problem differently. We
are working on the circle, which is the parametric curve
\begin{equation}
x(t) = \cos(t),\quad\mbox{and}\quad
y(t) = \sin(t).
\end{equation}
Thus the function we are optimizing becomes
\begin{equation}
\begin{aligned}
f(t) &= f\bigl(x(t),y(t)\bigr)\\
&=\cos(t)\sin(t)
\end{aligned}
\end{equation}
We see
\begin{equation}
f'(t) = -\sin^{2}(t)+\cos^{2}(t) = 0.
\end{equation}
We need to solve for $t$, to do so we rearrange terms
\begin{equation}
\sin^{2}(t)=\cos^{2}(t)
\end{equation}
and divide through by $\cos^{2}(t)$, getting
\begin{equation}
\tan^{2}(t) = 1.
\end{equation}
But this implies $t=(2n+1)\pi/4$ where $n=0$, $1$, $2$, or
$3$. Look: that's precisely describing
$(x,y)=(\pm1/\sqrt{2},\pm1/\sqrt{2})$. 


\N{Lagrange Multipliers, Constraints}
One way to consider this situation ``Optimize $f$ subject to the
constraint $g=0$'' is to say \emph{Okay, so suppose $g=0$, then
wouldn't we have}
\begin{equation}
f + \lambda g\approx f?
\end{equation}
The $g$ term vanishes anyways, so intuitively it seems ``equal-ish''.

\begin{example}[Minimizing Surface Area]
Find the dimensions of the cylinder with smallest surface area
whose volume is fixed at $16\pi$.

\emph{Solution}: The outline takes several steps, namely, (1)
construct the function, (2) take the derivatives, (3) solve.

\emph{Step One: Construct the Functions}. We first write
\begin{equation}
A(r,h) = \pi r^{2} + \pi r^{2} + 2\pi rh
\end{equation}
for the surface area, and
\begin{equation}
V(r,h) = \pi r^{2}h
\end{equation}
describes the area. The constraint is
\begin{equation}
C(r,h) = V(r,h) - 16\pi.
\end{equation}
Thus we construct the function
\begin{equation}
F(r,h) = A(r,h) - \lambda C(r,h).
\end{equation}
This concludes the first step.

\emph{Step Two: Take the Derivatives}. We find
\begin{equation}
\nabla F(r,h) = \langle \partial_{r}A(r,h)
- \lambda\partial_{r}C(r,h), \partial_{h}A(r,h)-\lambda\partial_{h}C(r,h)\rangle
  = 0.
\end{equation}
Observe
\begin{equation}
\partial_{r}A(r,h)=4\pi r+2\pi h,\quad\mbox{and}\quad
\partial_{r}C(r,h)=2\pi rh.
\end{equation}
We also have
\begin{equation}
\partial_{h}A(r,h)=2r\pi,\quad\mbox{and}\quad
\partial_{h}C(r,h)=\pi r^{2}.
\end{equation}
The derivative with respect to the Lagrange multiplier gives us
\begin{equation}
\partial_{\lambda}F(r,h) = C(r,h) = 0.
\end{equation}
So we can set up our equations as
\begin{equation}
\begin{aligned}
(\partial_{r}F&=0)&\quad &\quad &2\pi r+2\pi h&=\lambda 2\pi rh\\
(\partial_{h}F&=0)&\quad &\quad &2\pi r &=\lambda\pi r^{2}\\
(\partial_{\lambda}F&=0)&\quad &\quad &\pi r^{2}h-16\pi&=0.
\end{aligned}
\end{equation}
That concludes our second step.

\emph{Step Three: Solve}. We see immediately from the
$\partial_{h}F$ equation that
\begin{equation}
2\pi r = \lambda \pi r^{2}\implies \lambda=\frac{2}{r}.
\end{equation}
We plug this into the $\partial_{r}F$ equation, we get
\begin{equation}
\begin{aligned}
2\pi r+2\pi h &= \lambda2\pi rh\\
&=\left(\frac{2}{r}\right)2\pi rh\\
&=4\pi h
\end{aligned}
\end{equation}
and subtracting $2\pi h$ from both sides yields
\begin{equation}
2\pi r=2\pi h\implies r=h.
\end{equation}
Now we use the constraint
\begin{equation}
\pi r^{2}h = 16\pi
\end{equation}
substituting $r=h$ we get
\begin{equation}
\pi h^{3} = 16\pi\implies h = \sqrt[3]{16}.
\end{equation}
Thus when we take $r=h=\sqrt[3]{16}$, we minimize the surface area.
\end{example}


\begin{exercise}
Find the extrema of $f(x,y,z)=xy+yz+zx$ subject to the constraint
$g(x,y,z)=x^{2}+y^{2}+z^{2}-1$.
\end{exercise}
\begin{exercise}
Find the extrema of $f(x,y)=\exp(-xy)$ subject to the constraint
$g(x,y)=x^{2}+y^{2}-1$.
\end{exercise}



\subsection{Total Differential}
%%
%% totalDifferential.tex
%% 
%% Made by Alex Nelson
%% Login   <alex@black-cherry>
%% 
%% Started on  Fri Jun 29 14:21:33 2012 Alex Nelson
%% Last update Fri Jun 29 14:21:58 2012 Alex Nelson
%%
\N{Total Differential}
Lets consider some function
\begin{equation}
f\colon\RR^{2}\to\RR.
\end{equation}
What is $\D f$? Well, we can write it as
\begin{equation}
\D\vec{x}\cdot\vec{\nabla}f = (\D x\,\partial_{x}+\D
y\,\partial_{y})f(x,y)
\end{equation}
in Cartesian coordinates. We can write a ``linear approximation''
to $f(x,y)$ as
\begin{equation}
f(x+\Delta x, y+\Delta y)\approx f(x,y) + \langle\Delta x,\Delta
y\rangle\cdot\vec{\nabla}f(x,y).
\end{equation}
Again, this should remind us of the linear approximation when we
used the tangent line as an approximation to a curve.

\M
Another way to consider to total differential $\D{f}$ is by using
the ``chain rule'' (wink wink). We write
\begin{equation}
\D f = \frac{\partial f(x,y)}{\partial x}\D x
+\frac{\partial f(x,y)}{\partial y}\D y.
\end{equation}
This can be quite useful!

\begin{example}
Lets consider a cylinder whose radius is $r=10$ and height is
$h=100$. Its volume is $V(r,h)=\pi r^{2}h$. The measurement is
correct to $0.1$ precision, what's the error in our measurement?

Well, we approximate it as
\begin{equation}
\begin{aligned}
\D V(r,h) &\approx 0.1\partial_{r}V(r,h) + 0.1\partial_{h}
V(r,h)\\
&= 0.1(2\pi rh) + 0.1(\pi r^{2})
\end{aligned}
\end{equation}
The error is thus
\begin{equation}
\D V(10,100) \approx 0.1 (2\pi\cdot10\cdot100) + 0.1(\pi\cdot 10^{2})
\approx 210\pi.
\end{equation}
The error is approximately $659.7339$, and our estimated volume
is $31415$. The error is about 2\%, which is quite good.
\end{example}



\nocite{*}
\bibliographystyle{utcaps}
\bibliography{chain}
\end{document}
