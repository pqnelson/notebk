%%
%% calc.sphere.tex
%% 
%% Made by Alex Nelson
%% Login   <alex@tomato>
%% 
%% Started on  Thu Sep  3 13:26:42 2009 Alex Nelson
%% Last update Thu Sep  3 13:26:42 2009 Alex Nelson
%%

Consider the path integral expression for the 3-sphere, which
requires us to first consider the cobordism
\begin{equation}%\label{eq:}
\begin{CD}
\{0\}         @<<<     \{0\}     @<<<     \{0\}\\
@VVV                   @VVV                @VVV\\
\mathbb{Z}^{2}@<<<     \mathbb{Z}^{2}     @<<<     \mathbb{Z}^{2} @<<< \mathbb{Z}^{2}\\
@VVV                   @VVV                @VVV\\
\{0\}         @<<<     \{0\}     @<<<     \{0\}
\end{CD}
\end{equation}

For the sphere, we have the time evolution operator be described
by
\begin{equation}%\label{eq:}
Z(S):\mathbb{C}\to\mathbb{C}.
\end{equation}
We consider the path integral expression
\begin{equation}%\label{eq:}
\<1,Z(S)\cdot1\>=\int_{\mathcal{A}(S)}e^{-S(A)}\mathcal{D}A
\end{equation}
where the action term is (as we have 2 $X^{p+1}$ cells)
\begin{equation}%\label{eq:}
e^{-S(A)}=\sum_{n^{i}\in\mathbb{Z}^{2}}e^{-(F_{i}+2\pi n_{i})(F^{i}+2\pi n^{i})/(2e^{2}V)}.
\end{equation}
%\marginpar{We see that this exponential really has the exponent be $(F_{1}+2\pi n)^{2}+(F_{2}+2\pi m)^{2}-(F_{1}-F_{2})\pi(m-n)$ but we see $F_{2}=F_{1}$ so that extra term vanishes.}
This is equivalent to
\begin{equation}%\label{eq:}
e^{-S(A)} = \sum_{m,n\in\mathbb{Z}}e^{-(F_{1}+2\pi
  m)^{2}/2e^{2}V}e^{-(F_{2}+2\pi n)^{2}/2e^{2}V}
\end{equation}
which turns our expression into
\begin{equation}%\label{eq:}
\<1,Z(S)\cdot1\> = \int^{2\pi}_{0}\int^{2\pi}_{0}\sum_{m,n}e^{-m^{2}e^{2}V/2}e^{-n^{2}e^{2}V/2}e^{i(m+n)(A_{2}-A_{1})}\frac{dA_{1}}{2\pi}\frac{dA_{2}}{2\pi}.
\end{equation}
We see this becomes
\begin{equation}%\label{eq:}
\<1,Z(S)\cdot1\> = \int^{2\pi}_{0}\sum_{m,n}e^{-m^{2}e^{2}V/2}e^{-n^{2}e^{2}V/2}e^{i(m+n)(-A_{1})}\delta_{-m,n}\frac{dA_{1}}{2\pi}.
\end{equation}
But we see, setting $m:=-n$ and summing over $n$, that
\begin{equation}%\label{eq:}
\<1,Z(S)\cdot1\>=\sum_{n\in\mathbb{Z}}e^{-n^{2}e^{2}V/2}e^{-n^{2}e^{2}V/2}\int^{2\pi}_{0}\frac{dA_{1}}{2\pi}
\end{equation}
This reduces to
\begin{equation}%\label{eq:}
\<1,Z(S)\cdot1\>=\sum_{n\in\mathbb{Z}}e^{-n^{2}e^{2}V}
\end{equation}
and thus
\begin{equation}%\label{eq:}
Z(S):1\mapsto\sum_{n\in\mathbb{Z}}e^{-n^{2}e^{2}V}
\end{equation}
describes our time evolution operator.




\begin{comment}
We see that by expanding this out, and referring to Eq (13) of \S
2.325 of Gradshteyn and Ryzhik~\cite{gradshteyn2000table} (which is:
\begin{equation}%\label{eq:}
\int e^{-(ax^{2}+bx+c)}dx = \frac{1}{2}\sqrt{\frac{\pi}{a}}\exp\left(\frac{b^{2}-ac}{a}\right)\erf\left(\frac{b}{\sqrt{a}}+\sqrt{a}x\right)
\end{equation}
reproduced for reference) we have our path integral expression become 
\begin{equation}%\label{eq:}
\<1,Z(S)1\>=\sum_{n\in\mathbb{Z}}e^{-4\pi^{2}n^{2}}\int\left[\int
  e^{-(x^{2}+2[2\pi n+y]x + [y^{2}-4\pi ny])}\frac{dx}{2\pi}\right]\frac{dy}{2\pi}.
\end{equation}
We can either carry out this integral, or use the results from Eq \eqref{eq:cheatEqn}
in our last section to conclude that
\begin{equation}%\label{eq:}
\begin{split}
&\int^{2\pi}_{0}e^{-\left[A_{1}^{2}+2(-2\pi-A_{2})A_{1}+(A_{2}^{2}+4n^{2}\pi^{2}-4nA_{2}\pi)\right]}dA_{1}\\
&=\frac{\sqrt{e^{2}V\pi }}{2}  \left(\erf\left(\frac{2 \pi 
    n+A_{2}}{e \sqrt{V}}\right)-\erf\left(\frac{2 \pi  (n-1)+A_{2}}{e
    \sqrt{V}}\right)\right).
\end{split}
\end{equation}
In this situation, we have the following result (omitting the
forgotten factors of $1/2\pi$)
\begin{align}
Z(T):1\mapsto &\frac{1}{2} e^{-4 (n-1)^2 \pi ^2/e^2 V} V e^2
-e^{-4 n^2 \pi^2/e^2 V} V e^2
+\frac{1}{2} e^{-4 (n+1)^2 \pi ^2/e^2 V} V e^2\nonumber\\
&+n\pi ^{3/2} \sqrt{V} \erf\left(\frac{2 (n-1) \pi }{e \sqrt{V}}\right) e
-\pi ^{3/2} \sqrt{V} \erf\left(\frac{2 (n-1) \pi }{e\sqrt{V}}\right) e\nonumber\\
&-2 n \pi ^{3/2} \sqrt{V} \erf\left(\frac{2 n \pi }{e \sqrt{V}}\right) e
+n \pi ^{3/2} \sqrt{V} \erf\left(\frac{2 (n+1) \pi }{e \sqrt{V}}\right) e\nonumber\\
&+\pi ^{3/2} \sqrt{V} \erf\left(\frac{2 (n+1) \pi }{e \sqrt{V}}\right) e
\end{align}
\end{comment}
