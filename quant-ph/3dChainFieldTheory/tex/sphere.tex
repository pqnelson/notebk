%%
%% sphere.tex
%% 
%% Made by Alex Nelson
%% Login   <alex@tomato>
%% 
%% Started on  Sat Aug 29 16:03:43 2009 Alex Nelson
%% Last update Sat Aug 29 16:03:43 2009 Alex Nelson
%%

Observe the inductive procedure to constructing an $n$-sphere. We
start with $B^{0}$, the 0-ball, which is just a humble point. We
have $S^{0}$, the 0-sphere, which consists of 2 points. The
1-ball is a line segment between 2 points. Now here is the first
inductive case study: the 1-sphere is a circle. It consists of 2
$B^{1}$ with their boundaries glued together: 
\begin{figure}[h]
\begin{center}
\includegraphics{img/img.3}
\end{center}
\end{figure}
The region whose boundary is $S^{1}$ is precisely $B^{2}$. We
should consider the chain complex describing $S^{1}$ before
moving on. We have already drawn in the orientation of the edges
(and labeled them too!), so we need to pick an orientation of the
face itself. We see first of all that there are 2 edges and 2
vertices. This allows us to set up the chain complex as
\begin{equation}%\label{eq:}
\mathbb{Z}^{2}\xleftarrow{\partial_{0}}\mathbb{Z}^{2}
\end{equation}
We have picked an orientation for the edges. We see that
\begin{equation}%\label{eq:}
ae_{1}+be_{2}=\begin{bmatrix}a\\b\end{bmatrix}\quad\text{and}\quad 
av_{1}+bv_{2}=\begin{bmatrix}a\\b\end{bmatrix}
\end{equation}
We can describe the boundary operator by
\begin{equation}%\label{eq:}
\partial_{0}\left(\begin{bmatrix}1\\0\end{bmatrix}\right)=\begin{bmatrix}-1\\1\end{bmatrix}\quad\text{and}\quad
\partial_{0}\left(\begin{bmatrix}0\\1\end{bmatrix}\right)=\begin{bmatrix}1\\-1\end{bmatrix}
\end{equation}
which means that
\begin{equation}%\label{eq:}
\partial_{0}:\begin{bmatrix}a\\b\end{bmatrix}\mapsto\begin{bmatrix}b-a\\a-b\end{bmatrix}
\end{equation}
describes the behavior of the boundary operator.

We can now similarly construct $S^{2}$. We have 2 $B^{2}$ and glue their
boundaries together, or diagrammaticaly:
\begin{figure}[H]
\includegraphics{img/img.4}
\end{figure}
\noindent We should consider now the chain complex that describes
$S^{3}$. Observe that the two vertices on each of the boundaries
of the $B^{2}$ are identified to be the same. We are thus left
with two vertices for $S^{3}$ (as doodled on the right hand side
of our figure). The two edges for the boundary of $B^{2}$ are
likewise identified to be the same. We have, however, two
\emph{distinct faces} for $S^{3}$ which corresponds to the two
distinct $B^{2}$. This produces the following chain
\begin{equation}%\label{eq:}
\mathbb{Z}^{2}\xleftarrow{\partial_{0}}\mathbb{Z}^{2}\xleftarrow{\partial_{1}}\mathbb{Z}^{2}
\end{equation}
now we are left to deduce the boundary operators. This demands a
choice of orientation of edges, and faces. We have picked an
orientation for the edges as drawn in our figure, let $e_{1}$
begin at $v_{1}$ and end at $v_{2}$; $e_{2}$ begin at $v_{2}$ and
end at $v_{1}$. We then let
\begin{equation}%\label{eq:}
ae_{1}+be_{2}=\begin{bmatrix}a\\b\end{bmatrix}\quad\text{and}\quad 
av_{1}+bv_{2}=\begin{bmatrix}a\\b\end{bmatrix}
\end{equation}
and then deduce that
\begin{equation}%\label{eq:}
\partial_{0}\left(\begin{bmatrix}1\\0\end{bmatrix}\right)=\begin{bmatrix}-1\\1\end{bmatrix}\quad\text{and}\quad
\partial_{0}\left(\begin{bmatrix}0\\1\end{bmatrix}\right)=\begin{bmatrix}-1\\1\end{bmatrix}
\end{equation}
which implies that
\begin{equation}%\label{eq:}
\partial_{0}:\begin{bmatrix}a\\b\end{bmatrix}\mapsto\begin{bmatrix}-a-b\\a+b\end{bmatrix}
\end{equation}
determines one of the boundary operators completely. To determine
the other, we can cheat and figure out what square matrix $A$ is such
that
\begin{equation}%\label{eq:}
\begin{bmatrix} -1 & -1\\1 & 1\end{bmatrix}A = 0
\end{equation}
so the desired relation $\partial^{2}=0$ holds. This would
determine $A$ up to an arbitrary sign (which is determined by the
orientation of the faces). We see that
\begin{equation}%\label{eq:}
A = \begin{bmatrix}a & b\\
-a & -b\end{bmatrix}
\end{equation}
is the structure of $\partial_{1}$. This in fact implies
$(\partial_{0})^{2}=0$. We are forced to pick orientations for
the actual open balls $B^{2}$, but open doing so we see that it's
simply one $B^{2}$ minus the other, implying that (up to a sign
determined by orientation!) $a=b=\pm1$.

The inductive ritual should be clear: we take two copies of
$B^{n}$ and glue the boundaries, the $S^{n-1}$, together to
create a $S^{n}$. More formally, we present it as a theorem.

\begin{thm}%\label{thm:}
Consider the sphere $S^{n}$. Then the chain complex describing it
is
\begin{equation}%\label{eq:}
\mathbb{Z}^{2} \xleftarrow{\partial_{0}}\mathbb{Z}^{2}\xleftarrow{\partial_{1}}\cdots\xleftarrow{\partial_{n-1}}\mathbb{Z}^{2}
\end{equation}
where $\partial_{i}$ are the boundary operators.
\end{thm}
\begin{proof}
\noindent\textbf{Base Case (n=1):} We have already done this, two $B^{1}$
(line segments) glued together at their boundary $S^{0}$ (two
points) results in the chain complex
\begin{equation}%\label{eq:}
\mathbb{Z}^{2}\xleftarrow{\partial_{0}}\mathbb{Z}^{2}
\end{equation}
where
\begin{equation}%\label{eq:}
\partial_{0} = \begin{bmatrix}-1 & -1\\1 & 1\end{bmatrix}
\end{equation}
up to a sign determined by orientation.

\noindent\textbf{Inductive Hypothesis:} assume this works for arbitrary
$n$. That is, we have
\begin{equation}%\label{eq:}
\mathbb{Z}^{2}\xleftarrow{\partial_{0}}\cdots\xleftarrow{\partial_{n-1}}\mathbb{Z}^{2}
\end{equation}
for our chain complex.

\noindent\textbf{Inductive Case (n+1):} We have the boundaries of the
$B^{n+1}$, which takes care of the first $n-1$ boundary operators
and the first $n$ $\mathbb{Z}^{2}$ components of the chain. We
are still left with two copies of the \emph{interior} of
$B^{n+1}$. This constitutes an additional $\mathbb{Z}^{2}$
component in the chain diagram, and implies that the chain
complex looks like
\begin{equation}%\label{eq:}
\underbracket[0.25pt]{\left(\mathbb{Z}^{2}\xleftarrow{\partial_{0}}\cdots\xleftarrow{\partial_{n-1}}\mathbb{Z}^{2}\right)}_{\text{inductive hypothesis}}\xleftarrow{\partial_{n}}\mathbb{Z}^{2}
\end{equation}
where $\partial$ is the boundary operator. This concludes the
inductive case, and the proof.
\end{proof}
For our intentions however, we need to consider
two $B^{3}$, and glue their boundaries together!
