%%
%% conclusion.tex
%% 
%% Made by Alex Nelson
%% Login   <alex@tomato>
%% 
%% Started on  Sat Jul 25 14:56:27 2009 Alex Nelson
%% Last update Sat Jul 25 14:56:27 2009 Alex Nelson
%%

We've reviewed some notions from quantum mechanics, such as the
Rigged Hilbert Space and using unitary operators for
observables. When using representation theory, we need a unitary
representation of a group for use in quantum theory.

We've introduced various aspects of making quantum mechanics
relativistic. The main approach is to take advantage of the fact
that special relativity is basically ``just'' the Poincar\'e
group. We then proceeded to find a unitary representation of the
Lorentz group and the group of spacetime translations, then
combined them in a suitably nice way.

We've considered the situation of making the position operator
relativistic, and concluded after a few naive attempts that it
wouldn't work. 

The interested reader is free to peruse the resources cited in
the bibliography for further reading (specifically, the notion of
measurement relative to an observer is tackled beautifully in
Gambini and Porto~\cite{Gambini:2001pq}).
