%%
%% cap.tex
%% 
%% Made by Alex Nelson
%% Login   <alex@black-cherry>
%% 
%% Started on  Mon Aug 24 13:27:24 2009 Alex Nelson
%% Last update Mon Aug 24 13:27:24 2009 Alex Nelson
%%

\begin{wrapfigure}{r}{1.3in}
\includegraphics{img/img.7}
%\caption{A cap.}
%\caption{}\label{fig:img7}
\end{wrapfigure}

We see for the situation as doodled on the right where we start
from ``nothing'' and end up with the final state being a
``cap''. This is a little tricky, not because it is difficult but
because it is so vacuous. We can see that the initial state
consists of no vertices, no edges, no $p$-cells. That means that
$X_{p}=\emptyset$ for the initial state's data. The free group
generated by this is then
\begin{equation}%\label{eq:}
\mathbb{Z}^{\emptyset} = \{0\}
\end{equation}
since there is exactly one map $\emptyset\to\mathbb{Z}$, it
generates the trivial group. (As we are working with commutative
groups, the group operation is addition, and in this context the
identity element is generically denoted by 0 --- thus the trivial
group is $\{0\}$.) We are working in two dimensions, the chain in
this context for the initial state would be
\begin{equation}%\label{eq:}
\mathbb{Z}^{\emptyset}\leftarrow\mathbb{Z}^{\emptyset}=\{0\}\leftarrow\{0\}.
\end{equation}
This is the boring part of the chain calculations.

The more interesting part of the chain calculations comes from
the final state. We have 1 edge and 1 vertex, which implies the
chain complex would be for the final state
\begin{equation}%\label{eq:}
\mathbb{Z}\leftarrow\mathbb{Z}
\end{equation}
and thus for the entire process as a whole
\begin{equation}%\label{eq:}
\begin{CD}
\{0\}       @<<< \{0\}\\
@VVV              @VVV\\
\mathbb{Z}  @<<< \mathbb{Z} @<<< \mathbb{Z}\\
@AAA              @AAA\\
\mathbb{Z}  @<<< \mathbb{Z}
\end{CD}
\end{equation}
This is completely trivial. 

Now we should worry about the Hilbert spaces for the initial and
final states. For the initial state, we have a Hilbert space for
$L^{2}(\text{connections on }\emptyset)$. More precisely we can
write this as
\begin{subequations}
\begin{align}
L^{2}(\text{connections on }\emptyset) &= L^{2}(\hom(\mathbb{Z}^{\emptyset},U(1)))\\
&= L^{2}(\hom(\<e\>,U(1)))\\
&= L^{2}(\<e\>)\\
&= \mathbb{C}
\end{align}
\end{subequations}
The last step is justified since the set of complex valued
functions on a singleton is all complex numbers.

The Hilbert space for the final state is --- as we have seen two
times before --- really just
\begin{subequations}
\begin{align}
L^{2}(\hom(\mathbb{Z},U(1))) = L^{2}(U(1)).
\end{align}
\end{subequations}
This concludes the portion of deducing what the Hilbert spaces
are.

Now we consider the expression describing the probability
amplitude for the time evolution of this particular
process. Observe
\begin{equation}%\label{eq:}
\<\!\!\!\!\!\!\!\!{\color{red}\overbracket[0.25pt]{\phi\;\;\;\;}^{\;\;\;\;\;\in L^{2}(U(1))}}\!\!\!\!\!\!\!\!\!\!\!\!\!\!\!\!,\,\,Z(\iota)\!\!\!\!{\color{blue}\;\;\;\underbracket[0.25pt]{\psi\;\;\;\;}_{\in\mathbb{C}}}\!\!\!\!\!\> =
\int_{\substack{\text{connections}\\\text{on }\iota}} \overline{\phi(A|_{S'})}\psi(A|_{\emptyset})e^{-S(A)}\mathcal{D}A.
\end{equation}
We choose our basis vectors to test this expression on to be
$\psi=1$ and $\phi=\exp(ikA)$. Note that this equation implies
that 
\begin{equation}%\label{eq:}
Z(\iota):\mathbb{C}\to L^{2}(U(1)).
\end{equation}
We plug in our expression for $\mathcal{D}A=dA/2\pi$, among our
deductions above, to get the following:
\begin{equation}%\label{eq:}
\<e^{ikA},Z(\iota)1\> = \int^{2\pi}_{0} e^{-ikA}\cdot 1e^{-S(A)}\frac{dA}{2\pi}.
\end{equation}
We will manipulate this as before to deduce how $Z(\iota)$
behaves.

As before, we compute
\begin{equation}%\label{eq:}
e^{-S(A)} = \sum_{n\in\mathbb{Z}} e^{\frac{-1}{2e^{2}}(F+2n\pi)^{2}/V}.
\end{equation}
We need to calculate the curvature, which is easy in our
situation. It is, up to an arbitrary sign, $F=A$. Putting this
into our expression, we deduce
\begin{subequations}
\begin{align}
\<e^{ikA},Z(\iota)1\> &=
\int^{2\pi}_{0}e^{-ikA}\sum_{n\in\mathbb{Z}}e^{\frac{-1}{2e^{2}}(A+2n\pi)^{2}/V}\frac{dA}{2\pi}\\
&= \int^{2\pi}_{0}e^{-ikA}\underbracket[0.25pt]{\sum_{n\in\mathbb{Z}}e^{\frac{-n^{2}e^{2}V}{2}}e^{inA}}_{=Z(\iota)1}\frac{dA}{2\pi}.
\end{align}
\end{subequations}
We identify the underbracketed term with $[Z(\iota)1]\in L^{2}(U(1))$.
