%%
%% altRatioTest.tex
%% 
%% Made by Alex Nelson
%% Login   <alex@black-cherry>
%% 
%% Started on  Sun Jun 17 14:33:38 2012 Alex Nelson
%% Last update Sun Jun 17 15:36:44 2012 Alex Nelson
%%
%% \documentclass{article}
%% \pdfinfo{ /CreationDate (20120617143338)}
%% \usepackage{blog}
%% \begin{document}
\M
We have another couple of methods testing if an alternating
series converges. It's worth knowing as many different ways as
possible, because sometimes one doesn't work well (or at all).

We will consider a couple tests. For each test we provide a
proof that it works, and a couple examples.

\N{Alternating Ratio Test}
Consider the series 
\begin{equation}
\sum^{\infty}_{n=0}(-1)^{n}a_{n}
\end{equation}
Let
\begin{equation}
L = \lim_{n\to\infty}\frac{a_{n+1}}{a_{n}}.
\end{equation}
\begin{enumerate}
\item If $L<1$, then the series $\sum (-1)^{n}a_{n}$ conveges
  absolutely.
\item If $L>1$ (or $L=\infty$), then the series
  $\sum (-1)^{n}a_{n}$ diverges.
\end{enumerate}

\begin{proof}[Proof of Convergence]
We will prove if $L<1$, then the series
\begin{equation}
\sum^{\infty}_{n=0}a_{n}
\end{equation}
converges. The absolute convergence test implies the alternating
series will converge. What to do?

We first consider some $\varepsilon$ satisfying
$L<\varepsilon<1$. (Can we do this? Sure, pick $(L+1)/2$, and
we're good!) Since we suppose $L<1$, then there exists an $N$
such that
\begin{equation}
\left|\frac{a_{n+1}}{a_{n}}\right|<r\quad\mbox{for any }n\geq N.
\end{equation}
We have
\begin{equation}
\begin{aligned}
a_{N+1}&<r a_{N}\\
a_{N+2}&<r a_{N+1}<r^{2}a_{N}\\
a_{N+3}&<r^{3}a_{N}\\
a_{N+k}&<r^{k}a_{N}
\end{aligned}
\end{equation}
So we form a geometric series
\begin{equation}
\sum^{\infty}_{k=0}a_{N}r^{k} = \frac{a_{N}}{1-r}
\end{equation}
which bounds the ``most'' of our series
\begin{equation}
0\leq\sum^{\infty}_{k=0}a_{N+k}\leq\sum^{\infty}_{k=0}a_{N}r^{k}
\end{equation}
The comparison test tells us that ``most'' of our series
converges. But what about our whole series? We write it as
\begin{equation}
\sum^{\infty}_{n=0}a_{n} = \underbrace{\sum^{N-1}_{n=0}a_{n}}_{\text{finite}}+\underbrace{\sum^{\infty}_{k=0}a_{N+k}}_{\text{converges}}
\end{equation}
so it converges.
\end{proof}

\begin{proof}[Proof of Divergence]
We assume that $L>1$. The ratio $a_{n+1}/a_{n}$ will eventually
be greater than 1, too. So there exists an $N$ such that
\begin{equation}
\left|\frac{a_{n+1}}{a_{n}}\right|>1\quad\mbox{for any }n\geq N.
\end{equation}
So we see $|a_{n+1}|>|a_{n}|$ whenever $n\geq N$, thus
\begin{equation}
\lim_{n\to\infty}a_{n}\not=0.
\end{equation}
Thus it's impossible for the series $\sum a_{n}$ to converge!
\end{proof}


\begin{remark}
We only really proved that the series doesn't converge
\emph{absolutely}. If we pick some $r$ between $1<r<L$, then
there exists an $N$ such that
\begin{equation}
\left|\frac{a_{n+1}}{a_{n}}\right|>r\quad\mbox{for any }n\geq N.
\end{equation}
So we have
\begin{equation}
a_{N+k} > r^{k}a_{N}.
\end{equation}
We have our series
\begin{equation}
\begin{aligned}
\sum^{\infty}_{k=1}(-r)^{k}a_{N}
&= a_{N} \sum^{\infty}_{k=1}(r^{2k}-r^{2k-1})\\
&= a_{N} (r-1)\sum^{\infty}_{k=1}r^{2k}.
\end{aligned}
\end{equation}
We see since $r>1$ that the series
\begin{equation}
\sum^{\infty}_{k=1}r^{2k}\quad\mbox{diverges}
\end{equation}
Thus
\begin{equation}
\sum^{\infty}_{k=1}(-r)^{k}a_{N}\quad\mbox{diverges}
\end{equation}
We see then that the series
\begin{equation}
\sum^{\infty}_{k=1}a_{N+k}\geq\sum^{\infty}_{k=1}(-r)^{k}a_{N}
\end{equation}
diverges by the comparison test.
\end{remark}


\N{The Root Test}
Consider the series 
\begin{equation}
\sum^{\infty}_{n=1}(-1)^{n}a_{n}
\end{equation}
Let
\begin{equation}
L = \lim_{n\to\infty}\sqrt[n]{a_{n}}
\end{equation}
\begin{enumerate}
\item If $L<1$, then the series converges absolutely;
\item If $L>1$ (or $L=\infty$), then the series diverges.
\end{enumerate}

\begin{proof}
It's similar to the ratio test, for the convergent case we pick
some $r$ satisfying $L<r<1$. Then we have some $N$ satisfying
\begin{equation}
\sqrt[n]{a_{n}}<r\quad\mbox{for any }n\geq N.
\end{equation}
which implies
\begin{equation}
a_{n}<r^{n}.
\end{equation}
Thus we have
\begin{equation}
\sum^{\infty}_{k=1}a_{N+k}<\sum^{\infty}_{k=1}r^{N+k}<\sum^{\infty}_{k=0}r^{k}=\frac{1}{1-r}
\end{equation}
which by the  comparison test implies convergence. The proof for
divergence is similar.
\end{proof}
\begin{remark}
When $L=1$, the ratio test doesn't say anything about convergence
or divergence for the series.
\end{remark}

%\end{document}
