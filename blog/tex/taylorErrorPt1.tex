%%
%% taylorErrorPt1.tex
%% 
%% Made by Alex Nelson
%% Login   <alex@black-cherry>
%% 
%% Started on  Mon Jun 18 10:34:28 2012 Alex Nelson
%% Last update Mon Jun 18 10:51:51 2012 Alex Nelson
%%

\M
So last time we constructed a polynomial using the first $n$
terms in the Taylor series. This ``Taylor polynomial''
approximated our function, and we want to know \emph{how well does it approximate?}

We will derive the Taylor series differently. Our derivation will
give us a natural error term. \more{}

\N{Set Up} Lets consider $f(x)$ which is sufficiently nice around
$x=0$ (i.e., it has enough derivatives at $x=0$). The fundamental
theorem of calculus tells us that
\begin{equation}\label{eq:taylorErr1:line1}
f(x) = f(0) + \int^{x}_{0}f'(t)\,\D t.
\end{equation}
Wonderful.

\N{Linear Approximation}
We now use integration by parts for the integral in Equation \eqref{eq:taylorErr1:line1}:
\begin{equation}
\begin{aligned}
\int^{x}_{0}f'(t)\,\D t &=
\left.tf'(t)\right|^{x}_{0}-\int^{x}_{0}tf''(t)\,\D t\\
&= xf'(x)-0f'(0) - \int^{x}_{0}tf''(t)\,\D t\\
&= xf'(x) - \int^{x}_{0}tf''(t)\,\D t.
\end{aligned}
\end{equation}
Now we can plug in Equation \eqref{eq:taylorErr1:line1} for the
first term in the right hand side:
\begin{equation}
\begin{aligned}
xf'(x) - \int^{x}_{0}tf''(t)\,\D t &=
x\left(f'(0)+\int^{x}_{0}f'(t)\,\D t\right) - \int^{x}_{0}tf''(t)\,\D t\\
&= xf'(0) - \int^{x}_{0}(t-x)f''(t)\,\D t.
\end{aligned}
\end{equation}
We plug this back into Equation \eqref{eq:taylorErr1:line1}
\begin{equation}
\begin{aligned}
f(x) &= f(0) + \int^{x}_{0}f'(t)\,\D t\\
&= f(0) +\left[ xf'(0) - \int^{x}_{0}(t-x)f''(t)\,\D t\right].
\end{aligned}
\end{equation}
Observe the first two terms are precisely the linear
approximation to $f(x)$. What's the third term? \emph{The error term!}
But we're not done yet!

\N{Inductive Procedure}
We can iterate a procedure to get a better and better
approximation. Performing integration by parts on
\begin{equation}
\int^{x}_{0}(t-x)f''(t)\,\D t
\end{equation}
will give us the next term in our approximation plus an
integral. The integral gives us the error for using a quadratic
approximation. 

Lets do it! We use integration by parts 
\begin{equation}
\begin{aligned}
\int^{x}_{0}(t-x)f''(t)\,\D t &=
\left.\frac{(t-x)^{2}}{2!}f''(t)\right|^{x}_{0}-\int^{x}_{0}\frac{(t-x)^{2}}{2!}f'''(t)\,\D t\\
&=\frac{-(-x)^{2}}{2!}f''(0)-\int^{x}_{0}\frac{(t-x)^{2}}{2!}f'''(t)\,\D t
\end{aligned}
\end{equation}
So our approximation becomes
\begin{equation}
\begin{aligned}
f(x) &= f(0) +  xf'(0) - \left[\int^{x}_{0}(t-x)f''(t)\,\D  t\right]\\
&=f(0) + xf'(0) - \left[\frac{-(-x)^{2}}{2!}f''(0)-\int^{x}_{0}\frac{(t-x)^{2}}{2!}f'''(t)\,\D t\right]\\
&=f(0) + xf'(0) + \frac{x^{2}}{2!}f''(0)+\left[\int^{x}_{0}\frac{(t-x)^{2}}{2!}f'''(t)\,\D t\right]
\end{aligned}
\end{equation}
The brackted term in the final line is the error term for using
the quadratic approximation. But look: the quadratic
approximation corresponds to the first 3 terms of the Taylor series!
When we do this iterative procedure on the error term,
integrating by parts will produce a higher order
approximation. And for free, we get the error term telling us how
good our approximation is!
