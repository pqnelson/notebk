%%
%% lagrangeMultiplier.tex
%% 
%% Made by Alex Nelson
%% Login   <alex@black-cherry>
%% 
%% Started on  Fri Jun 29 13:44:22 2012 Alex Nelson
%% Last update Tue Jul  3 10:48:45 2012 Alex Nelson
%%

\M
So, we considered finding extrema for some function
$f\colon\RR^{n}\to\RR$, but what if we constrain our focus to
some surface $g\colon\RR^{n}\to\RR$? For example, the unit circle
would have
\begin{equation}
g(x,y) = x^{2}+y^{2} - 1=0
\end{equation}
How do we find extrema for $f(x,y)=xy$ on the unit circle?\more

\M What can we do? First we can consider the level curves
$f(x,y)=c$. These are precisely the curves $y=c/x$. The gradient
vector for $f$ is precisely
\begin{equation}
\nabla f=\langle y,x\rangle.
\end{equation}
This points in the direction of increasing values of $f$. 

\M
We want to consider the situation when $\nabla f=\lambda\nabla
g$, i.e., when the gradient of $f$ is precisely a scaled tangent
of $g$. Why? Because we want the gradient of $f$ to point in the
direction of a tangent to our surface. We draw the circle, the
gradient vector $\nabla f$ in red, and $\nabla g$ in
blue. Remember, the red vectors point in the direction of
increasing values of $f$, and we restrict our movement along the circle:
\begin{center}
  \includegraphics{img/lagrangeMultiplier.0}
\end{center}
Observe when the red and blue vectors are perpendicular,
$f=0$. But when they overlap as a purple vector or point in
completely opposite direction, what happens?

This happens when
\begin{equation}
\langle y,x\rangle = \lambda\langle 2x,2y\rangle
\end{equation}
or equivalently
\begin{equation}
y=2\lambda x,\quad\mbox{and}\quad
x=2\lambda y.
\end{equation}
Solving for $2\lambda$, we find
\begin{equation}
2\lambda = \frac{y}{x} = \frac{x}{y}
\end{equation}
which implies
\begin{equation}
x^{2}=y^{2}.
\end{equation}
But we're not quite done!

\M
We must remain on the circle, so we also must demand that
$x^{2}+y^{2}=1$. This equivalently implies
\begin{equation}
x^{2}=\frac{1}{2}\implies x=\pm\frac{\sqrt{2}}{2}.
\end{equation}

\M
Is this optimal? Lets try approaching the problem differently. We
are working on the circle, which is the parametric curve
\begin{equation}
x(t) = \cos(t),\quad\mbox{and}\quad
y(t) = \sin(t).
\end{equation}
Thus the function we are optimizing becomes
\begin{equation}
\begin{aligned}
f(t) &= f\bigl(x(t),y(t)\bigr)\\
&=\cos(t)\sin(t)
\end{aligned}
\end{equation}
We see
\begin{equation}
f'(t) = -\sin^{2}(t)+\cos^{2}(t) = 0.
\end{equation}
We need to solve for $t$, to do so we rearrange terms
\begin{equation}
\sin^{2}(t)=\cos^{2}(t)
\end{equation}
and divide through by $\cos^{2}(t)$, getting
\begin{equation}
\tan^{2}(t) = 1.
\end{equation}
But this implies $t=(2n+1)\pi/4$ where $n=0$, $1$, $2$, or
$3$. Look: that's precisely describing
$(x,y)=(\pm1/\sqrt{2},\pm1/\sqrt{2})$. 


\N{Lagrange Multipliers, Constraints}
One way to consider this situation ``Optimize $f$ subject to the
constraint $g=0$'' is to say \emph{Okay, so suppose $g=0$, then
wouldn't we have}
\begin{equation}
f + \lambda g\approx f?
\end{equation}
The $g$ term vanishes anyways, so intuitively it seems ``equal-ish''.

\begin{example}[Minimizing Surface Area]
Find the dimensions of the cylinder with smallest surface area
whose volume is fixed at $16\pi$.

\emph{Solution}: The outline takes several steps, namely, (1)
construct the function, (2) take the derivatives, (3) solve.

\emph{Step One: Construct the Functions}. We first write
\begin{equation}
A(r,h) = \pi r^{2} + \pi r^{2} + 2\pi rh
\end{equation}
for the surface area, and
\begin{equation}
V(r,h) = \pi r^{2}h
\end{equation}
describes the area. The constraint is
\begin{equation}
C(r,h) = V(r,h) - 16\pi.
\end{equation}
Thus we construct the function
\begin{equation}
F(r,h) = A(r,h) - \lambda C(r,h).
\end{equation}
This concludes the first step.

\emph{Step Two: Take the Derivatives}. We find
\begin{equation}
\nabla F(r,h) = \langle \partial_{r}A(r,h)
- \lambda\partial_{r}C(r,h), \partial_{h}A(r,h)-\lambda\partial_{h}C(r,h)\rangle
  = 0.
\end{equation}
Observe
\begin{equation}
\partial_{r}A(r,h)=4\pi r+2\pi h,\quad\mbox{and}\quad
\partial_{r}C(r,h)=2\pi rh.
\end{equation}
We also have
\begin{equation}
\partial_{h}A(r,h)=2r\pi,\quad\mbox{and}\quad
\partial_{h}C(r,h)=\pi r^{2}.
\end{equation}
The derivative with respect to the Lagrange multiplier gives us
\begin{equation}
\partial_{\lambda}F(r,h) = C(r,h) = 0.
\end{equation}
So we can set up our equations as
\begin{equation}
\begin{aligned}
(\partial_{r}F&=0)&\quad &\quad &2\pi r+2\pi h&=\lambda 2\pi rh\\
(\partial_{h}F&=0)&\quad &\quad &2\pi r &=\lambda\pi r^{2}\\
(\partial_{\lambda}F&=0)&\quad &\quad &\pi r^{2}h-16\pi&=0.
\end{aligned}
\end{equation}
That concludes our second step.

\emph{Step Three: Solve}. We see immediately from the
$\partial_{h}F$ equation that
\begin{equation}
2\pi r = \lambda \pi r^{2}\implies \lambda=\frac{2}{r}.
\end{equation}
We plug this into the $\partial_{r}F$ equation, we get
\begin{equation}
\begin{aligned}
2\pi r+2\pi h &= \lambda2\pi rh\\
&=\left(\frac{2}{r}\right)2\pi rh\\
&=4\pi h
\end{aligned}
\end{equation}
and subtracting $2\pi h$ from both sides yields
\begin{equation}
2\pi r=2\pi h\implies r=h.
\end{equation}
Now we use the constraint
\begin{equation}
\pi r^{2}h = 16\pi
\end{equation}
substituting $r=h$ we get
\begin{equation}
\pi h^{3} = 16\pi\implies h = \sqrt[3]{16}.
\end{equation}
Thus when we take $r=h=\sqrt[3]{16}$, we minimize the surface area.
\end{example}


\begin{exercise}
Find the extrema of $f(x,y,z)=xy+yz+zx$ subject to the constraint
$g(x,y,z)=x^{2}+y^{2}+z^{2}-1$.
\end{exercise}
\begin{exercise}
Find the extrema of $f(x,y)=\exp(-xy)$ subject to the constraint
$g(x,y)=x^{2}+y^{2}-1$.
\end{exercise}


