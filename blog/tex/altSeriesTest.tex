%%
%% altSeriesTest.tex
%% 
%% Made by Alex Nelson
%% Login   <alex@black-cherry>
%% 
%% Started on  Sat Jun 16 17:09:03 2012 Alex Nelson
%% Last update Sun Jun 17 14:45:03 2012 Alex Nelson
%%
%\documentclass{article}
%\usepackage{blog}
%\begin{document}
\N{Proposition} Let
\begin{equation}
\sum^{\infty}_{n=1}(-1)^{n+1}a_{n} =
a_{1}-a_{2}+a_{3}-a_{4}+\dots
\end{equation}
be a given series where $a_{n}>0$. Then the series converges if
\begin{enumerate}
\item $a_{n}\geq a_{n+1}$ for each $n$;
\item $\displaystyle\lim_{n\to\infty}a_{n}=0$.
\end{enumerate}
\begin{proof}
We have in the sequence of partial sums
\begin{equation}
S_{2n}=\underbrace{(a_{1}-a_{2})}_{\geq0}
+\underbrace{(a_{3}-a_{4})}_{\geq0}+\dots+\underbrace{(a_{2n-1}-a_{2n})}_{\geq0}\geq0
\end{equation}
so
\begin{equation}
0\leq S_{2}\leq S_{4}\leq \dots\leq S_{2n}\leq\dots
\end{equation}
But we also have
\begin{equation}
\begin{aligned}
S_{2n} &= a_{1} - (a_{2}-a_{3})-(a_{4}-a_{5})-(\dots)-a_{2n}\\
&\leq a_{1}.
\end{aligned}
\end{equation}
So
\begin{equation}
\lim_{n\to\infty}S_{2n}=L\leq a_{1}
\end{equation}
and
\begin{equation}
\begin{aligned}
\lim_{n\to\infty}S_{2n+1} &=\lim_{n\to\infty}(S_{2n}+a_{2n+1})\\
&=\left(\lim_{n\to\infty}S_{2n}\right)+\left(\lim_{n\to\infty}a_{2n+1}\right)\\
&=L+0=L.
\end{aligned}
\end{equation}
Conclusion: $\displaystyle\sum^{\infty}_{n=1}(-1)^{n+1}a_{n}=L$.
\end{proof}
\begin{example}
Consider the series
\begin{equation}
\sum^{\infty}_{n=1}\frac{(-1)^{n+1}}{n}
\end{equation}
We see that $1/n\geq1/(n+1)$ for each $n$, and 
\begin{equation}
\lim_{n\to\infty}\frac{1}{n}=0.
\end{equation}
Thus the Alternating Series Test implies the alternating Harmonic
series converges.
\end{example}
\begin{example}
Consider the series
\begin{equation}
\sum^{\infty}_{n=1}\frac{(-1)^{n+1}n}{(n+1)(n+2)}
\end{equation}
We see
\begin{equation}
\frac{n}{(n+1)(n+2)}\geq\frac{n+1}{(n+2)(n+3)};
\end{equation}
why? Well, multiply both sides by $(n+2)$ and we get
\begin{equation}
\frac{n}{n+1}\geq \frac{n+1}{n+3}
\end{equation}
Cross multiplication gives us
\begin{equation}
n(n+3)\geq (n+1)^{2} \iff n^{2}+3n \geq n^{2}+2n+1.
\end{equation}
Subtracting $n^{2}+2n$ from both sides gives us
\begin{equation}
n\geq1.
\end{equation}
So our series satisfies the first condition for the alternating
series test.

We also see that
\begin{equation}
\frac{n}{(n+1)(n+2)}\approx\frac{1}{n}
\end{equation}
for ``large $n$''. So we see 
\begin{equation}
\lim_{n\to\infty}\frac{n}{(n+1)(n+2)}=0.
\end{equation}
Thus our series satisfies the criteria for the alternating series
test, which implies convergence.
\end{example}

\N{Definitions} A series $\sum a_{n}$ is \textbf{``Absolutely Convergent''}
if $\sum|a_{n}|$ converges.

On the other hand, if $\sum|a_{n}|$ is divergent, then we call
the series $\sum a_{n}$ \textbf{``Conditionally Convergent''}.

\begin{example}
We see that $\sum(-1)^{n}/n$ is conditionally convergent, since
$\sum 1/n$ diverges.
\end{example}
\begin{example}
The series $\sum (-1)^{n}n^{-3/2}$ is absolutely convergent since
the integral test tells us $\sum n^{-3/2}$ converges.
\end{example}

\medbreak\noindent\emph{Question:} let $\sum a_{n}$ be a
convergent series, and $a_{n}\geq0$ for each $n$. Does the series
$\sum (-1)^{n}a_{n}$ converge? 

Stop and think before continuing!

\N{Absolute Convergence Test} 
If $\sum |a_{n}|$ conveges, then $\sum a_{n}$ converges.

\begin{proof}
We see first that 
\begin{equation}
-|a_{n}|\leq a_{n}\leq|a_{n}|\quad\mbox{for each }n
\end{equation}
So what? Well, we see that
\begin{equation}
0\leq a_{n}+|a_{n}|\leq2|a_{n}|\quad\mbox{for each }n
\end{equation}
Since $\sum|a_{n}|$ converges, we see $\sum2|a_{n}|$ converges
too. But by the comparison test, we see
\begin{equation}
\sum^{\infty}_{n=1}a_{n}+|a_{n}|
\end{equation}
converges. So what? We haven't proven $\sum a_{n}$ converges,
have we? Consider the following trick
\begin{equation}
\sum a_{n} = \underbrace{\sum
  a_{n}+|a_{n}|}_{\text{converges}}-\underbrace{\sum
  |a_{n}|}_{\text{converges}}.
\end{equation}
Therefore the series $\sum a_{n}$ converges.
\end{proof}
\begin{example}
Consider the series
\begin{equation}\label{eq:altSeriesTest:ex4:eq1}
\sum^{\infty}_{n=1}\frac{\cos(n)}{n^{3/2}}.
\end{equation}
What to do? We know
\begin{equation}
\left|\frac{\cos(n)}{n^{3/2}}\right|\leq\frac{1}{n^{3/2}}.
\end{equation}
So what? We know $\sum n^{-3/2}$ converges by the integral
test. Then
\begin{equation}
\sum^{\infty}_{n=1}\left|\frac{\cos(n)}{n^{3/2}}\right|
\end{equation}
converges by comparison. By the absolute convergence test, we know
our series in Equation \eqref{eq:altSeriesTest:ex4:eq1} converges.
\end{example}


%\end{document}


