%%
%% powerSeries.tex
%% 
%% Made by Alex Nelson
%% Login   <alex@black-cherry>
%% 
%% Started on  Sat Jun 16 17:37:15 2012 Alex Nelson
%% Last update Sun Jun 17 15:45:52 2012 Alex Nelson
%%
%\documentclass{article}
%\usepackage{blog}
%\begin{document}

\N{Definitions}
Let $x$ be a variable. a series of the form
\begin{equation}
\sum^{\infty}_{n=0}a_{n}x^{n}=a_{0}+a_{1}x+a_{2}x^{2}+\dots+a_{n}x^{n}+\dots
\end{equation}
is called a \textbf{``Power Series about $x=0$''}.

A series of the form
\begin{equation}
\sum^{\infty}_{n=0}b_{n}(x-a)^{n}=b_{0}+b_{1}(x-a)+b_{2}(x-a)^{2}+\dots+b_{n}(x-a)^{n}+\dots
\end{equation}
is called a \textbf{``Power Series about $x=a$''}.

\begin{example}
Consider the series
\begin{equation}
f(x)=\sum^{\infty}_{n=0}\frac{n+1}{2^{n}}x^{n}.
\end{equation}
For what values of $x$ does this series converge? When will the
series diverge?\more{}

\emph{Solution.} Using the absolute ratio test, we see
\begin{equation}
\begin{aligned}
\lim_{n\to\infty}\left|\frac{\displaystyle\frac{n+2}{2^{n+1}}x^{n+1}}{\displaystyle\frac{n+1}{2^{n}}x^{n}}\right|
&=\lim_{n\to\infty}\frac{n+2}{n+1}\cdot\frac{2^{n}}{2^{n+1}}\cdot|x|\\
&=(1)\cdot\left(\frac{1}{2}\right)\cdot|x|\\
&=\frac{1}{2}|x|
\end{aligned}
\end{equation}
For convergence, we need
\begin{equation}
\frac{1}{2}|x|<1\quad\Longrightarrow\quad|x|<2.
\end{equation}
So divergence would be when
\begin{equation}
|x|>2.
\end{equation}
We need to check the $|x|=2$ case. Observe, for $x=2$, we have
\begin{equation}
f(2)=\sum^{\infty}_{n=0}\frac{n+1}{2^{n}}2^{n}=\sum^{\infty}_{n=0}n+1
\end{equation}
which diverges. And at $x=-2$ we have
\begin{equation}
f(-2)=\sum^{\infty}_{n=0}\frac{n+1}{2^{n}}(-2)^{n}=\sum^{\infty}_{n=0}(-1)^{n}(n+1)
\end{equation}
which still diverges!
\end{example}

\N{Theorem}
Let $c\not=0$ and 
\begin{equation}
f(x)=\sum^{\infty}_{n=0}a_{n}x^{n}
\end{equation}
converge at $x=c$. Then $f(x)$ converges absolutely for
$|x|<|c|$.

\begin{proof}
Let $x$ be any value such that $|x|<|c|$. Since $f(c)$ converges,
there exists an $M$ such that
\begin{equation}
|a_{n}c^{n}|\leq M\quad\mbox{for any }n.
\end{equation}
Well, we see
\begin{equation}
|x/c|\leq 1
\end{equation}
so
\begin{equation}
a_{n}x^{n}=a_{n}c^{n}\left(\frac{x}{c}\right)^{n}
\end{equation}
and moreover
\begin{equation}
|a_{n}x^{n}| = |a_{n}c^{n}|\cdot\left|\frac{x}{c}\right|^{n}\leq M\left|\frac{x}{c}\right|^{n}
\end{equation}
But observe the series
\begin{equation}
g(x)=\sum^{\infty}_{n=0}M\cdot\left|\frac{x}{c}\right|^{n}
\end{equation}
is a geometric series which converges since $|x/c|\leq1$.

Therefre the series $\sum|a_{n}x^{n}|$ converges by comparison,
and the absolute convergence test tells us $\sum a_{n}x^{n}$
converges. Therefore $\sum a_{n}x^{n}$ is absolutely convergent.
\end{proof}


\begin{example}
Find the values of $x$ for which
\begin{equation}
\sum^{\infty}_{n=1}\frac{(x+7)^{n}}{\sqrt{n}}
\end{equation}
converge.

\emph{Solution}: Using the absolute ratio test, we find
\begin{equation}
\begin{aligned}
\lim_{n\to\infty}\left|\frac{\left(\displaystyle\frac{(x+7)^{n+1}}{\sqrt{n+1}}\right)}{\left(\displaystyle\frac{(x+7)^{n}}{\sqrt{n}}\right)}\right|
&=\lim_{n\to\infty}\frac{\sqrt{n}}{\sqrt{n+1}}|x+7|\\
&=|x+7|.
\end{aligned}
\end{equation}
We get convergence for $|x+7|<1$. So
\begin{equation}
-1<x+7<1\quad\Longrightarrow\quad-8<x<-6.
\end{equation}
We have to check the boundary cases.

When $x=-6$, we have
\begin{equation}
\sum^{\infty}\frac{(7-6)^{n}}{\sqrt{n}}=\sum^{\infty}_{n=1}\frac{1}{\sqrt{n}}
\end{equation}
which diverges by the integral test (or the comparison test with
the Harmonic series). For $x=-8$ we have our series become
\begin{equation}
\sum^{\infty}_{n=1}\frac{(-1)^{n}}{\sqrt{n}}
\end{equation}
which converges by the alternating series test.
\end{example}


%\end{document}
