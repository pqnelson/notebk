%%
%% curves.tex
%% 
%% Made by Alex Nelson
%% Login   <alex@black-cherry>
%% 
%% Started on  Fri Jun 29 11:50:28 2012 Alex Nelson
%% Last update Sat Jun 30 20:32:51 2012 Alex Nelson
%%
\N{Curves}
We are interested in describing the motion of my car. Well,
\emph{everyone} is interested in the motion of my car. How can we
describe it mathematically? 

First we approximate the car as a point. The point-like car moves
in time, so the value of its components are functions of
time. More precisely, the position of my car is 
\begin{equation}
\vec{r}(t) = \langle f(t),g(t),h(t)\rangle =
f(t)\widehat{\textbf{\i}} +
g(t)\widehat{\textbf{\j}} +
h(t)\widehat{\textbf{k}}
\end{equation}
where the functions $f(t)$, $g(t)$, and $h(t)$ are sometimes
called \emph{component functions}. Another way to think about
this is writing
\begin{equation}
\vec{r}\colon[0,1]\to\RR^{3}
\end{equation}
where $0\leq t\leq1$. 

Classical mechanics studies such curves under various
circumstances. We will discuss some notions of kinematics, and
study what it means to differentiate curves.

\begin{example}
Consider a point traveling in circular motion in the
$xy$-plane. What does this look like? 

Well, it's a paramteric curve, using trigonometric functions we write
\begin{equation}
\vec{r}(t) = \cos(t)\widehat{\textbf{\i}}+\sin(t)\widehat{\textbf{\j}}.
\end{equation}
This descrivbes an anti-clockwise circular motion with radius 1,
lying in the $xy$-plane.
\end{example}

\begin{example}
Suppose a particle travels along a parabolic curve, what does the
curve look like? We can write it explicitly as
\begin{equation}
\vec{r}(t) =
t\,\widehat{\textbf{\i}}+(t^{2}-1)\widehat{\textbf{\j}}
\end{equation}
This is precisely aa parabola.
\end{example}

\N{Calculus with Vector-Valued Functions}
We should recall the construction of the tangent line to a curve
$y=f(x)$ at a point $(x_{0},f(x_{0})=y_{0})$ had us write
\begin{equation}
t(h) = y_{0} + f'(x_{0}) \cdot h.
\end{equation}
When we consider the situation when we work with $\vec{r}(t)$
instead of a function $f(x)$. We have
$\vec{r}_{0}=\vec{r}(t_{0})$ be the base point for the tangent to
the curve, then we have
\begin{equation}
\vec{T}(h) = \vec{r}_{0} + \vec{r}'(t_{0})\cdot h
\end{equation}
The problem: what exactly is $\vec{r}'(t_{0})$?

\M
We can let $\vec{r}\colon(0,1)\to\RR^{3}$ (or more generally the
codomain can be $\RR^{n}$ for any positive integer $n\in\NN$). We
have
\begin{equation}
\frac{\D\vec{r}(t)}{\D t}=\vec{v}(t) = \lim_{\Delta t\to0}\frac{\vec{r}(t+\Delta t)-\vec{r}(t)}{\Delta t}
\end{equation}
describe the rate of change of the position vector $\vec{r}(t)$
with respect to time. What does this look like? Well, writiing out
\begin{equation}
\vec{r}(t) = \langle f(t),g(t),h(t)\rangle =
f(t)\widehat{\textbf{\i}} +
g(t)\widehat{\textbf{\j}} +
h(t)\widehat{\textbf{k}}
\end{equation}
we have
\begin{equation}
\frac{\D\vec{r}(t)}{\D t} = \langle f'(t), g'(t), h'(t)\rangle =
f'(t)\widehat{\textbf{\i}} +
g'(t)\widehat{\textbf{\j}} +
h'(t)\widehat{\textbf{k}}
\end{equation}
where primes denote differentiation with respect to time.

\begin{remark}
We can keep iterating this procedure to obtain higher order
derivatives of a curve.
\end{remark}

\N{Kinematics}
We have $\vec{r}(t)$ describe the position of a particle. The
velocity of the particle is a vector-valued function
\begin{equation}
\begin{aligned}
\vec{v}(t)
&=\lim_{\Delta t\to 0}\frac{\vec{r}(t+\Delta
  t)-\vec{r}(t)}{\Delta t}\\
&=\frac{\D\vec{r}(t)}{\D t}
\end{aligned}
\end{equation}
However, we also can consider the \emph{speed} or the magnitude
of the velocity
\begin{equation}
\|\vec{v}(t)\|=\frac{\D s}{\D t} = \begin{pmatrix}
\mbox{rate of change of distance}\\
\mbox{with respect to time}
\end{pmatrix}
\end{equation}
Observe the speed is a scalar quantity: it's just some function
of time. The velocity is a vector-valued function of time. 

We have one last kinematical quantity to consider: the
acceleration. This is just the rate of change of velocity with
respect to time:
\begin{equation}
\vec{a}(t) = \frac{\D\vec{v}(t)}{\D t} =
\frac{\D^{2}\vec{r}(t)}{\D t^{2}}
\end{equation}
Observe we can reconstruct the position from the velocity by
considering
\begin{equation}
\vec{r}(t) = \vec{r}(t_{0}) +
\int^{t}_{t_{0}}\frac{\D\vec{r}(\tau)}{\D\tau}\,\D\tau
\end{equation}
which when we consider $\vec{r}(t)=\langle x(t),y(t),z(t)\rangle$
we have the integral evaluated ``component-wise'':
\begin{equation}
\vec{r}(t) = \vec{r}(t_{0}) +
\left\langle \int^{t}_{t_{0}}\frac{\D x(\tau)}{\D\tau}\,\D\tau,
\int^{t}_{t_{0}}\frac{\D y(\tau)}{\D\tau}\,\D\tau,
\int^{t}_{t_{0}}\frac{\D z(\tau)}{\D\tau}\,\D\tau\right\rangle.
\end{equation}
We can similarly reconstruct velocity from acceleration.

\begin{example}
Consider the curve describing circular motion
\begin{equation}
\vec{r}(t) = \cos(t)\widehat{\textbf{\i}}
+\sin(t)\widehat{\textbf{\j}}
\end{equation}
What is its velocity vector, acceleration vector, and speed?

\emph{Solution}: We find its velocity
\begin{equation}
\begin{aligned}
\vec{v}(t) &= \frac{\D\vec{r}(t)}{\D t}\\
&=-\sin(t)(t)\widehat{\textbf{\i}}
+\cos(t)\widehat{\textbf{\j}}
\end{aligned}
\end{equation}
From this we can compute its speed as
\begin{equation}
\begin{aligned}
\|\vec{v}(t)\| &= \sqrt{\vec{v}(t)\cdot\vec{v}(t)}\\
&=\sqrt{\sin^{2}(t)+\cos^{2}(t)} = 1.
\end{aligned}
\end{equation}
The acceleration is precisely the derivative of the velocity
vector
\begin{equation}
\frac{\D\vec{v}(t)}{\D t}
= -\cos(t)\widehat{\textbf{\i}}-\sin(t)\widehat{\textbf{\j}}
\end{equation}
That concludes our example.
\end{example}

\begin{exercise}
Find the velocity vector, speed, and acceleration of the
parabolic curve $\vec{r}(t) =
t\,\widehat{\textbf{\i}}+(t^{2}-1)\widehat{\textbf{\j}}$
\end{exercise}

\begin{exercise}
Let $\vec{u}(t)$ and $\vec{v}(t)$ be differentiable vector-valued
functions of time. Prove or find a counter-example that
\begin{equation}
\frac{\D}{\D t}\bigl[\vec{u}(t)\times\vec{v}(t)\bigr]=
\frac{\D\vec{u}(t)}{\D t}\times\vec{v}(t) +
\vec{u}(t)\times\frac{\D\vec{v}(t)}{\D t}.
\end{equation}
\end{exercise}
\begin{exercise}
Calculate $\displaystyle\frac{\D}{\D t}\bigl[\vec{a}(t)\cdot\bigl(\vec{b}(t)\times\vec{c}(t)\bigr)\bigr]$
\end{exercise}
