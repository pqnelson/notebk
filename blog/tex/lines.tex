%%
%% lines.tex
%% 
%% Made by Alex Nelson
%% Login   <alex@black-cherry>
%% 
%% Started on  Thu Jun 21 17:27:08 2012 Alex Nelson
%% Last update Thu Jun 21 17:28:45 2012 Alex Nelson
%%
\N{Constructing Lines}
Suppose we have two points
\begin{equation}
A = (4,2,-1)\quad\mbox{and}\quad
B = (3,5,7).
\end{equation}
We want to find a line $\ell$ passing through these points. What to do?

First we form the vector
\begin{equation}
\begin{aligned}
\vec{v}=\overrightarrow{AB}
&= (3-4)\widehat{\textbf{\i}}+(5-2)\widehat{\textbf{\j}}+(7+1)\widehat{\textbf{k}}\\
&=-\widehat{\textbf{\i}}+3\widehat{\textbf{\j}}+8\widehat{\textbf{k}}
\end{aligned}
\end{equation}
This vector is parallel to $\ell$; the numbers given by this
vector's components (i.e., -1, 3, 8) are called
the \textbf{``Direction Numbers''} of $\ell$.  

In general, we have two distinct points $P_{0}=(x_0,y_0,z_0)$ and
$P=(x,y,z)$ on the line $\ell$, then we construct the vector
\begin{equation}
\overrightarrow{P_{0}P} = (x-x_{0})\widehat{\textbf{\i}}+
(y-y_{0})\widehat{\textbf{\j}}+(z-z_{0})\widehat{\textbf{k}}
\end{equation}
and this is equal to some scalar multiple of $\vec{v}$ (i.e.,
it's a dilation of the vector).  We write
\begin{equation}
(x-x_{0})\widehat{\textbf{\i}}+
(y-y_{0})\widehat{\textbf{\j}}+(z-z_{0})\widehat{\textbf{k}}
=t\vec{v}
\end{equation}
which lets us write
\begin{equation}
\left.\begin{array}{rl}
x-x_{0} &=tv_{1}\\
y-y_{0} &=tv_{2}\\
z-z_{0} &=tv_{3}
\end{array}
\right\}\implies
\left\{\begin{array}{rl}
x &=x_{0}+tv_{1}\\
y &=y_{0}+tv_{2}\\
z &=z_{0}+tv_{3}
\end{array}\right.
\end{equation}
This is the parametric equations of $\ell$. So returning to our
example, we have
\begin{equation}
\begin{aligned}
x &= 4 - t\\
y &= 2 + 3t\\
z &= -1 + 8t
\end{aligned}
\end{equation}
where the constant terms are precisely the values of the
components of $A$, and the coefficients of $t$ are the components
of the vector $\overrightarrow{AB}$.

\N{Distance From a Point to a Line}
What's the distance from any point $S$ in 3-space to a given line
$\ell$? 

We pick a point $P$ on $\ell$ and form a vector
$\overrightarrow{PS}$. The distance from $S$ to $\ell$ can be
given as $d$, as in the following diagram:
\begin{center}
\includegraphics{img/moreVectors.2}
\end{center}
We see $d=\|\overrightarrow{PS}\|\sin(\theta)$. If $\vec{v}$ is a
vector parallel to $\ell$, then we have
\begin{equation}
\|\overrightarrow{PS}\times\widehat{v}\|=\|\overrightarrow{PS}\|\sin(\theta)
\end{equation}
This is all abstract, lets consider an example.

\begin{example}
Find the distance from $S=(2,1,3)$ to the line given by
\begin{equation}
\begin{aligned}
x &= -1+t\\
y &= 2+t\\
z &= 1+2t
\end{aligned}
\end{equation}
First we pick the point when $t=0$, we call it
\begin{equation}
P = (-1,2,1).
\end{equation}
Observe
\begin{equation}
\overrightarrow{PS} =
3\widehat{\textbf{\i}}-\widehat{\textbf{\j}}+2\widehat{\textbf{k}}. 
\end{equation}
Now we need to find a vector parallel to the line. What to do?
Construct a vector by considering the point when $t=1$, which
would be $P_{1}=(0,3,3)$. Thus
\begin{equation}
\begin{aligned}
\vec{v}
&=\overrightarrow{PP_{1}}\\
&=(-1-0)\widehat{\textbf{\i}}+(2-3)\widehat{\textbf{\j}}+(1-3)\widehat{\textbf{k}}\\
&=-\widehat{\textbf{\i}}-\widehat{\textbf{\j}}-2\widehat{\textbf{k}}
\end{aligned}
\end{equation}
Its unit vector is
\begin{equation}
\begin{aligned}
\widehat{v} &= \vec{v}/\|\vec{v}\|\\
&= \vec{v}/\sqrt{1+1+4}\\ 
&= \vec{v}/\sqrt{6}
\end{aligned}
\end{equation}
We have
\begin{equation}
\begin{aligned}
\overrightarrow{PS}\times\widehat{v}
&= \frac{1}{\sqrt{6}}\begin{vmatrix}
\widehat{\textbf{\i}} & \widehat{\textbf{\j}} & \widehat{\textbf{k}}\\
3 & -1 & 2\\
-1 & -1 & -2
\end{vmatrix}\\
&= \frac{4\widehat{\textbf{\i}}+4\widehat{\textbf{\j}}-4\widehat{\textbf{k}}}{\sqrt{6}}
\end{aligned}
\end{equation}
This has its magnitude be
\begin{equation}
\|\overrightarrow{PS}\times\widehat{v}\|
= \frac{4\sqrt{3}}{\sqrt{6}}=2\sqrt{2},
\end{equation}
which describes the distance between our point $S$ and the given line.
\end{example}

