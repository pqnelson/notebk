%%
%% surfaces.tex
%% 
%% Made by Alex Nelson
%% Login   <alex@black-cherry>
%% 
%% Started on  Fri Jun 22 13:32:11 2012 Alex Nelson
%% Last update Thu Jun 28 12:18:49 2012 Alex Nelson
%%
\M
Let $w=f(x_{1},x_{2},\dots,x_{n})$ where $x_{1}$, $x_{2}$, \dots,
$x_{n}$ are all independent variables. Then the
\textbf{``Domain''} of $f$ is the set of $n$-tuples
$\RR^{n}$. Note that an ordered pair is 
\begin{equation}
(x,y)=\mbox{2-tuple}.
\end{equation}
The set of corresponding values is the \textbf{``Range''} (or
\emph{Codomain}) of the function. So we have
\begin{equation}
%% \begin{aligned}
%% f\colon &\RR^{n}&\to&\RR\\
%% &\mbox{(Domain)}&\to&\mbox{(Range)}
%% \end{aligned}
f\colon\underbrace{\RR^{n}}_{\text{domain}}\to\underbrace{\RR}_{\text{range}}
\end{equation}
Sometimes we write $\mathrm{dom}(f)$ for the domain of $f$, and
$\mathrm{ran}(f)$ or $\mathrm{cod}(f)$ for the range (or
codomain) of $f$. We \emph{DO NOT} write $f(\RR^{2})$ for the
range, because this is the collection of all points mapped by
$f$. 

\begin{example}
Consider $f\colon\RR^{2}\to\RR$ defined by
\begin{equation}
f(x,y) = \sqrt{x^{2}+y^{2}}
\end{equation}
Notice that $f(x,y)\geq0$ for any $x,y\in\RR$. So
\begin{equation}
f(\RR^{2})=\{ u\in\RR : u\geq0\}\not=\RR.
\end{equation}
This is not the codomain! It's contained in the codomain,
though. The image is \emph{always} a subset of the codomain.
\end{example}
\more

\begin{example}
Let $g(x,y)=\ln(x^{2}-y)$. For $g$ to be defined, we need
\begin{equation}
x^{2}-y>0\quad\mbox{or}\quad x^{2}>y.
\end{equation}
The boundary of the domain is $x^{2}=y$, which we can doodle:
\begin{center}
  \includegraphics{img/surfaces.0}
\end{center}
Since the boundary is not in the domain, then the domain of $g$
is open.
\end{example}

\begin{example}
Consider 
\begin{equation}
h(x,y)=\sqrt{25-x^{2}-y^{2}}.
\end{equation}
The domain is the set of $(x,y)$ such that
\begin{equation}
x^{2}+y^{2}\leq25.
\end{equation}
We can doodle this:
\begin{center}
\includegraphics{img/surfaces.1}
\end{center}
Observe that the boundary is the circle; the disc is the circle
and everything enclosed in it.
\end{example}

\M
So we have just discussed domains and codomains, but we have not
discussed the graph of the function $z=f(x,y)$. What would this
look like? Well, when we plot $y=g(x)$, it's on the plane
$\RR^{2}$. So plotting $z=f(x,y)$ would be on the 3-space
$\RR^{3}$. Lets start considering examples of what this looks
like. 

\begin{example}
Consider the graph given by
\begin{equation}
z = f(x,y) = 9 - x^{2}-y^{2}.
\end{equation}
How can we draw this? Well, the first trick is to draw when $x=0$
and $y=0$:
\begin{equation}
\begin{aligned}
z &= f(0,y) = 9-y^{2}\\
z &= f(x,0) = 9-x^{2}
\end{aligned}
\end{equation}
These are parabolas in the $yz$ and $xz$ planes,
respectively. Now we can start drawing \textbf{``Level Curves''},
i.e., curves where we fix $z$ to be some constant. For example,
when $z=0$, we have a circle
\begin{equation}
x^{2}+y^{2}=9
\end{equation}
Observe then that $z$ controls the radius of the circles: for
nonzero $z$, we have
\begin{equation}
x^{2}+y^{2}=9-z.
\end{equation}
The left hand side must be non-negative, and can be zero only
when $z=9$. So we get a surface that looks like
\begin{center}
 \includegraphics{img/paraboloid.0}
\end{center}
We call this surface a \textbf{``Paraboloid''} as we have
parabolas along the $x$- and $y$-axes.

This is the general scheme for picturing a surface: draw level
curves $f(x,y)=c$ for some constant $c$, which produces the level
curve for $f$ corresponding to $c$.
\end{example}

\begin{example}
Consider $x^{2}+y^{2}-z^{2}=0$. What does this surface look like?

\emph{Solution}: First we observe the curves along the $x$-axis,
i.e., when $y=0$ is $x^{2}=z^{2}$. Similarly when $x=0$ we have
$y^{2}=z^{2}$. What sort of curves are these? Well, $(\pm t,0,t)$
and $(0,\pm t,t)$ are the curves, for $t\in\RR$. 

Also note that $x^{2}+y^{2}$ describes a circle, whose radius
happens to be $z^{2}$. This tells us the level curves are simply
circles. So we have a cone: 
\begin{center}
  \includegraphics{img/surfaces.2}  
\end{center}
This surface is precisely a \textbf{``Cone''}.
\end{example}
\begin{example}
What if we deform the previous example, writing
\begin{equation}
x^{2}+y^{2}-z^{2}=1.
\end{equation}
What surface does this describe?

\emph{Solution}: Well, we see that the radius of the level curves
deform $z^{2}\to z^{2}+1$. So the resulting surface looks like a
cone, but along the $x$-axis we don't have a straight-line: we
have a hyperbola $x^{2}-z^{2}=1$. Similarly along the $y$-axis we
have another hyperbola $y^{2}-z^{2}=1$. Thus our surface is
doodled as:
\begin{center}
\includegraphics{img/surfaces.3}
\end{center}
This ``deformed cone'' is called a \textbf{``One Sheeted Hyperboloid''}.
\end{example}
\begin{example}
Another variation, consider the surface
\begin{equation}
x^{2}+y^{2}-z^{2}=-1
\end{equation}
What does it look like?

\emph{Solution}: Well, we see that $x^{2}+y^{2}\geq0$ always,
whereas $z^{2}-1\geq0$ only when $|z|\geq1$. So we have two
``surfaces'' or \textbf{``Sheets''} here. The level curves are
again circles, and this enables us to doodle the graph:
\begin{center}
\includegraphics{img/surfaces.4}
\end{center}
Along the $x$-axis and $y$-axis, we have hyperbolas. For this
reason, we call the surface a \textbf{``Two-Sheeted Hyperbaloid''}.
\end{example}

\begin{example}
Suppose we have a surface given by $z=x^{2}-y^{2}$. What does it
look like?

\emph{Solution}: Observe the level curves for $z>0$ gives us
hyperbolas, and for $z<0$ we again have hyperbolas (the same as
before but reflected about the line $x=y$). For $z=0$ we have a
cone. 

When we take $y=\pm1,0$ we see $z=x^{2}-C$ for some constant
$C$. These curves look like smiles. For $x=\pm1,0$, we have
$z=C-y^{2}$ for some constant $C$. These curves look like
frowns. 

This surface is a saddle, and we can doodle it as
\begin{center}
\includegraphics{img/surfaces.5}
\end{center}
Since the curves when we hold $z$ constant form hyperbolas, and
the curves when we hold $x$ or $y$ constaant are parabolas,
people sometimes call this saddle a \textbf{``Hyperbolic Paraboloid''}.
\end{example}
