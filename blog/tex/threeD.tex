%%
%% threeD.tex
%% 
%% Made by Alex Nelson
%% Login   <alex@black-cherry>
%% 
%% Started on  Wed Jun 20 13:38:30 2012 Alex Nelson
%% Last update Thu Jun 21 17:40:32 2012 Alex Nelson
%%

\N{The 3 Dimensional Coordinate System}
The $x$, $y$, $z$ axes are perpendicular to each other. We will
doodle three dimensions as follows:
\begin{center}
\includegraphics{img/threeD.0}
\end{center}
The formula for distance between two points $(x_{0}, y_{0},
z_{0})$ and $(x_{1},y_{1},z_{1})$ would be
\begin{equation}
s = \sqrt{(x_{1}-x_{0})^{2}+(y_{1}-y_{0})^{2}+(z_{1}-z_{0})^{2}}.
\end{equation}
A sphere with its center at $(a,b,c)$ would be the points which
are a distance $r$ away from the center. So we would have
\begin{equation}
S^{2} = \{\,(x,y,z) : \sqrt{(x-a)^{2}+(y-b)^{2}+(z-c)^{2}}=r\,\}
\end{equation}
Usually the formula is given as
\begin{equation}
(x-a)^{2}+(y-b)^{2}+(z-c)^{2}=r^{2}
\end{equation}
describes a sphere.

\N{Vectors}
A \textbf{``Vector''} is a directed line segment.

We know a directed line segment has both length (magnitude) and
direction, so any two directed line segments with the same length
\emph{and} direction represent the same vector.

Vectors are ``transportable'' in the sense that we may translate
their base point. We will represent the length of a vector
$\vec{u}$ as $\|\vec{u}\|$ or $|\vec{u}|$.

The notation for a vector would be $\langle x,y\rangle$ (in two
dimensions) or $\langle x,y,z\rangle$ (in three dimensions). The
vector from $(0,0,0)$ to $(x,y,z)$ is given as $\langle x,y,z\rangle$.
For $P=(-1,4,7)$ and $Q=(2,5,3)$, then the vector from $P$ to $Q$
is denoted $\overrightarrow{PQ}$.

We can add vectors graphically:
\begin{center}
\includegraphics{img/threeD.1}
\end{center}
Subtraction would amount to $\vec{A}-\vec{B}=\vec{A}+(-\vec{B})$,
and graphically this is:
\begin{center}
\includegraphics{img/threeD.2}
\end{center}
Algebraically, if
\begin{equation}
\vec{A}=\langle x_{a},y_{a},z_{a}\rangle,\quad
\vec{B}=\langle x_{b},y_{b},z_{b}\rangle
\end{equation}
then
\begin{equation}
\begin{aligned}
\vec{A}+\vec{B} &= \langle x_{a}+x_{b}, y_{a}+y_{b},
z_{a}+z_{b}\rangle\\
\vec{A}-\vec{B} &= \langle x_{a}-x_{b}, y_{a}-y_{b},
z_{a}-z_{b}\rangle
\end{aligned}
\end{equation}
This describes vector addition and subtraction on the components.

Note in two-dimensional space, the vectors
\begin{equation}
\widehat{\textbf{\i}}=\langle 1,0\rangle
\quad\mbox{and}\quad
\widehat{\textbf{\j}}=\langle 0,1\rangle
\end{equation}
are unit vectors (i.e., vectors whose length is $1$). They are
also called \emph{basis vectors} since any other vector $\vec{v}$
in two-dimensions can be written as
\begin{equation}
\begin{aligned}
\vec{v}&=\langle v_{x}, v_{y}\rangle\\
&=\langle v_{x},0\rangle + \langle0,v_{y}\rangle\\
&=v_{x}\langle1,0\rangle + v_{y}\langle0,1\rangle\\
&=v_{x}\widehat{\textbf{\i}}+v_{y}\widehat{\textbf{\j}}
\end{aligned}
\end{equation}
where $\langle v_{x}$ and $v_{y}$ are called the vector's 
\emph{components}. Note that the components of the vector depends on a
choice of coordinates (i.e., a choice of basis vectors).

The last notion we will discuss: given any vector $\vec{v}$ which
is nonzero, then we can construct the unit vector
\begin{equation}
\widehat{v} = \frac{\vec{v}}{\|\vec{v}\|}
\end{equation}
which has magnitude 1. We use hats to indicate unit vectors, and
arrows for arbitrary vectors.

\N{Caution:} Everything stated about vectors is a
half-truth. Really, these are ``tangent vectors'' which has a
base point and a vector part (i.e., where we stick the line
segment, and the directed line segment itself). We can only
add/subtract two tangent vectors if they have the same base point. 
But since we work in Euclidean space (which is flat), we can
transport vectors without a problem. This is a very special
situation! 

Since this is never mentioned, often students become confused
when they finish vector calculus and begin studying linear
algebra. Linear algebra fixes a base point, and considers the
collection of all vectors sharing the same base point. This is
the honest definition of a vector. 

\begin{remark}
We will also use the phrase ``three-space'' instead of
``three-dimensional space'', and ``two-space'' replacing
``two-dimensional space''. In general $n$-space is
$n$-dimensional Euclidean space.
\end{remark}

\subsection{Dot Products}
\N{Definition}
Given two vectors
\begin{equation}
\vec{u}=u_{1}\widehat{\textbf{\i}}+u_{2}\widehat{\textbf{\j}}+u_{3}\widehat{\textbf{k}},\quad\mbox{and}\quad
\vec{v}=v_{1}\widehat{\textbf{\i}}+v_{2}\widehat{\textbf{\j}}+v_{3}\widehat{\textbf{k}}
\end{equation}
their \textbf{``Dot Product''} is the number
\begin{equation}
\begin{aligned}
\vec{u}\cdot\vec{v} &= \sum_{i} u_{i}v_{i}\\
&= u_{1}v_{1}+u_{2}v_{2}+u_{3}v_{3}
\end{aligned}
\end{equation}
Let $\vec{u}$ and $\vec{v}$ be given vectors in three-space. 

\N{Angles}
How do we find the angle $\theta$ between the two vectors?
\begin{center}
\includegraphics{img/threeD.3}
\end{center}
We find that
\begin{equation}
\theta = \arccos\left(\frac{\|\vec{v}\|}{\|\vec{u}\|}\right)
\end{equation}
How is this? Well, we should recall the law of cosines
\begin{equation}
\|\vec{w}\|^{2}=\|\vec{u}\|^{2}+\|\vec{v}\|^{2}-2\|\vec{u}\|\cdot\|\vec{v}\|\cos(\theta)
\end{equation}
which can be written as
\begin{equation}
\begin{aligned}
\vec{w}\cdot\vec{w} &=
(u_{1}-v_{1})^{2}+(u_{2}-v_{2})^{2}+(u_{3}-v_{3})^{2}\\
&= \|\vec{u}\|^{2}-2(\vec{u}\cdot\vec{v})+\|\vec{v}\|^{2}
\end{aligned}
\end{equation}
Setting equals to equals gives us
\begin{equation}
-2(\vec{u}\cdot\vec{v})=-2\|\vec{u}\|\cdot\|\vec{v}\|\cos(\theta)
\end{equation}
and thus
\begin{equation}
\frac{\vec{u}\cdot\vec{v}}{\|\vec{u}\|\cdot\|\vec{v}\|}=\cos(\theta)
\end{equation}
Taking the arc cosine of both sides yields
\begin{equation}
\arccos\left(\frac{\vec{u}\cdot\vec{v}}{\|\vec{u}\|\cdot\|\vec{v}\|}\right)=\theta.
\end{equation}
A useful formula worth remembering 
\begin{equation}
(\vec{u}\cdot\vec{v})=\|\vec{u}\|\cdot\|\vec{v}\|\cos(\theta)
\end{equation}
\begin{example}
Let $\vec{u}=\langle3,-1,4\rangle$ and
$\vec{v}=\langle1,5,-2\rangle$. What's the angle between them?

\emph{Solution}: We first find
\begin{equation}
\begin{aligned}
\vec{u}\cdot\vec{v} &= (3\cdot1)+(-1\cdot5)+(4\cdot-2)\\
&=3-5-8=-10.
\end{aligned}
\end{equation}
We then compute
\begin{equation}
\|\vec{u}\|=\sqrt{9+1+16}=\sqrt{26}
\end{equation}
and
\begin{equation}
\|\vec{v}\|=\sqrt{1+25+4}=\sqrt{30}.
\end{equation}
Thus the angle between $\vec{u}$ and $\vec{v}$ is
\begin{equation}
\theta=\arccos\left(\frac{-10}{\sqrt{26}\sqrt{30}}\right)\approx1.937
\end{equation}
(radians).
\end{example}

\subsection{Orthogonality}

\M
Vectors are perpendicular or \textbf{``Orthogonal''} if
$\vec{u}\cdot\vec{v}=0$. Sometimes this is denoted $\vec{u}\bot\vec{v}$.

\begin{example}
Consider
\begin{equation}
\vec{u}=\langle6,-3,8\rangle,\quad\mbox{and}\quad
\vec{v}=\langle-2,4,3\rangle.
\end{equation}
We see
\begin{equation}
\vec{u}\cdot\vec{v}=-12-12+24=0
\end{equation}
which implies $\vec{u}$ and $\vec{v}$ are orthogonal.
\end{example}

\M
Consider the following diagram
\begin{center}
\includegraphics{img/threeD.4}
\end{center}
We're given vectors $\vec{u}=\overrightarrow{PQ}$ and
$\vec{v}=\overrightarrow{PS}$ in 3-space. Notice that if the
angle between the vectors $\theta$ is acute, as doodled, then
$\overrightarrow{PR}$ is the projection of $\vec{u}$ onto
$\vec{v}$. 

However, if $\theta$ is obtuse, we doodle the situation thus:
\begin{center}
\includegraphics{img/threeD.5}
\end{center}
Observe the projection of $\vec{u}$ onto $\vec{v}$ will not fall
on $\vec{v}$. The projection of $\vec{u}$ onto $\vec{v}$ is
syntactically 
\begin{equation}
\proj_{\vec{v}}\vec{u}
\end{equation}
The natural question: \emph{what is the formula for projecting
$\vec{u}$ onto $\vec{v}$?} We have
\begin{equation}
\|\overrightarrow{PR}\|=\begin{cases}\|\vec{u}\|\cos(\theta)
&\mbox{for $\theta$ acute}\\
-\|\vec{u}\|\cos(\theta)&\mbox{for $\theta$ obtuse}
\end{cases}
\end{equation}
The direction of $\overrightarrow{PR}$ depends on whether
$\theta$ is acute or obtuse; we have its unit vector be
\begin{equation}
\widehat{PR}=\begin{cases}\widehat{v}&\mbox{for $\theta$ acute}\\
-\widehat{v}&\mbox{for $\theta$ obtuse}
\end{cases}
\end{equation}
But now look, for both obtuse and acute $\theta$ we have
\begin{equation}
\begin{aligned}
\overrightarrow{PR}
&=\|\vec{u}\|\cos(\theta)\frac{\vec{v}}{\|\vec{v}\|}\\
&=\frac{\|\vec{u}\|\|\vec{v}\|\cos(\theta)}{\|\vec{v}\|}\frac{\vec{v}}{\|\vec{v}\|}\\
&=\frac{\vec{u}\cdot\vec{v}}{\|\vec{v}\|}\frac{\vec{v}}{\|\vec{v}\|}
\end{aligned}
\end{equation}
This describes the projection of $\vec{u}$ onto $\vec{v}$
for \emph{any} $\theta$:
\begin{equation}
\proj_{\vec{v}}\vec{u}=\frac{\vec{u}\cdot\vec{v}}{\|\vec{v}\|}\frac{\vec{v}}{\|\vec{v}\|}
\end{equation}
Notice that its magnitude is $\vec{u}\cdot\vec{v}/\|\vec{v}\|$. 

\M
Consider the same situation again. We have a vector
$\vec{w}=\overrightarrow{RQ}$ as doodled
\begin{center}
\includegraphics{img/threeD.6}
\end{center}
This vector $\vec{w}$ is orthogonal to the projection of
$\vec{u}$ onto $\vec{v}$. For this reason, we write
\begin{equation}
\vec{w} = \mathop{\mathrm{orth}}\nolimits_{\vec{v}}\vec{u}
\end{equation}
What is it? Well, using basic vector arithmetic, we find
\begin{equation}
\vec{w} = \vec{u} - \proj_{\vec{v}}\vec{u}.
\end{equation}
We will conclude our discussion of vectors here, but continue
next time.
