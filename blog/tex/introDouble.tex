%%
%% introDouble.tex
%% 
%% Made by Alex Nelson
%% Login   <alex@black-cherry>
%% 
%% Started on  Wed Jul  4 12:56:05 2012 Alex Nelson
%% Last update Mon Jul  9 12:40:21 2012 Alex Nelson
%%

\M
Consider $z=f(x,y)$. What is the volume of the region 
\begin{equation}
S = \{ (x,y,z)\in\RR^{3}\mid z=f(x,y),\; a\leq x\leq b,\;
c\leq y\leq d\}
\end{equation}
The first thing we do: consider the rectangle
$R=[a,b]\times[c,d]$ and form a partition of $[a,b]$ into $M-1$
segments and $[c,d]$ into $N-1$ segments. This gives us a mesh of
rectangles $R_{ij} = [x_{i-1},x_{i}]\times[y_{j-1},y_{j}]$ as
specified by the following diagram:
\begin{center}
\includegraphics{img/introDouble.0}
\end{center}
Observe the area of $R_{ij}$ is 
\begin{equation}
\Delta A = \Delta x\Delta y.
\end{equation}
We can approximate the volume $V$ of $S$ by a sort of Riemann
sum, picking points $(x_{ij}^{*}, y_{ij}^{*})$ in $R_{ij}$ and
taking
\begin{equation}
\sum^{M}_{i=1}\sum^{N}_{j=1}f(x^{*}_{ij},y^{*}_{ij})\Delta
A\approx V.
\end{equation}
However, as with Riemann sums, we recover the exact volume when
we take the limits $M,N\to\infty$:
\begin{equation}
V = \lim_{M,N\to\infty}\sum^{M}_{i=1}\sum^{N}_{j=1}f(x^{*}_{ij},y^{*}_{ij})\Delta
A
\end{equation}
if the limit exists. \more

\N{Definition}
We define the \textbf{``Double Integral''} of  $f$ over the integral $R$ as
\begin{equation}
\iint_{R}f(x,y)\,\D A = \lim_{M,N\to\infty}\sum^{M}_{i=1}\sum^{N}_{j=1}f(x^{*}_{ij},y^{*}_{ij})\Delta
A
\end{equation}
if this limit exists. 

Note this sum is called a \emph{double Riemann sum}, just as a
for the single integral we had a \emph{Riemann Sum}.

\N{Properties of Integrals}
We won't prove, but note, there are three important properties
double integrals satisfy. Two can be grouped together as
linearity:
\begin{equation}
\iint_{R}\bigl[f(x,y)+g(x,y)\bigr]\,\D A=\iint_{R}f(x,y)\,\D
A+\iint_{R}g(x,y)\,\D A
\end{equation}
and
\begin{equation}
\iint_{R}cf(x,y)\,\D A = c\iint_{R}f(x,y)\,\D A
\end{equation}
where $f$ and $g$ are continuous on $R$, and $c\in\RR$ is a
constant.

If $f(x,y)\geq g(x,y)$ for all $(x,y)\in R$, then
\begin{equation}
\iint_{R}f(x,y)\,\D A\geq\iint_{R}g(x,y)\,\D A.
\end{equation}

\N{Problem:} How do we compute $\iint_{R}f(x,y)\,\D A$?

\ifblog\section{Iterated Integrals}\fi\iftex\subsection{Iterated Integrals}\fi

\N{Fubini's Theorem} If $f$ is continuous on the integral
$R=[a,b]\times[c,d]$, then
\begin{equation}
\iint_{R}f(x,y)\,\D A=\int^{b}_{a}\left(\int^{d}_{c}f(x,y)\,\D
y\right)\D x
=\int^{d}_{c}\left(\int^{b}_{a}f(x,y)\,\D x\right)\D y.
\end{equation}
We won't prove it (that's what real analysis discusses!), but the
parenthetic terms are functions of a single variable and describe
the area of a ``sheet''. By stacking ``sheets'' we can compute
the volume of the region. 

\M This is fine for rectangular regions, but over arbitrary
regions what do we do? We bound the region $D$ by a pair of
curves $g_{1}(x)\leq y\leq g_{2}(x)$:
\begin{center}
\includegraphics{img/introDouble.1}\qquad
\includegraphics{img/introDouble.2}
\end{center}
In these situations, we write
\begin{equation}
\iint_{D}f(x,y)\,\D
A=\int^{b}_{a}\int^{g_{2}(x)}_{g_{1}(x)}f(x,y)\,\D y\,\D x.
\end{equation}
Please note the order of integration, and corresponding boundary
of integration!

\M
We can likewise bound the region by curves $h_{1}(y)\leq x\leq
h_{2}(y)$, for example:
\begin{center}
\includegraphics{img/introDouble.3}\quad
\includegraphics{img/introDouble.4}
\end{center}
The trick lies with writing these integrals as
\begin{equation}
\iint_{D}f(x,y)\,\D
A=\int^{d}_{c}\int^{h_{2}(y)}_{h_{1}(y)}f(x,y)\,\D x\,\D y.
\end{equation}
Again, note the order of integration!

\begin{example}
Lets consider the domain 
\begin{equation}
D = \{(x,y)\mid -1\leq x\leq1,\; 2x^{2}\leq y\leq 1+x^{2}\}.
\end{equation}
Evaluate the integral
\begin{equation}
I = \iint_{D}(x-y)\,\D A.
\end{equation}

\noindent\emph{Solution:\quad}\ignorespaces We first doodle this domain,
shade it in red: 
\begin{center}
\includegraphics{img/introDouble.5}
\end{center}
Now we use Fubini's theorem writing
\begin{equation}
I = \int^{1}_{-1}\int^{1+x^{2}}_{2x^{2}}(x-y)\,\D y\,\D x.
\end{equation}
Observe the order of integration, and the bounds of integration
are determined by specifying $y\geq 2x^{2}$ and $y\leq1+x^{2}$
for the $\int\D y$ quantity. Similarly, since $-1\leq x\leq1$, we
determine the bounds of integration for the $x$ integral.

Using linearity, we can write
\begin{subequations}
\begin{equation}
I = I_{1}+I{2}
\end{equation}
where
\begin{equation}
I_{1}=\int^{1}_{-1}\int^{1+x^{2}}_{2x^{2}}x\,\D y\,\D x
\end{equation}
and
\begin{equation}
I_{2}= \int^{1}_{-1}\int^{1+x^{2}}_{2x^{2}}-y\,\D y\,\D x
\end{equation}
\end{subequations}
Now we can evaluate each of these separately. 

We see for $I_{1}$, there are no $y$ expressions, so we have
\begin{equation}
\int^{1}_{-1}\int^{1+x^{2}}_{2x^{2}}x\,\D y\,\D x=
\int^{1}_{-1}x\int^{1+x^{2}}_{2x^{2}}\,\D y\,\D x.
\end{equation}
Performing this inner integral
\begin{equation}
\int^{1+x^{2}}_{2x^{2}}\D y
= \bigl(1+x^{2}\bigr)-(2x^{2})=1-x^{2}.
\end{equation}
We plug this back in:
\begin{equation}
\int^{1}_{-1}x\left(\int^{1+x^{2}}_{2x^{2}}\,\D y\right)\D x
=\int^{1}_{-1}x\bigl(1-x^{2}\bigr)\D x.
\end{equation}
This can be solved quickly as
\begin{equation}
\begin{aligned}
\int^{1}_{-1}x\bigl(1-x^{2}\bigr)\D x
&= \left.\frac{x^{2}}{2}-\frac{x^{4}}{4}\right|^{1}_{-1}\\
&= \left(\frac{(1)^{2}}{2}-\frac{(1)^{4}}{4}\right)
-\left(\frac{(-1)^{2}}{2}-\frac{(-1)^{4}}{4}\right)\\
&=0.
\end{aligned}
\end{equation}
Thus
\begin{equation}
I_{1}=0.
\end{equation}
Moreover this implies $I=I_{2}$.

What to do about $I_{2}$? Recall
\begin{equation*}
I_{2}= \int^{1}_{-1}\int^{1+x^{2}}_{2x^{2}}-y\,\D y\,\D x
\end{equation*}
We first perform the inner integral
\begin{equation}
\begin{aligned}
\int^{1+x^{2}}_{2x^{2}}y\,\D y &= \left.\frac{y^{2}}{2}\right|^{1+x^{2}}_{2x^{2}}\\
&=\frac{(1+x^{2})^{2} - (2x^{2})^{2}}{2}\\
&=\frac{1+2x^{2}+x^{4}-4x^{4}}{2} = \frac{1+2x^{2}-3x^{4}}{2}.
\end{aligned}
\end{equation}
We plug this back in
\begin{equation}
\int^{1}_{-1}-\left(\int^{1+x^{2}}_{2x^{2}}y\,\D y\right)\D x
=-\int^{1}_{-1}\frac{1+2x^{2}-3x^{4}}{2}\D x.
\end{equation}
We can calculate this out directly as
\begin{equation}
\begin{aligned}
\int^{1}_{-1}(1+2x^{2}-3x^{4})\,\D x
&=\left.x+\frac{2x^{3}}{3}-\frac{3x^{5}}{5}\right|^{1}_{-1}\\
&=\left(1+\frac{2}{3}-\frac{3}{5}\right)-\left(-1-\frac{2}{3}+\frac{3}{5}\right)\\
&=2+\frac{4}{3}-\frac{6}{5}=\frac{32}{15}.
\end{aligned}
\end{equation}
But we forgot the coefficient $-1/2$ in front of the integral!
Thus our integral is 
\begin{subequations}
\begin{equation}
I_{2}=\frac{-32}{30}
\end{equation}
giving us our final result
\begin{equation}
I = 0 - \frac{32}{30},
\end{equation}
\end{subequations}
which concludes our example.
\end{example}
