%%
%% moreVectors.tex
%% 
%% Made by Alex Nelson
%% Login   <alex@black-cherry>
%% 
%% Started on  Thu Jun 21 15:16:44 2012 Alex Nelson
%% Last update Thu Jun 21 17:28:46 2012 Alex Nelson
%%

\M
Last time we ended with discussing how to project a vector onto
another. So if we consider projecting $\vec{u}$ onto $\vec{v}$,
we can write this as
\begin{equation}
\proj_{\vec{v}}\vec{u} = \vec{u}_{\|}
\end{equation}
Observe, we have another vector constructed
\begin{equation}
\vec{u}_{\bot} = \vec{u}-\vec{u}_{\|}
\end{equation}
which is orthogonal to $\vec{v}$. We have a closed form
expression for projection, namely
\begin{equation}
\proj_{\vec{v}}\vec{u} = (\vec{u}\cdot\widehat{v})\widehat{v}
\end{equation}
where $\widehat{v}=\vec{v}/\|\vec{v}\|$ is a unit vector. But do
we have a closed form expression for $\vec{u}_{\bot}$?\more

\N{Cross-Product}
The cross product of $\vec{u}$ and $\vec{v}$ is 
\begin{equation}
\vec{u}\times\vec{v} = \begin{vmatrix}
\widehat{\textbf{\i}} & \widehat{\textbf{\j}} & \widehat{\mathbf{k}}\\
       u_{1}          &            u_{2}      & u_{3}\\
       v_{1}          &            v_{2}      & v_{3}
\end{vmatrix} =
\bigl(\|\vec{u}\|\|\vec{v}\|\sin(\theta)\bigr)\widehat{u}
\end{equation}
Note we are using notation from linear algebra writing, recursively,
\begin{equation}
\begin{aligned}
\det(A) &= \begin{vmatrix} a_{11} & a_{12} & a_{13}\\
a_{21} & a_{22} & a_{23}\\
a_{31} & a_{32} & a_{33}
\end{vmatrix} \\
&= a_{11}\begin{vmatrix} a_{22} & a_{23} \\ a_{32} &
  a_{33}
\end{vmatrix}
- a_{12} \begin{vmatrix} a_{21} & a_{23}\\ a_{31} & a_{33}
\end{vmatrix}
+a_{13}\begin{vmatrix} a_{21} & a_{22}\\ a_{31} & a{32}
\end{vmatrix}
\end{aligned}
\end{equation}
where
\begin{equation}
\begin{vmatrix} a & b\\ c & d
\end{vmatrix} = ad-bc.
\end{equation}
\begin{remark}
Observe this implies
$\widehat{\textbf{\i}}\times\widehat{\textbf{\j}}=\widehat{\textbf{k}}$, 
$\widehat{\textbf{\j}}\times\widehat{\textbf{k}}=\widehat{\textbf{\i}}$,
and 
$\widehat{\textbf{k}}\times\widehat{\textbf{\i}}=\widehat{\textbf{\j}}$.
\end{remark}
\begin{remark}
The cross-product takes two vectors, and \emph{produces a third vector}.
It \emph{does not} produce a scalar (a number, unlike the dot product).
\end{remark}
\emph{Pop quiz}: let $\vec{u}$ and $\vec{v}$ be vectors. Is $\vec{u}\times\vec{v}=\vec{v}\times\vec{u}$?

\begin{example}
Consider $\vec{u}=\langle2,1,-3\rangle$ and
$\vec{v}=\langle1,-2,1\rangle$. What is $\vec{u}\times\vec{v}$?

\emph{Solution}: we find
\begin{equation}
\begin{aligned}
\vec{u}\times\vec{v} &= \begin{vmatrix}
\widehat{\textbf{\i}} & \widehat{\textbf{\j}} &\widehat{\textbf{k}}\\
2 & 1 & -3\\
1 & -2 & 1
\end{vmatrix}\\
&= (1\cdot1-(-3)\cdot(-2))\widehat{\textbf{\i}} 
- (2\cdot1-(-3)\cdot1)\widehat{\textbf{\j}} 
+(2\cdot(-2)-1\cdot1)\widehat{\textbf{k}}\\
&=(1-6)\widehat{\textbf{\i}} 
- (2+3)\widehat{\textbf{\j}} 
+(-4-1)\widehat{\textbf{k}}\\
&=\langle-5,5,-5\rangle
\end{aligned}
\end{equation}
Another approach would have been to write
\begin{equation}
\vec{u}\times\vec{v} 
= (2\widehat{\textbf{\i}} + \widehat{\textbf{\j}} -3\widehat{\textbf{k}})
\times(\widehat{\textbf{\i}} -2 \widehat{\textbf{\j}} +\widehat{\textbf{k}})
\end{equation}
and used the cross-product's anticommutativity to do the calculations.
\end{example}

\N{Parallelogram Area}
Consider three distinct points $P$, $Q$, and $R$. We can
construct a parallelograph, as in the following diagram:
\begin{center}
\includegraphics{img/moreVectors.0}
\end{center}
We see that
$\overrightarrow{PQ}\times\overrightarrow{PR}=\vec{N}$, then the
area of the parallelogram is $\|\vec{N}\|$.

\N{Parallepiped Volume}
If we work in 3-space, and we have a six-sided region whose sides
are each parallelograms, we call this region a
parallepiped. Observe that we only need 3 vectors to specify the
vertices: $\vec{u}$, $\vec{v}$, and $\vec{w}$. Then we consider
$\vec{u}+\vec{v}$, $\vec{u}+\vec{w}$, $\vec{v}+\vec{w}$, and
$\vec{u}+\vec{v}+\vec{w}$ for the remaining vertices. What is the
volume of this region?

Lets draw a diagram:
\begin{center}
\includegraphics{img/moreVectors.1}
\end{center}
Lets first consider the face described by $\overrightarrow{PQ}=\vec{u}$ and
$\overrightarrow{PR}=\vec{v}$. We see the parallepiped may be considered as a
``stack'' of such faces, whose height is given by the third
vector $\vec{w}$. Then we see the area of the face is shaded in
the diagram, and algebraically it's given by
$\vec{u}\times\vec{v}$, and this produces a vector whose
magnitude is the area of the face. When we ``dot'' this with
$\proj_{\vec{u}\times\vec{v}}\vec{w}$, it's intuitively taking the product of the ``area of
aa face'' ($\vec{u}\times\vec{v}$) and the ``height of the
parallepiped'' ($\proj_{\vec{u}\times\vec{v}}\vec{w}$) producing the volume 
\begin{equation}
\mbox{volume } = (\vec{u}\times\vec{v})\cdot\vec{w}.
\end{equation}
Note this can be negative, and this just tells us information
regarding the parallepiped's \emph{orientation}.

