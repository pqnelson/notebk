%%
%% swapOrderDouble.tex
%% 
%% Made by Alex Nelson
%% Login   <alex@black-cherry>
%% 
%% Started on  Tue Jul 31 10:29:59 2012 Alex Nelson
%% Last update Tue Jul 31 13:04:33 2012 Alex Nelson
%%
\M
So we had introduced double integrals. We perform double
integration thanks to Fubini's theorem, stating
\begin{equation}
\iint_{R}f(x,y)\,\D A
=\iint_{R}f(x,y)\,\D x\,\D y
=\iint_{R}f(x,y)\,\D y\,\D x.
\end{equation}
Sometimes we cannot solve a double integral problem without
swapping the order of integration. Lets consider this in more
detail.\more

\N{Finding the Limits of Integration}
Lets review how we find limits of integration. We will work
through the example considering the domain $R$ bounded by the
curves
\begin{equation}
1=y+x,\quad\mbox{and}\quad 1=y^{2}+x^{2}.
\end{equation}
We consider some arbitrary function $f(x,y)$, and we will express
the integral $\iint_{R}f(x,y)\,\D A$ as $\iint_{R}f(x,y)\,\D
y\,\D x$. What's the algorithm?

\emph{Step One: Sketch}. We sketch the region of integation, and
label the bounding curves.
\begin{center}
  \includegraphics{img/swapOrder.0}
\end{center}

\emph{Step Two: Find the $y$-Limits of Integration}. We consider
a vertical line $L$ cutting through the region $R$ in the
direction of increasing $y$. Mark the $y$ values where $L$ enters
and leaves. These are the $y$-limits of integration and are
usually functions of $x$ (instead of constants).
\begin{center}
  \includegraphics{img/swapOrder.1}
\end{center}
This tells us that $1-x\leq y\leq\sqrt{1-x^{2}}$. Also note this
constrains $x^{2}\leq 1$ if $y$ is a real number.

\emph{Step Three: Find the $x$-Limits of Integration}. We choose
the $x$-limits that include all the vertical lines through
$R$. For us, this constrains $0\leq x\leq 1$. Thus we write the
double integral 
\begin{equation}
\iint_{R}f(x,y)\,\D
A=\int^{1}_{0}\int^{\sqrt{1-x^{2}}}_{1-x}f(x,y)\,\D y\,\D x.
\end{equation}
This concludes the generic procedure for finding the limits of
integration. 

\N{Swapping the Limits of Integration} Consider the double integral
\begin{equation}
I=\int^{1}_{0}\int^{3}_{3y}\exp(-x^{2})\,\D x\,\D y.
\end{equation}
How can we find $I$? We need to swap the order of integration. 

Lets doodle the domain of integration:
\begin{center}
  \includegraphics{img/swapOrder.2}
\end{center}
We draw a vertical line to find the $y$-limits of integration
\begin{center}
  \includegraphics{img/swapOrder.3}
\end{center}
We have the $y$-limits of integration be
\begin{equation}
0\leq y\leq x/3.
\end{equation}
Observe the $x$-limits of integration are quite simple
\begin{equation}
0\leq x\leq 3.
\end{equation}
We have the limits of integration, and we can write
\begin{equation}
I=\int^{3}_{0}\int^{x/3}_{0}\exp(-x^{2})\,\D y\,\D x.
\end{equation}
This may be solved quickly
\begin{equation}
I = \int^{3}_{0}\frac{x}{3}\exp(-x^{2})\,\D x
\end{equation}
and let $u=x^{2}$, so $\D u=2x\,\D x$ lets us write
\begin{equation}
I=\int^{\sqrt{3}}_{0}\frac{\exp(-u)}{6}\,\D u = \left.\frac{-1}{6}\exp(-u)\right]^{\sqrt{3}}_{0}.
\end{equation}
Thus
\begin{equation}
I = \frac{1-\exp(-\sqrt{3})}{6} \approx 0.13718.
\end{equation}
We couldn't have calculated this without swapping the order of integration.
