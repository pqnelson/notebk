%%
%% directionalDerivatives.tex
%% 
%% Made by Alex Nelson
%% Login   <alex@black-cherry>
%% 
%% Started on  Fri Jun 29 12:30:18 2012 Alex Nelson
%% Last update Tue Jul  3 11:02:55 2012 Alex Nelson
%%
\M
Suppose we have a scalar function of several variables
\begin{equation}
f\colon\RR^{3}\to\RR
\end{equation}
Let $\widehat{u}$ be some unit vector. How does $f$ change in the
$\widehat{u}$ direction?

We can consider this quantity as a function
\begin{equation}
g(\vec{x}) =
\lim_{h\to0}\frac{f(\vec{x}+h\widehat{u})-f(\vec{x})}{h}
\end{equation}
What does this look like?

\M Lets restrict our attention to the
smallest non-boring case: $f\colon\RR^2\to\RR$. Then we write
$\widehat{u} = \langle p,q\rangle$. We have
\begin{equation}
f(\vec{x}+h\widehat{u}) = f(x+hp,y+hq).
\end{equation}
Expanding this to first order in $h$ lets us write
\begin{equation}
\begin{aligned}
f(x+hp,y+hq) &= f(x,y+hq) +
hp\partial_{x}f(x,y+hq)+\bigO(h^{2})\\
&= \bigl(f(x,y)+hq\partial_{y}f(x,y)+\bigO(h^{2})\bigr)\\
&\quad+hp\partial_{x}\bigl(f(x,y)+hq\partial_{y}f(x,y)+\bigO(h^{2})\bigr)\\
&\quad+\bigO(h^{2})\\
&= f(x,y) + h\bigl(q\partial_{y}f(x,y) +
p\partial_{x}f(x,y)\bigr) +\bigO(h^{2}).
\end{aligned}
\end{equation}
But what does this look like? It's simply
\begin{equation}
f(\vec{x}+h\widehat{u}) = f(\vec{x}) +
h\widehat{u}\cdot\langle\partial_{x},\partial_{y}\rangle
f(\vec{x}) + \bigO(h^{2}).
\end{equation}

\begin{example}
What is the derivative of $f(x,y)=x/y$ in the direction of
$\vec{v}=\langle 1,3\rangle$ at $(5,3)$?

\emph{Solution}: We find the directional derivative is
\begin{equation}
v_{1}\partial_{x}f + v_{2}\partial_{y}f
\end{equation}
We compute
\begin{equation}
\partial_{x}f = 1/y
\end{equation}
and
\begin{equation}
\partial_{y}f = -x/y^{2}
\end{equation}
Thus we have
\begin{equation}
v_{1}\partial_{x}f + v_{2}\partial_{y}f = v_{1}(1/y) + v_{2}(-x/y^{2}).
\end{equation}
We plug in $v_{1}=1$, $v_{2}=3$
\begin{equation}
v_{1}\partial_{x}f + v_{2}\partial_{y}f = (1/y) + 3(-x/y^{2}).
\end{equation}
Then we evaluate $(x,y)=(5,3)$ to get
\begin{equation}
v_{1}\partial_{x}f + v_{2}\partial_{y}f = (1/3) + 3(-5/3^{2})
=-4/3.
\end{equation}
This gives us the directional derivative of $f$.
\end{example}


\M We denote
\begin{equation}
\vec{\nabla} = \langle\partial_{1},\dots,\partial_{n}\rangle
\end{equation}
and call it the \textbf{``Gradient''}. Note we will write
$\nabla$ interchangeably with the vector arrow $\vec{\nabla}$,
and they mean the same thing. The vector arrow doesn't add
anything semantically, it's just different syntax.

The directional derivative is then
\begin{equation}
\widehat{u}\cdot\vec{\nabla}f(\vec{x}) = \widehat{u}\cdot\langle
\partial_{1}f,\dots,\partial_{n}f\rangle.
\end{equation}
Note that the gradient acting on a scalar function produces a
vector-valued function of several variables, but we can also take
the dot product of the gradient with such a monstrosity.

\N{Question:} What is a vector-valued function of several variables?

For us, in practice, we think of this as a \emph{Vector Field}: a
``function'' which assigns to each point a vector. Each
vector-component is a function, usually smooth (i.e., infinitely
differentiable). 

(Again, just as we warned the reader with vectors, this too is a
lie. A vector field is a bit more than just a function
$\RR^{n}\to\RR^{n}$, it's a more complicated beast which is
studied further in differential geometry.)

\N{Meaning of Gradient}
Consider a family of level curves $f(\vec{x})=c$. The gradient
points towards the direction of increasing $c$. How can we see
this? Well, consider the function
\begin{equation}
f(x,y)=x^{2}-y^{2}.
\end{equation}
We see its gradient is
\begin{equation}
\vec{\nabla}f(x,y) = \langle 2x, -2y\rangle.
\end{equation}
Lets draw a few level-curves and see what the vectors point to:
\begin{center}
\includegraphics{img/gradient.0}
\end{center}
We see the vectors point towards $(x,y)\to(\pm\infty,0)$. 

\begin{exercise}
Consider the function $f\colon\RR^{2}\to\RR$ defined by
$f(x,y)=\ln(x^{2}+y^{2})$. What is its gradient?
\end{exercise}
\begin{exercise}
Let $g\colon\RR^{3}\to\RR$ be defined by $g(x,y,z) =
(x^{2}+y^{2}+z^{2})^{-1/2}$. Find its gradient.
\end{exercise}
\begin{exercise}
Let $f(x,y,z)=x/(y+z)$. Find its derivative in the direction
$\vec{u}=\langle 1,1,1\rangle$ at the point $(1,6,2)$. 
\end{exercise}
\begin{exercise}
Let $f(x,y)=x\exp(-y) + 3y$. Find its gradient.
\end{exercise}
