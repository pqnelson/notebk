%%
%% totalDifferential.tex
%% 
%% Made by Alex Nelson
%% Login   <alex@black-cherry>
%% 
%% Started on  Fri Jun 29 14:21:33 2012 Alex Nelson
%% Last update Fri Jun 29 14:21:58 2012 Alex Nelson
%%
\N{Total Differential}
Lets consider some function
\begin{equation}
f\colon\RR^{2}\to\RR.
\end{equation}
What is $\D f$? Well, we can write it as
\begin{equation}
\D\vec{x}\cdot\vec{\nabla}f = (\D x\,\partial_{x}+\D
y\,\partial_{y})f(x,y)
\end{equation}
in Cartesian coordinates. We can write a ``linear approximation''
to $f(x,y)$ as
\begin{equation}
f(x+\Delta x, y+\Delta y)\approx f(x,y) + \langle\Delta x,\Delta
y\rangle\cdot\vec{\nabla}f(x,y).
\end{equation}
Again, this should remind us of the linear approximation when we
used the tangent line as an approximation to a curve.

\M
Another way to consider to total differential $\D{f}$ is by using
the ``chain rule'' (wink wink). We write
\begin{equation}
\D f = \frac{\partial f(x,y)}{\partial x}\D x
+\frac{\partial f(x,y)}{\partial y}\D y.
\end{equation}
This can be quite useful!

\begin{example}
Lets consider a cylinder whose radius is $r=10$ and height is
$h=100$. Its volume is $V(r,h)=\pi r^{2}h$. The measurement is
correct to $0.1$ precision, what's the error in our measurement?

Well, we approximate it as
\begin{equation}
\begin{aligned}
\D V(r,h) &\approx 0.1\partial_{r}V(r,h) + 0.1\partial_{h}
V(r,h)\\
&= 0.1(2\pi rh) + 0.1(\pi r^{2})
\end{aligned}
\end{equation}
The error is thus
\begin{equation}
\D V(10,100) \approx 0.1 (2\pi\cdot10\cdot100) + 0.1(\pi\cdot 10^{2})
\approx 210\pi.
\end{equation}
The error is approximately $659.7339$, and our estimated volume
is $31415$. The error is about 2\%, which is quite good.
\end{example}
