%%
%% polar.tex
%% 
%% Made by Alex Nelson
%% Login   <alex@black-cherry>
%% 
%% Started on  Wed Aug  1 15:26:24 2012 Alex Nelson
%% Last update Wed Aug  1 16:02:11 2012 Alex Nelson
%%

\M
Recall with trigonometric functions, we considered a unit
circle. Given any point $(x_{0}, y_{0})$, we see it lies on a
circle with radius $r=\sqrt{x_{0}^{2}+y_{0}^{2}}$. We can
generalize this idea inventing a new coordinate system: any point
in $\RR^{2}$ can be described using the radial distance to the
origin, and an angular component.

\M
We fix a point (usually the origin) then use the $x$-axis to
determine an angle $\theta$ to a point $r$ away from the
origin. So we can doodle the generic situation as:
\begin{center}
\includegraphics{img/polar.0}
\end{center}
Note this is how people usually describe distances. For example
``New York is 3000 miles to the North-East'' where $r$ is given
as ``3000 miles'' and the angle $\theta$ we can determine from
``North-East''. 

Note this is a \emph{completely different} coordinate system,
i.e., we describe coordinates using different numeric values and
a different approach. We previously used the Cartesian coordinate
system, describing points in the ``obvious'' way
(component-wise). However, we can relate Cartesian coordinates to
polar coordinates, writing
\begin{equation}\label{eq:cartInPolar}
x=r\cos(\theta),\quad\mbox{and}\quad y=r\sin(\theta).
\end{equation}
Similarly, we can obtain polar coordinates from Cartesian
coordinates by
\begin{equation}
r=\sqrt{x^{2}+y^{2}},\quad\mbox{and}\quad \theta=\arctan(y/x).
\end{equation}
We can use either polar coordinates or Cartesian coordinates when
describing the plane $\RR^2$.

\M
How do we describe curves using polar coordinates? We would
effectively have
\begin{equation}
r=r(\theta),\quad\mbox{or}\quad \theta=\theta(r).
\end{equation}
The naive approach would convert to Cartesian coordinates. We
also have several heuristic rules describing the behaviour of the
curve. 

\M
The tangent line to $r=r(\theta)$ at $\theta=\theta_{0}$. We can
use implicit differentiation to find its slope as
\begin{equation}
\frac{\D y}{\D x}=\frac{\D y/\D\theta}{\D x/\D\theta}
\end{equation}
Recall Eq \eqref{eq:cartInPolar} and we can find the slope's
numerator
\begin{equation}
\frac{\D y}{\D\theta}=\frac{\D r(\theta)}{\D\theta}\sin(\theta)+r(\theta)\cos(\theta)
\end{equation}
and the denominator
\begin{equation}
\frac{\D x}{\D\theta}=\frac{\D r(\theta)}{\D\theta}\cos(\theta)-r(\theta)\sin(\theta)
\end{equation}
giving us
\begin{equation}
\frac{\D y}{\D x}=\frac{\displaystyle\frac{\D r(\theta)}{\D\theta}\sin(\theta)+r(\theta)\cos(\theta)}{\displaystyle\frac{\D r(\theta)}{\D\theta}\cos(\theta)-r(\theta)\sin(\theta)}.
\end{equation}
Thus the tangent line is
\begin{equation}
t(x) = y(\theta_{0}) + \left.\frac{\D y}{\D x}\right|_{\theta=\theta_{0}}\cdot\bigl(x-x(\theta_{0})\bigr).
\end{equation}


\begin{exercise}[Ovals of Cassini]
Consider the curve described by
$r^{4}-2c^{2}r^{2}\cos(2\theta)+c^{4}-a^{4}=0$. 
\end{exercise}
