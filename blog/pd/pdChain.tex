%%
%% pdChain.tex
%% 
%% Made by Alex Nelson
%% Login   <alex@black-cherry>
%% 
%% Started on  Thu Jun 28 15:42:27 2012 Alex Nelson
%% Last update Tue Jul  3 10:54:12 2012 Alex Nelson
%%

\N{Problem}
Consider a function $f(x,y)$ where we parametrize 
\begin{equation}
x=x(t,u),\quad\mbox{and}\quad y=y(t,u).
\end{equation}
If $t\to t+\Delta t$, how does $f\to f+\Delta f$ change?

\M
We first note
\begin{equation}
f(x+\Delta x,y) = f(x,y)+\Delta x\partial_{x}f(x,y)+\bigO(\Delta
x^{2}).
\end{equation}
Similarly 
\begin{equation}
f(x,y+\Delta y)= f(x,y)+\Delta y\partial_{y}f(x,y)+\bigO(\Delta
y^{2}).
\end{equation}
Thus we find
\begin{equation}
\begin{aligned}
f(x+\Delta x,y+\Delta y)
&=f(x,y+\Delta y)+\Delta x\partial_{x}f(x,y + \Delta y)+\bigO(\Delta
x^{2})\\
&=\bigl(f(x,y) + \Delta y\partial_{y}f(x,y) + \bigO(\Delta
y^{2})\bigr)\\
&\quad+ \Delta x\partial_{x}\bigl(f(x,y) + \Delta y\partial_{y}f(x,y) + \bigO(\Delta
y^{2})\bigr)\\
&\quad+\bigO(\Delta x^{2})\\
&= f(x,y) + \Delta y\partial_{y}f(x,y) + \Delta
x\partial_{x}f(x,y)\\
&\quad + \bigO(\Delta x\Delta y)%\partial_{x}\partial_{y}f(x,y)
+\bigO(\Delta x^{2})+\bigO(\Delta y^{2}).
\end{aligned}
\end{equation}
But specifically, we are interested in
\begin{equation}
\Delta x = \Delta t\partial_{t}x(t,u)+\bigO(\Delta t^{2})
\end{equation}
and
\begin{equation}
\Delta y = \Delta t\partial_{t}y(t,u)+\bigO(\Delta t^{2})
\end{equation}
Plugging this in allows us to write
\begin{equation}
\Delta f=\Delta y\partial_{y}f(x,y) + \Delta
x\partial_{x}f(x,y) + \bigO(\Delta x\Delta y)%\partial_{x}\partial_{y}f(x,y)
+\bigO(\Delta x^{2})+\bigO(\Delta y^{2})
\end{equation}
as
\begin{equation}
\Delta f = \bigl(\Delta t\partial_{t}y\bigr)\bigl(\partial_{y}f\bigr) +
\bigl(\Delta t\partial_{t}x\bigr)\bigl(\partial_{x}f\bigr) +
\bigO(\Delta t^{2})
\end{equation}
Observe under our substitution, we have the $\bigO(\Delta
x\Delta y)$ and other big O terms be gathered into the
$\bigO(\Delta t^{2})$ term.

So what? Observe
\begin{equation}
\begin{aligned}
\frac{\partial f}{\partial t} &= \lim_{\Delta t\to 0}\frac{f\bigl(x(t+\Delta t,u),y(t+\Delta t)\bigr)-f\bigl(x(t,u),y(t,u)\bigr)}{\Delta t} \\
&=\frac{\partial f}{\partial x}\frac{\partial x}{\partial t}+
\frac{\partial f}{\partial y}\frac{\partial y}{\partial t}.
\end{aligned}
\end{equation}
This is precisely the chain rule. Similarly, we find
\begin{equation}
\frac{\partial f}{\partial u}
=\frac{\partial f}{\partial x}\frac{\partial x}{\partial u}+
\frac{\partial f}{\partial y}\frac{\partial y}{\partial u}.
\end{equation}

\N{Implicit Differentiation Revisited}
Recall implicit differentiation required us to find $\D y/\D x$
from some complicated expression like
\begin{equation}
\E^{xy}+4y^{2}+\tan(x+y)=0
\end{equation}
What to do? First we write
\begin{equation}
z = F(x,y) = \E^{xy}+4y^{2}+\tan(x+y)=0.
\end{equation}
Next we say $y=y(x)$. So we find
\begin{equation}
\begin{aligned}
\frac{\D z}{\D x}
&= \frac{\partial F(x,y)}{\partial x}\frac{\D x}{\D x}+\frac{\partial F(x,y)}{\partial y}\frac{\D y}{\D x}\\
&=0
\end{aligned}
\end{equation}
where we set the derivative of $F$ to be zero since it's equal to the
derivative of zero. We can then write (taking $\D x/\D x=1$)
\begin{equation}
-\frac{\partial F(x,y)}{\partial x}=\frac{\partial
  F(x,y)}{\partial y}\frac{\D y}{\D x}
\end{equation}
and divide both sides by $\partial_{y}F$ to get
\begin{equation}
\frac{-\partial_{x}F}{\partial_{y}F} = \frac{\D y}{\D x}.
\end{equation}
But this is precisely what implicit differentiation gives us!

\N{Warning for Physicists} 
Physicists often use partial derivative notation slightly
differently. If $q(t)$ is the position of a particle, and $p(t)$
is its momentum, physicists consider arbitrary functions of the
form
\begin{equation}
f=f(q,p,t)
\end{equation}
and write
\begin{equation}
\frac{\partial f}{\partial t} = \lim_{\Delta t\to0}
\frac{f\bigl(q(t),p(t),t+\Delta t\bigr)-f\bigl(q(t),p(t),t\bigr)}{\Delta t}.
\end{equation}
This is strictly speaking not quite true. The error committed
lies in treating $q$ and $p$ as functions of time: really they
are variables whom we are trying to express as functions of
time. 

\begin{exercise}
Let $w=\sqrt{x}+y^{2}/z$ where $x=\exp(2t)$, $y=t^{3}+4t$, and
$z=t^{2}-t$. Find $\D w/\D t$.
\end{exercise}
\begin{exercise}
Let $z=\cos(xy)+y\sin(x)$ where $x=v^{2}+u$ and $y=u-v$. Find
$\partial_{u} z$ and $\partial_{v}z$.
\end{exercise}
