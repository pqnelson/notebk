%%
%% multivarContinuity.tex
%% 
%% Made by Alex Nelson
%% Login   <alex@black-cherry>
%% 
%% Started on  Wed Jun 27 11:16:54 2012 Alex Nelson
%% Last update Thu Jun 28 12:47:00 2012 Alex Nelson
%%
\M
We considered differentiating and integrating functions of a
single-variable. How? We began with the notion of a limit, and
then considered the derivative. If we have a, e.g., polynomial
\begin{equation}
p(x,y) = x^{3}+x^{2}y+xy^{2}+y^{3}
\end{equation}
we see
\begin{equation}
p(x+\Delta x,y) = x^{3}+x^{2}y+xy^{2}+y^{3}+\bigl(3x^{2}+2xy+y\bigr)\Delta x+
\mathcal{O}(\Delta x^{2})
\end{equation}
Again we stop and reflect: this treats $y$ as if it were
constant. So the derivative formed by
\begin{equation}
\lim_{\Delta x\to0}\frac{p(x+\Delta x,y)-p(x,y)}{\Delta x}=3x^{2}+2xy+y
\end{equation}
are ``incomplete'' or \textbf{partial}. There is some subtlety
here due to using multiple variables, and we have to discuss the
problems of limits first.\more

\N{Definition}
The function $z=f(x,y)$ is \textbf{``Continuous''} at
$(x_{0},y_{0})$ if

(i) $f(x_{0},y_{0})$ is defined and finite;

(ii) $\displaystyle\lim_{(x,y)\to(x_{0},y_{0})}f(x,y)=f(x_{0},y_{0})$ is
defined;

(iii) $\displaystyle\lim_{(x,y)\to(x_{0},y_{0})}f(x,y)$ is defined (and finite).

\noindent{Note:} this can be determined by picking any curve
$\gamma\colon[0,1]\to\RR^{2}$ which satisfies
\begin{equation}
\gamma(t_{0}) = (x_{0},y_{0})
\end{equation}
for some $0\leq t_{0}\leq 1$, then taking 
\begin{equation}
\lim_{t\to t_{0}}f\bigl(\gamma(t)\bigr) = \lim_{(x,y)\to(x_{0},y_{0})}f(x,y).
\end{equation}
The subtletly here lies with $\gamma$ being \emph{arbitrary}. If
two different curves produce two different results, the limit
\emph{does not exist}. Lets consider some examples and non-examples.

\begin{example}[Limit Exists]
Find
\begin{equation}
\lim_{(x,y)\to(2,4)}\frac{y+4}{x^{2}y-xy+4x^{2}-4x}.
\end{equation}

\emph{Solution}: for this, we can simply plug in the values
\begin{equation}
\begin{aligned}
\lim_{(x,y)\to(2,4)}\frac{y+4}{x^{2}y-xy+4x^{2}-4x}
&=\frac{(4)+4}{(2)^{2}(4)-(2)(4)+4(2^{2})-4(2)}\\
&=\frac{8}{16-8+16-8}=\frac{1}{2}.
\end{aligned}
\end{equation}
This is because the function is sufficiently nice.
\end{example}

\begin{example}[Limit Doesn't Exist]
What is
\begin{equation}
\lim_{(x,y)\to(0,0)}\frac{x^{4}}{x^{4}+y^{2}}=?
\end{equation}

\emph{Solution}: Lets first approach it along the $x$-axis,
i.e. first setting $y=0$. We find
\begin{equation}
\lim_{(x,y)\to(0,0)}\frac{x^{4}}{x^{4}+y^{2}}=\lim_{x\to0}\frac{x^{4}}{x^{4}}=1.
\end{equation}
Now lets approach it on the $y$-axis, i.e. first setting
$x=0$. We see
\begin{equation}
\lim_{(x,y)\to(0,0)}\frac{x^{4}}{x^{4}+y^{2}}=\lim_{y\to0}\frac{0}{0+y^{2}}=0.
\end{equation}
Still, approaching along the curve $y=x^{2}$ we see
\begin{equation}
\lim_{(x,y)\to(0,0)}\frac{x^{4}}{x^{4}+y^{2}}=\lim_{x\to0}\frac{x^{4}}{x^{4}+x^{4}}=\frac{1}{2}.
\end{equation}
But we have a problem: this implies $0=1/2=1$. This cannot be! So
the limit \emph{cannot exist!} Very sad.
\end{example}


\N{Definition} Let $z=f(x,y)$ be defined on a region $R$ in the
$xy$-plane, and let $(x_{0},y_{0})$ be an inerior point of $R$,
we just don't want a boundary point!

If
\begin{equation}
\lim_{\Delta x\to0}\frac{f(x_{0}+\Delta
  x,y_{0})-f(x_{0},y_{0})}{\Delta x}
\end{equation}
exists, then it is called the \textbf{``Partial Derivative''} of
$z=f(x,y)$ at $(x_{0},y_{0})$ with respect to $x$. It is denoted
\begin{equation}
\frac{\partial}{\partial x}f = \frac{\partial}{\partial x}z
=f_{x} = \partial_{x}f = \partial_{x}z
\end{equation}
evaluated at $(x_{0},y_{0})$. NB: the subscripts in the
$\partial_{x}$ indicate what variable we are taking the partial
derivative of, i.e., it's shorthand for
$\partial_{x}=\partial/\partial x$.

Under similar conditions,
\begin{equation}
\lim_{\Delta y\to0}\frac{f(x_{0},y_{0}+\Delta y)-f(x_{0},y_{0})}{\Delta y}
\end{equation}
is the partial derivative of $z=f(x,y)$ with respect to $y$ at
$(x_{0},y_{0})$. We denote this by
\begin{equation}
\frac{\partial f}{\partial y}=\frac{\partial z}{\partial
  y}=\partial_{y}f = \partial_{y}z
\end{equation}
among a myriad of different conventions.

\M Higher order partial derivatives are done by taking it one at
a time. So if 
\begin{equation}
z=\E^{xy}
\end{equation}
for example, we have
\begin{equation}
\partial_{y}z=x\E^{xy}
\end{equation}
and taking its derivative again yields
\begin{equation}
\begin{aligned}
\partial_{y}^{2}z &= \partial_{y}\left(x\E^{xy}\right)\\
&=x\partial_{y}(\E^{xy})
\end{aligned}
\end{equation}
Note we factor $x$ out in front of the partial derivative with
respect to $y$ because $x$ is constant with respect to $y$. So we
then obtain
\begin{equation}
\partial_{y}^{2}z = x^{2}\E^{xy}.
\end{equation}
We take partial derivatives one at a time, from right to left:
\begin{equation}
\partial_{x}\partial_{y}z
= \partial_{x}\bigl(\partial_{y}z\bigr).
\end{equation}
\emph{Question}: do partial derivatives commute? I.e., is
$\partial_{x}\partial_{y}=\partial_{y}\partial_{x}$ always?
Lets first consider an example calculation before considering an answer.

\begin{example}
Consider the function $u=x^{2}-y^{2}$. Find
$\partial_{x}^{2}u+\partial_{y}^{2}u$.

\emph{Solution}: We find that
\begin{equation}
\partial_{x}u = 2x\implies \partial_{x}^{2}u = 2.
\end{equation}
Similarly, we find
\begin{equation}
\partial_{y}u=-2y\implies \partial_{y}^{2}y=-2.
\end{equation}
Thus we conclude
\begin{equation}
\partial_{x}^{2}u+\partial_{y}^{2}u=2-2=0.
\end{equation}
\end{example}

\N{Do Partial Derivatives Commute?}
Answer: not always. The conditions are fairly weak: if
$\partial_{x}z$, $\partial_{y}z$, $\partial_{x}\partial_{y}z$ and
$\partial_{y}\partial_{x}z$ are continuous throughout their
respective domains, then 
\begin{equation}
\partial_{x}\partial_{y}z = \partial_{y}\partial_{x}z.
\end{equation}

\begin{exercise}
Let $u(x,t) = f(x+vt) + g(x-vt)$ where $v\not=0$ is some
constant. Prove
\begin{equation}
\partial_{t}^{2}u(x,t)=v^{2}\partial_{x}^{2}u(x,t).
\end{equation}
\end{exercise}
\begin{exercise}
Let $f(x,y)=\ln(x^{2}+y^{2})$. What is
$\partial_{x}^{2}f(x,y)+\partial_{y}^{2}f(x,y)$? 
\end{exercise}
\begin{exercise}
Consider $g(x,y)=1/\sqrt{x^{2}+y^{2}}$. What is
$\partial_{x}^{2}g(x,y)$? What is $\partial_{y}^{2}g(x,y)$? Is
$\partial_{x}\partial_{y}g(x,y)=\partial_{y}\partial_{x}g(x,y)$? 
\end{exercise}
\begin{exercise}
Let $f(x,y)=3x^{2}+4y^{3}++x^{2}y^{3}+\sin(xy)$. What is
$\partial_{x}f$? What is $\partial_{y}f$?
\end{exercise}
\begin{exercise}
Let $z=\arctan(x^{2}\E^{2y})$. What is $\partial_{x}z$? What is
$\partial_{y}z$? 
\end{exercise}
