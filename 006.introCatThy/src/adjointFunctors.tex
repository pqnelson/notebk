%%
%% adjointFunctors.tex
%% 
%% Made by Alex Nelson
%% Login   <alex@tomato>
%% 
%% Started on  Sun Jul 19 13:55:59 2009 Alex Nelson
%% Last update Sun Jul 19 13:55:59 2009 Alex Nelson
%%

\subsection{Definition via Hom-Set Adjunction}

Consider two functors $F:\ms{D}\to\ms{C}$ and
$G:\ms{C}\to\ms{D}$, and the natural isomorphism
\begin{equation}%\label{eq:}
\Phi:\hom_{\ms{C}}(F-,-)\cong{}\hom_{\ms{D}}(-,G-).
\end{equation}
This specifies a family of bijections for each pair of objects
$X\in\ms{C}$ and $Y\in\ms{D}$ 
\begin{equation}%\label{eq:}
\Phi_{X,Y}:\hom_{\ms{C}}(F(Y),X)\cong{}\hom_{\ms{D}}(Y,G(X)).
\end{equation}
In this situation, we say that $F$ is \define{Left Adjoint} to
$G$ and $G$ is \define{Right Adjoint} to $F$.

\subsection{Definition via Unit and Counit}

We'll introduce the notion of an adjunction as a weaker form of
equivalence. That is, we have two categories $\ms{C},\ms{D}$. We
have a small hierarchy so far of the notion of $\ms{C}$ being
``the same'' as $\ms{D}$. We have the notion they are isomorphic
if there is an isomorphism $F:\ms{C}\to\ms{D}$, which happens if
it is invertible i.e. there is a $G:\ms{D}\to\ms{C}$ such that
$F\circ{G}=\id{\ms{D}}$ and $G\circ{F}=\id{\ms{C}}$.

We can weaken this to the notion of $\ms{C}$ and $\ms{D}$ are
``equivalent'' if there are two functors $F:\ms{C}\to\ms{D}$ and
$G:\ms{D}\to\ms{C}$ such that we have two natural isomorphisms
\begin{equation}%\label{eq:}
F\circ{G}\cong{\id{\ms{C}}}
\end{equation}
and
\begin{equation}%\label{eq:}
\id{\ms{D}}\cong{G\circ{F}}.
\end{equation}
Now, why did we choose to write it this way?

Well, we can weaken the notion of an equivalence of two
categories even further. Instead of demanding that we have a
pair of natural isomorphisms, we can demand we have ``some''
arbitrary pair of natural transformations
\begin{equation}%\label{eq:}
\varepsilon:F\circ{G}\Rightarrow{\id{\ms{C}}}
\end{equation}
and
\begin{equation}%\label{eq:}
\eta:\id{\ms{D}}\Rightarrow{G\circ{F}}.
\end{equation}
We call them the counit and unit (respectively). We demand they
satisfy the demand that
\begin{equation}\label{eq:triangleIdentitiesOne}
F\xrightarrow{\;F\eta\;}F\circ{G\circ{F}}\xrightarrow{\;\varepsilon F\,}F
\end{equation}
and
\begin{equation}\label{eq:triangleIdentitiesTwo}
G\xrightarrow{\;\eta G\;}G\circ{F\circ{G}}\xrightarrow{\;G \varepsilon\,}G
\end{equation}
when composed are the identity natural transformations on $F$ and
$G$ (respectively). We call eqs \eqref{eq:triangleIdentitiesOne}
\eqref{eq:triangleIdentitiesTwo} the \define{Triangle Identities}.
When this happens we say that $F$ is \define{Left Adjoint} to
$G$, and $G$ is \define{Right Adjoint} to $F$.

\subsection{Equivalence of Two Definitions}
This derivation is really inspired from
Baez~\cite{BaezWeek79}. We have
the definition of an ``adjunction'' as two functors
\begin{equation}
L:\ms{C}\to\ms{D}\quad\text{and}\quad{}R:\ms{D}\to\ms{C}
\end{equation}
and a natural isomorphism between $\hom_{\ms{D}}(Lc,d)$ and
$\hom_{\ms{C}}(c,Rd)$. So it's just a pair of functors $L,R$ and
a natural isomorphism.

What happens if we take $c=Rd$? This could only affect one of
three things (either one of the functors or the natural
isomorphism). We see that our natural isomorphism becomes
\begin{equation}
\hom(LRd,d)\cong\hom(Rd,Rd)
\end{equation}
which is interesting. We know there is a special object in
$\hom(Rd,Rd)$, namely the identity morphism
$\id{Rd}:Rd\to{}Rd$. This implies that there is a special object
in $\hom(LRd,d)$ since the two objects are
isomorphic. Specifically we'll denote
\begin{equation}
e_{d}:LRd\to d
\end{equation}
denote this special object.

What is this ``special object''? What does it do, what intuition
should we have when we see it? Well, take $L:\ms{Set}\to\ms{Mon}$
be the functor which associates to each set $S\in\ms{Set}$ the
free monoid generated by it. It behaves in the obvious way, the
elements of $L(S)$ are lists of elements in $S$, and the binary
operator of $L(S)$ simply concatenates two lists together. We'll
let $R:\ms{Mon}\to\ms{Set}$ be the forgetful functor.  It simply
forgets the binary operator, and we have -- to no great surprise
-- the set of elements of the monoid. Now our ``special
morphism'' maps $LRd$ to $d$. Step by step we see that $Rd$ is
the set underlying $d$ and $L(Rd)$ to be the free monoid
generated by $Rd$. So the morphism maps $LRd$ to $d$, what can do
this? Well, our binary operator is usually written tacitly as
just multiplication, so if we look at these lists as strings of
elements of $d$, what could map strings of elements of $d$ to
$d$? The simplest answer would be to evaluate the string of
elements of $d$, that is carry out the multiplication. This
necessarily yields an element in $d$. That is what our ``special
morphism'' does.

In fact, this is not just the ``simplest'' choice of morphisms,
it's fairly (dare I say) ``\emph{natural}'' to choose such a
morphism. It's not too much of a stretch to say that the morphism
$e_{d}$ defines a natural transformation
\begin{equation}
e:LR\Rightarrow{}\id{\ms{D}}
\end{equation}
where $\id{\ms{D}}$ is the identity functor on $\ms{D}$.

We can similarly ask what happens if we take $d=Lc$? Then we have
a natural transformation between $\hom(c,RLc)$ and
$\hom(Lc,Lc)$. As before, we have a ``special'' morphism in
$\hom(Lc,Lc)$ which is $\id{Lc}$. This gives a special object in
$\hom(c,RLc)$. We'll denote this by
\begin{equation}
i_{c}:c\to{}RLc.
\end{equation}
Again, we ask ``What does it do?''

Using notation from the previous example, we have $i_{c}$ taking
a set of ``stuff'' $c$ to the set underlying the free monoid
generated by this ``stuff''.

As before, this yields a natural transformation
\begin{equation}
i:\id{\ms{C}}\Rightarrow{}RL.
\end{equation}
In other words, we end up with a pair of natural transformations
from our definition of adjoint functors. When one sees this
notion of a pair of natural transformations, one should be
reminded of an equivalence of categories. With adjunctions, we
have a sort of weaker form of equivalence. We no longer demand
that the natural transformations are invertible. (Just as a group
is a monoid, so too is an equivalence an adjunction?)

Recall, an equivalence of categories $\ms{C}$ and $\ms{D}$ is
defined as a pair of functors $F:\ms{C}\to\ms{D}$ and
$G:\ms{D}\to\ms{C}$ equipped with a pair of natural
transformations $e:FG\Rightarrow{}\id{\ms{D}}$ and
$i:\id{\ms{C}}\Rightarrow{}GF$ such that these natural
transformations are invertible.

\begin{rmk}
Some people define adoint functors in this manner as a sort of
distant cousin to equivalences, others prefer using the ``pair of
functors equipped with a natural isomorphism'' definition. We see
that we have derived the former from the latter. 
\end{rmk}
