%%
%% cones.tex
%% 
%% Made by Alex Nelson
%% Login   <alex@tomato>
%% 
%% Started on  Mon Jul 20 13:51:20 2009 Alex Nelson
%% Last update Mon Jul 20 13:51:20 2009 Alex Nelson
%%

\subsection{Diagrams}

We have an intuition in our heart of hearts of what a diagram in
category theory is, but is there a rigorous way to describe it?
The answer is ``Yes, there is a rigorous way to describe it.''
What we do is we typically have a category which describes the
shape of the diagram. It is typically finite (or at least small),
we shall denote the shape category by $\mathbb{I}$.

Now, the diagram is actually a functor which embeds the shape
into our category $\ms{C}$. That is, a diagram
$$ D:\mathbb{I}\to\ms{C} $$
is a functor that identifies the diagram in $\ms{C}$.

\subsection{Cones}

We need to first introduce the constant functor before leaping to
cones. Consider a functor $\Delta_{U}:\mathbb{I}\to\ms{C}$ be
such that
\begin{equation}%\label{eq:}
\Delta_{U}\left(A\xrightarrow{\;\;f\;\;}B\right) = U\xrightarrow{\;\;\id{U}\;\;}U
\end{equation}
or in other words, every object is mapped to $U$ and every
morphism is mapped to $\id{U}$. This shouldn't be too surprising
behavior for a functor dubbed ``the constant functor''!

Now, we can define a cone:
\begin{defn}%\label{defn:}
Let $D:\mathbb{I}\to\ms{C}$ be a diagram in $\ms{C}$ and
$\Delta_{U}:\mathbb{I}\to\ms{C}$ be the constant functor.
A \define{Cone over $D$ with Vertex $U$} is a natural
transformation $\Delta_{U}\Rightarrow{D}$ with components (for
all $I\in\mathbb{I}$)
\begin{equation}\label{eq:coneNaturalityCondition}
\vcenter{\xymatrix{
U\ar[d]_{\id{U}}\ar[r]^{P_{I}}&D(I)\ar[d]^{D(f)}\\
U\ar[r]_{P_{I'}}&D(I')
}}
\end{equation}
Observe that this is secretly a triangle with vertex $U$! The
naturality condition encodes all the information about the cone.
\end{defn}

(We can also have the notion of a \emph{cocone} which is dual to
the cone; we just have the \emph{target} of the arrows be $U$
instead of having the \emph{source} of the arrows be $U$.)

Now, the reason for calling it a cone is somewhat clear, it
secretly is a triangle with vertex $U$ after all. But the reason
why it's important is not so clear at the moment. What happens is
that we have a diagram and an object $U$. We basically have
morphisms from $U$ to each object in the diagram, and demand the
resulting larger diagram commutes. But this larger diagram can be
broken up into many smaller diagrams similar to eq
\eqref{eq:coneNaturalityCondition}. Why is this a good thing? Why
should we care about such a gadget that has morphisms from $U$ to
each object in a diagram? It turns out that we can express many
notions (structures we can equip a category with) in category
theory as a cone (or more precisely, a ``limit''). We will now
turn our focus to limits.

\subsection{Limits}

Now in analysis, we typically have limits of sequences of numbers
\begin{equation}%\label{eq:}
\lim_{n\to\infty}x_{n}=x
\end{equation}
or we have a limit of a function
\begin{equation}%\label{eq:}
\lim_{x\to{x_{0}}}f(x)=L.
\end{equation}
We (should) have a good intuition about these procedures. But in
category theory, what can we take the limit of? How about taking
the limit \emph{of diagrams?}

It turns out that a limit for a diagram in $\ms{C}$ is a
universal cone. What do we mean by a ``universal cone''? Well,
universal objects usually means that every other object ``factors
uniquely'' through it. So what would this mean for our precious
cone? It means that if we have a universal cone with vertex $U$,
and another cone with vertex $V$, that $V$ factors through $U$,
i.e. there is a unique morphism $h:V\to{U}$ such that
\begin{equation}%\label{eq:}
\vcenter{\xymatrix{
V\ar[dr]_{f}\ar@{-->}[rr]^{h}&&\ar[dl]^{g}U\\
&X&
}}
\end{equation}
commutes for all $X$. This might seem innocent enough, but it
turns out that many many important notions in category theory can
be expressed as limits or colimits over some vertex and some
simple diagram. We will turn our attention to the rest of the
chapter to examples.

