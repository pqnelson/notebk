%%
%% lecture22.tex
%% 
%% Made by Alex Nelson
%% Login   <alex@tomato3>
%% 
%% Started on  Sun Dec 12 11:32:08 2010 Alex Nelson
%% Last update Sun Dec 12 12:18:03 2010 Alex Nelson
%%
Now we will give another construction of the Verma
module. Really, it will be a general case of our particular
construction called \define{Induced Representations}. We will
talk about representations of associative algebras, it is more
general than representations of Lie Algebras (since we may
construct for any Lie Algebra $\mathscr{G}$ with its universal
enveloping algebra $\mathcal{U}(\mathscr{G})$\thinspace). So
representations of $\mathscr{G}$ and $\mathcal{U}(\mathscr{G})$
are precisely the same. Therefore we will consider
representations of an associative algebra $\mathscr{A}$. To ever
$x\in\mathscr{A}$ we assign a linear operator
\begin{equation}
\widehat{x}\colon V\to V.
\end{equation}
We have
\begin{equation}
\widehat{xy}=\widehat{x}\widehat{y}.
\end{equation}
We would like to say that representations of $\mathscr{A}$ is the
same as a left $\mathscr{A}$-module. An $\mathscr{A}$-module is a
vector space $V$ and we have multiplication by elements of
$\mathscr{A}$ (so we may consider $x\vec{v}$ for
$x\in\mathscr{A}$ and $\vec{v}\in V$). We may think of it as a
vector space over $\mathscr{A}$. We should have the standard
relations for a sort of associativity
\begin{equation}
(xy)\vec{v}=x(y\vec{v}).
\end{equation}
This is precisely what we have if we write
\begin{equation}
x\vec{v}=\widehat{x}\vec{v}.
\end{equation}
We see
\begin{equation}
(xy)\vec{v}=x(y\vec{v})\quad\iff\quad\widehat{xy}=\widehat{x}\widehat{y}
\end{equation}

If we can talk about left modules, we may talk about \emph{right}
modules. What does it mean? Well $\mathscr{A}$-scalar
multiplication occurs on the right, i.e. we have
\begin{equation}
(\vec{v}x)y=\vec{v}(xy)
\end{equation}
We write
\begin{equation}
\vec{v}x=\widetilde{x}\vec{v}
\end{equation}
is it a representation? Not really, observe
\begin{equation}
\widetilde{xy}=\widetilde{y}\widetilde{x}.
\end{equation}
We may say a right module is a representation of
$\mathscr{A}^{op}$, where $\mathscr{A}^{op}$ has multiplication
defined by $x\cdot y=yx$.

\begin{ex}
A left module could be $\mathscr{A}$, but this is also a right
module. Or we may say that $\mathscr{A}$ is a bimodule, i.e. a
left and a right module such that
\begin{equation}
(xa)y=x(ay)
\end{equation}
for all $x,y,a\in\mathscr{A}$.
\end{ex}

Now we would like to define the tensor product of algebras. If we
have a right module $V_{\mathscr{A}}$ and a left module
${}_{\mathscr{A}}W$, both over the associative algebra
$\mathscr{A}$, then we may take the tensor product as modules
\begin{equation}
V_{\mathscr{A}}\mathop{\otimes}\limits_{\mathscr{A}} {}_{\mathscr{A}}W =
\frac{V\otimes W}{Va\otimes W\sim V\otimes aW}
\end{equation}
as vector spaces, and we identify 
\begin{equation}
\vec{v}a\otimes\vec{w}\sim\vec{v}\otimes a\vec{w}
\end{equation}
which is the definition of taking the tensor product over
$\mathscr{A}$. The tensor product describes bilinear functions
\begin{subequations}
\begin{align}
f\colon&V\otimes W\to Z\\
v\otimes w\mapsto f(v,w)
\end{align}
\end{subequations}
be bilinear.

This means that bilinear functions of $\mathscr{A}$-modules would
be such that
\begin{equation}
f(va,w)=f(v,aw).
\end{equation}
A really nice picture occurs when we work with bimodules, or
$(\mathscr{A},\mathscr{B})$-modules.  We have $\mathscr{A}$,
$\mathscr{B}$ be associative algebras. An
$(\mathscr{A},\mathscr{B})$-module is simultaneously a left
$\mathscr{A}$-module \emph{and} a right $\mathscr{B}$-module. The
corresponding operation should commute. We have
\begin{equation}
(av)b=a(vb)
\end{equation}
be such a condition. In particular, what is an
$(\mathscr{A},\mathscr{A})$-module? It is something we called a
``bimodule''. Suppose we have an
$(\mathscr{A},\mathscr{B})$-module
${}_{\mathscr{A}}V_{\mathscr{B}}$ and a
$(\mathscr{B},\mathscr{C})$-module
${}_{\mathscr{B}}W_{\mathscr{C}}$ then we may consider
\begin{equation}
{}_{\mathscr{A}}V_{\mathscr{B}}\mathop{\otimes}\limits_{\mathscr{B}}{}_{\mathscr{B}}W_{\mathscr{C}}
\end{equation}
by identifying
\begin{equation}
vb\otimes w\sim v\otimes bw.
\end{equation}
We can say that $V\mathop{\otimes}\limits_{\mathscr{B}}W$ is an
$(\mathscr{A},\mathscr{C})$-module. We see that
\begin{equation}
a(v\otimes w)=(av)\otimes w.
\end{equation}
Is it possible to do this with the relation
\begin{equation}
vb\otimes w\sim v\otimes bw?
\end{equation}
We assert the multiplication by $a$ is compatible with this
equivalence because, well, it is:
\begin{subequations}
\begin{align}
a(vb\otimes w) &= a(vb)\otimes w\\
&= (av)b\otimes w\\
&\sim av\otimes bw\\
&= a(v\otimes bw)
\end{align}
\end{subequations}

\begin{ex}
Lets take a simple example $\mathscr{B}\subset\mathscr{A}$ a
subalgebra. Suppose we have a $\mathscr{B}$-module
${}_{\mathscr{B}}V$. We would like to get an
$\mathscr{A}$-module. How to do this? Look, we consider
$\mathscr{A}$ our algebra as an
$(\mathscr{A},\mathscr{B})$-module, now we take the tensor
product of these guys over $\mathscr{B}$
\begin{equation}
{}_{\mathscr{A}}\mathscr{A}_{\mathscr{B}}\mathop{\otimes}\limits_{\mathscr{B}}{}_{\mathscr{B}}V
\end{equation}
which is an $\mathscr{A}$-module.
\end{ex}

We have $\mathscr{G}\supset\mathscr{G}'$. Let
$\mathscr{A}=\mathcal{U}(\mathscr{G})$,
$\mathscr{B}=\mathcal{U}(\mathscr{G}')$. It is obvious that
\begin{equation}
\mathscr{A}\supset\mathscr{B}
\end{equation}
Suppose we have a representation of $\mathscr{G}'$ we recall this
is a $\mathscr{G}'$-module, which is precisely the same as a
$\mathcal{U}(\mathscr{G}')$-module; lets call this thing $V$. We
get the
\begin{equation}
\mathcal{U}(\mathscr{G})\mathop{\otimes}\limits_{\mathscr{B}}V
\end{equation}
a $\mathscr{G}$-module. This is precisely the notion of an
\define{Induced Representation of Lie Algebras}.

For the Verma module, we have
\begin{equation}
\mathscr{G}=\mathscr{G}_{+}\oplus\mathscr{H}\oplus\mathscr{G}_{-}
\end{equation}
there is a highest weight vector $\vec{v}$. We have for
$h\in\mathscr{H}$,
\begin{equation}
h\vec{v}=\lambda(h)\vec{v}.
\end{equation}
For
\begin{equation}
\mathscr{G}_{+}\vec{v}=0
\end{equation}
So we write
\begin{equation}
\mathscr{G}'=\mathscr{G}\oplus\mathscr{H}
\end{equation}
which is a subalgebra. The one-dimensional $\mathscr{G}'$-module
is $\Span\{\vec{v}\}$. This gives us a representation of
$\mathscr{G}_{+}$ too. We can write that the Verma module is an
induced module
\begin{equation}\label{eq:thingWeWannaWorkWith}
\mathcal{U}(\mathscr{G})\bigotimes_{\mathcal{U}(\mathscr{G}')}V_{\lambda}
\end{equation}
where $V_{\lambda}$ is precisely the $\Span\{\vec{v}\}$,
$\lambda$ is precisely the highest weight. The only that remains
is to prove that \eqref{eq:thingWeWannaWorkWith} is precisely the
Verma module. We see that
\begin{equation}
\mathcal{U}(\mathscr{G}_{-})\vec{v}=\mathcal{U}(\mathscr{G})\bigotimes_{\mathcal{U}(\mathscr{G}')}V_{\lambda}
\end{equation}
we did prove something before (although perhaps not in this
name): the Poincar\'e-Birkhoff-Witt theorem. We proved something
about the structure of $\mathcal{U}(\mathscr{G})$, namely
\begin{equation}
\mathscr{G}=\{E_{1},\dots,E_{n}\}
\end{equation}
so \marginpar{Poincar\'e-Birkhoff-Witt theorem}
\begin{equation}
\mathcal{U}(\mathscr{G})=\{c_{0}+c^{i}E_{i}+c^{ij}E_{i}E_{j}+\cdots\}
\end{equation}
but such a representation is not unique: it may be made unique by
(1) demanding $i\leq j$; or (2) that $c^{ij}$, $c^{ijk}$, and all
the other coefficients, be symmetric in its indices. This is the
Poincar\'e-Birkhoff-Witt theorem. Then it is clear that
$\mathcal{U}(\mathscr{G})$ is on the left hand side of the tensor
product; what is not clear is this identity. We speak of all
possible roots $E=\{e_{i},f_{i},h_{i}\}$.
