%%
%% lecture10.tex
%% 
%% Made by Alex Nelson
%% Login   <alex@tomato3>
%% 
%% Started on  Mon Jun 14 15:20:31 2010 Alex Nelson
%% Last update Mon Jun 14 15:55:14 2010 Alex Nelson
%%
We will finish the construction of the Lie group from the Lie
algebra. There is an important formula called the
\define{Baker-Campbell-Hausdorff formula.} Recall we considered a
correspondence between a curve in the Lie group and a curve in
the Lie algebra. The question is can we only consider curves
$\exp(tA)$ one-parameter families of the group; in the
neighborhood of the identity, the correspondence is
one-to-one. It would follow that multiplication in the group
\begin{equation}
e^{A}\cdot e^{B}=e^{C(A,B)}
\end{equation}
goes to some operation in the algebra. We have
\begin{equation}
C(A,B) = \log(e^{A}\cdot e^{B})
\end{equation}
we know
\begin{equation}
e^{A}=\sum^{\infty}_{n=0}\frac{A^{n}}{n!},
\end{equation}
but there is no such series for the logarithm. There is some
expansion for the logarithm \emph{in some neighborhood.} We need
to be careful about the order of multiplication, we can express
the series in terms of the commutators
\begin{equation}
C(A,B)=A+B+\frac{1}{2}[A,B]+\frac{1}{12}[A,[A,B]]-\frac{1}{12}[B,[A,B]]-\frac{1}{24}[B,[A,[A,B]]]+\cdots.
\end{equation}
Really, the most important part of this formula is
\begin{equation}
C(A,B)=A+B+\frac{1}{2}[A,B]+\cdots
\end{equation}
We expect \emph{ab initio} that $C(A,B)\in\lie(G)$ is in the Lie
algebra. This formula permits us to construct a \define{local Lie group},
i.e. a group with induced group operation in the neighborhood of
the unit element.

There is a special situation with this being global. Consider a
nilpotent Lie algebra. The Baker-Campbell-Hausdorff formula
becomes a polynomial with finitely many terms, which gives rise
to a Lie group from the Lie algebra.

\subsection{Representations of \texorpdfstring{$\mathfrak{sl}(2)$}{sl2}}

This is really quite important in math and in physics. We know
\begin{equation}
\Bbb{C}\otimes\frak{su}(2)\iso\frak{sl}(2).
\end{equation}
So representations of $\frak{su}(2)$ may be studies by
representations of $\frak{sl}(2)$. We know
\begin{equation}
SO(3)\iso SU(2)/\Bbb{Z}_{2},
\end{equation}
so we can study representations of $\frak{so}(3)$ too!

For $\frak{sl}(2)$, we have generators $e$, $f$, $h$ with the
commutation relations
\begin{subequations}
\begin{align}
[e,f]&=h\\
[h,e]&=2e\\
[h,f]&=-2f.
\end{align}
\end{subequations}
We would like to describe all representations of
$\frak{sl}(2)$. For a general Lie Algebra, we take its Cartan
subalgebra $\frak{h}\subset\lie(G)$. Here $h$ is a generator of
the Cartan subalgebra. We will take any rep
\begin{equation}
\varphi\colon\mathscr{G}\to\frak{gl}(n).
\end{equation}
So
\begin{equation}
f\mapsto\varphi(f)=F,\quad
e\mapsto\varphi(e)=E,\quad
h\mapsto\varphi(h)=H,
\end{equation}
and the commutation relations are
\begin{subequations}
\begin{align}
[E,F]&=H\label{eq:lec10:comm1}\\
[H,E]&=2E\label{eq:lec10:comm2} \\
[H,F]&=-2F.\label{eq:lec10:comm3}
\end{align}
\end{subequations}
We need to find 3 such matrices. We will consider eigenvectors of
$H$ called \define{Weight Vectors}
\begin{equation}
H\vec{x}=\lambda\cdot\vec{x}.
\end{equation}
Once we have one weight vector $\vec{x}$, we can construct others
via use of $E$ and $F$. We have from eq \eqref{eq:lec10:comm2}
\begin{equation}
HE=E(H+2)
\end{equation}
which, when applied to the weight vector, yields
\begin{equation}
HE\vec{x}=E(H+2)\vec{x}=(\lambda+2)E\vec{x}.
\end{equation}
This implies that $E\vec{x}$ is also an eigenvector of $H$ with
eigenvalue $\lambda+2$. Thus we have infinitely many weight
vectors, right? Well, this is {\bf wrong} since $E\vec{x}$ could
vanish! If $E\vec{x}=0$, then $\vec{x}$ is called the
\define{Highest Weight Vector}.

We also have
\begin{equation}
H(F\vec{x})=(\lambda-2)F\vec{x}
\end{equation}
by the exact same reasoning. This means that $F\vec{x}$ is also a
weight vector. We will now describe all finite dimensional
representations of $\frak{sl}(2)$.

\begin{rmk}
In finite dimensional representations, $H$ always has an eigenvector.
\end{rmk}

Lets apply $E$ to $\vec{x}$ many times, so we get new
eigenvectors. Then at some moment
\begin{equation}
HE^{k}\vec{x}=0
\end{equation}
for some $k$ since we cannot have an infinite number of distinct
eigenvectors. Let
\begin{equation}
\vec{v}:=E^{k-1}\vec{x}
\end{equation}
be the highest weight vector, so
\begin{equation}
E\vec{v}=0.
\end{equation}
Let
\begin{equation}
H\vec{v}=m\vec{v}.
\end{equation}
Let
\begin{equation}
\vec{v}_{k}=F^{k}\vec{v},
\end{equation}
we know
\begin{equation}
H\vec{v}_{k}=(m-2k)\vec{v}_{k}.
\end{equation}
This is a weight vector. We have
\begin{equation}
F\vec{v}_{k}=\vec{v}_{k+1}
\end{equation}
by definition. We should apply
\begin{equation}
E\vec{v}_{k}=EF\vec{v}_{k-1}=(FE+H)\vec{v}_{k-1}.
\end{equation}
We can guess that $E$ \emph{raises} the eigenvalue. That is
\begin{equation}
E\vec{v}_{k}=\gamma_{k}\vec{v}_{k}
\end{equation}
where $\gamma_{k}$ is some factor.

We can compute
\begin{subequations}
\begin{align}
E\vec{v}_{k} &= FE\vec{v}_{k-1}+H\vec{v}_{k-1}\\
&= F(\gamma_{k-1}\vec{v}_{k-2})+(m+2-2k)\vec{v}_{k-1}\\
&= \gamma_{k-1}F\vec{v}_{k-2}+(m+2-2k)\vec{v}_{k-1}\\
&= (m+2-2k+\gamma_{k-1})\vec{v}_{k-1}\\
&= \gamma_{k}\vec{v}_{k-1}
\end{align}
\end{subequations}
This implies that
\begin{equation}
\gamma_{k}=\gamma_{k-1}+m+2-2k
\end{equation}
a recursion relation which permits us to compute $\gamma_{k}$, an
arithmetic progression./ We have our representation be
irreducible if and only if
\begin{equation}
{\rm span}\{\vec{v}_{k}\}\iso\Bbb{C}^{n}.
\end{equation}
We have everything, we just need to compute the $\gamma_{k}$
constants. It turns out that the weights range from $m$, $m-2$,
..., $-m$.
