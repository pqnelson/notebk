%%
%% lecture12.tex
%% 
%% Made by Alex Nelson
%% Login   <alex@tomato3>
%% 
%% Started on  Sun Jun 20 11:23:40 2010 Alex Nelson
%% Last update Sun Jun 20 12:07:27 2010 Alex Nelson
%%
Today we will talk about representations, and this is the main
part of the class. First some things about representations. We
havea representation be a morphism $G\to{\rm GL}(n,\Bbb{C})$ for Lie
groups, a Lie algebra representation is a morphism
$\mathscr{G}\to\mathfrak{gl}(n,\Bbb{C})$. How do we classify
representations? Well, the most important group representations
are orthogonal or unitary in the sense of using an inner product
$\<\cdot,\cdot\>$, i.e. morphisms
\begin{equation}
G\to{\rm O}(n)\quad\mbox{and}\quad G\to{\rm U}(n)
\end{equation}
respectively. However, in the language of Lie algebras
\begin{equation}
A = 1 + a
\end{equation}
and the condition is as follows
\begin{equation}
\<ax,y\>+\<x,ay\>=0,
\end{equation}
which for algebras results in morphisms
\begin{equation}
\mathscr{G}\to\mathfrak{so}(n)\quad\mbox{and}\quad\mathscr{G}\to\mathfrak{u}(n)
\end{equation}
respectively for orthogonal and unitary representations.

\begin{thm}\label{lecture12:thm:reducibility}
Unitary and orthogonal representations are completely reducible
(i.e. a direct sum of irreps).
\end{thm}
\begin{lem}\label{lecture12:lemma:invariantSubspace}
If $W$ is an invariant subspace, then $W^{\bot}$ is also an
invariant subspace.
\end{lem}
\begin{proof}[Lemma \ref{lecture12:lemma:invariantSubspace}]
We have $AW\subset W$, where $A=\varphi(X)$ for some $X\in G$. If
$x\in W^{\bot}$,
\begin{equation}
\<x,w\>=0
\end{equation}
for all $w\in W$. But 
\begin{equation}
\<x,Aw\>=0
\end{equation}
since $Aw\in W$. But
\begin{equation}
\<x,Aw\>=\<A^{\dagger}x,w\>=0
\end{equation}
if and only if
\begin{equation}
A^{\dagger}x\in W^{\bot}
\end{equation}
for every $A$ in our representations. But
$A^{\dagger}=A^{-1}=\varphi(X)^{-1}=\varphi(X^{-1})$, so we are
done. If $\<x,x\>\geq0$ for all $x$, then $V=W\oplus W^{\bot}$.
\end{proof}
\begin{proof}[Theorem \ref{lecture12:thm:reducibility}]
We have by our lemma, if our representation is irreducible we're
done; otherwise, we write
\begin{equation}
V=W\oplus W^{\bot}.
\end{equation}
We iterate on our summands, until we end up writing
\begin{equation}
V = \bigoplus_{i\in I}V_{i}
\end{equation}
for our representation as a direct sum of mutually orthogonal
irreps. For finite dimensional $V$, there is only finitely many
recursions; \textbf{note} this is the first time finite
dimensionality is used. There is an analogous notion for infinite
dimensionality, but we need to use the direct integral.
\end{proof}

Now are all representations unitary or orthogonal?
Or...equivalent to unitary or orthogonal representations? The
answer is ``No'' for the simple reason they are not completely
reducible!
\begin{ex}
Consider the one-dimensional Lie algebra $\Bbb{R}$. It has one
generator $e$, and its commutator
\begin{equation}
[e,e] = 0.
\end{equation}
Now to construct a representation of this, how to do this? Send
this generator to somewhere satisfying the commutation
relations...but they are \emph{always} satisfied. So
representations of this Lie algebra are merely all linear operators.
\end{ex}
We should classify these representations. Two representations are
``involutive'' if they are, as matrices, similar
\begin{equation}
E' = A^{-1}EA.
\end{equation}
Every matrix may be written in Jordan normal form, and that's it!
We've classified everything. We get a series of eigenspaces
embedded in each other, but no direct sum of invariant subspaces
for the simple reason every invariant subspace should have an
eigenvector. So this representation \emph{is not} equivalent to a
unitary or orthogonal representation, and it is not completely
irreducible. Only when all the Jordan cells are one dimensional,
then the representation is completely reducible.

The corresponding Lie group to $\Bbb{R}$ is
$\Bbb{R}^{\times}_{>0}$, but there is one more group which
corresponds to the same Lie algebra viz. U(1). Geometrically U(1)
is a circle, and its elements are of the form
$\exp(i\varphi)$...there is a whole in this circle, it's not
simply connected. So the reperesentations of the Lie algebra
$\Bbb{R}$ does not coincide with representations of the Lie group
U(1). They are related but do not coincide. This group has
representations that are equivalent to unitary representations,
moreover are completely reducible; if $z=\exp(i\varphi)$, then
the representations are given by
\begin{equation}
z\mapsto z^{n}.
\end{equation}
How to prove all this stuff?

First how to prove all representations of this group are
equivalent to unitary ones. Lets consider any representation
\begin{equation}
\varphi\colon {\rm U}(1)\to V
\end{equation}
and introduce an inner product on $V$,
i.e. $\<\cdot,\cdot\>$. There are two possibilities: inner
product is invariant (we're happy, the representation is
unitary), or it's not invariant (it's not equivalent to a unitary
representation). Then we can make the inner product invariant,
namely by introducing a new inner product
\begin{equation}
\<\!\< v_{2},v_{1}\>\!\> = \int \<\varphi(g)v_{1},\varphi(g)v_{2}\>dg
\end{equation}
which is already invariant with respect to the group $G$. Namely
\begin{subequations}
\begin{align}
\<\!\<\varphi(h) v_{2},\varphi(h)v_{1}\>\!\> &= \int \<\varphi(g)\varphi(h)v_{1},\varphi(g)\varphi(h)v_{2}\>dg\\
&=\int\<\varphi(gh)v_{2},\varphi(gh)v_{1}\>dg\quad\mbox{g'=gh,
  dg'=dg}\\
&=\int\<\varphi(g')v_{2},\varphi(g')v_{1}\>dg'\\
&=\<\!\<v_{2},v_{1}\>\!\>
\end{align}
\end{subequations}
For us $g=\exp(i\varphi)$,
$gh=\exp(i\varphi)\exp(i\theta)=\exp\big(i(\varphi+\theta)\big)$
is merely a shift. So we need a measure on the circle which is
invariant with respect to translations; we have it. We proved
more, really, we almost proved the theorem:
\begin{thm}
Every complex representation of a compact group is equivalent to a
unitary representation.
\end{thm}
The proof is exactly the same, the only thing missing is a lemma.
\begin{lem}
Every compact group can be equipped with an invariant measure
(measure invariant with respect to shifts).
\end{lem}
Really a measure invariant under right (or left) shifts always
exists, but may not be finite for noncompact groups. To fix the
measure, fix it in the tangent spaces, but we know the tangent
spaces for Lie groups! For compact groups, left invariant measure
coincides with right measure, and are finite.
