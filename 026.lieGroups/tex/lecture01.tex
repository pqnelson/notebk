%%
%% lecture01.tex
%% 
%% Made by Alex Nelson
%% Login   <alex@tomato3>
%% 
%% Started on  Tue Jan 19 10:41:41 2010 Alex Nelson
%% Last update Tue Jan 19 11:05:10 2010 Alex Nelson
%%

Lie algebras and Lie groups, and their representations, will be
the subject of the quarter. We  will start with an outline for
the course.

First of all, an explanation of what is a Lie group. Well, group
theory is simply a theory of symmetry. We've studied mostly
finite and discrete symmetries, but in the real world symmetries
are continuous. For example in 4-dimensional space, everything is
invariant to translations labeled by elements of $\Bbb{R}^{3}$
(this is a continuous symmetry). Everything is also invariant
under rotations (symmetry denoted by SO(3)). So really the study
of symmetries should be the study of continuous groups. First of
all, group of symmetries should be a topological group. (It is a
group object in the category of topological spaces.) We have a
notion of continuity therefore.
\begin{defn}
A \define{Topological Group} consists of a topological space $M$
equipped with
\begin{enumerate}
\item a continuous mapping $\mu\colon M\times M\to M$ called the
  ``Law of composition'';
\item a continuous mapping $e\colon{\bf 1}\to M$ called the
  ``Identity Element'';
\item a continuous mapping $\xi\colon M\to M$ called the
  ``Inversion Operation''.
\end{enumerate}
\end{defn}
We can further demand that $M$ is a manifold. So if we have a
group $G$, t hen we can take a neighborhood of the unit element
$e\in U\subseteq G$ where $U\sim\Bbb{R}^{n}$ is ``topologically
equivalent'' (i.e. we can introduce a ``good'' coordinate system
in $U$ denoted by $x^1$, ..., $x^n$ which are continuous and have
continuous inverses).

A manifold is a topological space where every point $x\in M$ has
a neighborhood $U_{x}\subseteq M$ which is equivalent to
$\Bbb{R}^{n}$. We have a preferred point, namely the unit point
(identity element of the group), since we have \emph{shifts}
\begin{equation}
T_{g}(x)=g\cdot x.
\end{equation}
These shifts are continuous topological transformations with
continuous inverses, a continuous identity, etc. We have 
\begin{equation}
T_{g}(e)=g\cdot e=g \qquad\forall g\in G.
\end{equation}
We have coordinates in the neighborhood of any point. A Lie group
is a topological group with a coordinizable neighborhood at every
point.

Note that this is not a good definition. The coordinates of
$x\cdot y$ are coordinates of these two factors
$f^{j}(x^1,\ldots,x^n,y^1,\ldots,y^n)$ continuous, but that's not
enough. We would like these functions to be differentiable
(moreover smooth, i.e. $C^{\infty}$). This is not included in the
definition of a Lie group. First could we correct our coordinates
to be differentiable (there is a theorem which says we
can). Second, we'd like to incorporate smoothness into the
definition of a Lie group. A Lie group is then a topological
group with a neighborhood at each point with coordinates
permitting some smooth structure.

Now what of Lie algebras. If the Lie group is connected, if we
know the group in the neighborhood of $e$ (the identity element),
then we know it everywhere. The main thing is we can take
infinitesimally small neighborhoods\footnote{Do not worry about
  what this rigorously means, we will use the tools of
  differential geometry to make such a notion explicit and
  rigorous.} of $e$ which is how we get a Lie algebra.

Take the group $\GL{n}$. It is easy to see it is a Lie group. If
we take any closed subgroup $G\subset\GL{n}$, then we will get a
Lie group. It is a kind of submanifold of $\GL{n}$. We can
consider the tangent space of $e$ in this submanifold, and it
turns out it is a Lie algebra. So it is a vector space closed
under a commutator operation. This is the Lie algebra of the
group $G$, denoted $\Lie(G)$.

\begin{defn}
A \define{Lie Algebra} consists of a vector space $V$
equipped with a commutator
\begin{equation}
[\cdot,\cdot]\colon V\times V\to V
\end{equation}
such that, for all $a,b,c\in V$, we have
\begin{enumerate}
\item distributivity $[a,b+c]=[a,b]+[a,c]$;
\item anticommutativity $[a,b]=-[b,a]$;
\item Jacobi identity $[a,[b,c]]+[b,[c,a]]+[c,[a,b]]=0$.
\end{enumerate}
\end{defn}
\begin{prop}\label{prop:lieAlgebraMorphism}
For every Lie group we can construct a corresponding Lie
algebra. Moreover, if $\Phi\colon G\to G'$ is a Lie group
morphism, then we can construct a Lie algebra morphism $\phi\colon\Lie(G)\to\Lie(G')$.
\end{prop}

\begin{quest}
If we have a Lie algebra, does it come from a Lie group? Does it
only come from only one Lie group?
\end{quest}
The answer is that a finite dimensional Lie algebra gives rise to
a simply connected Lie group corresponding to the Lie
algebra. Omitting the criteria of finite dimensionality, we don't
have an answer --- no notion of infinite dimensional Lie groups
currently exist!

We have, lastly, a notion of representation. It is very simple,
namely a group homomorphism
\begin{equation}
\rho\colon G\to\GL{n}.
\end{equation}
By proposition \ref{prop:lieAlgebraMorphism} this has a
corresponding  morphism of Lie algebras $\Lie(G)\to\Lie(\GL{n})$
which is the representation of Lie algebras. How to restore, how
to classify the representations of Lie algebras? It is very easy,
contrasted to asking the same question for Lie groups.

We will mostly work with compact Lie groups, classical Lie
groups, and their representations. We will also consider infinite
dimensional Lie algebras. We will consider applications to
physics.

A Kac-Moody algebra is in general infinite dimensional, given by
some commutation relations. Every classical Lie Algebra may be
considered as a Kac-Moody algebra. Every Lie algebra of a compact
Lie group is a Kac-Moody algebra, to be more precise.

Semisimple and reductive Lie algebras are very closely related ot
the theory of compact Lie algebras of compact Lie groups. This is
Hermann Weyl's so-called unitary trick. By default we will
consider Lie algebras over $\Bbb{R}$, i.e. consider the
underlying vector space to be over $\Bbb{R}$. We can consider
over any field (viz. over $\Bbb{C}$). At some moment we may
switch to work over $\Bbb{C}$, as we are interested in complex
representations.
