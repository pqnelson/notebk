%%
%% lecture23.tex
%% 
%% Made by Alex Nelson
%% Login   <alex@tomato3>
%% 
%% Started on  Sun Dec 12 12:21:40 2010 Alex Nelson
%% Last update Sun Dec 12 16:48:30 2010 Alex Nelson
%%
\noindent\ignorespaces %
Today we will talk about spinor representations.

Now we wil repeat what we know about $\mathfrak{sl}(n)$. For this
Lie algebra, we can construct all ireeducible representations as
tensor representations. What does this mean? We have the
fundamental representation $V$, we can take tensor powers
$V\otimes\cdots\otimes V$ and there are a lot of invariant
subspaces there. All irreducible representations are equivalent
to representations in one of these invariant subspaces, we can
give a more precise result. The algebra $\mathfrak{sl}(n)$ has
Cartan subalgebra of rank $n-1$, we can draw the Dynkin diagram
for $\ClassicalGroup{A}_{n-1}$ as
\begin{center}
\includegraphics{img/LieImg.8}
\end{center}
We assign to each node a number. When we take $n-2$ nodes to be
zero, and one node to be 1, we can get an elementary representation.
This corresponds to $\Antisymmetric^{k}V$. Here we have the
highest weight vector
\begin{equation}
\vec{v}_{k}\in\Antisymmetric^{k}V
\end{equation}
then we take
\begin{equation}
\bigotimes(\Antisymmetric^{k}V)^{\otimes
  n_{k}}\quad\mbox{and}\quad\bigotimes \vec{v}_{k}^{\otimes n_{k}}
\end{equation}
will be the highest weight vector with weight
$(n_{1},\dots,n_{k},\dots)$. This representation is reducible,
but we can get an irreducible part.

We can restrict our attention to any subgroup of $\SL{n}$, in
particular $\SO{n}$, or more precisely $\mathfrak{so}(n)$. Not
all representations of $\mathfrak{so}(n)$ are tensor
representations, not all of them may be embedded in this tensor
representation. Remember for $\ClassicalGroup{D}_{n}=\SO{2n}$ the
Dynkin diagram is of the form
\begin{center}
\includegraphics{img/LieImg.4}
\end{center}
and for $\ClassicalGroup{B}_{n}=\SO{2n+1}$ the Dynkin diagram is
\begin{center}
\includegraphics{img/LieImg.6}
\end{center}
There are some special nodes. We have really from the left
$(n-2)$ nodes in $\ClassicalGroup{D}_{n}$:
\begin{center}
\includegraphics{img/LieImg.9}
\end{center}
the solid nodes are precisely when we get the same situation as
$\ClassicalGroup{A}_{n-2}$. For the special nodes, putting 1 as
the value on the black nodes gives special representations called
\define{Spinor Representations}. We want to construct spinor
representations. The main tool will be Clifford Algebras.

What is the Clifford Algebra? It is a unital associative algebra
with generators $e_{1}$, \dots, $e_{n}$ and relations
\begin{equation}
e_{\alpha}e_{\beta}+e_{\beta}e_{\alpha}=2\eta_{\alpha\beta}\cdot\1
\end{equation}
where $\eta_{\alpha\beta}$ are numbers and form a symmetric
matrix
\begin{equation}
\eta_{\alpha\beta}=\eta_{\beta\alpha}.
\end{equation}
We will require the matrix be degenerate
\begin{equation}
\det(\eta)\not=0
\end{equation}
but really we can impose the opposite condition
\begin{equation}
\eta=0
\end{equation}
and we get what is called a \define{Grassmann Algebra}. We can
reduce everything to these two opposite conditions. We can
consider any field, but we will work with $\CC$.

What is important is that $\eta_{\alpha\beta}$ may be
diagonalized. For complex numbers this shows all Clifford
Algebras of a given dimension are isomorphic. So we may choose
$\eta_{\alpha\beta}$ as we'd like. And we choose
\begin{equation}
\eta = \begin{bmatrix} 0 & 1\\
1 & 0
\end{bmatrix}
\end{equation}
when $n=2m$ (for some $m\in\NN$). We have two sorts of
generators, lets denote generators by symbols $a^{\dagger}_{i}$,
$a_{i}$. Then the defining relations will be of the following
form
\begin{subequations}
\begin{align}
a_{i}a_{j}^{\dagger}+a_{j}^{\dagger}a_{i}=2\delta_{ij}\\
a_{i}^{2}=0
\end{align}
\end{subequations}
which gives us
\begin{equation}
a_{i}a_{j}+a_{j}a_{i}=a_{i}^{\dagger}a_{j}^{\dagger}+a_{j}^{\dagger}a_{i}^{\dagger}=0.
\end{equation}
Physicists give a different name for this stuff, referring to it
as \define{Canonical Anticommutation Relations}, and $a_{i}$,
$a_{i}^{\dagger}$ play the role of creation and annihilation
operators. What may be said of the representations for this
Clifford Algebra. It is extremely simple: when $n=2m$, $\cliff(n)$
has only one irreducible representation. By the way, Physicists
also ahve another name for representations of $\cliff(n)$:
representations of Clifford Algebras are called \define{Dirac
Matrices}. We are sending
\begin{equation}
e_{\alpha}\mapsto\Gamma_{\alpha}
\end{equation}
generators to matrices which obey
\begin{equation}
\Gamma_{\alpha}\Gamma_{\beta}+\Gamma_{\beta}\Gamma_{\alpha}=2\eta_{\alpha\beta}\cdot\1.
\end{equation}
We would now like to prove: if we require irreducibility, then we
will have only one irreducible representation of Clifford
Algebras.

So how to prove this? It is very simple and more or less based on
the idea of the highest weight vector, although in this situation
it is named the vacuum vector $\phi$\marginpar{Vaccum Vector
$\phi$ generalized version of highest weight vector}. The Fock
Vacuum vector $\phi$ is annihilated by all $a_{i}$
\begin{equation}
a_{i}\phi = 0
\end{equation}
for all $i$. Here we can consider $a_{i_{1}}^{\dagger}\cdots
a_{i_{n}}^{\dagger}\phi$, we may take arbitrary linear
combinations of these guys. We can define the action of $a$ on
these guys. We just use the relations
\begin{equation}
a_{i}a_{j}^{\dagger}=2\delta_{ij}\1-a_{j}^{\dagger}a_{i}
\end{equation}
and we are golden.

We have some questions. Maybe this vacuum vector does not exist
at all. We construct this sum
\begin{equation}
\widehat{N}=\sum_{i}a_{i}^{\dagger}a_{i}
\end{equation}
observe that we may consider eigenvectors of $\widehat{N}$:
\begin{equation}
\widehat{N}\psi=N\psi
\end{equation}
which always exist for non-trivial finite-dimensional
representations. We see
\begin{equation}
a_{i}^{\dagger}\colon N\mapsto N+1 \quad\mbox{and}\quad
a_{i}\colon N\mapsto N-1.
\end{equation}
We have
\begin{subequations}
\begin{align}
\widehat{N}(a_{\alpha}\psi)
&= \sum_{i}a_{i}^{\dagger}a_{i}a_{\alpha}\psi\\
&= a_{\alpha}(N-1)\psi
\end{align}
\end{subequations}
We cannot go down indefinitely, and the number operator does
indeed exist. The representation is irreducible iff the vacuum
vector is unique (it is more or less obvious, if we had
reducibility we'd have another vacuum vector). It is very easy to
show that there is only one irreducible representation, and all
other representations are direct sums of irreducible
representations.

\begin{rmk}
Hoiw to realize this representation? We can consider the
Grassmann algebra with $n$ variables, we have ``functions of
anticommuting variables'' in the jargon of physicists. We may
multiply by $\alpha_{i}$ and differentiate with respect to them
$\partial/\partial\alpha_{i}$. Differentiation amounts to
``canceling after moving to the left.'' These two operators give
a Clifford Algebra.
\end{rmk}

The question is why do we need this? Elements of the orthogonal group
generate automorphisms of the Clifford algebra. What does this
mean? It means if we take generators $e_{i}$ of the Clifford
algebra, and we use the transformation
\begin{equation}
\widetilde{e}_{j} = {a_{j}}^{i}e_{i}
\end{equation}
and we assume ${a_{j}}^{i}$ is an orthogonal matrix, then it is
clear that the commutation relations for $\widetilde{e}$ are
precisely the same as for $e$. We see
\begin{equation}
\widetilde{e}_{i}\widetilde{e}_{j}+\widetilde{e}_{j}\widetilde{e}_{i}=2\eta_{ij}\cdot\1
\end{equation}
In other words, if we have Dirac Matrices, then we can obtain
``new'' Dirac matrices by using precisely the same orthogonal
transformation:
\begin{equation}
\widetilde{\Gamma}_{i} = {a_{i}}^{j}\Gamma_{j}.
\end{equation}
We only have one irreducible representation, so
\begin{equation}
\widetilde\Gamma_{i}\sim\Gamma_{i}
\end{equation}
should be similar. Iss this matrix unique? That is we have
matrices $U_{a}$ indexed by elements of the orthogonal group such that
\begin{equation}
\widetilde\Gamma_{i}=U_{a}\Gamma_{i}U_{a^{-1}},
\end{equation}
we can replace
\begin{equation}
U_{a}\mapsto\lambda U_{a}.
\end{equation}
We demand
\begin{equation}
U_{a}U_{b}=\lambda U_{ab}
\end{equation}
for the simple reason that we may perform 2 change of
coordinates. This is a small problem, the $U_{a}$ speceify a
representation of the orthogonal group called the \define{Spinor
Representation} of the orthogonal group. It is a projecctive
representation due to this pesky $\lambda$.
