%%
%% lecture15.tex
%% 
%% Made by Alex Nelson
%% Login   <alex@tomato3>
%% 
%% Started on  Fri Dec 10 11:39:14 2010 Alex Nelson
%% Last update Fri Dec 10 12:54:35 2010 Alex Nelson
%%

Today we will start with some general examples. First some simple
constructions of representations which are quite general. We will
consider a representation
\begin{equation}
\varphi\colon G\to\GL{V}.
\end{equation}
If we have one representation, we may consider manyt others
related to it. We may use any natural construction, any functor,
will give you something. For example we may consider the dual
space 
\begin{equation}
V^{*}=\left\{ v\colon V\to\Bbb{F}\right\}
\end{equation}
This is a contravariant functor. Remember $\varphi(g)\in{\rm
  GL}_{n}$ if $V$ is finite dimensional; duality is related by
the transpose $\varphi(g)^{T}$, we may ask ourselves if it is a
representation?

We see immediately \emph{no it isn't!} Because we may say the
transpose
\begin{equation}
\big(\varphi(g)\varphi(h)\big)^{T}\not=\varphi(g)^{T}\varphi(h)^{T}
\end{equation}
therefore we do not have a representation. It is simple to cure
this, we take
\begin{equation}
\big(\varphi(g)^{T}\big)^{-1}=\big(\varphi(g)^{-1}\big)^{T}
\end{equation}
which is the \define{Dual Representation}, i.e.\ the
representation on the dual space. We demand then that
\begin{equation}
\big(\varphi(g)^{-1}\big)^{T}=\varphi(g^{-1})^{T}
\end{equation}
and then the character of the dual representation is
\begin{equation}
\chi_{\text{dual}}(g)=\tr\big(\varphi(g^{-1})^{T}\big)=\tr\big(\varphi(g^{-1})\big)
\end{equation}
so
\begin{equation}
\chi_{\text{dual}}(g)=\chi(g^{-1}).
\end{equation}

There is another operation that is important, namely taking the
tensor product. Suppose $V$ has basis $(v_{1},\dots,v_{m})$ and
$W$ has basis $(w_{1},\dots,w_{n})$, then $V\otimes W$ has basis 
\begin{equation}
(v_{1}\otimes w_{1},\dots, v_{1}\otimes w_{n},\dots,v_{m}\otimes
  w_{1},\dots, v_{m}\otimes w_{n})
\end{equation}
and a vector in $V\otimes W$ is of the form
\begin{equation}
z = z^{ij}v_{i}\otimes w_{j} = \sum_{i=1}^{m}\sum_{j=1}^{n}z^{ij}v_{i}\otimes w_{j}
\end{equation}
where we use Einstein summation conventions where indices
upstairs are summed over the indices downstairs. The dependence
on the choice of basis is fictitious. If we have a change of
coordinates in $V$ have the components transform by
\begin{equation}
x^{i}\mapsto\widetilde{x}^{i} = {a^{i}}_{j}x^{j}
\end{equation}
and we consider some arbitrary vector
\begin{equation}
v=x^{j}v_{j}\quad\text{in }V
\end{equation}
and if we do likewise consider a change of coordinates in $W$ by
\begin{equation}
y^{l}\mapsto\widetilde{y}^{l}={b^{l}}_{k}y^{k}
\end{equation}
where we implicitly sum over $k$, then we have
\begin{equation}
w=y^{k}w_{k}
\end{equation}
describe an arbitrary element. What is the transformation in the
coordinates of the tensor product? It is very simple. We obtain
them by
\begin{equation}
\widetilde{z}^{il}={a^{i}}_{j}{b^{l}}_{k}z^{jk}
\end{equation}
so if
\begin{equation}
a=a(g)\quad\mbox{and}\quad b=b(g)
\end{equation}
for some group element $g\in G$, then
\begin{equation}
a\otimes b=(a\otimes b)(g)
\end{equation}
depends on $g$ too. This gives rise to a tensor product of
representations, which is a representation by functoriality.

If we have
\begin{equation}
\varphi\colon G\to\GL{V}
\end{equation}
and
\begin{equation}
\psi\colon G\to\GL{W}
\end{equation}
be representations, then we have the tensor product of
representations as
\begin{equation}
(\varphi\otimes\psi)_{g}(v\otimes w)=\big(\varphi(g)v\big)\otimes\big(\psi(g)w\big).
\end{equation}
What about vectors that are not basis vectors? We can use
distributivity, if
\begin{equation}
v=x^{i}v_{i}\quad\mbox{and}\quad w=y^{j}w_{j}
\end{equation}
then by definition
\begin{equation}
v\otimes w=x^{i}y^{j}(v_{i}\otimes w_{j}).
\end{equation}
In other words, if $V$ and $W$ are $G$-modules, then $V\otimes W$
is a $G$-module. We may iterate for as many $G$-modules tensored
together as possible. We may recall 
\begin{equation}
W\otimes V\iso V\otimes W
\end{equation}
naturally.

We may construct more representations via some gadget called an
\define{Intertwiner} which is a morphism of $G$-modules
(i.e. preserves commutator, group operation). Sometimes we use
shorthand
\begin{equation}
\varphi_{g}v = gv
\end{equation}
for the group action. Then a morphism is
\begin{equation}
\alpha(gv)=g(\alpha v).
\end{equation}
If an intertwiner is invertible, we have an equivalence of
representations. 

If we consider $V\otimes V$, then we have a natural intertwiner
namely
\begin{equation}
v\otimes w\mapsto w\otimes v.
\end{equation}
This is a natural isomorphism of representations, so nothing
changes. If we have $V^{\otimes n}$, then we have the symmetric
group $S_{n}$ consisting of intertwiners. So a permutation is an
intertwiner. We may consider vectors $x$ such that
\begin{equation}
\alpha(x)=x
\end{equation}
is invariant under such permutations; they form a subspace. More
precisely
\begin{equation}
x=z^{ij}v_{i}\otimes v_{j}
\end{equation}
and the coefficients are tensors, what we do is consider
symmetric tensors which are fixed points of the intertwiner which
implies the coefficients obey
\begin{equation}
z^{ij}=z^{ji}
\end{equation}
for all $i$, $j$.

We may consider the subspace obeying
\begin{equation}
\alpha(x)=-x
\end{equation}
then the coefficients are
\begin{equation}
z^{ij}=-z^{ji}
\end{equation}
antisymmetric tensors. The symmetric one is denoted by
$\Sym^{2}V$ and the antisymmetric by $\Antisymmetric^{2}V$. We
generalize to the tensor product of $n$ spaces
\begin{equation}
V^{\otimes n}=\underbracket[0.5pt]{V\otimes\dots\otimes V}_{\text{$n$ times}}
\end{equation}
then we get guys with $n$ indices $z^{i_{1}\dots i_{n}}$. We can
apply various demands of indices. We use the notation
\begin{equation}
z^{[ij]} = \frac{1}{2!}(z^{ij}-z^{ji})
\end{equation}
and
\begin{equation}
z^{(ij)} = \frac{1}{2!}(z^{ij}+z^{ji}).
\end{equation}

We may also take tensor products including the dual space and the
vector space, for example
\begin{equation}
V^{\otimes m}\otimes (V^{*})^{\otimes n} = 
\underbracket[0.5pt]{V\otimes\dots\otimes V}_{\text{$m$ times}}\otimes
\underbracket[0.5pt]{V^{*}\otimes\dots\otimes V^{*}}_{\text{$n$ times}}
\end{equation}
which results in guys
\begin{equation}
a^{i_{1}\dots i_{m}}_{j_{1}\dots j_{n}}
\end{equation}
with mixed indices.

We have been talking about groups, but we may consider analogous
gadgetry for the Lie algebra. If we have
\begin{equation}
(\varphi_{g}\otimes\psi_{g})(u\otimes v)=\big(\varphi(g)(u)\big)\otimes\big(\psi(g)(v)\big)
\end{equation}
for the Lie group, and we take 
\begin{equation}
g=1+\gamma
\end{equation}
where $\gamma$ is ``small.'' We obtain from the Lie group
representation $\varphi_{1+\gamma}$ a Lie algebra representation
$\widetilde{\varphi}_{\gamma}$, but how does the representation
behave under the tensor product of Lie algebra representations?
We have
\begin{equation}
(\widetilde\varphi_{\gamma}\otimes\widetilde\psi_{\gamma})(u\otimes v)=
\big(\widetilde\varphi_{\gamma}(u)\big)\otimes v + u\otimes\big(\widetilde\psi_{\gamma}(v)\big).
\end{equation}
Why? Well observe that
\begin{subequations}
\begin{align}
(\varphi_{1+\gamma}\otimes\psi_{1+\gamma}) 
&= (\1+\widetilde\varphi_{\gamma})\otimes(\1+\widetilde\psi_{\gamma})\\
&= {\bf 1} +
  \underbracket[0.5pt]{\widetilde{\varphi}_{\gamma}\otimes{\bf 1}
+{\bf 1}\otimes\widetilde{\psi}_{\gamma}}_{\text{Lie Algebra rep.}} + \mathcal{O}(\varepsilon^{2})
\end{align}
\end{subequations}
where $\varepsilon$ is the ``magnitude'' of $\gamma$, which is
negligibly small in comparison to 1, and $\1$ is the
identity operator.

Consider the simplest example $\U{n}$ and its fundamental
representation $\GL{\CC^{n}}$. The maximal torus is
\begin{equation}
T = \left\{\begin{bmatrix}
 e^{i\varphi_{1}} &        &  \\
              & \ddots &  \\
              &        & e^{i\varphi_{n}}
\end{bmatrix}
\right\},
\end{equation}
this corresponds to the Cartan subalgebra consisting of diagonal
matrices. The weight vectors are the standard basis
\begin{equation}
v_{i} = e_{i}
\end{equation}
which is 1 for the $i^{th}$ component, 0 for all others. Now it
is clear what are the weights, merely the corresponding
components. We may consider the tensor product of two such
representations. The basis is by our definition $v_{i}\otimes
v_{j}$, and it is very easy to understand
\begin{equation}
v_{i}\otimes v_{j}\mapsto (\varphi_{i}+\varphi_{j})(v_{i}\otimes v_{j})
\end{equation}
is a weight vector. There is a rule for the characters
\begin{equation}
\chi_{\varphi\otimes\psi}=\chi_{\varphi}\chi_{\psi}
\end{equation}
using the characters of the ``component'' representations. We
also have a representation of symmetric tensors with the basis
\begin{equation}
v_{i}\otimes v_{j}+v_{j}\otimes v_{i}
\end{equation}
and for a representation of antisymmetric tensors
\begin{equation}
v_{i}\otimes v_{j}-v_{j}\otimes v_{i}
\end{equation}
up to some overall factor of $1/2$. Both bases have almost the
same weights $\varphi_{i}+\varphi_{j}$, but for the antisymmetric
tensors we require $i\not=j$.



%% \begin{picture}(50,7)
%% \multiput(5,1)(20,0){5}{\circle{6}}
%% \multiput(8,1)(20,0){4}{\line(1,0){14}}
%% \end{picture}
