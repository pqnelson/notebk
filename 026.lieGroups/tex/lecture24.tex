%%
%% lecture24.tex
%% 
%% Made by Alex Nelson
%% Login   <alex@tomato3>
%% 
%% Started on  Sun Dec 12 20:50:41 2010 Alex Nelson
%% Last update Mon Dec 13 12:59:05 2010 Alex Nelson
%%
We have a Clifford algebra over any field $\FF$, we take some
nondegenerate matrix $\eta$, and we look for generators obeying
\begin{equation}
\Gamma_{\alpha}\Gamma_{\beta}+\Gamma_{\beta}\Gamma_{\alpha}=2\eta_{\alpha\beta}\cdot\1
\end{equation}
This is a representation of a Clifford algebra, they may be
classified quite easily:
\begin{enumerate}
\item all representations are irreducible;
\item the symmetric matrix $\eta_{\alpha\beta}$ is a symmetric
  $(m,n)$-form, which corresponds to $\cliff(m,n,\RR)$.
\end{enumerate}
We have orthogonal groups $\ORTH{m,n,\RR}$ for the metric which
is similar to
\begin{equation}
g = \diag(\underbrace{+1,\dots,+1}_{\text{$m$
    $+1$'s}},\underbrace{-1,\dots,-1}_{\text{$n$ $-1$'s}})
\end{equation}
If we take
\begin{equation}
\widetilde\Gamma_{\alpha}={a_{\alpha}}^{\beta}\Gamma_{\beta}
\end{equation}
(using Einstein summation convention) and
${a_{\alpha}}^{\beta}\in\ORTH{\eta,\FF}$, then the commutation
relations are preserved.

\begin{ClassificationCliff}
There is only one irreducible representation for $\cliff(2n)$ for
each dimension, and other irreducible representations in the same
dimensiona are equivalent:
\begin{equation}
\widetilde{\Gamma}_{\alpha}=U_{a}\Gamma_{\alpha}{U_{a}}^{-1}\quad\mbox{and}\quad
U_{ab}=(\mbox{constant})U_{a}U_{b}
\end{equation}
and we cannot set this constant to be unity.
\end{ClassificationCliff}

Consider $\SO{3}$, odd-dimensional guys. The situation is a
little different. How to construct representations of the group
in odd dimension? Recall
\begin{equation}
\SO{2n,\CC}\subset\SO{2n+1,\CC}
\end{equation}
is an embedding. We may construct $\Gamma_{1}$, \dots,
$\Gamma_{2n}$ which obey the desired relation. After that, the
theorem is: we may add the last matrix in such a way that the
commutation relations are satisfied. The construction is very
simple:
\begin{equation}
\Gamma_{2n+1} = \Gamma_{1}(\cdots)\Gamma_{2n}
\end{equation}
up to some constant. We should quickly verify that
$\Gamma_{2n+1}$ satisfy the desired properties, e.g.
\begin{equation}
\Gamma_{2n+1}^{2} = 1.
\end{equation}
It is very clear that
\begin{equation}
\Gamma_{2n+1}^{2}=\pm 1
\end{equation}
so we merely choose the coefficient to make it unity. Therefore
we may extend any irreducible representation of the Clifford
Algebra over $2n$-dimensions to be an irreducible representation
of $\cliff(2n+1)$, with an appropriate choice of
coefficient. This may be done in two ways really (up to sign).

Lets compute the dimensions. $\cliff(2n)$ has one irreducible
representation. What is its dimensions? It is
\begin{equation}
2^{n} = \dim\big(\cliff(2n)\big)
\end{equation}
Why? We have our guys divided into 2 parts, creation and
annihilation operators, and we may apply them to a vacuum
state. The rest follows trivially.

Consider $\cliff(2n+1)$ we have 2 irreducible representations
both of dimension $2^{n}$, since we considered irreducible
representations and added 1 Dirac matrix generator. For, e.g.,
$n=1$, we get $\cliff(3)$ and moreover we have an irreducible
representation of $\SO{3}$ of dimension $2^{1}=2$. But a
2-dimensional representation of $\SO{3}$ does not exist! What
does exist is
\begin{equation}
\SU{2}\to\SO{3}
\end{equation}
which is a 2-sheeted covering. We may take the inverse
mapping:
\begin{equation}
\SO{3}\to\SU{2}
\end{equation}
which is a 2-valued representation. This is \emph{precisely} the
spinor representation.

\subsection{Review of Previous Stuff}
Here we will stress several points which should be
stressed. There is a notion of a \define{Reducible
  Representation}, i.e. a representation with a nontrivial
subrepresentation (an invariant subspace). There is a notion of
\define{Completely Reducible Representation} which is the direct
sum of irreducible representatio. Is an irreducible
representation completely reducible? Yes! It is! Really, it is
$\rho=\rho$. 

Remember unitary and orthogonal representations are completely
reducible, representations of a Lie algebra/group are not
necessarily reducible (standard example: Abelian Lie algebra,
$\mathfrak{u}(1)^{k}$ which has more representations since
$\U{1}$ is not simply connected).

Irreducible representations of an Abelian Lie algebra are
\emph{always} 1-dimensional, but this doesn't mean that every
representation of the Abelian Lie Algebras are direct sums of
irreducible representations. For a representation onto a matrix
with not normal Jordan form, the representation is not completely
reducible; for the representation of the single generator by a
diagonalizable matrix, we get the direct sum of 1-dimensional
irreducible representations.

Remember
\begin{equation}
\mbox{Reductive Lie
  Algebra}=(\mbox{Center})\oplus(\mbox{semisimple part})
\end{equation}
It has a non-trivial center which is an Abelian Lie Algebra.

The Cartan subalgebra is a maximal Abelian subalgebra, but this
is not the end of the story because we should have some sort of
requirement of semisimplicity. It was defined as the Lie Algebra
of the maximal Torus for a compact group. What does ``maximal''
mean? If we add anything else, it become non-Abelian. But it is
\textbf{WRONG} to state it contains every Abelian
subalgebra. Consider $\mathfrak{gl}(n)$, the Cartan subalgebra
consists of diagonal matrices\dots but we should have a basis. If
we change basis, we get a completely different Cartan
subalgebra\dots well, a \emph{conjugate} Cartan
subalgebra. Really, 2 maximal Tori are conjugate; this is a
nontrivial statement. Care needs to be taken when working with a
Cartan subalgebra.

We have asimilar situation with simple roots, and the highest
weight representation. We need additional data to introduce
these notions. There is \emph{no} intrinsic information in the
Lie Algebra. The first choice is to fix the Cartan subalgebra
$\mathscr{H}$. Then we decompose the Lie Algebra into 3 parts:
\begin{equation}
\mathscr{G}=\mathscr{G}_{+}\oplus\mathscr{H}\oplus\mathscr{G}_{-}
\end{equation}
where this direct sum is as vector spaces. Each summand is a
subalgebra. We have a basis consisting of 3 types of elements
\begin{subequations}
\begin{align}
E_{i}\in\mathscr{G}_{+}\\
H_{k}\in\mathscr{H}\\
F_{j}\in\mathscr{G}_{-}
\end{align}
\end{subequations}
The commutation relations are nontrivial. The computations may
not be made componentwise. We can define the highest weight
vector $\vec{v}$ by
\begin{equation}
E_{i}\vec{v}=0
\end{equation}
for all $E_{i}\in\mathscr{G}_{+}$ \textbf{and}
\begin{equation}
H_{k}\vec{v}=\lambda(H_{k})\vec{v}
\end{equation}
where
\begin{equation}
\lambda\colon\mathscr{H}\to\FF
\end{equation}
is a linear functional called the \define{Highest Weight}. This
piucture is not terribly convenient, we'd like one with more
information. Namely a system of multiplicative generators denoted
by lowercase letters: $e_{\alpha}$, $h_{\alpha}$, and
$f_{\alpha}$. We take $E_{i}$ as the roots of Lie Algebra, so
\begin{equation}
[H,\mathscr{G}_{+}]\subset\mathscr{G}_{+}
\end{equation}
thus
\begin{equation}
[H_{k},E_{i}] = \alpha(H_{k})E_{i}
\end{equation}
are root vectors $E_{i}$ and roots $\alpha(H_{k})$. We see the
commmutator of root vectors is a root vector again
$[E_{i},E_{j}]$ with root $\alpha_{i}+\alpha_{j}$. We consider
the $e_{\alpha}$ sufficient to generate all $E_{i}$, and $f_{j}$
sufficient to generate all $F_{j}$.
