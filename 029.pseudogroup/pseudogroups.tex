%%
%% pseudogroups.tex
%% 
%% Made by Alex Nelson
%% Login   <alex@tomato3>
%% 
%% Started on  Sun Jan 17 10:24:22 2010 Alex Nelson
%% Last update Mon Jan 18 12:27:59 2010 Alex Nelson
%%
\documentclass{article}
\usepackage{notebk}

\title{Notes on Pseudogroups of Transformations}
\date{January 17, 2010}
\begin{document}
\maketitle
\tableofcontents

\section{Introduction}

We will review the traditional definition of pseudogroups.

Then we will ``internalize'' it in the language of category
theory as a groupoid.

\section{Traditional Notion of Pseudogroups}

We use the definition from Kobayashi and Nomizu~\cite{kobayashi1963foundations}.

\begin{defn}
A \define{Pseudogroup of Transformations} on a topological space
$S$ is a set $\Gamma$ of transformations satisfying the following
axioms:
\begin{enumerate}
\item Each $f\in\Gamma$ is a homeomorphism of an open set (called
  the domain of $f$) of $S$ onto another open set (called the
  range of $f$) of $S$;
\item If $f\in\Gamma$, then the restriction of $f$ to an
  arbitrary open subset of the domain of $f$ is in $\Gamma$;
\item Let $U=\cup_{i}U_{i}$ where each $U_{i}$ is an open set of
  $S$. A homeomorphism $f$ of $U$ onto an open set of $S$ belongs
  to $\Gamma$ if the restriction of $f$ to $U_{i}$ is in $\Gamma$
  for every $i$;
\item For every open set $U$ of $S$, the identity transformation
  of $U$ is in $\Gamma$;
\item If $f\in\Gamma$, then $f^{-1}\in\Gamma$;
\item If $f\in\Gamma$ is a homeomorphism of $U$ onto $V$ and
  $f'\in\Gamma$ is a homeomorphism of $U'$ onto $V'$ and if
  $V\cap U'$ is non-empty, then the homeomorphism $f'\circ f$ of
  $f^{-1}(V\cap U')$ onto $f'(V\cap U')$ is in $\Gamma$.
\end{enumerate}
\end{defn}

So what's the significance of being a homeomorphism? It means
that topologically, the open subsets of $S$ are ``the same''
locally. That is, we can deform any open subset into any other
open subset.

We have some consistency conditions too. This is precisely the
importance of properties 2 and 3. More precisely these are
conditions specifying ``consistency on restrictions''. The last
condition, property 6, is an explicit condition for ``consistency
on overlaps''. 

The property 4 makes a pseudogroup ``sufficiently nice'';
similarly, property 5 indicates that this is kind of like a
groupoid. However, a pseudogroup is \emph{a single set} of
transformations; a groupoid is more structured than this.

\section{Internalization}

\begin{quote}
When you are collecting mushrooms, you only see the mushroom
itself. But if you are a mycologist, you know that the real
mushroom is in the Earth. There's an enormous thing down there,
and you just see the fruit, the body that you eat. In
mathematics, the upper part of the mushroom corresponds to the
theorems that you see, but you don't see the things that are
below, that is: \emph{problems, conjectures, mistakes, ideas},
etc. --- V.~I.~Arnold~\cite{arnol2004hilbert}
\end{quote}

We will try several attempts to internalize the notion of a
pseudogroup of transformations. It may take several attempts,
leading to wrong turns, wrong guesses, etc. 

Recall, given a topological space $X$ (or more generally any set
$X$), we can construct a category $\mathcal{O}(X)$ whose objects
are open subsets of $X$ and morphisms are restrictions and
inclusions. We will need to use this to construct the notion of a
pseudogroup of transformations. Weinstein notes~\cite{weinstein1996groupoids}
that germs of elements of a pseudogroup form a groupoid.


\nocite{*}
\bibliographystyle{elements}
\bibliography{pseudogroups}
\end{document}
