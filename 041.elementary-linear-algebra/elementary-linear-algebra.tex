\documentclass{article}



\usepackage[12hr,us]{datetime}
\usepackage{macros}
\usepackage{danger}
\usepackage{notation}
\usepackage[equation,leqno]{deriv}

\def\homeurl{\url{https://pqnelson.github.io/notebk/}}



\title{Elementary Linear Algebra}
\author{Alex Nelson\thanks{This is a page from \homeurl{}\hfil\break\indent\;\, Compiled:\enspace\today\ at \currenttime\ (PST)}}
\date{October 26, 2022}



\begin{document}%\tracingall
\maketitle

\tableofcontents
\chapter{Preface}

This introduces differential topology, which studies global aspects of
smooth manifolds. It's roughly at the level of a graduate course,
assuming the reader knows elementary differential geometry of
curves\footnote{Roughly at the level of my notes \url{http://pqnelson.github.io/assets/notebk/dg.pdf}},
real analysis, and so on. These notes are based on Dr Fuchs' course on
differential topology at UC Davis, Math 239, held Fall quarter of 2010.

The conventions are slightly old-fashioned, namely: charts are defined
as ordered pairs of maps from ``patches'' of open subsets of $\RR^{n}$
to a set along with the open subsets of $\RR^{n}$ themselves. This is
done so we can induce a topology using a collection of charts, demanding
the charts be \emph{continuous}; then their images form a topological
basis. We get, for free, a topological manifold from this demand on
maximal atlases.

Most ``modern'' books use the opposite convention, a chart on a set $M$
is a pair $(\varphi,U)$ consisting of a subset $U\subset M$ --- \emph{not}
$U\subset\RR^{n}$ (!!!) --- and a mapping $\varphi\colon U\to\RR^{n}$.
This is fine, but requires more work to induce a topology.

\part{Problem Statement}
\section{Solving Systems of Equations}

\begin{example}[{Euler~\cite[Ch.IV \S4 question 3 \P612]{euler:algebra}}]
A mule and donkey carry a large load. The donkey complained of his load
and said to the mule, ``I need only 100 pounds of your load to make mine
twice as heavy as yours would be.'' To which the mule answered, ``But if you gave
me 100 pounds of your load, I'd be carrying three times you would
carry.''
How much did they carry?

Let $d$ be the donkey's load, $m$ be the mule's load. The donkey's
statement could be presented in the equation,
\begin{subequations}
\begin{equation}
d + 100 = 2(m - 100),
\end{equation}
whereas the mule's response,
\begin{equation}
m + 100 = 3(d - 100).
\end{equation}
\end{subequations}
From the first of these equations, we find
\begin{equation}
d = 2m - 300.
\end{equation}
We plug this into the second equation to find
\begin{equation}
\begin{split}
  m + 100 &= 3(2m - 400)\\
  &= 6m - 1200.
\end{split}
\end{equation}
Hence the mule carries,
\begin{subequations}
\begin{equation}
m = \frac{1300}{5} = 260~\mbox{pounds},
\end{equation}
and this means the donkey carries,
\begin{equation}
d = 520 - 300 = 220~\mbox{pounds}.
\end{equation}
\end{subequations}
\end{example}

\begin{example}[{Sraffa~\cite{sraffa}}]\label{ex:systems-of-equations:sraffa}
One class of systems of linear equations occurs naturally in
Neo-Ricardian economics. Consider a hypothetical economy consisting of
two industry sectors, wheat and iron, whose production processes are
described by the system of equations (where the left of the arrow refers
to the inputs at the start of the production period, and the quantities
to the right of the arrows are produced at the end of the period, and
then an annual market takes place where iron is traded for wheat):
\begin{equation}
\begin{array}{rcl} \mbox{inputs} & \to & \mbox{outputs}\\
280~\mbox{qr. wheat} + 12~\mbox{t. iron} & \to & 400~\mbox{qr. wheat}\\
120~\mbox{qr. wheat} + 8~\mbox{t. iron}  & \to &  20~\mbox{t. iron}.
\end{array}
\end{equation}
The question we want to know: what is the exchange rate between
quarters\footnote{A ``quarter'' of wheat refers to an obscure
measurement choice of 13th century England, because it can be found in
the \textit{Magna Carta}. By the 18th century, the measurement of 1
quarter of wheat in Britain varied port to port. It was finally
standardized in 1824 to be 8 bushels (or 64 gallons).} 
of wheat and tons of iron? We can create a system of equations by
introducing two unknowns $p_{w}$ for the price of 1 quarter of wheat,
and $p_{i}$ for the price of 1 ton of iron. We assume society is in a
``self reproducing state'', in the sense that the wheat sector exchanges
enough wheat-for-iron to continue producing wheat (and the iron sector
exchanges enough iron-for-wheat to continue producing iron):
\begin{subequations}
\begin{equation}
  \begin{array}{rcl}
    280p_{w} + 12p_{i} &=& 400p_{w}\\
    120p_{w} + 8p_{i} &=& 20p_{i}.
  \end{array}
\end{equation}
We can rewrite the first equation (by subtracting both sides by $280p_{w}$):
\begin{equation}
  \begin{array}{rcl}
    12p_{i} &=& 120p_{w}\\
    120p_{w} + 8p_{i} &=& 20p_{i}.
  \end{array}
\end{equation}
Then we rewrite the second equation (by subtracting both sides by $8p_{i}$):
\begin{equation}
  \begin{array}{rcl}
    12p_{i} &=& 120p_{w}\\
    120p_{w} &=& 12p_{i}.
  \end{array}
\end{equation}
These are redundant equations! We end up with a class of solutions,
described by
\begin{equation}
\boxed{10p_{w} = p_{i},}
\end{equation}
\end{subequations}
that is to say, the price of 1 ton iron is equal to the price of 10
quarters of wheat.
\end{example}

\begin{remark}
This method of solving such equations in Neo-Ricardian economics works
fine when there is no surplus; that is to say, when there is exactly as
much produced (for each commodity) used as inputs in the
economy. However, when there is surplus, we need more sophisticated
techniques of linear algebra, because we will have an eigenvalue problem.
\end{remark}

\begin{example}[Motivating Example]
  Suppose we have two equations with two variables $x$ and $y$ given by:
  \begin{subequations}
    \begin{align}
      -x-2y  &= -2\\
      5x+6y  &= 1.
    \end{align}
  \end{subequations}
  Are there [real] values for $x$ and $y$ which makes this hold? We can
  add 5 times the first equation to the second equation, giving us:
  \begin{subequations}
    \begin{align}
      -x-2y  &= -2\\
      0-4y  &= -9.
    \end{align}
  \end{subequations}
  The second equation may be solved for $y=9/4$, then we may substitute
  this into the first equation giving us
  \begin{equation}
-x-2(9/4) = -2\implies -x=2(9/4)-2=\frac{9}{2}-\frac{4}{2}=\frac{5}{2}.
  \end{equation}
  Hence we obtain the solutions
  \begin{equation}
\boxed{x=\frac{5}{2}}\quad\mbox{and}\quad\boxed{y=\frac{9}{4}.}
  \end{equation}
\end{example}

\N{Index Notation}
We will rely heavily on index notation. Why? Well, there are 26 possible
symbols (using lowercase Latin letters), which means we could not
discuss anything beyond systems of 5 equations in 5 variables; if we
also use the 26 uppercase Latin letters, then we cannot discuss anything
beyond systems of 7 equations in 7 unknowns. Rather than struggle with
new symbols, we will use indices: appending numbers as subscripts to
letters to refer to distinct quantities. Thus we write $x_{1}$, $x_{2}$,
$x_{3}$ instead of $x$, $y$, $z$.

We may use \define{Dummy Variables} to affix a variable as a subscript,
writing $x_{i}$ where $i$ could be $1$, $2$, or $3$. Here $i$ is the
dummy variable in the expression $x_{i}$. We often will write ``$x_{i}$
where $i=1,2,3$'' to indicate $i$ is a dummy variable ranging over the
values $1$, $2$, $3$.

We can have several indices [plural of ``index''] affixed to a
symbol. We have seen this with coefficients, which had two indices
separated by a comma. Some authors use a comma to separate indices,
other authors do not.

Also we should note, some authors use superscripts indices. This will
occur in some parts of mathematics, but we will avoid it in these notes
whenever possible (because it's hard to tell if a superscript number is an
exponent or an index).

\N{Basic Terminology}
A \define{linear equation in one variable} $x$ looks like
\begin{equation}
b = ax
\end{equation}
where $a$ and $b$ are real constants.

A single \define{linear equation in two variables} $x$ and $y$ looks like
\begin{equation}
  ax + by = c
\end{equation}
where $a$, $b$, and $c$ are real constants.

A single linear equation in $n$ variables $x_{1}$, \dots, $x_{n}$ looks
like
\begin{equation}
a_{1}x_{1} + \dots + a_{n}x_{n} = b
\end{equation}
where $a_{1}$, \dots, $a_{n}$, and $b$ are real constants.

\N{Terminology: Systems of equations}
More generally, a \define{System of $m$ Linear Equations in $n$ Unknowns} 
$x_{1}$, \dots, $x_{n}$ (where the subscripts are \define{Indices} to
append to one symbol $x$ a number or ``dummy variable'' [a variable
ranging over the index values], giving us any
number of variables for the price of one symbol) looks like:
\begin{equation}
\begin{array}{rcl}
a_{1,1}x_{1} + a_{1,2}x_{2} + \dots + a_{1,n}x_{n} &=& b_{1}\\
a_{2,1}x_{1} + a_{2,2}x_{2} + \dots + a_{2,n}x_{n} &=& b_{2}\\
\vdots\quad \qquad\vdots\quad\qquad\ddots\quad\qquad\vdots\quad & \vdots & \vdots\\
a_{m,1}x_{1} + a_{m,2}x_{2} + \dots + a_{m,n}x_{n} &=& b_{m}
\end{array}
\end{equation}
where $b_{1}$, \dots, $b_{m}$ are $m$ real constants, and we have
$a_{1,1}$, $a_{1,2}$, \dots, $a_{m,n}$ be $mn$ real constants called the
\define{Coefficients}. 

A \define{Solution} to a system of equations in $n$ variables $x_{1}$,
\dots, $x_{n}$ is an assignment of $n$ real values $s_{1}$, \dots,
$s_{n}$ to the variables (so $x_{1}=s_{1}$, \dots, $x_{n}=s_{n}$) which
satisfies the system of equations.


\begin{example}[System of Nonlinear Equations]
We have to distinguish a system of linear equations from nonlinear
equations. For example, the following is a nonlinear equation
\begin{equation}
3x^{2} + x = 2,
\end{equation}
because a polynomial of order $n>1$ will be nonlinear. If you have a
function $f(x)$ and you don't know if it's linear or not, check if
$f(ax+by)=af(x)+bf(y)$ holds or not --- this is the criteria of linearity.

The following equation is also nonlinear:
\begin{equation}
x + yz =3.
\end{equation}
We cannot multiply two [or more] unknowns together in a linear equation.

So the following equation is nonlinear:
\begin{equation}
x + y\cos(z) = 3,
\end{equation}
on two counts: first, $\cos(z)$ is nonlinear; and second, $y\cos(z)$ is
not allowed in a linear equation.
\end{example}

\N{Method of Elimination}
One method to find a solution, which has been taught to high school
students for decades, is to take the first equation and use it to
rewrite one variable in terms of the remaining. For example, in:
\begin{equation}
  \begin{array}{rcrcrcl}
    0      & + & 2x_{2} & + & 3x_{3} & = & 5\\
    2x_{1} & + & 0      & + & 3x_{3} &=& -1\\
    x_{1}  & + & x_{2}  & + & x_{3} &=& 1.
  \end{array}
\end{equation}
We would use the first equation to substitute $x_{2}=(5 - 3x_{3})/2$ in
the remaining equations, and discard the first equation, giving us:
\begin{equation}
  \begin{array}{rcrcrcl}
    2x_{1} & + & 0      & + & 3x_{3} &=& -1\\
    x_{1}  & + & \displaystyle\left(\frac{5 - 3x_{3}}{2}\right)  & + & x_{3} &=& 1.
  \end{array}
\end{equation}
We do this again, to use the first equation to write $x_{1} = (-1-3x_{3})/2$,
and plug this into the last equation, which gives us a value for
$x_{3}$.
We backsubstitute this into the second equation to obtain a value for
$x_{1}$, then together these are backsubstituted into either the first
or third equation of our original system to find the solution for
$x_{2}$.

\N{Issues with Method of Elimination}
Doing this ``method of elimination'' is tedious. \emph{I} [the author]
does not want to do this, it's too much work. Is there a better way to
find a solution to a system of linear equations?

A second issue with the method of elimination, it does not tell us when
a solution exists. Or if there is a family of solutions, which we saw in
Example~\ref{ex:systems-of-equations:sraffa} (the price of 1 ton of iron
was 10 times the price of a quarter of wheat, whatever that is).

\N{Roadmap: Enter the Matrix}
Remember what a system of 1 linear equation in 1 variable looks like? We
saw it was $ax=b$. We could solve this easily when $a\neq0$, simply take
$x=b/a$ or $x = a^{-1}b$. It would be lovely if all systems of linear
equations could be so easy.

We will try to make it easy by generalizing the notion of a
``number''. Instead of writing a system of equations as
\begin{equation}
\begin{array}{rcl}
a_{1,1}x_{1} + a_{1,2}x_{2} + \dots + a_{1,n}x_{n} &=& b_{1}\\
a_{2,1}x_{1} + a_{2,2}x_{2} + \dots + a_{2,n}x_{n} &=& b_{2}\\
\vdots\quad \qquad\vdots\quad\qquad\ddots\quad\qquad\vdots\quad & \vdots & \vdots\\
a_{m,1}x_{1} + a_{m,2}x_{2} + \dots + a_{m,n}x_{n} &=& b_{m}
\end{array}
\end{equation}
we will introduce a notion of a matrix $\mat{A}$ as an array of $m$ rows and
$n$ columns, and vectors $\vec{x}$, $\vec{b}$, then rewrite our system
as
\begin{equation}
\mat{A}\vec{x}=\vec{b}.
\end{equation}
Just as with our simpler system with one equation in one unknown $ax=b$
where we multiplied both sides by the number $a^{-1}$ (assuming $a$ is
invertible), we can imitate this and try ``multiplying'' both sides by a
matrix $\mat{B}$ to transform our system to something like
\begin{equation}
\vec{x} = \mat{B}\vec{b}.
\end{equation}
The problem: we will have to define a notion of matrix multiplication,
and prove it satisfies the familiar properties which the multiplication
of numbers enjoy.

\phantomsection
\subsection*{Exercises}
\addcontentsline{toc}{subsection}{Exercises}

\begin{exercise}[{Sraffa~\cite[\S2]{sraffa}}]
Find the exchange-values which ensure a
self-reproducing state for the following economy:
\begin{equation}
  \begin{array}{rcl}
240~\mbox{qr. wheat} + 12~\mbox{t. iron} + 18~\mbox{pigs} &\to& 450~\mbox{qr. wheat}\\
 90~\mbox{qr. wheat} +  6~\mbox{t. iron} + 12~\mbox{pigs} &\to& 21~\mbox{t. iron}\\
120~\mbox{qr. wheat} +  3~\mbox{t. iron} + 30~\mbox{pigs} &\to& 60~\mbox{pigs}.
  \end{array}
  \end{equation}
\end{exercise}


\part{Matrices}\label{part:matrices}
\section{Matrix Zoo}

We need to introduce the terminology first, before introducing notions
of matrix addition (or matrix multiplication).

\begin{definition}
Let $m$, $n$ be positive integers.
We define an $m$-by-$n$ \define{Matrix} $A$ to be a rectangular array of
$mn$ real [or complex] numbers arranged in $m$ horizontal \define{Rows}
and $n$ vertical \define{Columns}, written:
\begin{equation}
\mat{A}
= \left(\begin{array}{ccc>{\columncolor{olive!20}}ccc}
  a_{11} & a_{12} & \dots & a_{1j} & \dots & a_{1n} \\
a_{21} & a_{22} & \dots & a_{2j} & \dots & a_{2n}  \\
\vdots & \vdots & \ddots & \vdots & \ddots & \vdots \\
\rowcolor{olive!20}a_{i1} & a_{i2} & \dots & a_{ij} & \dots & a_{in} \\
\vdots & \vdots & \ddots & \vdots & \ddots & \vdots \\
a_{m1} & a_{m2} & \dots   & a_{mj} & \dots  & a_{mn} 
\end{array}\right)
\begin{array}{c}
  \\
  \gets i^{\text{th}}\mbox{\ row}
\end{array}
\end{equation}
We have highlighted the $i^{\text{th}}$ row, and the $j^{\text{th}}$ column.

If further $m=n$ (so we have an $n$-by-$n$ matrix), then we call it a
\define{Square Matrix}. In this case, we can refer to the \define{Main Diagonal} 
as the components $a_{1,1}$, $a_{2,2}$, \dots, $a_{n,n}$.
\end{definition}

\begin{remark}
We abbreviate ``$m$-by-$n$ matrix'' as ``$m\times n$ matrix''. The
pair of numbers $(m,n)$ is called the \define{Dimensions} of the matrix.

Also note, some people use parentheses in writing matrices, other people
use square brackets. Both are acceptable.
\end{remark}

\N{Matrix and Components}
If we want to refer to a generic component of an $m$-by-$n$ matrix $\mat{A}$,
we write $\mat{A}=(a_{ij})$ to indicate we will use $a_{ij}$ to refer to the
component found in the $i^{\text{th}}$ row and $j^{\text{th}}$ column.

Also, very important, if we have some $m$-by-$n$ matrix
$\mat{A}=(a_{i,j})$ and we wanted to refer to its component in row $r$,
column $c$, we may refer to that component by writing $(\mat{A})_{r,c}$.
This is, of course, the same as $a_{r,c}$, but it will be useful [much
later] to take a matrix and refer to certain components without
introducing $(a_{i,j})$ first.

\N{Example: Zero Matrix}
For any positive integers $m$, $n$, we have the $m\times n$ \define{Zero Matrix}
to be the matrix whose components are all zero. We denote this by
$\mat{0}$ or $0$
because it will play the analogous counterpart that $0\in\RR$ plays. 

\N{Example: Identity Matrix}
For any positive integer $n$, we have the $n\times n$ \define{Identity Matrix}
to be
\begin{equation}
  \mat{I}_{n} = (\delta_{i,j}) = \begin{cases}1 &\mbox{if }i=j,\\
    0 &\mbox{if }i\neq j
  \end{cases}
\end{equation}
where the components $\delta_{i,j}$ are referred to as the
\define{Kronecker Delta}. If $n$ is clear from context, we may suppress
the subscript and write $\mat I$ instead of $\mat{I}_{n}$. Some authors write
$\mathbf{1}$ for the identity matrix, because (as we will see) it is
analogous to the number $1\in\RR$.

\begin{definition}
A \define{Column Vector} refers to an $n\times1$ matrix.

A \define{Row Vector} refers to a $1\times m$ matrix.

A \define{Vector} may refer to either row vectors or column vectors; in
these notes, we will reserve the word ``vector'' specifically for
``column vector''.
\end{definition}

\begin{ddanger}
In vector calculus, we were quite cavalier about our notion of vectors. 
There a vector consisted of a \textsc{base point} and an ``arrow''
of some magnitude pointing in some direction from the base point. We
could freely move the base point around willy-nilly. In linear algebra,
we will be working with vectors, but they all share \emph{the same} base
point. 
\end{ddanger}

\begin{definition}
A square matrix $\mat{A}=(a_{i,j})$ for which $a_{i,j}=0$ when $i\neq j$ is
called a \define{Diagonal Matrix}. We may write
$\mat{A}=\diag(a_{1,1}, a_{2,2},\dots, a_{n,n})$ to indicate $\mat{A}$
is a diagonal matrix. 

If further $a_{i,i}=c$ for all $i$, then we call $\mat{A}$ a
\define{Scalar Matrix}. 
\end{definition}

\begin{remark}[``Scalars'']
The word ``scalar'' means ``number''. We motivated our diversion into
matrices by trying to create some gadget which resembles numbers. Scalar
matrices are the ``most number-like'' among matrices.
\end{remark}

\N{Examples}
\begin{enumerate}
\item 
The following is a diagonal matrix which is not a scalar matrix
\begin{subequations}
\begin{equation}
  \mat{A} = \begin{pmatrix}1 & 0\\
    0 & 2
  \end{pmatrix}.
\end{equation}
\item 
The following is not a diagonal matrix,
\begin{equation}
  \mat{J} = \begin{bmatrix}0 & -1\\
    1 & 0
  \end{bmatrix}.
\end{equation}
\item The following is a scalar matrix,
\begin{equation}
  \mat{B} = \begin{pmatrix}-\pi & 0\\
    0 & -\pi
  \end{pmatrix}.
\end{equation}
\item The identity matrix $\mat{I}_{n}$ is a scalar matrix.
\item The $n\times n$ zero matrix is a scalar matrix.
\end{subequations}
\end{enumerate}

\begin{definition}
Let $\mat{A}=[a_{i,j}]$ and $\mat{B}=[b_{i,j}]$ be two $m\times n$ matrices.
We say $\mat{A}$ and $\mat{B}$ are \define{Equal} if they have identical components
in identical positions, i.e., if for each $i=1,\dots,m$ and
$j=1,\dots,n$ we have $a_{i,j}=b_{i,j}$. We indicate the matrices are
equal by writing $\mat{A}=\mat{B}$.

If $\mat{A}$ and $\mat{B}$ are not equal, then we write $\mat{A}\neq\mat{B}$.
\end{definition}

\begin{remark}
We are technically introducing a new binary relation, \define{Matrix Equality}.
\end{remark}

\begin{remark}[Same dimensions]
If $\mat{A}$ and $\mat{B}$ do not have the same number of rows and columns, then
they cannot possibly be equal. In this case (if they have different
dimensions), we can still sensibly write $\mat{A}\neq\mat{B}$.
\end{remark}

\begin{definition}\label{defn:matrix-zoo:triangular}
We call an $m\times n$ matrix $\mat{A}=(a_{i,j})$:
\begin{enumerate}
\item \define{Upper Triangular} if $a_{i,j}=0$ for $i>j$, e.g., when
  $m=n$ it looks like:
  \[ \mat{A} = \begin{bmatrix}
    a_{1,1} & a_{1,2} & a_{1,3} & \dots & a_{1,n}\\
    0      & a_{2,2} & a_{2,3} & \dots & a_{2,n}\\
    0      & 0       & a_{3,3} & \dots & a_{3,n}\\
    \vdots & \vdots & \vdots & \ddots & \vdots\\
    0      & 0      & 0      & \dots  & a_{n,n}
  \end{bmatrix} \]
\item \define{Lower Triangular} if $a_{i,j}=0$ for $i<j$, e.g., when
  $m=n$ it looks like:
  \[ \mat{A} = \begin{bmatrix}
    a_{1,1} & 0      & 0       & \dots & 0\\
    a_{2,1} & a_{2,2} & 0       & \dots & 0\\
    a_{3,1} & a_{3,2} & a_{3,3} & \dots & 0\\
    \vdots & \vdots & \vdots & \ddots & \vdots\\
    a_{n,1} & a_{n,2} & a_{n,3} & \dots  & a_{n,n}
  \end{bmatrix} \]
\item If $\mat{A}$ is upper-triangular and the diagonal entries are all zero,
  then we call $\mat{A}$ \define{Strictly Upper Triangular}.
\item Similarly, if $\mat{A}$ is lower-triangular and the diagonal entries are
  all zero, then we call $\mat{A}$ \define{Strictly Lower Triangular}.
\end{enumerate}
\end{definition}

\N{Block matrix}
It is useful to draw horizontal and vertical lines (either dashed or
slid, it makes no difference except for typography), to partition a
matrix into blocks, permitting us to write
\begin{equation}
\mat{A} = \left[
    \begin{array}{c:c}
        \mat{B} & \mat{C} \\ \hdashline
        \mat{D} & \mat{E} 
    \end{array}
\right] = \left[
    \begin{array}{c|c}
        \mat{B} & \mat{C} \\ \hline
        \mat{D} & \mat{E} 
    \end{array}
\right].
\end{equation}
We could partition a matrix however we want, and we call the blocks
\define{Submatrices} of $\mat{A}$. Writing $\mat{A}$ in this manner is
called a \define{Block Decomposition} of $\mat{A}$.

\begin{example}
  Let us consider some $3\times 5$ matrix $\mat{A}$ whose components are
  partitioned into block form:
  \begin{equation}
    \mat{A} = \left[\begin{array}{cc:ccc}
      a_{11} & a_{12} & a_{13} & a_{14} & a_{15}\\
      a_{21} & a_{22} & a_{23} & a_{24} & a_{25}\\ \hdashline
      a_{31} & a_{32} & a_{33} & a_{34} & a_{35}
    \end{array}\right] = \left[
    \begin{array}{c:c}
        \mat{B} & \mat{C} \\ \hdashline
        \mat{D} & \mat{E} 
    \end{array}
\right]
  \end{equation}
  This partitions $\mat{A}$ into a $2\times 2$ matrix $\mat{B}$, a
  $2\times3$ matrix $\mat{C}$, a $1\times2$ matrix $\mat{D}$, and a
  $1\times 3$ matrix $\mat{E}$. We could also partition it in other
  ways, for example,
\begin{equation}
    \mat{A} = \left[\begin{array}{c:ccc:c}
      a_{11} & a_{12} & a_{13} & a_{14} & a_{15}\\
      a_{21} & a_{22} & a_{23} & a_{24} & a_{25}\\ \hdashline
      a_{31} & a_{32} & a_{33} & a_{34} & a_{35}
    \end{array}\right].
\end{equation}
\end{example}

\begin{exercise}
Can we have a (row or column) vector which is also a scalar matrix?
\end{exercise}
\section{Matrix Algebra}

\M
Now, we want to define binary operators on matrices.
But we don't want these definitions to be arbitrary. We want:
\begin{enumerate}
\item Matrix multiplication produce a system of linear equations,
\item Matrix addition of vectors recovers the familiar operation in
  vector calculus.
\end{enumerate}

\begin{definition}
Let $\mat{A}=(a_{i,j})$ and $\mat{B}=(b_{i,j})$ be two $m\times n$ matrices.
We define \define{Matrix Addition} of $\mat{A}$ and $\mat{B}$ to produce a third
$m\times n$ matrix $\mat{C}=(c_{i,j})$ (called the \emph{sum} of $\mat{A}$ and $\mat{B}$)
whose components are defined by $c_{i,j}=a_{i,j}+b_{i,j}$.  

We write $\mat{A}+\mat{B}$ for the sum of $\mat{A}$ and $\mat{B}$.
\end{definition}

\begin{remark}
We should prove, for any $m\times n$ matrices $\mat{A}$ and $\mat{B}$, their sum
exists and is unique. This is ``obvious enough''.
\end{remark}

\begin{example}
  Let
  \begin{subequations}
\begin{equation}
  \mat{A} = \begin{bmatrix}-4 & -4 & 5\\
    3 & -1 & 4
  \end{bmatrix}
\end{equation}
and
\begin{equation}
\mat{B} = \begin{bmatrix}-3 & 1 & 2\\ 6 & -3 & -5
\end{bmatrix}.
\end{equation}
Then
\begin{equation}
\mat{A}+\mat{B} =\begin{bmatrix}-4-3 & -4+1 & 5+2\\
3+6 & -1-3 & 4-5
\end{bmatrix}=\begin{bmatrix}-7 & -3 & 7\\
9 & -4 & -1
\end{bmatrix}.
\end{equation}
  \end{subequations}
\end{example}

\begin{proposition}[Commutativity of Matrix Addition]
For any two $m\times n$ matrices $\mat{A}=(a_{i,j})$ and $\mat{B}=(b_{i,j})$, we
have
$\mat{A}+\mat{B}=\mat{B}+\mat{A}$.
\end{proposition}

\begin{proof}
We see $\mat{A}+\mat{B}=(a_{i,j}+b_{i,j})$ and for each component we have $a_{i,j}+b_{i,j}=b_{i,j}+a_{i,j}$
since componentwise we have addition of numbers (which is commutative).
But the matrix with $(b_{i,j}+a_{i,j})=\mat{B}+\mat{A}$.
Hence we find $\mat{A}+\mat{B}=\mat{B}+\mat{A}$.
\end{proof}

\begin{proposition}
For any $m\times n$ matrix $\mat{A}=(a_{i,j})$, the sum of $\mat{A}$ with the
$m\times n$ zero matrix $0$ is $\mat{A}$.
\end{proposition}

\begin{proof}
We see that the components of the zero matrix are all identically zero,
$\mat 0=[(0)_{i,j}]$. So
\begin{subequations}
\begin{calculation}
  \mat{A}+\mat{0}
\step{by definition of matrix addition}
   [a_{i,j}+(0)_{i,j}]
\step{since $(0)_{i,j} = 0$ for all $i$, $j$}
   [a_{i,j}+0]
\step{arithmetic}
   [a_{i,j}]
\step{definition of $\mat{A}$}
   \mat{A}.
\end{calculation}
\end{subequations}
Thus $\mat{A}+\mat{0}=\mat{A}$.
\end{proof}

\begin{definition}
Let $\mat{A}=(a_{i,j})$ be an $m\times n$ matrix, let $r$ be a real [or
complex] number.
We define the \define{Scalar Multiple} of $\mat{A}$ by $r$ to
be the $m\times n$ matrix $r\mat{A} = (ra_{i,j})$ whose components are
obtained by multiplying every component of $\mat{A}$ by $r$. We can also
write this as $r\cdot \mat{A}=r\mat{A}$ to make the scalar
multiplication explicit. 
\end{definition}

\begin{remark}
``Scalar multiplication'' refers to the fact we are multiplying a matrix
  by a ``scalar'' [number], as opposed to multiplying by a vector or
  another matrix.
\end{remark}

\begin{example}
Let $r$ be a positive integer, let $\mat{A}$ be any matrix. Then
\begin{equation}
\underbrace{\mat{A} + \dots + \mat{A}}_{r~\text{times}} = rA.
\end{equation}
We can see this by induction on $r$.

\textbf{Base case:} $r=1$ means $\mat{A} = 1\cdot\mat{A}$ which is
trivially true.

\textbf{Inductive Hypothesis:} we now assume for any positive integer
$r$ that $\mat{A}+\dots+\mat{A}=r\mat{A}$.

\textbf{Inductive Case:} we now prove that $r+1$ case.
We want to prove
\begin{equation}
\underbrace{\mat{A}+\dots+\mat{A}}_{r+1~\text{times}}=(r+1)\mat{A}.
\end{equation}
We can use the inductive hypothesis to write:
\begin{equation}
\underbrace{\mat{A}+\dots+\mat{A}}_{r~\text{times}}+\mat{A}=r\mat{A}+\mat{A}.
\end{equation}
Then we observe $r\mat{A} + \mat{A} = (ra_{i,j}) + (a_{i,j}) = ((r+1)a_{i,j})$
using the definition of matrix addition.
Then by definition of scalar multiplication this is precisely $(r+1)\mat{A}$.
\end{example}

\begin{remark}
We have a sort of consistency result between scalar multiplication and
adding a matrix to itself finitely many times. That's a good sign.
\end{remark}

\begin{example}
  Let $r=-2$ and
  \begin{subequations}
    \begin{equation}
\mat{A} = \begin{bmatrix}
  -2 & 6\\
  6 & -5
\end{bmatrix}.
    \end{equation}
    Then
    \begin{equation}
r\mat{A} =\begin{bmatrix}
  -2\cdot(-2) & -2\cdot6\\
  -2\cdot 6 & -2\cdot (-5)
\end{bmatrix} = \begin{bmatrix}
  4 & -12\\
  -12 & 10
\end{bmatrix}.
    \end{equation}
  \end{subequations}
\end{example}

\begin{example}
If $\mat{A}=(a_{i,j})$ is a scalar $n\times n$ matrix $\mat{A}=\diag(a,a,\dots,a)$, then
$\mat{A} = a\mat{I}_{n}$.
\end{example}

\N{Properties of the Scalar Product}
Let $r$, $s$ be numbers and $\mat{A}$, $\mat{B}$ be appropriately sized
matrices. Then the following hold:
\begin{enumerate}
\item $r(s\mat{A})=(rs)\mat{A}$
\item $(r+s)\mat{A}=r\mat{A}+s\mat{A}$ (distributivity)
\item $r(\mat{A}+\mat{B})=r\mat{A}+r\mat{B}$
\end{enumerate}

\begin{definition}
Let $\mat{A}=(a_{i,j})$, $\mat{B}=(b_{i,j})$ be two $m\times n$
matrices. We define their \define{Difference} to be the matrix $\mat{A}-\mat{B} = \mat{A} + -1\cdot\mat{B}$.
Similarly, the \define{Negation} of $\mat{A}$ is the matrix $-\mat{A}=-1\cdot\mat{A}$.
\end{definition}

\begin{proposition}
For any matrix $\mat{A}$, we have $\mat{A}-\mat{A}=\mat{0}$.
\end{proposition}

\phantomsection
\subsection*{Exercises}
\addcontentsline{toc}{subsection}{Exercises}

\begin{exercise}
Prove the sum of two diagonal matrices is another diagonal matrix. Is
this true for scalar matrices (the sum of two scalar matrices is a
scalar matrix)?
\end{exercise}

\subsection{Matrix Transpose}

\begin{definition}
Let $\mat{A}=(a_{i,j})$ be an $m\times n$ matrix.
We define the \define{Transpose} of $\mat A$ to be the $n\times m$ matrix $\transpose{\mat{A}}=(\transpose{a}_{j,i})$
where $\transpose{a}_{j,i}=a_{i,j}$.
\end{definition}

\begin{example}
  If we have
\begin{subequations}
\begin{equation}
\mat{A} = %\begin{bmatrix}
\left[\begin{array}{ccccc}
\rowcolor{olive!20}a_{1,1} & {\color{BrickRed}a_{1,2}} & {\color{DarkGreen}a_{1,3}} & \dots & a_{1,n}\\
a_{2,1} & a_{2,2} & a_{2,3} & \dots & a_{2,n}\\
a_{3,1} & a_{3,2} & a_{3,3} & \dots & a_{3,n}\\
\vdots & \vdots & \vdots & \ddots & \vdots\\
a_{m,1} & a_{m,2} & a_{m,3} & \dots & a_{m,n}
  \end{array}\right],
%\end{bmatrix},
\end{equation}
(highlighting the first row to see where it goes, along with several
select entries), then we find
\begin{equation}
\transpose{\mat{A}} = %\begin{bmatrix}
\left[\begin{array}{>{\columncolor{olive!20}}ccccc}
a_{1,1} & a_{2,1} & a_{3,1} & \dots & a_{m,1}\\
{\color{BrickRed}a_{1,2}} & a_{2,2} & a_{3,2} & \dots & a_{m,2}\\
{\color{DarkGreen}a_{1,3}} & a_{2,3} & a_{3,3} & \dots & a_{m,3}\\
\vdots & \vdots & \vdots & \ddots & \vdots\\
a_{1,n} & a_{2,n} & a_{3,n} & \dots & a_{m,n}
\end{array}\right].
%  \end{bmatrix}.
\end{equation}
\end{subequations}
\end{example}

\begin{proposition}[Transpose is idempotent]
For any matrix $\mat{A}$, we have $\transpose{(\transpose{\mat{A}})}=\mat{A}$.
\end{proposition}

\begin{proof}
Let $\mat{B}=\transpose{\mat{A}}$
Then $(\transpose{\mat{B}})_{i,j} = (\mat{B})_{j,i}$ by definition of
the transpose, and $(\mat{B})_{j,i} = (\transpose{\mat{A}})_{j,i} = (\mat{A})_{i,j}$.
Hence $\left(\transpose{(\transpose{\mat{A}})}\right)_{i,j} = (\mat{A})_{i,j}$,
as desired.
\end{proof}

\begin{problem}
If $\mat{U}$ is an upper triangular matrix, then is
$\transpose{\mat{U}}$ upper triangular or lower triangular?
\end{problem}

\begin{problem}
If $\mat{L}$ is an lower triangular matrix, then is
$\transpose{\mat{L}}$ upper triangular or lower triangular?
\end{problem}

\subsection{Dot Product and Matrix Multiplication}

\begin{definition}
Recall, if we have two $n$-vectors $\vec{a}=(a_{1},\dots,a_{n})$ and
$\vec{b}=(b_{1},\dots,b_{n})$, we define their \define{Dot Product} to
be the scalar [number]
\begin{equation}
\vec{a}\cdot\vec{b} = a_{1}b_{1} + a_{2}b_{2} + \dots + a_{n}b_{n} = \sum^{n}_{j=1}a_{j}b_{j}.
\end{equation}
\end{definition}

\begin{example}
  Let
  \begin{equation}
\vec{a} = \begin{bmatrix}-1\\4\\-5
\end{bmatrix},\quad\mbox{and}\quad\vec{b}=\begin{bmatrix}
-6\\2\\4
\end{bmatrix}.
  \end{equation}
  Then
  \begin{equation}
    \begin{split}
    \vec{a}\cdot\vec{b} &= (-1)\cdot(-6) + 4\cdot2 + (-5)\cdot 4\\
    &= 6+8-20 = -6.
    \end{split}
  \end{equation}
\end{example}

\begin{proposition}
The dot product is symmetric, $\vec{a}\cdot\vec{b}=\vec{b}\cdot\vec{a}$.
\end{proposition}

\begin{definition}
Let $\mat{A}=(a_{i,j})$ be an $m\times p$ matrix,
let $\mat{B}=(b_{j,k})$ be an $p\times n$ matrix.
We define the \define{Matrix Multiplication} of $\mat{A}$ and $\mat{B}$
to be an $m\times n$ matrix $\mat{C}=(c_{i,k})$ whose components are
defined by the equation
\begin{equation}
c_{i,k} = \sum^{p}_{j=1}a_{i,j}b_{j,k}.
\end{equation}
That is to say, it is formed by taking the dot product of the
$i^{\text{th}}$ row of $\mat{A}$ with the $j^{\text{th}}$ column of $\mat{B}$,
\begin{equation*}
\left[\begin{array}{cccc}
a_{1,1} & a_{1,2} & \dots & a_{1,p}\\
a_{2,1} & a_{2,2} & \dots & a_{2,p}\\ 
\vdots & \vdots &   & \vdots\\
\rowcolor{olive!20}a_{i,1} & a_{i,2} & \dots & a_{i,p}\\
\vdots & \vdots &   & \vdots\\
a_{m,1} & a_{m,2} & \dots & a_{m,p}
  \end{array}\right]
\left[\begin{array}{ccc>{\columncolor{olive!20}}ccc}
b_{1,1} & b_{1,2} & \dots & b_{1,j} & \dots & b_{1,n}\\
b_{2,1} & b_{2,2} & \dots & b_{2,j} & \dots & b_{2,n}\\
\vdots & \vdots &       & \vdots  &       & \vdots\\
b_{p,1} & b_{p,2} & \dots & b_{p,j} & \dots & b_{p,n}
  \end{array}\right]
=
\left[\begin{array}{cccccc}
    c_{1,1} & c_{1,2} & \dots & c_{1,j} & \dots & c_{1,n}\\
    c_{2,1} & c_{2,2} & \dots & c_{2,j} & \dots & c_{2,n}\\
    \vdots & \vdots &       & \vdots  &       & \vdots\\
    c_{i,1} & c_{i,2} & \dots & {\colorbox{olive!20}{$c_{i,j}$}} & \dots & c_{i,n}\\
    \vdots & \vdots &        & \vdots &         & \vdots\\
    c_{m,1} & c_{m,2} & \dots & c_{m,j} & \dots & c_{m,n}
  \end{array}\right]
\end{equation*}
We denote the matrix multiplication of $\mat{A}$ and $\mat{B}$ by
$\mat{A}\mat{B}$. 
\end{definition}

\begin{example}
Let us consider an example just to show the ``mechanics'' of matrix
multiplication. Let
\begin{subequations}
\begin{equation}
\mat{A} = \begin{bmatrix}3 & 1 & 6\\2 & 1 & 3
\end{bmatrix}
\end{equation}
and
\begin{equation}
  \mat{B} = \begin{bmatrix}
    2 & 1 & 7\\
    6 & 1 & 0\\
    1 & -1 & 5
\end{bmatrix}.
\end{equation}
Then we find the component in the first column, first row, of the
product is:
\begin{equation}
  \begin{split}
  \mat{A}\mat{B} = \left[\begin{array}{ccc}\rowcolor{olive!20}3 & 1 & 6\\
      2 & 1 & 3
  \end{array}\right]\left[\begin{array}{>{\columncolor{olive!20}}ccc}
    2 & 1 & 7\\
    6 & 1 & 0\\
    1 & -1 & 5
    \end{array}\right]
  &= \begin{bmatrix}\colorbox{olive!20}{$3\cdot2 + 1\cdot6 + 6\cdot1$} & ? & ?\\
    ? & ? & ?
  \end{bmatrix}\\
  &= \begin{bmatrix}\colorbox{olive!20}{$18$} & ? & ?\\
    ? & ? & ?
  \end{bmatrix}
  \end{split}
\end{equation}
The next entry in the first row is:
\begin{equation}
  \begin{split}
  \mat{A}\mat{B} = \left[\begin{array}{ccc}\rowcolor{olive!20}3 & 1 & 6\\
      2 & 1 & 3
  \end{array}\right]\left[\begin{array}{c>{\columncolor{olive!20}}cc}
    2 & 1 & 7\\
    6 & 1 & 0\\
    1 & -1 & 5
    \end{array}\right]
  &= \begin{bmatrix}18 & \colorbox{olive!20}{$3\cdot1 + 1\cdot1 + 6\cdot-1$} & ?\\
    ? & ? & ?
  \end{bmatrix}\\
  &= \begin{bmatrix}18 & \colorbox{olive!20}{$-2$} & ?\\
    ? & ? & ?
  \end{bmatrix}
  \end{split}
\end{equation}
The last entry on the first row:
\begin{equation}
  \begin{split}
  \mat{A}\mat{B} = \left[\begin{array}{ccc}\rowcolor{olive!20}3 & 1 & 6\\
      2 & 1 & 3
  \end{array}\right]\left[\begin{array}{cc>{\columncolor{olive!20}}c}
    2 & 1 & 7\\
    6 & 1 & 0\\
    1 & -1 & 5
    \end{array}\right]
  &= \begin{bmatrix}18 & -2 & \colorbox{olive!20}{$3\cdot7 + 1\cdot0 + 6\cdot5$}\\
    ? & ? & ?
  \end{bmatrix}\\
  &= \begin{bmatrix}18 & -2 & \colorbox{olive!20}{$51$}\\
    ? & ? & ?
  \end{bmatrix}
  \end{split}
\end{equation}
We can continue to the second row:
\begin{equation}
  \begin{split}
  \mat{A}\mat{B} = \left[\begin{array}{ccc}3 & 1 & 6\\
\rowcolor{olive!20}2 & 1 & 3
  \end{array}\right]\left[\begin{array}{>{\columncolor{olive!20}}ccc}
    2 & 1 & 7\\
    6 & 1 & 0\\
    1 & -1 & 5
    \end{array}\right]
  &= \begin{bmatrix}18 & -2 & 51\\
    \colorbox{olive!20}{$2\cdot2 + 1\cdot6 + 3\cdot1$} & ? & ?
  \end{bmatrix}\\
  &= \begin{bmatrix}18 & -2 & 51\\
    \colorbox{olive!20}{$13$} & ? & ?
  \end{bmatrix}
  \end{split}
\end{equation}
The next column in the second row:
\begin{equation}
  \begin{split}
  \mat{A}\mat{B} = \left[\begin{array}{ccc}3 & 1 & 6\\
\rowcolor{olive!20}2 & 1 & 3
  \end{array}\right]\left[\begin{array}{c>{\columncolor{olive!20}}cc}
    2 & 1 & 7\\
    6 & 1 & 0\\
    1 & -1 & 5
    \end{array}\right]
  &= \begin{bmatrix}18 & -2 & 51\\
    13 & \colorbox{olive!20}{$2\cdot1 + 1\cdot1 + 3\cdot-1$} & ?
  \end{bmatrix}\\
  &= \begin{bmatrix}18 & -2 & 51\\
    13 & \colorbox{olive!20}{$0$} & ?
  \end{bmatrix}
  \end{split}
\end{equation}
Finally, the remaining entry:
\begin{equation}
  \begin{split}
  \mat{A}\mat{B} = \left[\begin{array}{ccc}3 & 1 & 6\\
\rowcolor{olive!20}2 & 1 & 3
  \end{array}\right]\left[\begin{array}{c>{\columncolor{olive!20}}cc}
    2 & 1 & 7\\
    6 & 1 & 0\\
    1 & -1 & 5
    \end{array}\right]
  &= \begin{bmatrix}18 & -2 & 51\\
    13 & 0 & \colorbox{olive!20}{$2\cdot7 + 1\cdot0 + 3\cdot5$}
  \end{bmatrix}\\
  &= \begin{bmatrix}18 & -2 & 51\\
    13 & 0 & \colorbox{olive!20}{$29$}
  \end{bmatrix}
  \end{split}
\end{equation}
\end{subequations}
\end{example}

\N{Recovering System of Linear Equations}
This is pretty random, why on Earth should we accept it?
Well, the biggest reason is because we recover a system of linear
equations by multiplying an $m\times n$ matrix by an $n\times 1$ column
vector.
If we write our $m\times n$ matrix $\mat{A}$ as $m$ row vectors
\begin{equation*}
  \mat{A} = \begin{bmatrix}\transpose{\vec{a}}_{1}\\
    \transpose{\vec{a}}_{2}\\
    \vdots\\
    \transpose{\vec{a}}_{m}
  \end{bmatrix}
\end{equation*}
then we see that multiplying it by our column vector $\vec{x}$ produces
\begin{equation}
  \mat{A}\vec{x} = \begin{bmatrix}\vec{a}_{1}\cdot\vec{x}\\
    \vec{a}_{2}\cdot\vec{x}\\
    \vdots\\
    \vec{a}_{m}\cdot\vec{x}
  \end{bmatrix}.
\end{equation}
If we have an $m$-vector of constants $\vec{b}$, and $\vec{x}$ were a
vector of unknowns, then
\begin{equation}
\mat{A}\vec{x} = \vec{b}
\end{equation}
is precisely a system of $m$ linear equations in $n$ unknowns. This is
wonderful: it's what we were trying to do all along!

\begin{example}
Let $\mat{A}=(a_{j,k})$ be an $m\times n$ matrix, consider the matrix
multiplication of the $m\times m$ identity matrix $\mat{I}_{m}$ with
$\mat{A}$. We find
\begin{calculation}
  (\mat{I}_{m}\mat{A})_{i,k}
\step{definition of matrix multiplication}
  \sum_{j=1}^{m}\delta_{i,j}a_{j,k}
  \step{breaking up the sum}
  \left(\sum_{j=1}^{i-1}\delta_{i,j}a_{j,k}\right)+\delta_{i,i}a_{i,k}+\left(\sum_{j=i+1}^{m}\delta_{i,j}a_{j,k}\right)
  \step{since $\delta_{i,j}=0$ if $i\neq j$}
  0 + \delta_{i,i}a_{i,k} + 0
\step{since $\delta_{i,i}=1$}
  a_{i,k}
\step{folding back $\mat{A}$ into the result}
  (\mat{A})_{i,k}.
\end{calculation}
Hence $\mat{I}_{m}\mat{A} = \mat{A}$.
\end{example}

\begin{example}
  Let
  \begin{subequations}
\begin{equation}
\mat{A} = \begin{bmatrix}1 & 2\\3 & 4
\end{bmatrix},
\end{equation}
and $\mat{I}_{2}$ be the 2-by-2 identity matrix. Then
\begin{equation}
\mat{A}\mat{I}_{2} = \mat{A}.
\end{equation}
  \end{subequations}
\end{example}

\begin{example}
Let $\mat{A}$ be a $1\times n$ matrix (i.e., a row $n$-vector) and
$\mat{B}$ be a $n\times 1$ matrix (i.e., a column $n$ vector). Let us
write $\mat{A}=(a_{1},\dots,a_{n})$ and $\mat{B}=\transpose{(b_{1},\dots,b_{n})}$.
Then
\begin{equation}
\mat{A}\mat{B} = \sum^{n}_{j=1}a_{j}b_{j}=a_{1}b_{1}+a_{2}b_{2}+\dots+a_{n}b_{n}.
\end{equation}
Does this look familiar? It's the dot product of two vectors $\vec{a}$,
$\vec{b}$. We specifically have, if $\vec{a}$ and $\vec{b}$ are both
column $n$-vectors, then
\begin{equation}
\vec{a}\cdot\vec{b} = \transpose{\vec{a}}\vec{b}.
\end{equation}
This relates matrix multiplication to the transpose and dot product.
\end{example}

\N{Rephrasing the definition}
If we have an $m\times n$ matrix $\mat{A}$ and an $n\times 2$ matrix
$\mat{B}$, we could view $\mat{B}$ as a pair of column $n$-vectors
\begin{equation}
\mat{B} = (\vec{b}_{1}, \vec{b}_{2}).
\end{equation}
In this case, we see that matrix multiplication amounts to,
\begin{equation}
\mat{A}\mat{B} = (\mat{A}\vec{b}_{1}, \mat{A}\vec{b}_{2}).
\end{equation}
We can continue in this manner, for an arbitrary $n\times p$ matrix
$\mat{C}$ thinking of it as $p$ column $n$-vectors
$\mat{C}=(\vec{c}_{1},\dots,\vec{c}_{p})$. Then matrix multiplication
amounts to
\begin{equation}
\mat{A}\mat{C} = (\mat{A}\vec{c}_{1}, \dots, \mat{A}\vec{c}_{p}).
\end{equation}
This is just rephrasing the definition more vividly.

\N{Concerns about the definition}
We should not get too ahead of ourselves here, we want to check matrix
multiplication has nice properties. Presumably not all the properties of
multiplying numbers, but hopefully enough of them. We should check for
associativity (very important property), commutativity (nice but
inessential), and distributivity over [matrix] addition.

\begin{theorem}[Associativity]
Let $\mat{A}$ be an $m\times n$ matrix, $\mat{B}$ be an $n\times p$
matrix, and $\mat{C}$ a $p\times q$ matrix.
Then matrix multiplication is associative, i.e.,
\begin{equation}
\mat{A}(\mat{B}\mat{C}) = (\mat{A}\mat{B})\mat{C}.
\end{equation}
\end{theorem}

There are two ways to prove this. The first is to define
$\mat{R}=\mat{B}\mat{C}$ and $\mat{L}=\mat{A}\mat{B}$, then unfold the
definitions to show $\mat{A}(\mat{B}\mat{C})=\mat{A}\mat{R}$ (by
definition of $\mat{R}$) and $(\mat{A}\mat{B})\mat{C}=\mat{L}\mat{C}$,
and then unfolding the definition of matrix multiplication we would
prove $\mat{A}\mat{R}=\mat{L}\mat{C}$. This involves rather nasty nested
sums.

The second approach is by induction on $q$. When $q=1$, we have
$\mat{C}$ be a column $p$-vector $\mat{C}=\vec{c}_{1}$. Matrix
multiplication becomes far more intuitive in this case. Proving
$\mat{A}(\mat{B}\vec{c}_{1})=(\mat{A}\mat{B})\vec{c}_{1}$ is the base
case of induction. Then we assume the inductive hypothesis (i.e., this
works for arbitrary $q$, that
$(\mat{A}\mat{B})\mat{C}_{q}=\mat{A}(\mat{B}\mat{C}_{q})$). Then we have
to prove the inductive case, when $\mat{C}=(\mat{C}_{q},\vec{c}_{q+1})$
is the block structure of $\mat{C}$. Intuitively this describes the
process of ``adding another column to $\mat{C}$, and proving
associativity still holds''. When combined with the base case, it
suffices to prove this works for any $q$.

\begin{remark}
One intuition of vectors is that they describe a certain state or
configuration, where each component refers to a different
degree-of-freedom. In Neo-Ricardian economics, this would be one
possible stock of goods (each component referring to a different commodity).
In physics, the components would be the coordinates for the positions of
various bodies.

Matrices then are used to transform states $\mat{A}\vec{x}_{\text{old}}=\vec{x}_{\text{new}}$. In Neo-Ricardian economics,
this is precisely the production process. In physics, this could be the
effect of rotation about an axis by a certain angle, or time-evolution
forward a certain amount of time.

Matrix multiplication describes composing these transformations (from
right to left), if $\mat{A}_{1}\vec{x}_{0}=\vec{x}_{1}$ describes how
$\mat{A}_{1}$ transforms $\vec{x}_{0}$, then
$\mat{A}_{2}\mat{A}_{1}\vec{x}_{0} = \mat{A}_{2}\vec{x}_{1}$. We could
use associativity to create a composite process
$\mat{A}_{\text{composite}}=\mat{A}_{2}\mat{A}_{1}$. But what's more, we
could create a composite process from any (finite number) of
intermediate processes, by matrix multiplication!
\end{remark}

\begin{example}[Matrix Multiplication is NOT commutative]
Lets compute
\begin{subequations}
\begin{equation}
\begin{bmatrix}1 & 1\\0 & 1 \end{bmatrix}
\begin{bmatrix}1 & 0\\1 & 1 \end{bmatrix}
=\begin{bmatrix}2 & 1\\1 & 1
\end{bmatrix}
\end{equation}
But at the same time, if we try commuting these two matrices, we get
\begin{equation}
\begin{bmatrix}1 & 0\\1 & 1 \end{bmatrix}
\begin{bmatrix}1 & 1\\0 & 1 \end{bmatrix}
=\begin{bmatrix}1 & 1\\1 & 2
\end{bmatrix}.
\end{equation}
\end{subequations}
Hence we conclude matrix multiplication is noncommutative, because we
have found a counter-example. And one counter-example is all we need to
disprove the hypothesis ``Matrix multiplication is commutative''.
\end{example}

\begin{proposition}
Let $r$ be a number, let $\mat{A}$ be an $m\times n$ matrix, let
$\mat{B}$ be an $n\times p$ matrix.
Then $\mat{A}(r\mat{B})=r(\mat{A}\mat{B})=(r\mat{A})\mat{B}$.
\end{proposition}

\begin{proposition}\label{prop:product-of-transpose}
Let $\mat{A}$ be an $m\times n$ matrix, let $\mat{B}$ be an $n\times p$ matrix.
Then the transpose of matrix product is the matrix multiplication of the
transposes, $\transpose{(\mat{A}\mat{B})} = \transpose{\mat{B}}\transpose{\mat{A}}$.
\end{proposition}

The proof will involve two steps: (1) expanding out
$(\transpose{(\mat{A}\mat{B})})_{k,i}$, and
(2) expanding out $(\transpose{\mat{B}}\transpose{\mat{A}})_{k,i}$.
Then we will find they are identical for every $i$, $k$.

\begin{proof}
  First we find
\begin{subequations}
\begin{calculation}
  (\transpose{\mat{B}}\transpose{\mat{A}})_{k,i}
\step*{definition of matrix multiplication}
  \sum_{j=1}^{n}(\transpose{\mat{B}})_{k,j}(\transpose{\mat{A}})_{j,i}
\step*{definition of transpose}
  \sum_{j=1}^{n}(\mat{B})_{j,k}(\mat{A})_{i,j}
\step{commutativity of multiplying components}
  \sum_{j=1}^{n}(\mat{A})_{i,j}(\mat{B})_{j,k}.
\end{calculation}

We similarly find, starting ``from the other end'' of the desired result,
\begin{calculation}
  \left(\transpose{(\mat{A}\mat{B})}\right)_{k,i}
\step*{definition of transpose}
  (\mat{A}\mat{B})_{i,k}
\step{definition of matrix multiplication}
  \sum^{n}_{j=1}(\mat{A})_{i,j}(\mat{B})_{j,k}.
\end{calculation}
But by comparing these two results, we find them identical, and since
it's true for every component of the product, we are forced to conclude
\begin{equation}
\transpose{(\mat{A}\mat{B})} = \transpose{\mat{B}}\transpose{\mat{A}}
\end{equation}
\end{subequations}
Hence the result.
\end{proof}

\begin{problem}
Is it true that $\transpose{(\mat{A}\mat{B}\mat{C})} = \transpose{\mat{C}}\transpose{\mat{B}}\transpose{\mat{A}}$?
What if we had $n$ factors, is it true that $\transpose{(\mat{A}_{1}\dots\mat{A}_{n})} = \transpose{\mat{A}}_{n}\dots\transpose{\mat{A}}_{1}$?
\end{problem}

\begin{example}[Plane geometry]\label{ex:matrix-algebra:plane-geometry}
We can look at $2\times2$ matrices as acting on points $(x,y)\in\RR^{2}$
on the plane. We turn $(x,y)$ into a column vector.

We have rotation anticlockwise by an angle
$\theta$ given by
\begin{equation}
\mat{A}_{\theta} = \begin{bmatrix}\cos(\theta) & -\sin(\theta)\\
\sin(\theta) & \cos(\theta)
\end{bmatrix}.
\end{equation}
Reflection about the $x$-axis is
\begin{equation}
\mat{R}  =\begin{bmatrix}1 & 0\\0 & -1
\end{bmatrix}.
\end{equation}
Dilation by a positive real number $\lambda>0$ is a diagonal matrix
\begin{equation}
  \mat{D}_{\lambda} =\begin{bmatrix}\lambda & 0\\
  0 & \lambda
  \end{bmatrix}.
\end{equation}
Observe for $0<\lambda<1$, $\mat{D}_{\lambda}$ is a contraction.
\end{example}

\begin{definition}
Let $n$ be a non-negative integer, let $\mat{A}$ be a square matrix.
We define the \define{Matrix Power} of $\mat{A}$ raised to the
$n^{\text{th}}$ power as the matrix $\mat{A}^{n}$ inductively defined by:
\begin{enumerate}
\item $\mat{A}^{0}=\mat{I}$, and
\item $\mat{A}^{n+1}=\mat{A}^{n}\mat{A}$ (and in particular $\mat{A}^{1}=\mat{A}$).
\end{enumerate}
So we have
\begin{equation}
\underbrace{\mat{A}\cdots\mat{A}}_{n~\text{times}}=\prod^{n}_{j=1}\mat{A}=\mat{A}^{n}.
\end{equation}
\end{definition}

\begin{example}
  Consider
  \begin{equation}
\mat{J} = \begin{bmatrix}0 & -1\\1 & 0
\end{bmatrix}.
  \end{equation}
  Then
  \begin{equation}
    \mat{J}^{2} = \begin{bmatrix}-1 & 0\\
      0 & -1
    \end{bmatrix} = -\mat{I}_{2}.
  \end{equation}
  Thus we have discovered some matrix which ``acts like'' $\sqrt{-1}$.
There is something profound here, if we examine Example~\ref{ex:matrix-algebra:plane-geometry},
we will find $J$ amounts to a rotation in $\RR^{2}$ anticlockwise by
$90^{\circ}$. This is precisely what happens if we multiply numbers in
the complex plane by $\I=\sqrt{-1}$.
\end{example}

\begin{example}
  The Fibonacci sequence is defined by $F_{0}=0$, $F_{1}=1$, and
  \begin{equation}
F_{n+1} = F_{n} + F_{n-1}.
  \end{equation}
  We see that
  \begin{equation}
F_{n+2} = F_{n+1} + F_{n} = (F_{n} + F_{n-1}) = 2F_{n}+F_{n-1}.
  \end{equation}
  Then we can describe it using a system of equations
  \begin{equation}
\begin{pmatrix}
1 & 1\\
1 & 2
\end{pmatrix}
\begin{pmatrix}F_{n-1}\\ F_{n}
\end{pmatrix} = \begin{pmatrix}F_{n+1}\\ F_{n+2}
\end{pmatrix}.
  \end{equation}
  We can use matrix power to simplify calculations to
  \begin{equation}
\begin{pmatrix}
1 & 1\\
1 & 2
\end{pmatrix}^{n}
\begin{pmatrix}0\\1
\end{pmatrix} = \begin{pmatrix}F_{n}\\ F_{n+1}
\end{pmatrix}.
\end{equation}
We will later find a really slick way to compute the powers of a matrix
quickly, because right now all we've done is rephrased the recurrence
relation in new notation.
\end{example}

\begin{problem}
  For any square matrix $\mat{A}$ and non-negative integers $p$ and $q$,
  prove $(\mat{A}^{p})^{q} = \mat{A}^{pq}$
  and $\mat{A}^{p}\mat{A}^{q}=\mat{A}^{p+q}$.
\end{problem}

\begin{problem}
Let $\displaystyle\mat{A}=\begin{bmatrix}1 & 1\\0 & 1 \end{bmatrix}$.
Compute $\mat{A}^{n}$. Start with $n=2$, $3$, $4$, then try to
generalize the results.
\end{problem}

\N{Puzzle}
Suppose we have an $m\times n$ matrix $\mat{A}$. When will there be an
$n\times m$ matrix $\mat{B}$ such that $\mat{B}\mat{A}=\mat{I}_{n}$? Or
$\mat{A}\mat{B}=\mat{I}_{m}$?

\subsection{Matrix Inverse}

\begin{definition}
Let $\mat{A}$ be an $n\times n$ matrix.
We call an $n\times n$ matrix $\mat{B}$ an \define{Inverse} of $\mat{A}$
if
\begin{equation}
\mat{A}\mat{B}=\mat{B}\mat{A}=\mat{I}_{n}.
\end{equation}
In this case, we call $\mat{A}$ \define{Invertible} (or \define{Nonsingular}).

Otherwise, if no such $\mat{B}$ exists, then $\mat{A}$ is called a
\define{Noninvertible} (or \define{Singular}) matrix.
\end{definition}

\begin{example}
The identity matrix is its own inverse. The zero matrix has no inverse.
\end{example}

\begin{theorem}[Uniqueness of inverse matrix]
Let $\mat{A}$ be an $n\times n$ matrix. Assume $\mat{B}$ is the inverse
matrix for $\mat{A}$. Then $\mat{B}$ is unique.
\end{theorem}

By ``unique'', we mean if we happen to come across another inverse of
$\mat{A}$, then it will be equal (by matrix equality) to $\mat{B}$.

\begin{proof}
Let $\mat{B}_{1}$ and $\mat{B}_{2}$ be inverse matrices for $\mat{A}$. 
We will prove $\mat{B}_{1}=\mat{B}_{2}$.

Consider the following calculation:
\begin{subequations}
\begin{calculation}
  \mat{B}_{1}
\step{defining property of identity matrix}
  \mat{B}_{1}\mat{I}_{n}
\step{definition of inverse matrix}
  \mat{B}_{1}(\mat{A}\mat{B}_{2})
\step{associativity of matrix multiplication}
  (\mat{B}_{1}\mat{A})\mat{B}_{2}
\step{definition of matrix inverse}
  \mat{I}_{n}\mat{B}_{2}
\step{defining property of identity matrix}
  \mat{B}_{2}.
\end{calculation}
\end{subequations}
Hence we conclude $\mat{B}_{1}=\mat{B}_{2}$, as desired.
\end{proof}

\N{Notation for Inverse Matrix}
Since the inverse matrix is unique (if it exists), we will denote the
matrix inverse of $\mat{A}$ by $\mat{A}^{-1}$.

\begin{example}
  Consider the matrix
  \begin{equation}
\varepsilon = \begin{bmatrix}0 & 1\\
0 & 0
\end{bmatrix}.
  \end{equation}
  This is noninvertible. How can we see it? Well, we observe
  \begin{equation}\label{eq:example-of-nilpotent-matrix}
\varepsilon^{2}=\mat{0}.
  \end{equation}
  If $\varepsilon$ had an inverse matrix $\mat{A}$, then
\begin{subequations}
\begin{calculation}
  \varepsilon
\step{defining property of identity matrix}
  \mat{I}_{2}\varepsilon
\step{using the fact $\mat{A}$ is an inverse matrix}
  (\mat{A}\varepsilon)\varepsilon 
\step{associativity of matrix multiplication}
  \mat{A}(\varepsilon^{2})
\step{by Eq~\eqref{eq:example-of-nilpotent-matrix}}
  \mat{A}(\mat{0})
\step{defining property of zero matrix}
  \mat{0}.
\end{calculation}
\end{subequations}
Hence if $\varepsilon$ were invertible, we would have
$\varepsilon=\mat{0}$. But this contradicts the definition of $\varepsilon$,
which is a nonzero matrix. Thus we are forced to conclude $\varepsilon$
is noninvertible.
\end{example}

\N{Solving Systems of Linear Equations}
Returning to our original motivation for this diversion into matrix
algebra, we can now use matrix inversion to solve systems of linear
equations. We do it thus:\footnote{Since we are proving equations are
logically equivalent to each other, we write $(A=B)\equiv(A'=B')$ for
``$(A=B)$ is logically equivalent to $(A'=B')$, in the sense that one
implies the other \emph{and} vice-versa''.}
\begin{subequations}
\begin{calculation}\gdef\defaultrelation{\equiv}
  \mat{A}\vec{x} = \vec{b}
\step{multiply both sides by $\mat{A}^{-1}$}
  \mat{A}^{-1}(\mat{A}\vec{x}) = \mat{A}^{-1}\vec{b}
\step{associativity of matrix multiplication}
  (\mat{A}^{-1}\mat{A})\vec{x} = \mat{A}^{-1}\vec{b}
\step{definition of matrix inverse}
  \mat{I}_{n}\vec{x} = \mat{A}^{-1}\vec{b}
\step{defining property of identity matrix}
  \vec{x} = \mat{A}^{-1}\vec{b}
\end{calculation}
\end{subequations}
But we have just traded notation for notation: we currently have no
algorithm to compute the inverse of a matrix! While this is true, we can
still meaningfully prove properties about the matrix inverse (a useful
skill for mathematicians to know, how to prove things despite lacking an
algorithm for computing it). But we will address this problem in the
next couple sections.

\begin{theorem}[Inverse of products]
Let $\mat{A}$, $\mat{B}$ be invertible [square] matrices.
Then $(\mat{A}\mat{B})^{-1} = \mat{B}^{-1}\mat{A}^{-1}$.
\end{theorem}

\begin{proof}
  We compute directly
\begin{subequations}
\begin{calculation}\gdef\defaultrelation{\equiv}
  (\mat{A}\mat{B})^{-1}\mat{A}\mat{B} = \mat{I}
\step{multiply on right by $\mat{B}^{-1}$}
  (\mat{A}\mat{B})^{-1}\mat{A}\mat{B}\mat{B}^{-1} = \mat{I}\mat{B}^{-1}
\step{associativity of matrix multiplication, defining property of $\mat{I}$}
  (\mat{A}\mat{B})^{-1}\mat{A}(\mat{B}\mat{B}^{-1}) = \mat{B}^{-1}
\step{definition of matrix inverse of $\mat{B}$}
  (\mat{A}\mat{B})^{-1}\mat{A}\mat{I} = \mat{B}^{-1}
\step{defining property of identity matrix}
  (\mat{A}\mat{B})^{-1}\mat{A} = \mat{B}^{-1}
\step{multiply both sides on right by $\mat{A}^{-1}$}
  (\mat{A}\mat{B})^{-1}\mat{A}\mat{A}^{-1} = \mat{B}^{-1}\mat{A}^{-1}
\step{associativity of matrix multiplication}
  (\mat{A}\mat{B})^{-1}(\mat{A}\mat{A}^{-1}) = \mat{B}^{-1}\mat{A}^{-1}
\step{definition of matrix inverse}
  (\mat{A}\mat{B})^{-1}\mat{I} = \mat{B}^{-1}\mat{A}^{-1}
\step{defining property of identity matrix}
  (\mat{A}\mat{B})^{-1} = \mat{B}^{-1}\mat{A}^{-1}
\end{calculation}
\end{subequations}
Hence we obtain the desired result.
\end{proof}

\begin{proposition}[Inverse commutes with transpose]
  Let $\mat{A}$ be an invertible matrix. Then
  $\transpose{(\mat{A}^{-1})} = (\transpose{\mat{A}})^{-1}$.
\end{proposition}

\begin{proof}
Recall Proposition~\ref{prop:product-of-transpose} which established the
product of the transpose is the product-in-reverse-order of transposes
$\transpose{(\mat{A}\mat{B})} = \transpose{\mat{B}}\transpose{\mat{A}}$.
Set $\mat{B}=\mat{A}^{-1}$.
We find the left hand side becomes,
\begin{equation}
\transpose{(\mat{A}\mat{B})} = \transpose{(\mat{A}\mat{A}^{-1})} = \transpose{\mat{I}}=\mat{I};
\end{equation}
the right-hand side becomes
\begin{equation}
\transpose{\mat{B}}\transpose{\mat{A}} = \transpose{(\mat{A}^{-1})}\transpose{\mat{A}}.
\end{equation}
Setting these equal yields
\begin{subequations}
\begin{calculation}\gdef\defaultrelation{\equiv}
  \transpose{(\mat{A}^{-1})}\transpose{\mat{A}} = \mat{I}
\step{multiply both sides on the right by $(\transpose{\mat{A}})^{-1}$}
  \transpose{(\mat{A}^{-1})}\transpose{\mat{A}}(\transpose{\mat{A}})^{-1} = \mat{I}(\transpose{\mat{A}})^{-1}
\step{associativity of matrix multiplication, defining property of identity matrix}
  \transpose{(\mat{A}^{-1})}(\transpose{\mat{A}}(\transpose{\mat{A}})^{-1}) = (\transpose{\mat{A}})^{-1}
\step{definition of matrix inverse}
  \transpose{(\mat{A}^{-1})}\mat{I} = (\transpose{\mat{A}})^{-1}
\step{defining property of identity matrix}
  \transpose{(\mat{A}^{-1})} = (\transpose{\mat{A}})^{-1}
\end{calculation}
\end{subequations}
Hence we obtain the desired result.
\end{proof}

\begin{proposition}[Matrix inversion is idempotent]
Let $\mat{A}$ be an invertible matrix. Then $(\mat{A}^{-1})^{-1}=\mat{A}$
\end{proposition}

\begin{proof}
Let $\mat{B}=\mat{A}^{-1}$. Then $\mat{A}\mat{B}=\mat{B}\mat{A}=\mat{I}$.
But this implies $\mat{A} = \mat{B}^{-1} = (\mat{A}^{-1})^{-1}$, as desired.
\end{proof}
\section{Solving Linear Equations with Augmented Matrices}

\M
Recall (\S\ref{par:matrix-algebra:solving-systems-of-equations})
we found a way to encode a system of linear equations using matrix
notation as
\begin{equation}
\mat{A}\vec{x}=\vec{b}.
\end{equation}
But we got no farther than that. Now we will begin to solve such a
system in an algorithmic manner.

Our first step will be to work with an \define{Augmented Matrix}, which
has block form
\begin{equation}
(\mat{A}\mid\vec{b}).
\end{equation}
Since the unknowns (encoded in the vector $\vec{x}$) are not known, we
omit them from the augmented matrix. For a concrete example:
\begin{equation}
  \begin{pmatrix}
  4 &  3 &  3\\
 -4 & -1 &  1\\
 -1 & -1 & -1\\
  0 &  2 &  1\\
  \end{pmatrix}\begin{pmatrix}x_{1}\\x_{2}\\x_{3}\\x_{4}
  \end{pmatrix} =
  \begin{pmatrix}-5\\1\\-2\\-1
  \end{pmatrix}
\mapsto
  \left[\begin{array}{ccc|c}
  4 &  3 &  3 & -5\\
 -4 & -1 &  1 &  1\\
 -1 & -1 & -1 & -2\\
  0 &  2 &  1 & -1 
  \end{array}\right]
\end{equation}
This is our first step, now what?

We will apply certain elementary row operations to repeatedly modify our
augmented matrix until it's in a form suitable for solving. Let us
define our elementary row operations, then lay out the criteria for an
augmented matrix being ``suitably nice''.

\begin{definition}
Let $\mat{A}$ be a matrix. We define an \define{Elementary Row Operation}
to be one of the following:
\begin{enumerate}
\item Multiply row $i$ by a nonzero scalar $r$,
\item Add a multiple of row $i$ to row $j$,
\item Swap rows $i$ and $j$.
\end{enumerate}
\end{definition}

\begin{remark}
These elementary row operations correspond to (respectively):
\begin{enumerate}
\item Multiply both sides of equation $i$ by a nonzero number $r$,
\item Add a multiple of equation $i$ to equation $j$,
\item Swap equations $i$ and $j$.
\end{enumerate}
We see this is permitted by elementary algebra.
\end{remark}

\begin{definition}
Let $\mat{A}$ be a matrix. We say it is in \define{Reduced Row Echelon Form}
\begin{enumerate}
\item All rows consisting of zeroes are at the bottom of the matrix, and
\item The leading coefficient of a nonzero row is strictly ``to the
  right'' of the leading coefficient of the row above it, and
\item The leading coefficient is 1.
\end{enumerate}
\end{definition}

\begin{remark}
Some authors do not require ``the leading coefficient is 1'' to the
criteria, but we require it.
\end{remark}

\begin{example}
  A generic reduced row echelon form matrix looks like
  \begin{equation}
\begin{bmatrix}
1 & a_{1,2} & a_{1,3} & \dots & a_{1,j} & a_{1,j+1} & a_{1,j+2} & a_{1,j+3} & \dots& a_{1,n}\\
0 & 1 & a_{2,3} & \dots & a_{2,j} & a_{2,j+1} & a_{2,j+2} & a_{2,j+3} & \dots& a_{2,n}\\
0 & 0 & 1 & \dots & a_{3,j} & a_{3,j+1} & a_{3,j+2} & a_{3,j+3} & \dots& a_{3,n}\\
\vdots & \vdots & \vdots & \ddots & \vdots & \vdots & \vdots & \vdots & \ddots & \vdots\\
0 & 0 & 0 & \dots & 0 & 1 & a_{i,j+2} & a_{i,j+3} & \dots & a_{i,n}\\
0 & 0 & 0 & \dots & 0 & 0 & 0        & 1        & \dots & a_{i+1,n}\\ 
0 & 0 & 0 & \dots & 0 & 0 & 0 & 0 & \dots & 0\\
0 & 0 & 0 & \dots & 0 & 0 & 0 & 0 & \dots & 0\\
0 & 0 & 0 & \dots & 0 & 0 & 0 & 0 & \dots & 0
\end{bmatrix}
  \end{equation}
\end{example}

\begin{example}
  For an augmented matrix in reduced row echelon form, we have
  \begin{equation}
\left[\begin{array}{ccc|c}
1 & 2 & 0 & 1\\
0 & 1 & 0 & 1\\
0 & 0 & 1 & 2
  \end{array}\right]  
  \end{equation}
  This would correspond to the system of equations
\begin{alignat*}{7}
1x_{1} && + && 2x_{2} && + && 0x_{3} &&\;=\;&&1\\
0x_{1} && + && 1x_{2} && + && 0x_{3} &&\;=\;&&1\\
0x_{1} && + && 0x_{2} && + && 1x_{3} &&\;=\;&&2
\end{alignat*}
We can solve this immediately, finding $x_{3}=2$, $x_{2}=1$, and $x_{1}=-1$.
\end{example}

\subsection{Gauss--Jordan Elimination}

\M
The idea is to use elementary row operations to transform our augmented
matrix $(\mat{A}\mid\vec{b})$ to reduced row echelon form $(\mat{B}\mid\vec{c})$,
then solve the system of equations $\mat{B}\vec{x}=\vec{c}$ which will
also be a solution to $\mat{A}\vec{x}=\vec{b}$.

We see that the two augmented matrices $(\mat{A}\mid\vec{b})$ and
$(\mat{B}\mid\vec{c})$ are ``the same'', very analogous to how the
fractions $1/2$ and $2/4$ are ``the same'' despite having different
numerators and denominators. Before demonstrating Gaussian elimination,
let us formalize this notion of ``sameness''.

\begin{definition}
We call two augmented matrices $(\mat{A}\mid\vec{b})$ and
$(\mat{B}\mid\vec{c})$ \define{Equivalent} if they have the same
solutions, and denote this by $(\mat{A}\mid\vec{b})\sim(\mat{B}\mid\vec{c})$.

More generally, any two $m\times n$ matrices $\mat{A}$, $\mat{B}$
are \define{Equivalent} if there is a finite sequence of elementary row
operations that transforms $\mat{A}$ into $\mat{B}$. (Further,
``equivalence of augmented matrices'' coincides with ``equivalence of matrices'',
so we will use the same symbol $\sim$ for both of them.)
\end{definition}

\begin{proposition}
  Matrix equivalence satisfies the following properties:
  \begin{enumerate}
  \item Reflexivity: for any matrix $\mat{A}$, we have $\mat{A}\sim\mat{A}$
  \item Symmetry: for any $m\times n$ matrices $\mat{A}$ and $\mat{B}$,
    we have $\mat{A}\sim\mat{B}$ implies $\mat{B}\sim\mat{A}$
  \item Transitivity: for any $m\times n$ matrices $\mat{A}$, $\mat{B}$, $\mat{C}$,
    we have $\mat{A}\sim\mat{B}$ and $\mat{B}\sim\mat{C}$ implies $\mat{A}\sim\mat{C}$.
  \end{enumerate}
\end{proposition}


\N{Gauss--Jordan Elimination}
Consider the augmented matrix
\begin{subequations}
\begin{equation}
\left[\begin{array}{ccc;{2pt/2pt}c}
    0 & 0 & 1 & 6\\
    2 & 4 & 6 & 8\\
    2 & 3 & 4 & 5
  \end{array}\right]
\end{equation}
The algorithm for transforming it to reduced row echelon form consists
of the following steps.

\N*{Step 1: Locate nonzero column} Look for the first column that is not all zeroes. We can
see this is the first column (highlighted):
\begin{equation}
\left[\begin{array}{>{\columncolor{red!20}}ccc;{2pt/2pt}c}
    0 & 0 & 1 & 6\\
    2 & 4 & 6 & 8\\
    2 & 3 & 4 & 5
  \end{array}\right]
\end{equation}
If, for some reason, the first column is a zero column vector, move to
the next column to the right (and keep moving until you find the first
column vector which is not a zero column vector); we will ignore the
zero columns, and refer to the first nonzero column vector as ``the
first column'' or ``the leading column''.

\N*{Step 2: Pivot} If the first row has a zero entry in the leading
column, swap a row with a nonzero entry in the leading column. We could
pick \emph{any} such row with a nonzero entry in the leading column, we
will swap the second row
\begin{equation}
\left[\begin{array}{ccc;{2pt/2pt}c}
\rowcolor{olive!20}    0 & 0 & 1 & 6\\
\rowcolor{red!20}    2 & 4 & 6 & 8\\
    2 & 3 & 4 & 5
  \end{array}\right]\sim
\left[\begin{array}{ccc;{2pt/2pt}c}
\rowcolor{red!20}    2 & 4 & 6 & 8\\
\rowcolor{olive!20}    0 & 0 & 1 & 6\\
    2 & 3 & 4 & 5
  \end{array}\right].
\end{equation}
\N*{Step 3: Normalize} We now normalize the top row by diving through by
the nonzero pivot (or, equivalent, multiplying by
$1/(\mbox{pivot})$). For us, this is dividing the first row by 2:
\begin{equation}
\left[\begin{array}{ccc;{2pt/2pt}c}
\rowcolor{red!20}    2 & 4 & 6 & 8\\
0 & 0 & 1 & 6\\
    2 & 3 & 4 & 5
  \end{array}\right]\sim
\left[\begin{array}{ccc;{2pt/2pt}c}
\rowcolor{red!20}    1 & 2 & 3 & 4\\
0 & 0 & 1 & 6\\
    2 & 3 & 4 & 5
  \end{array}\right]
\end{equation}

\N*{Step 4: Transform lower rows.} For each row \emph{below the first row} with a nonzero
leading component, we will subtract a multiple of the first row from
it. The idea is to transform each row \emph{below the first row} to have
a zero entry in the leading column.
For us, we see there is one row [beneath the first row] with a nonzero
entry in the leading column, highlighted in blue:
\begin{equation}
\left[\begin{array}{ccc;{2pt/2pt}c}
    1 & 2 & 3 & 4\\
0 & 0 & 1 & 6\\
\rowcolor{blue!20}    2 & 3 & 4 & 5
  \end{array}\right]
\end{equation}
We then add $-2$ times the first row to the third row, giving us
\begin{equation}
\left[\begin{array}{ccc;{2pt/2pt}c}
    1 & 2 & 3 & 4\\
0 & 0 & 1 & 6\\
\rowcolor{blue!20}    2 & 3 & 4 & 5
  \end{array}\right]\sim
\left[\begin{array}{ccc;{2pt/2pt}c}
    1 & 2 & 3 & 4\\
0 & 0 & 1 & 6\\
\rowcolor{blue!20}    0 & -1 & -2 & -3
  \end{array}\right]
\end{equation}
At this point, the first column with a nonzero entry will have its
general form look like $(1,0,0,\dots,0)$ (i.e., it leads with 1 and all
entries below it are zero).

\N*{Step 5: Repeat on the submatrix}
We take the submatrix for the remaining rows and remaining columns, then
return to step 1, and transform the submatrix to reduced row echelon
form. 
\begin{equation}
\left[\begin{array}{ccc;{2pt/2pt}c}
1 & 2 & 3 & 4\\ \cline{2-4}
0 & \multicolumn{1}{|c}{0} & 1 & 6\\
0 & \multicolumn{1}{|c}{-1} & -2 & -3
  \end{array}\right]
\end{equation}
We will find, for our particular example, the matrix has reduced row
echelon form
\begin{equation}\dots\sim
\left[\begin{array}{ccc;{2pt/2pt}c}
1 & 2 & 3 & 4\\
0 & 1 & 2 & 3\\
0 & 0 & 1 & 6
\end{array}\right].
\end{equation}
\end{subequations}

\begin{remark}
We should prove that Gauss-Jordan elimination produces a unique
augmented matrix, in some sense. It's obviously not true the results
will be identical, since we had some choice in swapping rows. But the
solutions produced from the resulting augmented matrices \emph{will} be
identical. After a few examples, we will prove this.
\end{remark}

\N{Gaussian Elimination}
We may optionally skip normalizing the ``pivot'' (step 3), but continue
to transform the lower rows so everything below the pivot is zero. The
slightly modified algorithm is referred to as \define{Gaussian Elimination},
\emph{not} to be confused with Gauss--Jordan elimination.

Gaussian elimination produces a matrix in \define{Row Echelon Form}.
The matrix is no longer ``reduced'', because the pivots are not
necessarily $1$.

Is there a reason to prefer one or the other? Well, with Gaussian
elimination, we could do it with a number system without division (like,
the integers). But with Gauss--Jordan elimination, when solving the
system of equations, step 3 avoids a division operation later on (which
can be convenient). One method is not ``better'' than the other, but
they are \emph{different} ``siblings''.

\begin{example}
  Suppose we have a $2\times2$ matrix $\mat{A}$ with generic entries
  \begin{equation}
\mat{A} = \begin{bmatrix}a & b\\c & d
\end{bmatrix}.
  \end{equation}
  Suppose $\mat{A}$ is invertible. What are the components of
  $\mat{A}^{-1}$?

  Right now, they are unknowns
\begin{subequations}
  \begin{equation}
\mat{A}^{-1} = \begin{bmatrix}x & y\\z & w
\end{bmatrix}.
  \end{equation}
  Matrix multiplication gives us
  \begin{equation}
\mat{A}\mat{A}^{-1} = \begin{bmatrix}a & b\\c & d
\end{bmatrix} \begin{bmatrix}x & y\\z & w
\end{bmatrix}
= \begin{bmatrix}
ax+bz & ay+bw\\
cx+dz & cy+dw
\end{bmatrix}.
  \end{equation}
  Which\dots does not really seem to improve the situation. Or does it?
  We expect this to be equal to the identity matrix, and really this is
  a system of 4 linear equations
\begin{equation}
\left.\begin{array}{cccccc}
ax &    & +bz &     & = & 1\\
   & ay &     & +bw & = & 0\\
cx &    & +dz &     & = & 0\\
   & cy &     & +dw & = & 1
\end{array}\right\}\equiv
\left(\begin{array}{cccc;{2pt/2pt}c}
  a &   & b &   & 1\\
    & a &   & b & 0\\
  c &   & d &   & 0\\
    & c &   & d & 1
\end{array}\right)
\end{equation}
\end{subequations}
We can now transform this augmented matrix to row echelon form (not
reduced, but still row echelon form).
\begin{subequations}
\begin{calculation}\gdef\defaultrelation{\sim}
\left(\begin{array}{cccc;{2pt/2pt}c}
  a & 0 & b & 0 & 1\\
  0 & a & 0 & b & 0\\
  c & 0 & d & 0 & 0\\
  0 & c & 0 & d & 1
\end{array}\right)
\step{add $(-c/a)(\mbox{row 1})$ to row 3} 
\left(\begin{array}{cccc;{2pt/2pt}c}
  a & 0 & b & 0 & 1\\
  0 & a & 0 & b & 0\\
  0 & 0 & \displaystyle d-\frac{bc}{a} & 0 & \displaystyle \frac{-c}{a}\\
  0 & c & 0 & d & 1
\end{array}\right)
\step{add $(-c/a)(\mbox{row 2})$ to row 4} 
\left(\begin{array}{cccc;{2pt/2pt}c}
  a & 0 & b & 0 & 1\\
  0 & a & 0 & b & 0\\
  0 & 0 & \displaystyle d-\frac{bc}{a} & 0 & \displaystyle \frac{-c}{a}\\
  0 & 0 & 0 & \displaystyle  d-\frac{bc}{a} & 1
\end{array}\right)
\step{since $d - (bc/a) = (ad - bc)/a$}
\left(\begin{array}{cccc;{2pt/2pt}c}
  a & 0 & b & 0 & 1\\
  0 & a & 0 & b & 0\\
  0 & 0 & \displaystyle\frac{ad - bc}{a} & 0 & \displaystyle \frac{-c}{a}\\
  0 & 0 & 0 & \displaystyle\frac{ad-bc}{a} & 1
\end{array}\right)
\end{calculation}
\end{subequations}
This has solution
\begin{subequations}
\begin{align}
w &= \frac{a}{ad-bc}\\
z &= \frac{-c}{ad-bc}\\
y &= \frac{-b}{ad-bc}\\
ax - \frac{bc}{ad - bc} &= 1\nonumber\\
\implies ax &= \frac{bc}{ad-bc}+1 = \frac{ad}{ad-bc}\nonumber\\
\implies x &= \frac{d}{ad-bc}.
\end{align}
\end{subequations}
Hence
\begin{equation}\label{eq:augmented-matrix:inverse-of-2-by-2-matrix}
\boxed{\mat{A}^{-1} = \frac{1}{ad-bc}\begin{pmatrix}d & -b\\
-c & a
\end{pmatrix}.}
\end{equation}
\end{example}

\N{Algorithm to determine inverse matrix}
From the previous example, we see that we could rephrase our process as
considering the $n\times 2n$ augmented matrix $(\mat{A}\mid\mat{I})$,
then ``applying Gauss--Jordan elimination and backsubstituting'' (i.e.,
applying elementary row operations until) we obtain
\begin{equation}
(\mat{A}\mid\mat{I})\sim(\mat{I}\mid\mat{A}^{-1}).
\end{equation}
If elementary row operations cannot transform $\mat{A}$ to the identity
matrix $\mat{A}\not\sim\mat{I}$, then $\mat{A}$ is singular.

\subsection{LU-Decomposition}

\M
Some matrices are easier to work with than others. Scalar matrices are
particular easy to invert, for example. We will consider what perhaps
could be called the ``silver bullet of linear algebra'': factorizing a
given matrix into a product of nicer matrices (i.e., writing $\mat{A}=\mat{M}_{1}\cdots\mat{M}_{n}$).

\begin{definition}
Let $\mat{A}$ be an $n\times n$ matrix.
Recall Definition~\ref{defn:matrix-zoo:triangular} where we defined
upper triangular and lower triangular matrices. 
We define the \define{LU-Decomposition} of $\mat{A}$ to consist of
\begin{enumerate}
\item a lower-triangular $n\times n$ matrix $\mat{L}$
\item an upper-triangular $n\times n$ matrix $\mat{U}$
\end{enumerate}
such that
\begin{enumerate}
\item $\mat{A}=\mat{L}\mat{U}$
\end{enumerate}
\end{definition}

\begin{remark}
Some texts add the condition that the diagonals of the lower-triangular
matrix are either 1 or 0. Other texts change that condition to be
imposed on the upper-triangular matrix instead. The reason for this is
because there is a unique LU decomposition in those situations. Without
either condition, there are many different possible LU decompositions.
The reader should be aware that not one of these definitions is
``better'' than another --- one is not ``the correct definition'' and
the others are ``wrong definitions'' --- but under certain
circumstances, it is preferable to \emph{adopt the convention} that we are
working with the lower-triangular matrix with 1 or 0 on the diagonal (or
whatever). 
\end{remark}

\M
Why on Earth is this an improvement? Well, recall Gauss--Jordan
elimination transformed $\mat{A}$ into an upper triangular matrix
$\mat{U}$, and that was how we could algorithmically solve systems of
equations. More explicitly, if $\mat{U}$ is upper triangular, we can
solve the system of equations
\begin{subequations}
\begin{equation}
\mat{U}\vec{x}=\vec{b}
\end{equation}
or, written explicitly with components,
\begin{equation}
\begin{pmatrix}
u_{1,1} & u_{1,2} & u_{1,3} & \dots & u_{1,n}\\
0      & u_{2,2} & u_{2,3} & \dots & u_{2,n}\\
0      &   0    & u_{3,3} & \dots & u_{3,n}\\
\vdots & \vdots & \vdots & \ddots & \vdots\\
0      &   0    &   0    &   0    & u_{n,n}
\end{pmatrix}\begin{pmatrix}x_{1}\\x_{2}\\x_{3}\\\vdots\\x_{n}
\end{pmatrix}
=\begin{pmatrix}b_{1}\\b_{2}\\b_{3}\\\vdots\\b_{n}
\end{pmatrix}
\end{equation}
\end{subequations}
The solution looks like, starting with the last equation in row $n$,
\begin{subequations}
  \begin{align}
    x_{n} &= \frac{b_{n}}{u_{n,n}}\\
    \intertext{then we plug this into the $n-1$ row to get,}
    x_{n-1} &= \frac{1}{u_{n-1,n-1}}(b_{n-1}-u_{n-1,n}x_{n})\\
    \vdots & \nonumber\\
    \intertext{and continuing until we get to row $j$, which has the
      generic solution:}
    x_{j} &= \frac{1}{u_{j,j}}\left(b_{j} - \sum^{n}_{k=j+1}u_{j,k}x_{k}\right).
  \end{align}
\end{subequations}
This procedure is known as \define{Backsubstitution}.

\M
We have a completely analogous way to handle lower-triangular matrices
$\mat{L}$ in systems of equations
\begin{equation}
\begin{pmatrix}
  \ell_{1,1} &  0        & 0         & \dots & 0\\
  \ell_{2,1} & \ell_{2,2} & 0         & \dots & 0\\
  \ell_{3,1} & \ell_{3,2} & \ell_{3,3} & \dots & 0\\
  \vdots    & \vdots    & \vdots     & \ddots & \vdots\\
  \ell_{n,1} & \ell_{n,2} & \ell_{n,3} & \dots & \ell_{n,n}
\end{pmatrix}
\begin{pmatrix}x_{1}\\x_{2}\\x_{3}\\\vdots\\x_{n}
\end{pmatrix}
=\begin{pmatrix}b_{1}\\b_{2}\\b_{3}\\\vdots\\b_{n}
\end{pmatrix}
\end{equation}
This time, we start with the first row, to find its solution,
\begin{subequations}
\begin{align}
x_{1} &= \frac{1}{\ell_{1,1}}(b_{1})\\
\intertext{then plugging this into the second row gives the solution,}
x_{2} &= \frac{1}{\ell_{2,2}}(b_{2} - \ell_{2,1}x_{1})\\
\intertext{and continuing to the generic row $j$}
x_{j} &= \frac{1}{\ell_{j,j}}\left(b_{j} - \sum^{j-1}_{k=1}\ell_{j,k}x_{k}\right).
\end{align}
\end{subequations}
This is called \define{Forward Substitution}.

\N{Algorithm: LU Factorization}
The generic algorithm for LU decomposition is best illustrated by
example. Given an $n\times n$ matrix $\mat{A}$.
We will generate a sequence of matrices $\mat{L}_{1}$,
$\mat{U}_{1}$, $\mat{L}_{2}$, $\mat{U}_{2}$, \dots, $\mat{L}_{n}$,
$\mat{U}_{n}$ which progressively get ``more triangular'' until we reach
the end result. Then we will call $\mat{U}:=\mat{U}_{n}$ and
$\mat{L}:=\mat{L}_{n}$. Consider the matrix
\begin{equation}
  \mat{A} =
  \begin{bmatrix}
    -2 &  1 & 3  &  4 \\
     1 &  2 & -5 & -2 \\
    -1 &  1 & 25 &  3 \\
     4 & -4 & 20 &  2 
  \end{bmatrix} = \begin{bmatrix}\transpose{\vec{r}_{1}}\\
    \transpose{\vec{r}_{2}}\\
    \transpose{\vec{r}_{3}}\\
    \transpose{\vec{r}_{4}}
  \end{bmatrix}
\end{equation}
where $\transpose{\vec{r}_{1}}$ is the first row,
$\transpose{\vec{r}_{2}}$ is the second row, and so on.

\N*{Step 1}
We form $\mat{U}_{1}$ by a process similar to Gauss-Jordan elimination,
we will ``zero out'' the first column beneath the first row.
We take subtract $a_{2,1}/a_{1,1}=(-1/2)$ times the first row to the second row, 
subtract $a_{3,1}/a_{1,1}=(1/2)$ times the first row to the third row,
and subtract $a_{4,1}/a_{1,1}=(-4/2)=-2$ times the first row to the fourth
row, to obtain $\mat{U}_{1}$:
\begin{equation}
\mat{U}_{1} = \begin{pmatrix}\transpose{\vec{r}_{1}}\\
  \transpose{\vec{r}_{2}} - (-1/2)\transpose{\vec{r}_{1}}\\
  \transpose{\vec{r}_{3}} - (1/2)\transpose{\vec{r}_{1}}\\
  \transpose{\vec{r}_{4}} - (-2)\transpose{\vec{r}_{1}}
\end{pmatrix} = \begin{pmatrix}
    -2 &  1  & 3    &  4 \\
     0 & 5/2 & -7/2 &  0\\
     0 & 1/2 & 47/2 &  1\\
     0 &  -2 &   26 & 10
\end{pmatrix} = \begin{pmatrix}\transpose{(\vec{u}^{(1)}_{1})}\\
  \transpose{(\vec{u}^{(1)}_{2})}\\
  \transpose{(\vec{u}^{(1)}_{3})}\\
  \transpose{(\vec{u}^{(1)}_{4})}
\end{pmatrix}
\end{equation}
We denote the rows of $\mat{U}_{1}$ using $\transpose{(\vec{u}^{(1)}_{j})}$.

We construct $\mat{L}_{1}$ by taking $1$ along the diagonal, and in the
first column setting $\ell_{j,1}$ equal to the these multipliers, i.e.,
$\ell_{j,1}=a_{j,1}/a_{1,1}$:
\begin{equation}
\mat{L}_{1} = \begin{pmatrix}
             1 & 0 & 0 & 0 \\
a_{2,1}/a_{1,1} & 1 & 0 & 0\\
a_{3,1}/a_{1,1} & ? & 1 & 0\\
a_{4,1}/a_{1,1} & ? & ? & 1
\end{pmatrix}= \begin{pmatrix}
             1 & 0 & 0 & 0 \\
-1/2 & 1 & 0 & 0\\
1/2 & ? & 1 & 0\\
-2 & ? & ? & 1
\end{pmatrix}
\end{equation}
Why does this work? Well, when we examine the resulting first column
from the product $\mat{L}_{1}\mat{U}_{1}$ we see we recover the first
column of our original matrix $\mat{A}$. 

\N*{Step 2} Now we assemble $\mat{U}_{2}$ from $\mat{U}_{1}$.
We take $\mat{U}_{1}$ and ``zero out'' the entries in the second column
beneath the second row. We do this by subtracting
$(u^{(1)}_{3,2}/u^{(1)}_{2,2})=1/5$ times the second row from the third row,
and subtracting
$(u^{(1)}_{4,2}/u^{(1)}_{2,2})=-4/5$ time the second row from the fourth
row:
\begin{equation}
\mat{U}_{2} = \begin{pmatrix}\transpose{(\vec{u}^{(1)}_{1})}\\
  \transpose{(\vec{u}^{(1)}_{2})}\\
  \transpose{(\vec{u}^{(1)}_{3})} - (1/5)\transpose{(\vec{u}^{(1)}_{2})}\\
  \transpose{(\vec{u}^{(1)}_{4})} - (-4/5)\transpose{(\vec{u}^{(1)}_{2})}
\end{pmatrix}
=\begin{pmatrix}
    -2 &  1  &     3 &  4 \\
     0 & 5/2 &  -7/2 &  0\\
     0 &  0  & 121/5 & 1\\
     0 &  0  & 116/5 & 10
\end{pmatrix} = \begin{pmatrix}\transpose{(\vec{u}^{(2)}_{1})}\\
  \transpose{(\vec{u}^{(2)}_{2})}\\
  \transpose{(\vec{u}^{(2)}_{3})}\\
  \transpose{(\vec{u}^{(2)}_{4})}
\end{pmatrix}.
\end{equation}
We denote the rows of $\mat{U}_{2}$ using $\transpose{(\vec{u}^{(2)}_{j})}$.

Now we enter the multiples in the second column beneath the diagonal of
$\mat{L}_{1}$ to obtain $\mat{L}_{2}$:
\begin{equation}
\mat{L}_{2} = \begin{pmatrix}
1    & 0 & 0 & 0 \\
-1/2 & 1 & 0 & 0\\
1/2  & (u^{(1)}_{3,2}/u^{(1)}_{2,2}) & 1 & 0\\
-2   & (u^{(1)}_{4,2}/u^{(1)}_{2,2}) & ? & 1
\end{pmatrix}
=\begin{pmatrix}
1    &    0 & 0 & 0 \\
-1/2 &    1 & 0 & 0\\
1/2  &  1/5 & 1 & 0\\
-2   & -4/5 & ? & 1
\end{pmatrix}.
\end{equation}
The reader can verify that the product of $\mat{L}_{2}$ and the second
column of $\mat{U}_{2}$ gives the second column of $\mat{A}$.

\N*{Step 3}
The last step for us, we need to subtract $116/121$ times the third row
of $\mat{U}_{2}$ from the last row of $\mat{U}_{2}$ to construct
$\mat{U}_{3}$:
\begin{equation}
\mat{U}_{3} = \begin{pmatrix}\transpose{(\vec{u}^{(2)}_{1})}\\
  \transpose{(\vec{u}^{(2)}_{2})}\\
  \transpose{(\vec{u}^{(2)}_{3})}\\
  \transpose{(\vec{u}^{(2)}_{4})}-(116/121)\transpose{(\vec{u}^{(2)}_{3})}
\end{pmatrix} =\begin{pmatrix}
    -2 &  1  &     3 &  4\\
     0 & 5/2 &  -7/2 &  0\\
     0 &  0  & 121/5 &  1\\
     0 &  0  &     0 & 1094/121.
\end{pmatrix}
\end{equation}
We see this is upper-triangular.

Now, the last entry in the lower-triangular matrix, we set it equal to
\begin{equation}
\mat{L}_{3} = \begin{pmatrix}
1    &    0 & 0 & 0 \\
-1/2 &    1 & 0 & 0\\
1/2  &  1/5 & 1 & 0\\
-2   & -4/5 & (u^{(2)}_{4,3}/u^{(2)}_{3,3}) & 1
\end{pmatrix} = \begin{pmatrix}
1    &    0 &       0 & 0 \\
-1/2 &    1 &       0 & 0\\
1/2  &  1/5 &       1 & 0\\
-2   & -4/5 & 116/121 & 1
\end{pmatrix}.
\end{equation}
The reader can verify that $\mat{L}_{3}$ times the third column of
$\mat{U}_{3}$ produces the third column of $\mat{A}$, \emph{and} that
$\mat{L}_{3}$ times the last column of $\mat{U}_{3}$ produces the last
column of $\mat{A}$.

\N*{Step 4} We then call $\mat{L} := \mat{L}_{3}$ and
$\mat{U} := \mat{U}_{3}$, then return this as the LU-decomposition of
$\mat{A}$.

\begin{remark}
The basic pattern is revealed, we are constructing $\mat{L}_{j+1}$ and
$\mat{U}_{j+1}$ by modifying the $j+1$ columns of $\mat{L}_{j}$ and
$\mat{U}_{j}$ such that the matrix multiplication of $\mat{L}_{j+1}$
times the $(j+1)^{\text{th}}$ column of $\mat{U}_{j+1}$ equals the
$(j+1)^{\text{th}}$ column of $\mat{A}$. Programmers call this
``property which holds after every step of the loop'' an
``invariant property'' of the algorithm.
\end{remark}

\begin{remark}
This LU decomposition algorithm is analogous to ``long division'' of
numbers we learn in elementary school: we construct the quotient one
digit at a time, multiplying the latest digit by the divisor, then
subtracting from the residue of the dividend this product.
\end{remark}

\N{LU Decomposition works on any matrix}
There is no reason why we need $\mat{A}$ to be a square matrix. We could
have let it be an $m\times n$ matrix.
Then $\mat{L}$ would be an $m\times m$ matrix and $\mat{U}$ would be an
$m\times n$ matrix. Or $\mat{L}$ could be an $m\times n$ matrix, and
$\mat{U}$ would be an $n\times n$ matrix. The algorithm works the same,
regardless.

\N{Block LU Decomposition}
Suppose we have a $2\times 2$ matrix
\begin{equation}
  \mat{M} = \begin{pmatrix}
    a & b\\
    c & d
\end{pmatrix}.
\end{equation}
The reader can verify
\begin{equation}
 \begin{pmatrix}
    a & b\\
    c & d
 \end{pmatrix} =
 \begin{pmatrix}
   1 & 0\\
   c/a & 1
 \end{pmatrix}
 \begin{pmatrix}a & b\\
   0 & d - (c/a)b
 \end{pmatrix}.
\end{equation}
We can go farther, and observe:
\begin{equation}
 \begin{pmatrix}
    a & b\\
    c & d
 \end{pmatrix} = 
 \begin{pmatrix}
   1 & 0\\
   c/a & 1
 \end{pmatrix}
 \begin{pmatrix}
   a & 0\\
   0 & d - (cb/a)
 \end{pmatrix}
 \begin{pmatrix}
   1 & b/a\\
   0 & 1
 \end{pmatrix}.
\end{equation}
What if we have a block matrix
\begin{equation}
  \mat{M} = \begin{pmatrix}
    \mat{A} & \mat{B}\\
    \mat{C} & \mat{D}
  \end{pmatrix},
\end{equation}
where $\mat{A}$, $\mat{B}$, $\mat{C}$, and $\mat{D}$ are all square
matrices of the same dimension --- say, they are all $n\times n$ matrices.
could we factorize it similarly? Let us try!

We, first of all, see that we will need to find the inverse of
$\mat{A}$ (since this is the analog to dividing by $a$). So before we
can do anything, we \emph{must} have $\mat{A}$ be an invertible
matrix. If it is singular, then there is no hope of an analogous
decomposition. 

There is some ambiguity in trying to figure out the analog to $c/a$ ---
should it be $\mat{A}^{-1}\mat{C}$ or $\mat{C}\mat{A}^{-1}$? For us to
answer this question, we should ask ourselves, ``What are the dimensions
of $\mat{A}^{-1}$ and $\mat{C}$?'' We assumed they are both $n\times n$,
so either would work ostensibly. The $L$ matrix would be multiplied by a
diagonal matrix
\begin{equation}
L\begin{pmatrix}
\mat{A} & 0\\
      0 & \mat{D} - \mbox{``$cb/a$''}
\end{pmatrix}
\end{equation}
For this to make sense, we would have the ``$c/a$'' block multiplied on
the right by $\mat{A}$. This means $\mat{C}\mat{A}^{-1}$ is the
analogous quantity, and we find
\begin{equation}
L = \begin{pmatrix}\mat{I} & 0\\\mat{C}\mat{A}^{-1} & \mat{I}
\end{pmatrix}.
\end{equation}
Similar reasoning suggests the upper triangular matrix should be
\begin{equation}
U = \begin{pmatrix}\mat{I} & \mat{A}^{-1}\mat{B}\\0 & \mat{I}
\end{pmatrix}.
\end{equation}
Now, the diagonal matrix is the thorn in our side. If we multiply it out
with the $U$ matrix, we find
\begin{equation}
\begin{pmatrix}
\mat{A} & 0\\
      0 & \mat{D} - \mbox{``$cb/a$''}
\end{pmatrix}\begin{pmatrix}\mat{I} & \mat{A}^{-1}\mat{B}\\0 & \mat{I}
\end{pmatrix}
= \begin{pmatrix}
\mat{A} & \mat{B}\\
0 & (\mat{D} - \mbox{``$cb/a$''})\mat{I}
\end{pmatrix}
\end{equation}
Multiplying this on the left by the $L$ matrix,
\begin{equation}
\begin{pmatrix}\mat{I} & 0\\\mat{C}\mat{A}^{-1} & \mat{I}
\end{pmatrix}\begin{pmatrix}
\mat{A} & \mat{B}\\
0 & \mat{D} - \mbox{``$cb/a$''}
\end{pmatrix} =\begin{pmatrix}
\mat{A} & \mat{B}\\
\mat{C} & \mat{C}\mat{A}^{-1}\mat{B} + (\mat{D} - \mbox{``$cb/a$''})
\end{pmatrix}
\end{equation}
For this to equal our original matrix, we need
\begin{equation}
\mat{C}\mat{A}^{-1}\mat{B}- \mbox{``$cb/a$''}=0
\end{equation}
that is to say,
\begin{equation}
\mbox{``$cb/a$''} =\mat{C}\mat{A}^{-1}\mat{B}.
\end{equation}
This gives us the block LU decomposition of $\mat{M}$:
\begin{equation}
\boxed{
\begin{pmatrix}
  \mat{A} & \mat{B}\\
  \mat{C} & \mat{D}
\end{pmatrix}
 = 
\begin{pmatrix}
             \mat{I} & 0\\
  \mat{C}\mat{A}^{-1} & \mat{I}
\end{pmatrix}
\begin{pmatrix}
  \mat{A} & 0\\
        0 & \mat{D} - \mat{C}\mat{A}^{-1}\mat{B}
\end{pmatrix}
\begin{pmatrix}
  \mat{I} & \mat{A}^{-1}\mat{B}\\
        0 & \mat{I}
\end{pmatrix}.}
\end{equation}




\phantomsection
\subsection*{Exercises}
\addcontentsline{toc}{subsection}{Exercises}

\begin{exercise}
What happens if $\mat{A}$ is $m\times q$, $\mat{B}$ is $m\times p$,
$\mat{C}$ is $n\times q$, and $\mat{C}$ is $n\times p$? For what values
of $m$, $n$, $p$, and $q$ would the block LU decomposition
work? [Hint: $\mat{A}$ must be invertible, what constraints does that
  place on its dimension?]\footnote{We could make this needlessly
complicated, working with the three cases $m<q$, $m=q$, and $m>q$.
For the sake of discussion, consider the two separate subcases of $m\neq q$
[for a total of 4 cases --- 2 subcases of $m<q$ and 2 subcases of $m>q$]
when there exists:
\begin{enumerate}
\item a matrix $\mat{A}_{L}$ is a $q\times m$
matrix such that it acts like an inverse from the left
$\mat{A}_{L}\mat{A}=\mat{I}_{q}$ (but not necessarily from the right),
and separately
\item a matrix $\mat{A}_{R}$ is an $q\times m$ matrix which acts like an inverse
from the right $\mat{A}\mat{A}_{R}=\mat{I}_{m}$ (but not necessarily
from the left).
\end{enumerate}
What happens if you try computing $\mat{A}_{R}\mat{A}$ (and
$\mat{A}\mat{A}_{L}$) is --- at best! --- you end up with a block matrix
of the form $(\mat{I}|0)$ or its transpose.
You can work this out, if you want, but we will develop more tools later
that will help you answer this pathological variant problem.}
\end{exercise}

\begin{exercise}
Let $\mat{A}$ be an $m\times n$ matrix and $\mat{U}$ be an
upper-triangular matrix. Will $\mat{A}\mat{U}$ be [upper or lower] triangular?
\end{exercise}

\begin{exercise}
  Let
  \[\mat{A} = \begin{pmatrix}
    1 & 1 & 1\\
    0 & 1 & 1\\
    0 & 0 & 1
  \end{pmatrix} \]
  Compute $\mat{A}^{n}$ by induction on $n\in\NN$.
\end{exercise}

\N{STOP!!!} And take a break, go out for a walk, drink a glass of water.
We've covered a lot in this section, introducing a general algorithm to
solve systems of linear equations, and we introduced for the first
time(!) a notion of ``factorization'' of matrices. That's a lot of new
stuff.
\section{Determinant}

\M
Remember when we figured out the inverse of a $2\times 2$ matrix, in
Eq~\eqref{eq:augmented-matrix:inverse-of-2-by-2-matrix} there was a
funny common factor of $ad-bc$. That was odd.

But if we try the same thing with a $3\times 3$ matrix, we end up with
\begin{equation}
\mat{A}^{-1} = \begin{pmatrix}a & b & c\\
  d & e & f\\
  g & h & i
\end{pmatrix}^{-1} = \frac{1}{a(ei-fh)-b(di-fg)+c(dh-eg)}\begin{pmatrix}A & B & C\\
  D & E & F\\
  G & H & I
\end{pmatrix}
\end{equation}
where $A$, $B$, $C$, etc., are all some mess. We're focused on the
scalar factor out in front: if $b=c$ or $d=g=0$, then we recover a similar
formula as in the $2\times 2$ case. Similarly, if $a=b=0$, we recover a
similar factor.

There is a recurring pattern here where, for a general $n\times n$
matrix $\mat{A}$, there is a common factor when computing its inverse
\begin{equation}
\mat{A}^{-1} = \frac{1}{\mbox{(suspicious factor)}}\mat{B}.
\end{equation}
If we set $(n-1)$ entries in the first row to zero, we recover the
suspicious factor from the $(n-1)\times(n-1)$ matrix inverse.

This suspicious factor turns out to be extremely important.

\N{Geometric Intuition}
Think about $2$-dimensional space $\RR^{2}$. We could begin by examining
the unit square:
\begin{equation}
C^{2} = \{(x_{1},x_{2})\in\RR^{2}\mid 0\leq x_{j}\leq 1, j=1,2\}.
\end{equation}
We read the right-hand side as the set (demarcated by squiggle brackets
$\{\dots\}$) consisting of points in the plane $(x_{1},x_{2})\in\RR^{2}$
(``$\in$'' takes an object to its left and a set to its right) such that
(``$\mid$'' encodes ``such that'') $0\leq x_{j}\leq 1, j=1,2$ the
components $x_{1}$ and $x_{2}$ lie between $0$ and $1$ inclusive. This
notation is known as ``set-builder notation''. If the reader is
unfamiliar with it, then they should consult Appendix~\cite{section:appendix-sets}.

What is its area? This is very silly, everyone learns in High School it
is the product of the lengths of its sides,
\begin{equation}
\operatorname{Area}(C^{2})=1\times1=1.
\end{equation}

But now suppose we ``deform'' the square to form a parallelogram with
one vertex fixed at the origin. Then we have the two sides described by
endpoints located at $\vec{a}=(a_{1},a_{2})$ and
$\vec{b}=(b_{1},b_{2})$.

What is its area? We can recall from vector
calculus that it is the magnitude of the their cross product
$|\vec{a}\times\vec{b}|=|a_{1}b_{2}-a_{2}b_{1}|$.

Does it look familiar? It should: it's precisely the suspicious factor
in Eq~\eqref{eq:augmented-matrix:inverse-of-2-by-2-matrix}.

\M What about the case in $\RR^{3}$? What if we start with the unit cube
\begin{equation}
C^{3} = \{(x_{1},x_{2},x_{3})\in\RR^{3}\mid 0\leq x_{j}\leq 1, j=1,2,3\}.
\end{equation}
The analogous quantity of interest is now its \emph{volume}. What is the
volume of the cube with side length equal to $1$? We recall
\begin{equation}
\operatorname{Vol}(C^{3})=1\times1\times1=1.
\end{equation}
Very simple.

What if we deform the $(x_{1},x_{2})$ cross sections to be a
parallelogram with sides at endpoints $(a_{1},a_{2},x_{3})$ and
$(b_{1},b_{2},x_{3})$, like we did in the $\RR^{2}$ case? Intuitively, this is
``dragging'' a parallelogram in the $(x_{1},x_{2})$-plane ``up'' the
$x_{3}$-axis for a ways of 1 unit of length. What is the volume of this?
It turns out to be the area of the parallelogram multiplied by the
distance we drag it. Its volume would be
\begin{equation}
\operatorname{Vol}(P) = (a_{1}b_{2}-a_{2}b_{1})\times 1.
\end{equation}
This is not terribly surprising, so let us deform further.

Specifically, we deform the unit cube to be a parallelepiped $P$ with a
corner fixed at the origin. The endpoints for our parallelepepiped $P$ along
its length, width, and height would be $\vec{a}=(a_{1},a_{2},a_{3})$,
$\vec{b}=(b_{1},b_{2},b_{3})$ and $\vec{c}=(c_{1},c_{2},c_{3})$. What is
its volume? The trick is to cheat and rotate axes until $\vec{a}$ and
$\vec{b}$ live in the $(x_{1},x_{2})$ plane.

But if we don't know that, then we can form the unit normal to the face
formed by edges $\vec{a}$ and $\vec{b}$ by taking their cross-product,
\begin{equation}
\widehat{\vec{n}} = \frac{\vec{a}\times\vec{b}}{|\vec{a}\times\vec{b}|}.
\end{equation}
We find the height to be the projection of $\vec{c}$ onto this normal
vector
\begin{equation}
h = \vec{c}\cdot\widehat{\vec{n}}.
\end{equation}
We can find the base area $B$ of the parallelogram formed by $\vec{a}$
and $\vec{b}$ by
\begin{equation}
B = |\vec{a}\times\vec{b}|,
\end{equation}
and, as always, base times height yields volume:
\begin{equation}
\operatorname{Vol}(P) = Bh = |\vec{a}\times\vec{b}|\;\vec{c}\cdot\widehat{\vec{n}}.
\end{equation}
Great, but this doesn't seem to help much. We just unfold the definition
of the unit normal, and we find
\begin{equation}
\operatorname{Vol}(P) = |\vec{a}\times\vec{b}|\frac{\vec{c}\cdot(\vec{a}\times\vec{b})}{|\vec{a}\times\vec{b}|}
= \vec{c}\cdot(\vec{a}\times\vec{b}).
\end{equation}
What is this explicitly? Let's write it out:
\begin{equation}
\operatorname{Vol}(P) = c_{1}(a_{2}b_{3}-a_{3}b_{2}) - c_{2}(a_{1}b_{3}-a_{3}b_{1})
+ c_{3}(a_{1}b_{2}-a_{2}b_{1}).
\end{equation}
Does it look familiar? It should: it's the suspicious factor in the case
of finding the inverse for a $3\times 3$ matrix.

\M
In short, this suspicious factor --- the exact same quantity --- appears
in the volume of $n$-dimensional parallelograms, and when computing the
inverse for an $n\times n$ matrix. This is strange, because one
situation is geometry whereas the other situation is
algebra. ``Strange'' is not the correct word for it: ``Profound'' should
be employed.

\N{Problem: How to compute this ``suspicious factor''?}
Now that we've established the profundity of this ``suspicious factor'',
how exactly do we compute it? What properties does it have? Why does it
matter in linear algebra? And what should we call it?

\begin{definition}
Let $\mat{A}=(a_{i,j})$ be an $n\times n$ matrix.
We recursively define its \define{Determinant} to be a scalar, defined by:
\begin{enumerate}
\item if $n=1$, we just take $\det(\mat{A})=a_{1,1}$;
\item if $n=2$, we simply take
  \begin{equation}
    \det(\mat{A}) = \det\begin{pmatrix}a_{1,1} & a_{1,2}\\
    a_{2,1} & a_{2,2}
    \end{pmatrix} = a_{1,1}a_{2,2} - a_{2,1}a_{1,2};
  \end{equation}
\item for $n>2$, we recursively define it using the formula
  \begin{equation}
\det(\mat{A}) = \sum_{j=1}^{n}(-1)^{j+1}a_{1,j}\det(\mat{M}_{1,j})
  \end{equation}
  where $\mat{M}_{1,j}$ is called a \emph{minor} of $\mat{A}$, obtained
  from $\mat{A}$ by deleting its first row and its $j^{\text{th}}$
  column --- more generally $\mat{M}_{i,j}$ is obtained by deleting row
  $i$ and column $j$ from $\mat{A}$.
\end{enumerate}
\end{definition}

\begin{lemma}
Let $\mat{A}$ be an $n\times n$ matrix with a column or row consisting of
zeros. Then $\det(\mat{A})=0$.
\end{lemma}

\begin{theorem}
Let $\mat{T}=(t_{i,j})$ be a triangular $n\times n$ matrix (either
upper-triangular, or lower-triangular, it doesn't matter).
Then its determinant is just the product of diagonal entries
$\det(\mat{T}) = \prod_{j=1}^{n}t_{j,j}$.
\end{theorem}

\begin{proof}
  We do this by induction on $n$.

  \textbf{Base Case:} $n=2$, for upper-triangular matrices we find
  \begin{equation}
\det\begin{pmatrix}t_{1,1} & t_{1,2}\\0 & t_{2,2}
\end{pmatrix} = t_{1,1}t_{2,2} - t_{1,2}0 = t_{1,1}t_{2,2}.
  \end{equation}
  For lower-triangular matrices
  \begin{equation}
\det\begin{pmatrix}t_{1,1} & 0\\t_{2,1} & t_{2,2}
\end{pmatrix} = t_{1,1}t_{2,2} - 0t_{2,1} = t_{1,1}t_{2,2}.
  \end{equation}
  Hence we establish the base case.

  \textbf{Inductive Hypothesis:} We assume this is true for general $n$.

  \textbf{Inductive Case:} For the $n\to n+1$ case, we will examine two
  subcases. Subcase 1 is when $\mat{T}$ is upper-triangular, where we
  write the block components out:
  \begin{equation}
    \mat{T} = \left(\begin{array}{cc}
      t_{1,1} & \transpose{\vec{t}} \\
      \vec{0} & \mat{T}^{(n)}
    \end{array}\right)
  \end{equation}
  The inductive hypothesis assumed $\det(\mat{T}^{(n)}$ is the product
  of diagonal components. We see that the minors $\mat{M}_{1,i}$ has a
  zero column for $i>1$. Hence only the $i=1$ minor contributes to the
  determinant, giving us
  \begin{subequations}
\begin{calculation}
    \det(\mat{T})
\step{definition of determinant}
    \sum^{n+1}_{i=1}(-1)^{i+1}t_{1,i}\det(\mat{M}_{1,i})
\step{pulling out the first term from the sum}
    t_{1,1}\det(\mat{M}_{1,1}) + \sum^{n+1}_{i=2}(-1)^{i+1}t_{1,i}\det(\mat{M}_{1,i})
\step{but $\det(\mat{M}_{1,i})=0$ for $i>1$}
    t_{1,1}\det(\mat{M}_{1,1}) + \sum^{n+1}_{i=2}(-1)^{i+1}t_{1,i}\cdot0
\step{the sum of zeros is zero}
    t_{1,1}\det(\mat{M}_{1,1}) + 0
\step{arithmetic}
    t_{1,1}\det(\mat{M}_{1,1})
\step{the minor $\mat{M}_{1,1} = \mat{T}^{(n)}$}
    t_{1,1}\det(\mat{T}^{(n)})
\step{inductive hypothesis}
    t_{1,1}\prod^{n+1}_{j=2}t_{j,j}
\step{associativity of multiplication}
    \prod^{n+1}_{j=1}t_{j,j}.
\end{calculation}
\end{subequations}
Hence we conclude for upper-triangular matrices, its determinant is
just the product of diagonal components.

\textsc{Subcase 2: Lower-triangular matrix.}
The reasoning for lower-triangular matrices is similar. Its block
components would look like
\begin{equation}
    \mat{T} = \left(\begin{array}{cc}
      t_{1,1} & \transpose{\vec{0}} \\
      \vec{t} & \mat{T}^{(n)}
    \end{array}\right)
  \end{equation}
The steps would be exactly the same, but this time because $t_{1,j}=0$
for $j>1$.
\end{proof}

\N{Levi-Civita Symbol}
It is an unfortunate fact that the definition we have given for the
determinant, while useful for \emph{performing calculations}, cannot
easily be used to prove properties concerning the
determinant. Consequently, we introduce an equivalent definition using a
terrifying quantity known as the Levi--Civita symbol. Its terror stems
from having multiple indices, but do not worry: it is used for book-keeping.

Let us define the Levi--Civita symbol for three-dimensions as
$\epsilon_{i,j,k}$ such that
\begin{enumerate}
\item $\epsilon_{1,2,3}=+1$, and
\item swapping two adjacent indices costs us a sign: $\epsilon_{j,i,k}=-\epsilon_{i,j,k}=\epsilon_{i,k,j}$;
$\epsilon_{k,i,j}=\epsilon_{i,j,k}=\epsilon{j,k,i}$.
\end{enumerate}
As a consequence of the second property, any repeated index will
correspond to a zero entry: $\epsilon_{i,i,k}=0$, $\epsilon_{i,j,j}=0$,
and so on. The Levi--Civita symbol in $n$ indices has analogous
properties:
\begin{enumerate}
\item $\epsilon_{1,2,3,\dots,n-1,n}=+1$, and
\item swapping two adjacent indices costs us a sign:
  $\epsilon_{i_{1},\dots,i_{j},i_{j+1},\dots,i_{n}}=-\epsilon_{i_{1},\dots,i_{j+1},i_{j},\dots,i_{n}}$.
\end{enumerate}
Yes, we have subscripts \emph{on our subscripts} --- we've indexed the
indexing variables!

The determinant in three-dimensions for matrix $\mat{A}=(a_{i,j})$ is then
\begin{equation}\label{eq:determinant:levi-civita:three-dim}
\boxed{\det(\mat{A}) = \sum^{3}_{i=1}\sum^{3}_{j=1}\sum^{3}_{k=1}
\epsilon_{i,j,k}a_{1,i}a_{2,j}a_{3,k}.}
\end{equation}
We can see this by performing the sums when $i=1$, we get
\begin{equation}
\sum^{3}_{j=1}\sum^{3}_{k=1}\epsilon_{1,j,k}a_{1,1}a_{2,j}a_{3,k}
=a_{1,1}\left(\sum^{3}_{k=1}\epsilon_{1,2,k}a_{2,2}a_{3,k}+\epsilon_{1,3,k}a_{2,3}a_{3,k}\right)
\end{equation}
But the antisymmetry property forces us to have $k=3$ in the first term
and $k=2$ in the second term as the only nonzero contribution in the
right-hand side:
\begin{equation}
\sum^{3}_{j=1}\sum^{3}_{k=1}\epsilon_{1,j,k}a_{1,1}a_{2,j}a_{3,k}
=a_{1,1}\left(\epsilon_{1,2,3}a_{2,2}a_{3,3}+\epsilon_{1,3,2}a_{2,3}a_{3,2}\right)
\end{equation}
Invoking antisymmetry to rearrange indices:
\begin{equation}
\sum^{3}_{j=1}\sum^{3}_{k=1}\epsilon_{1,j,k}a_{1,1}a_{2,j}a_{3,k}
=a_{1,1}\left(\epsilon_{1,2,3}a_{2,2}a_{3,3}-\epsilon_{1,2,3}a_{2,3}a_{3,2}\right)
\end{equation}
then invoking the first property simplifies the right-hand side:
\begin{equation}
\sum^{3}_{j=1}\sum^{3}_{k=1}\epsilon_{1,j,k}a_{1,1}a_{2,j}a_{3,k}
=a_{1,1}\left(a_{2,2}a_{3,3}-a_{2,3}a_{3,2}\right).
\end{equation}
This is precisely what we had as the coefficient to $a_{1,1}$ in the
familiar definition of the determinant. If we continue along, we will
find our new definition coincides with our old definition.

If we had a $4\times4$ matrix $\mat{A}=(a_{i,j})$, then we would have
\begin{equation}
\det(\mat{A}) = \sum^{4}_{i=1}a_{1,i}\det(M_{1,i})
\end{equation}
using our familiar definition. But we can rewrite $\det(M_{1,i})$ using
the Levi--Civita symbol as (remembering the minors are $3\times3$ matrices):
\begin{equation}
\det(\mat{A}) =
\sum^{4}_{i=1}(-1)^{i+1}a_{1,i}\left(\sum^{3}_{j=1}\sum^{3}_{k=1}\sum^{3}_{\ell=1}\epsilon_{j,k,\ell}(M_{1,j})_{2,j}(M_{1,k})_{3,k}(M_{1,\ell})_{4,\ell}\right),
\end{equation}
or by suppressing the column $i$ explicitly and reindexing accordingly:
\begin{equation}
\det(\mat{A}) =
\sum^{4}_{i=1}(-1)^{i+1}a_{1,i}\left(\sum^{4}_{\stackrel{j=1}{j\neq i}}\sum^{4}_{\stackrel{k=1}{k\neq i}}\sum^{4}_{\stackrel{\ell=1}{\ell\neq i}}\epsilon_{j,k,\ell}a_{2,j}a_{3,k}a_{4,\ell}\right),
\end{equation}
We observe the factor of $(-1)^{i+1}\epsilon_{j,k,\ell}$ could be
replaced by $\epsilon_{i,j,k,\ell}$ (which will also enforce the
conditions $i\neq j$ and $k\neq i$ and so on). Thus we find:
\begin{equation}
\det(\mat{A}) =
\sum^{4}_{i=1}\epsilon_{i,j,k,\ell}a_{1,i}\sum^{4}_{j=1}\sum^{4}_{k=1}\sum^{4}_{\ell=1}a_{2,j}a_{3,k}a_{4,\ell}.
\end{equation}
We invoke distributivity to tidy up the right-hand side as:
\begin{equation}
\boxed{\det(\mat{A}) =
\sum^{4}_{i=1}\sum^{4}_{j=1}\sum^{4}_{k=1}\sum^{4}_{\ell=1}\epsilon_{i,j,k,\ell}a_{1,i}a_{2,j}a_{3,k}a_{4,\ell}.}
\end{equation}
There seems to be a pattern emerging, when we examing the
three-dimensional case in Eq~\eqref{eq:determinant:levi-civita:three-dim}
and compare it to the four-dimensional case: we have a Levi--Civita
symbol with $n$ indices, and $n$ factors of $a_{1,i_{1}}a_{2,i_{2}}(\cdots)a_{n,i_{n}}$.

Thus we could argue, for any $n\times n$ matrix $\mat{A}=(a_{i,j})$, we have:
\begin{equation}\label{eq:determinant:using-levi-civita}
\boxed{\det(\mat{A}) = \sum^{n}_{i_{1}=1}\sum^{n}_{i_{2}=1}\cdots\sum^{n}_{i_{n}=1}\epsilon_{i_{1},i_{2},\dots,i_{n}}a_{1,i_{1}}a_{2,i_{2}}(\cdots)a_{n,i_{n}}.}
\end{equation}
We've shown this is true for $n=3$ and $n=4$, and we've shown how $n=4$
boils down to $n=3$. The general argument is similar, we would argue by
induction --- our base case has been established, we just need to prove
the inductive case $n+1$ in terms of the ``arbitrary $n$'' inductive
hypothesis. But this is precisely what we've done when moving from $n=3$
to $n=4$. The only difference will be slight, writing
\begin{align}
    \det\mat{A} &= \sum^{n+1}_{i_{1}=1}(-1)^{i_{1}+1}a_{1,i_{1}}\det(\mat{M}_{1,i_{1}})\\
    &= \sum^{n+1}_{i_{1}=1}(-1)^{i_{1}+1}a_{1,i_{1}}\left(\sum^{n}_{\mathclap{i_{2}=1}}(\cdots)\sum^{n}_{\mathclap{i_{n+1}=1}}\epsilon_{i_{2},\dots,i_{n},i_{n+1}}(\mat{M}_{i_{1},i_{2}})_{2,i_{2}}(\cdots)(\mat{M}_{i_{1},i_{n}})_{n,i_{n}}(\mat{M}_{i_{1},i_{n+1}})_{n+1,i_{n+1}}\right).\nonumber
\end{align}
The argument is exactly the same: rewrite the minors by components and
explicitly enforce $i_{j}\neq i_{1}$ in the sums, then we would
replace the
$(-1)^{i_{1}+1}\epsilon_{i_{2},\dots,i_{n+1}}$ by
$\epsilon_{i_{1},i_{2},\dots,i_{n+1}}$, and find the $i_{j}\neq i_{1}$
conditions redundant (so we'd remove them),
then invoking distributivity to obtain Eq~\eqref{eq:determinant:using-levi-civita}.
We will therefore take Eq~\eqref{eq:determinant:using-levi-civita}
to be proven.

\begin{lemma}
The $n\times n$ identity matrix $\mat{I}$ has determinant 1.
\end{lemma}

\begin{proof}
  We find
\begin{calculation}
\det(\mat{I})
    \step{using Eq~\eqref{eq:determinant:using-levi-civita}}
\sum^{n}_{i_{1}=1}\dots\sum^{n}_{i_{n}=1}\epsilon_{i_{1},\dots,i_{n}}\delta_{1,i_{1}}\dots \delta_{n,i_{n}}
    \step{summing over $i_{n}$}
\sum^{n}_{i_{1}=1}\dots\sum^{n}_{i_{n-1}=1}\epsilon_{i_{1},\dots,i_{n-1},n}\delta_{1,i_{1}}\dots \delta_{n-1,i_{n-1}}
    \step{induction}
\sum^{n}_{i_{1}=1}\epsilon_{i_{1},2,\dots,n-1,n}\delta_{1,i_{1}}
    \step{defining property of the Kronecker--delta}
\epsilon_{1,2,\dots,n}
    \step{defining property of Levi--Civita}
+1
\end{calculation}
Hence we conclude, for any $n$, the $n\times n$ identity matrix has
determinant $\det(\mat{I})=1$.
\end{proof}

\begin{theorem}
If $\mat{D}$ is a diagonal $n\times n$ matrix
$\mat{D}=\diag(d_{1},\dots,d_{n})$, then its determinant is the product
of the diagonal entries
\begin{equation}
\det(\mat{D}) = \prod^{n}_{j=1}d_{j}.
\end{equation}
\end{theorem}

This theorem is important, its proof is lengthy and involved. There are
no tricks to learn from it, so its importance is just to ensure the
result is true.

\begin{proof}
  We find
  \begin{calculation}
\det(\mat{D})
    \step{Eq~\eqref{eq:determinant:using-levi-civita}}
\sum^{n}_{i_{1}=1}\dots\sum^{n}_{i_{n}=1}\epsilon_{i_{1},\dots,i_{n}}d_{1,i_{1}}\dots d_{n,i_{n}}
    \step{since $d_{1,i}=d_{1}\delta_{1,i}$, etc.}
\sum^{n}_{i_{1}=1}\dots\sum^{n}_{i_{n}=1}\epsilon_{i_{1},\dots,i_{n}}\prod^{n}_{j=1}d_{j}\delta_{j,i_{j}}
     \step{defining property of Kronecker-delta applied in each summation}
\epsilon_{1,2,\dots,n}\prod^{n}_{j=1}d_{j}
     \step{by definition of the Levi--Civita symbol}
(+1)\prod^{n}_{j=1}d_{j}.
  \end{calculation}
This concludes the proof.
\end{proof}

\begin{theorem}
For any $n\times n$ matrices $\mat{A}$, $\mat{B}$,
the determinant of their product is the product of their determinants
\begin{equation}
\det(\mat{A}\mat{B}) = \det(\mat{A})\det(\mat{B}).
\end{equation}
\end{theorem}

\begin{proof}
Let $\mat{C}=(c_{i,k})=\mat{A}\mat{B}$, $\mat{A}=(a_{i,j})$ and $\mat{B}=(b_{i,j})$.
We know from the definition of matrix multiplication
\begin{equation}\label{eq:determinant:pf-of-product:components-of-product}
c_{i,k} = \sum^{n}_{j=1}a_{i,j}b_{j,k}.
\end{equation}
We find
\begin{subequations}
\begin{calculation}
\det(\mat{C})
    \step{using Eq~\eqref{eq:determinant:using-levi-civita}}
\sum^{n}_{j_{1}=1}\cdots\sum^{n}_{j_{n}=1}\epsilon_{j_{1},\dots,j_{n}}c_{1,j_{1}}(\cdots)c_{n,j_{n}}
    \step{using the formula for $c_{i,k}$ from Eq~\eqref{eq:determinant:pf-of-product:components-of-product}}
\sum^{n}_{j_{1}=1}\cdots\sum^{n}_{j_{n}=1}\epsilon_{j_{1},\dots,j_{n}}\left(\sum^{n}_{k=1}a_{1,k}b_{k,j_{1}}\right)\left(\sum^{n}_{k=1}a_{2,k}b_{k,j_{2}}\right)
(\cdots)\left(\sum^{n}_{k=1}a_{n,k}b_{k,j_{n}}\right)
    \step{collapsing sums}
\sum^{n}_{\substack{j_{1}=1\\\svdots\\j_{n}=1}}\epsilon_{j_{1},\dots,j_{n}}\left(\sum^{n}_{k=1}a_{1,k}b_{k,j_{1}}\right)\left(\sum^{n}_{k=1}a_{2,k}b_{k,j_{2}}\right)
(\cdots)\left(\sum^{n}_{k=1}a_{n,k}b_{k,j_{n}}\right)
    \step{reindexing}
\sum^{n}_{\substack{j_{1}=1\\\svdots\\j_{n}=1}}\epsilon_{j_{1},\dots,j_{n}}\left(\sum^{n}_{i_{1}=1}a_{1,i_{1}}b_{i_{1},j_{1}}\right)\left(\sum^{n}_{i_{2}=1}a_{2,i_{2}}b_{i_{2},j_{2}}\right)
(\cdots)\left(\sum^{n}_{i_{n}=1}a_{n,i_{n}}b_{i_{n},j_{n}}\right)
    \step{distributivity}
\sum^{n}_{\substack{i_{1}=1\\\svdots\\i_{n}=1}}\sum^{n}_{\substack{j_{1}=1\\\svdots\\j_{n}=1}}\epsilon_{j_{1},\dots,j_{n}}a_{1,i_{1}}b_{i_{1},j_{1}}a_{2,i_{2}}b_{i_{2},j_{2}}(\cdots)a_{n,i_{n}}b_{i_{n},j_{n}}
\end{calculation}
\end{subequations}
If we start at the other end, we find
\begin{subequations}
\begin{calculation}
\det(\mat{A})\det(\mat{B})
    \step{using Eq~\eqref{eq:determinant:using-levi-civita}}
\left(\sum^{n}_{i_{1}=1}\cdots\sum^{n}_{i_{n}=1}\epsilon_{i_{1},\dots,i_{n}}a_{1,i_{1}}(\cdots)a_{n,i_{n}}\right)
\left(\sum^{n}_{j_{1}=1}\cdots\sum^{n}_{j_{n}=1}\epsilon_{j_{1},\dots,j_{n}}b_{1,j_{1}}(\cdots)b_{n,j_{n}}\right)
    \step{collapsing indices into a single summation symbol}
\left(\sum^{n}_{\substack{i_{1}=1\\\svdots\\i_{n}=1}}\epsilon_{i_{1},\dots,i_{n}}a_{1,i_{1}}(\cdots)a_{n,i_{n}}\right)
\left(\sum^{n}_{\substack{j_{1}=1\\\svdots\\j_{n}=1}}\epsilon_{j_{1},\dots,j_{n}}b_{1,j_{1}}(\cdots)b_{n,j_{n}}\right)
    \step{using distributivity}
\sum^{n}_{\substack{i_{1}=1\\\svdots\\i_{n}=1}}\sum^{n}_{\substack{j_{1}=1\\\svdots\\j_{n}=1}}\left(\epsilon_{i_{1},\dots,i_{n}}a_{1,i_{1}}(\cdots)a_{n,i_{n}}\epsilon_{j_{1},\dots,j_{n}}b_{1,j_{1}}(\cdots)b_{n,j_{n}}\right)
    \step{reindexing the $j$ indices}
\sum^{n}_{\substack{i_{1}=1\\\svdots\\i_{n}=1}}\sum^{n}_{\substack{j_{i_{1}}=1\\\svdots\\j_{i_{n}}=1}}%
\left(\epsilon_{i_{1},\dots,i_{n}}a_{1,i_{1}}(\cdots)a_{n,i_{n}}
\epsilon_{j_{i_{1}},\dots,j_{i_{n}}}b_{i_{1},j_{i_{1}}}(\cdots)b_{i_{n},j_{i_{n}}}\right)
\end{calculation}
We want to show that reindexing the $j_{i_{k}}$ indices as $j_{k}$ will
cost us a factor of the Levi--Civita symbol. Well, we would be applying
a permutation $\pi$ to the $j$ indices, which \emph{would} cost us a
Levi--Civita symbol.
\begin{calculation}
\sum^{n}_{\substack{i_{1}=1\\\svdots\\i_{n}=1}}\sum^{n}_{\substack{j_{i_{1}}=1\\\svdots\\j_{i_{n}}=1}}%
\left(\epsilon_{i_{1},\dots,i_{n}}a_{1,i_{1}}(\cdots)a_{n,i_{n}}
\epsilon_{j_{i_{1}},\dots,j_{i_{n}}}b_{i_{1},j_{i_{1}}}(\cdots)b_{i_{n},j_{i_{n}}}\right)
\step{permuting $j_{i_{k}}$ with $j_{k}$ costs us a Levi--Civita symbol}
\sum^{n}_{\substack{i_{1}=1\\\svdots\\i_{n}=1}}\sum^{n}_{\substack{j_{1}=1\\\svdots\\j_{n}=1}}%
\left(\epsilon_{i_{1},\dots,i_{n}}a_{1,i_{1}}(\cdots)a_{n,i_{n}}
\epsilon_{i_{1},\dots,i_{n}}\epsilon_{j_{1},\dots,j_{n}}b_{i_{1},j_{1}}(\cdots)b_{i_{n},j_{n}}\right)
\step{the $\epsilon_{j_{1},\dots,j_{n}}(\epsilon_{i_{1},\dots,i_{n}})^{2}$ is just $\epsilon_{j_{1},\dots,j_{n}}$}
\sum^{n}_{\substack{i_{1}=1\\\svdots\\i_{n}=1}}\sum^{n}_{\substack{j_{1}=1\\\svdots\\j_{n}=1}}%
\left(a_{1,i_{1}}(\cdots)a_{n,i_{n}}\epsilon_{j_{1},\dots,j_{n}}b_{i_{1},j_{1}}(\cdots)b_{i_{n},j_{n}}\right)
\step{associativity}
\sum^{n}_{\substack{i_{1}=1\\\svdots\\i_{n}=1}}\sum^{n}_{\substack{j_{1}=1\\\svdots\\j_{n}=1}}%
\epsilon_{j_{1},\dots,j_{n}}
a_{1,i_{1}}(\cdots)a_{n,i_{n}}b_{i_{1},j_{1}}(\cdots)b_{i_{n},j_{n}}
\end{calculation}
\end{subequations}
The fact we could replace $\epsilon_{i_{1},\dots,i_{n}}\epsilon_{i_{1},\dots,i_{n}}\epsilon_{j_{1},\dots,j_{n}}$
with $\epsilon_{j_{1},\dots,j_{n}}$ stems from the fact that we require
the $i_{1}\neq i_{2}\neq\dots\neq i_{n}$ to be distinct in the sum, but
this is also equivalent to demanding the $j_{1}\neq j_{2}\neq\dots\neq j_{n}$
be distinct; however, the $\epsilon_{j_{1},\dots,j_{n}}$ factor
guarantees this is always the case. So the $\epsilon_{i_{1},\dots,i_{n}}^{2}$
would contribute only its magnitude (i.e., a factor of $+1$), hence we
can drop it.
But we see, this is precisely what we worked out for $\det(\mat{A}\mat{B})$.
Hence the result follows.
\end{proof}

\begin{lemma}
Let $\mat{A}=(a_{i,j})$ be an $n\times n$ matrix, let $r\in\RR$ be some
number, and let $\mat{B}=(b_{i,j})$ be obtained from $\mat{A}$ by adding
$r$ times row $\mu$ to row $\nu$ $b_{i,j} = a_{i,j} + ra_{\mu,j}\delta_{i,\nu}$.
Then $\det(\mat{A})=\det(\mat{B})$.
\end{lemma}
\begin{proof}
We see that $\mat{B} = (\mat{I} + r\mat{E}_{\mu,\nu})\mat{A}$, where
$\mat{E}_{\mu\nu}$ has a single nonzero entry located at row $\mu$,
column $\nu$, which is equal to 1. Then
$\det(\mat{B}) = \det(\mat{I} + r\mat{E}_{\mu,\nu})\det(\mat{A})$ since
the determinant of products is the product of determinants.
But $\mat{I} + r\mat{E}_{\mu,\nu}$ is either upper-triangular (when
$\mu<\nu$) or lower-triangular (when $\mu>\nu$), and in both cases the
determinant would be just the product of the diagonal
$\det(\mat{I} + r\mat{E}_{\mu,\nu})=\det(\mat{I})$ and we determined
this is 1. Hence $\det(\mat{B})=(1)\det(\mat{A})$.
\end{proof}

\begin{theorem}
Let $\mat{A}$ be an $n\times n$ matrix such that row $i$ is a multiple
of row $j$. Then $\det(\mat{A})=0$.

Let $\mat{B}$ be an $n\times n$ matrix such that column $i$ is a
multiple of column $j$. Then $\det(\mat{B})=0$.
\end{theorem}
\begin{proof}
Since we can add multiples of rows (or columns) to other rows without
affecting the determinant, we see if we subtract a multiple of row $i$
from row $j$ in $\mat{A}$ we will have a row consisting of zeroes. This
obviously has determinant 0 (it was the first thing we proved after the
definition of determinant).

Similar reasoning holds for $\mat{B}$.
\end{proof}

\begin{lemma}
Let $\mat{S}(i,j)$ be the $n\times n$ matrix obtained from the identity
matrix, swapping row $i$ and row $j$. Then
$\det(\mat{S}(i,j)) = (-1)^{i-j}$.
\end{lemma}

\begin{proof}
Let us prove this for the case when $j=i+1$. The general case follows by
applying this particular case repeatedly.
In this case, our matrix would look, in block form, like
\begin{equation}
  \det(\mat{S}(i,i+1)) = \left(\begin{array}{c|cc|c}
    \mat{I} & 0  & 0 & 0\\\hline
    0 & 0 & 1 & 0\\
    0 & 1 & 0 & 0\\\hline
    0 & 0 & 0 & \mat{I}
  \end{array}\right)
\end{equation}
We then add row $i+1$ to row $i$
\begin{equation}
(\mat{I} + \mat{E}_{i+1,i})(\mat{S}(i,i+1)) = \left(\begin{array}{c|cc|c}
    \mat{I} & 0  & 0 & 0\\\hline
    0 & 1 & 1 & 0\\
    0 & 1 & 0 & 0\\\hline
    0 & 0 & 0 & \mat{I}
  \end{array}\right)
\end{equation}
Then we subtract row $i$ from row $i+1$:
\begin{equation}
(\mat{I} - \mat{E}_{i,i+1})(\mat{I} + \mat{E}_{i+1,i})(\mat{S}(i,i+1)) = \left(\begin{array}{c|cc|c}
    \mat{I} & 0  & 0 & 0\\\hline
    0 & 1 & 1 & 0\\
    0 & 0 & -1 & 0\\\hline
    0 & 0 & 0 & \mat{I}
  \end{array}\right)
\end{equation}
Since this is upper-triangular,its determinant is just the product of
diagonal entries:
\begin{equation}
\det\left((\mat{I} - \mat{E}_{i,i+1})(\mat{I} + \mat{E}_{i+1,i})(\mat{S}(i,i+1))\right)=-1.
\end{equation}
However, since adding a multiple of one row to another does not affect
the determinant, we find
\begin{equation}
\det\left((\mat{I} - \mat{E}_{i+1,i})(\mat{I} + \mat{E}_{i,i+1})\mat{S}(i,i+1)\right)=\det(\mat{S}(i,i+1)).
\end{equation}
Hence we establish the case when $j=i+1$.

This general reasoning holds when we restore $j$, namely that
\begin{equation}
\det\left((\mat{I} - \mat{E}_{i,j})(\mat{I} + \mat{E}_{j,i})(\mat{S}(i,j))\right)=-1,
\end{equation}
hence
\begin{equation}
\det(\mat{S}(i,j)) = -1.
\end{equation}
Precisely as desired.
\end{proof}

\begin{proposition}
Let $\mat{A}$ be an $n\times n$ matrix, let $\mat{B}$ be obtained by
swapping rows $i$ and $j$ in $\mat{A}$.
Then $\det(\mat{B})=(-1)^{i-j}\det(\mat{A})$.
\end{proposition}

\begin{proof}
We see that $\mat{B}=\mat{S}(i,j)\mat{A}$. Then by the determinant of
products is the product of determinants, we find
\begin{equation}
\det(\mat{B})=\det(\mat{S}(i,j))\det(\mat{A}).
\end{equation}
We just proved $\det(\mat{S}(i,j))=(-1)^{i-j}$, so substituting this in
yields the result.
\end{proof}

\begin{theorem}
Let $\mat{A}$ be an invertible $n\times n$ matrix.
If $\det(\mat{A})\neq0$,
then $\det(\mat{A}^{-1}) = \left(\det(\mat{A})\right)^{-1}$.
\end{theorem}

\begin{proof}
We know $\mat{A}\mat{A}^{-1}=\mat{I}$, and then taking the determinant
of both sides yields
\begin{equation}
\det(\mat{A}\mat{A}^{-1}) = 1.
\end{equation}
We know the determinant of products is the product of determinants
\begin{equation}
\det(\mat{A}\mat{A}^{-1}) = \det(\mat{A})\det(\mat{A}^{-1}) = 1.
\end{equation}
Dividing both sides by $\det(\mat{A})$ yields the result.
\end{proof}



\begin{theorem}\label{thm:determinant:singular-matrices-have-zero-det}
  Let $\mat{A}$ be an $n\times n$ matrix.
  The determinant is zero if and only if $\mat{A}$ is singular;
  equivalently, the determinant is nonzero if and only if $\mat{A}$ is
  invertible.
\end{theorem}

\begin{proof}
Suppose $\mat{A}$ is invertible. Then
$\det(\mat{A}\mat{A}^{-1})=\det(\mat{I})=1\neq0$. In particular, this
means $\det(\mat{A})\neq0$.

The other direction is just the contrapositive: ``if $p$, then $q$'' is
logically equivalent to its contrapositive ``if not-$q$, then not-$p$''.
Here $q$ is ``$\det(\mat{A})\neq0$'', and $p$ is ``$\mat{A}$ is invertible''.
Hence the contrapositive is ``If $\det(\mat{A})=0$, then $\mat{A}$ is
not invertible''. And that's what we wanted to prove! So, we're done.
\end{proof}

\begin{theorem}
Let $\mat{A}$ be an $n\times n$ matrix.
The determinant of the transpose is the determinant of the original matrix:
\begin{equation*}
\det(\transpose{\mat{A}}) = \det(\mat{A}).
\end{equation*}
\end{theorem}

\begin{proof}
  Either $\mat{A}$ is invertible or not. If not, then it is singular and
  has zero determinant. Its transpose will be singular, and have zero
  determinant. Hence the result holds for singular matrices.

  For nonsingular matrices, we recall the LU-factorization
  $\mat{A}=\mat{L}\mat{U}$ where $\mat{U}$ has nonzero diagonal entries
  (because $\mat{A}$ is nonsingular) and $\mat{L}$ has 1 along its
  diagonal (hence $\det(\mat{L})=1$. Hence
  \begin{subequations}
  \begin{calculation}
    \det(\transpose{\mat{A}})
\step{LU decomposition}
    \det(\transpose{(\mat{L}\mat{U})})
\step{transpose of products is reverse product of transposes}
    \det(\transpose{\mat{U}}\transpose{\mat{L}})
\step{determinant of product is product of determinants}
    \det(\transpose{\mat{U}})\det(\transpose{\mat{L}})
\step{$\transpose{\mat{L}}$ is triangular with 1 on diagonal}
    \det(\transpose{\mat{U}})1 =\det(\transpose{\mat{U}})
\step{$\transpose{\mat{U}}$ is triangular with the same diagonal entries
  as $\mat{U}$}
  \det(\mat{U})
\step{$\mat{L}$ is triangular with 1 on diagonal}
  1\det(\mat{U}) = \det(\mat{L})\det(\mat{U})
\step{product of determinants is determinant of products}
  \det(\mat{L}\mat{U})
\step{by LU decomposition of $\mat{A}$}
  \det(\mat{A}).
  \end{calculation}
  \end{subequations}
  Hence the result.
\end{proof}

\begin{remark}
This proof is important, because it shows the basic gambit in linear
algebra: we take a matrix, and try factorizing it into a product of nice
matrices. Then we use this factorization to prove the desired result.
\end{remark}

\phantomsection
\subsection*{Exercises}
\addcontentsline{toc}{subsection}{Exercises}

\begin{exercise}
Let $\mat{M}$ be an $n\times n$ matrix.
We call $\mat{M}$ \define{Antisymmetric} if $\transpose{\mat{M}}=-\mat{M}$.
\begin{enumerate}
\item What would the diagonal values be for an antisymmetric matrix?
\item Write down a $3\times 3$ antisymmetric matrix.
\item What would the determinant of a $3\times3$ antisymmetric matrix
  be? What about a $4\times 4$ antisymmetric matrix?
\end{enumerate}
\end{exercise}

\begin{exercise}
Let $\mat{A}$ be an $n\times n$ matrix, let $c\neq 0$ be a nonzero number.
Is $\det(c\mat{A})$ a multiple of $\det(\mat{A})$? If so, what is that multiple?
If not, what is the determinant of a scalar matrix?
\end{exercise}

\begin{exercise}
Prove or find a counter-example: if $\vec{a}=(a_{j})$ and
$\vec{b}=(b_{k})$ are vectors in $\RR^{3}$, then their cross product has
components $\vec{a}\times\vec{b} = (\sum_{j}\sum_{k}\epsilon_{i,j,k}a_{j}b_{k})$.
\end{exercise}

(This gives one generalization of the cross-product to other dimensions
using the Levi--Civita symbol. We see the difficulty generalizing it to
higher dimensions: the Levi--Civita symbol in $n$ dimensions has $n$
indices. In other words, for $n\geq3$ dimensions, we need $n-1$ vectors
to form a cross-product.)

\begin{exercise}
We call an $n\times n$ matrix $\mat{N}$ \define{Nilpotent} if there is a
positive integer $m$ such that $\mat{N}^{m}=0$. If $\mat{N}$ is
nilpotent, then what is its determinant?
\end{exercise}

\begin{exercise}
We call an $n\times n$ matrix $\mat{A}$ \define{Idempotent} if
$\mat{A}^{2}=\mat{A}$. What constraints does this impose on $\det(\mat{A})$?
\end{exercise}

\begin{exercise}
Suppose $\mat{B}$ is an $n\times n$ matrix obtained from $\mat{A}$ by
multiplying row $i$ in $\mat{A}$ by a nonzero number $\gamma$. What is
$\det(\mat{B})$ in terms of $\det(\mat{A})$ and $\gamma$?

[Hint: they are not equal to each other in general, only for $\gamma=1$
  are the determinants equal.]
\end{exercise}

\begin{exercise}
Let $D$ be a function which eats in an $n\times n$ matrix $\mat{A}$
treated as $n$ column vectors $\mat{A}=(\vec{a}_{1},\vec{a}_{2}, \dots,\vec{a}_{n})$;
so $D(\mat{A}) = D(\vec{a}_{1},\vec{a}_{2},\dots,\vec{a}_{n})$ such that
\begin{enumerate}
\item $D(\mat{I})=1$ for the identity matrix
\item it is linear in every slot: for any $\vec{u}$, $\vec{v}\in\RR^{n}$,
  and every $a,b\in\RR$, we have
  $D(\vec{a}_{1},\dots,a\vec{u}+b\vec{v},\dots,\vec{a}_{n})=aD(\vec{a}_{1},\dots,\vec{u},\dots,\vec{a}_{n}) + bD(\vec{a}_{1},\dots,\vec{v},\dots,\vec{a}_{n})$
\item it is alternating: for any $i=1,2,\dots,n-1$, we have
  $D(\vec{a}_{1},\dots,\vec{a}_{i},\vec{a}_{i+1},\dots,\vec{a}_{n})=-D(\vec{a}_{1},\dots,\vec{a}_{i+1},\vec{a}_{i},\dots,\vec{a}_{n})$.
\end{enumerate}
Prove or find a counter-example: the function $D$ is just the determinant $D(\mat{A})=\det(\mat{A})$.
\end{exercise}

\subsection{Trace of a Matrix}

\begin{proposition}\label{prop:determinant:trace-as-infinitesimal-determinant}
Let $\mat{X}$ be an $n\times n$ matrix, let $\varepsilon>0$ be ``small''
(in the sense that we discard terms of order $\varepsilon^{2}$ or higher).
Then
\begin{equation}
\det(\mat{I} + \varepsilon\mat{X}) = 1 + \varepsilon\sum^{n}_{j=1}(\mat{X})_{j,j}.
\end{equation}
\end{proposition}

\begin{proof}
This follows from the fact that $1\gg\varepsilon^{2}$ and the product of
off-diagonal components would contribute a $\varepsilon^{2}$ term (or
some higher power of $\varepsilon$). So
\begin{equation}
\det(\mat{I} + \varepsilon\mat{X}) = \prod^{n}_{j=1}(1 + \varepsilon(\mat{X})_{j,j})+\mathcal{O}(\varepsilon^{2}).
\end{equation}
Expanding out the product, and keeping only the first-order
$\varepsilon$ terms gives us the result.
\end{proof}

\begin{remark}
From this perspective, the ``linear approximation'' to the determinant
``near the identity matrix'' is precisely the sum of diagonal entries in
the ``perturbation'' about the identity matrix. For this reason, it
deserves to be defined.
\end{remark}

\begin{definition}
Let $\mat{A}=(a_{i,j})$ be an $n\times n$ matrix.
We define its \define{Trace} to be the sum of its diagonal components
\begin{equation}
\tr(A) = \sum^{n}_{j=1}a_{j,j}.
\end{equation}
\end{definition}

\begin{proposition}
The trace of the transpose is the trace of the original matrix,
$\tr(\transpose{\mat{A}})=\tr(\mat{A})$.
\end{proposition}
\begin{proof}
Let $\mat{B}=\transpose{\mat{A}}$. Then its components would be
$(\mat{B})_{i,j}=(\mat{A})_{j,i}$ and in particular the diagonal
components are the same. So the sum of the diagonal components would be
equal.
\end{proof}

\begin{proposition}
Let $\mat{A}$ be an $n\times n$ matrix, let $c$ be some number.
Then $\tr(c\mat{A})=c\tr(\mat{A})$.
\end{proposition}

\begin{proposition}
Let $\mat{A}$ be an $m\times n$ matrix, let $\mat{B}$ be an $n\times m$ matrix.
Then $\tr(\mat{A}\mat{B})=\tr(\mat{B}\mat{A})$.
\end{proposition}

\begin{corollary}[Cyclic property]
Let $\mat{A}_{1}$, \dots, $\mat{A}_{k}$ be matrices of appropriate
dimension. Then $\tr(\mat{A}_{1}\mat{A}_{2}\dots\mat{A}_{k})=\tr(\mat{A}_{2}\dots\mat{A}_{k}\mat{A}_{1})$.
\end{corollary}

\begin{proof}
  We use associativity of matrix multiplication to write
  \begin{equation}
\tr(\mat{A}_{1}\mat{A}_{2}\dots\mat{A}_{k}) = \tr(\mat{A}_{1}(\mat{A}_{2}\dots\mat{A}_{k})),
  \end{equation}
  and then using the previous proposition we have
  \begin{equation}
\tr(\mat{A}_{1}(\mat{A}_{2}\dots\mat{A}_{k})) = \tr((\mat{A}_{2}\dots\mat{A}_{k})\mat{A}_{1}).
  \end{equation}
  Invoking associativity of matrix multiplication again yields the result.
\end{proof}

\begin{proposition}
Let $\mat{A}$, $\mat{B}$ be $n\times n$ matrices.
Then $\tr(\mat{A} + \mat{B}) = \tr(\mat{A}) + \tr(\mat{B})$.
\end{proposition}


\phantomsection
\subsection*{Exercises}
\addcontentsline{toc}{subsection}{Exercises}

\begin{exercise}
Prove or find a counter-example: if $\mat{A}$ is an invertible $n\times n$
matrix, $0<\varepsilon\ll1$ is a small real number, and $\mat{B}$ is an
arbitrary $n\times n$ matrix, then
$\det(\mat{A} + \varepsilon\mat{B})\approx \det(\mat{A})(1 + \varepsilon\tr(\mat{A}^{-1}\mat{B}))$
(plus higher order corrections in $\varepsilon$).
\end{exercise}

\begin{exercise}
If $\mat{A}$ is an $n\times n$ matrix such that $\transpose{\mat{A}}=\mat{A}^{-1}$,
then what values could $\det(\mat{A})$ be?
\end{exercise}

\begin{exercise}
In abstract algebra, the Vandermonde determinant is useful for studying
the roots of polynomials. The Vandermonde determinant in three variables
is given by the determinant
\[ \det\begin{pmatrix}
1     & 1     & 1 \\
a     & b     & c\\
a^{2} & b^{2} & c^{2}
\end{pmatrix}=(a-b)(b-c)(c-a). \]
Prove this formula actually holds. [Hint: expand both sides
  independently, then show they are equal to each other.]
\end{exercise}

% \section{Eigenstuff}

\N{Note to self}
I wanted to include this as the ending of part II, but I realized the
usefulness of eigenstuff is in having eigenvectors forming a basis, and
then diagonalizing the matrix. This requires putting this section in part III.

\begin{example}[Motivating Example]
  Consider the matrix
  \begin{equation}
\mat{M} = \begin{pmatrix}2 & -1\\
-1 & 2
\end{pmatrix}.
  \end{equation}
  We find that there are two ``directions'' (distinct vectors) which are
  just dilated when we multiply by $\mat{M}$,
  \begin{subequations}
    \begin{equation}
\mat{M}\begin{pmatrix}1\\1\end{pmatrix} = \begin{pmatrix}1\\1\end{pmatrix},
    \end{equation}
    and
    \begin{equation}
\mat{M}\begin{pmatrix}1\\-1\end{pmatrix} = 3\begin{pmatrix}1\\-1\end{pmatrix}.
    \end{equation}
  \end{subequations}
  This isn't a neat parlor trick: it turns out any invertible $n\times n$
  matrix will have at most $n$ vectors which are ``dilated'' by the matrix.
\end{example}

\begin{definition}
Let $\mat{A}$ be an $n\times n$ matrix.
We define an \define{Eigenvector} of $\mat{A}$ to be a [column]
$n$-vector $\vec{v}$ such that there is a nonzero $\lambda\in\RR$
[called the \define{Eigenvalue} associated with $\vec{v}$] satisfying
\begin{equation}\label{eq:defn:eigenvector}
\mat{A}\vec{v}=\lambda\vec{v}.
\end{equation}
\end{definition}

\N{Finding Eigenvalues and Eigenvectors}
This is great, but how do we find eigenvalues and eigenvectors?
The first thing to note is we can rewrite Eq~\eqref{eq:defn:eigenvector}
by subtracting $\lambda\vec{v}$ from both sides:
\begin{equation}
\mat{A}\vec{v}-\lambda\vec{v}=\vec{0}.
\end{equation}
We insert a secret identity operator (the matrix analog of ``multiply by $1$''):
\begin{equation}
\mat{A}\vec{v}-\lambda\mat{I}\vec{v}=\vec{0}.
\end{equation}
We can factor out $\vec{v}$ by distributivity:
\begin{equation}
(\mat{A}-\lambda\mat{I})\vec{v}=\vec{0}.
\end{equation}
For this equation to hold, either $\vec{v}=0$ or
$(\mat{A}-\lambda\mat{I})=0$, right?

Wrong: $(\mat{A}-\lambda\mat{I})$ could be nonzero and noninvertible.
That is when
\begin{equation}
\det(\mat{A}-\lambda\mat{I})=0.
\end{equation}
But the left-hand side is not identically zero. In fact, the left-hand
side is a polynomial in $\lambda$.
\emph{This polynomial is how we find eigenvalues for matrices.}

Once we have an eigenvalue, we can plug it in and then solve the system
of equations for the eigenvector. But first, let us define this
polynomial quantity.

\begin{definition}
Let $\mat{A}$ be an $n\times n$ matrix.
The \define{Characteristic Polynomial} of $\mat{A}$ is 
\begin{equation}
p(\lambda) = \det(\mat{A}-\lambda\mat{I}).
\end{equation}
Some authors use $\det(\lambda\mat{I}-\mat{A})$, it doesn't matter since
they have the same roots (which are the eigenvalues of $\mat{A}$ and the
\emph{actual} quantity of interest).
\end{definition}

\begin{example}
Recall our motivating example at the start of this section, we had
\begin{equation}
\mat{M} = \begin{pmatrix}2 & -1\\
-1 & 2
\end{pmatrix}.
\end{equation}
Its characteristic polynomial is
\begin{equation}
\det\begin{pmatrix}2-\lambda & -1\\
-1 & 2-\lambda
\end{pmatrix} = (2-\lambda)^{2}-1 = \lambda^{2}-4\lambda+3.
\end{equation}
We find this has roots $\lambda=1$ and $\lambda=3$.

We can find the eigenvector for $\lambda=1$ by solving
\begin{subequations}
  \begin{align}
    2x_{1} -x_{2} &= x_{1}\\
    -x_{1} + 2x_{2} &= x_{2}.
  \end{align}
\end{subequations}
These give us 2 copies of the same line described by
\begin{equation}
-x_{1} = -x_{2},\quad\mbox{or}\quad x_{1}=x_{2}.
\end{equation}
We have a generic eigenvector look like
\begin{equation*}
\vec{v}_{1} = m\begin{pmatrix}1\\1
\end{pmatrix}
\end{equation*}
where $m\in\RR$. Usually we normalize the eigenvector to be a unit
vector (so we fix any such parameters), which gives us
\begin{equation}
  \vec{v}_{1} = \begin{pmatrix}1/\sqrt{2}\\
    1/\sqrt{2}
  \end{pmatrix}.
\end{equation}
This is one eigenvector.

The other eigenvector, the one associated with $\lambda=3$, requires
solving the system of equations
\begin{subequations}
  \begin{align}
    2x_{1} -x_{2} &= 3x_{1}\\
    -x_{1} + 2x_{2} &= 3x_{2}.
  \end{align}
\end{subequations}
This gives us two copies of the same line, described by the equation
\begin{equation}
x_{1} = -x_{2}.
\end{equation}
The unit eigenvector is then
\begin{equation}
\vec{v}_{2} = \begin{pmatrix}1/\sqrt{2}\\
-1/\sqrt{2}
\end{pmatrix}.
\end{equation}
The reader may verify these satisfy the equation $\mat{M}\vec{v}=\lambda\vec{v}$
for eigenvectors of $\mat{M}$.
\end{example}

\vfill\eject
\part{Vector Spaces}

\N*{Roadmap}
We could stop here and die happily (after all, we have introduced a way
to solve systems of linear equations using matrices, which was our
mission statement), but it would be a shallow
life. Instead, we will discuss one more layer of abstraction, one more
shiny gadget that will help us understand systems of linear equations
more profoundly: vector spaces. We can meaningfully discuss ``spaces of
solutions'' to a problem, and it's the laboratory of mathematics which
showcases ``how mathematicians do stuff''.

% Vectors in R^{n}
\section{Vectors in \texorpdfstring{$\RR^{n}$}{Rn}}\label{section:vectors-in-r-n}

\M
We learn about vectors at university specifically so we could do vector
calculus. But we have only mentioned vectors in our notes on linear
algebra in passing as a particular ``species'' of matrices. Is this an
unfortunate collision of language? That is to say, are these two notions
distinct [not secretly the same] but unfortunately use the same term?

Let us review what a ``vector'' was for the real plane $\RR^{2}$.

\N{Real Number Line}
Let us recall what happens in the simplest case of all: the real number
line. The first thing we do is draw a line (in the Euclidean sense,
which extends in both directions infinitely far) and pick some point $O$
on the line:
\begin{center}
  \includegraphics{img/img.0}
\end{center}
We then pick a point $A\neq O$ on the line (which is not the same point
as $O$):
\begin{center}
  \includegraphics{img/img.1}
\end{center}
We take the distance between point $O$ and $A$ to be 1 unit. We then
construct the rest of the ``ticks'' along this line, and identify every
point with a number in the real numbers. Intuitively corresponding to
the picture:
\begin{center}
  \includegraphics{img/img.2}
\end{center}
How do we make this identification? We take a point $P$ on the line,
find its distance [from $O$ to $P$] as a multiple of the distance from
$O$ to $A$. In the case where, like point $Q$, it lies ``in the other
direction'', we identify point $Q$ with the negative real numbers by
dividing the distance between $O$ and $Q$ with the distance between $O$
and $A$ (and multiplying the result by $-1$). This is a real number
$x\in\RR$. Conversely, if $x\in\RR$, then we can identify with a point
on the line by dilating the line segment $\overline{OA}$ by $x$.

When we identify point $P$ on the line by the real number $x$, we call
$x$ the \define{Coordinate} of $P$. The point $O$ has coordinate $0$,
and the distance between points $P$ and $Q$ (who have coordinates
$x_{P}$ and $x_{Q}$, respectively) by the magnitude of the difference in
their coordinates $|x_{Q}-x_{P}|$.

\N{Constructing the Plane}
We can construct the plane by taking a line which passes through $O$ and
forms a $90^{\circ}$ angle with our real line. If we just ``plop'' down
such a line, we get:
\begin{center}
\includegraphics{img/img.3}
\end{center}
We can then form ``ticks'' along this new line, and call it the
\define{$y$ Axis} (our old line is the \define{$x$ Axis}):
\begin{center}
\includegraphics{img/img.4}
\end{center}
Instead of identifying a point in the plane with \emph{one} real number,
we now have a \emph{pair} of real numbers. These are obtained by
projecting onto the axes (the projection to the $x$-axis is in red, the
projection to the $y$-axis is in green):
\begin{center}
\includegraphics{img/img.5}
\end{center}
These give us points on the axes, which then give us coordinates $x_{P}$
and $y_{P}$ for those points on the axes. We combine these in an ordered
pair $(x_{P}, y_{P})$ and call them the \define{Coordinates} of $P$.
Conversely, given any pair of numbers $(x,y)\in\RR^{2}$, we can
construct a point in the plane by reversing the steps in this procedure.

\N{Vectors in the Plane}
Recall, we define a column 2-vector as a $2\times1$ matrix
\begin{equation}
\vec{v} = \begin{pmatrix}x_{1}\\x_{2}
\end{pmatrix}.
\end{equation}
We identify $\vec{p}$ with a point in the plane by constructing an
oriented line segment from $O$ (the point with coordinates $(0,0)$ in
the plane) and the point $P$ (with coordinates $(x_{1},x_{2})$). We
write this oriented line segment using the notation
$\overrightarrow{OP}$. We call $O$ the \define{Base Point} and $P$ the
\define{Endpoint} of $\overrightarrow{OP}$.
\begin{center}
\includegraphics{img/img.6}
\end{center}
The directed line segment $\overrightarrow{OP}$ has a \define{Direction} (given
by the angle formed with the positive $x_{1}$-axis) and a
\define{Magnitude} (given by its length).

We identify any directed line segment $\overrightarrow{OP}$ with base
point $O$ with a vector $\vec{v}$ by taking the components of $\vec{v}$
to be the coordinates of $P$ relative to base-point $O$.

Conversely, given any 2-vector $\vec{v}=(v_{1},v_{2})$, we identify it
with a directed line segment [in the plane with axes and units] from $O$
to $P$ where $P$ has coordinates $(v_{1},v_{2})$.

\begin{ddanger}\textsc{Caution:}
In physics, we often work with vectors with base point $Q$ and endpoint
$P$ in the plane (or in $\RR^{3}$ or wherever) and quite cavalierly
identify the oriented line segment $\overrightarrow{QP}$ with the line
segment of equal length and direction $\overrightarrow{OP'}$ located at
base-point $O$. This works because of the magic of $\RR^{n}$ being a
flat manifold. It doesn't work in general. In fact, in linear algebra,
we anchor all vectors to the same base point $O$.
\end{ddanger}

\begin{definition}
Let $\vec{v}=(v_{1},v_{2})$ be a vector in the plane. We define its
\define{Magnitude} by the non-negative real number
$\|\vec{v}\| = \sqrt{v_{1}^{2}+v_{2}^{2}}$.
\end{definition}

\N{Parallelogram Law}
When we have two vectors $\vec{u}$ and $\vec{v}$ in the plane (or in space), we can form a
parallelogram using vectors $\vec{u}$ and $\vec{v}$ as edges and their
shared base point as the origin. Their sum
is the vector to the opposite diagonal vertex from the origin:
\begin{center}
  \includegraphics{img/img.7}
\end{center}
The reader may verify the origin plus the two vectors give us three
vertices, and demanding a parallelogram gives us the remaining
vertex. Further, the coordinates of the remaining vertex may be obtained
by adding componentwise the coordinates of the endpoints for the vectors.

\begin{remark}
Fascinatingly, this ``parallelogram law'' is mistakenly attributed to
Pseudo-Aristotle, but recent scholarship shows this is a
misunderstanding. The first uses of the parallelogram law may be found
in Fermat (of ``Last Theorem'' fame) and Thomas Hobbes (the pessimistic ``rude,
short, and brutish'' philosopher). For further details, the reader is
invited to enjoy David Marshall Miller's
``The Parallelogram Rule from Pseudo-Aristotle to Newton''
\journal{Archive for History of Exact Sciences} \volume{71} (2017) pp.157--191.
\end{remark}

\N{Scalar Multiplication}
If we have a vector $\vec{v}$ corresponding to the oriented line segment
$\overrightarrow{OP}$ and a real number $r$, we may form the
scalar multiple $r\vec{v}$ by three cases:
\begin{enumerate}
\item Case 1 $r>0$: we just move the endpoint along the line passing
  through $O$ and $P$ to have a magnitude $r$ times greater than what
  the magnitude of $\overrightarrow{OP}$ is;
\item Case 2 $r=0$: we have the coordinates of the resulting vector be
  all zeroes.
\item Case 3 $r<0$: we find $Q$ the point along the line passing
  through $O$ and $P$ of the same magnitude but opposite direction of
  $\overrightarrow{OP}$, and multiply the magnitude of the line segment
  from $O$ to this new point $Q$ by $|r|>0$ to produce a point $R$ and
  identify $\overrightarrow{OR}$ with our new vector.
\end{enumerate}

\N{Consistency and Coherence Checks: Vector Subtraction}
We now may observe, for any vectors $\vec{u}$ and $\vec{v}$ in the
plane, that $\vec{u}-\vec{v}$ corresponds to the vector sum of $\vec{u}$
with the scalar multiple $-1\vec{v}$. Similarly, repeatedly adding a
vector $\vec{u}$ to itself $n\in\NN$ times gives us the scalar multiple
$n\vec{u}$.

In other words, scalar multiplication produces results which make sense
in light of vector addition \emph{via} the parallelogram law.

\N{Angles Between Vectors}
We can recall the law of cosines from trigonometry:
\begin{equation}
\|\vec{u}-\vec{v}\|^{2} = \|\vec{u}\|^{2} + \|\vec{v}\|^{2} - 2 \|\vec{u}\|\|\vec{v}\|\cos(\theta)
\end{equation}
where $\theta$ is the angle formed between the edges of the vector. We
refresh our memory of the location of the variables with the sketch:
\begin{center}
  \includegraphics{img/img.8}
\end{center}
Now, we can compute using coordinates
\begin{subequations}
\begin{calculation}
  \|\vec{u}-\vec{v}\|^{2}
\step{by definition of the magnitude of a vector}
  (u_{1}-v_{1})^{2} + (u_{2}-v_{2})^{2}
\step{expanding the terms}
  (u_{1}^{2}-2u_{1}v_{1}+v_{1}^{2}) + (u_{2}^{2}-2u_{2}v_{2}+v_{2}^{2})
\step{associativity of addition}
  (u_{1}^{2} + u_{2}^{2}) + (v_{1}^{2} + v_{2}^{2}) -2u_{1}v_{1}-2u_{2}v_{2}
\step{distributivity}
  (u_{1}^{2} + u_{2}^{2}) + (v_{1}^{2} + v_{2}^{2}) -2(u_{1}v_{1}+u_{2}v_{2}).
\end{calculation}
\end{subequations}
We see this is just
$\|\vec{u}\|^{2}+\|\vec{v}\|^{2}-2(\mbox{stuff})$. But we know what the
``(stuff)'' is, by the law of cosines,
\begin{equation}
-2(u_{1}v_{1}+u_{2}v_{2}) = - 2 \|\vec{u}\|\|\vec{v}\|\cos(\theta),
\end{equation}
hence \emph{for nonzero vectors} we find,
\begin{equation}
\frac{(u_{1}v_{1}+u_{2}v_{2})}{\|\vec{u}\|\|\vec{v}\|} = \cos(\theta).
\end{equation}
But what's more, we see the numerator of the right-hand side is just the
dot product of vectors $\vec{u}\cdot\vec{v}$. We conclude
\begin{equation}
\frac{\vec{u}\cdot\vec{v}}{\|\vec{u}\|\|\vec{v}\|} = \cos(\theta).
\end{equation}
We see this formula is symmetric if we switched places of $\vec{u}$ and
$\vec{v}$, in the sense we get exactly the same result.

But further, we can now \emph{define} the angle $\theta$ between
\emph{any} two \emph{nonzero}(!!) vectors $\vec{u}$ and $\vec{v}$ by the
relation: 
\begin{equation}
\boxed{\cos(\theta) := \frac{\vec{u}\cdot\vec{v}}{\|\vec{u}\|\|\vec{v}\|}.}
\end{equation}
We stress these must be nonzero vectors (otherwise we divide by zero and
horrible results follow).

\begin{definition}
If two nonzero vectors $\vec{u}$ and $\vec{v}$ are such that
\begin{equation}
\vec{u}\cdot\vec{v}=0,
\end{equation}
then we call them \define{Orthogonal} (usually we say one is orthogonal
to the other).
\end{definition}

\N{Unit Vectors and Directions}
If we wish to make a statement about ``directions'', then we can make it
precise by using \define{Unit Vectors} --- vectors of ``unit magnitude''
(i.e., length one). We follow tradition and use hats to indicate we have
a unit vector. For any nonzero vector $\vec{u}$ of arbitrary length, we
can \define{Normalize} it to produce the unit vector
\begin{equation}
\widehat{\vec{u}} = \frac{\vec{u}}{\|\vec{u}\|}.
\end{equation}
We see then that the angle between vectors may be found using their unit
vectors's dot product, without worrying about dividing by anything:
\begin{equation}
\widehat{\vec{u}}\cdot\widehat{\vec{v}}=\cos(\theta).
\end{equation}

\N{Generalization to $n>2$}
Everything we have discussed so far has been restricted to the real
plane $\RR^{2}$. But we could generalize it to $\RR^{n}$ for $n\in\NN$
arbitrary [but fixed]. Vectors then have $n$ components, but all our
definitions generalize accordingly. Vector addition is given by adding
components, scalar multiplication is given by multiplying each component
by the given scalar, and so on.
\section{Vector Spaces over \texorpdfstring{$\RR$}{R}}

\M
We have seen how $n$-tuples of real numbers form a vector space over the
real numbers. Now we will create an abstraction that captures the
crucial aspects of that family of examples.

\begin{definition}
A \define{Real Vector Space} consists of a set $V$ equipped with two
operations $\odot\colon\RR\times V\to V$ and $\oplus\colon V\times V\to V$
such that
\begin{enumerate}[label=(\arabic*)]
\item Closure of $\oplus$: If $\vec{u}$, $\vec{v}\in V$ are two arbitrary elements of $V$,
  then $\vec{u}\oplus\vec{v}\in V$; in other words, $V$ is closed under
  the $\oplus$ operation.
\item Commutativity of $\oplus$: for any $\vec{u}$, $\vec{v}\in V$,
  we have $\vec{u}\oplus\vec{v}=\vec{v}\oplus\vec{u}$
\item Associativity of $\oplus$: for any $\vec{u}$, $\vec{v}$, $\vec{w}\in V$,
  we have $\vec{u}\oplus(\vec{v}\oplus\vec{w})=(\vec{u}\oplus\vec{v})\oplus\vec{w}$
\item\label{axiom:vector-space:existence-of-zero-vector} Unit of $\oplus$: there exists an element $\vec{0}\in V$ such that
  for every $\vec{u}\in V$, $\vec{0}\oplus\vec{u}=\vec{u}\oplus\vec{0}=\vec{u}$.
\item Existence of negation: for each $\vec{u}\in V$, there exists a
  $-\vec{u}\in V$ such that $\vec{u}\oplus-\vec{u}=-\vec{u}\oplus\vec{u}=\vec{0}$
\item Closure of $\odot$: for any real number $c\in\RR$ and element
  $\vec{u}\in V$, we have $c\odot\vec{v}\in V$.
\item Left distributivity of $\odot$ over $\oplus$:
  for any $c\in\RR$ and $\vec{u}$, $\vec{v}\in V$, we have
  $c\odot(\vec{u}\oplus\vec{v}) = (c\odot\vec{u})\oplus(c\odot\vec{v})$.
\item Right distributivity of $\oplus$ over $\odot$:
  for any $c$, $d\in\RR$ and $\vec{u}\in V$, we have
  $(c+d)\odot\vec{u} = (c\odot\vec{u})\oplus(d\odot\vec{u})$.
\item Unit of $\odot$: for any $\vec{u}\in V$, $1\odot\vec{u}=\vec{u}$.
\end{enumerate}
We call elements of $V$ \define{Vectors}, we call the operators $\oplus$
\define{Vector Addition} and $\odot$ \define{Scalar Multiplication}.
The vector $\vec{0}$ in condition~\ref{axiom:vector-space:existence-of-zero-vector}
is called the \define{Zero Vector} of $V$.
\end{definition}

\begin{remark}
We could replace all instances of $\RR$ by $\CC$, any real number by
complex number, and we would obtain the definition of a \define{Complex Vector Space}.
This suggests the notion of a ``real vector space'' could be generalized
\emph{even further} to a ``vector space over [a suitably nice number system]''.
\end{remark}

\begin{example}
When $V=\RR^{n}$, and $\oplus$ is defined componentwise, and $\odot$ is
defined as multiplying each component by the given real number, then we
see we have formed a real vector space. This is an important example,
even if it seems trivial. Why? Because our definition is a
\emph{generalization} of this; if $\RR^{n}$ were not an example of a
real vector space, then our definition would \emph{fail as a generalization}.
\end{example}

\begin{example}
Let $\Mat(\RR;m,n)$ be the set of all $m\times n$ matrices with real components.
Then it forms a real vector space with $\oplus$ being addition of
matrices, and $\odot$ being multiplying each component by a scalar.
It is not enough to \emph{assert} this forms a vector space: we must
\emph{prove} it satisfies \emph{every condition} in the definition.
\begin{enumerate}[label=(\arabic*)]
\item Closure of $\oplus$: Let $\mat{A}=(a_{i,j})\in\Mat(\RR;m,n)$ and
  $\mat{B}=(b_{i,j})\in\Mat(\RR;m,n)$ be elements of $\Mat(\RR;m,n)$.
  We see $\mat{A}\oplus\mat{B}=(a_{i,j}+b_{i,j})$ which is an $m\times n$ 
  real matrix. Thus $\mat{A}\oplus\mat{B}\in\Mat(\RR;m,n)$, and since
  this was for \emph{arbitrary} $\mat{A}$ and $\mat{B}$, we have
  established closure.
\item Commutativity of $\oplus$: Let $\mat{A}=(a_{i,j})\in\Mat(\RR;m,n)$ and
  $\mat{B}=(b_{i,j})\in\Mat(\RR;m,n)$ be elements of $\Mat(\RR;m,n)$.
  We want to prove $\mat{A}\oplus\mat{B}=\mat{B}\oplus\mat{A}$.
  But we see $(\mat{A}\oplus\mat{B})_{i,j} = a_{i,j}+b_{i,j} = b_{i,j}+a_{i,j}$
  by the commutativity of addition for real numbers, and this is
  precisely
  $b_{i,j}+a_{i,j} = (\mat{B}\oplus\mat{A})_{i,j}$. Since this is true
  for \emph{every} component of $\mat{A}$ and $\mat{B}$, then it follows
  $\mat{A}\oplus\mat{B}=\mat{B}\oplus\mat{A}$. 
  And since this is true for \emph{arbitrary} $\mat{A}$ and
  $\mat{B}$, then it follows that $\oplus$ is commutative. 
\item Associativity of $\oplus$:
Let $\mat{A}=(a_{i,j})\in\Mat(\RR;m,n)$,
  $\mat{B}=(b_{i,j})\in\Mat(\RR;m,n)$, and
  $\mat{C}=(c_{i,j})\in\Mat(\RR;m,n)$ be elements of $\Mat(\RR;m,n)$.
  We see every component of the sum $(\mat{A}\oplus(\mat{B}\oplus\mat{C}))_{i,j} = a_{i,j} + (b_{i,j}+c_{i,j}) = (a_{i,j} + b_{i,j})+c_{i,j}$
  by associativity of addition over the real numbers,
  and this equals $(a_{i,j} + b_{i,j})+c_{i,j} = ((\mat{A}\oplus\mat{B})\oplus\mat{C})_{i,j}$.
  Since this holds for every component of the matrices, it follows that
  $\mat{A}\oplus(\mat{B}\oplus\mat{C}) = (\mat{A}\oplus\mat{B})\oplus\mat{C}$.
  Since this is true for arbitrary matrices $\mat{A}$, $\mat{B}$,
  $\mat{C}$ in $V$, then the condition ``$\oplus$ is assocative'' is satisfied.
\item Unit of $\oplus$: there exists an element $\mat{0}\in\Mat(\RR;m,n)$ such that
  for every $\mat{A}\in\Mat(\RR;m,n)$,
  $\mat{0}\oplus\mat{A}=\mat{A}\oplus\mat{0}=\mat{A}$. It suffices to
  prove $\mat{0}\oplus\mat{A}=\mat{A}$, because commutativity guarantees $\mat{A}\oplus\mat{0}=\mat{0}\oplus\mat{A}$.
  We see $(\mat{O}\oplus\mat{A})_{i,j} = 0 + (\mat{A})_{i,j} = (\mat{A})_{i,j}$.
  Since this is true for every component $i$ and $j$, we conclude
  $\mat{0}\oplus\mat{A}=\mat{A}$. Since this is true for arbitrary
  $\mat{A}\in\Mat(\RR;m,n)$, then it follows that the zero matrix
  $\mat{0}$ is the ``vector'' which satisfies the desired condition.
\item Existence of negation: for every $\mat{A}\in\Mat(\RR;m,n)$,
  there exists a $-\mat{A}\in\Mat(\RR;m,n)$ such that
  $(-\mat{A})_{i,j} = -(\mat{A})_{i,j}$ hence
  $(\mat{A}\oplus-\mat{A})_{i,j} = (\mat{A})_{i,j}+(-(\mat{A})_{i,j}) = 0$
  for each component $i$ and $j$. Thus we conclude
  $\mat{A}\oplus-\mat{A}=\mat{0}$ and by commutativity
  $-\mat{A}\oplus\mat{A}=\mat{0}$.
  Since this is true for arbitrary element $\mat{A}\in\Mat(\RR;m,n)$,
  the condition is satisfied.
\item Closure of $\odot$: for any real number $c\in\RR$ and element
  $\mat{A}\in\Mat(\RR;m,n)$, we have $c\odot\mat{A}\in\Mat(\RR;m,n)$.
  Multiplying the components of a $m\times n$ matrix by a real number
  produces a new matrix with components $(c\odot\mat{A})_{i,j}=c(\mat{A})_{i,j}$
  which is what we wanted to prove.
\item Left distributivity of $\odot$ over $\oplus$:
  for any $c\in\RR$ and
  $\mat{A}$, $\mat{B}\in\Mat(\RR;m,n)$, we have
  $c\odot(\mat{A}\oplus\mat{B}) = (c\odot\mat{A})\oplus(c\odot\mat{B})$.
  We see the components of the left-hand side are
  $(c\odot(\mat{A}\oplus\mat{B}))_{i,j} = c((\mat{A})_{i,j} + (\mat{B})_{i,j}) = c(\mat{A})_{i,j} + c(\mat{B})_{i,j}$
  by distributivity of multiplying real numbers over adding real
  numbers, and we see this is just the components of
  $((c\odot\mat{A})\oplus(c\odot\mat{B}))_{i,j}$. That is to say, every
  component of the matrices satisfy
  $(c\odot(\mat{A}\oplus\mat{B}))_{i,j} = ((c\odot\mat{A})\oplus(c\odot\mat{B}))_{i,j}$,
  hence by the definition of matrix equality
  $c\odot(\mat{A}\oplus\mat{B}) = (c\odot\mat{A})\oplus(c\odot\mat{B})$.
  Thus the condition is satisfied.
\item Right distributivity of $\oplus$ over $\odot$:
  for any $c$, $d\in\RR$ and $\mat{A}\in\Mat(\RR;m,n)$, we have
  $(c+d)\odot\mat{A} = (c\odot\mat{A})\oplus(d\odot\mat{A})$.

  We see every component of the left-hand side satisfies
  $((c+d)\odot\mat{A})_{i,j} = (c+d)(\mat{A})_{i,j} = c(\mat{A})_{i,j} + d(\mat{A})_{i,j}$.
  But this is precisely the component
  $((c\odot\mat{A})\oplus(c\odot\mat{B}))_{i,j}$ of the right hand
  side. Thus
  $((c+d)\odot\mat{A})_{i,j} = ((c\odot\mat{A})\oplus(c\odot\mat{B}))_{i,j}$
  holds for every component, and so by the definition of matrix equality
  we have $(c+d)\odot\mat{A} = (c\odot\mat{A})\oplus(d\odot\mat{A})$.
  Thus the condition is satisfied.
\item Unit of $\odot$: for any $\mat{A}\in\Mat(\RR;m,n)$, we have
  $1\odot\mat{A}=\mat{A}$.

  We find the components of the left-hand side are
  $(1\odot\mat{A})_{i,j} = 1(\mat{A})_{i,j} = (\mat{A})_{i,j}$ equal to
  the corresponding components of the right-hand side, for every
  component. Hence by the definition of matrix equality we have
  $1\odot\mat{A}=\mat{A}$.
  Since this is for arbitrary $\mat{A}$, the condition is satisfied.
\end{enumerate}
We have proven $\Mat(\RR;m,n)$ satisfies the conditions specified in our
definition for a real vector space, and thus find it \emph{is} a real
vector space.
\end{example}

\N{Puzzle}\label{puzzle:vector-spaces:solution-space}
Consider the solutions to the linear equation in $n$ unknowns
\begin{equation}
V = \{(x_{1},\dots,x_{n})\in\RR\mid a_{1}x_{1}+\cdots+a_{n}x_{n}=b\}
\end{equation}
where $a_{1}$, \dots, $a_{n}$, $b\in\RR$ are fixed constants.
Does $V$ form a vector space under the usual vector addition and scalar
multiplication borrowed from $\RR^{n}$?

\phantomsection
\subsection*{Exercises}
\addcontentsline{toc}{subsection}{Exercises}

\begin{exercise}
Let $A$ be any non-empty set (say, the set of letters in the alphabet,
or whatever). Consider the collection of real-valued functions
$V=\{f\colon A\to\RR\}$. Define $\oplus$ and $\odot$ on $V$ by:
\begin{enumerate}
\item for any $f$, $g\in V$, for any $a\in A$, $(f\oplus g)(a) = f(a) + g(a)$,
where the $\oplus$ is defined for functions and $+$ is the addition of
real numbers, and
\item for any $f\in V$ and $c\in\RR$ and $a\in A$, $(c\odot f)(a)=cf(a)$.
\end{enumerate}
Prove or find a counter-example: $V$ is a real vector space.

[Hint: different choices of $A$ will not provide counter-examples, but
  the condition that $A$ is nonempty is non-negotiable.]
\end{exercise}

\begin{exercise}
  Recall a \define{Polynomial} with real coefficients looks like
  \begin{equation}
p(x) = p_{0} + p_{1}x + p_{2}x^{2} + \dots + p_{n}x^{n}
  \end{equation}
  where $n\in\NN$ is called the \define{Degree} of $p$. (We write
  $\deg(p)$ if we wanted to refer to the degree of $p$.) The set of
  \emph{all} polynomials with real coefficients in the same unknown $x$
  (of all degrees) is denoted
  \begin{equation}
\RR[x] = \{p_{0} + p_{1}x + p_{2}x^{2} + \dots + p_{n}x^{n}\mid n\in\NN,
  p_{j}\in\RR, j=1,\dots,j\}.
  \end{equation}
  \textsc{Prove or find a counter-example:} the set of polynomials with real
  coefficients $\RR[x]$ is a real vector space.
  Vector addition and scalar multiplication both are done componentwise.
\end{exercise}


\section{Subspaces}

\M
Example~\ref{puzzle:vector-spaces:solution-space} motivates the
following definition:

\begin{definition}
Let $V$ be a real vector space with vector addition $\oplus$ and scalar
multiplication $\odot$.
Let $U\subset V$ be some subset. We call $U$ a \define{Subspace} of $V$
if
it forms a real vector space using
the vector addition $\oplus$ from $V$ and the scalar multiplication $\odot$ from $V$.
\end{definition}

\begin{remark}
  Several things to note about this definition:
  \begin{enumerate}
  \item The zero vector of $U$ must be the zero vector of $V$.
  \item The binary operators on $U$ must be those of $V$ restricted to
    $U$. We cannot change them.
  \end{enumerate}
\end{remark}

\begin{remark}
This is a clunky definition. No one wants to prove $U$ satisfies 9
conditions when we know $V$ satisfies them. One of our first goals is to
simplify the criteria for determining if $U$ is a subspace of $V$ or not.
\end{remark}

\begin{theorem}\label{thm:subspaces:subset-closed-under-linear-combos-is-a-subspace}
Let $V$ be a real vector space with vector addition $\oplus$ and scalar
multiplication $\odot$. Then the subset $U\subset V$ forms a subspace of
$V$ provided:
\begin{enumerate}
\item for any $\vec{u}$, $\vec{v}\in U$, we have
  $\vec{u}\oplus\vec{v}\in U$; and
\item for any $c\in\RR$ and $\vec{u}\in U$, we have $c\odot\vec{u}\in U$.
\end{enumerate}
\end{theorem}

In other words, if $U$ is closed under the vector addition
$\oplus$ and scalar multiplication $\odot$ from $V$, then $U$ is a
subspace of $V$.

\begin{proof}
Assume $U\subset V$ satisfies the two conditions
\begin{enumerate}[label=(\alph*)]
\item\label{assume:closed-under-addition} for any $\vec{u}$, $\vec{v}\in U$, we have
  $\vec{u}\oplus\vec{v}\in U$; and
\item\label{assume:closed-under-scalar-multiplication} for any $c\in\RR$ and $\vec{u}\in U$, we have $c\odot\vec{u}\in U$.
\end{enumerate}
We want to prove the following:
\begin{enumerate}[label=(\arabic*)]
\item Closure of $\oplus$: If $\vec{u}$, $\vec{v}\in U$ are two arbitrary elements of $U$,
  then $\vec{u}\oplus\vec{v}\in U$; in other words, $U$ is closed under
  the $\oplus$ operation.

  \textsc{Proof:} this is precisely assumption \ref{assume:closed-under-addition}.
\item Commutativity of $\oplus$: for any $\vec{u}$, $\vec{v}\in U$,
  we have $\vec{u}\oplus\vec{v}=\vec{v}\oplus\vec{u}$.

  \textsc{Proof:} any elements of $U$ are elements of $V$, so reconsider
  $\vec{u}$ and $\vec{v}$ as elements of $V$. Then the condition holds
  (since $V$ is a real vector space).
\item Associativity of $\oplus$: for any $\vec{u}$, $\vec{v}$, $\vec{w}\in U$,
  we have $\vec{u}\oplus(\vec{v}\oplus\vec{w})=(\vec{u}\oplus\vec{v})\oplus\vec{w}$.

  \textsc{Proof:} any elements of $U$ are elements of $V$, so reconsider
  $\vec{u}$, $\vec{v}$, and $\vec{w}$ as elements of $V$. Then the
  condition holds (since $V$ is a real vector space).
\item Unit of $\oplus$: there exists an element $\vec{0}\in U$ such that
  for every $\vec{u}\in U$, $\vec{0}\oplus\vec{u}=\vec{u}\oplus\vec{0}=\vec{u}$.

  \textsc{Proof:} since $U$ is closed under scalar multiplication by assumption~\ref{assume:closed-under-scalar-multiplication}, we
  know $0\odot\vec{u}=\vec{0}$ which must be in $U$.
\item Existence of negation: for each $\vec{u}\in U$, there exists a
  $-\vec{u}\in U$ such that $\vec{u}\oplus-\vec{u}=-\vec{u}\oplus\vec{u}=\vec{0}$

  \textsc{Proof:} since $U$ is closed under scalar multiplication by assumption~\ref{assume:closed-under-scalar-multiplication}, we
  identify $-\vec{u} = -1\odot\vec{u}$ which would be in $U$ and
  satisfies the desired properties.
\item Closure of $\odot$: for any real number $c\in\RR$ and element
  $\vec{u}\in U$, we have $c\odot\vec{v}\in U$.

  \textsc{Proof:} This is assumption~\ref{assume:closed-under-scalar-multiplication}.
\item Left distributivity of $\odot$ over $\oplus$:
  for any $c\in\RR$ and $\vec{u}$, $\vec{v}\in U$, we have
  $c\odot(\vec{u}\oplus\vec{v}) = (c\odot\vec{u})\oplus(c\odot\vec{v})$.

  \textsc{Proof:} any elements of $U$ are elements of $V$, so reconsider
  $\vec{u}$ and $\vec{v}$ as elements of $V$. Then the condition holds
  (since $V$ is a real vector space).
\item Right distributivity of $\oplus$ over $\odot$:
  for any $c$, $d\in\RR$ and $\vec{u}\in U$, we have
  $(c+d)\odot\vec{u} = (c\odot\vec{u})\oplus(d\odot\vec{u})$.

  \textsc{Proof:} any elements of $U$ are elements of $V$, so reconsider
  $\vec{u}$ as elements of $V$. Then the condition holds
  (since $V$ is a real vector space).
\item Unit of $\odot$: for any $\vec{u}\in U$, $1\odot\vec{u}=\vec{u}$.

  \textsc{Proof:} any elements of $U$ are elements of $V$, so reconsider
  $\vec{u}$ as elements of $V$. Then the condition holds
  (since $V$ is a real vector space).
\end{enumerate}
Hence $U$ satisfies the conditions to be a real vector space when
equipped with $\oplus$ and $\odot$ from $V$.
\end{proof}

\begin{example}
Let $V$ be any real vector space. We can define the \define{Trivial Subspace}
of $V$ to be the set $0=\{\vec{0}\in V\}$ consisting of just the zero
vector. This is closed under addition $\vec{0}\oplus\vec{0}=\vec{0}$ and
under scalar multiplication $\forall c\in\RR,c\odot\vec{0}=\vec{0}$.
Hence $0$ is a subspace of $V$.
\end{example}

\begin{corollary}[Subspace iff closed under arbitrary linear combinations]\label{cor:subspaces:subspace-iff-closed-under-linear-combos}
Let $V$ be a real vector space and $U\subset V$.
Then $U$ is a subspace of $V$ if and only if for every $u_{1},u_{2}\in U$
and $c_{1},c_{2}\in\RR$ we have $(c_{1}\odot u_{1})\oplus(c_{2}\odot u_{2})\in U$.
\end{corollary}

\begin{proof}
$(\implies)$ Assume $U$ is a subspace of $V$.
Then for any $\vec{u}_{1},\vec{u}_{2}\in U$
and $c_{1},c_{2}\in\RR$ we see $c_{1}\odot\vec{u}_{1}$ and $c_{2}\odot\vec{u}_{2}$
are both in $U$ and therefore their vector sum is in $U$ as well, i.e.,
$(c_{1}\odot\vec{u}_{1})\oplus(c_{2}\odot\vec{u}_{2})\in U$.

$(\impliedby)$ Assume for every $\vec{u}_{1},\vec{u}_{2}\in U$
and $c_{1},c_{2}\in\RR$ we have $(c_{1}\odot \vec{u}_{1})\oplus(c_{2}\odot \vec{u}_{2})\in U$.
Then we see when $c_{1}=1$ and $c_{2}=1$ we have our assumption become:
\begin{enumerate}[label=(\alph*)]
\item for every $\vec{u}_{1},\vec{u}_{2}\in U$ their vector sum
  $\vec{u}_{1}\oplus\vec{u}_{2}\in U$.
\end{enumerate}
When $c_{2}=0$ (and/or $\vec{u}_{2}=\vec{0}$), our assumption becomes:
\begin{enumerate}[resume*]
\item For every $c_{1}\in\RR$ and $\vec{u}_{1}\in U$,
  $c_{1}\odot\vec{u}_{1}\in U$.
\end{enumerate}
These are precisely stating $U$ is closed under $\oplus$ and $\odot$,
which implies $U$ is a subspace of $V$.
\end{proof}

\begin{remark}
The notion of ``arbitrary linear combinations [of elements of a subset
  of a vector space]'' turns out to be the critical key idea here. We
want to give it a name, because we will use it quite a bit.
\end{remark}

\begin{definition}
Let $\vec{v}_{1}$, \dots, $\vec{v}_{k}\in V$ be vectors. We call
$\vec{u}\in V$ a \define{Linear Combination} of $\vec{v}_{1}$, \dots,
$\vec{v}_{k}$ if there exists $c_{1},\dots,c_{k}\in\RR$ such that
\begin{equation}
\vec{u} = c_{1}\vec{v}_{1} + \cdots + c_{k}\vec{v}_{k}.
\end{equation}
\end{definition}

\begin{definition}
Let $V$ be a real vector space, let $S\subset V$ be a (possibly
infinite) set of a vectors. Then the \define{Span} of $S$ is the set of
\emph{finite} linear combinations of elements of $S$:
\begin{equation}
\Span(S) =
\{\sum^{k}_{j=1}\lambda_{j}\vec{v}_{j}\mid\lambda_{j}\in\RR,\vec{v}_{j}\in S,k\in\NN,j=1,\dots,k\}
\end{equation}
If $W\subset V$ is a subspace of $V$ such that $\Span(S) = W$, then we
call $S$ a \define{Spanning Set} of $W$.
\end{definition}

\begin{remark}
Did I mention that only finite linear combinations of elements of $S$
are allowed in the $\Span(S)$? Because that ``finite'' part is critical.
\end{remark}

\begin{proposition}[Spans are subspaces]
Let $V$ be a real vector space, $S\subset V$ a nonempty subset.
Then $\Span(S)$ is a subspace of $V$.
\end{proposition}

\begin{proof}
This follows from Corollary~\ref{cor:subspaces:subspace-iff-closed-under-linear-combos}.
The sum of two linear combinations is another linear combination.
\end{proof}

\N{Puzzle: ``Best'' Spanning Sets?}
If we have a vector space $V$, is there a ``best'' spanning set
$S\subset V$ (such that $\Span(S)=V$)? We may have ``redundancies'', for
example if $s_{1}\in S$ and $s_{2}\in S$, then we don't really need
$s_{1}+s_{2}\in S$. So it seems ``minimal'' is a good measure of
``best-ness''. Is there a ``minimal'' spanning set (in some appropriate
sense)?

\phantomsection
\subsection*{Exercises}
\addcontentsline{toc}{subsection}{Exercises}

\begin{exercise}
Consider the set of polynomials with real coefficients
  of degree less than or equal to $2$,
  $V=\{p\in\RR[x]\mid\deg(p)\leq2\}$ forms a real vector space
  where $(p_{0}+p_{1}x+p_{2}x^{2})\oplus(q_{0}+q_{1}x+q_{2}x^{2})=(p_{0}+q_{0})+(p_{1}+q_{1})x+(p_{2}+q_{2})x^{2}$
  and $c\odot(p_{0}+p_{1}x+p_{2}x^{2})=(cp_{0})+(cp_{1})x+(cp_{2})x^{2}$
  for arbitrary $c\in\RR$,
  $(p_{0}+p_{1}x+p_{2}x^{2})$, $(q_{0}+q_{1}x+q_{2}x^{2})\in V$.
  Is $V$ a subspace of the set of all polynomials with real coefficients $\RR[x]$?
  We know it's a subset, but is it a sub\emph{space}?
\end{exercise}

\begin{exercise}
Let $W$ be a real vector space.

Prove or find a counter-example: If $U$ is a subspace of $V$ and $V$ is
a subspace of $W$, then is $U$ a subspace of $W$?
\end{exercise}

\begin{exercise}
Consider the set $C^{\infty}(\RR)$ of smooth real functions.
Is this a subspace of $C(\RR)$ all continuous functions from the reals
to the real numbers?
Is it a subspace of $\{f\colon\RR\to\RR\}$ all functions (all of them --- continuous,
discontinuous, nowhere continuous --- it doesn't matter, all of them)?
\end{exercise}

\begin{exercise}
Consider the set of Riemann integrable functions on some interval
$[a,b]\subset\RR$. Prove or find a counter-example: this a subspace of all functions $\{f\colon[a,b]\to\RR\}$?

[Hint: can you construct a sequence of integrable functions which
  converges to a nonintegrable function?]
\end{exercise}

\begin{exercise}
Consider the subset $S = \{1,x,1-x^{2}\}\subset\RR[x]$. Compute $\Span(S)$.
\end{exercise}

\subsection{Orthogonal Complements and Direct Sums}

\begin{definition}
Let $U\subset V$ be a subspace. We define its \define{Orthogonal Complement}
of $U$ to be the set (not subspace, but a set) of vectors
\begin{equation}
U^{\perp} = \{\vec{w}\in V\mid \mbox{for every}~\vec{u}\in U, \vec{w}\cdot\vec{u}=0\}.
\end{equation}
\end{definition}

\begin{proposition}
For any $U\subset V$ subspace, its orthogonal complement $U^{\perp}$ is
a subspace of $V$.
\end{proposition}

\begin{proof}
  We will invoke Theorem~\ref{thm:subspaces:subset-closed-under-linear-combos-is-a-subspace}
  and prove $U^{\perp}$ is closed under vector addition and scalar
  multiplication.

  Let $\vec{w}_{1},\vec{w}_{2}\in U^{\perp}$ be arbitrary. Then for any
$\vec{u}\in U$ we have
\begin{calculation}
  \vec{u}\cdot(\vec{w}_{1}+\vec{w}_{2})
  \step{distributivity of scalar multiplication over vector addition}
  \vec{u}\cdot\vec{w}_{1}+\vec{u}\cdot\vec{w}_{2}
  \step{since $\vec{w}_{1},\vec{w}_{2}\in U^{\perp}$, definition of $U^{\perp}$}
  0 + 0 = 0
\end{calculation}
hence $(\vec{w}_{1}+\vec{w}_{2})\in U^{\perp}$. So it is closed under
vector addition.

Let $\vec{w}\in U^{\perp}$ and $c\in\RR$ be arbitrary. For any
$\vec{u}\in U$, we have
\begin{calculation}
  (c\vec{w})\cdot\vec{u}
  \step{associativity}
  c(\vec{w}\cdot\vec{u})
  \step{definition of orthogonal complement}
  c(0) = 0.
\end{calculation}
Hence $c\vec{w}\in U^{\perp}$ and moreover $U^{\perp}$ is closed under
scalar multiplication.

Thus $U^{\perp}$ is a subspace of $V$ by Theorem~\ref{thm:subspaces:subset-closed-under-linear-combos-is-a-subspace}.
\end{proof}

\begin{definition}
Let $U_{1}\subset V$ and $U_{2}\subset V$ be subspaces with a trivial
intersection $U_{1}\cap U_{2}=\{\vec{0}_{V}\}$.
Then we define the \define{Direct Sum} of $U_{1}$ with $U_{2}$ to be the
set
$U_{1}\oplus U_{2}=\{\vec{u}_{1} + \vec{u}_{2}\in V\mid \vec{u}_{1}\in U_{1}, \vec{u}_{2}\in U_{2}\}$.
\end{definition}

\begin{proposition}
If $U_{1}\subset V$ and $U_{2}\subset V$ are subspaces with trivial
intersection $U_{1}\cap U_{2}=\{\vec{0}_{V}\}$, then their direct sum
$U_{1}\oplus U_{2}$ is a subspace of $V$.
\end{proposition}

\begin{proposition}
For any subspace $U\subset V$, we have $V=U^{\perp}\oplus U$.
\end{proposition}

\section{Linear Dependence and Bases}\label{section:basis}

\M
We want to address the question of whether this is a ``best'' spanning
set for a subspace, and we saw in some sense ``redundant elements''
should be avoided. If $\vec{s}_{1}\in S$ and $\vec{s}_{2}\in S$, then it
would be redundant to have $\vec{s}_{1}+\vec{s}_{2}\in S$. Let us try to
formalize this intuition of ``redundant combinations''.

\begin{definition}\label{defn:basis:linearly-dependent}
Let $V$ be a vector space, let $\vec{v}_{1}$, \dots, $v_{n}\in V$ be
nonzero vectors $\vec{v}_{j}\neq\vec{0}$ for $j=1,\dots,n$.
We call them \define{Linearly Dependent} if there are coefficients (not
all zero) $c_{1},\dots,c_{n}\in\RR$ such that
\begin{equation}
c_{1}\vec{v}_{1} + c_{2}\vec{v}_{2}+\cdots+c_{n}\vec{v}_{n}=\vec{0}.
\end{equation}
If the only solution for this is $c_{1}=c_{2}=\cdots=c_{n}=0$ for all
coefficients to be zero, then we call the vectors \define{Linearly Independent}.
\end{definition}

\begin{example}
  In $\RR^{2}$, consider the vectors
  \begin{equation}
\vec{v}_{1} = \begin{pmatrix} 1\\0 \end{pmatrix},
\vec{v}_{2} = \begin{pmatrix} 0\\1 \end{pmatrix},
\vec{v}_{3} = \begin{pmatrix} 1\\1 \end{pmatrix},
\vec{v}_{4} = \begin{pmatrix} 1\\-1 \end{pmatrix}.
  \end{equation}
  Any three or more vectors from this list are linearly dependent since
  $\vec{v}_{3}=\vec{v}_{1}+\vec{v}_{2}$ and
  $\vec{v}_{4}=\vec{v}_{1}-\vec{v}_{2}$. But
  any two vectors from this list are linearly independent.
\end{example}

\begin{theorem}[Criterion for Linear Dependence]
A set of nonzero vectors $\{\vec{v}_{1},\dots,\vec{v}_{n}\}$ is linearly
dependent if and only if at least one of the vectors $\vec{v}_{k}$ is
expressible as a linear combination of the others
\begin{equation}
\vec{v}_{k} = \sum^{n}_{\substack{j=1\\j\neq k}}c_{j}\vec{v}_{j} =c_{1}\vec{v}_{1} + \cdots + c_{k-1}\vec{v}_{k-1} + c_{k+1}\vec{v}_{k+1} +
  \cdots + c_{n}\vec{v}_{n},
\end{equation}
where not all coefficients $c_{j}\in\RR$ are zero.
\end{theorem}

\begin{proof}
  $(\implies)$ Assume the vectors $\vec{v}_{1}$, \dots, $\vec{v}_{n}$
  are linearly dependent. Then by Definition~\ref{defn:basis:linearly-dependent},
  there are coefficients $c_{1}$, \dots, $c_{n}$ (not all zero) such that
\begin{equation}
c_{1}\vec{v}_{1} + c_{2}\vec{v}_{2}+\cdots+c_{n}\vec{v}_{n}=\vec{0}.
\end{equation}
Let $k$ be the last index for which $c_{k}\neq0$ (so for indices $\ell$
such that  $k<\ell\leq n$, then $c_{\ell}=0$). Then we can subtract
$c_{k}\vec{v}_{k}$ from both sides to get
\begin{equation}
c_{1}\vec{v}_{1} + c_{2}\vec{v}_{2}+\cdots+c_{k}\vec{v}_{k}-c_{k}\vec{v}_{k}=\vec{0}-c_{k}\vec{v}_{k},
\end{equation}
and dividing both sides by $-c_{k}$ gives us $\vec{v}_{k}$ as a linear
combination of $\vec{v}_{1}$, \dots, $\vec{v}_{k-1}$. This concludes the
forward direction of the proof.

  $(\impliedby)$ Assume there exists a vector $\vec{v}_{k}$ such that we
  can write it as a linear combination of the remaining vectors
\begin{equation}
\begin{split}
  \vec{v}_{k} &= \sum^{n}_{\substack{j=1\\j\neq k}}c_{j}\vec{v}_{j}\\
  &=c_{1}\vec{v}_{1} + \cdots + c_{k-1}\vec{v}_{k-1} + c_{k+1}\vec{v}_{k+1} +
  \cdots + c_{n}\vec{v}_{n},
\end{split}
\end{equation}
where not all $c_{j}\in\RR$ are zero. Then subtracting $\vec{v}_{k}$
from both sides gives us
\begin{equation}
\vec{0} = c_{1}\vec{v}_{1} + \cdots + c_{k-1}\vec{v}_{k-1} - \vec{v}_{k} + c_{k+1}\vec{v}_{k+1} +
  \cdots + c_{n}\vec{v}_{n}.
\end{equation}
Then by Definition~\ref{defn:basis:linearly-dependent}, since not all
coefficients $c_{j}$ are zero, we have the vectors are linearly dependent.
\end{proof}

\begin{theorem}[Nonzero determinant iff columns are linearly independent]
Let $\{\vec{v}_{1},\dots,\vec{v}_{n}\}\subset\RR^{n}$ be a list of $n$
distinct $n$-vectors, and
\begin{equation}
\mat{M} = (\vec{v}_{1}|\dots|\vec{v}_{n})
\end{equation}
be a matrix whose columns are the given $n$ column vectors.
Then $\det(\mat{M})\neq0$ if and only if $\{\vec{v}_{1},\dots,\vec{v}_{n}\}$
are linearly independent.
\end{theorem}

\begin{proof}
$(\implies)$ Assume $\det(\mat{M})\neq0$. Then $\mat{M}$ is invertible
  (by Theorem~\ref{thm:determinant:singular-matrices-have-zero-det}).
Then $\mat{M}\vec{x}=\vec{0}$ has a unique solution (\S\ref{par:matrix-algebra:solving-systems-of-equations}), namely $\vec{x}=\vec{0}$.
This is equivalent to saying
\begin{equation}
x_{1}\vec{v}_{1} + x_{2}\vec{v}_{2} + \cdots + x_{n}\vec{v}_{n} = 0
\end{equation}
implies $x_{1}=x_{2}=\cdots=x_{n}=0$. But by Definition~\ref{defn:basis:linearly-dependent},
this is precisely the condition for $\vec{v}_{1}$, \dots, $\vec{v}_{n}$
being linearly independent.

$(\impliedby)$ Assume $\vec{v}_{1}$, \dots, $\vec{v}_{n}$ are linearly
independent. Then by Definition~\ref{defn:basis:linearly-dependent}, the
only solution to
\begin{equation}
c_{1}\vec{v}_{1} + \cdots + c_{n}\vec{v}_{n} = \vec{0}
\end{equation}
is $c_{1}=\cdots=c_{n}=0$. In matrix form, if $\vec{x}=(c_{1},\dots,c_{n})$
is a column $n$-vector, then
\begin{equation}
\mat{M}\vec{x} = \vec{0}
\end{equation}
has $\vec{x}=\vec{0}$ be its only solution. This is true if and only if
$\mat{M}$ is invertible. But $\mat{M}$ is invertible if and only if
$\det(\mat{M})\neq0$. And by our assumption, $\vec{x}=\vec{0}$ is the
only solution, hence the result.
\end{proof}

\begin{corollary}\label{cor:basis:invertible-matrix-iff-columns-are-linearly-independent}
An $n\times n$ matrix $\mat{M}$ is invertible if and only if its columns are
linearly independent vectors.
\end{corollary}
\begin{proof}
We know from the previous theorem $\mat{M}$ has nonzero determinant if
and only if its columns are linearly independent vectors. We know from Theorem~\ref{thm:determinant:singular-matrices-have-zero-det}
$\mat{M}$ has a nonzero determinant if and only if $\mat{M}$ is
invertible.
Therefore, we know $\mat{M}$ is invertible if and only if its columns
are linearly independent vectors.
\end{proof}

\begin{definition}
Let $V$ be a real vector space and $B$ a set of vectors from $V$ such
that
\begin{enumerate}
\item it spans $V$: $\Span(B)=V$
\item there is no $A\subset B$ such that $\Span(A)=V$.
\end{enumerate}
Then we call $B$ a \define{Basis} of $V$.
\end{definition}

\begin{remark}[Need to prove existence of basis]
We have just defined a word, ``basis'', but we have no guarantee that a
basis will exist. This must be proven. The proof is not enlightening,
and requires the axiom of choice (pick some nonzero vector, now pick
another which is linearly independent of the first, keep picking
linearly independent vectors --- how? By the axiom of choice, it's
always possible \emph{somehow}; then show an arbitrary vector may be
written as a linear combination of our collection of chosen vectors).
\end{remark}

\begin{example}
  In $\RR^{2}$, the vectors
  \begin{equation}
\vec{z} = \begin{pmatrix}1\\ 1
\end{pmatrix},\quad\mbox{and}\quad\bar{\vec{z}} = \begin{pmatrix}1\\ -1
\end{pmatrix}.
  \end{equation}
  Then $\{\vec{z}, \bar{\vec{z}}\}$ form a basis for $\RR^{2}$.
\end{example}
\begin{proof}
  We need to show
  \begin{enumerate}
  \item $\Span\{\vec{z}, \bar{\vec{z}}\}=\RR^{2}$
  \item there is no $A\subset\{\vec{z}, \bar{\vec{z}}\}$ such that $\Span(A)=\RR^{2}$.
  \end{enumerate}
  The first claim may be proven by picking any element
  $\vec{v}\in\RR^{2}$, then showing it may be written as a linear
  combination of $\vec{z}$ and $\bar{\vec{z}}$. We see, if
  \begin{equation}
\vec{v} = \begin{pmatrix}v_{1}\\v_{2}
\end{pmatrix},
  \end{equation}
  then
  \begin{calculation}
    \displaystyle\frac{v_{1}+v_{2}}{2}\vec{z} + \frac{v_{1}-v_{2}}{2}\bar{\vec{z}}
\step{unfolding the definition of $\vec{z}$, $\bar{\vec{z}}$}
    \displaystyle\frac{v_{1}+v_{2}}{2}\begin{pmatrix}1\\1
    \end{pmatrix}
    + \frac{v_{1}-v_{2}}{2}\begin{pmatrix}1\\-1
    \end{pmatrix}
\step{scalar multiplication}
    \displaystyle\frac{1}{2}\begin{pmatrix}v_{1}+v_{2}\\v_{1}+v_{2}
    \end{pmatrix}
    + \frac{1}{2}\begin{pmatrix}v_{1}-v_{2}\\-v_{1}+v_{2}
    \end{pmatrix}
\step{distributivity}
    \displaystyle\frac{1}{2}\left[\begin{pmatrix}v_{1}+v_{2}\\v_{1}+v_{2}
    \end{pmatrix}
    + \begin{pmatrix}v_{1}-v_{2}\\-v_{1}+v_{2}
    \end{pmatrix}\right]
\step{vector addition}
    \displaystyle\frac{1}{2}\begin{pmatrix}(v_{1}+v_{2})+(v_{1}-v_{2})\\(v_{1}+v_{2})+(-v_{1}+v_{2})
    \end{pmatrix}
\step{arithmetic}
    \displaystyle\frac{1}{2}\begin{pmatrix}2v_{1}\\2v_{2}
      \end{pmatrix}
\step{scalar multiplication}
    \displaystyle\begin{pmatrix}v_{1}\\v_{2}
      \end{pmatrix} = \vec{v}
  \end{calculation}
  as desired. Hence any element of $\RR^{2}$ may be written as a linear
  combination of $\vec{z}$ and $\bar{\vec{z}}$, hence $\Span(\{\vec{z},\bar{\vec{z}}\})=\RR^{2}$.

  As to the second claim, there is no $A\subset\{\vec{z},\bar{\vec{z}}\}$,
  suppose there were such an $A$. Then either $A=\{\vec{z}\}$ or
  $A=\{\bar{\vec{z}}\}$. Pick $\vec{v}\in\{\vec{z},\bar{\vec{z}}\}$ but
  $\vec{v}\notin A$. Then we claim $\vec{v}\notin\Span(A)$.

  It suffices to show $\vec{z}$ is not a multiple of $\bar{\vec{z}}$
  (which corresponds to $A=\{\bar{\vec{z}}\}$ --- in the other case, it
  boils down to the same proof). If $\vec{z}$ were a multiple of
  $\bar{\vec{z}}$, then there is a $c\in\RR$ nonzero such that
  \begin{equation}
c\begin{pmatrix}1\\1
\end{pmatrix} = \begin{pmatrix}1\\-1
\end{pmatrix}
  \end{equation}
  This is a system of 2 equations in 1 unknown:
  \begin{equation}
c=1,\quad\mbox{and}\quad c=-1.
  \end{equation}
  But this is impossible. So $\vec{z}$ cannot be a multiple of
  $\bar{\vec{z}}$, which means $\vec{z}\notin\Span(\{\bar{\vec{z}}\})$.
  The same reasoning shows $\bar{\vec{z}}$ is not a multiple of
  $\vec{z}$, which means $\bar{\vec{z}}\notin\Span(\{\vec{z}\})$.

  Hence there is no $A\subset\{\vec{z},\bar{\vec{z}}\}$ such that $\Span(A)=\RR^{2}$.
\end{proof}

\begin{example}\label{ex:basis:canonical-basis}
Let $\vec{e}_{j}\in\RR^{n}$ have $1$ in its $j^{\text{th}}$ component
and $0$ in all other components. Then the set
$\{\vec{e}_{1},\dots,\vec{e}_{n}\}$ forms a basis of $\RR^{n}$ and is
called its \define{Canonical Basis}.
\end{example}

\N{Vector spaces have many bases}
We see that a vector space may have more than one basis. In fact, they
will have many different possible bases (plural of basis). We saw one
basis in $\RR^{2}$ given by $\vec{z}=(1,1)$ and
$\bar{\vec{z}}=(1,-1)$. We also see there is the canonical basis for
$\RR^{2}$, which is different from the first basis.

The moral of the story is that we may have many inequivalent bases for
any given vector space.

\begin{lemma}
Let $V$ be a real vector space.
Let $B=\{\vec{v}_{1},\dots,\vec{v}_{n}\}$ form a basis for $V$.
Let $T=\{\vec{w}_{1},\dots,\vec{w}_{m}\}$ be a set of linearly
independent vectors from $V$.
Then $m\leq n$ (i.e., $|T|\leq|B|$).
\end{lemma}

\begin{proof}
Since $T$ consists of linearly independent vectors, we can write
$\vec{w}_{m}$ as a linear combination of basis vectors
\begin{equation}
\vec{w}_{m} = c^{(m)}_{1}\vec{b}_{1}+c^{(m)}_{2}\vec{b}_{2}+\cdots+c^{(m)}_{n}\vec{b}_{n}.
\end{equation}
We can reindex the basis vectors such that $c^{(m)}_{1}\neq0$. In that
case, we ``swap out'' $\vec{b}_{1}$ for $\vec{w}_{m}$ since we can write
$\vec{b}_{1}$ as a linear combination:
\begin{equation}
\vec{b}_{1}=\frac{1}{c^{(m)}_{1}}\vec{w}_{m} - \frac{c^{(m)}_{2}\vec{b}_{2}+\cdots+c^{(m)}_{n}\vec{b}_{n}}{c^{(m)}_{1}}.
\end{equation}
This gives us a new set of basis vectors $B_{1}$, and we consider $T_{1}=T\setminus\{\vec{w}_{m}\}$
the collection of elements from $T$ which are not $\vec{w}_{m}$. In
particular, there are $m-1$ elements of $T_{1}$.

We can reiterate this step, swapping one element out of $T_{1}$ and
putting it into $B_{1}$ (and throwing away an element from $B_{1}$ which
has been replaced) to produce a new basis $B_{2}$. We produce $T_{2}$
from $T_{1}$ by taking the remaining elements of $T_{1}$ which are not
in $B_{2}$ into $T_{2}$. We see $T_{2}$ has $m-2$ elemeents.

Eventually one of two possibilities occurs:
\begin{enumerate}
\item We'll reach $B_{k}$ which no longer has any original basis
  elements from $B$ in it --- they are disjoint $B\cap B_{k}=\emptyset$.
  But this would imply there are elements in $T_{k}$ which cannot be
  written as a linear combination of the basis, which is a
  contradiction; or
\item We'll exhaust $T_{k}$ and have no more elements from $T$ to add to $B_{k}$.
\end{enumerate}
We iterate this until we get to $B_{m}$ and $T_{m}$, because $T_{m+1}$
will be empty.
\end{proof}

\begin{theorem}[Any two bases have same number of elements]
Let $V$ be a real vector space. Suppose there exists at least one basis
$B$ for $V$, and suppose $B$ has finitely many element.
Then any two bases for $V$ have the same number of elements as each other.
\end{theorem}

\begin{proof}
Let $B_{1}$, $B_{2}$ be any two bases for $V$. Let $m=|B_{1}|$ and $n=|B_{2}|$.
We claim
\begin{enumerate}
\item $m\leq n$ by the previous lemma, and
\item $n\leq m$ by the previous lemma.
\end{enumerate}
Hence $m=n$.
\end{proof}

\begin{definition}
Let $V$ be a real vector space, let $B$ be a basis for $V$. If $B$ has
finitely many elements, then we say $V$ is
\define{Finite-Dimensional}. In that case, we call the number of vectors
in $B$ the \define{Dimension} of $V$.
\end{definition}

\begin{remark}
Reasoning about infinite-dimensional spaces can be tricky. We have
already seen one example, $\RR[x]$ the space of polynomials. The linear
algebra of inifnite-dimensional spaces usually goes by the name
``functional analysis''. A particularly friendly subfield is ``Fourier
analysis'', where many intuitions from finite-dimensional linear algebra
carries over.
\end{remark}

\begin{definition}
Let $V$ be a finite-dimensional vector space, let
$B=\{\vec{f}_{1},\dots,\vec{f}_{n}\}$ be a basis for $V$.
We define an \define{Ordered Basis} to be a tuple $(\vec{f}_{1},\dots,\vec{f}_{n})$.

Futhermore, if the vectors $\vec{f}_{1}$, \dots, $\vec{f}_{n}$ form a
basis such that
\begin{equation}
\vec{f}_{i}\cdot\vec{f}_{j}=0\quad\mbox{if }i\neq j,
\end{equation}
then we call it an \define{Orthogonal Basis} for $V$. If, even further,
we have
\begin{equation}
\vec{f}_{i}\cdot\vec{f}_{j}=\delta_{i,j}=\begin{cases}0&\mbox{if }i\neq j\\
1 & \mbox{if }i=j
\end{cases}
\end{equation}
then we call it an \define{Orthonormal Basis}.
\end{definition}

\begin{remark}
This might seem silly (and, I guess, it is), but sets are not
ordered. There are times when we will want to specifically note the
order of basis vectors. The ordering may be arbitrary (for example, an
accidental artifact induced by the indexing), but important.
\end{remark}

\begin{remark}
In differential geometry, we sometimes see the term ``frame'' used for
an ordered basis.
\end{remark}

\subsection{Coordinates Relative to a Basis}

\M
Recall, when we began discussing vector spaces like $\RR^{1}$ and
$\RR^{2}$ (back in section~\ref{section:vectors-in-r-n}), we began by
drawing a line, picking a point $O$ (calling it the origin), and then
picking another point $P$. We identified the oriented line segment
$\overrightarrow{OP}$ as a unit vector. Then any other point $Q$ could
be identified as a multiple of $\overrightarrow{OP}$ such that
$\|\overrightarrow{OQ}\|/\|\overrightarrow{OP}\|=|x|\in\RR$ and if $Q$
is not ``in the $P$ direction'', we said
$-|x|\overrightarrow{OP}=\overrightarrow{OQ}$. In this way, we
identified $x$ as the coordinate of $Q$.

\M In $\RR^{2}$, we did the same thing, but we now have two axes. Any
point $S$ on the plane could be identified by a pair of real numbers
$(x,y)\in\RR^{2}$ by similar means.

\M
In general, this is what happens with a vector in a vector space
relative to a basis. We obtain ``coordinates'' for the vector, enabling
us to write our vector as a linear combination of basis vectors. This is
a crucial point, because an $n$-dimensional real vector space $V$ \emph{is not}
\emph{identical} to $\RR^{n}$ --- but by choosing a basis, we can
identify vectors in $V$ with tuples of real numbers in $\RR^{n}$, namely
their coordinates. This should be familiar, we've been working with
$n\times1$ matrices and calling them vectors (but really, they're just
inhabitants of $\RR^{n}$). We have been stretching the truth all this
time.

\begin{ddanger}
This is a really critical point to appreciate: vectors are ``points in
the plane'' \emph{not} an $n\times 1$ matrix. \textbf{Vectors are not matrices}.
But we can turn a vector in a finite-dimensional vector space
\emph{into} a column $n\times1$ matrix, and vice-versa, \emph{given some basis}
for $V$.
\end{ddanger}

Let us try to make things concrete.

\begin{definition}\label{defn:basis:coordinates}
Let $V$ be a finite-dimensional vector space, let
$B=(\vec{f}_{1},\dots,\vec{f}_{n})$ be an ordered basis for $V$, and let
$\vec{v}\in V$ be an arbitrary vector. Then we call the coefficients
$\lambda_{1},\dots,\lambda_{n}\in\RR$ in
\begin{equation}
\vec{v} = \lambda_{1}\vec{f}_{1} + \cdots + \lambda_{n}\vec{f}_{n}
\end{equation}
the \define{Coordinates} of $\vec{v}$ over $B$ (or ``relative to $B$'').
We may write
\begin{equation}
[\vec{v}]_{B} = \begin{pmatrix}\lambda_{1}\\\vdots\\\lambda_{n}
\end{pmatrix}
\end{equation}
for the coordinates of $\vec{v}$ relative to basis $B$.
\end{definition}

\begin{remark}
If we can write $\vec{v} = \lambda_{1}\vec{f}_{1} + \lambda_{2}\vec{f}_{2} + \cdots + \lambda_{n}\vec{f}_{n}$,
then we can identify $\vec{v}$ with the column vector of its coordinates
relative to the $\vec{f}_{j}$:
\begin{equation}
  \vec{v} \mathrel{\mbox{``=''}}
  \begin{pmatrix}\lambda_{1}\\\lambda_{2}\\\vdots\\\lambda_{n}
  \end{pmatrix}.
\end{equation}
The equality is in quotation marks because it's not an equality but an \emph{isomorphism},
a slightly weaker notion of ``the same''.
We can prove (once we formalize the notion of an isomorphism) that any
finite-dimensional real vector space $V$ is isomorphic to $\RR^{n}$ the
collection of column $n$-vectors with real entries; this is done by
identifying a vector with its coordinates relative to some basis. 
But this \emph{does not} mean $V$ is equal to $\RR^{n}$.
\end{remark}

\begin{problem}
Suppose we have a finite-dimensional vector space $V$.
Suppose we have one ordered basis $B=(\vec{e}_{1},\dots,\vec{e}_{n})$
and a distinct ordered basis $B'=(\vec{f}_{1},\dots,\vec{f}_{n})$
which share no vectors --- we have $\vec{f}_{i}\neq\vec{e}_{j}$
for all $i$, $j=1,\dots,n$.

If we have a vector $\vec{v}\in V$ and have found its coordinates
relative to $\vec{f}_{j}$,
\begin{equation}
\vec{v} = \lambda_{1}\vec{f}_{1} + \cdots + \lambda_{n}\vec{f}_{n},
\end{equation}
then how do we transform these into coordinates relative to $\vec{e}_{i}$?
\end{problem}

\subsection{Changing Bases}

\N{Change of Basis Matrix}
Let $V$ be a finite-dimensional real vector space.
Let $B=(\vec{e}_{1},\dots,\vec{e}_{n})$ and
$C=(\vec{f}_{1},\dots,\vec{f}_{n})$ be two ordered bases for $V$.
Then the change of bases from $B$ to $C$ transforms coordinates for
vectors $[\vec{v}]_{B}$ according to the matrix
\begin{equation}
  \mat{M}^{B}_{C} = \left(\begin{array}{c|c|c}
      [\vec{e}_{1}]_{C} & \dots & [\vec{e}_{n}]_{C}
  \end{array}\right),
\end{equation}
so any vector $\vec{v}\in V$ expressed in coordinates $[\vec{v}]_{B}$
relative to $B$ may be expressed in coordinates relative to $C$ as
\begin{equation}
[\vec{v}]_{C} = \mat{M}^{B}_{C}[\vec{v}]_{B}.
\end{equation}
Why would this work? Well, if we expand the right-hand side, we would
find
\begin{equation}
 \left(\begin{array}{c|c|c}
      [\vec{e}_{1}]_{C} & \dots & [\vec{e}_{n}]_{C}\\
  \end{array}\right)\begin{pmatrix}\lambda_{1}\\\vdots\\\lambda_{n}
 \end{pmatrix} =
      \lambda_{1}[\vec{e}_{1}]_{C} + \lambda_{2}[\vec{e}_{2}]_{C} +
      \dots + \lambda_{n} [\vec{e}_{n}]_{C}.
\end{equation}
But since $[\vec{e}_{j}]_{C}$ is a linear combination of the basis
vectors $\vec{f}_{1}$, \dots, $\vec{f}_{n}$, this expands out to form a
linear combination of $\vec{f}_{1}$, \dots, $\vec{f}_{n}$.

\begin{theorem}
If $V$ is a finite dimensional vector space with ordered bases $B$ and
$C$, if $[\mat{M}]^{B}_{C}$ is the change-of-coordinate matrix from $B$
to $C$, then the change-of-coordinate matrix from $C$ back to $B$
(i.e., $[\mat{M}]^{C}_{B}$) satisfies
\begin{equation}
[\mat{M}]^{C}_{B} = ([\mat{M}]^{B}_{C})^{-1}.
\end{equation}
That is to say, they are inverses of each other.
\end{theorem}

This partly motivated the bizarre superscript/subscript convention, it
resembles a fraction.

We can ketch the proof out in (\S\ref{chunk:basis:change-of-basis-matrix-among-orthonormal-bases}).

\N{Relating Coordinates Between Orthonormal Bases}
If $B$ and $C$ are both orthonormal bases for $V$, there is no reason to
believe they consist of the same basis vectors. One way to obtain a
different set of orthonormal basis vectors from $B$ is by an arbitrary
rotation, and possibly reflection about an axis (or about a nonzero
vector). These transformations produce a different basis, but do not
affect the orthonormality of the new basis.

What does the change of coordinates matrix look like between them?

\M\label{chunk:basis:change-of-basis-matrix-among-orthonormal-bases}
We can write out the matrix components explicitly for orthonormal
coordinates $\vec{e}_{1}$, \dots, $\vec{e}_{n}$ and other orthonormal
coordinates $\vec{f}_{1}$, \dots, $\vec{f}_{n}$ as
\begin{align}
  \vec{f}_{1} &= (\vec{f}_{1}\cdot\vec{e}_{1})\vec{e}_{1} + (\vec{f}_{1}\cdot\vec{e}_{2})\vec{e}_{2} + \cdots + (\vec{f}_{1}\cdot\vec{e}_{n})\vec{e}_{n}\\
  \vec{f}_{2} &= (\vec{f}_{2}\cdot\vec{e}_{1})\vec{e}_{1} + (\vec{f}_{2}\cdot\vec{e}_{2})\vec{e}_{2} + \cdots + (\vec{f}_{2}\cdot\vec{e}_{n})\vec{e}_{n}\\
  \vdots &\mathrel{\phantom{=(\vec{f}_{2}\cdot\vec{e}_{1})}}\vdots\quad\phantom{+ (\vec{f}_{1}\cdot\vec{e}_{2})}\vdots\qquad\ddots\qquad\vdots\nonumber\\
  \vec{f}_{n} &= (\vec{f}_{n}\cdot\vec{e}_{1})\vec{e}_{1} + (\vec{f}_{n}\cdot\vec{e}_{2})\vec{e}_{2} + \cdots + (\vec{f}_{n}\cdot\vec{e}_{n})\vec{e}_{n}
\end{align}
The components of the matrix
$[\mat{M}]^{B}_{C}=(\vec{f}_{i}\cdot\vec{e}_{j})$.
What is the inverse of this matrix?

We could do some complicated math, or we could wonder the simpler
problem: what is $[\mat{M}]^{B}_{C}\transpose{([\mat{M}]^{B}_{C})}$?
\begin{calculation}
  ([\mat{M}]^{B}_{C}\transpose{([\mat{M}]^{B}_{C})})_{i,k}
\step{unfold the definition of matrix multiplication}
  \sum^{n}_{j=1}([\mat{M}]^{B}_{C})_{i,j}(\transpose{([\mat{M}]^{B}_{C})})_{j,k}
\step{unfold the definition of $[\mat{M}]^{B}_{C}$ into components}
  \sum^{n}_{j=1}(\transpose{\vec{f}}_{i}\vec{e}_{j})(\transpose{\vec{e}}_{j}\vec{f}_{k})
\step{distributivity}
  \transpose{\vec{f}}_{i}\left(\sum^{n}_{j=1}\vec{e}_{j}\transpose{\vec{e}}_{j}\right)\vec{f}_{k}
\step{matrix multiplication, see Lemma~\ref{lemma:basis:outer-product-of-orthonormal-basis} below}
  \transpose{\vec{f}}_{i}\left(\mat{I}_{n}\right)\vec{f}_{k}
\step{defining property of the identity matrix, associativity of multiplication}
  \transpose{\vec{f}}_{i}\vec{f}_{k}
\step{by definition of orthonormality}
  \delta_{i,k}
\end{calculation}
In other words,
\begin{equation}
[\mat{M}]^{B}_{C}\transpose{([\mat{M}]^{B}_{C})} = \mat{I}.
\end{equation}
This implies
\begin{equation}
([\mat{M}]^{B}_{C})^{-1} = \transpose{([\mat{M}]^{B}_{C})}.
\end{equation}
In other words, the change of basis matrix is an orthogonal matrix
(c.f., Exercise~\ref{xca:matrix-algebra:orthogonal-matrix}).

\begin{lemma}\label{lemma:basis:outer-product-of-orthonormal-basis}
Let $V$ be a finite-dimensional vector space, let $\vec{e}_{1}$, \dots,
$\vec{e}_{n}$ be an orthonormal basis for $V$. Then
\begin{equation}
\sum^{n}_{j=1}\vec{e}_{j}\transpose{\vec{e}}_{j} = \mat{I}_{n}.
\end{equation}
\end{lemma}
\begin{proof}
We consider how this acts on an arbitrary vector $\vec{v}\in V$.
\begin{calculation}
  \left(\sum^{n}_{j=1}\vec{e}_{j}\transpose{\vec{e}}_{j}\right)\vec{v}
\step{expanding $\vec{v}$ in the basis}
  \left(\sum^{n}_{j=1}\vec{e}_{j}\transpose{\vec{e}}_{j}\right)\left(\sum^{n}_{k=1}c_{k}\vec{e}_{k}\right)
\step{by linearity}
  \sum^{n}_{k=1}c_{k}\left(\sum^{n}_{j=1}\vec{e}_{j}\transpose{\vec{e}}_{j}\vec{e}_{k}\right)
\step{by definition of orthonormality}
  \sum^{n}_{k=1}c_{k}\left(\sum^{n}_{j=1}\vec{e}_{j}\delta_{j,k}\right)
\step{unrolling the inner sum over $j$}
  \sum^{n}_{k=1}c_{k}\left(\vec{e}_{1}\delta_{1,k}+\dots+\vec{e}_{k-1}\delta_{k-1,k}+\vec{e}_{k}\delta_{k,k}+\vec{e}_{k+1}\delta_{k+1,k}+\dots+\vec{e}_{n}\delta_{n,k}\right)
\step{definition of $\delta_{j,k}$}
  \sum^{n}_{k=1}c_{k}\left(\vec{e}_{1}0+\dots+\vec{e}_{k-1}0+\vec{e}_{k}1+\vec{e}_{k+1}0+\dots+\vec{e}_{n}0\right)
\step{arithmetic}
  \sum^{n}_{k=1}c_{k}\left(\vec{e}_{k}\right)
\step{multiplication}
  \sum^{n}_{k=1}c_{k}\vec{e}_{k}
\step{since this is the expansion of $\vec{v}$ in the basis $\vec{e}_{k}$, ``undoing'' the first step of this chain of calculations}
  \vec{v}.
\end{calculation}
Since this was for arbitrary $\vec{v}\in V$, it follows that
\begin{equation}
\sum^{n}_{j=1}\vec{e}_{j}\transpose{\vec{e}}_{j} = \mat{I}_{n},
\end{equation}
as desired.
\end{proof}

\subsection{Graham--Schmidt Method}

\N{Puzzle}
If we have an $n$-dimensional vector space $V$ with $n$ linearly
independent [nonzero] vectors $\vec{x}_{1}$, \dots, $\vec{x}_{n}$, then is there
any way to construct an orthonormal basis out of them?

\N{Solution}
We will construct an orthonormal basis, one vector at a time.

The first step is to construct our initial vector
\begin{equation}
\widehat{\vec{v}_{1}} = \frac{\vec{x}_{1}}{\|\vec{x}_{1}\|}.
\end{equation}
This is a unit vector, and now we will use it to start our collection.
We could have easily have chosen $\vec{v}_{1}=\vec{x}_{1}$, as well, it
would just make things a little longer.

We now find the second vector $\vec{v}_{2}$. Since $\vec{x}_{2}$ and
$\vec{x}_{1}$ are linearly independent, it follows that $\vec{x}_{2}$
and $\vec{v}_{1}$ are linearly independent. Then we hope to find
coefficients $c_{1}$ and $c_{2}$ such that
\begin{equation}
\vec{v}_{2} = c_{1}\widehat{\vec{v}_{1}} + c_{2}\vec{x}_{2}
\end{equation}
is a unit vector orthogonal to $\vec{v}_{1}$. So
\begin{equation}
\widehat{\vec{v}_{1}}\cdot\vec{v}_{2} = 0,
\end{equation}
which forces us to admit
\begin{subequations}
  \begin{align}
    0 &= \widehat{\vec{v}_{1}}\cdot(c_{1}\widehat{\vec{v}_{1}} + c_{2}\vec{x}_{2})\\
    &= c_{1}\widehat{\vec{v}_{1}}\cdot\widehat{\vec{v}_{1}} + c_{2}\widehat{\vec{v}_{1}}\cdot\vec{x}_{2}\\
    &=c_{1} + c_{2}\widehat{\vec{v}_{1}}\cdot\vec{x}_{2}
  \end{align}
  hence
  \begin{equation}
c_{1} = -c_{2}\widehat{\vec{v}_{1}}\cdot\vec{x}_{2}.
  \end{equation}
  Setting $c_{2}=1$ (since it's arbitrary), we find
  \begin{equation}
\vec{v}_{2} = -(\widehat{\vec{v}_{1}}\cdot\vec{x}_{2})\widehat{\vec{v}_{1}}+\vec{x}_{1}
=\vec{x}_{2} -(\widehat{\vec{v}_{1}}\cdot\vec{x}_{2})\widehat{\vec{v}_{1}}.
  \end{equation}
\end{subequations}
We can quickly check that $\widehat{\vec{v}_{1}}\cdot\vec{v}_{2}=0$.

We now can see that $\vec{v}_{1}$, $\vec{v}_{2}$ span everything which
$\vec{x}_{1}$, $\vec{x}_{2}$ spanned. Since $\vec{x}_{3}$ was
independent of $\vec{x}_{1}$ and $\vec{x}_{2}$, we have
$\vec{x}_{3}\notin\Span(\{\vec{x}_{1},\vec{x}_{2})$. Therefore we will
use $\vec{x}_{3}$ to construct $\vec{v}_{3}$ by writing
\begin{equation}
  \vec{v}_{3} = c_{1}\vec{v}_{1} + c_{2}\vec{v}_{2} + c_{3}\vec{x}_{3}.
\end{equation}
We are trying to determine the unknown coefficients $c_{1}$, $c_{2}$,
$c_{3}$. We know $\vec{v}_{3}$ will be orthogonal to $\vec{v}_{1}$ and
$\vec{v}_{2}$:
\begin{equation}
\vec{v}_{3}\cdot\vec{v}_{2}=0,\quad\mbox{and}\quad\vec{v}_{3}\cdot\vec{v}_{1}=0.
\end{equation}
The first of these give us (recalling $\vec{v}_{2}\cdot\vec{v}_{1}=0$),
\begin{equation}
\vec{v}_{3}\cdot\vec{v}_{2} = 0 = c_{2}\vec{v}_{2}\cdot\vec{v}_{2} + c_{3}\vec{x}_{3}\cdot\vec{v}_{2},
\end{equation}
hence
\begin{equation}
c_{2} = -c_{3}\frac{\vec{x}_{3}\cdot\vec{v}_{2}}{\vec{v}_{2}\cdot\vec{v}_{2}}.
\end{equation}
Similarly, we find
\begin{equation}
\vec{v}_{3}\cdot\vec{v}_{1} = 0 = c_{1}\vec{v}_{1}\cdot\vec{v}_{1} + c_{3}\vec{x}_{3}\cdot\vec{v}_{1},
\end{equation}
give us
\begin{equation}
c_{1} = -c_{3}\frac{\vec{x}_{3}\cdot\vec{v}_{1}}{\vec{v}_{1}\cdot\vec{v}_{1}}.
\end{equation}
Setting $c_{3}=1$ give us
\begin{equation}
\vec{v}_{3} = \vec{x}_{3} - \left(\frac{\vec{x}_{3}\cdot\vec{v}_{1}}{\vec{v}_{1}\cdot\vec{v}_{1}}\right)\vec{v}_{1}
-\left(\frac{\vec{x}_{3}\cdot\vec{v}_{2}}{\vec{v}_{2}\cdot\vec{v}_{2}}\right)\vec{v}_{2}.
\end{equation}
Now we have three basis vectors in our collection.

We see that
\begin{equation}
\vec{v}_{3} = \vec{x}_{3} - (\vec{x}_{3}\cdot\widehat{\vec{v}_{1}})\widehat{\vec{v}_{1}} - (\vec{x}_{3}\cdot\widehat{\vec{v}_{2}})\widehat{\vec{v}_{2}}.
\end{equation}
The general pattern seems to be
\begin{equation}
\vec{v}_{n+1} = \vec{x}_{n+1} - \sum^{n}_{k=1}\left(\frac{\vec{x}_{n+1}\cdot\vec{v}_{k}}{\vec{v}_{k}\cdot\vec{v}_{k}}\right)\vec{v}_{k}.
\end{equation}
Is this actually true?

Well, we've proven it works for $n=2$ and $n=3$, so we can try proving
it by induction. We assume this works for arbitrary $(n+1)\in\NN$. Then
the inductive case, supposing we have $n+1$ orthogonal vectors
$\vec{v}_{1}$, $\vec{v}_{2}$, \dots, $\vec{v}_{n+1}$, and we have a
vector $\vec{x}_{n+2}\notin\Span(\{\vec{v}_{1},\dots,\vec{v}_{n+1}\})$.
Then we claim
\begin{equation}
\vec{v}_{n+2} = \vec{x}_{n+2} - \sum^{n+1}_{k=1}\left(\frac{\vec{x}_{n+1}\cdot\vec{v}_{k}}{\vec{v}_{k}\cdot\vec{v}_{k}}\right)\vec{v}_{k}
\end{equation}
is orthogonal to $\vec{v}_{j}$ for $j=1,\dots,n+1$. To see this, we
compute,
\begin{calculation}
  \vec{v}_{j}\cdot\vec{v}_{n+2}
  \step{unfolding definition of $\vec{v}_{n+2}$}
  \vec{v}_{j}\cdot\left(\vec{x}_{n+2} - \sum^{n+1}_{k=1}\left(\frac{\vec{x}_{n+1}\cdot\vec{v}_{k}}{\vec{v}_{k}\cdot\vec{v}_{k}}\right)\vec{v}_{k}\right)
  \step{distributivity of dot product}
  \vec{v}_{j}\cdot\vec{x}_{n+2} - \sum^{n+1}_{k=1}\left(\frac{\vec{x}_{n+1}\cdot\vec{v}_{k}}{\vec{v}_{k}\cdot\vec{v}_{k}}\right)\vec{v}_{j}\cdot\vec{v}_{k}
  \step{orthogonality of $\vec{v}_{j}\cdot\vec{v}_{k}=\delta_{j,k}\vec{v}_{j}\cdot\vec{v}_{j}$}
  \vec{v}_{j}\cdot\vec{x}_{n+2} - \sum^{n+1}_{k=1}\left(\frac{\vec{x}_{n+1}\cdot\vec{v}_{k}}{\vec{v}_{k}\cdot\vec{v}_{k}}\right)\delta_{j,k}\vec{v}_{j}\cdot\vec{v}_{j}
  \step{defining property of $\delta_{j,k}$}
  \vec{v}_{j}\cdot\vec{x}_{n+2} - \left(\frac{\vec{x}_{n+1}\cdot\vec{v}_{j}}{\vec{v}_{j}\cdot\vec{v}_{j}}\right)\vec{v}_{j}\cdot\vec{v}_{j}
  \step{commutativity of multiplication}
  \vec{v}_{j}\cdot\vec{x}_{n+2} - \vec{x}_{n+1}\cdot\vec{v}_{j}\left(\frac{\vec{v}_{j}\cdot\vec{v}_{j}}{\vec{v}_{j}\cdot\vec{v}_{j}}\right)
  \step{since $\vec{v}_{j}\neq\vec{0}$ hence $\vec{v}_{j}\cdot\vec{v}_{j}\neq0$}
  \vec{v}_{j}\cdot\vec{x}_{n+2} - \vec{x}_{n+1}\cdot\vec{v}_{j}
  \step{arithmetic}
  0.
\end{calculation}
Hence $\vec{v}_{n+2}$ is orthogonal to $\vec{v}_{j}$ for every
$j=1,\dots,n+1$.

\N{Graham--Schmidt Algorithm}\label{chunk:graham-schmidt}
Given $m$ linearly independent nonzero vectors $\vec{x}_{1}$, \dots,
$\vec{x}_{m}$, we can construct an orthonormal basis for
$\Span(\{\vec{x}_{1},\dots,\vec{x}_{m}\})$ as follows:

\N*{Step 1.} Set $\vec{f}_{1}=\widehat{\vec{x}_{1}}$. Then go to step 2.

\N*{Step 2.} For each $\vec{x}_{2}$, \dots, $\vec{x}_{m}$, compute
$\vec{v}_{k}$ by
\begin{equation}
\vec{v}_{k} = \vec{x}_{k} - \sum^{k}_{j=1}(\vec{x}_{k}\cdot\vec{f}_{j})\vec{f}_{j},
\end{equation}
and then set
\begin{equation}
\vec{f}_{k} = \widehat{\vec{v}_{k}} = \frac{\vec{v}_{k}}{\|\vec{v}_{k}\|}.
\end{equation}
This produces an orthonormal basis $B=\{\vec{f}_{1},\dots,\vec{f}_{m}\}$.

\begin{remark}
If we have a subspace $U\subset V$ and have found an orthonormal basis
$B_{U}$ of $B$, then we can extend it to a basis $B$ of $V$ by taking
$n=\dim(V)$ linearly independent vectors in $V$, and applying the
Graham--Schmidt algorithm to add them to $B_{V}$.
\end{remark}

\begin{definition}\label{defn:basis:projection}
Let $\vec{u},\vec{v}\in V$ be vectors. Assume $\vec{u}\neq\vec{0}_{V}$.
We define the \define{Projection} of $\vec{v}$ along the $\vec{u}$
direction to be the vector
\begin{equation}
\operatorname{Proj}_{\vec{u}}(\vec{v}) = \frac{\vec{u}\cdot\vec{v}}{\vec{u}\cdot\vec{u}}\vec{u}=(\widehat{\vec{u}}\cdot\vec{v})\widehat{\vec{u}}
\end{equation}
where $\widehat{\vec{u}}=\vec{u}/\|\vec{u}\|$ is a unit vector.
\end{definition}

\begin{remark}
We can see that the Graham--Schmidt algorithm can be rephrased as taking
$n$ linearly independent vectors $\vec{x}_{1}$, $\vec{x}_{2}$, \dots,
$\vec{x}_{n}$, and forming
\begin{subequations}
\begin{align}
  \vec{u}_{1} &= \vec{x}_{1}\\
  \vec{u}_{2} &= \vec{x}_{2} - \operatorname{Proj}_{\vec{u}_{1}}(\vec{x}_{2})\\
  \vec{u}_{3} &= \vec{x}_{3} - \operatorname{Proj}_{\vec{u}_{1}}(\vec{x}_{3}) - \operatorname{Proj}_{\vec{u}_{2}}(\vec{x}_{3})\\
  \vdots \nonumber\\
  \vec{u}_{n} &= \vec{x}_{n} - \sum^{n-1}_{k=1}\operatorname{Proj}_{\vec{u}_{k}}(\vec{x}_{n}).
\end{align}
\end{subequations}
This produces a set of $n$ orthogonal vectors, and we could normalize
them (``put hats on them'') to obtain $n$ orthonormal vectors.
\end{remark}


\begin{definition}
Let $\vec{u},\vec{v}\in V$ be vectors. Assume $\vec{u}\neq\vec{0}_{V}$.
We define the \define{Orthogonal Decomposition} of $\vec{v}$ with
respect to $\vec{u}$ as consisting of two vectors:
\begin{enumerate}
\item the parallel part $\vec{v}^{\parallel} = \operatorname{Proj}_{\vec{u}}(\vec{v})$,
and
\item the perpendicular part $\vec{v}^{\perp} = \vec{v} - \vec{v}^{\parallel}$.
\end{enumerate}
\end{definition}

\section{Linear Transformations}

\M
We have, so far, introduced a new shiny toy, a new mathematical gadget
called a ``vector space''. So far, we have discussed quite a few aspects
about them, except for one very important thing: how do we ``map'' one
vector space into another?

\begin{definition}
Let $V$, $W$ be real vector spaces.
A \define{Linear Transformation} is a function $L\colon V\to W$ such
that for each $\vec{u}$, $\vec{v}\in V$ and for any $c_{1},c_{2}\in\RR$,
we have it map linear combinations in $V$ to linear combinations in $W$:
\begin{equation}
L(c_{1}\vec{u} + c_{2}\vec{v}) = c_{1}L(\vec{u}) + c_{2}L(\vec{v}).
\end{equation}
If further $V=W$, then we call $L$ a \define{Linear Operator}.
We may also just refer to the function $f$ as \emph{Linear}.
\end{definition}

\begin{remark}
  Some authors give two conditions for $L$ being a linear transformation:
  \begin{enumerate}
  \item homogeneity: $L(c\vec{v})=cL(\vec{v})$ for every $c\in\RR$ and
    $\vec{v}\in V$
  \item additive: $L(\vec{u}+\vec{v})=L(\vec{u})+L(\vec{v})$ for every
    $\vec{u}$, $\vec{v}\in V$.
  \end{enumerate}
  Can you prove these two versions are secretly the same?
\end{remark}

\begin{example}
Let $V$ be a real vector space. The identity function $\id\colon V\to V$,
defined by $\id(\vec{v})=\vec{v}$ for all $\vec{v}\in V$, is a linear
operator on $V$.

\begin{proof}
Let $\vec{u}$, $\vec{v}\in V$ be arbitrary. Let $c_{1},c_{2}\in\RR$ be arbitrary.
We want to prove
\begin{equation}
\id(c_{1}\vec{u} + c_{2}\vec{v}) = c_{1}\id(\vec{u}) + c_{2}\id(\vec{v}).
\end{equation}
We see that the left-hand side expands to
\begin{equation}
\id(c_{1}\vec{u} + c_{2}\vec{v}) = c_{1}\vec{u} + c_{2}\vec{v},
\end{equation}
and the right-hand side expands to
\begin{equation}
c_{1}\id(\vec{u}) + c_{2}\id(\vec{v}) = c_{1}\vec{u} + c_{2}\vec{v}.
\end{equation}
Then we invoke the well-known fact that,
\begin{equation}
c_{1}\vec{u} + c_{2}\vec{v} = c_{1}\vec{u} + c_{2}\vec{v}
\end{equation}
to conclude we must have
\begin{equation}
\id(c_{1}\vec{u} + c_{2}\vec{v}) = c_{1}\id(\vec{u}) + c_{2}\id(\vec{v}).
\end{equation}
as desired.
\end{proof}
\end{example}

\begin{non-example}[Shifting is not a linear transformation]
Let $V$ be a vector space, let $\vec{v}_{0}\in V$ be a nonzero vector.
Define $f\colon V\to V$ by
\begin{equation}
f(\vec{v})=\vec{v}+\vec{v}_{0}.
\end{equation}
This is not a linear transformation. Why not? Well, suppose we have
$\vec{u}\in V$ and $\vec{v}\in V$, then
\begin{calculation}
  f(\vec{u} + \vec{v})
\step{unfold definition of $f$}
  (\vec{u} + \vec{v})+\vec{v}_{0}
\end{calculation}
However,
\begin{calculation}
  f(\vec{u}) + f(\vec{v})
\step{unfold definition of $f$}
  (\vec{u} +\vec{v}_{0}) + (\vec{v}+\vec{v}_{0})
\end{calculation}
We see that, in general,
\begin{equation}
(\vec{u} +\vec{v}_{0}) + (\vec{v}+\vec{v}_{0})\neq(\vec{u} + \vec{v})+\vec{v}_{0}.
\end{equation}
Hence
\begin{equation}
f(\vec{u} + \vec{v})\neq f(\vec{u}) + f(\vec{v}).
\end{equation}
Or to summarize it in a single equation
\begin{equation}
\begin{array}{ccc}
f(\vec{u} + \vec{v}) &\stackrel{???}{=}& f(\vec{u}) + f(\vec{v})\\
\verteq & & \verteq\\
(\vec{u} + \vec{v})+\vec{v}_{0} & \neq & (\vec{u} +\vec{v}_{0}) + (\vec{v}+\vec{v}_{0})
\end{array}
\end{equation}
which means $f$ cannot be linear.
\end{non-example}

\begin{example}
Let $c\in\RR$ be a nonzero constant $c\neq0$, let $V$ be a real vector space.
Let $f\colon V\to V$ be defined by
\begin{equation}
f(\vec{v}) = c\vec{v}.
\end{equation}
Then $f$ is a linear operator.

\begin{proof}
Let $\vec{u},\vec{v}\in V$ be arbitrary, let $c_{1},c_{2}\in\RR$ be arbitrary.
We will show $f(c_{1}\vec{u}+c_{2}\vec{v})=c_{1}f(\vec{u})+c_{2}f(\vec{v})$.
We begin by expanding
\begin{calculation}
  f(c_{1}\vec{u}+c_{2}\vec{v})
\step{unfold definition of $f$}
  c(c_{1}\vec{u}+c_{2}\vec{v})
\step{distributivity of scalar multiplication}
  cc_{1}\vec{u}+cc_{2}\vec{v}
\step{associativity of scalar multiplication}
  c(c_{1}\vec{u})+c(c_{2}\vec{v})
\step{folding the definition of $f$}
  f(c_{1}\vec{u})+f(c_{2}\vec{v})
\end{calculation}
This proves $f$ is linear.
\end{proof}
\end{example}

\begin{example}
  Let $\vec{u}\in V$ be a nonzero vector. Then projection along
  $\vec{u}$,
  \begin{subequations}
  \begin{equation}
\operatorname{Proj}_{\vec{u}}\colon V\to V
  \end{equation}
  sends
\begin{equation}
\vec{v}\mapsto\frac{\vec{u}\cdot\vec{v}}{\vec{u}\cdot\vec{u}}\vec{u}.
\end{equation}
  \end{subequations}
This is a linear operator.
\end{example}

\begin{example}
Let $\widehat{\vec{x}}\in V$ be a unit vector.
We define the \define{Householder transformation}
\begin{subequations}
  \begin{align}
    H_{\vec{x}}\colon &V\to V
    \intertext{by}
    &\vec{v}\mapsto \vec{v} - \widehat{\vec{x}}(\widehat{\vec{x}}\cdot\vec{v}).
  \end{align}
\end{subequations}
This describes a reflection about the plane whose normal vector is $\widehat{\vec{x}}$.
We see this is the sum of two linear operators, the identity matrix, and
the projection in the direction of $-\vec{x}$. The sum of linear
operators is itself a linear operator.
\end{example}

\N{Other examples}
There are a few other examples which are useful, in particular, if
$V$ is a vector space and $U\subset V$ is a subspace, then the inclusion
map
\begin{equation}
\iota\colon U\to V
\end{equation}
defined by $\iota(\vec{u})=\vec{u}$ for every $\vec{u}\in U$, this is a
linear transformation. We could also go the other way around: we could
construct a mapping
\begin{equation}
p\colon V\to U
\end{equation}
which projects to $U$. It's a little trickier \emph{at this point} to
prove it, though.

\begin{definition}
  When $L\colon V\to W$ is a linear transformation, we call:
\begin{itemize}
\item the inputs $V$ the \define{Domain} of $L$,
\item the possible outputs $W$ the \define{Codomain} of $L$,
\item the \define{Image} of $L$ is the subset $L(V) = \{L(\vec{v})\in W\mid\vec{v}\in V\}$
of vectors in $W$ which are the output of $L$.
\end{itemize}
Note: the codomain consists of the \emph{possible} outputs, the image
consists of the \emph{actual} outputs.
\end{definition}

\begin{definition}
If $L\colon V\to W$ is a linear transformation, we can define its
\define{Inverse} to be a linear transformation $L^{-1}\colon W\to V$
such that $L^{-1}\circ L=\id_{V}$ and $L\circ L^{-1}=\id_{W}$.
\end{definition}

\M
A lot of the terminology for matrices carries over to linear
transformations. For example, if $L\colon V\to V$ is a linear operator,
we call it \define{Similar} to a linear operator $M\colon V\to V$ if
there exists an invertible linear operator $P\colon V\to V$ such that
$L = P^{-1}\circ M \circ P$.
Compare this to similar matrices, $\mat{A} \sim\mat{B}$ if there exists
an invertible $\mat{P}$ such that $\mat{A}=\mat{P}^{-1}\mat{B}\mat{P}$.
We may abuse notation and write $L\sim M$ when two linear operators are
similar.

\subsection{Matrices as ``Coordinates'' of Linear Transformations}

\M
We saw how basis vectors, for a finite-dimensional vector space $V$,
allows us to introduce coordinates for vectors (in
Definition~\ref{defn:basis:coordinates}).
This was great because it allowed us to use some of our matrix machinery
we built in Part~\ref{part:matrices} when computing vector operations.
But there is more: basis vectors let us express a linear transformation
as a matrix, and ``the linear transformation acting on a vector'' as
mere matrix multiplication.

\begin{definition}
Let $V$, $W$ be [finite-dimensional] vector spaces with bases
$B_{V}=(\vec{e}_{1},\dots,\vec{e}_{m})$ and $B_{W}=(\vec{f}_{1},\dots,\vec{f}_{n})$
respectively. Let $L\colon V\to W$ be a linear transformation.
Then the \define{Matrix for $L$ relative to $B_{V}$ and $B_{W}$}
is the matrix $\mat{M}$ whose $j^{\text{th}}$ column is the coordinate vector
for $L(\vec{e}_{j})$ relative to $B_{W}$.
Then, for any $\vec{v}\in V$, we may compute $L(\vec{v})$ using
$\mat{M}$ multiplied by the coordinate vector of $\vec{v}$ relative to
basis $\vec{e}_{j}$.
\end{definition}

\begin{remark}
Just because we \emph{defined} something doesn't mean it works as
intended, or even exists. We need to \emph{prove it} in a theorem.
\end{remark}

\begin{theorem}[This stuff works]
Let $L\colon V\to W$ be a linear transformation from an $m$-dimensional
real vector space $V$ to an $n$-dimensional real vector space $W$.
Let $B_{V}=(\vec{e}_{1}, \dots, \vec{e}_{m})$ be a basis for $V$; let
$B_{W}=(\vec{f}_{1}, \dots, \vec{f}_{n})$ be a basis for $W$.

Then the $n\times m$ matrix $\mat{M}$ --- whose $j^{\text{th}}$ column is
the coordinate vector $[L(\vec{e}_{j})]_{B_{W}}$ of $L(\vec{e}_{j})$
relative to the basis $B_{W}$ --- is associated with $L$ and has the
following property: For any $\vec{v}\in V$, we have
\begin{equation}
[L(\vec{v})]_{B_{W}} = \mat{M}[\vec{v}]_{B_{V}}
\end{equation}
where $[\vec{v}]_{B_{V}}$ is the $m\times1$ coordinate vector of
$\vec{v}$ relative to $B_{V}$, and $[L(\vec{v})]_{B_{W}}$ is the $n\times1$
coordinate vector of $L(\vec{v})$ relative to $B_{W}$. Moreover,
$\mat{M}$ is the only matrix with this property.
\end{theorem}

\begin{proof}[Proof sketch]
The idea is to to construct such a matrix $\mat{M}$ from $L\colon V\to W$
and bases $B_{V}=(\vec{e}_{1},\dots,\vec{e}_{m})$ and
$B_{W}=(\vec{f}_{1},\dots,\vec{f}_{n})$, as follows:
\begin{itemize}
\item[Step 1.] Compute $L(\vec{e}_{j})$ for $j=1,\dots,m$.
\item[Step 2.] Find the coordinate vector $[L(\vec{e}_{j})]_{B_{W}}$ for
  $L(\vec{e}_{j})$ relative to the basis $B_{W}$. This means expressing
  $L(\vec{e}_{j})$ as a linear combination of the $\vec{f}_{i}$.
\item[Step 3.] The matrix $\mat{M}$ of $L$ with respect to $B_{V}$ and
  $B_{W}$ is formed by choosing $[L(\vec{e}_{j})]_{B_{W}}$ as the
  $j^{\text{th}}$ column of $\mat{M}$.
\end{itemize}
In this way, any vector $\vec{v} = c_{1}\vec{e}_{1}+\dots+c_{m}\vec{e}_{m}$
is mapped to $L(\vec{v}) = c_{1}L(\vec{e}_{1}) + c_{2}L(\vec{e}_{2}) + \dots + c_{m}L(\vec{e}_{m})$
by definition of a linear transformation, and this coincides with matrix
multiplication $\mat{M}[\vec{v}]_{B_{V}}$.
\end{proof}

\M
What do we do if the bases for either vector space is not canonical or
orthogonal? The trick is to form an augmented matrix
\begin{equation}
[\vec{f}_{1}\quad\dots\quad f_{n}\mid L(\vec{e}_{1})\quad\dots\quad L(\vec{e}_{m})]
\end{equation}
then applying elementary row operations, we transform it to reduced row
echelon form, and then keep transforming it until it becomes
\begin{equation}
[\vec{f}_{1}\quad\dots\quad f_{n}\mid L(\vec{e}_{1})\quad\dots\quad L(\vec{e}_{m})]\sim[\mat{I}\mid\mat{M}].
\end{equation}
We then identify $\mat{M}$ as the matrix for $L$ relative to the basis
$B_{V}$ and $B_{W}$.

\subsection{Injective, Surjective, Bijective Linear Transformations}

\begin{definition}
  Let $L\colon V\to W$ be a linear transformation. We call $L$
  \begin{itemize}
  \item \define{Surjective} (or \emph{Onto}) if, to each and every $\vec{w}\in W$,
    there exists a $\vec{v}\in V$ such that $L(\vec{v})=\vec{w}$
  \item \define{Injective} (or, confusingly, \emph{into}) if for any
    $\vec{v}_{1},\vec{v}_{2}\in V$ such that
    $L(\vec{v}_{1})=L(\vec{v}_{2})$, then we have
    $\vec{v}_{1}=\vec{v}_{2}$ --- equivalently, if
    $\vec{v}_{1}\neq\vec{v}_{2}$, then $L(\vec{v}_{1})\neq L(\vec{v}_{2})$.
  \item \define{Bijective} if it is both surjective and injective.
  \end{itemize}
\end{definition}

\begin{remark}
These terms (injective, surjective, bijective) hold for \emph{any}
function on sets. That is to say, these are not ``linear algebra
specific terms''.
\end{remark}

\begin{example}[Surjective map]
Let $U\subset V$ be a subspace. The map $L\colon V\to U$, defined by for
any $\vec{u}\in U$, $L(\vec{u})=\vec{u}$, and for any $\vec{v}\in V$ but
$\vec{v}\notin U$ we have $L(\vec{v})=\vec{0}$. This is a surjective
linear transformation.

\begin{proof}
Surjectivity isn't hard, we're told every $\vec{u}\in U$ are mapped to
$L(\vec{u})=\vec{u}$. Linearity may be a bit more difficult. If we had
some basis $B_{U}$ of $U$ which extends to a basis of $V$, then we see
\begin{equation}
L(\sum_{j=1}^{m}c_{j}\vec{b}_{j} + \sum_{k=m+1}^{n}d_{k}\vec{f}_{k}) = \sum^{m}_{j=1}c_{j}\vec{b}_{j}
\end{equation}
where $\vec{b}_{j}\in B_{U}$ for each $j=1,\dots,m$ and
$\vec{f}_{k}\notin B_{U}$ are basis vectors for the rest of $V$ (which
would be mapped to zero), and $c_{j}\in\RR$ for $j=1,\dots,m$ and
$d_{k}\in\RR$ for $k=m+1,\dots,n$. This is indeed linear, by definition.
\end{proof}
\end{example}

\N{Surjective Linear Maps}
A surjective linear map is sometimes denoted with a two-headed arrow
$L\colon V\onto W$. We intuitively think of surjectivity as ``covering''
the entire codomain.

\begin{example}[Injective map]
Let $U\subset V$ be a subspace. Then $L\colon U\to V$, defined by $L(\vec{u})=\vec{u}$,
is an injective linear map.

\begin{proof}
Let $\vec{u}_{1},\vec{u}_{2}\in U$ be arbitrary. Assume $\vec{u}_{1}\neq\vec{u}_{2}$.
Then $L(\vec{u}_{1})=\vec{u}_{1}\neq\vec{u}_{2}=L(\vec{u}_{1})$, hence
$L(\vec{u}_{1})\neq L(\vec{u}_{2})$. Then by definition, $L$ is injective.
\end{proof}
\end{example}

\N{Injective maps embed}
An injective linear map is sometimes denoted with the arrow $L\colon V\into W$
with a hooked arrow. This intuitively corresponds to embedding $V$ into $W$.
That is to say, there exists a subspace of $W$ which is ``the same'' as
$V$.

\N{Isomorphisms}
A bijective linear transformation is also called a \define{Isomorphism}
of vector spaces. If $L\colon V\to W$ is a bijective linear
transformation, then we may write $V\cong W$ to indicate there exists a
bijective linear transformation between them.

What this means is that, when viewed as vector spaces, $V$ and $W$ are
``the same''. To every element $\vec{v}\in V$ there is a unique
$\vec{w}\in W$ such that $L(\vec{v})=\vec{w}$.

\N{General Picture}
We can combine these insights to get some sense of what these terms
mean. For example, a surjective linear map is one whose codomain is
isomorphic to a subspace of the domain. An injective linear map is one
whose image is isomorphic with the domain.

\subsection{Kernels and Images}

\begin{definition}
Let $L\colon V\to W$ be a linear transformation. We define the
\define{Kernel} of $L$ to be the collection of vectors mapped to the
zero vector of $W$: $\ker(L)=\{\vec{v}\in V\mid L(\vec{v})=\vec{0}_{W}\}$.
\end{definition}

\begin{remark}
We can also talk about the ``kernel'' of a matrix, or other mathematical
mappings when there is a notion of ``zero'' (or a ``zero-like quantity'').
\end{remark}

\begin{proposition}
If $L\colon V\to W$ is a linear transformation, then $\ker(L)$ is a
subspace of $V$.
\end{proposition}

\begin{proof}
Claim 1: for any $\vec{v}$, $\vec{w}\in\ker(L)$, we have $\vec{v}+\vec{w}\in\ker(L)$.

We can see this from $L(\vec{v}+\vec{w}) = L(\vec{v})+L(\vec{w})$ by
linearity, and then $L(\vec{v})+L(\vec{w}) = \vec{0}_{W}+\vec{0}_{W} = \vec{0}_{W}$.
Hence $\vec{v}+\vec{w}\in\ker(L)$.

Claim 2: for any $\vec{v}\in\ker(L)$ and $c\in\RR$, we have
$(c\vec{v})\in\ker(L)$.

We can see this from $L(c\vec{v})=cL(\vec{v})$ due to linearity, and
$cL(\vec{v})=c\vec{0}_{W}=\vec{0}_{W}$. Hence $c\vec{v}\in\ker(L)$.

Then by Theorem~\ref{thm:subspaces:subset-closed-under-linear-combos-is-a-subspace},
$\ker(L)$ is a subspace of $V$.
\end{proof}

\begin{proposition}
Let $L\colon V\to W$ be a linear transformation.
If $\vec{v}_{1},\vec{v}_{2}\in V$ are mapped to the same element
$L(\vec{v}_{1})=L(\vec{v}_{2})$, then their difference lives in the
kernel of $L$, $\vec{v}_{2}-\vec{v}_{1}\in\ker(L)$.
\end{proposition}

\begin{proof}
  Assume $L(\vec{v}_{1})=L(\vec{v}_{2})$. Then
  \begin{calculation}
    L(\vec{v}_{1})=L(\vec{v}_{2})
    \step[\equiv]{subtracting $L(\vec{v}_{1})$ from both sides}
    L(\vec{v}_{2})-L(\vec{v}_{1})=\vec{0}_{W}
    \step[\equiv]{linearity}
    L(\vec{v}_{2}-\vec{v}_{1})=\vec{0}_{W}
  \end{calculation}
  hence $\vec{v}_{2}-\vec{v}_{1}\in\ker(L)$ as desired.
\end{proof}

\begin{theorem}
  Let $L\colon V\to W$ be a linear transformation.
  Then $L$ is an injective linear map if and only if $\ker(L)=0$ is the trivial subspace.
\end{theorem}
\begin{proof}
$(\implies)$ Assume $L$ is an injective map. For every $\vec{v}\in V$
  such that $\vec{v}\neq\vec{0}_{V}$ we have $L(\vec{v})\neq L(\vec{0}_{V})=\vec{0}_{W}$.
  Then $L(\vec{v})\neq\vec{0}_{W}$, which implies $\vec{v}\notin\ker(L)$
  when $\vec{v}\neq\vec{0}_{V}$. Hence $\ker(L)=\{\vec{0}_{V}\}$.

$(\impliedby)$ Assume $\ker(L)=0=\{\vec{0}_{V}\}$.
  Then for any $\vec{v}_{1},\vec{v}_{2}\in V$ such that
  $L(\vec{v}_{1})=L(\vec{v}_{2})$, we see $L(\vec{v}_{2})-L(\vec{v}_{1})=\vec{0}_{W}$.
  By linearity, we know $L(\vec{v}_{2})-L(\vec{v}_{1})=L(\vec{v}_{2}-\vec{v}_{1})$,
  hence $\vec{v}_{2}-\vec{v}_{1}\in\ker(L)$. But this implies $\vec{v}_{2}-\vec{v}_{1}=\vec{0}_{V}$,
  which means $\vec{v}_{2}=\vec{v}_{1}$. Hence $L$ is injective.
\end{proof}

\N{Dimensions of Kernel}
Let $L\colon V\to W$ be a linear transformation, $V$ a
finite-dimensional vector space. Suppose $\dim\ker(L)=k<\dim(V)$.
What does this mean? Well, we have a $k$-dimensional subspace of $V$
which ``collapses'' under application of $L$, in the sense that
$L(\ker(L))=0$ is the trivial subspace of $W$.

But if $n=\dim(V)$, then what happens to the other $n-k$ dimensions of
$V$ under $L$? Well, there are two possibilities:
\begin{enumerate}
\item some elements of $V\setminus\ker(L)$ would be mapped to
  $\vec{0}_{W}$, or
\item no element of $V\setminus\ker(L)$ could be mapped to $\vec{0}_{W}$.
\end{enumerate}
If some element $\vec{v}$ in $V\setminus\ker(L)$ [i.e. $\vec{v}\in V$
  but $\vec{v}\notin\ker(L)$] is mapped to $\vec{0}_{W}$, then
$L(\vec{v})=\vec{0}_{W}$ which by definition makes it
$\vec{v}\in\ker(L)$. This is impossible, so we are in the second
possibility: no element of $V\setminus\ker(L)$ could be mapped to
$\vec{0}_{W}$.

Could $V\setminus\ker(L)$ form a subspace of $V$?
Technically, no, because $\vec{0}_{V}\in\ker(L)$, so it would be
impossible for $\vec{0}_{V}\in V\setminus\ker(L)$. Alright, well, what
about the set $U=\{\vec{0}_{V}\}\cup(V\setminus\ker(L))$, could this
form a subspace of $V$?

Let us consider a basis for $\ker(L)$. We could do this by finding $k$
linearly independent vectors, then applying the Graham--Schmidt
algorithm (\S\ref{chunk:graham-schmidt}) to form a basis $B_{K}$ of
$\ker(L)$. We can consider the canonical basis for $V$, then apply the
Graham--Schmidt algorithm to extend $B_{K}$ to an orthonormal basis $B$ of
all of $V$. The elements $B_{U}=\{\vec{b}\in B\mid\vec{b}\notin B_{K}\}$
form a basis for $U=\Span(B_{U})$. Moreover, $L(V)=L(U)$.

We see that $\dim(U)=|B_{U}|$ is the number of basis elements which do
not belong to the kernel, and $\dim(\ker(L))$ is the number of the
remaining basis vectors. Consequently,
\begin{equation}
\dim(U) + \dim(\ker(L)) = \dim(V).
\end{equation}
We also see that $\dim(U)=\dim(L(V))$. This gives us the celebrated
result
\begin{equation}
\boxed{\dim(L(V)) + \dim(\ker(L)) = \dim(V).}
\end{equation}

\section{Eigenstuff}

\N{Note to self}
I wanted to include this as the ending of part II, but I realized the
usefulness of eigenstuff is in having eigenvectors forming a basis, and
then diagonalizing the matrix. This requires putting this section in part III.

\begin{example}[Motivating Example]
  Consider the matrix
  \begin{equation}
\mat{M} = \begin{pmatrix}2 & -1\\
-1 & 2
\end{pmatrix}.
  \end{equation}
  We find that there are two ``directions'' (distinct vectors) which are
  just dilated when we multiply by $\mat{M}$,
  \begin{subequations}
    \begin{equation}
\mat{M}\begin{pmatrix}1\\1\end{pmatrix} = \begin{pmatrix}1\\1\end{pmatrix},
    \end{equation}
    and
    \begin{equation}
\mat{M}\begin{pmatrix}1\\-1\end{pmatrix} = 3\begin{pmatrix}1\\-1\end{pmatrix}.
    \end{equation}
  \end{subequations}
  This isn't a neat parlor trick: it turns out any invertible $n\times n$
  matrix will have at most $n$ vectors which are ``dilated'' by the matrix.
\end{example}

\begin{definition}
Let $\mat{A}$ be an $n\times n$ matrix.
We define an \define{Eigenvector} of $\mat{A}$ to be a [column]
$n$-vector $\vec{v}$ such that there is a nonzero $\lambda\in\RR$
[called the \define{Eigenvalue} associated with $\vec{v}$] satisfying
\begin{equation}\label{eq:defn:eigenvector}
\mat{A}\vec{v}=\lambda\vec{v}.
\end{equation}
\end{definition}

\N{Finding Eigenvalues and Eigenvectors}
This is great, but how do we find eigenvalues and eigenvectors?
The first thing to note is we can rewrite Eq~\eqref{eq:defn:eigenvector}
by subtracting $\lambda\vec{v}$ from both sides:
\begin{equation}
\mat{A}\vec{v}-\lambda\vec{v}=\vec{0}.
\end{equation}
We insert a secret identity operator (the matrix analog of ``multiply by $1$''):
\begin{equation}
\mat{A}\vec{v}-\lambda\mat{I}\vec{v}=\vec{0}.
\end{equation}
We can factor out $\vec{v}$ by distributivity:
\begin{equation}
(\mat{A}-\lambda\mat{I})\vec{v}=\vec{0}.
\end{equation}
For this equation to hold, either $\vec{v}=0$ or
$(\mat{A}-\lambda\mat{I})=0$, right?

Wrong: $(\mat{A}-\lambda\mat{I})$ could be nonzero and noninvertible.
That is when
\begin{equation}
\det(\mat{A}-\lambda\mat{I})=0.
\end{equation}
But the left-hand side is not identically zero. In fact, the left-hand
side is a polynomial in $\lambda$.
\emph{This polynomial is how we find eigenvalues for matrices.}

Once we have an eigenvalue, we can plug it in and then solve the system
of equations for the eigenvector. But first, let us define this
polynomial quantity.

\begin{definition}
Let $\mat{A}$ be an $n\times n$ matrix.
The \define{Characteristic Polynomial} of $\mat{A}$ is 
\begin{equation}
p(\lambda) = \det(\mat{A}-\lambda\mat{I}).
\end{equation}
Some authors use $\det(\lambda\mat{I}-\mat{A})$, it doesn't matter since
they have the same roots (which are the eigenvalues of $\mat{A}$ and the
\emph{actual} quantity of interest).
\end{definition}

\begin{example}
Recall our motivating example at the start of this section, we had
\begin{equation}
\mat{M} = \begin{pmatrix}2 & -1\\
-1 & 2
\end{pmatrix}.
\end{equation}
Its characteristic polynomial is
\begin{equation}
\det\begin{pmatrix}2-\lambda & -1\\
-1 & 2-\lambda
\end{pmatrix} = (2-\lambda)^{2}-1 = \lambda^{2}-4\lambda+3.
\end{equation}
We find this has roots $\lambda=1$ and $\lambda=3$.

We can find the eigenvector for $\lambda=1$ by solving
\begin{subequations}
  \begin{align}
    2x_{1} -x_{2} &= x_{1}\\
    -x_{1} + 2x_{2} &= x_{2}.
  \end{align}
\end{subequations}
These give us 2 copies of the same line described by
\begin{equation}
-x_{1} = -x_{2},\quad\mbox{or}\quad x_{1}=x_{2}.
\end{equation}
We have a generic eigenvector look like
\begin{equation*}
\vec{v}_{1} = m\begin{pmatrix}1\\1
\end{pmatrix}
\end{equation*}
where $m\in\RR$. Usually we normalize the eigenvector to be a unit
vector (so we fix any such parameters), which gives us
\begin{equation}
  \vec{v}_{1} = \begin{pmatrix}1/\sqrt{2}\\
    1/\sqrt{2}
  \end{pmatrix}.
\end{equation}
This is one eigenvector.

The other eigenvector, the one associated with $\lambda=3$, requires
solving the system of equations
\begin{subequations}
  \begin{align}
    2x_{1} -x_{2} &= 3x_{1}\\
    -x_{1} + 2x_{2} &= 3x_{2}.
  \end{align}
\end{subequations}
This gives us two copies of the same line, described by the equation
\begin{equation}
x_{1} = -x_{2}.
\end{equation}
The unit eigenvector is then
\begin{equation}
\vec{v}_{2} = \begin{pmatrix}1/\sqrt{2}\\
-1/\sqrt{2}
\end{pmatrix}.
\end{equation}
The reader may verify these satisfy the equation $\mat{M}\vec{v}=\lambda\vec{v}$
for eigenvectors of $\mat{M}$.
\end{example}
% Sections left to write:
% - Linear Transformations
% - Change of Coordinates
% - Orthogonal Complement of Subspaces
% - Constructing Orthonormal Bases
% - Eigendecomposition(???)
% - Rank-Nullity Theorem

\vfill\eject
\part{Appendices}
\appendix
%\section{Structured Derivations}
\M I am thinking about writing a text about elementary linear algebra
using structured derivations. Instead of writing a sequence of
equations, possibly with hints, like:
\begin{subequations}
\begin{align}
  A &= B \qquad\mbox{(hint 1)}\\
  &= C \qquad\mbox{(hint 2)}\\
  &= D \qquad\mbox{(hint 3)}
\end{align}
\end{subequations}
where it is unclear if ``hint 2'' refers to why $B=C$ follows from
$A=B$, or perhaps why $A=C$ follows from $A=B$; and what ``hint 3''
attempts to explain is anyone's guess. Instead, we would have:
\begin{calculation}
  A
  \step{hint why $A=B$}
  B
  \step*{hint why $B=C$}
  C
  \step{hint why $B=D$}
  D
\end{calculation}
I have endeavoured to imitate the vertical spacing of equations (using
\verb#\displayskip# and \verb#\belowdisplayskip#), as well as the
horizontal spacing amount which the \verb#fleqn# document option would
use.
\begin{align*}
  A &= B \qquad\mbox{(hint 1)}\\
  &= C \qquad\mbox{(hint 2)} \\
  &= D \qquad\mbox{(hint 3)}
\end{align*}


\section{Neo-Ricardian Economics}

\N{Matrix form of equations}
We can express an economy in terms of its input matrix $\mat{A}$, and
its output matrix $\mat{B}$
\begin{equation}
\mat{A}\vec{x}_{n} = \mat{B}\vec{x}_{n+1}
\end{equation}
where $\vec{x}_{n}$ is the stock of commodities at time $t=n$ production
cycles (after some initial production cycle), and $\vec{x}_{n+1}$ is the
stock of commodities \emph{after} the production cycle. We are trying to
find price vectors $\vec{p}$ such that
\begin{equation}
\mat{A}\vec{p} = \mat{B}\vec{p}.
\end{equation}
We assume that, for each sector $i$, the total inputs is no less than
the amount produced
\begin{equation}
\sum_{j=1}^{n}a_{i,j} \leq\sum_{j=1}^{n}b_{i,j}.
\end{equation}
Since usually $\mat{B}$ is diagonal, this simplifies to
\begin{equation*}
\sum_{j=1}^{n}a_{i,j} \leq b_{j,j}.
\end{equation*}
The trick is to first introduce a new vector
\begin{equation}
\vec{q} = \mat{B}\vec{p}\implies \vec{p}=\mat{B}^{-1}\vec{q},
\end{equation}
then rewrite our problem as
\begin{equation}
\mat{A}\vec{p} = \mat{B}\vec{p}\iff \mat{A}\mat{B}^{-1}\vec{q} = \vec{q}.
\end{equation}
For a subsistence economy, when no industry sector has an output, this
is the situation we are trying to solve.

\N{Production with a Surplus}
When at least one industry sector produces more output than is needed
across the entire economy, then there is surplus produced. In this case,
there is a rate of profit $r$, and we are trying to solve the system of
equations:
\begin{equation}
(1+r)\mat{A}\vec{p} = \mat{B}\vec{p}.
\end{equation}
We do the same trick, by writing
\begin{equation}
\vec{q} = \mat{B}\vec{p},
\end{equation}
then plugging this into the equations for an economy with a surplus
gives us
\begin{equation}
(1+r)\mat{A}\vec{p} = \mat{B}\vec{p} \iff (1+r)\mat{A}\mat{B}^{-1}\vec{q}=\vec{q}.
\end{equation}
This is an eigenvalue problem! We divide both sides by $(1+r)$ to find
\begin{equation}
\mat{A}\mat{B}^{-1}\vec{q}=\frac{1}{1+r}\vec{q},
\end{equation}
where $\lambda=1/(1+r)$ is the eigenvalue and $\vec{q}$ is the
eigenvector. We are looking for real $\lambda$ such that
$0<\lambda\leq1$ --- when $\lambda=1$, there is no profit $r=0$;
similarly, it is rare to find $r>1$ (i.e., $\lambda<1/2$). The real
constraint is that every entry of $\vec{q}$ must be positive (otherwise
some commodities have negative value, which makes no sense).

\begin{example}[{Sraffa~\cite[see \S5]{sraffa}}]
  Consider the economy given by the equations of production
  \begin{subequations}
    \begin{align}
      280~\mbox{qr. wheat} + 12~\mbox{t. iron} &\to 575~\mbox{qr. wheat}\\
      120~\mbox{qr. wheat} + 8~\mbox{t. iron} &\to 20~\mbox{t. iron}
    \end{align}
  \end{subequations}
  We have
  \begin{equation}
    \mat{A} = \begin{pmatrix}280 & 12\\
      120 & 8
    \end{pmatrix},
    \quad\mbox{and}\quad
    \mat{B} = \begin{pmatrix}575 & 0\\
      0 & 20
    \end{pmatrix},
  \end{equation}
  and
  \begin{equation}
    \mat{A}\mat{B}^{-1} = \begin{pmatrix}280/575 & 12/20\\
      120/575 & 8/20
    \end{pmatrix}.
  \end{equation}
  We can find the eigenvalues using the characteristic polynomial
  \begin{equation}
p(\lambda) = \left(\frac{280}{575}-\lambda\right)\left(\frac{8}{20}-\lambda\right)-\frac{12}{20}\cdot\frac{120}{575},
  \end{equation}
  which has solutions $\lambda_{1}=2/23$ and $\lambda_{2}=4/5$. These correspond
  to $1+r=23/2$ and $1+r=5/4$ --- or $r=1050\%$ and $r=25\%$, respectively.
  The price vectors are
  \begin{equation}
\vec{p}_{1} = p_{w}\begin{pmatrix}1\\-115/6
\end{pmatrix},\quad\mbox{and}\quad\vec{p}_{2}= p_{w}\begin{pmatrix}1\\15
\end{pmatrix}.
  \end{equation}
  The viable price vector is $\vec{p}_{2}$ where the price of 1 ton of iron
  $p_{i}$ is equal to the price of 15 quarters of wheat, $p_{i}=15p_{w}$.
\end{example}


\begin{example}[{Sraffa~\cite[see \S25]{sraffa}}]
  Consider the more complicated equations of production:
  \begin{subequations}
    \begin{align}
      200~\mbox{qr. wheat} + 40~\mbox{t. iron} + 40~\mbox{t. coal} &\to 480~\mbox{qr. wheat}\\
      60~\mbox{qr. wheat} + 90~\mbox{t. iron} + 120~\mbox{t. coal} &\to 180~\mbox{t. iron}\\
      150~\mbox{qr. wheat} + 50~\mbox{t. iron} + 125~\mbox{t. coal} &\to 450~\mbox{t. coal}
    \end{align}
  \end{subequations}
  Observe the inputs are 410 qr wheat, 285 tons coal, 180 tons iron. We
  can compute the product $\mat{A}\mat{B}^{-1}$ as
  \begin{equation}
\mat{A}\mat{B}^{-1} = \begin{pmatrix}
  (5/12) & (2/9) & (4/45)\\
  (1/8) & (1/2) & (4/15)\\
  (5/16) & (5/18) & (5/18)
\end{pmatrix}.
  \end{equation}
  This has characteristic polynomial
\begin{equation}
p(\lambda) = \frac{35}{1296} - \frac{\lambda}{3} + \frac{43}{36}\lambda^{2}-\lambda^{3}.
\end{equation}
  We find its eigenvalues are $\lambda_{1}=5/6$, $\lambda_{2}=7/36$, and
  $\lambda_{3}=1/6$ which correspond to rates of profit $r_{1}=1/5$,
  $r_{2}=29/7$, and $r_{3}=5$.%%  The rates of profit $r_{2}$ and $r_{3}$
%%   lead to negative prices, so $r_{1}$ is the economically realistic rate
%%   of profit. The exchange rate is
%%   \begin{equation}
%% 8~\mbox{qr.\ wheat} = 11~\mbox{t.\ iron} = 10~\mbox{t.\ coal}
%%   \end{equation}

  If we consider $r_{1}=1/5$ as the rate of profit, the first equation
\begin{subequations}
  \begin{equation}
(6/5)(200p_{w} + 40 p_{i} + 40 p_{c}) = 480p_{w}
  \end{equation}
  may be rewritten as
  \begin{equation}
40p_{w} + 8p_{i} + 8p_{c} = 80p_{w}
  \end{equation}
  which reduces to
  \begin{equation}\label{eq:appendix-sraffa:system-2:wheat-sector}
p_{i} + p_{c} = 5p_{w}.
  \end{equation}
\end{subequations}
Now if we look at the second equation,
\begin{subequations}
  \begin{equation}
(6/5)(60p_{w} + 90 p_{i} + 120 p_{c}) = 180p_{i}
  \end{equation}
  which simplifies to
  \begin{equation}
12p_{w} + 18p_{i} + 24p_{c} = 30p_{i},
  \end{equation}
  which gives us
  \begin{equation}
p_{w} + 2p_{c} = p_{i}.
  \end{equation}
\end{subequations}
But if we add $p_{c}$ to both sides,
\begin{equation}
p_{w} + 3p_{c} = p_{i} + p_{c},
\end{equation}
we can use Eq~\eqref{eq:appendix-sraffa:system-2:wheat-sector} to
rewrite the right-hand side as
\begin{equation}
p_{w} + 3p_{c} = 5p_{w},
\end{equation}
hence
\begin{equation}
\boxed{p_{c} = \frac{4}{3}p_{w}.}
\end{equation}
If we plug this back into Eq~\eqref{eq:appendix-sraffa:system-2:wheat-sector}
we find
\begin{equation}
\boxed{p_{i} = \frac{11}{3}p_{w}.}
\end{equation}
Or if we want it in one big equation
\begin{equation}
\boxed{11p_{c} = 4p_{i} = \frac{44}{3}p_{w}.}
\end{equation}

We briefly mention the other choices of the rate of profit leads to
negative prices. For example, if we took $r_{3}$, the first equation of
production gives us
\begin{subequations}
\begin{equation}
6(200p_{w} + 40p_{i} + 40p_{c}) = 480p_{w}
\end{equation}
which simplifies to
\begin{equation}
5p_{w} + p_{i} + p_{c} = 2p_{w}
\end{equation}
and thus
\begin{equation}
p_{i} + p_{c} = -3p_{w}.
\end{equation}
\end{subequations}
If $p_{w}>0$, then either $p_{i}<0$ or $p_{c}<0$. If $p_{w}<0$, then $p_{w}<0$.
Either way, we have negative prices.

For $r_{2}=29/7$, we would have
\begin{subequations}
\begin{equation}
(36/7)(200p_{w}+40p_{i}+40p_{c}) = 480p_{w}
\end{equation}
which reduces to
\begin{equation}
(36/7)(5p_{w}+p_{i}+p_{c}) = 12p_{w},
\end{equation}
or equivalently
\begin{equation}
35p_{w} + 7p_{i} + 7p_{c} = \frac{1}{3}p_{w}
\end{equation}
hence
\begin{equation}
7p_{i} + 7p_{c} = -\frac{104}{3}p_{w}.
\end{equation}
\end{subequations}
And we're in exactly the same situation as before, we must have a
negative price.
\end{example}

\N{Numerical values}
A lot of these examples have nice numerical values. But if we, for
example, reduced the surplus of the wheat sector to zero in the previous
example, giving us the equations of production:
  \begin{subequations}
    \begin{align}
      200~\mbox{qr. wheat} + 40~\mbox{t. iron} + 40~\mbox{t. coal} &\to 410~\mbox{qr. wheat}\\
      60~\mbox{qr. wheat} + 90~\mbox{t. iron} + 120~\mbox{t. coal} &\to 180~\mbox{t. iron}\\
      150~\mbox{qr. wheat} + 50~\mbox{t. iron} + 125~\mbox{t. coal} &\to 450~\mbox{t. coal}
    \end{align}
  \end{subequations}
  Then the matrix $\mat{A}\mat{B}^{-1}$ has characteristic polynomial
  \begin{equation}
p(\lambda) = \frac{35}{1107} - \frac{1679}{4428}\lambda + \frac{467}{369}\lambda^{2}-\lambda^{3}.
  \end{equation}
  The closed form expression for the eigenvalues is rather unpleasant
  and shockingly involves imaginary quantities
\begin{subequations}
\begin{align}
\lambda_{1} &= \frac{467}{1107}+\frac{252805}{2214 \sqrt[3]{118352413+738\I\sqrt{3946737549}}}
+\frac{\sqrt[3]{118352413+738\I\sqrt{3946737549}}}{2214}
\end{align}
\begin{multline}
\lambda_{2} = \frac{467}{1107}
-\frac{252805}{4428 \sqrt[3]{118352413+738 \I\sqrt{3946737549}}}
-\frac{252805\I}{1476 \sqrt{3} \sqrt[3]{118352413+738\I\sqrt{3946737549}}}\\
-\frac{\sqrt[3]{118352413+738\I\sqrt{3946737549}}}{4428}\
+\frac{\I\sqrt[3]{118352413+738\I\sqrt{3946737549}}}{1476 \sqrt{3}}
\end{multline}
\begin{multline}
  \lambda_{3} = \frac{467}{1107}
  -\frac{252805}{4428 \sqrt[3]{118352413+738\I\sqrt{3946737549}}}
  +\frac{252805\I}{1476 \sqrt{3}\sqrt[3]{118352413+738\I\sqrt{3946737549}}}\\
  -\frac{\sqrt[3]{118352413+738\I\sqrt{3946737549}}}{4428}
  -\frac{\I \sqrt[3]{118352413+738\I\sqrt{3946737549}}}{1476 \sqrt{3}}
\end{multline}
\end{subequations}
  There is no closed form expression for the eigenvalues of
$\mat{A}\mat{B}^{-1}$ \emph{as real numbers}. We need to use numerical
  approximations:
  \begin{subequations}
    \begin{align}
      \lambda_{1} &\approx0.872546\\
      \lambda_{2} &\approx0.147691\\
      \lambda_{3} &\approx0.245346.
    \end{align}
  \end{subequations}
  Only $\lambda_{1}$ corresponds to a rate of profit between 0 and 100\%
  (namely, a rate of profit $r_{1}\approx0.146072$). This gives the
  exchange rate of
  \begin{equation}
p_{i}(\lambda_{1})\approx2.85062p_{w},\quad\mbox{and}\quad
p_{c}(\lambda_{1})\approx1.09298p_{w}.
  \end{equation}
If we tried the other rates of profit, we would find 
  \begin{equation}
    p_{i}(\lambda_{2})\approx-3.2122p_{w},\quad\mbox{and}\quad
    p_{c}(\lambda_{2})\approx0.726998p_{w},
  \end{equation}
  which is implausible; and lastly
  \begin{equation}
    p_{i}(\lambda_{3})\approx-6.33286p_{w},\quad\mbox{and}\quad
    p_{c}(\lambda_{3})\approx2.84669p_{w}.
  \end{equation}
  Again, this last case is implausible.


  \N{Perturbing Coal surplus, supply and demand}
  Suppose we restore the surplus of wheat.
  If we parametrize the surplus of coal as a variable $\Delta c$, then the
  equations of production become:
  \begin{subequations}
    \begin{align}
      200~\mbox{qr. wheat} + 40~\mbox{t. iron} + 40~\mbox{t. coal} &\to 480~\mbox{qr. wheat}\\
      60~\mbox{qr. wheat} + 90~\mbox{t. iron} + 120~\mbox{t. coal} &\to 180~\mbox{t. iron}\\
      150~\mbox{qr. wheat} + 50~\mbox{t. iron} + 125~\mbox{t. coal} &\to (285+\Delta c)~\mbox{t. coal}
    \end{align}
  \end{subequations}
  We see that as $\Delta c$ increases, the value of coal (as measured in its
  exchange-rate with wheat) decreases. We perform these calculations,
  at increments of $\Delta c=10$ tons of coal, and find:
  \begin{center}
    \begin{tabular}{cccc}
      \toprule
      &\multicolumn{2}{c}{Value of 1 ton of coal}&\\
      \cmidrule(r){2-3}
    $\Delta c$ & $p_{c}$ in multiples of $p_{w}$ & $p_{c}$ in multiples of $p_{i}$&rate of profit\\ \midrule
 $0$ & $ 2.39536 p_{w}$ & $ 0.57506 p_{i}$ & $3.79928\%$\\
$10$ & $ 2.28951 p_{w}$ & $ 0.555297 p_{i}$ & $3.93706\%$\\
$20$ & $ 2.19192 p_{w}$ & $ 0.536871 p_{i}$ & $4.07416\%$\\
$30$ & $ 2.10173 p_{w}$ & $ 0.519652 p_{i}$ & $4.2106\%$\\
$40$ & $ 2.01814 p_{w}$ & $ 0.503524 p_{i}$ & $4.34639\%$\\
$50$ & $ 1.94051 p_{w}$ & $ 0.488387 p_{i}$ & $4.48152\%$\\
$60$ & $ 1.86823 p_{w}$ & $ 0.474151 p_{i}$ & $4.616\%$\\
$70$ & $ 1.80081 p_{w}$ & $ 0.460739 p_{i}$ & $4.74984\%$\\
$80$ & $ 1.73777 p_{w}$ & $ 0.448081 p_{i}$ & $4.88303\%$\\
$90$ & $ 1.67874 p_{w}$ & $ 0.436114 p_{i}$ & $5.0156\%$\\
$100$ & $ 1.62335 p_{w}$ & $ 0.424783 p_{i}$ & $5.14753\%$\\
    \bottomrule
  \end{tabular}
  \end{center}
  We see as the surplus of coal $c$ increases, its exchange rate with
  iron and wheat decreases. In other words, \emph{as supply increases, price decreases}.
  That is to say, the ``law of supply'' is an emergent phenomenon in
  Neo-Ricardian economics.

  The curious reader may experiment with the Mathematica code used to
  produce this table:
\begin{Verbatim}
inverse_profit[c_]:=Eigenvalues[{{200/480,40/180,40/(285+c)},
 {60/480,90/180,120/(285+c)},
 {150/480,50/180,125/(285+c)}}]

prices[c_,profit_]:=Reduce[{profit*(60 wheat + 90 iron + 120 coal) == 180 iron,
 profit*(150 wheat + 50 iron + 125 coal) == (285 + c) coal},
 {iron, coal}]

exchange[c_] := prices[c, 1/(inverse_profit[c])[[1]]]

Table[{10*c,
   N[exchange[10*c]][[2]][[2]],
   N[exchange[10*c]][[2]][[2]]/N[exchange[10*c]][[1]][[2]] iron,
   N[(1/inverse_profit[c] - 1)[[1]]},
 {c,0,10}]
\end{Verbatim}
The \verb#inverse_profit# produces a list of eigenvalues for
$\mat{A}\mat{B}^{-1}$, the first eigenvalue corresponds to the
economically significant solution. We then find the exchange-rate of
iron, wheat, and coal using \verb#prices#, and finally produce a
\verb#Table# of rows consisting of $\Delta c$, the price of 1 ton of
coal in terms of the value of wheat $p_{w}$, and the price of 1 ton of
coal in terms of the value of iron $p_{i}$.\vfill\eject
\begin{thebibliography}{99}
\footnotesize%  \small

\bibitem{adams1996:ex}
J.F.\ Adams,
\textit{Lectures on Exceptional Lie Groups}.
University of Chicago Press, 1996.

%\cite{Baez:2001dm}
\bibitem{Baez:2001dm}
John C.~Baez,
``The Octonions''.
\journal{Bull.Am.Math.Soc.} \textbf{39} (2002) 145--205
[erratum: \journal{Bull.Am.Math.Soc.} \textbf{42} (2005) 213]
{\tt\doi{10.1090/S0273-0979-01-00934-X}}
[\arXiv{math/0105155} [math.RA]].
%396 citations counted in INSPIRE as of 30 May 2023

%\cite{Ekins:1975yu}
\bibitem{Ekins:1975yu}
J.~M.~Ekins and J.~F.~Cornwell,
``Semisimple Real Subalgebras of Noncompact Semisimple Real Lie Algebras. 5.,''
\journal{Rept.Math.Phys.} \textbf{7} (1975) 167--203
{\tt\doi{10.1016/0034-4877(75)90026-9}}
%4 citations counted in INSPIRE as of 02 Jun 2023


%\cite{Figueroa-OFarrill:2007jcv}
\bibitem{Figueroa-OFarrill:2007jcv}
Jos\'e Figueroa-O'Farrill,
``A Geometric construction of the exceptional Lie algebras $F_{4}$ and $E_{8}$''.
\journal{Commun.Math.Phys.} \textbf{283} (2008) 663--674
{\tt\doi{10.1007/s00220-008-0581-7}}
[\arXiv{0706.2829} [math.DG]].
%14 citations counted in INSPIRE as of 02 Jun 2023

%\cite{Garling:2011zz}
\bibitem{Garling:2011zz}
D.~J.~H.~Garling,
\textit{Clifford Algebras: An Introduction}.
Cambridge University Press, 2011.
%4 citations counted in INSPIRE as of 02 Jun 2023

%\cite{Gogberashvili:2019ojg}
\bibitem{Gogberashvili:2019ojg}
Merab Gogberashvili and Alexandre Gurchumelia,
``Geometry of the Non-Compact $G(2)$''.
\journal{J.Geom.Phys.} \textbf{144} (2019) 308--313
{\tt\doi{10.1016/j.geomphys.2019.06.015}}
[\arXiv{1903.04888} [physics.gen-ph]].
%3 citations counted in INSPIRE as of 02 Jun 2023

%\cite{Gunaydin:2001bt}
\bibitem{Gunaydin:2001bt}
M.~Gunaydin, K.~Koepsell and H.~Nicolai,
``The Minimal unitary representation of $\mathtt{E}_{8(8)}$''.
\journal{Adv.Theor.Math.Phys.} \textbf{5} (2002) 923--946
{\tt\doi{10.4310/ATMP.2001.v5.n5.a3}}
[\arXiv{hep-th/0109005} [hep-th]].
%59 citations counted in INSPIRE as of 02 Jun 2023

\bibitem{1212.3182}
Aaron Wangberg, Tevian Dray,
``$E_{6}$, the Group: The structure of $\SL(3,\OO)$''.
\arXiv{1212.3182}

%\cite{Zhang:2011ym}
\bibitem{Zhang:2011ym}
R.B.~Zhang,
``Serre presentations of Lie superalgebras''.
[\arXiv{1101.3114} [math.RT]].
%4 citations counted in INSPIRE as of 30 May 2023

\end{thebibliography}

% Tits construction of the exceptional simple Lie algebras
% https://arxiv.org/abs/0907.3789




\end{document}
