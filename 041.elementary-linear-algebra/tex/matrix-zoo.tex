\section{Matrix Zoo}

We need to introduce the terminology first, before introducing notions
of matrix addition (or matrix multiplication).

\begin{definition}
Let $m$, $n$ be positive integers.
We define an $m$-by-$n$ \define{Matrix} $A$ to be a rectangular array of
$mn$ real [or complex] numbers arranged in $m$ horizontal \define{Rows}
and $n$ vertical \define{Columns}, written:
\begin{equation}
\mat{A}
= \left(\begin{array}{ccc>{\columncolor{olive!20}}ccc}
  a_{11} & a_{12} & \dots & a_{1j} & \dots & a_{1n} \\
a_{21} & a_{22} & \dots & a_{2j} & \dots & a_{2n}  \\
\vdots & \vdots & \ddots & \vdots & \ddots & \vdots \\
\rowcolor{olive!20}a_{i1} & a_{i2} & \dots & a_{ij} & \dots & a_{in} \\
\vdots & \vdots & \ddots & \vdots & \ddots & \vdots \\
a_{m1} & a_{m2} & \dots   & a_{mj} & \dots  & a_{mn} 
\end{array}\right)
\begin{array}{c}
  \\
  \gets i^{\text{th}}\mbox{\ row}
\end{array}
\end{equation}
We have highlighted the $i^{\text{th}}$ row, and the $j^{\text{th}}$ column.

If further $m=n$ (so we have an $n$-by-$n$ matrix), then we call it a
\define{Square Matrix}. In this case, we can refer to the \define{Main Diagonal} 
as the components $a_{1,1}$, $a_{2,2}$, \dots, $a_{n,n}$.
\end{definition}

\begin{remark}
We abbreviate ``$m$-by-$n$ matrix'' as ``$m\times n$ matrix''. The
pair of numbers $(m,n)$ is called the \define{Dimensions} of the matrix.

Also note, some people use parentheses in writing matrices, other people
use square brackets. Both are acceptable.
\end{remark}

\N{Matrix and Components}
If we want to refer to a generic component of an $m$-by-$n$ matrix $\mat{A}$,
we write $\mat{A}=(a_{ij})$ to indicate we will use $a_{ij}$ to refer to the
component found in the $i^{\text{th}}$ row and $j^{\text{th}}$ column.

Also, very important, if we have some $m$-by-$n$ matrix
$\mat{A}=(a_{i,j})$ and we wanted to refer to its component in row $r$,
column $c$, we may refer to that component by writing $(\mat{A})_{r,c}$.
This is, of course, the same as $a_{r,c}$, but it will be useful [much
later] to take a matrix and refer to certain components without
introducing $(a_{i,j})$ first.

\N{Example: Zero Matrix}
For any positive integers $m$, $n$, we have the $m\times n$ \define{Zero Matrix}
to be the matrix whose components are all zero. We denote this by
$\mat{0}$ or $0$
because it will play the analogous counterpart that $0\in\RR$ plays. 

\N{Example: Identity Matrix}
For any positive integer $n$, we have the $n\times n$ \define{Identity Matrix}
to be
\begin{equation}
  \mat{I}_{n} = (\delta_{i,j}) = \begin{cases}1 &\mbox{if }i=j,\\
    0 &\mbox{if }i\neq j
  \end{cases}
\end{equation}
where the components $\delta_{i,j}$ are referred to as the
\define{Kronecker Delta}. If $n$ is clear from context, we may suppress
the subscript and write $\mat I$ instead of $\mat{I}_{n}$. Some authors write
$\mathbf{1}$ for the identity matrix, because (as we will see) it is
analogous to the number $1\in\RR$.

\begin{definition}
A \define{Column Vector} refers to an $n\times1$ matrix.

A \define{Row Vector} refers to a $1\times m$ matrix.

A \define{Vector} may refer to either row vectors or column vectors; in
these notes, we will reserve the word ``vector'' specifically for
``column vector''.
\end{definition}

\begin{ddanger}
In vector calculus, we were quite cavalier about our notion of vectors. 
There a vector consisted of a \textsc{base point} and an ``arrow''
of some magnitude pointing in some direction from the base point. We
could freely move the base point around willy-nilly. In linear algebra,
we will be working with vectors, but they all share \emph{the same} base
point. 
\end{ddanger}

\begin{definition}
A square matrix $\mat{A}=(a_{i,j})$ for which $a_{i,j}=0$ when $i\neq j$ is
called a \define{Diagonal Matrix}. We may write
$\mat{A}=\diag(a_{1,1}, a_{2,2},\dots, a_{n,n})$ to indicate $\mat{A}$
is a diagonal matrix. 

If further $a_{i,i}=c$ for all $i$, then we call $\mat{A}$ a
\define{Scalar Matrix}. 
\end{definition}

\begin{remark}[``Scalars'']
The word ``scalar'' means ``number''. We motivated our diversion into
matrices by trying to create some gadget which resembles numbers. Scalar
matrices are the ``most number-like'' among matrices.
\end{remark}

\N{Examples}
\begin{enumerate}
\item 
The following is a diagonal matrix which is not a scalar matrix
\begin{subequations}
\begin{equation}
  \mat{A} = \begin{pmatrix}1 & 0\\
    0 & 2
  \end{pmatrix}.
\end{equation}
\item 
The following is not a diagonal matrix,
\begin{equation}
  \mat{J} = \begin{bmatrix}0 & -1\\
    1 & 0
  \end{bmatrix}.
\end{equation}
\item The following is a scalar matrix,
\begin{equation}
  \mat{B} = \begin{pmatrix}-\pi & 0\\
    0 & -\pi
  \end{pmatrix}.
\end{equation}
\item The identity matrix $\mat{I}_{n}$ is a scalar matrix.
\item The $n\times n$ zero matrix is a scalar matrix.
\end{subequations}
\end{enumerate}

\begin{definition}
Let $\mat{A}=[a_{i,j}]$ and $\mat{B}=[b_{i,j}]$ be two $m\times n$ matrices.
We say $\mat{A}$ and $\mat{B}$ are \define{Equal} if they have identical components
in identical positions, i.e., if for each $i=1,\dots,m$ and
$j=1,\dots,n$ we have $a_{i,j}=b_{i,j}$. We indicate the matrices are
equal by writing $\mat{A}=\mat{B}$.

If $\mat{A}$ and $\mat{B}$ are not equal, then we write $\mat{A}\neq\mat{B}$.
\end{definition}

\begin{remark}
We are technically introducing a new binary relation, \define{Matrix Equality}.
\end{remark}

\begin{remark}[Same dimensions]
If $\mat{A}$ and $\mat{B}$ do not have the same number of rows and columns, then
they cannot possibly be equal. In this case (if they have different
dimensions), we can still sensibly write $\mat{A}\neq\mat{B}$.
\end{remark}

\begin{definition}\label{defn:matrix-zoo:triangular}
We call an $m\times n$ matrix $\mat{A}=(a_{i,j})$:
\begin{enumerate}
\item \define{Upper Triangular} if $a_{i,j}=0$ for $i>j$, e.g., when
  $m=n$ it looks like:
  \[ \mat{A} = \begin{bmatrix}
    a_{1,1} & a_{1,2} & a_{1,3} & \dots & a_{1,n}\\
    0      & a_{2,2} & a_{2,3} & \dots & a_{2,n}\\
    0      & 0       & a_{3,3} & \dots & a_{3,n}\\
    \vdots & \vdots & \vdots & \ddots & \vdots\\
    0      & 0      & 0      & \dots  & a_{n,n}
  \end{bmatrix} \]
\item \define{Lower Triangular} if $a_{i,j}=0$ for $i<j$, e.g., when
  $m=n$ it looks like:
  \[ \mat{A} = \begin{bmatrix}
    a_{1,1} & 0      & 0       & \dots & 0\\
    a_{2,1} & a_{2,2} & 0       & \dots & 0\\
    a_{3,1} & a_{3,2} & a_{3,3} & \dots & 0\\
    \vdots & \vdots & \vdots & \ddots & \vdots\\
    a_{n,1} & a_{n,2} & a_{n,3} & \dots  & a_{n,n}
  \end{bmatrix} \]
\item If $\mat{A}$ is upper-triangular and the diagonal entries are all zero,
  then we call $\mat{A}$ \define{Strictly Upper Triangular}.
\item Similarly, if $\mat{A}$ is lower-triangular and the diagonal entries are
  all zero, then we call $\mat{A}$ \define{Strictly Lower Triangular}.
\end{enumerate}
\end{definition}

\N{Block matrix}
It is useful to draw horizontal and vertical lines (either dashed or
slid, it makes no difference except for typography), to partition a
matrix into blocks, permitting us to write
\begin{equation}
\mat{A} = \left[
    \begin{array}{c:c}
        \mat{B} & \mat{C} \\ \hdashline
        \mat{D} & \mat{E} 
    \end{array}
\right] = \left[
    \begin{array}{c|c}
        \mat{B} & \mat{C} \\ \hline
        \mat{D} & \mat{E} 
    \end{array}
\right].
\end{equation}
We could partition a matrix however we want, and we call the blocks
\define{Submatrices} of $\mat{A}$. Writing $\mat{A}$ in this manner is
called a \define{Block Decomposition} of $\mat{A}$.

\begin{example}
  Let us consider some $3\times 5$ matrix $\mat{A}$ whose components are
  partitioned into block form:
  \begin{equation}
    \mat{A} = \left[\begin{array}{cc:ccc}
      a_{11} & a_{12} & a_{13} & a_{14} & a_{15}\\
      a_{21} & a_{22} & a_{23} & a_{24} & a_{25}\\ \hdashline
      a_{31} & a_{32} & a_{33} & a_{34} & a_{35}
    \end{array}\right] = \left[
    \begin{array}{c:c}
        \mat{B} & \mat{C} \\ \hdashline
        \mat{D} & \mat{E} 
    \end{array}
\right]
  \end{equation}
  This partitions $\mat{A}$ into a $2\times 2$ matrix $\mat{B}$, a
  $2\times3$ matrix $\mat{C}$, a $1\times2$ matrix $\mat{D}$, and a
  $1\times 3$ matrix $\mat{E}$. We could also partition it in other
  ways, for example,
\begin{equation}
    \mat{A} = \left[\begin{array}{c:ccc:c}
      a_{11} & a_{12} & a_{13} & a_{14} & a_{15}\\
      a_{21} & a_{22} & a_{23} & a_{24} & a_{25}\\ \hdashline
      a_{31} & a_{32} & a_{33} & a_{34} & a_{35}
    \end{array}\right].
\end{equation}
\end{example}

\begin{exercise}
Can we have a (row or column) vector which is also a scalar matrix?
\end{exercise}