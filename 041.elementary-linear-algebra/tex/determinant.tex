\section{Determinant}

\M
Remember when we figured out the inverse of a $2\times 2$ matrix, in
Eq~\eqref{eq:augmented-matrix:inverse-of-2-by-2-matrix} there was a
funny common factor of $ad-bc$. That was odd.

But if we try the same thing with a $3\times 3$ matrix, we end up with
\begin{equation}
\mat{A}^{-1} = \begin{pmatrix}a & b & c\\
  d & e & f\\
  g & h & i
\end{pmatrix}^{-1} = \frac{1}{a(ei-fh)-b(di-fg)+c(dh-eg)}\begin{pmatrix}A & B & C\\
  D & E & F\\
  G & H & I
\end{pmatrix}
\end{equation}
where $A$, $B$, $C$, etc., are all some mess. We're focused on the
scalar factor out in front: if $b=c$ or $d=g=0$, then we recover a similar
formula as in the $2\times 2$ case. Similarly, if $a=b=0$, we recover a
similar factor.

There is a recurring pattern here where, for a general $n\times n$
matrix $\mat{A}$, there is a common factor when computing its inverse
\begin{equation}
\mat{A}^{-1} = \frac{1}{\mbox{(suspicious factor)}}\mat{B}.
\end{equation}
If we set $(n-1)$ entries in the first row to zero, we recover the
suspicious factor from the $(n-1)\times(n-1)$ matrix inverse.

This suspicious factor turns out to be extremely important.

\N{Geometric Intuition}
Think about $2$-dimensional space $\RR^{2}$. We could begin by examining
the unit square:
\begin{equation}
C^{2} = \{(x_{1},x_{2})\in\RR^{2}\mid 0\leq x_{j}\leq 1, j=1,2\}.
\end{equation}
We read the right-hand side as the set (demarcated by squiggle brackets
$\{\dots\}$) consisting of points in the plane $(x_{1},x_{2})\in\RR^{2}$
(``$\in$'' takes an object to its left and a set to its right) such that
(``$\mid$'' encodes ``such that'') $0\leq x_{j}\leq 1, j=1,2$ the
components $x_{1}$ and $x_{2}$ lie between $0$ and $1$ inclusive. This
notation is known as ``set-builder notation''. If the reader is
unfamiliar with it, then they should consult Appendix~\cite{section:appendix-sets}.

What is its area? This is very silly, everyone learns in High School it
is the product of the lengths of its sides,
\begin{equation}
\operatorname{Area}(C^{2})=1\times1=1.
\end{equation}

But now suppose we ``deform'' the square to form a parallelogram with
one vertex fixed at the origin. Then we have the two sides described by
endpoints located at $\vec{a}=(a_{1},a_{2})$ and
$\vec{b}=(b_{1},b_{2})$.

What is its area? We can recall from vector
calculus that it is the magnitude of the their cross product
$|\vec{a}\times\vec{b}|=|a_{1}b_{2}-a_{2}b_{1}|$.

Does it look familiar? It should: it's precisely the suspicious factor
in Eq~\eqref{eq:augmented-matrix:inverse-of-2-by-2-matrix}.

\M What about the case in $\RR^{3}$? What if we start with the unit cube
\begin{equation}
C^{3} = \{(x_{1},x_{2},x_{3})\in\RR^{3}\mid 0\leq x_{j}\leq 1, j=1,2,3\}.
\end{equation}
The analogous quantity of interest is now its \emph{volume}. What is the
volume of the cube with side length equal to $1$? We recall
\begin{equation}
\operatorname{Vol}(C^{3})=1\times1\times1=1.
\end{equation}
Very simple.

What if we deform the $(x_{1},x_{2})$ cross sections to be a
parallelogram with sides at endpoints $(a_{1},a_{2},x_{3})$ and
$(b_{1},b_{2},x_{3})$, like we did in the $\RR^{2}$ case? Intuitively, this is
``dragging'' a parallelogram in the $(x_{1},x_{2})$-plane ``up'' the
$x_{3}$-axis for a ways of 1 unit of length. What is the volume of this?
It turns out to be the area of the parallelogram multiplied by the
distance we drag it. Its volume would be
\begin{equation}
\operatorname{Vol}(P) = (a_{1}b_{2}-a_{2}b_{1})\times 1.
\end{equation}
This is not terribly surprising, so let us deform further.

Specifically, we deform the unit cube to be a parallelepiped $P$ with a
corner fixed at the origin. The endpoints for our parallelepepiped $P$ along
its length, width, and height would be $\vec{a}=(a_{1},a_{2},a_{3})$,
$\vec{b}=(b_{1},b_{2},b_{3})$ and $\vec{c}=(c_{1},c_{2},c_{3})$. What is
its volume? The trick is to cheat and rotate axes until $\vec{a}$ and
$\vec{b}$ live in the $(x_{1},x_{2})$ plane.

But if we don't know that, then we can form the unit normal to the face
formed by edges $\vec{a}$ and $\vec{b}$ by taking their cross-product,
\begin{equation}
\widehat{\vec{n}} = \frac{\vec{a}\times\vec{b}}{|\vec{a}\times\vec{b}|}.
\end{equation}
We find the height to be the projection of $\vec{c}$ onto this normal
vector
\begin{equation}
h = \vec{c}\cdot\widehat{\vec{n}}.
\end{equation}
We can find the base area $B$ of the parallelogram formed by $\vec{a}$
and $\vec{b}$ by
\begin{equation}
B = |\vec{a}\times\vec{b}|,
\end{equation}
and, as always, base times height yields volume:
\begin{equation}
\operatorname{Vol}(P) = Bh = |\vec{a}\times\vec{b}|\;\vec{c}\cdot\widehat{\vec{n}}.
\end{equation}
Great, but this doesn't seem to help much. We just unfold the definition
of the unit normal, and we find
\begin{equation}
\operatorname{Vol}(P) = |\vec{a}\times\vec{b}|\frac{\vec{c}\cdot(\vec{a}\times\vec{b})}{|\vec{a}\times\vec{b}|}
= \vec{c}\cdot(\vec{a}\times\vec{b}).
\end{equation}
What is this explicitly? Let's write it out:
\begin{equation}
\operatorname{Vol}(P) = c_{1}(a_{2}b_{3}-a_{3}b_{2}) - c_{2}(a_{1}b_{3}-a_{3}b_{1})
+ c_{3}(a_{1}b_{2}-a_{2}b_{1}).
\end{equation}
Does it look familiar? It should: it's the suspicious factor in the case
of finding the inverse for a $3\times 3$ matrix.

\M
In short, this suspicious factor --- the exact same quantity --- appears
in the volume of $n$-dimensional parallelograms, and when computing the
inverse for an $n\times n$ matrix. This is strange, because one
situation is geometry whereas the other situation is
algebra. ``Strange'' is not the correct word for it: ``Profound'' should
be employed.

\N{Problem: How to compute this ``suspicious factor''?}
Now that we've established the profundity of this ``suspicious factor'',
how exactly do we compute it? What properties does it have? Why does it
matter in linear algebra? And what should we call it?

\begin{definition}
Let $\mat{A}=(a_{i,j})$ be an $n\times n$ matrix.
We recursively define its \define{Determinant} to be a scalar, defined by:
\begin{enumerate}
\item if $n=1$, we just take $\det(\mat{A})=a_{1,1}$;
\item if $n=2$, we simply take
  \begin{equation}
    \det(\mat{A}) = \det\begin{pmatrix}a_{1,1} & a_{1,2}\\
    a_{2,1} & a_{2,2}
    \end{pmatrix} = a_{1,1}a_{2,2} - a_{2,1}a_{1,2};
  \end{equation}
\item for $n>2$, we recursively define it using the formula
  \begin{equation}
\det(\mat{A}) = \sum_{j=1}^{n}(-1)^{j+1}a_{1,j}\det(\mat{M}_{1,j})
  \end{equation}
  where $\mat{M}_{1,j}$ is called a \emph{minor} of $\mat{A}$, obtained
  from $\mat{A}$ by deleting its first row and its $j^{\text{th}}$
  column --- more generally $\mat{M}_{i,j}$ is obtained by deleting row
  $i$ and column $j$ from $\mat{A}$.
\end{enumerate}
\end{definition}

\begin{lemma}
Let $\mat{A}$ be an $n\times n$ matrix with a column or row consisting of
zeros. Then $\det(\mat{A})=0$.
\end{lemma}

\begin{theorem}
Let $\mat{T}=(t_{i,j})$ be a triangular $n\times n$ matrix (either
upper-triangular, or lower-triangular, it doesn't matter).
Then its determinant is just the product of diagonal entries
$\det(\mat{T}) = \prod_{j=1}^{n}t_{j,j}$.
\end{theorem}

\begin{proof}
  We do this by induction on $n$.

  \textbf{Base Case:} $n=2$, for upper-triangular matrices we find
  \begin{equation}
\det\begin{pmatrix}t_{1,1} & t_{1,2}\\0 & t_{2,2}
\end{pmatrix} = t_{1,1}t_{2,2} - t_{1,2}0 = t_{1,1}t_{2,2}.
  \end{equation}
  For lower-triangular matrices
  \begin{equation}
\det\begin{pmatrix}t_{1,1} & 0\\t_{2,1} & t_{2,2}
\end{pmatrix} = t_{1,1}t_{2,2} - 0t_{2,1} = t_{1,1}t_{2,2}.
  \end{equation}
  Hence we establish the base case.

  \textbf{Inductive Hypothesis:} We assume this is true for general $n$.

  \textbf{Inductive Case:} For the $n\to n+1$ case, we will examine two
  subcases. Subcase 1 is when $\mat{T}$ is upper-triangular, where we
  write the block components out:
  \begin{equation}
    \mat{T} = \left(\begin{array}{cc}
      t_{1,1} & \transpose{\vec{t}} \\
      \vec{0} & \mat{T}^{(n)}
    \end{array}\right)
  \end{equation}
  The inductive hypothesis assumed $\det(\mat{T}^{(n)}$ is the product
  of diagonal components. We see that the minors $\mat{M}_{1,i}$ has a
  zero column for $i>1$. Hence only the $i=1$ minor contributes to the
  determinant, giving us
  \begin{subequations}
\begin{calculation}
    \det(\mat{T})
\step{definition of determinant}
    \sum^{n+1}_{i=1}(-1)^{i+1}t_{1,i}\det(\mat{M}_{1,i})
\step{pulling out the first term from the sum}
    t_{1,1}\det(\mat{M}_{1,1}) + \sum^{n+1}_{i=2}(-1)^{i+1}t_{1,i}\det(\mat{M}_{1,i})
\step{but $\det(\mat{M}_{1,i})=0$ for $i>1$}
    t_{1,1}\det(\mat{M}_{1,1}) + \sum^{n+1}_{i=2}(-1)^{i+1}t_{1,i}\cdot0
\step{the sum of zeros is zero}
    t_{1,1}\det(\mat{M}_{1,1}) + 0
\step{arithmetic}
    t_{1,1}\det(\mat{M}_{1,1})
\step{the minor $\mat{M}_{1,1} = \mat{T}^{(n)}$}
    t_{1,1}\det(\mat{T}^{(n)})
\step{inductive hypothesis}
    t_{1,1}\prod^{n+1}_{j=2}t_{j,j}
\step{associativity of multiplication}
    \prod^{n+1}_{j=1}t_{j,j}.
\end{calculation}
\end{subequations}
Hence we conclude for upper-triangular matrices, its determinant is
just the product of diagonal components.

\textsc{Subcase 2: Lower-triangular matrix.}
The reasoning for lower-triangular matrices is similar. Its block
components would look like
\begin{equation}
    \mat{T} = \left(\begin{array}{cc}
      t_{1,1} & \transpose{\vec{0}} \\
      \vec{t} & \mat{T}^{(n)}
    \end{array}\right)
  \end{equation}
The steps would be exactly the same, but this time because $t_{1,j}=0$
for $j>1$.
\end{proof}

\N{Levi-Civita Symbol}
It is an unfortunate fact that the definition we have given for the
determinant, while useful for \emph{performing calculations}, cannot
easily be used to prove properties concerning the
determinant. Consequently, we introduce an equivalent definition using a
terrifying quantity known as the Levi--Civita symbol. Its terror stems
from having multiple indices, but do not worry: it is used for book-keeping.

Let us define the Levi--Civita symbol for three-dimensions as
$\epsilon_{i,j,k}$ such that
\begin{enumerate}
\item $\epsilon_{1,2,3}=+1$, and
\item swapping two adjacent indices costs us a sign: $\epsilon_{j,i,k}=-\epsilon_{i,j,k}=\epsilon_{i,k,j}$;
$\epsilon_{k,i,j}=\epsilon_{i,j,k}=\epsilon{j,k,i}$.
\end{enumerate}
As a consequence of the second property, any repeated index will
correspond to a zero entry: $\epsilon_{i,i,k}=0$, $\epsilon_{i,j,j}=0$,
and so on. The Levi--Civita symbol in $n$ indices has analogous
properties:
\begin{enumerate}
\item $\epsilon_{1,2,3,\dots,n-1,n}=+1$, and
\item swapping two adjacent indices costs us a sign:
  $\epsilon_{i_{1},\dots,i_{j},i_{j+1},\dots,i_{n}}=-\epsilon_{i_{1},\dots,i_{j+1},i_{j},\dots,i_{n}}$.
\end{enumerate}
Yes, we have subscripts \emph{on our subscripts} --- we've indexed the
indexing variables!

The determinant in three-dimensions for matrix $\mat{A}=(a_{i,j})$ is then
\begin{equation}\label{eq:determinant:levi-civita:three-dim}
\boxed{\det(\mat{A}) = \sum^{3}_{i=1}\sum^{3}_{j=1}\sum^{3}_{k=1}
\epsilon_{i,j,k}a_{1,i}a_{2,j}a_{3,k}.}
\end{equation}
We can see this by performing the sums when $i=1$, we get
\begin{equation}
\sum^{3}_{j=1}\sum^{3}_{k=1}\epsilon_{1,j,k}a_{1,1}a_{2,j}a_{3,k}
=a_{1,1}\left(\sum^{3}_{k=1}\epsilon_{1,2,k}a_{2,2}a_{3,k}+\epsilon_{1,3,k}a_{2,3}a_{3,k}\right)
\end{equation}
But the antisymmetry property forces us to have $k=3$ in the first term
and $k=2$ in the second term as the only nonzero contribution in the
right-hand side:
\begin{equation}
\sum^{3}_{j=1}\sum^{3}_{k=1}\epsilon_{1,j,k}a_{1,1}a_{2,j}a_{3,k}
=a_{1,1}\left(\epsilon_{1,2,3}a_{2,2}a_{3,3}+\epsilon_{1,3,2}a_{2,3}a_{3,2}\right)
\end{equation}
Invoking antisymmetry to rearrange indices:
\begin{equation}
\sum^{3}_{j=1}\sum^{3}_{k=1}\epsilon_{1,j,k}a_{1,1}a_{2,j}a_{3,k}
=a_{1,1}\left(\epsilon_{1,2,3}a_{2,2}a_{3,3}-\epsilon_{1,2,3}a_{2,3}a_{3,2}\right)
\end{equation}
then invoking the first property simplifies the right-hand side:
\begin{equation}
\sum^{3}_{j=1}\sum^{3}_{k=1}\epsilon_{1,j,k}a_{1,1}a_{2,j}a_{3,k}
=a_{1,1}\left(a_{2,2}a_{3,3}-a_{2,3}a_{3,2}\right).
\end{equation}
This is precisely what we had as the coefficient to $a_{1,1}$ in the
familiar definition of the determinant. If we continue along, we will
find our new definition coincides with our old definition.

If we had a $4\times4$ matrix $\mat{A}=(a_{i,j})$, then we would have
\begin{equation}
\det(\mat{A}) = \sum^{4}_{i=1}a_{1,i}\det(M_{1,i})
\end{equation}
using our familiar definition. But we can rewrite $\det(M_{1,i})$ using
the Levi--Civita symbol as (remembering the minors are $3\times3$ matrices):
\begin{equation}
\det(\mat{A}) =
\sum^{4}_{i=1}(-1)^{i+1}a_{1,i}\left(\sum^{3}_{j=1}\sum^{3}_{k=1}\sum^{3}_{\ell=1}\epsilon_{j,k,\ell}(M_{1,j})_{2,j}(M_{1,k})_{3,k}(M_{1,\ell})_{4,\ell}\right),
\end{equation}
or by suppressing the column $i$ explicitly and reindexing accordingly:
\begin{equation}
\det(\mat{A}) =
\sum^{4}_{i=1}(-1)^{i+1}a_{1,i}\left(\sum^{4}_{\stackrel{j=1}{j\neq i}}\sum^{4}_{\stackrel{k=1}{k\neq i}}\sum^{4}_{\stackrel{\ell=1}{\ell\neq i}}\epsilon_{j,k,\ell}a_{2,j}a_{3,k}a_{4,\ell}\right),
\end{equation}
We observe the factor of $(-1)^{i+1}\epsilon_{j,k,\ell}$ could be
replaced by $\epsilon_{i,j,k,\ell}$ (which will also enforce the
conditions $i\neq j$ and $k\neq i$ and so on). Thus we find:
\begin{equation}
\det(\mat{A}) =
\sum^{4}_{i=1}\epsilon_{i,j,k,\ell}a_{1,i}\sum^{4}_{j=1}\sum^{4}_{k=1}\sum^{4}_{\ell=1}a_{2,j}a_{3,k}a_{4,\ell}.
\end{equation}
We invoke distributivity to tidy up the right-hand side as:
\begin{equation}
\boxed{\det(\mat{A}) =
\sum^{4}_{i=1}\sum^{4}_{j=1}\sum^{4}_{k=1}\sum^{4}_{\ell=1}\epsilon_{i,j,k,\ell}a_{1,i}a_{2,j}a_{3,k}a_{4,\ell}.}
\end{equation}
There seems to be a pattern emerging, when we examing the
three-dimensional case in Eq~\eqref{eq:determinant:levi-civita:three-dim}
and compare it to the four-dimensional case: we have a Levi--Civita
symbol with $n$ indices, and $n$ factors of $a_{1,i_{1}}a_{2,i_{2}}(\cdots)a_{n,i_{n}}$.

Thus we could argue, for any $n\times n$ matrix $\mat{A}=(a_{i,j})$, we have:
\begin{equation}\label{eq:determinant:using-levi-civita}
\boxed{\det(\mat{A}) = \sum^{n}_{i_{1}=1}\sum^{n}_{i_{2}=1}\cdots\sum^{n}_{i_{n}=1}\epsilon_{i_{1},i_{2},\dots,i_{n}}a_{1,i_{1}}a_{2,i_{2}}(\cdots)a_{n,i_{n}}.}
\end{equation}
We've shown this is true for $n=3$ and $n=4$, and we've shown how $n=4$
boils down to $n=3$. The general argument is similar, we would argue by
induction --- our base case has been established, we just need to prove
the inductive case $n+1$ in terms of the ``arbitrary $n$'' inductive
hypothesis. But this is precisely what we've done when moving from $n=3$
to $n=4$. The only difference will be slight, writing
\begin{align}
    \det\mat{A} &= \sum^{n+1}_{i_{1}=1}(-1)^{i_{1}+1}a_{1,i_{1}}\det(\mat{M}_{1,i_{1}})\\
    &= \sum^{n+1}_{i_{1}=1}(-1)^{i_{1}+1}a_{1,i_{1}}\left(\sum^{n}_{\mathclap{i_{2}=1}}(\cdots)\sum^{n}_{\mathclap{i_{n+1}=1}}\epsilon_{i_{2},\dots,i_{n},i_{n+1}}(\mat{M}_{i_{1},i_{2}})_{2,i_{2}}(\cdots)(\mat{M}_{i_{1},i_{n}})_{n,i_{n}}(\mat{M}_{i_{1},i_{n+1}})_{n+1,i_{n+1}}\right).\nonumber
\end{align}
The argument is exactly the same: rewrite the minors by components and
explicitly enforce $i_{j}\neq i_{1}$ in the sums, then we would
replace the
$(-1)^{i_{1}+1}\epsilon_{i_{2},\dots,i_{n+1}}$ by
$\epsilon_{i_{1},i_{2},\dots,i_{n+1}}$, and find the $i_{j}\neq i_{1}$
conditions redundant (so we'd remove them),
then invoking distributivity to obtain Eq~\eqref{eq:determinant:using-levi-civita}.
We will therefore take Eq~\eqref{eq:determinant:using-levi-civita}
to be proven.

\begin{lemma}
The $n\times n$ identity matrix $\mat{I}$ has determinant 1.
\end{lemma}

\begin{proof}
  We find
\begin{calculation}
\det(\mat{I})
    \step{using Eq~\eqref{eq:determinant:using-levi-civita}}
\sum^{n}_{i_{1}=1}\dots\sum^{n}_{i_{n}=1}\epsilon_{i_{1},\dots,i_{n}}\delta_{1,i_{1}}\dots \delta_{n,i_{n}}
    \step{summing over $i_{n}$}
\sum^{n}_{i_{1}=1}\dots\sum^{n}_{i_{n-1}=1}\epsilon_{i_{1},\dots,i_{n-1},n}\delta_{1,i_{1}}\dots \delta_{n-1,i_{n-1}}
    \step{induction}
\sum^{n}_{i_{1}=1}\epsilon_{i_{1},2,\dots,n-1,n}\delta_{1,i_{1}}
    \step{defining property of the Kronecker--delta}
\epsilon_{1,2,\dots,n}
    \step{defining property of Levi--Civita}
+1
\end{calculation}
Hence we conclude, for any $n$, the $n\times n$ identity matrix has
determinant $\det(\mat{I})=1$.
\end{proof}

\begin{theorem}
If $\mat{D}$ is a diagonal $n\times n$ matrix
$\mat{D}=\diag(d_{1},\dots,d_{n})$, then its determinant is the product
of the diagonal entries
\begin{equation}
\det(\mat{D}) = \prod^{n}_{j=1}d_{j}.
\end{equation}
\end{theorem}

This theorem is important, its proof is lengthy and involved. There are
no tricks to learn from it, so its importance is just to ensure the
result is true.

\begin{proof}
  We find
  \begin{calculation}
\det(\mat{D})
    \step{Eq~\eqref{eq:determinant:using-levi-civita}}
\sum^{n}_{i_{1}=1}\dots\sum^{n}_{i_{n}=1}\epsilon_{i_{1},\dots,i_{n}}d_{1,i_{1}}\dots d_{n,i_{n}}
    \step{since $d_{1,i}=d_{1}\delta_{1,i}$, etc.}
\sum^{n}_{i_{1}=1}\dots\sum^{n}_{i_{n}=1}\epsilon_{i_{1},\dots,i_{n}}\prod^{n}_{j=1}d_{j}\delta_{j,i_{j}}
     \step{defining property of Kronecker-delta applied in each summation}
\epsilon_{1,2,\dots,n}\prod^{n}_{j=1}d_{j}
     \step{by definition of the Levi--Civita symbol}
(+1)\prod^{n}_{j=1}d_{j}.
  \end{calculation}
This concludes the proof.
\end{proof}

\begin{theorem}
For any $n\times n$ matrices $\mat{A}$, $\mat{B}$,
the determinant of their product is the product of their determinants
\begin{equation}
\det(\mat{A}\mat{B}) = \det(\mat{A})\det(\mat{B}).
\end{equation}
\end{theorem}

\begin{proof}
Let $\mat{C}=(c_{i,k})=\mat{A}\mat{B}$, $\mat{A}=(a_{i,j})$ and $\mat{B}=(b_{i,j})$.
We know from the definition of matrix multiplication
\begin{equation}\label{eq:determinant:pf-of-product:components-of-product}
c_{i,k} = \sum^{n}_{j=1}a_{i,j}b_{j,k}.
\end{equation}
We find
\begin{subequations}
\begin{calculation}
\det(\mat{C})
    \step{using Eq~\eqref{eq:determinant:using-levi-civita}}
\sum^{n}_{j_{1}=1}\cdots\sum^{n}_{j_{n}=1}\epsilon_{j_{1},\dots,j_{n}}c_{1,j_{1}}(\cdots)c_{n,j_{n}}
    \step{using the formula for $c_{i,k}$ from Eq~\eqref{eq:determinant:pf-of-product:components-of-product}}
\sum^{n}_{j_{1}=1}\cdots\sum^{n}_{j_{n}=1}\epsilon_{j_{1},\dots,j_{n}}\left(\sum^{n}_{k=1}a_{1,k}b_{k,j_{1}}\right)\left(\sum^{n}_{k=1}a_{2,k}b_{k,j_{2}}\right)
(\cdots)\left(\sum^{n}_{k=1}a_{n,k}b_{k,j_{n}}\right)
    \step{collapsing sums}
\sum^{n}_{\substack{j_{1}=1\\\svdots\\j_{n}=1}}\epsilon_{j_{1},\dots,j_{n}}\left(\sum^{n}_{k=1}a_{1,k}b_{k,j_{1}}\right)\left(\sum^{n}_{k=1}a_{2,k}b_{k,j_{2}}\right)
(\cdots)\left(\sum^{n}_{k=1}a_{n,k}b_{k,j_{n}}\right)
    \step{reindexing}
\sum^{n}_{\substack{j_{1}=1\\\svdots\\j_{n}=1}}\epsilon_{j_{1},\dots,j_{n}}\left(\sum^{n}_{i_{1}=1}a_{1,i_{1}}b_{i_{1},j_{1}}\right)\left(\sum^{n}_{i_{2}=1}a_{2,i_{2}}b_{i_{2},j_{2}}\right)
(\cdots)\left(\sum^{n}_{i_{n}=1}a_{n,i_{n}}b_{i_{n},j_{n}}\right)
    \step{distributivity}
\sum^{n}_{\substack{i_{1}=1\\\svdots\\i_{n}=1}}\sum^{n}_{\substack{j_{1}=1\\\svdots\\j_{n}=1}}\epsilon_{j_{1},\dots,j_{n}}a_{1,i_{1}}b_{i_{1},j_{1}}a_{2,i_{2}}b_{i_{2},j_{2}}(\cdots)a_{n,i_{n}}b_{i_{n},j_{n}}
\end{calculation}
\end{subequations}
If we start at the other end, we find
\begin{subequations}
\begin{calculation}
\det(\mat{A})\det(\mat{B})
    \step{using Eq~\eqref{eq:determinant:using-levi-civita}}
\left(\sum^{n}_{i_{1}=1}\cdots\sum^{n}_{i_{n}=1}\epsilon_{i_{1},\dots,i_{n}}a_{1,i_{1}}(\cdots)a_{n,i_{n}}\right)
\left(\sum^{n}_{j_{1}=1}\cdots\sum^{n}_{j_{n}=1}\epsilon_{j_{1},\dots,j_{n}}b_{1,j_{1}}(\cdots)b_{n,j_{n}}\right)
    \step{collapsing indices into a single summation symbol}
\left(\sum^{n}_{\substack{i_{1}=1\\\svdots\\i_{n}=1}}\epsilon_{i_{1},\dots,i_{n}}a_{1,i_{1}}(\cdots)a_{n,i_{n}}\right)
\left(\sum^{n}_{\substack{j_{1}=1\\\svdots\\j_{n}=1}}\epsilon_{j_{1},\dots,j_{n}}b_{1,j_{1}}(\cdots)b_{n,j_{n}}\right)
    \step{using distributivity}
\sum^{n}_{\substack{i_{1}=1\\\svdots\\i_{n}=1}}\sum^{n}_{\substack{j_{1}=1\\\svdots\\j_{n}=1}}\left(\epsilon_{i_{1},\dots,i_{n}}a_{1,i_{1}}(\cdots)a_{n,i_{n}}\epsilon_{j_{1},\dots,j_{n}}b_{1,j_{1}}(\cdots)b_{n,j_{n}}\right)
    \step{reindexing the $j$ indices}
\sum^{n}_{\substack{i_{1}=1\\\svdots\\i_{n}=1}}\sum^{n}_{\substack{j_{i_{1}}=1\\\svdots\\j_{i_{n}}=1}}%
\left(\epsilon_{i_{1},\dots,i_{n}}a_{1,i_{1}}(\cdots)a_{n,i_{n}}
\epsilon_{j_{i_{1}},\dots,j_{i_{n}}}b_{i_{1},j_{i_{1}}}(\cdots)b_{i_{n},j_{i_{n}}}\right)
\end{calculation}
We want to show that reindexing the $j_{i_{k}}$ indices as $j_{k}$ will
cost us a factor of the Levi--Civita symbol. Well, we would be applying
a permutation $\pi$ to the $j$ indices, which \emph{would} cost us a
Levi--Civita symbol.
\begin{calculation}
\sum^{n}_{\substack{i_{1}=1\\\svdots\\i_{n}=1}}\sum^{n}_{\substack{j_{i_{1}}=1\\\svdots\\j_{i_{n}}=1}}%
\left(\epsilon_{i_{1},\dots,i_{n}}a_{1,i_{1}}(\cdots)a_{n,i_{n}}
\epsilon_{j_{i_{1}},\dots,j_{i_{n}}}b_{i_{1},j_{i_{1}}}(\cdots)b_{i_{n},j_{i_{n}}}\right)
\step{permuting $j_{i_{k}}$ with $j_{k}$ costs us a Levi--Civita symbol}
\sum^{n}_{\substack{i_{1}=1\\\svdots\\i_{n}=1}}\sum^{n}_{\substack{j_{1}=1\\\svdots\\j_{n}=1}}%
\left(\epsilon_{i_{1},\dots,i_{n}}a_{1,i_{1}}(\cdots)a_{n,i_{n}}
\epsilon_{i_{1},\dots,i_{n}}\epsilon_{j_{1},\dots,j_{n}}b_{i_{1},j_{1}}(\cdots)b_{i_{n},j_{n}}\right)
\step{the $\epsilon_{j_{1},\dots,j_{n}}(\epsilon_{i_{1},\dots,i_{n}})^{2}$ is just $\epsilon_{j_{1},\dots,j_{n}}$}
\sum^{n}_{\substack{i_{1}=1\\\svdots\\i_{n}=1}}\sum^{n}_{\substack{j_{1}=1\\\svdots\\j_{n}=1}}%
\left(a_{1,i_{1}}(\cdots)a_{n,i_{n}}\epsilon_{j_{1},\dots,j_{n}}b_{i_{1},j_{1}}(\cdots)b_{i_{n},j_{n}}\right)
\step{associativity}
\sum^{n}_{\substack{i_{1}=1\\\svdots\\i_{n}=1}}\sum^{n}_{\substack{j_{1}=1\\\svdots\\j_{n}=1}}%
\epsilon_{j_{1},\dots,j_{n}}
a_{1,i_{1}}(\cdots)a_{n,i_{n}}b_{i_{1},j_{1}}(\cdots)b_{i_{n},j_{n}}
\end{calculation}
\end{subequations}
The fact we could replace $\epsilon_{i_{1},\dots,i_{n}}\epsilon_{i_{1},\dots,i_{n}}\epsilon_{j_{1},\dots,j_{n}}$
with $\epsilon_{j_{1},\dots,j_{n}}$ stems from the fact that we require
the $i_{1}\neq i_{2}\neq\dots\neq i_{n}$ to be distinct in the sum, but
this is also equivalent to demanding the $j_{1}\neq j_{2}\neq\dots\neq j_{n}$
be distinct; however, the $\epsilon_{j_{1},\dots,j_{n}}$ factor
guarantees this is always the case. So the $\epsilon_{i_{1},\dots,i_{n}}^{2}$
would contribute only its magnitude (i.e., a factor of $+1$), hence we
can drop it.
But we see, this is precisely what we worked out for $\det(\mat{A}\mat{B})$.
Hence the result follows.
\end{proof}

\begin{lemma}
Let $\mat{A}=(a_{i,j})$ be an $n\times n$ matrix, let $r\in\RR$ be some
number, and let $\mat{B}=(b_{i,j})$ be obtained from $\mat{A}$ by adding
$r$ times row $\mu$ to row $\nu$ $b_{i,j} = a_{i,j} + ra_{\mu,j}\delta_{i,\nu}$.
Then $\det(\mat{A})=\det(\mat{B})$.
\end{lemma}
\begin{proof}
We see that $\mat{B} = (\mat{I} + r\mat{E}_{\mu,\nu})\mat{A}$, where
$\mat{E}_{\mu\nu}$ has a single nonzero entry located at row $\mu$,
column $\nu$, which is equal to 1. Then
$\det(\mat{B}) = \det(\mat{I} + r\mat{E}_{\mu,\nu})\det(\mat{A})$ since
the determinant of products is the product of determinants.
But $\mat{I} + r\mat{E}_{\mu,\nu}$ is either upper-triangular (when
$\mu<\nu$) or lower-triangular (when $\mu>\nu$), and in both cases the
determinant would be just the product of the diagonal
$\det(\mat{I} + r\mat{E}_{\mu,\nu})=\det(\mat{I})$ and we determined
this is 1. Hence $\det(\mat{B})=(1)\det(\mat{A})$.
\end{proof}

\begin{theorem}
Let $\mat{A}$ be an $n\times n$ matrix such that row $i$ is a multiple
of row $j$. Then $\det(\mat{A})=0$.

Let $\mat{B}$ be an $n\times n$ matrix such that column $i$ is a
multiple of column $j$. Then $\det(\mat{B})=0$.
\end{theorem}
\begin{proof}
Since we can add multiples of rows (or columns) to other rows without
affecting the determinant, we see if we subtract a multiple of row $i$
from row $j$ in $\mat{A}$ we will have a row consisting of zeroes. This
obviously has determinant 0 (it was the first thing we proved after the
definition of determinant).

Similar reasoning holds for $\mat{B}$.
\end{proof}

\begin{lemma}
Let $\mat{S}(i,j)$ be the $n\times n$ matrix obtained from the identity
matrix, swapping row $i$ and row $j$. Then
$\det(\mat{S}(i,j)) = (-1)^{i-j}$.
\end{lemma}

\begin{proof}
Let us prove this for the case when $j=i+1$. The general case follows by
applying this particular case repeatedly.
In this case, our matrix would look, in block form, like
\begin{equation}
  \det(\mat{S}(i,i+1)) = \left(\begin{array}{c|cc|c}
    \mat{I} & 0  & 0 & 0\\\hline
    0 & 0 & 1 & 0\\
    0 & 1 & 0 & 0\\\hline
    0 & 0 & 0 & \mat{I}
  \end{array}\right)
\end{equation}
We then add row $i+1$ to row $i$
\begin{equation}
(\mat{I} + \mat{E}_{i+1,i})(\mat{S}(i,i+1)) = \left(\begin{array}{c|cc|c}
    \mat{I} & 0  & 0 & 0\\\hline
    0 & 1 & 1 & 0\\
    0 & 1 & 0 & 0\\\hline
    0 & 0 & 0 & \mat{I}
  \end{array}\right)
\end{equation}
Then we subtract row $i$ from row $i+1$:
\begin{equation}
(\mat{I} - \mat{E}_{i,i+1})(\mat{I} + \mat{E}_{i+1,i})(\mat{S}(i,i+1)) = \left(\begin{array}{c|cc|c}
    \mat{I} & 0  & 0 & 0\\\hline
    0 & 1 & 1 & 0\\
    0 & 0 & -1 & 0\\\hline
    0 & 0 & 0 & \mat{I}
  \end{array}\right)
\end{equation}
Since this is upper-triangular,its determinant is just the product of
diagonal entries:
\begin{equation}
\det\left((\mat{I} - \mat{E}_{i,i+1})(\mat{I} + \mat{E}_{i+1,i})(\mat{S}(i,i+1))\right)=-1.
\end{equation}
However, since adding a multiple of one row to another does not affect
the determinant, we find
\begin{equation}
\det\left((\mat{I} - \mat{E}_{i+1,i})(\mat{I} + \mat{E}_{i,i+1})\mat{S}(i,i+1)\right)=\det(\mat{S}(i,i+1)).
\end{equation}
Hence we establish the case when $j=i+1$.

This general reasoning holds when we restore $j$, namely that
\begin{equation}
\det\left((\mat{I} - \mat{E}_{i,j})(\mat{I} + \mat{E}_{j,i})(\mat{S}(i,j))\right)=-1,
\end{equation}
hence
\begin{equation}
\det(\mat{S}(i,j)) = -1.
\end{equation}
Precisely as desired.
\end{proof}

\begin{proposition}
Let $\mat{A}$ be an $n\times n$ matrix, let $\mat{B}$ be obtained by
swapping rows $i$ and $j$ in $\mat{A}$.
Then $\det(\mat{B})=(-1)^{i-j}\det(\mat{A})$.
\end{proposition}

\begin{proof}
We see that $\mat{B}=\mat{S}(i,j)\mat{A}$. Then by the determinant of
products is the product of determinants, we find
\begin{equation}
\det(\mat{B})=\det(\mat{S}(i,j))\det(\mat{A}).
\end{equation}
We just proved $\det(\mat{S}(i,j))=(-1)^{i-j}$, so substituting this in
yields the result.
\end{proof}

\begin{theorem}
Let $\mat{A}$ be an invertible $n\times n$ matrix.
If $\det(\mat{A})\neq0$,
then $\det(\mat{A}^{-1}) = \left(\det(\mat{A})\right)^{-1}$.
\end{theorem}

\begin{proof}
We know $\mat{A}\mat{A}^{-1}=\mat{I}$, and then taking the determinant
of both sides yields
\begin{equation}
\det(\mat{A}\mat{A}^{-1}) = 1.
\end{equation}
We know the determinant of products is the product of determinants
\begin{equation}
\det(\mat{A}\mat{A}^{-1}) = \det(\mat{A})\det(\mat{A}^{-1}) = 1.
\end{equation}
Dividing both sides by $\det(\mat{A})$ yields the result.
\end{proof}



\begin{theorem}\label{thm:determinant:singular-matrices-have-zero-det}
  Let $\mat{A}$ be an $n\times n$ matrix.
  The determinant is zero if and only if $\mat{A}$ is singular;
  equivalently, the determinant is nonzero if and only if $\mat{A}$ is
  invertible.
\end{theorem}

\begin{proof}
Suppose $\mat{A}$ is invertible. Then
$\det(\mat{A}\mat{A}^{-1})=\det(\mat{I})=1\neq0$. In particular, this
means $\det(\mat{A})\neq0$.

The other direction is just the contrapositive: ``if $p$, then $q$'' is
logically equivalent to its contrapositive ``if not-$q$, then not-$p$''.
Here $q$ is ``$\det(\mat{A})\neq0$'', and $p$ is ``$\mat{A}$ is invertible''.
Hence the contrapositive is ``If $\det(\mat{A})=0$, then $\mat{A}$ is
not invertible''. And that's what we wanted to prove! So, we're done.
\end{proof}

\begin{theorem}
Let $\mat{A}$ be an $n\times n$ matrix.
The determinant of the transpose is the determinant of the original matrix:
\begin{equation*}
\det(\transpose{\mat{A}}) = \det(\mat{A}).
\end{equation*}
\end{theorem}

\begin{proof}
  Either $\mat{A}$ is invertible or not. If not, then it is singular and
  has zero determinant. Its transpose will be singular, and have zero
  determinant. Hence the result holds for singular matrices.

  For nonsingular matrices, we recall the LU-factorization
  $\mat{A}=\mat{L}\mat{U}$ where $\mat{U}$ has nonzero diagonal entries
  (because $\mat{A}$ is nonsingular) and $\mat{L}$ has 1 along its
  diagonal (hence $\det(\mat{L})=1$. Hence
  \begin{subequations}
  \begin{calculation}
    \det(\transpose{\mat{A}})
\step{LU decomposition}
    \det(\transpose{(\mat{L}\mat{U})})
\step{transpose of products is reverse product of transposes}
    \det(\transpose{\mat{U}}\transpose{\mat{L}})
\step{determinant of product is product of determinants}
    \det(\transpose{\mat{U}})\det(\transpose{\mat{L}})
\step{$\transpose{\mat{L}}$ is triangular with 1 on diagonal}
    \det(\transpose{\mat{U}})1 =\det(\transpose{\mat{U}})
\step{$\transpose{\mat{U}}$ is triangular with the same diagonal entries
  as $\mat{U}$}
  \det(\mat{U})
\step{$\mat{L}$ is triangular with 1 on diagonal}
  1\det(\mat{U}) = \det(\mat{L})\det(\mat{U})
\step{product of determinants is determinant of products}
  \det(\mat{L}\mat{U})
\step{by LU decomposition of $\mat{A}$}
  \det(\mat{A}).
  \end{calculation}
  \end{subequations}
  Hence the result.
\end{proof}

\begin{remark}
This proof is important, because it shows the basic gambit in linear
algebra: we take a matrix, and try factorizing it into a product of nice
matrices. Then we use this factorization to prove the desired result.
\end{remark}

\phantomsection
\subsection*{Exercises}
\addcontentsline{toc}{subsection}{Exercises}

\begin{exercise}
Let $\mat{M}$ be an $n\times n$ matrix.
We call $\mat{M}$ \define{Antisymmetric} if $\transpose{\mat{M}}=-\mat{M}$.
\begin{enumerate}
\item What would the diagonal values be for an antisymmetric matrix?
\item Write down a $3\times 3$ antisymmetric matrix.
\item What would the determinant of a $3\times3$ antisymmetric matrix
  be? What about a $4\times 4$ antisymmetric matrix?
\end{enumerate}
\end{exercise}

\begin{exercise}
Let $\mat{A}$ be an $n\times n$ matrix, let $c\neq 0$ be a nonzero number.
Is $\det(c\mat{A})$ a multiple of $\det(\mat{A})$? If so, what is that multiple?
If not, what is the determinant of a scalar matrix?
\end{exercise}

\begin{exercise}
Prove or find a counter-example: if $\vec{a}=(a_{j})$ and
$\vec{b}=(b_{k})$ are vectors in $\RR^{3}$, then their cross product has
components $\vec{a}\times\vec{b} = (\sum_{j}\sum_{k}\epsilon_{i,j,k}a_{j}b_{k})$.
\end{exercise}

(This gives one generalization of the cross-product to other dimensions
using the Levi--Civita symbol. We see the difficulty generalizing it to
higher dimensions: the Levi--Civita symbol in $n$ dimensions has $n$
indices. In other words, for $n\geq3$ dimensions, we need $n-1$ vectors
to form a cross-product.)

\begin{exercise}
We call an $n\times n$ matrix $\mat{N}$ \define{Nilpotent} if there is a
positive integer $m$ such that $\mat{N}^{m}=0$. If $\mat{N}$ is
nilpotent, then what is its determinant?
\end{exercise}

\begin{exercise}
We call an $n\times n$ matrix $\mat{A}$ \define{Idempotent} if
$\mat{A}^{2}=\mat{A}$. What constraints does this impose on $\det(\mat{A})$?
\end{exercise}

\begin{exercise}
Suppose $\mat{B}$ is an $n\times n$ matrix obtained from $\mat{A}$ by
multiplying row $i$ in $\mat{A}$ by a nonzero number $\gamma$. What is
$\det(\mat{B})$ in terms of $\det(\mat{A})$ and $\gamma$?

[Hint: they are not equal to each other in general, only for $\gamma=1$
  are the determinants equal.]
\end{exercise}

\begin{exercise}
Let $D$ be a function which eats in an $n\times n$ matrix $\mat{A}$
treated as $n$ column vectors $\mat{A}=(\vec{a}_{1},\vec{a}_{2}, \dots,\vec{a}_{n})$;
so $D(\mat{A}) = D(\vec{a}_{1},\vec{a}_{2},\dots,\vec{a}_{n})$ such that
\begin{enumerate}
\item $D(\mat{I})=1$ for the identity matrix
\item it is linear in every slot: for any $\vec{u}$, $\vec{v}\in\RR^{n}$,
  and every $a,b\in\RR$, we have
  $D(\vec{a}_{1},\dots,a\vec{u}+b\vec{v},\dots,\vec{a}_{n})=aD(\vec{a}_{1},\dots,\vec{u},\dots,\vec{a}_{n}) + bD(\vec{a}_{1},\dots,\vec{v},\dots,\vec{a}_{n})$
\item it is alternating: for any $i=1,2,\dots,n-1$, we have
  $D(\vec{a}_{1},\dots,\vec{a}_{i},\vec{a}_{i+1},\dots,\vec{a}_{n})=-D(\vec{a}_{1},\dots,\vec{a}_{i+1},\vec{a}_{i},\dots,\vec{a}_{n})$.
\end{enumerate}
Prove or find a counter-example: the function $D$ is just the determinant $D(\mat{A})=\det(\mat{A})$.
\end{exercise}

\subsection{Trace of a Matrix}

\begin{proposition}\label{prop:determinant:trace-as-infinitesimal-determinant}
Let $\mat{X}$ be an $n\times n$ matrix, let $\varepsilon>0$ be ``small''
(in the sense that we discard terms of order $\varepsilon^{2}$ or higher).
Then
\begin{equation}
\det(\mat{I} + \varepsilon\mat{X}) = 1 + \varepsilon\sum^{n}_{j=1}(\mat{X})_{j,j}.
\end{equation}
\end{proposition}

\begin{proof}
This follows from the fact that $1\gg\varepsilon^{2}$ and the product of
off-diagonal components would contribute a $\varepsilon^{2}$ term (or
some higher power of $\varepsilon$). So
\begin{equation}
\det(\mat{I} + \varepsilon\mat{X}) = \prod^{n}_{j=1}(1 + \varepsilon(\mat{X})_{j,j})+\mathcal{O}(\varepsilon^{2}).
\end{equation}
Expanding out the product, and keeping only the first-order
$\varepsilon$ terms gives us the result.
\end{proof}

\begin{remark}
From this perspective, the ``linear approximation'' to the determinant
``near the identity matrix'' is precisely the sum of diagonal entries in
the ``perturbation'' about the identity matrix. For this reason, it
deserves to be defined.
\end{remark}

\begin{definition}
Let $\mat{A}=(a_{i,j})$ be an $n\times n$ matrix.
We define its \define{Trace} to be the sum of its diagonal components
\begin{equation}
\tr(A) = \sum^{n}_{j=1}a_{j,j}.
\end{equation}
\end{definition}

\begin{proposition}
The trace of the transpose is the trace of the original matrix,
$\tr(\transpose{\mat{A}})=\tr(\mat{A})$.
\end{proposition}
\begin{proof}
Let $\mat{B}=\transpose{\mat{A}}$. Then its components would be
$(\mat{B})_{i,j}=(\mat{A})_{j,i}$ and in particular the diagonal
components are the same. So the sum of the diagonal components would be
equal.
\end{proof}

\begin{proposition}
Let $\mat{A}$ be an $n\times n$ matrix, let $c$ be some number.
Then $\tr(c\mat{A})=c\tr(\mat{A})$.
\end{proposition}

\begin{proposition}
Let $\mat{A}$ be an $m\times n$ matrix, let $\mat{B}$ be an $n\times m$ matrix.
Then $\tr(\mat{A}\mat{B})=\tr(\mat{B}\mat{A})$.
\end{proposition}

\begin{corollary}[Cyclic property]
Let $\mat{A}_{1}$, \dots, $\mat{A}_{k}$ be matrices of appropriate
dimension. Then $\tr(\mat{A}_{1}\mat{A}_{2}\dots\mat{A}_{k})=\tr(\mat{A}_{2}\dots\mat{A}_{k}\mat{A}_{1})$.
\end{corollary}

\begin{proof}
  We use associativity of matrix multiplication to write
  \begin{equation}
\tr(\mat{A}_{1}\mat{A}_{2}\dots\mat{A}_{k}) = \tr(\mat{A}_{1}(\mat{A}_{2}\dots\mat{A}_{k})),
  \end{equation}
  and then using the previous proposition we have
  \begin{equation}
\tr(\mat{A}_{1}(\mat{A}_{2}\dots\mat{A}_{k})) = \tr((\mat{A}_{2}\dots\mat{A}_{k})\mat{A}_{1}).
  \end{equation}
  Invoking associativity of matrix multiplication again yields the result.
\end{proof}

\begin{proposition}
Let $\mat{A}$, $\mat{B}$ be $n\times n$ matrices.
Then $\tr(\mat{A} + \mat{B}) = \tr(\mat{A}) + \tr(\mat{B})$.
\end{proposition}


\phantomsection
\subsection*{Exercises}
\addcontentsline{toc}{subsection}{Exercises}

\begin{exercise}
Prove or find a counter-example: if $\mat{A}$ is an invertible $n\times n$
matrix, $0<\varepsilon\ll1$ is a small real number, and $\mat{B}$ is an
arbitrary $n\times n$ matrix, then
$\det(\mat{A} + \varepsilon\mat{B})\approx \det(\mat{A})(1 + \varepsilon\tr(\mat{A}^{-1}\mat{B}))$
(plus higher order corrections in $\varepsilon$).
\end{exercise}

\begin{exercise}
If $\mat{A}$ is an $n\times n$ matrix such that $\transpose{\mat{A}}=\mat{A}^{-1}$,
then what values could $\det(\mat{A})$ be?
\end{exercise}

\begin{exercise}
In abstract algebra, the Vandermonde determinant is useful for studying
the roots of polynomials. The Vandermonde determinant in three variables
is given by the determinant
\[ \det\begin{pmatrix}
1     & 1     & 1 \\
a     & b     & c\\
a^{2} & b^{2} & c^{2}
\end{pmatrix}=(a-b)(b-c)(c-a). \]
Prove this formula actually holds. [Hint: expand both sides
  independently, then show they are equal to each other.]
\end{exercise}
