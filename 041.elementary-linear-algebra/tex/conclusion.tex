\section{Conclusion: The End?}

\epigraph{\itshape If stories have ends, then this story ends here.

I think, though it is not my field of specialisation, that some stories
end, but others carry on. They are eternal. They secretly carry on after
the story appears to be finished, continuing in silence. These stories
do not talk. They are never heard. I think my story may be like
that.}{Dan Abnett, \textit{Saturnine} (2020)}

\M
We've come quite a ways from ``solving a system of linear equations'',
wading through not just one, but two new abstractions: matrix algebras
and vector spaces. The destination for elementary linear algebra is
always the spectral theorem for real matrices --- a symmetric square
real matrix is diagonalizable.

The next topics of discussion would be, well, what do we do with a real
square matrix which is not symmetric? Can we get it into a
``diagonal-ish'' form?

\N{Other Number Fields}
The other direction for discussion is, we have been working with
\emph{real} matrices, \emph{real} vector spaces, everything using the
\emph{real} numbers. What if we used the complex numbers instead? We
need to abstract away the notion of the dot product and use something
similar, called the \emph{inner product}. The spectral theorem holds for
complex matrices, but with some slight changes.

If I ever get around to it, this is the subject of my notes on
\emph{intermediate linear algebra}: now, \emph{right now}, having
finished reading the previous sections, you know how to multiply the
hell out of matrices. We can use this knowledge as a ``toy model'' to
demonstrate how mathematicians present knowledge and figure out what
theorems to prove.

\N{Numerical Linear Algebra}
There is also a lot of nuance behind the calculations we do on the
computer, especially with linear algebra. This opens up an exciting new
field to us: numerical linear algebra. Furthermore, this may be studied
as a gateway to functional analysis. It is beautiful, and there are many
lovely good books on the subject.

\N{Functional Analysis}
If the reader knows real analysis,\footnote{For readers who do not know
real analysis but want to learn it, I recommend starting with Stephen
Abott's \textit{Understanding Analysis} as your first book.} then the reader will be aware that
many definitions could be applied to square matrices. Conversely, if we
view indices as function arguments, so $v\colon\{1,2,3\}\to\RR$ sends
$j\in\{1,2,3\}$ to a real number denoted $v_{j}$ --- well, what is
stopping us from changing the domain to, say, an open interval? Or all
of $\RR$? Or $\RR^{n}$? From this perspective, we can look at a
real-valued function $f(\vec{x})$ as a vector and $\vec{x}$ as a dummy
variable. What does the dot product look like in this case? Matrix
multiplication? Presumably, ``sums are replaced by integrals'', but are
there bases?

Since it's usually hard to reason about infinities, we could view
numerical linear algebra as ``finite functional analysis''. The two
complement each other like peanut butter and jelly.

\N{Goodbye}
Wow, look how far we've come! You could see your house from here. And
you didn't even complain once. Or maybe you did, I can't say, I can't
hear that well\dots

\vfill\textit{Dixi}.