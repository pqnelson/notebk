\section{Neo-Ricardian Economics}

\N{Matrix form of equations}
We can express an economy in terms of its input matrix $\mat{A}$, and
its output matrix $\mat{B}$
\begin{equation}
\mat{A}\vec{x}_{n} = \mat{B}\vec{x}_{n+1}
\end{equation}
where $\vec{x}_{n}$ is the stock of commodities at time $t=n$ production
cycles (after some initial production cycle), and $\vec{x}_{n+1}$ is the
stock of commodities \emph{after} the production cycle. We are trying to
find price vectors $\vec{p}$ such that
\begin{equation}
\mat{A}\vec{p} = \mat{B}\vec{p}.
\end{equation}
We assume that, for each sector $i$, the total inputs is no less than
the amount produced
\begin{equation}
\sum_{j=1}^{n}a_{i,j} \leq\sum_{j=1}^{n}b_{i,j}.
\end{equation}
Since usually $\mat{B}$ is diagonal, this simplifies to
\begin{equation*}
\sum_{j=1}^{n}a_{i,j} \leq b_{j,j}.
\end{equation*}
The trick is to first introduce a new vector
\begin{equation}
\vec{q} = \mat{B}\vec{p}\implies \vec{p}=\mat{B}^{-1}\vec{q},
\end{equation}
then rewrite our problem as
\begin{equation}
\mat{A}\vec{p} = \mat{B}\vec{p}\iff \mat{A}\mat{B}^{-1}\vec{q} = \vec{q}.
\end{equation}
For a subsistence economy, when no industry sector has an output, this
is the situation we are trying to solve.

\N{Production with a Surplus}
When at least one industry sector produces more output than is needed
across the entire economy, then there is surplus produced. In this case,
there is a rate of profit $r$, and we are trying to solve the system of
equations:
\begin{equation}
(1+r)\mat{A}\vec{p} = \mat{B}\vec{p}.
\end{equation}
We do the same trick, by writing
\begin{equation}
\vec{q} = \mat{B}\vec{p},
\end{equation}
then plugging this into the equations for an economy with a surplus
gives us
\begin{equation}
(1+r)\mat{A}\vec{p} = \mat{B}\vec{p} \iff (1+r)\mat{A}\mat{B}^{-1}\vec{q}=\vec{q}.
\end{equation}
This is an eigenvalue problem! We divide both sides by $(1+r)$ to find
\begin{equation}
\mat{A}\mat{B}^{-1}\vec{q}=\frac{1}{1+r}\vec{q},
\end{equation}
where $\lambda=1/(1+r)$ is the eigenvalue and $\vec{q}$ is the
eigenvector.

\begin{example}[{Sraffa~\cite[see \S5]{sraffa}}]
  Consider the economy given by the equations of production
  \begin{subequations}
    \begin{align}
      280~\mbox{qr. wheat} + 12~\mbox{t. iron} &\to 575~\mbox{qr. wheat}\\
      120~\mbox{qr. wheat} + 8~\mbox{t. iron} &\to 20~\mbox{t. iron}
    \end{align}
  \end{subequations}
  We have
  \begin{equation}
    \mat{A} = \begin{pmatrix}280 & 12\\
      120 & 8
    \end{pmatrix},
    \quad\mbox{and}\quad
    \mat{B} = \begin{pmatrix}575 & 0\\
      0 & 20
    \end{pmatrix},
  \end{equation}
  and
  \begin{equation}
    \mat{A}\mat{B}^{-1} = \begin{pmatrix}280/575 & 12/20\\
      120/575 & 8/20
    \end{pmatrix}.
  \end{equation}
  We can find the eigenvalues using the characteristic polynomial
  \begin{equation}
p(\lambda) = \left(\frac{280}{575}-\lambda\right)\left(\frac{8}{20}-\lambda\right)-\frac{12}{20}\cdot\frac{120}{575},
  \end{equation}
  which has solutions $\lambda_{1}=2/23$ and $\lambda_{2}=4/5$. These correspond
  to $1+r=23/2$ and $1+r=5/4$ --- or $r=1050\%$ and $r=25\%$, respectively.
  The price vectors are
  \begin{equation}
\vec{p}_{1} = p_{w}\begin{pmatrix}1\\-115/6
\end{pmatrix},\quad\mbox{and}\quad\vec{p}_{2}= p_{w}\begin{pmatrix}1\\15
\end{pmatrix}.
  \end{equation}
  The viable price vector is $\vec{p}_{2}$ where the price of 1 ton of iron
  $p_{i}$ is equal to the price of 15 quarters of wheat, $p_{i}=15p_{w}$.
\end{example}


\begin{example}[{Sraffa~\cite[see \S25]{sraffa}}]
  Consider the more complicated equations of production:
  \begin{subequations}
    \begin{align}
      200~\mbox{qr. wheat} + 40~\mbox{t. iron} + 40~\mbox{t. coal} &\to 480~\mbox{qr. wheat}\\
      60~\mbox{qr. wheat} + 90~\mbox{t. iron} + 120~\mbox{t. coal} &\to 180~\mbox{t. iron}\\
      150~\mbox{qr. wheat} + 50~\mbox{t. iron} + 125~\mbox{t. coal} &\to 450~\mbox{t. coal}
    \end{align}
  \end{subequations}
  Observe the inputs are 410 qr wheat, 285 tons coal, 180 tons iron. We
  can compute the product $\mat{A}\mat{B}^{-1}$ as
  \begin{equation}
\mat{A}\mat{B}^{-1} = \begin{pmatrix}
  (5/12) & (2/9) & (4/45)\\
  (1/8) & (1/2) & (4/15)\\
  (5/16) & (5/18) & (5/18)
\end{pmatrix}.
  \end{equation}
  This has characteristic polynomial
\begin{equation}
p(\lambda) = \frac{35}{1296} - \frac{\lambda}{3} + \frac{43}{36}\lambda^{2}-\lambda^{3}.
\end{equation}
  We find its eigenvalues are $\lambda_{1}=5/6$, $\lambda_{2}=7/36$, and
  $\lambda_{3}=1/6$ which correspond to rates of profit $r_{1}=1/5$,
  $r_{2}=29/7$, and $r_{3}=5$.%%  The rates of profit $r_{2}$ and $r_{3}$
%%   lead to negative prices, so $r_{1}$ is the economically realistic rate
%%   of profit. The exchange rate is
%%   \begin{equation}
%% 8~\mbox{qr.\ wheat} = 11~\mbox{t.\ iron} = 10~\mbox{t.\ coal}
%%   \end{equation}

  If we consider $r_{1}=1/5$ as the rate of profit, the first equation
\begin{subequations}
  \begin{equation}
(6/5)(200p_{w} + 40 p_{i} + 40 p_{c}) = 480p_{w}
  \end{equation}
  may be rewritten as
  \begin{equation}
40p_{w} + 8p_{i} + 8p_{c} = 80p_{w}
  \end{equation}
  which reduces to
  \begin{equation}\label{eq:appendix-sraffa:system-2:wheat-sector}
p_{i} + p_{c} = 5p_{w}.
  \end{equation}
\end{subequations}
Now if we look at the second equation,
\begin{subequations}
  \begin{equation}
(6/5)(60p_{w} + 90 p_{i} + 120 p_{c}) = 180p_{i}
  \end{equation}
  which simplifies to
  \begin{equation}
12p_{w} + 18p_{i} + 24p_{c} = 30p_{i},
  \end{equation}
  which gives us
  \begin{equation}
p_{w} + 2p_{c} = p_{i}.
  \end{equation}
\end{subequations}
But if we add $p_{c}$ to both sides,
\begin{equation}
p_{w} + 3p_{c} = p_{i} + p_{c},
\end{equation}
we can use Eq~\eqref{eq:appendix-sraffa:system-2:wheat-sector} to
rewrite the right-hand side as
\begin{equation}
p_{w} + 3p_{c} = 5p_{w},
\end{equation}
hence
\begin{equation}
\boxed{p_{c} = \frac{4}{3}p_{w}.}
\end{equation}
If we plug this back into Eq~\eqref{eq:appendix-sraffa:system-2:wheat-sector}
we find
\begin{equation}
\boxed{p_{i} = \frac{11}{3}p_{w}.}
\end{equation}
Or if we want it in one big equation
\begin{equation}
\boxed{11p_{c} = 4p_{i} = \frac{44}{3}p_{w}.}
\end{equation}

We briefly mention the other choices of the rate of profit leads to
negative prices. For example, if we took $r_{3}$, the first equation of
production gives us
\begin{subequations}
\begin{equation}
6(200p_{w} + 40p_{i} + 40p_{c}) = 480p_{w}
\end{equation}
which simplifies to
\begin{equation}
5p_{w} + p_{i} + p_{c} = 2p_{w}
\end{equation}
and thus
\begin{equation}
p_{i} + p_{c} = -3p_{w}.
\end{equation}
\end{subequations}
If $p_{w}>0$, then either $p_{i}<0$ or $p_{c}<0$. If $p_{w}<0$, then $p_{w}<0$.
Either way, we have negative prices.

For $r_{2}=29/7$, we would have
\begin{subequations}
\begin{equation}
(36/7)(200p_{w}+40p_{i}+40p_{c}) = 480p_{w}
\end{equation}
which reduces to
\begin{equation}
(36/7)(5p_{w}+p_{i}+p_{c}) = 12p_{w},
\end{equation}
or equivalently
\begin{equation}
35p_{w} + 7p_{i} + 7p_{c} = \frac{1}{3}p_{w}
\end{equation}
hence
\begin{equation}
7p_{i} + 7p_{c} = -\frac{104}{3}p_{w}.
\end{equation}
\end{subequations}
And we're in exactly the same situation as before, we must have a
negative price.
\end{example}

\N{Numerical values}
A lot of these examples have nice numerical values. But if we, for
example, reduced the surplus of the wheat sector to zero in the previous
example, then there is no closed form expression for the eigenvalues of
$\mat{A}\mat{B}^{-1}$. We need to use numerical approximations
$\lambda_{1}\approx 0.955884$,
$\lambda_{2}\approx 0.337684$, and
$\lambda_{3}\approx-0.0279857$. The iron and coal sectors give us the
relations
\begin{subequations}
\begin{equation}
p_{i} = \frac{2(1+r)(18 + 7(1+r))}{108 - 84(1+r) + 7 (1+r)^{2}}p_{w}
\end{equation}
and
\begin{align}
p_{c} &= \frac{14}{72}(1+r)p_{i} + \frac{23-7r}{72}p_{w}\\
&= \frac{2(11 + 4r - 7r^{2})}{31 - 70r + 7r^{2}}p_{w}
\end{align}
\end{subequations}
which holds provided
\begin{equation}
31 - 70r + 7r^{2}\neq0.
\end{equation}
That is to say,
\begin{equation}
r\neq\frac{35\pm12\sqrt{7}}{7}=5\pm\frac{12}{\sqrt{7}}.
\end{equation}
We find the approximate prices, for
$\lambda_{1}=(1+r_{1})^{-1}\approx 0.955884$ [$r_{1}\approx 4.61515\%$] we have
$p_{i}(r_{1})\approx1.38421p_{w}$ and 
$p_{c}(r_{1})\approx0.633981p_{w}$;
for $\lambda_{2}\approx 0.337684$ [$r_{2}\approx196.135\%$], we find
$p_{i}(r_{2})\approx0.170988p_{w}$ and $p_{c}(r_{2})\approx0.131292 p_{w}$; and for
$\lambda_{3}\approx-0.0279857$ [$r_{3}\approx-3673.25\%$], we find
$p_{i}(r_{3})\approx -0.00903004p_{w}$ and
$p_{c}(r_{3})\approx-0.00922875p_{w}$.

If we demand, as output of wheat continuously increased from 410
quarters to 480 quarters, that the prices and rates of profit
continuously vary as well, then the there is n$\lambda_{1}$ corresponds to the
economically significant solution. This doesn't always work, leading to
a mathematical phenomena known as ``bifurcations''.

\N{Bifurcations}
Suppose we have continuously varied the output of the coal sector with a
parameter $c\in\RR$ such that $0\leq c$, so the economy's equations of
production look like:
  \begin{subequations}
    \begin{align}
      200~\mbox{qr. wheat} + 40~\mbox{t. iron} + 40~\mbox{t. coal} &\to 480~\mbox{qr. wheat}\\
      60~\mbox{qr. wheat} + 90~\mbox{t. iron} + 120~\mbox{t. coal} &\to 180~\mbox{t. iron}\\
      150~\mbox{qr. wheat} + 50~\mbox{t. iron} + 125~\mbox{t. coal} &\to (285+c)~\mbox{t. coal}
    \end{align}
  \end{subequations}
Then if we try to numerically solve the characteristic polynomial for
$\mat{A}\mat{B}(c)^{-1}$, there will be 1 real solution and 2 complex
solutions. The real solution leads to a rate of profit
$r(c=0)\approx0.03799=3.799\%$ and $r(c=450-285)=6$. This doesn't match
any rate of profit to the initial system in our second example.

What does this mean? As we ``continuously vary'' the amount of surplus
coal produced, there's a ``jump'' or ``discontinuity'' between the
situation with no coal surplus ($c=0$) and the situation we are given
($c=485-280$). This is called a \define{Bifurcation} and reflects some
complexity in the model (in the sense of chaos theory). Robert Vieanneau
researches this phenomena in Neo-Ricardian economics.
  
\N{Modeling supply shocks}
We can model supply shocks by replacing $\mat{B}$ with
$\mat{B}+\delta\mat{B}$ for some ``shock matrix'' $\delta\mat{B}$.

\begin{example}
If we consider a modification of the second example
  \begin{subequations}
    \begin{align}
      200~\mbox{qr. wheat} + 40~\mbox{t. iron} + 40~\mbox{t. coal} &\to 480~\mbox{qr. wheat}\\
      60~\mbox{qr. wheat} + 90~\mbox{t. iron} + 120~\mbox{t. coal} &\to 180~\mbox{t. iron}\\
      150~\mbox{qr. wheat} + 50~\mbox{t. iron} + 125~\mbox{t. coal} &\to (285+\delta)~\mbox{t. coal}
    \end{align}
  \end{subequations}
  where $\delta\geq0$ is the ``shock'' to the coal supply, we see that
  there are some rough upper bounds on the rate of profit from the coal
  sector:
  \begin{subequations}
    \begin{equation}
(1+r)(150p_{w} + 50p_{i} + 125p_{c}) = (285 + \delta)p_{c}
    \end{equation}
    has positive prices provided
    \begin{equation}
    (1+r)(125p_{c}) < (285 + \delta)p_{c}
    \end{equation}
    or equivalently
    \begin{equation}
r < 1.28 + \frac{\delta}{125}.
    \end{equation}
    The wheat sector induces a similar upper bound on the rate of profit
    to be no more than $1 + r < 12/5$ or $r < 7/5$, and the iron sector
    induces the upper bound $1 + r < 2$ or $r < 1$.
  \end{subequations}
\end{example}

