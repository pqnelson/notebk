\section{Set Theory in a Nutshell}\label{section:appendix-sets}

\M
Modern mathematics is based on [Zermelo--Frankel] set theory. We will
review the notation and properties here without proof. We also present
this as a ``naive set theory'' using words instead of dry logic notation.
The interested reader is encouraged to read Halmos' \textit{Naive Set Theory}
for a more in-depth treatment.

\N{``Definition''}
A set is a ``well-defined'' unordered
collection of ``stuff'' (possibly other sets, possibly numbers, or
matrices, or whatever). We define a set in terms of its members, and we
denote ``$x$ is a member of a set $A$'' by writing ``$x\in A$'' (or
rarely, ``$A\ni x$'' for ``$A$ contains element $x$'').

If we have an element $x$ which does not belong to $A$, we write this as
$x\notin A$.

Note: ``$\in$'' (and ``$\notin$'') take an object to its left, a set to
its right, and produces a proposition (or ``formula'').

\N{``Well-Definedness''}
Although we didn't specify what exactly a ``well-defined unordered collection''
means, we can loosely describe it by the law that there are no circular
chains of elements: so things like
$A_{1}\in A_{2}\in\dots\in A_{1}$ are illegal. It is grammatically fine,
but so is the sentence ``Colorless green ideas sleep furiously''.

A consequence to this is we cannot form the \emph{set} of all sets (why
would anyone want to?). Instead, we construct a ``bigger'' species of
collections known as ``classes'', whose members are sets. This leads us
to things like NBG axioms (or Morse--Kelly axioms) for set theory. In
every sense possible, it's beyond the scope of any concern involving sets,
especially our own.

\N{What's allowed as an element}
Any \emph{object} is allowed as an element: sets, functions, numbers,
matrices, vectors, spaces, points, etc. We can combine any of these
things together into a set --- there is no ``type restriction'' saying
we cannot combine a set of numbers with a set of matrices.

However, propositions, predicates, truth-hood, falsehood, etc., are not allowed.

If we were studying a language, its words could be elements of a
set. The word ``truth'' within the language may be an element of a
set. But this is different than \emph{truth}. This is the situation with
a finger pointing at the moon, and the actual moon itself.

\begin{remark}
It may be circular to say, ``Sets are collections of objects'' and
``Objects are whatever we can permit as elements of sets''\dots and
you'd be right. The general heuristic is, a proposition is something
which is true or false, a predicate is something which takes an object
[or several objects] and produces a proposition, and an object is
everything else.
\end{remark}

\begin{remark}[Sets are objects]
Sets are objects, of course. When discussing the syntax (``grammar'') of
the notation, we may specifically restrict inputs to be sets of some kind.
We did this by stating ``$\in$'' expects a set to its right (as opposed
to an arbitrary object is allowed to its left).
\end{remark}

\N{Notation} We distinguish predicates from functions by writing
predicates with square brackets $P[x]$ whereas functions use parentheses
$f(x)$. 

\N{Subsets}
We can encode the idea of one set being entirely contained in another as
follows: if every $a\in A$ satisfies $a\in B$, then $A$ is called a
\define{Subset} of $B$ and we write $A\subset B$. If further there
exists at least one $b\in B$ such that $b\notin A$, then we call $A$ a
\define{Proper Subset} of $B$ and indicate this by $A\propersubset B$.
Otherwise we sometimes call $B$ an \emph{improper subset} of $A$. The
reader should be careful, not all authors distinguish proper subsets
from improper subsets.

Again, the symbol ``$\subset$'' takes one set to its left, a set to its
right, and produces a proposition.

\begin{theorem}
Every set is a subset of itself: For every set $A$, $A\subset A$.
\end{theorem}

\begin{theorem}
If $A\subset B$ and $B\subset C$, then $A\subset C$.
\end{theorem}

For example: ``Every cat is a mammal, and every mammal is an animal, it
follows every cat is an animal''.

\N{Set Equality}
We call two sets $A$ and $B$ \define{Equal} if every element $a\in A$
also belongs to $B$, and if every element $b\in B$ also belongs to $A$.
In this case, we write $A=B$.

On the other hand, if there exists an $a\in A$ such that $a\notin B$,
or if there exists a $b\in B$ such that $b\notin A$, then we write
$A\neq B$ to indicate $A$ is not equal to $B$.

\begin{theorem}
Set equality is an equivalence relation, i.e., satisfies all of the
following:
\begin{enumerate}
\item Reflexivity: for any set $A$, $A=A$
\item Symmetry: for any set $A$ and $B$, if $A=B$, then $B=A$
\item Transitivity: for any sets $A$, $B$, $C$, if $A=B$ and $B=C$, then $A=C$.
\end{enumerate}
\end{theorem}

\begin{theorem}
Let $A$ and $B$ be sets.
Then $A=B$ if and only if $A\subset B$ and $B\subset A$.
\end{theorem}

\M
For a set consisting of finitely many elements, we can write the set
explicitly as a comma separated list of its elements sandwiched between
squiggle braces:
\begin{equation}
A = \{x_{1}, x_{2}, \dots, x_{n}\}.
\end{equation}
Usually this is tedious, so we try to write a description of sets in
other ways.

\N{Empty Set}
The set with zero elements is called the empty set, and denoted $\emptyset=\{\}$.
The empty set is always a subset of everything: for any set $A$, we have
$\emptyset\subset A$.

\N{Famous Sets}
We have the following notation for familiar sets:
\begin{enumerate}
\item $\NN=\{1,2,3,\dots\}$ for the natural numbers (i.e., positive integers)
\item $\ZZ=\{\dots,-2,-1,0,1,2,\dots\}$ for the integers
\item $\QQ$ for the rational numbers
\item $\RR$ for the real numbers
\item $\CC$ for the complex numbers
\end{enumerate}
We typically assume $\NN\subset\ZZ$, $\ZZ\subset\QQ$, $\QQ\subset\RR$,
$\RR\subset\CC$.

\N{Union (merging sets)}
If $A$ and $B$ are sets, then we can form a new set $A\cup B$ whose
elements are precisely those that belong either to $A$ or to $B$ (or both).
So every $a\in A$ also belongs to $a\in A\cup B$,
and every $b\in B$ also belongs to $b\in A\cup B$.

\begin{example}
We can write $\NN_{0}=\{0\}\cup\NN$ for the non-negative integers. We
see $\NN\subset\NN_{0}$ is a proper subset.
\end{example}

\N{Intersection}
If $A$ and $B$ are sets, then we can form the collection $A\cap B$ of elements
belonging to both $A$ and $B$: $x\in A\cap B$ if and only if $x\in A$
and $x\in B$.

When $A\cap B=\emptyset$, there are no elements that belong to
\emph{both} $A$ and $B$, then we call $A$ and $B$ \define{Disjoint}.

\N{Functions}
If $A$ and $B$ are sets, then we can define a \define{Function}
$f\colon A\to B$ to be such that for each $a\in A$ there is exactly one
$b\in B$ such that $b=f(a)$.

\N{Ordered Pair}
We can define an ordered pair $(a,b)$ to be such that $(a,b)=(x,y)$ if
and only if $a=x$ and $b=y$. We can generalize this idea from a pair to
a tuple of $n$ guys $(a_{1}, a_{2}, \dots, a_{n})$ defined such that
$(a_{1},a_{2},\dots,a_{n})=(b_{1},b_{2},\dots,b_{n})$ if and only if
$a_{1}=b_{1}$ and $a_{2}=b_{2}$ and \dots and $a_{n}=b_{n}$.

\N{Cartesian Product}
If we have sets $A$ and $B$, we can define the collection of ordered
pairs $(a,b)$ for each $a\in A$ and for each $b\in B$. This is precisely
the Cartesian product $A\times B$. We have $(a,b)\in A\times B$ for each
$a\in A$ and $b\in B$.

We can generalize this to as many factors as we want: the set of ordered triples
$A\times B\times C$, quadrouples $A\times B\times C\times D$, and so
on. This is a bit of an abuse of notation, but it's ok.

\N{Notation}
If $f\colon A\times B\to C$, instead of writing $f((a,b))$, we will
write $f(a,b)$. This is convention.

\N{Infix Notation}
Sometimes we introduce a function $f\colon A\times B\to C$ with the
understanding it is infixed, i.e., we will use it as $a~f~b$ instead of
$f(a,b)$. This happens a lot with binary operators. I mean, honestly,
who thinks $+(a,b)$ makes any sense?

\N{Composing Functions} If $f\colon A\to B$ and $g\colon B\to C$ are
functions, then we may ``feed'' the ``output'' of $f$ into $g$ and
produce a new function denoted $g\circ f$ (and we read it from right to left).
So for any $a\in A$, we have $(g\circ f)(a) = g\left(f(a)\right)$.

This is associative: for any $f\colon A\to B$, $g\colon B\to C$,
$h\colon C\to D$, we have $(h\circ g)\circ f = h\circ(g\circ f)$.

\N{Set Builder Notation}
If we want to form a subset of $A$ satisfying elements $a\in A$ some
property or formula $P[a]$, then we write this using the notation:
\begin{equation}
\{a\in A\mid P[a]\}
\end{equation}
and read this as:
\begin{itemize}
\item[``$\{$''] the set of
\item[``$a\in A$''] the elements $a\in A$
\item[``$\mid$''] such that
\item[``{$P[a]$}''] $a$ satisfies the property $P$, i.e., $P[a]$ is true
\item[``$\}$''] [breath].
\end{itemize}
If we have some function $f\colon A\to B$, we can write
\begin{equation}
\{f(a)\mid a\in A, P[a]\}
\end{equation}
or if there's ambiguity
\begin{equation}
\{f(a)\in B\mid a\in A, P[a]\}
\end{equation}
We read this as
\begin{itemize}
\item[``$\{$''] the set of
\item[``$f(a)$''] the result of $f$ applied to $a$
\item[``$\mid$''] such that
\item[``$a\in A$''] the set is generated by every $a\in A$
\item[``,''] and
\item[``{$P[a]$}''] $a$ satisfies the property $P$, i.e., $P[a]$ is true
\item[``$\}$''] [breath].
\end{itemize}
This notation is fairly standard. Some authors use colons instead of
vertical bars to indicate ``such that''; other authors use \emph{both}
interchangeably.

\N{Concluding Remarks}
There are alternative foundations to mathematics, but in practice it
doesn't matter much. There are various different axiomatizations to set
theory, most of whom concern themselves to incorporating classes (and
even \emph{bigger} collections) into the framework. It seems that set
theory is a ``kludge'' designed to work around the shortcomings of
first-order logic.\footnote{For example, quantifiers in first-order
logic only quantify over constants, not functions. But if we encode a
function as a set of ordered pairs $(x,f(x))$, then a ``function of sets'' is 
a constant and may be quantified over.}

Type theory is gaining attention, but it is quirky enough to avoid
discussing it much. It involves a very clever way to encode a
proposition as a type, the proof as a term. The disadvantage with type
theory lies with discussing ``heterogeneous'' collections, among other
constraints.

Higher-order logic is kind of a ``half way'' between type theory and set
theory. There are two types, one for propositions, the second for
``objects''.

The honest truth is that mathematicians work with a linear combination
of these three foundations without realizing it, or caring. In these
notes, we continue this time honored tradition.