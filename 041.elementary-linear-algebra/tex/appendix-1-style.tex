\section{Structured Derivations}
\M I am thinking about writing a text about elementary linear algebra
using structured derivations. Instead of writing a sequence of
equations, possibly with hints, like:
\begin{subequations}
\begin{align}
  A &= B \qquad\mbox{(hint 1)}\\
  &= C \qquad\mbox{(hint 2)}\\
  &= D \qquad\mbox{(hint 3)}
\end{align}
\end{subequations}
where it is unclear if ``hint 2'' refers to why $B=C$ follows from
$A=B$, or perhaps why $A=C$ follows from $A=B$; and what ``hint 3''
attempts to explain is anyone's guess. Instead, we would have:
\begin{calculation}
  A
  \step{hint why $A=B$}
  B
  \step*{hint why $B=C$}
  C % should be not numbered
  \step{hint why $B=D$}
  D
  \step[\sim]{hint why $D\sim E$}
  E
\end{calculation}
I have endeavoured to imitate the vertical spacing of equations (using
\verb#\displayskip# and \verb#\belowdisplayskip#), as well as the
horizontal spacing amount which the \verb#fleqn# document option would
use.
\begin{align*}
  A &= B \qquad\mbox{(hint 1)}\\
  &= C \qquad\mbox{(hint 2)} \\
  &= D \qquad\mbox{(hint 3)}
\end{align*}

