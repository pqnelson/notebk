\section{Vectors in \texorpdfstring{$\RR^{n}$}{Rn}}\label{section:vectors-in-r-n}

\epigraph{``There are vectors,'' Kit replied, ``and vectors. Over in Dr.\ Prandtl's shop, they're all straightforward lift and drift, velocity and so forth. You can draw pictures, of good old three-dimensional space if you like, or on the Complex plane, if Zhukovsky's Transformation is your glass of tea. Flights of arrows, teardrops. In Geheimrat Klein's shop, we were more used to expressing vectors without pictures, purely as an array of coefficients, no relation to anything physical, not even space itself, and writing them in any number of dimensions---according to Spectral Theory, up to infinity.''}{Thomas Pynchon, \textit{Against the Day} (2006)}

\M
We learn about vectors at university specifically so we could do vector
calculus. But we have only mentioned vectors in our notes on linear
algebra in passing as a particular ``species'' of matrices. Is this an
unfortunate collision of language? That is to say, are these two notions
distinct [not secretly the same] but unfortunately use the same term?

Let us review what a ``vector'' was for the real plane $\RR^{2}$.

\N{Real Number Line}
Let us recall what happens in the simplest case of all: the real number
line. The first thing we do is draw a line (in the Euclidean sense,
which extends in both directions infinitely far) and pick some point $O$
on the line:
\begin{center}
  \includegraphics{img/img.0}
\end{center}
We then pick a point $A\neq O$ on the line (which is not the same point
as $O$):
\begin{center}
  \includegraphics{img/img.1}
\end{center}
We take the distance between point $O$ and $A$ to be 1 unit. We then
construct the rest of the ``ticks'' along this line, and identify every
point with a number in the real numbers. Intuitively corresponding to
the picture:
\begin{center}
  \includegraphics{img/img.2}
\end{center}
How do we make this identification? We take a point $P$ on the line,
find its distance [from $O$ to $P$] as a multiple of the distance from
$O$ to $A$. In the case where, like point $Q$, it lies ``in the other
direction'', we identify point $Q$ with the negative real numbers by
dividing the distance between $O$ and $Q$ with the distance between $O$
and $A$ (and multiplying the result by $-1$). This is a real number
$x\in\RR$. Conversely, if $x\in\RR$, then we can identify with a point
on the line by dilating the line segment $\overline{OA}$ by $x$.

When we identify point $P$ on the line by the real number $x$, we call
$x$ the \define{Coordinate} of $P$. The point $O$ has coordinate $0$,
and the distance between points $P$ and $Q$ (who have coordinates
$x_{P}$ and $x_{Q}$, respectively) by the magnitude of the difference in
their coordinates $|x_{Q}-x_{P}|$.

\N{Constructing the Plane}
We can construct the plane by taking a line which passes through $O$ and
forms a $90^{\circ}$ angle with our real line. If we just ``plop'' down
such a line, we get:
\begin{center}
\includegraphics{img/img.3}
\end{center}
We can then form ``ticks'' along this new line, and call it the
\define{$y$ Axis} (our old line is the \define{$x$ Axis}):
\begin{center}
\includegraphics{img/img.4}
\end{center}
Instead of identifying a point in the plane with \emph{one} real number,
we now have a \emph{pair} of real numbers. These are obtained by
projecting onto the axes (the projection to the $x$-axis is in red, the
projection to the $y$-axis is in green):
\begin{center}
\includegraphics{img/img.5}
\end{center}
These give us points on the axes, which then give us coordinates $x_{P}$
and $y_{P}$ for those points on the axes. We combine these in an ordered
pair $(x_{P}, y_{P})$ and call them the \define{Coordinates} of $P$.
Conversely, given any pair of numbers $(x,y)\in\RR^{2}$, we can
construct a point in the plane by reversing the steps in this procedure.

\N{Vectors in the Plane}
Recall, we define a column 2-vector as a $2\times1$ matrix
\begin{equation}
\vec{v} = \begin{pmatrix}x_{1}\\x_{2}
\end{pmatrix}.
\end{equation}
We identify $\vec{p}$ with a point in the plane by constructing an
oriented line segment from $O$ (the point with coordinates $(0,0)$ in
the plane) and the point $P$ (with coordinates $(x_{1},x_{2})$). We
write this oriented line segment using the notation
$\overrightarrow{OP}$. We call $O$ the \define{Base Point} and $P$ the
\define{Endpoint} of $\overrightarrow{OP}$.
\begin{center}
\includegraphics{img/img.6}
\end{center}
The directed line segment $\overrightarrow{OP}$ has a \define{Direction} (given
by the angle formed with the positive $x_{1}$-axis) and a
\define{Magnitude} (given by its length).

We identify any directed line segment $\overrightarrow{OP}$ with base
point $O$ with a vector $\vec{v}$ by taking the components of $\vec{v}$
to be the coordinates of $P$ relative to base-point $O$.

Conversely, given any 2-vector $\vec{v}=(v_{1},v_{2})$, we identify it
with a directed line segment [in the plane with axes and units] from $O$
to $P$ where $P$ has coordinates $(v_{1},v_{2})$.

\begin{ddanger}\textsc{Caution:}
In physics, we often work with vectors with base point $Q$ and endpoint
$P$ in the plane (or in $\RR^{3}$ or wherever) and quite cavalierly
identify the oriented line segment $\overrightarrow{QP}$ with the line
segment of equal length and direction $\overrightarrow{OP'}$ located at
base-point $O$. This works because of the magic of $\RR^{n}$ being a
flat manifold. It doesn't work in general. In fact, in linear algebra,
we anchor all vectors to the same base point $O$.
\end{ddanger}

\begin{definition}
Let $\vec{v}=(v_{1},v_{2})$ be a vector in the plane. We define its
\define{Magnitude} by the non-negative real number
$\|\vec{v}\| = \sqrt{v_{1}^{2}+v_{2}^{2}}$.
\end{definition}

\N{Parallelogram Law}
When we have two vectors $\vec{u}$ and $\vec{v}$ in the plane (or in space), we can form a
parallelogram using vectors $\vec{u}$ and $\vec{v}$ as edges and their
shared base point as the origin. Their sum
is the vector to the opposite diagonal vertex from the origin:
\begin{center}
  \includegraphics{img/img.7}
\end{center}
The reader may verify the origin plus the two vectors give us three
vertices, and demanding a parallelogram gives us the remaining
vertex. Further, the coordinates of the remaining vertex may be obtained
by adding componentwise the coordinates of the endpoints for the vectors.

\begin{remark}
Fascinatingly, this ``parallelogram law'' is mistakenly attributed to
Pseudo-Aristotle, but recent scholarship shows this is a
misunderstanding. The first uses of the parallelogram law may be found
in Fermat (of ``Last Theorem'' fame) and Thomas Hobbes (the pessimistic ``rude,
short, and brutish'' philosopher). For further details, the reader is
invited to enjoy David Marshall Miller's
``The Parallelogram Rule from Pseudo-Aristotle to Newton''
\journal{Archive for History of Exact Sciences} \volume{71} (2017) pp.157--191.
\end{remark}

\N{Scalar Multiplication}
If we have a vector $\vec{v}$ corresponding to the oriented line segment
$\overrightarrow{OP}$ and a real number $r$, we may form the
scalar multiple $r\vec{v}$ by three cases:
\begin{enumerate}
\item Case 1 $r>0$: we just move the endpoint along the line passing
  through $O$ and $P$ to have a magnitude $r$ times greater than what
  the magnitude of $\overrightarrow{OP}$ is;
\item Case 2 $r=0$: we have the coordinates of the resulting vector be
  all zeroes.
\item Case 3 $r<0$: we find $Q$ the point along the line passing
  through $O$ and $P$ of the same magnitude but opposite direction of
  $\overrightarrow{OP}$, and multiply the magnitude of the line segment
  from $O$ to this new point $Q$ by $|r|>0$ to produce a point $R$ and
  identify $\overrightarrow{OR}$ with our new vector.
\end{enumerate}

\N{Consistency and Coherence Checks: Vector Subtraction}
We now may observe, for any vectors $\vec{u}$ and $\vec{v}$ in the
plane, that $\vec{u}-\vec{v}$ corresponds to the vector sum of $\vec{u}$
with the scalar multiple $-1\vec{v}$. Similarly, repeatedly adding a
vector $\vec{u}$ to itself $n\in\NN$ times gives us the scalar multiple
$n\vec{u}$.

In other words, scalar multiplication produces results which make sense
in light of vector addition \emph{via} the parallelogram law.

\N{Angles Between Vectors}
We can recall the law of cosines from trigonometry:
\begin{equation}
\|\vec{u}-\vec{v}\|^{2} = \|\vec{u}\|^{2} + \|\vec{v}\|^{2} - 2 \|\vec{u}\|\|\vec{v}\|\cos(\theta)
\end{equation}
where $\theta$ is the angle formed between the edges of the vector. We
refresh our memory of the location of the variables with the sketch:
\begin{center}
  \includegraphics{img/img.8}
\end{center}
Now, we can compute using coordinates
\begin{subequations}
\begin{calculation}
  \|\vec{u}-\vec{v}\|^{2}
\step{by definition of the magnitude of a vector}
  (u_{1}-v_{1})^{2} + (u_{2}-v_{2})^{2}
\step{expanding the terms}
  (u_{1}^{2}-2u_{1}v_{1}+v_{1}^{2}) + (u_{2}^{2}-2u_{2}v_{2}+v_{2}^{2})
\step{associativity of addition}
  (u_{1}^{2} + u_{2}^{2}) + (v_{1}^{2} + v_{2}^{2}) -2u_{1}v_{1}-2u_{2}v_{2}
\step{distributivity}
  (u_{1}^{2} + u_{2}^{2}) + (v_{1}^{2} + v_{2}^{2}) -2(u_{1}v_{1}+u_{2}v_{2}).
\end{calculation}
\end{subequations}
We see this is just
$\|\vec{u}\|^{2}+\|\vec{v}\|^{2}-2(\mbox{stuff})$. But we know what the
``(stuff)'' is, by the law of cosines,
\begin{equation}
-2(u_{1}v_{1}+u_{2}v_{2}) = - 2 \|\vec{u}\|\|\vec{v}\|\cos(\theta),
\end{equation}
hence \emph{for nonzero vectors} we find,
\begin{equation}
\frac{(u_{1}v_{1}+u_{2}v_{2})}{\|\vec{u}\|\|\vec{v}\|} = \cos(\theta).
\end{equation}
But what's more, we see the numerator of the right-hand side is just the
dot product of vectors $\vec{u}\cdot\vec{v}$. We conclude
\begin{equation}
\frac{\vec{u}\cdot\vec{v}}{\|\vec{u}\|\|\vec{v}\|} = \cos(\theta).
\end{equation}
We see this formula is symmetric if we switched places of $\vec{u}$ and
$\vec{v}$, in the sense we get exactly the same result.

But further, we can now \emph{define} the angle $\theta$ between
\emph{any} two \emph{nonzero}(!!) vectors $\vec{u}$ and $\vec{v}$ by the
relation: 
\begin{equation}
\boxed{\cos(\theta) := \frac{\vec{u}\cdot\vec{v}}{\|\vec{u}\|\|\vec{v}\|}.}
\end{equation}
We stress these must be nonzero vectors (otherwise we divide by zero and
horrible results follow).

\begin{definition}
If two nonzero vectors $\vec{u}$ and $\vec{v}$ are such that
\begin{equation}
\vec{u}\cdot\vec{v}=0,
\end{equation}
then we call them \define{Orthogonal} (usually we say one is orthogonal
to the other).
\end{definition}

\N{Unit Vectors and Directions}
If we wish to make a statement about ``directions'', then we can make it
precise by using \define{Unit Vectors} --- vectors of ``unit magnitude''
(i.e., length one). We follow tradition and use hats to indicate we have
a unit vector. For any nonzero vector $\vec{u}$ of arbitrary length, we
can \define{Normalize} it to produce the unit vector
\begin{equation}
\widehat{\vec{u}} = \frac{\vec{u}}{\|\vec{u}\|}.
\end{equation}
We see then that the angle between vectors may be found using their unit
vectors's dot product, without worrying about dividing by anything:
\begin{equation}
\widehat{\vec{u}}\cdot\widehat{\vec{v}}=\cos(\theta).
\end{equation}

\N{Generalization to $n>2$}
Everything we have discussed so far has been restricted to the real
plane $\RR^{2}$. But we could generalize it to $\RR^{n}$ for $n\in\NN$
arbitrary [but fixed]. Vectors then have $n$ components, but all our
definitions generalize accordingly. Vector addition is given by adding
components, scalar multiplication is given by multiplying each component
by the given scalar, and so on.