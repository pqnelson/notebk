\section{Solving Linear Equations with Augmented Matrices}

\M
Recall (\S\ref{par:matrix-algebra:solving-systems-of-equations})
we found a way to encode a system of linear equations using matrix
notation as
\begin{equation}
\mat{A}\vec{x}=\vec{b}.
\end{equation}
But we got no farther than that. Now we will begin to solve such a
system in an algorithmic manner.

Our first step will be to work with an \define{Augmented Matrix}, which
has block form
\begin{equation}
(\mat{A}\mid\vec{b}).
\end{equation}
Since the unknowns (encoded in the vector $\vec{x}$) are not known, we
omit them from the augmented matrix. For a concrete example:
\begin{equation}
  \begin{pmatrix}
  4 &  3 &  3\\
 -4 & -1 &  1\\
 -1 & -1 & -1\\
  0 &  2 &  1\\
  \end{pmatrix}\begin{pmatrix}x_{1}\\x_{2}\\x_{3}\\x_{4}
  \end{pmatrix} =
  \begin{pmatrix}-5\\1\\-2\\-1
  \end{pmatrix}
\mapsto
  \left[\begin{array}{ccc|c}
  4 &  3 &  3 & -5\\
 -4 & -1 &  1 &  1\\
 -1 & -1 & -1 & -2\\
  0 &  2 &  1 & -1 
  \end{array}\right]
\end{equation}
This is our first step, now what?

We will apply certain elementary row operations to repeatedly modify our
augmented matrix until it's in a form suitable for solving. Let us
define our elementary row operations, then lay out the criteria for an
augmented matrix being ``suitably nice''.

\begin{definition}
Let $\mat{A}$ be a matrix. We define an \define{Elementary Row Operation}
to be one of the following:
\begin{enumerate}
\item Multiply row $i$ by a nonzero scalar $r$,
\item Add a multiple of row $i$ to row $j$,
\item Swap rows $i$ and $j$.
\end{enumerate}
\end{definition}

\begin{remark}
These elementary row operations correspond to (respectively):
\begin{enumerate}
\item Multiply both sides of equation $i$ by a nonzero number $r$,
\item Add a multiple of equation $i$ to equation $j$,
\item Swap equations $i$ and $j$.
\end{enumerate}
We see this is permitted by elementary algebra.
\end{remark}

\begin{definition}
Let $\mat{A}$ be a matrix. We say it is in \define{Reduced Row Echelon Form}
\begin{enumerate}
\item All rows consisting of zeroes are at the bottom of the matrix, and
\item The leading coefficient of a nonzero row is strictly ``to the
  right'' of the leading coefficient of the row above it, and
\item The leading coefficient is 1.
\end{enumerate}
\end{definition}

\begin{remark}
Some authors do not require ``the leading coefficient is 1'' to the
criteria, but we require it.
\end{remark}

\begin{example}
  A generic reduced row echelon form matrix looks like
  \begin{equation}
\begin{bmatrix}
1 & a_{1,2} & a_{1,3} & \dots & a_{1,j} & a_{1,j+1} & a_{1,j+2} & a_{1,j+3} & \dots& a_{1,n}\\
0 & 1 & a_{2,3} & \dots & a_{2,j} & a_{2,j+1} & a_{2,j+2} & a_{2,j+3} & \dots& a_{2,n}\\
0 & 0 & 1 & \dots & a_{3,j} & a_{3,j+1} & a_{3,j+2} & a_{3,j+3} & \dots& a_{3,n}\\
\vdots & \vdots & \vdots & \ddots & \vdots & \vdots & \vdots & \vdots & \ddots & \vdots\\
0 & 0 & 0 & \dots & 0 & 1 & a_{i,j+2} & a_{i,j+3} & \dots & a_{i,n}\\
0 & 0 & 0 & \dots & 0 & 0 & 0        & 1        & \dots & a_{i+1,n}\\ 
0 & 0 & 0 & \dots & 0 & 0 & 0 & 0 & \dots & 0\\
0 & 0 & 0 & \dots & 0 & 0 & 0 & 0 & \dots & 0\\
0 & 0 & 0 & \dots & 0 & 0 & 0 & 0 & \dots & 0
\end{bmatrix}
  \end{equation}
\end{example}

\begin{example}
  For an augmented matrix in reduced row echelon form, we have
  \begin{equation}
\left[\begin{array}{ccc|c}
1 & 2 & 0 & 1\\
0 & 1 & 0 & 1\\
0 & 0 & 1 & 2
  \end{array}\right]  
  \end{equation}
  This would correspond to the system of equations
\begin{alignat*}{7}
1x_{1} && + && 2x_{2} && + && 0x_{3} &&\;=\;&&1\\
0x_{1} && + && 1x_{2} && + && 0x_{3} &&\;=\;&&1\\
0x_{1} && + && 0x_{2} && + && 1x_{3} &&\;=\;&&2
\end{alignat*}
We can solve this immediately, finding $x_{3}=2$, $x_{2}=1$, and $x_{1}=-1$.
\end{example}

\subsection{Gauss--Jordan Elimination}

\M
The idea is to use elementary row operations to transform our augmented
matrix $(\mat{A}\mid\vec{b})$ to reduced row echelon form $(\mat{B}\mid\vec{c})$,
then solve the system of equations $\mat{B}\vec{x}=\vec{c}$ which will
also be a solution to $\mat{A}\vec{x}=\vec{b}$.

We see that the two augmented matrices $(\mat{A}\mid\vec{b})$ and
$(\mat{B}\mid\vec{c})$ are ``the same'', very analogous to how the
fractions $1/2$ and $2/4$ are ``the same'' despite having different
numerators and denominators. Before demonstrating Gaussian elimination,
let us formalize this notion of ``sameness''.

\begin{definition}
We call two augmented matrices $(\mat{A}\mid\vec{b})$ and
$(\mat{B}\mid\vec{c})$ \define{Equivalent} if they have the same
solutions, and denote this by $(\mat{A}\mid\vec{b})\sim(\mat{B}\mid\vec{c})$.

More generally, any two $m\times n$ matrices $\mat{A}$, $\mat{B}$
are \define{Equivalent} if there is a finite sequence of elementary row
operations that transforms $\mat{A}$ into $\mat{B}$. (Further,
``equivalence of augmented matrices'' coincides with ``equivalence of matrices'',
so we will use the same symbol $\sim$ for both of them.)
\end{definition}

\begin{proposition}
  Matrix equivalence satisfies the following properties:
  \begin{enumerate}
  \item Reflexivity: for any matrix $\mat{A}$, we have $\mat{A}\sim\mat{A}$
  \item Symmetry: for any $m\times n$ matrices $\mat{A}$ and $\mat{B}$,
    we have $\mat{A}\sim\mat{B}$ implies $\mat{B}\sim\mat{A}$
  \item Transitivity: for any $m\times n$ matrices $\mat{A}$, $\mat{B}$, $\mat{C}$,
    we have $\mat{A}\sim\mat{B}$ and $\mat{B}\sim\mat{C}$ implies $\mat{A}\sim\mat{C}$.
  \end{enumerate}
\end{proposition}


\N{Gauss--Jordan Elimination}
Consider the augmented matrix
\begin{subequations}
\begin{equation}
\left[\begin{array}{ccc;{2pt/2pt}c}
    0 & 0 & 1 & 6\\
    2 & 4 & 6 & 8\\
    2 & 3 & 4 & 5
  \end{array}\right]
\end{equation}
The algorithm for transforming it to reduced row echelon form consists
of the following steps.

\N*{Step 1: Locate nonzero column} Look for the first column that is not all zeroes. We can
see this is the first column (highlighted):
\begin{equation}
\left[\begin{array}{>{\columncolor{red!20}}ccc;{2pt/2pt}c}
    0 & 0 & 1 & 6\\
    2 & 4 & 6 & 8\\
    2 & 3 & 4 & 5
  \end{array}\right]
\end{equation}
If, for some reason, the first column is a zero column vector, move to
the next column to the right (and keep moving until you find the first
column vector which is not a zero column vector); we will ignore the
zero columns, and refer to the first nonzero column vector as ``the
first column'' or ``the leading column''.

\N*{Step 2: Pivot} If the first row has a zero entry in the leading
column, swap a row with a nonzero entry in the leading column. We could
pick \emph{any} such row with a nonzero entry in the leading column, we
will swap the second row
\begin{equation}
\left[\begin{array}{ccc;{2pt/2pt}c}
\rowcolor{olive!20}    0 & 0 & 1 & 6\\
\rowcolor{red!20}    2 & 4 & 6 & 8\\
    2 & 3 & 4 & 5
  \end{array}\right]\sim
\left[\begin{array}{ccc;{2pt/2pt}c}
\rowcolor{red!20}    2 & 4 & 6 & 8\\
\rowcolor{olive!20}    0 & 0 & 1 & 6\\
    2 & 3 & 4 & 5
  \end{array}\right].
\end{equation}
\N*{Step 3: Normalize} We now normalize the top row by diving through by
the nonzero pivot (or, equivalent, multiplying by
$1/(\mbox{pivot})$). For us, this is dividing the first row by 2:
\begin{equation}
\left[\begin{array}{ccc;{2pt/2pt}c}
\rowcolor{red!20}    2 & 4 & 6 & 8\\
0 & 0 & 1 & 6\\
    2 & 3 & 4 & 5
  \end{array}\right]\sim
\left[\begin{array}{ccc;{2pt/2pt}c}
\rowcolor{red!20}    1 & 2 & 3 & 4\\
0 & 0 & 1 & 6\\
    2 & 3 & 4 & 5
  \end{array}\right]
\end{equation}

\N*{Step 4: Transform lower rows.} For each row \emph{below the first row} with a nonzero
leading component, we will subtract a multiple of the first row from
it. The idea is to transform each row \emph{below the first row} to have
a zero entry in the leading column.
For us, we see there is one row [beneath the first row] with a nonzero
entry in the leading column, highlighted in blue:
\begin{equation}
\left[\begin{array}{ccc;{2pt/2pt}c}
    1 & 2 & 3 & 4\\
0 & 0 & 1 & 6\\
\rowcolor{blue!20}    2 & 3 & 4 & 5
  \end{array}\right]
\end{equation}
We then add $-2$ times the first row to the third row, giving us
\begin{equation}
\left[\begin{array}{ccc;{2pt/2pt}c}
    1 & 2 & 3 & 4\\
0 & 0 & 1 & 6\\
\rowcolor{blue!20}    2 & 3 & 4 & 5
  \end{array}\right]\sim
\left[\begin{array}{ccc;{2pt/2pt}c}
    1 & 2 & 3 & 4\\
0 & 0 & 1 & 6\\
\rowcolor{blue!20}    0 & -1 & -2 & -3
  \end{array}\right]
\end{equation}
At this point, the first column with a nonzero entry will have its
general form look like $(1,0,0,\dots,0)$ (i.e., it leads with 1 and all
entries below it are zero).

\N*{Step 5: Repeat on the submatrix}
We take the submatrix for the remaining rows and remaining columns, then
return to step 1, and transform the submatrix to reduced row echelon
form. 
\begin{equation}
\left[\begin{array}{ccc;{2pt/2pt}c}
1 & 2 & 3 & 4\\ \cline{2-4}
0 & \multicolumn{1}{|c}{0} & 1 & 6\\
0 & \multicolumn{1}{|c}{-1} & -2 & -3
  \end{array}\right]
\end{equation}
We will find, for our particular example, the matrix has reduced row
echelon form
\begin{equation}\dots\sim
\left[\begin{array}{ccc;{2pt/2pt}c}
1 & 2 & 3 & 4\\
0 & 1 & 2 & 3\\
0 & 0 & 1 & 6
\end{array}\right].
\end{equation}
\end{subequations}

\begin{remark}
We may optionally skip normalizing the ``pivot'' (step 3), but continue
to transform the lower rows so everything below the pivot is zero. This
produced \define{Row Echelon Form}. It is no longer ``reduced'', because
the pivots are not $1$.
\end{remark}

\begin{remark}
We should prove that Gauss-Jordan elimination produces a unique
augmented matrix, in some sense. It's obviously not true the results
will be identical, since we had some choice in swapping rows. But the
solutions produced from the resulting augmented matrices \emph{will} be
identical. After a few examples, we will prove this.
\end{remark}

\begin{example}
  Suppose we have a $2\times2$ matrix $\mat{A}$ with generic entries
  \begin{equation}
\mat{A} = \begin{bmatrix}a & b\\c & d
\end{bmatrix}.
  \end{equation}
  Suppose $\mat{A}$ is invertible. What are the components of
  $\mat{A}^{-1}$?

  Right now, they are unknowns
\begin{subequations}
  \begin{equation}
\mat{A}^{-1} = \begin{bmatrix}x & y\\z & w
\end{bmatrix}.
  \end{equation}
  Matrix multiplication gives us
  \begin{equation}
\mat{A}\mat{A}^{-1} = \begin{bmatrix}a & b\\c & d
\end{bmatrix} \begin{bmatrix}x & y\\z & w
\end{bmatrix}
= \begin{bmatrix}
ax+bz & ay+bw\\
cx+dz & cy+dw
\end{bmatrix}.
  \end{equation}
  Which\dots does not really seem to improve the situation. Or does it?
  We expect this to be equal to the identity matrix, and really this is
  a system of 4 linear equations
\begin{equation}
\left.\begin{array}{cccccc}
ax &    & +bz &     & = & 1\\
   & ay &     & +bw & = & 0\\
cx &    & +dz &     & = & 0\\
   & cy &     & +dw & = & 1
\end{array}\right\}\equiv
\left(\begin{array}{cccc;{2pt/2pt}c}
  a &   & b &   & 1\\
    & a &   & b & 0\\
  c &   & d &   & 0\\
    & c &   & d & 1
\end{array}\right)
\end{equation}
\end{subequations}
We can now transform this augmented matrix to row echelon form (not
reduced, but still row echelon form).
\begin{subequations}
\begin{calculation}\gdef\defaultrelation{\sim}
\left(\begin{array}{cccc;{2pt/2pt}c}
  a & 0 & b & 0 & 1\\
  0 & a & 0 & b & 0\\
  c & 0 & d & 0 & 0\\
  0 & c & 0 & d & 1
\end{array}\right)
\step{add $(-c/a)(\mbox{row 1})$ to row 3} 
\left(\begin{array}{cccc;{2pt/2pt}c}
  a & 0 & b & 0 & 1\\
  0 & a & 0 & b & 0\\
  0 & 0 & \displaystyle d-\frac{bc}{a} & 0 & \displaystyle \frac{-c}{a}\\
  0 & c & 0 & d & 1
\end{array}\right)
\step{add $(-c/a)(\mbox{row 2})$ to row 4} 
\left(\begin{array}{cccc;{2pt/2pt}c}
  a & 0 & b & 0 & 1\\
  0 & a & 0 & b & 0\\
  0 & 0 & \displaystyle d-\frac{bc}{a} & 0 & \displaystyle \frac{-c}{a}\\
  0 & 0 & 0 & \displaystyle  d-\frac{bc}{a} & 1
\end{array}\right)
\step{since $d - (bc/a) = (ad - bc)/a$}
\left(\begin{array}{cccc;{2pt/2pt}c}
  a & 0 & b & 0 & 1\\
  0 & a & 0 & b & 0\\
  0 & 0 & \displaystyle\frac{ad - bc}{a} & 0 & \displaystyle \frac{-c}{a}\\
  0 & 0 & 0 & \displaystyle\frac{ad-bc}{a} & 1
\end{array}\right)
\end{calculation}
\end{subequations}
This has solution
\begin{subequations}
\begin{align}
w &= \frac{a}{ad-bc}\\
z &= \frac{-c}{ad-bc}\\
y &= \frac{-b}{ad-bc}\\
ax - \frac{bc}{ad - bc} &= 1\nonumber\\
\implies ax &= \frac{bc}{ad-bc}+1 = \frac{ad}{ad-bc}\nonumber\\
\implies x &= \frac{d}{ad-bc}.
\end{align}
\end{subequations}
Hence
\begin{equation}
\boxed{\mat{A}^{-1} = \frac{1}{ad-bc}\begin{pmatrix}d & -b\\
-c & a
\end{pmatrix}.}
\end{equation}
\end{example}

\N{Algorithm to determine inverse matrix}
From the previous example, we see that we could rephrase our process as
considering the $2n\times n$ augmented matrix $(\mat{A}\mid\mat{I})$,
then ``applying Gauss--Jordan elimination and backsubstituting'' (i.e.,
applying elementary row operations until) we obtain
\begin{equation}
(\mat{A}\mid\mat{I})\sim(\mat{I}\mid\mat{A}^{-1}).
\end{equation}
If elementary row operations cannot transform $\mat{A}$ to the identity
matrix $\mat{A}\not\sim\mat{I}$, then $\mat{A}$ is singular.