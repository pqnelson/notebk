\section{Solving Systems of Equations}

\begin{example}[{Euler~\cite[Ch.IV \S4 question 3 \P612]{euler:algebra}}]
A mule and donkey carry a large load. The donkey complained of his load
and said to the mule, ``I need only 100 pounds of your load to make mine
twice as heavy as yours would be.'' To which the mule answered, ``But if you gave
me 100 pounds of your load, I'd be carrying three times you would
carry.''
How much did they carry?

Let $d$ be the donkey's load, $m$ be the mule's load. The donkey's
statement could be presented in the equation,
\begin{subequations}
\begin{equation}
d + 100 = 2(m - 100),
\end{equation}
whereas the mule's response,
\begin{equation}
m + 100 = 3(d - 100).
\end{equation}
\end{subequations}
From the first of these equations, we find
\begin{equation}
d = 2m - 300.
\end{equation}
We plug this into the second equation to find
\begin{equation}
\begin{split}
  m + 100 &= 3(2m - 400)\\
  &= 6m - 1200.
\end{split}
\end{equation}
Hence the mule carries,
\begin{subequations}
\begin{equation}
m = \frac{1300}{5} = 260~\mbox{pounds},
\end{equation}
and this means the donkey carries,
\begin{equation}
d = 520 - 300 = 220~\mbox{pounds}.
\end{equation}
\end{subequations}
\end{example}

\begin{example}[{Sraffa~\cite{sraffa}}]\label{ex:systems-of-equations:sraffa}
One class of systems of linear equations occurs naturally in
Neo-Ricardian economics. Consider a hypothetical economy consisting of
two industry sectors, wheat and iron, whose production processes are
described by the system of equations (where the left of the arrow refers
to the inputs at the start of the production period, and the quantities
to the right of the arrows are produced at the end of the period, and
then an annual market takes place where iron is traded for wheat):
\begin{equation}
\begin{array}{rcl} \mbox{inputs} & \to & \mbox{outputs}\\
280~\mbox{qr. wheat} + 12~\mbox{t. iron} & \to & 400~\mbox{qr. wheat}\\
120~\mbox{qr. wheat} + 8~\mbox{t. iron}  & \to &  20~\mbox{t. iron}.
\end{array}
\end{equation}
The question we want to know: what is the exchange rate between
quarters\footnote{A ``quarter'' of wheat refers to an obscure
measurement choice of 13th century England, because it can be found in
the \textit{Magna Carta}. By the 18th century, the measurement of 1
quarter of wheat in Britain varied port to port. It was finally
standardized in 1824 to be 8 bushels (or 64 gallons).} 
of wheat and tons of iron? We can create a system of equations by
introducing two unknowns $p_{w}$ for the price of 1 quarter of wheat,
and $p_{i}$ for the price of 1 ton of iron. We assume society is in a
``self reproducing state'', in the sense that the wheat sector exchanges
enough wheat-for-iron to continue producing wheat (and the iron sector
exchanges enough iron-for-wheat to continue producing iron):
\begin{subequations}
\begin{equation}
  \begin{array}{rcl}
    280p_{w} + 12p_{i} &=& 400p_{w}\\
    120p_{w} + 8p_{i} &=& 20p_{i}.
  \end{array}
\end{equation}
We can rewrite the first equation (by subtracting both sides by $280p_{w}$):
\begin{equation}
  \begin{array}{rcl}
    12p_{i} &=& 120p_{w}\\
    120p_{w} + 8p_{i} &=& 20p_{i}.
  \end{array}
\end{equation}
Then we rewrite the second equation (by subtracting both sides by $8p_{i}$):
\begin{equation}
  \begin{array}{rcl}
    12p_{i} &=& 120p_{w}\\
    120p_{w} &=& 12p_{i}.
  \end{array}
\end{equation}
These are redundant equations! We end up with a class of solutions,
described by
\begin{equation}
\boxed{10p_{w} = p_{i},}
\end{equation}
\end{subequations}
that is to say, the price of 1 ton iron is equal to the price of 10
quarters of wheat.
\end{example}

\begin{remark}
This method of solving such equations in Neo-Ricardian economics works
fine when there is no surplus; that is to say, when there is exactly as
much produced (for each commodity) used as inputs in the
economy. However, when there is surplus, we need more sophisticated
techniques of linear algebra, because we will have an eigenvalue problem.
\end{remark}

\begin{remark}
As of 28 October 2022, it appears the going rate is 1.18925 quarters of wheat
exchanges for 1 ton of iron.
\end{remark}

\begin{example}[Motivating Example]
  Suppose we have two equations with two variables $x$ and $y$ given by:
  \begin{subequations}
    \begin{align}
      -x-2y  &= -2\\
      5x+6y  &= 1.
    \end{align}
  \end{subequations}
  Are there [real] values for $x$ and $y$ which makes this hold? We can
  add 5 times the first equation to the second equation, giving us:
  \begin{subequations}
    \begin{align}
      -x-2y  &= -2\\
      0-4y  &= -9.
    \end{align}
  \end{subequations}
  The second equation may be solved for $y=9/4$, then we may substitute
  this into the first equation giving us
  \begin{equation}
-x-2(9/4) = -2\implies -x=2(9/4)-2=\frac{9}{2}-\frac{4}{2}=\frac{5}{2}.
  \end{equation}
  Hence we obtain the solutions
  \begin{equation}
\boxed{x=\frac{5}{2}}\quad\mbox{and}\quad\boxed{y=\frac{9}{4}.}
  \end{equation}
\end{example}

\N{Index Notation}
We will rely heavily on index notation. Why? Well, there are 26 possible
symbols (using lowercase Latin letters), which means we could not
discuss anything beyond systems of 5 equations in 5 variables; if we
also use the 26 uppercase Latin letters, then we cannot discuss anything
beyond systems of 7 equations in 7 unknowns. Rather than struggle with
new symbols, we will use indices: appending numbers as subscripts to
letters to refer to distinct quantities. Thus we write $x_{1}$, $x_{2}$,
$x_{3}$ instead of $x$, $y$, $z$.

We may use \define{Dummy Variables} to affix a variable as a subscript,
writing $x_{i}$ where $i$ could be $1$, $2$, or $3$. Here $i$ is the
dummy variable in the expression $x_{i}$. We often will write ``$x_{i}$
where $i=1,2,3$'' to indicate $i$ is a dummy variable ranging over the
values $1$, $2$, $3$.

We can have several indices [plural of ``index''] affixed to a
symbol. We have seen this with coefficients, which had two indices
separated by a comma. Some authors use a comma to separate indices,
other authors do not.

Also we should note, some authors use superscripts indices. This will
occur in some parts of mathematics, but we will avoid it in these notes
whenever possible (because it's hard to tell if a superscript number is an
exponent or an index).

\N{Basic Terminology}
A \define{linear equation in one variable} $x$ looks like
\begin{equation}
b = ax
\end{equation}
where $a$ and $b$ are real constants.

A single \define{linear equation in two variables} $x$ and $y$ looks like
\begin{equation}
  ax + by = c
\end{equation}
where $a$, $b$, and $c$ are real constants.

A single linear equation in $n$ variables $x_{1}$, \dots, $x_{n}$ looks
like
\begin{equation}
a_{1}x_{1} + \dots + a_{n}x_{n} = b
\end{equation}
where $a_{1}$, \dots, $a_{n}$, and $b$ are real constants.

\N{Terminology: Systems of equations}
More generally, a \define{System of $m$ Linear Equations in $n$ Unknowns} 
$x_{1}$, \dots, $x_{n}$ (where the subscripts are \define{Indices} to
append to one symbol $x$ a number or ``dummy variable'' [a variable
ranging over the index values], giving us any
number of variables for the price of one symbol) looks like:
\begin{equation}
\begin{array}{rcl}
a_{1,1}x_{1} + a_{1,2}x_{2} + \dots + a_{1,n}x_{n} &=& b_{1}\\
a_{2,1}x_{1} + a_{2,2}x_{2} + \dots + a_{2,n}x_{n} &=& b_{2}\\
\vdots\quad \qquad\vdots\quad\qquad\ddots\quad\qquad\vdots\quad & \vdots & \vdots\\
a_{m,1}x_{1} + a_{m,2}x_{2} + \dots + a_{m,n}x_{n} &=& b_{m}
\end{array}
\end{equation}
where $b_{1}$, \dots, $b_{m}$ are $m$ real constants, and we have
$a_{1,1}$, $a_{1,2}$, \dots, $a_{m,n}$ be $mn$ real constants called the
\define{Coefficients}. 

A \define{Solution} to a system of equations in $n$ variables $x_{1}$,
\dots, $x_{n}$ is an assignment of $n$ real values $s_{1}$, \dots,
$s_{n}$ to the variables (so $x_{1}=s_{1}$, \dots, $x_{n}=s_{n}$) which
satisfies the system of equations.


\begin{example}[System of Nonlinear Equations]
We have to distinguish a system of linear equations from nonlinear
equations. For example, the following is a nonlinear equation
\begin{equation}
3x^{2} + x = 2,
\end{equation}
because a polynomial of order $n>1$ will be nonlinear. If you have a
function $f(x)$ and you don't know if it's linear or not, check if
$f(ax+by)=af(x)+bf(y)$ holds or not --- this is the criteria of linearity.

The following equation is also nonlinear:
\begin{equation}
x + yz =3.
\end{equation}
We cannot multiply two [or more] unknowns together in a linear equation.

So the following equation is nonlinear:
\begin{equation}
x + y\cos(z) = 3,
\end{equation}
on two counts: first, $\cos(z)$ is nonlinear; and second, $y\cos(z)$ is
not allowed in a linear equation.
\end{example}

\N{Method of Elimination}
One method to find a solution, which has been taught to high school
students for decades, is to take the first equation and use it to
rewrite one variable in terms of the remaining. For example, in:
\begin{equation}
  \begin{array}{rcrcrcl}
    0      & + & 2x_{2} & + & 3x_{3} & = & 5\\
    2x_{1} & + & 0      & + & 3x_{3} &=& -1\\
    x_{1}  & + & x_{2}  & + & x_{3} &=& 1.
  \end{array}
\end{equation}
We would use the first equation to substitute $x_{2}=(5 - 3x_{3})/2$ in
the remaining equations, and discard the first equation, giving us:
\begin{equation}
  \begin{array}{rcrcrcl}
    2x_{1} & + & 0      & + & 3x_{3} &=& -1\\
    x_{1}  & + & \displaystyle\left(\frac{5 - 3x_{3}}{2}\right)  & + & x_{3} &=& 1.
  \end{array}
\end{equation}
We do this again, to use the first equation to write $x_{1} = (-1-3x_{3})/2$,
and plug this into the last equation, which gives us a value for
$x_{3}$.
We backsubstitute this into the second equation to obtain a value for
$x_{1}$, then together these are backsubstituted into either the first
or third equation of our original system to find the solution for
$x_{2}$.

\N{Issues with Method of Elimination}
Doing this ``method of elimination'' is tedious. \emph{I} [the author]
does not want to do this, it's too much work. Is there a better way to
find a solution to a system of linear equations?

A second issue with the method of elimination, it does not tell us when
a solution exists. Or if there is a family of solutions, which we saw in
Example~\ref{ex:systems-of-equations:sraffa} (the price of 1 ton of iron
was 10 times the price of a quarter of wheat, whatever that is).

\N{Roadmap: Enter the Matrix}
Remember what a system of 1 linear equation in 1 variable looks like? We
saw it was $ax=b$. We could solve this easily when $a\neq0$, simply take
$x=b/a$ or $x = a^{-1}b$. It would be lovely if all systems of linear
equations could be so easy.

We will try to make it easy by generalizing the notion of a
``number''. Instead of writing a system of equations as
\begin{equation}
\begin{array}{rcl}
a_{1,1}x_{1} + a_{1,2}x_{2} + \dots + a_{1,n}x_{n} &=& b_{1}\\
a_{2,1}x_{1} + a_{2,2}x_{2} + \dots + a_{2,n}x_{n} &=& b_{2}\\
\vdots\quad \qquad\vdots\quad\qquad\ddots\quad\qquad\vdots\quad & \vdots & \vdots\\
a_{m,1}x_{1} + a_{m,2}x_{2} + \dots + a_{m,n}x_{n} &=& b_{m}
\end{array}
\end{equation}
we will introduce a notion of a matrix $\mat{A}$ as an array of $m$ rows and
$n$ columns, and vectors $\vec{x}$, $\vec{b}$, then rewrite our system
as
\begin{equation}
\mat{A}\vec{x}=\vec{b}.
\end{equation}
Just as with our simpler system with one equation in one unknown $ax=b$
where we multiplied both sides by the number $a^{-1}$ (assuming $a$ is
invertible), we can imitate this and try ``multiplying'' both sides by a
matrix $\mat{B}$ to transform our system to something like
\begin{equation}
\vec{x} = \mat{B}\vec{b}.
\end{equation}
The problem: we will have to define a notion of matrix multiplication,
and prove it satisfies the familiar properties which the multiplication
of numbers enjoy.

\phantomsection
\subsection*{Exercises}
\addcontentsline{toc}{subsection}{Exercises}

\begin{exercise}[{Sraffa~\cite[\S2]{sraffa}}]
Find the exchange-values which ensure a
self-reproducing state for the following economy:
\begin{equation}
  \begin{array}{rcl}
240~\mbox{qr. wheat} + 12~\mbox{t. iron} + 18~\mbox{pigs} &\to& 450~\mbox{qr. wheat}\\
 90~\mbox{qr. wheat} +  6~\mbox{t. iron} + 12~\mbox{pigs} &\to& 21~\mbox{t. iron}\\
120~\mbox{qr. wheat} +  3~\mbox{t. iron} + 30~\mbox{pigs} &\to& 60~\mbox{pigs}.
  \end{array}
  \end{equation}
\end{exercise}

