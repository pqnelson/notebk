\section{Subspaces}

\M
Example~\ref{puzzle:vector-spaces:solution-space} motivates the
following definition:

\begin{definition}
Let $V$ be a real vector space with vector addition $\oplus$ and scalar
multiplication $\odot$.
Let $U\subset V$ be some subset. We call $U$ a \define{Subspace} of $V$
if
it forms a real vector space using
the vector addition $\oplus$ from $V$ and the scalar multiplication $\odot$ from $V$.
\end{definition}

\begin{remark}
  Several things to note about this definition:
  \begin{enumerate}
  \item The zero vector of $U$ must be the zero vector of $V$.
  \item The binary operators on $U$ must be those of $V$ restricted to
    $U$. We cannot change them.
  \end{enumerate}
\end{remark}

\begin{remark}
This is a clunky definition. No one wants to prove $U$ satisfies 9
conditions when we know $V$ satisfies them. One of our first goals is to
simplify the criteria for determining if $U$ is a subspace of $V$ or not.
\end{remark}

\begin{theorem}\label{thm:subspaces:subset-closed-under-linear-combos-is-a-subspace}
Let $V$ be a real vector space with vector addition $\oplus$ and scalar
multiplication $\odot$. Then the subset $U\subset V$ forms a subspace of
$V$ provided:
\begin{enumerate}
\item for any $\vec{u}$, $\vec{v}\in U$, we have
  $\vec{u}\oplus\vec{v}\in U$; and
\item for any $c\in\RR$ and $\vec{u}\in U$, we have $c\odot\vec{u}\in U$.
\end{enumerate}
\end{theorem}

In other words, if $U$ is closed under the vector addition
$\oplus$ and scalar multiplication $\odot$ from $V$, then $U$ is a
subspace of $V$.

\begin{proof}
Assume $U\subset V$ satisfies the two conditions
\begin{enumerate}[label=(\alph*)]
\item\label{assume:closed-under-addition} for any $\vec{u}$, $\vec{v}\in U$, we have
  $\vec{u}\oplus\vec{v}\in U$; and
\item\label{assume:closed-under-scalar-multiplication} for any $c\in\RR$ and $\vec{u}\in U$, we have $c\odot\vec{u}\in U$.
\end{enumerate}
We want to prove the following:
\begin{enumerate}[label=(\arabic*)]
\item Closure of $\oplus$: If $\vec{u}$, $\vec{v}\in U$ are two arbitrary elements of $U$,
  then $\vec{u}\oplus\vec{v}\in U$; in other words, $U$ is closed under
  the $\oplus$ operation.

  \textsc{Proof:} this is precisely assumption \ref{assume:closed-under-addition}.
\item Commutativity of $\oplus$: for any $\vec{u}$, $\vec{v}\in U$,
  we have $\vec{u}\oplus\vec{v}=\vec{v}\oplus\vec{u}$.

  \textsc{Proof:} any elements of $U$ are elements of $V$, so reconsider
  $\vec{u}$ and $\vec{v}$ as elements of $V$. Then the condition holds
  (since $V$ is a real vector space).
\item Associativity of $\oplus$: for any $\vec{u}$, $\vec{v}$, $\vec{w}\in U$,
  we have $\vec{u}\oplus(\vec{v}\oplus\vec{w})=(\vec{u}\oplus\vec{v})\oplus\vec{w}$.

  \textsc{Proof:} any elements of $U$ are elements of $V$, so reconsider
  $\vec{u}$, $\vec{v}$, and $\vec{w}$ as elements of $V$. Then the
  condition holds (since $V$ is a real vector space).
\item Unit of $\oplus$: there exists an element $\vec{0}\in U$ such that
  for every $\vec{u}\in U$, $\vec{0}\oplus\vec{u}=\vec{u}\oplus\vec{0}=\vec{u}$.

  \textsc{Proof:} since $U$ is closed under scalar multiplication by assumption~\ref{assume:closed-under-scalar-multiplication}, we
  know $0\odot\vec{u}=\vec{0}$ which must be in $U$.
\item Existence of negation: for each $\vec{u}\in U$, there exists a
  $-\vec{u}\in U$ such that $\vec{u}\oplus-\vec{u}=-\vec{u}\oplus\vec{u}=\vec{0}$

  \textsc{Proof:} since $U$ is closed under scalar multiplication by assumption~\ref{assume:closed-under-scalar-multiplication}, we
  identify $-\vec{u} = -1\odot\vec{u}$ which would be in $U$ and
  satisfies the desired properties.
\item Closure of $\odot$: for any real number $c\in\RR$ and element
  $\vec{u}\in U$, we have $c\odot\vec{v}\in U$.

  \textsc{Proof:} This is assumption~\ref{assume:closed-under-scalar-multiplication}.
\item Left distributivity of $\odot$ over $\oplus$:
  for any $c\in\RR$ and $\vec{u}$, $\vec{v}\in U$, we have
  $c\odot(\vec{u}\oplus\vec{v}) = (c\odot\vec{u})\oplus(c\odot\vec{v})$.

  \textsc{Proof:} any elements of $U$ are elements of $V$, so reconsider
  $\vec{u}$ and $\vec{v}$ as elements of $V$. Then the condition holds
  (since $V$ is a real vector space).
\item Right distributivity of $\oplus$ over $\odot$:
  for any $c$, $d\in\RR$ and $\vec{u}\in U$, we have
  $(c+d)\odot\vec{u} = (c\odot\vec{u})\oplus(d\odot\vec{u})$.

  \textsc{Proof:} any elements of $U$ are elements of $V$, so reconsider
  $\vec{u}$ as elements of $V$. Then the condition holds
  (since $V$ is a real vector space).
\item Unit of $\odot$: for any $\vec{u}\in U$, $1\odot\vec{u}=\vec{u}$.

  \textsc{Proof:} any elements of $U$ are elements of $V$, so reconsider
  $\vec{u}$ as elements of $V$. Then the condition holds
  (since $V$ is a real vector space).
\end{enumerate}
Hence $U$ satisfies the conditions to be a real vector space when
equipped with $\oplus$ and $\odot$ from $V$.
\end{proof}

\begin{example}
Let $V$ be any real vector space. We can define the \define{Trivial Subspace}
of $V$ to be the set $0=\{\vec{0}\in V\}$ consisting of just the zero
vector. This is closed under addition $\vec{0}\oplus\vec{0}=\vec{0}$ and
under scalar multiplication $\forall c\in\RR,c\odot\vec{0}=\vec{0}$.
Hence $0$ is a subspace of $V$.
\end{example}

\begin{corollary}[Subspace iff closed under arbitrary linear combinations]\label{cor:subspaces:subspace-iff-closed-under-linear-combos}
Let $V$ be a real vector space and $U\subset V$.
Then $U$ is a subspace of $V$ if and only if for every $u_{1},u_{2}\in U$
and $c_{1},c_{2}\in\RR$ we have $(c_{1}\odot u_{1})\oplus(c_{2}\odot u_{2})\in U$.
\end{corollary}

\begin{proof}
$(\implies)$ Assume $U$ is a subspace of $V$.
Then for any $\vec{u}_{1},\vec{u}_{2}\in U$
and $c_{1},c_{2}\in\RR$ we see $c_{1}\odot\vec{u}_{1}$ and $c_{2}\odot\vec{u}_{2}$
are both in $U$ and therefore their vector sum is in $U$ as well, i.e.,
$(c_{1}\odot\vec{u}_{1})\oplus(c_{2}\odot\vec{u}_{2})\in U$.

$(\impliedby)$ Assume for every $\vec{u}_{1},\vec{u}_{2}\in U$
and $c_{1},c_{2}\in\RR$ we have $(c_{1}\odot \vec{u}_{1})\oplus(c_{2}\odot \vec{u}_{2})\in U$.
Then we see when $c_{1}=1$ and $c_{2}=1$ we have our assumption become:
\begin{enumerate}[label=(\alph*)]
\item for every $\vec{u}_{1},\vec{u}_{2}\in U$ their vector sum
  $\vec{u}_{1}\oplus\vec{u}_{2}\in U$.
\end{enumerate}
When $c_{2}=0$ (and/or $\vec{u}_{2}=\vec{0}$), our assumption becomes:
\begin{enumerate}[resume*]
\item For every $c_{1}\in\RR$ and $\vec{u}_{1}\in U$,
  $c_{1}\odot\vec{u}_{1}\in U$.
\end{enumerate}
These are precisely stating $U$ is closed under $\oplus$ and $\odot$,
which implies $U$ is a subspace of $V$.
\end{proof}

\begin{remark}
The notion of ``arbitrary linear combinations [of elements of a subset
  of a vector space]'' turns out to be the critical key idea here. We
want to give it a name, because we will use it quite a bit.
\end{remark}

\begin{definition}
Let $\vec{v}_{1}$, \dots, $\vec{v}_{k}\in V$ be vectors. We call
$\vec{u}\in V$ a \define{Linear Combination} of $\vec{v}_{1}$, \dots,
$\vec{v}_{k}$ if there exists $c_{1},\dots,c_{k}\in\RR$ such that
\begin{equation}
\vec{u} = c_{1}\vec{v}_{1} + \cdots + c_{k}\vec{v}_{k}.
\end{equation}
\end{definition}

\begin{definition}
Let $V$ be a real vector space, let $S\subset V$ be a (possibly
infinite) set of a vectors. Then the \define{Span} of $S$ is the set of
\emph{finite} linear combinations of elements of $S$:
\begin{equation}
\Span(S) =
\{\sum^{k}_{j=1}\lambda_{j}\vec{v}_{j}\mid\lambda_{j}\in\RR,\vec{v}_{j}\in S,k\in\NN,j=1,\dots,k\}
\end{equation}
If $W\subset V$ is a subspace of $V$ such that $\Span(S) = W$, then we
call $S$ a \define{Spanning Set} of $W$.
\end{definition}

\begin{remark}
Did I mention that only finite linear combinations of elements of $S$
are allowed in the $\Span(S)$? Because that ``finite'' part is critical.
\end{remark}

\begin{proposition}[Spans are subspaces]
Let $V$ be a real vector space, $S\subset V$ a nonempty subset.
Then $\Span(S)$ is a subspace of $V$.
\end{proposition}

\begin{proof}
This follows from Corollary~\ref{cor:subspaces:subspace-iff-closed-under-linear-combos}.
The sum of two linear combinations is another linear combination.
\end{proof}

\N{Puzzle: ``Best'' Spanning Sets?}
If we have a vector space $V$, is there a ``best'' spanning set
$S\subset V$ (such that $\Span(S)=V$)? We may have ``redundancies'', for
example if $s_{1}\in S$ and $s_{2}\in S$, then we don't really need
$s_{1}+s_{2}\in S$. So it seems ``minimal'' is a good measure of
``best-ness''. Is there a ``minimal'' spanning set (in some appropriate
sense)?

\phantomsection
\subsection*{Exercises}
\addcontentsline{toc}{subsection}{Exercises}

\begin{exercise}
Consider the set of polynomials with real coefficients
  of degree less than or equal to $2$,
  $V=\{p\in\RR[x]\mid\deg(p)\leq2\}$ forms a real vector space
  where $(p_{0}+p_{1}x+p_{2}x^{2})\oplus(q_{0}+q_{1}x+q_{2}x^{2})=(p_{0}+q_{0})+(p_{1}+q_{1})x+(p_{2}+q_{2})x^{2}$
  and $c\odot(p_{0}+p_{1}x+p_{2}x^{2})=(cp_{0})+(cp_{1})x+(cp_{2})x^{2}$
  for arbitrary $c\in\RR$,
  $(p_{0}+p_{1}x+p_{2}x^{2})$, $(q_{0}+q_{1}x+q_{2}x^{2})\in V$.
  Is $V$ a subspace of the set of all polynomials with real coefficients $\RR[x]$?
  We know it's a subset, but is it a sub\emph{space}?
\end{exercise}

\begin{exercise}
Let $W$ be a real vector space.

Prove or find a counter-example: If $U$ is a subspace of $V$ and $V$ is
a subspace of $W$, then is $U$ a subspace of $W$?
\end{exercise}

\begin{exercise}
Consider the set $C^{\infty}(\RR)$ of smooth real functions.
Is this a subspace of $C(\RR)$ all continuous functions from the reals
to the real numbers?
Is it a subspace of $\{f\colon\RR\to\RR\}$ all functions (all of them --- continuous,
discontinuous, nowhere continuous --- it doesn't matter, all of them)?
\end{exercise}

\begin{exercise}
Consider the set of Riemann integrable functions on some interval
$[a,b]\subset\RR$. Prove or find a counter-example: this a subspace of all functions $\{f\colon[a,b]\to\RR\}$?

[Hint: can you construct a sequence of integrable functions which
  converges to a nonintegrable function?]
\end{exercise}

\begin{exercise}
Consider the subset $S = \{1,x,1-x^{2}\}\subset\RR[x]$. Compute $\Span(S)$.
\end{exercise}

\subsection{Orthogonal Complements and Direct Sums}

\begin{definition}
Let $U\subset V$ be a subspace. We define its \define{Orthogonal Complement}
of $U$ to be the set (not subspace, but a set) of vectors
\begin{equation}
U^{\perp} = \{\vec{w}\in V\mid \mbox{for every}~\vec{u}\in U, \vec{w}\cdot\vec{u}=0\}.
\end{equation}
\end{definition}

\begin{proposition}
For any $U\subset V$ subspace, its orthogonal complement $U^{\perp}$ is
a subspace of $V$.
\end{proposition}

\begin{proof}
  We will invoke Theorem~\ref{thm:subspaces:subset-closed-under-linear-combos-is-a-subspace}
  and prove $U^{\perp}$ is closed under vector addition and scalar
  multiplication.

  Let $\vec{w}_{1},\vec{w}_{2}\in U^{\perp}$ be arbitrary. Then for any
$\vec{u}\in U$ we have
\begin{calculation}
  \vec{u}\cdot(\vec{w}_{1}+\vec{w}_{2})
  \step{distributivity of scalar multiplication over vector addition}
  \vec{u}\cdot\vec{w}_{1}+\vec{u}\cdot\vec{w}_{2}
  \step{since $\vec{w}_{1},\vec{w}_{2}\in U^{\perp}$, definition of $U^{\perp}$}
  0 + 0 = 0
\end{calculation}
hence $(\vec{w}_{1}+\vec{w}_{2})\in U^{\perp}$. So it is closed under
vector addition.

Let $\vec{w}\in U^{\perp}$ and $c\in\RR$ be arbitrary. For any
$\vec{u}\in U$, we have
\begin{calculation}
  (c\vec{w})\cdot\vec{u}
  \step{associativity}
  c(\vec{w}\cdot\vec{u})
  \step{definition of orthogonal complement}
  c(0) = 0.
\end{calculation}
Hence $c\vec{w}\in U^{\perp}$ and moreover $U^{\perp}$ is closed under
scalar multiplication.

Thus $U^{\perp}$ is a subspace of $V$ by Theorem~\ref{thm:subspaces:subset-closed-under-linear-combos-is-a-subspace}.
\end{proof}

\begin{definition}
Let $U_{1}\subset V$ and $U_{2}\subset V$ be subspaces with a trivial
intersection $U_{1}\cap U_{2}=\{\vec{0}_{V}\}$.
Then we define the \define{Direct Sum} of $U_{1}$ with $U_{2}$ to be the
set
$U_{1}\oplus U_{2}=\{\vec{u}_{1} + \vec{u}_{2}\in V\mid \vec{u}_{1}\in U_{1}, \vec{u}_{2}\in U_{2}\}$.
\end{definition}

\begin{proposition}
If $U_{1}\subset V$ and $U_{2}\subset V$ are subspaces with trivial
intersection $U_{1}\cap U_{2}=\{\vec{0}_{V}\}$, then their direct sum
$U_{1}\oplus U_{2}$ is a subspace of $V$.
\end{proposition}

\begin{proposition}
For any subspace $U\subset V$, we have $V=U^{\perp}\oplus U$.
\end{proposition}
