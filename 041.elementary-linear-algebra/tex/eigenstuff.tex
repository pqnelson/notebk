\section{Eigenstuff}

\N{Note to self}
I wanted to include this as the ending of part II, but I realized the
usefulness of eigenstuff is in having eigenvectors forming a basis, and
then diagonalizing the matrix. This requires putting this section in part III.

\subsection{Eigenstuff for Matrices}

\begin{example}[Motivating Example]
  Consider the matrix
  \begin{equation}
\mat{M} = \begin{pmatrix}2 & -1\\
-1 & 2
\end{pmatrix}.
  \end{equation}
  We find that there are two ``directions'' (distinct vectors) which are
  just dilated when we multiply by $\mat{M}$,
  \begin{subequations}
    \begin{equation}
\mat{M}\begin{pmatrix}1\\1\end{pmatrix} = \begin{pmatrix}1\\1\end{pmatrix},
    \end{equation}
    and
    \begin{equation}
\mat{M}\begin{pmatrix}1\\-1\end{pmatrix} = 3\begin{pmatrix}1\\-1\end{pmatrix}.
    \end{equation}
  \end{subequations}
  This isn't a neat parlor trick: it turns out any invertible $n\times n$
  matrix will have at most $n$ vectors which are ``dilated'' by the matrix.
\end{example}

\begin{definition}
Let $\mat{A}$ be an $n\times n$ matrix.
We define an \define{Eigenvector} of $\mat{A}$ to be a [column]
$n$-vector $\vec{v}$ such that there is a nonzero $\lambda\in\RR$
[called the \define{Eigenvalue} associated with $\vec{v}$] satisfying
\begin{equation}\label{eq:defn:eigenvector}
\mat{A}\vec{v}=\lambda\vec{v}.
\end{equation}
\end{definition}

\begin{remark}[Linear Operators can have eigenstuff]
We also have an eigenvector for a linear operator $L\colon V\to V$
be a vector $\vec{v}\in V$ such that there is a nonzero $\lambda\in\RR$
satisfying $L(\vec{v})=\lambda\vec{v}$.
We don't even require $V$ to be finite-dimensional (but it's far easier
when $\dim(V)<\infty$).
\end{remark}

\N{Finding Eigenvalues and Eigenvectors}
This is great, but how do we find eigenvalues and eigenvectors?
The first thing to note is we can rewrite Eq~\eqref{eq:defn:eigenvector}
by subtracting $\lambda\vec{v}$ from both sides:
\begin{equation}
\mat{A}\vec{v}-\lambda\vec{v}=\vec{0}.
\end{equation}
We insert a secret identity operator (the matrix analog of ``multiply by $1$''):
\begin{equation}
\mat{A}\vec{v}-\lambda\mat{I}\vec{v}=\vec{0}.
\end{equation}
We can factor out $\vec{v}$ by distributivity:
\begin{equation}
(\mat{A}-\lambda\mat{I})\vec{v}=\vec{0}.
\end{equation}
For this equation to hold, either $\vec{v}=0$ or
$(\mat{A}-\lambda\mat{I})=0$, right?

Wrong: $(\mat{A}-\lambda\mat{I})$ could be nonzero and noninvertible.
That is when
\begin{equation}
\det(\mat{A}-\lambda\mat{I})=0.
\end{equation}
But the left-hand side is not identically zero. In fact, the left-hand
side is a polynomial in $\lambda$.
\emph{This polynomial is how we find eigenvalues for matrices.}

Once we have an eigenvalue, we can plug it in and then solve the system
of equations for the eigenvector. But first, let us define this
polynomial quantity.

\begin{definition}
Let $\mat{A}$ be an $n\times n$ matrix.
The \define{Characteristic Polynomial} of $\mat{A}$ is
\begin{equation}
p(\lambda) = \det(\mat{A}-\lambda\mat{I}).
\end{equation}
Some authors use $\det(\lambda\mat{I}-\mat{A})$, it doesn't matter since
they have the same roots (which are the eigenvalues of $\mat{A}$ and the
\emph{actual} quantity of interest).
\end{definition}

\N{Pop quiz.} If we multiply a row of $\mat{A}$ by a nonzero scalar
$c\in\RR$, then will this change the characteristic polynomial for $\mat{A}$?

\begin{example}
Recall our motivating example at the start of this section, we had
\begin{equation}
\mat{M} = \begin{pmatrix}2 & -1\\
-1 & 2
\end{pmatrix}.
\end{equation}
Its characteristic polynomial is
\begin{equation}
\det\begin{pmatrix}2-\lambda & -1\\
-1 & 2-\lambda
\end{pmatrix} = (2-\lambda)^{2}-1 = \lambda^{2}-4\lambda+3.
\end{equation}
We find this has roots $\lambda=1$ and $\lambda=3$.

We can find the eigenvector for $\lambda=1$ by solving
\begin{subequations}
  \begin{align}
    2x_{1} -x_{2} &= x_{1}\\
    -x_{1} + 2x_{2} &= x_{2}.
  \end{align}
\end{subequations}
These give us 2 copies of the same line described by
\begin{equation}
-x_{1} = -x_{2},\quad\mbox{or}\quad x_{1}=x_{2}.
\end{equation}
We have a generic eigenvector look like
\begin{equation*}
\vec{v}_{1} = m\begin{pmatrix}1\\1
\end{pmatrix}
\end{equation*}
where $m\in\RR$. Usually we normalize the eigenvector to be a unit
vector (so we fix any such parameters), which gives us
\begin{equation}
  \vec{v}_{1} = \begin{pmatrix}1/\sqrt{2}\\
    1/\sqrt{2}
  \end{pmatrix}.
\end{equation}
This is one eigenvector.

The other eigenvector, the one associated with $\lambda=3$, requires
solving the system of equations
\begin{subequations}
  \begin{align}
    2x_{1} -x_{2} &= 3x_{1}\\
    -x_{1} + 2x_{2} &= 3x_{2}.
  \end{align}
\end{subequations}
This gives us two copies of the same line, described by the equation
\begin{equation}
x_{1} = -x_{2}.
\end{equation}
The unit eigenvector is then
\begin{equation}
\vec{v}_{2} = \begin{pmatrix}1/\sqrt{2}\\
-1/\sqrt{2}
\end{pmatrix}.
\end{equation}
The reader may verify these satisfy the equation $\mat{M}\vec{v}=\lambda\vec{v}$
for eigenvectors of $\mat{M}$.
\end{example}

% \subsection{Eigenstuff for Linear Operators}

\M
Everything we've said carries over to linear operators acting on a
vector space $L\colon V\to V$.

\begin{lemma}\label{lemma:linear-transformations:det-and-tr-using-eigenvalues}
If $\mat{A}$ is an $n\times n$ matrix with eigenvalues $\lambda_{1}$,
$\dots$, $\lambda_{n}$, then
\begin{equation}
\det(\mat{A}) = \prod^{n}_{j=1}\lambda_{j}
\end{equation}
and
\begin{equation}
\tr(\mat{A}) = \sum^{n}_{j=1}\lambda_{j}.
\end{equation}
\end{lemma}

\begin{proof}
We suppose there are
$n$ eigenvalues for our $n\times n$ matrix $\lambda_{1}$, \dots, $\lambda_{n}$.
We have the following, starting from the characteristic polynomial for $\mat{A}$:
\begin{calculation}
  p_{\mat{A}}(\lambda)
  \step{definition of characteristic polynomial}
  \det(\mat{A} - \lambda\mat{I})
  \step{fundamental theorem of algebra, Theorem~\ref{thm:fundamental-theorem-of-algebra}}
  (\lambda_{1} - \lambda)(\cdots)(\lambda_{n} - \lambda)
\end{calculation}
(Recall (\S\ref{thm:fundamental-theorem-of-algebra}), the fundamental
theorem of algebra states if $p(\lambda)$ is a polynomial of degree $n$
with roots $\lambda_{1}$, \dots, $\lambda_{n}$, then $p(\lambda)=c(\lambda-\lambda_{1})(\cdots)(\lambda-\lambda_{n})$.)
Now, if we set $\lambda=0$ to the characteristic polynomial, we find
\begin{equation}
\det(\mat{A}) = \prod^{n}_{j=1}\lambda_{j},
\end{equation}
as desired.

For the trace, we again start with the characteristic polynomial, but
divided by $(-\lambda)^{n}$,
\begin{equation}
\frac{p_{\mat{A}}(\lambda)}{(-\lambda)^{n}} =\det\left(\mat{I} + \frac{\mat{A}}{\lambda}\right).
\end{equation}
Set $\varepsilon=1/\lambda$ and take $\lambda$ large enough so
$0<\varepsilon\ll1$, then by Proposition~\ref{prop:determinant:trace-as-infinitesimal-determinant},
\begin{equation}
  \det\left(\mat{I} + \frac{\mat{A}}{\lambda}\right) =
  1 + \frac{\tr(\mat{A})}{\lambda} + \mathcal{O}(\lambda^{-2}).
\end{equation}
Here $\mathcal{O}(\lambda^{-2})$ means there are terms involving factors
of $\lambda^{-2}$ or some power of it, which we do not care about, and
``sweep under the rug''.

But we see, by expanding out the left-hand side,
\begin{equation}
\frac{\det\left(\lambda\mat{I} + \mat{A}\right)}{\lambda^{n}} =
\prod^{n}_{j=1}\frac{\lambda + \lambda_{j}}{\lambda}
=\prod^{n}_{j=1}\left(1 + \frac{\lambda_{j}}{\lambda}\right).
\end{equation}
Again, taking $\varepsilon=1/\lambda$ and expanding out the product
gives us
\begin{equation}
\frac{\det\left(\lambda\mat{I} + \mat{A}\right)}{\lambda^{n}}
=\prod^{n}_{j=1}\left(1 + \frac{\lambda_{j}}{\lambda}\right)
=1 + \frac{1}{\lambda}\sum^{n}_{j=1}\lambda_{j} + \mathcal{O}(\lambda^{-2}).
\end{equation}
Then equating results, we find
\begin{equation}
\tr(\mat{A}) = \sum^{n}_{j=1}\lambda_{j}
\end{equation}
as desired.
\end{proof}

\begin{remark}
As a consistency check, observe if $\mat{A}$ is singular, then it has a
zero eigenvalue. The product of its eigenvalues would be zero. Its
determinant (as a singular matrix) would be zero. So our lemma is
consistent with this special case.
\end{remark}

\subsection{Diagonalization}

\begin{definition}
  We call an $n\times n$ matrix $\mat{M}$ \define{Diagonalizable} if
  there exists a diagonal $n\times n$ matrix $\mat{D}$ such that
  $\mat{M}\sim\mat{D}$; i.e., if there is an invertible $n\times n$
  matrix $\mat{P}$ such that
  $\mat{M} = \mat{P}^{-1}\mat{D}\mat{P}$.
\end{definition}

\begin{lemma}
The identity matrix is similar only to itself.
\end{lemma}
\begin{proof}
  For $\mat{I}_{n}$, pick any invertible matrix $\mat{P}$. We find
\begin{calculation}
  \mat{P}^{-1}\mat{I}_{n}\mat{P}
  \step{associativity of matrix multiplication}
  \mat{P}^{-1}(\mat{I}_{n}\mat{P})
  \step{defining property of identity matrix}
  \mat{P}^{-1}\mat{P}
  \step{defining property of matrix inverse}
  \mat{I}_{n}.
\end{calculation}
Hence the result.
\end{proof}

\begin{theorem}
Similar $n\times n$ matrices $\mat{A}\sim\mat{B}$ have the same eigenvalues.
\end{theorem}
\begin{proof}
Let $\mat{P}$ be such that $\mat{A}=\mat{P}^{-1}\mat{B}\mat{P}$.
We will prove the characteristic polynomial for $\mat{A}$ is the same as
for $\mat{B}$. We just unfold the characteristic polynomial for $\mat{B}$,
\begin{calculation}
  p_{\mat{B}}(\lambda)
\step{definition of characteristic polynomial}
  \det(\lambda\mat{I}_{n} - \mat{B})
\step{unfolding definition of $\mat{B}$}
  \det(\lambda\mat{I}_{n} - \mat{P}^{-1}\mat{A}\mat{P})
\step{identity matrix is always similar to itself}
  \det(\lambda\mat{P}^{-1}\mat{I}_{n}\mat{P} - \mat{P}^{-1}\mat{A}\mat{P})
\step{distributivity}
  \det(\mat{P}^{-1}(\lambda\mat{I}_{n} - \mat{A})\mat{P})
\step{determinant of product is product of determinants}
  \det(\mat{P}^{-1})\det(\lambda\mat{I}_{n} - \mat{A})\det(\mat{P})
\step{commutativity of multiplication of numbers}
  \det(\lambda\mat{I}_{n} - \mat{A})\det(\mat{P}^{-1})\det(\mat{P})
\step{product of determinants is determinant of product}
  \det(\lambda\mat{I}_{n} - \mat{A})\det(\mat{P}^{-1}\mat{P})
\step{definition of matrix inverse}
  \det(\lambda\mat{I}_{n} - \mat{A})\det(\mat{I})
\step{determinant of identity matrix is $1$}
  \det(\lambda\mat{I}_{n} - \mat{A})
\step{definition of characteristic polynomial}
  p_{\mat{A}}(\lambda)
\end{calculation}
Then $\mat{A}$ and $\mat{B}$ have identical characteristic polynomials
and, moreover, this means they have identical eigenvalues.
\end{proof}

\begin{theorem}\label{thm:linear-transformations:diagonalizable-iff-eigenvectors-linearly-independent}
An $n\times n$ matrix $\mat{A}$ can be diagonalized if and only if it
has $n$ linearly independent eigenvectors.
\end{theorem}

\begin{proof}
\begin{subequations}
$(\implies)$ Assume $\mat{A}$ can be diagonalized. Then there exists an
  invertible matrix $\mat{P}$ and a diagonal matrix $\mat{D}$ such that
\begin{equation}
\mat{P}^{-1}\mat{A}\mat{P}=\mat{D}.
\end{equation}
  We can multiply both sides on the left by $\mat{P}$ to get
\begin{equation}
\mat{A}\mat{P}=\mat{P}\mat{D}.
\end{equation}
If $\mat{P} = (\vec{p}_{1}|\dots|\vec{p}_{n})$ is written out as $n$
column vectors, and if $\mat{D} = \diag(\lambda_{1},\dots,\lambda_{n})$,
then we have our equation become:
\begin{equation}
  (\mat{A}\vec{p}_{1}|\dots|\mat{A}\vec{p}_{n}) = (\lambda_{1}\vec{p}_{1}|\dots|\lambda_{n}\vec{p}_{n}).
\end{equation}
This is a system of $n$ eigenvectors $\vec{p}_{1}$, \dots, $\vec{p}_{n}$
and $n$ eigenvalues $\lambda_{1}$, \dots, $\lambda_{n}$. Since $\mat{P}$
is invertible, its columns are linearly independent by Corollary~\ref{cor:basis:invertible-matrix-iff-columns-are-linearly-independent}. Hence the columns
of $\mat{P}$ (which are the eigenvectors of $\mat{A}$) form a basis and,
in particular, are all nonzero.
\end{subequations}

\begin{subequations}
$(\impliedby)$ Assume $\mat{A}$ has $n$ linearly independent
eigenvectors $\vec{x}_{1}$, \dots, $\vec{x}_{n}$ with associated
eigenvalues $\lambda_{1}$, \dots, $\lambda_{n}$. Construct the matrix
\begin{equation}
\mat{P} = (\vec{x}_{1}\mid\dots\mid\vec{x}_{n}).
\end{equation}
The columns of $\mat{P}$ are linearly independent, hence $\mat{P}$ is
invertible by Corollary~\ref{cor:basis:invertible-matrix-iff-columns-are-linearly-independent}. We find
\begin{equation}
\mat{A}\mat{P} = (\mat{A}\vec{x}_{1}\mid\dots\mid\mat{A}\vec{x}_{n})
\end{equation}
and since the $\vec{x}_{i}$ are all eigenvectors, we have,
\begin{equation}
\mat{A}\mat{P} = (\lambda_{1}\vec{x}_{1}\mid\dots\mid\lambda_{n}\vec{x}_{n}).
\end{equation}
But in the first half of the proof, we saw the right-hand side is
precisely equal to $\mat{P}\mat{D}$ where
$\mat{D}=\diag(\lambda_{1},\dots,\lambda_{n})$. This establishes
$\mat{A}\sim\mat{D}$ and that $\mat{A}$ is diagonalizable.
\end{subequations}
\end{proof}

\begin{remark}
We accidentally proved too much in the forward direction. Namely, we
established the eigenvectors form a basis.
\end{remark}

\begin{theorem}
If the roots of the characteristic polynomial for an $n\times n$ matrix
$\mat{A}$ are all distinct (i.e., there are $n$ distinct roots with no
duplicates), then $\mat{A}$ is diagonalizable.
\end{theorem}

In other words, if $\mat{A}$ has $n$ distinct eigenvalues, then
$\mat{A}$ is diagonalizable.

\begin{proof}
Let $\mat{A}$ have eigenvalues $\lambda_{1}$, \dots, $\lambda_{n}$ and
associated eigenvectors $\vec{x}_{1}$, \dots, $\vec{x}_{n}$. Our goal is
to prove the eigenvectors are linearly independent (since they would
form the columns of a matrix $\mat{P}$, as outlined in the proof of the
previous theorem).

Assume for contradiction that the eigenvectors are not all linearly
independent. Then we can find some $\vec{x}_{j}$ which is a linear
combination of the vectors $\vec{x}_{1}$, \dots, $\vec{x}_{j-1}$ and
that these $j-1$ vectors $S=\{\vec{x}_{1}, \dots, \vec{x}_{j-1}\}$
are linearly independent.

So we find
\begin{equation}\label{eq:linear-transformations:diagonalizable-iff-distinct-eigenvalues:first-step}
\vec{x}_{j} = c_{1}\vec{x}_{1} + \cdots + c_{j-1}\vec{x}_{j-1}.
\end{equation}
This is the first step.

Now we multiply on the left of both sides of Eq~\eqref{eq:linear-transformations:diagonalizable-iff-distinct-eigenvalues:first-step}
by $\mat{A}$ which gives us
\begin{equation}
\mat{A}\vec{x}_{j} = c_{1}\mat{A}\vec{x}_{1} + \cdots + c_{j-1}\mat{A}\vec{x}_{j-1}.
\end{equation}
But these are eigenvectors, so we can plug in their eigenvalues
\begin{equation}\label{eq:linear-transformations:diagonalizable-iff-distinct-eigenvalues:second-step}
\lambda_{j}\vec{x}_{j} = c_{1}\lambda_{1}\vec{x}_{1} + \cdots + c_{j-1}\lambda_{j-1}\vec{x}_{j-1}.
\end{equation}
This is the second step.

Now we return to Eq~\eqref{eq:linear-transformations:diagonalizable-iff-distinct-eigenvalues:first-step},
and multiply both sides by $\lambda_{j}$
\begin{equation}
\lambda_{j}\vec{x}_{j} = c_{1}\lambda_{j}\vec{x}_{1} + \cdots + c_{j-1}\lambda_{j}\vec{x}_{j-1}.
\end{equation}
We subtract from this equation the result of our second step, i.e., Eq~\eqref{eq:linear-transformations:diagonalizable-iff-distinct-eigenvalues:second-step}
\begin{subequations}
\begin{align}
    \vec{0} &= \lambda_{j}\vec{x}_{j} - \lambda_{j}\vec{x}_{j}\\
    &= (c_{1}\lambda_{j}\vec{x}_{1} + \cdots + c_{j-1}\lambda_{j}\vec{x}_{j-1})
 - (c_{1}\lambda_{1}\vec{x}_{1} + \cdots + c_{j-1}\lambda_{j-1}\vec{x}_{j-1})\\
&= c_{1}(\lambda_{j}-\lambda_{1})\vec{x}_{1} + \cdots + c_{j-1}(\lambda_{j}-\lambda_{j-1})\vec{x}_{j-1}.
\end{align}
\end{subequations}
We see for this to work, we need every term to vanish. Since
$\vec{x}_{i}\neq\vec{0}$ for every $i=1,\dots,n$, we need the
coefficients to vanish. But since $\lambda_{j}\neq\lambda_{i}$ for
$i\neq j$, it would be impossible for $\lambda_{j}-\lambda_{i}=0$. We are
forced to conclude that $c_{i}=0$ for $i=1,\dots,j-1$.

This contradicts the assumption that the eigenvectors are linearly dependent,
and we conclude they are all linearly independent. Then by Theorem~\ref{thm:linear-transformations:diagonalizable-iff-eigenvectors-linearly-independent}
it follows $\mat{A}$ is diagonalizable.
\end{proof}

\N{Importance of diagonalizable matrices}
If $\mat{A}$ is diagonalizable, then
$\mat{A}=\mat{P}\mat{D}\mat{P}^{-1}$. We see
\begin{subequations}
\begin{calculation}
  \mat{A}^{2}
\step{unfolding $\mat{A}$}
  (\mat{P}\mat{D}\mat{P}^{-1})(\mat{P}\mat{D}\mat{P}^{-1})
\step{associativity of matrix multiplication}
  \mat{P}\left(\mat{D}(\mat{P}^{-1}\mat{P})\mat{D}\right)\mat{P}^{-1}
\step{defining property of matrix inverse}
  \mat{P}\left(\mat{D}\mat{I}\mat{D}\right)\mat{P}^{-1}
\step{defining property of identity matrix}
  \mat{P}\left(\mat{D}\mat{D}\right)\mat{P}^{-1}
\step{definition of matrix power}
  \mat{P}\mat{D}^{2}\mat{P}^{-1}
\end{calculation}
\end{subequations}
We can similarly find
\begin{subequations}
\begin{calculation}
  \mat{A}^{3}
\step{unfolding definition of matrix power}
  \mat{A}^{2}\mat{A}
\step{unfolding definition of $\mat{A}$ and previous calculation}
  (\mat{P}\mat{D}^{2}\mat{P}^{-1})(\mat{P}\mat{D}\mat{P}^{-1})
\step{associativity of matrix multiplication}
  \mat{P}\left(\mat{D^{2}}(\mat{P}^{-1}\mat{P})\mat{D}\right)\mat{P}^{-1}
\step{defining property of matrix inverse}
  \mat{P}\left(\mat{D}^{2}\mat{I}\mat{D}\right)\mat{P}^{-1}
\step{defining property of identity matrix}
  \mat{P}\left(\mat{D}^{2}\mat{D}\right)\mat{P}^{-1}
\step{definition of matrix power}
  \mat{P}\mat{D}^{3}\mat{P}^{-1}.
\end{calculation}
\end{subequations}
The pattern is clear, now; if we assume (the so-called ``inductive hypothesis''), for arbitrary $n\in\NN$,
\begin{equation}
\mat{A}^{n} = \mat{P}\mat{D}^{n}\mat{P}^{-1}.
\end{equation}
Then we have
\begin{subequations}
\begin{calculation}
  \mat{A}^{n+1}
\step{unfolding definition of matrix power}
  \mat{A}^{n}\mat{A}
\step{unfolding definition of $\mat{A}$ and inductive hypothesis}
  (\mat{P}\mat{D}^{n}\mat{P}^{-1})(\mat{P}\mat{D}\mat{P}^{-1})
\step{associativity of matrix multiplication}
  \mat{P}\left(\mat{D^{n}}(\mat{P}^{-1}\mat{P})\mat{D}\right)\mat{P}^{-1}
\step{defining property of matrix inverse}
  \mat{P}\left(\mat{D}^{n}\mat{I}\mat{D}\right)\mat{P}^{-1}
\step{defining property of identity matrix}
  \mat{P}\left(\mat{D}^{n}\mat{D}\right)\mat{P}^{-1}
\step{definition of matrix power}
  \mat{P}\mat{D}^{n+1}\mat{P}^{-1}.
\end{calculation}
\end{subequations}
This is an inductive proof for the claim:
\begin{equation}
\boxed{\mat{A}^{n} = \mat{P}\mat{D}^{n}\mat{P}^{-1}.}
\end{equation}

\begin{ddanger}
\textsc{Caution:} not all invertible matrices are diagonalizable. Do not
make this mistake! The following example provides a counter-example.
\end{ddanger}

\begin{example}
  The matrix
  \begin{equation}
\mat{J} = \begin{pmatrix}0 & -1\\1 & 0
\end{pmatrix}
  \end{equation}
  is not diagonalizable in $\Mat(\RR;2)$.
\end{example}
\begin{proof}
  We see the characteristic polynomial for $\mat{J}$ is
  \begin{equation}
p(\lambda) = \det\begin{pmatrix}-\lambda & -1\\1 & -\lambda
\end{pmatrix}=\lambda^{2}+1.
  \end{equation}
  This has roots at $\lambda_{\pm}=\pm\sqrt{-1}$, which is not a real
  number. In $\mat(\CC;2)$, we could diagonalize $\mat{J}$ with
  eigenvectors $\vec{v}_{+}=(\I,1)$ and $\vec{v}_{-}=(-\I,1)$. There is
  no way to construct a matrix $\mat{P}\in\Mat(\RR;2)$ that would
  diagonalize $\mat{J}$.
\end{proof}

\N{Puzzle} When will a real $n\times n$ matrix be diagonalizable?

\begin{lemma}\label{lemma:linear-transformations:symmetric-matrices-have-orthogonal-eigenvectors-associated-with-different-eigenvalues}
  Let $\mat{A}$ be a real $n\times n$ matrix.
  If $\mat{A}$ is symmetric (i.e., $\mat{A} = \transpose{\mat{A}}$),
  then eigenvectors associated with distinct eigenvalues are orthogonal.
\end{lemma}

\begin{proof}
Assume $\mat{A}$ is symmetric. We will prove the
eigenvectors (associated with distinct eigenvalues) of $\mat{A}$ are
orthogonal. Let $\lambda_{i}$, $\lambda_{j}$ be eigenvalues with respective eigenvectors $\vec{x}_{i}$,
$\vec{x}_{j}$ which are distinct $\lambda_{i}\neq\lambda_{j}$. Then
\begin{calculation}
  \lambda_{i}(\vec{x}_{i}\cdot\vec{x}_{j})
  \step{compatibility of scalar multiplication and dot product}
  (\lambda_{i}\vec{x}_{i})\cdot\vec{x}_{j}
  \step{since $\vec{x}_{i}$ is an eigenvector with eigenvalue $\lambda_{i}$}
  (\mat{A}\vec{x}_{i})\cdot\vec{x}_{j}
  \step{definition of dot product using matrices}
  \transpose{(\mat{A}\vec{x}_{i})}\vec{x}_{j}
  \step{transpose of product is reverse product of transposes}
  (\transpose{\vec{x}_{i}}\transpose{\mat{A}})\vec{x}_{j}
  \step{associativity of matrix multiplication, and restore dot product}
  \vec{x}_{i}\cdot(\transpose{\mat{A}}\vec{x}_{j})
  \step{since $\mat{A}$ is symmetric}
  \vec{x}_{i}\cdot(\mat{A}\vec{x}_{j})
  \step{since $\vec{x}_{j}$ is an eigenvalue}
  \vec{x}_{i}\cdot(\lambda_{j}\vec{x}_{j})
  \step{scalar multiplication is compatible with dot product}
  \lambda_{j}(\vec{x}_{i}\cdot\vec{x}_{j})
\end{calculation}
then we have (subtracting one side from the other)
\begin{equation}
(\lambda_{i} - \lambda_{j})(\vec{x}_{i}\cdot\vec{x}_{j}) = 0.
\end{equation}
Either $\lambda_{i}=\lambda_{j}$ or $\vec{x}_{i}\cdot\vec{x}_{j}=0$.
But we assumed the eigenvalues are distinct, so $\lambda_{i}\neq\lambda_{j}$.
Hence $\vec{x}_{i}\cdot\vec{x}_{j}=0$ as desired.
\end{proof}

\begin{theorem}[Fundamental Theorem of Algebra]\label{thm:fundamental-theorem-of-algebra}
Any polynomial of degree $n$ $p(x)=p_{0} + p_{1}x + p_{2}x^{2} + \cdots + x^{n}$
may be factorized as a product of its roots $p(x)=(x-x_{1})(x-x_{2})(\cdots)(x-x_{n})$.
\end{theorem}

I am taking this as ``given'', its proof lies outside the scope of these notes.
In upcoming notes on intermediate linear algebra, a proof will be given.

\begin{definition}
Let $\mat{A}$ be an $n\times n$ matrix. We say an eigenvalue
$\lambda_{j}$ of $\mat{A}$ has \define{Multiplicity} $k_{j}$ if the
characteristic polynomial of $\mat{A}$ has $k_{j}$ factors of
$(\lambda-\lambda_{j})$ appearing in it, i.e.,
\begin{equation}
p(\lambda) = (\lambda-\lambda_{j})^{k_{j}}\prod^{n-k_{j}}_{\substack{i\neq j\\\lambda_{i}\neq\lambda_{j}}}(\lambda - \lambda_{i})
\end{equation}
\end{definition}

\begin{definition}
Let $\mat{A}$ be an $n\times n$ matrix with eigenvalue $\lambda_{j}$.
We define the \define{Eigenspace} of $\mat{A}$ for eigenvalue
$\lambda_{j}$ to be the \hyperref[defn:linear-transformations:kernel]{kernel} of $\lambda_{j}\mat{I} - \mat{A}$ (or,
equivalently, the span of all vectors whose eigenvalue is $\lambda_{j}$).
\end{definition}

\begin{remark}[Checking the definition]
We see this is a ``good definition'', since a vector $\vec{v}$ is an
eigenvector (or linear combination of eigenvectors) with eigenvalue $\lambda$,
then $\mat{A}\vec{v}=\lambda\mat{I}\vec{v}$ or equivalently
$(\mat{A}-\lambda\mat{I})\vec{v}=\vec{0}$. Hence $\vec{v}\in\ker(\mat{A}-\lambda\mat{I})$.
This matches our intuition of an ``eigenspace'', we should prove that it
is actually a subspace (hence actually merits the suffix ``-space'').
\end{remark}

\begin{lemma}
If $\lambda_{j}$ is an eigenvalue of $\mat{A}$ with multiplicity $k_{j}$,
then its eigenspace has dimension $k_{j}=\dim(\ker(\lambda_{j}\mat{I}-\mat{A}))$.
\end{lemma}

\begin{lemma}
If $\mat{A}$ is an $n\times n$ matrix with eigenvalue $\lambda$ and two
linearly independent eigenvectors $\vec{v}_{1}$, $\vec{v}_{2}$ both with
eigenvalue $\lambda$, then $\vec{v}_{2} - (\vec{v}_{2}\cdot\widehat{\vec{v}}_{1})\widehat{\vec{v}}_{1}$
is an eigenvector of $\mat{A}$ with eigenvalue $\lambda$.
\end{lemma}

\begin{proof}
  We find by direct computation,
  \begin{calculation}
    \mat{A}(\vec{v}_{2} - (\vec{v}_{2}\cdot\widehat{\vec{v}}_{1})\widehat{\vec{v}}_{1})
    \step{distributivity}
    \mat{A}\vec{v}_{2} - \mat{A}(\vec{v}_{2}\cdot\widehat{\vec{v}}_{1})\widehat{\vec{v}}_{1}
    \step{since $\vec{v}_{2}$ is an eigenvector}
    \lambda\vec{v}_{2} - \mat{A}(\vec{v}_{2}\cdot\widehat{\vec{v}}_{1})\widehat{\vec{v}}_{1}
    \step{linearity}
    \lambda\vec{v}_{2} - (\vec{v}_{2}\cdot\widehat{\vec{v}}_{1})\mat{A}\widehat{\vec{v}}_{1}
    \step{since $\widehat{\vec{v}}_{1}$ is an eigenvector}
    \lambda\vec{v}_{2} - (\vec{v}_{2}\cdot\widehat{\vec{v}}_{1})\lambda\widehat{\vec{v}}_{1}
    \step{commutativity of multiplication for numbers}
    \lambda\vec{v}_{2} - \lambda(\vec{v}_{2}\cdot\widehat{\vec{v}}_{1})\widehat{\vec{v}}_{1}
    \step{distributivity}
    \lambda(\vec{v}_{2} - (\vec{v}_{2}\cdot\widehat{\vec{v}}_{1})\widehat{\vec{v}}_{1})
  \end{calculation}
  Hence we conclude $(\vec{v}_{2} - (\vec{v}_{2}\cdot\widehat{\vec{v}}_{1})\widehat{\vec{v}}_{1})$
  is an eigenvector of $\mat{A}$ with eigenvalue $\lambda$.
\end{proof}

\begin{remark}
As an immediate corollary to this, we can apply the Graham--Schmidt
algorithm (\S\ref{chunk:graham-schmidt}) to eigenvectors of the same
eigenvalue $\lambda$ to construct an orthonormal basis for the
eigenspace of $\lambda$.
\end{remark}

\begin{theorem}[Spectral theorem for real matrices]\label{thm:spectral-theorem}
  Let $\mat{A}$ be a real $n\times n$ matrix. Assume $\det(\mat{A})\neq0$.
  Then $\mat{A}$ is diagonalizable by an orthogonal matrix $\mat{P}$
  (i.e., $\mat{A}=\mat{P}\mat{D}\mat{P}^{-1}$ where $\mat{D}$ is diagonal) if and only if
  $\mat{A}$ is symmetric, that is,
\begin{equation}
\mat{A} = \transpose{\mat{A}}.
\end{equation}
\end{theorem}

\begin{proof}
$(\implies)$ Assume $\mat{A}$ is diagonalizable, so $\mat{A}=\mat{P}\mat{D}\mat{P}^{-1}$
  where $\mat{P}$ is orthogonal. Then $\mat{P}^{-1} = \transpose{\mat{P}}$.
  Then $\mat{A} = \mat{P}\mat{D}\transpose{\mat{P}}$ is clearly
  symmetric, since $\transpose{\mat{A}}=\transpose{(\transpose{\mat{P}})}\mat{D}\transpose{\mat{P}}$.
  Recall the transpose is idempotent $\transpose{(\transpose{\mat{P}})}=\mat{P}$
  which implies $\transpose{\mat{A}}=\mat{P}\mat{D}\transpose{\mat{P}}=\mat{A}$.
  Hence $\mat{A}$ is symmetric.

  \medbreak
  $(\impliedby)$ Assume $\mat{A}$ is symmetric.
For each eigenvalue $\lambda_{j}$ of multiplicity $k_{j}>1$ (i.e., there
are $k_{j}>1$ linearly independent eigenvectors with the same eigenvalue
$\lambda_{j}$), we find $k_{j}$ linearly independent eigenvectors.

Then applying our beloved Graham--Schmidt algorithm
(\S\ref{chunk:graham-schmidt}), we transform these eigenvectors into
orthonormal vectors for the eigenspace.


Then we have the eigenvectors for distinct eigenvalues be orthogonal by Lemma~\ref{lemma:linear-transformations:symmetric-matrices-have-orthogonal-eigenvectors-associated-with-different-eigenvalues},
and we have just constructed orthonormal eigenvectors when they share
the same eigenvalue, and taken altogether these form an orthonormal
basis. These
form the columns of matrix $\mat{P}$, which is an orthogonal matrix (\S\ref{chunk:basis:change-of-basis-matrix-among-orthonormal-bases}), and
we have $\mat{D}=\mat{P}^{-1}\mat{A}\mat{P}=\transpose{\mat{P}}\mat{A}\mat{P}$
be the diagonal matrix of eigenvalues. Then we have the data for
diagonalizing $\mat{A}=\mat{P}\mat{D}\mat{P}^{T}$.
\end{proof}

\subsection{Matrix Exponential}

\N{Matrix Exponential}
In particular, we can define the matrix exponential as
\begin{equation}
\exp(\mat{A}) = \mat{P}\exp(\mat{D})\mat{P}^{-1}
\end{equation}
where
\begin{equation}
\exp(\mat{D}) = \diag(\exp(d_{1}), \dots, \exp(d_{n}))
\end{equation}
the matrix exponential of the diagonal matrix is just the diagonal of
exponentials.

If further $\mat{D}$ consists of strictly positive entries, we could
define the matrix logarithm as
\begin{equation}
\log(\mat{A}) = \mat{P}\log(\mat{D})\mat{P}^{-1}
\end{equation}
where, again, the matrix logarithm of the diagonal is just the
component-wise logarithm of its entries,
\begin{equation}
\log(\mat{D}) = \diag(\log(d_{1}), \dots, \log(d_{n})).
\end{equation}


\begin{theorem}
  For any matrix $\mat{A}$, we have
  \begin{equation}
\det(\exp(\mat{A})) = \E^{\tr(\mat{A})}.
  \end{equation}
\end{theorem}
\begin{proof}
We see that $\exp(\mat{A})=\mat{P}\exp(\mat{D})\mat{P}^{-1}$ has
eigenvalues $\E^{\lambda_{j}}$ where $\lambda_{j}$ are the eigenvalues
of $\mat{A}$. Then by Lemma~\ref{lemma:linear-transformations:det-and-tr-using-eigenvalues},
\begin{equation}
\det(\exp(\mat{A})) = \prod^{n}_{j=1}\E^{\lambda_{j}}
\end{equation}
and the right-hand side can be written as
\begin{equation}
\det(\exp(\mat{A})) = \E^{\sum^{n}_{j=1}\lambda_{j}}.
\end{equation}
But by Lemma~\ref{lemma:linear-transformations:det-and-tr-using-eigenvalues},
\begin{equation}
  \begin{split}
    \det(\exp(\mat{A})) &= \E^{\sum^{n}_{j=1}\lambda_{j}}\\
    &= \E^{\tr(\mat{A})}.
  \end{split}
\end{equation}
Hence the result.
\end{proof}

\begin{remark}
In quantum field theory, we often work with infinite-dimensional vector
spaces (like the smooth functions on spacetime). Linear operators on
these vector spaces include derivatives. For peculiar reasons,
physicists want to compute the ``determinant'' of the differentiation
operator. The previous theorem offers one way to approach such a
concept. We could define ``eigenfunctions'' for a differential operator
$D$ as $D(f)=\lambda f$, then the trace is just the sum over all eigenvalues.
We could define $\det(\exp(D))=\exp(\sum\lambda)$, for example. This is
done more precisely with zeta functions, and is known as the zeta
functional regularization of the functional determinant.
\end{remark}

\begin{corollary}
  For any matrix $\mat{A}$ with real or complex entries,
  we have $\exp(\mat{A})$ be nonsingular.
\end{corollary}

\begin{proof}
We see $\tr(\mat{A})$ is finite --- call it $z=\tr(\mat{A})$ --- and
by the previous theorem
\begin{equation}
\det(\exp(\mat{A})) = \E^{\tr(\mat{A})} = \E^{z}.
\end{equation}
But $\E^{z}\neq0$ ever for $z\in\CC$ (or $z\in\RR$), which means the
determinant of $\exp(\mat{A})$ is never zero. Hence $\exp(\mat{A})$ must
be invertible.
\end{proof}

\N{Analytic Function of a Matrix}
More generally, if $f\colon\RR\to\RR$ is a function with a Taylor series about $x_{0}$
given by
\begin{equation}
f(x_{0}+h)=\sum^{\infty}_{n=0}\frac{a_{n}}{n!}h^{n},
\end{equation}
then it may be applied to a diagonalizable matrix
$\mat{A}=\mat{P}\mat{D}\mat{P}^{-1}$ provided every entry of $\mat{D}$
lies within the neighborhood of convergence for the Taylor series. If
$\mat{D} = \diag(x_{0} + d_{1}, \dots, x_{0} + d_{n})$, then
\begin{equation}
f(\mat{A}) = \mat{P}f(\mat{D})\mat{P}^{-1}
\end{equation}
where $f(\mat{D}) = \diag(f(x_{0} + d_{1}), \dots, f(x_{0}+d_{n}))$.
This follows from $\mat{A}^{n}=\mat{P}\mat{D}^{n}\mat{P}^{-1}$ for every
$n\in\NN_{0}$ and just plugging it into the series expansion.
