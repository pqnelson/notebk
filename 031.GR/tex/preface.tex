%%
%% intro.tex
%% 
%% Made by alex
%% Login   <alex@tomato>
%% 
%% Started on  Tue Feb 14 08:32:16 2012 alex
%% Last update Tue Feb 14 08:32:16 2012 alex
%%
\preface
These are my collected notes on classical general
relativity. These are graduate level notes, and I
have reformatted, merged, and edited them into a cohesive
whole. I doubt these notes could take the place of a textbook,
but may make wonderful supplement to one.

The references used are either books I own, or eprinted
articles. This is the guideline I tried maintaining, but there
are exceptions to the rule of ``free eprinted articles.'' (An
additional problem: some articles are so old that they are not
[yet] eprinted and published online. Sad, I know, but still\dots)

Also be forewarned: the bibliography consists of two
sections. The first consisting of books, which are recommended
for the reader. The second consisting of technical articles,
relevant for points made. 

Strictly speaking, the math used in the first part (the
pedagogical part) is not correct. We will be sloppy, as sloppy as
physicists are. It's not ``incorrect'' per se, but it may give
mathematicians indigestion.

%The plan is to divide this text into three parts (or perhaps
%release three texts):
I hope to write three texts: the first (which you are reading) is
a pedagogical introduction to classical general relativity. The
second concerns advanced portions of general relativity,
preparing the reader for the ADM formalism, numerical relativity,
treatment of spinors, and so on. The third deals with quantum gravity.
