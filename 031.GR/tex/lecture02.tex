%%
%% lecture02.tex
%% 
%% Made by alex
%% Login   <alex@tomato>
%% 
%% Started on  Tue Feb 14 11:53:25 2012 alex
%% Last update Tue Feb 14 11:53:25 2012 alex
%%
Lets review the basic setup: gravity determines paths in
spacetime, a set of preferred paths determine geometry, so we
can try to go backwards and determine the geometry of spacetime
from geodesics. Or given the curvature of spacetime, we can
determine the geodesics. 


If we consider charged bodies in the electromagnetic field, it is
done in two steps: (1) use Maxwell's equations to determine the
electric and magnetic fields; (2) use the Lorentz force Law to
compute trajectories. If we are really careful, general
relativity does the whole thing in a single step. If we have the
field equations, we only get a consistent answer if everything
moves (along a geodesic). In electromagnetism, we can hold
something still with an uncharged body; yet for general
relativity, the equivalence principle says (the analogous
procedure) cannot happen.

\subsection{Geodesics in Special Relativity}

We were talking about deriving geodesics. Lets review spacetime
in special relativity (see, e.g.,
Carroll~\cite[\normalfont\S\S1.1--1.4]{Carroll:2004st}, Gibbons~\cite{gibbons:2010}, or
Giulini~\cite{Giulini:2006uy,Giulini:2008pu} for more
mathematically oriented discussions). The most basic feature:
distances are relative, and time is relative. Events (things that
occur at some place and time) are dependent on the observer, but
the proper time $s$ (or $\tau$) is defined in special relativity
as
\begin{equation}
\D s^2=c^2\D t^2-\D x^2-\D y^2-\D z^2
\end{equation}
very much like Pythagoras' theorem but space and time come in
different signs.\marginpar{Set $c=1$} We will work with units
where $c=1$.

If we have two events, then we can construct the geodesic from
$(x_0,t_0)$ to $(x_1,t_1)$ as 
\begin{equation}
s=\int^{(x_1,t_1)}_{(x_0,t_0)}\D s
\end{equation}
then obtain from $\delta s=0$ the equations of motion for the
geodesic. These are determined, as by last time, to be the
solution of the differential equation
\begin{equation}
\frac{\D^2}{\D s^2}X^{\mu}(s)=0.
\end{equation}
In special relativistic spacetime, geodesics are the trajectories
with the longest proper time, \emph{not} the shortest! We will
consider two examples and then the general case.

\begin{ex}[$\RR^2$ revisited]
Lets consider geodesics in $\RR^2$ using polar coordinates:
\begin{equation}
x=r\cos(\theta),\quad y=r\sin(\theta).
\end{equation}
We then see
\begin{equation}
\begin{split}
\D s^2 &=\D x^2+\D y^2\\
&=\D r^2+r^2\D\theta^2
\end{split}
\end{equation}
We are interested in paths from $(r_0,\theta_0)$ to
$(r_1,\theta_1)$. We label the path by some parameter $u$ and
write
\begin{equation}
s=\int\D s.
\end{equation}
Note: we change the parameter $u=s$ \emph{after} we've chosen the
path. So we write
\begin{equation}
r=r(u),\quad\theta=\theta(u).
\end{equation}
The integral becomes
\begin{equation}
\begin{split}
s
&=\int\frac{\D s}{\D u}\D u\\
&=\int\sqrt{\left(\frac{\D r}{\D u}\right)^{2}+r^{2}\left(\frac{\D\theta}{\D u}\right)^{2}}\D u
\end{split}
\end{equation}
which we extremize. We define
\begin{equation}
E=\left(\frac{\D r}{\D u}\right)^{2}+r^{2}\left(\frac{\D\theta}{\D u}\right)^{2}
\end{equation}
and take the variation
\begin{equation}\label{eq:lec02:ex01:variation}
\begin{split}
\delta s
&=\delta\int E^{1/2}\D u\\
&=\frac{1}{2}\int E^{-1/2}\delta E\D u\\
&=0.
\end{split}
\end{equation}
First we need to compute $\delta E$, which is a triviality:
\begin{equation}
\delta E = 2\frac{\D r}{\D u}\frac{\D\delta r}{\D u}
+2r\delta r\left(\frac{\D\theta}{\D u}\right)^{2}
+2r^{2}\frac{\D\theta}{\D u}\frac{\D\delta\theta}{\D u}.
\end{equation}
Then what? Well, plug it back into Eq \eqref{eq:lec02:ex01:variation}
to find
\begin{align}
\delta s
&=\frac{1}{2}\int\left[
-2\frac{\D}{\D u}\left(E^{-1/2}\frac{\D r}{\D u}\right)\delta r+
E^{-1/2}2r\left(\frac{\D\theta}{\D u}\right)^{2}\delta r-
2\frac{\D}{\D u}\left(E^{-1/2}r^{2}\frac{\D\theta}{\D u}\right)\delta\theta
\right]\D u\nonumber\\
&=0
\end{align}
We require the coefficients of $\delta r$, $\delta\theta$ must
vanish\footnote{Mathematicians may feel uneasy about this, but it is due to the fundamental lemma of variational calculus.}.
We get a set of equations, and we have our particular path. This
allows us \emph{now} to set $u=s$, thus $E=1$, and our equations
\begin{align*}
\frac{\D}{\D s}\left(r^{2}\frac{\D\theta}{\D s}\right)
&=0\tag{$\delta\theta$\text{ coefficient}}\\
-\frac{\D^{2}r}{\D s^{2}}+r\left(\frac{\D\theta}{\D s}\right)^{2}
&=0\tag{$\delta r$\text{ coefficient}}
\end{align*}
Trick \#1: we have 2 second-order Ordinary Differential
Equations. We can do some of the integration without even
thinking about it (although this trick will give mathematicians
indigestion). We have
\begin{equation}\label{eq:ex01:firstInt}
\D s^2=\D r^2+r^2\D\theta^2\implies
\left(\frac{\D r}{\D s}\right)^{2}+r^{2}\left(
\frac{\D\theta}{\D s}\right)^{2}=1
\end{equation}
is the first integral of our two given equations.

The $\delta\theta$ coefficient is easy. It says
\begin{equation}
\begin{split}
r^{2}\frac{\D\theta}{\D s}&=\mbox{(constant)}\\
&\eqdef a.
\end{split}
\end{equation}
We plug this into the Eq \eqref{eq:ex01:firstInt} to find
\begin{equation}
\left(\frac{\D r}{\D s}\right)^{2}+\frac{a^{2}}{r^{2}}=1.
\end{equation}
Thus we have, rearranging and manipulating, the following
expression
\begin{equation}
\begin{split}
\D s&=
\frac{\D r}{\sqrt{1-(a/r)^{2}}}\\
&=\frac{r\D r}{\sqrt{r^{2}-a^{2}}}
\end{split}
\end{equation}
and integration yields
\begin{equation}\label{eq:ex01:rEqS}
s-s_{0}=\sqrt{r^{2}-a^{2}}\implies r^{2}=a^{2}+(s-s_{0})^{2}.
\end{equation}
What to do? Well, we do the only thing we can do! We plug this
expression for $r$ into
\begin{equation*}
r^{2}\frac{\D\theta}{\D s}=a
\end{equation*}
and we obtain
\begin{equation}
\left(a^{2}+(s-s_{0})^{2}\right)\frac{\D\theta}{\D s}=a.
\end{equation}
We know how to solve first order differential equations, so we
just integrate
\begin{equation}
\begin{split}
\theta-\theta_{0}&=\int\frac{a\D s}{a^{2}+(s-s_{0})^{2}}\\
&=\arctan\left(\frac{s-s_{0}}{a}\right).
\end{split}
\end{equation}
But we want to write $s-s_{0}$ in terms of $\theta$, so we can
plug it into Eq \eqref{eq:ex01:rEqS}. What to do? Well, we can
manipulate our result to obtain
\begin{equation}
\theta-\theta_{0}=\arctan\left(\frac{s-s_{0}}{a}\right)
\implies s-s_{0}=a\tan(\theta-\theta_{0}).
\end{equation}
So what? Well, plug this into Eq \eqref{eq:ex01:rEqS}
\begin{equation}
\begin{split}
r^{2}
&=a^{2}\left(1+\tan^{2}(\theta-\theta_{0})\right)\\
&=\frac{a^{2}}{\cos^{2}(\theta-\theta_{0})}.
\end{split}
\end{equation}
So what? Well, this implies
\begin{equation}
r\cos(\theta-\theta_{0})=a
\end{equation}
is constant, which is precisely a straight line. Thus a geodesic
in $\RR^2$ using polar coordinates is precisely a straight line,
the same result we obtained from considering a geodesic using
Cartesian coordinates!
\end{ex}
\begin{ex}[Hyperbolic Plane]\index{Hyperbolic Plane!Geodesics|(}
Lets consider the hyperbolic plane, where
\begin{equation}
\D s^{2}=\frac{1}{y^{2}}\left(\D x^{2}+\D y^{2}\right).
\end{equation}
This is related to de Sitter space.\index{de Sitter space!Relation to Hyperbolic Plane}
We want to find geodesics, so we parametrize a path
\begin{equation}
x=x(u),\quad y=y(u)
\end{equation}
then take
\begin{equation}
E=\frac{1}{y^{2}}\left[
\left(\frac{\D x}{\D u}\right)^{2}+
\left(\frac{\D y}{\D u}\right)^{2}
\right].
\end{equation}
We take the variation
\begin{equation}
\begin{split}
\delta s&=\delta\int E^{1/2}\D u\\
&=\frac{1}{2}\int E^{-1/2}\delta E\D u\\
&=0.
\end{split}
\end{equation}
First we consider the variation
\begin{equation}
\delta E=
\frac{-2}{y^{3}}\delta y\left[
\left(\frac{\D x}{\D u}\right)^{2}+
\left(\frac{\D y}{\D u}\right)^{2}
\right]
+\frac{2}{y^{2}}\left[
\frac{\D x}{\D u}\frac{\D\delta x}{\D u}+
\frac{\D y}{\D u}\frac{\D\delta y}{\D u}
\right]
\end{equation}
Now we plugging this monstrous result into the variation of the
length yields
\begin{equation}
\begin{split}
\delta s&=
\int
E^{-1/2}\left[
\frac{-1}{y}E\delta y
+\frac{1}{y^{2}}\frac{\D x}{\D u}\frac{\D\delta x}{\D u}+
\frac{1}{y^{2}}\frac{\D y}{\D u}\frac{\D\delta y}{\D u}
\right]\D u\\
&=0
\end{split}
\end{equation}
Integration by parts gives the $\delta x$ coefficient
\begin{equation*}\tag{$\delta x$\text{ coefficient}}
\frac{\D}{\D u}\left(\frac{1}{y^{2}}\frac{\D x}{\D u}\right)=0
\end{equation*}
and the first-integral trick gives
\begin{equation}\label{eq:ex02:firstInt}
\frac{1}{y^{2}}\left(\frac{\D x}{\D u}\right)^{2}+
\frac{1}{y^{2}}\left(\frac{\D y}{\D u}\right)^{2}=1.
\end{equation}
The $\delta x$ coefficient may be solved explicitly as
\begin{equation}
\frac{\D x}{\D u}=ky^{2}
\end{equation}
where $k$ is a constant. We plug this into the first-integral
equation \eqref{eq:ex02:firstInt}
\begin{equation}
k^{2}y^{2}+\frac{1}{y}^{2}\left(\frac{\D y}{\D u}\right)^{2}=1.
\end{equation}
There are two possible families of geodesics: when $k\not=0$ and
when $k=0$.

If $k\not=0$, then
\begin{equation}
\frac{\D y}{\D u}=\frac{\D y}{\D x}\frac{\D x}{\D u}
\end{equation}
and we can think of $y$ as a function
\begin{equation}
y=y(x).
\end{equation}
The differential equation becomes
\begin{equation}
\frac{\D y}{\D u}=ky^{2}\frac{\D y}{\D x}
\end{equation}
which we plug into the first-integral
\begin{equation}
k^2y^2+k^2y^2\left(\frac{\D y}{\D u}\right)^{2}=1\implies
\left(\frac{\D y}{\D u}\right)^{2}=\frac{1}{(ky)^{2}}-1.
\end{equation}
We may solve this
\begin{equation}
k^{2}\left[
y^{2}+(x-x_{0})^{2}
\right]=1
\end{equation}
which is a circle! This family of geodesics are circles centered
at $(x_{0},0)$.

If, on the other hand, $k=0$ what happens? We see that the
differential equation
\begin{equation}
\frac{\D x}{\D u}=ky^{2}=0
\end{equation}
implies $x$ is a constant. Thus it is a straight line.
\index{Hyperbolic Plane!Geodesics|)}\end{ex}

\subsection{General Geodesic Equation}
We have\footnote{Remember, we are using the Einstein summation
  convention, so $x_{a}y^{ab}=\sum_{a}x_{a}y^{ab}$. When the
  index appears both ``downstairs'' and ``upstairs'', we sum over
  it implicitly. But note: the indices must have the same dummy
  variable, and one must be downstairs while another upstairs!!!}
\begin{equation}
\D s^{2}=g_{ab}\D x^{a}\D x^{b}
\end{equation}
where $g_{ab}$\marginpar{$g_{ab}$ is metric, $\D s^{2}$ line element}
is called the \define{Metric}\index{Metric Tensor}\index{$g_{ab}$}
and $\D s^2$ is called the \define{Line Element}.
Indices we sum over are called \define{Dummy Indices}, and we may
relabel them as we please
\begin{equation}
\begin{split}
\D s^{2} &= g_{ab}\D x^{a}\D x^{b}\\
&=g_{cd}\D x^{c}\D x^{d}
\end{split}
\end{equation}
The \define{Signature} of the metric means the number of positive
and negative eigenvalues. There are two conventions for general
relativity: $(+---)$ called the \define{West Coast}
convention or \define{Particle Physicists} convention; and
$(-+++)$ called the \define{East Coast} convention or
\define{Relativists Convention}.

Now lets derive the geodesic equation. We want the path which
extremizes the arc-length. This is determined by demanding the
variation of the arc-length vanishes
\begin{equation}
\delta s=0.
\end{equation}
As usual, we define
\begin{equation}
\boxed{
E\eqdef g_{ab}\frac{\D x^{a}}{\D u}\frac{\D x^{b}}{\D u}
}
\end{equation}
where $x^{a}=x^{a}(u)$ labels a path. Thus we have
\begin{equation}
s=\int E^{1/2}\D u
\end{equation}
imply
\begin{equation}
\delta s = \frac{1}{2}\int E^{-1/2}\delta E\D u
\end{equation}
where
\begin{equation}
\delta E=(\delta g_{ab})\frac{\D x^{a}}{\D u}\frac{\D x^{b}}{\D u}
+2g_{ab}\frac{\D x^{a}}{\D u}\frac{\D\delta x^{b}}{\D u}.
\end{equation}
Note that
\begin{equation}
\delta g_{ab}=\frac{\partial g_{ab}}{\partial x^{c}}\delta x^{c}
\end{equation}
since we can ``wiggle'' around a path, the metric varies along
that path. We can notationally write
\begin{equation}
\delta g_{ab}=(\partial_{c} g_{ab})\delta x^{c}
\end{equation}
where
\begin{equation}
\partial_{c}=\frac{\partial}{\partial x^{c}}
\end{equation}
and no, that is not a typo. The index on $\partial_{c}$ is
supposed to be downstairs provided the denominator is $\partial
x^{c}$. The reason is due to how this quantity behaves when we
change coordinates.

Thus
\begin{equation}
\delta E=(\partial_{c}g_{ab})\frac{\D x^{a}}{\D u}\frac{\D x^{b}}{\D u}
+2g_{bc}\frac{\D x^{b}}{\D u}\frac{\D\delta x^{c}}{\D u}
\end{equation}
where we intentionally relabel the dummy indices. This renders
\begin{equation}
\begin{split}
\delta\int\D s&=
\frac{1}{2}\int E^{-1/2}\left[\delta
  x^{c}\partial_{c}g_{ab}\frac{\D x^{a}}{\D u}\frac{\D x^{b}}{\D u}
+2g_{cb}\frac{\D\delta x^{c}}{\D u}\frac{\D x^{b}}{\D u}\right]\D u\\
&=\int\left[\frac{1}{2}E^{-1/2}\partial_{c}g_{ab}\frac{\D x^{a}}{\D u}\frac{\D x^{b}}{\D u}
-\frac{\D}{\D u}\left(E^{-1/2}g_{bc}\frac{\D x^{b}}{\D u}\right)\right]\delta x^{c}\D u\\
&=0
\end{split}
\end{equation}
where the second line, we integrated by parts and threw away the
boundary terms. This implies the bracketed term must vanish:
\begin{equation}
\frac{1}{2}E^{-1/2}\partial_{c}g_{ab}\frac{\D x^{a}}{\D u}\frac{\D x^{b}}{\D u}
-\frac{\D}{\D u}\left(E^{-1/2}g_{bc}\frac{\D x^{b}}{\D u}\right)=0
\end{equation}
and this is our geodesic equation.

We can choose $u=s$ and thus $E=1$, giving us
\begin{equation}
\boxed{
\frac{\D}{\D s}
\left(g_{bc}\frac{\D x^{b}}{\D s}\right)
-\frac{1}{2}\partial_{c}g_{ab}\frac{\D x^{a}}{\D s}\frac{\D x^{b}}{\D s}=0.
}
\end{equation}
This is the geodesic equation. But also note we may pick another
paramter $u$ for which
\begin{equation}
\frac{\D E}{\D u}=0
\end{equation}
and $u$ is called an \define{Affine Parameter}. For light rays it
is conventional to use a $\lambda$ for this parameter. Also
notice
\begin{equation}
\D s^{2}=\begin{cases}
1&\mbox{for massive bodies}\\
0&\mbox{for photons}
\end{cases}
\end{equation}
which is our last observation for now.
