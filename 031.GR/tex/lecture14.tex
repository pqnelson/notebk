%%
%% lecture14.tex
%% 
%% Made by alex
%% Login   <alex@tomato>
%% 
%% Started on  Tue Mar  6 08:04:47 2012 alex
%% Last update Tue Mar  6 08:04:47 2012 alex
%%

Parallel transport $v^{a}$ around a closed loop, and we get
\begin{equation}
v^{a}_{\|}={H^{a}}_{b}v^{b}
\end{equation}
where
\begin{equation}
{H^{a}}_{b}={\delta^{a}}_{b}+\int_{S}{{R_{\mu\nu}}^{a}}_{b}
\,\D x^{\mu}\,\D x^{\nu}+\dots
\end{equation}
is the holonomy (as before).

\begin{wrapfigure}{r}{4pc}
  \vspace{-1pc}
  \includegraphics{img/lecture14.0}
  \vspace{-2pc}
\end{wrapfigure}
We can get the Holonomy as a product of holonomies of arbitrarily
small curves. This idea is doodled on the right, which resembles
the Fibonacci rectangle. In this sense, the curvature tensor
tells you everything. But we have to assume this curvature
encloses a surface. It is possible this is not the case.

\begin{wrapfigure}{l}{4pc}
  \vspace{-1pc}
  \includegraphics{img/lecture14.1}
  \vspace{-2pc}
\end{wrapfigure}
\noindent In the case of the torus, the circle curve $\gamma$
cannot be broken into smaller closed curves. A space is called
\define{Simply Connected} if every loop can be continuously
shrunk to a point (``contracted'').

Curvature is a measure of the inability to define a (universal)
Cartesian coordinate system.

If we look at the commutator of covariant derivatives,we find
\begin{equation}
(\nabla_{\mu}\nabla_{\nu}-\nabla_{\nu}\nabla_{\mu})v^{a}
={{R_{\mu\nu}}^{a}}_{b}v^{b}.
\end{equation}
\begin{xca}
Check this explicitly!
\end{xca}
\noindent\ignorespaces Recall
\begin{equation}
\nabla_{\nu}v^{a}=\partial_{\nu}v^{a}-{{\Gamma_{\nu}}^{a}}_{b}v^{b}.
\end{equation}
We may think of this as an integrability condition, or an
infinitesimal holonomy.


\subsection{Geodesic Deviation}
We would like to compare a one-parameter family of geodesics. One
parameter is $s$, the proper time along the geodesic, and $t$
which labels the geodesic we are on. There are two parameters
\begin{subequations}
\begin{align}
u^{\mu} &= \frac{\partial x^{\mu}}{\partial s}\\
X^{\mu} &= \frac{\partial x^{\mu}}{\partial t}
\end{align}
\end{subequations}
where $u$ tells us the velocity, and $X$ points from one geodesic
to its neighboring geodesic. (Carroll~\cite{Carroll:2004st} refers to these as
$T^{\mu}$ and $S^{\mu}$, respectively.)

We can define a relative velocity
\begin{equation}\label{eq:defnRelativeVelocity}
V^{\mu} = u^{\rho}\nabla_{\rho}X^{\mu}
\end{equation}
which is the relative velocity rate the separation is changing
in time. Similarly, we may define
\begin{equation}
A^{\mu} = u^{\rho}\nabla_{\rho}V^{\mu}
\end{equation}
which is the relative acceleration.

Let us start with an identity
\begin{equation}\label{eq:lec14:id}
u^{\rho}\nabla_{\rho}X^{\mu}=X^{\rho}\nabla_{\rho}u^{\mu}
\end{equation}
%As derivatives
If we consider the difference between the right hand side and the
left hand side
\begin{equation}
u^{\rho}\nabla_{\rho}X^{\mu}-X^{\rho}\nabla_{\rho}u^{\mu}=\dots
\end{equation}
What happens? Well, the first thing to observe is that there are
no terms involving connection components (they drop out). So we
are left with:
\begin{equation}
u^{\rho}\nabla_{\rho}X^{\mu}-X^{\rho}\nabla_{\rho}u^{\mu}=
u^{\rho}\partial_{\rho}X^{\mu}-X^{\rho}\partial_{\rho}u^{\mu}.
\end{equation}
We also use the geodesic equation for autoparallel situations:
\begin{equation}
u^{\mu}\nabla_{\mu}u^{\rho}=0.
\end{equation}
Now what? Well, we will consider
\begin{subequations}
\begin{equation}
u^{\rho}\nabla_{\rho}V^{\mu}=A^{\mu}.
\end{equation}
Using Equation \eqref{eq:defnRelativeVelocity} yields
\begin{equation}
u^{\rho}\nabla_{\rho}V^{\mu}=u^{\rho}\nabla_{\rho}(X^{\nu}\nabla_{\nu}u^{\mu}).
\end{equation}
Invoking Leibniz's rule
\begin{equation}
u^{\rho}\nabla_{\rho}V^{\mu}=\underbracket[0.5pt]{(u^{\rho}\nabla_{\rho}X^{\nu})}_{=V^{\nu}}\nabla_{\nu}u^{\mu}+
X^{\nu}u^{\rho}\nabla_{\rho}\nabla_{\nu}u^{\mu}.
\end{equation}
We use commutation relations on the second term, yielding
\begin{equation}\label{eq:lec14:whatWeHaveSoFar}
u^{\rho}\nabla_{\rho}V^{\mu}=V^{\nu}\nabla_{\nu}u^{\mu}+
X^{\nu}u^{\rho}\nabla_{\nu}\nabla_{\rho}u^{\mu} + 
X^{\nu}u^{\rho}\underbracket[0.5pt]{(\nabla_{\rho}\nabla_{\nu}-\nabla_{\nu}\nabla_{\rho})u^{\mu}}_{{{R_{\rho\nu}}^{\mu}}_{\sigma}u^{\sigma}}
\end{equation}
Now we observe
\begin{equation}
X^{\nu}u^{\rho}\nabla_{\nu}\nabla_{\rho}u^{\mu}=
X^{\nu}\nabla_{\nu}\underbracket[0.5pt]{(u^{\rho}\nabla_{\rho}u^{\mu})}_{=0}-(X^{\nu}\nabla_{\nu}u^{\rho})\nabla_{\rho}u^{\mu}.
\end{equation}
Plugging these expressions back into Equation \eqref{eq:lec14:whatWeHaveSoFar}
\begin{equation}
\begin{split}
u^{\rho}\nabla_{\rho}V^{\mu}&=V^{\nu}\nabla_{\nu}u^{\mu}+0-(X^{\nu}\nabla_{\nu}u^{\rho})\nabla_{\rho}u^{\mu}
+X^{\nu}u^{\rho}{{R_{\rho\nu}}^{\mu}}_{\sigma}u^{\sigma}\\
&=X^{\nu}u^{\rho}{{R_{\rho\nu}}^{\mu}}_{\sigma}u^{\sigma}
\end{split}
\end{equation}
Thus we obtain
\begin{equation}
\boxed{
A^{\mu}=X^{\nu}u^{\rho}{{R_{\rho\nu}}^{\mu}}_{\sigma}u^{\sigma}.
}
\end{equation}
\end{subequations}

Remember $\D x^{\mu}/\D s$ for slow bodies has
\begin{equation}
u^{0}\sim 1,\quad\mbox{and}\quad
u^{i}\sim v/c\ll1.
\end{equation}
Then the relative acceleration
\begin{equation}
A^{i}\sim {{R_{0\nu}}^{i}}_{0}X^{\nu}
\end{equation}
is proportional to the spatial derivative of the gradient of the
potential. In classical Newtonian gravity, this gives tidal forces.

\subsection{Symmetries of Riemann Tensor}
Consider\marginpar{Symmetries of Riemann tensor:\\(1) Skew Symmetries} the curvature tensor $R_{\mu\nu\rho\sigma}$, there are
some symmetries it has (or more precisely, its \emph{indices}
have).  We see that
\begin{subequations}
\begin{align}
R_{\mu\nu\rho\sigma}
&=-R_{\nu\mu\rho\sigma}\\
&=-R_{\mu\nu\sigma\rho}\\
&=R_{\rho\sigma\mu\nu}
\end{align}
\end{subequations}
There are 2 pairs of antisymmetric indices, and those pairs are
symmetric. Thus in 4-dimensions, there are only 21 independent
components of the Riemann tensor. We also have the
Jacobi\marginpar{(2) Jacobi Identity}
identity in the last 3 indices:
\begin{equation}
R_{\mu\nu\rho\sigma}+R_{\mu\rho\sigma\nu}+R_{\mu\sigma\nu\rho}=0.
\end{equation}
We have, in 4-dimensions, only 20 independent components.

The last identity is known as\marginpar{(3) Bianchi Identity}
the \emph{Bianchi Identity}:
\begin{subequations}
\begin{equation}
\nabla_{[\mu}R_{\nu\rho]\sigma\tau}=0
\end{equation}
or equivalently
\begin{equation}
\nabla_{\mu}R_{\nu\rho\sigma\tau}+
\nabla_{\nu}R_{\rho\mu\sigma\tau}+
\nabla_{\rho}R_{\mu\nu\sigma\tau}=0.
\end{equation}
\end{subequations}
This actually follows from
\begin{equation}
\bigl[\nabla_{\mu},[\nabla_{\nu},\nabla_{\rho}]\bigr]+
\bigl[\nabla_{\nu},[\nabla_{\rho},\nabla_{\mu}]\bigr]+
\bigl[\nabla_{\rho},[\nabla_{\mu},\nabla_{\nu}]\bigr]=0.
\end{equation}
Note that if we include torsion, we need to modify the Bianchi
identity to include some term proportional to the torsion.

\subsection{Related Tensors}
We\marginpar{Ricci Tensor $R_{\mu\nu}$} should note that the only nontrivial contraction for the
Riemann curvature tensor is
\begin{equation}
g^{\mu\nu}R_{\mu\alpha\nu\beta}=R_{\alpha\beta}.
\end{equation}
We call it the Ricci tensor. It follows from
\begin{equation}
R_{abcd}=R_{cdab}
\end{equation}
that the Ricci tensor is symmetric. We can\marginpar{Scalar Curvature $R$} contract
again to get
\begin{equation}
R=g^{\mu\nu}R_{\mu\nu}
\end{equation}
which is the \emph{Scalar Curvature} (sometimes called the Ricci scalar).
It turns out that Einstein tensor (i.e., the traceless Ricci
tensor)\marginpar{Einstein Tensor $G_{\mu\nu}$}
\begin{equation}
G_{\mu\nu}=R_{\mu\nu}-\frac{1}{2}g_{\mu\nu}R
\end{equation}
is interesting, since contracting the Bianchi identities gives
\begin{equation}
\nabla_{\nu}G^{\mu\nu}=0.
\end{equation}
This is called the \emph{contracted Bianchi identity}.

The Ricci tensor gives us the volume changing aspects of
curvature. More precisely, given a ``geodesic ball'' in a
manifold, the Ricci curvature tells us how it differs from a
``flat ball''. For General Relativity, the Ricci curvature
determines the degree to which matter will tend to converge or
diverge in time\footnote{This is precisely Raychaudhuri's
  equation, see Baez~\cite{Baez:2001qy}, Kar and SenGupta~\cite{Kar:2006ms}, Eric Poisson~\cite[\normalfont\S2]{poisson}, Hawking and Ellis~\cite[\normalfont\S4.1]{Hawking:1973uf}.}. 
The remainder of the Riemann tensor is known as the Weyl tensor,
which gives us the shear.

The last tensor worth mentioning is the Weyl
tensor\marginpar{Weyl Tensor $C_{\alpha\beta\mu\nu}$}. In
$n$-dimensions, we have
\begin{equation}
C_{\rho\sigma\mu\nu}=R_{\rho\sigma\mu\nu}-\frac{2}{n-2}(g_{\rho[\mu}R_{\nu]\sigma}-g_{\sigma[\mu}R_{\nu]\rho})+\frac{2}{(n-1)(n-2)}(g_{\rho[\mu}g_{\nu]\sigma}R).
\end{equation}
Note that the Weyl tensor vanishes in 3-dimensions\footnote{The Cotton tensor is used instead. See Garcia et al.~\cite{Garcia:2003bw}.}. One interesting property the
Weyl tensor possess is conformal invariance, i.e., under local
rescalings
\begin{equation}
g_{\mu\nu}\to e^{2f}g_{\mu\nu}
\end{equation}
where $f$ is an arbitrary smooth function, the Weyl tensor
remains the same.
