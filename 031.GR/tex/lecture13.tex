%%
%% lecture13.tex
%% 
%% Made by alex
%% Login   <alex@tomato>
%% 
%% Started on  Fri Mar  2 13:10:48 2012 alex
%% Last update Fri Mar  2 13:10:48 2012 alex
%%

\begin{wrapfigure}{r}{6pc}
  \vspace{-1pc}
  \includegraphics{img/lecture13.0}
  \vspace{-2pc}
\end{wrapfigure}
We will discuss a bit more about parallel transport, since it is
the main ingredient in defining curvature. 

Parallel transport a vector $v^{a}$ at $p$ along a curve through
$p$. We find the parallel transport by considering
\begin{equation}
\frac{\D x^{\mu}}{\D s}\nabla_{\mu}v^{a}=0,
\end{equation}
which is a first order differential equation.

Consider flat space in Cartesian coordinates, this equation
becomes
\begin{equation}
\begin{split}
\frac{\D x^{\mu}}{\D s}\frac{\partial v^{a}}{\partial x^{\mu}}
&=\frac{\D v^{a}}{\D s}\\
&=0
\end{split}
\end{equation}
which means we keep components constant. In flat space, using
arbitrary coordinates, this is \emph{necessarily} true too!

In curved space, using the Christoffel connection, a geodesic is
given by
\begin{equation}
\frac{\D x^{\mu}}{\D s}\nabla_{\mu}
\left(\frac{\D x^{\nu}}{\D s}
\right)=0.
\end{equation}
This demands that the tangent to the geodesic remains parallel to
itself. Further, if
\begin{equation}
\frac{\D x^{\mu}}{\D s}\nabla_{\mu}v^{a}=0
\end{equation}
this implies the length of $v$ is constant and the angle between
$v$ and the tangent is constant.

We can go backwards: first defining parallel transport, then
obtaining the covariant derivative. Lets define Cartesian
coordinates in flat space, which is trivial. We just parallel
transport the orthonormal basis to each point in space.

Lets start with a curved space and an orthonormal frame at a
point. We can then parallel transport it to every point. So
there's a tad of a paradox. The problem is there are many ways to
go from one point to another. It can yield two different bases,
given two different parallel transports from one point to
another. It can yield two different bases given two different
parallel transports from one point to another. We can measure
curvature by the difference. An equivalent procedure begins with
a point, make a loop, then transport the frame around the
loop. How it changes yields information about the
curvature. Recall the geodesic equation
\begin{equation}
\frac{\D v^{a}}{\D s}
+{{\Gamma_{\mu}}^{a}}_{b}\frac{\D x^{\mu}}{\D s}v^{b}=0
\end{equation}
although for simplicity we write
\begin{equation}
{{\Gamma_{\mu}}^{a}}_{b}\frac{\D x^{\mu}}{\D s}={A^{a}}_{b}.
\end{equation}
Thus the geodesic equation is the familiar
\begin{equation}
\frac{\D v^{a}}{\D s}+{A^{a}}_{b}v^{b}=0
\end{equation}
and $v^{a}(0)$ is given.

We can use standard methods for solving coupled differential
equations, or we can integrate
\begin{align}
v^{a}(s)
&= v^{a}(0)-\int^{s}_{0} {A^{a}}_{b}(s_{1})v^{b}(s_{1})\,\D s_{1}\\
&= v^{a}(0)-
\int^{s}_{0} {A^{a}}_{b}(s_{1})\left[v^{b}(0)-\int^{s_{1}}_{0}{A^{b}}_{c}(s_{2})v^{c}(s_{s})\,\D s_{2}\right]\,\D s_{1}
\nonumber\tag{\mbox{iterate}}\\
&= v^{a}(0)-
\int^{s}_{0}
    {A^{a}}_{b}(s_{1})\left[v^{b}(0)-\int^{s_{1}}_{0}{A^{b}}_{c}(s_{2})
\left(v^{c}(0)-\int^{s_{2}}_{0}{A^{c}}_{d}(s_{3})v^{d}(s_{3})\D s_{3}\right)
\D s_{2}\right]\D s_{1}
\nonumber\tag{\mbox{iterate again}}
\end{align}
This is the origin of holonomy\marginpar{Holonomy}. This
iterative process yields
\begin{equation}
v^{a}(s)=\sum_{n=0}^{\infty}\int^{s}_{0}\int^{s_{1}}_{0}\dots\int^{s_{n-1}}_{0}
{\bigl(A(s_{1})A(s_{2})\dots A(s_{n})\bigr)^{a}}_{b}v^{b}(0)
\,\D s_{n}\dots\D s_{1}.
\end{equation}
We define the path-ordering operator\marginpar{Path-Ordering Operator $\pathOrder$}
\begin{equation}
\pathOrder\bigl(A(s_{1})B(s_{2})\bigr)
=\begin{cases}A(s_{1})B(s_{2}) & \mbox{if }s_{1}>s_{2}\\
B(s_{2})A(s_{1}) &\mbox{if }s_{2}>s_{1}
\end{cases}
\end{equation}
The mnemonic is ``later is last'' (where last is the left-most).

So we have
\begin{equation}
\begin{split}
v^{a}(s)
&= \sum\frac{(-1)^{n}}{n!}\pathOrder\left(
\int^{s}_{0}\dots\int^{s}_{0}
{\bigl(A(s_{1})\dots A(s_{n})\bigr)^{a}}_{b}v^{b}(0)\,\D
s_{n}\dots\D s_{1}
\right)\\
&=\pathOrder\bigg(\underbracket[0.5pt]{\exp(-\int^{s}_{0}A\,\D
  s_{1}){{}^{a}}_{b}}_{\mathclap{\text{Parallel Transport Matrix}}}v^{b}(0)\bigg).
\end{split}
\end{equation}
For a closed loop, it is called the \define{Holonomy} of the
loop. We can write the holonomy as
\begin{equation}
H =
\pathOrder\bigg(\exp(-\oint\Gamma_{\mu}\,\D x^{\mu}){{}^{a}}_{b}\,v^{b}(0)\bigg).
\end{equation}
For an infinitesimal curve, we find
\begin{equation}
{H^{a}}_{b}={\delta^{a}}_{b}
+{{R_{\mu\nu}}^{a}}_{b}\,\D x^{\mu}\wedge\D x^{\nu}
+\begin{pmatrix}\mbox{Higher %}\\\mbox{
Order}\\
\mbox{Terms}
\end{pmatrix}
\end{equation}
where using the ordinary Stokes' theorem\footnote{Mathematicians
  call \emph{everything} ``Stokes' theorem'', so be forewarned
  gentle physicist!}
\begin{equation}
{{R_{\mu\nu}}^{a}}_{b}=\partial_{\mu}{{\Gamma_{\nu}}^{a}}_{b}
-\partial_{\nu}{{\Gamma_{\mu}}^{a}}_{b}
+{{\Gamma_{\mu}}^{a}}_{c}{{\Gamma_{\nu}}^{c}}_{b}
-{{\Gamma_{\nu}}^{a}}_{c}{{\Gamma_{\mu}}^{c}}_{b}
\end{equation}
which is the curvature tensor. It's antisymmetric in the first
two indices, and if we lower the $a$ index the last two indices
are antisymmetric in an orthonormal frame.

\begin{rmk}
We have the Holonomy group $\GL{4,\RR}$ in general, but we can
impose symmetry. The $A$ is a Lie-algebra valued one-form.
\end{rmk}

The curvature form\marginpar{Curvature Form} is thus
\begin{equation}\label{eq:curvatureTwoFormDefn}
{\mathcal{R}^{a}}_{b}=\frac{1}{2}{{R_{\mu\nu}}^{a}}_{b}\,\D x^{\mu}\wedge\D x^{\nu}
\end{equation}
in that case. In an orthonormal basis, the equation for curvature becomes:
\begin{equation}\label{eq:cartanSecondStructEqn}
{\mathcal{R}^{a}}_{b}=\D{\omega^{a}}_{b}+{\omega^{a}}_{c}\wedge{\omega^{c}}_{b}.
\end{equation}
This is the second Cartan structure equation.

\begin{danger}
Note that Equation \eqref{eq:cartanSecondStructEqn} holds only
when we work with an orthonormal frame!
\end{danger}

\medskip
\noindent\ignorespaces Once we have the curvature two-form (i.e., we know what
${\mathcal{R}^{a}}_{b}$ is) we can go back to Equation
\eqref{eq:curvatureTwoFormDefn} to obtain
${{R_{\mu\nu}}^{a}}_{b}$ in some basis. 


\begin{danger}
Using symmetries of the curvature tensor, some authors (e.g.,
Carroll~\cite{Carroll:2004st}) write
${\mathcal{R}^{a}}_{b}=\frac{1}{2}{R^{a}}_{b\mu\nu}\,\D x^{\mu}\wedge\D x^{\nu}$.
Just be forewarned on the different equivalent ways of defining
the curvature two-form!
\end{danger}
