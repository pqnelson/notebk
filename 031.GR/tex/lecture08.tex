%%
%% lecture08.tex
%% 
%% Made by alex
%% Login   <alex@tomato>
%% 
%% Started on  Thu Feb 23 13:33:17 2012 alex
%% Last update Thu Feb 23 13:33:17 2012 alex
%%
A type $(k,l)$-tensor $T$ is a multilinear map from $k$ dual
vectors and $l$ vectors to $\RR$:
\begin{equation}
T\colon\underbrace{\mathrm{T}^{*}\mathcal{M}
\times\dots\times\mathrm{T}^{*}\mathcal{M}}_{\text{$k$ times}}\times
\underbrace{\mathrm{T}\mathcal{M}
\times\dots\times\mathrm{T}\mathcal{M}}_{\text{$l$ times}}
\to\RR.
\end{equation}
This is a linear map, so if we know what it does on the basis
vectors (and covectors), we know everything. We have
\begin{equation}
T(\D x^{\mu_{1}},\dots,\D x^{\mu_{k}},\partial_{\nu_{1}},\dots,,\partial_{\nu_{\ell}})={T^{\mu_{1}\dots\mu_{k}}}_{\nu_{1}\dots\nu_{\ell}}
\end{equation}
are the components of $T$ in a coordinate basis.

We have this method of constructing new tensors out of old ones:
the tensor product. The idea is simple, basically multiply the
components together. More formally, if we take a $(k,\ell)$
tensor and a $(m,n)$ tensor, their tensor product gives us 
a $(k+m,\ell+n)$ tensor denoted
\begin{equation}
\mathop{S}\limits^{(k,\ell)}\otimes
\mathop{T}\limits^{(m,n)}=\mathop{U}\limits^{(k+m,\ell+n)}
\end{equation}
and it has components given by
\begin{equation}
\begin{split}
(S\otimes T)&(\omega_{1},\dots,\omega_{k+m},v^{1},\dots,v^{\ell+n})=\\
&\quad S(\omega_{1},\dots,\omega_{k},v^{1},\dots,v^{\ell})T(\omega_{k+1},\dots,\omega_{k+m},v^{\ell+1},\dots,v^{\ell+n}).
\end{split}
\end{equation}
So what happens in practice? Well, consider a $(1,1)$ tensor 
\begin{equation}
T={T^{\mu}}_{\nu}\partial_{\mu}\otimes\D x^{\nu}
\end{equation}
we say
\begin{subequations}
\begin{align}
T(\omega,v)
&= T(\omega_{\rho}\D x^{\rho},v^{\sigma}\partial_{\sigma})\\
&={T^{\mu}}_{\nu}\partial_{\mu}\otimes\D x^{\nu}(\omega_{\rho}\D x^{\rho},v^{\sigma}\partial_{\sigma})\\
&={T^{\mu}}_{\nu}(\omega_{\rho}\partial_{\mu}\D x^{\rho})\otimes(v^{\sigma}\D x^{\nu}\partial_{\sigma})\\
&={T^{\mu}}_{\nu}(\omega_{\rho}{\delta_{\mu}}^{\rho})(v^{\sigma}{\delta^{\nu}}_{\sigma})\\
&=T^{\mu}_{\nu}\omega_{\mu}v^{\nu}.
\end{align}
\end{subequations}
Again, this is what physicists say. Mathematicians would be a
little more cautious, but get the same result.

\medbreak\noindent\textbf{Warning:\quad}\ignorespaces %
It looks like anything with an index is a tensor, in some sense
this is true but a tensor is independent of what basis you're using.
So lets consider a misleading non-example: a type $(2,0)$ tensor
$(\partial\omega)$ which satisfies
\begin{equation}
(\partial\omega)(\partial_{\mu},\partial_{\nu})=\partial_{\mu}\omega_{\nu}.
\end{equation}
Suppose we choose a different coordinate system, then
\begin{equation}
(\partial\omega)(\partial_{\mu},\partial_{\nu})\not=
(\partial\omega)(\partial_{\mu'},\partial_{\nu'})
\end{equation}
In other words: it is not even linear!

\noindent\textbf{Moral:\quad}\ignorespaces %
Indices don't make something an tensor!

\medbreak
\begin{ex}[Kronecker Delta]
A type $(1,1)$ tensor is the Kronecker delta
\begin{subequations}
\begin{align}
\delta
&={\delta^{\mu}}_{\nu}\partial_{\mu}\otimes\D x^{\nu}\\
&=\partial_{\mu}\otimes\D x^{\mu}.
\end{align}
\end{subequations}
So 
\begin{equation}
\delta(\omega,v)=\omega_{\mu}v^{\mu}.
\end{equation}
We can show this by showing $\delta$ is linear, or we can show
that it's independent of coordinates.
\end{ex}
\begin{ex}[Field Strength Tensor]
The field strength tensor is a $(2,0)$ tensor 
\begin{equation}
F=F_{\mu\nu}\D x^{\mu}\otimes\D x^{\nu}
\end{equation}
with components
\begin{equation}
F_{0i}\sim E_{i},\quad\mbox{and}\quad F_{ij}\sim B_{k}
\end{equation}
Usually it is written $F=\D A+A\wedge A$.
\end{ex}
\begin{ex}[Metric Tensor]
A $(2,0)$ tensor we have seen before is the metric tensor
\begin{equation}
g=g_{\mu\nu}\D x^{\mu}\otimes\D x^{\nu}
\end{equation}
which behaves on vectors as
\begin{equation}
g(v,w)=g_{\mu\nu}v^{\mu}w^{\nu}.
\end{equation}
This is a generalization of the inner product.

The metric lets us change a vector to a dual vector. Consider (in
some basis) a vector $v=v^{\mu}\partial_{\mu}$, then $g(v,-)$ is
an object taking a vector to a real number:
\begin{equation}
\begin{split}
g(v,-)\colon &\mathrm{T}\mathcal{M}\to\RR,\\
&\omega\mapsto g(v,\omega).
\end{split}
\end{equation}
Component-wise this looks like
\begin{subequations}
\begin{align}
g(v,-)
&=(g_{\mu\nu}\D x^{\mu}\otimes\D
x^{\nu})(v^{\rho}\partial_{\rho},-)\\
&=g_{\mu\nu}\<\D x^{\mu}|v^{\rho}\partial_{\rho}\>\D x^{\nu}\\
&=g_{\mu\nu}(v^{\rho}{\delta_{\rho}}^{\mu})\D x^{\nu}\\
&=(g_{\mu\nu}v^{\mu})\D x^{\nu}.
\end{align}
\end{subequations}
\textbf{NOTATION:\quad}\ignorespaces $g_{\mu\nu}v^{\mu}=v_{\nu}$.

Note that we need one more condition for $g$ to be a metric: it
must be nondegenerate. So
\begin{equation}
g(u,v)=0\quad\mbox{for all }v
\end{equation}
only when $u=0$.
Equivalently, $g_{\mu\nu}$ must be invertible. Its inverse is
denoted $g^{\mu\nu}$ so 
\begin{equation}
g_{\alpha\mu}g^{\mu\beta}={\delta_{\alpha}}^{\beta}.
\end{equation}
The metric has to be symmetric. The inverse metric tensor
\begin{equation}
g=g^{\mu\nu}\partial_{\mu}\otimes\partial_{\nu}
\end{equation}
is also an honest tensor.
\end{ex}
