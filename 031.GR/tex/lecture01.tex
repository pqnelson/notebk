%%
%% lecture01.tex
%% 
%% Made by alex
%% Login   <alex@tomato>
%% 
%% Started on  Tue Feb 14 08:29:36 2012 alex
%% Last update Tue Feb 14 08:29:36 2012 alex
%%
\subsection{General Relativity's Importance in Physics}
The first 50 years after Einstein published his field equations,
physicists held one of two opinions:

1) It was a beautiful model for how physics ought to be.

2) It was largely irrelevant unless you specialize in it.

\noindent\ignorespaces%
Most people imagine it's a model for how physics ought to be,
unless gravity's emergent. The second view is more or less
disregarded. In high energy physics, the coupling constants
converge to the same value at high enough energies where gravity
is significant (perhaps it unifies with the other forces, and
perhaps that's why it is significant). Trivially, General
Relativity is useful in cosmology.

There exists a sizeable group of people in condensed matter physics
where analog models\footnote{For a review, see Barcelo \emph{et al.}~\cite{Barcelo:2005fc}.} are used; e.g., an event horizon for sound
as an analog to Black Holes. There are attempts to make
predictions for, e.g., quantum gravity (using analogs of Hawking
radiation, etc.).

In the next 5 years, there should be experimental evidence for
gravitational radiation. In 15 years there will be more sensitive
tests available. We can ask questions like ``Does $E/c^{2}$
contribute to mass?'' There are interesting anomalies, e.g.,
measurements\footnote{See, e.g., Gundlach's measurements~\cite{gundlach},
the CODATA 2002 recommended values~\cite{CODATA2002}}
 of Newton's constant $G$ differ from each other by
$10\sigma$ to $15\sigma$, the Pioneer satellite feels
accelerations that's still not accounted for~\cite{Nieto:2007ng}, the predicted
energy level for Dark Energy is off by 120 orders of magnitude~\cite{Martin:2012bt}.

\subsection{Geometry and Physics}
Lets recall Newton's second law 
\begin{equation}\label{eq:lec01:newtonSecondLaw}
F=ma.
\end{equation}
Initially there was some contraversy whether it's a ``true
natural law'' or just a definition of force 
(see Spivak~\cite{spivak:2010} for details). We understand the
mass on the right hand side of Newton's second law describes
\emph{inertial mass}, the body's resistance to acting forces. But
we may say a couple other things.

First, since Newton's second law involves only acceleration, we
work with the second time derivative of position. Higher order time
derivative models are unstable since they have energy unbounded
from below, as Ostrogradski\index{Ostrogradski Theorem} proved\footnote{See
  Woodard~\cite[\normalfont\S2]{Woodard:2006nt} for a review of
  Ostrogradski's theorem for classical mechanics.}. 

The second thing to say is that Newton divided the world in two:
the object we are examining, and the rest of the world affecting
it. But gravity is now an exception. We consider gravitational
force of a body with mass $M$ acting on another body (with mass
$m$) in Newton's second law \eqref{eq:lec01:newtonSecondLaw},
writing
\begin{equation}
F\eqdef\frac{GmM}{r^{2}}
\end{equation}
for the gravitational force, and we invoke the second law writing
\begin{equation}\label{eq:lec01:gravity:newtonSecondLaw}
\frac{GmM}{r^{2}}=ma.
\end{equation}
We observe \emph{mathematically} the masses $m$ cancels out on
both sides. We thus obtain
\begin{equation}
\frac{GM}{r^{2}}=a.
\end{equation}
That makes gravity different from everything else, in that it
makes gravity a \emph{theory of paths}.

Almost\marginpar{Equivalence Principle gives us geometry} automatically, this makes gravity a theory of
geometry. But first note that really we have the right hand side
of \eqref{eq:lec01:gravity:newtonSecondLaw} be
\begin{equation}
ma=m_{i}a
\end{equation}
where $m_{i}$ is the \emph{inertial mass}, whereas the
gravitational force the body with mass $m$ feels is
\begin{equation}
\vec{F}_{12}=\frac{GMm_{g}}{r^{2}}\widehat{e}_{r}
\end{equation}
where $\widehat{e}_{r}$ is the unit vector from the body $M$ to
the body $m$, $m_{g}$ is the \emph{(passive) gravitational
  mass}\footnote{Note that the way to think of ``gravitational
  mass'' is that it is the \emph{``charge'' gravity feels.}}
which experiences the gravitational force, and $M$
is the \emph{(active) gravitational mass} exerting the
gravitational force. We have two conceptually different masses:
the inertial mass $m_{i}$, and the gravitational mass
$m_{g}$. The basic ingredient for gravity is the idea 
\begin{equation}\label{eq:lec01:weakEquiv}
m_{i}=m_{g}
\end{equation}
called the \define{Principle of (Weak) Equivalence}\index{Weak Equivalence Principle}\index{Equivalence Principle!Weak}. 

There is a \emph{Strong Equivalence Principle}\index{Strong Equivalence Principle}\index{Equivalence Principle!Strong}. Using
Newton's third Law, we find
\begin{equation}
\vec{F}_{21}=-\vec{F}_{12}.
\end{equation}
The role of ``active'' and ``passive'' gravitational masses swap.
%If active and passive gravitational masses were not
Active and passive gravitational masses are ``equivalent'' in the sense
\begin{equation}
\frac{m^{(a)}}{m^{(p)}}=\frac{M^{(a)}}{M^{(p)}}
\end{equation}
where $m$, $M$ are gravitational masses and the superscript
indicates whether they are active or passive gravitational
masses. This can be checked by the Earth-Moon\index{Lunar Laser Ranging} system.
Since 1968, when NASA attached lasers and reflectors (i.e., three plane mirrors
meeting mutually at right angles) to the moon, we have timed the
delay of a laser pulse sent to the Moon. If the strong
equivalence principle didn't hold, we'd expect the
Earth--Moon system's center of mass would oscillate with the
Lunar period. But we have not observed this\footnote{See Williams, et al.,~\cite{Williams:2005rv} for more data on this. Will~\cite{Will:1998dx} has a more broad discussion of experimental foundations underlying the equivalence principle in its various forms.}.

Reiterating the main point: the equality
\begin{equation*}\tag{\ref*{eq:lec01:weakEquiv}}
m_{i}=m_{g}
\end{equation*}
is what makes the geometric picture possible. The present tests
(as of 2010) suggest they're equal to parts in $10^{12}$ or $10^{13}$.

But Galileo knew the equivalence principle, did he suspect the
geometrical aspects? No, because gravity determines acceleration,
and paths depend on initial velocity too. Gravity doesn't
determine paths in space, instead it determines \emph{paths in spacetime}.
We need to articulate our vocabulary regarding paths before we can
continue discussing gravity.

First a \define{Extremal Path} between two points is referred to
as a \define{Geodesic}. We will set up the framework to discuss
geodesics, then proceed to consider calculations.

\marginpar{Flat, Intrinsic, Extrinsic Geometries}We need to know a little about what it means for a space to be
``curved'', so we will first consider what it means for space to
be ``flat''. We consider it to be the usual Euclidean
geometry. There is an important distinction between ``intrinsic geometry''\index{Geometry!Intrinsic}\index{Intrinsic Geometry}
(the curvature of space in itself without reference to higher
dimensions, e.g., the sum of angles of a triangle on Earth
doesn't add up to $\pi$, but without reference to 3 dimensions)
and ``extrinsic geometry''\index{Geometry!Extrinsic} \index{Extrinsic Geometry}
(curvature as seen in higher dimensions). We mostly care about
intrinsic curvature with General Relativity. There are times
(e.g., in the ADM formalism) when extrinsic curvature is
important. \TODO{figure out some transition motivating arclength}

With paths, we really need an idea of distance. Recall for flat
space, the Pythogoras' theorem gives us a path's length (more
or less) as
\begin{equation}
s^{2}=x^{2}+y^{2}.
\end{equation}
If we knew infinitesimal distances, that's enough: we can
integrate to get the distance
\begin{equation}
s=\int\D s,
\end{equation}
where
\begin{equation}
(\D s)^{2}=(\D x)^{2}+(\D y)^{2}.
\end{equation}
\textbf{WARNING:} the notation used is $\D s^2=(\D s)^2$, which
may confuse neophytes. The distance between $(x_0,y_0)$ and
$(x_1,y_1)$ is determined by the variation
\begin{equation}
\delta\int^{(x_1,y_1)}_{(x_0,y_0)}\D s=0.
\end{equation}
Think of it like the Euler-Lagrange equation. We will now
consider some special cases.\marginpar{Geodesic Equation: Examples}

\begin{ex}[Sphere]
Recall a sphere is described by
\begin{equation}
x^2+y^2+z^2=R^2.
\end{equation}
The distance is
\begin{equation}
\D s^2=\D x^2 + \D y^2+\D z^2
\end{equation}
with a constraint. In spherical coordinates we have
\begin{equation}
\begin{split}
r^{2}&=x^2+y^2+z^2\\
&=R^2
\end{split}
\end{equation}
Thus $r$ is constant. By substitution, we find
\begin{equation}
\D s^2=\D r^2+r^2(\D\theta^2+\sin^{2}(\theta)\,\D\varphi^2)
\end{equation}
This is in a flat 3-dimensional space. But if we set $r=R$ and
thus $\D r=0$, we obtain
\begin{equation}
\D s^2=R^2(\D\theta^2+\sin^{2}(\theta)\,\D\varphi^2)
\end{equation}
So we plug this into the variation
\begin{equation}
\delta\int\D s=0
\end{equation}
to get the geodesic equation.
\end{ex}
\begin{ex}
Consider a surface in $\RR^3$ defined by
\begin{equation}
z=f(x,y).
\end{equation}
So now we have
\begin{subequations}
\begin{align}
\D s^2 
&=\D x^2+\D y^2+\D z^2\\
&=\D x^2+\D y^2+\left(\frac{\partial f}{\partial x}\,\D x+\frac{\partial f}{\partial y}\,\D y\right)^{2}\\
&=g_{xx}\,\D x^{2}+2g_{xy}\,\D x\,\D y+g_{yy}\,\D y^{2}
\end{align}
\end{subequations}
where we have
\begin{equation}
g_{xx}=1+\left(\frac{\partial f}{\partial x}\right)^{2},\quad
g_{xy}=\frac{\partial f}{\partial x}\frac{\partial f}{\partial y},\quad
g_{yy}=1+\left(\frac{\partial f}{\partial y}\right)^{2}
\end{equation}
are the coefficients. These coefficients $g_{ab}$ are called
\define{Components of the Metric Tensor}\index{Metric Tensor!Components of ---}
These are the basic physical variables. There is one
subtlety---we can have the same geometry described by
\emph{different coordinates!} For example, in Cartesian
coordinates the plane $\RR^2$ is described by
\begin{subequations}
\begin{equation}
\D s^2=\D x^2+\D y^2,
\end{equation}
whereas in Polar coordinates it is
\begin{equation}
\D s^2=\D r^2+r^2\,\D\theta^2,
\end{equation}
\end{subequations}
and although they describe the same geometry (a flat plane), the
metric tensor is different.
\end{ex}
\begin{rmk}\index{Metric Tensor!and Rotation}\index{Rotation!and Metric Tensor}
Note that the way to tell there is an object experience rotation
in spacetime is when the metric has a nonzero
$g_{t\varphi}\not=0$ term.
\end{rmk}

\begin{ex}[Flat $\RR^2$]
Consider flat 2-dimensional space. We have
\begin{equation}
\D s^2=\D x^2+\D y^2
\end{equation}
We want to describe a path, so we parametrize it:
\begin{equation}
(x,y)=(x(u),y(u)).
\end{equation}
We want to extremize
\begin{equation}
\D s=\left[\left(\frac{\D x}{\D u}\right)^{2}+
\left(\frac{\D y}{\D u}\right)^{2}\right]^{1/2}\D u
\end{equation}
Lets call the bracketed term, say,
\begin{equation}
E\eqdef\left(\frac{\D x}{\D u}\right)^{2}+
\left(\frac{\D y}{\D u}\right)^{2}.
\end{equation}
This is just an assignment of variables. Intuitively, it plays
the role of ``kinetic energy''. We want to extremize
\begin{equation}
s=\int E^{1/2}\,\D u
\end{equation}
What to do? Well,
\begin{subequations}
\begin{align}
\delta\int E^{1/2}\D u
&=\int\frac{1}{2}E^{-1/2}\delta E\,\D u\\
&=\int\frac{1}{2}E^{-1/2}\left[2\frac{\D x}{\D u}\delta\frac{\D x}{\D u}+
\frac{\D y}{\D u}\delta\frac{\D y}{\D u}\right]^{1/2}\D u\\
&=\int\frac{1}{2}E^{-1/2}\left[2\frac{\D x}{\D u}\frac{\D }{\D u}\delta x+
\frac{\D y}{\D u}\frac{\D}{\D u}\delta y\right]^{1/2}\D u
\end{align}
\end{subequations}
We integrate by parts, and demand the variation vanishes at its
endpoints, thus
\begin{equation}
\delta\int\D s=-\int\left[
\frac{\D}{\D u}\left(E^{-1/2}\frac{\D x}{\D u}\right)\delta x
+
\frac{\D}{\D u}\left(E^{-1/2}\frac{\D y}{\D u}\right)\delta y
\right]\D u.
\end{equation}
So this is supposed to vanish, which implies for the coefficients
\begin{equation}
\frac{\D}{\D u}\left(E^{-1/2}\frac{\D x}{\D u}\right)=0\quad
\mbox{and}\quad
\frac{\D}{\D u}\left(E^{-1/2}\frac{\D y}{\D u}\right)=0
\end{equation}
We can integrate directly to find $E=$constant. There is a trick
we never specified anything about $u$. So let us choose $u=s$,
it's a perfectly kosher choice. Then
\begin{equation}
E=1
\end{equation}
which makes the equations of motion
\begin{equation}
\frac{\D^2x}{\D u^2}=0,\quad\mbox{and}\quad
\frac{\D^2y}{\D u^2}=0.
\end{equation}
This has its solution be
\begin{equation}
x(s)=as+b,\quad y(s)=\alpha s+\beta.
\end{equation}
That's the geodesic for flat space in Cartesian coordinates.
\end{ex}
