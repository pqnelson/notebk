%%
%% lecture18.tex
%% 
%% Made by alex
%% Login   <alex@tomato>
%% 
%% Started on  Sat Mar 10 15:25:11 2012 alex
%% Last update Sat Mar 10 15:25:11 2012 alex
%%

The weak field equations. The most important test of general
relativity is that it gives us back Newtonian gravity. Lets
consider a weak field, which can be thought of as a perturbation
of a background $\eta_{\mu\nu}$. So
\begin{equation}
g_{\mu\nu}=\eta_{\mu\nu}+h_{\mu\nu}
\end{equation}
and the inverse metric is given\footnote{We should recall that
  the Neumann series gives us
  $(I+X)^{-1}=I+X+X^{2}+\dots+X^{n}+\dots$, which is employed here.} by
\begin{equation}
g^{\mu\nu}=\eta^{\mu\nu}-h^{\mu\nu}+\bigO(h^{2}).
\end{equation}
We raise and lower indices with $\eta$ in this approximation. The
Christoffel connection
\begin{equation}
\begin{split}
\Gamma^{\rho}_{\mu\nu}
&=\frac{1}{2}g^{\rho\sigma}(\partial_{\mu}g_{\sigma\nu}
+\partial_{\nu}g_{\mu\sigma}-\partial_{\sigma}g_{\mu\nu})\\
&=\frac{1}{2}\eta^{\rho\sigma}(\partial_{\mu}h_{\sigma\nu}
+\partial_{\nu}h_{\mu\sigma}-\partial_{\sigma}h_{\mu\nu})
+\bigO(h^{2}).
\end{split}
\end{equation}
For the Ricci tensor, we only have the $\partial\Gamma$ terms to
worry about, since $\Gamma\Gamma\sim\bigO(h^{2})$. The components
of the Ricci tensor are
\begin{equation}
R_{\mu\nu}=\frac{1}{2}(\partial^{\sigma}\partial_{\mu}h_{\sigma\nu}
+\partial^{\sigma}\partial_{\nu}h_{\sigma\mu}
-\partial^{\sigma}\partial_{\sigma}h_{\mu\nu}
-\partial_{\mu}\partial_{\nu}{h^{\sigma}}_{\sigma}).
\end{equation}
If we write
\begin{equation}
h={h^{\sigma}}_{\sigma}
\end{equation}
for the trace, then the ``trace-reversed $h$'' is
\begin{equation}
\bar{h}_{\mu\nu}=h_{\mu\nu}-\frac{1}{2}\eta_{\mu\nu}h
\end{equation}
Observe
\begin{equation}
\begin{split}
\bar{h}&=\eta^{\mu\nu}\bar{h}_{\mu\nu}\\
&=h-2h=-h.
\end{split}
\end{equation}
The Einstein tensor becomes
\begin{equation}
G_{\mu\nu}=\frac{1}{2}(-\Box\bar{h}_{\mu\nu}+
\partial_{\mu}\partial^{\sigma}\bar{h}_{\sigma\nu}+
\partial_{\nu}\partial^{\sigma}\bar{h}_{\sigma\mu}-
\eta_{\mu\nu}\partial^{\sigma}\partial^{\tau}\bar{h}_{\sigma\tau}).
\end{equation}
We can choose coordinates such that
\begin{equation}
G_{\mu\nu}=\frac{-1}{2}\Box\bar{h}_{\mu\nu}.
\end{equation}
%% The basic trick is that any metric at a given point has an
%% expansion
%% \begin{equation}
%% g_{\mu\nu}=\eta_{\mu\nu}+0+(\partial_{\mu}g_{\nu\alpha}\partial_{\nu}g_{\mu\beta})\frac{g^{\alpha\beta}}{2!}+\dots
%% \end{equation}
When
\begin{equation}
x^{\mu}\to x^{\mu}+\zeta^{\mu}
\end{equation}
for infinitesimal $\zeta$, then
\begin{equation}
g_{\mu\nu}\to
g_{\mu\nu}+\nabla_{\mu}\zeta_{\nu}+\nabla_{\nu}\zeta_{\mu}
\end{equation}
and the perturbation transforms as
\begin{equation}
h_{\mu\nu}\to h_{\mu\nu}+\partial_{\mu}\zeta_{\nu}+\partial_{\nu}\zeta_{\mu}
+\bigO(h^{2}).
\end{equation}
Observe this implies
\begin{equation}
\partial^{\sigma}\bar{h}_{\sigma\nu}\to\partial^{\sigma}\bar{h}_{\sigma\nu}+\Box\zeta_{\nu}.
\end{equation}
We may choose $\zeta$ so that
\begin{equation}
\partial^{\sigma}\bar{h}_{\sigma\nu}\to0.
\end{equation}
We just have to solve
\begin{equation}
\Box\zeta_{\nu}=-\partial^{\sigma}\bar{h}^{\text{(old)}}_{\sigma\nu}
\end{equation}
which is the wave equation with a source. We know how to solve
that! See, e.g., Jackson's electrodynamics text. We can choose
coordinates such that
\begin{equation}
\partial^{\sigma}\bar{h}_{\sigma\nu}=0
\end{equation}
which we call the harmonic gauge, Fock gauge, Lorenz gauge, de
Donder gauge, etc.

\begin{rmk}[Physical Ramifications of Choice of Coordinates]
There is no physical meaning for this choice of gauge (i.e., this
particular choice of coordinates), nor does any other choice have
physical meaning unless there exists strong symmetries which
enable a canonical choice.
\end{rmk}

The harmonic gauge has Einstein's field equations read
\begin{equation}
\begin{split}
\frac{-1}{2}\Box\bar{h}_{\mu\nu}=\frac{\kappa^{2}}{2}T_{\mu\nu}.
\end{split}
\end{equation}
Among other things in life, this tells us (1) there exists
gravity waves, (2) they travel at the speed of light because of
the D'Alembertian.

\subsection{Newtonian Limit}
Lets consider the Newtonian limit, when
\begin{equation}
v/c\lll1
\end{equation}
where $v$ is the velocity of gravitating bodies.
For non-interacting matter (``dust'') we had the stress energy
tensor
\begin{equation}
T^{\mu\nu}=\rho u^{\mu}u^{\nu}
\end{equation}
which has components
\begin{equation}
T^{00}\approx\rho
\end{equation}
and
\begin{equation}
T^{ij}\approx T^{i0}\approx T^{0j}\approx 0.
\end{equation}
So we have
\begin{equation}
\Box\bar{h}^{00}=-\kappa^{2}\rho.
\end{equation}
Observe the D'Alembertian is written
\begin{equation}
\Box=\frac{1}{c^{2}}\frac{\partial^{2}}{\partial t^{2}}
-\nabla^{2}
\end{equation}
but in the Newtonian approximation
\begin{equation}
\frac{1}{c^{2}}\frac{\partial^{2}}{\partial t^{2}}\approx0.
\end{equation}
Thus
\begin{equation}
\Box\approx-\nabla^{2}.
\end{equation}
Our field equation becomes
\begin{equation}
-\nabla^{2}\bar{h}_{00}=-\kappa^{2}\rho.
\end{equation}
This is precisely Poisson's equation for Newtonian gravity! That
is
\begin{equation}
\nabla^{2}\Phi=4\pi G\rho
\end{equation}
thus by inspection
\begin{equation}
\bar{h}_{00}=\frac{\kappa^{2}}{4\pi G}\Phi
\end{equation}
We see that
\begin{equation}
h_{\mu\nu}=\bar{h}_{\mu\nu}-\frac{1}{2}\eta_{\mu\nu}\bar{h}
\end{equation}
Hence
\begin{equation}
h_{\mu\nu}\frac{\kappa^{2}}{8\pi G}\Phi\eta_{\mu\nu}.
\end{equation}
The line element in the Newtonian approximation is
\begin{equation}
\D s^{2}=\left(1+\frac{\kappa^{2}}{8\pi G}\Phi\right)\D t^{2}
-\left(1-\frac{\kappa^{2}}{8\pi G}\Phi\right)\D\mathbf{x}\cdot\D\mathbf{x}.
\end{equation}
This is good: General Relativity contains Newtonian gravity at
appropriate limits!

\begin{rmk}
When we have very light masses moving close to the speed of
light, we need to include other components of $h$; but we can
still use the weak field approximation!
\end{rmk}

We now know that the field equations are
\begin{equation}
\Box\bar{h}_{\mu\nu}=-16\pi GT_{\mu\nu}
\end{equation}
we pull out our copy of Jackson, or Afkren (or whatever), and use
the Green's function for the D'Alembertian
\begin{equation}
\begin{split}
\bar{h}_{\mu\nu}(\mathbf{x},t) &= 
4
G\int \frac{T_{\mu\nu}(\vec{y},t-|\vec{x}-\vec{y}|)}{|\vec{x}-\vec{y}|}\D^{3}y\\
&=4
G\int \frac{T_{\mu\nu}^{\text{(ret)}}}{|\vec{x}-\vec{y}|}\D^{3}y
\end{split}
\end{equation}
We can interpret the next order corrections as gravity's coupling
to the stress-energy tensor. To conclude our discussion, we will
write a table comparing the multipole expansion in
electromagnetism\footnote{C.f., Jackson's \emph{Classical
Electrodynamics}~\cite[{\normalfont p.145 \emph{et seq.}}]{jackson:1999}.} and in gravity:

\medbreak
\noindent\begin{tabular}{p{7pc}|p{11pc}p{11pc}}%{p{7pc}|p{11pc}|p{11pc}}
\toprule
Multipole Term&Electromagnetism & Gravity\\\midrule
Monopole Moment 
  & The total charge $q$; charge conserved, monopole moment is constant 
  & The total mass $m$; Newtonian limit has mass conserved, fixed
  field unchanging in time. \\    
Dipole Moment 
  & $\sum q_{i}r_{i}$, $\dot{D}=\sum q_{i}\dot{r}_{i}$; Fix two
  charges to the ends of a spring and oscillate.
  & $\sum m_{i}r_{i}$, $\sum m_{i}\dot{r}_{i} = \sum p_{i}=0$ in
  the center of mass frame. Try to oscillate total momentum, but
  this is fixed!\\
Magnetic Dipole
  & $\sum q_{i}\vec{v}_{i}\times\vec{r}_{i}$ 
  & $\sum m_{i}\vec{v}_{i}\times\vec{r}_{i}=\vec{L}=\mbox{constant}$
    Gravity has no mass dipole or magnetic dipole by conservation
    laws.\\
Quadrapole 
  & (None)
  & $\sum m_{i}{r_{i}}^{\mu}{r_{i}}^{\nu}$ This is the lowest
  order radiation for gravity, but we have to take the
  appropriate number of derivatives. The power is $\sim v/c^{8}$
  (This is a strong restriction on corrections to gravity!)\\
\bottomrule
\end{tabular}
