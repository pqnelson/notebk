%%
%% lecture04.tex
%% 
%% Made by alex
%% Login   <alex@tomato>
%% 
%% Started on  Sun Feb 19 11:51:19 2012 alex
%% Last update Sun Feb 19 11:51:19 2012 alex
%%
There are a few differences with lightlike geodesics and timelike
geodesics. First we use an affine parameter $\lambda$ instead of
$s$. Second $\D s^{2}=0$ between events. So we have
\begin{align*}
g_{ab}\frac{\D x^{a}}{\D s}\frac{\D x^{b}}{\D s}=1
\tag{\text{for a planet}}\\
g_{ab}\frac{\D x^{a}}{\D\lambda}\frac{\D x^{b}}{\D\lambda}=0.
\tag{\text{for light}}
\end{align*}
The only thing that changes is
\begin{equation}
\left(\frac{\D r}{\D\lambda}\right)^{2}=
E^{2}-\left(1-\frac{2m}{r}\right)\frac{L^{2}}{r^{2}}.
\end{equation}
The convention for light is to \emph{not} use tildes on $L$ and
$E$. We also have
\begin{equation}
\left(\frac{\D r}{\D\varphi}\right)^{2}=\frac{L^{2}}{r^{4}}.
\end{equation}
We can see the angle as a function of distance, instead of the
other way around:
\begin{equation}
\left(\frac{\D\varphi}{\D u}\right)^{2}=
\frac{L^{2}}{E^{2}-L^{2}u^{2}(1-2mu)}
\end{equation}
which is the same sort of problem we've seen before. In
particular, the ``Newtonian Approximation''\marginpar{Newtonian Approximation}
is
\begin{equation}
\begin{split}
\frac{\D\varphi}{\D u} 
&=\frac{1}{\sqrt{(E/L)^{2}-u^{2}}}\\
&=\frac{1}{\sqrt{b^{-2}-u^{2}}}.
\end{split}
\end{equation}
The solution for our differential equation is
\begin{equation}
\varphi-\varphi_{0}=\arcsin(bu)
\end{equation}
and thus
\begin{equation}
r\sin(\varphi-\varphi_{0})=b.
\end{equation}
This is a straight line! In this approximation, light moves in a
straight line.

\subsection{First Approximation}
It is useful to use the approximation
\begin{equation}
u^{2}-2mu^{3}\approx u^{2}(1-2mu)^{2}-m^{2}u^{4}.
\end{equation}
Let us define
\begin{equation}
y=u(1-mu),
\end{equation}
we can ignore $m^{2}y^{4}$ relative to $y^{2}$. So
\begin{equation}
\D y=\D u (1-2mu),
\end{equation}
thus
\begin{equation}
\begin{split}
\D u 
&= (1-2mu)^{-1}\D y\\
&\approx (1+2my)\D y.
\end{split}
\end{equation}
Then
\begin{equation}
\begin{split}
\D\varphi
&=\frac{\pm\D u}{\sqrt{(E/L)^{2}-u^{2}(1-2mu)}}\\
&\approx\pm\frac{(1+2my)}{\sqrt{b^{-2}-y^{2}}}\D y.
\end{split}
\end{equation}
So
\begin{equation}
\begin{split}
\frac{1}{2}\Delta\varphi
&=\int^{b^{-1}}_{0}\left(\frac{1+2my}{\sqrt{b^{-2}-y^{2}}}\right)
\D y\\
&=\pi+\frac{4m}{b}.
\end{split}
\end{equation}
This means light is bent, taking a trajectory roughly doodled
thus: 

\begin{center}
  \includegraphics{img/lecture04.0}
\end{center}

\begin{rmk}
Please avoid the temptation to Taylor expand in
\begin{equation}
\frac{\D\varphi}{\D u}=\frac{\pm1}{\sqrt{b^{-2}-u^{2}+2mu^{3}}}
\end{equation}
Do not Taylor expand the right hand side, specifically involving
the $mu^{3}$ term, about 0. We get something circuitous if we try.
\end{rmk}

\subsection{Second Approximation}
An older technique no longer taught, perhaps the most
straightforward, is
\begin{equation}\label{eq:lec3:approx2:omegaSquared}
\omega^{2}=u^{2}-2mu^{3}
\end{equation}
so we have
\begin{equation}
\varphi=\int^{1}_{0}\frac{\D u}{\sqrt{1-\omega^{2}}}.
\end{equation}
%% We have equation \eqref{eq:lec3:approx2:omegaSquared} imply
%% \begin{equation}
%% \omega=u\sqrt{1-2mu}
%% \end{equation}
%% and 
By expanding
\begin{equation}
u=\omega+\alpha_{1}\omega^{2}+\alpha_{2}\omega^{3}
\end{equation}
we get a nice systematic perturbation. Although it is possible to
solve equation \eqref{eq:lec3:approx2:omegaSquared}, that is not
the point! No, what we do is rewrite it as
\begin{equation}
\omega=u\sqrt{1-2mu}
\end{equation}
then Taylor expand the squareroot on the right hand side up to
some term. This is when we make our approximation:
\begin{equation}\label{eq:lec3:approx2:omegaAfterTaylorExpansion}
\omega\approx u-mu^{2}.
\end{equation}
We consider
\begin{equation}
m^{2}u^{4}\lll u
\end{equation}
as our approximation, so squaring equation
\eqref{eq:lec3:approx2:omegaAfterTaylorExpansion} recovers equation \eqref{eq:lec3:approx2:omegaSquared}.


Observe equation
\eqref{eq:lec3:approx2:omegaAfterTaylorExpansion} is a quadratic equation in $u$, which has its
solution
\begin{equation}
u_{\pm}=\frac{1}{2m}(1\pm\sqrt{1-4m\omega}).
\end{equation}
We Taylor expand this to third order in $\omega$, taking the
physically meaningful root $u=u_{-}$
\begin{equation}
\begin{split}
u&\approx\frac{1}{2m}\left(2m\omega+\frac{1}{8}(4m\omega)^{2}+\frac{1}{16}(4m\omega)^{3}
\right)\\
&\approx \omega+m\omega^{2}+\frac{1}{2}m^{2}\omega^{3}.
\end{split}
\end{equation}
Thus
\begin{equation}
\D u=\D\omega+2m\omega\D\omega+\frac{3}{2}m^{2}\omega^{2}\D\omega,
\end{equation}
and our integral becomes
\begin{equation}
\begin{split}
\Delta\varphi
&=\int^{1}_{0}\frac{(1+2m\omega+\frac{3}{2}m^{2}\omega^{2})}{\sqrt{1-\omega^{2}}}\D\omega\\
&=2m+\left(1+\frac{3m^{2}}{4}\right)\pi
\end{split}
\end{equation}
Observe we have an additional term involving $m^{2}$ in this approximation.
\begin{ddanger}
These calculations should be carefully double checked, and
re-examined to make certain we did everything consistently. This
is left as an exercise to you, gentle reader!
\end{ddanger}
The astute reader probably feels discomfort at $b$
disappearing. Observe that half the angle of deflection is
\begin{equation}
\begin{split}
\frac{\Delta\varphi}{2} &=
\int^{1/b}_{0}\frac{(1+2m\omega+\frac{3}{2}m^{2}\omega^{2})}{\sqrt{b^{-2}-\omega^{2}}}\D\omega\\
&=\frac{\pi}{2}+\frac{2m}{b}+\frac{3\pi m^{2}}{8b^{2}}.
\end{split}
\end{equation}
Thus the total angle of deflection is
\begin{equation}
\Delta\varphi=\pi+\frac{4m}{b}+\frac{3\pi m^{2}}{4b^{2}}.
\end{equation}
Notice this agrees, to first order in $m$, with the first
approximation we made.

\subsection{Third Approximation}
Most introductory texts perform the following approximation
\begin{equation}
\begin{split}
u^{2}-2mu^{3}&=u^{2}(1-2mu)\\
&\approx u^{2}(1-mu)^{2}
\end{split}
\end{equation}
Choose a new variable
\begin{equation}
y=u(1-mu),
\end{equation}
and then our integral becomes
\begin{equation}
\varphi=\int\frac{\D u}{\sqrt{b^{-2}-y^{2}+(\mbox{small factor})}}
\end{equation}
To lowest order, this is the same trick as the first
approximation. Higher order terms needs Newtonian corrections. We
find
\begin{equation}
\D y=(1-2mu)\D u
\end{equation}
and so
\begin{subequations}
\begin{align}
\D u
&\approx\frac{\D y}{1-2mu}\\
&\approx(1+2mu)\D y\\
&\approx(1+2my)\D y
\end{align}
\end{subequations}
thus
\begin{equation}
\varphi=\pm\int\frac{(1+2my)\D y}{\sqrt{b^{-2}-y^{2}}}.
\end{equation}
The first term is the Newtonian integral, and the second term is
straightforward. Consider half of the path
\begin{equation}
\begin{split}
\varphi
&=\int^{y=1/b}_{y=0}\frac{(1+2my)\D y}{\sqrt{b^{-2}-y^{2}}}\\
&=\frac{\pi}{2}+\frac{2m}{b}.
\end{split}
\end{equation}
So the total deflection $\Delta\varphi$ is twice this:
\begin{equation}
\Delta\varphi=\pi+4\frac{m}{b}.
\end{equation}
This is the first order correction.

\subsection{Shapiro Time Delay}

\begin{wrapfigure}{r}{10pc}
  \vspace{-1pc}
  \includegraphics{img/lecture04.1}
  \vspace{-3pc}
\end{wrapfigure}

Here's the idea: send a radio signal from the Earth to the
satellite. There is a time delay from receiving the reflection.
The physical problem is doodled on the right, with the light's
trajectory as the dashed line.

Lets assess the problem. Since this is light, we have
\begin{equation}
\begin{split}
\D s^{2} &= 0\\
&= \left(1-\frac{2m}{r}\right)\D t^{2}
-\left(1-\frac{2m}{r}\right)^{-1}\D r^{2}
-r^{2}\D\varphi^{2}
\end{split}
\end{equation}
With a particular choice of coordinates we use the Newtonian approximation
\begin{equation}
r\sin(\varphi)=b.
\end{equation}
What to do? Well, we can derive a geodesic equation for this
approximation:
\begin{equation}
\sin(\varphi)\D r+r\cos(\varphi)\D\varphi=0
\end{equation}
which is rearranged to become
\begin{equation}
\D\varphi=\frac{-1}{r}\tan(\varphi)\D r.
\end{equation}
We square both sides and use basic trigonometry
\begin{equation}
\begin{split}
r^{2}(\D\varphi)^{2}&=\tan^{2}(\varphi)\;(\D r)^{2}\\
&=\frac{b^{2}}{r^{2}-b^{2}}\D r^{2}.
\end{split}
\end{equation}
Why do this? Because we can replace the $r^{2}\D\varphi^{2}$ term
in the $\D s^{2}$ expression:
\begin{equation}
\left(1-\frac{2m}{r}\right)\D t^{2}
=
\left(1-\frac{2m}{r}\right)^{-1}\D r^{2}
+\frac{b^{2}}{r^{2}-b^{2}}\D r^{2}.
\end{equation}
Remember we want to find the \emph{time delay}, so we get rid of
$\D t^{2}$ coefficient:
\begin{equation}
\D t^{2}=\left(1-\frac{2m}{r}\right)^{-2}\D r^{2}
+\left(1-\frac{2m}{r}\right)^{-1}\frac{b^{2}}{r^{2}-b^{2}}\D r^{2}.
\end{equation}
We make the approximation
\begin{equation}
\begin{split}
\left(1-\frac{2m}{r}\right)^{-2}&\approx
\left(1+\frac{2m}{r}\right)^{2}\\
&\approx 1+\frac{4m}{r}+\underbrace{\mathcal{O}(m^{2}/r^{2})}_{\text{negligible}}
\end{split}
\end{equation}
which simplifies our expression to be
\begin{equation}
\D t^{2}
\approx
\left[1+\frac{4m}{r}+\left(1+\frac{2m}{r}\right)\frac{b^{2}}{r^{2}-b^{2}}\right]\D
r^{2}.
\end{equation}
Now, we will perform a long and tedious calculation. The
uninterested reader may skip its proof.

\begin{prop}\label{prop:lec04:simplifyingCalc:shapiroTimeDelay}
We have
\begin{equation}
\left[1+\frac{4m}{r}+\left(1+\frac{2m}{r}\right)\frac{b^{2}}{r^{2}-b^{2}}\right]
=
\frac{r^{2}}{r^{2}-b^{2}}\left[1
+\frac{4m}{r}-\frac{2m}{r}\frac{b^{2}}{r^{2}}\right]
\end{equation}
\end{prop}
\begin{proof}
We see that
\begin{equation}
\left(1+\frac{2m}{r}\right)\left(\frac{b^{2}}{r^{2}-b^{2}}\right)
=\left(\frac{b^{2}}{r^{2}-b^{2}}+\frac{2m}{r}\frac{b^{2}}{r^{2}-b^{2}}\right)
\end{equation}
Adding $(1+4m/r)$ to this yields
\begin{equation}
\begin{split}
\left(1+\frac{4m}{r}\right)+\left(\frac{b^{2}}{r^{2}-b^{2}}+\frac{2m}{r}\frac{b^{2}}{r^{2}-b^{2}}\right)
&=\left(1+\frac{4m}{r}+\frac{b^{2}}{r^{2}-b^{2}}+\frac{2m}{r}\frac{b^{2}}{r^{2}-b^{2}}\right)\\
&=\left(\frac{r^{2}}{r^{2}-b^{2}}+\frac{4m}{r}+\frac{2m}{r}\frac{b^{2}}{r^{2}-b^{2}}\right)
\end{split}
\end{equation}
Factoring out $r^{2}/(r^{2}-b^{2})$ gives us
\begin{subequations}
\begin{align}
\left(\frac{r^{2}}{r^{2}-b^{2}}+\frac{4m}{r}+\frac{2m}{r}\frac{b^{2}}{r^{2}-b^{2}}\right)
&=
\frac{r^{2}}{r^{2}-b^{2}}\left(1+\frac{4m(r^{2}-b^{2})}{r^{3}}+\frac{2m}{r}\frac{b^{2}}{r^{2}}\right)\\
&=\frac{r^{2}}{r^{2}-b^{2}}\left(1+\frac{4m}{r}-\frac{4mb^{2})}{r^{3}}+\frac{2mb^{2}}{r^{3}}\right)\\
&=\frac{r^{2}}{r^{2}-b^{2}}\left(1+\frac{4m}{r}-\frac{2mb^{2})}{r^{3}}\right).
\end{align}
\end{subequations}
This concludes the proof.
\end{proof}

Proposition \eqref{prop:lec04:simplifyingCalc:shapiroTimeDelay} yields
\begin{equation}
\D t^{2}\approx\frac{r^{2}}{r^{2}-b^{2}}\left[1
+\frac{4m}{r}-\frac{2m}{r}\frac{b^{2}}{r^{2}}\right]\D r^{2}.
\end{equation}
Thus we obtain (taking the Taylor series for the square root on
the bracketed terms)
\begin{equation}
\D t\approx
\frac{\pm r}{\sqrt{r^{2}-b^{2}}}\left[
1+\frac{2m}{r}-\frac{mb^{2}}{r^{3}}
\right]\D r.
\end{equation}
What now?

We evaluate the integral
\begin{equation}
\int\D t=\pm\biggl[
\underbracket[0.5pt]{\sqrt{r^{2}-b^{2}}}_{\substack{\text{length of line}\\\text{in flat geometry}}}
+\underbracket[0.5pt]{2m\ln\left(\frac{r}{b}+\sqrt{\frac{r^{2}}{b^{2}}-1}\right)}_{\text{correction to first term}}
-\frac{m}{r}\sqrt{r^{2}-b^{2}}
\biggr].
\end{equation}

\subsection{Time Dilation}

\begin{wrapfigure}{r}{4pc}
  \vspace{-1pc}
  \includegraphics{img/lecture04.2}
\end{wrapfigure}

This is the last experiment we will consider: gravitational time
dilation, or gravitational redshifting. Most books for the lay
person describe it as ``time runs more slowly in a gravitational
field'' (although the immediate question we should ask is:
\emph{relative to what?}).

So for that to make sense, we need to describe how we are
measuring the rate of time, and how to compare these. We will
work again with the Schwarzschild metric. An observer from clock
1 sends a signal to clock 2. We doodle a spacetime diagram, so
it's at the same angle (i.e., we assume $\theta=\varphi=$constant). 

\begin{center}
  \includegraphics{img/lecture04.3}
\end{center}

\noindent\ignorespaces%
So the metric reads
\begin{equation}
\D s^{2}=-\left(1-\frac{2m}{r}\right)^{-1}\D r^{2}
+\left(1-\frac{2m}{r}\right)\D t^{2}=0
\end{equation}
for light. After re-arranging terms we find
\begin{equation}
\D t=\left(1-\frac{2m}{r}\right)^{-1}\D r
\end{equation}
Integration yields
\begin{equation}
t_{2}-t_{1}=\int^{r_{2}}_{r_{1}}\left(1-\frac{2m}{r}\right)^{-1}\D r.
\end{equation}
But look, we also have
\begin{equation}
(t_{2}+\Delta t_{2})-(t_{1}+\Delta t_{1})=\int^{r_{2}}_{r_{1}}\left(1-\frac{2m}{r}\right)^{-1}\D r.
\end{equation}
This implies
\begin{equation}
\Delta t_{2}=\Delta t_{1}.
\end{equation}

Now, an observer measures proper time, so
\begin{subequations}
\begin{align}
\Delta s_{1}
&=\int\D s\\
&=\int\sqrt{1-\frac{2m}{r_{1}}}\D t\\
&=\sqrt{1-\frac{2m}{r_{1}}}\Delta t_{1}
\end{align}
\end{subequations}
where we consider the observer sitting at $r_{1}$ and \emph{is not}
a photon.
Similarly, the observer at clock 2 will observe the interval
between ticks as
\begin{subequations}
\begin{align}
\Delta s_{2}
&=\sqrt{1-\frac{2m}{r_{2}}}\Delta t_{2}\\
&=\left(\frac{\sqrt{1-\frac{2m}{r_{2}}}}{\sqrt{1-\frac{2m}{r_{2}}}}\right)\Delta s_{1}\\
&\approx\left(1-\frac{m}{r_{2}}+\frac{m}{r_{1}}\right)\Delta s_{1}
\end{align}
\end{subequations}
for weak gravitational fields. We see that a photon wave is
redshifted
\begin{equation}
\frac{\Delta\lambda}{\lambda}\approx
-\frac{m}{r_{2}}+\frac{m}{r_{1}}.
\end{equation}
An observer far from the black hole would see a clock on the
Black Hole's horizon stop. There is nothing deep about this, however.



\begin{exercises}
\begin{xca}[The Newtonian Approximation]
In the Newtonian approximation, the spacetime metric is
\begin{equation}
\D s^2 = (1 + 2\phi)\,\D t^2 - (1 - 2\phi)(\D x^2 + \D y^2 + \D z^2)
\end{equation}
where $\phi$ is the Newtonian gravitational potential. This
approximation holds when $\phi$ is small compared to $1$ and
velocities $v^i = \D x^i/\D t$ are also small compared to $1$,
with $\phi$ of the same order as $v^2$. 

(Notation: Latin indices from the middle of the alphabet---$i$,
$j$, $k$, \dots---are spatial indices, going from 1 to
3. Remember that we are using units $c = 1$.) 

Show that to lowest order, the geodesics are the standard paths of Newtonian gravity, that
is, $\mathbf{a} = -\nabla\phi$. 
\end{xca}
\begin{xca}[Geodesics and the Christoffel connection]
Let $g^{ab}$ be the matrix inverse of the metric tensor, that is,
$g^{ab} g_{bc} = {\delta^{a}}_{c}$. Show that the geodesic
equation can be written in the form 
\begin{equation}
\frac{\D^{2}x^{a}}{\D s^{2}}
\Gamma^{a}_{bc}\frac{\D x^{b}}{\D s}\frac{\D x^{c}}{\D s}=0
\end{equation}
where
\begin{equation}
\Gamma^{a}_{bc} = g^{ad} (\partial_{b} g_{dc} + \partial_{c}
g_{db} - \partial_{d} g_{bc}) 
\end{equation}
$\Gamma^{a}_{bc}$ is known as the Christoffel connection, or the ``Christoffel symbols.''

(Hint: you will encounter an expression of the form $\D g_{ab}/\D s$. 
Remember that in the geodesic equation, $g_{ab}$ is the metric
along the geodesic, and is therefore a function of
$x^{c}(s)$. Use the chain rule.) 
\end{xca}
\begin{xca}[Deflection of a massive particle by the Sun]
In this problem, you will (approximately) compute the deflection
of a \emph{massive} particle in the Schwarzschild metric. Note:
some of this is quite hard! 

\textbf{(a)}
Recall that for a massive particle, we defined
\begin{equation}
\widetilde{E}=-\left(
1-\frac{2m}{r}\right)\frac{\D t}{\D s}.
\end{equation}
Find the relationship between $\widetilde{E}$ and the particle speed
\begin{equation}
v^{2} =
\left|\frac{\D\mathbf{x}}{\D t}\right|^{2}
\end{equation}
at $r\to\infty$. (Hint: at infinity, the Schwarzschild metric
reduces to the spherical coordinate 
form of the flat spacetime Minkowski metric $\D s^2 = \D t^2-\D x^2-\D y^2-\D z^2$.)

\textbf{(b)}
For a massive particle, the equation of motion we derived was
\begin{equation}
\left(\frac{\D u}{\D\varphi}\right)^{2}=
\frac{\widetilde{E}^{2}-(1-2mu)(1+\widetilde{L}^{2}u^{2})}{\widetilde{L}^{2}}
\end{equation}
with $u = 1/r$. Consider a particle coming in from infinity,
being deflected, and returning to infinity. Find the deflection
$\Delta\varphi$ in the Newtonian approximation, that is,
neglecting the  term $mu^{3}$. (The solution of the equation of motion is a hyperbola, and can be derived by  a number of methods, but I suggest that you use the technique we saw in class, integrating $\D\varphi$, since this will help in part c.)

\textbf{(c)}
Find the next order approximate expression for the deflection
$\Delta\varphi$, treating the relativistic term $mu^{3}$ as a
small perturbation. You can use the same method that we did in
deriving the deflection of light, including the
definition of a new variable $y = u(1 - mu)$, although the
integral will now be somewhat different---be careful about the
slightly tricky limits of integration! As in the case of light,
assume that $mu\ll1$. 

\textbf{(d)}
The impact parameter $b$ is defined as the minimum value of $r$
on the trajectory. You should already have worked this out in
step (b) to find your integration range. (Note that $b$ is the
turning point, the value at which the derivative $\D u/\D\varphi$
changes sign.) Write $\widetilde{L}$ as a function of $b$, and
rewrite the deflection $\Delta\varphi$ in terms of
$\widetilde{E}$ and $b$. You may assume that
$\widetilde{E}^{2}-1\gg m/b$.

\textbf{(e)}
Show that for speeds near the speed of light--that is,
$v\lesssim1$---the deflection is approximately 
\begin{equation}
\Delta\varphi\approx\frac{2m}{b}\left(1+\frac{1}{v^{2}}\right)
\end{equation}
and agrees with our result for light when $v = 1$.
\end{xca}
\end{exercises}
