%%
%% lecture15.tex
%% 
%% Made by alex
%% Login   <alex@tomato>
%% 
%% Started on  Tue Mar  6 12:20:26 2012 alex
%% Last update Tue Mar  6 12:20:26 2012 alex
%%

Recall that the geodesic equation for nearby geodesics has the
relative acceleration
\begin{equation}
A^{\mu} = {R^{\mu}}_{\sigma\rho\nu}u^{\rho}X^{\nu}u^{\sigma}.
\end{equation}
Lets examine the Newtonian approximation $u^{0}\approx 1$, $u^{i}\approx0$.
We see the spatial components of acceleration is
\begin{equation}\label{eq:spatialComponentsOfAcceleration}
A^{i}\approx {R^{i}}_{00\nu}X^{\nu}={R^{i}}_{00j}X^{j}.
\end{equation}
Note that
\begin{equation}
{R^{i}}_{00j}=\partial_{0}\Gamma^{i}_{0j}-\partial_{j}\Gamma^{i}_{00}
+\underbrace{\Gamma^{i}_{0\mu}\Gamma^{\mu}_{0j}-\Gamma^{i}_{j\mu}\Gamma^{\mu}_{00}}_{\approx0}
\end{equation}
and
\begin{equation}
\partial_{0}\Gamma^{i}_{0j}=0
\end{equation}
since $\Gamma^{i}_{0j}$ is time-independent. Thus
\begin{equation}
{R^{i}}_{00j}\approx-\partial_{j}\Gamma^{i}_{00}
\end{equation}
for our approximation.

In Newtonian gravity, the acceleration for a particle is
\begin{equation}\label{eq:deviationFirstParticle}
\ddot{x}^{i}=-\partial^{i}\Phi(x)
\end{equation}
Consider a separation vector $\vec{X}=X^{j}$, where we have
one test particle described by $x^{i}$ and another described by
$x^{i}+X^{i}$. The second particle's acceleration is
\begin{equation}\label{eq:deviationSecondParticle}
\frac{\D^{2}(x^i+X^i)}{\D
  t^{2}}=-g^{ij}\partial_{j}\Phi(x^{k}+X^k)
\end{equation}
We Taylor expand to find
\begin{equation}
\partial_{j}\Phi(x^{k}+X^k)=\partial_{j}\Phi(x^i)
+\partial_{k}\partial_{j}\Phi(x^{i})X^{k}+\dots
\end{equation}
Substitute the Taylor expansion into Equation
\eqref{eq:deviationSecondParticle}, and subtract out the
acceleration of the first particle described in Equation
\eqref{eq:deviationFirstParticle}, we obtain
\begin{equation}
\frac{\D^{2}X^{i}}{\D t^{2}}=-\delta^{ij}\left(\partial_{j}\partial_{k}\Phi\right)X^{k}.
\end{equation}
Compare this to Equation
\eqref{eq:spatialComponentsOfAcceleration}, we find
\begin{equation}
-{R^{i}}_{00k}=-\delta^{ij}\partial_{j}\partial_{k}\Phi
\end{equation}
Observe that we can find one component of the Ricci tensor:
\begin{equation}
\begin{split}
R_{00} &\approx {R^{i}}_{00i}\\
&=\nabla^{2}\Phi=4\pi G\rho
\end{split}
\end{equation}
where $\rho$ is the mass-density.

This is wonderful, but we really want to consider some action
\begin{equation}
I=\int\mathcal{L}\,\D^{4}x
\end{equation}
such that its vanishing first variation $\delta I=0$ yields the
equations of motion. This should also be
coordinate-independent. We recall that an $n$-form
$L_{0\dots(n-1)}\,\D x^{0}\wedge\dots\wedge\D x^{n-1}$ is
coordinate-independent, and we may integrate it over an
$n$-dimensional manifold. Since it is totally antisymmetric, we
know it has one component. We can write this out as
\begin{equation}
\begin{split}
\mathcal{L}&=\mbox{(Scalar)}\sqrt{-g}\\
&=L\sqrt{-g}.
\end{split}
\end{equation}
We also have a few other requirements. A derivation of the
action is given in Box \ref{box:EHAction}, but the resulting
Lagrangian is
\begin{equation}
\mathcal{L}_{EH}=(g^{\mu\nu}R_{\mu\nu}-2\Lambda)\sqrt{-g}
\end{equation}
giving us the field equations
\begin{equation}
R_{\mu\nu}-\frac{1}{2}g_{\mu\nu}R+\Lambda g_{\mu\nu}=\frac{\kappa^{2}}{2}T_{\mu\nu}
\end{equation}
where $\Lambda$ is the cosmological constant, and $T_{\mu\nu}$ is
the stress-energy tensor (describing the distribution of
energy-momentum density). 

\begin{Boxed}{Einstein--Hilbert Action}\label{box:EHAction}
Starting principles:
\begin{enumerate}
\item Action should be a coordinate-independent integral of a
local Lagrangian.
\item Gravitational part should depend on metric only (no
``background structure'')
\item Geometry should be pseudo-Riemannian (no torsion or
nonmetricity)
\item Field equations should contain no more than two derivatives
of metric.
\end{enumerate}

In four dimensions, most general action obeying these principles
is 
\begin{equation}
I_{EH} = \frac{1}{\kappa^{2}}\int\sqrt{-g}(R-2\Lambda)\,\D^{4}x
       = \frac{1}{\kappa^{2}}\int\sqrt{-g}(g^{\mu\nu}R_{\mu\nu}-2\Lambda)\,\D^{4}x
\end{equation}
where $\kappa$ and $\Lambda$ are constants.

Variation of the action:
\begin{equation}
\delta I_{EH} = \frac{1}{\kappa^{2}}\int[\delta(\sqrt{-g})(R-2\Lambda)
+ \sqrt{-g}\delta g^{\mu\nu} R_{\mu\nu}+\sqrt{-g}g^{\mu\nu}\delta
  R_{\mu\nu}]\,\D^{4}x
\end{equation}
Look at three terms separately:
\begin{enumerate}
\item $\delta g$: basic relationship $\ln\det M=\tr\ln M$
\begin{equation}
\delta\ln\det M = (\delta\det M)/\det M = \tr\delta\ln
M=\tr(M^{-1}\delta M)=-\tr(M\delta M^{-1})
\end{equation}
So $\delta g=-g_{\mu\nu}\delta g^{\mu\nu}$, \quad
$\delta\sqrt{-g}=-\frac{1}{2}\sqrt{-g}g_{\mu\nu}\delta
g^{\mu\nu}$.
\item second term is already in right form,
$\sqrt{-g}R_{\mu\nu}\delta g^{\mu\nu}$.
\item $\delta R_{\mu\nu}$: first note that although the
connection is not a tensor, $\delta\Gamma^{\rho}_{\mu\nu}$
\emph{is} a tensor.

(To see this, consider the difference between two covariant
derivatives, one with connection $\Gamma^{\rho}_{\mu\nu}$ and one
with connection
$\Gamma^{\rho}_{\mu\nu}+\delta\Gamma^{\rho}_{\mu\nu}$.)

Next check that
\begin{equation}
\delta R_{\mu\nu}
=\nabla_{\rho}\delta\Gamma^{\rho}_{\mu\nu}-\nabla_{\mu}\delta\Gamma^{\rho}_{\nu\rho}
\end{equation}
(You can check the variation explicitly, or you can look in
Riemann normal coordinates, where $\Gamma^{\rho}_{\mu\nu}=0$ at
some chosen point.)

Hence
\begin{equation}
\begin{split}
g^{\mu\nu}\delta R_{\mu\nu} 
&=\nabla_{\rho}(g^{\mu\nu}\delta\Gamma^{\rho}_{\mu\nu})-\nabla_{\mu}(g^{\mu\nu}\delta\Gamma^{\rho}_{\rho\nu})
=\nabla_{\rho}[g^{\mu\nu}\delta\Gamma^{\rho}_{\mu\nu}-g^{\rho\nu}\delta\Gamma^{\sigma}_{\nu\sigma}]\\ &=\frac{1}{\sqrt{-g}}\partial_{\rho}\left(\sqrt{-g}\left[g^{\mu\nu}\Gamma^{\rho}_{\mu\nu}-g^{\rho\nu}\delta\Gamma^{\sigma}_{\nu\sigma}\right]\right)
\end{split}
\end{equation}
where the last step uses the fact that for a vector $\nabla_{\mu}v^{\mu}=\frac{1}{\sqrt{-g}}\partial_{\mu}(\sqrt{-g}v^{\mu})$.
\end{enumerate}

Now combine the three terms. The last term gives a total
derivative,
\begin{equation}
\partial_{\rho}\left(\sqrt{-g}\left[g^{\mu\nu}\delta\Gamma^{\rho}_{\mu\nu}-g^{\rho\nu}\delta\Gamma^{\sigma}_{\nu\sigma}\right]\right)
\end{equation}
which integrates to zero as long as $\delta\Gamma$ goes to zero
fast enough at any boundaries. That leaves the first two terms:
\begin{equation}
\delta I_{EH} = \frac{1}{\kappa^{2}}\int\sqrt{-g}
\left[R_{\mu\nu}-\frac{1}{2}g_{\mu\nu}(R-2\Lambda)\right]\delta g^{\mu\nu}\,\D^{4}x
\end{equation}
Now assume there is an additional ``matter'' contribution $I_{m}$
to the action, and \emph{define}
\begin{equation}
\delta I_{m} = -\frac{1}{2}\int\sqrt{-g}T_{\mu\nu}\delta g^{\mu\nu}\,\D^{4}x
\end{equation}
Then
\begin{equation}
\delta I_{total} = \int \sqrt{-g}\left\{\frac{1}{\kappa^{2}}\left[R_{\mu\nu}-\frac{1}{2}g_{\mu\nu}(R-2\Lambda)\right]-\frac{1}{2}T_{\mu\nu}\right\}\delta g^{\mu\nu}
\,\D^{4}x
\end{equation}
and the field equations are 
\begin{equation}
R_{\mu\nu}-\frac{1}{2}g_{\mu\nu}R+\Lambda
g_{\mu\nu}=\frac{\kappa^{2}}{2}T_{\mu\nu}
\end{equation}
\end{Boxed}
\begin{rmk}[Further Reading]
We have discussed the Einstein--Hilbert action, but there are
other Lagrangians out there. Additionally, we have not even
discussed other choice of variables, nor have we discussed the
Hamiltonian formalism. These are reviewed in Peldan's
``Actions for Gravity''~\cite{Peldan:1993hi}. We will discuss the
Hamiltonian formalism in a follow up paper.
\end{rmk}
