%%
%% lecture21.tex
%% 
%% Made by alex
%% Login   <alex@tomato>
%% 
%% Started on  Thu Jul 12 10:40:45 2012 alex
%% Last update Thu Jul 12 10:40:45 2012 alex
%%

We have the Schwarzschild metric
\begin{equation}
\D s^{2} = \left(1-\frac{2m}{r}\right)\D t^{2}
-\left(1-\frac{2m}{r}\right)^{-1}\D r^{2}
-r^{2}\,\D\Omega^{2}
\end{equation}
where
\begin{equation}
\D\Omega^{2} = \D\theta^{2}+\sin^{2}\theta\D\varphi^{2}
\end{equation}
is the usual notation for the metric on $(n-2)$-sphere. 

\begin{thm}[Birkhoff]
Spherically symmetric vacuum field equations imply the
Schwarzschild solution.
\end{thm}

Physically if we have a spherically symmetric field, we can treat
it as concentrated at a point, all gravitation waves have to be
spherical, but they're really quadrapole or higher order.

We chose $t$ by demanding a time variable such that everything's
independent of it. Our solution is still perfectly good. We would
like some physical time component with some physical meaning.

We have an observer shooting off light to the cylinder of
constant radius. We could equally make this baseballs instead of
photons, which is useful for collapsing spherical shells. The
diagram is
\begin{center}
\includegraphics{img/lecture21.0}
\end{center}
We have light (null geodesics) and it's only radial (so we have
$\D\Omega^{2}=0$). Then we have 
\begin{equation}
\D s^{2} = 0 = \left(1-\frac{2m}{r}\right)\D t^{2}
-\left(1-\frac{2m}{r}\right)^{-1}\D r^{2}
\end{equation}
which implies
\begin{equation}
\pm\left(1-\frac{2m}{r}\right)^{-1}\D r = \D t.
\end{equation}
We introduce a coordinate $r_{*}$ such that
\begin{equation}
\D t= \pm\D r_{*}
\end{equation}
so
\begin{equation}
r_{*} = r + 2m\ln\left|\frac{r}{2m}-1\right|.
\end{equation}
Either $t-r_{*}=u$ or $t+r_{*}=v$ where $u$, $v$ are constants
and the same as geodesics $v$-labeling.

We\marginpar{advanced Eddington--Finkelstein Coordinates} eliminate $t$ from the metric, so we get
\begin{equation}
\D s^{2} = \left(1-\frac{2m}{r}\right)\D v^{2}-2\,\D v\,\D
r-r^{2}\,\D\Omega^{2}
\end{equation}
This is the same metric expressed in different coordinates. They
are called the \define{advanced Eddington--Finkelstein Coordinates}.
If we used $u$ instead of $v$, we'd get \emph{retarded}
Eddington--Finkelstein coordinates.

The coordinates with $v$ yields a bit of information. The null
radial geodesics satisfy
\begin{equation}
\left(1-\frac{2m}{r}\right)\D v^{2}=2\,\D r\,\D v
\end{equation}
The solutions are either
\begin{equation}
v=\mbox{const.},\quad\mbox{or}\qquad
\frac{\D r}{\D v}=\frac{1}{2}\left(1-\frac{2m}{r}\right).
\end{equation}
Outgoing geodesics asymptotically approach $r=2m$. One thing to
note is that $r=2m$ is a null geodesic (i.e., it's lightlike)!

\begin{defn}
A \define{Killing Horizon} is when a Killing vector changes from
timelike to lightlike.
\end{defn}

The\marginpar{Kruskal--Szekeres Coordinates} next thing to do is
try replacing $r$ with $u$. It's easier to first define
\begin{equation}
U=\exp(-u/4m),\quad\mbox{and}\quad V=\exp(v/4m).
\end{equation}
We find (plugging these back into the Schwarzschild solution, we
have
\begin{equation}
\D s^{2}=\frac{32m^{3}}{r}\exp(-r/2m)\,\D U\,\D V -r^{2}\,\D\Omega^{2}
\end{equation}
where $r=r(U,V)$ is defined by
\begin{equation}
\left(\frac{r}{2m}-1\right)\exp(r/2m)=-UV,\quad\mbox{and}
\quad\frac{U}{V}=\exp(-t/2m).
\end{equation}
These coordinates are called \define{Kruskal--Szekeres Coordinates}.
One of the nice things about these coordinates: nothing in
particular goes horribly awry when the metric goes to zero,
everything's nicely behaved.

\begin{wrapfigure}{r}{13pc}
\vspace{-1pc}
\includegraphics{img/lecture21.2}
\vspace{-1pc}
\end{wrapfigure}
Consider $r=2m$, in our new coordinates this is
\begin{equation}
UV=0.
\end{equation}
Our event horizon has two solutions
\begin{equation}
U=0,\quad\mbox{or}\quad V=0.
\end{equation}
These are null geodesics, so $U=0$, $V=0$ gives two lines are
$45^{\circ}$ angles.

When $r=0$, we have $UV=1$. This is a hyperboloid. The
hyperboloid is drawn to the right with dashed lines to denote a
genuine singularity (usually, it's with a ``squiggly'' line).
We also have the situation when $r$ is a constant and $r>2m$;
then $UV<0$ is also constant. Conversely, when $r<2m$ is a
constant, we have $UV>0$ be constant. These situations are drawn
in red and blue to the right.


\begin{wrapfigure}{l}{11pc}
\vspace{-1pc}
\includegraphics{img/lecture21.3}
\vspace{-1pc}
\end{wrapfigure}
We have 4 regions labeled as shown on the left. Regions I and IV
are outside of the black hole. Regions II and III are inside of
the black hole. If we enter these regions, we necessarily hit the
singularity (we'd need to travel faster than light to escape the
region). Note that a ``white hole'' is just a time-reversed black
hole. The regions relevant for black holes are I and II, whereas
I and III are relevant for white holes.

We see black holes but not white holes. Why? Well, we're working
with $T^{\mu\nu}=0$. We're working with matter collapsing, all we
really have for the vacuum is part of region I and part of region II.

We think of white hole/black hole as an eternal black hole
perhaps formed by early fluctuations of the young universe (Hsu
suggests something along these lines~\cite{Hsu:2010vp}), perhaps
this is a wrong intuition. 

There is no solution of the vacuum with an isometry which takes
region III into any other region. Presumably the white holes
radiate away.

Now\marginpar{Penrose Diagrams} we had
\begin{equation}
\D s^{2} = (\dots)\D U\,\D V - r^{2}\,\D\Omega^{2}.
\end{equation}
If we multiply by a function, it doesn't change null
geodesics. Penrose invented a trick to make $r=\infty$ into a
finite distance, a doodle called a \define{Penrose Diagram}. For
the Schwarzschild metric, we have the diagram:
\begin{center}
\includegraphics{img/lecture21.4}
\end{center}
This distorts area but preserves the causal
structure\footnote{This is because the ``causal structure'' is
  determined by the angles between intersecting curves; it's a
  conformal transformation.}. 

\begin{exercises}
\begin{xca}[Black holes and trapped surfaces]
The Schwarzschild metric in Kruskal-Szekeres coordinates is
\begin{equation}
\D s^{2} =
\frac{32m^{2}}{r}\E^{-r/2m}
\bigl(-\D T^{2}+\D X^{2}\bigr)
 + r^2 \,\D\Omega^2
\end{equation}
where $r$ is viewed as a function of $X$ and $T$: 
\[
\left(\frac{r}{2m}-1\right)\E^{r/2m}=X^{2}-T^{2}
\]
\noindent\textbf{a.\quad}\ignorespaces Show that radial null
geodesics emitted from the two-sphere $(T_{0}, X_{0}, \theta,
\varphi)$ are described by the equation of motion
\begin{equation}
X - X_{0} = \epsilon(T - T_{0}),\quad
\theta=\mbox{const.},\quad\varphi=\mbox{const.}
\end{equation}
where $\epsilon=1$ for outgoing geodesics and $\epsilon=-1$ for
ingoing geodesics.

\noindent\textbf{b.\quad}\ignorespaces Consider the new
two-sphere formed by the wave front at time $T$ of these radial
geodesics. Show that the area of this sphere is $A =
4\pi{r^{2}(X, T )}$. (Hint: this is not completely obvious; you
need to think about how area is defined in a curved spacetime.)

\noindent\textbf{c.\quad}\ignorespaces In region I, $X_{0}>0$ and
$-X_{0}<T_{0}<X_{0}$. By considering $\D{A}/\D{T}$, show that the
the area $A$ increases with $T$ for outgoing geodesics, and
decreases for ingoing geodesics. 

\noindent\textbf{d.\quad}\ignorespaces In region II (inside the
event horizon), $T_{0}>0$ and $-T_{0}<X_{0}<T_{0}$. Show that in
this region, $A$ decreases with $T$ for both ingoing and outgoing
geodesics. This is the condition that the initial sphere $(T_{0},
X_{0}, \theta, \varphi)$ is a trapped surface. 
\end{xca}
\end{exercises}


