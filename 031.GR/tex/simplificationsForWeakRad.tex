%%
%% simplificationsForWeakRad.tex
%% 
%% Made by alex
%% Login   <alex@tomato>
%% 
%% Started on  Sun Mar 11 14:00:37 2012 alex
%% Last update Sun Mar 11 14:00:37 2012 alex
%%


\begin{Boxed}{Some Simplifications for Weak Gravitational Radiation}
We saw that to first order in perturbation theory
\begin{equation}
\bar{h}_{\mu\nu}(\mathbf{x},t)=
4G\int\frac{T_{\mu\nu}(\mathbf{y},t-|\mathbf{y}-\mathbf{x}|)}{|\mathbf{y}-\mathbf{x}|}\,\D^{3}y
\end{equation}
Let us concentrate on the purely spatial components
$\bar{h}_{ij}$ since the remaining components $\bar{h}_{0\mu}$
may be obtained by using the harmonic gauge condition
$\partial^{\mu}\bar{h}_{\mu\nu}=0$.

First, suppose an isolated source is at a distance $R$, and has
linear size $r\lll R$. Then to a good approximation,
\begin{equation}\label{eq:box2:goodApprox}
\bar{h}_{\mu\nu}(\mathbf{x},t)=
\frac{4G}{R}\int T_{\mu\nu}(\mathbf{y},t-|\mathbf{y}-\mathbf{x}|)\,\D^{3}y
\end{equation}
Now, to lowest order in $h$, energy conservation implies that
\begin{equation}
\partial_{\mu}T^{\mu\nu}=0=\partial_{i}T^{i\nu}+\partial_{t}T^{t\nu}
\end{equation}
We can now use a trick. Note the identities
\begin{equation}\label{eq:box2:trick:id1}
\begin{split}
\partial_{k}(x^{i}T^{kj})&=\delta^{i}_{k}T^{kj}+x^{i}\partial_{k}T^{kj}\\
&=T^{ij}-x^{i}\partial_{t}T^{tj}
\end{split}
\end{equation}
\begin{equation}\label{eq:box2:trick:id2}
\begin{split}
\partial_{\ell}(x^{i}x^{j}T^{t\ell})
&=\delta^{i}_{\ell}x^{j}T^{t\ell}+\delta^{j}_{\ell}x^{i}T^{t\ell}
+x^{i}x^{j}\partial_{\ell}T^{t\ell}\\
&=x^{j}T^{ti}+x^{i}T^{tj}-x^{i}x^{j}\partial_{t}T^{t\ell}
\end{split}
\end{equation}
Solving \eqref{eq:box2:trick:id1} for $T^{ij}$, using the
symmetry of $T^{ij}$, and inserting \eqref{eq:box2:trick:id2},
we see that
\begin{align}
T^{ij}
&=x^{i}\partial_{t}T^{tj}+\partial_{k}(x^{i}T^{kj})\nonumber\\
&=\frac{1}{2}\partial_{t}(x^{i}T^{tj}+x^{j}T^{ti})+
\frac{1}{2}\partial_{k}(x^{i}T^{kj}+x^{j}T^{ik})\nonumber\\
&=\frac{1}{2}\partial_{t}\Bigl(\partial_{\ell}(x^{i}x^{j}T^{t\ell})+x^{i}x^{j}\partial_{t}T^{tt}\Bigr)
+\frac{1}{2}\partial_{k}\Bigl(x^{i}T^{kj}+x^{j}T^{ik}\Bigr)\nonumber\\
&=\frac{1}{2}\partial_{t}^{2}(x^{i}x^{j}T^{tt})
+\frac{1}{2}\partial_{\ell}\Bigl(\partial_{t}(x^{i}x^{j}T^{t\ell})+x^{i}T^{\ell j}+x^{j}T^{i\ell}\Bigr)
\end{align}
We plug this back into Equation \eqref{eq:box2:goodApprox}. By
stokes theorem, the term involving $\partial_{\ell}$ integrates
to zero---by assumption, the source is isolated, so the integral
can be converted to a surface integral over a surface
\emph{outside} the source, where $T^{\mu\nu}=0$. Hence
\begin{equation}
\begin{split}
\bar{h}_{ij}(\mathbf{x},t)
&=\frac{2G}{R}\int\partial^{2}_{t}(y^{i}y^{j}T^{tt})\,\D^{3}y\\
&=\frac{2G}{R}\frac{\D^{2}}{\D t^{2}}\int y^{i}y^{j}T^{tt}(\mathbf{y},t-|\mathbf{y}-\mathbf{x}|)\,\D^{3}y.
\end{split}
\end{equation}
The integral is the quadrupole moment; thus, the metric
perturbation goes as the second time derivative of the quadrupole
moment.

For an isolated system of a few gravitating bodies (say, stars)
with masses of order $m$ and velocities of order $v$, the
quadrupole moment is $\sim mr^{2}$, and thus $\bar{h}\sim
Gmv^{2}/R$. Furthermore, if the system is gravitationally bound,
$v^{2}\sim Gm/r$, so $\bar{h}\sim v^{4}r/R$.

For a typical binary neutron star, $r\sim 10^{7}$ km and
$v^{2}\sim10^{-7}$; for such a system at a distance of a
kiloparsec, this gives $\bar{h}\sim10^{-21}$.
\end{Boxed}
