%%
%% prob4.tex
%% 
%% Made by alex
%% Login   <alex@tomato>
%% 
%% Started on  Wed Mar  7 09:06:46 2012 alex
%% Last update Wed Mar  7 09:06:46 2012 alex
%%
\begin{xca}[Bases and connections]
Given a basis of tangent vectors $e_{a}$, the connection
$\Gamma^{b}_{\mu a}$ can be determined by that prescription that
\begin{equation}\label{eq:prob4:basesConn:eqn1}
\frac{\D e_{a}}{\D s}=e_{b}\Gamma^{b}_{\mu a}\frac{\D x^{\mu}}{\D s}
\end{equation}
In general, this has to be either given from the start or
determined from the metric, but it's useful to look at a simple
example where it can be determined by what you already know. 

A coordinate basis for Cartesian coordinates in two dimensions is
\begin{equation}
e_{x}=\frac{\partial}{\partial x},\quad
e_{y}=\frac{\partial}{\partial y}
\end{equation}
The corresponding coordinate basis for polar coordinates is
\begin{equation}
\begin{split}
e_{r}&=\frac{\partial}{\partial r}=\frac{x}{r}e_{x}+\frac{y}{r}e_{y}\\
e_{\theta}&=\frac{\partial}{\partial\theta}=-ye_{x}+xe_{y}.
\end{split}
\end{equation}

\noindent\textbf{a.\quad}Show that the relation between $e_x$, $e_y$
and $e_r$, $e_\theta$ given above is correct, that is, that
$\partial/\partial r$ and $\partial/\partial\theta$ can be
expressed in terms of $\partial/\partial x$ and
$\partial/\partial y$ as shown.

\medbreak
\noindent\textbf{b.\quad}Suppose that $\D e_{x}/\D s = \D e_y/\D s =
0$. (This means the space is flat; we'll see this later.) 
Using Equation \eqref{eq:prob4:basesConn:eqn1}, find the
connection coefficients $\Gamma^{a}_{\mu b}$ in the polar coordinate basis.

\medbreak
\noindent\textbf{c.\quad}A different, ``noncoordinate'' basis for
polar coordinates is 
\begin{equation}
e_1 = \frac{\partial}{\partial r},\quad
e_{2}=\frac{1}{r}\frac{\partial}{\partial\theta}
\end{equation}
(This basis is useful in part because it's orthonormal; we'll see
this later in the course.) Find the connection coefficients in this basis.
\end{xca}
\begin{xca}[Killing vectors]\label{xca:prob4:killing}
A Killing vector $\chi^{\mu}$ is a vector that satisfies the
Killing equation
\begin{equation}\label{eq:prob4:killing}
\nabla_{\mu}\chi_{\nu}+\nabla_{\nu}\chi_{\mu}=0
\end{equation}

\noindent\textbf{a.\quad}Show that for any Killing vector
\begin{equation}
\nabla^{\mu}\nabla_{\mu}\chi^{\rho}=-{R^{\rho}}_{\sigma}\chi^{\sigma}
\end{equation}
(Hint: the way to get a curvature tensor in this kind of equation is by commuting covariant
derivatives somewhere.)

\medbreak\noindent\textbf{b.\quad}Show that if the connection is the standard Christoffel connection, then the Killing
Equation \eqref{eq:prob4:killing}
\begin{equation}
g_{\mu\rho}\partial_{\nu}\chi^{\rho}
+g_{\nu\rho}\partial_{\mu}\chi^{\rho}
+\chi^{\rho}\partial_{\rho}g_{\mu\nu}=0
\end{equation}
\end{xca}

\begin{xca}[Christoffel connection and curvature in two dimensions]\label{xca:prob4:2d}
Any Lorentzian metric on a two-dimensional manifold $M$ can be
locally put in the form 
\begin{equation}
\D s^2 = e^{2\phi} (-\D t^2 +\D x^2)
\end{equation}
by a suitable choice of coordinates in an open set on $M$. Here,
$\phi$ is an arbitrary function of $x$ and $t$. Find the
Christoffel connection for this metric (in a coordinate
basis). Find the curvature tensor. What condition must $\phi$
satisfy for the curvature to vanish? 

(Hint: use the symmetry of the curvature tensor to see that there
is only one independent nonzero component in two dimensions. This
will save a lot of work.) 
\end{xca}
\begin{xca}[``Moving frames'' in two dimensions]
Starting with the metric of exercise \ref{xca:prob4:2d} find an
orthonormal basis of one-forms, and use the Cartan structure
equations to find the connection one-form and the curvature
two-form. Compare your result to exercise \ref{xca:prob4:2d}. (The answers had
better be equivalent!)
\end{xca}
