%%
%% lecture22.tex
%% 
%% Made by alex
%% Login   <alex@tomato>
%% 
%% Started on  Sat Jul 14 11:32:52 2012 alex
%% Last update Sat Jul 14 11:32:52 2012 alex
%%

We observe over long distances the universe is homogeneous and
isotropic. ``Homogeneous'' means if we take two regions of space,
we cannot distinguish them. Certainly this is not true at small
distances (e.g., compare a human being and a rock). ``Isotropic''
says if you're at one point, every direction appears the same.

What is a homogeneous non-isotropic shape? A cylinder!

A sphere with a variable density depending on the distance from
the equator is isotropic but non-homogeneous.

Lets look at metrics that are homogeneous and isotropic. First
thing to notice is that this is a coordinate dependent statement.

Lets Choose a time $t'$ such that space at constant $t'$ is
homogeneous. So
\begin{equation}
\D s^{2}=A\,(\D t')^{2} + 2B_{i}\,\D x^{i}\,\D t' + g_{ij}\,\D
x^{i}\,\D x^{j}
\end{equation}
Homogeneity and isotropy requires $B_{i}=0$, otherwise $B_{i}$
picks out a direction. So we throw it away. Similarly, $A$ must
be a function of time only.

We choose a $t$ such that
\begin{equation}
\D t=\sqrt{-A(t')}\,\D t'.
\end{equation}
This locally rescales the $t'$ coordinate. We now can write
\begin{equation}
\D s^{2}=-\D t^{2}+g_{ij}\,\D x^{i}\,\D x^{j}
\end{equation}
Lets examine the (fixed time $t$) spatial metric's curvature:
\begin{equation}
{R^{ij}}_{kl}\to {R^{A}}_{B}
\end{equation}
where $A=12,13,23=ij$. We have a $3\times3$ matrix we can look at
its eigenvalues $\lambda_1$, $\lambda_2$, $\lambda_3$ and its
eigenvectors. We claim they are all equal, otherwise we could
pick out the largest eigenvector $v^{B}=v^{ij}$ then obtain a
3-vector $\varepsilon_{kij}v^{ij}$ and this picks out a
direction. But we cannot allow this! So we have
\begin{equation}
{R^{A}}_{B} = k{\delta^{A}}_{B}
\end{equation}
and $k$ cannot depend on spatial directions (otherwise its
gradient picks out some preferred direction). At best we have
$k=k(t)$ be a function of time. So
\begin{equation}
R_{ij\ell m}=k(g_{i\ell}g_{jm}-g_{mi}g_{j\ell})
\end{equation}
and we have a ``space of constant curvature''. We then find
\begin{equation}
g_{ij}\,\D x^{i}\,\D x^{j}=\frac{\D
  r^{2}}{1-k^{2}}+r^{2}(\D\theta^{2}+\sin^{2}\theta\,\D\varphi^{2})
\end{equation}
where $k=-1,0,1$. For $k=1$, we write $r=\sin\psi$ and thus we
obtain
\begin{equation}
g_{ij}\,\D x^{i}\,\D x^{j}=\D\psi^{2}+\sin^{2}\psi(\D\theta^{2}+\sin^{2}\theta\,\D\varphi^{2})
\end{equation}
which describes $S^{3}$. But $k=0$ gives us
\begin{equation}
g_{ij}\,\D x^{i}\,\D x^{j}=\D r^{2}+r^{2}(\D\theta^{2}+\sin^{2}\theta\,\D\varphi^{2})
\end{equation}
which is flat. Last $k=-1$ we pick $r=\sinh(\psi)$ and obtain
\begin{equation}
g_{ij}\,\D x^{i}\,\D x^{j}=\D\psi^{2}+\sinh^{2}\psi(\D\theta^{2}+\sin^{2}\theta\,\D\varphi^{2})
\end{equation}
which describes hyperbolic 3-space $\mathbb{H}^{3}$.

We have the metric as (at fixed time)
\begin{equation}
\D s^{2}=-\D t^{2}+\underbrace{a^{2}(t)}_{\mathrlap{\text{scale
      factor as function of time}}}\widetilde{g}_{ij}\,\D
x^{i}\,\D x^{j}
\end{equation}
What about the stress-energy tensor? We have
\begin{subequations}
\begin{equation}
T^{0i}=0
\end{equation}
otherwise we'll have a preferred direction, and
%\begin{equation}
\begin{align}
T^{00} &=\rho(t)\\
{T^{i}}_{j} &= p(t){\delta^{i}}_{j}
\end{align}
%\end{equation}
\end{subequations}
otherwise if we have ${T^{i}}_{j}$ be a function of position, its
gradient would determine some preferred direction. Note
\begin{align*}
\rho&=\mbox{energy density}\\
p&=\mbox{pressure}.
\end{align*}
We plug this into the Einstein field equation (using units where
$G_{N}=1$), we end up with the\marginpar{Friedmann Equations} Friedmann equations: 
\begin{subequations}\label{eq:friedmann}
\begin{align}
\frac{3\dot{a}^{2}}{a^{2}} &= 8\pi\rho-\frac{3k}{a^{2}}\\
\frac{3\ddot{a}}{a} &= -4\pi(\rho+3p).
\end{align}
\end{subequations}

We need\marginpar{Equation of State} a third equation, which is an equation of state, i.e.,
looks like
\begin{equation}
p=p(\rho).
\end{equation}
The simplest choice is
\begin{equation}
p=w\rho
\end{equation}
for some constant of proportionality $w$. If $w=0$, we have dust;
for $w=1/3$ we have radiation (it follows from Maxwell's
equations); and if $w=-1$, then we have a cosmological
constant. We can have some combination (e.g., a universe with
dust, radiation, and a cosmological constant). To get a
cosmology, we choose some equation of state then plug it back
in. 

The\marginpar{Perturbations} next step is to examine
perturbations. We need to do a weak-field approximation to a
background metric which is not necessarily flat.

From the Friedmann equations \eqref{eq:friedmann}, we see
\begin{equation}
\begin{aligned}
\frac{\D}{\D t}(3\dot{a}^{2}) 
&= 6\dot{a}\ddot{a} = -8\pi(\rho+3p)a\dot{a}\\
&=\frac{\D}{\D t}(8\pi\rho a^{2})
\end{aligned}
\end{equation}
thus
\begin{equation}
\dot{\rho}+3(\rho+p)\frac{\dot{a}}{a}=0.
\end{equation}
If $p=w\rho$, then this is easy to solve:
\begin{equation}
\rho=\rho_{0}a^{-3(1+w)}.
\end{equation}
If $w=0$, this is a conservation of dust
$\rho=\rho_{0}a^{-3}$. For radiation ($w=1/3$) we get an extra
factor for redshift. Observe when we have a cosmological constant
($w=-1$), we have $\rho$ be constant.

Suppose $\rho+3p>0$, there is an initial singularity. This isn't
due to the homogeneity and isotropy conditions, years ago
Hawking, Penrose and others proved if $\rho+3p>0$, then there is
an initial singularity. If $\rho+3p<0$, we have a big bounce.

\begin{exercises}
\begin{xca}[de Sitter and anti-de Sitter space]
Consider a homogeneous, isotropic cosmology with a nonzero
cosmological constant $\Lambda$ and an otherwise vanishing
stress-energy tensor. The cosmological constant can be thought of
as part of the stress-energy tensor, so this setting amounts to
saying that
\begin{equation*}
\rho =-p=\frac{\Lambda}{8\pi}
\end{equation*}
(in units $G_{N} = 1$).

\noindent\textbf{a.\quad}\ignorespaces Solve the Friedmann
equations (see Carroll~\cite{Carroll:2004st} section 8.3) for the case $\Lambda>0$,
considering all values of the spatial curvature parameter
$k$. This solution is called de Sitter space.

\noindent\textbf{b.\quad}\ignorespaces Do the same for
$\Lambda<0$. This solution is called anti-de Sitter space. 
\end{xca}
\end{exercises}
