%%
%% lecture16.tex
%% 
%% Made by alex
%% Login   <alex@tomato>
%% 
%% Started on  Tue Mar  6 15:06:59 2012 alex
%% Last update Tue Mar  6 15:06:59 2012 alex
%%
We will assume the cosmological constant vanishes $\Lambda=0$. We
see the field equations look like
\begin{equation}
G_{\mu\nu}=\frac{\kappa^{2}}{2}T_{\mu\nu}
\end{equation}
Ten components of the curvature tensor directly depend on $T_{\mu\nu}$.
The Weyl tensor indirectly depends on it. The $G_{\mu\nu}$
vanishes for flat space, yet the Weyl tensor describes the free
propagation of gravity waves.

Recall in electromagnetism, the source of the electric field is
charge $e$ or in the field equations charge density
$\rho_{e}$. With Lorentz transformation (viz.~length
contraction), charge density increases because volume decreases:
\begin{equation}
\begin{split}
&\rho_{e}\to\frac{1}{\sqrt{1-v^{2}}}\rho_{e}\\
\implies&\rho_{e}\sim J^{0}
\end{split}
\end{equation}
where $J^{\mu}$ is the 4-current.

We know mass is responsible for gravity, but rest mass or total
mass? Observationally, all forms of energy contributes to the
gravitational field. So the energy $E$ has energy density
$\rho_{m}$. How does this transform? Well, we see:
\begin{equation}
\begin{split}
E&\to\frac{1}{\sqrt{1-v^{2}}}E\\
\rho_{m}&\to\left(\frac{1}{1-v^{2}}\right)\rho_{m}
\end{split}
\end{equation}
This is what happens for a $00$ components of a rank-2 tensor. So
\begin{subequations}
\begin{equation}
\rho\approx T^{00}
\end{equation}
and similarly
\begin{equation}
\begin{split}
T^{0i}&\approx\mbox{``Energy Current''}\\
&\approx\mbox{``Momentum Density''}
\end{split}
\end{equation}
and
\begin{equation}
T^{ij}\approx\mbox{``Pressure''}.
\end{equation}
\end{subequations}
The field equations were thought of as
\begin{equation}
R_{\mu\nu}+g_{\mu\nu}R\propto T_{\mu\nu}
\end{equation}
just as for Newtonian gravity
\begin{equation}
\nabla^{2}\Phi=4\pi G\rho_{m}.
\end{equation}
Einstein at one point proposed
\begin{equation}
R_{\mu\nu}=T_{\mu\nu}
\end{equation}
but we can't change coordinates correctly, as these equations are
under-determined. We know in special relativity the conservation
of energy states
\begin{equation}
\partial_{\mu}T^{\mu\nu}=0
\end{equation}
So by the comma-goes-to-semicolon rule, we expect
\begin{equation}
\nabla_{\mu}T^{\mu\nu}=0.
\end{equation}
But only
\begin{equation}
\nabla_{\mu}G^{\mu\nu}=0
\end{equation}
whereas
\begin{equation}
\nabla_{\mu}R^{\mu\nu}\not=0.
\end{equation}

In general relativity, we use $I$ for the action and $S$ for the
entropy (when we Wick rotate $t\to\tau=-\I t$, the Euclidean action for a
black hole describes \emph{is} its entropy). We have the action
\begin{equation}
I=\int\sqrt{-g}L\,\D^{4}x
\end{equation}
and its variation is
\begin{equation}
\delta I=\int\sqrt{-g}E_{\mu\nu}\delta g^{\mu\nu}\,\D^{4}x
\end{equation}
This action is diffeomorphism-invariant. If $\delta g^{\mu\nu}$
is just a coordinate transformation, then $\delta I=0$ identically.
This is true ``on shell'' (when the equations of motion are
satisfied). What is $\delta g^{\mu\nu}$ under a change of
coordinates?
Consider
\begin{equation}
x^{\mu}\to x^{\mu}+\zeta^{\mu}
\end{equation}
We see then that
\begin{equation}
g_{\mu\nu}(x)\D x^{\mu}\D x^{\nu}
\to g_{\mu\nu}(x+\zeta)
\bigl(\D x^{\mu}+\partial_{\rho}\zeta^{\mu}\D x^{\rho}\bigr)
\bigl(\D x^{\nu}+\partial_{\sigma}\zeta^{\nu}\D x^{\sigma}\bigr)
\end{equation}
where we Taylor expand to first order the metric
\begin{equation}
g_{\mu\nu}(x+\zeta)=g_{\mu\nu}(x)+\zeta^{\tau}\partial_{\tau}g_{\mu\nu}(x).
\end{equation}
Observe this tells us how the metric changes, after some index
gymnastics we obtain
\begin{equation}
\begin{split}
g_{\mu\nu}&\to g_{\mu\nu}+(g_{\mu\rho}\partial_{\nu}\zeta^{\rho}+g_{\rho\nu}\partial_{\mu}\zeta^{\rho}+\zeta^{\rho}\partial_{\rho}g_{\mu\nu})\\
&\qquad=g_{\mu\nu}+\nabla_{\mu}\zeta_{\nu}+\nabla_{\nu}\zeta_{\mu}
\end{split}
\end{equation}
Thus
\begin{equation}\label{eq:lec16:killingVec}
\delta_{\zeta}g_{\mu\nu}=\nabla_{\mu}\zeta_{\nu}+\nabla_{\nu}\zeta_{\mu}.
\end{equation}
If the right hand side vanishes, we have a Killing vector (c.f.,
Exercise~\ref{xca:prob4:killing}). If we say the metric is
time-independent, then this is equivalent to stating there exists
some time-like Killing vector. Similarly, spherical symmetry
means that we have Killing vectors generate the spherical symmetries.

So, we see that
\begin{equation}
\delta_{\zeta}g^{\mu\nu}=-\nabla^{\mu}\zeta^{\nu}
-\nabla^{\nu}\zeta^{\mu}
\end{equation}
So under a coordinate transformation
\begin{equation}
\delta I=-\int\sqrt{-g}E_{\mu\nu}(\nabla^{\mu}\zeta^{\nu}+\nabla^{\nu}\zeta^{\mu})\,\D^{4}x.
\end{equation}
We have $E_{\mu\nu}=E_{\nu\mu}$ which simplifies the integrand
\begin{equation}
\delta I=-2\int\sqrt{-g}E_{\mu\nu}\nabla^{\mu}\zeta^{\nu}\,\D^{4}x
\end{equation}
since we're summing over dummy indices and $E$ is symmetric. Now
we may write this as
\begin{equation}
\delta I=-2\int\sqrt{-g}\Bigl[
\nabla^{\mu}(E_{\mu\nu}\zeta^{\nu})-(\nabla^{\mu}E_{\mu\nu})\zeta^{\nu}
\Bigr]\,\D^{4}x
\end{equation}
Recall
\begin{equation}
\nabla_{\mu}v^{\mu}=\frac{1}{\sqrt{-g}}\partial_{\mu}(\sqrt{-g}v^{\mu})
\end{equation}
thus the first term in the integrand becomes
\begin{equation}
-2\int\sqrt{-g}\nabla^{\mu}(E_{\mu\nu}\zeta^{\nu})\,\D^{4}x=-2\int\partial_{\mu}(\sqrt{-g}{E^{\mu}}_{\nu}\zeta^{\nu})\,\D^{4}x
\end{equation}
which we can always do, since the metric's covariant derivative
vanishes. This is a surface integral! Thus if $\zeta\to0$ ``fast
enough'' (or, equivalently, $\zeta=0$ on the boundary), the first
term vanishes.

Therefore
\begin{equation}
\int\sqrt{-g}(\nabla^{\mu}E_{\mu\nu})\zeta^{\nu}\,\D^{4}x=0
\end{equation}
which is true for arbitrary $\zeta$. This implies
\begin{equation}
\nabla^{\mu}E_{\mu\nu}=0
\end{equation}
which is a conservation law. But we cannot change it into
integral form. This is a special case of Noether's theorem. We
can run this backwards to get the equations of motion.
