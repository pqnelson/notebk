%%
%% lecture05.tex
%% 
%% Made by alex
%% Login   <alex@tomato>
%% 
%% Started on  Tue Feb 21 10:14:53 2012 alex
%% Last update Tue Feb 21 10:14:53 2012 alex
%%
A curved space is ``locally like'' $\RR^n$. What does this mean?
Well, we can take open discs in $\RR^{n}$ and paste them together
to form our curved space. The basic doodle describing this is
thus:

\begin{center}
  \includegraphics{img/lecture05.0}
\end{center}

\noindent\ignorespaces%
We have our curved space $\mathcal{M}$, and a neighborhood
$\mathcal{U}\propersubset\mathcal{M}$ which is ``like'' a
neighborhood $\mathcal{V}$ of $\RR^n$. We make this rigorous by a
mapping
\begin{equation}
\varphi\colon\mathcal{U}\to\mathcal{V}
\end{equation}
and demand it is bijective (one-to-one and onto). The map
$\varphi$ is called a \define{Coordinate Map}. Note that some
conventions have $\varphi$ going in the \emph{opposite}
direction, just a warning when reading other texts.

Note that since $\varphi$ is invertible, we can express an point
$p\in\mathcal{U}$ in terms of coordinates induced from
$\varphi(p)\in\RR^{n}$. 

The question we should ask is: what happens on overlapping
charts? We have two different descriptions, and we should hope
that the descriptions are ``the same.'' Lets consider the situation:

\begin{center}
  \includegraphics{img/lecture05.1}
\end{center}

The minimal condition on $\varphi_{2}\circ\varphi_{1}^{-1}$ is
that it is continuous and has a continuous inverse (i.e., it's a
``homeomorphism''). If $\varphi_{2}\circ\varphi_{1}^{-1}$ is
differentiable (or $C^{n}$ or analytic or \dots) and has a
differentiable ($C^{n}$, analytics, \dots) inverse, then
$\mathcal{M}$ is a \emph{Differentiable Manifold} (or a $C^{n}$
Manifold, analytic manifold, etc.). 
In practice, it is sufficient using a $C^2$ or $C^3$ manifold for
general relativity; however, most people prefer using
$C^{\infty}$ for as long as possible.
