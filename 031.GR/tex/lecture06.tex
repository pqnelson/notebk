%%
%% lecture06.tex
%% 
%% Made by alex
%% Login   <alex@tomato>
%% 
%% Started on  Wed Feb 22 10:02:42 2012 alex
%% Last update Wed Feb 22 10:02:42 2012 alex
%%

The main lessons to take home:
\begin{enumerate}
\item Manifolds have coordinates and in practice the actual
  details of a calculation depends on using coordinates;
\item Coordinates gives a set of maps, the coordinate system are
  not the manifold itself. Are the properties found the properties of the
  coordinate systems or of the manifold?
\end{enumerate}
We can answer the second point directly. Recall in $\RR^2$ we
have the line element be, in Polar coordinates,
\begin{equation}\label{eq:lec06:polarCoordinates}
\D s^{2}=\D r^{2}+r^{2}\D\theta^{2}.
\end{equation}
But look, it has a peculiar value for $\theta$ when $r=0$
(namely: distance is $\theta$-independent when $r=0$). But that's
dependent on the coordinate system! On the other hand, if we
change coordinate systems to write the line element as
\begin{equation}
\D s^{2}=\D\theta^{2}+\theta^{2}\D r^{2}.
\end{equation}
This is the same as equation \eqref{eq:lec06:polarCoordinates},
but $\theta$ is the radial distance and $r$ is the angular
component. Is there anything deep about $\RR^2$ detected here?
No, just faulty coordinate systems which break down at a single
point. 

\medskip
{\bf Moral:\quad}\ignorespaces Coordinate System's properties
$\not=$ Manifold's properties.

\medskip
In a manifold, there is no preferred basis, so we need to first
define a basis prior to defining a vector. There are several ways
to do this:
\begin{description}
\item[Old School:]
Deal with a manifold with coordinate system, and we use certain
rules describing how vectors (and friends) behave under a change
of coordinates.
\item[New School:]
We observe the directional derivative in a particular direction
is the same in any basis. So we say that a vector is
$\vec{v}\cdot\nabla$, e.g., in two dimensions
\begin{subequations}
\begin{align}
\vec{v}\cdot\nabla
&=v^{x}\partial_{x}+v^{y}\partial_{y}\\
&=v^{r}\partial_{r}+v^{\theta}\partial_{\theta}
\end{align}
\end{subequations}
We find that 
\begin{equation}
\begin{split}
v^{x}&=v(x)\\
&=v^{r}\frac{\partial x}{\partial r}+v^{\theta}\frac{\partial x}{\partial\theta}
\end{split}
\end{equation}
We have a quantity that is independent of our choice of basis and
it has a natural way to change under a change of coordinates.
\end{description}

In practice, we start with some manifold $\mathcal{M}$. We
consider some curve
\begin{equation}
\gamma\colon[0,1]\to\mathcal{M}.
\end{equation}
Let $p\in\mathcal{M}$, and consider a smooth function
\begin{equation}
f\colon\mathcal{M}\to\RR
\end{equation}
We see that the mathematical definition for the vector $v$ at $p$ is
\begin{equation}
v_{p}(f)=\left.\frac{\D}{\D\lambda}\bigl(f\circ\gamma(\lambda)\bigr)\right|_{\gamma(\lambda)=p}.
\end{equation}
This is fairly abstract.

We may use local coordinates to make things a bit easier. Lets
draw the doodle of the geometric situation:

\begin{center}
  \includegraphics{img/lecture06.0}
\end{center}

\noindent\ignorespaces %
We have our vector
\begin{subequations}
\begin{align}
v_{p}(f)
&=\frac{\D}{\D\lambda}(f\circ\gamma)\\
&=\frac{\D}{\D\lambda}(f\circ\varphi^{-1}\circ\varphi\circ\gamma)\\
&=\frac{\D}{\D\lambda}\bigl(\widetilde{f}\circ(\varphi\circ\gamma)\bigr)\\
&=\frac{\partial\widetilde{f}}{\partial\varphi^{\mu}}
\frac{\D(\varphi\circ\gamma)^{\mu}}{\D\lambda}\\
&\mbox{``=''}\partial_{\mu}\frac{\D\varphi^{\mu}}{\D\lambda}
\end{align}
\end{subequations}
where we ignore the distinction between the manifold and the
coordinates on the manifold in this last step. Observe this looks
like $\vec{v}\cdot\nabla\widetilde{f}$ the directional derivative!

The vector may be written as
\begin{equation}
v=v^{\mu}e_{\mu},
\end{equation}
with basis vectors
\begin{equation}
\{\partial_{\mu}\}=\{e_{\mu}\}.
\end{equation}
We can have what is called an\marginpar{Anholonomic Basis} anholonomic (or ``non-coordinate'')
basis\index{Anholonomic Basis}\index{Basis!Anholonomic} 
\begin{equation}
f_{a}={f_{a}}^{\mu}e_{\mu}
\end{equation}
where $\det(f)\not=0$ and $\mu,a=1,\dots,n$. The coordinate basis
satisfies 
\begin{equation}
[e_{\mu},e_{\nu}]=\partial_{\mu}\partial_{\nu}
-\partial_{\nu}\partial_{\mu}=0.
\end{equation}
The converse is true, if given any basis $f_{a}$ and
\begin{equation}
[f_{a},f_{b}]=0,
\end{equation}
then there is some coordinate system 
\begin{equation}
f_{a}=\partial/\partial y^{a}.
\end{equation}
This follows from the existence theorem on partial differential equations.

Lets consider how vectors transform under coordinate
changes.\marginpar{Vectors under change of coordinates} We
see that in two different coordinate systems $\partial_{\mu}$ and
$\partial_{\mu'}$ we can write
\begin{equation}
v=v^{\mu}\partial_{\mu}=v^{\mu'}\partial_{\mu'}.
\end{equation}
Observe how it acts on $x^{\nu}$:
\begin{subequations}
\begin{align}
v(x^{\nu})
&=v^{\mu}\partial_{\mu}x^{\nu}\\
&=v^{\mu}{\delta_{\mu}}^{\nu}\\
&=v^{\nu}
\end{align}
\end{subequations}
and using the other coordinate representation
\begin{subequations}
\begin{align}
v(x^{\nu})
&=v^{\mu'}\partial_{\mu'}x^{\mu}\\
&=v^{\mu'}\frac{\partial x^{\nu}}{\partial x^{\mu'}}
\end{align}
\end{subequations}
and setting equals to equals tells us
\begin{equation}
v^{\nu}=v^{\mu'}\frac{\partial x^{\nu}}{\partial x^{\mu'}}.
\end{equation}
This gives us the transformation law between our two coordinate systems.

\begin{rmk}
Note that all the vectors living at a single base point
$p\in\mathcal{M}$ form a linear space $\mathrm{T}_{p}\mathcal{M}$
called the \define{Tangent Space at $p$}.
\end{rmk}
