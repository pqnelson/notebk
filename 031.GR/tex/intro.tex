%%
%% intro.tex
%% 
%% Made by alex
%% Login   <alex@tomato>
%% 
%% Started on  Tue Feb 14 08:32:16 2012 alex
%% Last update Tue Feb 14 08:32:16 2012 alex
%%
\renewcommand{\leftmark}{Introduction}\phantomsection\addcontentsline{toc}{section}{Introduction}
\section*{Introduction}
These are my collected notes on classical general relativity. I
have reformatted, merged, and edited them into a cohesive whole.
The various resources used are either ``cited'' in-text, in
footnotes, or collected into the ``annotated'' bibliography in
the back. 

The references used are either books I own, or free eprinted
articles. This is the guideline I tried maintaining, but there
are exceptions to the rule of ``free eprinted articles.'' (An
additional problem: some articles are so old that they are not
[yet] eprinted and published online. Sad, I know, but still\dots)

Strictly speaking, the math used in the first part (the
pedagogical part) is not correct. We will be sloppy, as sloppy as
physicists are. Alas we use physicist's theorems too (e.g.,
Birkhoff's theorem is ``Spherically symmetric vacuum field
equations imply the Schwarzschild solution'')/ It's not
``incorrect'' per se, but it may give mathematicians indigestion.

%The plan is to divide this text into three parts (or perhaps
%release three texts):
I hope to write three texts: the first (which you are reading) is
a pedagogical introduction to classical general relativity. The
second concerns advanced portions of general relativity,
preparing the reader for the ADM formalism, numerical relativity,
treatment of spinors, and so on. The third deals with quantum gravity.
