%%
%% sets.tex
%% 
%% Made by Alex Nelson
%% Login   <alex@tomato>
%% 
%% Started on  Sun May 31 13:51:53 2009 Alex Nelson
%% Last update Sun May 31 13:51:53 2009 Alex Nelson
%%

Just a review of a few properties of sets, and some basic
approaches to proofs.

\begin{prop}\label{prop:emptySets}
  For any set $A$
\begin{enumerate}
\item The empty set is a subset of $A$:
\begin{equation*}
  \emptyset\subseteq A,\qquad\forall A.
\end{equation*}
\item The union of $A$ with the empty set is $A$
\begin{equation*}
  A\cup\emptyset=A,\qquad\forall A.
\end{equation*}
\item The intersection of $A$ with the empty set is the empty et
\begin{equation*}%\label{eq:}
  A\cap\emptyset=\emptyset,\qquad\forall A.
\end{equation*}
\item The Cartesian product of $A$ and the empty set is empty
\begin{equation*}%\label{eq:}
  A\times\emptyset=\emptyset,\qquad\forall A.
\end{equation*}
\item The only subset of the empty set is itself
\begin{equation*}%\label{eq:}
  A\subseteq\emptyset\Rightarrow A=\emptyset,\qquad\forall A
\end{equation*}
\item The power set of the empty set is a set containing only the
  empty set
\begin{equation*}%\label{eq:}
  2^{\emptyset}=\{\emptyset\}
\end{equation*}
\item The empty set has cardinality zero (it has zero
  elements). Moreover, the empty set is finite
\begin{equation*}%\label{eq:}
  |\emptyset|=0.
\end{equation*}
\end{enumerate}
\end{prop}

Given some statement
\begin{equation}%\label{eq:}
\text{If }P,\text{ then }Q.
\end{equation}
its \textbf{contrapositive} is 
\begin{equation}%\label{eq:}
\text{If }Q\text{ is not true, then }P\text{ is not true.} 
\end{equation}
Its \textbf{converse} is 
\begin{equation}%\label{eq:}
\text{If }P\text{, then }Q.
\end{equation}
Sometimes it's easier to prove the converse than to prove a given
proposition.

To show that two sets $A, B$ are equal, we need to show $A\subset
B$ and $B\subset A$.
