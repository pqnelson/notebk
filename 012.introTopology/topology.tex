%%
%% topology.tex
%% 
%% Made by Alex Nelson
%% Login   <alex@tomato>
%% 
%% Started on  Sun May 31 10:34:17 2009 Alex Nelson
%% Last update Sun May 31 10:34:17 2009 Alex Nelson
%%
\begin{prob}
The fundamental aim of topology, for all practical purposes, is
to generalize real analysis to sets other than $\mathbb{R}$. We
begin by generalising the notion of what an ``open set'' is. That
is, previously we thought of an open set as a set that doesn't
include its boundary, for example
\begin{equation}
S = \{x\in\mathbb{R}: -1<x<1\}
\end{equation}
is an open interval since it doesn't include -1 or 1. We wish to
generalize this notion, so lets begin by introducing a collection
of open subsets of a given set $X$ called a ``\emph{topology on $X$}''.
\end{prob}

\begin{defn}\label{defn:topology}
A \textbf{Topology} on a set $X$ is a collection $\mathcal{T}$ of
subsets of $X$ having the following properties:
\begin{enumerate}
\item $\emptyset$ and $X$ are in $\mathcal{T}$
\item The union of the elements of any subcollection of
  $\mathcal{T}$ is in $\mathcal{T}$
\item The intersection of the elements of any finite
  subcollection of $\mathcal{T}$ is in $\mathcal{T}$.
\end{enumerate}
A set $X$ for white a topology has $\mathcal{T}$ been specified
is called a \textbf{topological space}.
\end{defn}

We generalize the notion of an open set to be any subset
$U\subset X$ such that $U\in\mathcal{T}$. This ``weakens'' our
previous notion of an open set, i.e. this topological definition
includes our previous intuition about open sets and extends
beyond it. We call elements of $X$ ``\emph{points}'' of the
space. In fact, we will \emph{define} open sets as elements of
$\mathcal{T}$:

\begin{defn}\label{defn:openSets}
  Let $(X,\mathcal{T})$ be a topological space. Then the members
  of $\mathcal{T}$ are defined as \textbf{open sets}.
\end{defn}

Now this definition of a topological space is rather
abstract. Lets consider a few examples using finite sets.
\begin{ex}
Consider the set $X=\{a,b\}$ where $a,b$ are arbitrary
objects. The possible topologies on this are
\begin{enumerate}
\item $\left\{\emptyset,\{a,b\}\right\}$ This is trivially a
  topology, since it includes $X$, $\emptyset$. Their union
  $X\cup\emptyset=X$, their intersection
  $X\cap\emptyset=\emptyset$ are both in the topology. This is
  dubbed the ``\textbf{trivial topology}''. The only open sets are
  empty and $X$.
\item $\left\{\emptyset,\{a\},\{a,b\}\right\}$ This is also a
  topology, since $\{a\}\cap\{a,b\}=\{a\}$, $\{a\}\cup X=X$,
  $\{a\}\cup\emptyset=\{a\}$, $\{a\}\cap\emptyset=\emptyset$. All
  of these are in the topology.
\item $\left\{\emptyset,\{b\},\{a,b\}\right\}$ This is just
  switching $a$ with $b$, which is also a topology.
\item $\left\{\emptyset,\{a\},\{b\},\{a,b\}\right\}$ This is also
  a topology, it's all possible subsets of $X$. This is given a
  special name known as the ``\emph{power set}'' or
  ``\textbf{discrete topology}''. Here, \emph{every} subset is open.
\end{enumerate}  
We can see a diagrammatic scheme of these topologies in fig \eqref{fig:img1}.
\end{ex}

\begin{figure}[t]
\includegraphics{img.1}
\caption{An example of all the possible topologies on the set
  $X=\{a, b\}$ corresponding to (from top to bottom, left to right)
  the discrete topology, $\{\emptyset,\{b\},X\}$,
  $\{\emptyset,\{a\},X\}$, and the trivial topology.}\label{fig:img1}
\end{figure}

\begin{ex}
This is a nonexample. Consider the set $X=\{a,b,c\}$. The collection
\begin{equation}
\mathcal{T} = \left\{\emptyset,\{a,b\},\{b,c\},X\right\}.
\end{equation}
Observe that
\begin{equation}
\{a,b\}\cap\{b,c\}=\{b\}\notin\mathcal{T}
\end{equation}
So it's not closed under finite intersections, so $\mathcal{T}$
isn't a topology.
\end{ex}

\begin{prop}\label{prop:discreteTopologyFromSingletons}
  If we have some topological space $(X,\mathcal{T})$, and for
  each $x\in X$ we have $\{x\}\in\mathcal{T}$, then $\mathcal{T}$
  is the discrete topology.
\end{prop}
\begin{proof}
We want to show that $\mathcal{T}$ is the power set of $X$. We
know that arbitrary unions of subsets of $X$ in $\mathcal{T}$ is
also in $\mathcal{T}$, and by hypothesis all singletons are
elements of $\mathcal{T}$. It follows that any arbitrary subset
of $U\subset X$ can be written as an arbitrary union of
singletons
\begin{equation}
U = \bigcup_{x\in U} \{x\}.
\end{equation}
Since this is any arbitrary subset of $X$, it follows that
\emph{every} subset of $X$ is in $\mathcal{T}$. Thus
$\mathcal{T}$ is the discrete topology.
\end{proof}

\begin{defn}\label{defn:comparisonTopologies}
  Let $\mathcal{T}$, $\mathcal{T}'$ be two topologies on the same
  set $X$. We say that
\begin{equation}
\mathcal{T}\supset\mathcal{T}'\iff\begin{cases}
\mathcal{T}\text{ is \textbf{finer} than }\mathcal{T}'\\
\mathcal{T}'\text{ is \textbf{coarser} than }\mathcal{T}.
\end{cases}
\end{equation}
\end{defn}
\begin{rmk}
This notion of ``finer'' and ``coarser'' topologies are worthy of
explanation. The typical explanation is to think of sand paper,
or gravel. Munkres~\cite{munkres} gives the example of having a
topology be visualized by a collection of large rocks. If we
break it up into finer gravel, we can ``union'' them together to
get back the ``coarser topology'' and it takes up less
volume. Intuitively, this is a ``finer'' configuration of
pebbles. 
\end{rmk}
