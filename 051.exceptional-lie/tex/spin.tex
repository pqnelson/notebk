\chapter{Spin Groups and their Representations}

\section{Review of Clifford Algebras}

\M
Clifford algebras may be formed from a vector space $V$ over $\FF$. It's
especially intuitive when $V$ is finite-dimensional. We just take some
canonical basis $\Gamma_{1}$, \dots, $\Gamma_{n}$ for $V$, then we turn it into an
algebra by defining multiplication on vectors by linearity and the
following anticommutator relation:
\begin{equation}
\Gamma_{i}\Gamma_{j} + \Gamma_{j}\Gamma_{i} = -\delta_{i,j}.
\end{equation}
Observe that $\Gamma_{i}^{2}=-1$ and $\Gamma_{i}\Gamma_{j}=-\Gamma_{j}\Gamma_{i}$ when $i\neq j$.
The resulting algebra is precisely the \define{Clifford Algebra} $\Cl(V)$.

We can generalize this by altering the anticommutation relation to be
\begin{equation}
\Gamma_{i}\Gamma_{j} + \Gamma_{j}\Gamma_{i} = \eta_{i,j}
\end{equation}
for some matrix $\eta$. When $\eta_{i,j}=0$, we obtain the
\define{Grassmann Algebra}. We can also have ``split'' Clifford algebras
by having the signature of $\eta$ being indefinite.

\begin{remark}
Historically, people describe a Clifford algebra using a quadratic form
(a generalized ``norm squared'' quantity). This is completely equivalent
to what we have done. Namely, $\eta_{i,j}=-Q(\Gamma_{i})\delta_{i,j}$ where
$Q$ is the quadratic form on $V$.
\end{remark}

\N{Even Part}
The even part of the Clifford algebra is written as $\Cl^{0}(V)$ or
$\Cl(V)_{0}$. It consists of guys that look like
\[ a_{0}\cdot\mathbf{1} + a^{\mu\nu}\Gamma_{\mu}\Gamma_{\nu} + a^{\alpha\beta\mu\nu}\Gamma_{\alpha}\Gamma_{\beta}\Gamma_{\mu}\Gamma_{\nu}+\dots.\]
Each term has an even number of Clifford generators appearing in it. The
coefficients $a^{\mu\nu}$, $a^{\alpha\beta\mu\nu}$, etc., are completely
antisymmetric in all its indices; equivalently, we could simply sum over
$\alpha<\beta<\mu<\nu<\dots$. 

\begin{definition}
Let $A$ be an $\FF$-algebra (usually $\FF=\RR$, $\CC$, or $\HH$). An
\define{$A$-Representation} of the Clifford algebra $\Cl(V)$ is an algebra morphism
\[ \rho\colon\Cl(V)\to\hom_{A}(W, W) \]
where $W$ is a $A$-algebra representation, $\hom_{A}(W, W)$ is the space
of endomorphisms of $W$ commuting with the $A$-action.
\end{definition}

\N{Semisimple Algebras}
We call an algebra $A$ \define{Semisimple} if all its modules are
completely reducible (= direct sums of simple modules). This is one
possible criterion. We can consider $A$ as a left $A$-module when $A$ is
associative\footnote{If I recall correctly, we need associativity.}, and
then this coincides with $A$ as a semisimple $A$-module. Then $A$ is
semisimple if and only if the left regular representation for $A$ is
completely reducible.

In particular, $\Cl(V)$ is a semisimple algebra. Recall semisimple rings
generalize the concept of a group algebra $\CC[G]$ for finite groups
$G$. For a review of the structure theorem for Clifford algebras, see
\arXiv{1610.02418}. In particular, all irreducible representations of a
semisimple associative algebra may be found as subrepresentations of its
regular representation.\footnote{See Theorem 3.2.13 in \url{https://studenttheses.uu.nl/bitstream/handle/20.500.12932/29895/thesis.pdf?sequence=2}}

For more on this, see chapter 5 of Varadarajan~\cite{Varadarajan:2004yz}.
Specifically, when $V$ is even-dimensional, $\Cl(V)$ is semisimple ---
this is proven in Theorem~5.2.7, and the odd-dimensional case is proven
in Theorem~5.2.10.

\N{Notation: ``Volume Element''}
We write $\omega=\prod^{m}_{j=1}\Gamma_{j}\in\Cl(V)$ and refer to it as the
``\emph{Volume Element}''. Adams does not use this notation (or
terminology), but it caught on in the mathematical literature.
Physicists call it the \emph{Chiral Element}.

\N{Notation}
We write $\FF(n)$ for the algebra of $n\times n$ matrices with entries
in the division algebra $\FF$ (usually $\RR$, $\CC$, $\HH$).

We also write $\Cl_{p,q}$ for $\Cl(\RR^{p,q})$ using the indefinite norm
of signature $(p,q)$ with quadratic form
\begin{equation}
Q(x) = (x_{1})^{2}+\cdots+(x_{p})^{2}-(x_{p+1})^{2}-(\cdots)-(x_{p+q})^{2}.
\end{equation}
We write $\Cl_{p}=\Cl_{p,0}$. This is all fairly standard notation.

\begin{lemma}
Let $\FF=\RR$, $\CC$, or $\HH$.
Consider $\FF(n)$ as an $\RR$-algebra.
Then $\FF(n)$ has a natural representation on $\FF^{n}$ and, up to
equivalence, this is the only irreducible representation.
\end{lemma}

\begin{proof}
Matrix algebras are simple. Simple algebras have a single representation
up to isomorphism.
\end{proof}

\begin{proposition}
\begin{enumerate}
\item $\RR(n)\otimes\RR(m)\iso\RR(nm)$ for all $n$, $m\in\NN$
\item $\RR(n)\otimes_{\RR}\FF\iso\FF(n)$ for all $n\in\NN$
\item $\CC\otimes_{\RR}\CC\iso\CC\oplus\CC$
\item $\CC\otimes_{\RR}\HH\iso\CC(2)$
\item $\HH\otimes_{\RR}\HH\iso\RR(4)$.
\end{enumerate}
\end{proposition}

\N{Facts}
We can compute:
\begin{subequations}
\begin{align}
\Cl_{1,0} &= \CC\\
\Cl_{0,1} &= \RR\oplus\RR\\
\Cl_{2,0} &= \HH\\
\Cl_{0,2} &= \RR(2)\\
\Cl_{1,1} &= \RR(2).
\end{align}
\end{subequations}
More helpfully, arranged as a table
\begin{equation}
\begin{array}{c|ccc}
(\Cl_{0,n})   & \huh   & \huh & \huh\\
  \RR(2)     & \huh   & \huh & \huh\\
\RR\oplus\RR & \RR(2) & \huh & \huh\\\hline
\RR          &  \CC   &  \HH & (\Cl_{m,0})
\end{array}
\end{equation}

\begin{theorem}
For all $n$, $r$, $s\geq0$, we have the following isomorphisms:
\begin{enumerate}
\item $\Cl_{n,0}\otimes\Cl_{0,2}\iso\Cl_{0,n+2}$
\item $\Cl_{0,n}\otimes\Cl_{2,0}\iso\Cl_{n+2,0}$
\item $\Cl_{r,s}\otimes\Cl_{1,1}\iso\Cl_{r+1,s+1}$.
\end{enumerate}
\end{theorem}

\begin{remark}
The third condition, $\Cl_{r,s}\otimes\Cl_{1,1}\iso\Cl_{r+1,s+1}$,
allows us to ``propagate'' a known entry ``diagonally'' across the table
of Clifford algebras.
\end{remark}

\N{Bott Periodicity}
We also have the following isomorphisms:
\begin{enumerate}
\item $\Cl_{n+8,0}\iso\Cl_{n,0}\otimes\Cl_{8,0}$
\item $\Cl_{0,n+8}\iso\Cl_{0,n}\otimes\Cl_{0,8}$
\item For complex Clifford Algebras: $\CCl_{n+2}\iso\CCl_{n}\otimes_{\CC}\CCl_{2}$.
\end{enumerate}

\M
This gives us the following table for real $\Cl_{p,q}$:
\begin{equation*}
  \begin{footnotesize}\hskip-3em
\begin{array}{c|ccccccccc}
  8 & \RR(16)            & \CC(16) & \HH(16) & \HH(16)\oplus\HH(16) & \HH(32) & \CC(64) & \RR(128) & \RR(128)\oplus\RR(128) & \RR(256)\\
  7 & \CC(8)             & \HH(8) & \HH(8)\oplus\HH(8) & \HH(16) & \CC(32) & \RR(64) & \RR(64)\oplus\RR(64) & \RR(128) & \CC(128)\\
  6 & \HH(4)             & \HH(4)\oplus\HH(4) & \HH(8) & \CC(16) & \RR(32) & \RR(32)\oplus\RR(32) & \RR(64) & \CC(64) & \HH(64)\\
  5 & \HH(2)\oplus\HH(2) & \HH(4) & \CC(8) & \RR(16) & \RR(16)\oplus\RR(16) & \RR(32) & \CC(32) & \HH(32) & \HH(32)\oplus\HH(32)\\
  4 & \HH(2)             & \CC(4) & \RR(8) & \RR(8)\oplus\RR(8) & \RR(16) & \CC(16) & \HH(16) & \HH(16)\oplus\HH(16)  & \HH(32)\\
  3 & \CC(2)             & \RR(4) & \RR(4)\oplus\RR(4) & \RR(8) & \CC(8) & \HH(8) & \HH(8)\oplus\HH(8) & \HH(16) & \CC(32)\\
  2 & \RR(2)             & \RR(2)\oplus\RR(2) & \RR(4) & \CC(4) & \HH(4) & \HH(4)\oplus\HH(4) & \HH(8) & \CC(16) & \RR(32)\\
  1 & \RR\oplus\RR       & \RR(2) & \CC(2) & \HH(2) & \HH(2)\oplus\HH(2) & \HH(4) & \CC(8) & \RR(16)  & \RR(16)\oplus\RR(16) \\
  0 & \RR                & \CC    & \HH & \HH\oplus\HH & \HH(2) & \CC(4) & \RR(8) & \RR(8)\oplus\RR(8) & \RR(16) \\\hline
    & 0                  & 1  & 2 & 3 & 4 & 5 & 6 & 7 & 8
\end{array}
  \end{footnotesize}
\end{equation*}

\begin{theorem}\label{thm:lie:spin:even-clifford-isomorphisms}
  \begin{enumerate}
  \item $\Cl_{n+1}^{0}\iso\Cl_{n}$ for each $n\in\NN_{0}$.
  \item $\Cl_{s,t}^{0}\iso\Cl_{s-1,t}$ for $s, t\in\NN$.
  \item $\Cl_{s,t}^{0}\iso\Cl_{t-1,s}$ for $s, t\in\NN$.
  \end{enumerate}
\end{theorem}

In particular, we have the following table describing the even
subalgebra as the following real associative algebras:

\begin{table}[h!]
\centering
\begin{tabular}{|c|c|c|} \hline
  $s-t$ mod $8$ & $\Cl_{s,t}^{0}$ & $N$ \\\hline
           1, 7 &       $\RR(N)$ & $2^{(s+t-1)/2}$\\
           3, 5 &       $\HH(N)$ & $2^{(s+t-3)/2}$\\ \hline
           2, 6 &       $\CC(N)$ & $2^{(s+t-2)/2}$\\
              4 & $\HH(N)\oplus\HH(N)$ & $2^{(s+t-4)/2}$\\
              0 & $\RR(N)\oplus\RR(N)$ & $2^{(s+t-2)/2}$\\ \hline
\end{tabular}
\caption{Even Clifford Subalgebras in $\RR^{s,t}$}
\end{table}

\section{The $\spin$ Representation}

\M
We consider the Clifford algebra $\Cl(\RR^{n})$ which has $n$ generators
$\Gamma_{1}$, \dots, $\Gamma_{n}$ which satisfy the usual relations
\begin{equation}\label{eq:spin:clifford-anticommutator}
\{\Gamma_{i}, \Gamma_{j}\} = \Gamma_{i}\Gamma_{j} + \Gamma_{j}\Gamma_{i} = -2\delta_{ij} = \eta_{ij}.
\end{equation}
If we wanted, we could replace the right-hand side by some nondegenerate
$n\times n$ matrix $\eta_{ij}$, but we will not need it for our purposes.
These generators $\Gamma_{i}$ generalize the Dirac gamma matrices. Then we
can consider the Lie algebra $\spin(n)$ spanned by $\Gamma_{i}\Gamma_{j}$ for
$i\neq j$.

\begin{lemma}\label{lemma:spin:clifford-algebra:commutator-of-one-and-two}
We have the commutator of generators for the Clifford algebra satisfy
$$[\Gamma_{\alpha}, \Gamma_{\beta}\Gamma_{\gamma}]=\Gamma_{\gamma}\eta_{\beta\alpha}-\Gamma_{\beta}\eta_{\alpha\gamma}.$$
\end{lemma}

\begin{proof}
By direct calculation,
\begin{calculation}
[\Gamma_{\alpha}, \Gamma_{\beta}\Gamma_{\gamma}]
\step{definition of commutator}
\Gamma_{\alpha}\Gamma_{\beta}\Gamma_{\gamma} - \Gamma_{\beta}\Gamma_{\gamma}\Gamma_{\alpha}
\step{associativity}
\Gamma_{\alpha}\Gamma_{\beta}\Gamma_{\gamma} - \Gamma_{\beta}(\Gamma_{\gamma}\Gamma_{\alpha})
\step{using anticommutation relations from Eq~\eqref{eq:spin:clifford-anticommutator}}
\Gamma_{\alpha}\Gamma_{\beta}\Gamma_{\gamma} - \Gamma_{\beta}(\eta_{\alpha\gamma}-\Gamma_{\alpha}\Gamma_{\gamma})
\step{distributivity}
\Gamma_{\alpha}\Gamma_{\beta}\Gamma_{\gamma} - \Gamma_{\beta}\eta_{\alpha\gamma}+\Gamma_{\beta}\Gamma_{\alpha}\Gamma_{\gamma}
\step{associativity}
\Gamma_{\alpha}\Gamma_{\beta}\Gamma_{\gamma} - \Gamma_{\beta}\eta_{\alpha\gamma}+(\Gamma_{\beta}\Gamma_{\alpha})\Gamma_{\gamma}
\step{using anticommutation relations from Eq~\eqref{eq:spin:clifford-anticommutator}}
\Gamma_{\alpha}\Gamma_{\beta}\Gamma_{\gamma} - \Gamma_{\beta}\eta_{\alpha\gamma}+(\eta_{\beta\alpha}-\Gamma_{\alpha}\Gamma_{\beta})\Gamma_{\gamma}
\step{distributivity}
\Gamma_{\alpha}\Gamma_{\beta}\Gamma_{\gamma} - \Gamma_{\beta}\eta_{\alpha\gamma}+\eta_{\beta\alpha}\Gamma_{\gamma}-\Gamma_{\alpha}\Gamma_{\beta}\Gamma_{\gamma}
\step{subtraction}
- \Gamma_{\beta}\eta_{\alpha\gamma}+\eta_{\beta\alpha}\Gamma_{\gamma}.\qedhere
\end{calculation}
\end{proof}

\begin{lemma}
The commutator obeys the product rule $[A,BC]=[A,B]C+B[A,C]$
and $[AB,C]=A[B,C] + [A,C]B$.
\end{lemma}

\begin{proposition}
We have the commutator of generators for the Clifford algebra satisfy
$$[\Gamma_{\alpha}\Gamma_{\beta}, \Gamma_{\gamma}\Gamma_{\delta}]=\bigl(\eta_{\beta\delta}\eta_{\gamma\alpha}-\eta_{\beta\gamma}\eta_{\alpha\delta}\bigr)
+\bigl(\Gamma_{\alpha}\Gamma_{\delta}\eta_{\gamma\beta}-\Gamma_{\alpha}\Gamma_{\gamma}\eta_{\beta\delta}
-\Gamma_{\beta}\Gamma_{\delta}\eta_{\gamma\alpha}
+\Gamma_{\beta}\Gamma_{\gamma}\eta_{\alpha\delta}\bigr).$$
\end{proposition}

\begin{proof}
By direct calculation,
\begin{calculation}
[\Gamma_{\alpha}\Gamma_{\beta}, \Gamma_{\gamma}\Gamma_{\delta}]
\step{using the product rule for commutator}
\Gamma_{\alpha}[\Gamma_{\beta}, \Gamma_{\gamma}\Gamma_{\delta}] + [\Gamma_{\alpha}, \Gamma_{\gamma}\Gamma_{\delta}]\Gamma_{\beta}
\step{using Lemma~\ref{lemma:spin:clifford-algebra:commutator-of-one-and-two}}
\Gamma_{\alpha}[\Gamma_{\beta}, \Gamma_{\gamma}\Gamma_{\delta}]
+ \bigl(\Gamma_{\delta}\eta_{\gamma\alpha}-\Gamma_{\gamma}\eta_{\alpha\delta}\bigr)\Gamma_{\beta}
\step{using Lemma~\ref{lemma:spin:clifford-algebra:commutator-of-one-and-two}}
\Gamma_{\alpha}\bigl(\Gamma_{\delta}\eta_{\gamma\beta}-\Gamma_{\gamma}\eta_{\beta\delta}\bigr)
+ \bigl(\Gamma_{\delta}\eta_{\gamma\alpha}-\Gamma_{\gamma}\eta_{\alpha\delta}\bigr)\Gamma_{\beta}
\step{distributivity}
\Gamma_{\alpha}\Gamma_{\delta}\eta_{\gamma\beta}-\Gamma_{\alpha}\Gamma_{\gamma}\eta_{\beta\delta}
+ \Gamma_{\delta}\Gamma_{\beta}\eta_{\gamma\alpha}-\Gamma_{\gamma}\Gamma_{\beta}\eta_{\alpha\delta}
\step{associativity}
\Gamma_{\alpha}\Gamma_{\delta}\eta_{\gamma\beta}-\Gamma_{\alpha}\Gamma_{\gamma}\eta_{\beta\delta}
+ (\Gamma_{\delta}\Gamma_{\beta})\eta_{\gamma\alpha}-(\Gamma_{\gamma}\Gamma_{\beta})\eta_{\alpha\delta}
\step{using anticommutation relations from Eq~\eqref{eq:spin:clifford-anticommutator}}
\Gamma_{\alpha}\Gamma_{\delta}\eta_{\gamma\beta}-\Gamma_{\alpha}\Gamma_{\gamma}\eta_{\beta\delta}
+ (\eta_{\beta\delta}-\Gamma_{\beta}\Gamma_{\delta})\eta_{\gamma\alpha}-(\Gamma_{\gamma}\Gamma_{\beta})\eta_{\alpha\delta}
\step{using anticommutation relations from Eq~\eqref{eq:spin:clifford-anticommutator}}
\Gamma_{\alpha}\Gamma_{\delta}\eta_{\gamma\beta}-\Gamma_{\alpha}\Gamma_{\gamma}\eta_{\beta\delta}
+ (\eta_{\beta\delta}-\Gamma_{\beta}\Gamma_{\delta})\eta_{\gamma\alpha}
-(\eta_{\beta\gamma}-\Gamma_{\beta}\Gamma_{\gamma})\eta_{\alpha\delta}
\step{distributivity}
\Gamma_{\alpha}\Gamma_{\delta}\eta_{\gamma\beta}-\Gamma_{\alpha}\Gamma_{\gamma}\eta_{\beta\delta}
+ \eta_{\beta\delta}\eta_{\gamma\alpha}-\Gamma_{\beta}\Gamma_{\delta}\eta_{\gamma\alpha}
-\eta_{\beta\gamma}\eta_{\alpha\delta}+\Gamma_{\beta}\Gamma_{\gamma}\eta_{\alpha\delta}
\step{collecting terms}
\bigl(\eta_{\beta\delta}\eta_{\gamma\alpha}-\eta_{\beta\gamma}\eta_{\alpha\delta}\bigr)
+\bigl(\Gamma_{\alpha}\Gamma_{\delta}\eta_{\gamma\beta}-\Gamma_{\alpha}\Gamma_{\gamma}\eta_{\beta\delta}
-\Gamma_{\beta}\Gamma_{\delta}\eta_{\gamma\alpha}
+\Gamma_{\beta}\Gamma_{\gamma}\eta_{\alpha\delta}\bigr).\qedhere
\end{calculation}
\end{proof}

\begin{remark}
We can interpret this as $\Gamma_{i}\Gamma_{j}$ satisfy the same commutation
relations as the generators $L_{ij}$ for the orthogonal Lie algebra.
\end{remark}

\M
What we tend to do is consider
\begin{equation}
\Gamma_{\alpha\beta} = \frac{1}{2}(\Gamma_{\alpha}\Gamma_{\beta} - \Gamma_{\beta}\Gamma_{\alpha}).
\end{equation}
Observe
\begin{equation}
\Gamma_{\alpha}\Gamma_{\beta} = \frac{1}{2}(\Gamma_{\alpha}\Gamma_{\beta} -
\Gamma_{\beta}\Gamma_{\alpha}) + \frac{1}{2}(\Gamma_{\alpha}\Gamma_{\beta} + \Gamma_{\beta}\Gamma_{\alpha})
= \Gamma_{\alpha\beta} - \eta_{\alpha\beta}\mathbf{1}.
\end{equation}
%% It is not hard to find
%% \begin{equation}
%% \begin{split}
%% \Gamma_{\alpha\beta} &= \frac{1}{2}(\Gamma_{\alpha}\Gamma_{\beta} + \Gamma_{\alpha}\Gamma_{\beta} - \eta_{\alpha\beta})\\
%% &= \Gamma_{\alpha}\Gamma_{\beta} - \frac{1}{2}\eta_{\alpha\beta}.
%% \end{split}
%% \end{equation}
Why do this?

Well, consider the representation of $\so(n)$ using 2-forms, writing
\begin{equation}
\rho(a) = a_{\mu\nu}\D x^{\mu}\wedge\D x^{\nu},
\end{equation}
using summation convention (repeated indices, with one downstairs
[subscript] and another upstairs [superscript], are summed over). Then
$a_{\mu\nu}$ is antisymmetric, which means it belongs to $\so(n)$. Thus
$\rho\colon\so(n)\to\Exterior^{2}\CC^{n}$ is a faithful representation.

But we're not done: identify $\D x^{\mu}\wedge\D x^{\nu}$ with
$\Gamma_{\mu\nu}$, and we obtain the $\spin(n)$ representation for $\so(n)$.
To be clear, we have a mapping
\begin{equation}
a_{\mu\nu}\D x^{\mu}\wedge\D x^{\nu}\mapsto\sum_{\mu,\nu}a_{\mu\nu}\Gamma_{\mu\nu}.
\end{equation}
This produces the spin representation
\begin{equation}
\rho(a) = \sum_{\mu,\nu}a_{\mu\nu}\Gamma_{\mu\nu}.
\end{equation}
It turns out to be quite useful.

\section{Spin Group}

\begin{definition}
Let $V$ be a finite-dimensional vector space over $\RR$ or $\CC$.
The \define{Pin Group} $\Pin(V)$ consists of elements of $\Cl(V)$
with unit norm, multiplication induced from $\Cl(V)$.
\end{definition}

\M
The inverse of $c\, \Gamma_{1}(\cdots)\Gamma_{k}\in\Pin(V)$ (for some nonzero $c\in\FF$)
is $(-1)^{k}c^{-1}\Gamma_{k}(\cdots)\Gamma_{2}\Gamma_{1}$ when we have $\Gamma_{i}^{2}=-1$
for each $i=1$, \dots, $n$.

Why would we have a coefficient $c\neq\pm1$ at all? Well, if we
consider, e.g.,
\begin{equation}
\begin{pmatrix}
v_{1}\\ v_{2}
\end{pmatrix} = \begin{pmatrix}\cos(\theta) & -\sin(\theta)\\
\sin(\theta) & \cos(\theta)
\end{pmatrix}
\begin{pmatrix}
\Gamma_{1}\\ \Gamma_{2}
\end{pmatrix},
\end{equation}
then $v_{1}$ is length 1 and inverible. But it is the sum of two
generators with nontrivial coefficients (for $\theta\notin\pi\QQ$, for
example). This is an example of a random element which is a pinor
(element of $\Pin(V)$).

\begin{remark}
Adams~\cite{adams1996:ex} gives an equivalent construction of the Pin
group, but is rather confusing: he uses $\beta$ to reverse the order of
multiplication, for example, instead of using the transpose of the
representation. Adams's $\alpha$ simply maps $\Gamma_{i}\mapsto-\Gamma_{i}$ on the
canonical generators for $\Cl(V)$. His
$\gamma=\alpha\circ\beta=\beta\circ\alpha$ called \define{Conjugation}
by DJH Garlin~\cite[pg.94]{Garling:2011zz}. Observe that $\alpha$ acts
trivially on even guys, and only on odd guys will something nontrivial
happen. 
\end{remark}

\begin{remark}
Adams defines $\Pin(V)$ as those $x\in\Cl(V)$ such that:
\begin{enumerate}
\item $x(\gamma(x))=(\gamma(x))x=1$, i.e., its conjugate is its inverse; and
\item the map $\pi_{x}\colon V\subset\Cl(V)\to\Cl(V)$ defined by $\pi_{x}(v)=xv(\beta(x))$
is injective.
\end{enumerate}
Then he proves $\pi$ is a 2-to-1 morphism of $\Pin(V)\onto\O(V)$. The
preimage of $\SO(V)$ is then defined as $\Spin(V)$.

This coincides with what Garlin~\cite[ch.8]{Garling:2011zz} does, since
$x^{-1}=\gamma(x)$ and so Garlin's approach [rewritten to use Adams's
  notation] using $v\mapsto x v \alpha(x^{-1})$, and direct calculation
shows
\begin{calculation}
  x v \alpha(x^{-1})
\step{since $x^{-1} = \gamma(x)$ by Adams's first criterion}
  x v ((\alpha\circ\gamma)(x))
\step{since $\gamma=\alpha\circ\beta$}
x v ((\alpha\circ\alpha\circ\beta)(x))
\step{since $\alpha\circ\alpha=\id$}
xv(\beta(x)).
\end{calculation}
But Garlin is slightly more general.
\end{remark}

\N{Adjoint Action of $\Pin(V)$ on $V$}
We can construct a representation of $\Pin(V)$ using $V$ by means of
\begin{equation}
\begin{split}
  \Ad\colon\Pin(V)&\to\aut(V)\\
  x&\mapsto\Ad_{x}
\end{split}
\end{equation}
where $\pi_{x}$ is defined as acting on $v\in V$ by using the inclusion
$i\colon V\into\Cl(V)$ as
\begin{equation}
\Ad_{x}(v) = x\,i(v)\,x^{-1}.
\end{equation}
If we observe that $x^{-1}=-x/Q(x)$ using the quadratic form $Q(-)$ [for
$x\in\Pin(V)$, we have $Q(x)=\pm1$],
then
\begin{calculation}
\Ad_{x}(v)
\step{using the definition of the adjoint representation}
xvx^{-1}
\step{since $x^{-1}=-x/Q(x)$}
\frac{-1}{Q(x)}xvx
\step{associativity}
\frac{-1}{Q(x)}(xv)x
\step{using anticommutator relations and induced bilinear form $B(-,-)$}
\frac{-1}{Q(x)}(-vx - 2B(v,x)\mathbf{1})x
\step{distributivity}
\frac{-1}{Q(x)}(vQ(x) - 2B(v,x)x)
\step{scalar distributivity}
-v + 2\frac{B(v,x)}{Q(x)}x
\end{calculation}
The reader may confirm this is, in fact, the negation of the reflection
of $v$ about $x$. We will write this as
\begin{equation}
\Ad_{x}(v) = -R_{x}(v)
\end{equation}
for $R_{x}$ be the linear operator encoding reflection about the
hyperplane perpendicular to $x$.

\N{Twisted Adjoint Representation}
We can construct the Twisted Adjoint action $\widetilde{\Ad}_{x}(v)=(-x)vx^{-1}$,
or more generally $\widetilde{\Ad}_{x}(v)=\alpha(x)\,v\,x^{-1}$ using
Adams's notation. This coincides with $\pi_{x}(v)$. When $a=u_{1}\dots u_{k}\in\Pin(V)$,
we have
\begin{equation}
\widetilde{\Ad}_{a}=R_{u_{1}}\circ(\cdots)\circ R_{u_{k}}.
\end{equation}
Reflections are orthogonal transformations, therefore $\widetilde{\Ad}$
defines a group morphism
\begin{equation}
\widetilde{\Ad}\colon\Pin(V)\to\O(V).
\end{equation}
We also see that $\widetilde{\Ad}$ is surjective by the Cartan--Dieudonn\'e
theorem.

\begin{theorem}[Cartan--Dieudonn\'e]
Every $g\in\O(V)$ is the product of a finite number of reflections
$R_{u_{1}}\circ(\cdots)\circ R_{u_{k}}$ along non-null lines (i.e.,
$Q(u_{i})\neq0$ for each $i=1,\dots,k$) and moreover $k\leq\dim(V)$.
\end{theorem}

\begin{theorem}
  The following is an exact sequence:
  \begin{equation}
1\to\{\pm1\}\into\Pin(V)\xonto{\widetilde{\Ad}}\O(V)\to1.
  \end{equation}
\end{theorem}

\N{Spin Group}
When we restrict attention to $\Pin(V)\cap\Cl^{0}(V)$ elements with an
even number of factors, then we obtain the spin group, denoted $\Spin(V)$.
This is a double cover of $\SO(V)$, everything works out nicely.

\N{Lie Algebra of $\Pin(V)$}
For $V\iso\FF^{n}$ (here $\FF$ being $\RR$ or $\CC$), $\Pin(V)$ has an
associated Lie algebra spanned by $\Gamma_{i}\Gamma_{j}\in\Cl(V)$ for $i<j$, and
whose Lie bracket is the commutator.

\M We can then consider the subgroup of $\Pin(V)$ consisting of an even
number of generators multiplied together. This is precisely the Spinor
group $\Spin(V)$.

Observe, this works out nicely, since the exponentiation of the Lie
algebra for $\Pin(V)$ should give us the component connected to the
identity, which is precisely $\Spin(V)$.

\M
Observe that, for $i < j$, we have:
\begin{calculation}
  (\Gamma_{i}\Gamma_{j})^{2}
\step{unfold}
  \Gamma_{i}\Gamma_{j}\Gamma_{i}\Gamma_{j}
\step{since $i\neq j$ we have $\Gamma_{j}\Gamma_{i}=-\Gamma_{i}\Gamma_{j}=(-1)\Gamma_{i}\Gamma_{j}$}
  \Gamma_{i}(-1)\Gamma_{i}\Gamma_{j}\Gamma_{j}
\step{using $\Gamma_{j}^{2}=-1$ and $\Gamma_{i}(-1)\Gamma_{i}=+1$}
  (+1)(-1) = -1.
\end{calculation}
Therefore for any $t\in\RR$, we have:
\begin{equation}
\E^{\Gamma_{i}\Gamma_{j}t} = \cos(t) + \Gamma_{i}\Gamma_{j}\sin(t),
\end{equation}
using Euler's formula. But when acting on $V$, we see that
$x=\exp(\Gamma_{i}\Gamma_{j}t)$ is represented by the matrix
\begin{equation}%\renewcommand{\arraystretch}{1.3}
\begin{split}
  \phantom{\pi_{x}} &\\
  \pi_{x}&=\quad \left[\begin{array}{ccccccccccc}
% 1 & 2    & 3 &  4       & 5 &  6     & 7 & 8         & 9 & 10 & 11
1 &        &   & \UP{i}{} &   &        &   & \UP{j}{}  &   &        & \\
  & \ddots &   &          &   &        &   &           &   &        & \\
  &        & 1 &          &   &        &   &           &   &        & \\
\LF{i}  &        &   & \cos(2t) &   &        &   & -\sin(2t) &   &        & \\
  &        &   &          & 1 &        &   &           &   &        & \\
  &        &   &          &   & \ddots &   &           &   &        & \\
  &        &   &          &   &        & 1 &           &   &        & \\
\LF{j}  &        &   & \sin(2t) &   &        &   & \cos(2t)  &   &        & \\
  &        &   &          &   &        &   &           & 1 &        & \\
  &        &   &          &   &        &   &           &   & \ddots & \\
  &        &   &          &   &        &   &           &   &        & 1
  \end{array}\right]
\end{split}
\end{equation}
Believe me? Well, even I have my doubts, so let us see how it acts on a
unit vector $\Gamma_{k}\in V$:
\begin{calculation}
(\cos(t) + \Gamma_{i}\Gamma_{j}\sin(t))\Gamma_{k}(\cos(t) - \Gamma_{i}\Gamma_{j}\sin(t))
\step{distributivity}
(\Gamma_{k}\cos(t) + \Gamma_{i}\Gamma_{j}\Gamma_{k}\sin(t))(\cos(t) - \Gamma_{i}\Gamma_{j}\sin(t))
\step{distributivity}
\Gamma_{k}\cos^{2}(t) + \Gamma_{i}\Gamma_{j}\Gamma_{k}\sin(t)\cos(t)
-\Gamma_{k}\Gamma_{i}\Gamma_{j}\cos(t)\sin(t) - \Gamma_{i}\Gamma_{j}\Gamma_{k}\Gamma_{i}\Gamma_{j}\sin^{2}(t)
\end{calculation}
Now, if $\Gamma_{k}=\Gamma_{i}$ or if $\Gamma_{k}=\Gamma_{j}$, then we have
\begin{calculation}
(\cos(t) + \Gamma_{i}\Gamma_{j}\sin(t))\Gamma_{k}(\cos(t) - \Gamma_{i}\Gamma_{j}\sin(t))
\step{continuing from last step}
\Gamma_{k}\cos^{2}(t) + \Gamma_{i}\Gamma_{j}\Gamma_{k}2\sin(t)\cos(t) - \Gamma_{i}\Gamma_{j}\Gamma_{k}\sin^{2}(t)
\step{using trig identities}
\Gamma_{k}\cos(2t) + \Gamma_{i}\Gamma_{j}\Gamma_{k}\sin(2t)
\step{collecting terms}
\bigl(\cos(2t) + \Gamma_{i}\Gamma_{j}\sin(2t)\bigr)\Gamma_{k}.
\end{calculation}
Hence we find if $\Gamma_{k}=\Gamma_{i}$ or if $\Gamma_{k}=\Gamma_{j}$,
\begin{subequations}
\begin{equation}
(\cos(t) + \Gamma_{i}\Gamma_{j}\sin(t))\Gamma_{k}(\cos(t) - \Gamma_{i}\Gamma_{j}\sin(t))
= \bigl(\cos(2t) + \Gamma_{i}\Gamma_{j}\sin(2t)\bigr)\Gamma_{k}.
\end{equation}
If $\Gamma_{k}\neq \Gamma_{i}$ and $\Gamma_{k}\neq \Gamma_{j}$, then we find by a
straightforward calculation (and using the $\cos^{2}(t)+\sin^{2}(t)=1$
trig identity):
\begin{equation}
(\cos(t) + \Gamma_{i}\Gamma_{j}\sin(t))\Gamma_{k}(\cos(t) - \Gamma_{i}\Gamma_{j}\sin(t))=\Gamma_{k}.
\end{equation}
\end{subequations}
Combining these two results, and rewriting it using a matrix, we obtain
precisely the desired matrix.

Coincidentally, this is why we have a 2-to-1 covering of $\SO(n)$
\textbf{as groups}, because we have $\cos(2t)$ and $\sin(2t)$ as the
entries of our matrix. But \textbf{as Lie algebras}, we have an
isomorphism $\spin(n)\iso\so(n)$.

\N{Maximal Torus of $\Spin(V)$}
The maximal torus $T$ for $\Spin(V)$, $V\iso\RR^{m}$, are elements of
the form
\begin{equation}
y = \prod^{n}_{r=1}[\cos(x_{r}/2) + \Gamma_{2r-1}\Gamma_{2r}\sin(x_{r}/2)]
\end{equation}
where $x_{r}\in\RR$ are parameters (``Euler angles''). These correspond
to matrices
\begin{equation}
  \pi_{y} = 
\begin{pmatrix}
\cos(x_{1}) & -\sin(x_{1}) &            &              & & & & \\
\sin(x_{1}) &  \cos(x_{1}) &            &              & & & & \\
           &              & \cos(x_{2}) & -\sin(x_{2}) & & & & \\
           &              & \sin(x_{2}) &  \cos(x_{2}) & & & & \\
 & & & & \ddots &            &              & \\
 & & & &        & \cos(x_{n}) & -\sin(x_{n}) & \\
 & & & &        & \sin(x_{n}) &  \cos(x_{n}) &  \\
 & & & &        &             &             & 1
\end{pmatrix}
\end{equation}
If $m=2n$, we just delete the last row and column (i.e., the row and
column with $1$ in the lower right corner).

\begin{remark}
Adams~\cite{adams1996:ex} concludes chapter 3 with a remark proving, for
$\RR^{n}$ guys, every element of $\Spin(\RR^{n})$ is conjugate to an
element in its torus. This seems intuitive, but I may be missing
something. 
\end{remark}

\section{Representations}

\M
Since $\Spin(V)\subset\Cl^{0}(V)$ is a subset of even Clifford
subalgebra, then any $\Cl^{0}(V)$-module is a representation of
$\Spin(V)$. In fact, irreducible representations of $\Cl_{r,s}^{0}$ 
induces irreducible representations for $\Spin_{r,s}$.

\begin{proposition}[{Adams~\cite[Prop.4.1]{adams1996:ex}}]
The algebras $\Cl(V)$ and $\Cl^{0}(V)$ are semi-simple, and hence all
their representations are completely reducible.
\end{proposition}

Adams proves the representations are completely reducible.

\begin{proof}[Proof sketch]
Adams begins by considering the standard basis $\Gamma_{1}$, \dots, $\Gamma_{m}$
for $V\iso\FF^{m}$, then constructs the subgroup
\begin{equation}
E = \{\pm\prod^{m}_{j=1}e^{i_{j}}_{j} \mid i_{j}=0\mbox{ or }1\},
\end{equation}
considered as a subgroup of order $2^{m+1}$ of $\Cl(V)$, which should
correspond to matrices $\diag(\pm1,\dots,\pm1)$ of $\O(m)$.

Then in $\Cl^{0}(V)$ we have the subgroup $E_{0}$ of $2^{m}$ elements
with $\sum_{j}i_{j}$ being even, specifically we have $E_{0}\subset E\subset\Pin(V)$.

We construct an operator $\nu$ corresponding to $-1\in\Cl^{0}(V)$
considered as an element of $E_{0}$ (or $E$ or $\Pin(V)$).

Adams observes $\nu$ is the generator of the $\pi$ representation's kernel.

Now, there are two claims made:
\begin{enumerate}
\item A module over $\Cl(V)$ gives a representation of $E$ where $\nu$
  acts as $-1$.
\item A representation of $E$ where $\nu$ acts as $-1$ gives a module
  over $\FF[E]/(\nu+1)\iso\Cl(V)$ the quotient of the group algebra
  $\FF[E]$ modulo $\nu+1\sim0$ (i.e., $\nu\sim-1$).
\end{enumerate}
In particular, these claims tell us the representation theory of
$\Cl(V)$ may be deduced from the representations of the finite group $E$.
This means all representations are completely reducible. A similar
argument holds for $\Cl^{0}(V)$ using $E_{0}$ instead of $E$.
\end{proof}

\begin{remark}
Adams will make heavy usage of these groups $E$, $E_{0}$, and the $\nu$ operator.
\end{remark}

\begin{remark}
Crucially, the insight that $\FF[E]/(\nu+1)\iso\Cl(V)$ allows us to find
representations for $E$, then induce a representation for $\Cl(V)$. This
is used in several proofs. The mnemonic appears to be that ``$\nu$ is
\emph{\textbf{n}egative one}.''
\end{remark}

\begin{lemma}\label{lemma:spin:center-of-e0}
The center of $E_{0}$ is $\{\pm1\}$ if $m=2n+1$ is odd dimensional, and
$\{\pm1,\pm\prod^{m}_{j=1}e_{j}\}$ if $m=2n$ is even dimensional.
\end{lemma}

\begin{proof}
Recall, the group center consists of $z\in Z(G)$ such that for all $g\in G$,
conjugation $z^{g}=g^{-1}zg=z$. It suffices to check conjugation by
every generator of the group $E_{0}$, which are precisely $\Gamma_{r}\Gamma_{s}$. Now if we conjugate
$g=\prod^{m}_{j=1}\Gamma_{j}^{i_{j}}$ by $\Gamma_{r}\Gamma_{s}$ where $i_{r}=1$ and $i_{s}=0$,
we change its sign. So if $g\in Z(E_{0})$, then $g=\pm1$ or $\pm \Gamma_{1}\Gamma_{2}\dots \Gamma_{m}$.
The latter is in the center only for $m$ even.
\end{proof}

\N{Fact}\label{fact:spin:complex-irrep-abelian-group-one-dim} Every $\CC$ irreducible representation for Abelian groups are
one-dimensional.

\N{Induced Representations}\label{chunk:spin:induced-rep}
Just to review induced representations, let $G$ be a finite group, let
$H\leq G$ be any subgroup, let $[G : H]=n$.
Suppose we have a representation $\pi\colon H\to\aut(V)$.
Suppose we have a complete set of coset representatives $g_{1}$, \dots,
$g_{n}$ for $H$ in $G$.

First, we should observe, for any $g\in G$ and any coset representative
$g_{i}$, we have $gg_{i}\in G$ (by closure of
multiplication). Therefore, $gg_{i}$ lives in a left coset of $H$, say
$g_{j}H$. More explicitly, $gg_{i}=g_{j}h_{i}$ for some $h_{i}\in H$.
This can be done for each coset representative, giving us a function of
indices $j(i)$, and we write:
\begin{equation}
g\cdot g_{i} = g_{j(i)}h_{i}.
\end{equation}
So far, so good.

Now we construct our representation space,
\begin{equation}
\widetilde{V} = \bigoplus^{n}_{i=1}g_{i}V.
\end{equation}
Then we define a $G$-action on $\widetilde{V}$ by using the insights
from the previous paragraph:
\begin{equation}
g\cdot\sum^{n}_{i=1}g_{i}\vec{v}_{i}=\sum^{n}_{i=1}g_{j(i)}\bigl(\pi(h_{i})\vec{v}_{i}\bigr).
\end{equation}
The resulting representation for $G$ is denoted $\operatorname{Ind}_H^G\pi$
or $\operatorname{Ind}_H^GV$ depending on author.

We can generalize this construction, using the insight that --- as a
representation space for $H$, $V$ is a $\FF[H]$-module --- the induced
representation is precisely the tensor product
\begin{equation}
\operatorname{Ind}_H^GV = \FF[G]\otimes_{\FF[H]}V.
\end{equation}
No conditions of finiteness are necessary in this generalized version.

\begin{proposition}[{Adams~\cite[Prop.4.2]{adams1996:ex}}]
\begin{enumerate}
\item If $m=\dim(V)=2n+1$ is odd, then $\Cl^{0}(V)$ has one irreducible
  representation $\Delta$ of degree $2^{n}$ affording a representation
  $\Delta$ of $\Spin(2n+1)$ with weights $\frac{1}{2}(\pm x_{1}\pm x_{2}\pm\cdots\pm x_{n})$
  with each sign being chosen independently, so there are $2^{n}$ of
  these weights.
\item If $m=\dim(V)=2n$ is even, then $\Cl^{0}(V)$ has two irreducible
  representations $\Delta^{+}$, $\Delta^{-}$ of degree $2^{n-1}$
  affording representations $\Delta^{+}$, $\Delta^{-}$ of $\Spin(2n)$
  with weights $\frac{1}{2}(\pm x_{1}\pm x_{2}\pm\cdots\pm x_{n})$
  with an even number of $-$ minus signs for $\Delta^{+}$ and an odd number of
  $-$ minus signs for $\Delta^{-}$. There are $2^{n-1}$ weights.
\item If $V$ is a complex vector space, then these constructions give
  complex-analytic representations for $\Spin_{\CC}(m)$.
\end{enumerate}
\end{proposition}

Physicists call $\Delta^{+}$ and $\Delta^{-}$ the \define{Half-Spin Representations},
some (e.g., Deligne) call them the \define{Semi-Spin Representations}.

\begin{remark}
It appears, thanks to Theorem~\ref{thm:lie:spin:even-clifford-isomorphisms},
an irreducible representation for $\Cl_{s,t}^{0}$ could be obtained from
$\Cl_{s-1,t}$. 
\end{remark}

\begin{remark}
I haven't seen a good explanation of Adams's proof, but it appears to be
fleshed out a bit more fully here: \url{https://www2.math.upenn.edu/~brweber/Courses/2013/Math651/Notes/L18_CliffSpin.pdf}
\end{remark}

\begin{remark}
This proposition is used later when constructing a Lie algebra of type
$\mathtt{E}_{8}$. 
\end{remark}

Adams's proof is rather convoluted, but amounts to constructing an
induced representation for the finite group $E_{0}$, then observes it
gives an irreducible representation for $\Spin(V)$.

\begin{proof}[Proof sketch]
By Schur's lemma, $\nu$ acts on any irreducible representation as either
$1$ or $-1$. The representations in which it acts as $1$ are
representations of $E_{0}/\langle\nu\rangle$ which is an Abelian group
of order $2^{m-1}$. Therefore there are $2^{m-1}$ one-dimensional
representations of $E_{0}$ in which $\nu$ acts as $1$.

Since the kernel of $E_{0}\to E_{0}/\langle\nu\rangle$ has exactly two
elements, the conjugacy classes in $E_{0}$ are either one element (if
the element is central) or two elements $\pm g$.

Recall: isomorphism classes of irreducible representations (over $\CC$)
of a finite group are in one-to-one correspondence [i.e., bijective with]
the conjugacy classes.

We see that $E_{0}$ has one (resp., two) more irreducible class(es) of
representation(s) than $E_{0}/\langle\nu\rangle$ if $m=2n+1$ (resp., $m=2n$).

Let $F\subset E_{0}$ be the subgroup generated by $\Gamma_{1}\Gamma_{2}$,
$e_{3}e_{4}$, \dots, $e_{2r-1}e_{2r}$, \dots, $e_{2n-1}e_{2n}$. This is
an Abelian group of order $2^{n+1}$, so
\begin{equation}
[E_{0} : F] = \begin{cases}2^{n} &\mbox{if }m=2n+1\\
2^{n-1} &\mbox{if }m=2n.
\end{cases}
\end{equation}
Is this really Abelian? Yes, we can calculate (assuming $i\neq j\neq k\neq\ell$):
\begin{calculation}
  e_{i}e_{j}e_{k}e_{\ell}
\step{anticommutator}
  -e_{i}e_{k}e_{j}e_{\ell}
\step{anticommutator}
  e_{i}e_{k}e_{\ell}e_{j}
\step{anticommutator}
  -e_{k}e_{i}e_{\ell}e_{j}
\step{anticommutator}
  e_{k}e_{\ell}e_{i}e_{j}.
\end{calculation}
Now we will begin to construct the induced representations.

We know (\S\ref{fact:spin:complex-irrep-abelian-group-one-dim})
every complex irreducible representation for $F$ will be one-dimensional.
Choose a complex one-dimensional representation $W$ of $F$ in which
$\nu$ acts as $-1$, and $e_{2r-1}e_{2r}$ acts as $\I\varepsilon_{r}$,
where $\varepsilon_{r}=\pm1$ and $\I=\sqrt{-1}$.

Form the induced representation (\S\ref{chunk:spin:induced-rep})
\begin{equation}
\operatorname{Ind}^{E_{0}}_{F}W = \CC[E_{0}]\otimes_{\CC[F]}W,
\end{equation}
where $\CC[E_{0}]$ is the complex group algebra for $E_{0}$.

Since $\deg(\operatorname{Ind}^{E_{0}}_{F}W)=[E_{0}:F]\deg(W)$ and
$\deg(W)=1$, we find the induced representation for $E_{0}$ is degree
$2^{n}$ for $m=2n+1$ and degree $2^{n-1}$ for $m=2n$. Observe, since
$\nu$ acts as $-1$ on $W$, that in the induced representation $\nu$
continues to act as $-1$.

Adams argues without proof that the basis for the representation is
given by
\begin{subequations}
\begin{align}
\prod^{2n+1}_{\substack{i=1\\i~\text{odd}}}e_{i}^{j_{i}}\quad\mbox{with}\quad\sum^{2n+1}_{1}j_{i}~\mbox{even},\quad\mbox{if }
m=2n+1,\\
\intertext{and}
\prod^{2n-1}_{\substack{i=1\\i~\text{odd}}}e_{i}^{j_{i}}\quad\mbox{with}\quad\sum^{2n-1}_{1}j_{i}~\mbox{even},\quad\mbox{if }
m=2n.
\end{align}
\end{subequations}
\textbf{I do not immediately see how to obtain this result.} (I also
think that each $j_{i}\in\{0,1\}$ but Adams does not explicitly state this.)
My intuition is that it is [somehow] due to $e_{2r-1}e_{2r}$ being
represented as $\I\varepsilon_{r}$.

Now, there are two happy endings depending on $m$. The story is
basically the same: we will obtain an irreducible representation (or
two) for $E_{0}$ and this induces an irreducible representation for
$\Spin(V)$. And they live happily ever after.

\textbf{Case 1: $m=2n+1$}. There are $2^{n}$ choices for $W$ (because we
can choose the sign of $\varepsilon_{r}$), so $E_{0}/F$ permutes them
transitively and, by conjugating with $e_{2r}e_{2r+1}$, we can change
the sign of $\varepsilon_{r}$ without changing anything else. Each of
these appears in $\CC[E_{0}]\otimes_{\CC[F]}W$. Therefore, we get a
representation $\Delta$ of $E_{0}$ with characters $2^{n}$ at $1$,
$-2^{n}$ at $-1$, and $0$ everywhere else. By orthogonality relations,
$\Delta$ is an irreducible representation of $E_{0}$.

\textbf{Case 2: $m=2n$}. There are $2^{n}$ choices for $W$  and under
$E_{0}/F$ falls into two orbits: those with $\prod\varepsilon_{r}=+1$
and those with $\prod\varepsilon_{r}=-1$. We define $\varepsilon:=\prod\varepsilon_{r}$.

The character for these representations is $2^{n-1}$ at $1$, $-2^{n-1}$
at $-1$, $\I^{n}\varepsilon$ at $\prod^{2n}e_{i}$, and $-\I^{n}\varepsilon$
at $-\prod^{2n}e_{i}$, and $0$ everywhere else. These are irreducible
representations by orthogonality relations. These are also inequivalent
representations denoted $\Delta^{\pm}$, which concludes the proof.

(Also, observe from Lemma~\ref{lemma:spin:center-of-e0} that
$\prod^{2n}e_{i}$ lives in the center of $E_{0}$ and must commute with
everything. This implies it is represented by a diagonal matrix, and I
think $\Delta^{\pm}$ are the eigenspaces for it.)
\end{proof}

\M
The basic idea is that, when $m=2n+1$ is odd, then
$\omega\notin\Cl^{0}(m)$. Therefore,
\begin{equation}
\Cl(m) = \Cl^{0}(m) \oplus \omega\,\Cl^{0}(m)\iso\Cl^{0}(m)\otimes_{\RR}\RR[\omega].
\end{equation}
We can induce a Pinor representation $P$ of $\Cl^{0}(m)$ from a Spinor
representation $S$ of $\Cl^{0}(m)$ by tensoring it
with $\RR[\omega]$:
\begin{equation}
P = \Cl(m)\otimes_{\Cl^{0}(m)} S.
\end{equation}
Observe
\begin{equation}
\RR[\omega]\iso\CC\quad\mbox{when}\quad m=1,5\pmod8
\end{equation}
since $\omega^{2}=-1$ in that case; and
\begin{equation}
\RR[\omega]\iso\RR\oplus\RR\quad\mbox{when}\quad m=3,7\pmod8
\end{equation}
since $\omega^{2}=1$ in those cases.

\N{``Definition''}
In even dimensions, there are two inequivalent irreducible
representations for the $\Spin(2n)$ group. Physicists call them
\define{Weyl Spinors}. The volume element $\omega$ plays the role of
picking out the irreducible representations; more precisely
$(1\pm\omega)$ projects onto the two irreducible representations.

\M
There are two inequivalent complex spinor representations when $s-t=2,6\pmod8$
namely $S(s,t)$ and $\overline{S(s,t)}$ of complex dimension
$2^{(s+t-2)/2}$. They are determined by the value of $\Gamma_{d+1}$: it is
$i$ on $S(s,t)$ and $-i$ on $\overline{S(s,t)}$.

\N{Majorana--Weyl Spinors}
When $s-t=0\pmod8$, there are two inequivalent real spinor
representations $S(s,t)_{\pm}$ of real dimension $2^{(s+t-2)/2}$ and
distinguished by the value of $\Gamma_{d+1}$: being $\pm1$ on $S(s,t)_{\pm}$.
These are known as the Majorana--Weyl Spinors.

\N{Symplectic Majorana--Weyl Spinors}
When $s-t=4\pmod8$, there are two inequivalent quaternionic spinor
representations $S(s,t)_{\pm}$ of quaternionic dimension $2^{(s+t-4)/2}$
and distinguished by the value of $\Gamma_{d+1}$ being $\pm1$ on
$S(s,t)_{\pm}$.
These are the \emph{symplectic Majorana--Weyl Spinors}.

\N{References}
See also \S3 of Li's ``Geometry of Supergravity''\footnote{\url{https://sili-math.github.io/Supersymmetry.pdf}}.

\begin{example}
\begin{enumerate}
\item $\Spin(2)\iso\U(1)$ is a circle
\item $\Spin(3)\iso\SU(2)\iso\Sp(1)$ using $S^{3}\subset\HH$
\item $\Spin(4)\iso\SU(2)\times\SU(2)$ using two copies of $S^{3}\subset\HH$
\item $\Spin(5)\iso\Sp(2)$ corresponds to unitary group of quaternionic
  $2\times2$ matrices
\end{enumerate}
\end{example}

\M
Adams then uses these representations to construct exceptional Lie
groups. For example, he constructs [the compact real] $\mathtt{G}_{2}$
as a subgroup of $\Spin(7)$ in Adams's Theorem~5.5.

Its split form may be constructed analogously, but using 7-dimensional
Minkowski space instead.
Gogberashvili and Gurchumelia~\cite{Gogberashvili:2019ojg} construct
this quite explicitly. Ekins and Cornwell~\cite{Ekins:1975yu} do the
calculations for the associated Lie algebra.

\M
Also, Adams makes heavy usage of the $\Spin(16)$ subgroup of
$\mathtt{E}_{8}$ when constructing other exceptional Lie groups.
We have a sequence of mappings $G\to\Spin(16)\to\mathtt{E}_{8}$,
then take the centralizer of the image of $G$ under these maps, and the
identity component of that centralizer turns out to be a compact form of
an exceptional Lie group (depending on the choice of $G$).
This provides compact forms for $\mathtt{E}_{6}$, $\mathtt{E}_{7}$,
$\mathtt{F}_{4}$.

The only direct construction for $\mathtt{F}_{4}$
that I can find is Figueroa-O'Farrill~\cite{Figueroa-OFarrill:2007jcv}
using $S^{8}$ and Killing spinors.
