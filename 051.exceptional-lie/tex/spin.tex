\chapter{Spin Groups and their Representations}

\section{The $\spin$ Representation}

\M
We consider the Clifford algebra $\Cl(\RR^{n})$ which has $n$ generators
$e_{1}$, \dots, $e_{n}$ which satisfy the usual relations
\begin{equation}\label{eq:spin:clifford-anticommutator}
\{e_{i}, e_{j}\} = e_{i}e_{j} + e_{j}e_{i} = -2\delta_{ij} = \eta_{ij}.
\end{equation}
If we wanted, we could replace the right-hand side by some nondegenerate
$n\times n$ matrix $\eta_{ij}$, but we will not need it for our purposes.
These generators $e_{i}$ generalize the Dirac gamma matrices. Then we
can consider the Lie algebra $\spin(n)$ spanned by $e_{i}e_{j}$ for
$i\neq j$.

\begin{lemma}\label{lemma:spin:clifford-algebra:commutator-of-one-and-two}
We have the commutator of generators for the Clifford algebra satisfy
$$[e_{\alpha}, e_{\beta}e_{\gamma}]=e_{\gamma}\eta_{\beta\alpha}-e_{\beta}\eta_{\alpha\gamma}.$$
\end{lemma}

\begin{proof}
By direct calculation,
\begin{calculation}
[e_{\alpha}, e_{\beta}e_{\gamma}]
\step{definition of commutator}
e_{\alpha}e_{\beta}e_{\gamma} - e_{\beta}e_{\gamma}e_{\alpha}
\step{associativity}
e_{\alpha}e_{\beta}e_{\gamma} - e_{\beta}(e_{\gamma}e_{\alpha})
\step{using anticommutation relations from Eq~\eqref{eq:spin:clifford-anticommutator}}
e_{\alpha}e_{\beta}e_{\gamma} - e_{\beta}(\eta_{\alpha\gamma}-e_{\alpha}e_{\gamma})
\step{distributivity}
e_{\alpha}e_{\beta}e_{\gamma} - e_{\beta}\eta_{\alpha\gamma}+e_{\beta}e_{\alpha}e_{\gamma}
\step{associativity}
e_{\alpha}e_{\beta}e_{\gamma} - e_{\beta}\eta_{\alpha\gamma}+(e_{\beta}e_{\alpha})e_{\gamma}
\step{using anticommutation relations from Eq~\eqref{eq:spin:clifford-anticommutator}}
e_{\alpha}e_{\beta}e_{\gamma} - e_{\beta}\eta_{\alpha\gamma}+(\eta_{\beta\alpha}-e_{\alpha}e_{\beta})e_{\gamma}
\step{distributivity}
e_{\alpha}e_{\beta}e_{\gamma} - e_{\beta}\eta_{\alpha\gamma}+\eta_{\beta\alpha}e_{\gamma}-e_{\alpha}e_{\beta}e_{\gamma}
\step{subtraction}
- e_{\beta}\eta_{\alpha\gamma}+\eta_{\beta\alpha}e_{\gamma}.\qedhere
\end{calculation}
\end{proof}

\begin{lemma}
The commutator obeys the product rule $[A,BC]=[A,B]C+B[A,C]$
and $[AB,C]=A[B,C] + [A,C]B$.
\end{lemma}

\begin{proposition}
We have the commutator of generators for the Clifford algebra satisfy
$$[e_{\alpha}e_{\beta}, e_{\gamma}e_{\delta}]=\bigl(\eta_{\beta\delta}\eta_{\gamma\alpha}-\eta_{\beta\gamma}\eta_{\alpha\delta}\bigr)
+\bigl(e_{\alpha}e_{\delta}\eta_{\gamma\beta}-e_{\alpha}e_{\gamma}\eta_{\beta\delta}
-e_{\beta}e_{\delta}\eta_{\gamma\alpha}
+e_{\beta}e_{\gamma}\eta_{\alpha\delta}\bigr).$$
\end{proposition}

\begin{proof}
By direct calculation,
\begin{calculation}
[e_{\alpha}e_{\beta}, e_{\gamma}e_{\delta}]
\step{using the product rule for commutator}
e_{\alpha}[e_{\beta}, e_{\gamma}e_{\delta}] + [e_{\alpha}, e_{\gamma}e_{\delta}]e_{\beta}
\step{using Lemma~\ref{lemma:spin:clifford-algebra:commutator-of-one-and-two}}
e_{\alpha}[e_{\beta}, e_{\gamma}e_{\delta}]
+ \bigl(e_{\delta}\eta_{\gamma\alpha}-e_{\gamma}\eta_{\alpha\delta}\bigr)e_{\beta}
\step{using Lemma~\ref{lemma:spin:clifford-algebra:commutator-of-one-and-two}}
e_{\alpha}\bigl(e_{\delta}\eta_{\gamma\beta}-e_{\gamma}\eta_{\beta\delta}\bigr)
+ \bigl(e_{\delta}\eta_{\gamma\alpha}-e_{\gamma}\eta_{\alpha\delta}\bigr)e_{\beta}
\step{distributivity}
e_{\alpha}e_{\delta}\eta_{\gamma\beta}-e_{\alpha}e_{\gamma}\eta_{\beta\delta}
+ e_{\delta}e_{\beta}\eta_{\gamma\alpha}-e_{\gamma}e_{\beta}\eta_{\alpha\delta}
\step{associativity}
e_{\alpha}e_{\delta}\eta_{\gamma\beta}-e_{\alpha}e_{\gamma}\eta_{\beta\delta}
+ (e_{\delta}e_{\beta})\eta_{\gamma\alpha}-(e_{\gamma}e_{\beta})\eta_{\alpha\delta}
\step{using anticommutation relations from Eq~\eqref{eq:spin:clifford-anticommutator}}
e_{\alpha}e_{\delta}\eta_{\gamma\beta}-e_{\alpha}e_{\gamma}\eta_{\beta\delta}
+ (\eta_{\beta\delta}-e_{\beta}e_{\delta})\eta_{\gamma\alpha}-(e_{\gamma}e_{\beta})\eta_{\alpha\delta}
\step{using anticommutation relations from Eq~\eqref{eq:spin:clifford-anticommutator}}
e_{\alpha}e_{\delta}\eta_{\gamma\beta}-e_{\alpha}e_{\gamma}\eta_{\beta\delta}
+ (\eta_{\beta\delta}-e_{\beta}e_{\delta})\eta_{\gamma\alpha}
-(\eta_{\beta\gamma}-e_{\beta}e_{\gamma})\eta_{\alpha\delta}
\step{distributivity}
e_{\alpha}e_{\delta}\eta_{\gamma\beta}-e_{\alpha}e_{\gamma}\eta_{\beta\delta}
+ \eta_{\beta\delta}\eta_{\gamma\alpha}-e_{\beta}e_{\delta}\eta_{\gamma\alpha}
-\eta_{\beta\gamma}\eta_{\alpha\delta}+e_{\beta}e_{\gamma}\eta_{\alpha\delta}
\step{collecting terms}
\bigl(\eta_{\beta\delta}\eta_{\gamma\alpha}-\eta_{\beta\gamma}\eta_{\alpha\delta}\bigr)
+\bigl(e_{\alpha}e_{\delta}\eta_{\gamma\beta}-e_{\alpha}e_{\gamma}\eta_{\beta\delta}
-e_{\beta}e_{\delta}\eta_{\gamma\alpha}
+e_{\beta}e_{\gamma}\eta_{\alpha\delta}\bigr).\qedhere
\end{calculation}
\end{proof}

\M
What we tend to do is consider
\begin{equation}
e_{\alpha\beta} = \frac{1}{2}(e_{\alpha}e_{\beta} - e_{\beta}e_{\alpha}).
\end{equation}
It is not hard to find
\begin{equation}
\begin{split}
e_{\alpha\beta} &= \frac{1}{2}(e_{\alpha}e_{\beta} + e_{\alpha}e_{\beta} - \eta_{\alpha\beta})\\
&= e_{\alpha}e_{\beta} - \frac{1}{2}\eta_{\alpha\beta}.
\end{split}
\end{equation}
Why do this?

Well, consider the representation of $\so(n)$ using 2-forms, writing
\begin{equation}
\rho(a) = a_{\mu\nu}\D x^{\mu}\wedge\D x^{\nu},
\end{equation}
using summation convention (repeated indices, with one downstairs
[subscript] and another upstairs [superscript], are summed over). Then
$a_{\mu\nu}$ is antisymmetric, which means it belongs to $\so(n)$. Thus
$\rho\colon\so(n)\to\Exterior^{2}\CC^{n}$ is a faithful representation.

But we're not done: identify $\D x^{\mu}\wedge\D x^{\nu}$ with
$e_{\mu\nu}$, and we obtain the $\spin(n)$ representation for $\so(n)$.
To be clear, we have a mapping
\begin{equation}
a_{\mu\nu}\D x^{\mu}\wedge\D x^{\nu}\mapsto\sum_{\mu,\nu}a_{\mu\nu}e_{\mu\nu}.
\end{equation}
This produces the spin representation
\begin{equation}
\rho(a) = \sum_{\mu,\nu}a_{\mu\nu}e_{\mu\nu}.
\end{equation}
It turns out to be quite useful.

\section{Spin Group}

\begin{definition}
Let $V$ be a finite-dimensional vector space over $\RR$ or $\CC$.
The \define{Pin Group} $\Pin(V)$ consists of elements of $\Cl(V)$
with unit norm, multiplication induced from $\Cl(V)$.
\end{definition}

\M
The inverse of $c\, e_{1}(\cdots)e_{k}\in\Pin(V)$ (for some nonzero $c\in\FF$)
is $(-1)^{k}c^{-1}e_{k}(\cdots)e_{2}e_{1}$ when we have $e_{i}^{2}=-1$
for each $i=1$, \dots, $n$.

Why would we have a coefficient $c\neq\pm1$ at all? Well, if we
consider, e.g.,
\begin{equation}
\begin{pmatrix}
v_{1}\\ v_{2}
\end{pmatrix} = \begin{pmatrix}\cos(\theta) & -\sin(\theta)\\
\sin(\theta) & \cos(\theta)
\end{pmatrix}
\begin{pmatrix}
e_{1}\\ e_{2}
\end{pmatrix},
\end{equation}
then $v_{1}$ is length 1 and inverible. But it is the sum of two
generators with nontrivial coefficients (for $\theta\notin\pi\QQ$, for
example). This is an example of a random element which is a pinor
(element of $\Pin(V)$).

\begin{remark}
Adams~\cite{adams1996:ex} gives an equivalent construction of the Pin
group, but is rather confusing: he uses $\beta$ to reverse the order of
multiplication, for example, instead of using the transpose of the
representation. Adams's $\alpha$ simply maps $e_{i}\mapsto-e_{i}$ on the
canonical generators for $\Cl(V)$. His
$\gamma=\alpha\circ\beta=\beta\circ\alpha$ called \define{Conjugation}
by DJH Garlin~\cite[pg.94]{Garling:2011zz}. Observe that $\alpha$ acts
trivially on even guys, and only on odd guys will something nontrivial
happen. 
\end{remark}

\begin{remark}
Adams defines $\Pin(V)$ as those $x\in\Cl(V)$ such that:
\begin{enumerate}
\item $x(\gamma(x))=(\gamma(x))x=1$, i.e., its conjugate is its inverse; and
\item the map $\pi_{x}\colon V\subset\Cl(V)\to\Cl(V)$ defined by $\pi_{x}(v)=xv(\beta(x))$
is injective.
\end{enumerate}
Then he proves $\pi$ is a 2-to-1 morphism of $\Pin(V)\onto\O(V)$. The
preimage of $\SO(V)$ is then defined as $\Spin(V)$.

This coincides with what Garlin~\cite[ch.8]{Garling:2011zz} does, since
$x^{-1}=\gamma(x)$ and so Garlin's approach [rewritten to use Adams's
  notation] using $v\mapsto x v \alpha(x^{-1})$, and direct calculation
shows
\begin{calculation}
  x v \alpha(x^{-1})
\step{since $x^{-1} = \gamma(x)$ by Adams's first criterion}
  x v ((\alpha\circ\gamma)(x))
\step{since $\gamma=\alpha\circ\beta$}
x v ((\alpha\circ\alpha\circ\beta)(x))
\step{since $\alpha\circ\alpha=\id$}
xv(\beta(x)).
\end{calculation}
But Garlin is slightly more general.
\end{remark}

\N{Lie Algebra of $\Pin(V)$}
For $V\iso\FF^{n}$ (here $\FF$ being $\RR$ or $\CC$), $\Pin(V)$ has an
associated Lie algebra spanned by $e_{i}e_{j}\in\Cl(V)$ for $i<j$, and
whose Lie bracket is the commutator.

\M We can then consider the subgroup of $\Pin(V)$ consisting of an even
number of generators multiplied together. This is precisely the Spinor
group $\Spin(V)$.

Observe, this works out nicely, since the exponentiation of the Lie
algebra for $\Pin(V)$ should give us the component connected to the
identity, which is precisely $\Spin(V)$.

\M
Observe that, for $i < j$, we have:
\begin{calculation}
  (e_{i}e_{j})^{2}
\step{unfold}
  e_{i}e_{j}e_{i}e_{j}
\step{since $i\neq j$ we have $e_{j}e_{i}=-e_{i}e_{j}=(-1)e_{i}e_{j}$}
  e_{i}(-1)e_{i}e_{j}e_{j}
\step{using $e_{j}^{2}=-1$ and $e_{i}(-1)e_{i}=+1$}
  (+1)(-1) = -1.
\end{calculation}
Therefore for any $t\in\RR$, we have:
\begin{equation}
\E^{e_{i}e_{j}t} = \cos(t) + e_{i}e_{j}\sin(t),
\end{equation}
using Euler's formula. But when acting on $V$, we see that
$x=\exp(e_{i}e_{j}t)$ is represented by the matrix
\begin{equation}%\renewcommand{\arraystretch}{1.3}
\begin{split}
  \phantom{\pi_{x}} &\\
  \pi_{x}&=\quad \left[\begin{array}{ccccccccccc}
% 1 & 2    & 3 &  4       & 5 &  6     & 7 & 8         & 9 & 10 & 11
1 &        &   & \UP{i}{} &   &        &   & \UP{j}{}  &   &        & \\
  & \ddots &   &          &   &        &   &           &   &        & \\
  &        & 1 &          &   &        &   &           &   &        & \\
\LF{i}  &        &   & \cos(2t) &   &        &   & -\sin(2t) &   &        & \\
  &        &   &          & 1 &        &   &           &   &        & \\
  &        &   &          &   & \ddots &   &           &   &        & \\
  &        &   &          &   &        & 1 &           &   &        & \\
\LF{j}  &        &   & \sin(2t) &   &        &   & \cos(2t)  &   &        & \\
  &        &   &          &   &        &   &           & 1 &        & \\
  &        &   &          &   &        &   &           &   & \ddots & \\
  &        &   &          &   &        &   &           &   &        & 1
  \end{array}\right]
\end{split}
\end{equation}
Believe me? Well, even I have my doubts, so let us see how it acts on a
unit vector $e_{k}\in V$:
\begin{calculation}
(\cos(t) + e_{i}e_{j}\sin(t))e_{k}(\cos(t) - e_{i}e_{j}\sin(t))
\step{distributivity}
(e_{k}\cos(t) + e_{i}e_{j}e_{k}\sin(t))(\cos(t) - e_{i}e_{j}\sin(t))
\step{distributivity}
e_{k}\cos^{2}(t) + e_{i}e_{j}e_{k}\sin(t)\cos(t)
-e_{k}e_{i}e_{j}\cos(t)\sin(t) - e_{i}e_{j}e_{k}e_{i}e_{j}\sin^{2}(t)
\end{calculation}
Now, if $e_{k}=e_{i}$ or if $e_{k}=e_{j}$, then we have
\begin{calculation}
(\cos(t) + e_{i}e_{j}\sin(t))e_{k}(\cos(t) - e_{i}e_{j}\sin(t))
\step{continuing from last step}
e_{k}\cos^{2}(t) + e_{i}e_{j}e_{k}2\sin(t)\cos(t) - e_{i}e_{j}e_{k}\sin^{2}(t)
\step{using trig identities}
e_{k}\cos(2t) + e_{i}e_{j}e_{k}\sin(2t)
\step{collecting terms}
\bigl(\cos(2t) + e_{i}e_{j}\sin(2t)\bigr)e_{k}.
\end{calculation}
Hence we find if $e_{k}=e_{i}$ or if $e_{k}=e_{j}$,
\begin{subequations}
\begin{equation}
(\cos(t) + e_{i}e_{j}\sin(t))e_{k}(\cos(t) - e_{i}e_{j}\sin(t))
= \bigl(\cos(2t) + e_{i}e_{j}\sin(2t)\bigr)e_{k}.
\end{equation}
If $e_{k}\neq e_{i}$ and $e_{k}\neq e_{j}$, then we find by a
straightforward calculation (and using the $\cos^{2}(t)+\sin^{2}(t)=1$
trig identity):
\begin{equation}
(\cos(t) + e_{i}e_{j}\sin(t))e_{k}(\cos(t) - e_{i}e_{j}\sin(t))=e_{k}.
\end{equation}
\end{subequations}
Combining these two results, and rewriting it using a matrix, we obtain
precisely the desired matrix.

Coincidentally, this is why we have a 2-to-1 covering of $\SO(n)$
\textbf{as groups}, because we have $\cos(2t)$ and $\sin(2t)$ as the
entries of our matrix. But \textbf{as Lie algebras}, we have an
isomorphism $\spin(n)\iso\so(n)$.

\N{Maximal Torus of $\Spin(V)$}
The maximal torus $T$ for $\Spin(V)$, $V\iso\RR^{m}$, are elements of
the form
\begin{equation}
y = \prod^{n}_{r=1}[\cos(x_{r}/2) + e_{2r-1}e_{2r}\sin(x_{r}/2)]
\end{equation}
where $x_{r}\in\RR$ are parameters (``Euler angles''). These correspond
to matrices
\begin{equation}
  \pi_{y} = 
\begin{pmatrix}
\cos(x_{1}) & -\sin(x_{1}) &            &              & & & & \\
\sin(x_{1}) &  \cos(x_{1}) &            &              & & & & \\
           &              & \cos(x_{2}) & -\sin(x_{2}) & & & & \\
           &              & \sin(x_{2}) &  \cos(x_{2}) & & & & \\
 & & & & \ddots &            &              & \\
 & & & &        & \cos(x_{n}) & -\sin(x_{n}) & \\
 & & & &        & \sin(x_{n}) &  \cos(x_{n}) &  \\
 & & & &        &             &             & 1
\end{pmatrix}
\end{equation}
If $m=2n$, we just delete the last row and column (i.e., the row and
column with $1$ in the lower right corner).

\begin{remark}
Adams~\cite{adams1996:ex} concludes chapter 3 with a remark proving, for
$\RR^{n}$ guys, every element of $\Spin(\RR^{n})$ is conjugate to an
element in its torus. This seems intuitive, but I may be missing
something. 
\end{remark}

\section{Representations}

\M
Since $\Spin(V)\subset\Cl_{0}(V)$ is a subset of even Clifford
subalgebra, then any $\Cl_{0}(V)$-module is a representation of
$\Spin(V)$.

\begin{proposition}[{Adams~\cite[Prop.4.1]{adams1996:ex}}]
The algebras $\Cl(V)$ and $\Cl_{0}(V)$ are semi-simple, and hence all
their representations are completely reducible.
\end{proposition}

\begin{proposition}[{Adams~\cite[Prop.4.2]{adams1996:ex}}]
\begin{enumerate}
\item If $m=\dim(V)=2n+1$ is odd, then $\Cl_{0}(V)$ has one irreducible
  representation $\Delta$ of degree $2^{n}$ affording a representation
  $\Delta$ of $\Spin(2n+1)$ with weights $\frac{1}{2}(\pm x_{1}\pm x_{2}\pm\cdots\pm x_{n})$
  with each sign being chosen independently, so there are $2^{n}$ of
  these weights.
\item If $m=\dim(V)=2n$ is even, then $\Cl_{0}(V)$ has two irreducible
  representations $\Delta^{+}$, $\Delta^{-}$ of degree $2^{n-1}$
  affording representations $\Delta^{+}$, $\Delta^{-}$ of $\Spin(2n)$
  with weights $\frac{1}{2}(\pm x_{1}\pm x_{2}\pm\cdots\pm x_{n})$
  with an even number of $-$ minus signs for $\Delta^{+}$ and an odd number of
  $-$ minus signs for $\Delta^{-}$. There are $2^{n-1}$ weights.
\item If $V$ is a complex vector space, then these constructions give
  complex-analytic representations for $\Spin_{\CC}(m)$.
\end{enumerate}
\end{proposition}

\M
Adams then uses these representations to construct exceptional Lie
groups. For example, he constructs [the compact real] $\mathtt{G}_{2}$
as a subgroup of $\Spin(7)$ in Adams's Theorem~5.5.

Its split form may be constructed analogously, but using 7-dimensional
Minkowski space instead.
Gogberashvili and Gurchumelia~\cite{Gogberashvili:2019ojg} construct
this quite explicitly. Ekins and Cornwell~\cite{Ekins:1975yu} do the
calculations for the associated Lie algebra.

\M
Also, Adams makes heavy usage of the $\Spin(16)$ subgroup of
$\mathtt{E}_{8}$ when constructing other exceptional Lie groups.
We have a sequence of mappings $G\to\Spin(16)\to\mathtt{E}_{8}$,
then take the centralizer of the image of $G$ under these maps, and the
identity component of that centralizer turns out to be a compact form of
an exceptional Lie group (depending on the choice of $G$).
This provides compact forms for $\mathtt{E}_{6}$, $\mathtt{E}_{7}$,
$\mathtt{F}_{4}$.

The only direct construction for $\mathtt{F}_{4}$
that I can find is Figueroa-O'Farrill~\cite{Figueroa-OFarrill:2007jcv}
using $S^{8}$ and Killing spinors.