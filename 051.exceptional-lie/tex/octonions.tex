\chapter{Octonions}

\section{Cayley--Dickson Construction}

\N{Complex Numbers}
We start with $\RR$. We adjoin to it a number $\I$, and define $\CC$ as
consisting of pairs of real numbers $x$, $y\in\RR$ written as $x + \I y$.
Multiplication works by
\begin{equation}
(a + \I b)(c + \I d) = (ac + \lambda d\bar{b}) + \I(\bar{a}d + cb) 
\end{equation}
where $\lambda\in\{-1,+1\}$ is a constant such that
\begin{equation}
\I^{2} = \lambda,
\end{equation}
and for real numbers $\bar{x}=x$. We have $\overline{x + \I y} = x - \I y$
for complex numbers.

\N{Quaternions}
Now we have $\CC$, we can iterate this construction. We adjoint a new
number $\J $ and define $\HH$ as a pair of complex numbers $x$, $y\in\CC$
written as $x + \J y$. We define conjugation by
\begin{equation}
\overline{x + \J y} = \overline{x} - \J y,
\end{equation}
and multiplication by
\begin{equation}
(a + \J b)(c + \J d) = (ac + \lambda d\bar{b}) + \J(\bar{a}d + cb) 
\end{equation}
where $\lambda\in\{-1,+1\}$ is a constant such that
\begin{equation}
\J^{2} = \lambda.
\end{equation}
When $\lambda=+1$, these are the split quaternions; when $\lambda=-1$,
these are the usual quaternions $\HH$.

Really? Let us see:
\begin{calculation}
  (a + \J b)(c + \J d)
\step{distributivity}
a(c + \J d) + \J b(c + \J d)
\step{writing complex numbers out explicitly}
(a_{1} + \I a_{2})(c_{1} + \I c_{2} + \J d_{1} + \K d_{2})
+ (\J b_{1} + \K b_{2})(c_{1} + \I c_{2} + \J d_{1} + \K d_{2})
\step{distributivity}
(a_{1} + \I a_{2})(c_{1} + \I c_{2}) + (a_{1} + \I a_{2})j(d_{1} +\I d_{2})
+ \J (b_{1} + \I b_{2})(c_{1} + \I c_{2}) + \J (b_{1} + \I b_{2})j(d_{1} +\I d_{2})
\step{since $\I\J=-\J\I$}
(a_{1} + \I a_{2})(c_{1} + \I c_{2}) + (a_{1} + \I a_{2})\J(d_{1} +\I d_{2})
+ \J(b_{1} + \I b_{2})(c_{1} + \I c_{2}) + \J^{2}(b_{1} - \I b_{2})(d_{1} +\I d_{2})
\step{collecting factors of $j$}
[(a_{1} + \I a_{2})(c_{1} + \I c_{2}) + \J^{2}(b_{1} - \I b_{2})(d_{1} +\I d_{2})]
+ (a_{1} + \I a_{2})\J(d_{1} +\I d_{2})
+ \J(b_{1} + \I b_{2})(c_{1} + \I c_{2})
\step{since $\I\J=-\J\I$}
[(a_{1} + \I a_{2})(c_{1} + \I c_{2}) + \J^{2}(b_{1} - \I b_{2})(d_{1} +\I d_{2})]
+ \J(a_{1} - \I a_{2})(d_{1} +\I d_{2})
+ \J(b_{1} + \I b_{2})(c_{1} + \I c_{2})
\step{collecting factors of $\J$}
[(a_{1} + \I a_{2})(c_{1} + \I c_{2}) + \J^{2}(b_{1} - \I b_{2})(d_{1} +\I d_{2})]
+ \J[(a_{1} - \I a_{2})(d_{1} +\I d_{2}) + (b_{1} + \I b_{2})(c_{1} + \I c_{2})]
\step{rewriting these guys with complex numbers}
[ac + \lambda\bar{b}d] + \J[\bar{a}d + bc].
\end{calculation}

\N{Caution about Conventions}
There are different (but equivalent) presentations of the
Cayley--Dickson construction. For example, Wikipedia consistently uses
the definition for multiplication
\begin{equation}
(a, b)(c, d) = (ac - \overline{d}b, da + b\overline{c}).
\end{equation}
This turns out to be equivalent to our presentation.

If we suppose we have a number system $\FF$, then the Cayley--Dickson
construction adjoins a number $\widehat{\i}$ to it satisfying
$\widehat{\i}^{2}=-1$ and define a $*$ operation satisfying:
\begin{subequations}
\begin{align}
a(\widehat{\i}b) &= \widehat{\i}(a^{*}b)\\
(a\widehat{\i})b &= (a(b^{*}))\widehat{\i}\\
(\widehat{\i}a)(b\widehat{\i}^{-1}) &= (ab)^{*},\\
\intertext{for all $a$, $b\in\FF$. We also have}
\widehat{\i}^{*} &= -\widehat{\i}.
\end{align}
\end{subequations}
This suffices for constructing a new algebra $\FF'$ from $\FF$ and
$\widehat{\i}$. If there is no ``natural'' $*$ operation on $\FF$ (e.g.,
the real numbers don't appear to have one), then $*$ may be defined as
doing nothing. For $\CC$, it's complex conjugation.

Observe further that, if we do the math, we have:
\begin{equation}
a^{*} = (\widehat{\i}a)\widehat{\i}^{-1} = \widehat{\i}(a\widehat{\i}^{-1}),
\end{equation}
for all $a\in\FF$.

\N{Octonions}
We iterate our construction, using a pair of quaternions $a$, $b\in\HH$
to write an octonion as a pair $(a, b)$. Then multiplication of
octonions is defined as:
\begin{equation}
(a, b)(c,  d) =(ac - d\bar{b}, \bar{a}d+cb).
\end{equation}
We may define the conjugate of an octonion as
\begin{equation}
\overline{(a,b)} = (\overline{a}, -b).
\end{equation}
This is equivalent to writing
\begin{equation}
(a, b) = a + \ell b,
\end{equation}
where $\ell$ is the new ``number'' adjoined to the quaternions to obtain
the octonions $\OO$.

\section{Automorphisms}

\M
It is not too hard to see that $\aut(\RR)$ is trivial, and
$\aut(\CC)\iso\ZZ_{2}$ consists of complex conjugation. However,
quaternions and octonions have ``bigger'' automorphism groups. 

\subsection{For Quaternions}

\N{``Definition''} We recall an \emph{automorphism} of $\HH$ is an
$\RR$-linear map $f\colon\HH\to\HH$ such that:
\begin{enumerate}
\item $f(xy)=f(x)f(y)$ for all $x$, $y\in\HH$
\item $f(x\pm y)=f(x)\pm f(y)$ for all $x$, $y\in\HH$
\item $f$ is bijective.
\end{enumerate}

\N{Positive-definite inner product}
We can form the inner product $\langle x, y\rangle$ on $\HH$ using the
polarization of the norm $\|x\|=x\overline{x}$. This inner product is
positive definite, since $\langle x, x\rangle\geq x\overline{x}$.

\N{Automorphisms preserving inner product}
We can consider which automorphisms preserve this inner product. We know
for any two unit quaternions $u$, $v\in\HH$ (i.e.,
$u\overline{u}=v\overline{v}=1$) we have a linear map
\begin{equation}
f_{u,v}(x) = ux\overline{v}.
\end{equation}
This preserves the inner product, since
\begin{equation}
f_{u,v}(x)\overline{f_{u,v}(x)}=x\overline{x}
\end{equation}
for any $x\in\HH$.

Using associativity and $\overline{x\cdot y}=\overline{y}\cdot\overline{x}$, we
find
\begin{equation}
f_{u_{1},v_{1}}\circ f_{u_{2},v_{2}} = f_{u_{1}u_{2},\,v_{1}v_{2}}.
\end{equation}
Therefore $f$ defines a morphism
\begin{equation}
f\colon S^{3}\times S^{3}\to\SO(4),
\end{equation}
where $S^{3}=\{u\in\HH\mid u\overline{u}=1\}\subset\HH$. The reader can
verify $\ker(f)=\{(\pm1,\pm1)\}\iso\ZZ_{2}$ and also that $f$ is
surjective.

\begin{exercise}
Prove or find a counter-example:
the automorphism group $\aut(H)$ for the quaternions consist of elements
of $\SO(4)$ of the form $f_{u,u}$.
\end{exercise}

\begin{remark}
Using the language of spin groups, the mappings $f$ form the elements of
$\Spin(3)\times\Spin(3)$, and we have asked the reader to prove
$\aut(\HH)\iso\Spin(3)$. 
\end{remark}

\subsection{For Octonions}

\begin{definition}
An \define{Automorphism} of $\OO$ is a $\RR$-linear function
$f\colon\OO\to\OO$ such that
\begin{enumerate}
\item $f(xy)=f(x)f(y)$ for all $x$, $y\in\OO$,
\item $f(x\pm y)=f(x)\pm f(y)$ for all $x$, $y\in\OO$,
\item $f$ is a bijection.
\end{enumerate}
\end{definition}

\begin{remark}
This definition assumes we constructed the normed division algebra using
the Cayley--Dickson construction starting with the real numbers. We
could have started with, say, the integers (or some other commutative
ring). In this case, an automorphism of the Octonions (resulting from
this modified construction) needs to be $R$-linear instead of $\RR$-linear,
where $R$ is our initial commutative ring replacing the reals.

This is why our definition for an automorphism of $\OO$ has the
redundant conditions.
\end{remark}

\begin{exercise}[{Coxeter~\cite{Coxeter:1946int}}]
If we demand $f(u)=v$ where $u$ is any unit of $\OO$ and $v$ is any
corresponding unit of $\OO$, prove there are only $168 = 2^{3}3^{1}7^{1}$
such automorphisms $f$.
[Hint: consider the permutations of units generated by $(12)(47)$ and
  $(2143576)$ and possibly changes of signs.]
\end{exercise}

\N{Inner Automorphisms}
We could consider the inner automorphisms for the Octonions, which look
like
\begin{equation}
f(x) = gxg^{-1},
\end{equation}
where $g\neq0$ is a fixed Octonion.
Does this make sense? Well, no, because the Octonions are not
associative. But we \emph{can} use the fact that
\begin{equation}
g^{-1} = \frac{1}{\|g\|^{2}}\overline{g}.
\end{equation}
Then we have
\begin{equation}
(gx)g^{-1} = g(xg^{-1}),
\end{equation}
and so our proposed notion of an inner automorphism \emph{does} make
sense.

When is it an automorphism?

\begin{theorem}[{Lamont~\cite[Th.2.1]{Lamont:1963}}]
The function $f(x)=gxg^{-1}$ is an automorphism of the Octonions if and
only if either
(1) $g$ is real, or (2)
\begin{equation}
\bigl(\Re(g)\bigr)^{2} = \frac{1}{4}\|g\|^{2}.
\end{equation}
That is, the octonion $g$ lies at an angle of $0^{\circ}$, $60^{\circ}$,
$120^{\circ}$, or $180^{\circ}$ from the positive real axis.
\end{theorem}

That is, $(g + \overline{g})^{2} = \|g\|^{2} = \overline{g}g$ iff
$x\mapsto gxg^{-1}$ is an automorphism.