\chapter{Octonions}

\section{Cayley--Dickson Construction}

\N{Complex Numbers}
We start with $\RR$. We adjoin to it a number $\I$, and define $\CC$ as
consisting of pairs of real numbers $x$, $y\in\RR$ written as $x + \I y$.
Multiplication works by
\begin{equation}
(a + \I b)(c + \I d) = (ac + \lambda d\bar{b}) + \I(\bar{a}d + cb) 
\end{equation}
where $\lambda\in\{-1,+1\}$ is a constant such that
\begin{equation}
\I^{2} = \lambda,
\end{equation}
and for real numbers $\bar{x}=x$. We have $\overline{x + \I y} = x - \I y$
for complex numbers.

\N{Quaternions}
Now we have $\CC$, we can iterate this construction. We adjoint a new
number $j$ and define $\HH$ as a pair of complex numbers $x$, $y\in\CC$
written as $x + jy$. We define conjugation by
\begin{equation}
\overline{x + jy} = \overline{x} - jy,
\end{equation}
and multiplication by
\begin{equation}
(a + j b)(c + j d) = (ac + \lambda d\bar{b}) + j(\bar{a}d + cb) 
\end{equation}
where $\lambda\in\{-1,+1\}$ is a constant such that
\begin{equation}
j^{2} = \lambda.
\end{equation}
When $\lambda=+1$, these are the split quaternions; when $\lambda=-1$,
these are the usual quaternions $\HH$.

Really? Let us see:
\begin{calculation}
  (a + jb)(c + jd)
\step{distributivity}
a(c + jd) + jb(c + jd)
\step{writing complex numbers out explicitly}
(a_{1} + \I a_{2})(c_{1} + \I c_{2} + jd_{1} + kd_{2})
+ (jb_{1} + kb_{2})(c_{1} + \I c_{2} + jd_{1} + kd_{2})
\step{distributivity}
(a_{1} + \I a_{2})(c_{1} + \I c_{2}) + (a_{1} + \I a_{2})j(d_{1} +\I d_{2})
+ j(b_{1} + \I b_{2})(c_{1} + \I c_{2}) + j(b_{1} + \I b_{2})j(d_{1} +\I d_{2})
\step{since $\I j=-j\I$}
(a_{1} + \I a_{2})(c_{1} + \I c_{2}) + (a_{1} + \I a_{2})j(d_{1} +\I d_{2})
+ j(b_{1} + \I b_{2})(c_{1} + \I c_{2}) + j^{2}(b_{1} - \I b_{2})(d_{1} +\I d_{2})
\step{collecting factors of $j$}
[(a_{1} + \I a_{2})(c_{1} + \I c_{2}) + j^{2}(b_{1} - \I b_{2})(d_{1} +\I d_{2})]
+ (a_{1} + \I a_{2})j(d_{1} +\I d_{2})
+ j(b_{1} + \I b_{2})(c_{1} + \I c_{2})
\step{since $\I j=-j\I$}
[(a_{1} + \I a_{2})(c_{1} + \I c_{2}) + j^{2}(b_{1} - \I b_{2})(d_{1} +\I d_{2})]
+ j (a_{1} - \I a_{2})(d_{1} +\I d_{2})
+ j(b_{1} + \I b_{2})(c_{1} + \I c_{2})
\step{collecting factors of $j$}
[(a_{1} + \I a_{2})(c_{1} + \I c_{2}) + j^{2}(b_{1} - \I b_{2})(d_{1} +\I d_{2})]
+ j [(a_{1} - \I a_{2})(d_{1} +\I d_{2}) + (b_{1} + \I b_{2})(c_{1} + \I c_{2})]
\step{rewriting these guys with complex numbers}
[ac + \lambda\bar{b}d] + j[\bar{a}d + bc].
\end{calculation}

\N{Octonions}
We iterate our construction, using a pair of quaternions $a$, $b\in\HH$
to write an octonion as a pair $(a, b)$. Then multiplication of
octonions is defined as:
\begin{equation}
(a, b)(c,  d) =(ac - d\bar{b}, \bar{a}d+cb).
\end{equation}
We may define the conjugate of an octonion as
\begin{equation}
\overline{(a,b)} = (\overline{a}, -b).
\end{equation}
This is equivalent to writing
\begin{equation}
(a, b) = a + \ell b,
\end{equation}
where $\ell$ is the new ``number'' adjoined to the quaternions to obtain
the octonions $\OO$.