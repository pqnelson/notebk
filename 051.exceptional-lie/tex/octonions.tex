\chapter{Octonions}

\section{Cayley--Dickson Construction}

\N{Complex Numbers}
We start with $\RR$. We adjoin to it a number $\I$, and define $\CC$ as
consisting of pairs of real numbers $x$, $y\in\RR$ written as $x + \I y$.
Multiplication works by
\begin{equation}
(a + \I b)(c + \I d) = (ac + \lambda d\bar{b}) + \I(\bar{a}d + cb) 
\end{equation}
where $\lambda\in\{-1,+1\}$ is a constant such that
\begin{equation}
\I^{2} = \lambda,
\end{equation}
and for real numbers $\bar{x}=x$. We have $\overline{x + \I y} = x - \I y$
for complex numbers.

\N{Quaternions}
Now we have $\CC$, we can iterate this construction. We adjoint a new
number $\J $ and define $\HH$ as a pair of complex numbers $x$, $y\in\CC$
written as $x + \J y$. We define conjugation by
\begin{equation}
\overline{x + \J y} = \overline{x} - \J y,
\end{equation}
and multiplication by
\begin{equation}
(a + \J b)(c + \J d) = (ac + \lambda d\bar{b}) + \J(\bar{a}d + cb) 
\end{equation}
where $\lambda\in\{-1,+1\}$ is a constant such that
\begin{equation}
\J^{2} = \lambda.
\end{equation}
When $\lambda=+1$, these are the split quaternions; when $\lambda=-1$,
these are the usual quaternions $\HH$.

Really? Let us see:
\begin{calculation}
  (a + \J b)(c + \J d)
\step{distributivity}
a(c + \J d) + \J b(c + \J d)
\step{writing complex numbers out explicitly}
(a_{1} + \I a_{2})(c_{1} + \I c_{2} + \J d_{1} + \K d_{2})
+ (\J b_{1} + \K b_{2})(c_{1} + \I c_{2} + \J d_{1} + \K d_{2})
\step{distributivity}
(a_{1} + \I a_{2})(c_{1} + \I c_{2}) + (a_{1} + \I a_{2})j(d_{1} +\I d_{2})
+ \J (b_{1} + \I b_{2})(c_{1} + \I c_{2}) + \J (b_{1} + \I b_{2})j(d_{1} +\I d_{2})
\step{since $\I\J=-\J\I$}
(a_{1} + \I a_{2})(c_{1} + \I c_{2}) + (a_{1} + \I a_{2})\J(d_{1} +\I d_{2})
+ \J(b_{1} + \I b_{2})(c_{1} + \I c_{2}) + \J^{2}(b_{1} - \I b_{2})(d_{1} +\I d_{2})
\step{collecting factors of $j$}
[(a_{1} + \I a_{2})(c_{1} + \I c_{2}) + \J^{2}(b_{1} - \I b_{2})(d_{1} +\I d_{2})]
+ (a_{1} + \I a_{2})\J(d_{1} +\I d_{2})
+ \J(b_{1} + \I b_{2})(c_{1} + \I c_{2})
\step{since $\I\J=-\J\I$}
[(a_{1} + \I a_{2})(c_{1} + \I c_{2}) + \J^{2}(b_{1} - \I b_{2})(d_{1} +\I d_{2})]
+ \J(a_{1} - \I a_{2})(d_{1} +\I d_{2})
+ \J(b_{1} + \I b_{2})(c_{1} + \I c_{2})
\step{collecting factors of $\J$}
[(a_{1} + \I a_{2})(c_{1} + \I c_{2}) + \J^{2}(b_{1} - \I b_{2})(d_{1} +\I d_{2})]
+ \J[(a_{1} - \I a_{2})(d_{1} +\I d_{2}) + (b_{1} + \I b_{2})(c_{1} + \I c_{2})]
\step{rewriting these guys with complex numbers}
[ac + \lambda\bar{b}d] + \J[\bar{a}d + bc].
\end{calculation}

\N{Caution about Conventions}
There are different (but equivalent) presentations of the
Cayley--Dickson construction. For example, Wikipedia consistently uses
the definition for multiplication
\begin{equation}
(a, b)(c, d) = (ac - \overline{d}b, da + b\overline{c}).
\end{equation}
This turns out to be equivalent to our presentation.

If we suppose we have a number system $\FF$, then the Cayley--Dickson
construction adjoins a number $\widehat{\i}$ to it satisfying
$\widehat{\i}^{2}=-1$ and define a $*$ operation satisfying:
\begin{subequations}
\begin{align}
a(\widehat{\i}b) &= \widehat{\i}(a^{*}b)\\
(a\widehat{\i})b &= (a(b^{*}))\widehat{\i}\\
(\widehat{\i}a)(b\widehat{\i}^{-1}) &= (ab)^{*},\\
\intertext{for all $a$, $b\in\FF$. We also have}
\widehat{\i}^{*} &= -\widehat{\i}.
\end{align}
\end{subequations}
This suffices for constructing a new algebra $\FF'$ from $\FF$ and
$\widehat{\i}$. If there is no ``natural'' $*$ operation on $\FF$ (e.g.,
the real numbers don't appear to have one), then $*$ may be defined as
doing nothing. For $\CC$, it's complex conjugation.

Observe further that, if we do the math, we have:
\begin{equation}
a^{*} = (\widehat{\i}a)\widehat{\i}^{-1} = \widehat{\i}(a\widehat{\i}^{-1}),
\end{equation}
for all $a\in\FF$.

\N{Octonions}
We iterate our construction, using a pair of quaternions $a$, $b\in\HH$
to write an octonion as a pair $(a, b)$. Then multiplication of
octonions is defined as:
\begin{equation}
(a, b)(c,  d) =(ac - d\bar{b}, \bar{a}d+cb).
\end{equation}
We may define the conjugate of an octonion as
\begin{equation}
\overline{(a,b)} = (\overline{a}, -b).
\end{equation}
This is equivalent to writing
\begin{equation}
(a, b) = a + \ell b,
\end{equation}
where $\ell$ is the new ``number'' adjoined to the quaternions to obtain
the octonions $\OO$.
