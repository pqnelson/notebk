\chapter{$G_{2}$}

\M
Since $\mathtt{G}_{2}$ is the smallest exceptional Lie group, it's
usually the first one studied. Whether this is fair or not, there are at
least three ways to construct the group, which are secretly related to
each other.

\section{Constructing the Group}

\subsection{As Stabilizer of Point}

\begin{theorem}[{Adams~\cite[Th5.5]{adams1996:ex}}]\label{thm:g2:as-subgroup-of-spin-7}
Consider the subgroup $G$ of $\Spin(7)$ which fixes a point $z\in S^{7}\subset\Delta$.
Then $G$ is a compact, connected, simply-connected Lie group of rank 2,
of dimension 14, is commonly called $\mathtt{G}_{2}$, and has the Dynkin
diagram: 
\begin{center}
\includegraphics{img/img.0}
\end{center}
Moreover, $\mathtt{G}_{2}$ is
transitive on pairs $(x,y)$ of orthogonal vectors in $S^{6}\subset\RR^{7}$.
\end{theorem}

Adams's proof is an unhappy one, for me (at least), because it doesn't
give us $\mathtt{G}_{2}$ explicitly. Instead, Adams proves the subgroup
of $\Spin(7)$ fixing $z$ has the desired properties and then proves
uniqueness. 

\subsection{Automorphism Group for Octonions}

\M
We can recall, in subsection~\ref{subsec:octonions:automorphisms},
the notion of automorphisms for Octonions.

\begin{definition}\label{defn:g2:as-aut-oo}
The automorphism group for the Octonions is $\mathtt{G}_{2}$. These are
precisely linear maps $A\colon\OO\to\OO$ such that
\begin{enumerate}
\item $A(1)=1$ and more generally $A(r)=r$ for all $r\in\RR$;
\item $A(x\pm y)=A(x)\pm A(y)$ for all $x,y\in\OO$; and
\item $A(xy)=A(x)A(y)$ for all $x,y\in\OO$.
\end{enumerate}
\end{definition}

\N{Relation to previous definition}
We can form the Octonions from any base triple $(e_{1}, e_{2}, e_{3})$
of orthogonal imaginary unit Octonions which intuitively play the role
of $\I$, $j$, $\ell$. Remember, $k$ is just an abbreviation for $\I j$.
We can pick $e_{1}$ from the unit sphere $S^{6}\subset S^{7}\subset\OO$ of imaginary
Octonions, then pick $e_{2}$ from the unit sphere $S^{5}\subset S^{6}\subset\OO$
consisting of unit imaginary Octonions orthogonal to $e_{1}$.
The remaining possible member of a base triple would belong to $S^{7}$
which is orthogonal to $e_{1}$, $e_{2}$, and $e_{1}e_{2}$ --- which is
precisely a sphere $S^{4}\subset S^{6}\subset\OO$.

The automorphism group for the Octonions then acts transitively on the
space of basic triples. Geometrically, this is precisely the description
of $\mathtt{G}_{2}$ from our first definition. The ``fixed point'' of
$S^{7}$ is precisely $1\in\OO$.

This establishes the logical equivalence of
the group defined in Theorem~\ref{thm:g2:as-subgroup-of-spin-7}
and the group in Definition~\ref{defn:g2:as-aut-oo}.

\subsection{Stabilizer of Three-Form}

\N{Multiplication Table for Octonions}
Let us first write down the multiplication table for the Octonions,
writing $e_{0}=1$, $e_{1}=\I$, $e_{2}=j$, $e_{3}=k$, $e_{4}=\ell$,
$e_{5}=\I\ell$, $e_{6}=j\ell$, $e_{7}=k\ell$ as the canonical basis for 
$\RR^{8}$ used as generators for the Octonions, we have:
\begin{equation}
\begin{array}{c||c|c|c|c|c|c|c}
      & e_{1}  & e_{2}  & e_{3} & e_{4} & e_{5} & e_{6} & e_{7}\\ \hline\hline
e_{1} & -1     & e_{3}  & -e_{2} & e_{5} & -e_{4} & -e_{7} & e_{6}\\ \hline
e_{2} & -e_{3} & -1     & e_{1} & e_{6}  & e_{7} & -e_{4} & -e_{5}\\ \hline
e_{3} & e_{2} & -e_{1} & -1    & e_{7}  & -e_{6} & e_{5} & -e_{4}\\ \hline
e_{4} & -e_{5}  & -e_{6} & -e_{7} & -1    & e_{1} & e_{2} & e_{3}\\ \hline
e_{5} & e_{4} & -e_{7} & e_{6} & -e_{1} & -1   & -e_{3} & e_{2}\\ \hline
e_{6} & e_{7}  & e_{4} & -e_{5}  & -e_{2} & e_{3} & -1 & -e_{1}\\ \hline
e_{7} & -e_{6}  & e_{5}  & e_{4} & -e_{3}  & -e_{2} & e_{1} & -1
\end{array}
\end{equation}

\N{Associated Calibrated 3-form}
There is a 3-form $\varphi\in\Exterior^{3}(\Im\OO)^{*}$ defined by
\begin{equation}
\varphi(a, b, c) = \frac{1}{2}\langle a, [b,c]\rangle =\frac{1}{2}\langle[a,b],c\rangle
\end{equation}
for any $a$, $b$, $c\in\Im\OO$. Here $[a,b]=ab-ba$ is just the
commutator, and $\langle-,-\rangle$ is the usual inner product on
$\RR^{7}$. This is called the \define{Associated Calibrated 3-Form}.

If we let $\omega^{i}$ be the dual basis, so
$\omega^{i}[e_{j}]={\delta^{i}}_{j}$, and if we write
$\omega^{ijk}=\omega^{i}\wedge\omega^{j}\wedge\omega^{k}$, then let us
try to write $\varphi$ as a linear combination of these $\omega^{ijk}$.
We find:
\begin{subequations}
\begin{align}
  \varphi(e_{1},e_{2},e_{3})&=1\\
  \varphi(e_{1},e_{4},e_{5})&=1\\
  \varphi(e_{1},e_{6},e_{7})&=-1\\
  \varphi(e_{2},e_{4},e_{6})&=1\\
  \varphi(e_{2},e_{5},e_{7})&=1\\
  \varphi(e_{3},e_{4},e_{7})&=1\\
  \varphi(e_{3},e_{5},e_{6})&=-1
\end{align}
\end{subequations}
This gives us
\begin{equation}
\varphi = \omega^{123} + \omega^{145} - \omega^{167} + \omega^{246} +
\omega^{257} + \omega^{347} - \omega^{356}.
\end{equation}

\begin{ddanger}
Be careful with the expression for $\varphi$ using local coordinates,
since we could use a different basis and different multiplication table
for the Octonions.
\end{ddanger}

\N{Representation on 3-Forms}
We have $\GL(\RR^{7})$ act on $\Exterior^{3}(\RR^{7})^{*}$ with
$g\in\GL(\RR^{7})$ by means of the pullback, so for any $a$, $b$,
$c\in\RR^{7}$, and $\Omega\in\Exterior^{3}(\RR^{7})^{*}$, we have
\begin{equation}
g^{*}\Omega(a,b,c) = \Omega\bigl(g^{-1}(a), g^{-1}(b), g^{-1}(c)\bigr).
\end{equation}
This should be contrasted with the naive dual representation on the
third exterior power of $V^{*}$, which would result in using
$\transpose{(g^{-1})}$ instead of $g^{-1}$, yielding an incorrect
expression for the group.

\begin{definition}\label{defn:g2:isotropy-group-of-3-form}
We can define the group $\mathtt{G}_{2}$ as the subgroup of $\GL(\RR^{7})$
as the isotropy subgroup for the associated calibrated 3-form, i.e., as:
\begin{equation}
\mathtt{G}_{2} = \{g\in\GL(\RR^{7})\mid g^{*}\varphi=\varphi\}.
\end{equation}
Explicitly, this means, for any $a$, $b$, $c\in\RR^{7}$ and $g\in\GL(\RR^{7})$, that
$\varphi\bigl(g^{-1}(a), g^{-1}(b), g^{-1}(c)\bigr)=\varphi(a, b, c)$ iff $g\in\mathtt{G}_{2}$.
Of course, since $g\in\mathtt{G}_{2}$ iff $g^{-1}\in\mathtt{G}_{2}$, so
we could easily demand the equivalent condition
$\varphi(g(a),g(b),g(c))=\varphi(a,b,c)$ holds.
\end{definition}

\begin{remark}
Engel~\cite{engel1900:talk} proved in 1900 that a generic complex
3-form's isotropy subgroup is isomorphic to $\mathtt{G}_{2}$. We find in
his notes for his June 11, 1900 talk to the Royal Saxonian Academy of
Sciences: ``Zudem ist hiermit eine direkte Definition unsrer
vierzehngliedrigen einfachen Gruppe gegeben, die an Eleganz nichts zu
w\"{u}nschen \"{u}brig l\"{a}sst.'' [Roughly: 
  ``In addition, this gives a direct definition of our fourteen-membered
  simple group [i.e., $\mathtt{G}_{2}$], which leaves nothing to be
  desired in terms of elegance.'']

Of course, Engel's result is in a related-but-different setting.
When we work with the complex 3-form
\begin{equation}\label{eq:g2:engel-3-form}
\Omega = \omega^{147} + \omega^{257} + \omega^{367} - 2\omega^{123}+2\omega^{456},
\end{equation}
which is what Engels used in his talk, we end up with a group isomorphic
to the complex $\mathtt{G}_{2}$ group. When we work with this same
3-form in the split $\RR^{4,3}$ space, we obtain the isotropy group is
isomorphic to the split real form of $\mathtt{G}_{2}^{\text{split}}$. 
\end{remark}

\begin{remark}
Up to a $\mathtt{G}_{2}$ symmetry (or possibly multiplication by a
nonzero real number), there are only two distinct
``generic'' 3-forms for $\Exterior^{3}(\RR^{7})^{*}$: Engel's 3-form in
Eq~\eqref{eq:g2:engel-3-form}, and the associated calibrated
3-form. Here, we call a $p$-form over $\RR^{n}$ is ``generic'' if its
orbit under $\GL(n)$ is an open set. There are a variety of proofs for
this claim, see section 4 of Draper~\cite{draper:2017g2} for references.

If we work with these 3-forms over the complex numbers instead (i.e., as
elements of $\Exterior^{3}(\CC^{7})^{*}$), then there is only one orbit
for the generic 3-forms.
\end{remark}

\N{Relation to other definitions}
If we take our definition for $\mathtt{G}_{2}$, we can relate it to the
group $\aut(\OO)$ as follows: write any element $\alpha\in\OO$ as
$\alpha=r+\vec{m}$ where $r\in\RR$ is the real part and $\vec{m}$ is the
imaginary part. Then if $g\in\mathtt{G}_{2}$, it acts on $\OO$ by
$\rho_{g}(1)=1$ and $\rho_{g}(\vec{m})=g\cdot\vec{m}$. This relates any
element of $\mathtt{G}_{2}$ to an automorphism of the
Octonions. Conversely, given any automorphism of the Octonions, we may
obtain an element of $\mathtt{G}_{2}$ by examining its action on the
imaginary part of an Octonion and pretending it's a vector in $\RR^{7}$.

This establishes the logical equivalence of
Definition~\ref{defn:g2:as-aut-oo} and
Definition~\ref{defn:g2:isotropy-group-of-3-form}.

