\chapter{$F_{4}$}

\N{Origin: Useless Curio}
The $\mathtt{F}_{4}$ exceptional group is a bit more convoluted. The
earliest discussion of it I could find is in Elie Cartan's doctoral
thesis (part II, \S19) and later in his 1914 article ``\textit{Les groupes r\'eels simples, finis et continus}''\footnote{\url{http://www.numdam.org/item?id=ASENS_1914_3_31__263_0}},
left in the penultimate section of the article, and a sad few paragraphs
in his thesis, which dryly proves its
simplicity as a Lie algebra. It is quite likely Killing discovered it
first and published it in one of his articles in the 1890s, I could not
acquire them to verify this suspicion. To the best of my knowledge, this
is the first instance of its discussion: a curio which is an exception to an
otherwise beautiful classification theorem, as a burden to be dealt
with, rather than as the symmetry group for some object.

\N{Connection with Octonions}
It wasn't until 1949 that any insight into $\mathtt{F}_{4}$ could be
gleaned: Jordan constructed the octionic plane, then the next year Borel~\cite{Borel:1950}
proved its isometry group is precisely $\mathtt{F}_{4}$.
Simultaneously, as Borel made this insight, Chevalley and
Schafer~\cite{Chevalley:1950} constructed the Lie algebra
$\mathfrak{f}_{4}$ using the exceptional Jordan algebra
$\mathfrak{h}_{3}(\OO)$ and, moreover, that the Lie group
$\mathtt{F}_{4}$ is isomorphic to the automorphism group
$\mathtt{F}_{4}\iso\aut(\mathfrak{h}_{3}(\OO))$. Adams~\cite{adams1996:ex}
constructs $\mathtt{F}_{4}$ in this manner, as well.

\M
There is one, last, construction of the $\mathtt{F}_{4}$ group I am
aware of, which is a direct analogy to the construction of the
$\mathtt{E}_{8}$ Lie group: start with the Lie algebra $\mathfrak{spin}(9)$
Lie algebra and add to it its fundamental spin representation
$\Delta_{9}$, then form a Lie algebra out of the underlying vector space
$\mathfrak{spin}(9)\oplus\Delta_{9}$.

This is secretly related to the construction as the automorphism group
of the exceptional Jordan algebra, since
$\Delta_{9}\iso\Delta^{+}_{8}\oplus\Delta^{-}_{8}\iso\OO^{2}$ and
$\Spin(8)$ acts on it.

We can also relate it to triality, by the vector space isomorphisms
\begin{subequations}
  \begin{align}
    \mathfrak{f}_{4} &\iso\mathfrak{so}(9)\oplus\OO^{2}\\
    &\iso\mathfrak{so}(8)\oplus\RR^{8}\oplus\Delta^{+}_{8}\oplus\Delta^{-}_{8}.
  \end{align}
\end{subequations}
We interpret the $\RR^{8}$ as the vector representation for
$\mathfrak{so}(8)$, and $\Delta^{\pm}_{8}$ as the spinor
representations. We construct a bracket
$[-,-]\colon\mathfrak{so}(8)\times\mathfrak{f}_{4}\to\mathfrak{f}_{4}$
using the obvious actions, and it acts on the representating spaces via
the triality map.

I am being a bit terse here, but if I were honest, I should admit these
constructions are all really ``the same under the hood''.

\N{Killing Spinor Construction}
Jos\'e Figueroa-O'Farrill~\cite{Figueroa-OFarrill:2007jcv} published a
construction of the Lie algebra $\mathfrak{f}_{4}$ using a geometric
construction routine from Supergravity called Killing spinors (analogous
quantities to Killing vectors, but on spinor bundles). I do not understand
the details yet, but I would like to learn more when time allows.
Figueroa-O'Farrill assures us that this underpins the construction we
find in Adams' gambit turning $\mathfrak{spin}(n)\oplus\Delta_{n}$ 
into a Lie algebra.