\chapter{Clifford Algebras}

\N{TODO: fix notation}
I should use $e_{\alpha}$ for an orthonormal basis, $\gamma_{\alpha}$
for the generators of $\Cl(V)$, and $\Gamma_{\alpha}$ for matrices in
the representation $\End(W)$ of $\Cl(p,q)$.

\N{TODO: super tensor products}
I believe the calculations below of the Clifford algebra irreducible
representations are correct, but I should be using the super tensor
product for the isomorphisms like
$\Cl(n+2)\iso\Cl(n)\super\otimes\Cl(2)$, and so on.

\section{Review of Clifford Algebras}

\M
Clifford algebras may be formed from a vector space $V$ over $\FF$. It's
especially intuitive when $V$ is finite-dimensional. We just take some
canonical basis $\Gamma_{1}$, \dots, $\Gamma_{n}$ for $V$, then we turn it into an
algebra by defining multiplication on vectors by linearity and the
following anticommutator relation:
\begin{equation}
\Gamma_{i}\Gamma_{j} + \Gamma_{j}\Gamma_{i} = -\delta_{i,j}.
\end{equation}
Observe that $\Gamma_{i}^{2}=-1$ and $\Gamma_{i}\Gamma_{j}=-\Gamma_{j}\Gamma_{i}$ when $i\neq j$.
The resulting algebra is precisely the \define{Clifford Algebra} $\Cl(V)$.

We can generalize this by altering the anticommutation relation to be
\begin{equation}
\Gamma_{i}\Gamma_{j} + \Gamma_{j}\Gamma_{i} = \eta_{i,j}
\end{equation}
for some matrix $\eta$. When $\eta_{i,j}=0$, we obtain the
\define{Grassmann Algebra}. We can also have ``split'' Clifford algebras
by having the signature of $\eta$ being indefinite.

\begin{remark}
Historically, people describe a Clifford algebra using a quadratic form
(a generalized ``norm squared'' quantity). This is completely equivalent
to what we have done. Namely, $\eta_{i,j}=-Q(\Gamma_{i})\delta_{i,j}$ where
$Q$ is the quadratic form on $V$.
\end{remark}

\N{Even Part}
The even part of the Clifford algebra is written as $\Cl^{0}(V)$ or
$\Cl(V)_{0}$. It consists of guys that look like
\[ a_{0}\cdot\mathbf{1} + a^{\mu\nu}\Gamma_{\mu}\Gamma_{\nu} + a^{\alpha\beta\mu\nu}\Gamma_{\alpha}\Gamma_{\beta}\Gamma_{\mu}\Gamma_{\nu}+\dots.\]
Each term has an even number of Clifford generators appearing in it. The
coefficients $a^{\mu\nu}$, $a^{\alpha\beta\mu\nu}$, etc., are completely
antisymmetric in all its indices; equivalently, we could simply sum over
$\alpha<\beta<\mu<\nu<\dots$. 

\begin{definition}
Let $A$ be an $\FF$-algebra (usually $\FF=\RR$, $\CC$, or $\HH$). An
\define{$A$-Representation} of the Clifford algebra $\Cl(V)$ is an algebra morphism
\[ \rho\colon\Cl(V)\to\hom_{A}(W, W) \]
where $W$ is a $A$-algebra representation, $\hom_{A}(W, W)$ is the space
of endomorphisms of $W$ commuting with the $A$-action.
\end{definition}

\N{Semisimple Algebras}
We call an algebra $A$ \define{Semisimple} if all its modules are
completely reducible (= direct sums of simple modules). This is one
possible criterion. We can consider $A$ as a left $A$-module when $A$ is
associative\footnote{If I recall correctly, we need associativity.}, and
then this coincides with $A$ as a semisimple $A$-module. Then $A$ is
semisimple if and only if the left regular representation for $A$ is
completely reducible.

In particular, $\Cl(V)$ is a semisimple algebra. Recall semisimple rings
generalize the concept of a group algebra $\CC[G]$ for finite groups
$G$. For a review of the structure theorem for Clifford algebras, see
\arXiv{1610.02418}. In particular, all irreducible representations of a
semisimple associative algebra may be found as subrepresentations of its
regular representation.\footnote{See Theorem 3.2.13 in \url{https://studenttheses.uu.nl/bitstream/handle/20.500.12932/29895/thesis.pdf?sequence=2}}

For more on this, see chapter 5 of Varadarajan~\cite{Varadarajan:2004yz}.
Specifically, when $V$ is even-dimensional, $\Cl(V)$ is semisimple ---
this is proven in Theorem~5.2.7, and the odd-dimensional case is proven
in Theorem~5.2.10.

\N{Notation: ``Volume Element''}
We write $\omega=\prod^{m}_{j=1}\Gamma_{j}\in\Cl(V)$ and refer to it as the
``\emph{Volume Element}''. Adams does not use this notation (or
terminology), but it caught on in the mathematical literature.
Physicists call it the \emph{Chiral Element}.

\N{Notation}
We write $\FF(n)$ for the algebra of $n\times n$ matrices with entries
in the division algebra $\FF$ (usually $\RR$, $\CC$, $\HH$).

We also write $\Cl_{p,q}$ for $\Cl(\RR^{p,q})$ using the indefinite norm
of signature $(p,q)$ with quadratic form
\begin{equation}
Q(x) = (x_{1})^{2}+\cdots+(x_{p})^{2}-(x_{p+1})^{2}-(\cdots)-(x_{p+q})^{2}.
\end{equation}
We write $\Cl_{p}=\Cl_{p,0}$. This is all fairly standard notation.

\begin{lemma}
Let $\FF=\RR$, $\CC$, or $\HH$.
Consider $\FF(n)$ as an $\RR$-algebra.
Then $\FF(n)$ has a natural representation on $\FF^{n}$ and, up to
equivalence, this is the only irreducible representation.
\end{lemma}

\begin{proof}
Matrix algebras are simple. Simple algebras have a single representation
up to isomorphism.
\end{proof}

\begin{proposition}
\begin{enumerate}
\item $\RR(n)\otimes\RR(m)\iso\RR(nm)$ for all $n$, $m\in\NN$
\item $\RR(n)\otimes_{\RR}\FF\iso\FF(n)$ for all $n\in\NN$
\item $\CC\otimes_{\RR}\CC\iso\CC\oplus\CC$
\item $\CC\otimes_{\RR}\HH\iso\CC(2)$
\item $\HH\otimes_{\RR}\HH\iso\RR(4)$.
\end{enumerate}
\end{proposition}
\begin{proof}
  (1) Obvious.

  (2) Also obvious, we just ``componentwise'' extend the scalars.

  (3) Write $\CC\iso\RR[x]/(x^{2}+1)$, then by direct calculation
\begin{calculation}
\CC\otimes_{\RR}\CC
\step[\iso]{since $\CC\iso\RR[x]/(x^{2}+1)$}
\CC\otimes_{\RR}\RR[x]/(x^{2}+1)
\step[\iso]{extension of scalars}
\CC[x]/(x^{2}+1)
\step[\iso]{Chinese remainder theorem}
\bigl(\CC[x]/(x+\I)\bigr)\oplus \bigl(\CC[x]/(x-\I)\bigr)
\step[\iso]{since $\CC[x]/(x\pm\I)\iso\CC$}
\CC\oplus\CC.
\end{calculation}

(4) Recall that $\HH$ is generated by $\I$ and $j$, since
  $k=ij$. Write
  \begin{equation}
\widehat{I} = \begin{bmatrix}\sqrt{-1} & 0\\
0 & \sqrt{-1}
\end{bmatrix},\quad\mbox{and}\quad
\widehat{J} = \begin{bmatrix} 0 & -1\\
1 & 0
\end{bmatrix}.
  \end{equation}
  Then we can observe $\widehat{I}\widehat{J}=-\widehat{J}\widehat{I}$
  squares to $-1$. Then
\begin{equation}
\begin{split}
  \HH\otimes_{\RR}\CC&\to\CC(2)\\
  \I\otimes1 &\mapsto\widehat{I}\\
  j\otimes1 &\mapsto\widehat{J}.
\end{split}
\end{equation}
This is a homomorphism and moreover an isomorphism.

(5) We have
\begin{equation}
\HH\otimes_{\RR}\HH\iso\End(\HH),
\end{equation}
and since $\HH\iso\RR^{4}$ as a vector space, this implies the result.
\end{proof}

\N{Facts}
We can compute:
\begin{subequations}
\begin{align}
\Cl_{1,0} &= \CC\\
\Cl_{0,1} &= \RR\oplus\RR\\
\Cl_{2,0} &= \HH\\
\Cl_{0,2} &= \RR(2)\\
\Cl_{1,1} &= \RR(2).
\end{align}
\end{subequations}
More helpfully, arranged as a table
\begin{equation}
\begin{array}{c|ccc}
(\Cl_{0,n})   & \huh   & \huh & \huh\\
  \RR(2)     & \huh   & \huh & \huh\\
\RR\oplus\RR & \RR(2) & \huh & \huh\\\hline
\RR          &  \CC   &  \HH & (\Cl_{m,0})
\end{array}
\end{equation}

\begin{theorem}
For all $n$, $r$, $s\geq0$, we have the following isomorphisms:
\begin{enumerate}
\item $\Cl_{n,0}\otimes\Cl_{0,2}\iso\Cl_{0,n+2}$
\item $\Cl_{0,n}\otimes\Cl_{2,0}\iso\Cl_{n+2,0}$
\item $\Cl_{r,s}\otimes\Cl_{1,1}\iso\Cl_{r+1,s+1}$.
\end{enumerate}
\end{theorem}

\begin{remark}
The third condition, $\Cl_{r,s}\otimes\Cl_{1,1}\iso\Cl_{r+1,s+1}$,
allows us to ``propagate'' a known entry ``diagonally'' across the table
of Clifford algebras.
\end{remark}

\N{Bott Periodicity}
We also have the following isomorphisms:
\begin{enumerate}
\item $\Cl_{n+8,0}\iso\Cl_{n,0}\otimes\Cl_{8,0}$
\item $\Cl_{0,n+8}\iso\Cl_{0,n}\otimes\Cl_{0,8}$
\item For complex Clifford Algebras: $\CCl_{n+2}\iso\CCl_{n}\otimes_{\CC}\CCl_{2}$.
\end{enumerate}

\M
This gives us the following table for real $\Cl_{p,q}$:
\begin{equation*}
  \begin{footnotesize}\hskip-3em
\begin{array}{c|ccccccccc}
  8 & \RR(16)            & \CC(16) & \HH(16) & \HH(16)\oplus\HH(16) & \HH(32) & \CC(64) & \RR(128) & \RR(128)\oplus\RR(128) & \RR(256)\\
  7 & \CC(8)             & \HH(8) & \HH(8)\oplus\HH(8) & \HH(16) & \CC(32) & \RR(64) & \RR(64)\oplus\RR(64) & \RR(128) & \CC(128)\\
  6 & \HH(4)             & \HH(4)\oplus\HH(4) & \HH(8) & \CC(16) & \RR(32) & \RR(32)\oplus\RR(32) & \RR(64) & \CC(64) & \HH(64)\\
  5 & \HH(2)\oplus\HH(2) & \HH(4) & \CC(8) & \RR(16) & \RR(16)\oplus\RR(16) & \RR(32) & \CC(32) & \HH(32) & \HH(32)\oplus\HH(32)\\
  4 & \HH(2)             & \CC(4) & \RR(8) & \RR(8)\oplus\RR(8) & \RR(16) & \CC(16) & \HH(16) & \HH(16)\oplus\HH(16)  & \HH(32)\\
  3 & \CC(2)             & \RR(4) & \RR(4)\oplus\RR(4) & \RR(8) & \CC(8) & \HH(8) & \HH(8)\oplus\HH(8) & \HH(16) & \CC(32)\\
  2 & \RR(2)             & \RR(2)\oplus\RR(2) & \RR(4) & \CC(4) & \HH(4) & \HH(4)\oplus\HH(4) & \HH(8) & \CC(16) & \RR(32)\\
  1 & \RR\oplus\RR       & \RR(2) & \CC(2) & \HH(2) & \HH(2)\oplus\HH(2) & \HH(4) & \CC(8) & \RR(16)  & \RR(16)\oplus\RR(16) \\
  0 & \RR                & \CC    & \HH & \HH\oplus\HH & \HH(2) & \CC(4) & \RR(8) & \RR(8)\oplus\RR(8) & \RR(16) \\\hline
    & 0                  & 1  & 2 & 3 & 4 & 5 & 6 & 7 & 8
\end{array}
  \end{footnotesize}
\end{equation*}

\begin{theorem}\label{thm:lie:spin:even-clifford-isomorphisms}
  \begin{enumerate}
  \item $\Cl_{n+1}^{0}\iso\Cl_{n}$ for each $n\in\NN_{0}$.
  \item $\Cl_{s,t}^{0}\iso\Cl_{s-1,t}$ for $s, t\in\NN$.
  \item $\Cl_{s,t}^{0}\iso\Cl_{t-1,s}$ for $s, t\in\NN$.
  \end{enumerate}
\end{theorem}

In particular, we have the following table describing the even
subalgebra as the following real associative algebras:

\begin{table}[h!]
\centering
\begin{tabular}{|c|c|c|} \hline
  $s-t$ mod $8$ & $\Cl_{s,t}^{0}$ & $N$ \\\hline
           1, 7 &       $\RR(N)$ & $2^{(s+t-1)/2}$\\
           3, 5 &       $\HH(N)$ & $2^{(s+t-3)/2}$\\ \hline
           2, 6 &       $\CC(N)$ & $2^{(s+t-2)/2}$\\
              4 & $\HH(N)\oplus\HH(N)$ & $2^{(s+t-4)/2}$\\
              0 & $\RR(N)\oplus\RR(N)$ & $2^{(s+t-2)/2}$\\ \hline
\end{tabular}
\caption{Even Clifford Subalgebras in $\RR^{s,t}$}
\end{table}

