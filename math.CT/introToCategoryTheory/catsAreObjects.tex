\mysubsection{A Morphism is a Mathematical Object}

Now what is a morphism? Well, one would suspect that it's
something else that's not an object. That would require defining
something in math that isn't a mathematical object, which is a
contradiction. So here we must be clever. We really just had the
condition that, let $\ms{C}$ be a category, $X,Y\in\ob{\ms{C}}$, a morphism
$f\in\hom(X,Y)$ is such that
\begin{equation}%\label{eq:}
f:X\to Y
\end{equation}
which might mean something to the reader, it might not. It
doesn't mean a thing to the author, so lets make our notion of a
morphism mathematically rigorous! 

Naively, one would normally do the following: check to see for
similar examples, try to generalize them, then call it a day. We
see that Baez and Wise~\cite{BaezWise:2004ln} give the following
example for a function as a mathematical object:
\begin{ex}
We will go through the list of criteria for a mathematical object
to demonstrate a function is in fact a mathematical object. We
can specify a function as:
\begin{description}
\item[Stuff] a pair of sets $X,$ $Y$;
\item[Structure] equipped with $f\subseteq X\times Y$;
\item[Properties] such that for each $x\in X$ there is exactly
  one corresponding $y\in Y$ such that $(x,y)\in f$.
\end{description}
\noindent Thus a function $f$ is a mathematical object. QEF.
\end{ex}

This more or less defined a function by a set of ordered pairs
$(x,y)$ specified by (as a physicist could understand it)
$y=f(x)$. Now, can we generalize this notion to from a pair of
sets to a pair of arbitrary mathematical objects?

We want it to be more of a ``black box'' than a function. That 
is, we want it to be the following:
\begin{description}
\item[Stuff] We see that we have a pair of mathematical objects 
  $f=(X,Y)$ (where $X,Y\in\ob{\ms{C}}$)
\item[Structure] equipped with two functions $Dom(f), Cod(f)$
\item[Properties] such that $Dom(f)=X$ and $Cod(f)=Y$.
\end{description}
Observe this preserves the structure of the object since both 
objects are in the same category. 


