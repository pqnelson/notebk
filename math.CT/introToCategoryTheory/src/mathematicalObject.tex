%%
%% object.tex
%% 
%% Made by Alex Nelson
%% Login   <alex@tomato>
%% 
%% Started on  Wed Jun 17 15:28:51 2009 Alex Nelson
%% Last update Wed Jun 17 17:43:46 2009 Alex Nelson
%%
\begin{prob}
Suppose we wanted to describe all of math by some common
``meta-structure''. How would we do it? It has to be general enough to
encompass all of the diverse aspects of every mathematical field,
but the danger is becoming \emph{too general}. This scares the
physicists. 

An alternate way to think of the problem is perhaps
this programming problem: we want to program an object oriented
computer algebra system with a common base class, with
functions/homomorphisms/operators/etc as instances of a 
{\tt function} class which extends our base class. Is there an
elegant way to solve this problem?
\end{prob}

So, to answer the initial problem, how can we describe some
generic mathematical object? We will follow the examples of
Baez~\cite{Baez:2004pa}, Baez and Wise~\cite{BaezWise:2004ln},
Baez and Shulman~\cite{baez-2006}, and Baez and Dolan~\cite{baez-2001-1} in
the answer to this question. Well, in linear algebra when we
introduce the notion of a vector space, it is defined as the set
of vectors. Similarly, in abstract algebra, a group is defined as
a set with a binary operator with various properties. The
recurring theme appears to be that there is some set underlying
the mathematical object, but lets not be so strict. Lets instead
give the following proposition:
\begin{prop}%\label{prop:}
A \define{Mathematical Object} is defined by at least specifying
some underlying ``\emph{stuff}'' (e.g. a set, several sets, etc.).
\end{prop}
Well, returning to our example of a vector space, what else makes
this set a vector space? There is some ``extra structure'' that
makes it so. Specifically we can do two things: 1) we can add two
vectors together to get a third vector, 2) we can multiply a
vector by a scalar to get another vector. These are binary
operators, which really are just functions
\begin{equation}%\label{eq:}
f:X\times Y\to Z
\end{equation}
where $X$, $Y$, and $Z$ are the underlying sets, $f$ is the
binary operator in question. It's just that in practice, it looks
kinda funny writing $+(\vec{x},\vec{y})$ for vector addition.
This suggestion of binary operators as functions is -- at first
glance -- foreign. So, being general (again), we suggest that
there are functions, relations, some specified elements, etc. all
fit in this ``extra structure'' description. In e.g. a topology
or a sigma algebra, we are worried about collections of subsets,
which should be taken into account as ``extra structure'' as well
since we are working with distinguished subsets. So to be fully
general we propose the following revised proposition
\begin{prop}%\label{prop:}
A \define{Mathematical Object} is defined by at least specifying
\begin{enumerate}
\item some underlying ``\emph{stuff}'' (e.g. a set, several sets, etc.)
\item equipping this ``stuff'' with some ``\emph{structure}''
  (e.g. functions, binary operators, collections of subsets,
  distinguished elements, etc.).
\end{enumerate}
\end{prop}
This still is not enough. We can't just have any old
``structure'', we need to specify conditions our desired
structure satisfies. If we are considering a binary operator, is
our ``stuff'' closed under this binary operator? Is there some
identity element $e$ so when we consider the binary operator
applied to $e$ and some $x$ (i.e. $f(e,x)$) we end up with our
$x$? If so, does every $x$ in our stuff have an inverse? And so
on, the list is endless.

These conditions are really just demanding certain equations (or
in some cases inequalities or inclusions) holds. These are just
algebraically described as ``\emph{properties}'' of our
structure. This is enough to give a fully general account of a
mathematical object:
\begin{framed}
\begin{defn}\label{defn:object}
\addcontentsline{toc}{section}{*\quad\;Definition: Mathematical Object}
A \define{Mathematical Object} is defined by:
\begin{enumerate}
\item specifying some underlying ``\emph{stuff}'' (e.g. a set, several sets, etc.)
\item equipping this ``stuff'' with some ``\emph{structure}''
  (e.g. functions, binary operators, collections of subsets,
  distinguished elements, etc.).
\item with this ``structure'' satisfying certain
  ``\emph{properties}'' (e.g. equations, inequalities, etc.).
\end{enumerate}
\end{defn}
\end{framed}
This is a sufficient generalized notion of a mathematical
object. We can use this in the definition of a category.

\begin{rmk}
If one really pushed, I doubt that any of these conditions can be
rigorously defined (what do you mean by ``equations''?
``functions''? etc.). The point is, as with every definition, to
give some intuition behind the concept as well as some defining
characteristics of the object. But if one really pushed, I
suspect this is where Godel's incompleteness comes into play.
\end{rmk}

\begin{rmk}
As Baez and Wise point out~\cite{BaezWise:2004ln}, we can always
\emph{check} the properties -- they are either true or false. We
can also \emph{choose} structures from a set of
possibilities. And we can \emph{choose} stuff from a category of
possibilities. But each step depends on the following
ones. Structure depends on stuff, and properties depends on
structure. This should be somewhat intuitive, we can't demand
conditions on structure we don't have, nor can we have structure
depend on stuff we don't have.
\end{rmk}

\begin{prop}
Every definition in math is defining one of the following: a
mathematical object, some ``stuff'', some ``structure'', or some
``property''.
\end{prop}
\begin{con} In relating definitions to mathematical objects, we
  conjecture:
\begin{enumerate}
\item The definition of some ``stuff'' or ``structure'' can be
  reformulated as a definition of a mathematical object.
\item The definition of some ``property'' can be reformulated as
  a definition of some ``structure''; moreover, it can be
  reformulated as a definition of a mathematical object.
\end{enumerate}
Or every definition defines a mathematical
object --- defining a mathematical object is sometimes more
circuitous and not as interesting or useful. 
\end{con}
This conjecture is summarized in the single phrase
``\emph{everything defined in math is-a mathematical object}''.
