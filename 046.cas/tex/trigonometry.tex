\chapter{Trigonometry}

\M I am curious about symbolically defining trigonometric
functions.\footnote{We can consider a different approach using the
binomial theorem and infinitesimals, following Euler; see McKinzie and
Tuckey's ``Higher Trigonometry, Hyperreal Numbers, and Euler’s Analysis of Infinities''~\cite{mckinzie}.} It
suffices to consider defining sine and cosine by certain defining
properties. What are they?

We can axiomatically characterize cosine and sine using the following
properties:
\begin{enumerate}
\item Angle subtraction for cosine: $\cos(x-y) = \cos(x)\cos(y) + \sin(x)\sin(y)$ for all $x\in\RR$, $y\in\RR$
\item $\sin(\pi/2)=1$
\item for any $x\in[0,\pi/2]$, $\sin(x)\geq0$.
\end{enumerate}
In fact, we have a theorem.

\begin{theorem}[{Robinson~\cite{robinson}}]
Let $p\in\RR$ be a positive real number (fixed). Let $C\colon\RR\to\RR$
and $S\colon\RR\to\RR$ be continuous functions such that
\begin{enumerate}
\item $C(x-y) = C(x)C(y) + S(x)S(y)$ for all $x$, $y\in\RR$
\item $S(p)=1$
\item for any $x\in[0,p]$, $S(x)\geq0$.
\end{enumerate}
Then $C$ and $S$ exist and are unique.
\end{theorem}

Here $p$ generalizes the constant $\pi/2$. If we let $p\in\RR$ be fixed
but arbitrary, nothing changes: for negative $p$, the sign will change
for $S(x)$; for $p=0$, we will find $C(x)=S(x)=0$. So the restriction
that $p>0$ is inconsequential book-keeping.

\N{On $\pi$}
If we take this synthetic approach, how do we actually compute $\pi$? We
could easily do so using Machin's formula
\begin{equation}
\frac{\pi}{4} = 4\arctan(1/5) - \arctan(1/239),
\end{equation}
or some other version. Euler popularized this approach with the series
expansion for arctangent applied to the Machin-like formula,
\begin{equation}
\frac{\pi}{4} = 5\arctan(1/7) + 2 \arctan(3/77).
\end{equation}
Euler discovered a series expression, which converges faster than the
Taylor series expansion,
\begin{equation}
\arctan(z) = \frac{z}{1 + z^{2}}\sum^{\infty}_{n=0}\prod^{n}_{k=1}\frac{2kz^{2}}{(2k + 1)(1 + z^{2})}
= \sum^{\infty}_{n = 0}\frac{2^{2n}(n!)^{2}}{(2n + 1)!}\frac{z^{2n + 1}}{(1 + z^{2})^{n+1}}.
\end{equation}

\N{Euler's Formula for Arctangent}
Following Chien-Lih~\cite{chien-lih2005:elementary}, we compute
\begin{equation}
\int^{1}_{0}\frac{x}{1 + x^{2}t^{2}}\D t = \left.\arctan(xt)\right|^{t=1}_{t=0}
=\arctan(x).
\end{equation}
Substitute $t=\cos(u)$,
\begin{subequations}
\begin{align}
\arctan(x) &= \int^{1}_{0}\frac{x}{1 + x^{2}t^{2}}\D t\\
&= \int^{\pi/2}_{0}\frac{x\sin(u)}{1 + x^{2}\cos^{2}(u)}\D u\\
&=\int^{\pi/2}_{0}\frac{x\sin(u)}{1 + x^{2}}\frac{1}{\left(1 - \frac{x^{2}\sin^{2}(u)}{1 + x^{2}}\right)}\D u.
\end{align}
\end{subequations}
We then Taylor expand the second factor using the geometric series
\begin{equation}
\frac{1}{1 - z} = 1 + z + z^{2} + \dots
\end{equation}
to obtain
\begin{equation}
\arctan(x) = \int^{\pi/2}_{0}\left[\sum^{\infty}_{n=0}\frac{x^{2n+1}}{(1 + x^{2})^{n + 1}}
\sin^{2n+1}(u)\right]\D u.
\end{equation}
Using Lebesgue's monotone convergence theorem for infinite series, we
can swap the sum and integral, obtaining
\begin{equation}
\arctan(x) =\sum^{\infty}_{n=0}
 \left[\int^{\pi/2}_{0}\sin^{2n+1}(u)\D u\right]\frac{x^{2n+1}}{(1 + x^{2})^{n + 1}}
= \sum^{\infty}_{n = 0}\frac{2^{2n}(n!)^{2}}{(2n + 1)!}\frac{x^{2n+1}}{(1 + x^{2})^{n + 1}},
\end{equation}
where we use the fact that
\begin{equation}
\int^{\pi/2}_{0}\sin^{2n+1}(u)\D u = \frac{2^{2n}(n!)^{2}}{(2n + 1)!}.
\end{equation}

\section{Results}

\begin{axiom}[Angle subtraction for cosine]\label{axiom:trigonometry:cos-subtraction-law}
For any $x,y\in\RR$,  $\cos(x-y) = \cos(x)\cos(y) + \sin(x)\sin(y)$.
\end{axiom}

\begin{axiom}\label{axiom:trigonometry:sine-pi-over-2-equals-1}
  $\sin(\pi/2)=1$
\end{axiom}

\begin{axiom}\label{axiom:trigonometry:sine-is-positive-between-0-and-pi-over-2}
  For any $x\in[0,\pi/2]$, $\sin(x)\geq0$.
\end{axiom}

\N{Notation}
For any positive $n\in\NN$ and $x\in\RR$, we will write $\sin^{n}(x)$
for $[\sin(x)]^{n}$, and $\cos^{n}(x)$ for $[\cos(x)]^{n}$. We will use
$\arcsin(x)$ for the inverse function of $\sin(x)$, and $\arccos(x)$ for
the inverse function of $\cos(x)$.

\begin{lemma}\label{lemma:trigonometry:a+b=c-and-b-geq-0-implies-a-geq-c}
  For any $a,b,c\in\RR$,
  if $a+b=c$ and $b\geq0$, then $a\geq c$.
\end{lemma}

\begin{lemma}\label{lemma:trigonometry:cos-zero-geq-one}
  $\cos(0)\geq1$.
\end{lemma}
\begin{proof}
  \begin{calculation}
    \cos(0)
    \step{since $\pi/2 - \pi/2 = 0$}
    \cos\left(\frac{\pi}{2}-\frac{\pi}{2}\right)
    \step{by Axiom~\ref{axiom:trigonometry:cos-subtraction-law}}
    \cos^{2}(\pi/2) + \sin^{2}(\pi/2)
    \step{by Axiom~\ref{axiom:trigonometry:sine-pi-over-2-equals-1} and $1^{2}=1$}
    \cos^{2}(\pi/2) + 1
  \end{calculation}
  We have then $\cos(0) = \cos^{2}(\pi/2) + 1$ and
  $\cos^{2}(\pi/2)\geq0$,
  which implies $\cos(0) \geq 1$ by Lemma~\ref{lemma:trigonometry:a+b=c-and-b-geq-0-implies-a-geq-c}.
\end{proof}

\begin{lemma}\label{lemma:trigonometry:cos-zero-leq-one}
$\cos(0)\leq1$.
\end{lemma}
\begin{proof}
  \begin{calculation}
    \cos(0)
    \step{since $0-0=0$}
    \cos(0-0)
    \step{by Axiom~\ref{axiom:trigonometry:cos-subtraction-law}}
    \cos^{2}(0) + \sin^{2}(0)
  \end{calculation}
  We claim $\cos^{2}(0) + \sin^{2}(0)\geq\cos^{2}(0)$, hence by Lemma~\ref{lemma:trigonometry:a+b=c-and-b-geq-0-implies-a-geq-c},
  \begin{equation}
\cos(0)\geq\cos^{2}(0).
  \end{equation}
  Dividing both sides by $\cos(0)$ gives the result.
\end{proof}

\begin{proposition}\label{prop:trigonometry:cos-of-zero-is-one}
$\cos(0) = 1$
\end{proposition}
\begin{proof}
  From Lemma~\ref{lemma:trigonometry:cos-zero-geq-one} we know
  $\cos(0)\geq1$, and from
  Lemma~\ref{lemma:trigonometry:cos-zero-leq-one} we know
  $\cos(0)\leq1$. So we have $1\leq\cos(0)\leq1$ imply $\cos(0)=1$.
\end{proof}

\begin{lemma}\label{lemma:equality:substitute-summand-for-equal-term}
For any $a,b,c,d\in\RR$, if $a+b=d$ and $b=c$, then $a+c=d$.
\end{lemma}

\begin{lemma}\label{lemma:equality:substitute-left-summand-for-equal-term}
For any $a,b,c,d\in\RR$, if $a+b=d$ and $a=c$, then $c+b=d$.
\end{lemma}

\begin{proposition}\label{prop:cosine-is-even}
For any $x\in\RR$, $\cos(-x) = \cos(x)$.
\end{proposition}
\begin{proof}
Using $-x=0-x$ and Axiom~\ref{axiom:trigonometry:cos-subtraction-law},
\begin{equation}
\cos(-x) = \cos(0 - x) = \cos(0)\cos(x) + \sin(0)\sin(x).
\end{equation}
But we also know that, from $x=x-0$ and Axiom~\ref{axiom:trigonometry:cos-subtraction-law},
\begin{equation}
\cos(x) = \cos(x - 0) = \cos(x)\cos(0) + \sin(x)\sin(0).
\end{equation}
Then using commutativity of multiplication, we find these expressions
are identical.
\begin{calculation}
  \cos(-x)
  \step{since $-x=0-x$}
  \cos(0-x)
  \step{subtraction law for cosine [Axiom~\ref{axiom:trigonometry:cos-subtraction-law}]}
  \cos(0)\cos(x) + \sin(0)\sin(x)
  \step{since $\sin(0)\sin(x)=\sin(x)\sin(0)$ by commutativity, and Lemma~\ref{lemma:equality:substitute-summand-for-equal-term}}
  \cos(0)\cos(x) + \sin(x)\sin(0)
  \step{since $\cos(0)\cos(x)=\cos(x)\cos(0)$ by commutativity, and Lemma~\ref{lemma:equality:substitute-left-summand-for-equal-term}}
  \cos(x)\cos(0) + \sin(x)\sin(0)
  \step{subtraction law for cosine [Axiom~\ref{axiom:trigonometry:cos-subtraction-law}]}
  \cos(x-0)
  \step{since $x-0=x$}
  \cos(x)\qedhere
\end{calculation}
\end{proof}

\begin{proposition}\label{prop:cos-sq-plus-sine-sq-equal-one}
$\cos^{2}(x) + \sin^{2}(x) = 1$.
\end{proposition}
\begin{proof}
  \begin{calculation}
    1
    \step{using Proposition~\ref{prop:trigonometry:cos-of-zero-is-one}}
    \cos(0)
    \step{since $0=x-x$}
    \cos(x-x)
    \step{subtraction law for cosine [Axiom~\ref{axiom:trigonometry:cos-subtraction-law}]}
     \cos^{2}(x) + \sin^{2}(x).\qedhere
  \end{calculation}
\end{proof}

\begin{proposition}\label{prop:sine-leq-one-when-x-leq-pi-over-two}
For any $x\in[0,\pi/2]$ we have $0\leq \sin(x)\leq 1$.
\end{proposition}

\begin{proof}
  Let $x\in[0,\pi/2]$ be arbitrary.

  We know $\sin(x)\geq0$ by Axiom~\ref{axiom:trigonometry:sine-is-positive-between-0-and-pi-over-2}, we just need to
  prove $\sin(x)\leq1$. We will prove $\sin^{2}(x)\leq 1$ (which, since
  $\sin(x)\geq0$, implies the result).

  Observe $\sin(x)^{2} = 1 - \cos(x)^{2}$, and $\cos(x)^{2}\geq 0$
  implies $1-\cos(x)^{2}\leq 1$.
  Hence we have $\sin(x)^{2}\leq 1$ and $0\leq \sin(x)$ give the result.
\end{proof}

\begin{proposition}\label{prop:sine-of-zero}\label{prop:cos-of-half-pi-is-zero}
  (a) $\sin(0) = 0$

  (b) $\cos(\pi/2) = 0$.
\end{proposition}
\begin{proof}
(b) We start with $\cos(\pi/2 - \pi/2) = \cos(\pi/2)^{2} + \sin(\pi/2)^{2}$
by Axiom~\ref{axiom:trigonometry:cos-subtraction-law}. Then:
\begin{calculation}
  \cos(\pi/2 - \pi/2) = \cos(\pi/2)^{2} + \sin(\pi/2)^{2}
\step[\equiv]{since $\cos(0)=1$ by Proposition~\ref{prop:trigonometry:cos-of-zero-is-one}}
  1 = \cos(\pi/2)^{2} + \sin(\pi/2)^{2}
\step[\equiv]{since $\sin(\pi/2)=1$ by Axiom~\ref{axiom:trigonometry:sine-pi-over-2-equals-1}}
  1 = \cos(\pi/2)^{2} + 1
\step[\equiv]{subtract $1$ from both sides}
  0 = \cos(\pi/2)^{2}
\step[\equiv]{since $x^{2}=0$ iff $x=0$}
  0 = \cos(\pi/2)  
\end{calculation}

(a) We start with $C(\pi/2 - 0) = C(\pi/2)C(0) + S(\pi/2)S(0)$
by Axiom~\ref{axiom:trigonometry:cos-subtraction-law}. Then:
\begin{calculation}
  \cos(\pi/2 - 0) = \cos(\pi/2)\cos(0) + \sin(\pi/2)\sin(0)
\step[\equiv]{since $\cos(0)=1$ by Proposition~\ref{prop:trigonometry:cos-of-zero-is-one}}
  \cos(\pi/2 - 0) = \cos(\pi/2) + \sin(\pi/2)\sin(0)
\step[\equiv]{since $\sin(\pi/2)=1$ by Axiom~\ref{axiom:trigonometry:sine-pi-over-2-equals-1}}
  \cos(\pi/2 - 0) = \cos(\pi/2) + \sin(0)
\step[\equiv]{since $\pi/2 - 0=\pi/2$}
  \cos(\pi/2) = \cos(\pi/2) + \sin(0)
\step[\equiv]{subtract $\cos(\pi/2)$ from both sides}
  0 = \sin(0).\qedhere
\end{calculation}
\end{proof}

\begin{proposition}\label{prop:cos-pi-over-2-and-sine}
$\cos(\pi/2 - x) = \sin(x)$ for any $x$.
\end{proposition}
\begin{proof}
Let $x$ be arbitrary.
We have $\cos(\pi/2 - x) = \cos(\pi/2)\cos(x) + \sin(\pi/2)\sin(x)$
by Axiom~\ref{axiom:trigonometry:cos-subtraction-law}. Then:
\begin{calculation}
  \cos(\pi/2 - x) = \cos(\pi/2)\cos(x) + \sin(\pi/2)\sin(x)
\step[\equiv]{by Axiom~\ref{axiom:trigonometry:sine-pi-over-2-equals-1}}
  \cos(\pi/2 - x) = \cos(\pi/2)\cos(x) + 1\sin(x)
\step[\equiv]{since $\cos(\pi/2)=0$ from Proposition~\ref{prop:cos-of-half-pi-is-zero}}
  \cos(\pi/2 - x) = 0\cos(x) + 1\sin(x)
\step[\equiv]{basic algebra}
  \cos(\pi/2 - x) = \sin(x).\qedhere
\end{calculation}
\end{proof}

\begin{proposition}\label{prop:sin-pi-over-2-and-cosine}
$\sin(\pi/2 - x) = \cos(x)$ for any $x$.
\end{proposition}
\begin{proof}
Set $y = \pi/2 - x$, then $\cos(\pi/2 - y) = \sin(y)$ by
Proposition~\ref{prop:cos-pi-over-2-and-sine}, and basic algebra
confirms $x = \pi/2 - y$ which means $\cos(\pi/2 - y) = \cos(x)$,
which gives the result.
\end{proof}

\begin{proposition}\label{prop:sine-pi-over-four-equals-cosine-pi-over-four}
$\sin(\pi/4) = \cos(\pi/4)$.
\end{proposition}
\begin{proof}
Since for any $x$ we have $\cos(\pi/2 - x) = \sin(x)$ by Proposition~\ref{prop:cos-pi-over-2-and-sine}, choose $x=\pi/4$ and we obtain the result.
\end{proof}

\begin{proposition}\label{prop:cosine-is-positive-on-0-to-pi-over-2}
For all $x\in[0,\pi/2]$, we have $0\leq \cos(x)\leq 1$.
\end{proposition}
\begin{proof}
  Let $x\in[0,\pi/2]$ be arbitrary, set $y=\pi/2-x$. Then $y\in[0,\pi/2]$.
  We claim $0\leq\sin(y)\leq 1$, due to Proposition~\ref{prop:sine-leq-one-when-x-leq-pi-over-two}.

  But we know from Proposition~\ref{prop:sin-pi-over-2-and-cosine} that
  $\cos(x)=\sin(y)$, which gives the result.
\end{proof}

\begin{proposition}\label{prop:sine-of-sum}
$\sin(x + y) = \sin(x)\cos(y) + \cos(x)\sin(y)$.
\end{proposition}
\begin{proof}
We have
\begin{calculation}
\sin(x + y)
  \step{since $\sin(u) = \cos((\pi/2) - u)$ using Proposition~\ref{prop:cos-pi-over-2-and-sine}}
\cos((\pi/2)-(x + y))
  \step{algebra}
\cos\left(\bigl((\pi/2)-x\bigr) - y\right)
  \step{angle-addition for cosine}
\cos\left(\frac{\pi}{2} - x\right)\cos(y) + \sin\left(\frac{\pi}{2} - x\right)\sin(y)
  \step{using Propositions~\ref{prop:cos-pi-over-2-and-sine} and \ref{prop:sin-pi-over-2-and-cosine}}
\sin(x)\cos(y) + \cos(x)\sin(y).\qedhere
\end{calculation}
\end{proof}

\begin{proposition}\label{prop:double-angle-for-sine}
$\sin(2x) = 2\sin(x)\cos(x)$.
\end{proposition}

\begin{proof}
  This follows immediately from the previous result, Proposition~\ref{prop:sine-of-sum}, by choosing $x=y$.
\end{proof}

\begin{proposition}
$\sin(\pi) = 0$.
\end{proposition}

\begin{proof}
We find by direct calculation,
\begin{calculation}
  \sin(\pi)
\step{double-angle law for sine from Proposition~\ref{prop:double-angle-for-sine}}
  2\sin(\pi/2)\cos(\pi/2)
\step{since $\cos(\pi/2)=0$ by Proposition~\ref{prop:cos-of-half-pi-is-zero}}
  0.\qedhere
\end{calculation}
\end{proof}

\begin{proposition}
$2\bigl(\cos(\pi/4)\bigr)^{2} = 2\bigl(\sin(\pi/4)\bigr)^{2} = 1$.
\end{proposition}
\begin{proof}
We know from Proposition~\ref{prop:sine-pi-over-four-equals-cosine-pi-over-four}
that $\cos(\pi/4)=\sin(\pi/4)$. We know from Axiom~\ref{axiom:trigonometry:sine-pi-over-2-equals-1} that $\sin(\pi/2)=1$.
When we combine this knowledge with the double-angle identity for Sine
(a.k.a., Proposition~\ref{prop:double-angle-for-sine}) we find
\begin{calculation}
  1
\step{Axiom~\ref{axiom:trigonometry:sine-pi-over-2-equals-1}}
  \sin(\pi/2)
\step{since $\pi/2=2(\pi/4)$ and double-angle identity for sine Proposition~\ref{prop:double-angle-for-sine}}
  2\sin(\pi/4)\cos(\pi/4)
\step{by Proposition~\ref{prop:sine-pi-over-four-equals-cosine-pi-over-four}}
  2\sin^{2}(\pi/4).\qedhere
\end{calculation}
\end{proof}

\begin{proposition}\label{prop:sine-of-pi-over-four}\label{prop:cosine-of-pi-over-four}
$\cos(\pi/4) = \sin(\pi/4) = \sqrt{2}/2$.
\end{proposition}

\begin{proof}
Since $\pi/4\in[0,\pi/2)$, we know $\sin(\pi/4)\geq 0$ by Axiom~\ref{axiom:trigonometry:sine-is-positive-between-0-and-pi-over-2}. Combined with the
  previous proposition, we have $\sin(\pi/4) = \sqrt{2}/2$.

Similarly, on that interval, we know from Proposition~\ref{prop:cosine-is-positive-on-0-to-pi-over-2}
that $\cos(\pi/4)$ is positive. Combined with the previous result, we have
proven the claim.
\end{proof}

\begin{proposition}\label{prop:sine-of-minus-pi-over-four}
$\sin(-\pi/4) = -\sin(\pi/4) = -\sqrt{2}/2$.
\end{proposition}
\begin{proof}
By direct calculation:
\begin{calculation}
0
\step{since $\sin(0)=0$ from Proposition~\ref{prop:sine-of-zero}}
\sin(0)
\step{since $0 = (\pi/4) + (-\pi/4)$}
\sin\bigl((\pi/4) + (-\pi/4)\bigr)
\step{angle-addition for sine Proposition~\ref{prop:sine-of-sum}}
\sin(\pi/4)\cos(-\pi/4) + \sin(-\pi/4)\cos(\pi/4)
\step{since $\cos(-x)=\cos(x)$ by Proposition~\ref{prop:cosine-is-even}}
\sin(\pi/4)\cos(\pi/4) + \sin(-\pi/4)\cos(\pi/4)
\step{distributivity}
\bigl(\sin(\pi/4) + \sin(-\pi/4)\bigr)\cos(\pi/4).
\end{calculation}
Since $\cos(\pi/4)\neq0$ by Proposition~\ref{prop:cosine-of-pi-over-four}, we must have $\sin(\pi/4) + \sin(-\pi/4)=0$
which implies the result.
\end{proof}

\begin{proposition}\label{prop:sine-half-pi-is-minus-one}
$-\sin(\pi/2)=-1$ and $\sin(-\pi/2)=-1$.
\end{proposition}
\begin{proof}
(a) We use the double-angle identity for
sine (a.k.a., Proposition~\ref{prop:double-angle-for-sine})
\begin{equation}
\sin(\pi/2) = 2\sin(\pi/4)\cos(\pi/4).
\end{equation}
Then we use Proposition~\ref{prop:sine-of-pi-over-four} to rewrite the
right-hand side as
\begin{equation}
\sin(\pi/2) = 2(\sqrt{2}/2)^{2} = 2(2/4) = 1.
\end{equation}
Thus we have the result $-\sin(\pi/2)=-1$.

(b) We likewise use the double-angle identity for sine
\begin{calculation}
\sin(-\pi/2)
\step{basic algebra}
\sin\bigl(2(-\pi/4)\bigr)
\step{double-angle law for sine from Proposition~\ref{prop:double-angle-for-sine}}
2\sin(-\pi/4)\cos(-\pi/4)
\step{cosine is even by Proposition~\ref{prop:cosine-is-even}}
2\sin(-\pi/4)\cos(\pi/4)
\step{$\cos(\pi/4)=\sqrt{2}/2$ from Proposition~\ref{prop:cosine-of-pi-over-four}}
\sqrt{2}\sin(-\pi/4)
\step{since $\sin(-\pi/4)=-\sqrt{2}/2$ by Proposition~\ref{prop:sine-of-minus-pi-over-four}}
\sqrt{2}(-\sqrt{2}/2)
\step{algebra}
-1.\qedhere
\end{calculation}
\end{proof}

\begin{proposition}\label{prop:cos-pi-is-minus-one}
$\cos(\pi)=-1$.
\end{proposition}
\begin{proof}
  By direct calculation
\begin{calculation}
\cos(\pi)
\step{since $\pi = (\pi/2) - (-\pi/2)$}
\cos\bigl(\pi/2 - (-\pi/2)\bigr)
\step{using Axiom~\ref{axiom:trigonometry:cos-subtraction-law}}
\cos(\pi/2)\cos(-\pi/2) + \sin(\pi/2)\sin(-\pi/2)
\step{$\sin(\pi/2)=1$ and $\sin(-\pi/2)=-1$ by Proposition~\ref{prop:sine-half-pi-is-minus-one}}
\cos(\pi/2)\cos(-\pi/2) - 1
\step{since cosine is even by Proposition~\ref{prop:cosine-is-even}}
\cos^{2}(\pi/2) - 1
\step{since $\cos(\pi/2)=0$ from Proposition~\ref{prop:cos-of-half-pi-is-zero}}
0 - 1
\step{algebra}
-1.\qedhere
\end{calculation}
\end{proof}

\begin{proposition}\label{prop:sine-is-odd-function}
$\sin(-x) = -\sin(x)$ for any $x$.
\end{proposition}
\begin{proof}
By direct calculation:
  \begin{calculation}
  \sin(-x)
\step{since $\cos(\pi/2 - u)=\sin(u)$ from Proposition~\ref{prop:cos-pi-over-2-and-sine}}
  \cos\bigl((\pi/2) - (-x)\bigr)
\step{algebra}
  \cos\bigl((\pi/2) + x\bigr)
\step{algebra}
  \cos\bigl(\pi - ((\pi/2) - x)\bigr)
\step{by Axiom~\ref{axiom:trigonometry:cos-subtraction-law}}
  \cos(\pi)\cos\bigl((\pi/2) - x\bigr) + \sin(\pi)\sin\bigl((\pi/2) - x\bigr) 
\step{since $\cos(\pi)=-1$ by Proposition~\ref{prop:cos-pi-is-minus-one}}
  -\cos\bigl((\pi/2) - x\bigr) + \sin(\pi)\sin\bigl((\pi/2) - x\bigr) 
\step{since $\sin(\pi)=0$ from Proposition~\ref{prop:sine-of-zero}}
-\cos\bigl((\pi/2) - x\bigr) + 0
\step{since $\cos(\pi/2 - x) = \sin(x)$ from Proposition~\ref{prop:cos-pi-over-2-and-sine}}
-\sin(x).\qedhere
\end{calculation}
\end{proof}

\begin{proposition}\label{prop:trigonometry:cos-addition-law}
$cos(x+y) = \cos(x)\cos(y) - \sin(x)\sin(y)$ for any $x$, $y$.
\end{proposition}
\begin{proof}
Let $x$, $y$ be arbitrary.
\begin{calculation}
  \cos(x + y)
\step{since $y = -(-y)$}
\cos(x - (-y))
\step{by Axiom~\ref{axiom:trigonometry:cos-subtraction-law}}
\cos(x)\cos(-y)+\sin(x)\sin(-y)
\step{by previous result, Proposition~\ref{prop:sine-is-odd-function}}
\cos(x)\cos(-y)-\sin(x)\sin(y)
\step{since $\cos(-x)=\cos(x)$ from Proposition~\ref{prop:cosine-is-even}}
\cos(x)\cos(y)-\sin(x)\sin(y).\qedhere
\end{calculation}
\end{proof}

\begin{proposition}\label{prop:trigonometry:cosine-double-angle-law}
$\cos(2x) = \cos^{2}(x) - \sin^{2}(x) = 2\cos^{2}(x) - 1$ for any $x$.
\end{proposition}

\begin{proof}
Let $x$ be arbitrary.
  First recall Proposition~\ref{prop:cos-sq-plus-sine-sq-equal-one},
$\cos^{2}(x)+\sin^{2}(x)=1$ implies
$\sin^{2}(x)=1-\cos^{2}(x)$. Therefore we find
\begin{equation}
\cos^{2}(x) - \sin^{2}(x) = 2\cos^{2}(x) - 1.
\end{equation}
Now, by direct calculation
\begin{calculation}
\cos(2x)
\step{algebra}
\cos(x + x)
\step{cosine addition law from Proposition~\ref{prop:trigonometry:cos-addition-law}}
\cos^{2}(x) - \sin^{2}(x).\qedhere
\end{calculation}
\end{proof}

\begin{proposition}[Half-angle formula for cosine]\label{prop:trigonometry:half-angle-formula-for-cosine}
For any $x\in\RR$, $\cos^{2}(x/2) = (1 + \cos(x))/2$.
\end{proposition}

\begin{proof}
  By direct calculation
\begin{calculation}
  \cos(x)
  \step{since $x = 2(x/2)$}
  \cos\bigl(2(x/2)\bigr)
  \step{double-angle law from Proposition~\ref{prop:trigonometry:cosine-double-angle-law}}
  2\cos^{2}(x/2) - 1.
\end{calculation}
Hence the result follows by adding $1$ to both sides.
\end{proof}

\begin{proposition}[Half-angle formula for sine]\label{prop:trigonometry:half-angle-formula-for-sine}
For any $x\in\RR$, $\sin^{2}(x/2) = (1 - \cos(x))/2$.
\end{proposition}

\begin{proof}
We use Proposition~\ref{prop:cos-sq-plus-sine-sq-equal-one},
\begin{equation}
\sin^{2}(x/2) + \cos^{2}(x/2) = 1,
\end{equation}
then rewrite $\cos^{2}(x/2)$ using the half-angle formula from Proposition~\ref{prop:trigonometry:half-angle-formula-for-cosine}
\begin{equation}
\sin^{2}(x/2) + \frac{1 + \cos(x)}{2} = 1.
\end{equation}
Rearranging terms gives the result.
\end{proof}

\begin{proposition}
$\sin(x-y) = \sin(x)\cos(y) - \cos(x)\sin(y)$ for any $x$, $y$.
\end{proposition}

\begin{proof}
This follows from Propositions~\ref{prop:sine-of-sum} and \ref{prop:sine-is-odd-function}.
\end{proof}

\section{Periodicity}

The remaining results follow from the angle-addition identities and the
knowledge that $\sin(\pi/2)=1$, $\cos(\pi/2)=0$, $\sin(-x)=-\sin(x)$, $\cos(-x)=\cos(x)$.

\begin{proposition}
$\sin(\pi/2 + x) = \cos(x)$.
\end{proposition}

\begin{proposition}
$\cos(\pi/2 + x) = -\sin(x)$.
\end{proposition}

The remaining results follow from angle-addition identities.

\begin{proposition}
$\sin(\pi + x) = -\sin(x)$.
\end{proposition}

\begin{proposition}
$\cos(\pi + x) = -\cos(x)$.
\end{proposition}

\begin{proposition}
$\sin(2\pi + x) = \sin(x)$.
\end{proposition}

\begin{proposition}
$\cos(2\pi + x) = \cos(x)$.
\end{proposition}

\section{Exact Values}

We have show in Proposition~\ref{prop:sine-half-pi-is-minus-one} $\sin(-\pi/2)=-1$,
and in Proposition~\ref{prop:sine-of-pi-over-four} $\sin(\pi/4)=\cos(\pi/4)=\sqrt{2}/2$.

\begin{proposition}
$\displaystyle{\sin(\pi/8) = \frac{1}{2}\sqrt{2 - \sqrt{2}}}$
\end{proposition}

\begin{proof}
Using half-angle formula for sine,
\begin{calculation}
  \sin^{2}(\pi/8)
\step{half-angle formula from Proposition~\ref{prop:trigonometry:half-angle-formula-for-sine}}
  \frac{1 - \cos(\pi/4)}{2}
\step{since $\cos(\pi/4)=\sqrt{2}/2$ from Proposition~\ref{prop:cosine-of-pi-over-four}}
  \frac{1 - \frac{1}{2}\sqrt{2}}{2}
\step{algebra}
  \frac{2 - \sqrt{2}}{4}.
\end{calculation}
We know $0\leq\pi/8\leq\pi/2$, so $\sin(\pi/8)\geq 0$ and then the
square root gives the result.
\end{proof}

\begin{proposition}\label{prop:cosine-of-pi-over-eight}
$\displaystyle{\cos(\pi/8) = \frac{1}{2}\sqrt{2 + \sqrt{2}}}$
\end{proposition}

\begin{proof}
Using half-angle formula for cosine,
\begin{calculation}
  \cos^{2}(\pi/8)
\step{half-angle formula from Proposition~\ref{prop:trigonometry:half-angle-formula-for-cosine}}
  \frac{1 + \cos(\pi/4)}{2}
\step{since $\cos(\pi/4)=\sqrt{2}/2$ from Proposition~\ref{prop:cosine-of-pi-over-four}}
  \frac{1 + \frac{1}{2}\sqrt{2}}{2}
\step{algebra}
  \frac{2 + \sqrt{2}}{4}.
\end{calculation}
We know $0\leq\pi/8\leq\pi/2$, so $\cos(\pi/8)\geq 0$ and then the
square root gives the result.
\end{proof}

\begin{proposition}
$\displaystyle{\cos(\pi/16) = \frac{1}{2}\sqrt{2 + \sqrt{2 + \sqrt{2}}}}$.
\end{proposition}

\begin{proof}
Using half-angle formula for cosine,
\begin{calculation}
  \cos^{2}(\pi/16)
\step{half-angle formula from Proposition~\ref{prop:trigonometry:half-angle-formula-for-cosine}}
  \frac{1 + \cos(\pi/8)}{2}
\step{since $\cos(\pi/8)=\sqrt{2 + \sqrt{2}}/2$ from Proposition~\ref{prop:cosine-of-pi-over-eight}}
  \frac{1 + \frac{1}{2}\sqrt{2 + \sqrt{2}}}{2}
\step{algebra}
  \frac{2 + \sqrt{2 + \sqrt{2}}}{4}.
\end{calculation}
We know $0\leq\pi/16\leq\pi/2$, so $\cos(\pi/16)\geq 0$ and then the
square root gives the result.
\end{proof}

\begin{remark}
Numerically, $\cos(\pi/16)\approx0.980785$.
\end{remark}

\begin{proposition}
$\displaystyle{\sin(\pi/16) = \frac{1}{2}\sqrt{2 - \sqrt{2 + \sqrt{2}}}}$.
\end{proposition}

\begin{proof}
Using half-angle formula for cosine,
\begin{calculation}
  \sin^{2}(\pi/16)
\step{half-angle formula from Proposition~\ref{prop:trigonometry:half-angle-formula-for-sine}}
  \frac{1 - \cos(\pi/8)}{2}
\step{since $\cos(\pi/8)=\sqrt{2 + \sqrt{2}}/2$ from Proposition~\ref{prop:cosine-of-pi-over-eight}}
  \frac{1 - \frac{1}{2}\sqrt{2 + \sqrt{2}}}{2}
\step{algebra}
  \frac{2 - \sqrt{2 + \sqrt{2}}}{4}.
\end{calculation}
We know $0\leq\pi/16\leq\pi/2$, so $\cos(\pi/16)\geq 0$ and then the
square root gives the result.
\end{proof}

\begin{remark}
Numerically, $\sin(\pi/16)\approx0.1950903$.
\end{remark}

\M
More generally, for $n\in\NN$, $n>1$, we have
\begin{subequations}
\begin{align}
\cos(\pi/2^{n+1}) &= \frac{\sqrt{2 + 2\cos(\pi/2^{n})}}{2}\\
\intertext{and}
\sin(\pi/2^{n+1}) &= \frac{\sqrt{2 - 2\cos(\pi/2^{n})}}{2}.
\end{align}
\end{subequations}

\section{Inequalities}

\begin{proposition}
Let $k$ be a constant. If $\cos(x)\leq 1 - k$, then $\cos(x/2)\leq 1 - k/2$.
\end{proposition}

\begin{proof}
Since $0\leq \cos(x/2)\leq 1$, we have $\cos(x/2)\geq\cos^{2}(x/2)$ and
\begin{calculation}
\cos^{2}(x/2)
\step{half-angle formula from Proposition~\ref{prop:trigonometry:half-angle-formula-for-cosine}}
\frac{1  +\cos(x)}{2}
\step[\leq]{hypothesis}
\frac{1 + 1 - k}{2},
\end{calculation}
hence the result.
\end{proof}

\begin{proposition}
For $0\leq x\leq \pi/4$, we have $\cos(x)\geq\sin(x)$.
\end{proposition}

\begin{proof}
This is because cosine is decreasing on this interval and $1\geq\cos(x)\geq\sqrt{2}/2$
and since is increasing on this interval and $\sqrt{2}/2\geq\sin(x)\geq0$.
\end{proof}

\begin{proposition}
Let $0\leq x\leq\pi/2$.
Then $\sin(x)\geq x\cos(x)$ if and only if $\sin(x)\geq x/\sqrt{1 + x^{2}}$.
\end{proposition}

\begin{proof}
Observe on this domain $\sin(x)\geq0$ and $\cos(x)\geq0$.
\begin{calculation}
\sin(x)\geq x\cos(x)
\step[\equiv]{square both sides}
\sin^{2}(x)\geq x^{2}\cos^{2}(x)
\step[\equiv]{using $\sin^{2}(x)+\cos^{2}(x)=1$}
\sin^{2}(x)\geq x^{2}(1 - \sin^{2}(x))
\step[\equiv]{adding $x^{2}\sin^{2}(x)$ to both sides}
(1 + x^{2})\sin^{2}(x)\geq x^{2}
\step[\equiv]{dividing $1 + x^{2}$ to both sides}
\sin^{2}(x)\geq\frac{x^{2}}{1 + x^{2}}
\step[\equiv]{take square root of both sides}
\sin(x)\geq\frac{x}{\sqrt{1 + x^{2}}}.\qedhere
\end{calculation}
\end{proof}

\begin{proposition}
Let $0\leq x\leq\pi/2$.
Then $\cos(x)\leq 1/\sqrt{1 + x^{2}}$ if and only if $\sin(x)\geq x/\sqrt{1 + x^{2}}$.
\end{proposition}

\begin{proof}
We see by direct calculation
\begin{calculation}
\sin(x)\geq x/\sqrt{1 + x^{2}}
\step[\equiv]{square both sides}
\sin^{2}(x)\geq\frac{x^{2}}{1 + x^{2}}
\step[\equiv]{since $\sin^{2}(x) = 1 - \cos^{2}(x)$}
1 - \cos^{2}(x)\geq\frac{x^{2}}{1 + x^{2}}
\step[\equiv]{add $\cos^{2}(x)$ to both sides}
1\geq\frac{x^{2}}{1 + x^{2}} + \cos^{2}(x)
\step[\equiv]{subtract $x^{2}/(1 + x^{2})$ from both sides}
1 - \frac{x^{2}}{1 + x^{2}}\geq\cos^{2}(x)
\step[\equiv]{algebra}
\frac{1}{1 + x^{2}}\geq\cos^{2}(x)
\step[\equiv]{take square root of both sides}
\frac{1}{\sqrt{1 + x^{2}}}\geq\cos(x).\qedhere
\end{calculation}
\end{proof}

\begin{proposition}
  Let $0\leq x\leq\pi/16$. Then
  \begin{equation}
1 - \frac{1}{2}x^{2} - \frac{1}{200} \leq\frac{1}{\sqrt{1 + x^{2}}}\leq1 - \frac{1}{2}x^{2} + \frac{1}{200}.
  \end{equation}
\end{proposition}

\begin{remark}
This requires the Binomial series expansion and an upper bound on the
Lagrange form of the remainder term.
\end{remark}

\begin{proposition}
Let $0\leq x\leq\pi/16$.
If $\cos(x)\leq 1 - \frac{1}{2}x^{2} - \frac{1}{200}$, then $\cos(x)\leq 1/\sqrt{1 + x^{2}}$.
\end{proposition}

\begin{lemma}
Let $x\in\RR$, $n\in\NN$ be such that $n\geq2$ and $0<x<nx\leq1$.
Let $f(x) = \sin(x)/x$.
Then for $k\in\NN$, $1\leq k < n$, we have $f(kx) < f((k+1)x)$.
\end{lemma}

\begin{proof}
Induction on $k$.

Base case ($k=1$). We find by direct calculation
\begin{calculation}
f(2x)
\step{unfold $f(x)=\sin(x)/x$}
\frac{\sin(2x)}{2x}
\step{angle-addition for sine}
\frac{2\sin(x)\cos(x)}{2x}
\step{folding $f(x)=\sin(x)/x$}
f(x)\cos(x)
\step[<]{since $\cos(x)<1$ on $0<x<1$}
f(x).
\end{calculation}

Inductive hypothesis: assume
\begin{equation}
f(kx) < f((k-1)x).
\end{equation}
We will note the angle-subtraction law for sine gives us
\begin{equation}
\sin(kx - x) = \sin(kx)\cos(x) - \sin(x)\cos(kx).
\end{equation}

\textsc{Claim 1:} $(k-1)\sin(kx) < k\sin(kx)\cos(x) - k\sin(x)\cos(kx).$
Now by direct calculation, we obtain the inequality from multiplying
both sides of the inductive hypothesis by $k(k-1)x$,
\begin{calculation}
k(k-1)xf(kx) < k(k-1)xf((k-1)x)
\step[\equiv]{unfolding $f(x)$}
(k-1)\sin(kx) < k\sin((k-1)x)
\step[\equiv]{from angle-subtraction law for sine}
(k-1)\sin(kx) < k\sin(kx)\cos(x) - k\sin(x)\cos(kx).
\end{calculation}
This proves the claim.

Now we see
\begin{calculation}
k\sin(kx)\cos(x) - \sin(kx)
\step[\leq]{since $0<\cos(x)<1$ here}
k\sin(kx) - \sin(kx)
\step[<]{using claim 1}
k\sin(kx)\cos(x) - k\sin(x)\cos(kx)
\step[\leq]{since $0\leq\cos(x)\leq1$}
k\sin(kx) - k\cos(kx)\sin(x).
\end{calculation}
Hence
\begin{calculation}
k\sin((k+1)x)
\step{angle-addition for sine}
k\sin(kx)\cos(x) + k\cos(kx)\sin(x)
\step[<]{from the previous set of calculations}
k\sin(kx) + \sin(kx).
\end{calculation}
Hence the result by dividing both sides by $k(k+1)x$.
\end{proof}

\begin{definition}
We define the \define{(Unnormalized) Sinc Function} as
\begin{equation}
\sinc(x) := \frac{\sin(x)}{x}.
\end{equation}
\end{definition}

\begin{remark}
For any $n\in\NN$, we will write $\sinc^{n}(x)$ as shorthand for $(\sinc(x))^{n}$.
\end{remark}

\begin{proposition}
Let $x\in\RR$.
Then
\begin{equation}
\sinc^{2}(x/2) = \frac{2}{1 + \cos(x)}\sinc^{2}(x).
\end{equation}
\end{proposition}

\begin{proof}
We see that
%% \begin{equation}
%% \sin^{2}(x/2) = \frac{1 - \cos(x)}{2}.
%% \end{equation}
%% We also see that
%% \begin{equation}
%% \sinc^{2}(x) = \frac{1 - \cos^{2}(x)}{x^{2}} = \frac{(1 - \cos(x))(1 + \cos(x))}{x^{2}}.
%% \end{equation}
%% Then we have
\begin{calculation}
  \sinc^{2}(x/2)
  \step{unfolding definition of sinc}
  \frac{\sin^{2}(x/2)}{(x/2)^{2}}
  \step{using the half-angle identity for $\sin^{2}(x/2)$}
  \frac{1 - \cos(x)}{2}\frac{1}{(x/2)^{2}}
  \step{multiply by $1$}
  \frac{1+\cos(x)}{1+\cos(x)}\frac{1 - \cos(x)}{2}\frac{1}{(x/2)^{2}}
  \step{algebra}
  \frac{1}{1 + \cos(x)}\frac{1}{2}\frac{1 - \cos^{2}(x)}{x^{2}/4}
  \step{rewriting}
  \frac{2}{1 + \cos(x)}\sinc^{2}(x).\qedhere
\end{calculation}
\end{proof}

\section{Approximations}

\M
When doing calculus, we want to approximate $\sin(\varepsilon)$ and
$\cos(\varepsilon)$ for $0<\varepsilon\ll1$. We hypothesize
\begin{equation}
\sin(\varepsilon) \approx \sin(0) + a_{1} \varepsilon + a_{2}\varepsilon^{2}
\end{equation}
and
\begin{equation}
\cos(\varepsilon) \approx \cos(0) + b_{1}\varepsilon + b_{2}\varepsilon^{2}
\end{equation}
where we just have to determine coefficients $a_{1}$, $a_{2}$, $b_{1}$,
$b_{2}$.
Since $\sin(-\varepsilon)=-\sin(\varepsilon)$, we expect $a_{2}=0$.

We have
\begin{equation}
\sin^{2}(\varepsilon)\approx \varepsilon^{2}a_{1}^{2}
\end{equation}
and
\begin{equation}
\cos^{2}(\varepsilon) \approx 1 + 2b_{1}\varepsilon + (b_{1}^{2} + 2 b_{2})\varepsilon^{2}.
\end{equation}
For
\begin{equation}
\cos^{2}(\varepsilon) + \sin^{2}(\varepsilon) = 1,
\end{equation}
we need
\begin{subequations}
\begin{align}
2b_{1} &= 0\\
a_{1}^{2} + b_{1}^{2} + 2b_{2} &= 0
\end{align}
\end{subequations}
For us to also have the angle-addition law for cosine, we have
\begin{equation}
\begin{split}
\cos(\varepsilon+\varepsilon)&=\cos^{2}(\varepsilon) - \sin^{2}(\varepsilon)\\
&\approx 1 + (2b_{2}-a_{1}^{2})\varepsilon^{2}\\
&=1 + b_{2}(2\varepsilon)^{2}.
\end{split}
\end{equation}
Hence we have another equation determining the coefficients
\begin{equation}
2b_{2} - a_{1}^{2} = 4b_{2}\implies b_{2} = \frac{-a_{1}^{2}}{2}.
\end{equation}
So we have
\begin{equation}
\cos(\varepsilon)\approx 1 - \frac{a_{1}^{2}}{2}\varepsilon^{2}.
\end{equation}

\N{Cubic Approximation}
If we extend our approximation to
\begin{subequations}
\begin{align}
\sin(\varepsilon) &\approx a_{1}\varepsilon + a_{3}\varepsilon^{3}\\
\cos(\varepsilon) &\approx 1 - \frac{a_{1}^{2}}{2}\varepsilon^{2} + b_{3}\varepsilon^{3},
\end{align}
\end{subequations}
then
\begin{equation}
\cos^{2}(\varepsilon) - \sin^{2}(\varepsilon)\approx 1 - 2a_{1}^{2}\varepsilon
+ 2b_{3}\varepsilon^{3}.
\end{equation}
Comparing coefficients to
\begin{equation}
\cos(2\varepsilon)\approx 1 - 2a_{1}^{2}\varepsilon^{2} + 8b_{3}\varepsilon^{3}
\end{equation}
gives us
\begin{equation}
2b_{3} = 8b_{3}\implies b_{3}=0.
\end{equation}
Angle-addition for sine gives us
\begin{equation}
2\cos(\varepsilon)\sin(\varepsilon)\approx 2a_{1}\varepsilon +
(-a_{1}^{3} + 2a_{3})\varepsilon^{3},
\end{equation}
comparing coefficients to
\begin{equation}
\sin(2\varepsilon)\approx 2a_{1}\varepsilon + 8a_{3}\varepsilon^{3}
\end{equation}
gives us
\begin{equation}
-a_{1}^{3} + 2a_{3} = 8a_{3}\implies a_{3} = \frac{-a_{1}^{3}}{6}.
\end{equation}
If we continue in this manner, we would recover the Taylor expansion for
sine and cosine, but with $a_{1}$ left to be determined.

\N{Infinitesimal}
Now, we recall
\begin{equation}
(\cos(x) + \I\sin(x))^{n} = \cos(nx) + \I\sin(nx),
\end{equation}
then for infinitesimals we have
\begin{equation}
(\cos(\varepsilon) + \I\sin(\varepsilon))^{n}\approx\cos(n\varepsilon)
+\I\sin(n\varepsilon).
\end{equation}
Taking $x=n\varepsilon$ and our linear approximations above, we find
\begin{equation}
(\cos(\varepsilon) + \I\sin(\varepsilon))^{n} = (\cos(x/n) + \I\sin(x/n))^{n}
\approx \left(1 + \frac{\I a_{1} x}{n}\right)^{n}.
\end{equation}
Taking $n\to\infty$ gives us the first step towards Euler's identity
\begin{equation}
\cos(x) + \I\sin(x) = \exp(\I a_{1}x).
\end{equation}

\N{TODO: Find Inequality}
For $0 < x < \pi/2$, I need to prove
\begin{equation}
x\geq \sin(x)\geq x\cos(x).
\end{equation}
We know $\sin^{2}(x) < \cos^{2}(x)$ on this domain, so $\sin(x) < \cos(x)$.
The difficulty is to show $\sin^{2}(x) < x^{2}\cos^{2}(x)$.
$\sin^{2}(x) < x^{2}(1 - \sin^{2}(x))$
$(1 + x^{2})\sin^{2}(x) < x^{2}$.

On $0 < x < \pi/2$, we have $\sin(x) < x$ and $\cos(x) > 1 - x$.
But can we prove it? We can prove if for any constant $k$ such that
$\cos(x)\leq 1 - k$, then $\cos(x/2)\leq 1 - (k/2)$ since
$\cos(x/2)\leq\cos^{2}(x/2) = (1 + \cos(x))/2 \leq (2 + k)/2$.
Similarly, if $\cos(x)\leq 1 - k$, then $\sin(x/2)\geq k/2$ since
$\sin(x/2)\geq\sin^{2}(x/2) = (1 - \cos(x))/2\geq (1 - 1 + k)/2 = k/2$.

We see $\sin(\pi/4 - y) = 2^{-1/2}(\cos(y) - \sin(y))$ and
$\cos(\pi/4 - y) = 2^{-1/2}(\cos(y) + \sin(y))$ for any $0<y<\pi/4$.
Since sine is increasing on the interval $[0,\pi/2]$, it follows that
cosine is decreasing on the same interval (since
$\sin(\pi + x) = \cos(x + \pi/2) = -\sin(x)$) and therefore
\begin{equation}
\cos(x)\geq\cos(\pi/4) = \sin(\pi/4)\geq\sin(x)
\end{equation}
for all $0\leq x\leq\pi/4$.

We also know
\begin{subequations}
\begin{align}
\sin((\pi/4) - \varepsilon) &= 2^{-1/2}(\cos(\varepsilon) - \sin(\varepsilon))\\
\intertext{and}
\cos((\pi/4) - \varepsilon) &= 2^{-1/2}(\cos(\varepsilon) + \sin(\varepsilon))
\end{align}
\end{subequations}
for any $0 < \varepsilon < \pi/4$.
%% From
%% \begin{equation}
%% \sin(2\varepsilon)=2\sin(\epsilon)\cos(\epsilon)
%% \end{equation}
%% we obtain
%% \begin{equation}
%% 2a_{1}\varepsilon + 4a_{2}\varepsilon^{2}\approx 2a_{1}\varepsilon + (a_{2}+a_{1}b_{1})\varepsilon^{2}.
%% \end{equation}