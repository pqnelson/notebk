\chapter{Trigonometry}

\M I am curious about symbolically defining trigonometric
functions.\footnote{We can consider a different approach using the
binomial theorem and infinitesimals, following Euler; see McKinzie and
Tuckey's ``Higher Trigonometry, Hyperreal Numbers, and Euler’s Analysis of Infinities''~\cite{mckinzie}.} It
suffices to consider defining sine and cosine by certain defining
properties. What are they?

We can axiomatically characterize cosine and sine using the following
properties:
\begin{enumerate}
\item Angle subtraction for cosine: $\cos(x-y) = \cos(x)\cos(y) + \sin(x)\sin(y)$ for all $x\in\RR$, $y\in\RR$
\item $\sin(\pi/2)=1$
\item for any $x\in[0,\pi/2]$, $\sin(x)\geq0$.
\end{enumerate}
In fact, we have a theorem.

\begin{theorem}[{Robinson~\cite{robinson}}]
Let $p\in\RR$ be a positive real number (fixed). Let $C\colon\RR\to\RR$
and $S\colon\RR\to\RR$ be continuous functions such that
\begin{enumerate}
\item $C(x-y) = C(x)C(y) + S(x)S(y)$ for all $x$, $y\in\RR$
\item $S(p)=1$
\item for any $x\in[0,p]$, $S(x)\geq0$.
\end{enumerate}
Then $C$ and $S$ exist and are unique.
\end{theorem}

Here $p$ generalizes the constant $\pi/2$. If we let $p\in\RR$ be fixed
but arbitrary, nothing changes: for negative $p$, the sign will change
for $S(x)$; for $p=0$, we will find $C(x)=S(x)=0$. So the restriction
that $p>0$ is inconsequential book-keeping.

Consequently, we will abbreviate $C(x)=\cos(x)$ and $S(x)=\sin(x)$ in
the following results.

\section{Results}

\begin{axiom}[Angle subtraction for cosine]\label{axiom:trigonometry:cos-addition-law}
For any $x,y\in\RR$,  $\cos(x-y) = \cos(x)\cos(y) + \sin(x)\sin(y)$.
\end{axiom}

\begin{axiom}\label{axiom:trigonometry:sine-pi-over-2-equals-1}
  $\sin(\pi/2)=1$
\end{axiom}

\begin{axiom}\label{axiom:trigonometry:sine-is-positive-between-0-and-pi-over-2}
  For any $x\in[0,\pi/2]$, $\sin(x)\geq0$.
\end{axiom}

\N{Notation}
For any positive $n\in\NN$ and $x\in\RR$, we will write $\sin^{n}(x)$
for $[\sin(x)]^{n}$, and $\cos^{n}(x)$ for $[\cos(x)]^{n}$. We will use
$\arcsin(x)$ for the inverse function of $\sin(x)$, and $\arccos(x)$ for
the inverse function of $\cos(x)$.

\begin{lemma}\label{lemma:trigonometry:a+b=c-and-b-geq-0-implies-a-geq-c}
  For any $a,b,c\in\RR$,
  if $a+b=c$ and $b\geq0$, then $a\geq c$.
\end{lemma}

\begin{lemma}\label{lemma:trigonometry:cos-zero-geq-one}
  $\cos(0)\geq1$.
\end{lemma}
\begin{proof}
  \begin{calculation}
    \cos(0)
    \step{since $\pi/2 - \pi/2 = 0$}
    \cos\left(\frac{\pi}{2}-\frac{\pi}{2}\right)
    \step{by Axiom~\ref{axiom:trigonometry:cos-addition-law}}
    \cos^{2}(\pi/2) + \sin^{2}(\pi/2)
    \step{by Axiom~\ref{axiom:trigonometry:sine-pi-over-2-equals-1} and $1^{2}=1$}
    \cos^{2}(\pi/2) + 1
  \end{calculation}
  We have then $\cos(0) = \cos^{2}(\pi/2) + 1$ and
  $\cos^{2}(\pi/2)\geq0$,
  which implies $\cos(0) \geq 1$ by Lemma~\ref{lemma:trigonometry:a+b=c-and-b-geq-0-implies-a-geq-c}.
\end{proof}

\begin{lemma}\label{lemma:trigonometry:cos-zero-leq-one}
$\cos(0)\leq1$.
\end{lemma}
\begin{proof}
  \begin{calculation}
    \cos(0)
    \step{since $0-0=0$}
    \cos(0-0)
    \step{by Axiom~\ref{axiom:trigonometry:cos-addition-law}}
    \cos^{2}(0) + \sin^{2}(0)
  \end{calculation}
  We claim $\cos^{2}(0) + \sin^{2}(0)\geq\cos^{2}(0)$, hence by Lemma~\ref{lemma:trigonometry:a+b=c-and-b-geq-0-implies-a-geq-c},
  \begin{equation}
\cos(0)\geq\cos^{2}(0).
  \end{equation}
  Dividing both sides by $\cos(0)$ gives the result.
\end{proof}

\begin{proposition}\label{prop:trigonometry:cos-of-zero-is-one}
$\cos(0) = 1$
\end{proposition}
\begin{proof}
  From Lemma~\ref{lemma:trigonometry:cos-zero-geq-one} we know
  $\cos(0)\geq1$, and from
  Lemma~\ref{lemma:trigonometry:cos-zero-leq-one} we know
  $\cos(0)\leq1$. So we have $1\leq\cos(0)\leq1$ imply $\cos(0)=1$.
\end{proof}

\begin{lemma}\label{lemma:equality:substitute-summand-for-equal-term}
For any $a,b,c,d\in\RR$, if $a+b=d$ and $b=c$, then $a+c=d$.
\end{lemma}

\begin{lemma}\label{lemma:equality:substitute-left-summand-for-equal-term}
For any $a,b,c,d\in\RR$, if $a+b=d$ and $a=c$, then $c+b=d$.
\end{lemma}

\begin{proposition}\label{prop:cosine-is-even}
For any $x\in\RR$, $\cos(-x) = \cos(x)$.
\end{proposition}
\begin{proof}
Using $-x=0-x$ and Axiom~\ref{axiom:trigonometry:cos-addition-law},
\begin{equation}
\cos(-x) = \cos(0 - x) = \cos(0)\cos(x) + \sin(0)\sin(x).
\end{equation}
But we also know that, from $x=x-0$ and Axiom~\ref{axiom:trigonometry:cos-addition-law},
\begin{equation}
\cos(x) = \cos(x - 0) = \cos(x)\cos(0) + \sin(x)\sin(0).
\end{equation}
Then using commutativity of multiplication, we find these expressions
are identical.
\begin{calculation}
  \cos(-x)
  \step{since $-x=0-x$}
  \cos(0-x)
  \step{subtraction law for cosine [Axiom~\ref{axiom:trigonometry:cos-addition-law}]}
  \cos(0)\cos(x) + \sin(0)\sin(x)
  \step{since $\sin(0)\sin(x)=\sin(x)\sin(0)$ by commutativity, and Lemma~\ref{lemma:equality:substitute-summand-for-equal-term}}
  \cos(0)\cos(x) + \sin(x)\sin(0)
  \step{since $\cos(0)\cos(x)=\cos(x)\cos(0)$ by commutativity, and Lemma~\ref{lemma:equality:substitute-left-summand-for-equal-term}}
  \cos(x)\cos(0) + \sin(x)\sin(0)
  \step{subtraction law for cosine [Axiom~\ref{axiom:trigonometry:cos-addition-law}]}
  \cos(x-0)
  \step{since $x-0=x$}
  \cos(x)\qedhere
\end{calculation}
\end{proof}

\begin{proposition}\label{prop:cos-sq-plus-sine-sq-equal-one}
$\cos^{2}(x) + \sin^{2}(x) = 1$.
\end{proposition}
\begin{proof}
  \begin{calculation}
    1
    \step{using Proposition~\ref{prop:trigonometry:cos-of-zero-is-one}}
    \cos(0)
    \step{since $0=x-x$}
    \cos(x-x)
    \step{subtraction law for cosine [Axiom~\ref{axiom:trigonometry:cos-addition-law}]}
     \cos^{2}(x) + \sin^{2}(x).\qedhere
  \end{calculation}
\end{proof}

\begin{proposition}\label{prop:sine-leq-one-when-x-leq-pi-over-two}
For any $x\in[0,\pi/2]$ we have $0\leq \sin(x)\leq 1$.
\end{proposition}

\begin{proof}
Let $x\in[0,\pi/2]$ be arbitrary. We know $S(x)\geq0$, we just need to
prove $S(x)\leq1$. We know this, because $S(x)^{2} = 1- C(x)^{2}$, and
$C(x)^{2}\geq 0$ implies $1-C(x)^{2}\leq 1$. Hence we have $S(x)^{2}\leq 1$
and $0\leq S(x)$ give the result.
\end{proof}

\begin{proposition}\label{prop:sine-of-zero}
  (a) $S(0) = 0$

  (b) $C(\pi/2) = 0$.
\end{proposition}
\begin{proof}
(b) We find $C(\pi/2 - \pi/2) = C(\pi/2)^{2} + S(\pi/2)^{2}$. We know
$C(0) = 1$ and $S(\pi/2) = 1$, which gives us $1 = C(\pi/2)^{2} + 1$,
hence $C(\pi/2)^{2} = 0$ and $C(\pi/2) = 0$.

(a)
Consider $C(\pi/2 - 0) = C(\pi/2)C(0) + S(\pi/2)S(0)$. We know $C(0)=1$,
and $S(\pi/2) = 1$, so we obtain $C(\pi/2 - 0) = C(\pi/2) + S(0)$.
Subtracting $C(\pi/2)$ from both sides gives $S(0)=0$.
\end{proof}

\begin{proposition}\label{prop:cos-pi-over-2-and-sine}
$C(\pi/2 - x) = S(x)$.
\end{proposition}
\begin{proof}
We have $C(\pi/2 - x) = C(\pi/2)C(x) + S(\pi/2) S(x)$
then using $S(\pi/2)=1$ and $C(\pi/2) = 0$ gives us
$C(\pi/2 - x) = 0 + S(x)$.
\end{proof}

\begin{proposition}\label{prop:sin-pi-over-2-and-cosine}
$S(\pi/2 - x) = C(x)$.
\end{proposition}
\begin{proof}
Set $y = \pi/2 - x$, then $C(\pi/2 - y) = S(y)$ and $C(\pi/2 - y) = C(x)$,
which gives the result.
\end{proof}

\begin{proposition}\label{prop:sine-pi-over-four-equals-cosine-pi-over-four}
$S(\pi/4) = C(\pi/4)$.
\end{proposition}
\begin{proof}
  Since $C(\pi/2 - x) = S(x)$ by Proposition~\ref{prop:cos-pi-over-2-and-sine}, choose $x=\pi/4$ and we obtain the result.
\end{proof}

\begin{proposition}\label{prop:cosine-is-positive-on-0-to-pi-over-2}
For all $x\in[0,\pi/2]$, we have $0\leq C(x)\leq 1$.
\end{proposition}
\begin{proof}
  Let $x\in[0,\pi/2]$ be arbitrary, set $y=\pi/2-x$. Then $y\in[0,\pi/2]$.
  We claim $0\leq\sin(y)\leq 1$, due to Proposition~\ref{prop:sine-leq-one-when-x-leq-pi-over-two}.

  But we know from Proposition~\ref{prop:sin-pi-over-2-and-cosine} that
  $\cos(x)=\sin(y)$, which gives the result.
\end{proof}

\begin{proposition}\label{prop:sine-of-sum}
$S(x + y) = S(x)C(y) + C(x)S(y)$.
\end{proposition}
\begin{proof}
We have $S(x + y) = C((\pi/2 - x) - y) = C(\pi/2 - x)C(y) + S(\pi/2 - x)S(y)$.
Using earlier results proves the claim.
\end{proof}

\begin{proposition}\label{prop:double-angle-for-sine}
$S(2x) = 2S(x)C(x)$.
\end{proposition}

\begin{proof}
  This follows immediately from the previous result, by choosing $x=y$.
\end{proof}

\begin{proposition}
$S(\pi) = 0$.
\end{proposition}

\begin{proof}
  Since $C(\pi/2)=0$, we find $S(\pi)=2S(\pi/2)C(\pi/2)=0$.
\end{proof}

\begin{proposition}
$2(C(\pi/4))^{2} = 2(S(\pi/4))^{2} = 1$.
\end{proposition}
\begin{proof}
We know from Proposition~\ref{prop:sine-pi-over-four-equals-cosine-pi-over-four}
that $C(\pi/4)=S(\pi/4)$. We know from our second axiom that $S(\pi/2)=1$.
When we combine this knowledge with the double-angle identity for Sine
(a.k.a., Proposition~\ref{prop:double-angle-for-sine}) we find
\begin{equation}
1 = S(\pi/2) = 2S(\pi/4)C(\pi/4) = 2S(\pi/4)^{2} = 2C(\pi/4)^{2}.
\end{equation}
Hence we prove the result.
\end{proof}

\begin{proposition}\label{prop:sine-of-pi-over-four}
$C(\pi/4) = S(\pi/4) = \sqrt{2}/2$.
\end{proposition}

\begin{proof}
Since $\pi/4\in[0,\pi/2)$, we know $S(\pi/4)\geq 0$. Combined with the
  previous proposition, we have $S(\pi/4) = \sqrt{2}/2$.

Similarly, on that interval, we know from Proposition~\ref{prop:cosine-is-positive-on-0-to-pi-over-2}
that $C(\pi/4)$ is positive. Combined with the previous result, we have
proven the claim.
\end{proof}

\begin{proposition}
$S(-\pi/4) = -S(\pi/4) = -\sqrt{2}/2$.
\end{proposition}
\begin{proof}
  Consider $S(0) = S(\pi/4 + (-\pi/4)) = S(\pi/4)C(-\pi/4) + S(-\pi/4)C(\pi/4)$
  by Proposition~\ref{prop:sine-of-sum}.
  We know $C(-x)=C(x)$ by Proposition~\ref{prop:cosine-is-even}, so we
  know
  \begin{equation}
S(0) = S(\pi/4)C(\pi/4) + S(-\pi/4)C(\pi/4) = (S(\pi/4) + S(-\pi/4))C(\pi/4).
  \end{equation}
  From Proposition~\ref{prop:sine-of-zero} we know $S(0)=0$, and we know
  $C(\pi/4)\neq 0$. Hence we conclude
  \begin{equation}
S(\pi/4) + S(-\pi/4) = 0,
  \end{equation}
  which implies the desired result.
\end{proof}

\begin{proposition}
$-S(\pi/2)=-1$ and $S(-\pi/2)=-1$.
\end{proposition}
\begin{proof}
(a) We use the double-angle identity for
sine (a.k.a., Proposition~\ref{prop:double-angle-for-sine})
\begin{equation}
S(\pi/2) = 2S(\pi/4)C(\pi/4).
\end{equation}
Then we use Proposition~\ref{prop:sine-of-pi-over-four} to rewrite the
right-hand side as
\begin{equation}
S(\pi/2) = 2(\sqrt{2}/2)^{2} = 2(2/4) = 1.
\end{equation}
Thus we have the result $-S(\pi/2)=-1$.

(b) We likewise use the double-angle identity for sine
\begin{subequations}
  \begin{align}
    S(-\pi/2) &= S(2(-\pi/4)) \\
    &= 2S(-\pi/4)C(-\pi/4)
  \end{align}
  but using Proposition~\ref{prop:cosine-is-even} we have
  $C(-\pi/4)=C(\pi/4)$ and combined with the previous result:
  \begin{align}
    S(-\pi/2) &= 2(-S(\pi/4))C(\pi/4)\\
    &= -(2S(\pi/4)C(\pi/4)).
  \end{align}
  Now we see the right-hand side is just $-1\cdot S(\pi/2)$.
\end{subequations}
\end{proof}

\begin{proposition}
$C(\pi)=-1$.
\end{proposition}
\begin{proof}
  We have
  \begin{subequations}
    \begin{align}
    C(\pi) &= C(\pi/2 - (-\pi/2))\\
    &= C(\pi/2)C(-\pi/2) + S(\pi/2)S(-\pi/2).
    \end{align}
    Now the previous result tells us $S(\pi/2)=1$ and $S(-\pi/2)=-1$, we
    have established in Proposition~\ref{prop:sine-of-zero} that
    $C(\pi/2)=0$, and from Proposition~\ref{prop:cosine-is-even} that
    $C(-\pi/2)=C(\pi/2)$ hence $C(-\pi/2)=0$. Combined this tells us
    \begin{equation}
C(\pi) = 0 + (+1)\cdot(-1) = -1,
    \end{equation}
  \end{subequations}
  as desired.
\end{proof}

\begin{proposition}\label{prop:sine-is-odd-function}
$S(-x) = -S(x)$.
\end{proposition}
\begin{proof}
  We have:
  \begin{subequations}
    \begin{align}
      C(\pi - (\pi/2 - x))
      &= C(\pi)C(\pi/2 - x) + S(\pi)S(\pi/2 - x)\\
      &= -1\cdot(C\pi/2 - x) + 0\cdot S(\pi/2 - x)\\
      &= -1\cdot S(x).
    \end{align}
  \end{subequations}
  We also have
  \begin{subequations}
    \begin{align}
      C(\pi - (\pi/2 - x))
      &= C(\pi/2 + x)\\
      &= C(\pi/2 - (-x))\\
      &= S(-x).
    \end{align}
  \end{subequations}
  Setting equals to equals yields the result.
\end{proof}
%% \begin{proof}
%% We have $S(-x) = C(\pi/2 - (-x)) = C(-\pi/2 - x)$.
%% \end{proof}

\begin{proposition}
$C(x+y) = C(x)C(y) - S(x)S(y)$.
\end{proposition}
\begin{proof}
Consider $C(x-z)=C(x)C(z)-S(x)S(z)$, set $z=-y$, then using the previous
result we find $C(x-z)=C(x+y)=C(x)C(y)+S(x)S(y)$ as desired.
\end{proof}

\begin{proposition}
$C(2x) = C(x)^{2} - S(x)^{2} = 2C(x)^{2} - 1$.
\end{proposition}

\begin{proof}
  We see $C(2x)=C(x+x)$, then using the previous proposition we find
  \begin{equation}
C(2x) = C(x)^{2} - S(x)^{2}.
  \end{equation}
  We recall Proposition~\ref{prop:cos-sq-plus-sine-sq-equal-one}
  establishing $C(x)^{2} + S(x)^{2} = 1$, which implies
  $S(x)^{2} = 1 - C(x)^{2}$. We can rewrite our result as
  \begin{equation*}
C(2x) = C(x)^{2} - (1 - C(x)^{2}) = 2C(x)^{2} - 1.\qedhere
  \end{equation*}
\end{proof}

\begin{proposition}
$(C(x) + 1)/2 = (C(x/2))^{2}$.
\end{proposition}

\begin{proof}
  We have $C(2y) = 2C(y)^{2} - 1$ from the previous result, so
  $C(2y) + 1 = 2C(y)^{2}$. Now we substitute $y=x/2$ and obtain the result.
\end{proof}

\begin{proposition}
$S(x-y) = S(x)C(y) - C(x)S(y)$.
\end{proposition}

\begin{proof}
This follows from Propositions~\ref{prop:sine-of-sum} and \ref{prop:sine-is-odd-function}.
\end{proof}

The remaining results follow from the angle-addition identities and the
knowledge that $S(\pi/2)=1$, $C(\pi/2)=0$, $S(-x)=-S(x)$, $C(-x)=C(x)$.

\begin{proposition}
$S(\pi/2 + x) = C(x)$.
\end{proposition}

\begin{proposition}
$C(\pi/2 + x) = -S(x)$.
\end{proposition}

The remaining results follow from angle-addition identities.

\begin{proposition}
$S(\pi + x) = - S(x)$.
\end{proposition}

\begin{proposition}
$C(\pi + x) = -C(x)$.
\end{proposition}

\begin{proposition}
$S(2\pi + x) = S(x)$.
\end{proposition}

\begin{proposition}
$C(2\pi + x) = C(x)$.
\end{proposition}
