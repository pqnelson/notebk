\chapter{Theorems and Lisp}

\M
It turns out that rewriting S-expressions suffices as the basis for a
theorem prover, which is precisely how ACL2 works. Carl Eastlund and
Daniel Friedman's \textit{The Little Prover} discuss a ``toy model'' of
such a prover for the Scheme dialect. The terminology is a little
strange at first, but the idea is to apply theorems to rewrite
subexpressions (called the \define{Focus}) until the result is
\lstinline[language=lisp]{t}. 

\M The precise description of the logic of ACL2 may be found in Kaufmann
and Moore's ``A Precise Description of the ACL2 Logic''.\footnote{Found online at \url{http://www.cs.utexas.edu/users/moore/publications/km97a.pdf}}
There are 11 axioms which specify the Lisp primitives, which behave as
we would expect --- things like which branch of an ``if'' expression to
take depending on the result of evaluating the test expression.

The 11 axioms for ACL2 are (the first 5 axioms specify equality, the
last 6 specify the primitives of ACL2):
\begin{enumerate}
\item \lstinline[language=lisp]{(not (equal t nil))}
\item \lstinline[language=lisp]{(implies (equal (not x) nil) (equal (if x y z) y))}
\item \lstinline[language=lisp]{(implies (equal x nil) (equal (if x y z) z))}
\item \lstinline[language=lisp]{(iff (equal x y) (or nil (equal (equal x y) t)))}
\item \lstinline[language=lisp]{(iff (= x y) (equal (equal x y) t))}
\item \lstinline[language=lisp]{(iff (consp x) (or nil (equal (consp x) t)))}
\item \lstinline[language=lisp]{(equal (consp (cons x y)) t)}
\item \lstinline[language=lisp]{(equal (consp nil) nil)}
\item \lstinline[language=lisp]{(equal (car (cons x y)) x)}
\item \lstinline[language=lisp]{(equal (cdr (cons x y)) y)}
\item \lstinline[language=lisp]{(implies (equal (consp x) t) (equal (cons (car x) (cdr x)) x))}
\end{enumerate}

%% \begin{enumerate}
%% \item \lstinline[language=lisp]{(not (equal t nil))}
%% \item \lstinline[language=lisp]{(equal (if t x y) x)}
%% \item \lstinline[language=lisp]{(equal (if nil x y) y)}
%% \item \lstinline[language=lisp]{(equal (consp (cons x y)) t)}
%% \item \lstinline[language=lisp]{(equal (consp nil) nil)}
%% \item \lstinline[language=lisp]{(equal (car (cons x y)) x)}
%% \item \lstinline[language=lisp]{(equal (cdr (cons x y)) y)}
%% \item \lstinline[language=lisp]{(implies (equal (consp x) t) (equal (cons (car x) (cdr x)) x))}
%%   %\item \lstinline[language=lisp]{}
%% \end{enumerate}

\N{Rewrite rules as theorems}
The generic rewrite rule may be written as a theorem whose statement is
of the form:
\begin{lisp-example}[caption={Template of theorem statement}]
(implies (and hypothesis-1 ... hypothesis-n)
         (equal lhs rhs))
\end{lisp-example}
When the hypotheses are applicable, we rewrite the focus using the rule 
\lstinline[language=lisp]{(equal lhs rhs)}.
This tries to match the focus with the left-hand side. If we can unify
the expressions, then we rewrite using the substituted ``\lstinline[language=lisp]{rhs}''
expression.

Specifically, when --- after substituting variables --- each hypothesis
evaluates to \lstinline[language=lisp]{'t}, we can apply the rewrite rule.
Care must be taken to handle situations when the focus is within an
if-expression which imposes conditions which affect hypotheses for a
theorem. The hypothesis satisfaction is done via backchaining.\footnote{For more about rewriting in ACL2, see \url{https://www.ccs.neu.edu/home/harshrc/courses/cs2800-fall2010/f10-lec26.pdf}}
Backchaining is done by looking at the hypothesis, then finds a theorem
of the form ``$p$ implies $\langle$hypothesis$\rangle$'' and then tries
to prove $p$ instead of the hypothesis.

\M We can implement a ``poor man's quasiquote'' which can be used to
define a proposition to produce a formula parametrized by variables.

\begin{code}[name=axioms, caption={Defining a proposition}]
(defun poor-man-quasiquote (e arglist &optional (cdr? nil))
  (cond
    ((and (symbolp e)
          (find e arglist)) e)
    ((consp e) (let ((result (cons (poor-man-quasiquote (car e) arglist)
                                   (if (endp (cdr e))
                                       nil
                                       (poor-man-quasiquote (cdr e) arglist 't)))))
                 (if cdr?
                     result
                     (cons 'list result))))
    (t (list 'quote e))))

(defmacro defproposition (name arglist e)
  (let ((body (poor-man-quasiquote e arglist)))
    `(defun ,name ,arglist ,body)))
\end{code}

\N{Axioms and Theorems}
We can then introduce axioms as ``just propositions'', and theorems
provided their proofs are valid. We just relegate the heavy lifting to a 
``\lstinline[language=lisp]{proves-claim?}'' function, which could be a
generic. We could abstract away the formulas proven to a class, with an
subclass for ``\lstinline[language=lisp]{implies}'', another subclass for
``\lstinline[language=lisp]{equals}'', and so on.

\begin{code}[name=axioms]
(defmacro defaxiom (name arglist body)
  (defproposition name arglist body))

(defmacro defthm (name arglist statement proof)
  (if (proves-claim? statement proof)
    (defproposition name arglist statement)
    (error "Invalid proof for theorem '" name "'")))
\end{code}

\N{Convention for Predicates}
I will distinguish predicates ``in the metalanguage'' (i.e., in Common
Lisp) from predicates ``in the object language'' (i.e., in the axioms)
by using the ``-p'' suffix for Common Lisp predicates and ``?'' suffix
for object language predicates.

\N{Equality}
We will use double equals for our equality. Although we can treat
equality as ``just another equivalence relation'', we need to also
explicitly include the Leibniz law of indiscernibles,
\begin{equation}
\forall P\forall x\forall y, (x=y)\land P[x]\implies P[y].
\end{equation}
We also need another analogous axiom schema for functions instead of
predicates.
\begin{equation}
\forall F\forall x\forall y, (x=y)\implies F(x) = F(y).
\end{equation}

\begin{code}[name=axioms]
(defaxiom ==/reflexive (x)
  (equal (== x x) 't))

(defaxiom ==/commutative (x y)
  (equal (== x y) (== y x)))
\end{code}

  
\N{TODO: Readable Proofs, Theorem Names}
I want to have proofs be explicit, or as explicit as possible. This
requires specifying the focus and telling Lisp to rewrite it using a
specific theorem (or axiom). Presumably this will be a macro that will
check the validity of a proof (if a theorem), then either raise an
exception (for invalid proofs) or define a function which will produce a
statement with the specified substitutions.
\begin{enumerate}
\item Think of a naming convention for theorems; should I use packages
  for namespacing (e.g., ``\lstinline[language=lisp]{add:associative}'')
  or should I use slashes (e.g., ``\lstinline[language=lisp]{thm:add/associative}'')?
\item How to make proofs readable? 
\item Presumably \lstinline[language=lisp]{(defthm thm-name (formals) statement proof...)}
  encodes a $\forall$-statement where we universally quantify over the
  formal arguments; how would we handle existential quantified statements?
\end{enumerate}

\M
Definitional theorems should be axioms, I suspect. For example,
\begin{lisp-example}
(defun + (&rest numbers)
  (reduce #'binary-+ numbers :initial-value 0))
\end{lisp-example}
This will produce the ``definitional theorem''
\begin{lisp-example}
(defaxiom defun-+ (xs)
  (equal (+ . xs) (reduce #'binary-+ xs :initial-value 0)))
\end{lisp-example}
More generally if we have defined a new function
``\lstinline[language=lisp]{fun}'' with formal arguments $\vec{x}$ and a
body $b(\vec{x})$, the
associated definitional theorem is an axiom of the form
``\lstinline[language=lisp]{(equal (fun} $\vec{x}\,$\lstinline[language=lisp]{)}
$b(\vec{x}\,)$\lstinline[language=lisp]{)}''.
We would also have the definitional axioms:
\begin{lisp-example}
(defaxiom defun-* (xs)
  (equal (* . xs) (reduce #'binary-* xs :initial-value 1)))

(defaxiom defun-- (x xs)
  (equal (- x . xs) (reduce #'binary-- xs :initial-value x)))

(defaxiom defun-negate (x)
  (equal (- x) (binary-- 0 x)))

(defaxiom defun-/ (x xs)
  (equal (/ x . xs) (reduce #'binary-/ xs :initial-value x)))

(defaxiom defun-invert (x)
  (equal (- x) (binary-/ 1 x)))
\end{lisp-example}

\begin{remark}
  Therefore, it makes sense to use the template of the form
  \begin{center}
    $\langle$\emph{definitional theorem name}$\rangle$ ::=
\texttt{axiom:}$\langle$\emph{package}$\rangle$\texttt{/}\texttt{defun-}$\langle$\emph{function name}$\rangle$
  \end{center}
for naming definitional theorems. This suggests using a separate package
for the axioms, and probably it would be wise to have another package
for theorems. But it also makes sense to have a helper function so I
could do something like use \lstinline[language=lisp]{(def-of '+ args...)}
as a justification for a proof step.
\end{remark}

\begin{remark}
The name ``definitional theorem'' is lifted from Mizar, shamelessly.
\end{remark}

\N{Reduce}
We can provide some axiomatic specification for
\lstinline[language=lisp]{reduce}.
\begin{lisp-example}
(defaxiom defun-reduce (f init-val xs)
  (if (null xs)
    init-val
    (equal (reduce f xs :initial-value init-val)
           (reduce f (cdr xs) :initial-value (f init-val (car xs))))))

(defaxiom right-reduce (f init-val xs)
  (if (null xs)
    init-val
    (equal (reduce f xs :initial-value init-val
                        :from-end t)
           (reduce f (butlast xs) :initial-value (f (last xs) init-val)
                                  :from-end t))))
\end{lisp-example}
I do not know if this suffices to prove:
\begin{lisp-example}
(defthm reduce-associative-fun (f head tail)
  (implies (and (associative? f)
                (list? tail))
           (equal (reduce f (cons head tail))
                  (reduce f (cons head tail)
                            :from-end t))))
\end{lisp-example}
This does beg the question: \emph{how would we define associativity?}
Na\"{\i}vely, we would hope something like:
\begin{lisp-example}
(defun associative? (f)
  (forall (x y z)
          (equal (f (f x y) z)
                 (f x (f y z)))))
\end{lisp-example}
But we do not have a clear notion of quantifiers in our attempt to
emulate ACL2.

\begin{remark}
We need to also have the reducing function be commutative, as well as associative.
\end{remark}