\chapter{Elementary Algebra}

\M
The axioms for elementary algebra include the following statements,
which possibly include redundancies. We summarize results which may be
found, e.g., in Appendix C of Mayer~\cite{mayer:calculus}. Also note
that the accepted notation has:
\begin{subequations}
\begin{align}
  a - b &:= a + (-b)
  \intertext{and}
  a/b &:= a\cdot(b^{-1}).
\end{align}
\end{subequations}

\N{Commutativity laws}
\begin{enumerate}
\item $\forall a,b\in\RR, a+b=b+a$
\item $\forall a,b\in\RR, a\cdot b=b\cdot a$
\end{enumerate}

\N{Associativity laws}
\begin{enumerate}[resume*]
\item $\forall a,b,c\in\RR, (a+b)+c=a+(b+c)$
\item $\forall a,b,c\in\RR, (a\cdot b)\cdot c=a\cdot(b\cdot c)$
\end{enumerate}

\N{Left Distributive property}
\begin{enumerate}[resume*]
\item $\forall a,b,c\in\RR, a\cdot(b+c)=(a\cdot b)+(a\cdot c)$
\item $\forall a,b,c\in\RR, a\cdot(b-c)=(a\cdot b)-(a\cdot c)$
\end{enumerate}

\N{Right Distributive property}
\begin{enumerate}[resume*]
\item $\forall a,b,c\in\RR, (a+b)\cdot c=(a\cdot c)+(b\cdot c)$
\item $\forall a,b,c\in\RR, (a-b)\cdot c=(a\cdot c)-(b\cdot c)$
\item $\forall a,b,c\in\RR, (a+b)/c=(a/c)+(b/c)$
\item  $\forall a,b,c\in\RR, (a-b)/c=(a/c)-(b/c)$
\end{enumerate}

\N{Axioms of $0$ and $1$} The rational numbers $0$ and $1$ satisfies the
following axioms.

\begin{enumerate}[resume*]
\item $\forall a\in\RR, a+0=a$
\item $\forall a\in\RR, 0+a=a$
\item $\forall a\in\RR, a\cdot1=a$
\item $\forall a\in\RR, 1\cdot a=a$
\item $\forall a\in\RR, a\cdot0=0$
\item $\forall a\in\RR, 0\cdot a=0$
\item $0\neq1$
\item $\forall a,b\in\RR$, if $ab=0$, then $a=0$ or $b=0$ (or both)
\end{enumerate}

\N{Additive and multiplicative Inverse}

\begin{enumerate}[resume*]
\item $\forall a\in\RR, a+(-a)=0$
\item $\forall a\in\RR, (-a)+a=0$
\item $\forall a\in\RR, a\neq0\implies a\cdot a^{-1} = 1$
\item $\forall a\in\RR, a\neq0\implies a^{-1}\cdot a = 1$
\item $-0=0$
\item $1^{-1}=1$
\end{enumerate}

\M We can prove the following results, but take them as axioms.
\begin{enumerate}[resume*]
\item $\forall a\in\RR, -(-a)=a$
\item $\forall a,b\in\RR, a-b = a+(-b)$
\item $\forall a,b\in\RR, b\neq0\implies a/b=a\cdot(b^{-1})$
\item $\forall a,b\in\RR, a\neq0\land b\neq0\implies (a\cdot b)^{-1}=(a^{-1})\cdot(b^{-1})$
\item $\forall a\in\RR, -a = (-1)\cdot a$
\item $\forall a\in\RR, a\neq=0\implies (a^{-1})^{-1}=a$
\item $\forall a,b\in\RR, (-a)\cdot(-b)=a\cdot b$
\item $\forall a,b\in\RR, (-a)\cdot b=-(a\cdot b)$
\item $\forall a,b\in\RR, a\cdot(-b)=-(a\cdot b)$
\item $\forall a,b\in\RR, b\neq0\implies (-a)/b=-(a/b)$
\item $\forall a,b\in\RR, b\neq0\implies a/(-b)=-(a/b)$
\end{enumerate}

\N{Cancellation Laws}
\begin{enumerate}[resume*]
\item $\forall a,b,c\in\RR, a+b=a+c\implies b=c$
\item $\forall a,b,c\in\RR, b+a=c+a\implies b=c$
\item $\forall a,b,c\in\RR, a\neq0\land a\cdot b=a\cdot c\implies b=c$
\item $\forall a,b,c\in\RR, a\neq0\land b\cdot a=c\cdot a\implies b=c$
\end{enumerate}

\N{Other identities}
These may be derived as theorems, but we can take them as axioms.
\begin{enumerate}[resume*]
\item $\forall a,b\in\RR, a^{2} - b^{2} = (a-b)(a+b)$
\item $\forall a,b\in\RR, a^{2} + 2ab + b^{2} = (a+b)^{2}$
\item $\forall a,b\in\RR, a^{2} - 2ab + b^{2} = (a-b)^{2}$
\item $\forall a,b,x\in\RR, x^{2} + (a+b)\cdot x + a\cdot b = (x+a)\cdot(x+b)$
\item $\forall a,b,c,d\in\RR, (a+b)(c+d) = ac+ad+bc+bd$
\item $\forall a,b,c,d\in\RR, b\neq0\land d\neq0\implies (a/b)\cdot(c/d) = (a\cdot c)/(b\cdot d)$
\item $\forall a,b,c,d\in\RR, b\neq0\land d\neq0\implies (a/b) + (c/d) = (a\cdot d + b\cdot c)/(b\cdot d)$
\item $\forall a,b,c,d\in\RR, b\neq0\land d\neq0\implies (a/b) - (c/d) = (a\cdot d - b\cdot c)/(b\cdot d)$
\end{enumerate}

\section{Order Laws}

\N{Trichotomy law} For any $a,b\in\RR$ exactly one of the following is
true:
\begin{equation}
a < b,\quad\mbox{or}\quad a=b,\quad\mbox{or}\quad a>b.
\end{equation}
We call $a$ \define{Positive} if $a>0$, and \define{Negative} if $a<0$.

\N{Sign Laws} We can derive the following from just $a<b\implies a+c<b+c$ and
$0<a\land 0<b\implies 0<a\cdot b$ (the axioms for an ordered field).
\begin{enumerate}[resume*]
\item $\forall a,b\in\RR, a>0\land b>0\implies ab>0\land a/b>0$
\item $\forall a,b\in\RR, a<0\land b>0\implies ab<0\land a/b<0$
\item $\forall a,b\in\RR, a>0\land b<0\implies ab<0\land a/b<0$
\item $\forall a,b\in\RR, a<0\land b<0\implies ab>0\land a/b>0$
\item $\forall a,b\in\RR, a>0\land b>0\implies a+b>0$
\item $\forall a,b\in\RR, a<0\land b<0\implies a+b<0$
\item $\forall a\in\RR, a>0\iff -a<0$
\item $\forall a\in\RR, a>0\iff a^{-1}>0$
\item $\forall a,b\in\RR, ab>0\implies (a>0\land b>0)\lor(a<0\land b<0)$
\item $\forall a,b\in\RR, ab<0\implies (a>0\land b<0)\lor(a<0\land b>0)$
\item $\forall a,b\in\RR, a/b>0\implies (a>0\land b>0)\lor(a<0\land b<0)$
\item $\forall a,b\in\RR, a/b<0\implies (a>0\land b<0)\lor(a<0\land b>0)$
\end{enumerate}

\M
We have as an immediate consequence the following result, which we take axiomatically
\begin{enumerate}[resume*]
\item $\forall a\in\RR, a^{2}\geq0$
\item $\forall a\in\RR, a^{2}>0\iff a\neq0$
\end{enumerate}

\N{Transitivity}
\begin{enumerate}[resume*]
\item $\forall a,b,c\in\RR,a<b\land b<c\implies a<c$
\end{enumerate}

\N{Addition of inequalities}
\begin{enumerate}[resume*]
\item $\forall a,b,c,d\in\RR, a<b\land c<d\implies a+c<b+d$
\item $\forall a,b,c,d\in\RR, a\leq b\land c\leq d\implies a+c\leq b+d$
\item $\forall a,b,c,d\in\RR, a<b\land c\leq d\implies a+c< b+d$
\item $\forall a,b,c\in\RR, a<b\implies a-c < b-c$
\item $\forall a,b\in\RR, a<b\implies -a > -b$
\item $\forall a,b,c\in\RR, b<c\implies a-b > a-c$
\end{enumerate}

\N{Multiplication of Inequalities}
\begin{enumerate}[resume*]
\item $\forall a,b,c,d\in\RR, a<b\land c<d\implies ac<bd$
\item $\forall a,b,c\in\RR, a<b\land c>0\implies a/c<b/c$
\item $\forall a,b,c,d\in\RR, 0<a<b\land 0<c<d\implies 0<ac<bd$
\item $\forall a,b,c\in\RR, a<b\land c<0\implies ac>bc$
\item $\forall a,b,c\in\RR, a<b\land c<0\implies a/c>b/c$
\item $\forall a,b\in\RR, 0<a\land a<b\implies a^{-1} > b^{-1}$
\end{enumerate}

\N{Arcimedean Property} This may be stated in various forms.
\begin{enumerate}[resume*]
\item $\forall a\in\RR,\exists x,y\in\QQ, x < a < y$
\item $\forall a\in\RR, a>0\implies \exists n\in\NN, a < n$
\item $\forall a,b\in\RR, a>0\land a < b\implies \exists n\in\NN, b < na$
\end{enumerate}

\section{Definitions}

\M There are a few notions worth defining, like the absolute value and
   [rational] powers.

\subsection{Absolute Value}

\begin{definition}
For any $x\in\RR$, we define the \define{Absolute Value} to be
the number $|x|\in\RR$ by
\begin{equation}
  |x| = \begin{cases}
     x & \mbox{if } x > 0,\\
     0 & \mbox{if } x = 0,\\
    -x & \mbox{if } x < 0.
  \end{cases}
\end{equation}
\end{definition}

\M Some results
\begin{enumerate}[resume*]
\item $\forall a,b\in\RR, a>0\implies (|x|<a \iff -a<x<a)$
\item $\forall a,b\in\RR, a>0\implies (|x|\leq a \iff -a\leq x\leq a)$
\item $\forall a\in\RR, |a|=|-a|$
\item $\forall a\in\RR, -|a| \leq a \leq |a|$
\item $\forall a,b\in\RR, |ab|=|a|\cdot|b|$
\item $\forall a,b\in\RR, b\neq0\implies |a/b| = |a|/|b|$
\end{enumerate}

\subsection{Powers}

\begin{definition}
Let $a\in\RR$ and $n\in\NN_{0}$, then the \define{Power} is defined by
the rules
\begin{subequations}
\begin{align}
a^{0} &= 1\\
a^{n+1} &= a^{n}\cdot a.
\end{align}
\end{subequations}
Further, if $a\neq0$ and $n\in\NN$ is nonzero, then we extend our
definition to the negative integers by
\begin{equation}
a^{-n} = \frac{1}{a^{n}}.
\end{equation}
\end{definition}

\begin{definition}
Let $a\in\RR$ and $n\in\NN$. Assume $a>0$. We define the fractional
power by
\begin{equation}
\exists!b\in\RR,b=a^{1/n}\land b^{n}=a.
\end{equation}
Now, if $a\geq0$ and $m\in\ZZ$ arbitrary and $n\in\NN$, we can define the
rational power of $a$ by
\begin{equation}
a^{m/n} = \begin{cases}
     (a^{1/n})^{m} & \mbox{if } a > 0\\
               0 & \mbox{if } a=0 \mbox{ and } m>0\\
\mbox{undefined} & \mbox{if } a=0\mbox{ and } m < 0.
\end{cases}
\end{equation}
\end{definition}

\begin{remark}
The question of $0^{0}$ is a contentious issue. We are inclined to agree
with Knuth (\arXiv{math/9205211}) that we should take $0^{0}=1$ and distinguish this
from the limits to these values. After all, $0^{0}$ permits the Binomial
theorem to make sense in more general settings, and the Binomial theorem
is too important to give up.
\end{remark}

\N{Monotonicity of Powers}
\begin{enumerate}[resume*]
\item $\forall x,y\in\RR,r\in\QQ, x\geq0\land y\geq0\land r>0\implies(x<y\iff x^{r}<y^{r})$
\item $\forall x,y\in\RR,r\in\QQ, x>0\land y>0\land r<0\implies(x<y\iff x^{r}>y^{r})$
\item $\forall a\in\RR,p,q\in\QQ, a>1\implies (p<q\iff a^{p}<a^{q})$
\item $\forall a\in\RR,p,q\in\QQ, a<1\implies (p<q\iff a^{p}>a^{q})$
\end{enumerate}

\N{Laws of Exponents} Whenever the powers involved are defined, then we
have the following identities.
\begin{enumerate}[resume*]
\item $\forall a\in\RR,\forall r,s\in\QQ,a^{r}a^{s}=a^{r+s}$
\item $\forall a\in\RR,\forall r,s\in\QQ,(a^{r})^{s}=a^{(rs)}$
\item $\forall a,b\in\RR,\forall r\in\QQ,(ab)^{r}=a^{s}b^{r}$
\item $\forall a\in\RR,\forall r\in\QQ, a^{-r} = 1/a^{r}$
\end{enumerate}

\section{Axiomatic Definition of Reals}

\subsection{Field Axioms}

\begin{axiom}
$\NN\subset\ZZ\subset\QQ\subset\RR$.
\end{axiom}

\begin{axiom}[Associativity of addition]\label{axiom:elementary-algebra:add-associativity}
For all $x$, $y$, $z\in\RR$, we have $x + (y + z) = (x + y) + z$.
\end{axiom}

\begin{axiom}[Commutativity of addition]\label{axiom:elementary-algebra:add-commutativity}
For all $x$, $y\in\RR$, we have $x + y = y + x$.
\end{axiom}

\begin{axiom}[Existence of $0$]\label{axiom:elementary-algebra:existence-of-0}
There exists a number $0\in\QQ$ such that for any $x\in\RR$ we have $x+0=x$.
\end{axiom}

\begin{proposition}\label{prop:elementary-algebra:additive-unit-identity}
Let $x\in\RR$. Then $0 + x = x$ and $x + 0 = x$.
\end{proposition}

\begin{proof}
We have $x + 0 = x$ by Axiom~\ref{axiom:elementary-algebra:existence-of-0}.
By direct calculation:
\begin{calculation}
x
\step{from Axiom~\ref{axiom:elementary-algebra:existence-of-0}}
x + 0
\step{from Axiom~\ref{axiom:elementary-algebra:add-commutativity}}
0 + x.\qedhere
\end{calculation}
\end{proof}

\begin{axiom}[Existence of additive inverse]
For all $x\in\RR$ there exists at least one $y\in\RR$ such that $x + y = 0$.
\end{axiom}

\begin{theorem}[Uniqueness of additive inverse]
Let $x\in\RR$. For any $y$, $y'\in\RR$ such that $x + y = 0$ and $x + y' = 0$,
we have $y = y'$. 
\end{theorem}

\begin{proof}
We have
\begin{calculation}
  y
\step{by $0$ being additive identity from Proposition~\ref{prop:elementary-algebra:additive-unit-identity}}
y + 0
\step{by hypothesis $0 = x + y'$}
y + (x + y')
\step{by associativity of addition from Axiom~\ref{axiom:elementary-algebra:add-associativity}}
(y + x) + y'
\step{by commutativity of addition from Axiom~\ref{axiom:elementary-algebra:add-commutativity}}
(x + y) + y'
\step{from hypothesis}
0 + y'
\step{by $0$ being additive identity from Proposition~\ref{prop:elementary-algebra:additive-unit-identity}}
y'.\qedhere
\end{calculation}
\end{proof}

\begin{remark}
Since the additive inverse is unique, we write $-x$ for the real number
such that $x + (-x) = 0$. We write $-x$ in Lisp as \lstinline[language=lisp]{(- x)}.
\end{remark}

\begin{corollary}\label{cor:elementary-algebra:add-inverse}
For any $x\in\RR$, we have $x + (-x) = 0$ and $(-x) + x = 0$.
\end{corollary}

\begin{axiom}[Associativity of multiplication]
For all $x$, $y$, $z\in\RR$, we have $x \cdot (y \cdot z) = (x \cdot y) \cdot z$.
\end{axiom}

\begin{axiom}[Commutativity of multiplication]\label{axiom:elementary-algebra:times-commutativity}
For all $x$, $y\in\RR$, we have $x \cdot y = y \cdot x$.
\end{axiom}

\begin{axiom}\label{axiom:elementary-algebra:times-distributes-over-add}
For all $x$, $y$, $z\in\RR$, we have $x\cdot(y + z) = x\cdot y + x\cdot z$.
\end{axiom}

\begin{theorem}\label{thm:elementary-algebra:times-distributes-over-add}
For all $x$, $y$, $z\in\RR$, we have $x\cdot(y + z) = x\cdot y + x\cdot z$
and $(x + y)\cdot z = x\cdot z + y\cdot z$.
\end{theorem}

\begin{proof}
We have $x\cdot(y + z) = x\cdot y + x\cdot z$ from Axiom~\ref{axiom:elementary-algebra:times-distributes-over-add}.
By direct calculation
\begin{calculation}
(x + y)\cdot z
\step{commutativity of multiplication from Axiom~\ref{axiom:elementary-algebra:times-commutativity}}
z\cdot(x + y)
\step{distributivity from Axiom~\ref{axiom:elementary-algebra:times-distributes-over-add}}
z\cdot x + z\cdot y
\step{commutativity of multiplication from Axiom~\ref{axiom:elementary-algebra:times-commutativity}}
x\cdot z + y\cdot z.\qedhere
\end{calculation}
\end{proof}

\begin{proposition}\label{prop:elementary-algebra:sub-from-both-sides}
Let $x$, $y$, $z\in\RR$. If $x + y = z$, then $x = z - y$.
\end{proposition}

\begin{axiom}[Existence of $1$]\label{axiom:elementary-algebra:times-1}
There exists a number $1\in\QQ$ such that for any $x\in\RR$ we have $x\cdot1=x$.
\end{axiom}

\begin{proposition}\label{prop:elementary-algebra:add-unit}
For all $x$, $y\in\RR$, if $x + y = x$ or $y + x = x$, then $y = 0$.
\end{proposition}

\begin{theorem}\label{thm:elementary-algebra:0-times-anything-is-0}
For any $x\in\RR$, we have $0\cdot x=0$.
\end{theorem}
\begin{proof}
By direct calculation, we find
\begin{calculation}
  x
  \step{by $1$ being multiplicative identity element}
  1\cdot x
  \step{using distributivity and $0$ being additive identity element}
  (0 + 1)\cdot x
  \step{distributivity}
  0\cdot x + 1\cdot x
  \step{since $1\cdot x=x$}
  0\cdot x + x.
\end{calculation}
We have $x = 0\cdot x + x$ imply $0\cdot x = 0$ by Proposition~\ref{prop:elementary-algebra:add-unit}.
\end{proof}

\begin{theorem}\label{thm:elementary-algebra:times-neg-1-is-add-inverse}
For any $x\in\RR$, we have $(-1)\cdot x = -x$.
\end{theorem}

\begin{proof}
We will prove $x + (-1)\cdot x = 0$, which implies $(-1)\cdot x = -x$.
By direct calculation,
\begin{calculation}
0
\step{using Theorem~\ref{thm:elementary-algebra:0-times-anything-is-0}}
0\cdot x
\step{since $0 = 1 - 1$}
(1 - 1)\cdot x
\step{distributivity}
1\cdot x + (-1)\cdot x
\step{since $1\cdot x = x$}
x + (-1)\cdot x.\qedhere
\end{calculation}
\end{proof}

\begin{proposition}
$(-1)^{2}=1$.
\end{proposition}

\begin{proof}
We find
\begin{calculation}
(-1)^{2} + (-1)
\step{distributivity}
(-1)\cdot((-1) + 1)
\step{since $(-1)+1=0$}
(-1)\cdot0
\step{since $x\cdot0=0$}
0.
\end{calculation}
It follows that $(-1)^{2}$ has the same additive invese as $1$, which
implies the result.
\end{proof}

\begin{axiom}\label{axiom:elementary-algebra:0-neq-1}
$0\neq1$.
\end{axiom}

\begin{remark}
This seems like a silly axiom, but without it we could end up with
models with the property ``Every real number is zero''. We also do not
have sufficient axioms to prove $0\neq1$.
\end{remark}

\begin{axiom}
For all $x\in\RR$, if $x\neq0$, then there exists at least one $y\in\RR$
such that $x\cdot y=1$.
\end{axiom}

\begin{theorem}
For all $x\in\RR$, if $x\neq0$, then there exists exactly one $y\in\RR$
such that $x\cdot y=1$.
\end{theorem}

\begin{proof}
Let $y$, $y'\in\RR$ be such that $x\cdot y=1$ and $x\cdot y'=1$.
Then
\begin{calculation}
  y
  \step{multiplicative unit}
  y\cdot 1
  \step{by hypothesis}
  y\cdot(x\cdot y')
  \step{associativity of multiplication}
  (y\cdot x)\cdot y'
  \step{commutativity of multiplication}
  (x\cdot y)\cdot y'
  \step{by hypothesis}
  1\cdot y'
  \step{multiplicative unit}
  y'.\qedhere
\end{calculation}
\end{proof}

\begin{remark}
We will write $x^{-1}\in\RR$ to denote the real number for which $x\cdot x^{-1}=1$.
\end{remark}

\subsection{Ordered Field Axioms}

\M
We assume there is an ordering relation $\leq$ on $\RR$. The axioms we
stipulate amount to demanding $\RR$ be totally ordered and compatibility
of $\leq$ with arithmetic operations.

\N{Notation}
For $x$, $y\in\RR$, we will write
\begin{equation}
x < y\qquad\mbox{if}\qquad x\leq y\quad\mbox{and}\quad x\neq y.
\end{equation}
We also have
\begin{equation}
y > x\quad\mbox{means}\quad x < y,
\end{equation}
and
\begin{equation}
y\geq x\quad\mbox{means}\quad x\leq y.
\end{equation}

\begin{axiom}[$\leq$ is Reflexive]\label{axiom:elementary-algebra:order-reflexivity}
For all $x\in\RR$, we have $x\leq x$
\end{axiom}

\begin{axiom}[$\leq$ is Antisymmetric]\label{axiom:elementary-algebra:order-antisymmetry}
For all $x$, $y\in\RR$, if $x\leq y$ and $y\leq x$, then $x=y$.
\end{axiom}

\begin{axiom}[$\leq$ is Transitive]\label{axiom:elementary-algebra:order-transitivity}
For all $x$, $y$, $z\in\RR$, if we have $x\leq y$ and $y\leq z$,
then $x\leq z$.
\end{axiom}

\begin{axiom}[$\leq$ is Total]\label{axiom:elementary-algebra:order-total}
For all $x$, $y\in\RR$, either $x\leq y$ or $y\leq x$.
\end{axiom}

\begin{proposition}\label{prop:elementary-algebra:order-total}
For all $x$, $y\in\RR$, either $x < y$ or $x = y$ or $x > y$.
\end{proposition}

\begin{axiom}[$\leq$ Preserved Under Addition]\label{axiom:elementary-algebra:add-preserves-order}
For all $x$, $y$, $z\in\RR$, if $x\leq y$, then $x + z\leq y + z$.
\end{axiom}

\begin{proposition}
For all $x$, $y$, $z\in\RR$, if $x < y$, then $x + z < y + z$.
\end{proposition}

\begin{theorem}
$-1 < 0$.
\end{theorem}

\begin{proof}
Add $1$ to both sides yields $0\leq 1$. We have $0\neq1$ by Axiom~\ref{axiom:elementary-algebra:0-neq-1}.
\end{proof}

\begin{axiom}[$\leq$ Preserved Under Multiplication]\label{axiom:elementary-algebra:times-preserves-order}
For all $x$, $y\in\RR$, if $0\leq x$ and $0\leq y$, then $0 \leq x\cdot y$.
\end{axiom}

\begin{definition}
Let $x\in\RR$.
\begin{enumerate}
\item We call $x$ \define{Positive} if $x > 0$.
\item We call $x$ \define{Negative} if $x < 0$.
\item We call $x$ \define{Non-negative} if $x\geq 0$.
\item We call $x$ \define{Non-positive} if $x\leq 0$.
\end{enumerate}
\end{definition}

\subsection{Exponentiation}

\begin{definition}
Let $x\in\RR$, let $n\in\NN_{0}$.
We inductively define the $n^{\text{th}}$ power of $x$ by:
\begin{enumerate}
\item $x^{0} = 1$
\item $x^{n + 1} = x^{n}\cdot x$.
\end{enumerate}
\end{definition}

\begin{remark}
This means $0^{0}=1$, when it should be undefined. We make this decision
so we can obtain the binomial theorem.
\end{remark}

\begin{remark}
The Lisp function for $x^{n}$ is \lstinline[language=lisp]{(expt} $x$ $n$\lstinline[language=lisp]{)}.
\end{remark}

\begin{proposition}
Let $x\in\RR$. Then $x^{1} = x$.
\end{proposition}

\begin{definition}
Let $x\in\RR$ be nonzero, let $n\in\NN$.
We extend the $n^{\text{th}}$ power of $x$ to negative values of $n$ by
\begin{equation}
x^{-n} = (x^{n})^{-1}.
\end{equation}
That is, $x^{-n}$ is the multiplicative inverse of $x^{n}$.
\end{definition}

\begin{remark}
This justifies the choice of notation that $x^{-1}$ is the
multiplicative inverse of $x$, since we would have $x^{-1} = (x^{1})^{-1}$
--- or to be explicitly clear in Lisp notation, 
\lstinline[language=lisp]{(expt x -1)} is equal to
\lstinline[language=lisp]{(/ (expt x 1))}. But since $x^{1}=x$, we have
\lstinline[language=lisp]{(expt x -1)} equal to \lstinline[language=lisp]{(/ x)}.
This identification is made readily by our choice of notation.
\end{remark}

\begin{proposition}
For any $x\in\RR$, $m$, $n\in\NN_{0}$, we have $(x^{m})\cdot(x^{n})=x^{m+n}$.
\end{proposition}

\begin{proof}
By induction on $n$.

Base case ($n=0$): by direct calculation
\begin{calculation}
  (x^{m})\cdot(x^{0})
  \step{since $x^{0}=1$}
  (x^{m})\cdot1
  \step{since $1$ is unit of multiplication}
  x^{m}
  \step{since $m = m+0$}
  x^{m+0}
  \step{since $n=0$}
  x^{m+n}.
\end{calculation}

Inductive hypothesis: assume $(x^{m})\cdot(x^{n})=x^{m + n}$.

Inductive case: we see
\begin{calculation}
  (x^{m})\cdot(x^{n+1})
  \step{by definition}
  (x^{m})\cdot(x^{n}\cdot x)
  \step{associativity of multiplication}
  \bigl((x^{m})\cdot(x^{n})\bigr)\cdot x
  \step{by inductive hypothesis}
  x^{m + n} \cdot x
  \step{by definition}
  x^{(m + n) + 1}
  \step{by associativity of addition}
  x^{m + n + 1}.\qedhere
\end{calculation}
\end{proof}

\begin{corollary}
For any nonzero $x\in\RR$, and for any $m$, $n\in\ZZ$,
we have $(x^{m})\cdot(x^{n})=x^{m+n}$.
\end{corollary}

\begin{proposition}
For any $x\in\RR$, $m$, $n\in\NN_{0}$, we have $(x^{m})^{n}=x^{mn}$.
\end{proposition}

\begin{proof}
By induction on $n$.

Base case ($n=0$): $(x^{m})^{0} = 1$ by definition.

Inductive hypothesis: assume $(x^{m})^{n}=x^{mn}$.

Inductive case: we see
\begin{calculation}
  (x^{m})^{n+1}
  \step{by definition}
  \bigl((x^{m})^{n}\bigr)\cdot(x^{m})
  \step{by inductive hypothesis}
  x^{mn}\cdot x^{m}
  \step{using $x^{a}x^{b}=x^{a+b}$}
  x^{mn + m}
  \step{distributivity gives $m\cdot(n + 1) = mn + m$}
  x^{m\cdot(n + 1)}.\qedhere
\end{calculation}
\end{proof}

\begin{lemma}
For any nonzero $x\in\RR$, and for any $n\in\NN$, we have $x^{-n}=(x^{n})^{-1}=(x^{-1})^{n}$.
\end{lemma}

\begin{proof}
Since $x^{-n} = (x^{n})^{-1}$ by definition of exponentiation to a
negative power, we are left to prove $x^{-n} = (x^{-1})^{n}$.

By induction on $n$. The base case ($n=1$) is $x^{-1} = (x^{1})^{-1}$
which is true, and $(x^{-1})^{1} = (x^{-1})$.

Inductive hypothesis: assume for arbitrary $n$ that $x^{-n}=(x^{n})^{-1}=(x^{-1})^{n}$.

Inductive case: we have $x^{-(n+1)} = (x^{n+1})^{-1}$ by
definition.
From $x^{n+1} = x^{n}x^{1}$ we have
$(x^{n+1})^{-1} = (x^{n}x^{1})^{-1} = (x^{n})^{-1}(x^{1})^{-1} = x^{-n} x^{-1}$,
Using the inductive hypothesis $x^{-n} = (x^{-1})^{n}$ we arrive at $(x^{n+1})^{-1}=(x^{-1})^{n}x^{-1}=(x^{-1})^{n+1}$,
as desired.
\end{proof}

\begin{corollary}\label{cor:elementary-algebra:expt-of-expt-is-expt-raised-to-product}
For any nonzero $x\in\RR$, and for any $m$, $n\in\ZZ$,
we have $(x^{m})^{n}=x^{mn}$.
\end{corollary}

\begin{proof}
Since $n\geq 0$ or $n < 0$, we prove it by cases.

Assume $n\geq 0$. Either $m\geq0$ or $m < 0$. For $m\geq 0$,
the result follows immediately. For $m < 0$, write $m = -k$, then
$x^{m} = x^{-k} = (x^{-1})^{k}$. But $x^{-1}$ is a real number and $k\geq0$.
So $((x^{-1})^{k})^{n}=(x^{-1})^{kn}$. And by definition $(x^{-1})^{kn}=x^{-kn}=x^{mn}$.

For $n < 0$, let us write $n=-q$. Then $(x^{m})^{n} = (x^{m})^{-q} = [(x^{m})^{q}]^{-1}=(x^{mq})^{-1}=x^{-mq}=x^{-mn}$.
\end{proof}

\begin{corollary}
For any nonzero $x\in\RR$ and for any $n\in\ZZ$, we have $x^{2n} = (x^{2})^{n}$.
\end{corollary}

\begin{proof}
This follows from Corollary~\ref{cor:elementary-algebra:expt-of-expt-is-expt-raised-to-product}
with $m=2$.
\end{proof}

\begin{remark}
This will be useful when computing $x^{n}$, since either $n=2k$ or
$n=2k+1$. In the former case, we recursively compute $(x^{2})^{k}$, and
the latter case we computer $x^{n} = x\cdot x^{2k} = x\cdot (x^{2})^{k}$.
\end{remark}

\begin{proposition}
For any $x\in\RR$, we have $0\leq x^{2}$.
\end{proposition}

\begin{proof}
From Proposition~\ref{prop:elementary-algebra:order-total},
we have $x < 0$ or $x = 0$ or $x > 0$. We will work in these three
cases.

If $x < 0$, then write $x = -y$ for $y\in\RR$ with $y > 0$. Then $x^{2} = (-1)^{2}y^{2}$.
Since $(-1)^{2} = 1$, this implies $x^{2} = y^{2}$ and $y^{2}\geq 0$ by Axiom~\ref{axiom:elementary-algebra:times-preserves-order}.

If $x = 0$, then $x^{2}=0$. Hence the result.

If $x > 0$, then $x\geq 0$, and the result follows from Axiom~\ref{axiom:elementary-algebra:times-preserves-order}.
\end{proof}

% \label{axiom:elementary-algebra:}

\N{On Completeness}
Typically there is discussion about the least-upper bound property
(``Dedekind completeness''). This is equivalent to assuming the
Archimedean property (for every positive $x\in\RR$, there exists an
$n\in\NN$ such that $1 < nx$), and that every Cauchy sequence of real numbers
converges to a real number (i.e., the reals are ``topologically complete'').


