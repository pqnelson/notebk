\chapter{Elementary Algebra}

\M
The axioms for elementary algebra include the following statements,
which possibly include redundancies.

\N{Commutativity laws}
\begin{enumerate}
\item $\forall a,b\in\RR, a+b=b+a$
\item $\forall a,b\in\RR, a\cdot b=b\cdot a$
\end{enumerate}

\N{Associativity laws}
\begin{enumerate}[resume*]
\item $\forall a,b,c\in\RR, (a+b)+c=a+(b+c)$
\item $\forall a,b,c\in\RR, (a\cdot b)\cdot c=a\cdot(b\cdot c)$
\end{enumerate}

\N{Left Distributive property}
\begin{enumerate}[resume*]
\item $\forall a,b,c\in\RR, a\cdot(b+c)=(a\cdot b)+(a\cdot c)$
\item $\forall a,b,c\in\RR, a\cdot(b-c)=(a\cdot b)-(a\cdot c)$
\end{enumerate}

\N{Right Distributive property}
\begin{enumerate}[resume*]
\item $\forall a,b,c\in\RR, (a+b)\cdot c=(a\cdot c)+(b\cdot c)$
\item $\forall a,b,c\in\RR, (a-b)\cdot c=(a\cdot c)-(b\cdot c)$
\item $\forall a,b,c\in\RR, (a+b)/c=(a/c)+(b/c)$
\item  $\forall a,b,c\in\RR, (a-b)/c=(a/c)-(b/c)$
\end{enumerate}

\N{Axioms of $0$ and $1$} The rational numbers $0$ and $1$ satisfies the
following axioms.

\begin{enumerate}[resume*]
\item $\forall a\in\RR, a+0=a$
\item $\forall a\in\RR, 0+a=a$
\item $\forall a\in\RR, a\cdot1=a$
\item $\forall a\in\RR, 1\cdot a=a$
\item $\forall a\in\RR, a\cdot0=0$
\item $\forall a\in\RR, 0\cdot a=0$
\item $0\neq1$
\item $\forall a,b\in\RR$, if $ab=0$, then $a=0$ or $b=0$ (or both)
\end{enumerate}

\N{Additive and multiplicative Inverse}

\begin{enumerate}[resume*]
\item $\forall a\in\RR, a+(-a)=0$
\item $\forall a\in\RR, (-a)+a=0$
\item $\forall a\in\RR, a\neq0\implies a\cdot a^{-1} = 1$
\item $\forall a\in\RR, a\neq0\implies a^{-1}\cdot a = 1$
\item $-0=0$
\item $1^{-1}=1$
\end{enumerate}

\M We can prove the following results, but take them as axioms.
\begin{enumerate}[resume*]
\item $\forall a\in\RR, -(-a)=a$
\item $\forall a,b\in\RR, a-b = a+(-b)$
\item $\forall a,b\in\RR, b\neq0\implies a/b=a\cdot(b^{-1})$
\item $\forall a,b\in\RR, a\neq0\land b\neq0\implies (a\cdot b)^{-1}=(a^{-1})\cdot(b^{-1})$
\item $\forall a\in\RR, -a = (-1)\cdot a$
\item $\forall a\in\RR, a\neq=0\implies (a^{-1})^{-1}=a$
\item $\forall a,b\in\RR, (-a)\cdot(-b)=a\cdot b$
\item $\forall a,b\in\RR, (-a)\cdot b=-(a\cdot b)$
\item $\forall a,b\in\RR, a\cdot(-b)=-(a\cdot b)$
\item $\forall a,b\in\RR, b\neq0\implies (-a)/b=-(a/b)$
\item $\forall a,b\in\RR, b\neq0\implies a/(-b)=-(a/b)$
\end{enumerate}

\N{Cancellation Laws}
\begin{enumerate}[resume*]
\item $\forall a,b,c\in\RR, a+b=a+c\implies b=c$
\item $\forall a,b,c\in\RR, b+a=c+a\implies b=c$
\item $\forall a,b,c\in\RR, a\neq0\land a\cdot b=a\cdot c\implies b=c$
\item $\forall a,b,c\in\RR, a\neq0\land b\cdot a=c\cdot a\implies b=c$
\end{enumerate}

\N{Other identities}
These may be derived as theorems, but we can take them as axioms.
\begin{enumerate}[resume*]
\item $\forall a,b\in\RR, a^{2} - b^{2} = (a-b)(a+b)$
\item $\forall a,b\in\RR, a^{2} + 2ab + b^{2} = (a+b)^{2}$
\item $\forall a,b\in\RR, a^{2} - 2ab + b^{2} = (a-b)^{2}$
\item $\forall a,b,x\in\RR, x^{2} + (a+b)\cdot x + a\cdot b = (x+a)\cdot(x+b)$
\item $\forall a,b,c,d\in\RR, (a+b)(c+d) = ac+ad+bc+bd$
\item $\forall a,b,c,d\in\RR, b\neq0\land d\neq0\implies (a/b)\cdot(c/d) = (a\cdot c)/(b\cdot d)$
\item $\forall a,b,c,d\in\RR, b\neq0\land d\neq0\implies (a/b) + (c/d) = (a\cdot d + b\cdot c)/(b\cdot d)$
\item $\forall a,b,c,d\in\RR, b\neq0\land d\neq0\implies (a/b) - (c/d) = (a\cdot d - b\cdot c)/(b\cdot d)$
\end{enumerate}

\section{Order Laws}

\N{Trichotomy law} For any $a,b\in\RR$ exactly one of the following is
true:
\begin{equation}
a < b,\quad\mbox{or}\quad a=b,\quad\mbox{or}\quad a>b.
\end{equation}
We call $a$ \define{Positive} if $a>0$, and \define{Negative} if $a<0$.

\N{Sign Laws} We can derive the following from just $a<b\implies a+c<b+c$ and
$0<a\land 0<b\implies 0<a\cdot b$ (the axioms for an ordered field).
\begin{enumerate}[resume*]
\item $\forall a,b\in\RR, a>0\land b>0\implies ab>0\land a/b>0$
\item $\forall a,b\in\RR, a<0\land b>0\implies ab<0\land a/b<0$
\item $\forall a,b\in\RR, a>0\land b<0\implies ab<0\land a/b<0$
\item $\forall a,b\in\RR, a<0\land b<0\implies ab>0\land a/b>0$
\item $\forall a,b\in\RR, a>0\land b>0\implies a+b>0$
\item $\forall a,b\in\RR, a<0\land b<0\implies a+b<0$
\item $\forall a\in\RR, a>0\iff -a<0$
\item $\forall a\in\RR, a>0\iff a^{-1}>0$
\item $\forall a,b\in\RR, ab>0\implies (a>0\land b>0)\lor(a<0\land b<0)$
\item $\forall a,b\in\RR, ab<0\implies (a>0\land b<0)\lor(a<0\land b>0)$
\item $\forall a,b\in\RR, a/b>0\implies (a>0\land b>0)\lor(a<0\land b<0)$
\item $\forall a,b\in\RR, a/b<0\implies (a>0\land b<0)\lor(a<0\land b>0)$
\end{enumerate}

\M
We have as an immediate consequence the following result, which we take axiomatically
\begin{enumerate}[resume*]
\item $\forall a\in\RR, a^{2}\geq0$
\item $\forall a\in\RR, a^{2}>0\iff a\neq0$
\end{enumerate}

\N{Transitivity}
\begin{enumerate}[resume*]
\item $\forall a,b,c\in\RR,a<b\land b<c\implies a<c$
\end{enumerate}

\N{Addition of inequalities}
\begin{enumerate}[resume*]
\item $\forall a,b,c,d\in\RR, a<b\land c<d\implies a+c<b+d$
\item $\forall a,b,c,d\in\RR, a\leq b\land c\leq d\implies a+c\leq b+d$
\item $\forall a,b,c,d\in\RR, a<b\land c\leq d\implies a+c< b+d$
\item $\forall a,b,c\in\RR, a<b\implies a-c < b-c$
\item $\forall a,b\in\RR, a<b\implies -a > -b$
\item $\forall a,b,c\in\RR, b<c\implies a-b > a-c$
\end{enumerate}

\N{Multiplication of Inequalities}
\begin{enumerate}[resume*]
\item $\forall a,b,c,d\in\RR, a<b\land c<d\implies ac<bd$
\item $\forall a,b,c\in\RR, a<b\land c>0\implies a/c<b/c$
\item $\forall a,b,c,d\in\RR, 0<a<b\land 0<c<d\implies 0<ac<bd$
\item $\forall a,b,c\in\RR, a<b\land c<0\implies ac>bc$
\item $\forall a,b,c\in\RR, a<b\land c<0\implies a/c>b/c$
\item $\forall a,b\in\RR, 0<a\land a<b\implies a^{-1} > b^{-1}$
\end{enumerate}

\N{Arcimedean Property} This may be stated in various forms.
\begin{enumerate}[resume*]
\item $\forall a\in\RR,\exists x,y\in\QQ, x < a < y$
\item $\forall a\in\RR, a>0\implies \exists n\in\NN, a < n$
\item $\forall a,b\in\RR, a>0\land a < b\implies \exists n\in\NN, b < na$
\end{enumerate}

\section{Definitions}

\M There are a few notions worth defining, like the absolute value and
   [rational] powers.

\subsection{Absolute Value}

\begin{definition}
For any $x\in\RR$, we define the \define{Absolute Value} to be
the number $|x|\in\RR$ by
\begin{equation}
  |x| = \begin{cases}
     x & \mbox{if } x > 0,\\
     0 & \mbox{if } x = 0,\\
    -x & \mbox{if } x < 0.
  \end{cases}
\end{equation}
\end{definition}

\M Some results
\begin{enumerate}[resume*]
\item $\forall a,b\in\RR, a>0\implies (|x|<a \iff -a<x<a)$
\item $\forall a,b\in\RR, a>0\implies (|x|\leq a \iff -a\leq x\leq a)$
\item $\forall a\in\RR, |a|=|-a|$
\item $\forall a\in\RR, -|a| \leq a \leq |a|$
\item $\forall a,b\in\RR, |ab|=|a|\cdot|b|$
\item $\forall a,b\in\RR, b\neq0\implies |a/b| = |a|/|b|$
\end{enumerate}

\subsection{Powers}

\begin{definition}
Let $a\in\RR$ and $n\in\NN_{0}$, then the \define{Power} is defined by
the rules
\begin{subequations}
\begin{align}
a^{0} &= 1\\
a^{n+1} &= a^{n}\cdot a.
\end{align}
\end{subequations}
Further, if $a\neq0$ and $n\in\NN$ is nonzero, then we extend our
definition to the negative integers by
\begin{equation}
a^{-n} = \frac{1}{a^{n}}.
\end{equation}
\end{definition}

\begin{definition}
Let $a\in\RR$ and $n\in\NN$. Assume $a>0$. We define the fractional
power by
\begin{equation}
\exists!b\in\RR,b=a^{1/n}\land b^{n}=a.
\end{equation}
Now, if $a\geq0$ and $m\in\ZZ$ arbitrary and $n\in\NN$, we can define the
rational power of $a$ by
\begin{equation}
a^{m/n} = \begin{cases}
     (a^{1/n})^{m} & \mbox{if } a > 0\\
               0 & \mbox{if } a=0 \mbox{ and } m>0\\
\mbox{undefined} & \mbox{if } a=0\mbox{ and } m < 0.
\end{cases}
\end{equation}
\end{definition}

\begin{remark}
The question of $0^{0}$ is a contentious issue. We are inclined to agree
with Knuth (\arXiv{math/9205211}) that we should take $0^{0}=1$ and distinguish this
from the limits to these values. After all, $0^{0}$ permits the Binomial
theorem to make sense in more general settings, and the Binomial theorem
is too important to give up.
\end{remark}

\N{Monotonicity of Powers}
\begin{enumerate}[resume*]
\item $\forall x,y\in\RR,r\in\QQ, x\geq0\land y\geq0\land r>0\implies(x<y\iff x^{r}<y^{r})$
\item $\forall x,y\in\RR,r\in\QQ, x>0\land y>0\land r<0\implies(x<y\iff x^{r}>y^{r})$
\item $\forall a\in\RR,p,q\in\QQ, a>1\implies (p<q\iff a^{p}<a^{q})$
\item $\forall a\in\RR,p,q\in\QQ, a<1\implies (p<q\iff a^{p}>a^{q})$
\end{enumerate}

\N{Laws of Exponents} Whenever the powers involved are defined, then we
have the following identities.
\begin{enumerate}[resume*]
\item $\forall a\in\RR,\forall r,s\in\QQ,a^{r}a^{s}=a^{r+s}$
\item $\forall a\in\RR,\forall r,s\in\QQ,(a^{r})^{s}=a^{(rs)}$
\item $\forall a,b\in\RR,\forall r\in\QQ,(ab)^{r}=a^{s}b^{r}$
\item $\forall a\in\RR,\forall r\in\QQ, a^{-r} = 1/a^{r}$
\end{enumerate}
