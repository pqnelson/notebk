\chapter{Introduction}

\M
Computer algebra systems, like Wolfram Mathematica, Maple, etc., tend to
do symbolic computation well. Their basic structure could be described
as using rewrite rules to manipulate expressions, according to the rules
of elementary algebra and calculus. The problem is, as Cauchy decried,
the results may not always make sense.

This has been a staple of computer science and mathematical literature
for decades now, chronicling bugs encountered in the various computer
algebra systems. Excellent work has been done by Fredrik Johansson~\cite{johansson2020:fungrim} and
friends in creating the opensource ``Mathematical Functions
Grimoire''\footnote{See \url{https://fungrim.org/}} in an attempt to
standardize rules and expected results. It is not a test suite, but it
can be used as the basis for one.

\N{Question: Automate Euler?}
This motivates the question we examine in this note: can we ``encode''
rules to simulate Eulerian reasoning? Can we create an ``automated Euler''?

\M
We are not particularly concerned about ``accuracy'', per se, in the
sense that if ``AE'' [automated Euler] generates a formal solution to a
symbolic problem, then we will be happy.

\N{Are the rules correct?}
This begs the next question, are the rules we encode ``correct''? It is
entirely plausible there was an error in transcription, for example. Can
we have some way to verify the correctness of our system (in some
suitable sense)?

I think we can create an ``ersatz theorem prover'' to prove a rule
follows from previous ones. We can use these proofs like unit tests, in
the sense that they are run ``on demand'' as opposed to everytime in
production.

\M
Since we have some ``vague ideas'' about what we're doing, it seems wise
to use a Lisp for our programming language. Lisp is one of the few (if not, the
only) programming language where the metalanguage is also the object
language (``homoiconic'' in a strong sense, not a buzzword sense).
Mathematics, sadly, also muddles its object language and metalanguage,
as Nick de Bruijn~\cite{de-bruijn:aut001} realized back in 1968:
\begin{quote}
As to the question what part of mathematics can be written in \textsc{Automath},
it should first be remarked that we do not possess a workable definition
of the word ``mathematics''. Quite often a mathematician jumps from his
mathematical language into a kind of metalanguage, obtains results
there, and uses these results in his original context. It seems to be
very hard to create a single language in which such things can be done
without any restriction. Of course it is possible to have a language in
which discussions about the language itself can be expressed, but that
is not the difficulty. The problem is to extend a given text $T_{1}$, by
means of a metalingual discussion $T_{2}$ ($T_{2}$ talks about $T_{1}$) and to put $T_{2}$
in the context of $T_{1}$, instead of creating a new context where both $T_{1}$
and $T_{2}$ can take place. For, if $T_{1}$ is placed in a new context, it is not
the same text any more; anyway, it is not available at the places where
the old $T_{1}$ was available.
\end{quote}
