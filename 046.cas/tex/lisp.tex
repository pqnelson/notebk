\chapter{Lisp}

\M
We will use Lisp as our programming language, because it is
\emph{Eulerian} in nature.

\N{Arithmetic Operators}
Neither Common Lisp nor Scheme provide a specification for how
associativity works with their arithmetic operators. We will stipulate
left-associativity:

\begin{lisp-example}
(+ x y z...) := (+ (+ x y) z...)
(- x y z...) := (- (- x y) z...)
(* x y z...) := (* (* x y) z...)
(/ x y z...) := (/ (/ x y) z...)
\end{lisp-example}

\begin{remark}
This only makes a difference when the operations are nonassociative,
e.g., with floating-point arithmetic. But who knows, someone may want to
extend this code to work for Octonions (or Lie algebras, or whatever).
\end{remark}

\begin{remark}
Clojure \emph{does} define its arithmetic operators in the
\verb#core.clj# library, which has no formal definition (which is just
as bad).
\end{remark}

\N{Notation}
We encode $x^{-1}$ as \lstinline[language=lisp]{(/ x)} since it is the
multiplicative inverse of $x$. Confusingly, $x^{m}$ is encoded as
\lstinline[language=lisp]{(expt x m)}, but the source of confusion stems
from the abuse of notation for $x^{-1}$ coincides with $x^{m}$ when $m=-1$.

\section{Pattern Matching}

\M
Much of the heavy lifting will be done for us via ``pattern matching''.
A pattern is just an S-expression with some ``unknowns'' which we match
against a given S-expression. The result will be some dictionary binding
these ``pattern variables'' to data from our input expression.

There are two types of variables appearing in a pattern:
\begin{enumerate}
\item Segment Variables \lstinline[language=lisp]{(?? segment)}, which
  match against a an arbitrary number [zero or more] of elements; and
\item Pattern Variables \lstinline[language=lisp]{(? pattern)}, which match against a single elements.
\end{enumerate}
There is an optional predicate which help us refine matches further, for
example
\lstinline[language=lisp]{(? pattern number?)}. In a segment variable,
this predicate applies to a single argument: the list of elements
matching the segment variable.

\N{Dictionary of Bindings}
When we match a pattern against an expression, we will construct a
key-value dictionary whose keys are the pattern variables and pattern
segments, and whose values are whatever they match against from the
input expression. This is the intended goal for our
\lstinline[language=lisp]{match} function, if it succeeds. If the
pattern is not-applicable to the input expression, then it produces a
\lstinline[language=lisp]{:fail}.

\N{Substitution}
Given some dictionary of bindings, and an expression involving pattern
variables and pattern segments found in the dictionary, we generally
want to replace the pattern variables (and segments) with their
associated values. In other words, we want to ``substitute'' their bound
values using this dictionary.
This will be done using the
\lstinline[language=lisp]{substitute-in} function.

\subsection{Term Rewriting}
\N{Term rewriting}
Term rewriting amounts to checking an element against given a list of ``rules'',
finding the first applicable rule, then produces a new element using the
data produced from the \lstinline[language=lisp]{match} function.

A rule, for us, consists of two or three elements:
\begin{enumerate}
\item A ``pattern'' to match against,
\item Some optional ``guard'' or extra conditions encoded as a predicate
  (with the convention \lstinline[language=lisp]{nil} being ``no guard''
  and not ``constantly false''), and
\item A ``skeleton'' which tells a rewrite-system what to do with the matches.
\end{enumerate}
We say a rule is ``applicable'' if \lstinline[language=lisp]{match}
applied to the pattern against the given input does not
\lstinline[language=lisp]{:fail}.

\N{Evaluation in the skeleton}
If we want to evaluate an expression in the skeleton involving pattern
or segment variables, then we have a special \lstinline[language=lisp]{:eval}
keyword. It substitutes in the values bound to the variables, then
evaluates the expression, and substitutes the result into the skeleton.

\N{Example}
We can consider four rules for what we have asserted for the arithmetic operators.

\begin{lisp-example}
12345678901234567890123456789012345678901234567890123456789012345678901234567890
         0         1         2         3         4         5         6         7
;; addition
((+ (? x number?) (? y number?) (?? z)) ;; pattern
 nil                                    ;; explicitly no guard
 (+ (:eval (+ (? x) (? y))) (?? z)))    ;; skeleton
;; subtraction, with no guard
((- (? x number?) (? y number?) (?? z)) ;; pattern
 (- (:eval (- (? x) (? y))) (?? z)))    ;; skeleton
;; multiplication, with no guard
((* (? x number?) (? y number?) (?? z)) ;; pattern
 (* (:eval (* (? x) (? y))) (?? z)))    ;; skeleton
;; division, with no guard
((/ (? x number?) (? y number?) (?? z)) ;; pattern
 (/ (:eval (/ (? x) (? y))) (?? z)))    ;; skeleton
\end{lisp-example}

\begin{remark}
We could have implemented these rules using segment variables and reduce
it explicitly, which will reduce the amount of time spent recursively
calling the term rewriting function. The new rules would look something like:
\begin{lisp-example}
;; addition
((+ (?? x number?) (?? y))              ;; pattern
 (+ (:eval (reduce + (?? x))) (?? y)))  ;; skeleton
;; subtraction
((- (?? x number?) (?? y))              ;; pattern
 (- (:eval (reduce - (?? x))) (?? y)))  ;; skeleton
;; multiplication
((* (?? x number?) (?? y))              ;; pattern
 (* (:eval (reduce * (?? x))) (?? y)))  ;; skeleton
;; division
((/ (?? x number?) (?? y))              ;; pattern
 (/ (:eval (reduce / (?? x))) (?? y)))  ;; skeleton
\end{lisp-example}
We could go farther and insist the segment variable is not null as the
guard condition, something like:
\begin{lisp-example}
(lambda (dict) (not (null (var-value (?? x) dict))))
\end{lisp-example}
\end{remark}

\begin{remark}
We could have added some optimized rules, like ``multiplication by zero
vanishes''; i.e., if we have any input
resembling \lstinline[language=lisp]{(* (?? x) (? y zero?) (?? z))},
then we can replace it by $0$.
\end{remark}

\subsection{Constructing a Dictionary of Bindings}

\N{Match signature}
The \lstinline[language=lisp]{match} function will take a pattern and an
expression, then produce either a dictionary of bindings (keys are the
pattern and segment variables, the values are the matching elements) or
a \lstinline[language=lisp]{:fail} keyword. In the latter case, we say
the input \emph{failed to match} the pattern; in all other cases, we say
the input \emph{matches} (or \emph{fits}) the pattern.

\begin{algorithm}{Match an S-Expression Against a Pattern}
Given a \lstinline[language=lisp]{pattern} and an
\lstinline[language=lisp]{input}, and an optional
\lstinline[language=lisp]{dict} of bindings [default value for the
  dictionary will be an empty dictionary], we will update the
dictionary, then recursively call the
\lstinline[language=lisp]{match} function with
the rest of the \lstinline[language=lisp]{pattern} and the rest of the
\lstinline[language=lisp]{input} (and the updated dictionary); or else
we \lstinline[language=lisp]{:fail} to make a match and return that.\setcounter{algorithmstep}{-1}

\step [Failure ends now.] If \lstinline[language=lisp]{dict} is
\lstinline[language=lisp]{:fail}, then we bail out here.

\step [Pattern equals input?] If
\lstinline[language=lisp]{pattern} is the same as
\lstinline[language=lisp]{input}, then we return the dictionary of
bindings (we're done matching pattern variables).

\step [Pattern element or segment?] If the \lstinline[language=lisp]{pattern} is a ``pattern
element'', we go to the next step. Otherwise we go to step~\ref{algorithm:step:lisp:matching:trying-to-match-pattern-segment}.

\step [Is the pattern element already in the dictionary?]
Check if the pattern element is already in the dictionary; if so, then
check it is bound to the same value as the first element in the
\lstinline[language=lisp]{input} data --- we
\lstinline[language=lisp]{:fail} if there is disagreement (we are trying
to match one variable against two distinct values). Otherwise, go to the
next step.

\step [Matching a pattern element not in the dictionary.]
If the pattern element \emph{is not} in the dictionary of bindings, then
add the binding, and recursively call \lstinline[language=lisp]{match}
using the rest of the \lstinline[language=lisp]{pattern},
the rest of the \lstinline[language=lisp]{input}, and the updated
dictionary bindings. (It's impossible to fail in this branch.)

\step [Pattern segments.]\label{algorithm:step:lisp:matching:trying-to-match-pattern-segment}
When we're trying to match a pattern segment, well, it's a mess, and I'm
going to carve out that particular case to its own algorithm. Otherwise
go to the next step.

\step [Input must start with a list.]
We should expect the
\lstinline[language=lisp]{input} to be a list whose first element is a list.
If that's not the case, then the pattern has failed (i.e.,
\lstinline[language=lisp]{match} returns \lstinline[language=lisp]{:fail}).
Otherwise go to the final step.

\step [Recurse against the input's head and tail.]
We have the head of the \lstinline[language=lisp]{input} be
a list, so we call \lstinline[language=lisp]{match} using the top of the
pattern and the top of the \lstinline[language=lisp]{input}. We check
the updated bindings to see if we have \lstinline[language=lisp]{:fail}
(in which case, we propagate this as the return value). Otherwise, we
recursively call \lstinline[language=lisp]{match}
with the rest of the \lstinline[language=lisp]{input} and the
rest of the \lstinline[language=lisp]{pattern} using the updated
dictionary bindings.\quad\qedsymbol
\end{algorithm}

\begin{algorithm}{Matching pattern segments}
Given the \lstinline[language=lisp]{pattern},
the \lstinline[language=lisp]{input}, and the current
\lstinline[language=lisp]{dict} bindings dictionary, we will either
\lstinline[language=lisp]{:fail} to make a match or recursely call the
\lstinline[language=lisp]{match} function with updated input, pattern,
and dictionary bindings.

\step [Is pattern segment already bound or not?]
We check if the pattern segment is already in the dictionary. If it is,
then go to step~\ref{step:algorithm:segment-match:check-binding-fits-current-situation}; otherwise, we go to step~\ref{step:algorithm:segment-match:depth-first-search}.

\step [Check the binding fits current situation]\label{step:algorithm:segment-match:check-binding-fits-current-situation}
If the pattern segment \emph{is already bound} (i.e., the dictionary has
a value associated to it), then we check the segment against the data.
If the sublist of \lstinline[language=lisp]{input} agrees with what is
already bound --- so input looks like \lstinline[language=lisp]{(x1 x2 ... xn y...)}
and suppose the pattern segment is bound to \lstinline[language=lisp]{(x1 x2 ... xn)}, then we
do not update the dictionary (we reuse it),
but we skip the \lstinline[language=lisp]{input} ahead to where the
pattern segment ends --- so we use \lstinline[language=lisp]{(y...)} as
the new input --- and we use the rest of the
\lstinline[language=lisp]{pattern} in a recursive call to
\lstinline[language=lisp]{match}. (The ``match pattern segments''
algorithm ends here.)

\step [Depth-first search for longest fit]\label{step:algorithm:segment-match:depth-first-search}
The pattern segment \emph{is not} in the dictionary. We perform a
depth-first search to try to match the \lstinline[language=lisp]{input}
against the \lstinline[language=lisp]{pattern} for as long as possible.
The longest matching segment is used, and since this is depth-first
search we will have recursively called \lstinline[language=lisp]{match}
to produce a dictionary binding for the rest of the pattern, too.
Or we have utterly failed and there is no possible match for the pattern
segment, in which case we propagate the \lstinline[language=lisp]{:fail}.\quad\qedsymbol
\end{algorithm}

\N{On Equality}
There has been a lot of ``if this element matches the value bound'',
which is left vague. What we mean is that the two elements are equal.
Lisp has many different possible interpretations for ``equal'':
\begin{enumerate}
\item equal as pointers,
\item equal as numbers,
\item equal as values,
\item recursively equality on lists and data structures.
\end{enumerate}
We could leave the test for equality as an optional parameter, or we
could follow something like Baker~\cite{baker1993equal} and invent an
\lstinline[language=lisp]{egal} generic function.

\subsection{Substitution}

\M
Now that we have constructed a dictionary of bindings from
\lstinline[language=lisp]{match}-ing some input against a
pattern, what do we do with it? We can construct a function to take an
expression called the \emph{skeleton} (involving the pattern variables
and pattern segments) and our dictionary, then replace all instances of
pattern variables (and pattern segments) with the values associated to
them in the dictionary.

\begin{algorithm}{Substitution in a rule skeleton}%
Given a skeleton and a dictionary of bindings, we will expand
(``hydrate'') the skeleton by replacing instances of pattern variables
(and pattern segments) with their values according to the dictionary.\setcounter{algorithmstep}{-1}

\step [Nil skeletons.] If the skeleton is \lstinline[language=lisp]{nil},
then we return \lstinline[language=lisp]{nil}.

\step [Pattern variable skeletons.] If the skeleton is a pattern variable
\lstinline[language=lisp]{(? x)}, then we lookup the value in our
dictionary and retun that.

\step [List skeletons.]
If the skeleton is a list, then we have several subcases which we
consider starting in the next step. Otherwise, we go to step~\ref{step:algorithm:substitute-in:finished-with-list-subcases}.

\step [Subcase: leads with pattern segment?] If the skeleton starts with
a pattern segment [so the skeleton looks like
\lstinline[language=lisp]{((?? x) e)}], then we lookup the value
associated to the pattern segment \emph{and append} to it the result of
substituting into the subskeleton \lstinline[language=lisp]{e} the
values bound in the dictionary.

\step [Subcase: evaluating expressions.] If the skeleton is
\lstinline[language=lisp]{(:EVAL (stuff))}, then we first recursively
call substitute using \lstinline[language=lisp]{stuff} as the skeleton,
and then we evaluate it; something like
\lstinline[language=lisp]{(eval (substitute-in stuff dict))}.

\step [Subcase: splice sublists.] If the skeleton is
\lstinline[language=lisp]{((:SPLICE (stuff)) e)}, then we hydrate
``stuff'' --- that is, we recursively call
\lstinline[language=lisp]{(substitute-in stuff dict)} --- and append to
it the result of hydrating the rest of the skeleton.

\step [Subcase: skeleton \emph{is} a list.] In the very last case, the
skeleton is a list and we need to process each element in the list. So
we recursively call substitute-in on the head and tail of the list; that
is, if the skeleton looks like \lstinline[language=lisp]{(h . tl)} then
we return
\lstinline[language=lisp]{(cons (substitute-in h dict) (substitute-in tl dict))}.

\step [Skeleton must be literal value]\label{step:algorithm:substitute-in:finished-with-list-subcases}
We have a skeleton which is neither a pattern element, nor a list. What
could it be? Well, we \emph{hope} it's a literal value (it could be
something like a CLOS constructor, or whatever). In this case, we want to ``fall
through'' and just return the skeleton.\quad\qedsymbol
\end{algorithm}


\begin{remark}[Class instances, objects]
There is the problem for Common Lisp (and Clojure) when a skeleton
involves objects --- instances of classes. Common Lisp can use its
sophisticated object system to modify the substitution algorithm to
treat instances using constructors. Clojure, well, I do not know how to
handle that situation.
\end{remark}
