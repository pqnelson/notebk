\chapter{Infinitesimals}

\M
It may be easier to use infinitesimals instead of limits to define
differentiation. This could be done using dual real numbers,
$\RR[\varepsilon]/(\varepsilon^{2})$; i.e., adjoining a ``number''
$\varepsilon\neq0$ such that $\varepsilon^{2}=0$. When we implement it
on a computer, we get something called ``automatic differentiation''.
Then we define the derivative $f'(x)$ of $f(x)$ as satisfying:
\begin{equation}
f(x + \varepsilon) = f(x) + \varepsilon\cdot f'(x).
\end{equation}
If we were being more faithful to the historic perspective, we would
allow for an arbitrary number of these infinitesimals, and redefine
equality so it obeys the ``epsilon principle'': if for each finite
$\varepsilon>0$ we have $|x - y|<\varepsilon$, then $x=y$. This leads us
to the hyperreal number system\index{Hyperreal Number}\index{$\mathbb{R}*$}
usually denoted $\RR*$. Keisler's \textit{Elementary Calculus: An Infinitesimal Approach}
discusses calculus using nonstandard reals.\footnote{\url{https://people.math.wisc.edu/~hkeisler/calc.html}}

\section{History}

\M
The history of infinitesimals (and infinities) is not as simple as
``There exists a `number' [wink wink] $\varepsilon\neq0$ such that
$\varepsilon^{2}=0$.''
Bair \textit{et al.}~\cite{Bair2016:interpreting} review the history of
infinitesimals and the different modern interpretations.

\N{Leibniz and Euclid}
Historically, Leibniz is credited with inventing infinitesimals, but
Leibniz cites Mercator as their inventor. Leibniz was critical of the
perspective that infinitesimals are nilpotent elements in his
\textit{Cum Prodiisset} (1701). Instead, Leibniz appears to have
conceived infinitesimals as a violation of the Archimedean property, as
Leibniz wrote in a 14/24 June 1695 letter to l'Hospital:
\begin{quote}
I use the term \emph{incomparable magnitudes} to refer to [magnitudes]
of which one multiplied by any finite number whatsoever, will be
unable to exceed the other, in the same way [adopted by] Euclid
in the fifth definition of the fifth book [of \textit{The Elements}]
\end{quote}
Modern editions of Euclid have this particular definition be numbered
differently. The modern reader should refer instead to Euclid's \textit{Elements} Book V, definition 4
which states:\footnote{Translation by David E. Joyce, from \url{http://aleph0.clarku.edu/~djoyce/elements/bookV/defV4.html}}
\begin{quote}
Magnitudes are said to have a ratio to one another which can, when multiplied, exceed one another.
\end{quote}
This is usually interpreted as stating the Archimedean property. But
Leibniz interpreted infinitesimals in light of this particular
definitions: we call the magnitude $x$ \define{Infinitesimal with respect to} $y$ (and
$y$ is \define{Infinite with respect to} $x$) if $x < y$ and
there is no integer $n\in\ZZ$ such that $nx > y$.

This, at least, is my reading of Katz and
Sherry~\cite{katz2012:leibniz}. See also \arXiv{2111.00922}
for the competing perspectives of Leibniz's theory of infinitesimals.

\M
Euler seems to treat them as useful fictions, but never uses the
phrase. But Euler also had a strange notion of numbers. For example,
$\sqrt{12}$ is irrational but it's a ``determinate quantity'' because we
can form sequences $a_{n}$ and $b_{n}$ such that for each $n\in\NN$ we have
\begin{enumerate}
\item $a_{n}^{2}<12$ and $12 < b_{n}^{2}$, and
\item $a_{n} < a_{n + 1} < b_{n+1} < b_{n}$.
\end{enumerate}
We know $3^{2}=9<12$ and $4^{2}=16 > 12$, which gives $a_{1}=3$ and $b_{1}=4$.
Then Newton's method applied to these values give us two convergent
sequences which converge to $\sqrt{12}$ and we interpret them as
effective methods of calculating the quantity.
Euler discusses this in \textit{Elements of Algebra} part I, chapter XII
(``Of square roots and of Irrational Numbers resulting from them'').