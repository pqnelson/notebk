\chapter{Integrals}

\M
Really, we should use a table of integrals, it's orders of magnitude
faster than trying to reason about integration. I am curious about how
much we can deduce from just the fundamental theorem of calculus alone.
Arguably, this is just playing around with a differential ring, since
we're doing formal calculations.

\section{Indefinite Integrals}

\N{Definition (Antiderivatives)}
Indefinite integrals (``antiderivatives'') are defined by the
``indefinite Leibniz theorem''
\begin{equation}
\frac{\D}{\D x}\int f(x)\,\D x = f(x).
\end{equation}
This is unique up to some constant of integration $C$, which we may omit
for simplicity.

\begin{theorem}[Fundamental Theorem of Calculus]\label{integrals:thm:ftc}
  Let $f$ be a differentiable function in a single variable. Then
  \[ \int f'(x)\,\D x = f(x) + C\]
  where $C$ is a constant of integration (which we will henceforth omit).
\end{theorem}

\begin{proof}
We begin by writing $F(x)=\int f(x)\,\D x$.
Then we find
\begin{equation}\label{eq:integrals:indefinite:ftc:proof-step}
F'(x) = f(x)
\end{equation}
by definition of the antiderivative (``indefinite Leibniz rule'').
So
\begin{calculation}
\int F'(x)\,\D x
  \step{using Eq~\eqref{eq:integrals:indefinite:ftc:proof-step}}
\int f(x)\,\D x
  \step{folding in the definition of $F(x)$}
F(x).
\end{calculation}
Hence the result.
\end{proof}

\begin{remark}
Our goal is, given $f(x)$, to find a function $F(x)$ such that its
derivative satisfies $F'(x)=f(x)$. We will build up a library of results
and use a few basic tricks.
\end{remark}

\begin{corollary}\label{integrals:u-substitution}
Let $\varphi$ be a differentiable function. Then
\[ \int f\bigl(\varphi(x)\bigr)\varphi'(x)\,\D x = \left.\int f(u)\,\D u\right|_{u=g(x)}.\]
\end{corollary}

\begin{proof}
Let $F(x) = \int f(x)\,\D x$.
Then
\begin{calculation}
(F\circ g)'(x)
  \step{chain rule of differentiation}
F'\bigl(g(x)\bigr)g'(x)
  \step{Leibniz property of antiderivatives}
f\bigl(g(x)\bigr)g'(x)
\end{calculation}
Hence, using Theorem~\ref{integrals:thm:ftc},
\[ \int f\bigl(g(x)\bigr)g'(x)\,\D x = \left.\int f(u)\,\D u\right|_{u=g(x)}.\qedhere\]
\end{proof}

\N{Power rule}\label{integrals:chunk:power-rule} $\displaystyle{\int x^{n}\,\D x = \frac{x^{n+1}}{n+1}}$
for $n\neq-1$.

This follows from the power rule for derivatives and
Theorem~\ref{integrals:thm:ftc}.

\N{Logarithm}\label{prop:integral:logarithm} $\displaystyle{\int x^{-1}\,\D x = \ln(x)}$.

This follows from the derivative of logarithm and
Theorem~\ref{integrals:thm:ftc}.

\begin{corollary}\label{cor:integrals:log-trick}
Let $f$ be a differentiable function. Then
\[ \int\frac{f'(x)}{f(x)}\D x = \ln\bigl(f(x)\bigr).\]
In particularly, if we can find a differentiable function $g(x)$ such
that
\[ \frac{g'(x)}{g(x)} = f(x),\]
then we have
\begin{equation}
\int\frac{g'(x)}{g(x)}\D x = \ln\bigl(f(x)\bigr).
\end{equation}
\end{corollary}

\begin{proof}
From the chain rule applied to $\ln\bigl(f(x)\bigr)$, then integrating
yields the result by Theorem~\ref{integrals:thm:ftc}.
\end{proof}

\N{Exponentials} $\displaystyle{\int \E^{x}\,\D x = \E^{x}}$
and for $a>0$ we have $\displaystyle{\int a^{x}\,\D x = \frac{a^{x}}{\ln(a)}}$.

\begin{proof}
  We recall
  \begin{equation}
\frac{\D}{\D x}\E^{x} = \E^{x}
  \end{equation}
  which proves the first claim by Theorem~\ref{integrals:thm:ftc}. The
  second claim follows from Theorem~\ref{integrals:thm:ftc} applied to
  the fact that
  \begin{equation}
\frac{\D}{\D x}a^{x} = a^{x}\ln(a),
  \end{equation}
  after dividing both sides by $\ln(a)$.
\end{proof}

\N{Sine and Cosine} $\displaystyle{\int \cos(x)\,\D x = \sin(x)}$
and $\displaystyle{\int \sin(x)\,\D x = -\cos(x)}$.

\begin{proof}
This follows from the derivative of sine (or cosine) and Theorem~\ref{integrals:thm:ftc}.
\end{proof}

\M $\displaystyle{\int \frac{\D x}{\sin^{2}(x)} = -\cot(x)}$

\begin{proof}
This follows from the derivative of cotangent being negative cosecant
squared and the fundamental theorem of calculus.
\end{proof}

\M $\displaystyle{\int \frac{\D x}{\cos^{2}(x)} = \tan(x)}$

\begin{proof}
This follows from the derivative of tangent being secant squared and the fundamental theorem
of calculus.
\end{proof}

\M $\displaystyle{\int \frac{\sin(x)}{\cos^{2}(x)}\D x = \sec(x)}$

\begin{proof}
Using $u$-substitution (\S\ref{integrals:u-substitution}),
let $u=\cos(x)$, then $\D u=-\sin(x)\,\D x$ and we have
\begin{equation}
\int \frac{\sin(x)}{\cos^{2}(x)}\D x=\int\frac{-\D u}{u^{2}}.
\end{equation}
Then the power rule (\S\ref{integrals:chunk:power-rule}) yields the result.
\end{proof}

\M $\displaystyle{\int \frac{\cos(x)}{\sin^{2}(x)}\D x = -\csc(x)}$

\begin{proof}
Using $u$-substitution (\S\ref{integrals:u-substitution}),
let $u=\sin(x)$, then $\D u=\cos(x)\,\D x$ and we have
\begin{equation}
\int \frac{\cos(x)}{\sin^{2}(x)}\D x=\int\frac{\D u}{u^{2}}.
\end{equation}
Then the power rule (\S\ref{integrals:chunk:power-rule}) yields the result.
\end{proof}

\M $\displaystyle{\int \tan(x)\,\D x = -\ln\bigl(\cos(x)\bigr)}$.

\begin{proof}
We will be using the logarithm trick (\S\ref{cor:integrals:log-trick}).
This is because we know
\begin{equation}
\frac{1}{\cos(x)}\frac{\D}{\D x}\cos(x) = -\tan(x),
\end{equation}
and
\begin{equation}
\frac{1}{\cos(x)}\frac{\D}{\D x}\cos(x) = \frac{\D}{\D x}\ln\bigl(\cos(x)\bigr).
\end{equation}
The result follows immediately.
\end{proof}

\M $\displaystyle{\int \cot(x)\,\D x = \ln\bigl(\sin(x)\bigr)}$.

\begin{proof}
We will be using the logarithm trick (\S\ref{cor:integrals:log-trick}).
This is because we know
\begin{equation}
\frac{1}{\sin(x)}\frac{\D}{\D x}\sin(x) = \cot(x),
\end{equation}
and
\begin{equation}
\frac{1}{\sin(x)}\frac{\D}{\D x}\sin(x) = \frac{\D}{\D x}\ln\bigl(\sin(x)\bigr).
\end{equation}
The result follows immediately.
\end{proof}

\M $\displaystyle{\int \frac{\D x}{\sin(x)} = \ln\bigl(\tan(x/2)\bigr)}$.

\begin{proof}
We will use the angle-addition formula for sine (\S\ref{prop:double-angle-for-sine})
and the logarithm trick (\S\ref{cor:integrals:log-trick}).
Why? Because
\begin{calculation}
  \frac{1}{\tan(x/2)}\frac{\D}{\D x}\tan(x/2)
  \step{derivative of tangent}
  \frac{1}{\tan(x/2)}\frac{\sec^{2}(x/2)}{2}
  \step{unfolding definitions back to sine, cosine}
  \frac{\cos(x/2)}{\sin(x/2)}\frac{1}{2\cos^{2}(x/2)}
  \step{algebra}
  \frac{1}{2\cos(x/2)\sin(x/2)}
  \step{angle-addition formula for sine (\S\ref{prop:double-angle-for-sine})}
  \frac{1}{\sin(x)}.
\end{calculation}
The result follows from the logarithm trick.
\end{proof}

\M $\displaystyle{\int \frac{\D x}{\cos(x)} = \ln\bigl(\sec(x)+\tan(x)\bigr)}$.

\begin{proof}
We will use the logarithm trick (\S\ref{cor:integrals:log-trick}).
Observe that
\begin{calculation}
  \frac{\D}{\D x}\bigl(\sec(x)+\tan(x)\bigr)
\step{unfolding secant}
  \frac{\D}{\D x}\left(\frac{1}{\cos(x)}+\tan(x)\right)
\step{linearity of derivative}
  \frac{\D}{\D x}\frac{1}{\cos(x)} + \frac{\D}{\D x}\tan(x)
\step{derivative of tangent is secant-squared}
  \frac{\D}{\D x}\frac{1}{\cos(x)} + \sec^{2}(x)
\step{chain rule plus power rule}
  \frac{-1}{\cos^{2}(x)}\frac{\D\cos(x)}{\D x} + \sec^{2}(x)
\step{derivative of cosine}
  \frac{-\sin(x)}{\cos^{2}(x)} + \sec^{2}(x)
\step{folding in the definition of tangent, unfolding one factor of secant}
  \frac{\tan(x)}{\cos(x)} + \frac{\sec(x)}{\cos(x)}
\step{algebra}
  \frac{\tan(x) + \sec(x)}{\cos(x)}.
\end{calculation}
Now, we see that (by dividing both sides by $\tan(x)+\sec(x)$):
\begin{equation}
  \frac{1}{\tan(x) + \sec(x)}\frac{\D}{\D x}\bigl(\sec(x)+\tan(x)\bigr)
= \frac{1}{\cos(x)}.
\end{equation}
The result follows from the logarithm trick (\S\ref{cor:integrals:log-trick}).
\end{proof}

\M\label{integrals:indefinite:arctanh} $\displaystyle{\int\frac{\D x}{1 + x^{2}} = \arctan(x)}$.

\begin{proof}
Let $x=\tan(t)$. Then $\D x = \sec^{2}(t)\,\D t = (1 + x^{2})\,\D t$.
Then
\begin{calculation}
  \int\frac{\D x}{1 + x^{2}}
\step{using $x=\tan(t)$}
  \int\D t
\step{power rule with $n=0$}
  t
\step{since $t=\arctan(x)$}
  \arctan(x).
\end{calculation}
Hence the result.
\end{proof}

\M\label{integrals:one-over-linear-combo-of-x} $\displaystyle{\int\frac{\D x}{a - bx}\,\D x = \frac{-1}{b}\ln(a - bx)}$ for $b\neq0$.

\begin{proof}
By direct calculation,
\begin{calculation}
  \int\frac{\D x}{a - bx}\,\D x
\step{$u= a - bx$, so $\D u = -b\D x$, and by $u$-substitution}
  \frac{-1}{b}\int\frac{\D u}{u}
\step{using Proposition~\ref{prop:integral:logarithm}}
  \frac{-1}{b}\ln(u)
\step{unfolding $u = a - bx$}
  \frac{-1}{b}\ln(a - bx).\qedhere
\end{calculation}
\end{proof}

\M $\displaystyle{\int\frac{\D x}{1 - x^{2}} = \arctanh(x) = \frac{1}{2}\ln\left(\frac{1+x}{1-x}\right)}$.

\begin{proof}
  We see
\begin{equation}\label{eq:integrals:indefinite:arctanh:intermediate-step}
\frac{1}{1 - x} + \frac{1}{1 + x}
= \frac{(1 + x) + (1 - x)}{1 - x^{2}}
= \frac{2}{1 - x^{2}}.
\end{equation}
We see use (\S\ref{integrals:one-over-linear-combo-of-x}) 
\begin{calculation}
  \int\frac{\D x}{1 - x^{2}}
\step{using Eq~\eqref{eq:integrals:indefinite:arctanh:intermediate-step}}
  \int\frac{1}{2}\left(\frac{1}{1 - x} + \frac{1}{1 + x}\right)\D x
\step{using linearity of integral}
  \frac{1}{2}\int\frac{\D x}{1 - x} + \frac{1}{2}\int\frac{\D x}{1+x}
\step{using result (\S\ref{integrals:one-over-linear-combo-of-x})}
  \frac{-1}{2}\ln(1 - x) + \frac{1}{2}\ln(1 + x)
\step{law of logarithms}
  \frac{1}{2}\ln\left(\frac{1+x}{1 - x}\right).
\end{calculation}
Hence one half of the result. The other half of the result follows from
the substitution $x=\I u$, the integral for $1/(1+u^{2})=\arctan(u)$
[i.e., result (\S\ref{integrals:indefinite:arctanh})],
and $\arctanh(x)=-\I\arctan(\I x)$.
\end{proof}

\M $\displaystyle{\int\frac{\D x}{\sqrt{1 - x^{2}}} = \arcsin(x)}$.

\begin{proof}
This follows from the fundamental theorem of calculus and
\begin{equation}
\frac{\D}{\D x}\arcsin(x) = \frac{1}{\sqrt{1 - x^{2}}}.
\end{equation}
(This latter follows from the inverse function theorem.)
\end{proof}

\M $\displaystyle{\int\frac{\D x}{\sqrt{x^{2} - 1}} = \ln\bigl(x + \sqrt{x^{2}-1}\bigr)=\arccosh(x)}$.

\begin{proof}
We see
\begin{calculation}
  \frac{\D}{\D x}(x + \sqrt{x^{2} - 1})
\step{differentiation}
  1 + \frac{x}{\sqrt{x^{2} - 1}}
\step{algebra}
  \frac{\sqrt{x^{2} - 1} + x}{\sqrt{x^{2} - 1}}.
\end{calculation}
Therefore
\begin{equation}
\frac{1}{x + \sqrt{x^{2} - 1}}\frac{\D}{\D x}(x + \sqrt{x^{2} - 1})
= \frac{1}{\sqrt{x^{2} - 1}}.
\end{equation}
The result follows from the logarithm trick (\S\ref{cor:integrals:log-trick}).
\end{proof}

\M $\displaystyle{\int\frac{\D x}{\sqrt{x^{2} + 1}} = \ln\bigl(x + \sqrt{x^{2}+1}\bigr)=\arcsinh(x)}$.

The proof is similar to the previous result.

\M $\displaystyle{\int\sinh(x)\,\D x = \cosh(x)}$.

This follows from the derivative of $\cosh(x)$ and Theorem~\ref{integrals:thm:ftc}.

\M $\displaystyle{\int\cosh(x)\,\D x = \sinh(x)}$.

This follows from the derivative of $\sinh(x)$ and Theorem~\ref{integrals:thm:ftc}.

\M $\displaystyle{\int\frac{\D x}{\sinh^{2}(x)} = -\coth(x)}$.

This follows from the derivative of $\coth(x)$ and Theorem~\ref{integrals:thm:ftc}.

\M $\displaystyle{\int\frac{\D x}{\cosh^{2}(x)} = \tanh(x)}$.

This follows from the derivative of $\coth(x)$ and Theorem~\ref{integrals:thm:ftc}.

\M $\displaystyle{\int\tanh(x)\,\D x = \ln\bigl(\cosh(x)\bigr)}$.

\begin{proof}
  We recall
  \begin{equation}
\frac{\D\cosh(x)}{\D x}=\sinh(x),
  \end{equation}
  so
  \begin{equation}
\frac{1}{\cosh(x)}\frac{\D\cosh(x)}{\D x}=\tanh(x).
  \end{equation}
The result follows from the logarithm trick (\S\ref{cor:integrals:log-trick}).
\end{proof}

\M $\displaystyle{\int\coth(x)\,\D x = \ln\bigl(\sinh(x)\bigr)}$.
\begin{proof}
  We recall
  \begin{equation}
\frac{\D\sinh(x)}{\D x}=\cosh(x),
  \end{equation}
  so
  \begin{equation}
\frac{1}{\sinh(x)}\frac{\D\sinh(x)}{\D x}=\coth(x).
  \end{equation}
The result follows from the logarithm trick (\S\ref{cor:integrals:log-trick}).
\end{proof}

\M $\displaystyle{\int\frac{\D x}{\sinh(x)} = \ln\bigl(\tanh(x/2)\bigr)}$.

\begin{proof}
  We recall
\begin{equation}
\frac{\D}{\D x}\tanh(x/2) = \frac{1}{2}\frac{1}{\cosh^{2}(x)}.
\end{equation}
  Then
\begin{calculation}
  \frac{1}{\tanh(x/2)}\frac{\D}{\D x}\tanh(x/2)
\step{differentiation}
  \frac{1}{2}\frac{1}{\tanh(x/2)}\frac{1}{\cosh^{2}(x)}
\step{unfolding tanh}
  \frac{1}{2}\frac{\cosh(x/2)}{\sinh(x/2)}\frac{1}{\cosh^{2}(x)}
\step{simplification}
  \frac{1}{2\sinh(x/2)\cosh(x/2)}
\step{angle-addition formula for hyperbolic sine}
  \frac{1}{\sinh(x)}.
\end{calculation}
The result follows from the logarithm trick (\S\ref{cor:integrals:log-trick}).
\end{proof}

% \M $\displaystyle{\int\,\D x = }$.