\begin{thebibliography}{99}\setlength{\parskip}{0pt}
  \setlength{\itemsep}{0pt plus 0.3ex}%\footnotesize
\bibitem{Bair2016:interpreting}
  Jacques Bair and Piotr B{\l}aszczyk and Robert Ely and Val{\'{e}}rie Henry and Vladimir Kanovei and Karin U. Katz and Mikhail G. Katz and Semen S. Kutateladze and Thomas McGaffey and Patrick Reeder and David M. Schaps and David Sherry and Steven Shnider,
  ``Interpreting the Infinitesimal Mathematics of Leibniz and Euler''.
  \journal{Journal for General Philosophy of Science}
  \volume{48} no.2 (2016) 195--238
  {\tt\doi{10.1007/s10838-016-9334-z}}
  \arXiv{1605.00455}
\bibitem{baker1993equal}
  Henry G.~Baker,
  ``Equal Rights for Functional Objects, or, the More Things Change, The More They Are the Same''.
  \journal{ACM SIGPLAN OOPS Messenger} \volume{4}, no 4 (1993)
  2--27.\newline
  {\tt\doi{10.1145/165593.165596}}
\bibitem{de-bruijn:aut001}
  N.G.~de Bruijn,
  ``AUTOMATH, a language for mathematics''.
  T.H. Report 66-WSK-05, November 1968.\newline
  \url{https://www.win.tue.nl/automath/archive/webversion/aut001/aut001.html}
\bibitem{chien-lih2005:elementary}
  Hwang Chien-Lih,
  ``89.67 An elementary derivation of Euler's series for the arctangent function''.
  \textit{The Mathematical Gazette}, \textbf{89} no.516 (2005), 469--470.
  {\tt\doi{10.1017/S0025557200178404}}
\bibitem{ferraro2004:coefficients}
  Giovanni Ferraro,
  ``Differentials and differential coefficients in the Eulerian foundations of the calculus''.
  \journal{Historia Mathematica}
  \volume{31} no.1 (2004) 34--61
  {\tt\doi{10.1016/S0315-0860(03)00030-2}}
\bibitem{johansson2020:fungrim}
  Fredrik Johansson,
  ``FunGrim: A Symbolic Library for Special Functions''.
  In: Bigatti, A., Carette, J., Davenport, J., Joswig, M., de Wolff, T.
  (eds)
  \textit{Mathematical Software --- ICMS 2020. ICMS 2020}.
  Lecture Notes in Computer Science(), vol 12097. Springer, 2020.\newline
  {\tt\doi{10.1007/978-3-030-52200-1_31}}.
\bibitem{katz2012:leibniz}
  Mikhail G.\ Katz and David Sherry,
  ``Leibniz's Infinitesimals: Their Fictionality, Their Modern Implementations, and Their Foes from Berkeley to Russell and Beyond''.
  \journal{Erkenntnis} \volume{78} no.3 (2012) 571--625
  {\tt\doi{10.1007/s10670-012-9370-y}} \arXiv{1205.0174}
\bibitem{mayer:calculus}
  Ray Mayer,
  ``Calculus I''.
  Lecture notes, Reed University, 2007.\newline
  \url{http://people.reed.edu/~mayer/math111.html/header/header.html}
\bibitem{mckinzie} Mark McKinzie and Curtis Tuckey,
  ``Higher Trigonometry, Hyperreal Numbers, and Euler's Analysis of Infinities''.
  \journal{Mathematics Magazine} \volume{74}, no.5 (2001) 339--368.\newline
\url{https://www.maa.org/sites/default/files/pdf/upload_library/22/Allendoerfer/2002/0025570x.di021222.02p0075s.pdf}
\bibitem{robinson} Gerson B.~Robison,
  ``A new approach to circular functions, $\pi$, and $\lim\sin(x)/x$".
  \journal{Math.\ Mag.} \volume{41}, no.2 (March 1968), 66--70.\newline
{\tt\doi{10.2307/2689051}}
\end{thebibliography}
