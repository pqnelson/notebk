%%
%% lecture01.tex
%% 
%% Made by alex
%% Login   <alex@tomato>
%% 
%% Started on  Wed Jan  4 11:29:14 2012 alex
%% Last update Wed Jan  4 11:29:14 2012 alex
%%
Today is a bit of motivation, and can easily be disregarded.

Integration amounts to considering cochains and chains. What does
this mean? Well, lets remember that an integral looks like
\begin{equation}
\int_{\Sigma}\omega
\end{equation}
where $\omega$ is the integrand (a differential form, in calculus
or differential geometry) and $\int_{\Sigma}$ specifies where we
take our integral. Alternatively, we could write it out as
\begin{equation}
\<\omega\mid\Sigma\>=\int_{\Sigma}\omega.
\end{equation}
But this requires quite a bit of structure! What if we don't work
with differential geometry? What if we work with topological
spaces? What do we do? We will answer these questions in this course.

For simplicity, we will work with differential forms. Remember
Stoke's theorem
\begin{equation}
\int_{\Sigma}\D\omega=\int_{\partial\Sigma}\omega.
\end{equation}
This is a topological statement. But we have something more:
consider
\begin{equation}
\alpha=\omega+\D\beta
\end{equation}
then we see
\begin{equation}
\begin{split}
\int_{\Sigma}\D\alpha &= \int_{\Sigma}\D(\omega+\D\beta)\\
&= \int_{\Sigma}\D\omega+\D^{2}\beta\\
&= \int_{\Sigma}\D\omega
\end{split}
\end{equation}
which is bad: two distinct differential forms integrate to the
same value. We want to avoid these sorts of problems. Also note
we tacitly used the relationship
\begin{equation}
\D^2=0.
\end{equation}
This is scary, we could have argued
\begin{equation}
\begin{split}
\int_{\Sigma}\D\D\beta 
&=\int_{\partial\Sigma}\D\beta\\
&=\int_{\partial\partial\Sigma}\beta\\
&=0
\end{split}
\end{equation}
which makes more intuitive sense: the boundary of a boundary is
zero. 

So the punchline is: we are working with a certain equivalence
class of differential forms. Really, we have
$\ker(\D)\propersupset\im(\D)$, so we may consider the cosets
$\ker(\D)/\im(\D)$. This is precisely the \define{Cohomology Classes}\index{Cohomology Class}.

%% Well, we work with cell complexes in algebraic topology, and that
%% is sufficient for us to start with. Now we can generalize the
%% notion of a path as a sequence of edges. This is an algebraic
%% nightmare! Why not consider the formal ring $\ZZ[e_1,\dots]$
%% (where $e_k$ is an edge)? Then a path amounts to 
%% \begin{equation}
%% \gamma = e_1 + e_2 - e_3 + \dots + e_k
%% \end{equation}
%% where we add or subtract according to the orientation of the edge
%% (where it points). 

