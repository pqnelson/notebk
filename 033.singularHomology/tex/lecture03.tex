%%
%% lecture03.tex
%% 
%% Made by alex
%% Login   <alex@tomato>
%% 
%% Started on  Thu Jan  5 07:57:17 2012 alex
%% Last update Thu Jan  5 07:57:17 2012 alex
%%

We introduced a notion of homology, for every topological space
$X$ the ntoion of the group of chains $C_{n}(X)$ consisting of
linear combinations $\sum c_n\varphi_n$ of singular cubes. We
then took
\begin{equation}
C(X)=\bigoplus_{n}C_{n}(X).
\end{equation}
We introduced the boundary operator
\begin{equation}
\bdry\colon C_{n}(X)\to C_{n-1}(X),
\end{equation}
and it obeys
\begin{equation}
\bdry^2=0.
\end{equation}
We introduced the group of cycles
\begin{equation}
Z(X)=\ker(\bdry)
\end{equation}
and the group of boundaries
\begin{equation}
B(X)=\im(X),
\end{equation}
thus the Homology group is defined as
\begin{equation}
H_{n}(X)=Z_{n}(X)/B_{n}(X).
\end{equation}
We computed $H_{0}(X)$.

\index{Fundamental Group!relation to $H_{1}(X)$|(}
Lets consider $H_{1}(X)$ and its relation with
$\pi_{1}(X)$. There exists a morphism 
\begin{equation}\label{eq:lec03:fundamentalGroupRelatedToHomology}
\pi_{1}(X)\to H_{1}(X). 
\end{equation}
Why? Well, first if we have the fundamental group, we
should mark a point to construct $\pi_{1}(X)$, but what is the
path? It is really a map
\begin{equation}
I\to X
\end{equation}
which is a singular 1-dimensional cube! So really, what is the
morphism described in Eq \eqref{eq:lec03:fundamentalGroupRelatedToHomology}?
It is a mapping
\begin{equation}
(\mbox{Paths})\to(\mbox{Chains}).
\end{equation}
Moreover, it is a mapping
\begin{equation}
(\mbox{Paths})\to(\mbox{Cycles}).
\end{equation}
Why? Well, the fundamental group is concerned with loops, which
has the starting and ending points be the same. But the boundary
of a loop vanishes. So a loop corresponds to a

\begin{wrapfigure}{r}{6pc}
  \vspace{-12pt}
  \includegraphics{B.img/lecture03.0}
  \vspace{-12pt}
\end{wrapfigure}
\noindent\ignorespaces %
cycle\index{Cycle!Relation with Loop}.
It is clear if we have the concatenation of paths, then it is
mapped to a singular cubed that may be ``divided in two''. So we
have two pictures. One gives a path divided in two pieces, and
another picture gives us a singular cube. So really, for us, it
should not be a problem to say they are the same curve. But
honestly, we should give a formula and prove formally these guys
are ``\emph{Homologous\/}''\index{Homologous}\footnote{The notion
of two guys being ``homologous'' amounts to stating \emph{they
  describe the same cycle.} What does this mean? Well, if $\alpha$ and $\beta$ are homologous $k$-chains, then $\alpha-\beta=\bdry\gamma$ for some $(k+1)$-chain $\gamma$. }. 
This is not hard to do, but we won't do it.

\begin{wrapfigure}{l}{6pc}
  \vspace{-12pt}
  \includegraphics{B.img/lecture03.1}
  \vspace{-12pt}
\end{wrapfigure}
So we have this morphism, what can we say?
We have homotopic paths. We should prove that homotopic paths
gives us homologous cycles. 
What does this mean? Well, lets draw the  homotopy. It is a
cylinder $S^1\times I$, what are the boundaries of this cylinder?
This gives us homologous cycles. Admittedly, this is somewhat
handwavy, but lets not worry about it for now.

We constructed a map $\pi_{1}(X)\to H_{1}(X)$. Now we will assume
$X$ is connected.\marginpar{If $X$ connected, we have $\pi_{1}(X)\onto H_{1}(X)$ surjective} This is a reasoanble assumption. We see that
this morphism is surjective:
\begin{equation}
\pi_{1}(X)\onto H_{1}(X).
\end{equation}
Look, suppose we have some cycle in $H_{1}(X)$, which consists of
several ``cubes'' with vanishing boundary. But really this gives
us something which is the image of a circle. Again, we see that a
cycle consists of several circles. We can always add some path
that comes from $x_{0}$, so we can cover $H_{1}(X)$ by
$\pi_{1}(X)$.
This is not a rigorous proof, just convincing handwaviness.

If $\pi_{1}(X)\onto H_{1}(X)$ is surjective, what about the
kernel of this mapping? Well, $\pi_{1}(X)$ is usually
noncommutative, so the commutator
\begin{equation}
[\pi_{1}(X),\pi_{1}(X)]\to(\mbox{Kernel}),
\end{equation}
thus
\begin{equation}
\pi_{1}(X)/[\pi_{1}(X),\pi_{1}(X)]\to H_{1}(X)
\end{equation}
and it is surjective. One can prove this map is injective. So
$H_{1}$ is simply the
\define{Abelianization}\index{Abelianization} of $\pi_{1}$,
provided $X$ is connected.
\index{Fundamental Group!relation to $H_{1}(X)$|)}

\begin{wrapfigure}{r}{4pc}
  \includegraphics{B.img/lecture03.2}
  \vspace{-12pt}
\end{wrapfigure}
Just an aside: we will give an example of something homotopic to
zero, but not homologous to zero. Consider a handle body, as
doodled to the right. Consider the path drawn in red, near the
boundary of the handle. It is homotopic to zero, but not
homologous to zero.

We\marginpar{Functorial Properties of Homology} will consider the
functorial properties of homology, which will help n performing
calculations. If
\begin{equation}
f\colon X\to Y
\end{equation}
is continuous, then it induces
\begin{equation}
H_{k}\bigl(f\colon X\to Y)\quad=\quad
f_{*}\colon H_{k}(X)\to H_{k}(Y)
\end{equation}
We may construct it by considering the behaviour on chains. We
see a chain is a linear combination $\sum c_n\varphi_n$ where
\begin{equation}
\varphi_n\colon I^k\to X,
\end{equation}
but we see by continuity
\begin{equation}
f\colon\varphi_n\colon I^k\to Y.
\end{equation}
Thus we map chains to chains by 
linearity. We have
\begin{equation}
C_{n}(X) = \left.\begin{pmatrix}\mbox{Linear}\\\mbox{Combination}
\end{pmatrix}\!\right/\!\!\begin{pmatrix}\mbox{Degenerate}\\\mbox{Chains}
\end{pmatrix}
\end{equation}
but by continuity $f_{*}$ maps degenerate cubes to degenerate
cubes. Thus $f_{*}$ maps degenerate chains to degenerate chains
by linearity.

Additionally, we have
\begin{equation}
\partial f_{*}=f_{*}\partial.
\end{equation}
As a consequence, we see it maps cycles to cycles and boundaries
to boundaries. So it maps homology to homology, by construction!

Moreover, $(f\circ g)_{*}=f_{*}\circ g_{*}$. That's obvious, and
if we have an identity map $(\id{})_{*}=\id{*}$. These are two
trivial properties.
All of these properties state we have homology be a functor
$\Top\to\Ab$.

Now, a less trivial statement. Suppose $f,g\colon X\to Y$ and we
have $f\homotopic g$ homotopic. That is, we have
\begin{equation}
F\colon X\times I\to Y
\end{equation}
such that
\begin{equation}
F(-,0)=f(-),\quad\mbox{and}\quad F(-,1)=g(-).
\end{equation}
Then the induced morphisms are the same. In other words:
\begin{thm}
If $f,g\colon X\to Y$ and $f\homotopic g$ homotopic, then they
generate the same map of homology $f_{*}=g_{*}$.
\end{thm}

\begin{wrapfigure}{r}{11pc}
  \vspace{-30pt}
  \centering
  \includegraphics{B.img/lecture03.3}
  \vspace{-24pt}
\end{wrapfigure}
\noindent\emph{Proof.\enspace}\ignorespaces %
Suppose we have a cycle in $X$, we call it $z$. We have a
cylinder obtained by $F_{*}$. This is the handwavy sketch of the
proof, and it is doodled on the right. But we will give the
rigorous details.

We have a regular cube $\varphi\colon I^{k}\to X$. Now, from this
singular cube, we have two singular cubes in $Y$, namely
\begin{subequations}
\begin{equation}
f\circ\varphi\colon I^k\to Y
\end{equation}
and
\begin{equation}
g\circ\varphi\colon I^k\to Y.
\end{equation}
\end{subequations}
But we have more! Namely
\begin{equation}
F\colon X\times I\to Y
\end{equation}
but what does this mean? We take
\begin{equation}
I^{k+1}=I^{k}\times I,
\end{equation}
and we compose
\begin{equation}
\widetilde{\varphi}=\varphi\times\id{I}\colon I^{k+1}\to X\times I
\end{equation}
with $F$, so we end up with
\begin{equation}
F\circ\widetilde{\varphi}\colon I^{k+1}\to Y.
\end{equation}

\begin{wrapfigure}{r}{7pc}
  \vspace{-12pt}
  \centering
  \includegraphics{B.img/lecture03.4}
  \vspace{-12pt}
\end{wrapfigure}      
\noindent\ignorespaces %
What will be the boundary $\bdry(F\circ\widetilde{\varphi})$?
We doodle the $k=1$ case on our right, and we immediately see
that
\begin{equation}
\bdry(F\circ\widetilde{\varphi})=g_{*}\varphi-f_{*}\varphi-F(\bdry\varphi).
\end{equation}
Then we may apply this to any chain. Apply it to $z$, we have
\begin{equation}
\bdry(F\widetilde{z})=g_{*}z-f_{*}z-F(\bdry z).
\end{equation}
In the case when $z$ is a cycle, i.e.\ $\bdry z=0$, we get
$g_{*}z-f_{*}z$ is the boundary to \emph{something!} But this
means $f_{*}=g_{*}$ for homology, which is what we wanted.\hfill\qedsymbol\break
\medskip
This proves the statement made earlier regarding homotopic paths
and homologies.
