%%
%% emergenceNewton.tex
%% 
%% Made by alex
%% Login   <alex@tomato>
%% 
%% Started on  Wed Feb 15 15:37:14 2012 alex
%% Last update Wed Feb 15 15:37:14 2012 alex
%%
Space is seen as ``literally just a storage space for
information'' (6). It turns out to be ``naturally'' associated
with matter. There is an interesting discussion that the maximal
allowed information is finite for each part of space, implying
that it is impossible to localize a particle with infinite
precision at a point of a continuum space; ``in fact'' points and
coordinates arise as derived concepts. ``One could assume that
information is stored in points of a discretized space'' which
has interesting applications with respect to causal sets.

Information is assumed to be stored on surfaces, called
``screens'' apparently. Screens seperate points, and ``thus'' are
``the natural place'' to store information about particles that
move from one side to the other.

\subsection{Force and Inertia}

In analogy to Bekenstein's derivation of black hole
entropy~\cite{Bekenstein:1973ur}, we will consider entropy
related to surface area in flat spacetime. We assume that the
change of entropy associated with the information on the boundary
equals
\begin{equation}
\Delta\entropy=2\pi k_{B}\qquad\hbox{when}\qquad\Delta x=\frac{\hbar}{mc}.
\end{equation}
The reason for the factor of $2\pi$ ``will become apparent
soon.'' Suppose that the change in entropy is linear in $\Delta
x$, that is to say
\begin{equation}
\Delta\entropy=2\pi k_{B}\frac{mc}{\hbar}\Delta x.
\end{equation}

\noindent$\langle$\emph{N.B. for an early proposal for a Goldstone graviton from
Lorentz-Invariance breaking, see
Ohanian~\cite{PhysRev.184.1305}. For more modern ideas, see \cite{Jenkins:2003hw}.}$\rangle$

Recall Unruh demonstrated that an observer in an accelerated
frame experiences a temperature
\begin{equation}
k_{B}T=\frac{1}{2\pi}\frac{\hbar a}{c}
\end{equation}
where $a$ is the acceleration. We see then that
\begin{subequations}
\begin{align}
T &= \frac{1}{2\pi}\frac{\hbar a}{ck_{B}}\\
T\Delta\entropy &= ma\Delta x\\
&=F\Delta x
\end{align}
\end{subequations}
Thus we get our expression
\begin{equation}
T\Delta\entropy = F\Delta x
\end{equation}
relating entropy to force, tacitly deriving Newton's second law.
