%%
%% quantizationScalarField.tex
%% 
%% Made by Alex Nelson
%% Login   <alex@tomato>
%% 
%% Started on  Tue Jul 28 12:07:19 2009 Alex Nelson
%% Last update Tue Jul 28 12:07:19 2009 Alex Nelson
%%

We will apply what we've learned about functional techniques of
quantization to the scalar field. The real aim here is to
demonstrate how to derive Feynman rules.

\subsection{Generalizing from Particles to Fields}

We recklessly claimed that eq
\eqref{eq:functionalProductManyDimensions} holds for any quantum
system (if not, we're claiming it now). So it should hold for a
quantum field theory. In our situation --- the beloved 
scalar field --- the position coordinates $q^i$ are replaced by
the scalar field amplitudes $\phi(\bar{x})$, and the Hamiltonian
is
\begin{equation}%\label{eq:}
H = \int[\frac{1}{2}\pi^{2}+\frac{1}{2}(\nabla\phi)^2+V(\phi)]d^{3}x.
\end{equation}
Thus making the suitable changes, our formula becomes
\begin{equation}%\label{eq:}
\begin{split}
\<\phi_{b}(\bar{x})|&e^{-iHT}|\phi_{a}(\bar{x})\> = \\&\int\mathcal{D}\phi\mathcal{D}\pi\exp\left[i\int^{T}_{0}[\pi\dot{\phi}-\frac{1}{2}\pi^{2}-\frac{1}{2}(\nabla\phi)^{2}-V(\phi)]d^{4}x\right]
\end{split}
\end{equation}
where the functions $\phi(x)$ over which we integrate are
constrained to the specific configurations $\phi_a(\bar{x})$ at
$x^0=0$ and $\phi_b(\bar{x})$ at $x^0=T$. We observe that the
exponential is quadratic in $\pi$ (the canonically conjugate
momenta to the field $\phi$), so we can complete the square and
evaluate to find
\begin{equation}\label{eq:pathIntegralForScalarFieldInGeneral}
\<\phi_b(\bar{x})|e^{-iHT}|\phi_{a}(\bar{x})\> = \int\mathcal{D}\phi\exp\left[i\int^{T}_{0}\mathcal{L}d^{4}x\right],
\end{equation}
where $\mathcal{L} = \frac{1}{2}(\partial_{\mu}\phi)^{2}-V(\phi)$
is the Lagrangian density. The masure $\mathcal{D}\phi$ here
again involves odd constants.

The time integral in the exponent of eq \eqref{eq:pathIntegralForScalarFieldInGeneral}
goes from 0 to $T$ as determined by our choice of what transition
function to computes is, in all other respects, ``manifestly
Lorentz invariant'' \textbf{we claim without proof.} Any other
symmetries that the Lagrangian has are also explicitly preserved
by the functional integral (again, claimed without
proof). Symmetries take an increasingly important role in quantum
field theory, so we take the following (at first seemingly rash)
step: \emph{completely abandon Hamiltonian dynamics} and claim
that eq \eqref{eq:pathIntegralForScalarFieldInGeneral}
\emph{DEFINES} the Hamiltonian dynamics. Any such formula
corresponds to ``some'' Hamiltonian; and we can alsways
differentiate with respect to $T$ to derive the Schrodinger
equation to figure out which, as we did in the previous
section. We can thus consider the Lagrangian $\mathcal{L}$ as the
most fundamental specification of a quantum field
theory\footnote{Although recently there has been some ``blasphemous'' dabbling in working \emph{sans} Lagrangian or Hamiltonian, see Kochan~\cite{Kochan:2008}.}
We'll focus on working with the functional integral to compute
from $\mathcal{L}$ directly, no dependence on $H$ at all.

\subsection{Correlation Functions} In Quantum Field theories, we
usually work with vacuum states denoted as $|0\>$ or sometimes
$|\Omega\>$. There is a special function of interest which can be
interpreted as the amplitude for proagation of a particle or
excitation from $y$ to $x$, we call them ``\emph{Correlation Functions}''.
Here we will often denote them by
\begin{equation}%\label{eq:}
\<\Omega|T\{\phi(x)\phi(y)\}|\Omega\>
\end{equation}
where $T\{-\}$ is the time ordering operator\footnote{The
  interested reader is invited to read Ticciati page
  83~\cite{Ticciati:1999qp}.}, we can remember what it does by
the mnemonic ``Later goes to the left'':
\begin{equation}%\label{eq:}
T\{\phi(x)\phi(y)\} = \begin{cases}
\phi(x)\phi(y)&\text{if }x^0>y^0\\
\phi(y)\phi(x)&\text{if }y^0>x^0.
\end{cases}
\end{equation}
This (the correlation function) is the standard tool we use in
Feynman diagrams, which is used everywhere in quantum field
theory.

The first step therefore towards making direct use of functional
integrals is to find a functional formula for computing
correlation functions. To deduce what it should be consider the
object
\begin{equation}\label{eq:functionalIntegralCorrelationFunction}
\int\mathcal{D}\phi(x)\phi(x_1)\phi(x_2)\exp\left[i\int^{T}_{-T}\mathcal{L}(\phi)d^{4}x\right],
\end{equation}
where the boundary conditions on the path integral are
$\phi(-T,\bar{x})=\phi_{a}(\bar{x})$ and
$\phi(T,\bar{x})=\phi_{b}(\bar{x})$ for some (specified)
$\phi_a$, $\phi_b$. We want to relate this to the two-point
correlation function
\begin{equation}%\label{eq:}
\<\Omega|T\{\phi_{H}(x_1)\phi_{H}(x_2)\}|\Omega\>
\end{equation}
where $\phi_H$ denotes the operator in the Heisenberg picture
(contrasted with $\phi_S$ the operator in the Schrodinger
picture). 

First thing to do is to break up the functional integral in eq
\eqref{eq:functionalIntegralCorrelationFunction} as follows:
\begin{equation}%\label{eq:}
\int\mathcal{D}\phi(x) =
\int\mathcal{D}\phi_{1}(\bar{x})\int\mathcal{D}\phi_{2}(\bar{x})\int_{\substack{
\phi(x^{0}_{1},\bar{x})=\phi_{1}(\bar{x})\\
\phi(x^{0}_{2},\bar{x})=\phi_{2}(\bar{x})
}}\mathcal{D}\phi(x)
\end{equation}
The main functional integral $\int\mathcal{D}\phi(x)$ is now
constrained at times \todo[color=red!40]{\textbf{To Do:}\\ \small{Elaborate on these intermediate states}}$x^{0}_{1}$ and $x^{0}_{2}$ (in addition to
the endpoints $-T$, $T$), but we must integrate seperately over
the intermediate configurations $\phi_1(\bar{x})$,
$\phi_{2}(\bar{x})$. The amplitude is now broken in three, each
being a simple transition amplitude according to eq
\eqref{eq:pathIntegralForScalarFieldInGeneral}. The times
$x^{0}_{1}$, $x^{0}_{2}$ automatically fall into order; e.g. if
$x^{0}_{1}<x^{0}_{2}$ then eq
\eqref{eq:functionalIntegralCorrelationFunction} becomes
\begin{equation}%\label{eq:}
\int\mathcal{D}\phi_{1}(\bar{x})
\int\mathcal{D}\phi_{2}(\bar{x})
\phi_{1}(\bar{x})\phi_{2}(\bar{x})
\<\phi_{b}|e^{-iH(T-x^{0}_{2})}|\phi_2\>
\<\phi_2|e^{-iH(x^{0}_{2}-x^{0}_{1})}|\phi_{1}\>
\<\phi_{1}|e^{-iH(x^{0}_{1}-T)}|\phi_{a}\>.
\end{equation}
We can turn the field $\phi_{1}(\bar{x})$ into a Schrodinger
operator by
\begin{equation}%\label{eq:}
\phi_{S}(\bar{x}_{1})|\phi_{1}\> =
\phi_{1}(\bar{x}_{1})|\phi_{1}\>.
\end{equation}
The completeness relations (which appears to be effectively just
a generalization of the resolution of the identity) demands
\begin{equation}%\label{eq:}
\int\mathcal{D}\phi_{1}|\phi_{1}\>\<\phi_{1}| = \mathbf{1}
\end{equation}
which allows us to eliminate the intermediate state
$|\phi_1\>$. A similar argument holds for eliminating the
intermediate state $|\phi_{2}\>$, which allows us to obtain the
final expression
\begin{equation}%\label{eq:}
\<\phi_b|e^{-iH(T-x^{0}_{2})}\phi_{S}(\bar{x}_2)e^{-iH(x^{0}_{2}-x^{0}_{1})}\phi_{S}(\bar{x}_{1})e^{-iH(x^{0}_{1}+T)}|\phi_{a}\>.
\end{equation}
We can now note most of the exponential factors cancel out to
yield Heisenberg operators. In the case when
$x^{0}_{1}>x^{0}_{2}$, the order of $x_1$ and $X_2$ would be
interchanged! We conclude that eq
\eqref{eq:functionalIntegralCorrelationFunction} is equal to
\begin{equation}%\label{eq:}
\<\phi_b|e^{-iHT}T\{\phi_{H}(x_1)\phi_{H}(x_2)\}e^{-iHT}|\phi_a\>.
\end{equation}
This expression is remarkably close to being correct.

We need to consider the limit as $T\to\infty(1-i\varepsilon)$ for
``small $\varepsilon$''. This trick picks out the vacuum state
$|\Omega\>$ from $|\phi_a\>$ and $|\phi_b\>$ (provided that we
have some overlap between the specified $|\phi\>$ and
$|\Omega\>$, which we assume). For example, if we write
$|\phi_a\>$ as a linear combination of eigenstates $|n\>$ of $H$
we then have
\begin{equation}%\label{eq:}
e^{-iHT}|\phi_{a}\> = \sum_{n}e^{-iE_{n}T}|n\>\<n|\phi_a\>\xrightarrow[T\to\infty(1-i\varepsilon)]{}\<\Omega|\phi_a\>e^{-iE_{0}\cdot\infty(1-i\varepsilon)}|\Omega\>.
\end{equation}
Here this should be taken with a grain of salt, since we have ---
in our usual hand wavy manner --- treated infinity ``as if'' it
were a number. Now explicitly what we do is the following
\begin{equation}%\label{eq:}
\left(\<\phi_{b}|e^{-iHT}\right)T\{\phi(x_1)\phi(x_2)\}\left(e^{-iHT}|\phi_{a}\>\right)\xrightarrow[T\to\infty(1-i\varepsilon)]{}\<\phi_b|\Omega\>\<\Omega|\phi_a\>>e^{-iE_{0}\cdot\infty(1-i\varepsilon)}e^{iE_{0}\cdot\infty(1-i\varepsilon)}\<\Omega|T\{\phi(x_1)\phi(x_2)\}|\Omega\>
\end{equation}
We obtain awkward phase and overlap factors, but
don't worry: this is actually a good thing! These factors cancel
out if we divide them out. That is, we obtain the formula
\begin{equation}\label{eq:functionalIntegralTwoPointCorrelationFunction}
\begin{split}
\<\Omega|&T\{\phi_{H}(x_1)\phi_{H}(x_2)\}|\Omega\> =\\& \lim_{T\to\infty(1-i\varepsilon)}\left(\frac{\displaystyle\int\mathcal{D}\phi(x)\phi(x_1)\phi(x_2)\exp[i\int^{T}_{-T}\mathcal{L}d^{4}x]}{\displaystyle\int\mathcal{D}\phi(x)\exp[i\int^{T}_{-T}\mathcal{L}d^{4}x]}\right)
\end{split}
\end{equation}
This is our desired formula, we have the two-point correlation
function purely from functional integration. We can ask ``What
about higher-point correlation functions?'' The trick is simple,
the numerator simply increases the number of $\phi(x_j)$ factors.

\subsection*{Outline of the Derivation of Feynman Rules}

So here's the general algorithm we'll use when deriving the
Feynman rules: first we are going to define the measure
$\mathcal{D}\phi$ by discretizing spacetime with a cubic lattice;
then we will switch via a Fourier \emph{series} (yes, series, due
to discretization) to change the field from $\phi(x_i)$ to
$\phi(k^{\mu}_{n})$. This simplifies the evaluation of the
discretized functional integral. We make use of analogies made
with discrete vector calculus when considering a generalization
of Gaussian integrals. This results in introducing the notion of
a functional determinant. Although we will try to avoid the
notion of the functional determinant for the most part (since it
turns out it's canceled out in practice), the interested reader
can refer to appendix \ref{appendix:functionDet}.

The approach of implementing a discretized lattice then taking
the continuum limit is clumsy, bulky, and inelegant, but it gives
a good intuition about what's really going on mathematically. But
that isn't a good reason to continue to do long, bulky
calculations. We are going to consider a slicker way to compute
the functional integral using a sort of generalization of the
notion of a generating function. We'll consider this derivation
of Feynman rules in its own subsection.

\subsection{Feynman Rules}

Our next task, to accomplish our dream of deriving Feynman rules
in a functional manner, is to compute various correlation
functions direclty from the RHS of eq \eqref{eq:functionalIntegralTwoPointCorrelationFunction}.
In other words, we will use eq
\eqref{eq:functionalIntegralTwoPointCorrelationFunction} to
derive the Feynman rules for scalar field theory. We'll start
slow, with the two-point function in the free Klein-Gordon
theory, then proceed to generalize to higher
correlations. Finally we'll consider the $\phi^{4}$ theory.

\subsubsection{Free Scalar Field} Consider a noninteracting real
scalar field
\begin{equation}\label{eq:actionFreeScalarField}
S_{0} = \int\mathcal{L}_{0}d^{4}x = \int[\frac{1}{2}(\partial_{\mu}\phi)^2-\frac{1}{2}m^{2}\phi^{2}]d^{4}x.
\end{equation}
Since $\mathcal{L}_{0}$ is quadratic in $\phi$, the functional
integrals in eq
\eqref{eq:functionalIntegralTwoPointCorrelationFunction} take the
form of generalized Gaussian integrals. We will, for this
situation, be able to calculate them exactly.

\subsubsection{Discretization with Cubic Lattice} We want to
define the integral $\mathcal{D}\phi$ over field
configurations. The way we defined it for quantum mechanics was
to consider the functional integral as the limit of a large
number number of integrals. We take the number of integrals to go
to infinity. But we were always working with the discrete setting
(i.e. finitely many integrals) with a discretized path, i.e. the
replacement 
\begin{equation}%\label{eq:}
x(t)\longmapsto x_{k}
\end{equation}
where $k$ indicates the time slice. We perform a similar ritual
and replace the field operators
\begin{equation}%\label{eq:}
\phi(x)\longmapsto\phi(x_k)
\end{equation}
where $x_k$ is some point on the lattice\footnote{It should be
  noted that the lattice shouldn't be pictured as a sort of
  collection of wires, where we can consider values ``on the
  wire'' and anything ``off the wire'' is zero, ignored, or
  undefined. The intuition should be we have a discrete set of
  points, and we specify the connectedness of these points by
  some graph, or network, or something similar.}. Since we have
assumed the lattice is cubic, the spacing is even in every
dimension. For brevity, we will call the spacing
$\varepsilon$. We will also let the four-dimensional volume of
the spacetime region be denoted by $L^4$, and we will define
\begin{equation}%\label{eq:}
\mathcal{D}\phi = \prod_{j}d\phi(x_j)
\end{equation}
up to some constant.

\subsubsection{Fourier Expansion} The field values $\phi(x_j)$
can be represented by a ``discrete'' Fourier series (instead of a
``continuous'' Fourier transform):
\begin{equation}\label{eq:fourierSeriesField}
\phi(x_j) = \frac{1}{V}\sum_{n}e^{-ik_{n}\cdot{x_{j}}}\tilde{\phi}(k_n)
\end{equation}
where $k^{\mu}_{n}=2\pi{n^{\mu}}/L$, with
$n^{\mu}\in\mathbb{Z}^{3,1}$, $|k^{\mu}|<\pi/\varepsilon$, and
$V=L^{4}$. The Fourier coefficients $\tilde{\phi}$ are
complex. However, $\phi$ is real, which implies the coefficients
are constrained by
\begin{equation}%\label{eq:}
\tilde{\phi}^{*}(k) = \phi(-k).
\end{equation}
We will consider the real and imaginary parts of $\tilde{\phi}(k_{n})$
(with $k_n>0$) as independent variables. The change of variables
from the $\phi(x_j)$ to the new $\tilde{\phi}(k_n)$ is a unitary
transformation, enabling us to rewrite the integral as
\begin{equation}%\label{eq:}
\int\mathcal{D}\phi(x) = \prod_{k^{0}_{n}>0}\int
d\re(\tilde{\phi}(k_{n}))
d\im(\tilde{\phi}(k_{n})).
\end{equation}
Later we will take the limit as $L\to\infty$,
$\varepsilon\to0$. The effect of the limit is to recover the
continuum, it converts discrete sums over $k_n$ to continuous
integrals over k:
\begin{equation}%\label{eq:}
\frac{1}{V}\sum_{n}\longmapsto\int\frac{d^{4}k}{(2\pi)^4}.
\end{equation}
In our discussion, this will produce Feynman perturbative theory;
but we will not eliminate the ultraviolet and infrared
divergences of Feynman diagrams. (Renormalization and
Regularization is beyond the scope of this note.)

\subsubsection{Action under Change of Variables} Having now
defined the functional measure, somewhat simplified by our
Fourier analysis, we can now compute the functional integral over
$\phi$ (or more precisely $\tilde{\phi}$). The action for the
free real scalar field \eqref{eq:actionFreeScalarField} can be
rewritten in terms of the Fourier coefficients as
\begin{subequations}
\begin{align}
\int[\frac{1}{2}(\partial_{\mu}\phi)^{2}-\frac{1}{2}m^{2}\phi^{2}]d^{4}x 
&=
\frac{-1}{V}\sum_{n}\frac{1}{2}(m^{2}-k_{n}^{2})|\tilde{\phi}(k_{n})|^{2}\\
&= 
\frac{-1}{V}\sum_{n}\frac{1}{2}(m^{2}-k_{n}^{2})[\re(\tilde{\phi}_{n})^{2}+\im(\tilde{\phi}_{n})^{2}]
\end{align}
\end{subequations}
where we have introduced
$\tilde{\phi}_{n}=\widetilde{\phi}(k_{n})$. The quantity
\begin{equation}%\label{eq:}
(m^{2}-k_{n}^{2}) = (m^{2}+\|\bar{k}_{n}\|^{2}-(k_{n}^{0})^{2})
\end{equation}
is positive provided that $k^{0}_{n}$ is ``not too large''. For
the rest of our calculations, we will assume that this quantity
is positive (i.e. we will evaluate it by analytic continuation
from the region where $\|\bar{k}_{n}\|>k^{0}_{n}$).

\subsubsection{The Gaussian Connection} The denominator of the
two-point correlation function can be written as a product of
Gaussian integrals:
\begin{subequations}
\begin{align}
\int\mathcal{D}\phi e^{iS_{0}} &=
\left(\prod_{k_{n}^{0}>0}\int d\re(\tilde{\phi}(k_{n}))
d\im(\tilde{\phi}(k_{n}))\right)\nonumber\\
\;\;&\times\exp\left[\frac{-i}{V}\sum_{k^{0}_{n}>0}(m^{2}-k_{n}^{2})\|\tilde{\phi}_{n}\|^{2}\right].\\
&= \prod_{k^{0}_{n}>0}\left(\int d\tilde{\phi}_{n} \exp\left[\frac{-i}{V}(m^{2}-k_{n}^{2})\|\tilde{\phi}_{n}\|^{2}\right]\right)
\end{align}
\end{subequations}
We can easily rewrite this by expanding the product, grouping the
real and imaginary terms, yielding the expression
\begin{equation}
\begin{split}
\int\mathcal{D}\phi e^{iS_{0}} =&  \prod_{k^{0}_{n}>0} 
\left(\int d\re(\tilde{\phi}(k_{n}))\exp\left[\frac{-i}{V}(m^{2}-k_{n}^{2})\re(\tilde{\phi}_{n})^{2}\right]\right)\\
&\times
\left(\int d\im(\tilde{\phi}(k_{n}))\exp\left[\frac{-i}{V}(m^{2}-k_{n}^{2})\im(\tilde{\phi}_{n})^{2}\right]\right).
\end{split}
\end{equation}
We can evaluate these integrals by using the usual Gaussian
tricks which yields
\begin{subequations}\label{eq:functionalIntegralAsProductOfGaussians}
\begin{align}
\int\mathcal{D}\phi e^{iS_{0}} =& \prod_{k^{0}_{n}>0}\sqrt{\frac{-i\pi{V}}{m^{2}-k_{n}^{2}}}\sqrt{\frac{-i\pi{V}}{m^{2}-k_{n}^{2}}}\\
=& \prod_{\text{all } k_{n}}\sqrt{\frac{-i\pi{V}}{m^{2}-k_{n}^{2}}}.
\end{align}
\end{subequations}
We used the Gaussian integration formula, but this really ought
to be justified since functional integrals are a foreign beast.

\subsubsection{Functional Determinant Detrimant.}
Recall that before we had to rotate the contour of integration to
$t\to t(1-i\varepsilon)$ to guarantee convergence for the time
integral. This means we should really have to change $k^0\to
k^{0}(1+i\varepsilon)$ in eq \eqref{eq:fourierSeriesField} and
everything that follows from it. In particular, this means that
\begin{equation}%\label{eq:}
(k_{n}^{2}-m^{2})\to(k_{n}^{2}-m^{2}+i\varepsilon).
\end{equation}
The $i\varepsilon$ term gives the necessary convergence factor
for the Gaussian integrals. It also defines the direction of the
analytic continuation that might be necessary to define the
squareroots in eq \eqref{eq:functionalIntegralAsProductOfGaussians}.

Consider an analogous situation with the general Gaussian
integral
\begin{equation}%\label{eq:}
\left(\prod_{k}\int d\xi_{k}\right)\exp[-\xi_{i}B^{ij}\xi_{j}]
\end{equation}
where $B$ is a symmetric matrix with eigenvalues $b_i$. To
evaluate this integral, let $\xi_{i} = {A_{i}}^{j}x_{j}$ where
${A_{i}}^{j}$ is an orthogonal matrix of eigenvectors that
diagonalizes $B$. Changing variables from $\xi_i$ to $x_i$ we
have
\begin{subequations}
\begin{align}
\left(\prod_{k}\int d\xi_{k}\right)\exp[-\xi_{i}B^{ij}\xi_{j}] &= \left(\prod_{k}\int dx_{k}\right)\exp[-\sum_{j}b_{j}x_{j}^{2}]\\
&= \prod_{k}\left(\int \exp[-b_{k}x_{k}^{2}]dx_{k}\right)\\
&= \prod_{k}\sqrt{\pi/b_{k}}\\
&= \operatorname{const}/\sqrt{\det(B)}.
\end{align}
\end{subequations}
The astute reader should note that the numerator diverges, it is
a constant that is of order $\mathcal{O}(k^{n/2})$ and we take
the limit as $n\to\infty$. We typically overlook this
inconvenient fact. Now we should apply this to our problem at hand.

The analogy is made clearer if we first perofrm an integration by
parts to write the Klein Gordon action as
\begin{equation}%\label{eq:}
S_{0}=\frac{1}{2}\int\phi(-\partial^{2}-m^{2})\phi d^{4}x +
\text{boundary terms}.
\end{equation}
Thus the matrix $B$ corresponds to the operator
$(m^2+\partial^2)$ and we can formally write our result as
\begin{equation}%\label{eq:}
\int\mathcal{D}\phi e^{iS_{0}} = (\text{const})[\det(m^2+\partial^2)]^{-1/2}.
\end{equation}
The object on the right hand side is called a ``\emph{functional
  determinant}'', to see one way to calculate it refer to
appendix \ref{appendix:functionDet}. It can, at first, look quite
ill defined; but rest assured, the functional determinant usually
cancels out.

\subsubsection{Two Point Function} 

Now consider the numerator of the two point function as defined
by eq
\eqref{eq:functionalIntegralTwoPointCorrelationFunction}. We need
to Fourier-expand the two extra factors of $\phi$:
\begin{equation}%\label{eq:}
\phi(x_1)\phi(x_2) = \left(\frac{1}{V}\sum_{j}e^{-ik_{j}\cdot{x_1}}\tilde{\phi}_{j}\right)\left(\frac{1}{V}\sum_{l}e^{-ik_{l}\cdot{x_2}}\tilde{\phi}_{l}\right)
\end{equation}
We can plug this in to find the numerator:
\begin{equation}%\label{eq:}
\begin{split}
&\left(\prod_{k^{0}_{n}>0}\int
d\re(\tilde{\phi}_{n})d\im(\tilde{\phi}_{n})\right)\left(\frac{1}{V}\sum_{j}e^{-ik_{j}\cdot{x_1}}\tilde{\phi}_{j}\right)\\
&\times\left(\frac{1}{V}\sum_{l}e^{-ik_{l}\cdot{x_2}}\tilde{\phi}_{l}\right)\exp\left[\frac{-i}{V}\sum_{k^{-}_{n}>0}(m^{2}-k_{n}^{2})\|\tilde{\phi}_{n}\|^{2}\right]\\
&= \frac{1}{V^{2}}\sum_{m,l}e^{-i(k_{m}\cdot x_{1}+k_{l}\cdot
  x_{2})}\\
&\times\prod_{k^{0}_{n}>0}\left(\int
d\re(\tilde{\phi}_{n})d\im(\tilde{\phi}_{n}) (\tilde{\phi}_{l}\tilde{\phi}_{m})\exp\left[\frac{-i}{V}(m^{2}-k_{n}^{2})\|\tilde{\phi}_{n}\|^{2}\right]\right)
\end{split}
\end{equation}
\todo[color=red!40]{\textbf{To Do:}\\ \small{I am skeptical of these calculations, sometime when I have
  the leisure of time to perform the calculations explicitly I
  would love to make certain this reasoning is absolutely correct.}}Now, most of the time the integrand will be odd. There is an
exception when $k_m = \pm k_l$. If $k_m = +k_l$ then the term
involving $\re(\tilde{\phi}_m)^2$ is nonzero but it's exactly
canceled by the term involving $\im(\tilde{\phi}_m)^2$. If
$k_m=-k_l$, we take advantage of the relation
$\tilde{\phi}^{*}(k)=\tilde{\phi}(-k)$. In this situation, the
two terms add. When $k^{0}_{m}<0$, we obtain the following
expression for the numerator:
\begin{equation}%\label{eq:}
\frac{1}{V^{2}}\sum_{m}e^{-ik_{m}\cdot(x_{1}+x_{2})}\left(\prod_{k^{0}_{n}}\frac{-i\pi{V}}{m^{2}-k_{n}^{2}}\right)\frac{-iV}{m^{2}-k_{m}^{2}+i\varepsilon}.
\end{equation}
Note the parenthetic term cancels with the denominator exactly,
leaving us the Feynman propagator when we take the continuum limit.

\todo[color=red!40]{\textbf{To Do:}\\ \small{Type up the
    four-point correlation function calculation, and the
    $\phi^{4}$ model in the discretization scheme.}}

\subsection{Generating Functional Tricks}
%%
%% functionalScalarFieldSlick.tex
%% 
%% Made by Alex Nelson
%% Login   <alex@tomato>
%% 
%% Started on  Sat Aug  1 12:23:40 2009 Alex Nelson
%% Last update Sat Aug  1 12:23:40 2009 Alex Nelson
%%
\subsubsection{Outline}
As promised, we'll introduce a slicker way to compute Feynman
rules using functional derivatives. It's a lot more
mathematically rigorous (and simpler) than the discretization
scheme. The method uses a mathematical gadget which generalizes
the notion of a generating function --- the generating
\emph{functional}. Recall the generating function is used to
compute constants and other useful numbers by taking its
$n^{\text{th}}$ derivative and evaluating it at 0. We
\emph{functionally} do the same thing, take the functional
derivative of the generating functional and evaluate it at 0.


\subsubsection{Properties of the Functional Derivative}
First lets try to review the properties of the functional
$\delta/\delta J(x)$. The functional derivative obeys the basic
property (in four dimensions)
\begin{equation}%\label{eq:}
\frac{\delta}{\delta J(x)}J(y)=\delta^{(4)}(x-y),
\quad\text{or}\quad
\frac{\delta}{\delta J(x)}\int J(y)\phi(y)d^{4}y = \phi(x).
\end{equation}
This can be viewed as a continuous generalization of the vector
calculus derivative 
\begin{equation}%\label{eq:}
\frac{\partial}{\partial x_i}x_j = \delta_{ij}
\quad\text{or}\quad
\frac{\partial}{\partial x_i}\sum_{j}x_{j}k_{j}=k_{i}.
\end{equation}
To take the functional derivatives of more complicated
situations, we use the basic properties of the chain rule and the
product rule. \textbf{Warning:} we \textbf{assume without proof}
that these properties hold, we'll not divulge into the proof
here. So we have situations like the following:
\begin{equation}%\label{eq:}
\frac{\delta}{\delta J(x)}\exp\left[i\int J(y)\phi(y)d^{4}y\right]
=i\phi(x)\exp\left[i\int J(y)\phi(y)d^{4}y\right].
\end{equation}
When the functional depends on the derivative of $J$ we integrate
by parts --- and for all practical purposes, we always can
integrate by parts in quantum field theory --- then apply the
functional derivative as follows:
\begin{equation}%\label{eq:}
\frac{\delta}{\delta J(x)}\int V^{\mu}(y)\partial_{\mu}J(y)d^{4}y
=
\frac{\delta}{\delta J(x)}\left(
\operatorname{bdry terms} + 
\int J(y)\partial_{\mu}V^{\mu}(y)d^{4}y
\right)
= - \partial_{\mu}V^{\mu}(x).
\end{equation}
This concludes our review of basic properties that we'll use
later on.

\subsubsection{Generating Functional}

As alluded to earlier, the basic object of interest is the
generating functional of correlation functions. We denote this
object of interest by $Z[J]$. In a scalar field theory, it's
defined as
\begin{equation}\label{eq:scalarFieldTheoryGeneratingFunctional}
Z[J] \stackrel{\text{def}}{=}\int\mathcal{D}\phi\exp\left[
i\int[\mathcal{L}+J(x)\phi(x)]d^{4}x
\right].
\end{equation}
This is a functional integral over $\phi$. We've merely added to
$\mathcal{L}$ in the exponent an extra term $J(x)\phi(x)$, which
we usually refer to as the ``\emph{source term}''.

\subsubsection{Derivation of Correlation Function}
Now we use it to compute the generating functions for the
Klein-Gordon field (the free scalar field). For example the
two-point function is
\begin{equation}%\label{eq:}
\<0|T\{\phi(x_1)\phi(x_2)\}|0\> = \left.\frac{1}{Z_0}
\left(-i\frac{\delta}{\delta J(x_{1})}\right)
\left(-i\frac{\delta}{\delta J(x_{2})}\right)
Z[J]\right|_{J=0}
\end{equation}
where $Z_0=Z[0]$. Each functional derivative brings down a factor
of $\phi$ in the numerator of $Z[J]$; setting $J=0$ we recover
our desired expression. In more explicit detail, we can compute
\begin{subequations}
\begin{align}
\frac{\delta}{\delta J(x_2)}Z[J] &= \frac{\delta}{\delta J(x_2)}
\int\mathcal{D}\phi\exp\left[
i\int[\mathcal{L}+J(x)\phi(x)]d^{4}x
\right] \\
&= \int\mathcal{D}\phi\frac{\delta}{\delta J(x_2)}
\exp\left[
i\int[\mathcal{L}+J(x)\phi(x)]d^{4}x
\right] \\
&= \int\mathcal{D}\phi i\phi(x_2)
\exp\left[
i\int[\mathcal{L}+J(x)\phi(x)]d^{4}x
\right]
\end{align}
\end{subequations}
This is the effect of one functional derivative, we multiply by
$-i$ to finish one functional operation. We need to do another to
get the expression
\begin{equation}%\label{eq:}
\left(-i\frac{\delta}{\delta J(x_{1})}\right)
\left(-i\frac{\delta}{\delta J(x_{2})}\right)
Z[J]
= \int\mathcal{D}\phi \phi(x_1)\phi(x_2)
\exp\left[
i\int[\mathcal{L}+J(x)\phi(x)]d^{4}x
\right].
\end{equation}
To get the final expression, we need to divide by $Z_0$ and
evaluate at $J=0$ to get
\begin{equation*}%\label{eq:}
\left.\frac{1}{Z_0}
\left(-i\frac{\delta}{\delta J(x_{1})}\right)
\left(-i\frac{\delta}{\delta J(x_{2})}\right)
Z[J]\right|_{J=0} =\displaystyle{
\frac{\displaystyle\int\mathcal{D}\phi\; \phi(x_1)\phi(x_2)\exp\left[i\int[\mathcal{L}]d^{4}x\right]}
{\displaystyle\int\mathcal{D}\phi\exp\left[i\int\mathcal{L}d^{4}x\right]}}
\end{equation*}
which is precisely what is expected.

\subsubsection{Slicker Way to Compute Correlation Functions}
We've seen that eq
\eqref{eq:scalarFieldTheoryGeneratingFunctional} recovers the
expected expression for two-point functions. It's pretty nifty
for us since the free scalar field can be written fairly
easily. It's explicitly written after integrating by parts (on
the first term)
\begin{equation}\label{eq:slickPartialIntegration}
\int[\mathcal{L}_{0}(\phi)+J\phi]d^{4}x 
= 
\int[\frac{1}{2}\phi(-\partial^{2}-m^{2}+i\varepsilon)\phi+J\phi]d^{4}x.
\end{equation}
The factor of $i\varepsilon$ is to guarantee convergence. We
complete the square by introducing a shifted scalar field
\begin{equation}%\label{eq:}
\phi'(x)\stackrel{\text{def}}{=}\phi(x)-i\int D_{F}(x-y)J(y)d^{4}y
\end{equation}
where $D_{F}(x-y)$ is the Feynman propagator --- i.e. the Green's
function of the Klein-Gordon operator, we find that our original
expression \eqref{eq:slickPartialIntegration} becomes
\begin{equation}%\label{eq:}
\begin{split}
\int[&\mathcal{L}_{0}(\phi)+J\phi]d^{4}x =\\
&\int[\frac{1}{2}\phi'(-\partial^{2}-m^{2}+i\varepsilon)\phi']d^{4}x
-\int\frac{1}{2}J(x)\left[-iD_{F}(x-y)\right]J(y)d^{4}y.
\end{split}
\end{equation}
More symbolically, we could rewrite the change of variables as
\begin{equation}%\label{eq:}
\phi'\stackrel{\text{def}}{=}\phi+(-\partial^{2}-m^{2}+i\varepsilon)^{-1}J,
\end{equation}
and the result becomes
\begin{equation}%\label{eq:}
\int[\mathcal{L}_0 + J\phi]d^{4}x = 
\int\left[\frac{1}{2}\phi'(-\partial^{2}-m^{2}+i\varepsilon)\phi'-\frac{1}{2}J(-\partial^{2}-m^{2}+i\varepsilon)^{-1}J\right]d^{4}x.
\end{equation}
This looks nasty, but we are not done yet. We have a few tricks
left.

When we change variables from $\phi$ to $\phi'$, it's just a
shift, so the Jacobian in the functional integral definition of
$Z[J]$ is the identity. The result is
\begin{equation}%\label{eq:}
\underbrace{\int\mathcal{D}\phi'
\exp\left[i\int\mathcal{L}_{0}(\phi')d^{4}x\right]}_{Z_{0}}
\underbrace{\exp\left[-i\int\frac{1}{2}J(x)[-iD_{F}(x-y)J(y)]d^{4}xd^{4}y\right]}_{\text{independent of }\phi'}
\end{equation}
As noted, the second integral is independent of $\phi'$ and the
first is precisely $Z_{0}$. The generating functional of the free
scalar field is thus
\begin{equation}\label{eq:generatingFunctionPostManip}
Z[J] = Z_{0}\exp\left[\frac{-1}{2}\int J(x)D_{F}(x-y)J(y)d^{4}xd^{4}y\right]
\end{equation}
Lets use \eqref{eq:scalarFieldTheoryGeneratingFunctional} and
\eqref{eq:generatingFunctionPostManip} to compute some
correlation functions. The two-point function is by definition
\begin{equation}%\label{eq:}
\<0|T\{\phi(x_1)\phi(x_2)\}|0\>
=\left.
\frac{-\delta}{\delta J(x_{1})}
\frac{\delta}{\delta J(x_{2})}
\exp\left[\frac{-1}{2}\int J(x)D_{F}(x-y)J(y)d^{4}xd^{4}y\right]\right|_{J=0}
\end{equation}
We can evaluate one functional derivative to find
\begin{equation}%\label{eq:}
\<0|T\{\phi(x_1)\phi(x_2)\}|0\>
=\left.
\frac{-\delta}{\delta J(x_{1})}
\left[\frac{-1}{2}\int D_{F}(x_{2}-y)J(y)d^{4}y 
-\frac{1}{2}\int J(x)D_{F}(x-x_{2})d^{4}x
\right]
\frac{Z[J]}{Z_{0}}\right|_{J=0}
\end{equation}
\begin{comment}
We can see that the negatives cancel out nicely, we end up with 2
terms that effectively look like
\begin{equation*}%\label{eq:}
\frac{-\delta}{\delta J(x_{1})}
\left[\frac{-1}{2}\int D_{F}(x_{2}-y)J(y)d^{4}y 
-\frac{1}{2}\int J(x)D_{F}(x-x_{2})d^{4}x
\right]
\propto
\frac{\delta}{\delta J(x_1)}\int D_{F}(x_2-y)J(y)d^{4}y
\end{equation*}
\end{comment}
We can see that this is just what happens after evaluating one
functional derivative of the definition of the two-point
function, which allows us to conclude that
\begin{equation}%\label{eq:}
\<0|T\{\phi(x_1)\phi(x_2)\}|0\> = D_{F}(x_1-x_2).
\end{equation}
This is good because it connects back to what we should already know.

\subsubsection{Example: Four Point Correlation Functions}

This is a rather space-consuming computation, so we need to
introduce some abbreviations. Namely we'll use the conventions
that arguments of functions are subscripts: $\phi_1 = \phi(x_1)$,
$J_x=J(x)$, $D_{x4}=D_{F}(x-x_{4})$, etc. Repeated subscripts
will be integrated over implicitly (like a continuous Einstein's
summation convention). The four-point function is thus
\begin{subequations}
\begin{align}
\<0|T\{\phi_1\phi_2\phi_3\phi_4\}|0\> 
&=\left. \frac{\delta}{\delta J_{1}}
\frac{\delta}{\delta J_{2}}
\frac{\delta}{\delta J_{3}}
[-J_{x}D_{x4}]e^{-\frac{1}{2}J_{x}D_{xy}J_{y}}\right|_{J=0}\\
&=\left. \frac{\delta}{\delta J_{1}}
\frac{\delta}{\delta J_{2}}
[-D_{34}+J_{x}D_{x4}J_{y}D_{y3}]e^{-\frac{1}{2}J_{x}D_{xy}J_{y}}\right|_{J=0}\\
=&\left. \frac{\delta}{\delta J_{1}}
[D_{34}J_{x}D_{x2}+D_{24}J_{y}D_{y3}+J_{x}D_{x4}J_{y}D_{23}]e^{-\frac{1}{2}J_{x}D_{xy}J_{y}}\right|_{J=0}\\
&=D_{34}D_{12}+D_{24}D_{13}+D_{14}D_{23}
\end{align}
\end{subequations}
which is precisely what we expect by Wick's theorem.

\subsubsection{The Beauty of the Generating Functional}

The beauty of the situation is that these calculations are
completely independent of whether things are free or
interacting. The catch is the $Z[J=0]$ factor is not trivial in
the interacting situation. It gives us the sum of the vacuum
diagrams.

