%%
%% quantizationEM.tex
%% 
%% Made by Alex Nelson
%% Login   <alex@tomato>
%% 
%% Started on  Sat Aug  1 13:45:29 2009 Alex Nelson
%% Last update Sat Aug  1 13:45:29 2009 Alex Nelson
%%

We assert the Feynman rule for the photon propagator to be
\begin{equation}%\label{eq:}
\parbox{20mm}{
\begin{fmfgraph*}(50,25)\fmfpen{0.2mm}
    \fmfleft{v1} \fmfright{v2}
    \fmf{photon,label=$\scriptstyle{k}\to$}{v1,v2}
\end{fmfgraph*}}
 = \frac{-ig_{\mu\nu}}{k^{2} + i\varepsilon}.
\end{equation}
Now that we have the basic tools of functional quantization, lets
try to prove it.

Consider the functional integral 
\begin{equation}%\label{eq:}
\int \mathcal{D}A\;e^{iS[A]}
\end{equation}
where $S[A]$ is the action for the free electromagnetic
field. The functional integral is taken over the four components,
i.e. we have
$\mathcal{D}A=\mathcal{D}A^{0}\mathcal{D}A^{1}\mathcal{D}A^{2}\mathcal{D}A^{3}$. We
integrate by parts and Fourier expand to find the action to be
\begin{subequations}
\begin{align}
S &= \int \left[\frac{-1}{4}F_{\mu\nu}F^{\mu\nu}\right]d^{4}x\\
&= \frac{1}{2}\int A_{\mu}(x)\left[\partial^{2}g^{\mu\nu}-\partial^{\mu}\partial^{\nu}\right]A_{\nu}(x)d^{4}x\\
&= \frac{-1}{2}\int\tilde{A}_{\mu}(k)\left[-k^{2}g^{\mu\nu}+k^{\mu}k^{\nu}\right]\tilde{A}_{\nu}(-k)\frac{d^{4}k}{(2\pi)^{4}}.
\end{align}
\end{subequations}
Observe that if $\tilde{A}_{\mu}(k)=k_{\mu}\alpha(k)$ for some
scalar function $\alpha$, the integrand becomes
\begin{subequations}
\begin{align}
\tilde{A}_{\mu}(k)\left[-k^{2}g^{\mu\nu}+k^{\mu}k^{\nu}\right]\tilde{A}_{\nu}(-k)
&=
\alpha(k)k_{\mu}\left[-k^{2}g^{\mu\nu}+k^{\mu}k^{\nu}\right](-k_{\nu})(\alpha(-k))\\
&=\alpha(k)\alpha(-k)\left[k_{\mu}k_{\nu}k^{2}g^{\mu\nu}-k_{\mu}k_{\nu}k^{\mu}k^{\nu}\right]\\
&=\alpha(k)\alpha(-k)[0] = 0.
\end{align}
\end{subequations}
This integrand vanishes for \textbf{any} choice of scalar
function $\alpha$. In this situation, the integrand of the
functional integral $\int \mathcal{D}A e^{iS[A]}$ is 1, which
more importantly implies it is a badly divergent functional
integral. We deduce that
\begin{equation}\label{eq:photonPropagatorSingularity}
\begin{split}
&(\partial^{2}g_{\mu\nu}-\partial_{\mu}\partial_{\nu})D^{\nu\rho}_{F}(x-y)=i{\delta_{\mu}}^{\rho}\delta^{(4)}(x-y)\\
&\text{or}\quad(-k^{2}g_{\mu\nu}+k_{\mu}k_{\nu})\tilde{D}^{\nu\rho}_{F}(k)=i{\delta_{\mu}}^{\rho}
\end{split}
\end{equation}
(which importantly defines the Feynman propagator
$D^{\mu\rho}_{F}$) has no solution. This shouldn't surprise
anyone since $(-k^{2}g_{\mu\nu}+k_{\mu}k_{\nu})$ has a singularity.

The real problem child here is gauge invariance. Recall that
$F_{\mu\nu}$ (and thus $\mathcal{L}$) is invariant under a
general U(1) gauge transformation of the form
\begin{equation}%\label{eq:}
A_{\mu}(x)\to A_{\mu}(x)+\frac{1}{e}\partial_{\mu}\alpha(x)
\end{equation}
where $\alpha(x)$ is any scalar function. This transformation
means that potentials of the form
$A_{\mu}(x)=\frac{1}{e}\partial_{\mu}\alpha(x)$ is gauge
equivalent to 0. The functional integral is badly defined because
we are redundantly integrating over a continuous infginity of
physically equivalent field configurations. That is, we haven't
gauge-fixed the action yet! We need to count each physically
interesting state once.

We can accomplish this gauge fixing by a method due to Faddeev
and Popov (for the original paper,
see~\cite{Faddeev:1967fc}). Let $G(A)$ be a sort of generalized
``indicating functional'' which is zero for a certain gauge
condition, e.g. for the Lorenz gauge we have 
\begin{equation}%\label{eq:}
G(A)=\partial_{\mu}A^{\mu}(x).
\end{equation} 
We can constrain the functional integral to be when $G(A)=0$ by
using a delta function $\delta\left(G(A)\right)$. Geometrically
we could have the intuition that we assign a delta function at
each point $x$. To do this legally we insert 1 in the functional
integral of the form
\begin{equation}\label{eq:insertOne}
1 = \int \mathcal{D}\alpha(x)
\delta\left(G(A^{\alpha})\right)\det\left(\frac{\delta
  G(A^{\alpha})}{\delta \alpha}\right)
\end{equation}
where we have $A^{\alpha}_{\mu}(x) = A_{\mu}(x) +
\frac{1}{e}\partial_{\mu}\alpha(x)$. We see that this condition
\eqref{eq:insertOne} generalizes the identity
\begin{equation}%\label{eq:}
1 = \left(\prod_{j}\int d a_{j}\right)
\delta^{(n)}(\bar{g}(\bar{a}))\det(\partial g_{j}/\partial a_{k})
\end{equation}
for discrete $n$-dimensional vectors. In the Lorenz gauge we have 
\begin{equation}%\label{eq:}
G(A^{\alpha}) = \partial^{\mu}A_{\mu} + \frac{1}{e}\partial^{2}\alpha
\end{equation}
so the functional determinant in our situation (due to our case
being the vacuum) is precisely $\det(\partial^{2}/e)$. For our
situation the only  thing that matters is that the functional
determinant is independent of $A$, i.e. that it's like a constant
term in our functional integral.

After inserting \eqref{eq:insertOne} the functional integral
becomes
\begin{equation}%\label{eq:}
\det\left(\frac{\delta G(A^{\alpha})}{\delta \alpha}\right)\int
\mathcal{D}\alpha\int\mathcal{D}A e^{iS[A]}\delta\left(G(A^{\alpha})\right)
\end{equation}
Now we will change the variable of integration from $A$ to
$A^{\alpha}$. This is a simple shift so
$\mathcal{D}A=\mathcal{D}A^{\alpha}$, and by gauge invariance
$S[A]=S[A^{\alpha}]$. We then obtain
\begin{equation}%\label{eq:}
\int\mathcal{D}A e^{iS[A]} = \det\left(\frac{\delta G(A^{\alpha})}{\delta \alpha}\right)\int
\mathcal{D}\alpha\int\mathcal{D}A^{\alpha} e^{iS[A^{\alpha}]}\delta\left(G(A^{\alpha})\right).
\end{equation}
The functional integral over $A$ is now restricted by the delta
function to physically distinct inequivalent states, as
desired. The infinity comes from a divergent integral over
$\alpha(x)$ which simply gives an infinite multiplicative factor.

To go any further in our investigation, we have to gauge fix
$G(A)$. We choose the general class of functions
\begin{equation}%\label{eq:}
G(A) = \partial^{\mu}A_{\mu}(x) - \omega(x)
\end{equation}
where $\omega(x)$ is any scalar function. Setting this $G(A)$
zero occurs as a generalization of the Lorenz gauge. The
functional determinant is the same as in the Lorenz gauge
\begin{equation}%\label{eq:}
\det\left(\frac{\delta G(A^{\alpha})}{\delta\alpha}\right)=\det(\partial^{2}/e).
\end{equation}
Thus the functional integral becomes:
\begin{equation}%\label{eq:}
\int\mathcal{D}A e^{iS[A]} =
\det(\partial^{2}/e)\left(\int\mathcal{D}\alpha\right)\left(\int
\mathcal{D}A^{\alpha} e^{iS[A^{\alpha}]}\delta\left(\partial^{\mu}A^{\alpha}_{\mu}(x)-\omega(x)\right)\right).
\end{equation}
This equality holds for any $\omega(x)$, so it will hold if we
replace the right hand side with any properly normalized linear
combination involving different functions $\omega(x)$. Our final
trick will be integrating over all $\omega(x)$ with a Gaussian
weighting function centered at $\omega=0$. Our expression will
become
\begin{equation*}%\label{eq:}
\begin{split}
&N(\xi)\int\mathcal{D}\omega\;
\exp\left[-i\int\frac{\omega^{2}}{2\xi}d^{4}x\right]\det(\frac{\partial^{2}}{e})\int\mathcal{D}\alpha\int\mathcal{D}A^{\alpha}\;e^{iS[A^{\alpha}]}\delta(\partial^{\mu}A_{\mu}^{\alpha}(x)-\omega(x))
=\\
&N(\xi)\det(\partial^{2}/e)\int\mathcal{D}\alpha\int\mathcal{D}A^{\alpha}\;e^{iS[A^{\alpha}]}\exp\left[-i\int\frac{(\partial^{\mu}A^{\alpha}_{\mu}(x))^{2}}{2\xi}d^{4}x\right]
\end{split}
\end{equation*}
where $N(\xi)$ is a normalization constant and we have used the
delta function to perform the integral over $\omega$. We can
choose $\xi$ to be any finite constant. What we have really done
is we've effectively added a new term
$-(\partial^{\mu}A_{\mu}^{\alpha})^{2}/(2\xi)$ to the Lagrangian.

What we have done thus far in our functional quantization of the
electromagnetic field is we have worked with the denominator of
our formula for the correlation functions:
\begin{equation}%\label{eq:}
\<\Omega|T\{\mathcal{O}(A)\}|\Omega\>=\lim_{T\to\infty(1-i\varepsilon)}\frac{\displaystyle\int\mathcal{D}A\;\mathcal{O}(A)\exp\left[-i\int^{T}_{-T}\mathcal{L}d^{4}x\right]}{\displaystyle\int\mathcal{D}A\;\exp\left[-i\int^{T}_{-T}\mathcal{L}d^{4}x\right]}.
\end{equation}
We can do the same manipulations for the numerator, provided that
the operator $\mathcal{O}(A)$ is gauge invariant. (If it isn't,
the change of variables trick $A\to A^{\alpha}$ won't work.)
Assuming that the operator $\mathcal{O}(A)$ is in fact gauge
invariant, we find that its correlation function becomes
\begin{equation}%\label{eq:}
\begin{split}
&\<\Omega|T\{\mathcal{O}(A)\}|\Omega\>\\
&\quad = \lim_{T\to\infty(1-i\varepsilon)}
\frac{\displaystyle\int\mathcal{D}A\;\mathcal{O}(A)
\exp\left[-i\int^{T}_{-T}\left(\mathcal{L}-\frac{1}{2\xi}(\partial^{\mu}A_{\mu})^{2}\right)d^{4}x\right]}
{\displaystyle\int\mathcal{D}A\;\exp\left[-i\int^{T}_{-T}\left(\mathcal{L}-\frac{1}{2\xi}(\partial^{\mu}A_{\mu})^{2}\right)d^{4}x\right]}.
\end{split}
\end{equation}
Mathemagically the awkward constants we had previously are all
canceled out. The only trace left behind of our meddling is the
extra $\xi$-term that is added to the action.

We concluded that eq \eqref{eq:photonPropagatorSingularity}
implies that it is not sensible to obtain a photon propagator
from the action $S[A]$. With our new $\xi$-term, however, that
equation becomes
\begin{equation}%\label{eq:}
(-k^{2}g_{\mu\nu}+(1-\frac{1}{\xi})k_{\mu}k_{\nu})\tilde{D}^{\nu\rho}_{F}(k)
  = i {\delta_{\mu}}^{\rho},
\end{equation}
which has the solution
\begin{equation}\label{eq:feynmanPropagatorFromFPprocedure}
\tilde{D}^{\mu\nu}_{F}(k) = \frac{-i}{k^{2}+i\varepsilon}\left(g^{\mu\nu}-(1-\xi)\frac{k^{\mu}k^{\nu}}{k^{2}}\right).
\end{equation}
This is our desired expression for the photon propagator. The
$i\varepsilon$ term in the denominator arises exactly in the same
way as in the Klein-Gordon free field case.

In practice, one usually chooses a specific value of $\xi$ to
actually perform computations. Two choices that are often
convenient are
\begin{align*}
\xi=0&\qquad\text{Landau Gauge}\\
\xi=1&\qquad\text{Feynman Gauge}
\end{align*}
In our notes on the subject of Feynman diagrams, we have used the
Feynman gauge.

The Faddeev-Popov procedure guarantees the value of any
correlation function of gauge invariant operators computed from
Feynman diagrams \emph{will be independent of the value of $\xi$ used
in the calculation} (provided the value of $\xi$ is used
consistently; we can't do half of one computation using the
Landau gauge, then finish up the rest in the Feynman gauge
because it makes things nicer!). It's easy to show that in QED
this assertion is true (I suspect due to the fact that it is an
Abelian gauge group). Note in eq \eqref{eq:feynmanPropagatorFromFPprocedure}
that $\xi$ multiplies a term in the photon propagator
proportional to $k^{\mu}k^{\nu}$. According to the Ward-Takahashi
identity, the replacement in a Green's function of any photon
propagator by $k^{\mu}k^{\nu}$ yields zero, except for terms
involving external off-shell fermions. These terms are equal and
opposite for particle and anti-particle, and vanish when the
fermions are grouped into gauge-invariant combinations.

Now, QED isn't just photon propagators. We have
something else we need to compute: S-matrix elements from
correlation functions of non-gauge-invariant operators $\psi(x)$,
$\overline{\psi}(x)$ and $A_{\mu}(x)$. We will assert that the
S-matrix elements are correctly computed by this procedure. The
S-matrix is really defined between asymptotic states, we can
compute S-matrix elements in a formalism where the coupling
constant is ``turned off'' some ``infinitely long time ago'' in
the past and far in the future. In the zero coupling limit, there
is a clean seperation between gauge-invariant and gauge-variant
states. On the one hand, single-particle states containing one
electron, one positron, or one transversely polzarized photon are
gauge-invariant; while, on the other, states with timelike and
longitudinal photon polarizations transform under gauge
motions. We can thus define a gauge-invariant S-matrix in the
following way: Let $S_{FP}$ be the S-matrix between general
asymptotic states, computed from the Faddeev-Popov procedure. The
matrix is unitary but not gauge invariant (proof?). Let $P_{0}$
be a projection onto the subspace of the space of asymptotic
states in which all particles are either electrons, positrons, or
transverse photons. Let
\begin{equation}\label{eq:SMatrixOnGaugeInvariantStates}
S = P_{0} S_{FP} P_{0}.
\end{equation}
This S-matrix is gauge-invariant by construction, \emph{it is projected
onto gauge-invariant states.} Now it is not obvious that it is
unitary. 

In a handwavy way, we'll summarize the reasoning that this matrix
is in fact unitary. Any matrix element
$\mathcal{M}^{\mu}\epsilon^{*}_{\mu}$ for photon emission
satisfies
\begin{equation}%\label{eq:}
\sum_{i=1,2}\epsilon^{*}_{i\mu}\epsilon_{i\nu}\mathcal{M}^{\mu}\mathcal{M}^{*\nu}=(-g_{\mu\nu})\mathcal{M}^{\mu}\mathcal{M}^{*\nu},
\end{equation}
where the sum on the left hand side runs over transverse
polarizations only. (The same argument applies if
$\mathcal{M}^{\mu}$ and $\mathcal{M}^{*\nu}$ are distinct
amplitudes, as long as they satisfy the Ward identity.) This is
exactly the information we need to see that
\begin{equation}%\label{eq:}
SS^{\dag}=P_{0}S_{FP}P_{0}S_{FP}^{\dag}P_{0}=P_{0}S_{FP}S_{FP}^{\dag}P_{0}.
\end{equation}
Now, we can use the unitarity of $S_{FP}$ to see that $S$ is
unitary, $SS^{\dag}=1$ on the subspace of gauge-invariant
states. It is easy to check explicitly that the formula \eqref{eq:SMatrixOnGaugeInvariantStates}
for the S-matrix is independent of $\xi$: \emph{the Ward identity
implies that any QED matrix element with all external fermions
on-shell is unchanged if we add to the photon propagator
$D^{\mu\nu}(q)$ any term proportional to $q^{\mu}$}.




