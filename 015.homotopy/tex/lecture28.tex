%%
%% lecture28.tex
%% 
%% Made by alex
%% Login   <alex@tomato>
%% 
%% Started on  Tue Dec 27 20:51:34 2011 alex
%% Last update Tue Dec 27 20:51:34 2011 alex
%%


\begin{wrapfigure}{r}{10pc}
  \vspace{-20pt}
  \centering
  \includegraphics{img/lecture28.0}
\end{wrapfigure}
Now we will prove there exists a homotopy exact sequence of a
fibration. Really we are proving the homotopy lifting property
for locally trivial fibations. Lets suppose we have $p\colon E\to
B$ for us the essential caase is a fibration where
\begin{equation*}
p^{-1}\{b\}\homotopic F;
\end{equation*}
and if we have
\begin{equation*}
\psi\colon X\to E
\end{equation*}
where $X$ is some arbitrary (but fixed) topological space, then
we can compose to get a map 
\begin{equation*}
p\circ\psi\colon X\to B.
\end{equation*}
Lets write $\varphi=p\circ\psi$. We may say that $\psi$ lies
above $\varphi$, or $\psi$ is a \define{Lifting}\index{Lifting}
of $\varphi$. Not every map can be lifted, in particular the map
\begin{equation}
\id{B}\colon B\to B
\end{equation}
this lifting map is prcisely what is called a
\define{Section}\index{Section!as Lift of Identity}.

For a locally trivial fibration, every homotopy may be lifted
provided the cell complex $X$ is ``good enough'' (i.e.,
polyhedral). Precisely this means that if we have a map
\begin{equation}
\varphi\colon X\to B
\end{equation}
and suppose we have lifted it to get a map
\begin{equation}
\psi\colon X\to E,
\end{equation}
then
\begin{equation}
\varphi=p\circ\psi.
\end{equation}
But now we know that $\psi$ does not always exist. Assume we have
a homotopy 
\begin{equation}
\varphi_{t}\colon X\to B
\end{equation}
with the property
\begin{equation}
\varphi_0=\varphi,
\end{equation}
then the statement is: if we start with a locally trivial
fibration, there exists a homotopy $\psi_t$ such that
$\psi_0=\psi$ and $p\circ\psi_t=\varphi_t$.

We will prove this by induction (in some sense). We will prove
something stronger, namely this lifting property for a pair. We
assume $X$ is a cell complex, $A\propersubset X$ is a
subcomplex. Or better, a polyhedron and subpolyhedron. This is
the homotopy lifting property for pairs. We have the additional
assumption that homotopy is lifted on $A$.

\begin{thm}
Let $(A,X)$ be a cellular pair, $\psi\colon X\to A$
There exists a homotopy $\psi_t$ such that $\psi_0=\psi$,
$p\circ\psi_t=\varphi_t$, and the lifted homotopy has the given
value on $A$.
\end{thm}

In other words, we may begin on $A$, then extend it to $X$. We
will now prove a particular case.

\bigskip
\noindent\textbf{TODO:} write up the proof.
