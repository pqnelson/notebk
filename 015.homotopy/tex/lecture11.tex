%%
%% lecture11.tex
%% 
%% Made by alex
%% Login   <alex@tomato>
%% 
%% Started on  Wed Dec 28 12:13:19 2011 alex
%% Last update Wed Dec 28 12:13:19 2011 alex
%%

So, recall we discussed van Kampen's theorem. The formulation
involved a notion of ``free product'' of two groups $G_1*G_2$. We
gave two descriptions. One took the generators
\begin{equation}
\<a_{i}\mid r_{k}\>
\end{equation}
of $G_1$ and
\begin{equation}
\<b_{j}\mid s_{m}\>
\end{equation}
of $G_2$. Then we defined $G_1*G_2$ to be 
\begin{equation}
\<a_i,b_j\mid r_k,s_m\>.
\end{equation}
The other explanation looked at the coproduct in the category $\Grp$.
It is then
\begin{equation}
\hom(G_1*G_2,G)=\hom(G_1,G)\times\hom(G_2,G)
\end{equation}
Sometimes it is useful to have a more explicit description,
namely elements of the free product $a_1b_1a_2b_2(\dots)$ where
$a_i\in G_1$ and $b_j\in G_2$ are arbitrary elements. The only
problem is how do we find the product of two guys? It's easy,:
\begin{equation}
(a_1b_1\cdots a_nb_n)(a'_1b'_1\cdots a'_mb'_m)=
a_1b_1\cdots a_nb_na'_1b'_1\cdots a'_mb'_m
\end{equation}
but if $a_{i}=1$ we don't write it. So, for example,
$a_11b_1=a_1b_1$. 

The notion of the free product is closely related to the notion
of free groups. Consider the free groups $\freeGrp{m}$,
$\freeGrp{n}$ with $m$ and $n$ generators respectively. We see
\begin{equation}
\freeGrp{m}*\freeGrp{n}=\freeGrp{m+n}.
\end{equation}
Since
\begin{equation}
\freeGrp{1}\iso\ZZ
\end{equation}
we deduce
\begin{equation}
\freeGrp{m}\iso\underbrace{\ZZ*\dots*\ZZ}_{\text{$m$ times}}.
\end{equation}
But enough humorless group theory!

We have a space $X$ covered by two open sets $A$, $B$. We also
have $A$, $B$, $A\cap B$ be connected. (Pop quiz: is $X$
connected?) We take $\star\in A\cap B$. Then
\begin{equation}
\pi_{1}(X,\star)=\bigl(\pi_{1}(A,\star)*\pi_{1}(B,\star)\bigr)/N
\end{equation}
where $N$ is some normal subgroup. Our question is then regarding
the description of the subgroup. First, a proof of our statement.

We should construct a morphism of the free product to
$\pi_{1}(X,\star)$:
\begin{equation}
\pi_{1}(A,\star)\times\pi_{1}(B,\star)\to\pi_{1}(X,\star)
\end{equation}
We should prove it is surjective. If we write an element of the
free product as $a_1b_1\cdots a_kb_k$. We can easily construct a
morphism
\begin{equation}
i_*\colon\pi_1(A,\star)\to\pi_1(X,\star),
\end{equation}
and we have another morphism
\begin{equation}
j_*\colon\pi_{1}(B,\star)\to\pi_1(X,\star).
\end{equation}
Then we have
\begin{equation}
a_1b_1\cdots a_kb_k\mapsto i_*(a_1)j_*(b_1)(\cdots)
i_*(a_k)j_*(b_k)
\end{equation}
It is a morphism, but we should prove it is surjective. We have
basically done it, because look: we have a space $X$ which is
covered by $A$ and $B$. We can take any loop that goes through
$A$ and $B$ by inserting a loop in $A\cap B$. Then we really have
two loops: one in $A$ and the other in $B$ that agree at the two
points obtained from the loop in $A\cap B$.

We will commit a crime---it's not a felony but it is a
misdemeanor. We have $a_1,\dots,a_n,b_1,\dots,b_m$ be the
generators with relations $r_i$ and $s_j$. We are saying we should
impose new relations, that's what happens when we factorize by
$N$. What are these relations? Consider
\begin{equation}
u\in\pi_1(A\cap B,\star),
\end{equation}
it can be mapped by 
\begin{subequations}
\begin{equation}
\alpha\colon\pi_1(A\cap B,\star)\to\pi_1(A,\star)
\end{equation}
and
\begin{equation}
\beta\colon\pi_1(A\cap B,\star)\to\pi_1(B,\star).
\end{equation}
\end{subequations}
There is no doubt we should add the relation
\begin{equation}
\alpha(u)\sim\beta(u)
\end{equation}
Why? This means they will give the same element in $\pi_1(X,\star)$.
The relations are in $N$, the question is: do we have something
else? No, $N$ is generated by these relations.
\begin{danger}
We are a little bit sloppy here. IF we are being completely
honest, we really mean $(i_*\circ\alpha)(u)=(j_*\circ\beta)(u)$. 
But this is nothing terrible, so we continue to write
$\alpha(u)=\beta(u)$. 
\end{danger}

\begin{thm}[van Kampen]\index{van Kampen Theorem|textbf}
If $a_i$ are generators of $\pi_1(A,\star)$, $r_i$ are the
relations for $\pi_1(A,\star)$, and if $b_i$ are the generators
of $\pi_1(B,\star)$ and $s_j$ are its relations; THEN
$\pi(X,\star)$ has generators $a_i,b_j$ with relations $r_i,s_j$
and $\alpha(u)\sim\beta(u)$ for any $u\in\pi_1(A\cap B,star)$.
\end{thm}

We need to prove that there are no other relations. A precise
proof can be found in
Hatcher~\cite[\S1.2]{hatcher2006algebraic}. A more general
version of the theorem may be found in May~\cite[\S2.7]{may}.

\begin{proof}[Sketch of Proof]
What should we do? We have the following picture: $A$, $B$,
$A\cap B$; they're all open, but this is irrelevant; if we have a
closed path in $X$, then we can divide our path into small pieces
in such a way that every small piece is in either $A$, $B$, or
$A\cap B$. We consider another such path. We consider the
deformation of these paths. But a deformation is a function of a
rectangle. We divide up the rectangle such that they are only in
$A$, $B$, or $A\cap B$. Then each rectangle is a deformation only
in $A$, $B$, or $A\cap B$. This is not a precise proof, which may
be found in Hatcher.
\end{proof}

Now we may apply this theorem in many ways. We may consider this
for connected cell complexes. Our cell complex may be described
in the following way:
\begin{equation}
X=X^1\cup(\mbox{cell})
\end{equation}
Really we may say it is the disjoint union of open sets. We take
\begin{equation}
A=X^1\sqcup\bigl(\mbox{cell}-\mbox{center}\bigr),
\end{equation}
so really we may say that
\begin{equation}
A=X-(\mbox{center of the cell})
\end{equation}
And 
\begin{equation}
B=(\mbox{open cell}).
\end{equation}
At this moment, let us say the dimension of the cell is 2. Let us
try to get the answer for this particular case. First
\begin{equation}
\pi_1(X,*)=\pi_1(X^1)
\end{equation}
since $A$ is homotopically equivalent to its boundary. So we have
$\pi_1(B,*)$ be trivial since it is contractible. What about
$\pi_1(A\cap B)$? Well, we see that 
\begin{equation}
A\cap B\homotopic S^1\times(0,1)\homotopic S^1
\end{equation}
homotopic, so
\begin{equation}
\pi_1(A\cap B)\iso\ZZ.
\end{equation}
We should factorize $\pi_1(X^1)$ by relations from $A\cap B$. If
$u\in A\cap B$, it gives us an element in $X^1$, so it gives us a
relation in $X^1$. Observe, higher dimensional cells do nothing.
