%%
%% lecture21.tex
%% 
%% Made by alex
%% Login   <alex@tomato>
%% 
%% Started on  Sat Dec 31 11:35:19 2011 alex
%% Last update Sat Dec 31 11:35:19 2011 alex
%%
We defined homotopy groups which generalize the fundamental
group. We would like to compute the simplest example. Homotopy
groups are Abelian, and Abelian groups are easy to describe ---
but hard to compute.

The first thing we will do is calculate the homotopy group of
$\RR^n$. It is immediately obvious: the homotopy group is
trivial, since $\RR^n$ is contractible. Alternatively: if we have
a spheroid, then we may contract it.

\index{Homotopy Group!of $S^{n}$|(}
What about $\pi_{k}(S^{n})$? Well, for $k<n$, we have a spheroid
\begin{equation}
f\colon I^k\to S^n
\end{equation}
such that $f(\partial I^{k})$ is the north pole. Well, if $k<n$
and $f$ is ``not too bad'', then it doesn't cover the whole
sphere. This means there is a point $x_0$ which is not contained
in $f(I^k)$. We recall
\begin{equation}
S^{n}-\{x_0\}=\RR^n
\end{equation}
and every spheroid is contractible. But what does it mean for $f$
to be ``not too bad''? We should approximate our map by a
polynomial, or a $C^{r}$ map, or a piecewise-linear map; then
this approximation is homotopic to $f$.

What about $\pi_{n}(S^{n})$? Well, we have
\begin{equation}
\pi_{n}(S^{n})=\homotopyClass(S^n,S^n),
\end{equation}
but why? This is normally not the case! In general we have
\begin{equation}
\homotopyClass(S^k,X)=\begin{pmatrix}
\mbox{orbit of}\\
\pi_{1}(X)\mbox{ in }\pi_{k}(X).
\end{pmatrix}
\end{equation}
%For $n\geq2$, we see
%\begin{
For $\pi_{1}(X)$ Abelian, we see
\begin{equation}
\homotopyClass(S^1,X)=\pi_{1}(X).
\end{equation}
(Recall how $\pi_{1}(X)$ acts on $\pi_{k}(X)$ via inner automorphisms.)
So what happens for $k>1$? Well, we see
\begin{equation}
\pi_{n}(S^{n})=\homotopyClass(S^n,S^n)\xrightarrow{1:1}\ZZ
\end{equation}
We even talked about this before: we proved it for $n=1$, but for
$n>1$ it is also true. If $f\colon S^n\to S^n$, then $f^{-1}(x)$
is an algebraic number called the \define{Degree}\index{Degree!of a Map} of $f$.

What happens for $\pi_{k}(S^n)$ and $k>n$? Consider
$\pi_{3}(S^{2})$, we constructed a map $S^{3}\to S^{2}$, the Hopf
map\index{Hopf Map} and it is not contractible. The explanation
for this, recall the preimages for points were circles. The
preimages for two points are two linked circles. One may see this
group is nontrivial. What about in general? This situation is
very nontrivial. We have
\begin{subequations}
\begin{equation}
\pi_{k}(S^{k})\iso\ZZ
\end{equation}
and
\begin{equation}
\pi_{4k-1}(S^{2k})\iso\ZZ+\begin{pmatrix}\mbox{finite}\\\mbox{group}\end{pmatrix}
\end{equation}
\end{subequations}
All others are finite!
\index{Homotopy Group!of $S^{n}$|)}

We will consider a covering $p\colon\widetilde{X}\to X$ where
$X$, $\widetilde{X}$ are connceted. We would like to consider the
situation with homotopy groups
\begin{equation}
\pi_{k}(\widetilde{X},*)=\pi_{k}(X,*)
\end{equation}
for $k\geq2$. The proof is as follows: ifwe have a path in $X$,
it may be uniquely lifted to a path in $\widetilde{X}$ --- well,
not uniquely, but if we lift the starting point then it's
\emph{unambiguous}. 

So we now have a spheroid in $X$. The trick is to look at a
spheroid\index{Spheroid!as family of paths} as a family of paths
in $X$, then we may lift it to $\widetilde{X}$.\index{Spheroid!Lifting}
But this lifts the paths, does it lift the spheroid? Well, the
initiall point of the paths are lifted to the marked point. When
we lift $f(\partial I^{k})$ to the marked point, then we may lfit
the spheroid. In principle, if
\begin{equation}
f(\partial I^k)\propersubset p^{-1}(*)
\end{equation}
so we have some freedom in lifting.

Remember we considered $G$ acting on a simply connected space
$\widetilde{X}$, we let
\begin{equation}
X=\widetilde{X}/G.
\end{equation}
Assume $G$ acts freely, we proved $G=\pi_{1}(X)$. We can get
every universal covering this way. We now know
$\pi_{n}(X)=\pi_{n}(\widetilde{X})$ for $n\geq2$. We see
therefore that
\begin{equation}
G\iso\pi_{1}(X)
\end{equation}
acts on $\pi_{n}(X)$. But we kknow $\pi_{1}(X)$ acts on
$\pi_{n}(X)$, we merely proved it in a different way. But this is
geometrically obvious. Lets use this for an interesting
calculation.

\index{Homotopy Group!and Universal Covering|(}
\index{Universal Covering!and Homotopy Group|(}
We know the universal covering of $S^1$ is $\RR$, so
\begin{equation}
S^{1}=\RR/\ZZ.
\end{equation}
The action of $\ZZ$ on $\RR$ is $x\mapsto x+n$. We know
\begin{equation}
\pi_{1}(S)=\ZZ,\quad\mbox{and}\quad\pi_{n}(S^{1})=0
\end{equation}
for $n\geq2$. We have $\homotopyClass(S^n,S^1)$ be trivial, for
$n\geq2$.

A more interesting case: we have $\ZZ_2$ act on $S^n$ by
$x\mapsto-x$. We see
\begin{equation}
S^{n}/\ZZ_2=\RP^n
\end{equation}
We know
\begin{equation}
\pi_{1}(\RP^n)=\ZZ_2,\quad\mbox\quad\pi_{k}(\RP^{n})\iso\pi_{k}(S^{n})
\end{equation}
otherwise. In particular
\begin{equation}
\pi_{n}(\RP^n)\iso\ZZ
\end{equation}
Consider $\homotopyClass(S^n,\RP^n)$, we see $\deg(\id{S^n})=1$
and
\begin{equation}
\begin{split}
\varepsilon\colon&S^n\to S^n\\
&x\mapsto-x
\end{split}
\end{equation}
has $\deg(\varepsilon)=\pm1$. The conclusion is, for $n$ even, we
have
\begin{equation}
\homotopyClass(S^n,\RP^n)=\ZZ/(n\sim-n)
\end{equation}
whereas for $n$ odd we have no identification.
\index{Homotopy Group!and Universal Covering|)}%
\index{Universal Covering!and Homotopy Group|)}
