%%
%% lecture25.tex
%% 
%% Made by alex
%% Login   <alex@tomato>
%% 
%% Started on  Tue Jan  3 11:01:27 2012 alex
%% Last update Tue Jan  3 11:01:27 2012 alex
%%

So, last time we discussed the ``\emph{Exact Homotopy Sequence Fibration\/}''
\begin{equation}
\pi_{k}(F)\xrightarrow{i_{*}}
\pi_{k}(E)\xrightarrow{p_{*}}
\pi_{k}(B)\xrightarrow{\partial}
\pi_{k-1}(F)
\end{equation}
where $p\colon E\to B$ forms the fibration, and
\begin{subequations}
\begin{equation}
\varphi\colon I^k\to B 
\end{equation}
is a spheroid, 
\begin{equation}
\psi\colon I^k\to E 
\end{equation}
\end{subequations}
is a spheroid such that 
\begin{equation}
p\circ\psi=\varphi
\end{equation}
holds.

Consider the Hopf fibration
\begin{equation}
S^{3}\xrightarrow{S^1}S^2.
\end{equation}
The sequence we had was
\begin{equation}
\pi_{k}(S^{1})\to\pi_{k}(S^{3})\to\pi_{k}(S^2)\to\pi_{k-1}(S^1).
\end{equation}
Notice for $k=2$ we get
\begin{equation}
\pi_{2}(S^{1})=0\to\pi_{2}(S^{3})\to\pi_{2}(S^2)\to\ZZ\iso\pi_{1}(S^1)\to0=\pi_{1}(S^{3}).
\end{equation}
We should remember that
\begin{equation}
\pi_{2}(S^2)\iso\ZZ
\end{equation}
thus we obtain
\begin{equation}
0\to\pi_{2}(S^{3})\to\ZZ\to\ZZ\to0.
\end{equation}
But we also have
\begin{equation}
\pi_{k}(S^{n})=0\quad\mbox{for }k<n.
\end{equation}
Thus we obtain
\begin{equation}
0\to\pi_{2}(S^{3})=0\to\ZZ\to\ZZ\to0.
\end{equation}
For $k>2$ we find $\pi_{k}(S^1)=0$, thus obtaining a sequence
\begin{equation}
\pi_{k}(S^{1})=0\to\pi_{k}(S^{3})\to\pi_{k}(S^{2})\to0=\pi_{k-1}(S^1).
\end{equation}
This implies
\begin{equation}
\pi_{k}(S^{3})\iso\pi_{k}(S^{2})
\end{equation}
for all $k\geq3$.

We would like to extract some interesting sequences. First we
will introduce a \define{Section}\index{Bundle!Section of a ---|textbf}% 
\index{Fibre Bundle!Section of a ---|textbf}% 
\index{Fibration!Section of a ---|textbf}% 
\index{Section!of a Fibre Bundle|textbf} %
of a fibre bundle $p\colon E\to B$. So we take a single point in
every fibre $F_{b}$, and we do this continuously. In othr words,
it is a mapping
\begin{equation}
q\colon B\to E
\end{equation}
such that
\begin{equation}
q(b)\in F_{b}\quad\forall b\in B.
\end{equation}
Thus
\begin{equation}
(p\circ q)(b)=b
\end{equation}
or simply $p\circ q=\id{B}$.

\begin{Boxed}{Sections in Fibre Bundles}
The intuition should really be guided by thinking of vector
fields in $\RR^3$. Here, our fibre is $F\iso\RR^3$ \emph{the vector space}
and the base is $B=\RR^3$ \emph{the topological space}. What do
we do? Well, it's quite simple: we have a continuous mapping
\begin{equation}
\begin{split}
\RR^{3}&\to E\iso\RR^{3}\times\RR^{3}\\
\boldsymbol{p}&\mapsto(\boldsymbol{p},\vec{v}\,)
\end{split}
\end{equation}
where $\boldsymbol{p}\in\RR^3$ is a point in our topological
space, and $\vec{v}\in\RR^3$ is a vector in our vector space. The
ordered pair $(\boldsymbol{p},\vec{v})$ is a tangent vector with
base point $\boldsymbol{p}$ and vector part $\vec{v}$.

\begin{ddanger}
Note that as a vector bundle, the space $E\iso\RR^3\times\RR^3$
is a \emph{trivial} vector bundle. This enables us to write guys
living in $E$ as an ordered pair. Also note, this is not the
tangent bundle! It resembles it an awful lot, but I am being
lazy and referring to something similar but different.
\end{ddanger}

What if we want to assign something else to each point
``continuously''? For example: assign to each point a
differential operator? Or a group element? How do we handle these
situations? The solution is obvious: work with a
\emph{topological gadget} (e.g., a topological group). What does
this mean? Well, it means the gadget has topological
structure. This enables us to \emph{continuously} assign data to
each point. 

The problem is \emph{not every bundle has sections}. If the
bundle is ``twisted'' too much (i.e., not ``trivial'' enough),
then the topology ``obstructs'' global sections existing. What's
a global section?\marginpar{Global Section defined on every $U\propersubset{B}$; local section defined on some $U$}\index{Section!Global} It is a section defined on $B$, not just a
neighborhood $U\propersubset B$. Compare this to a local section\index{Section!Local},
which is defined on some neighborhood $U\propersubset B$.
\end{Boxed}
The trivial bundle $p\colon B\times F\to B$ has sections, of
course. We just take $f=f(b)$ continuous. Then
\begin{equation}
\begin{split}
q\colon&B\to B\times F\\
&b\mapsto\bigl(b,f(b)\bigr)
\end{split}
\end{equation}
We have a section map $B\to F$.

\begin{wrapfigure}{r}{6pc}
  \centering
  \includegraphics{img/lecture25.0}
  \vspace{-12pt}
\end{wrapfigure}

The Mobius band\index{Mobius Band!as Fibration} is a fibration if
we take the dashed line, doodled to the right. We get a section,
where the fibres are doodled in grey. What's the base space $B$
for the Mobius band? Well, it's a circle. What's the fibre? Well,
$F_{b}\iso I$ it's homeomorphic to the unit interval. So
\emph{locally}, i.e.\ for a sufficiently small neighborhood
$U\propersubset B$, we have $p^{-1}(U)\iso U\times I$.

\index{Section!Existence of a ---|(}
Lets consider a surface, say in $\RR^3\supset\Sigma$. Lets specifically
consider the following picture: our surface is smooth, so we may
speak meaningfully of tangent vectors. If we set $B=\Sigma$, then
$E=T\Sigma$ is our \define{Tangent Bundle}.\index{Bundle!Tangent}\index{Tangent Bundle} 
For our surface $\Sigma$, a tangent vector has two components;
thus our tangent space $T_{x}\Sigma=p^{-1}(x)$ is
two-dimensional. What does this mean? Well,
$T_{x}\Sigma\iso\RR^2$ for all $x\in\Sigma$. Our tangent bundle
is a locally trivial fibration, so at some $U\propersubset\Sigma$
``small neighborhood'', we have $p^{-1}U\iso U\times\RR^2$.

Do we have a section for $T\Sigma$? Well, there is a trivial one:
$q(b)=0$ for any $b\in B$. But generally, a section for a tangent
bundle is a \define{Tangent Vector Field}.\index{Tangent Bundle!Section of}%
\index{Section!Tangent Vector Field}\index{Tangent Vector Field|textbf}
Can we have a nontrivial tangent vector field? Well, just get rid
of zero in the fibre:
\begin{equation*}
E'=E-(B\times\{0\})
\end{equation*}
and consider the fibration
\begin{equation}
p'\colon E'\to B.
\end{equation}
This is really the same guy as $p\colon E\to B$ with the demand
of nontriviality. A section would be a mapping
\begin{equation}
q\colon B\to(E-B\times\{0\}).
\end{equation}
Lets note that the fibre we are working with looks like
\begin{equation}
F'_{b}\iso\RR^{2}-\{0\}.
\end{equation}
So what? Well,
\begin{equation}
F'_{b}\iso S^{1}
\end{equation}
describes the fibre's topology. The million-dollar question: does
a section exist? How to prove that the section exists?

Lets first discuss some very trivial statements. If we have a
fibration $p\colon E\to B$ and a section on our fibration, it is
a map $q\colon B\to E$. Observe: these are maps. So what? Well,
functoriality induces morphisms
\begin{subequations}
\begin{equation}
q_{*}\colon\pi_{n}(B)\to\pi_{n}(E)
\end{equation}
and
\begin{equation}
p_{*}\colon\pi_{n}(E)\to\pi_{n}(B).
\end{equation}
\end{subequations}
We use functoriality and obtain
\begin{equation}
p\circ q=\id{B}\quad\implies\quad p_{*}\circ q_{*}=\id{\pi_{n}(B)},
\end{equation}
which implies $p_{*}$ is surjective. Recall our exact homotopy
sequence for a fibration has
\begin{equation}
\pi_{k}(E)\xrightarrow{p_{*}}\pi_{k}(B)\xrightarrow{\partial}\pi_{k-1}(F)
\end{equation}
So $p_{*}$ surjective gives us
\begin{equation}
\im(p_{*})=\pi_{k}(B)
\end{equation}
and exactness gives us
\begin{equation}
\im(p_{*})=\ker(\partial).
\end{equation}
Together, these imply
\begin{equation}
\pi_{k}(B)=\ker(\partial)
\end{equation}
provided a section exists. Lets reiterate this:
\begin{equation}
\boxed{\quad\mbox{If $\ker(\partial)=\pi_{k}(B)$, then a section exists.}\quad}
\end{equation}
For example, we have our Hopf fibration, does it have a section?
Well, $\pi_{2}(S^3)\to\pi_{2}(S^2)$ is not surjective, so it is
impossible for a section to exist.
\index{Section!Existence of a ---|)}

Lets introduce the notion of a \define{Stiefel
Manifold}\index{Stiefel Manifold} $V_{n,k}$.\marginpar{$V_{n,k}$ consists of $k\times n$ matrices whose columns are \emph{orthonormal} column vectors in $\RR^{n}$}
Here
\begin{equation}
V_{n,k} = \{ (e_1, \dots, e_k) \}
\end{equation}
where $e_{i}$ is a column vector living in $\RR^n$. Morever, we
demand orthonormality
\begin{equation}
\<e_i,e_j\>=\delta_{ij}.
\end{equation}
We may give a different definition: we require $e_1$, \dots,
$e_k$ be linearly independent. Lets denote this different
definition by $\widetilde{V}_{n,k}$.\marginpar{$\widetilde{V}_{n,k}$ consists of $k\times n$ matrices whose columns are \emph{linearly independent} column vectors in $\RR^{n}$}
We see that $V_{n,k}\propersubset\widetilde{V}_{n,k}$. But this
embedding is really a homotopy equivalence. How can we say this?
Well, the Grahm-Schmidt procedure\index{Grahm-Schmidt procedure}
gives us a mapping
\begin{equation}
\widetilde{V}_{n,k}\to V_{n,k}
\end{equation}
which is a homotopy equivalence.

We have a special case
\begin{equation}
V_{n,1}=S^{n-1}
\end{equation}
and
\begin{equation}
\widetilde{V}_{n,1}=\RR^{n}\setminus\{0\}.
\end{equation}
There are other fascinating examples, 
\begin{equation}
V_{n,2}=\{\mbox{normalized tangent vectors to }S^{n-1}\},
\end{equation}
but we also have a fibration
\begin{subequations}
\begin{equation}
V_{3,2}\to S^{2}
\end{equation}
constructed by
\begin{equation}
(e_1,e_2)\mapsto e_1.
\end{equation}
\end{subequations}
The fibre is $S^1$. Writing down the exact sequence for this
fibration is very easy, but it is a Hopf fibration if we replace
$S^{3}$ with $V_{3,2}$.

First $V_{3,2}=\SO{3}$ where $\ORTH{n}$ is the group of
orthogonal matrices, and $\SO{n}$ is the subgroup with unit
determinant. Why is this obvious? First, we have
\begin{equation}
V_{3,3}=\ORTH{3}
\end{equation}
trivially, since each column is orthonormal. We have, given
$(e_1,e_2)\in V_{3,2}$, the third vector $e_{3}$ be orthogonal to
$e_1$ and $e_2$. But we have two options $\pm e_{3}$ are both
orthogonal to $e_1$ and $e_2$! We get a mapping
\begin{equation}
\begin{split}
V_{3,2}\to&V_{3,3}\\
(e_1,e_2)\mapsto & X=(e_1,e_2,se_3)
\end{split}
\end{equation}
where $s=\pm1$ is such that $\det(X)=1$. This gives us a
one-to-one correspondence between $V_{3,2}$ and $\SO{3}$. But
that's a triviality. More generally, we have
\begin{equation}
V_{n,n-1}=\SO{n}
\end{equation}
but
\begin{equation}
V_{3,2}\iso\RP^2.
\end{equation}
Then $\pi_{1}(V_{3,2})\iso\ZZ_2$. We see our exact homotopy
sequence is
\begin{equation}
\pi_{2}(V_{3,2})\to\pi_{2}(S^{2})\iso\ZZ\to\ZZ_2\to 0
\end{equation}
but it then follows that
\begin{equation}
\pi_{2}(V_{3,2})\to\ZZ
\end{equation}
is \emph{not} surjective. Thus this fibration has no
section. There exists no nonzero tangent vector field to the
sphere. The moral: \emph{Don't bring hedehog to barber shop}.%
\index{Hedgehog!Should Avoid Barbers}

