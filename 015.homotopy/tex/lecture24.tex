%%
%% lecture24.tex
%% 
%% Made by alex
%% Login   <alex@tomato>
%% 
%% Started on  Tue Jan  3 09:30:19 2012 alex
%% Last update Tue Jan  3 09:30:19 2012 alex
%%

We will see how to apply our theorem. Recall we have a space $X$,
a subset $A\propersubset X$, and a point $*\in A$. We considered
homotopy groups $\pi_{n}(A,*)$ but every spheroid of $A$ is a
spheroid in $X$. We thus get a morphism
\begin{equation}
\pi_{n}(A,*)\to\pi_{n}(X,*).
\end{equation}
But we can consider an absolute spheroid in $X$ as a relative
spheroid, thus we get a morphism
\begin{equation}
\pi_{n}(X,*)\to\pi_{n-1}(X,A,*).
\end{equation}
Thus we get our exact sequence
\begin{equation}
\dots\to\pi_{n}(A,*)\to\pi_{n}(X,*)\to\pi_{n-1}(X,A,*)
\to\pi_{n-1}(A,*)\to\dots
\end{equation}
We have to check that
\begin{equation}
\pi_{n}(A,*)\to\pi_{n}(X,*)\to\pi_{n-1}(X,A,*)
\end{equation}
is exact. Lets first introduce some notation:
\begin{subequations}
\begin{align}
i_{*}\colon&\pi_{n}(A,*)\to\pi_{n}(X,*)\\
j_{*}\colon&\pi_{n}(X,*)\to\pi_{n}(X,A,*)\\
\partial\colon&\pi_{n}(X,A,*)\to\pi_{n-1}(A)
\end{align}
\end{subequations}
Then we have
\begin{equation}
\partial\circ j_{*}=0,\quad j_{*}\circ i_{*}=0,\quad
i_{*}\circ\partial=0.
\end{equation}
So we have
\begin{equation}
\im(i_{*})\subset\ker(j_*),\quad\im(j_*)\subset\ker(\partial),\quad
\im(\partial)\subset\ker(i_*).
\end{equation}
In other words, this sequence is a \define{Complex}\index{Complex|textbf}.
Last time we proved that
\begin{equation}
\im(i_{*})\supset\ker(j_*),\quad\im(j_*)\supset\ker(\partial),\quad
\im(\partial)\supset\ker(i_*).
\end{equation}
which imply that these are equal and our sequence is exact.

Consider  $X=\bar{D}^n$, $A=S^{n-1}$, and $*\in A$. We can write
down an exact sequence:
\begin{equation}
\pi_{k}(S^{n-1})\to\pi_{k}(\bar{D}^{n})\to\pi_{k}(\bar{D}^{n},S^{n-1})\to
\pi_{k-1}(S^{n-1})\to\dots 
\end{equation}
We definitely know that $\bar{D}^n$ is contractible, which means
they have trivial homotopy groups. We obtain
\begin{equation}
\pi_{k}(S^{n-1})\to0\to\pi_{k}(\bar{D}^{n},S^{n-1})\to
\pi_{k-1}(S^{n-1})\to\dots 
\end{equation}
which implies an isomorphism.\marginpar{Important: $0\to A\into B$ and $A\onto B\to 0$ for exact sequences}
If $0\to A\into B$ where we have $A\into B$ injective, then
$\im(A\into B)\propersubset B$. Similarly, we have $A\onto B\to
0$ imply the kernel is contained in the image and, being sloppy
with notation, have $B\propersubset A$. Thus $0\to A\to B\to 0$
implies $A\iso B$.

So what? Well, this implies
\begin{equation}
\pi_{k}(\bar{D}^{n},S^{n-1})\iso
\pi_{k-1}(S^{n-1})
\end{equation}
That is the end of the story.

\subsection{Homotopy Groups of Fibrations}
We have a notion of fibration $p\colon E\to B$ surjective, and
\begin{equation}
p^{-1}\{b\}=F_{b}\iso F
\end{equation}
for any $b\in B$, where $F$ is the fibre. A \define{Locally
  Trivial Fibration}\index{Locally Trivial Fibration}\index{Fibration!Locally Trivial}
is one which locally behaves as a direct product. So for
some neighborhood $U\propersubset B$ , we have
\begin{equation}
p^{-1}(U)\iso U\times F.
\end{equation}
We may consider a fibre
\begin{equation}
p^{-1}\{b\}=F_{b}\propersubset E
\end{equation}
we may compute the relative homotopy groups of this pair:
$\pi_{k}(E,F_{b},*)$ where $*\in F_{b}$. So
\begin{equation}
p(*)=b=*
\end{equation}
we mark a point in the base
\begin{equation}
b\in B.
\end{equation}
Lets clarify notation a bit: We have $b=*$ in base $B$, and in
$F_{b}$ the marked point denoted $*$. We have $F$ be the fibre
over the marked point. \emph{Immediately} relative homotopy
groups of $E$ relative to $F$ is mapped to the absolute homotopy
group of $B$:
\begin{equation}
p_{*}\colon\pi_{n}(E,F,*)\to\pi_{n}(B,*).
\end{equation}
We have a fibration over the base.

\begin{thm}
We have this morphism $p_{*}$ be an isomorphism if the fibration
is locally trivial.
\end{thm}

There is a notion of a cell fibration\index{Cell Complex!Fibration Over a ---}%
\index{Cell Fibration|textbf}\index{Bundle!Cellular ---}%
\index{Fibration!Cell}\index{Fibre Bundle!Cellular ---}%
 defined in this way. Using this theorem for complexes, if 
\begin{equation}
\pi_{n}(X,A)\iso\pi_{n}(B)
\end{equation}
then it's locally trivial, etc. So in other words, our cell
complex $X$ corresponds to $E$, and the subcomplex $A$ correspond
to the ``fibre''. When we use this analogy on our cell complex,
we get a cell fibration.
The only thing we need is the ability to lift from a spheroid in
$B$ to a relative spheroid. We will elaborate later, now we will
focus on examples.

If $(E,F)$ are a pair, we may consider the exact sequence
\begin{equation}
\dots\to\pi_{n}(F)\to\pi_{n}(E)\to\pi_{n}(E,F)\to\pi_{n-1}(F)\to\dots.
\end{equation}
Topological results are invariant with respect to letters used
for variables. But recall our theorem, that is
\begin{equation}
\pi_{n}(E,F)\iso\pi_{n}(B).
\end{equation}
So we can write
\begin{equation}
\dots\to\pi_{n}(F)\to\pi_{n}(E)\to\pi_{n}(B)\to\pi_{n-1}(F)\to\dots
\end{equation}
we have an \define{Exact Homotopy Sequence of Fibrations}.%
\index{Fibration!Exact Homotopy Sequence of ---}%
\index{Exact Sequence!Homotopy Groups of Fibration}%
\index{Fibre Bundle!Exact Homotopy Sequence of ---}%
\index{Homotopy Group!Sequence for Fibration}%
This permits me to consider, to calculate homotopy groups and
everyt fibration gives some information. So first of all, we need
examples of fibrations. We have one! The trivial fibration, but
that is a triviality.

\index{Hopf Fibration!Generalization of ---|(}
But we have also the Hopf fibration, where we have $E=S^3$,
$B=S^2$ and $F=S^1$. For a more general picture, we had the
complex projective space which is defined in terms of $(n+1)$
complex numbers defined up to a factor $(z_0 : \dots : z_n)$. It
is a sphere of dimension $S^{2n+1}$, when we factorize by an
action of $S^{1}$ we get
\begin{equation}
S^{2n+1}/S^{1}=\CP^{n}.
\end{equation}
We may consider a fibration $S^{2n+1}\to\CP^{n}$ with the fibre
$S^{1}$. In the case $n=1$, we have $\CP^1=S^2$. It is clear this
fibration is locally trivial. We see we get
\begin{equation}
\dots\to\pi_{k}(S^1)\to\pi_{k}(S^{2n+1})\to\pi_{k}(\CP^{n})\to\pi_{k-1}(S^{1})\to\dots
\end{equation}
describing the exact homotopy sequence.
\index{Hopf Fibration!Generalization of ---|)}

Remember that we already mentioned
\begin{equation}
\pi_{k}(\mbox{space})=\pi_{k}\begin{pmatrix}\mbox{covering}\\\mbox{space}
\end{pmatrix}
\end{equation}
for $k\geq2$. We thus have
\begin{equation}
\pi_{k}(B)\iso\pi_{k}(E)
\end{equation}
since $E$ is the covering space of the base $B$. We see $F$ is
discrete, so 
\begin{equation}
\pi_{k}(F)=0
\end{equation}
for $k\geq1$. This situation we explained, every third term of
our sequence vanishes. So
\begin{equation}
\pi_{k}(S^1)=0
\end{equation}
for $k\geq2$, and
\begin{equation}
\pi_{1}(S^1)\iso\ZZ.
\end{equation}
We may write down this sequence, we are in this wonderful
situation that lets us say
\begin{equation}
\pi_{k}(S^{2n+1})\iso\pi_{k}(\CP^n).
\end{equation}
But this is not always true. Why? Because
\begin{equation}
\pi_{1}(S^{1})\not=0.
\end{equation}
We should consider this portion of the exact sequence
\begin{equation}
\bigl(\pi_{2}(S^{1})=0\bigr)\to
\pi_{2}(S^{2n+1})\to
\pi_{2}(\CP^n)\to
\bigl(\pi_{1}(S^{1})\iso\ZZ\bigr)\to
\bigl(\pi_{1}(S^{2n+1})=0\bigr)
\end{equation}
But
\begin{equation}
\pi_{2}(S^{2n+1})=0
\end{equation}
which gives us
\begin{equation}
0\to\pi_{2}(\CP^n)\to\bigl(\pi_{1}(S^{1})\iso\ZZ\bigr)\to0
\end{equation}
We get
\begin{equation}
0\to\pi_{2}(\CP^n)\to\ZZ\to0
\end{equation}
or in other words
\begin{equation}
\pi_{2}(\CP^n)\iso\ZZ,
\end{equation}
and for $k>2$ we have
\begin{equation}
\pi_{k}(\CP^{n})\iso\pi_{k}(S^{2n+1})
\end{equation}
as desired.
