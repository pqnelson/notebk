%%
%% lecture06.tex
%% 
%% Made by alex
%% Login   <alex@tomato>
%% 
%% Started on  Tue Dec 27 09:58:36 2011 alex
%% Last update Tue Dec 27 09:58:36 2011 alex
%%

If $X$, $Y$ are topological spaces, we may consider $\hom(X,Y)$
and it is also a topological space. We may consider
$\pi_{0}\bigl(\hom(X,Y)\bigr)$ the connected components (homotopy
classes) of the space. We say $X\homotopic X'$ homotopic if there
exists maps
\begin{equation}
f\colon X\to X',\quad\mbox{and}\quad
g\colon X'\to X
\end{equation}
such that
\begin{equation}
g\circ f\homotopic\id{X}\quad\mbox{and}\quad
f\circ g\homotopic\id{X'}
\end{equation}
are homotopic.

\begin{ddanger}
Note that we are abusing language a little. We have the notion of
homotopic maps $f\homotopic g$ and we construct from this a weak
equivalence relation on topological spaces. We call this
``homotopy equivalence of topological spaces'' or simply
``homotopic spaces''.
\end{ddanger}

For example, if $X\homotopic(\mbox{point})$, then we say $X$ is
\define{Contractible}\index{Contractible!Space|textbf}. In
particular every convex set $C$ is contractible, it is obvious. We
assume without loss of generality that $0\in C$, we have
\begin{equation}
f_{t}(x)=tx
\end{equation}
describe the contraction. Thus $D^n$, $\bar{D}^n$, $\RR^n$ are
contractible. But $S^{n}$ is \emph{not} contractible.
If we delete a single point from a sphere, we get
\begin{equation}
S^{n}-\mbox{point}\iso\RR^{n}
\end{equation}
and that implies $S^{n}-(\mbox{point})$ is contractible.

There is a notion of retraction and deformation retraction. If
$A\propersubset X$, then it is a deformation retract if the
inclusion is a homotopy equivalence. We have
\begin{equation}
f_{0}(x)=x
\end{equation}
for all $x\in X$ and
\begin{equation}
f_{1}(x)\in A
\end{equation}
for all $x\in X$ be a deformation retract. If $X\propersupset A$
(where $A$ is a ``good'' subset, i.e. closed and we require if
$X$ is a cell complex that $A$ be a subcomplex), is contractible,
then $X\homotopic X/A$ homotopic.

We will consider homotopic classifications of graphs (we will
consider connected graphs, this is not really a restriction).

\begin{thm}\label{thm:bouquet}
Every connected graph is homotopically equivalent to a wedge sum
of cirlces (``bouquet'').\index{Bouquet}
\end{thm}
If we are working with sets with marked points $(X,x_0)$, then we
may consider\index{Marked Points}\index{Space!Pointed}\index{Topological Space!Pointed}
\begin{equation}
f\colon (X,x_0)\to (Y,y_0)
\end{equation}
which is a map $f\colon X\to Y$ such that $f(x_0)=y_0$.Then we
may consider the set $\hom\bigl((X,x_0),(Y,y_0)\bigr)$ which is
again a topological space, or part of the topological space
$\hom(X,Y)$. We may generalize this to a pair $(X,A)$ where
$A\propersubset X$, and we may consider maps
\begin{equation}
f\colon (X,A)\to (Y,B)
\end{equation}
which consists of a map $f\colon X\to Y$ such that
$f(A)\propersubset B$. This is a straightforward generalization
of marked points. Again, we consider
$\hom\bigl((X,A),(Y,B)\bigr)$
which is a topological space, and again we may speak of connected
components. 

\index{Disjoint Union|(}
If $X$, $Y$ are topological spaces we may consider a disjoint
union $X\sqcup Y$ which is a topological space with two
components $X$ and $Y$. We see
\begin{equation}
\hom(X\sqcup Y,Z)=\hom(X,Z)\times\hom(Y,Z).
\end{equation}
Also note we have two notions of disjoint unions: a set
theoretic\index{Disjoint Union!of Sets}
sense, where we have
\begin{subequations}
\begin{equation}
\bar{D}^{n}=S^{n-1}\sqcup D^{n}
\end{equation}
be true, whereas the topological version has
\begin{equation}
\bar{D}^{n}\not=S^{n-1}\sqcup D^{n}
\end{equation}
\end{subequations}
\index{Disjoint Union|)}%
We may consider the wedge sum\index{Wedge Sum} of $X$ and $Y$ to be
\begin{equation}
\begin{split}
(X\wedgeSum Y,*)&=(X,x_0)\wedgeSum (Y,y_0) \\
&= (X\sqcup Y)/(x_{0}\sim y_{0})
\end{split}
\end{equation}
where we identify the marked points of the pointed spaces
$(X,x_0)$ and $(Y,y_0)$. Thus
\begin{equation}
\hom\bigl((X\wedgeSum Y,*),(Z,z_0)\bigr) = 
\hom\bigl((X,*),(Z,z_0)\bigr)\times
\hom\bigl((Y,*),(Z,z_0)\bigr)
\end{equation}
The marked point needs to be taken into account. The wedge sum is
the coproduct in the category $\Top_{*}$ of topological spaces
with marked point (``pointed topological spaces'').

\begin{proof}[Proof (of Theorem \ref{thm:bouquet})]
It is trivial. Suppose we have a connected graph. Every edge is an
interval, but an interval is contractible. We can contract as
long as the starting point of and edge is different from its
ending point. We can contract any edge with two different
vertices, which decreases the number of vertices. At some point
we have only one vertex, and it's the wedge sum of circles.
\end{proof}

So to give an example of what this would look like, we doodle:
\begin{center}
  \includegraphics{img/lecture6.0}
\end{center}

\bigbreak
The second question: start with a completely general graph. How
to say what is an equivalent graph? The answer is very simple if
we know the Euler characteristic. The Euler characteristic of a
compact set is a homotopy invariant. We will prove it next
quarter. Suppose our graph is simply connected, then the Euler
characteristic is
\begin{equation}
\alpha_0-\alpha_1=(\mbox{\# vertices})-(\mbox{\# edges}),
\end{equation}
and the wedge sum of $k$ circles is
\begin{equation}
\chi(S^{1}\wedgeSum\dots\wedgeSum S^{1})=1-(\mbox{\# circles})
\end{equation}
as there is a single vertex. If we have homotopy equivalence,
then we may compute the number of circles in a moment.

Graphs are one-dimensional guys, and reduce to circles. Now we
should consider
\begin{equation}
\pi_{0}\bigl(\hom(S^1,S^1)\bigr)=\ZZ
\end{equation}
We should see this as obvious. We saw
\begin{equation}
S^1\to \RR^2-(\mbox{point})\homotopic S^{1}
\end{equation}
and the homotopy class of maps is characterized by the winding
number\index{Winding Number}\index{Winding Number!generalization of ---|see{Degree}}.
