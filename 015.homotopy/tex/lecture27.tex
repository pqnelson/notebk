%%
%% lecture27.tex
%% 
%% Made by alex
%% Login   <alex@tomato>
%% 
%% Started on  Tue Dec 27 20:51:49 2011 alex
%% Last update Tue Dec 27 20:51:49 2011 alex
%%

Consider a topological group $G$ and a closed subgroup
$H\propersubset G$, we are assuming that $G$ is a compact Lie
group for simplicity. Then we may say that $H$ acts on $G$ by
means of multiplication --- specifically, by means of
multiplication on the right. That is
\begin{equation}
\varphi_{h}(g)=gh
\end{equation}
for $g\in G$ and $h\in H$. We have cosets $gH$ by considering the
orbits of such maps. We also have a fibration where the fibres
are $H$ and 
\begin{equation}
p\colon G\to G/H
\end{equation}
is the fibration. So $G$ acts on $G/H$ but on the left. If we
take any $\gamma\in G$ and $\gamma(gH)=(\gamma g)H$. This action
is transitive. Ifwe start with $H$ and consider
\begin{equation}
\gamma(eH)=\gamma H
\end{equation}
we get every orbit, therefore it is transitive. If $G$ acts
transitively on $X$, then $X$ may be identified with the space of
orbits $X=G/H$ where $H$ is a stable subgroup. We may write down
the exact homotopy sequnce for this fibration.

\begin{ex}
We may consider $\U{n}/\U{n-1}=S^{2n-1}$ and this is simply
because $\U{n}$ acts on $\CC^n$ by definition. The orbits of this
action are spheres. The stable subgroup is $\SU{n}$. We may also
write 
\begin{equation}
\SU{n}/\SU{n-1}=S^{2n-1} 
\end{equation}
which is the same stuff.

But we may repeat the same consideration on $\RR^n$, where the
orthogonal group replaces the unitary group. We have that 
\begin{equation}
\ORTH{n}/\ORTH{n-1}=S^{n-1}.
\end{equation}
We may also consider instead
\begin{equation}
\SO{n}/\SO{n-1}=S^{n-1}.
\end{equation}
There is another consideration using $\HH^n$ quaternionic
space. The analog of unitary or orthogonal group, here, is the
symplectic group $\Sp{n}$. We obtain
\begin{equation}
\Sp{n}/\Sp{n-1}=S^{4n-1}
\end{equation}
by similar reasoning as the complex case.
\end{ex}

Recall that for the exact sequence for fibrations we have
\begin{equation}
\dots\to
\pi_{n}(F)\to
\pi_{n}(E)\to
\pi_{n}(B)\to
\pi_{n-1}(F)\to
\dots
\end{equation}
but usually one of these spaces is contractible, so its homotopy
group is trivial for all $n$. Thus we get an isomorphism between
homotopy groups.

We thus deduce that, for ``small $k$'', 
\begin{equation}
\pi_{k}\bigl(\U{n}\bigr)\iso\pi_{k}\bigl(\U{n-1}\bigr)
\end{equation}
This means we may consider \define{Stable Homotopy Groups}\index{Homotopy Group!Stable}\index{Stable Homotopy Group|see{Homotopy Group}}
which are denoted by $\pi_{k}(\mathrm{U})$,
$\pi_{k}(\mathrm{O})$, and $\pi_{k}(\mathrm{Sp})$. This
corresponds to $\pi_{k}\bigl(\U{n}\bigr)$, etc., for ``large enough $n$''.
We can compute how large $n$ has to be.
There then exists a wonderful statement:
\begin{bott}\index{Bott Periodicity Theorem}\index{Bott Periodicity!Theorem|(}
We have $\pi_{k}(\mathrm{U})\iso\pi_{k+2}(\mathrm{U})$,
$\pi_{k}(\mathrm{O})\iso\pi_{k+8}(\mathrm{O})$, and
$\pi_{k}(\mathrm{Sp})\iso\pi_{k+4}(\mathrm{O})$. 
\index{Bott Periodicity!Theorem|)}\end{bott}

So we should know a small amount of these guys. First of all, all
even guys
\begin{subequations}
\begin{equation}
\pi_{2k}(\mathrm{U})=0.
\end{equation}
We see $\pi_{0}(\mathrm{U})$ is the set of connected components,
and thus the unitary group is connected. We also can see
\begin{equation}
\pi_{2k}(\mathrm{U})\iso\ZZ
\end{equation}
\end{subequations}
We see $\SU{n}\propersubset\U{n}$ is a subgroup. Moreover, if
$u\in\U{n}$, $v\in\SU{n}$, we have
\begin{equation}
u=\lambda v
\end{equation}
for some $\lambda\in\CC$. So this quotient
\begin{equation}
\U{n}/\SU{n}=S^1
\end{equation}
as topological spaces, so we have
\begin{equation}
\pi_{1}\bigl(\SU{n}\bigr)=0.
\end{equation}
Thus we deduce that $\pi_{1}\bigl(\U{n}\bigr)\iso\ZZ$ and we may
use Bott periodicity.

For the orthogonal group, we must compute $\pi_{0}$, \dots,
$\pi_{7}$. We see that
\begin{equation}
\pi_{0}(\mathrm{O})\iso\ZZ_2
\end{equation}
since there are two connected components, by Euler's
theorem\index{Euler's Theorem}\footnote{Euler's theorem states: if $X$ is an orthogonal
matrix, then $\det(X)=\pm1$. This can be seen by
$\det(X^{\mathrm{T}}X)=\det(I)=1$ and
$\det(X^{\mathrm{T}})=\det(X)$. Thus $\det(X)^2=1$, and there are
only two real numbers that do this.}. We also observe
\begin{equation}
\pi_{1}\bigl(\SO{3}\bigr)\iso\pi_{1}\bigl(\RP^3\bigr)\iso\ZZ_2
\end{equation}
so
\begin{equation}
\pi_{1}(\mathrm{O})\iso\ZZ_2.
\end{equation}
Now we see that
\begin{equation}
\pi_{2}(\mathrm{O})=0
\end{equation}
is trivial. More interestingly, for any Lie group $G$ we have
\begin{equation}
\pi_{2}(G)=0.
\end{equation}
But we have
\begin{equation}
\pi_{3}(\mathrm{O})\iso\ZZ,
\end{equation}
which is a general fact for every \emph{simple} noncommutative
Lie group $G$ we have $\pi_{3}(G)\iso\ZZ$.
We may now write down a table
\begin{center}
\begin{tabular}{c|c|c|c|c|c|c|c}
$\pi_{0}(\mathrm{O})$ & 
$\pi_{1}(\mathrm{O})$ & 
$\pi_{2}(\mathrm{O})$ & 
$\pi_{3}(\mathrm{O})$ & 
$\pi_{4}(\mathrm{O})$ & 
$\pi_{5}(\mathrm{O})$ & 
$\pi_{6}(\mathrm{O})$ & 
$\pi_{7}(\mathrm{O})$\\ \hline
$\ZZ_2$ & $\ZZ_2$ & 0 & $\ZZ$ & 0 & 0 & 0 & $\ZZ$
\end{tabular}
\end{center}
Only $\pi_{0}$ depends on the components, all other homotopy
groups are computed on the component connected to the identity
element. 

We may compute $\pi_0$, $\pi_1$, $\pi_2$, $\pi_3$ for all simple
Lie groups $\mathrm{U}$, $\mathrm{O}$, $\mathrm{Sp}$. Every Lie
group is homotopy equivalent to a compact Lie group. So really,
look, first of all both $\U{n}$ and $\Sp{n}$ are connected but
$\ORTH{n}$ has two components. We know
\begin{equation}
\pi_{3}\bigl(\SU{2}\bigr)=\pi_{3}(S^3)=\ZZ
\end{equation}
and this gives us
\begin{equation}
\pi_{3}\bigl(\SO{3}\bigr)=\ZZ.
\end{equation}
We observe
\begin{equation}
\SO{4}\iso\bigl(\SU{2}\times\SU{2}\bigr)/\ZZ_2.
\end{equation}
But it follows that
\begin{equation}
\pi_{3}\bigl(\SO{4}\bigr)\iso\ZZ\oplus\ZZ.
\end{equation}
We have $\Sp{1}\iso\SU{2}$, so
\begin{equation}
\pi_{3}\bigl(\Sp{1}\bigr)\iso\ZZ.
\end{equation}
We can now apply stability to deduce
\begin{equation}
\pi_{2}\bigl(\SU{n}\bigr)=0
\end{equation}
and
\begin{equation}
\pi_{3}\bigl(\SU{n}\bigr)\iso\ZZ
\end{equation}
for $n\geq2$. We see we may compute
\begin{equation}
\SU{3}/\SU{2}\iso S^5
\end{equation}
We have
\begin{equation}
\SO{6}\iso\SU{4},
\end{equation}
so we now know the homotopy group for $\SO{6}$. We see
\begin{equation}
\SO{6}/\SO{5}\iso S^5
\end{equation}
and that enables us to deduce the homotopy groups for $\SO{5}$.


\exercises
\begin{xca}
Let us consider a locally trivial fibration with total space $E$,
base $B=S^2$ and fiber $S^{1}$. Express the homotopy groups of
$E$ in terms of homotopy groups of the sphere $S^2$.

Hint. The data you have do not specify completely the homotopy
groups of $E$. You should describe all possible answers.
\end{xca}
\begin{xca}\index{Classifying Space}\index{$B_{G}$}
A topological group $G$ acts freely on contractible space
$E$. Express the homotopy groups of the space of orbits $B_{G} =
E/G$ in terms of homotopy groups of $G$. (The space $B_{G}$ is
called classifying space of $G$.)
\end{xca}
\begin{xca}
The group $\U{k}\times\U{k}$ is embedded into group $\U{2k}$ as a
subgroup consisting of block-diagonal matrices with two $k\times
k$ blocks. Calculate the homotopy groups
$\pi_{n}(\U{2k}/\U{k}\times\U{k})$ for $n < 2k$.


Hint. You can use the fact that the natural homomorphism
$\pi_{n}(\U{k})\to\pi_{n}(\U{m})$ where $k<m$ is an isomorphism
for $n < 2k$. The groups $\pi_{n}\bigl(\U{k}\bigr)$ for $n<2k$
are called stable homotopy groups;%
\index{Stable Homotopy Groups}\index{Group!Stable Homotopy}\index{Homotopy Group!Stable} %
they are equal to $\ZZ$ for odd $n$ and to $0$ for even $n$
\end{xca}
