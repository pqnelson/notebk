%%
%% lecture15.tex
%% 
%% Made by alex
%% Login   <alex@tomato>
%% 
%% Started on  Thu Dec 29 09:46:23 2011 alex
%% Last update Thu Dec 29 09:46:23 2011 alex
%%

We covered van Kampen's theorem, which lets us calculate the
fundamental group. There is another approach which is just as
powerful. 

\begin{thm}\label{thm:lec15:1}
Let $X$ be simply connected, suppose a group $G$ acts freely on
$X$. IF we consider the space of orbits $X/G$, then
$\pi_{1}(X/G)=G$.
\end{thm}
\index{Action!Free|(}\index{Group Action!Free|(}
What does it mean ``$G$ acts freely''? Well, $G$ acts on $X$
means that for every element $g\in G$ we have a transformation
\begin{equation}
\varphi_{g}\colon X\to X
\end{equation}
and we require
\begin{subequations}
\begin{equation}
\varphi_{gh}=\varphi_{g}\varphi_{h}
\end{equation}
and
\begin{equation}
\id{X}=\varphi_{1}.
\end{equation}
\end{subequations}
But we can write this as
\begin{equation}
\varphi_{g}(x)=gx,
\end{equation}
where we have associativity
\begin{equation}
(gg')x=g(g'x).
\end{equation}
This is the left action; there is a right action which is more or
less the same stuff, but written on the right, e.g.,
$\varphi_{gh}=\varphi_{h}\varphi_{g}$. If we don't say explicitly
otherwise, we use the left action.

We have a notion of an \define{Orbit}\index{Orbit} of every
point. For one point, an orbit is
\begin{equation}
Gx=\{gx\mid g\in G\}.
\end{equation}
We also have a \define{Stabilizer}\index{Stabilizer}
\begin{equation}
\mbox{stab}(x)=\{g\in G\mid gx=x\}=H_{x}
\end{equation}
It is a subgroup of $G$. A free action (if the group is finite)
is very simple: all the stabilizers are trivial.
\index{Action!Free|)}\index{Group Action!Free|)}

When $X$ is a topological space, we require 
\begin{equation}
\varphi_{g}\colon X\to X
\end{equation}
to be continuous. We may say a little bit more, namely:
what does it mean we have a\index{Group Action!on Topological Space}\index{Action!Free on Topological Space} free action on a topological space? 
We have an orbit, and each point in the orbit is distinct. So if
$g,g'\in G$ and $g\not=g'$, then 
\begin{equation}
gx\not=g'x.
\end{equation}
What may we do? We may take a neighborhood of a point of the
orbit. Then we get a neighborhood of \emph{every} point of the
orbit. Here we should note we assume $X$ is at least
Hausdorff. We may take $U$ ``sufficiently small'' so that 
\begin{equation}
U\cap gU=\emptyset,
\end{equation}
we see that we may say that 
\begin{equation}
g_{i}U\cap g_{j}U=\emptyset.
\end{equation}
Now we may define the free group action.
\begin{defn}\index{Action!Free|textbf}
A group $G$ \define{Acts Freely} on $X$ if for any $x\in X$ we
may find a neighborhood $U\ni x$ such that $g U\cap
g'U=\emptyset$ for distinct $g,g'\in G$.
\end{defn}
Really we have already used this notion. Well, a particular case
of it. We computed the fundamental group of $S^{1}$. But really
we have
\begin{equation}
S^{1}\iso\RR/\ZZ
\end{equation}
where the action is
\begin{equation}
x\mapsto x+n.
\end{equation}
The orbit of $0\in\RR$ is precisely $\ZZ$. We have then
\begin{equation*}
[0,1]/(0\sim1)
\end{equation*}
we have a circle. So $\pi_{1}(S^{1})=\ZZ$ by our theorem.

The torus is a more complicated example, since $\RR^2/\ZZ^2$
describes the torus.
The action is
\begin{equation}
(x,y)\mapsto(x+n_1,y+n_2)
\end{equation}
So
\begin{equation}
\pi_{1}(T)=\ZZ^{2}.
\end{equation}
More generally, the multidimensional torus\index{Torus!Multidimensional} is obtained from
\begin{equation}
T^{n}=\RR^n/\ZZ^n
\end{equation}
and this gives us
\begin{equation}
\pi_{1}(T^{n})\iso\ZZ^n.
\end{equation}
One more example. On the sphere we have the action of the group
$\ZZ_2$. That is
\begin{equation}
S^2/\ZZ_2\iso\RP^2,
\end{equation}
so we have
\begin{equation}
\pi_{1}(S^{2}/\ZZ_{2})=\pi_{1}(\RP^{2})=\ZZ_{2}.
\end{equation}
We can generalize to $\RP^{n}=S^{n}/\ZZ_{2}$ for $n\geq2$. We see
\begin{equation}
\pi_{1}(\RP^{n})=\ZZ_{2}.
\end{equation}
But we cannot do this for $S^1$, since it is \emph{not} simply
connected! 

%\medbreak\noindent\emph{Proof} (Theorem \ref{thm:lec15:1}).\quad
Let us suppose we have a path on $X/G$. It can be lifted to
$X$. What does this mean? Recall we have a natural map
\begin{equation}
f\colon X\to X/G.
\end{equation}
When we have a point $y\in X/G$, it may be considered by a set of points
$\{x\in X\mid f(x)=y\}$. We may lift $y$ to $x$. We may do this
several different ways, and the number of different ways is
precisely the number of elements of $G$. In other words
\begin{equation}
|G|=|f^{-1}(y)|
\end{equation}
describes the cardinality. So this lifting of the path is not
unique. 

\begin{wrapfigure}{l}{6pc}
  \vspace{-12pt}
  \centering
  \includegraphics{img/lecture15.0}
\end{wrapfigure}
\noindent\ignorespaces
We divide our path into small pieces. This is doodled on the left.
We lift the end point.
Consider a neighborhood $U$ such that
\begin{equation}
gU\cap U=\emptyset
\end{equation}
for any $g\not=1$ in the group $G$. But 
\begin{equation}
f(U)=V=f(gU).
\end{equation}
We may say that
\begin{equation}
f^{-1}(V)=\bigsqcup_{g\in G}gU.
\end{equation}
We would like to lift our piece of path contained in $V$, but a
path is continuous. Therefore if we lift it, the piece of path
should be lifted and moreover the small piece is lifted
uniquely. We want to lift the \emph{whole} path. Now we use
compactness to have a finite covering that may be lifted in a
unique way.

\begin{lem}
For every path, there is a unique lifting of the path for every
lift of the starting point.
\end{lem}

So now we would like to say the following thing: suppose we have
a family of paths $g_{t}(\tau)$ and everything varies
continuously. Then the lifts also vary continuously. This
follows almost immediately from uniqueness. We assumed that the
starting points are lifted continuously. Now we can prove the
theorem.

\medbreak\noindent\emph{Proof} (Theorem \ref{thm:lec15:1}).\quad%
This correspondence is very simple. Take any fixed (i.e., not
varying) point $*$, we can consider its orbit $g(*)$. We may
consider the path $h(t)$ where $h(0)=*$ and $h(1)=g(*)$. We apply
our map $f$ and we have
\begin{equation}
f\bigl(h(0)\bigr)=f\bigl(h(1)\bigr)
\end{equation}
which is a loop!
This defines an element of the fundamental group. Now there is a
question whether we may say this element depends on the choice of
$g$. But all paths are homotopic, by simple connectedness of $X$.

\begin{center}
\includegraphics{img/lecture15.1}
\end{center}
We then have a map $G\to\pi_{1}(X/G)$. We need to prove it is
one-to-one, but we proved this: every point in the quotient may
be lifted. This map is surjective. We should prove this map is
injective (if we can lift the homotopy, then definitely $g$
remains the same). We should prove it preserves multiplication,
but we will do this next time.

