%%
%% lecture03.tex
%% 
%% Made by alex
%% Login   <alex@tomato>
%% 
%% Started on  Thu Sep 29 11:03:37 2011 alex
%% Last update Thu Sep 29 11:03:37 2011 alex
%%
We have a sequence of sets $X^{k}$ called a
\define{$k$-Dimensional Skeleton}\index{Skeleton}. We get a $(k+1)$-dimensional
skeleton by taking several closed balls $\overline{D}^{k+1}_{i}$
and take
\begin{equation}
f_{i}\colon S^{k}\to X^{k}
\end{equation}
to paste $\partial\overline{D}^{k}_{i}$ to the $k$-dimensional
skeleton. If we consider only the open balls, then
\begin{equation}
X^{k+1}=X^{k}\sqcup D^{k}_{1}\sqcup\dots\sqcup D^{k}_{n}
\end{equation}
(NB: this is the \emph{disjoint} union!) The skeleton is closed
\begin{equation}
\overline{X^{k}}=X^{k}
\end{equation}
This is more or less the definition given, which is fine for the
finite-dimensional case. In general, we should note exclude the
case when $k\in\NN$, then the cell complex is the union of all
the skeletons:
\begin{equation}
X=\bigcup_{k\in\NN}X^{k}
\end{equation}
But we should think about the topology. We simply say that
\begin{equation}
U\propersubset X\mbox{ is open if }U\cap  X^{k}\mbox{ is open
}\forall k.
\end{equation}
That's reasonable. We impose the condition that the embedding
\begin{equation}
X^{k}\into X
\end{equation}
is continuous. The preimage of an open set $U\propersubset X$ is
then such that $U\cap X^{k}$ is open too. That is,
$U\propersubset X$ is open if and only if $U\cap X^{k}$ is
open. This means we take the weakest possible topology in
$X$\dots well, the weakest one satisfying the requirement
$U\propersubset X$ is open $\iff U\cap X^{k}$ is open. Lets
consider some examples. 

\begin{ex}
Zero dimensional cell complexes are just collections of
vertices. One-dimensional cell complexes are called
\define{Graphs}\index{Graph!as One-Dimensional Complex}\index{Complex!Graph as a ---}. We have $\overline{D}^{1}$ be a line interval,
the endpoints are the vertices in the $X^{0}$ skeleton. There are
some cases when the graphs are topologically equivalent, e.g.,
$E$ and $T$ are topologically equivalent.
\end{ex}

\begin{wrapfigure}{r}{0.7in}
        \includegraphics{img/lecture3.0}
\end{wrapfigure}
We can prove some guys are topologically nonequivalent. The
number of components is a topological invariant, so the letter
``i'' is not equivalent to any uppercase letter. We introduce a
notion of the \define{Degree of a Vertex}\index{Degree!of a Vertex}\index{Vertex!Degree} which is the number of
edges going into the vertex, which is a bit ambiguous.
But the degree of a vertex is a topological invariant. Why?
Because we can give it a topological definition. The graph for
$T$ as doodled on the right, we can pick three points of the
graph such that
\begin{equation}
\mbox{graph}-(\mbox{3 points})=U\sqcup V
\end{equation}
where $U$ and $V$ are open sets. Degree should be a local notion,
defined by the behavior of the graph in a neighborhood.

We will say the degree of a vertex is less than or equal to $k$
if in any neighborhood of the vertex we can find $k$ points such
that after deleting our points, the connected component of the
vertex is smaller than the union of the edges coming into this
vertex. In graph theory, this operation of removing vertices in
this manner is called a \define{Cut}\index{Cut|see{Vertex Cut}} (or ``\emph{Vertex Cut\/}''\index{Vertex!Cut}\index{Vertex Cut}).

What is important is that degree of a vertex is defined in terms
of connectedness. Although this notion of ``connectedness'' is a
topological notion, Graph theorists use the confusingly obtuse
term \define{Vertex Connectivity}\index{Vertex!Connectivity}\index{Vertex Connectivity}. So vertices of degree $k$ must be mapped to
vertices of degree $k$. Thus $A\not\iso T$ is not a topological
equivalence of graphs. If we cut any point from $T$, it's
disconnected. However, if we cut a point in $A$ while retaining
connectedness, it becomes topologically equivalent to $H$ --- any
other cut renders it disconnected.

\begin{wrapfigure}{r}{1in}
        \includegraphics{img/lecture3.1}
\end{wrapfigure}
Consider the cell complex for the torus, as doodled on the right.
We see that the skeleton can be described by a rectangle as its
only cell in 2-dimensions.

For a sphere we can consider it as many different cell
complexes. For example, we can construct the sphere by
\begin{equation}
D^{2}/\partial D^{2}\iso S^{2}
\end{equation}
which has a single 2-cell, and a single vertex. On the other
hand, if we take two discs $D^{2}_{0}$ and $D^{2}_{1}$, then
consider
\begin{equation}
(D^{2}_{0}\sqcup D^{2}_{1})/(\partial D^{2}_{0}\sim \partial
D^{2}_{1})\iso S^{2}
\end{equation}
we have 2 vertices, 2 edges, and 2 faces.

Recall we discussed the Euler characteristic.\index{Euler Characteristic|textbf}
We may define it as
\begin{equation}\index{$\chi(X)$}
\chi(X)=\sum(-1)^{n}\alpha_{n}
\end{equation}
where $\alpha_{n}$ is the number of $n$-cells. This is a
topological invariant, it is the simplest one. It obeys
\begin{equation}
\chi(A\sqcup B)=\chi(A)+\chi(B)
\end{equation}
So
\begin{equation}
\begin{split}
\chi(\mbox{torus}) &= (\mbox{1 vertex})-(\mbox{2 edges})+(\mbox{1 face})\\
&=0.
\end{split}
\end{equation}
Observe for the sphere we have
\begin{equation}
\begin{split}
\chi(\mbox{sphere}) &= (\mbox{2 vertices})-(\mbox{2
edges})+(\mbox{2 faces})\\
&=2.
\end{split}
\end{equation}
Quite simple!

\begin{wrapfigure}{r}{1.68in}
  \vspace{-36pt}
  \includegraphics{img/lecture3.2}
  \vspace{-36pt}
\end{wrapfigure}
We consider a handle\index{Handle}, obtained by taking a torus and deleting an
open ball. It is an example of a manifold/surface with
boundary. We may take a sphere and delete several discs. We paste
in each cut-out disc a handle.
