%%
%% lecture09.tex
%% 
%% Made by alex
%% Login   <alex@tomato>
%% 
%% Started on  Tue Dec 27 14:20:05 2011 alex
%% Last update Tue Dec 27 14:20:05 2011 alex
%%
Now,
\begin{equation}
\homotopyClass(S^1,S^1)=\ZZ
\end{equation}
the homotopy classification of maps $S^1\to S^1$ are in
one-to-one correspondence with $\ZZ$. So we are a little bit
sloppy here. But the important question: what about
$\homotopyClass(S^1,S^2)$? Well, every map $f\colon S^1\to S^2$
is homotopically equivalent to the constant map. 

%% We can see this
%% by considering the map taking $S^1$ to the equator. Now what?
%% Well, suppose we continuously deform this fixing one point on the
%% equator, while shrinking our circle. We can continue shrinking
%% until the deformation contracts the equator to a single point.

The proof is as follows. Suppose $f\colon S^1\to S^2$, take
$x\in S^2$ which is not covered by the image $x\notin
f(S^1)$. Then we may delete this $x$. But we get
\begin{equation}
S^2-\{x\}=\RR^2
\end{equation}
which is contractible. We may contract the map to a point. For
physicists, it's a good proof; but for mathematicians---no! We
could have a Peano curve, where
\begin{equation}
f(S^1)=S^2
\end{equation}
i.e., a space filling curve. Differentiable, or even piecewise
differentiable,maps of this kind are impossible. We get a trivial
lemma:
\begin{lem}[Trivial]
If $S^1\to\RR^n$, every such map may be approximated as good as
you want by piecewise linear maps.
\end{lem}
\begin{proof}
But this is a triviality. Look, we have a circle, or an interval,
we may decompose it into small pieces. If we have a map to
$\RR^n$, we may approximate the map  along the subdivision by a
linear map. If the subdivision is in ``small pieces'', then our
approximation is ``good.''
\end{proof}
This lemma is pretty much universal, we may substitute the domain
by $S^2$ or any cell complex. We may consider a rectangle, divide
it up into ``small triangles''---this is called a
\define{Triangulation}\index{Triangulation}. If we consider
\begin{equation}
[0,1]\times[0,1]\to\RR^2,
\end{equation}
we may approximate it by piecewise linear functions. We extend it
uniquely by linearity (perhaps ``affine'' is a better choice of
words than ``linear''). We may approximate the maps that are
horrible by maps that are quite nice. But now, when we
approximate, if we have two closed maps to a sphere---they are
homotopic. We may say every map is homotopic to a good map.

A minor technicality with the codomain. We either generalize the
notion of piecewise linear map to $S^2$, or (the easier choice)
take a piece of $S^2$ homeomorphic to a square. We take the
preimage of this to construct a piecewise linear map. Then we use
the lemma on extension of homotopy, we get an extension homotopic
to the approximation, and so on. Consider
$\homotopyClass(S^k,S^n)$ for $k<n$. We see that all maps are
homotopically trivial.

\subsection{Fundamental Group}
\index{Loop Space|(}\index{Space!Loop|(}
A very important notion that may be explained as follows: we have
a space $X$, and maps $S^1\to X$. We want to consider their
homotopy classes, but we want some algebraic structure. So we
need some structure, some operation. These maps are loops. We
will assume that our $S^1$ has a marked point, and $X$ has a
marked point; we will say that we would like to consider maps
\begin{equation}
(S^1,*)\to(X,x_0)
\end{equation}
which take the marked point to the marked point. What do we have?
We have two loops that start and finish at the same point. People
use the word ``concatenation''\index{Concatenation!of Paths}. We
get an operation of two loops. %% We will write this out as
%% \begin{equation}
%% (f*g)(\alpha)=\begin{cases}
%% f(2\alpha) & 0\leq\alpha<\pi\\
%% g(2\alpha-2\pi) & \pi\leq\alpha<2\pi
%%   \end{cases}
%% \end{equation}
%% to indicate the concatenation of $f,g\colon S^1\to X$.
We write $f*g$ for the concatenation of $g$ followed by
$f$.\marginpar{$f*g$ = concatenation of paths}
Now we may define the fundamental
group\index{Fundamental Group}\index{Group!Fundamental}. We
consider
\begin{equation}
\homotopyClass({(S^1,*)},{(X,x_0)})
\end{equation}
so if $f\homotopic f'$ and $g\homotopic g'$, then
\begin{equation}
f*g\homotopic f'*g'
\end{equation}
homotopic. This operation $f*g$ is associative and has
inverses. We thus have a group, and we call it the
\define{Fundamental Group} denoted $\pi_{1}(X,x_0)$\index{$\pi_{1}(X,x_0)$}.
As usual, this is a sloppy definition. We will give a more
precise one.

First: what is a loop? It is easier, formally, to work with
intervals. So we will consider
\begin{subequations}
\begin{equation}
f\colon[0,1]\to X
\end{equation}
and we require 
\begin{equation}
f(0)=f(1)=x_0
\end{equation}
\end{subequations}
both endpoints are mapped to the marked point. But this is the
same as a map of a circle, which starts and stops at the marked
point. Why did we take $[0,1]$? I don't know! We could take
instead
\begin{equation*}
f\colon[0,a]\to X
\end{equation*}
for any $a>0$ with the condition $f(0)=f(a)=x_0$. But topology
doesn't care about this: $[0,a]\iso[0,1]$ for topologists. Now we
can consider the space of all these maps denoted by $\Omega(X,x_0)\propersubset\hom\bigl([0,1],(X,x_0)\bigr)$.%
\index{$\Omega(X,x_0)$}%
Now we can modify this if we we consider the space of maps
\begin{equation}\index{$\widetilde{\Omega}(X,x_0)$}
\widetilde{\Omega}(X,x_0)=\{ f\colon[0,a]\to(X,x_0)\mid a>0, f(0)=f(a)=x_0\}.
\end{equation}
Are these spaces the same? No, of course not. But
\begin{equation}
\Omega\iso\widetilde{\Omega}
\end{equation}
which is a triviality since
\begin{equation}
\Omega\propersubset\widetilde{\Omega}
\end{equation}
is embedded, we may retract $\widetilde{\Omega}$ to $\Omega$. We
may deform continuously $[0,a]\homotopic[0,1]$ which is why we
may confidently state $\Omega\homotopic\widetilde{\Omega}$
homotopic.

We now want to define multiplication in these spaces. This is
easy to do for $\widetilde{\Omega}$. If we have
\begin{subequations}
\begin{equation}
f\colon[0,a]\to(X,x_0)
\end{equation}
and
\begin{equation}
g\colon[0,b]\to(X,x_0)
\end{equation}
\end{subequations}
then we may define their concatenation\index{Concatenation|textbf} as
\begin{subequations}
\begin{equation}
h=f*g\colon[0,a+b]\to(X,x_0)
\end{equation}
where
\begin{equation}
h(x) = \begin{cases}f(x) & 0\leq x\leq a\\
g(x-a) & a\leq x\leq b+a.
\end{cases}
\end{equation}
\end{subequations}
It is very clear in this picture this concatenation is an
associative operation. We don't need to prove
\begin{equation}
(f*g)*h=f*(g*h).
\end{equation}
So $\widetilde{\Omega}$ has an operation called
``concatenation'', and it is an associative operation. People
usually work with $[0,1]$ but concatenation is still defined for
\begin{equation}
f,g\colon[0,1]\to(X,x_0).
\end{equation}
We have
\begin{equation}
h(x) = \begin{cases}f(2x) & 0\leq x\leq\frac{1}{2}\\
g(2x-1) & \frac{1}{2}\leq x\leq 1
\end{cases}
\end{equation}
for our concatenation operation. Is it associative? No, it will
not be associative, because look
\begin{equation}
(f*g)*h = \begin{cases}
f(4x) & 0\leq x\leq 1/4\\
g(4x-1) & 1/4\leq x\leq 1/2\\
h(2x-1) & 1/2\leq x\leq1
  \end{cases}
\end{equation}
but if we consider instead
\begin{equation}
f*(g*h) = 
\begin{cases}
f(2x) & 0\leq x\leq 1/2\\
g(4x-1) & 1/2\leq x\leq 3/4\\
h(4x-2) & 3/4\leq x\leq1
\end{cases}
\end{equation}
So we do not have associativity, strictly speaking. But if we
care up to homotopy, we have full associativity for
$\widetilde{\Omega}$. 

Now we may define the fundamental group. What are we doing? Well,
as a set $\pi_{1}(X,x_0)$\index{$\pi_{1}(X,x_0)$} is the set of components
\begin{equation}
\pi_{0}\bigl(\Omega(X,x_0)\bigr)=\pi_{0}\bigl(\widetilde{\Omega}(X,x_0)\bigr)
\end{equation}
but in $\Omega$ we have multiplication (and we have it in
$\widetilde{\Omega}$ too), and --- what a coincidence! --- we
have multiplication in $\pi_{0}(-)$. And, moreover,
$\pi_{0}(\Omega(X,x_0))$ has multiplication be associative, which
implies associativity in the others.

We have a unit element, namely
\begin{equation}
e(t)=x_0
\end{equation}
where we stay at the marked point. We may say
\begin{equation}
e*f\homotopic f.
\end{equation}
It is very easy to say
\begin{equation}
f^{-1}(t)=f(1-t)
\end{equation}
as the path that goes in the opposite direction. We end up with
the fundamental group.
\index{Loop Space|)}\index{Space!Loop|)}
