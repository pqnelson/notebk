%%% set-theory.tex --- 
%% 
%% Filename: set-theory.tex
%% Description: 
%% Author: alex
%% Maintainer: 
%% Created: Thu Jun  9 19:38:23 2016 (-0700)
%% Version: 
%% Package-Requires: ()
%% Last-Updated: 
%%           By: 
%%     Update #: 0
%% URL: 
%% Doc URL: 
%% Keywords: 
%% Compatibility: 

\section{Set Theory}
\M
As an appendix, we formalize various theorems used throughout the notes,
and standardize the notation we use.

\N{Definitions} We have a grocery list of definitions.
\begin{enumerate}
\item A \define{Set} consists of an unordered collection of objects without
  duplicates.

\item An \define{Element} of a set $A$ is an object $x$ belonging to the
  collection, denoted $x\in A$.

\item A \define{Subset} of a given set $A$ consists of a set $B$ such that
  every element $b\in B$ belongs to $A$. We denote $B$ is a subset of $A$
  as $B\subset A$.

\item Two sets are \define{Equal} if and only if $A\subset B$ and $B\subset A$.
  We denote this as $A=B$ if they are equal, and $A\neq B$ otherwise.

\item A \define{Proper Subset} is a subset $B\subset A$ such that $B\neq A$.
  We denote a proper subset as $B\propersubset A$. The notation is
  analogous to ordering natural numbers $m\leq n$ allows the possibility
  $m=n$, whereas $m<n$ does not.
\end{enumerate}

\N{More Definitions}
We have various operations involving many sets.

\begin{enumerate}
\item A \define{Family of Sets} consists of a collection $J$ called the
  \emph{indexing set}, and a collection of sets $\{U_{\alpha} : \alpha\in J\}$.
  Sometimes this is simply denoted $\{U_{\alpha}\}_{\alpha\in J}$ or
  simply $\{U_{\alpha}\}$.
\item The \define{Union} of a family of sets consists of a set
  \begin{equation}
    \bigcup_{\alpha}U_{\alpha} = \{x : x\in U_{\alpha}\mbox{ for some $\alpha$}\}.
  \end{equation}
\item The \define{Intersection} of a family of sets $\{U_{\alpha}\}_{\alpha\in{J}}$ consists of a set
  \begin{equation}
    \bigcap_{\alpha}U_{\alpha} = \{x : x\in U_{\alpha}\mbox{ for all
      $\alpha\in J$}\}.
  \end{equation}
\end{enumerate}

\begin{thm}
  The union of a family of sets contains any family member as a subset:
  \begin{equation}
    \forall\beta,\enspace U_{\beta}\subset\bigcup_{\alpha}U_{\alpha}
  \end{equation}
\end{thm}

\begin{thm}
  The intersection of a family of sets is contained in any family member
  as a subset:
  \begin{equation}
    \forall\beta,\enspace \bigcap_{\alpha}U_{\alpha}\subset U_{\beta}
  \end{equation}
\end{thm}
