%%% intro.tex --- 
%% 
%% Filename: intro.tex
%% Description: 
%% Author: alex
%% Maintainer: 
%% Created: Sat Jan 30 10:08:40 2016 (-0800)
%% Version: 
%% Package-Requires: ()
%% Last-Updated: 
%%           By: 
%%     Update #: 0
%% URL: 
%% Doc URL: 
%% Keywords: 
%% Compatibility: 
\section{Definition of a Topology}
\begin{defn}\label{defn:topology}
  A \define{Topology} on a set $X$ consists of a collection
  $\topology{T}$ of subsets of $X$ such that
  \begin{enumerate}
  \item It contains the emptyset $\emptyset\in\topology{T}$
  \item It contains the underlying set $X\in\topology{T}$
  \item The union of arbitrary subcollections of $\topology{T}$ are in
    $\topology{T}$
  \item The finite intersection of finitely many element of
    $\topology{T}$ are in $\topology{T}$.
  \end{enumerate}
  A set $X$ equipped with a topology $\topology{T}$ is called a
  \define{Topological Space}.
\end{defn}
\N*{Remarks}
(1) Elements of a topology are precisely open sets. A topology \emph{defines}
open sets: a subset $U\subset X$ of a topological space is open if and
only if it is an element of the topology $U\in\topology{T}$.

(2) A set may be equipped with multiple different topologies. Two
topological spaces cannot be considered ``the same'' if the underlying
set $X$ is the same. Because of this, we will sometimes write a
topological space as an ordered pair $(X,\topology{T})$ to clarify
\emph{which} topology we're working with.

%% (3) If we think of mathematical objects in terms of ``stuff, structure,
%% and properties'', then a topology is a ``structure''. But it is not like
%% generic algebraic structure: it is ``contravariant'', in the sense that
%% the morphisms are different.

\N[Trivial Topology]{Example}
Let $X$ be an arbitrary set. Then $\topology{T}=\{\,\emptyset, X\,\}$ is a
``trivial'' topology on $X$. Can we prove it? We must prove this trivial
topology satisfies the axioms:

\begin{enumerate}
  \item We see, by construction, the empty set belongs to the collection
    $\emptyset\in\topology{T}$.
  \item Also, by construction, the underlying set belongs to the
    collection $X\in\topology{T}$
  \item There are only finitely many families of subsets of
    $\mathcal{T}$ (namely 4 families), the singletons and emptyset
    trivially satisfy the property of arbitrary unions belong to
    $\mathcal{T}$ (otherwise we contradict the first two axioms). The
    last family would be $\mathcal{T}$ itself, the union of its elements
    are precisely $X$, which we know is in $\mathcal{T}$.
  \item The intersection of finitely many elements can only be
    $\emptyset$ or $X$, both of which belong to $\mathcal{T}$.
\end{enumerate}

Hence the collection $\mathcal{T}$ satisfies the desired properties. It
is called the \define{Trivial Topology} (or sometimes the
``\emph{Indiscrete Topology}'').

\N[Discrete Topology]{Example}
Let $X$ again be an arbitrary set. Then $\powerset{X}$ is also a topology.
We need to show it satisfies the axioms for a topology:

\begin{enumerate}
\item We know $\emptyset\in\powerset{X}$.
\item We also know $X\in\powerset{X}$
\item We need to show, if
  $\{U_{\lambda}\}_{\lambda\in\Lambda}\subset\powerset{X}$, then
  $\bigcup_{\lambda\in\Lambda}U_{\lambda}\in\powerset{X}$. We know
  $\bigcup_{\lambda\in\Lambda}U_{\lambda}\subset X$ since
  $U_{\lambda}\subset X$ for each $\lambda\in\Lambda$. Hence the claim
  follows immediately.
\item We know the intersection of two elements
  $U_{1},U_{2}\in\powerset{X}$ also belongs to the powerset $U_{1}\cap
  U_{2}\in\powerset{X}$. By induction, any finite intersection of
  finitely many elements of $\powerset{X}$ must also belong to
  $\powerset{X}$. 
\end{enumerate}

The collection $\powerset{X}$ forms a topology on $X$ called the
\define{Discrete Topology}.

\N{Example}
Consider $\RR$. The collection of open intervals $(a,b)=\{x\in\RR : a<
x<b\}$ gives us a topology if we take the finite intersection of open
intervals and arbitrary unions of open intervals. This is called the
\define{Standard Topology on $\RR$}.

\begin{defn}
  Let $X$ be a set, $\topology{T}$ and $\topology{T'}$ be two
  topologies on $X$.
  \begin{enumerate}
  \item If $\topology{T'}\supset\topology{T}$, then we say
    \textbf{$\topology{T'}$ is \define{Finer} than $\topology{T}$}
    (and \define{Strictly Fine} iff
    $\topology{T'}\propersupset\topology{T}$).
  \item If $\topology{T'}\subset\topology{T}$, then we say
    \textbf{$\topology{T'}$ is \define{Finer} than $\topology{T}$}
    (and \define{Strictly Fine} iff
    $\topology{T'}\propersubset\topology{T}$).
  \item If either $\mathcal{T}\propersupset\mathcal{T'}$ or
    $\mathcal{T}\propersubset\mathcal{T'}$, then we say $\mathcal{T}$ is
    \define{Comparable} to $\mathcal{T'}$. Otherwise they are
    \define{Incomparable} (or just ``\emph{Not Comparable}'').
  \end{enumerate}
\end{defn}

\N{Finite Complement Topology}
Let $X$ be a set, $\topology{T}_{f}$ be the collection of all subsets
$U$ of $X$ such that $X\setminus U$ is either finite or all of $X$. We
claim $\topology{T}_{f}$ is a topology on $X$.
\begin{enumerate}
  \item We see $X\setminus\emptyset=X$, so
    $\emptyset\in\topology{T}_{f}$
  \item We see $X\setminus X=\emptyset$ which is finite, so
    $X\in\topology{T}_{f}$.
  \item Let $\{U_{\alpha}\}_{\alpha\in A}$ be an indexed family of
    nonempty elements of $\topology{T}_{f}$. We want to show the union
    $bigcup_{\alpha}U_{\alpha}\in\topology{T}_{f}$. We use de Morgan's
    law to write
    \begin{equation}
      X \setminus\bigcup_{\alpha}U_{\alpha} = \bigcap_{\alpha}(X\setminus U_{\alpha})
    \end{equation}
    We see $X\setminus U_{\alpha}$ is finite for each $\alpha\in A$,
    and since the intersection of an arbitrary family of finite sets is
    finite, we have this set difference be a finite set. Hence the union
    $\bigcup_{\alpha}U_{\alpha}\in\topology{T}_{f}$.
  \item Let $U_{1}$, \dots, $U_{n}\in\topology{T}_{f}$ be nonempty
    sets. Then we want to show $U_{1}\cap\dots\cap U_{n}\in\topology{T}_{f}$.
    We again invoke de Morgan, writing
    \begin{equation}
      X\setminus \bigcap^{n}_{i=1}U_{i}=\bigcup^{n}_{i=1}(X\setminus U_{i})
    \end{equation}
    and observing since $U_{i}\in\topology{T}_{f}$ that $X\setminus U_{i}$
    must be finite (since it cannot be all of $X$, as we assumed
    $U_{i}$ is nonempty). The finite union of finite sets is again a
    finite set, hence de Morgan produced a finite set. This implies
    $U_{1}\cap\dots\cap U_{n}\in\topology{T}_{f}$.
\end{enumerate}
\begin{defn}
  Let $X$ be a set. A \define{Basis} for a topology on $X$ consists of a
  set $\topology{B}$ of \define{Basis Elements} such that
  \begin{description}
    \item[$\topology{B}$ Covers $X$:] For each $x\in X$, there is at
      least one $B\in\topology{B}$ such that $x\in B$.
    \item[Closed under Finite Intersection:]
      If $x$ belongs to the intersection of two basis elements $B_{1}$
      and $B_{2}$, then there is a $B_{3}\in\topology{B}$ such that
      $x\in B_{3}\subset B_{1}\cap B_{2}$.
  \end{description}
\end{defn}
\begin{defn}\label{defn:generated-topology}
  The \define{Topology generated by basis $\topology{B}$} consists of a
  collection $\mathcal{T}$ of subsets $U\subset X$ such that
  \begin{enumerate}
  \item for each $x\in U$, there is a $B\in\topology{B}$
    containing $x$ and contained in $U$: $x\in B$ and $B\subset U$
  \end{enumerate}
\end{defn}
\begin{lemma}\label{lemma:open-set-squeeze-lemma}
Let $X$ be a topological space, $A\subset X$. Suppose for each $x\in A$
there is an open set $U$ containing $x$ such that $U\subset A$. Then $A$
is open.
\end{lemma}
\begin{pfSketch}
  This is a ``squeeze-type'' lemma, so we'll find a family of sets
  $U_{x}$ such that $x\in U_{x}\subset A$. Then we'll simply take the
  union of all such $U_{x}$, and show how this contains the union of
  $\{x\}$ singleton points of $A$ (the union \emph{is} $A$), and since
  each $U_{x}\subset A$ we'll have $A\subset\cup U_{x}\subset A$.
\end{pfSketch}
\begin{spf}
  \Let $x\in A$
  \Consider $U_{x}\subset X$ open such that $x\in U_{x}\subset A$ by hypothesis.
  \Then $\bigcup_{x\in A}\{x\}\subset\bigcup_{x\in A}U_{x}\subset\bigcup_{x\in A}A$
  \step $\bigcup_{x\in A}\{x\}=A$, trivially.
  \Hence $A\subset\bigcup_{x\in A}U_{x}\subset A\implies \bigcup_{x\in A}U_{x}=A$
  \Thus $A$ is open, since the union of a family of open sets is open.
\end{spf}
\bigskip%
\smallskip
\begin{thm}
  Let $\topology{T}$ be the topology generated by a basis $\topology{B}$
  on $X$.
  Then $\topology{T}$ really is a topology.
\end{thm}
\begin{pfSketch}
We show $\topology{T}$ satisfies the axioms for a topology directly,
with the sets $U\in\topology{T}$ generated by basis elements via Lemma
\ref{lemma:open-set-squeeze-lemma}. 
\end{pfSketch}
\begin{spf}
  \step $\emptyset\in\topology{T}$ since the empty union is empty
  \step $X\in\topology{T}$
  \begin{spf}
    \Let $x\in X$
    \Consider $B_{x}\in\topology{B}$ such that $x\in B_{x}$ by the basis
    covers $X$
    \Hence $\bigcup_{x\in X}B_{x}=X$ by Lemma \ref{lemma:open-set-squeeze-lemma}.
  \end{spf}
  \step For each family of open sets $\{U_{\alpha}\in\topology{T}\}$,
  we have $\bigcup_{\alpha}U_{\alpha}\in\topology{T}$
  \begin{spf}
    \Let $\{U_{\alpha}\}\in\powerset{\topology{T}}$ be a family of open
    sets.
    \Consider $B_{x,\alpha}\in\topology{B}$ such that for $x\in U_{\alpha}$,
    we have $x\in B_{x,\alpha}\subset U_{\alpha}$
    \Hence the result by Lemma \ref{lemma:open-set-squeeze-lemma}.
  \end{spf}
  \step For any $\{U_{i}\}_{i=1}^{n}\subset\topology{T}$, we have
  $U_{1}\cap\dots\cap U_{n}\in\topology{T}$.
  \begin{spf}
    Directly
    \Suffices to show the intersection of any pair of sets $U_{1}$,
    $U_{2}\in\topology{T}$ belongs to the generated topology
    $U_{1}\cap U_{2}\in\topology{T}$.
    \Let $U_{1}$, $U_{2}\in\topology{T}$.
    \Assume $U_{1}\cap U_{2}\neq\emptyset$ (otherwise we've proven the
    empty set is in the generated topology).
    \Let $x\in U_{1}\cap U_{2}$
    \Consider $B_{x,j}\in\topology{B}$ such that
    $x\in U_{j}\implies x\in B_{x,j}\subset U_{j}$ where $j=1,2$.
    \Consider $B_{x}\in\topology{B}$ such that
    $x\in B_{x,1}\cap B_{x,2}\implies x\in B_{x}\subset B_{x,1}\cap B_{x,2}$
    \Hence the result by applying Lemma \ref{lemma:open-set-squeeze-lemma}
    to the family $\{B_{x} : x\in U_{1}\cap U_{2}\}$.\placeQED{}
  \end{spf}
\end{spf}

\N*{Remark}
The proof presented is overkill. The condition for $U$ to be open in the
generated topology is simply to have a basis element contain a point
in $U$, for each point in $U$. But really, if we are honest, this is
equivalent to saying $U$ is the union of some family of basis elements,
or the intersection of finitely many basis elements (or both).

\N{Exercise}
Let $X$ be a topological space with topology $\mathcal{T}$ generated by
basis $\mathcal{B}$. Show that any basis element $B\in\mathcal{B}$ is
open $B\in\mathcal{T}$.

\N{Puzzle} What is the basis for the trivial topology? For the discrete
topology?

\N*{Challenge} What is the basis for the finite-complement topology?

\begin{lemma}\label{lemma:basis-generates-topology}
  Let $X$ be a set, $\topology{B}$ be a basis for a topology
  $\topology{T}$ on $X$. Then $\topology{T}$ consists of the collection
  of all possible unions of basis elements.
\end{lemma}

\begin{pfSketch}
  Since the claim is about set equality, we will have two steps: (1)
  $\topology{T}\subset\mbox{(all possible unions of basis elements)}$,
  and (2)
  $\topology{T}\supset\mbox{(all possible unions of basis elements)}$.
  One direction we've proven in Lemma \ref{lemma:open-set-squeeze-lemma}.
\end{pfSketch}

\begin{spf}
  \step $\{\bigcup_{B\in\beta} B : \beta\in\powerset{\topology{B}}\}\subset\topology{T}$
  \begin{spf}
    Immediately by definitions of a topology (\S\ref{defn:topology}) and
    the topology generated by a basis (\S\ref{defn:generated-topology}).
    \step Every basis element is an open set in the generated topology.
    \Hence the union of an arbitrary family of basis elements is open
    (by definition of a topology).
  \end{spf}
  \step $\{\bigcup_{B\in\beta} B : \beta\in\powerset{\topology{B}}\}\supset\topology{T}$
  \begin{spf}
    \Let $U\in\topology{T}$
    \Let $x\in U$.
    \Consider $B_{x}\in\topology{B}$ such that $x\in B_{x}\subset U$.
    \Hence $U=\bigcup_{x\in U}B_{x}$ by Lemma \ref{lemma:open-set-squeeze-lemma}.
    \Thus $U\in\{\bigcup_{B\in\beta} B : \beta\in\powerset{\topology{B}}\}$.\placeQED{}
  \end{spf}
\end{spf}

\N{Remarks} (1) This tells us a simpler way to generate a topology from
a basis, simply consider $\bigcup_{B\in\beta}B$ for every possible
family of basis elements $\beta\in\powerset{\topology{B}}$.

(2) Caution: if $U\in\topology{T}$ is an open set, and $\topology{B}$
generates $\topology{T}$, then $U$ is \emph{not} necessarily a
\emph{unique} combination of basis elements. Compare this to linear
algebra, where a vector \emph{is} a unique linear combination of basis
vectors. 

\N{Obtaining a Basis for a Topology}%
{\itshape %
Let $X$ be a topological space equipped with topology $\topology{T}$.
Suppose $\mathcal{C}$ is a collection of open sets of $X$ such that for
each subset $U\subset X$ and each $x\in U$ there is an element
$C\in\mathcal{C}$ such that $x\in C\subset U$.
Then $\mathcal{C}$ is a basis for the topology $\topology{T}$.}

\smallskip

\begin{pfSketch}
  We'll first show that $\mathcal{C}$ is a basis for some topology
  $\topology{T'}$. This amounts to verifying the axioms for a basis. We
  will then show that $\mathcal{T'}=\mathcal{T}$. This is a set equality
  claim, so we need to show $\mathcal{T'}\subset\mathcal{T}$ and
  $\mathcal{T'}\supset\mathcal{T}$. 
\end{pfSketch}

\begin{spf}
  \step $\mathcal{C}$ is the basis for some topology $\mathcal{T'}$
  \begin{spf}
    Verifying the Axioms
    \step For each $x\in X$ there is a $B\in\mathcal{C}$ such that $x\in B$.
    \begin{spf}
      \Let $x\in X$
      \ThenConsider $B\in\mathcal{C}$ such that $x\in B\subset X$ by hypothesis.
      \Hence then claim.
    \end{spf}
    \step If $x\in B_{1}\in\mathcal{C}$ and
    $x\in B_{2}\in\mathcal{C}$, then there is a
    $B_{3}\in\mathcal{C}$ such that $x\in B_{3}\subset B_{1}\cap B_{2}$.
    \begin{spf}
      Directly from hypothesis.
      \Assume $x\in B_{1}\in\mathcal{C}$, $x\in B_{2}\in\mathcal{C}$
      \Thus $B_{1}\cap B_{2}\in\topology{T}$ by closure under finite
      intersection, and $\mathcal{C}\subset\mathcal{T}$.
      \ThenConsider $B_{3}\in\mathcal{C}$ such that
      $x\in B_{3}\subset B_{1}\cap B_{2}$ by hypothesis.
      \Hence the claim.
    \end{spf}
  \end{spf}
  \step $\topology{T'}=\topology{T}$
  \begin{spf}
    \step $\topology{T'}\subset\topology{T}$
    \begin{spf}
      \Let $U\in\topology{T'}$
      \Then $U=\bigcup C_{\alpha}$ for some family of elements $C_{\alpha}\in\mathcal{C}$ by Lemma \ref{lemma:basis-generates-topology}.
      \Hence $U\in\topology{T}$ by closure under arbitrary unions.
    \end{spf}
    \step $\topology{T}\subset\topology{T'}$
    \begin{spf}
      \Let $U\in\topology{T}$.
      \Let $x\in U$.
      \ThenConsider $C_{x}\in\mathcal{C}$ such that $x\in C_{x}\subset
      U$
      by hypothesis.
      \Hence by Lemma \ref{lemma:open-set-squeeze-lemma} we have the result.\placeQED
    \end{spf}
  \end{spf}
\end{spf}

\N{Finer Bases Equivalent to Finer Topologies} {\itshape
Let $\topology{B}$ and $\topology{B'}$ be bases for the topologies
$\topology{T}$ and $\topology{T'}$ (resp.) on $X$. Then the following
are equivalent:
\begin{enumerate}
\item $\topology{T'}$ is finer than $\topology{T}$
\item For each $x\in X$ and $B\in\topology{B}$, there is a
  $B'\in\topology{B'}$ such that $x\in B'\subset B$.
\end{enumerate}
}

\begin{spf}
  \step $(1)\implies(2)$
  \begin{spf}
    \Assume $\topology{T'}$ is finer than $\topology{T}$
    \Let $B\in\topology{B}$
    \Then $B\in\topology{T'}$ by assumption.
    \Hence the result by definition of a basis.
  \end{spf}
  \step $(1)\impliedby(2)$.
  \begin{spf}
    \Assume for each $x\in X$ and $B\in\topology{B}$, there is a
    $B'\in\topology{B'}$ such that $x\in B'\subset B$.
    \Let $U\in\topology{T}$
    \Then there is a family $\{B_{\alpha}\}\subset\topology{B}$ which
    fills $U=\bigcup_{\alpha}B_{\alpha}$ by Lemma \ref{lemma:open-set-squeeze-lemma}.
    \Then for each $B_{\alpha}$, there is another family
    $\{B'_{\beta,\alpha}\}\subset\topology{B'}$ such that
    $\bigcup_{\beta}B'_{\beta,\alpha}=B_{\alpha}$ by Lemma \ref{lemma:open-set-squeeze-lemma}.
    \Hence $U\in\topology{T'}$ by $\bigcup_{\alpha,\beta}B'_{\beta,\alpha}=U$.\placeQED
  \end{spf}
\end{spf}

\N{Standard Topology on $\RR$}
Consider
\begin{equation}
  \topology{B} = \{(a,b) : a,b\in\RR\}
\end{equation}
where
\begin{equation}
  (a,b) = \{x\in\RR : a<x<b\}
\end{equation}
is an open interval. Then $\topology{B}$ generates the
\define{Standard Topology} on $\RR$, and unless explicitly stated
otherwise we assume $\RR$ is equipped with the standard topology. But
lets prove this really \emph{is} a topology. It suffices to show that
$\topology{B}$ is a basis.

\begin{spf}
  \step For each $x\in\RR$, there is a $B\in\topology{B}$ such that
  $x\in B$.
  \begin{spf}
    \Let $x\in\RR$
    \ThenConsider $B\in\topology{B}$ such that $B=(x-1,x+1)$.
    \Hence $x\in B$.
  \end{spf}
  \step If $x\in B_{1}\cap B_{2}$, then there is a
  $B_{3}\in\topology{B}$ such that $x\in B_{3}\subseteq B_{1}\cap
  B_{2}$.
  \begin{spf}
    \Let $x\in\RR$.
    \Assume $x\in (a_{1},b_{1})\in\topology{B}$, and $x\in
    (a_{2},b_{2})\in\topology{B}$.
    \Then $\max(a_{1},a_{2})<x$ and $x<\min(b_{1},b_{2})$ by assumption.
    \Hence $x\in\bigl(\max(a_{1},a_{2}), \min(b_{1},b_{2})\bigr)=B_{3}\in\topology{B}$.\placeQED
  \end{spf}
\end{spf}

\N{Lower Limit Topology}
Let $\topology{B}$ consist of elements of the form
\begin{equation}
  [a,b) = \{x\in\RR : a\leq x<b\}.
\end{equation}
Then $\topology{B}$ generates a topology in $\RR$ called the lower limit
topology. When we consider $\RR$ equipped with this topology, we'll
indicate it with the notation $\lowerlimRR$. The proof $\topology{B}$ is
a basis boils down to the same major moments as the proof for the
standard topology.

\N{$K$-Topology}
Let
\begin{equation}
  K = \{1/n : n\in\NN\}.
\end{equation}
Let $\topology{B}$ consist of elements of the form $(a,b)$ \emph{and}
$(a,b)\setminus K$. Then $\topology{B}$ generates the
\define{$K$-Topology} on $\RR$, and when we work with this topology, we
denote the space as $\RR_{K}$. We must prove this is a valid basis.

\begin{pfSketch}
  The major difference from this basis and the one for the standard
  topology on $\RR$ has to do with the second property for a basis,
  namely there is a $B_{3}\subset B_{1}\cap B_{2}$, when
  $B_{j}=(a_{j},b_{j})\setminus K$ for $j=1,2$. But we see that it
  amounts to the intersection of the intervals, then subtract off $K$.
\end{pfSketch}
\begin{spf}
  \step $\topology{B}$ covers $\RR$
  \step If $x\in B_{1}\cap B_{2}$, then there is a $B_{3}$ such that
  $x\in B_{3}\subset B_{1}\cap B_{2}$
  \begin{spf}
    \Let $x\in\RR$
    \Assume $x\in B_{j}\in\topology{B}$ for $j=1,2$
    \Then without loss of generality, we may assume
    $B_{j}=(a_{j},b_{j})\setminus K$.
    \Hence $x\in\bigl(\max(a_{1},a_{2}), \min(b_{1},b_{2})\bigr)\setminus K=B_{3}\in\topology{B}$.\placeQED
  \end{spf}
\end{spf}

\begin{thm}
  $\lowerlimRR$ is strictly finer than $\RR$
\end{thm}
\begin{thm}
  $\RR_{K}$ is strictly finer than $\RR$.
\end{thm}
\begin{thm}
  $\RR_{K}$ is not comparable to $\lowerlimRR$.
\end{thm}
\begin{defn}
  A \define{Sub-basis} $\topology{S}$ for a topology on $X$ is the
  collection of $X$ whose union generates $X$.

  The \define{Topology generated by the sub-basis} $\topology{S}$ is
  defined to be the collection $\topology{T}$ generated by all unions of
  finite intersections of elements of $\topology{S}$.
\end{defn}

\begin{lemma}
  The collection $\topology{B}$ of finite intersection of elements of a
  sub-basis $\topology{S}$ is a basis for a topology.
\end{lemma}
\begin{thm}
  The ``topology generated by the sub-basis'' really is a \emph{bona
    fide} topology.
\end{thm}
