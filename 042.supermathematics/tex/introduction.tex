\section{Introduction}

\M
We will study supermathematics by means of puzzles and exercises.
``Puzzles'' are more ``longterm'' questions to ponder, which we may
answer in future sections. ``Exercises'' are questions I want you to
answer as you read them.

\begin{definition}
Let $\FF$ be a field (usually $\RR$ or $\CC$, but we could make $\FF$ a
suitably nice unital ring).
A \define{Grassmann Algebra} in $n$ generators $\theta_{1}$, \dots,
$\theta_{n}$ over $\FF$ consists of the free algebra subject to the
following restriction:
\begin{subequations}
\begin{equation}
\theta_{i}\theta_{j}+\theta_{j}\theta_{i}=0
\end{equation}
for $i$, $j=1,\dots,n$, and for any $x\in\FF$,
\begin{equation}
x\theta_{i}=\theta_{i}x.
\end{equation}
\end{subequations}
\end{definition}

\begin{remark}
In other words, we are introducing $n$ new ``numbers'' $\theta_{1}$,
\dots, $\theta_{n}$ such that they anticommute with each other (and
commute with ``actual numbers'' which belong to $\FF$).
\end{remark}

\begin{exercise}
Are Grassmann algebras unital?
\end{exercise}

\begin{exercise}
Prove $\theta_{i}^{2}=0$.
\end{exercise}

\begin{exercise}
How would you define $(x+\theta y)^{-1}$ for $x,y\in\FF$ and $\theta$
being a Grassmann generator?
\end{exercise}

\begin{exercise}
Prove, for each $n\in\NN$, there exists up to [unique?] isomorphism
exactly one Grassmann algebra in $n$ generators over $\FF$.
\end{exercise}

\N{Notation} As a consequence of Grassmann algebras being unique (up to
isomorphism), we will denote the Grassmann algebra in $n$ generators as $\Lambda^{n}$.

\begin{exercise}
Let $\alpha\in\Lambda^{n}$. Prove we can write it as a finite linear
combination of monomials of generators
\begin{equation}
a = a_{0} + \sum_{i}a_{i}\theta_{i} +
\sum_{i,j}a_{i,j}\theta_{i}\theta_{j} + \dots + a_{1,2,3,\dots,n}\theta_{1}(\dots)\theta_{n}.
\end{equation}
Are there any restrictions on the coefficients $a_{i,j}$, $a_{i,j,k}$,
etc.? [Can we have them be symmetric in their indices? Antisymmetric? Etc.]
\end{exercise}

\begin{definition}
Let $a\in\Lambda^{n}$ be an arbitrary element of a Grassmann algebra
with $n$ generators. We can write it as
\begin{equation}
a = a_{B} + a_{S}
\end{equation}
where
\begin{equation}
a_{S} = \sum_{n=1}\frac{1}{n!}a_{i_{1}\dots i_{n}}\theta_{i_{1}}(\dots)\theta_{i_{n}}
\end{equation}
is called the \define{Soul} of $a$ (and it contains all appearances of the
Grassmann generators $\theta_{1}$, \dots, $\theta_{n}$) and
$a_{B}\in\FF$ is called the \define{Body} of $a$.
\end{definition}

\begin{remark}
This terminology I found in Bryce DeWitt's \textit{Supermanifolds}
(Second ed., Cambridge university press, 1992).
\end{remark}

\begin{exercise}
Prove, for any Grassmann number $a\in\Lambda^{n}$, that its soul is nilpotent $a_{S}^{n+1}=0$.
\end{exercise}

\begin{definition}
We define $\Lambda^{m,n}$ to be a Grassmann algebra over the ring
$\FF[x_{1},\dots,x_{m}]$. In this case, we write a generic element
$f(x_{1},\dots,x_{m},\theta_{1},\dots,\theta_{n})\in\Lambda^{m,n}$ as
\begin{equation}
  \begin{split}
  f(x_{1},\dots,x_{m},\theta_{1},\dots,\theta_{n})
&= f_{B}(x_{1},\dots,x_{m}) + f_{S}(x_{1},\dots,x_{m},\theta_{1},\dots,\theta_{n})\\
&= f_{B}(x) + \sum_{N=1}f_{i_{1}\dots i_{N}}(x)\theta^{i_{1}}(\cdots)\theta^{i_{N}}.
  \end{split}
\end{equation}
Here the coefficients are polynomials (or rational functions).
\end{definition}

\begin{remark}[Abuse of notation]
We will frequently abuse notation and suppress indices, writing things
like $f(x)$ instead of $f(x_{1},\dots,x_{1})$, and $f(x,\theta)$ instead
of $f(x_{1},\dots,x_{m},\theta_{1},\dots,\theta_{n})$.

While we're abusing notation, we are fairly generous with what
$\FF[x_{1},\dots,x_{m}]$ could be replaced by; we could have
$C^{\infty}(\RR^{m})$ instead, for example.
\end{remark}

\begin{exercise}
Prove, for any $f\in\Lambda^{m,n}$ that $(f_{S}(x,\theta))^{n+1}=0$.
\end{exercise}

\M
The usefulness of $\Lambda^{m,n}$ is that we can now ``continuously
vary'' generators among themselves. For example, in $\Lambda^{3,2}$, we
have
\begin{equation}
  \begin{pmatrix}
    \cos(t) & -\sin(t)\\
    \sin(t) &  \cos(t)
  \end{pmatrix}\begin{pmatrix}\theta_{1}\\\theta_{2}
  \end{pmatrix} =: \begin{pmatrix}\theta_{1}(t)\\\theta_{2}(t)
  \end{pmatrix}
\end{equation}
give a continuous rotation among the Grassmann generators as we move
along the $t$-axis [i.e., forwards through time].

\begin{puzzle}
How can we do calculus (specifically differentiation and integration)
using $\Lambda^{m,n}$?
\end{puzzle}
