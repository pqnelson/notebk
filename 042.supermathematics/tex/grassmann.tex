\section{Grassmann Numbers and Algebras}

\begin{definition}
Let $\FF$ be a field (usually $\RR$ or $\CC$, but we could make $\FF$ a
suitably nice unital ring).
A \define{Grassmann Algebra} in $n$ generators $\theta_{1}$, \dots,
$\theta_{n}$ over $\FF$ consists of the free algebra subject to the
following restriction:
\begin{subequations}
\begin{equation}
\theta_{i}\theta_{j}+\theta_{j}\theta_{i}=0
\end{equation}
for $i$, $j=1,\dots,n$ (including $i=j$), and for any $x\in\FF$,
\begin{equation}
x\theta_{i}=\theta_{i}x.
\end{equation}
\end{subequations}
The generators $\theta_{i}$ are called \define{Anticommuting Variables}.
\end{definition}

\begin{remark}
In other words, we are introducing $n$ new ``numbers'' $\theta_{1}$,
\dots, $\theta_{n}$ such that they anticommute with each other (and
commute with ``actual numbers'' which belong to $\FF$).
\end{remark}

\begin{remark}
We find in the literature people usually work with $\FF=\CC$, rarely
work with $\FF=\RR$, and almost never any other field.
\end{remark}

\begin{exercise}
Are Grassmann algebras unital?\footnote{Recall, we call an algebra
$\mathcal{A}$ \define{Unital} if there is a number $1\in\mathcal{A}$
such that for every $x\in\mathcal{A}$, $1\cdot x=x\cdot1=x$.}
\end{exercise}

\begin{exercise}
Prove $\theta_{i}^{2}=0$. What happens if we work in a field $\FF$ with
characteristic 2?
\end{exercise}

\begin{exercise}
We could you define $(x+\theta y)^{-1}$ for $x,y\in\FF$ and $\theta$
being a Grassmann generator as $(x^{-1} - x^{-2}y\theta)$ --- verify
their product is $1$ provided $x\neq0$.

What would $(x_{0} + x_{1}\theta_{1} + x_{2}\theta_{2} + x_{1,2}\theta_{1}\theta_{2})^{-1}$
be (assuming there are only two generators $\theta_{1}$, $\theta_{2}$
for the algebra)? [Hint: setting $\theta_{i}=0$ while leaving the other
generator nonzero $\theta_{j}\neq0$ should reproduce the solution to the
previous scenario, i.e., $x_{0}^{-1}-x_{0}^{-2}x_{j}\theta_{j}$.]
\end{exercise}

%\begin{solution}
\begin{answer}
  We begin by trying to find the coefficients to
  $(x_{0} + x_{1}\theta_{1} + x_{2}\theta_{2} + x_{1,2}\theta_{1}\theta_{2})^{-1}=(y_{0}+y_{1}\theta_{1}+y_{2}\theta_{2}+y_{1,2}\theta_{1}\theta_{2})$
  by multiplying it out:
\begin{multline}
(x_{0} + x_{1}\theta_{1} + x_{2}\theta_{2} + x_{1,2}\theta_{1}\theta_{2})(y_{0}+y_{1}\theta_{1}+y_{2}\theta_{2}+y_{1,2}\theta_{1}\theta_{2})\\
    = x_{0}y_{0} + (x_{1}y_{0} + y_{0}x_{1})\theta_{1}
    + (x_{2}y_{0} + y_{2}x_{0})\theta_{2}\\
    + (x_{0}y_{1,2} + x_{1,2}y_{0} + x_{1}y_{2} - x_{2}y_{1})\theta_{1}\theta_{2}.
\end{multline}
From the hint, we can deduce $y_{0}=1/x_{0}$ and $y_{j}=-x_{j}/x_{0}^{2}$
for $j=1,2$. To get the coefficient of $\theta_{1}\theta_{2}$ to vanish,
we need to solve $x_{0}y_{1,2}+x_{1,2}y_{0}+x_{1}y_{2}-x_{2}y_{0}=0$,
which is 1 linear equation in 1 unknown. We obtain
$y_{1,2} = -x_{1,2}/x_{0}$ since $x_{1}y_{2}=x_{2}y_{1}$.
%$y_{1,2} = [(x_{1}^{2}/x_{0}) + (x_{2}^{2}/x_{0}) - x_{1,2}]/(x_{0}^{2})$.
%\end{solution}
\end{answer}

\begin{exercise}
More generally, if $\alpha$ is a Grassmann number with a ``nonzero
constant term'', say,
\begin{equation}
\alpha = a_{0} + \sum_{i}a_{i}\theta_{i} + \sum_{i<j}a_{i,j}\theta_{i}\theta_{j}
+\sum_{i<j<k}a_{i,j,k}\theta_{i}\theta_{j}\theta_{k}+\dots.
\end{equation}
What does the generic inverse $\alpha^{-1}$ look like? What does the
inverse for the ``linear Grassmann number'' $(a_{0} + \sum_{i}a_{i}\theta_{i})^{-1}$
look like when there are $n\in\NN$ Grassmann generators $\theta_{i}$?
\end{exercise}

\begin{exercise}
Prove, for each $n\in\NN$, there exists up to [unique?] isomorphism
exactly one Grassmann algebra in $n$ generators over $\FF$.
\end{exercise}

\N{Notation} As a consequence of Grassmann algebras being unique (up to
isomorphism), we will denote the Grassmann algebra in $n$ generators as $\Lambda^{n}$.

\begin{exercise}
Let $a\in\Lambda^{n}$. Prove we can write it as a finite linear
combination of monomials of generators
\begin{equation}
a = a_{0} + \sum_{i}a_{i}\theta_{i} +
\sum_{i,j}a_{i,j}\theta_{i}\theta_{j} + \dots + a_{1,2,3,\dots,n}\theta_{1}(\dots)\theta_{n}.
\end{equation}
Are there any restrictions on the coefficients $a_{i,j}$, $a_{i,j,k}$,
etc.? [Can we have them be symmetric in their indices? Antisymmetric? Etc.]
\end{exercise}

\N{Convention}
We will implicitly order indices, so we write $a\in\Lambda^{n}$ as
\begin{equation}
a = a_{0} + \sum_{i}a_{i}\theta_{i} +
\sum_{i<j}a_{i,j}\theta_{i}\theta_{j} +
\sum_{i<j<k}a_{i,j,k}\theta_{i}\theta_{j}\theta_{k} +
\dots + a_{1,2,3,\dots,n}\theta_{1}(\dots)\theta_{n}.
\end{equation}
We may be lazy and forget to order the dummy variables, but it is the
convention we will use from now on.

\begin{definition}
Let $a\in\Lambda^{n}$ be an arbitrary element of a Grassmann algebra
with $n$ generators. We can write it as
\begin{equation}
a = a_{B} + a_{S}
\end{equation}
where
\begin{equation}
a_{S} = \sum_{n=1}\frac{1}{n!}a_{i_{1}\dots i_{n}}\theta_{i_{1}}(\dots)\theta_{i_{n}}
\end{equation}
is called the \define{Soul} of $a$ (and it contains all appearances of the
Grassmann generators $\theta_{1}$, \dots, $\theta_{n}$) and
$a_{B}\in\FF$ is called the \define{Body} of $a$.
\end{definition}

\begin{remark}
This terminology I found in Bryce DeWitt~\cite{DeWitt:2012mdz}.
\end{remark}

\begin{exercise}
Prove, for any Grassmann number $a\in\Lambda^{n}$, that its soul is nilpotent $a_{S}^{n+1}=0$.
\end{exercise}

\begin{definition}
Let us define the \define{Degree} of a monomial
$\theta^{i_{1}}(\cdots)\theta^{i_{j}}$ of $j\in\NN$ [distinct] Grassmann generators
to be the integer $\deg(\theta^{i_{1}}(\cdots)\theta^{i_{j}})=j$.
\end{definition}

\begin{definition}
We define $\Lambda^{m,n}$ to be a Grassmann algebra over the ring
$\FF[x_{1},\dots,x_{m}]$. In this case, we write a generic element
$f(x_{1},\dots,x_{m},\theta_{1},\dots,\theta_{n})\in\Lambda^{m,n}$ as
\begin{equation}
  \begin{split}
  f(x_{1},\dots,x_{m},\theta_{1},\dots,\theta_{n})
&= f_{B}(x_{1},\dots,x_{m}) + f_{S}(x_{1},\dots,x_{m},\theta_{1},\dots,\theta_{n})\\
&= f_{B}(x) + \sum_{N=1}f_{i_{1}\dots i_{N}}(x)\theta^{i_{1}}(\cdots)\theta^{i_{N}}.
  \end{split}
\end{equation}
Here the coefficients are polynomials (or rational functions).
\end{definition}

\begin{remark}[Abuses of notation]
We will frequently abuse notation and suppress indices, writing things
like $f(x)$ instead of $f(x_{1},\dots,x_{1})$, and $f(x,\theta)$ instead
of $f(x_{1},\dots,x_{m},\theta_{1},\dots,\theta_{n})$.

While we're abusing notation, we are fairly generous with what
$\FF[x_{1},\dots,x_{m}]$ could be replaced by; we could have
$C^{\infty}(\RR^{m})$ instead, for example.
\end{remark}

\begin{remark}[Formal calculus]
We stress that we are working algebraically, so these are formal
calculations. We do not care about convergence or whatnot. I believe, in
practice, we usually work with the formal power series $\RR[[x_{1},\dots,x_{m}]]$
or formal Laurent series $\CC(\!(x_{1},\dots,x_{m})\!)$ (which intuitively
is just a formal Laurent polynomial plus a formal power series $\CC(x_{1},\dots,x_{m}) + \CC[[x_{1},\dots,x_{m}]]$).
But there are analysts who worry about the superdetails, e.g., Andrei
Khrennikov~\cite{Khrennikov:1999bd} discusses them further.
\end{remark}

\begin{exercise}
Prove, for any $f\in\Lambda^{m,n}$ that $(f_{S}(x,\theta))^{n+1}=0$.
\end{exercise}

\M
The usefulness of $\Lambda^{m,n}$ is that we can now ``continuously
vary'' generators among themselves. For example, in $\Lambda^{3,2}$, we
have
\begin{equation}
  \begin{pmatrix}
    \cos(t) & -\sin(t)\\
    \sin(t) &  \cos(t)
  \end{pmatrix}\begin{pmatrix}\theta_{1}\\\theta_{2}
  \end{pmatrix} =: \begin{pmatrix}\theta_{1}(t)\\\theta_{2}(t)
  \end{pmatrix}
\end{equation}
give a continuous rotation among the Grassmann generators as we move
along the $t$-axis [i.e., forwards through time].

\begin{definition}
Let $\Lambda^{m,n}$. We call a term $f\in\Lambda^{m,n}$ a
\define{Homogeneous Element of Degree $k$} if we can write it as a
linear combination of monomials of degree $k$:
\begin{equation}
f(x,\theta) = \sum_{i_{1}<\dots<i_{k}}f_{i_{1},\dots,i_{k}}(x)\theta^{i_{1}}(\cdots)\theta^{i_{k}}
\end{equation}
Moreover, we will write $\Lambda^{m,n}_{\text{even}}$ for guys which are
a linear combination of monomials of even degree,
\begin{subequations}
\begin{equation}
f(x,\theta) = \sum_{k~\text{even}}\sum_{i_{1}<\dots<i_{k}}f_{i_{1},\dots,i_{k}}(x)\theta^{i_{1}}(\cdots)\theta^{i_{k}}
\end{equation}
and call such $f\in\Lambda^{m,n}_{\text{even}}$ \define{Even} (or
\emph{bosonic}). Similarly, we write $\Lambda^{m,n}_{\text{odd}}$ for
guys which are a linear combination of monomials of odd degree,
\begin{equation}
f(x,\theta) = \sum_{k~\text{odd}}\sum_{i_{1}<\dots<i_{k}}f_{i_{1},\dots,i_{k}}(x)\theta^{i_{1}}(\cdots)\theta^{i_{k}}
\end{equation}
\end{subequations}
and call such $f\in\Lambda^{m,n}_{\text{odd}}$ \define{Odd} (or \emph{fermionic}).
\end{definition}

\begin{exercise}
Prove or find a counter-example: every $f\in\Lambda^{m,n}$ can be
written as a sum of $f_{0}\in\Lambda^{m,n}_{\text{even}}$ and $f_{1}\in\Lambda^{m,n}_{\text{odd}}$.
Moreover, these $f_{0}$ and $f_{1}$ are unique (for each $f$).
\end{exercise}

\begin{puzzle}
How can we do calculus (specifically differentiation and integration)
using $\Lambda^{m,n}$?
\end{puzzle}
