\section{Integrals of Grassmann-valued Functions}

\N{Desired properties of integration}
Zinn--Justin\footnote{J.Zinn-Justin, ``Quantum Field Theory \& Critical
Phenomena'', OUP, 2002.
See specifically online supplement \url{http://ckw.phys.ncku.edu.tw/public/pub/Notes/PhaseTransitions/Zinn-Justin/QFT-RG/Main.php}}
motivates integration of a function
$f\colon\Lambda^{m,n}\to\Lambda^{m,n}$ (denote it $I(f)$) with the following desiderata:
\begin{enumerate}
\item It is a linear operation: $I(\lambda_{1}f_{1} + \lambda_{2}f_{2}) = \lambda_{1}I(f_{1}) + \lambda_{2}I(f_{2})$
\item It's ``contour integral''-like, in the sense it has no boundary
  terms; which means, in particular, for any derivative $D$ with respect to the
  variable integrated, the integral of a derivative vanishes $ID=0$
\item The derivative of a definit integral should vanish $DI=0$
\item A constant factor [i.e., any $A$ such that $D(A)=0$] can be taken outside the integral
$I(fA) = I(f)A$
\item It changes the grading like $D$, so $PI+IP=0$ where the parity
  operator acts on odd variables like $P(\theta_{i})=-\theta_{i}$ and
  $P(\theta_{i_{1}}\dots\theta_{i_{j}})=P(\theta_{i_{1}})(\dots)P(\theta_{i_{j}})$,
  but on even variables like $P(x)=x$.
\end{enumerate}
Since differentiation with respect to an odd variable $\theta_{k}$ is a
nilpotent operation $D^{2}=0$, it follows from our desired criteria that
integration is differentiation $I=D$.

\M
Consequently, we have left-integration
\begin{subequations}
\begin{equation}
\int\D\theta\,(\theta) = 1,\quad\int\D\theta\,(1) = 0
\end{equation}
as well as right-integration,
\begin{equation}
\int(\theta)\,\D\theta=1,\quad\int(1)\,\D\theta = 0.
\end{equation}
\end{subequations}
These correspond to left-differentiation and right-differentiation. The
mnemonic is the ``left'' and ``right'' integration refers to where we
place the differential $\D\theta$ relative to the integrand.

\begin{exercise}
Do differentials of Grassmann generators commute? That is, would we
expect $\D\theta_{i}\,\D\theta_{j}=-\D\theta_{j}\,\D\theta_{i}$?
\end{exercise}

\begin{puzzle}
Recall Green's theorem states, for some simply-connected region $\Sigma$ in $\RR^{2}$,
we have
\begin{equation}
\int_{\partial\Sigma}(L\,\D x+M\,\D y) = \iint\left(\frac{\partial M}{\partial x}-\frac{\partial L}{\partial y}\right)\,\D x\,\D y.
\end{equation}
This generalizes to functions on the complex plane $f\colon\Sigma\subset\CC\to\CC$
defined by $f=u+\I v$,
\begin{equation}
\oint_{\partial\Sigma}f\,\D z = \iint_{\Sigma}\left(\frac{\partial f}{\partial
  z}\right)\,\D x\,\D(\I y).
\end{equation}
Does it generalize to to some ``simply-connected region'' of
$\Lambda^{1,1}$ with $f(x,\theta)=u(x)+v(x)\theta$?
\end{puzzle}

\begin{puzzle}
Consider $\Lambda^{m,n}$, let
$a_{i,j}\colon\Lambda^{m,n}\to\Lambda^{m,n}$ for $i,j=1,\dots,n$.
What is the ``fermionic Gaussian integral''
\begin{equation}
I =
\int\exp\left(\sum_{i,j}\theta_{i}a_{i,j}\theta_{j}\right)\,\D^{n}\theta
= \huh
\end{equation}
[Hint: Exercise~\ref{xca:differentiation:taylor-series-for-exp} may be useful.]
\end{puzzle}

%% 1 + (a + b\theta) + (a + b\theta)^{2} + (a + b\theta)^{3} + \dots
%% = a + b\theta + (a^{2} + 2ab\theta)/2! + (a^{3} + 3a^{2}b\theta)/3!
%% = 1+ (a + a^2/2! + a^3/3! + \dots) + (1 + a + a^2/2 + \dots)b\theta
%% = \exp(a)(1 + b\theta)


\begin{exercise}[A.~Schwarz 2008]
  Compute
  \begin{equation}
\int\log(1 + \theta_{1}\theta_{2} + \theta_{3}\theta_{4}
+ \theta_{1}\theta_{2}\theta_{3}\theta_{4})\,\D^{4}\theta
  \end{equation}
where the $\theta_{i}$ are all anticommuting variables.
\end{exercise}

\begin{exercise}[A.~Schwarz 2008]
  Calculate
  \begin{equation}
I = \int\exp(-x^{2}-2x\theta_{1}\theta_{2})\,\D x\,\D^{2}\theta.
  \end{equation}
\end{exercise}

\begin{exercise}[{Berezin~\cite[see pp.54--55]{Berezin:1966nc}}]
Let us consider $\Lambda^{m,2n}$ with Grassmann generators $\theta_{1}$,
\dots, $\theta_{n}$, $\psi_{1}$, \dots, $\psi_{n}$. Suppose we define
the ``Fourier transform'' to be
\begin{equation}
\widehat{f}(x,\psi) = \int f(x,\theta)\exp\left(\sum_{k=1}^{n}\theta_{k}\psi_{k}\right)\,\D\theta_{n}\cdots\D\theta_{1}.
\end{equation}
Is it true that
\begin{equation}
\psi_{i}\widehat{f}(x,\psi) \stackrel{???}{=} (-1)^{c} \frac{\partial}{\partial\theta_{i}}\left(\int f(x,\theta)\exp\left(\sum_{k=1}^{n}\theta_{k}\psi_{k}\right)\,\D\theta_{n}\cdots\D\theta_{1}\right),
\end{equation}
(for some $c\in\ZZ$, the reader will determine it),
or is there no similar analogue to Fourier transform turning
multiplication into differentiation for Grassmann variables?
\end{exercise}

\begin{exercise}[{Berezin~\cite[see p.54]{Berezin:1966nc}}]
Does integration by parts still work for Grassmann algebras? That is,
prove or find a counterexample for $f,g\in\Lambda^{m,n}$ that:
\begin{equation}
\int f(\theta)\left(\frac{\overrightarrow{\partial}}{\partial\theta_{i}}g(\theta)\right)\,\D\theta_{n}\cdots\,\D\theta_{1}
\stackrel{???}{=}
\int \left(f(\theta)\frac{\overleftarrow{\partial}}{\partial\theta_{i}}\right)g(\theta)\,\D\theta_{n}\cdots\,\D\theta_{1},
\end{equation}
where we indicate left differentiation and right differentiation with
the arrow telling us which way to differentiate. [Hint: it suffices to
  work with monomials $f$ and $g$, not necessarily of the same degree $\deg(f)\neq\deg(g)$.]
\end{exercise}
