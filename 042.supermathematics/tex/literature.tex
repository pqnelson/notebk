\section{Literature Review}

I am told from Hamilton~\arXiv{math/0510390} a ``standard modern
reference'' on supermathematics is:
\begin{enumerate}[label={[\arabic*]}]
\item Deligne and Morgan,
  ``Notes on Supersymmetry (following Joseph Bernstein)''.
  In \textit{Quantum Fields and Strings: a Course for Mathematicians}, Vol.~1, AMS, Providence, Rhode Island, 1999, pp. 41--97.
\end{enumerate}
I have particularly liked \S5 of:
\begin{enumerate}[resume,label={[\arabic*]}]
\item Alexander Karabegov, Yuri Neretin, Theodore Voronov,
  ``Felix Alexandrovich Berezin and his work''.
  \arXiv{1202.3930}
\end{enumerate}
It reminds me of Dr Schwarz's course on supermathematics.

I think the best strategy is to introduce Grassmann algebras, then
``supercalculus'' with the goal of trying to compute the ``super
Gaussian integral'' (in $n$ odd variables $\theta_{1}$, \dots, $\theta_{n}$):
\begin{equation}
I = \int\E^{a_{ij}\theta^{i}\theta^{j}}\,\D\theta_{n}\cdots\D\theta_{1}
= \sqrt{2\det(a_{ij})}
\end{equation}
This opens the door to \emph{linear superalgebra}. We can then relate
the superalgebra of supermatrices to construct super-analogues to Lie
algebras (so-called ``Lie superalgebras''). From there, it is natural to
think about the Lie algebra/Lie group correspondence, and seek a
super-analogue\dots which requires a notion of Lie supergroups and
supermanifolds. This is the barest of roadmaps.

Coincidentally, this is Berezin's approach in his book
\textit{Introduction to Superanalysis}.

For Dr Schwarz's course, we used Varadarajan's book
\textit{Supersymmetry for Mathematicians}.

\subsection{Lie Super-Algebra}

Victor Kac\footnote{Victor Kac, ``A sketch of Lie superalgebra theory'', \journal{Comm.Math.Phys.} \volume{53}, Number 1 (1977) 31--64. \url{https://projecteuclid.org/euclid.cmp/1103900590}} cites the first article
creating ``Lie Superalgebras'' to be:

\begin{enumerate}[resume,label={[\arabic*]}]
\item Felix Berezin, G. I. Kac, \journal{Math.~Sbornik} \volume{82} (1970) 343--351 (Russian)
\end{enumerate}
For a ``no nonsense'' review of the definitions:
\begin{enumerate}[resume,label={[\arabic*]}]
\item L. Frappat, A. Sciarrino, P. Sorba,
  ``Dictionary on Lie Superalgebras''.\newline
  \arXiv{hep-th/9607161}, 145 pages.
\end{enumerate}

\subsection{Supergeometry}

There are at least two distinct versions of supergeometry (or uses of
the phrase ``supergeometry''). One is algebraic geometry over Grassmann
algebras, the other is the ``usual'' differential geometry but locally
things look ``super''.

\begin{enumerate}[resume,label={[\arabic*]}]
\item Mikhail Kapranov, ``Supergeometry in mathematics and physics''.\newline
\arXiv{1512.07042} [math.AG] for the algebraic geometry version.
\end{enumerate}

Voronov taught a course on supergeometry recently\footnote{\url{https://itmp.msu.ru/en/mscgeometry/courses/intro-to-supergeometry}}, and the Syllabus
cited the following literature:
\begin{itemize}
\item F. A. Berezin. Introduction to superanalysis. Reidel, Dordrecht,1987.
\item D. A. Leites. Introduction to the theory of supermanifolds, Russian Math. Surveys 35 (1) (1980),13-64.
\item Yu. I. Manin. Gauge field theory and complex geometry, Springer-Verlag, Berlin,1997.
\item Th. Voronov. Geometric Integration Theory on Supermanifolds. Harwood Academic Publ., 1992. (2nd ed.: Cambridge Scientific Publ., 2014)
\item P. Deligne and J. Morgan. Notes on supersymmetry (following Joseph Bernstein). In book: Quantum fields and strings: a course for mathematicians, Vol.1, 41-97, Amer. Math. Soc., Providence, RI, 1999.
\item A. Rogers. Supermanifolds: Theory and applications. World Scientific Publishing. Ltd., Hackensack, NJ, 2007.
\item Additional literature:
  \begin{itemize}
  \item Th. Voronov.  Quantization on supermanifolds and the analytic proof of the Atiyah-Singer index theorem. J. Soviet Math. 64 (4) (1993),  993--1069.
  \item Th. Voronov. On volumes of  classical supermanifolds. Sbornik: Mathematics 207 (11) (2016), 1512-1536.
  \item Th. Voronov.  Graded geometry, Q-manifolds, and microformal geometry, Fortschritte der Physik 67 (2019), 1910023%, DOI 10.1002/prop.201910023.
  \end{itemize}
\end{itemize}

\subsection{Superanalysis}

\begin{enumerate}[resume,label={[\arabic*]}]
\item Atsushi Inoue,
  ``Lectures on Super Analysis -- Why necessary and What's that?''
  \arXiv{1504.03049}, 218 pages.
\end{enumerate}
For applications in physics, perhaps Dan Freed's ``Five Lectures on Supersymmetry''\footnote{For
audio and a draft of the notes, see
\url{https://www.on.kitp.ucsb.edu/online/geom/}}
may be interesting.