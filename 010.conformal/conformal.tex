\documentclass{amsart}
% term paper for Carlip's class on quantum gravity, on spontaneous symmetry breaking in conformal Weyl gravity
\usepackage{url}
\usepackage{hyperref}
\usepackage{paralist}
\numberwithin{equation}{section}
\title{Spontaneous Symmetry Breaking in Conformal Weyl Gravity}
\author{Alex Nelson ID 992174730}
\email{pqnelson@ucdavis.edu}

\begin{document}
\maketitle
\begin{abstract}
By spontaneously breaking conformal symmetry, a scalar field emerges
which we identify as the cosmological constant. The first instance is
in the Schwarzschild solution for the Weyl Gravity Field
Equations. The second is in the homogeneous and isotropic universe
with fermionic and bosonic matter. In the second case, an effective
gravitational constant emerges, as well as a positive cosmological
constant, both give theoretical justification within the conformal
gravity framework of an accelerating universe.
\end{abstract} 

\section{Introduction}
%%
%% intro.tex
%% 
%% Made by alex
%% Login   <alex@tomato>
%% 
%% Started on  Mon May 21 15:09:58 2012 alex
%% Last update Wed May 30 10:11:39 2012 Alex Nelson
%%
\chapter*{\phantomsection\addcontentsline{toc}{chapter}{Introduction}Introduction}

\textbf{Warning:\quad\ignorespaces} This introduction may be
skipped for most readers. The intended audience are logicians
interested in the foundations used throughout, and people
attempting to write ``formal mathematics''.

We will be working without much ``formal'' framework. That is, we
are largely working symbolically and not rigorously. The logical
foundation made could be described as ``High School Algebra''. We
use variables (uncontroversial for logicians). Our operations
are:
\begin{enumerate}
\item Addition. We have $a+b$ be commutative, so $a+b=b+a$ and
  associatve $(a+b)+c=a+(b+c)$. There is an identity element $0$
  such that $0+a=a$ for any $a$. Negation produces the additive
  inverse $-(a)=-a$. 
\item Subtraction. This is just adding by the additive inverse:
  $a-b = a+(-(b))$. 
\item Multiplication. Written $a\cdot{b}$, it's commutative
  $a\cdot{b}=b\cdot{a}$ and associative
  $a\cdot(b\cdot{c})=(a\cdot{b})\cdot{c}$. Its identity element
  is denoted $1\cdot{a}=a$ for any $a$. Multiplicative inverse is
  denoted $1/a$ for any nonzero $a$ (i.e., $a\not=0$).
\item Division. This, like subtraction, multiplies by the
  multiplicative inverse.
\item Exponentiation. We write $a^b$. It is the first
  noncommutative operation $a^b\not=b^a$ and it is not
  associative $a^{(b^{c})}\not=(a^{b})^{c}$. It has an identity
  element in the sense that $a^{1}=a$ for any $a$.
\end{enumerate}
Tacitly, we have an ordering of numbers:
\begin{equation}
a<b\iff b-a\mbox{ is positive}
\end{equation}
We have an ordered field. 

The astute student would realize that exponentiation could have
two inverses: the logarithm and the $n^{\rm th}$-root. We say
\begin{equation}
\log_{b}(x)=y\iff b^{y}=x.
\end{equation}
The root-approach specifies for any number $a$ another number is
produced $a^{1/n}$ satisfying
\begin{equation}
(a^{1/n})^{n}=a.
\end{equation}
What about $\sqrt{-1}$? We run into difficulties: the root
approach gives us unreal (or \emph{imaginary}) numbers. Really,
we work with real numbers ``embedded'' in an ambient complex
number system.


\section{Action Principle and Field Equations}
%%
%% action.tex
%% 
%% Made by Alex Nelson
%% Login   <alex@tomato>
%% 
%% Started on  ??? Dec  ? ??:??:?? 2008 Alex Nelson
%% Last update Wed Dec 10 02:03:11 2008 Alex Nelson
%%

One begins with the conformally-invariant fourth order action
\begin{equation}\label{conformalAction}
I_{W} = -\alpha_{g}\int
d^{4}x\sqrt{-g}C_{\alpha\beta\gamma\delta}C^{\alpha\beta\gamma\delta}
\end{equation}
where $\alpha_{g}$ is the coupling constant and
$C_{\alpha\beta\gamma\delta}$ is the Weyl tensor. This action is
invariant under conformal transformations of the metric
\begin{equation*}
g_{\mu\nu}(x)\to e^{2\alpha(x)}g_{\mu\nu}(x)
\end{equation*}
(Originally Weyl considered $\alpha$ in the action
\eqref{conformalAction} to be used for the both the conformal and
the electromagnetic gauge transformations. This way one has for the
covariant derivative $\nabla_{\mu}g_{\alpha\beta} =
A_{\mu}g_{\alpha\beta}$ where $A_{\mu}$ is the electromagnetic
4-potential. The problem with this approach is that conformal
invariance implied the particles are massless, which is observably
false.) We can now simplify this Lagrangian a bit.

By plugging in the definition of the Riemann tensor, and recalling
that any contraction of any pair of indices of the Weyl tensor
vanishes, we see
\begin{equation}
R_{\mu\nu\alpha\beta}R^{\mu\nu\alpha\beta} =
C_{\mu\nu\alpha\beta}C^{\mu\nu\alpha\beta} + 2R_{\mu\nu}R^{\mu\nu} - \frac{1}{3}R^{2}
\end{equation}
where $R$ is the Ricci scalar. Rearranging terms, we have
\begin{equation}
C_{\mu\nu\alpha\beta}C^{\mu\nu\alpha\beta} =
R_{\mu\nu\alpha\beta}R^{\mu\nu\alpha\beta} -  2R_{\mu\nu}R^{\mu\nu} + \frac{1}{3}R^{2}.
\end{equation}


Before beginning, note that the quantity~\cite{Kazanas:1988qa,Lanczos:1938sf}
\begin{equation}
\sqrt{-g}\left(R_{\alpha\beta\mu\nu}R^{\alpha\beta\mu\nu} -
4R_{\alpha\beta}R^{\alpha\beta} + R^{2}\right)
\end{equation}
is a total divergence. So instead of having our Lagrangian be
\begin{equation}
L = \sqrt{-g}C_{\alpha\beta\mu\nu}C^{\alpha\beta\mu\nu}
\end{equation}
we can \emph{equivalently} use the Lagrangian
\begin{equation}
L = -2\sqrt{-g}(R_{\alpha\beta}R^{\alpha\beta} - R^{2}/3)
\end{equation}
since we would be working with an extra term (by Stoke's theorem a surface
integral), and by demanding the variation vanishes on the boundary
the only nonzero contribution would be this Lagrangian.

Now, De Witt~\cite{dewitt1964} explicitly calculates out the
equations of motion for two Lagrangians:
\begin{equation}
L_{2} = \sqrt{-g}R^{2},\qquad\text{and}\qquad L_{1}=\sqrt{-g}R^{\mu\nu}R_{\mu\nu}
\end{equation}
Our Lagrangian is a linear combination of these two, so we use a
linear combination of the variation of their respective actions
\begin{equation}
\frac{\delta S_{2}}{\delta g^{\mu\nu}} = \frac{g_{\mu\nu}}{2}
\nabla^{\beta}\nabla_{\beta}({R^{\alpha}}_{\alpha})  +
\nabla^{\beta}\nabla_{\beta}R_{\mu\nu}  -
\nabla_{\beta}\nabla_{\nu}{R_{\mu}}^{\beta} - \nabla_{\beta}\nabla_{\mu}{R_{\nu}}^{\beta}
-2R_{\mu\beta}{R_{\nu}}^{\beta} + \frac{g_{\mu\nu}}{2}R_{\alpha\beta}R^{\alpha\beta}
\end{equation}
(where $\nabla_{\mu}$ is the covariant derivative operator) and
\begin{equation}
\frac{\delta S_{1}}{\delta g^{\mu\nu}} =
2g_{\mu\nu}\nabla^{\beta}\nabla_{\beta}{R^{\alpha}}_{\alpha}
-2\nabla_{\mu}\nabla_{\nu}{R^{\alpha}}_{\alpha} -
2{R^{\alpha}}_{\alpha}R_{\mu\nu} + g_{\mu\nu}R^{2}/2
\end{equation}
where $S_1$ and $S_2$ are the actions of the Lagrangians $L_1$
and $L_2$ respectively. In the literature, these two quantities
are typically referred to as $W^{(2)}_{\mu\nu} = \delta
S_{2}/\delta g_{\mu\nu}$ and $W^{(1)}_{\mu\nu} = \delta
S_{1}/\delta g_{\mu\nu}$. From them, we can construct the quantity
\begin{equation}
2\alpha_{g}W_{\mu\nu} = 2\alpha(W^{(2)}_{\mu\nu} -
\frac{1}{3}W^{(1)}_{\mu\nu})
\end{equation}
which is precisely the variation of the conformal action. So we
end up with the field equations being
\begin{equation}
4\alpha_{g}W_{\mu\nu} = T_{\mu\nu}
\end{equation}
where $T_{\mu\nu}$ is the stress-energy tensor we all know and love.

%%
% Break this part up into a new section on the non-uniqueness of the
% Schwarzschild metric?
%%
Although this is an intimidating system of coupled fourth order,
nonlinear partial differential equations, there are a few
solutions calculated out. Mannheim and
Kazanas~\cite{Mannheim:1988dj} have computed the exact solution
exterior to a static, spherically symmetric gravitating source,
which is
\begin{equation}
-g_{00} = 1/g_{rr} = 1 - \frac{\beta(2 - 3\beta\gamma)}{r} - 3\beta\gamma
+ \gamma r - kr^2
\end{equation}
where the parameters $\beta$, $\gamma$ and $k$ are three
dimensionful integrations constants which appear in the solution
but not in the equations of motion. They spontaneously break the
scale symmetry. This should look familiar as it resembles the
Schwarzschild solution with a cosmological constant
\begin{equation}\label{exteriorMetric}
-g_{00} = 1/g_{rr} = 1 + \frac{\Lambda}{3}r^{2} - \frac{2m}{r}
\end{equation}
in units where $G_{N} = 1$ and $c=1$. The only difference is a
constant term and a term that linearly depends on $r$.

Here we need to reiterate so one appreciates the beauty of the
situation. \marginpar{\footnotesize{Breaking symmetry gives information about cosmological constant}} In a Lagrangian of
the form \eqref{conformalAction} which has no boundary term or
constant term added in by hand, makes no assumptions about the
cosmological constant, one can solve for the spherically
symmetry, static gravitating body and one \emph{naturally} gets a
term which yields information about the cosmological constant and
a term which breaks symmetry to give masses to the massless
particles. 

\section{Conformal Cosmology}
%%
%% cosmology.tex
%% 
%% Made by Alex Nelson
%% Login   <alex@tomato>
%% 
%% Started on  Sun Dec  7 19:30:38 2008 Alex Nelson
%% Last update Wed Dec 10 02:01:26 2008 Alex Nelson
%%

We begin by thinking about breaking symmetry differently (read:
in the naive way) by considering\footnote{Note that this is for
  De-Sitter spacetime, to make this anti-de-Sitter spacetime we
  need to change the sign of the $\phi^{4}$ term. This has been
  calculated in \cite{Edery:2006hg}.} the Lagrangian of matter
  conformally coupled to gravity~\cite{Mannheim:1999nc,Mannheim:2007ki} 
\begin{equation}\label{symmetryBreakingAction}
I_{M} = -\int
d^{4}x\sqrt{-g}\left[\underbrace{\frac{1}{2}\nabla^{\mu}\phi\nabla_{\mu}\phi -
  \frac{1}{12}\phi^{2}R + \lambda \phi^{4}}_{\text{scalar}} +
  \underbrace{i\bar{\psi}\gamma^{\mu}(x)[\partial_{\mu} +
    \Gamma_{\mu}(x)]\psi}_{\text{fermion}} - \underbrace{g\phi\bar{\psi}\psi}_{\text{interaction}}\right]
\end{equation}
where $\Gamma_{\mu}(x)$ is the fermion spin connection, $\lambda$ and
$g$ are the dimensionless coupling constants, $\phi(x)$ is the (symmetry
breaking) scalar field and $\psi$ is a fermionic field. 

We will demonstrate that the scalar spontaneously breaks
symmetry. Observe that the potential term for the scalar field is 
\begin{equation}
V(\phi) = \frac{\phi^{2}R}{12} - \lambda\phi^{4}
\end{equation}
we take its derivative 
\begin{equation}
V'(\phi) = \frac{\phi R}{6} - 4\lambda\phi^3
\end{equation}
then set it to zero and solve for $\phi$. The resulting value is
\begin{equation}
v = \pm\sqrt{\frac{R}{24\lambda}}
\end{equation}
then we plug it back into the potential to find
\begin{equation}
V\left(\pm\sqrt{\frac{R}{24\lambda}}\right)
= \left(\frac{R}{24\lambda}\right)\frac{R}{12} - \lambda\left(\frac{R}{24\lambda}\right)^{2} = \frac{R^2}{576\lambda}
\end{equation}
which is nonzero, which implies that symmetry is spontaneously
broken. Note that if we included the fermion-scalar interaction term,
the results would not have changed as it would have been equivalent to
adding a term linear in $\phi$ into the potential (for explicit
calculations refer to appendix B). (Observe the dependence on $R$ is
directly proportional too.)

When the scalar field $\phi(x)$ in $I_M$ obtains a nonzero mass (which
we are free to rotate to some ``spacetime constant'' $\phi_{0}$ due to
conformal invariance), the fermion then obeys the curved space Dirac
equations
\begin{equation}
i\hbar\overline{\psi}\gamma^{\mu}(x)(\partial_{\mu}
+ \Gamma_{\mu}(x))\psi = \hbar g\phi_{0}\psi
\end{equation}
and acquires a mass $\hbar g\phi_{0}$. The scalar field's equation of
motion is
\begin{equation}
\nabla_{\mu}\nabla^{\mu} \phi + \frac{\phi R}{6} - 4\lambda \phi^{3} +
g\bar{\psi}\psi = 0.
\end{equation}

The corresponding stress-energy tensor to \eqref{symmetryBreakingAction} is
\begin{eqnarray}
T^{\mu\nu} &=& \hbar\Big[i\bar{\psi}\gamma^\mu(\partial^\nu
+ \Gamma^\nu)\psi + \frac{2}{3}\nabla^{\mu}\phi\nabla^{\nu}\phi
- \frac{g^{\mu\nu}}{6}\nabla^{\alpha}\phi\nabla_{\alpha}\phi
-\frac{\phi\nabla^{\mu}\nabla_{\nu}\phi}{3} \nonumber\\
& & + \frac{g^{\mu\nu}\phi\nabla^{\alpha}\nabla_{\alpha}\phi}{3}
- \frac{\phi^2}{6}(R^{\mu\nu} -\frac{g^{\mu\nu}}{2}R) - g^{\mu\nu}\lambda\phi^{4}\Big]
\end{eqnarray}
which can be rewritten grouping terms in a more elegant manner. If we
think of 
\begin{equation}
\rho u^\mu u^\nu = i\hbar\bar{\psi}\gamma^\mu(\partial^\nu
+ \Gamma^\nu)\psi + \frac{\hbar}{2}\nabla^{\mu}\phi\nabla^{\nu}\phi
\end{equation}
where $\rho$ is the ``pressure'' of an ideal fluid, $u^\mu$ is thought of
as the worldline (so it satisfies $g_{\mu\nu}u^\mu u^\nu=-1$), and
\begin{equation}
pu^\mu u^\nu = \frac{-\hbar}{3}\phi\nabla^{\mu}\nabla^{\nu}\phi + \frac{\hbar}{6}\nabla^{\mu}\phi\nabla^{\nu}\phi
\end{equation}
where $p$ is the ``pressure'' of an ideal fluid, then the stress
energy tensor may be written as
\begin{equation}\label{stressEnergyTensorConformalCosmology}
T_{\mu\nu} = (\rho + p)U_{\mu}U_{\nu} + pg_{\mu\nu} -
\frac{1}{6}\phi_{0}^{2}\left(R_{\mu\nu} -
\frac{\hbar}{2}g_{\mu\nu}R\right) - g_{\mu\nu}\hbar\lambda \phi_{0}^{4}.
\end{equation}
This is the right hand side of our fourth order field equations.

By working in an isotropic and homogeneous geometry, the left hand side of
\eqref{stressEnergyTensorConformalCosmology} necessarily
vanishes, giving us the equation
\begin{equation}
\frac{1}{6}\phi^{2}\left(R_{\mu\nu} -
\frac{1}{2}g_{\mu\nu}R\right) = (\rho + p)U_{\mu}U_{\nu} +
pg_{\mu\nu}  - g_{\mu\nu}\lambda \phi^{4}
\end{equation}
Thus conformal cosmology looks like the standard cosmology with a
perfect matter fluid and a nonzero cosmological constant with the
\marginpar{\footnotesize{Replace Newton's $G_{N}$ with an effective
one}}important exception that Newton's constant has been replaced by
an ``effective'' constant of the form 
\begin{equation}
G_\text{eff} = \frac{-3}{4\pi \phi_{0}^{2}}.
\end{equation}
\emph{This is not Newton's constant as Cavendish measured, but instead
a term which we identify to be analagous to the Newton gravitational
constant!} Observe that as we change scales, $G_\text{eff}$ changes in inverse
proportion. 

We can also identify the\marginpar{\footnotesize{Cosmological Constant Emerges}} $\Lambda = \lambda \phi_{0}^{4}$ term as being a
cosmological constant. Note that this term really is effectively a
cosmological constant since it is a homogeneous and isotropic global
scalar field. Observe that this cosmological constant
is \begin{inparaenum}
\item scale dependent (that is, quartic in $\phi$),
\item always positive (that is, we have de Sitter spacetime, so
gravity is repulsive \emph{but at this scale}).
\end{inparaenum}
The notion that the cosmological constant is scale dependent is novel,
but the important change is that the sign explains how gravity is
repulsive instead of attractive. Due to the sign, there is no initial
singularity in this model. Instead the universe expands from a finite
minimum radius, and is not subject to the same problems that one
encounters in the standard cosmological model.

Despite finding gravity being globally repulsive, it is locally
attractive. This reconciles the use of the fourth order Poisson
equation \eqref{fourthOrderPoisson} which merely adds an extra term linear in $r$
(radial distance) to the gravitational potential that would be
negligible at \emph{local} scales. It turns out that Mannheim et
al~\cite{Mannheim:1996jt} demonstrate the empirical strength of such a
proposition at the galactic level, but that is beyond the scope of
this article to review it too.

\section{Conclusion}
We introduced a different action which is based off of Weyl's attempt
to unify gravity and electromagnetism. Instead of attempting such a
unified field theory, we observed that it has interesting
gravitational properties. 

The vacuum satisfies the Schwarzschild solution for general relativity
with a nonzero cosmological constant, plus some nonzero term and a
term linear in $r$ negligibly small at the ``local'' scale. Due to
these extra terms, the scale invariance was spontaneously broken. This
was purely accidental.

We also observed that when we solve the fourth order field equations
for the isotropic and homogeneous case, we end up breaking symmetry
again. But in doing so, we recover the standard cosmological model,
and we explained why gravity is accelerating within the framework of
the Conformal gravity model. Further, we have an effective
gravitational constant that is scale dependent which allows gravity to
be repulsive globally but (due to inhomogeneities in the scalar field)
is locally attractive. This is consistent with the first investigation
of spontaneous symmetry breaking in solving the static, spherically
symmetric body's gravitational field as locally (``for small enough
$r$'') resembling Schwarzschild's solution.

Observe that this is really nothing surprising, since this is just
another version of the Brans-Dicke theory. The Brans-Dicke action is
\begin{equation}
I = \frac{1}{16\pi}\int d^{4}x\sqrt{-g}\left(\phi R
- \omega \frac{\partial_{\mu}\phi\partial^{\mu}\phi}{\phi} + L_{matter}\right)
\end{equation}
one can rearrange it by introducing $\Phi^2=\phi$ to look like
\begin{equation}
I = \frac{1}{16\pi}\int d^{4}x\sqrt{-g}\left(\Phi^2R -
4\omega\partial_\mu\Phi\partial^{\mu}\Phi + L_{matter}\right)
\end{equation}
which resembles the action in Eq \eqref{symmetryBreakingAction}. What
the Brans-Dicke theory effectively does is replace $k=16\pi G/c^4$
with a scalar field $\phi$. We did something similar, except our
scalar field spontaneously broke the scale invariance (so,
analogously, we had a bare minimum value for $k$) which gave rise
to a cosmological constant in addition to recovering the standard
cosmological model. Further, we used covariant derivatives instead of
partial derivatives, so we would need to include in the $L_{matter}$
the extra terms, the $\Phi^4$ term, and the coupling to
matter. Nonetheless, the cosmological constant naturally emerges when
we break symmetry. 

\appendix
%%
%% symmetryBreaking.tex
%% 
%% Made by Alex Nelson
%% Login   <alex@tomato>
%% 
%% Started on  Sat Nov 29 13:34:02 2008 Alex Nelson
%% Last update Sun Dec  7 15:27:16 2008 Alex Nelson
%%

\section{Spontaneous Symmetry Breaking}

Spontaneous symmetry breaking occurs whenever a given field in a given
Lagrangian has a nonzero vacuum expectation value. Why exactly is this
``breaking'' the symmetry? Well, the Lagrangian appears symmetric
under a symmetry group, but its vacuum state fails to be
symmetric. The system no longer behaves symmetrically. So we went from
symmetry to no symmetry due to a nonzero vacuum expectation value. It
came about from condensed matter physics (see~\cite{wen} for
applications of it in condensed matter physics) but has
since been applied to quantum field theory and particle physics
(see~\cite{peskinSchroeder} for examples in particle physics).

Consider the scalar Lagrangian given by
\begin{equation}
\mathcal{L} =
\underbrace{\frac{1}{2}(\partial_{\mu}\phi)^2}_{\text{``kinetic term''}} +
\underbrace{\frac{1}{2}\mu^{2}\phi^{2} -
  \frac{\lambda}{4!}\phi^{4}}_{\text{``potential term''}}
\end{equation}
where $\phi$ is the scalar field, $\mu$ is a sort of ``mass''
parameter, and $\lambda$ is the coupling. Observe that there is a
symmetry of $\phi\to-\phi$ (a discrete symmetry).  We can think of the
potential as being
\begin{equation}
V(\phi) = -\frac{1}{2}\mu^{2}\phi^2 + \frac{\lambda}{4!}\phi^{4}
\end{equation}
which has extrema when its derivative is zero. There are two, given by
\begin{equation}
\phi_{0} = \pm v = \pm \mu\sqrt{\frac{6}{\lambda}}
\end{equation}
where the constant $v$ is the ``\textbf{vacuum expectation value}''.

We can then write
\begin{equation}
\phi(x) = v + \sigma(x)
\end{equation}
and then rewrite the Lagrangian as
\begin{equation}
\mathcal{L} = \frac{1}{2}(\partial_{\mu}\sigma)^{2} -
\frac{1}{2}(2\mu^2)\sigma^2 - \sqrt{\frac{\lambda}{6}}\mu\sigma^3 -
\frac{\lambda}{4!}\sigma^{4}
\end{equation}
where we dropped the constant terms. We see that the symmetry
$\phi\to-\phi$ is no longer identifiable. 

%\section{Spontaneous Symmetry Breaking in Non-relativistic Conformal Gravity}\label{nonrelativisticSymmetryBreaking}
%%%
%% nonrelativisticSymmetryBreaking.tex
%% 
%% Made by Alex Nelson
%% Login   <alex@tomato>
%% 
%% Started on  Sun Dec  7 14:49:33 2008 Alex Nelson
%% Last update Sun Dec  7 16:38:38 2008 Alex Nelson
%%

Mannheim notes~\cite{Mannheim:1993rs} that from the fourth order
Poisson equation
\begin{equation}
\nabla^{4}B(r) = f(r)
\end{equation} 
where for a spherical source its exact exterior solution is
\begin{equation}
B(r>R) = \frac{-r}{2}\int^{R}_{0}d\rho f(\rho)\rho^2 -
\frac{1}{6r}\int^{R}_{0}d\rho f(\rho)\rho^4
\end{equation}
we can use the non-relativistic potential
\begin{equation}
V(r) = \frac{-\beta}{r} + \frac{\gamma}{2}r.
\end{equation}
(Numerically the $\beta\gamma$ term is negligible.) It turns
out~\cite{Mannheim:1994ph} that we can write the spherically
symmetric source function as
\begin{equation}
f(r) = \frac{3({T^{0}}_{0} - {T^{r}}_{r})}{4\alpha B(r)}
\end{equation}
which allows us to see, since the radial piece of
$\nabla^{4}B(r)$ is $(rB)''''/r$, that the exterior metric
\eqref{exteriorMetric} emerges as the most general solution to
the fourth order Laplace equation $\nabla^4 B(r)=0$ More
importantly, this lets us write
\begin{equation}
\gamma = \frac{-1}{2}\int^{R}_{0}d\rho f(\rho) \rho^2
\end{equation}
and
\begin{equation}
\beta(2-3\beta\gamma) = \frac{1}{6}\int^{R}_{0}d\rho
f(\rho)\rho^{4}
\end{equation}
which means that $\gamma$ is some negative constant and $\beta$
is some positive constant for a body with positive $f(r)$ (it
also has to be positive so the potential contains the correct
Newtonian limit).

We see then that
\begin{equation}
\frac{d}{dr}V(r) = \frac{\beta}{r^2} + \frac{\gamma}{2}
\end{equation}
has extrema at
\begin{equation}
r_{0} = \pm\sqrt{\frac{-2\beta}{\gamma}}
\end{equation}
which is nonzero for massive bodies. (Also, don't be fooled by
the negative in the squareroot, $\gamma$ is negative so the two
become positive thus $r_0\in\mathbb{R}$.) This is an indicator
that symmetry is spontaneously broken even nonrelativistically!

\section{Proof of Spontaneous Symmetry Breaking with extra Linear
  Term}
The potential we are investigating has the form
\begin{equation}
V(\phi) = c\phi + \frac{\phi^{2}R}{12} - \lambda\phi^{4}
\end{equation}
where $c$ is some constant term, so its first derivative would be
\begin{equation}
V'(\phi) = c + \frac{\phi R}{6} - 4\lambda\phi^3.
\end{equation}
By the fundamental theorem of algebra, it has exactly three roots at the values
\begin{eqnarray*}
v_{1} &=& -\frac{\sqrt[3]{2} R \lambda +\left(\sqrt{2}
    \sqrt{\lambda ^3 \left(5832 c^2 \lambda -R^3\right)}-108 c \lambda
    ^2\right)^{2/3}}{6 2^{2/3} \lambda  \sqrt[3]{\sqrt{2} \sqrt{\lambda ^3
    \left(5832 c^2 \lambda -R^3\right)}-108 c \lambda ^2}} \\
v_{2} &=& \frac{2 \sqrt[3]{-2} R \lambda +\left(1-i \sqrt{3}\right)
    \left(\sqrt{2} \sqrt{\lambda ^3 \left(5832 c^2 \lambda -R^3\right)}-108 c
    \lambda ^2\right)^{2/3}}{12 2^{2/3} \lambda  \sqrt[3]{\sqrt{2}
    \sqrt{\lambda ^3 \left(5832 c^2 \lambda -R^3\right)}-108 c \lambda
    ^2}} \\
v_{3} &=& \frac{\left(1+i \sqrt{3}\right)
    \left(\sqrt{2} \sqrt{\lambda ^3 \left(5832 c^2 \lambda -R^3\right)}-108 c
    \lambda ^2\right)^{2/3}-2 (-1)^{2/3} \sqrt[3]{2} R \lambda }{12 2^{2/3}
    \lambda  \sqrt[3]{\sqrt{2} \sqrt{\lambda ^3 \left(5832 c^2 \lambda
    -R^3\right)}-108 c \lambda ^2}}.
\end{eqnarray*}
We can plug each of these into the function with the extra linear term
and find
\begin{eqnarray*}
V(v_{1}) &=&    -\frac{\left(\sqrt[3]{2} R \lambda +\left(\sqrt{2} \sqrt{\lambda ^3
    \left(5832 c^2 \lambda -R^3\right)}-108 c \lambda
    ^2\right)^{2/3}\right)^4}{5184 2^{2/3} \lambda ^3 \left(\sqrt{2}
    \sqrt{\lambda ^3 \left(5832 c^2 \lambda -R^3\right)}-108 c \lambda
    ^2\right)^{4/3}}\\
& &+\frac{R \left(\sqrt[3]{2} R \lambda +\left(\sqrt{2}
    \sqrt{\lambda ^3 \left(5832 c^2 \lambda -R^3\right)}-108 c \lambda
    ^2\right)^{2/3}\right)^2}{864 \sqrt[3]{2} \lambda ^2 \left(\sqrt{2}
    \sqrt{\lambda ^3 \left(5832 c^2 \lambda -R^3\right)}-108 c \lambda
    ^2\right)^{2/3}} \\
& &-\frac{c \left(\sqrt[3]{2} R \lambda +\left(\sqrt{2}
    \sqrt{\lambda ^3 \left(5832 c^2 \lambda -R^3\right)}-108 c \lambda
    ^2\right)^{2/3}\right)}{6 2^{2/3} \lambda  \sqrt[3]{\sqrt{2} \sqrt{\lambda
    ^3 \left(5832 c^2 \lambda -R^3\right)}-108 c \lambda ^2}}
\end{eqnarray*}
\begin{eqnarray*}
V(v_2) &=& -\frac{\left(2 \sqrt[3]{-2} R \lambda +\left(1-i \sqrt{3}\right)
    \left(\sqrt{2} \sqrt{\lambda ^3 \left(5832 c^2 \lambda -R^3\right)}-108 c
    \lambda ^2\right)^{2/3}\right)^4}{82944 2^{2/3} \lambda ^3 \left(\sqrt{2}
    \sqrt{\lambda ^3 \left(5832 c^2 \lambda -R^3\right)}-108 c \lambda
    ^2\right)^{4/3}} \\
& & +\frac{R \left(2 \sqrt[3]{-2} R \lambda +\left(1-i
    \sqrt{3}\right) \left(\sqrt{2} \sqrt{\lambda ^3 \left(5832 c^2 \lambda
    -R^3\right)}-108 c \lambda ^2\right)^{2/3}\right)^2}{3456 \sqrt[3]{2}
    \lambda ^2 \left(\sqrt{2} \sqrt{\lambda ^3 \left(5832 c^2 \lambda
    -R^3\right)}-108 c \lambda ^2\right)^{2/3}} \\
& & +\frac{c \left(2 \sqrt[3]{-2} R
    \lambda +\left(1-i \sqrt{3}\right) \left(\sqrt{2} \sqrt{\lambda ^3
    \left(5832 c^2 \lambda -R^3\right)}-108 c \lambda
    ^2\right)^{2/3}\right)}{12 2^{2/3} \lambda  \sqrt[3]{\sqrt{2}
    \sqrt{\lambda ^3 \left(5832 c^2 \lambda -R^3\right)}-108 c \lambda ^2}}
\end{eqnarray*}
\begin{eqnarray*}
V(v_{3}) &=& -\frac{\left(\left(1+i \sqrt{3}\right) \left(\sqrt{2} \sqrt{\lambda ^3
    \left(5832 c^2 \lambda -R^3\right)}-108 c \lambda ^2\right)^{2/3}-2
    (-1)^{2/3} \sqrt[3]{2} R \lambda \right)^4}{82944 2^{2/3} \lambda ^3
    \left(\sqrt{2} \sqrt{\lambda ^3 \left(5832 c^2 \lambda -R^3\right)}-108 c
    \lambda ^2\right)^{4/3}} \\ 
& &+\frac{R \left(\left(1+i \sqrt{3}\right)
    \left(\sqrt{2} \sqrt{\lambda ^3 \left(5832 c^2 \lambda -R^3\right)}-108 c
    \lambda ^2\right)^{2/3}-2 (-1)^{2/3} \sqrt[3]{2} R \lambda \right)^2}{3456
    \sqrt[3]{2} \lambda ^2 \left(\sqrt{2} \sqrt{\lambda ^3 \left(5832 c^2
    \lambda -R^3\right)}-108 c \lambda ^2\right)^{2/3}} \\
& &+\frac{c
    \left(\left(1+i \sqrt{3}\right) \left(\sqrt{2} \sqrt{\lambda ^3 \left(5832
    c^2 \lambda -R^3\right)}-108 c \lambda ^2\right)^{2/3}-2 (-1)^{2/3}
    \sqrt[3]{2} R \lambda \right)}{12 2^{2/3} \lambda  \sqrt[3]{\sqrt{2}
    \sqrt{\lambda ^3 \left(5832 c^2 \lambda -R^3\right)}-108 c \lambda ^2}}.
\end{eqnarray*}
Observe that these are nonzero quantities, just a whole lot messier
than neglecting that linear term.

\bibliographystyle{utphys}
\bibliography{conformal}
\end{document}

