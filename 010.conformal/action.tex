%%
%% action.tex
%% 
%% Made by Alex Nelson
%% Login   <alex@tomato>
%% 
%% Started on  ??? Dec  ? ??:??:?? 2008 Alex Nelson
%% Last update Wed Dec 10 02:03:11 2008 Alex Nelson
%%

One begins with the conformally-invariant fourth order action
\begin{equation}\label{conformalAction}
I_{W} = -\alpha_{g}\int
d^{4}x\sqrt{-g}C_{\alpha\beta\gamma\delta}C^{\alpha\beta\gamma\delta}
\end{equation}
where $\alpha_{g}$ is the coupling constant and
$C_{\alpha\beta\gamma\delta}$ is the Weyl tensor. This action is
invariant under conformal transformations of the metric
\begin{equation*}
g_{\mu\nu}(x)\to e^{2\alpha(x)}g_{\mu\nu}(x)
\end{equation*}
(Originally Weyl considered $\alpha$ in the action
\eqref{conformalAction} to be used for the both the conformal and
the electromagnetic gauge transformations. This way one has for the
covariant derivative $\nabla_{\mu}g_{\alpha\beta} =
A_{\mu}g_{\alpha\beta}$ where $A_{\mu}$ is the electromagnetic
4-potential. The problem with this approach is that conformal
invariance implied the particles are massless, which is observably
false.) We can now simplify this Lagrangian a bit.

By plugging in the definition of the Riemann tensor, and recalling
that any contraction of any pair of indices of the Weyl tensor
vanishes, we see
\begin{equation}
R_{\mu\nu\alpha\beta}R^{\mu\nu\alpha\beta} =
C_{\mu\nu\alpha\beta}C^{\mu\nu\alpha\beta} + 2R_{\mu\nu}R^{\mu\nu} - \frac{1}{3}R^{2}
\end{equation}
where $R$ is the Ricci scalar. Rearranging terms, we have
\begin{equation}
C_{\mu\nu\alpha\beta}C^{\mu\nu\alpha\beta} =
R_{\mu\nu\alpha\beta}R^{\mu\nu\alpha\beta} -  2R_{\mu\nu}R^{\mu\nu} + \frac{1}{3}R^{2}.
\end{equation}


Before beginning, note that the quantity~\cite{Kazanas:1988qa,Lanczos:1938sf}
\begin{equation}
\sqrt{-g}\left(R_{\alpha\beta\mu\nu}R^{\alpha\beta\mu\nu} -
4R_{\alpha\beta}R^{\alpha\beta} + R^{2}\right)
\end{equation}
is a total divergence. So instead of having our Lagrangian be
\begin{equation}
L = \sqrt{-g}C_{\alpha\beta\mu\nu}C^{\alpha\beta\mu\nu}
\end{equation}
we can \emph{equivalently} use the Lagrangian
\begin{equation}
L = -2\sqrt{-g}(R_{\alpha\beta}R^{\alpha\beta} - R^{2}/3)
\end{equation}
since we would be working with an extra term (by Stoke's theorem a surface
integral), and by demanding the variation vanishes on the boundary
the only nonzero contribution would be this Lagrangian.

Now, De Witt~\cite{dewitt1964} explicitly calculates out the
equations of motion for two Lagrangians:
\begin{equation}
L_{2} = \sqrt{-g}R^{2},\qquad\text{and}\qquad L_{1}=\sqrt{-g}R^{\mu\nu}R_{\mu\nu}
\end{equation}
Our Lagrangian is a linear combination of these two, so we use a
linear combination of the variation of their respective actions
\begin{equation}
\frac{\delta S_{2}}{\delta g^{\mu\nu}} = \frac{g_{\mu\nu}}{2}
\nabla^{\beta}\nabla_{\beta}({R^{\alpha}}_{\alpha})  +
\nabla^{\beta}\nabla_{\beta}R_{\mu\nu}  -
\nabla_{\beta}\nabla_{\nu}{R_{\mu}}^{\beta} - \nabla_{\beta}\nabla_{\mu}{R_{\nu}}^{\beta}
-2R_{\mu\beta}{R_{\nu}}^{\beta} + \frac{g_{\mu\nu}}{2}R_{\alpha\beta}R^{\alpha\beta}
\end{equation}
(where $\nabla_{\mu}$ is the covariant derivative operator) and
\begin{equation}
\frac{\delta S_{1}}{\delta g^{\mu\nu}} =
2g_{\mu\nu}\nabla^{\beta}\nabla_{\beta}{R^{\alpha}}_{\alpha}
-2\nabla_{\mu}\nabla_{\nu}{R^{\alpha}}_{\alpha} -
2{R^{\alpha}}_{\alpha}R_{\mu\nu} + g_{\mu\nu}R^{2}/2
\end{equation}
where $S_1$ and $S_2$ are the actions of the Lagrangians $L_1$
and $L_2$ respectively. In the literature, these two quantities
are typically referred to as $W^{(2)}_{\mu\nu} = \delta
S_{2}/\delta g_{\mu\nu}$ and $W^{(1)}_{\mu\nu} = \delta
S_{1}/\delta g_{\mu\nu}$. From them, we can construct the quantity
\begin{equation}
2\alpha_{g}W_{\mu\nu} = 2\alpha(W^{(2)}_{\mu\nu} -
\frac{1}{3}W^{(1)}_{\mu\nu})
\end{equation}
which is precisely the variation of the conformal action. So we
end up with the field equations being
\begin{equation}
4\alpha_{g}W_{\mu\nu} = T_{\mu\nu}
\end{equation}
where $T_{\mu\nu}$ is the stress-energy tensor we all know and love.

%%
% Break this part up into a new section on the non-uniqueness of the
% Schwarzschild metric?
%%
Although this is an intimidating system of coupled fourth order,
nonlinear partial differential equations, there are a few
solutions calculated out. Mannheim and
Kazanas~\cite{Mannheim:1988dj} have computed the exact solution
exterior to a static, spherically symmetric gravitating source,
which is
\begin{equation}
-g_{00} = 1/g_{rr} = 1 - \frac{\beta(2 - 3\beta\gamma)}{r} - 3\beta\gamma
+ \gamma r - kr^2
\end{equation}
where the parameters $\beta$, $\gamma$ and $k$ are three
dimensionful integrations constants which appear in the solution
but not in the equations of motion. They spontaneously break the
scale symmetry. This should look familiar as it resembles the
Schwarzschild solution with a cosmological constant
\begin{equation}\label{exteriorMetric}
-g_{00} = 1/g_{rr} = 1 + \frac{\Lambda}{3}r^{2} - \frac{2m}{r}
\end{equation}
in units where $G_{N} = 1$ and $c=1$. The only difference is a
constant term and a term that linearly depends on $r$.

Here we need to reiterate so one appreciates the beauty of the
situation. \marginpar{\footnotesize{Breaking symmetry gives information about cosmological constant}} In a Lagrangian of
the form \eqref{conformalAction} which has no boundary term or
constant term added in by hand, makes no assumptions about the
cosmological constant, one can solve for the spherically
symmetry, static gravitating body and one \emph{naturally} gets a
term which yields information about the cosmological constant and
a term which breaks symmetry to give masses to the massless
particles. 
