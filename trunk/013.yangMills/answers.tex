
\ansno1.1:
 We find that \begin {equation}{\delta _{a}}^{a} = \sum _{a=1}^{n}{\delta _{a}}^{a}=\sum ^{n}_{a=1}1=n\end {equation} by simply using the Einstein summation convention.

\ansno1.2:
 We find that we can write the matrix $A^{ab}$ as \begin {equation} A^{ab} = \frac {1}{2}(A^{ab}-A^{ba}) \end {equation} so in particular, we can plug this into the question and find \begin {subequations} \begin {align} S_{ab}A^{ab} &= \frac {1}{2}\sum _{a,b}S_{ab}(A^{ab}-A^{ba})\\ &= \frac {1}{2}\sum _{a,b}(S_{ab}A^{ab}-S_{ab}A^{ba})\\ &=\frac {1}{2}\sum _{a,b}(S_{ab}A^{ab}-S_{ba}A^{ba})\\ &=\frac {1}{2}\sum _{a,b}(S_{ab}A^{ab}-S_{ab}A^{ab})\\ &= 0. \end {align} \end {subequations} We justify the second to last step by noting we are summing over \emph {dummy indices}, so we can relabel them how we want.

\ansno1.3:
 We find by definition that \begin {subequations} \begin {align} A^{ab}T_{ab} &= \frac {1}{2}(A^{ab}-A^{ba})T_{ab}\\ &=\frac {1}{2}(A^{ab}T_{ab}-A^{ba}T_{ab})\\ &=\frac {1}{2}(A^{ab}T_{ab}-A^{ab}T_{ba})\\ &=\frac {1}{2}A^{ab}(T_{ab}-T_{ba}) \end {align} \end {subequations} where we once again use the fact that we can switch dummy indices when we contract over them. Alternatively, we can observe that we can write \begin {equation}T_{ab} = \frac {1}{2}(T_{ab}+T_{ba})+\frac {1}{2}(T_{ab}-T_{ba}) \end {equation} that is, any arbitrary tensor can be written as the sum of a symmetric tensor (first term) and an antisymmetric tensor (second term). Upon this realization, we find that the first term contracted with $A^{ab}$ is --- by the previous exercise! --- precisely zero. We are left with \begin {equation}A^{ab}T_{ab} = 0 + \frac {1}{2}A^{ab}(T_{ab}-T_{ba}) \end {equation} precisely as desired.

\ansno1.4:
 We transform coordinates by $x^{\alpha }\mapsto y^{\alpha '}$ and then we find \begin {subequations} \begin {align} {\delta _{\beta '}}^{\alpha '} &= \left [\frac {\partial x^{\beta }}{\partial y^{\beta '}}\frac {\partial y^{\alpha '}}{\partial x^{\alpha }}\right ]{\delta _{\beta }}^{\alpha }\\ &= \left [\delta ^{\beta }_{\beta '}\delta ^{\alpha '}_{\alpha }\right ]{\delta _{\beta }}^{\alpha }\\ &= \delta ^{\beta }_{\beta '}{\delta _{\beta }}^{\alpha '}\\ &= {\delta _{\beta '}}^{\alpha '} \end {align} \end {subequations} precisely as desired.

\ansno1.5:
 We find that the partial derivatives transform covariantly \begin {equation}\partial _{a'} = \frac {\partial x^{a}}{\partial y^{a'}}\partial _{a} \end {equation} and the components of a vector transform contravariantly \begin {equation}v^{b'} = \frac {\partial y^{b'}}{\partial x^{b}}v^{b}. \end {equation} When we plug these into our expected equation we find \begin {subequations} \begin {align} \partial _{a'}v^{b'} &= \left (\frac {\partial x^{a}}{\partial y^{a'}}\partial _{a}\right )\left (\frac {\partial y^{b'}}{\partial x^{b}}v^{b}\right )\\ &= \frac {\partial x^{a}}{\partial y^{a'}}\left (v^{b}\partial _{a}\frac {\partial y^{b'}}{\partial x^{b}} + \frac {\partial y^{b'}}{\partial x^{b}}\partial _{a}v^{b}\right ) \end {align} \end {subequations} which we see has some extra term involving the derivatives of the Jacobian matrix.

\ansno1.6:
 We see from the previous exercise that \begin {equation}\partial _{a'}v^{b'} = \frac {\partial x^{a}}{\partial y^{a'}}\left (v^{b}\partial _{a}\frac {\partial y^{b'}}{\partial x^{b}} + \frac {\partial y^{b'}}{\partial x^{b}}\partial _{a}v^{b}\right ) \end {equation} so we find that \begin {equation}\partial _{a'}v_{b'} = \frac {\partial x^{a}}{\partial y^{a'}}\left (v_{b}\partial _{a}\frac {\partial x^{b}}{\partial y^{b'}} + \frac {\partial x^{b}}{\partial y^{b'}}\partial _{a}v_{b}\right ) \end {equation} Then by antisymmetrization we find that \begin {subequations} \begin {align} \partial _{a'}v_{b'}-\partial _{b'}v_{a'} &= \frac {\partial x^{a}}{\partial y^{a'}}\left (v_{b}\partial _{a}\frac {\partial x^{b}}{\partial y^{b'}} + \frac {\partial x^{b}}{\partial y^{b'}}\partial _{a}v_{b}\right ) \nonumber \\ & - \frac {\partial x^{b}}{\partial y^{b'}}\left (v_{a}\partial _{b}\frac {\partial x^{a}}{\partial y^{a'}}+\frac {\partial x^{a}}{\partial y^{a'}}\partial _{b}v_{a}\right ) \end {align} \end {subequations} We see that by index gymnastics the expression \begin {equation}\frac {\partial x^{a}}{\partial y^{a'}}v_{b}\partial _{a}\frac {\partial x^{b}}{\partial y^{b'}}-\frac {\partial x^{b}}{\partial y^{b'}} v_{a}\partial _{b}\frac {\partial x^{a}}{\partial y^{a'}} = 0 \end {equation} identically, since we are subtracting out one thing from itself. (Again, justified by switching dummy indices.) This then implies we get the expression \begin {equation}\partial _{a'}v_{b'}-\partial _{b'}v_{a'} = \frac {\partial x^{a}}{\partial y^{a'}}\frac {\partial x^{b}}{\partial y^{b'}}\partial _{a}v_{b} - \frac {\partial x^{b}}{\partial y^{b'}}\frac {\partial x^{a}}{\partial y^{a'}}\partial _{b}v_{a} \end {equation} which transforms precisely as a covariant tensor with two indices.
