%%
%% rules.tex
%% 
%% Made by Alex Nelson
%% Login   <alex@tomato>
%% 
%% Started on  Fri Sep 18 11:35:51 2009 Alex Nelson
%% Last update Fri Sep 18 11:35:51 2009 Alex Nelson
%%

Now, we wish to outline the ``rules'' to constructing some Yang-Mills
theory. The basic principle is really quite simple, but we would like
to make certain that we are really generalizing electromagnetism. To
check that we are, we will consider the situation when the gauge 
group has the Lie algebra $\mathfrak{u}(1)$ --- the gauge group 
corresponding to electromagnetism's. If we get back Maxwell's 
equations, then we're golden and our generalization is
``correct''.

The general structure of this section is thus a presentation of
the basic set up, followed by a demonstration that we recover
Maxwell's equations when we choose $\mathfrak{u}(1)$ for our
gauge group's Lie Algebra. Then we will consider the Canonical
(i.e. Hamiltonian) formalism applied to the Yang-Mills
situation. We will also try to present the Yang-Mills equations
in vector calculus form, analogous to Maxwell's equations.

We will have to introduce slightly odd notation. At first it will
seem incredibly foreign, but it should become more familiar the
more one works with it. We will use capital Latin letters for
group indices. When working with electromagnetism, our gauge
group was $U(1)$ which is precisely the roots of unity,
i.e. looked like $\exp(i\lambda)$ for some
$\lambda\in\mathbb{R}$. But when working with e.g. $SU(2)$, we
have 2 by 2 matrices be our group elements (that have determinant
one, and are equal to its complex conjugate transpose). We had 2
indices for a matrix, so a group element would look like
$g_{II'}$. The notation is borrowed from Penrose and Rindler~\cite{penroseRindlerOne,penroseRindlerTwo}.

\ssn{Basic Setup}
Let $\mathfrak{g}$ be a Lie Algebra. The gauge field $A$ is a 1-form
on spacetime $M$ with values in $\mathfrak{g}$ we write this as
$A\in\Omega^{1}(M,\mathfrak{g})$. We have its curvature be the
field strength
\begin{equation}%\label{eq:}
F=(d+A)\wedge A=dA+\frac{1}{2}[A,A]=\frac{1}{2}F_{\mu\nu}dx^{\mu}dx^{\nu}\in\Omega^{2}(M,\mathfrak{g}),
\end{equation}
and the Yang-Mills action
\begin{equation}
S_{YM}[A] =
\frac{-1}{2}\int_{M}(F,*F)=\frac{-1}{2g^{2}_{n}}\int_{M}\tr(F^{*}_{\mu\nu}F^{\mu\nu})dV.
\end{equation}
We can quantize this using the various tools of the quantization
of gauge theory (see Henneaux and Teitelboim~\cite{Henneaux:1985kr} and
Henneaux~\cite{Henneaux:1992ig} for details on this fascinating topic).
