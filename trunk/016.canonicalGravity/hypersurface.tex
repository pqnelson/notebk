%%
%% hypersurface.tex
%% 
%% Made by Alex Nelson
%% Login   <alex@tomato>
%% 
%% Started on  Sat Jun  6 14:24:25 2009 Alex Nelson
%% Last update Sat Jun  6 14:24:25 2009 Alex Nelson
%%

Lets briefly turn our attention to the geometry of hypersurfaces
in four dimensions. In a four dimensional manifold, a
hypersurface is a three dimensional submanifold that can be
timelike, spacelike, or null. We may consider a particular
hypersurface $\Sigma$ by restricting the coordinates
\begin{equation}%\label{eq:}
C(x^{\alpha})=0
\end{equation}
or by parametrizing the coordinates
\begin{equation}%\label{eq:}
x^{\alpha} = x^{\alpha}(y^a)
\end{equation}
where $y^a$ ($a=1,$ 2, 3) are the coordinates intrinsic to the
hypersurface. Think of the sphere in three dimensions, we can
specify it by the restriction of the coordinates
\begin{equation}%\label{eq:}
C(x,y,z) = R^2-x^2-y^2-z^2 = 0
\end{equation}
or in parametric form
\begin{subequations}
\begin{align}
x(\theta,\phi) &= R\sin(\theta)\cos(\phi)\\
y(\theta,\phi) &= R\sin(\theta)\sin(\phi)\\
z(\theta,\phi) &= R\cos(\theta)
\end{align}
\end{subequations}
where $\theta$, $\phi$ are the coordinates intrinsic to the sphere.

\marginpar{Unit normal} Now, if we consider the vector $\partial_{\alpha}C$, it is normal
to the hypersurface. This is due to the observation the only
value of $C$ changes in a direction orthogonal to $\Sigma$. If
the surface is not null, we can introduce the unit normal
$n_{\alpha}$ defined such that
\begin{equation}%\label{eq:}
n^{\alpha}n_{\alpha} = s = \begin{cases}+1,&\text{if $\Sigma$ is timelike}\\
-1,&\text{if $\Sigma$ is spacelike.}\end{cases}
\end{equation}
We demand the condition that $n^\alpha$ point in the direction of
increasing $C$,
\begin{equation}%\label{eq:}
n^{\alpha}\partial_{\alpha} C > 0.
\end{equation}
We see that given this condition on the sign of the unit normal,
and the fact that $\partial_{\alpha}C$ points in the direction of
the unit normal, that
\begin{equation}%\label{eq:}
n_{\alpha} = \frac{s\partial_{\alpha}C}{|g^{\mu\nu}\partial_{\mu}C\partial_{\nu}C|^{1/2}}
\end{equation}
provided that the hypersurface is not null (i.e. it must be
either timelike or spacelike). If it were null, then the
denominator vanishes:
\begin{equation}%\label{eq:}
g^{\mu\nu}\partial_{\mu}C\partial_{\nu}C=0
\end{equation}
which is bad.

\marginpar{Induced Metric}The metric intrinsic to
the hypersurface is obtained by restricting the line element $ds^2$ to
displacements confined to the hypersurface. Remember we
parametrized the four dimensional coordinates of the surface by
the equations
\begin{equation}\label{eq:intrinsicRelations}
x^{\alpha}=x^{\alpha}(y^a).
\end{equation}
We see that the vectors
\begin{equation}%\label{eq:}
{e^{\alpha}}_{a} = \frac{\partial x^{\alpha}}{\partial y^a}
\end{equation}
are tangent to the curves contained in $\Sigma$. Why? Well, the
relations described by Eq \eqref{eq:intrinsicRelations} describe
curves contained entirely in $\Sigma$, parametrized in $y^a$, so
differentiating with respect to the parameters would yield the
tangent vectors.

Now, for displacements contained in $\Sigma$, we can write the
infinitesimal line element as
\begin{equation}%\label{eq:}
ds_{\Sigma}^{2} = g_{\alpha\beta}dx^{\alpha}dx^{\beta} =
g_{\alpha\beta}\left(\frac{\partial x^{\alpha}}{\partial y^{a}}dy^{a}\right)\left(\frac{\partial x^{\beta}}{\partial y^{b}}dy^{b}\right)
= h_{ab}dy^ady^b.
\end{equation}
where
\begin{equation}%\label{eq:}
h_{ab} = g_{\alpha\beta}\frac{\partial x^{\alpha}}{\partial
  y^{a}}\frac{\partial x^{\beta}}{\partial y^{b}} = g_{\alpha\beta}{e^{\alpha}}_{a}{e^{\beta}}_{b}
\end{equation}
is called the ``induced metric''. It behaves as a scalar when we
change coordinates $x^{\mu}\to x^{\mu'}$ but it behaves as a
tensor when we change coordinates $y^{m}\to y^{m'}$ intrinsic to
the surface. We'll call such things \define{three-tensors}. 

\subsection{Lie Derivative of the Metric Along a Vector}\label{sstn:lieDerivativeOfMetricAlongVector}
%%
%% lieDerivativeOfMetricAlongVector.tex
%% 
%% Made by Alex Nelson
%% Login   <alex@tomato>
%% 
%% Started on  Fri Jun  5 13:49:18 2009 Alex Nelson
%% Last update Fri Jun  5 13:49:18 2009 Alex Nelson
%%
The Lie derivative of the metric along a vector $\xi^{a}$ is
\begin{equation}\label{eq:lieDerivativeOfMetric}
\mathscr{L}_{\xi}g_{ab} =
g_{ac}\partial_{b}\xi^{c} + 
g_{bc}\partial_{a}\xi^{c} +
\xi^{c}\partial_{c}g_{ab}.
\end{equation}
Observe that
\begin{equation}\label{eq:covariantDerivativeContraction}
g_{bc}\nabla_{a}\xi^{c} = g_{bc}(\partial_a\xi^c + \Gamma^{c}_{ad}\xi^{d})
\end{equation}
where $\Gamma$ is the Christoffel symbol, $\nabla$ is the
covariant derivative. We specifically find
\begin{equation}\label{eq:firstManipulation}
g_{bc}\Gamma^{c}_{ad}\xi^{d} = \Gamma_{bad}\xi^{d}.
\end{equation}
For the affine connection, we have
\begin{equation}\label{eq:affineConnectionConditions}
\partial_{c}g_{ab} = \Gamma_{acb} + \Gamma_{bca} = 0.
\end{equation}
So we plug this into eq \eqref{eq:lieDerivativeOfMetric} to find
\begin{equation}\label{eq:lieDerivativeMutatisMutandi}
\mathscr{L}_{\xi}g_{ab} =
g_{ac}\partial_{b}\xi^{c} + 
g_{bc}\partial_{a}\xi^{c} +
\xi^{c}\left(\Gamma_{acb} + \Gamma_{bca}\right)
\end{equation}
By the properties of the Christoffel symbol, specifically
\begin{equation}%\label{eq:}
\Gamma_{cab} = \Gamma_{cba}
\end{equation}
we can rewrite eq \eqref{eq:covariantDerivativeContraction}
as
\begin{equation}%\label{eq:}
g_{bc}\nabla_{a}\xi^{c} = g_{bc}\partial_a\xi^c + \Gamma_{bad}\xi^{d}.
\end{equation}
Now observe the Lie derivative of the metric along our vector
$\xi^{a}$ can be grouped in terms
\begin{equation}%\label{eq:}
\mathscr{L}_{\xi}g_{ab} =
(g_{ac}\partial_{b}\xi^{c} + \Gamma_{abc}\xi^{c}) + 
(g_{bc}\partial_{a}\xi^{c} + \Gamma_{bac}\xi^{c})
\end{equation}
since the $c$ index is summed over, it's a dummy index. We can
rewrite this in more familiar terms
\begin{equation}%\label{eq:}
\mathscr{L}_{\xi}g_{ab} =
(g_{ac}\partial_{b}\xi^{c} + \Gamma_{abd}\xi^{d}) + 
(g_{bc}\partial_{a}\xi^{c} + \Gamma_{bad}\xi^{d})
\end{equation}
thus
\begin{equation}%\label{eq:}
\mathscr{L}_{\xi}g_{ab} = g_{ac}\nabla_{b}\xi^{c} + g_{bc}\nabla_{a}\xi^{c}.
\end{equation}
Since the connection is metric compatible, we can ``bring the
metric inside the derivative''
\begin{equation*}%\label{eq:}
g_{ab}\nabla_{c}(\cdots) \to \nabla_{c}(g_{ab}\cdots)
\end{equation*}
since $\nabla g_{ab} = 0$. Now we can rewrite our Lie derivative
as
\begin{equation}%\label{eq:}
\mathscr{L}_{\xi}g_{ab} = \nabla_{b}\xi_{a} + \nabla_{a}\xi_{b}.
\end{equation}
This is precisely the Killing equation.


