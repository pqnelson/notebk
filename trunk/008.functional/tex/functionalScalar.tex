%%
%% functionalScalar.tex
%% 
%% Made by Alex Nelson
%% Login   <alex@tomato>
%% 
%% Started on  Sat Aug 15 13:06:34 2009 Alex Nelson
%% Last update Sat Aug 15 13:06:34 2009 Alex Nelson
%%

We would like to generalize our functional methods to field
theory by generalizing the parametrization from a 1 dimensional
parameter (time $t$) to a 4 dimensional parametrization
$x^{\mu}$. So this means that
\begin{align*}
t&\to x^{\mu}\\
\frac{d}{dt}&\to \partial_{\mu}\\
q(t)&\to \varphi(x^{\mu})
\end{align*}
In fact we can set up a dictionary for our generalization
\begin{center}
\begin{tabular}{|c|c|}\hline
mechanics          & field theory\\\hline
$q(t)$             & $\varphi(\bar{x},t)$\qquad{classical field}\\
$\widehat{Q}$      & $\varphi(\bar{x},t)$\qquad{operator field}\\
$f(t)$             & $J(\bar{x},t)$\qquad{classical source}\\\hline
\end{tabular}
\end{center}
We can write the Hamiltonian density for the free scalar field as
\begin{equation}%\label{eq:}
\mathcal{H}_{0} = 
\underbracket[0.25pt]{\frac{1}{2}\Pi^{2} +
\frac{1}{2}(\nabla\varphi)^{2}}_{\mathclap{\substack{=\partial^{\mu}\varphi\partial_{\mu}\varphi\\
=\text{kinetic energy}}}}+
\underbracket[0.25pt]{\frac{1}{2}m^{2}\varphi^{2}}_{\mathclap{\text{potential}}}
\end{equation}
The trick we've employed so many times before using
infinitesimals is done precisely the same way using the
Hamiltonian density. That is,
$\mathcal{H}_{0}\mapsto(1-i\varepsilon)\mathcal{H}_{0}$. For
simplicity, we tacitly assume we always mean
$(1-i\varepsilon)m^{2}$ whenever we write $m^{2}$, it's
completely equivalent to
$\mathcal{H}_{0}\mapsto(1-i\varepsilon)\mathcal{H}_{0}$. 

Consider the functional integral
\begin{equation}%\label{eq:}
Z[J] = \<0|0\>_{J} = \int\mathcal{D}\varphi\exp\left[i\int[\mathcal{L}_{0}+J\varphi]d^{4}x\right]
\end{equation}
where
\begin{equation}%\label{eq:}
\mathcal{L}_{0} =
\frac{1}{2}\partial^{\mu}\varphi\partial_{\mu}\varphi - \frac{1}{2}m^{2}\varphi^{2}
\end{equation}
is the Lagrangian density and
\begin{equation}%\label{eq:}
\mathcal{D}\varphi\propto
\prod_{\mathclap{\substack{
\text{space-}\\
\text{time}\\
\text{points $x$}}}}
d\varphi(x)
\end{equation}
is the functional measure.

\subsubsection{Remark on Peskin and Schroeder's ``Derivation''.}
In Chapter 9 of Peskin and Schroeder~\cite{Peskin:1995ev} there
is a ``derivation'' of the functional quantization of the scalar
field using the limit of a discrete lattice. There is some
mathematically dubious aspects of this derivation, for one they
use a Fourier series when they should be using a discrete Fourier
transform. What difference does this make? Well, the intuition of
the discrete Fourier transform is that the field exists only on
the nodes of the lattice; the intuition of a Fourier series is
that it exists everywhere continuously in a finite universe. The
latter is not in spirit with taking the continuum limit! In fact,
this beautifully motivates us to read the paper by Wise on chain field
theory~\cite{Wise:2005mp}, which resolves this problem by a
topological approach which yields the correct continuum limit by
taking advantage of recent work on discrete differential geometry.

\subsection{Slick Method of Computing Correlation Functions}
%%
%% functionalScalarFieldSlick.tex
%% 
%% Made by Alex Nelson
%% Login   <alex@tomato>
%% 
%% Started on  Sat Aug  1 12:23:40 2009 Alex Nelson
%% Last update Sat Aug  1 12:23:40 2009 Alex Nelson
%%
\subsubsection{Outline}
As promised, we'll introduce a slicker way to compute Feynman
rules using functional derivatives. It's a lot more
mathematically rigorous (and simpler) than the discretization
scheme. The method uses a mathematical gadget which generalizes
the notion of a generating function --- the generating
\emph{functional}. Recall the generating function is used to
compute constants and other useful numbers by taking its
$n^{\text{th}}$ derivative and evaluating it at 0. We
\emph{functionally} do the same thing, take the functional
derivative of the generating functional and evaluate it at 0.


\subsubsection{Properties of the Functional Derivative}
First lets try to review the properties of the functional
$\delta/\delta J(x)$. The functional derivative obeys the basic
property (in four dimensions)
\begin{equation}%\label{eq:}
\frac{\delta}{\delta J(x)}J(y)=\delta^{(4)}(x-y),
\quad\text{or}\quad
\frac{\delta}{\delta J(x)}\int J(y)\phi(y)d^{4}y = \phi(x).
\end{equation}
This can be viewed as a continuous generalization of the vector
calculus derivative 
\begin{equation}%\label{eq:}
\frac{\partial}{\partial x_i}x_j = \delta_{ij}
\quad\text{or}\quad
\frac{\partial}{\partial x_i}\sum_{j}x_{j}k_{j}=k_{i}.
\end{equation}
To take the functional derivatives of more complicated
situations, we use the basic properties of the chain rule and the
product rule. \textbf{Warning:} we \textbf{assume without proof}
that these properties hold, we'll not divulge into the proof
here. So we have situations like the following:
\begin{equation}%\label{eq:}
\frac{\delta}{\delta J(x)}\exp\left[i\int J(y)\phi(y)d^{4}y\right]
=i\phi(x)\exp\left[i\int J(y)\phi(y)d^{4}y\right].
\end{equation}
When the functional depends on the derivative of $J$ we integrate
by parts --- and for all practical purposes, we always can
integrate by parts in quantum field theory --- then apply the
functional derivative as follows:
\begin{equation}%\label{eq:}
\frac{\delta}{\delta J(x)}\int V^{\mu}(y)\partial_{\mu}J(y)d^{4}y
=
\frac{\delta}{\delta J(x)}\left(
\operatorname{bdry terms} + 
\int J(y)\partial_{\mu}V^{\mu}(y)d^{4}y
\right)
= - \partial_{\mu}V^{\mu}(x).
\end{equation}
This concludes our review of basic properties that we'll use
later on.

\subsubsection{Generating Functional}

As alluded to earlier, the basic object of interest is the
generating functional of correlation functions. We denote this
object of interest by $Z[J]$. In a scalar field theory, it's
defined as
\begin{equation}\label{eq:scalarFieldTheoryGeneratingFunctional}
Z[J] \stackrel{\text{def}}{=}\int\mathcal{D}\phi\exp\left[
i\int[\mathcal{L}+J(x)\phi(x)]d^{4}x
\right].
\end{equation}
This is a functional integral over $\phi$. We've merely added to
$\mathcal{L}$ in the exponent an extra term $J(x)\phi(x)$, which
we usually refer to as the ``\emph{source term}''.

\subsubsection{Derivation of Correlation Function}
Now we use it to compute the generating functions for the
Klein-Gordon field (the free scalar field). For example the
two-point function is
\begin{equation}%\label{eq:}
\<0|T\{\phi(x_1)\phi(x_2)\}|0\> = \left.\frac{1}{Z_0}
\left(-i\frac{\delta}{\delta J(x_{1})}\right)
\left(-i\frac{\delta}{\delta J(x_{2})}\right)
Z[J]\right|_{J=0}
\end{equation}
where $Z_0=Z[0]$. Each functional derivative brings down a factor
of $\phi$ in the numerator of $Z[J]$; setting $J=0$ we recover
our desired expression. In more explicit detail, we can compute
\begin{subequations}
\begin{align}
\frac{\delta}{\delta J(x_2)}Z[J] &= \frac{\delta}{\delta J(x_2)}
\int\mathcal{D}\phi\exp\left[
i\int[\mathcal{L}+J(x)\phi(x)]d^{4}x
\right] \\
&= \int\mathcal{D}\phi\frac{\delta}{\delta J(x_2)}
\exp\left[
i\int[\mathcal{L}+J(x)\phi(x)]d^{4}x
\right] \\
&= \int\mathcal{D}\phi i\phi(x_2)
\exp\left[
i\int[\mathcal{L}+J(x)\phi(x)]d^{4}x
\right]
\end{align}
\end{subequations}
This is the effect of one functional derivative, we multiply by
$-i$ to finish one functional operation. We need to do another to
get the expression
\begin{equation}%\label{eq:}
\left(-i\frac{\delta}{\delta J(x_{1})}\right)
\left(-i\frac{\delta}{\delta J(x_{2})}\right)
Z[J]
= \int\mathcal{D}\phi \phi(x_1)\phi(x_2)
\exp\left[
i\int[\mathcal{L}+J(x)\phi(x)]d^{4}x
\right].
\end{equation}
To get the final expression, we need to divide by $Z_0$ and
evaluate at $J=0$ to get
\begin{equation*}%\label{eq:}
\left.\frac{1}{Z_0}
\left(-i\frac{\delta}{\delta J(x_{1})}\right)
\left(-i\frac{\delta}{\delta J(x_{2})}\right)
Z[J]\right|_{J=0} =\displaystyle{
\frac{\displaystyle\int\mathcal{D}\phi\; \phi(x_1)\phi(x_2)\exp\left[i\int[\mathcal{L}]d^{4}x\right]}
{\displaystyle\int\mathcal{D}\phi\exp\left[i\int\mathcal{L}d^{4}x\right]}}
\end{equation*}
which is precisely what is expected.

\subsubsection{Slicker Way to Compute Correlation Functions}
We've seen that eq
\eqref{eq:scalarFieldTheoryGeneratingFunctional} recovers the
expected expression for two-point functions. It's pretty nifty
for us since the free scalar field can be written fairly
easily. It's explicitly written after integrating by parts (on
the first term)
\begin{equation}\label{eq:slickPartialIntegration}
\int[\mathcal{L}_{0}(\phi)+J\phi]d^{4}x 
= 
\int[\frac{1}{2}\phi(-\partial^{2}-m^{2}+i\varepsilon)\phi+J\phi]d^{4}x.
\end{equation}
The factor of $i\varepsilon$ is to guarantee convergence. We
complete the square by introducing a shifted scalar field
\begin{equation}%\label{eq:}
\phi'(x)\stackrel{\text{def}}{=}\phi(x)-i\int D_{F}(x-y)J(y)d^{4}y
\end{equation}
where $D_{F}(x-y)$ is the Feynman propagator --- i.e. the Green's
function of the Klein-Gordon operator, we find that our original
expression \eqref{eq:slickPartialIntegration} becomes
\begin{equation}%\label{eq:}
\begin{split}
\int[&\mathcal{L}_{0}(\phi)+J\phi]d^{4}x =\\
&\int[\frac{1}{2}\phi'(-\partial^{2}-m^{2}+i\varepsilon)\phi']d^{4}x
-\int\frac{1}{2}J(x)\left[-iD_{F}(x-y)\right]J(y)d^{4}y.
\end{split}
\end{equation}
More symbolically, we could rewrite the change of variables as
\begin{equation}%\label{eq:}
\phi'\stackrel{\text{def}}{=}\phi+(-\partial^{2}-m^{2}+i\varepsilon)^{-1}J,
\end{equation}
and the result becomes
\begin{equation}%\label{eq:}
\int[\mathcal{L}_0 + J\phi]d^{4}x = 
\int\left[\frac{1}{2}\phi'(-\partial^{2}-m^{2}+i\varepsilon)\phi'-\frac{1}{2}J(-\partial^{2}-m^{2}+i\varepsilon)^{-1}J\right]d^{4}x.
\end{equation}
This looks nasty, but we are not done yet. We have a few tricks
left.

When we change variables from $\phi$ to $\phi'$, it's just a
shift, so the Jacobian in the functional integral definition of
$Z[J]$ is the identity. The result is
\begin{equation}%\label{eq:}
\underbrace{\int\mathcal{D}\phi'
\exp\left[i\int\mathcal{L}_{0}(\phi')d^{4}x\right]}_{Z_{0}}
\underbrace{\exp\left[-i\int\frac{1}{2}J(x)[-iD_{F}(x-y)J(y)]d^{4}xd^{4}y\right]}_{\text{independent of }\phi'}
\end{equation}
As noted, the second integral is independent of $\phi'$ and the
first is precisely $Z_{0}$. The generating functional of the free
scalar field is thus
\begin{equation}\label{eq:generatingFunctionPostManip}
Z[J] = Z_{0}\exp\left[\frac{-1}{2}\int J(x)D_{F}(x-y)J(y)d^{4}xd^{4}y\right]
\end{equation}
Lets use \eqref{eq:scalarFieldTheoryGeneratingFunctional} and
\eqref{eq:generatingFunctionPostManip} to compute some
correlation functions. The two-point function is by definition
\begin{equation}%\label{eq:}
\<0|T\{\phi(x_1)\phi(x_2)\}|0\>
=\left.
\frac{-\delta}{\delta J(x_{1})}
\frac{\delta}{\delta J(x_{2})}
\exp\left[\frac{-1}{2}\int J(x)D_{F}(x-y)J(y)d^{4}xd^{4}y\right]\right|_{J=0}
\end{equation}
We can evaluate one functional derivative to find
\begin{equation}%\label{eq:}
\<0|T\{\phi(x_1)\phi(x_2)\}|0\>
=\left.
\frac{-\delta}{\delta J(x_{1})}
\left[\frac{-1}{2}\int D_{F}(x_{2}-y)J(y)d^{4}y 
-\frac{1}{2}\int J(x)D_{F}(x-x_{2})d^{4}x
\right]
\frac{Z[J]}{Z_{0}}\right|_{J=0}
\end{equation}
\begin{comment}
We can see that the negatives cancel out nicely, we end up with 2
terms that effectively look like
\begin{equation*}%\label{eq:}
\frac{-\delta}{\delta J(x_{1})}
\left[\frac{-1}{2}\int D_{F}(x_{2}-y)J(y)d^{4}y 
-\frac{1}{2}\int J(x)D_{F}(x-x_{2})d^{4}x
\right]
\propto
\frac{\delta}{\delta J(x_1)}\int D_{F}(x_2-y)J(y)d^{4}y
\end{equation*}
\end{comment}
We can see that this is just what happens after evaluating one
functional derivative of the definition of the two-point
function, which allows us to conclude that
\begin{equation}%\label{eq:}
\<0|T\{\phi(x_1)\phi(x_2)\}|0\> = D_{F}(x_1-x_2).
\end{equation}
This is good because it connects back to what we should already know.

\subsubsection{Example: Four Point Correlation Functions}

This is a rather space-consuming computation, so we need to
introduce some abbreviations. Namely we'll use the conventions
that arguments of functions are subscripts: $\phi_1 = \phi(x_1)$,
$J_x=J(x)$, $D_{x4}=D_{F}(x-x_{4})$, etc. Repeated subscripts
will be integrated over implicitly (like a continuous Einstein's
summation convention). The four-point function is thus
\begin{subequations}
\begin{align}
\<0|T\{\phi_1\phi_2\phi_3\phi_4\}|0\> 
&=\left. \frac{\delta}{\delta J_{1}}
\frac{\delta}{\delta J_{2}}
\frac{\delta}{\delta J_{3}}
[-J_{x}D_{x4}]e^{-\frac{1}{2}J_{x}D_{xy}J_{y}}\right|_{J=0}\\
&=\left. \frac{\delta}{\delta J_{1}}
\frac{\delta}{\delta J_{2}}
[-D_{34}+J_{x}D_{x4}J_{y}D_{y3}]e^{-\frac{1}{2}J_{x}D_{xy}J_{y}}\right|_{J=0}\\
=&\left. \frac{\delta}{\delta J_{1}}
[D_{34}J_{x}D_{x2}+D_{24}J_{y}D_{y3}+J_{x}D_{x4}J_{y}D_{23}]e^{-\frac{1}{2}J_{x}D_{xy}J_{y}}\right|_{J=0}\\
&=D_{34}D_{12}+D_{24}D_{13}+D_{14}D_{23}
\end{align}
\end{subequations}
which is precisely what we expect by Wick's theorem.

\subsubsection{The Beauty of the Generating Functional}

The beauty of the situation is that these calculations are
completely independent of whether things are free or
interacting. The catch is the $Z[J=0]$ factor is not trivial in
the interacting situation. It gives us the sum of the vacuum
diagrams.

