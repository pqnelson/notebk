%%
%% lecture18.tex
%% 
%% Made by alex
%% Login   <alex@tomato>
%% 
%% Started on  Tue Dec 27 20:36:56 2011 alex
%% Last update Tue Dec 27 20:36:56 2011 alex
%%

\begin{prop}\index{Covering!Uniqueness of Simply Connected ---}
Simply connected covering is unique.
\end{prop}
The ``unique'' means that two simply connected coverings are the
same. Nothing is said of their \emph{existence!}

\medbreak
\noindent\emph{Proof.}\quad\ignorespaces%
Let $p\colon\widetilde{X}\to X$ be a simply connected
covering. So $X$ is connected. We may declare any point in
$\widetilde{X}$ to correspond to the marked point in $X$, i.e.,
\begin{equation}
p(*)=*.
\end{equation}
Now we take $x\in\widetilde{X}$ so $p(x)\in X$.
We will take a path from $*$ to $x$ in $\widetilde{X}$,
\begin{equation}
\alpha\colon[0,1]\to\widetilde{X}
\end{equation}
such that $\alpha(0)=*$, $\alpha(1)=x$. Now this path may be
projected to $X$. Every path is homotopic in $\widetilde{X}$,
thus when projected to $X$ they remain homotopic. So we have
\begin{equation}
x\in\widetilde{X}\rightleftharpoons\begin{pmatrix}
\begin{matrix}p(x)\in X,
\end{matrix}
&
\begin{matrix}
\mbox{homotopy class of}\\
\mbox{paths from $*$ to $p(x)$}
\end{matrix}
\end{pmatrix}
\end{equation}
If we are given only $X$, we may construct
\begin{equation}
\widetilde{X}=(z\in X,\mbox{homotopy class of paths from $*$ to $z$}).
\end{equation}
But this means that $\widetilde{X}$ is basically unique, so there
exists a unique simply connected covering. Moreover this appears
to be a construction that always gives something we call a
``simply connected covering''---but this is wrong. It gives a
set. We need to prove it is a topological space, and that it is a
covering. Lets suppose a simply connected covering exists. If
$V\ni *$ is a neighborhood, then it is covered by disjoint sets
in $\widetilde{X}$
\begin{equation}
(U_1\sqcup U_2\sqcup\dots)=p^{-1}V,
\end{equation}
every map $U_i\to V$ is a homeomorphism. But then let us take any
closed path in $V$. It is a small path, it's covered by some path
$\widetilde{\alpha}\in U_i$ such that
\begin{equation}
p(\widetilde{\alpha})=\alpha.
\end{equation}
But this covering path $\widetilde{\alpha}$ is contractible in
$\widetilde{X}$. It follows that $\alpha$ is contractible in
$X$. This isn't always true, there is a pathological
counter-example.


\begin{wrapfigure}{l}{1in}
  \vspace{-12pt}
  \centering
  \includegraphics{img/lecture18.0}
\end{wrapfigure}
\noindent\ignorespaces %
A neighborhood $U$ of a pair $(z,\mbox{path})$. Let $V$ be a
neighborhood of $z$. Then $U$ has
\begin{equation}
(z',\mbox{path}')\in U
\end{equation}
where $z'\in V$. We have
\begin{equation}
\mbox{path}'=\mbox{path}*\gamma
\end{equation}
where $\gamma\colon[0,1]\to V$ has
\begin{equation}
\gamma(0)=z',\quad\gamma(1)=z.
\end{equation}
This gives us a topology, and it is simply connected, etc. What
should be required is really something a little bit stronger:

\medbreak
\noindent\textbf{Requirement:\quad}for every point $z\in X$ and
every neighborhood $V$ of $z$, there exists a smaller
neighborhood $U\propersubset V$ such that every closed path in
$U$ is contractible in $V$.

\medbreak
\noindent\ignorespaces %
On $\widetilde{X}$ we can define an action of $\pi_{1}(X,*)$ such
that (1) this action is free, (2)
$X=\widetilde{X}/\pi_{1}(X,*)$. How to prove this?

We lift everything we did in the opposite order. Remember
\begin{equation}
\widetilde{X}=(z\in X,\mbox{path from $*$ to $z$ in $X$})
\end{equation}
We see
\begin{equation}
\beta\colon I\to X
\end{equation}
with $\beta(0)=\beta(1)=*$, which specifies an element of
$\pi_{1}(X)$. Further, $\alpha$ is a path from $*$ to $z$. We
concatenate $\beta*\alpha$, we changed the path though not the
end points. We remain in the same fibre. Thus it's an action of $\pi_{1}(X)$.
So it justifies the statement $X=\widetilde{X}/\pi_{1}(X)$.

So we've shown: a simply connected covering is unique, and it is
a regular covering. But what about other coverings? The simply
connected covering is universal. It is easy to see without formal
proofs. Let us take any connected covering of $X$, denote it by
$Y$. But it also has a simply connected covering $\widetilde{Y}$
of $Y$. But now the covering of a covering is a covering. So it
is the simply connected covering of $X$, i.e.,
\begin{equation}
\widetilde{X}=\widetilde{Y}.
\end{equation}
We may therfore say
\begin{equation}
Y=\widetilde{X}/\pi_{1}(Y)
\end{equation}
and that's the end of the story.

We should check the simply connected covering exists. But we may
prov ethis. Instead, we'll do something different. Namely, look:
we take some covering $Y$ and we project
\begin{equation}
\pi\colon Y\to X.
\end{equation}
But from this construction it follows we may construct a map
$\widetilde{X}\to Y$ by definition of $\widetilde{X}$. It is
almost obvious this map is a covering.

%%%%%%%%%%%%%%%%%%%%%%%%%%%%%%%%%%%%%%%%%%%%%%%%%%%%%%%%%%%%%%%%%%%%%%%%%%%
\exercises
\begin{xca}\index{Mobius Band!Connected Coverings of ---}
Describe all connected coverings of Moebius band. Show that all
compact covering spaces are homeomorphic either to Moebius band
or to annulus.
\end{xca}
\begin{xca}
Let us consider a group $G$ of transformations of the plane
$\RR^2$ generated by transformations $(x, y)\to(x, y + 2)$ and
$(x, y)\to(x + 1, -y + 1)$. 
\begin{enumerate}
\item Prove that $G$ acts freely on $\RR^2$.
\item Prove that the quotient space $\RR^2/G$ is homeomorphic to
Klein bottle (i.e.\ it can be be obtained from two Moebius bands
by means of identifying their boundaries).\index{Klein Bottle!Homeomorphic to $\RR^2/G$}
\item Use this construction to show that there exists a covering
of Klein bottle homeomorphic to a torus.\index{Klein Bottle!Torus as Covering of ---}
\end{enumerate}
\end{xca}
\begin{xca}
One says that a map $p\colon X\to Y$ is a ramified covering\index{Ramified Covering}\index{Covering!Ramified} of the surface $Y$ if the map $p$ is a covering of the set $Y\setminus F$ where $F$ is a finite subset of $Y$ (i.e.\ the map $p\colon p^{-1}(Y)\to Y$ is a covering. Let us suppose that $p$ is $n$-sheeted covering of $Y\setminus F$ (i.e.\ every point of $Y\setminus F$ has $n$ preimages) and the ramification points (points of $F$) have $\mu_{1}$, \dots, $\mu_{k}$ preimages (here $k$ stands for the number of points in $F$). Prove the following formula connecting Euler characteristics of $X$ and $Y$:
\begin{equation}
\chi(X)=n\chi(Y)+\sum_{1\leq i\leq k}(\mu_{i}-n)
\end{equation}
(this is a version of Riemann-Hurwitz formula\index{Riemann-Hurwitz Formula}).

Hint. You can use additivity of Euler characteristic.
\end{xca}
\begin{xca}
Check the Riemann-Hurwitz formula for the map $p\colon S^2\to
S^2$ defined by the formula $p(z)=z^n$ (we consider the sphere as
the set of complex numbers with addition of a point at infinity:
$S^2 = \CC\cup\infty$). 
\end{xca}
\begin{xca}
Calculate the Euler characteristic of a surface defined by the
equation $w^2=q(z)$ where $q(z)$ is a polynomial of degree
$m$. (Such a surface is called elliptic if $m=3$ or $m=4$ and
hyperelliptic if $m>4$). Here $w, z \in S^2 = \CC\cup\infty$.

Hint. Consider the surface as a ramified covering of the
sphere. Take into account that for $m=2n$ the function
$w=\pm\sqrt{q(z)}$ looks like $w=\pm{}z^n$ at infinity and
therefore consists of two separate branches; this means that in
the neighborhood of infinity we have two-sheeted covering. For
$m=2n+1$ this function looks like $w=\pm z^{n}\sqrt{z}$, hence we
have a ramification point at infinity.
\end{xca}

