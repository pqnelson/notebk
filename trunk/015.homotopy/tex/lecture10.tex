%%
%% lecture10.tex
%% 
%% Made by alex
%% Login   <alex@tomato>
%% 
%% Started on  Mon Dec 26 21:20:24 2011 alex
%% Last update Mon Dec 26 21:20:24 2011 alex
%%

We considered a space with a marked point $(X,x_0)$ and we
constructed the loop space $\Omega(X,x_0)$ with loops that start
and end at $x_0\in X$. We reviewed two constructions which are
homotopically equivalent. We also considered a binary operation
called concatenation which is intuitively ``multiplication''. We
can use this multiplication to define the group
\begin{equation}
\pi_{1}(X,x_0)=\pi_{0}\bigl(\Omega(X,x_0)\bigr)
\end{equation}
which is just the set of components of the loop space. We proved
this multiplication is associative, the unit element exists, and
inversion exists. Well, we did not ``prove'' it, but it is
obvious. We see
\begin{equation}
f^{-1}(t)=f(1-t)
\end{equation}
and that $e(t)\homotopic f^{-1}*f$ homotopic.

The fundamental group is not, strictly speaking, an invariant of
a topological space. It may depend on the marked point; the
question is, can we (if we change the marked point) say
\begin{equation}
\pi_{1}(X,x_0)=\pi_{1}(X,\widetilde{x}_{0})?
\end{equation}
If $X$ is pathwise-connected, yes. If $X$ is disconnected, we
have no chance. We will give two proofs of this. One is really
short and (perhaps) the better, while the other is more
pedestrian. 

\begin{proof}
We have a connected space $X$, and a path
\begin{equation}
h\colon[0,1]\to X
\end{equation}
where $h(0)=x_0$ and $h(1)=x_1$. What we can do, we can say we
have a deformation from one marked point into the other. We have
the homotopy extension property\index{Homotopy Extension Property}.
So if we have a deformation of a subset, we may extend it to a
deformation of the \emph{whole} space. So 
\begin{equation}
(X,x_0)\homotopic(X,x_1)
\end{equation}
homotopic, and we may go in the other direction by using
$h^{-1}(t)$. It is very easy to see these maps are homotopically
equivalence. Then everything is fine. 
\end{proof}
This proof has the disadvantage that the homotopy extension
principles holds \emph{almost} for every space. But if $X$ is not
among them, this proof does not hold. 
\begin{proof}
We will give another proof that is perhaps a bit longer. Look we
have these two points $x_0$, $x_1$ and a connecting path. We
consider a path in $\Omega(X,x_0)$ but we want to get a path that
lives in $\Omega(X,x_1)$. If $h\colon[0,1]\to X$ such that
$h(0)=x_0$ and $h(1)=x_1$, then for any $f\in\Omega(X,x_0)$ we
induce a map $h*f*h^{-1}\in\Omega(X,x_1)$. Thus if we are working
with connected spaces, the fundamental group does not depend on
the marked point.
\end{proof}
\begin{rmk}
Is this true? Not really, because we cannot say that this
isomorphism
\begin{equation}
\pi_{1}(X,x_0)\iso\pi_{1}(X,x_1)
\end{equation}
is canonical. The isomorphism depends on the choice of the path
connecting marked points. Why? Lets explain. These points $x_0$,
$x_1$ can coincide, why not? No one said they had to be
different! If they coincide, then the path connecting the marked
points is a loop. Then our formula
\begin{equation}
f\mapsto h*f*h^{-1}
\end{equation}
can be understood at the level of the fundamental group. At the
level of the fundamental group, this gives us a nontrivial inner
automorphism of $\pi_{1}(X,x_0)$. It is a canonical isomorphism
when the group $\pi_{1}(X)$ is Abelian.
\end{rmk}

When working with the fundamental group, we will neglect the
marked point. It is a little dangerous.

Now, let us consider $\homotopyClass(S^1,X)$ the classification
of maps from the circle to $X$. We assume $X$ is connected. We
see now that
\begin{equation}
\pi_1\colon (X,x_0)\to\homotopyClass(S^1,X)
\end{equation}
we get a group. So lets say something that definitely is quite
trivial, namely: we can say that the conjugacy classes of
$\pi_1(X,x_0)$ is also a map to $\homotopyClass(S^1,X)$. We are
really saying if two guys are conjugate
\begin{equation}
h^{-1}gh=f
\end{equation}
then they are mapped to the samje homotopy class. We can identify
maps $S^1\to X$ with conjugacy classes in the fundamental group.

But now we would like to calculate the fundamental group. Two
fundamental groups are quite obvious. The first
is\index{Fundamental Group!of Circle}
\begin{equation}
\pi_{1}(S^1)=\ZZ.
\end{equation}
This is because we know
\begin{equation}
\homotopyClass(S^1,S^1)\to\ZZ
\end{equation}
is a one-to-one correspondence. 

The other fundamental group\index{Fundamental Group!of $n$-Sphere} 
that we know is
\begin{equation}
\pi_{1}(S^n)=0
\end{equation}
for $n>1$. It is the \emph{trivial} group!

There are (at least) two ways to compute the fundamental
group. One is by the van Kampen Theorem\index{van Kampen Theorem}.
We represent
\begin{equation}
X=A\cup B
\end{equation}
and we will assume $A$, $B$, and $A\cap B$ are connected; we also
assume $A$ and $B$ are open. (Pop quiz: is $A\cap B$ open?) We
will consider the simplest case when $A\cap B$ is \emph{simply}
connected. (Here \define{Simply Connected}\index{Connectedness!Simply}\index{Simply Connected|textbf}
means connected and the fundamental group is trivial.)
Without loss of generality, we will say that our marked point
$\star\in A\cap B$. 
The typical case is when considering the wedge sum of two
circles, the intersection is a single point, but then we work
with closed sets. Well, okay, but that doesn't matter! In this
case,
\begin{equation}
\pi_{1}(X,\star)=\mbox{free product of }\pi_{1}(A,\star)\mbox{ and }\pi_{1}(B,\star)
\end{equation}
The free product\index{Group!Free Product}\index{Free Product!of Groups}\index{Product!Free}
is a general notion in group theory. Let $G_1$, $G_2$ be
groups. Their free product can be deduced in terms of generators
$a_1,\dots,a_m$ of $G_1$ and $b_1,\dots,b_n$ of $G_2$; we have
some relations on $G_1$ as well as some on $G_2$. Then we combine
generators and relations. This isn't a very good definition,
since it depends on the choice of generators and relations.

We could define it in a slightly different way. Consider
\begin{equation*}
\hom(G_1*G_2,G)
\end{equation*}
where $G_1*G_2$ is the free product. We say
\begin{equation}
\hom(G_1*G_2,G)=\hom(G_1,G)\times\hom(G_2,G)
\end{equation}
which doesn't explicitly depend on the generators and relations
of $G_1$ and $G_2$. We have a pair of homomorphisms that act on
elements of $G_1$ and $G_2$ respectively. Now we'd like to
explain why we have this. 



\exercises
\begin{xca}
Let $f$ denote a differentiable map of a circle into another
circle. In angular coordinates the map $f$ can be represented by
a multivalued function, but its derivative $f(\alpha)$ is
well-defined. Prove that the degree of the map can be represented
as an integral 
\begin{equation}
\deg(f)=\frac{1}{2\pi}\int^{2\pi}_{0}f'(\alpha)\D\alpha.
\end{equation}
\end{xca}
\begin{xca}
Prove that a polynomial map $p\colon\RR\to\RR$ where $p$ is a
polynomial of degree $n$ can be extended by continuity to a map
of circles. (We use the fact that adding to $\RR$ one point at
infinity we get a circle). Calculate the degree of this map of
circles. 
\end{xca}
