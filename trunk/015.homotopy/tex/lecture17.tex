%%
%% lecture17.tex
%% 
%% Made by alex
%% Login   <alex@tomato>
%% 
%% Started on  Thu Dec 29 11:16:54 2011 alex
%% Last update Thu Dec 29 11:16:54 2011 alex
%%

We introduced a notion of fibration. If we have a map $p\colon
E\to B$ which is surjective $p(E)=B$, we have fibres 
\begin{equation}
F_{b}=f^{-1}\{b\}
\end{equation}
for $b\in B$. If $F_{b}\iso F_{b'}\iso F$, where $F$ is ``some
topological space'', then we have a fibration. The space $F$ is
called the typical fibre.

There are very nice fibrations called \define{Trivial Fibrations}
where $E=F\times B$ where all the fibres are canonically
homeomorphic.

A mapping of fibrations $\varphi\colon E\to E'$ is such that
such that
\begin{equation}
\varphi\colon F_{b}\to F'_{\varphi(b)}
\end{equation}
We also demand that, if $p\colon E\to B$ and $p'\colon E'\to B'$,
then 
\begin{equation}
p'\circ\varphi=p
\end{equation}
holds.

A \define{Locally Trivial Fibration} is precisely a fibre
bundle. We take a fibration $p\colon E\to B$ and require for
$U\propersubset B$ open that
\begin{equation}
p^{-1}(U)=U\times F.
\end{equation}
Another way to think of it is as pasted together from direct
products. We have a cover $\{U_{\alpha}\}$ of $B$, we consider
$U_{\alpha}\times F$ pasted together, i.e., $(U_{\alpha}\cap
U_{\beta})\times F\propersubset U_{\alpha}\times F$ and
$(U_{\alpha}\cap
U_{\beta})\times F\propersubset U_{\beta}\times F$, the
intersection is embedded in both. We should have a map that
identifies the overlap
\begin{equation}
(b,f)\sim\bigl(b,\varphi_{\alpha\beta}(b)f\bigr)
\end{equation}
where
\begin{equation}
\varphi_{\alpha\beta}(b)\colon F\to F
\end{equation}
which is defined for all $b\in U_{\alpha}\cap U_{\beta}$. We can
introduce different notation
\begin{equation}
\widetilde{\varphi}_{\alpha\beta}(b,f) =
(b,\varphi_{\alpha\beta}(b)f).
\end{equation}
These guys are called the \define{Transition Functions}\index{Transition Functions|textbf}
which obeys some properties of compatibility.

We may consider the case when the fibre is discrete. Then a
locally trivial fibration is a covering\index{Covering!as a Fibration}.
This is a definition. We have $p\colon E\to B$, consider
\begin{equation}
p^{-1}(U)=\begin{pmatrix}
\mbox{disjoint union}\\
\mbox{of homeomorphic}\\
\mbox{components}
\end{pmatrix}
\end{equation}
where $U\propersubset B$ is ``small.'' But $p^{-1}(U)=U\times F$,
but $F$ is a discrete space where every point is open. 

\begin{ex}
Let $G$ be a discrete group that acts freely on $X$. We have the
identification map
\begin{equation}
\pi\colon X\to X/G
\end{equation}
and this is a covering. This is called a \define{Regular Covering}\index{Covering!Regular}\index{Regular Covering}
\end{ex}
Remember we assumed $X$ is simply connected, so by definition it
is a regular covering --- but it has one more name: a
\define{Universal Covering}\index{Covering!Universal}\index{Universal Covering}.
Every connected covering may be obtained from the universal
one. We will focus exclusively on connected coverings for the rest
of this lecture.

Why is this universal? If we have a (regular) covering, we may
obtain other coverings in the following way: let 
\begin{equation}
G=\pi_{1}(X/G)=\pi_{1}(X^{1})
\end{equation}
We have $X^{1}=X/G$ be connected. We may take \emph{any}
subgroup $H\propersubset G$ which acts freely on $X$. Therefore
we may repeat the same construction, obtaining a map
\begin{equation}
X\xrightarrow{\alpha_{H}}(X/H)\xrightarrow{\pi_{H}}(X/G).
\end{equation}
The first remark is that $\alpha_h$, $\pi_H$ both are coverings,
moreover $\alpha_H$ is a regular covering. We take some
$V\propersubset X/G$. We have its preimage
\begin{equation}
\pi^{-1}(V)=\sqcup U_{i} = \sqcup g_{i}U
\end{equation}
where $g_i\in G$. This induces an open covering. When we consider
$\pi^{-1}_{H}(V)=\sqcup U_{i}$, we recall
$\pi^{-1}_{H}(V)\propersubset\pi^{-1}(V)$ is a proper subset. All
connected coverings of $X/H$ may be obtained in this way.

One more remark. Is this quotient $X/H$ always a regular
covering? Not always. But it is clear if $H\propersubset G$ is a
normal subgroup, then $\pi^{-1}_{H}$ is a regular
covering. Because $(G/H)$ acts on $(X/H)$. The points of $(X/H)$
are orbits $Hx$; and if $H\propersubset G$ is normal, and
$\gamma\in G/H$, then it acts on $(X/H)$ by taking $g\in\gamma$
and applying it to $hx$. We may represent it as
\begin{equation}
ghx=hgx
\end{equation}
so $ghg^{-1}=h'\in H$ by virtue of $H$ being a normal subgroup.

\begin{thm}
Connected coverings of a ``good'' space $Y$ are in one-to-one
correspondence with subgroups of $\pi_{1}(Y)$. Connected regular
coverings correspond to normal subgroups of $\pi_{1}(Y)$.
\end{thm}

A ``good'' space is one with at least one universal covering, but
in reality we need less. The idea is that it is a maximal
covering --- so a good space has small closed loops be
contractible. It is not precisely clear where they are
contractible. More precisely every point has a neighborhood such
that closed loops are contractible in a bigger neighborhood.

