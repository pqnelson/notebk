%%
%% lecture16.tex
%% 
%% Made by alex
%% Login   <alex@tomato>
%% 
%% Started on  Mon Dec 26 21:28:28 2011 alex
%% Last update Mon Dec 26 21:28:28 2011 alex
%%

I missed this lecture, but Steenrod's book~\cite{steenrod} is the
standard reference for fibre bundles. So instead, I will review
the topological notion of fibre bundles. 

In a sense it is a straightforward generalization of what we
have: a group $G$ acting freely on a topological space $X$. Then
we have a surjective continuous map 
\begin{equation}
p\colon X\to X/G. 
\end{equation}
We call the space $X/G$ the \define{Base Space}\index{Base Space}\index{Space!Base}, and note that
for any $x\in X/G$ we have
\begin{equation}
p^{-1}\{x\}\iso G
\end{equation}
be a homeomorphism. 

How do we generalize this? Well, the first steps is to let the
base space $B$ be any topological space. We have
the \define{Total Space}\index{Total Space}\index{Space!Total}
be a topological space $E$. A \define{Fibration}\index{Fibration|textbf}
is then a surjective continuous map
\begin{equation}
p\colon E\to B
\end{equation}
with the extra condition that, for any $b,b'\in B$, we have
$F_{b}=p^{-1}\{b\}$ and $F_{b'}=p^{-1}\{b'\}$ be the preimages
such that
\begin{equation}
F_{b}\iso F_{b'}
\end{equation}
are homeomorphic. 

\begin{ex}[Trivial Fibration]
Example number zero is quite simple. Let $F$ and $B$ be
topological spaces, and
\begin{equation}
E=F\times B
\end{equation}
be the total space. It's just the product space. This is a basic
fibre bundle; in fact, the fibre bundle \emph{generalizes} the
notion of a product space. But as a fibration, we call it
the \define{Trivial Fibration}\index{Fibration!Trivial|textbf}
whenever the fibration is just the product space.
\end{ex}
\begin{ex}[Tangent Bundle]\index{Bundle!Tangent}\index{Tangent Bundle}
Consider a smooth $n$-dimensional manifold $M$, at each $x\in M$ we may consider
the vector space $T_{x}M$ of all tangent vectors with base point
$x$. We can construct the total space
\begin{equation}
TM = \bigsqcup_{x\in M}T_{x}M
\end{equation}
which is a fibration. Really? Well, we see that we have a
continuous surjective function
\begin{equation}
p\colon TM\to M
\end{equation}
which gives the base point of the tangent vector. What's the
fibre? Well,
\begin{equation}
p^{-1}\{x\}=T_{x}M\iso\RR^{n}
\end{equation}
describes the fibre: it is the tangent vector space. The fibre is
then just the vector space $\RR^n$, where $n=\dim(M)$.
\end{ex}

A \define{Fibre Bundle}\index{Fibre Bundle}\index{Bundle!Fibre}
is then a fibration $(F,E,B,p)$ which is \define{Locally
Trivial}\index{Locally Trivial Fibration}\index{Fibration!Locally Trivial}
in the sense that, $U\propersubset B$ open implies
\begin{equation}
p^{-1}(U)\iso U\times F
\end{equation}
Another way to think of it is as pasting together a bunch of
direct products.
%%%%%%%%%%%%%%%%%%%%%%%%%%%%%%%%%%%%%%%%%%%%%%%%%%%%%%%%%%%%%%%%%%
\exercises
\begin{xca}
Let $X\propersubset\RR^3$ be the union of $n$ lines through the
origin. Compute $\pi_{1}(\RR^3\setminus X)$.
\end{xca}
\begin{xca}
Let $X$ be the quotient space of $S^2$ obtained by identifying
the north and south pole into a single point. Put a cell complex
structure on $X$ and use this to compute $\pi_{1}(X)$.
\end{xca}
\begin{xca}
Compute the fundamental group of the space obtained from two tori
$S^{1}\times S^{1}$ by identifying a circle $S^{1}\times x_{0}$
in one torus with the corresponding cycle $S^{1}\times x_{0}$ in
another torus. 
\end{xca}
