%%
%% lecture07.tex
%% 
%% Made by alex
%% Login   <alex@tomato>
%% 
%% Started on  Mon Dec 26 21:18:08 2011 alex
%% Last update Mon Dec 26 21:18:08 2011 alex
%%

\begin{lem}[Homotopy Extension Property]
Let us suppose we have a pair $A\propersubset X$ closed and we
have a map
\begin{equation}
F\colon X\to Y.
\end{equation}
We can restrict our $F$ to $A$,
\begin{equation}
f=\left.F\right|_{A}
\end{equation}
Suppose we have a deformation $f_{t}\colon A\to Y$ such that
$f_{0}(a)=F(a)$ for all $a\in A$. What we would like to do is
extend, construct a family of maps
\begin{equation}
F_{t}\colon X\to Y
\end{equation}
such that on $A$, we have $f_{t}=\left.F_{t}\right|_{A}$. We want
to extend this deformation from $A$ to the whole space $X$. We
say that this pair $(X,A)$ has the \define{Homotopy Extension Property}.% 
\index{Homotopy Extension Property}\index{Homotopy!Extension Property}
\end{lem}
\begin{rmk}\index{HEP|see{Homotopy Extension Property}}
We often abbreviate ``Homotopy Extension Property'' as HEP.
\end{rmk}
\begin{lem}
If $A$ is a (closed) subcomplex of the cell complex $X$, then the
pair $(X,A)$ has the Homotopy Extension Property.
\end{lem}

\begin{wrapfigure}{r}{1.06in}
  \vspace{-20pt}
  \includegraphics{img/lecture7.0}
\end{wrapfigure}
\noindent\emph{Proof} (Particular Case).
Take $X$ to be an interval and $A$ consisting of boundary
points. We have a map a map of the interval, and we have a
deformation over $A$. The situation is doodled to the right,
where the top picture shows $X$ and in light gray $A$; the bottom
picture is $A\times[0,1]\cup X\times\{0\}$. The parameter $t$ of
$I$ is labeled as well.
We would like $f_{t}\colon A\to Y$, which is $I\times A\to A$ a
function intuitively taking the bottom diagram as the domain. We
may construct a retraction
\begin{equation}
\varphi\colon X\times I\to X\times\{0\}\cup A\times I
\end{equation}
which is precisely the dilapidated rectangle to the right. If we
have such a map, we can construct $F_{t}=f\circ\varphi$ where
$f=f_{t}$ on $A\times I$ and $f=F$ on $X\times\{0\}$. Now it is
completely clear that $F_{t}$ is an extension of $f_{t}$ and
that's it. If we have that retraction, we may apply 

\begin{wrapfigure}{l}{0.75in}
  \vspace{-18pt}
  \includegraphics{img/lecture7.1}
  \vspace{-36pt}
\end{wrapfigure}
\noindent this to every
case. Moreover, we may construct the deformation retraction.
This is very easy. The procedure is doodled on the left, where we
have $A\times I$ in light gray and the deformation retraction is
dashed. 

\medbreak
\noindent\textbf{Case:} Take $X$ to be a disc, multiply the
boundary by $I$ and we get the cylinder.
Again we should take the retraction. We repeat the procedure,
taking a point above the cylinder and consider lines from
$A\times I\cup D^{2}\times\{0\}$ to the point. We may continue to
generalize to higher dimension cases.

The general case we take $A\propersubset X$ and take a
$k$-skeleton $X^{k}$ and perform this induction on $k$. We can
assume we took this for $(A\cup X^{k-1})\propersubset(A\cup X^{k})$.
We extend from $(k-1)$-cells to $k$-cells, but this is precisely
what we have done.\hfill\qedsymbol\break

Remember we had $A\propersubset X$ and $A$ was contractible,
i.e., we had a retraction from $A$ to a point. We claim
$X\homotopic X/A$ homotopic. We prove this when $A$ is ``nice'',
i.e., the pair $(X,A)$ has the Homotopy Extension Property.
\begin{proof}
We have $\id{X}\colon X\to X$ and on $A$ we can deform this
map---$A$ is contractible, so we have $f_{t}\colon A\to A$ such
that $f_{0}(a)=a$ and $f_{1}(a)=a_{0}$. But this is precisely the
picture of the Homotopy Extension Principle, we have a
deformation of the whole space. This permits us to extend from
$f_{t}$ to $F_{t}\colon X\to X$ which has the property
$f_{t}=\left.F_{t}\right|_{A}$ and in particular $f_{1}=\left.F_{1}\right|_{A}$
which means $F_{1}(a)=a_{0}$ for $a\in A$. We have a map $X/A\to X$,
because all of $A$ went to one point. This is the main step of
the proof, we should prove it's a homotopy equivalence but we'll
skip it.
\end{proof}

Recall we stated
\begin{equation}
\homotopyClass(S^{1},S^{1})=\ZZ.
\end{equation}
We will prove it, but it won't be absolutely rigorous (we'll be
rigorous later in a more general setting). On a circle we have an
angular coordinate $\alpha\in[0,2\pi)$, or $\alpha\in\RR$ subject
to the equivalence $\alpha\simeq\alpha+2\pi$. Therefore if we
have
\begin{equation}
f\colon S^1\to S^1
\end{equation}
we have two possibilities. One is, e.g.,
\begin{equation}
f(\alpha)=5\alpha
\end{equation}
which is discontinuous at $2\pi k/5$ for $k=1,\dots,4$. What to
do? Well, we relax the map a little bit to be
\begin{equation}
f\colon[0,2\pi]\to\RR
\end{equation}
such that
\begin{equation}
f(2\pi)\equiv f(0)\bmod 2\pi.
\end{equation}
So in other words
\begin{equation}
f(2\pi) = f(0) + k2\pi
\end{equation}
for some $k\in\ZZ$; then this $f$ specifies a map of circles. But
that is obvious. What is less obvious is any map of circles may
be written in this way.

This $k\in\ZZ$ is called the \define{Degree}\index{Degree!of a Map}
\textbf{of a Map}. It is a triviality that
\begin{equation}
\deg(f)=\deg(g)\implies f\homotopic g
\end{equation}
homotopic. Maps of the same degree are homotopic. To see this
triviality, take
\begin{equation}
h_{t}(x)=tg(x)+(1-t)f(x)
\end{equation}
and since $\deg(g)=\deg(f)$ we see that $\deg(h)=\deg(f)=\deg(g)$
as well. There is no problem here. Let $k=\deg(f)$ and
$k'=\deg(g)$, then
\begin{equation}
\begin{split}
h_{t}(2\pi) &= t\bigl(k'2\pi+g(0)\bigr)+(1-t)\bigl(k2\pi+f(0)\bigr)\\
&= \bigl(tk'+(1-t)k\bigr)2\pi + h_{t}(0).
\end{split}
\end{equation}
But this works if and only if $k=k'$, otherwise we have an
integer varying continuously while remaining integral\dots and
then anything is possible! We have $f(\varphi)=k\varphi$ be a map
of degree $k$, so all maps of degree $k$ are homotopic to it.

\exercises
\begin{xca}
Classify first 15 letters of English alphabet up to topological
equivalence and up to homotopy equivalence.
(Consider capital letters only.)
\end{xca}
\begin{xca}
Find a letter that is homotopy equivalent to a torus with one deleted point.
\end{xca}
\begin{xca}\index{Mobius Band!and Surfaces}
Calculate Euler characteristic of a sphere with $g$ handles and $h$
Moebius bands attached.
\end{xca}
\begin{xca}
Calculate the Euler characteristic of a T-shirt. Find a graph that is
homotopy equivalent to a T-shirt.
\end{xca}
\begin{xca}
5. Calculate Euler characteristic of projective plane. (We define
projective plane as two-dimensional sphere
$\|x\|=1$ where the point $x$ is identified with the point $-x$.)
\end{xca}
