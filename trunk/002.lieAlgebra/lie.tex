\documentclass[final]{amsart}
\usepackage{fly,cancel}
\usepackage{amsmath,amscd}
\usepackage{sidecap}
\title{Notes on Lie Groups and Algebras}
\DeclareMathOperator{\lie}{Lie}
\date{July 27, 2008}
\begin{document}
\maketitle
\tableofcontents
\section{Introduction}

We follow an approach similar to Artin~\cite{ArtinBk01}.

\section{Tangent Vectors}
\begin{defn}
\marginpar{Algebraic Sets often taught as ``surfaces'' in calculus courses in universities}Let $S\subseteq\mathbb{R}^n$ be a set. If our set $S$ is the locus of zeros of one or more polynomial functions $f(x_1,\ldots,x_n)$, it is called a \textbf{real algebraic set}:
\begin{equation}
S = \{ x\in\mathbb{R}^n: f(x) = 0\}.
\end{equation}
\end{defn}

\begin{ex}
The set $S\subset\mathbb{R}^3$ defined by the zeros of the polynomial
\begin{equation}
f(x,y,z) = x^2 + y^2 + z^2 - 1
\end{equation}
defines a sphere of radius 1. QEF.
\end{ex}

\begin{ex}
The set $S\subset\mathbb{R}^3$ defined by the zeros of the polynomial
\begin{equation}
f(x,y,z) = \frac{x^2}{a^2} + \frac{y^2}{b^2} + \frac{z^2}{c^2} - 1
\end{equation}
(where $a,b,c\in\mathbb{R}^{+}$) defines an ellipsoid. QEF.
\end{ex}

\begin{lem}\label{lemma1}
Let $S$ be a real algebraic set in $\mathbb{R}^n$, defined as the locus of zeros of one or more polynomial functions $f(x)$. The tangent vectors to $S$ at $x$ are orthogonal to the gradients $f(x)$.
\end{lem}

\section{An Introduction to Infinitesimals}

\subsection{A Brief Aside on Complex Numbers}

Recall with complex numbers (the set $\mathbb{C}$) we are dealing with a new
``number''
\begin{equation}\label{root}
i^2 + 1 = 0\quad\Rightarrow\quad i = \sqrt{-1}
\end{equation}
and so we have a new number system with elements of the form
\begin{equation}
(x + iy)\in\mathbb{C}\quad\textrm{where }x,y\in\mathbb{R}.
\end{equation}
We can formally define multiplication using the polynomial multiplication:
\begin{equation}
(a + ib)(x + iy) = (ax - by) + i(ay + bx)
\end{equation}
which results in a complex number.

\subsection{And Now For Something Slightly Different}

Suppose instead of Eq (\ref{root}) we worked with
\begin{equation}
\varepsilon^2 = 0\quad\textrm{and}\quad\varepsilon\neq 0
\end{equation}
which means we have elements called \textbf{infinitesimals} of the form
\begin{equation*}
x + \varepsilon y.
\end{equation*}
We can formally define multiplication in the same way we did for complex 
numbers
\begin{equation}
(a + b\varepsilon)(x+y\varepsilon) = ax + (xb + ya)\varepsilon + \cancelto{0}{by\varepsilon^2}.
\end{equation}
Note that $\varepsilon^2=0$ is the defining characteristic of our infinitesimal element!

Observe what mayhem we may do with this. We have some favorite Taylor series
\begin{equation}
f(x + h) = \sum_{n=0}^{\infty}\frac{f^{(n)}(x)}{n!}h^n
\end{equation}
which becomes with infinitesimals
\begin{equation}\label{classicTaylor}
f(x + \varepsilon) = f(x) + f'(x)\varepsilon + \cancelto{0}{\mathcal{O}(\varepsilon^2)}.
\end{equation}
Similarly for
\begin{equation}
\frac{1}{1 + x} = \sum_{n=0} (-x)^n = 1 - x + x^2 - \ldots
\end{equation}
we get
\begin{equation}
\frac{1}{1 + \varepsilon} = 1 - \varepsilon.
\end{equation}
For the famous euler's constant
\begin{equation}
e^x = \sum_{n=0}\frac{x^n}{n!}
\end{equation}
we get
\begin{equation}
e^\varepsilon = 1 + \varepsilon + \cancelto{0}{\mathcal{O}(\varepsilon^2/2)}.
\end{equation}
For trigonometric functions, which can be obtained by Euler's identity
\begin{equation}
e^{ix} = \cos(x) + i\sin(x)
\end{equation}
we deduce both
\begin{equation}
\cos(\varepsilon) = 1 + \cancelto{0}{\mathcal{O}(\varepsilon^2)}
\end{equation}
and
\begin{equation}
\sin(\varepsilon) = \varepsilon + \cancelto{0}{\mathcal{O}(\varepsilon^3)}.
\end{equation}
Thus we find that in this instance
\begin{equation}
\tan(\varepsilon) = \frac{\sin(\varepsilon)}{\cos(\varepsilon)} = \varepsilon/1
\end{equation}
which is remarkable!


\section{Tangent Vectors and Infinitesimals Together}

Recall Lemma (\ref{lemma1}) that tangent vectors are orthogonal to $\nabla f(x)$.
Given a point $x$ of $\mathbb{R}^n$ and a vector $\vec{v}\in\mathbb{R}^n$, the
sum $x + \vec{v}\varepsilon$ is a vector with infinitesimal entries which we
think of intuitively as \emph{an infinitesimal change in $x$ in the direction of
$\vec{v}$.} Observe the Taylor expansion of $f(x + \vec{v}\varepsilon)$ yields by
Eq (\ref{classicTaylor})
\begin{equation}
f(x + \vec{v}\varepsilon) = f(x) + \varepsilon\vec{v}\cdot\nabla f(x)
\end{equation}
since higher order terms vanish by our use of infinitesimal generator $\varepsilon$.

We want to find tangent vectors to some point $x\in\mathbb{R}^n$, that is vectors such that
\begin{equation}
\vec{v}\cdot\nabla f(x) = 0.
\end{equation}
If we demand that $x\in S$ is part of the algebraic set, then we necessarily
have
\begin{equation}\label{algebraicSet}
f(x) = 0
\end{equation}
since we specifically defined algebraic sets to be the sets of zeros of 
polynomials.

If we notice these two things, we have
\begin{equation}
f(x + \vec{v}\varepsilon) = f(x) + \varepsilon\vec{v}\cdot\nabla f(x) = \varepsilon\vec{v}\cdot\nabla f(x)
\end{equation}
since by Eq (\ref{algebraicSet}) we have $f(x)=0$ we are left with the right
hand side. Since we want to find all such vectors $\vec{v}$ that are orthogonal
to the gradient of $f$, we then can find it through the relationship
\begin{equation}
f(x + \vec{v}\varepsilon) = 0
\end{equation}
instead of solving a differential system of equations. We will be using this 
property a lot to find tangent vectors.

Here we must reiterate the beauty of this system. To find tangent vectors, we 
simply are looking for vectors $\vec{v}$ such that
\begin{equation}
f(x + \varepsilon\vec{v}) = 0
\end{equation}
where $f(x)=0$ generates an algebraic set $S$. By taylor expanding, we see that
\begin{equation}
f(x + \varepsilon\vec{v}) = \cancelto{0}{f(x)} + \varepsilon\vec{v}\cdot\nabla f(x)
\end{equation}
and moreover, if this is zero then we have found the tangent vectors to the
point $x\in S$!

\section{Matrix Groups}

If the reader is unfamiliar with either group theory or matrices, it is advised 
to refer to the appropriate appendix first. Whenever we discuss matrices, it is
assumed to be a square matrix.

Allow us to first introduce the \emph{general linear group}. If we have two 
matrices which are invertible $A$ and $B$, then their product is invertible.
We know this because if $\det(A)\neq 0$ then $A$ is invertible, and
\begin{equation}
\det(AB)=\det(A)\det(B).
\end{equation}
Thus the only way that $AB$ is not invertible would be if
\begin{equation}\label{necessaryConditionInvertibility}
\det(AB) = 0
\end{equation}
but
\begin{equation}
\det(A)\neq 0 \quad\textrm{and}\quad\det(B)\neq 0
\end{equation}
which means that Eq (\ref{necessaryConditionInvertibility}) is no longer true.
Thus we have a contradiction.

We therefore conclude the product of two invertible matrices is invertible. We
have a group! The identity element is the identity matrix, and the binary
operation is matrix multiplication. This group is called the \textbf{general
linear group} of $n\times n$ matrices over the field $\mathbb{F}$, or more
succinctly just $GL_{n}(\mathbb{F})$. We usually have $\mathbb{F}=\mathbb{R}$ or
$\mathbb{C}$.

\marginpar{Meaning of ``special''}We can then normalize the matrices by a simple
trick. If we have an $n\times n$ matrix $A$ with a determinant not equal to one,
we can simply map it to
\begin{equation}
A\mapsto\frac{1}{\sqrt[n]{\det(A)}}A
\end{equation}
which preserves the property of unit determinant. The subgroup of $GL_{n}(\mathbb{F})$
with this extra condition is called the \textbf{special linear group} and denoted
$SL_{n}(\mathbb{F})$.

The next property of matrices that we can investigate would be orthogonality. 
That is, when we diagonalize a matrix $X$, we end up multiplying it by $AXA^{-1}$.
The property of orthogonal matrices that we are interested in simply is
\begin{equation}
A^\textrm{T} = A^{-1}.
\end{equation}
The subgroup of $GL_{n}(\mathbb{F})$ that satisfies such a property is the
\textbf{orthogonal group} denoted as $O_{n}(\mathbb{F})$. And similarly, the
subgroup of $SL_{n}(\mathbb{F})$ that satisfies the orthogonality property is
the \textbf{special orthogonal group} denoted $SO_{n}(\mathbb{F})$.

The favorite property of quantum physicists, unitarity (``self-adjointness''), 
also forms a group. The basic property is this, for an $n\times n$ matrix $A$
with complex entries, we have
\begin{equation}
\bar{A}^\textrm{T} = A^{-1}
\end{equation}
the complex conjugate of the transpose is the inverse of $A$. This forms the
\textbf{unitary group} denoted as $U_{n}(\mathbb{C})$. We can form the 
\textbf{special unitary group} by the normalization routine outlined above, and
we denote this group as $SU_{n}(\mathbb{C})$. 

The most bizarre condition on a $2n\times 2n$ matrix $M$ (note even dimensions!) is that of being symplectic. What does it mean anyways? It means that $M$ satisfies
\begin{equation}
M^\textrm{T} \Omega M = \Omega
\end{equation}
where (typically)
\begin{equation}\label{usualSuspect}
\Omega = \begin{bmatrix} 0 & I_n \\ -I_n & 0 \\ \end{bmatrix}.
\end{equation} 
What is the consequenec of this? Well, it's a bit stronger than orthogonality because
\begin{equation}
M^{-1} = \Omega^{-1}M^\textrm{T}\Omega
\end{equation}
holds. We can form $M$ into block matrix form where
\begin{equation}
M = \begin{bmatrix}A & B \\ C & D\end{bmatrix}
\end{equation}
where $A,B,C,D$ are all $n\times n$ matrices satisfying the following properties:
\begin{subequations}
\begin{align}
A^\textrm{T}D - C^\textrm{T}B &= I \\
A^\textrm{T}C &= C^\textrm{T}A \\
D^\textrm{T}B &= B^\textrm{T}D.
\end{align}
\end{subequations}
The symplectic property of matrices is not as well known as the orthogonality condition or self-adjointness.

We have this neat little table of matrix groups:
\begin{center}
  \begin{tabular}{ | l | l | }
\hline
$GL_{n}(\mathbb{R})$ & all invertible matrices with real entries \\ \hline
$SL_{n}(\mathbb{R})$ & all matrices in $GL_{n}(\mathbb{R})$ with determinant 1 \\ \hline
$O_{n}(\mathbb{R})$ & all matrices with their inverse equal to their transpose \\ \hline
$SO_{n}(\mathbb{R})$ & all matrices in $O_{n}(\mathbb{R})$ with determinant 1 \\ \hline
$U_{n}$ & all unitary $n\times n$ matrices \\ \hline
$SU_{n}$ & all matrices in $SL_{n}(\mathbb{C})$ and in $U_{n}$. \\ \hline
$SP_{2n}(\mathbb{R})$ & $P\in GL_{2n}(\mathbb{R})$ such that $P^\textrm{T}JP = J$\\
 & for a given antisymmetric matrix $J$ (usually the one in Eq (\ref{usualSuspect})) \\ \hline
  \end{tabular}
\end{center} 

\section{Lie Algebras}

The notion of a Lie algebra is fairly straightforward, but to reiterate we just
generalized the notion of a tangent vector to a point $x$ in an algebraic set
$S$ generated by the zeros of the function $f$ to be the vector $\vec{v}$ such
that
\begin{equation}
f(x + \varepsilon\vec{v}) = 0.
\end{equation}
We will use this notion to find the ``tangent vectors'' of the matrix groups
tangent to the ``point'' $I$ (the identity matrix).

\begin{ex}
We have certain conditions on the groups which we have outlined in function form, 
e.g. the orthogonal group requires that
\begin{equation}
A^\textrm{T} = A^{-1}.
\end{equation}
We have the tangent vectors then satisfy
\begin{equation}
(I + \varepsilon V)^\textrm{T} = (I + \varepsilon V)^{-1}
\end{equation}
which reduces to
\begin{equation}
\varepsilon V^\textrm{T} = -\varepsilon V.
\end{equation}
How did we do this step? Well, with matrices if the spectral radius is less than
$\pm1$, then we can do the following
\begin{equation}
(I - X)^{-1} = I + X + \frac{X^2}{2!} + \ldots
\end{equation}
and since $\varepsilon^2 = 0$ we have it only to first order. Thus
\begin{equation}
(I + \varepsilon V)^{-1} = I - \varepsilon V
\end{equation}
and we are content.

This implies that
\begin{equation}
\varepsilon(V^\textrm{T} + V) = 0
\end{equation}
and since 
\begin{equation}
\varepsilon\neq 0
\end{equation}
we have \emph{necessarily}
\begin{equation}\label{orthogonalLieAlgebra}
(V^\textrm{T} + V) = 0\Rightarrow V^\textrm{T}=-V.
\end{equation}
Thus we are dealing with traceless matrices. QEF.
\end{ex}

It often occurs that the same infinitesmal tangents can correspond to different groups.
For example, for the special linear group we have the condition
\begin{equation}
\det(X) = 1
\end{equation}
but we have the infinitesmal tangents determined by
\begin{equation}
\det(I + \varepsilon A) = 1.
\end{equation}
We know for ``small'' $\varepsilon$ that
\begin{equation}
\det(I + \varepsilon A) = 1 + \varepsilon \operatorname{tr}(A)
\end{equation}
and to have this be equal to one we need
\begin{equation}
\operatorname{tr}(A) = 0
\end{equation}
which is satisfied by Eq (\ref{orthogonalLieAlgebra}).

\begin{thm}
Let $A$ be a real, $n\times n$ matrix. Then the following are equivalent
\begin{enumerate}
\item $\operatorname{tr}(A)=0$;
\item $\exp(tA)$ is a one-parameter subgroup of $SL_{n}(\mathbb{R})$;
\item $A$ is in the Lie Algebra of $SL_{n}(\mathbb{R})$;
\item $A$ is an infinitesimal tangent to $SL_{n}(\mathbb{R})$ at $I$.
\end{enumerate}
\end{thm}

Note note note that the infinitesmal tangents includes as a subset the Lie
Algebras. We denote the Lie Algebra of a Matrix Group $G$ as $\lie{G}$. We also
denote the infinitesmal tangents of $G$ by $\textrm{Inf}(G)$.  We denote the
one-parameter subgroup of $G$ as $\textrm{Exp}(G)$. We have
\begin{equation}
\textrm{Exp}(G)\subset\lie(G)\subset\textrm{Inf}(G).
\end{equation}
So we have the most general set being the tangents of $I$ in $G$.

\subsection{Lie Bracket}

The Lie Algebra of a linear group has additional structure, an operation called
a \textbf{Lie Bracket}. We formally induce a Lie Bracket on these collections 
of matrices by the expression
\begin{equation}
[A,B] = AB - BA
\end{equation}
where $A$ and $B$ are matrices. This satisfies the \textbf{Jacobi Identity}
\begin{equation}
[A,[B,C]] + [B,[C,A]] + [C,[A,B]] = 0.
\end{equation}
We see also that if $A,B$ are skew-symmetric (antisymmetric), then
\begin{equation}
[A,B]^\textrm{T} = (AB-BA)^\textrm{T} = B^\textrm{T}A^\textrm{T} - A^\textrm{T}B^\textrm{T} = BA-AB = -[A,B].
\end{equation}
That is, it results in a skew-symmetric matrix.

The bracket is important because it is the infinitesmal version of the commutator in Lie \emph{groups}
$PQP^{-1}Q^{-1}$. To see this, we introduce two infinitesimal parameters $\epsilon$
and $\delta$ such that
\begin{equation}
\epsilon^2 = \delta^2 = 0
\end{equation}
and
\begin{equation}
\epsilon\delta = \delta\epsilon.
\end{equation}
Recall that the inverse of $I+A\epsilon$ is $I-A\epsilon$. So if $P=I+A\epsilon$
and $Q=I+B\delta$, the commutator expands to
\begin{equation}
(I+A\epsilon)(I+B\delta)(I-A\epsilon)(I-B\delta) = I + (AB - BA)\epsilon\delta.
\end{equation}
Intuitively we can see this because the commutator of infinitesmal elements in
$G$ is still in $G$!

\begin{defn}
A \textbf{Lie Algebra} $V$ over a field $\mathbb{F}$ is a vector space together
with a law of composition
\begin{eqnarray*}
V\times V &\to& V \\
v,w & \mapsto & [v,w]
\end{eqnarray*}
called the \textbf{bracket} having the following properties:
\begin{enumerate}
\item (Bilinearity) $[v_1 + v_2,w] = [v_1,w] + [v_2,w]$, $[cv,w] = c[v,w]$, $[v,w_1 + w_2] = [v,w_1] + [v,w_2]$, $[v,cw] = c[v,w]$;
\item (Antisymmetry) $[v,v]=0$;
\item (Jacobi Identity) $[u,[v,w]] + [v,[w,u]] + [w,[u,v]] = 0$;
\end{enumerate}
for all $u,v,w\in V$ and all $c\in\mathbb{F}$.
\end{defn}


\appendix
%%
%% groups.tex
%% 
%% Made by Alex Nelson
%% Login   <alex@tomato>
%% 
%% Started on  Wed Dec 24 14:35:51 2008 Alex Nelson
%% Last update Wed Dec 24 14:35:51 2008 Alex Nelson
%%

\begin{figure}[ht!]
\begin{center}
\includegraphics{img/img.1}
\end{center}
\caption[Relation between Monoids and Groups]{A Venn Diagram
  to illustrate the relation of the monoids and the
  groups. Observe all groups are monoids, but not all
  monoids are groups. Similarly, all trout are fish, but not
  all fish are trouts.}
\end{figure}
A \textbf{group}\index{Group} $G$ is a monoid with some
extra structure. Namely, for each element $x\in G$ there is
an element $y\in G$ such that
\begin{equation}
xy = yx = e.
\end{equation}
This element $y$ is called an
\textbf{inverse}\index{Inverse} for $x$. Such an inverse is
unique, a simple proof to illustrate this: suppose that
there are two inverses $y,y'$ of $x$, then
\begin{equation}
y' = y'e = y'(xy) = (y'x)y = ey = y.
\end{equation}
For multiplication, we denote the inverse of $x$ by
$x^{-1}$, and for addition the inverse of $x$ is denoted by
$-x$. The \textbf{order}\index{Group!Order} of a group is
the number of elements in the group, the size of the set so
to speak.

It is probably worth going on without saying that $e$ is its
own inverse. But if we have (an identity $e$) an element $x$ and its inverse
$x^{-1}$ and a given law of composition, then we can create
a group consisting of the elements
\begin{equation}
G = \{ (x^{-1})^{n}:n\in\mathbb{N}\}\cup\{(x)^{n}:n\in\mathbb{N}\}\cup\{e\}
\end{equation}
which is trivial. 

\begin{ex}
Consider the matrix
\begin{equation}
z = \begin{bmatrix}1 & 0 \\0 & -1\end{bmatrix}
\end{equation}
observe that $z^2=I$. That tells us that $\{I,z\}$ is a
matrix group under matrix multiplication. \qef
\end{ex}

Notice we are being a bit ambiguous here in specifying
``inverses'' and ``unit elements''. E.g. with matrices, it makes
a difference if we multiply by the right or by the left
(e.g. $X$ multiplied on the right by $Y$ is $XY$ but
multiplied on the left is $YX$, and in general $XY-YX\neq0$
-- if it is zero, then $X$, $Y$ are diagonal matrices). But
we have been sufficiently general so left inverses and left
identity elements are also inverses and identity
elements. We can be precise in specification and prove it
too:
\begin{prop}
Let $G$ be a set with an associative law of composition, let
$e$ be a left unit for that law, and assume that every
element has a left inverse. Then $e$ is a unit and each left
inverse is also an inverse. In particular, $G$ is a group.
\end{prop}
\begin{proof}
Let $b\in G$ and let $a\in G$ be such that $ba=e$. Then
\begin{equation}
bab = eb = b.
\end{equation} 
This much should be trivial, it's just substitution. We can
see that
\begin{equation}
abab = a(ba)b = aeb = ab
\end{equation}
which is equivalent to
\begin{equation}
(ab)^2 = ab.
\end{equation}
We can multiply both sides by $(ab)^{-1}$ on the left to see
that
\begin{equation}
(ab) = e
\end{equation}
which implies that $a$ is the left inverse for $b$, or
equivalently that $b$ is the right inverse for $a$. We then
see that
\begin{equation}
aba = a(e) = (e)a
\end{equation}
which implies the left identity $e$ is a ``bi''-identity
(i.e. both a left and right identity).
\end{proof}
\begin{ex}
Let $G$ be a group, and $S$ be some nonempty set. The set of
maps from $S$ to $G$, denoted $M(S,G)$, is itself a
group. That is, for two maps $f,g:S\to G$, we define $fg$ to
be the map such that
\begin{equation}\label{eq:groups:exLawOfComposition}
(fg)(x) = f(x)g(x)
\end{equation}
The inverse $f^{-1}$ multiplied by $f$ is equal to the
identity element in the group ${\mathbbm 1}$. That means,
$f^{-1}(x)=f(x)^{-1}$. If $f(x)^{-1}$ is well defined,
meaning that $f(x)$ -- an element of the group $G$ -- has an
inverse, which is necessarily true because that's the
definition of a group, then each element of $M(S,G)$ has an
inverse under the law of composition defined by Eq
\eqref{eq:groups:exLawOfComposition}. We also have for
arbitrary $f\in M(S,G)$ the product of it well defined too
since $f(x)$ is an element in the group, so $f(x)^n$ is an
element in the group raised to the $n^{th}$ power -- which
is well defined because it's a monoid! So it is a group.\qef
\end{ex}
\begin{ex}
Let $S$ be a non-empty set. Let $G$ be the set of bijections
from $S$ to $S$. Then we see that, making the composition of
mappings the law of composition, $G$ is a group. (Remember a
bijection is invertible, so each map has an inverse and we
have the identity defined in the obvious way as the identity
map of $S$.) The elements of $G$ are called
\textbf{permutations}\index{Permutations} of $S$. We denote
$G$ by $\operatorname{Perm}(S)$. \qef
\end{ex}
\begin{ex}
Consider the group $\mathbb{Z}_{2}$, which consists of two
elements $0$ and $1$. One can imagine this as a ``bit'' or a
binary digit. It has the operation of addition
\begin{equation}
0+0=1+1=0,\quad 0+1=1+0=1
\end{equation}
which is commutative. It is a cyclic group.\index{Group!Cyclic}\index{Cyclic Group} We can further
generalize this to $\mathbb{Z}_{3}$ with 3 elements: 0, 1,
and 2. It has the laws of addition
\begin{equation}
0+0=1+2=2+1=0,\quad 0+1=1+0=2+2=1,\quad 0+2=2+0=1+1=2
\end{equation}
which is also commutative. We have inverses well defined,
etc. etc. etc. We can generalize this to $\mathbb{Z}_{n}$
where $n\in\mathbb{N}$. This is a family of finite groups. \qef
\end{ex}
\begin{rmk}
Typically we ``represent'' a number $p$ in a cyclic group
$\mathbb{Z}_{n}$ (where $0\leq p<n$ is some integer) by
$\exp[i2\pi (p/n)]$. In this ``representation'', we have
instead of addition simple multiplication. Note that the
``generator'' of the group (the only element we really need
that we can subject to the law of composition of the group
to) is the primitive $n^{th}$ root of unity
$\exp[i2\pi/n]$. So $p$``=''$(\exp[i2\pi/n])^p$.
\end{rmk}
\begin{ex}
Let $G_1$, $G_2$ be groups. Let $G_1\times G_2$ be the
\textbf{direct product}\index{Direct Product!Groups}\index{Group!Direct Product}
which is similar to the direct product of sets, so
$G_1\times G_2$ is the set of all pairs $(x_1, x_2)$ with
$x_1\in G_1$ and $x_2\in G_2$. We define the product
componentwise by
\begin{equation}
(x_1,x_2)(y_1,y_2) = (x_1y_1,x_2y_2).
\end{equation}
We see that $G_1\times G_2$ is a group, with unit element
$(e_1, e_2)$ (where $e_1$ is the unit of $G_1$, and $e_2$ is
the unit of $G_2$). We can do this for $n$ groups, with
componentwise multiplication. \qef
\end{ex}

Let $G$ be a group. A \textbf{subgroup}\index{Group!Subgroup}\index{Subgroup} $H$
of $G$ is a subset of $G$ containing theunit element, and
such that $H$ is closed under the law of composition and
inverse (or equivalently, it is a submonoid such that if
$x\in H$ then $x^{-1}\in H$). A subgroup is
\textbf{trivial}\index{Subgroup!Trivial} if it consists of
the unit element alone. And trivially, the intersection of
an arbitrary non-empty family of subgroups is a subgroup.

Let $G$ be a group and $S\subset G$ be a subset. We say that
$S$ \textbf{generates} $G$ or that $S$ is a set of
\textbf{generators} for $G$ if every element of $G$ can be
expressed as a product of elements of $S$ or inverses of
elements of $S$, i.e. as a product $x_1\cdots x_n$ where
each $x_i$ or $x_{i}^{-1}$ is in $S$. It is clear that the
set of al such products is a subgroup of $G$ (remember,
$x^0=e$) and is the smallest subgroup of $G$ containing
$S$. What the hell does this mean? Well, $S$ generates $G$
if and only if the smallest subgroup of $G$ containing $S$
is $G$ itself. 

Let's stop and reiterate what we have just introduced. We
have a subgroup $S$ of a group $G$ is a submonoid that is a
group. So the submonoid contains inverses for each element.
We have some finite subset $S'$ of a group $G$ which is such that
any element of $G$ can be written as a product of any finite
number of elements of $S'$. This is a bit vague, so perhaps
an example is needed. 

\begin{ex}
Consider the quaternions, which have elements
\begin{equation}
i^2=j^2=k^2=ijk=-1
\end{equation}
We see that
\begin{align*}
i(ijk) &= i(-1)\\
\Rightarrow jk &= i\\
\Rightarrow (jk)k &= ik\\
\Rightarrow -j &= ik\\
\Rightarrow ij &= k
\end{align*}
So from $i$ and $j$, we can get $1=(-1)^2=(i)^4=(j)^4$ as
well as $k=ij$. All we need are two elements to have the
rest of the quaternions. \qef
\end{ex}

Note typically, if $G$ is a group, and $S$ is a set of
generators, then we write $G=\<S\>$. \marginpar{cyclic group has one generator}By 
definition, a cyclic group is a group which has one
generator.

\begin{ex} 
There are two non-abelian groups of order 8. One is the
quaternions which we already say, the other is the
\textbf{symmetries of the square},\index{Symmetries of Square}\index{Symmetry!Square}\index{Square!Symmetry} generated
by two elements $\sigma$, $\tau$ such that
\begin{equation}
\sigma^4 = \tau^2 = e,\quad\text{and }\tau\sigma\tau^{-1} =
\sigma^{3}
\end{equation}
We see that
\begin{align*}
(\tau\sigma\tau^{-1})\tau &=
\sigma^{3}\tau\\
&= \tau\sigma
\end{align*}
but also that
\begin{equation}
\sigma^{-1} = \sigma^{3}\Rightarrow \sigma^{-1}\sigma = e =
\sigma^4
\end{equation}
and
\begin{equation}
\tau^{-1} = \tau\Rightarrow \tau\sigma\tau=\sigma^3.
\end{equation}
So that means that
\begin{equation}
\tau\sigma\tau = \sigma^{-1}
\end{equation}
and
\begin{equation}
\tau\sigma^{-1}\tau = \sigma.
\end{equation}
But doesn't this imply
\begin{equation}
\tau = \sigma\tau\sigma
\end{equation}
by multiplying by the right by $\tau\sigma$. So we can write
the inverses in terms of $\sigma$ and $\tau$ alone. \qef
\end{ex}
\begin{ex}
The quaternion group is more generally the group generated
by two elements $i$, $j$ and setting $ij=k$ and $m=i^2$, we
have
\begin{equation}
i^4 = j^4 = k^4 = e,\quad\text{and }i^2=j^2=k^2=m,\; ij=mji
\end{equation}
but observe that as we introduced it before, it works
perfectly fine setting $m=-1$ and $e=1$. \qef
\end{ex}

Let $G,G'$ be monoids. A \textbf{monoid-homomorphism} (or
simply \textbf{homomorphism}\index{Homomorphism}) of $G$
into $G'$ is a mapping $f:G\to G'$ such that
$f(xy)=f(x)f(y)$ for all $x,y\in G$ and mapping the unit
element of $G$ into that of $G'$. If additionally $G,G'$ are
groups, the mapping is given a special name called a
\textbf{group-homomorphism}. 

\begin{ex}
Consider the monoid $\mathbb{N}$, a map
\begin{equation}
f:\mathbb{N}\to\mathbb{Z}_2
\end{equation}
is a homomorphism defined such that
\begin{equation}
f(n) = \begin{cases}0 &\text{$n$ is even}\\
1 &\text{otherwise}\end{cases}
\end{equation}
Then $f$ is a homomorphism. It maps the 0 element to the 0
element, and it maps 
\begin{align*}
f(m+n) &= f(m)+f(n)
\end{align*} 
trivially (odd+odd=even, even+even=even, even+odd=odd, and
1+1=0, 0+0=0, 0+1=1 respectively). \qef
\end{ex}
\begin{ex}
Let $V$, $W$ be arbitrary vector spaces. Let
\begin{equation}
f:V\to W
\end{equation}
be a linear transformation. Trivially it is a homomorphism,
as
\begin{equation}
f(u+v)=f(u)+f(v)
\end{equation} 
and by the definition of a linear transformation, we have
\begin{equation}
f(0)=0.
\end{equation}
Why? Well, it's trivial, observe
\begin{equation}
f(0+0)=f(0)+f(0)=f(0)\Rightarrow f(0)=0.
\end{equation}
So it preserves vector addition, and it maps the identity of
vector addition to the identity of vector addition. \qef
\end{ex}

Observe that for a group homomorphism $f:G\to G'$, we have
\begin{align*}
f(xx^{-1}) &= f(e)\text{ since }xx^{-1}=e\\
&= f(x)f(x^-1)\text{ since $f$ is a homomorphism}\\
&= e'\text{ since $f$ is a homomorphism}\\
\Rightarrow \left(f(x)\right)^{-1}f(e) &= f(x^{-1})\\
&= f(x)^{-1}.
\end{align*}
It's trivial.

\begin{ex}
Let $G$ be a commutative group. Then for $x\in G$,
\begin{equation}
x\mapsto x^n
\end{equation}
for some fixed integer $n$, is a homomorphism called the
\textbf{$n$-th power map}\index{Homomorphism!$n$-th Power Map}. \qef
\end{ex}
\begin{ex}
Let $G_i$ be some collection of groups, and $i\in I$ be an
element of some indexing set. Let
\begin{equation}
G = \prod G_{i}
\end{equation}
be direct product of all $G_i$, for all $i\in I$. So an
element of $G$ would be a tuple consisting of components
from each of the $G_i$. Let
\begin{equation}
p_i:G\to G_i
\end{equation}
be the projection of the $i^{th}$ factor. It selects the
component of the tuples in $G$ which corresponds to the
contribution from the group $G_i$. Then $p_i$ is a
homomorphism. \qef
\end{ex}\begin{quote}\begin{thm}
Let $G$, $G'$ be groups, and $S$  be a set of generators of
$G$. Let
\begin{equation}
f:S\to G'
\end{equation}
be a map. If there exists a homomorphism $\widetilde{f}:G\to
G'$ such that when we restrict $G$ to be $S$,
$\widetilde{f}=f$, then there exists only one
$\widetilde{f}$. 
\end{thm}
\end{quote}
In other words, $f$ has at most one extension to a
homomorphism of $G$ into $G'$. 

Let $f:G\to G'$ and $g:G'\to G''$ be two
group-homomorphisms. Then the composition $g\circ f$ is also
a group homomorphism. If $f,g$ are isomorphisms then so is
$g\circ f$. Further, $f^{-1}:G'\to G$ is also an
isomorphism. In particular, the set of all automorphisms of
$G$ is itself a group, denoted as $\aut(G)$.
\begin{defn}
Let $f:G\to G'$ be a group homomorphism. Let $e,e'$ be the
respective unit elements of $G,G'$. We can then define the
\textbf{kernel}\index{Kernel!Group Homomorphism} of $f$ to
be the subset of $G$ consisting of all elements $x$ such
that $f(x)=e'$.
\end{defn}
We immediately see that the kernel forms a subgroup of $G$,
since for any two $x,y\in\ker(f)$, we have
\begin{align*}
f(x+y) &= f(x)+f(y)\\
&= e'+e'\\
&= e'.
\end{align*}
Furthermore, since $f$ is a homomorphism, we have $f(e)=e'$
and
\begin{equation}
f(e) = f(xx^{-1}) = f(x)f(x)^{-1} = e'f(x^{-1})=e'
\end{equation}
which implies $f(x^{-1})=e'$. So $x^{-1}$ is in $\ker(f)$.
\begin{prop}
Let $f:G\to G'$ be a group homomorphism. Let $H'$ be the
\textbf{image}\index{Homomorphism!Image} of $f$. Then $H'$
is a subgroup of $G'$.
\end{prop}
\begin{proof}
Observe that for $x,y\in G$, that
\begin{equation}
f(xy)=f(x)f(y)\in H'.
\end{equation}
Further, since $f(e)=e'\in H'$ (so $H'$ has an identity
element), we see
\begin{equation}
f(xx^{-1})=f(x)f(x^{-1})=e'\Rightarrow f(x^{-1})\in H'.
\end{equation}
So the inverse of an arbitrary element is in $H'$, as is the
identity element, which is sufficient for $H'$ to be a
subgroup. 
\end{proof}
\begin{rmk}
The kernel and image of $f$ are denoted by $\ker(f)$ and
$\imag(f)$ respectively.
\end{rmk}

\section{A Review of Matrices}\label{appendixMatrix}


\subsection{Matrix Operations}

There are two main matrix operations: matrix addition and matrix multiplication. Both are interesting.

First a digression. When working with matrices, we usually have the components represented by a value. We also want to be as general as possible, so these values are represented by variables. But the English language only has 26 letters...so anything beyond a 5 by 5 matrix is beyond hope of representation if we just use $a,b,\ldots,x,y,z$. What to do? We will use \textbf{indices} to indicate what entry we are talking about! So each matrix has only one letter, and two indices. For example, a 2 by 2 matrix could be represented by
\begin{equation}
A = [a_{ij}] = \left[
\begin{array}{cc}
a_{11} & a_{12} \\
a_{21} & a_{22}
\end{array}
\right]
\end{equation}
where the $i$ index tracks the \textbf{rows} and the $j$ index tracks the \textbf{columns}. \marginpar{Indices are dummy variables}\textbf{NOTE NOTE NOTE} that the indices are \emph{dummy variables}! This means that $b_{ij}$ can be rewritten as $b_{mn}$ (or with any other pair of letters one would like) and it would refer to the same thing.

So a 3 by 3 matrix can be represented by
\begin{equation}
B = [b_{kl}] =
\begin{bmatrix}
b_{11} & b_{12} & b_{13} \\
b_{21} & b_{22} & b_{23} \\
b_{31} & b_{32} & b_{33}
\end{bmatrix}
\end{equation}
and so on and so forth. It is increasingly common to see in Differential Geometry simply $b_{kl}$ to be used instead of $B$, so we can explicitly calculate things out.

Matrix addition is merely componentwise addition. So if $A = [a_{ij}]$ and $B = [b_{kl}]$ then
\begin{equation}
A + B = [a_{ij} + b_{ij}] = \begin{bmatrix}
a_{11}+b_{11} & \ldots \\
\vdots & \ddots
\end{bmatrix}.
\end{equation}
Matrix addition is straightforward, it's done component-wise.

The other operation, matrix multiplication, is a bit trickier! If $A$ is an m-by-n matrix and $B$ is an n-by-p matrix, then their product $C=AB$ is an m-by-p matrix defined component wise by
\begin{equation}
C = [c_{ij}] = [\sum_{k}^{n} a_{ik}b_{kj}]
\end{equation}
where $A=[a_{ij}]$ and $B=[b_{kj}]$.

Also note that we may think of a matrix as being composed of column vectors or row vectors. Why should we think of it like this? Well, matrix multiplication drastically simplifies 100 fold. Observe that it suddenly becomes little more than finding the dot product of the $i^\textrm{th}$ row of $A$ with the $j^\textrm{th}$ column of $B$ to find the component $c_{ij}$.\footnote{The rule I always remember is ``row-dot-column''.}


\begin{SCfigure}
  \centering
  \includegraphics[width=0.5\textwidth]%
    {mult.png}% picture filename
  \caption{ The ``row dot column'' technique of matrix multiplication illustrated. }
\end{SCfigure}

For a generalized example, consider the following illustration:
\begin{equation}
\mathbf{AB} =   \begin{bmatrix}    A_1 \\    A_2 \\    A_3 \\    \vdots \end{bmatrix} * \begin{bmatrix} B_1 & B_2 & B_3 & \dots \end{bmatrix} =  \begin{bmatrix} (A_1 \cdot B_1) & (A_1 \cdot B_2) & (A_1 \cdot B_3) & \dots \\ (A_2 \cdot B_1) & (A_2 \cdot B_2) & (A_2 \cdot B_3) & \dots \\ (A_3 \cdot B_1) & (A_3 \cdot B_2) & (A_3 \cdot B_3) & \dots \\ \vdots & \vdots & \vdots & \ddots  \end{bmatrix}.
\end{equation}
where $A_{i}$ are row vectors and $B_{k}$ are column vectors.

We will explicitly compute an example of matrix multiplication
\begin{equation}
 \begin{bmatrix}      1 & 0 & 2 \\       -1 & 3 & 1   \end{bmatrix} \cdot   \begin{bmatrix}      3 & 1 \\      2 & 1 \\      1 & 0   \end{bmatrix} = \begin{bmatrix}    1 \times 3 + 0 \times 2 + 2 \times 1 & 1 \times 1 + 0 \times 1 + 2 \times 0 \\   -1 \times 3 + 3 \times 2 + 1 \times 1 & -1 \times 1 + 3 \times 1 + 1 \times 0  \end{bmatrix} = \begin{bmatrix}     5 & 1 \\     4 & 2 \end{bmatrix}
\end{equation}

\begin{ex}
Given two vectors $\vec{a}=a_i,\vec{b}=b_j$ their dot product is explained through the use of matrix multiplication
\begin{equation}
\vec{a}\cdot \vec{b} = \sum_{i=1}^n a_ib_i = a_1b_1 + a_2b_2 + \cdots + a_nb_n 
\end{equation}
or more generally, if we are using complex-valued vectors
\begin{equation}
\vec{a}\cdot \vec{b} = \sum_{i=1}^n a_i\bar{b}_i
\end{equation}
where $\bar{b}_i$ are the complex conjugate components of $\vec{b}$. QEF.
\end{ex}

Let us note some properties of Matrix multiplication: it is associative
\begin{equation}
A(BC) = (AB)C
\end{equation}
it is distributive
\begin{equation}
A(B+C) = AB + AC
\end{equation}
and
\begin{equation}
(A+B)C = AC + BC.
\end{equation}
It is compatible with scalar multiplication too (let $c$ be a scalar)
\begin{equation}
\begin{array}{cc}
c(AB) &= (cA)B\\
(Ac)B &= A(cB)\\
(AB)c &= A(Bc).
\end{array}
\end{equation}

\subsection{Euclidean and Einstein Summation Conventions}

There is a convention which Einstein invented (or so I am told, I may be completely wrong!) where repeated indices are summed over. This occurs if and only if at least one of the repeating indices is a subscript and at least one is a superscript. For example, the dot product is represented as
\begin{equation}
\vec{a}\cdot\vec{b} = a_{i}b^{i} = \sum_{i}a_{i}b^{i}.
\end{equation}
Do not confuse the superscripted $b$ vector for ``$b$ to the $i^\textrm{th}$ power''! It is used to indicate which vectors are \textbf{covariant} and which vectors are \textbf{contravariant}.\footnote{There is some confusion and debate about the use of these terms in higher mathematics, since in category theory they mean the exact opposite of what they mean in the physics application of differential geometry. We will use the physics conventions here.} Here covariant vectors are linear functionals (see \S 4). Covariant vectors are denoted by the subscript indices, and contravariant vectors are denoted by the superscript indices.

We will resist the urge to go into details about the notion of covariance and contravariance until \S 4, feel free to skip ahead to find out more about it.

On the other hand, if we just sum whenever we see two indices of any kind with the same variable, this is called \textbf{Euclidean Summation Convention}\footnote{The origin of this phrase is, as far as the author is aware, from Misner, Thorne, and Wheeler's \emph{Gravitation}, Chapter 12.3, page 294: ``(`Euclidean' index conventions; repeated space indices to be summed even if both are down; dot denotes time derivative)''.}. For instance
\begin{equation}
a_{ijk}b_{klm} = \sum_{k}a_{ijk}b_{klm}
\end{equation}
is done by Euclidean convention, but this would never happen in Einstein convention!

\begin{rmk}
Most of the time, Euclidean summation convention and Einstein summation convention are both used in physics and math texts. Care should be used in the future!
\end{rmk}

\subsection{Transposes of Matrices}

When we have a matrix $A$, we can take its transpose, which is (for a 2 by 2 matrix)
\begin{equation}
\left[\begin{array}{cc}
a & b\\
c & d
\end{array}\right]^{\mathrm{T}}
= 
\left[\begin{array}{cc}
a & c\\
b & d
\end{array}\right].
\end{equation}
Note how the diagonal entries stay the same as we swap the off diagonal components. 

We can take the transpose of rectangular matrices too. Observe
\begin{equation}
\begin{bmatrix} 1 & 2 \\ 3 & 4 \\ 5 & 6 \end{bmatrix}^{\mathrm{T}}  = \begin{bmatrix} 1 & 3 & 5\\ 2 & 4 & 6 \end{bmatrix}. 
\end{equation}

Recall that we represent vectors as a column vector, which is a column matrix with 1 entry per row. That is, for a 2 dimensional vector we have
\begin{equation}
\vec{v} = \begin{bmatrix}
a \\
b
\end{bmatrix}
\end{equation}
and a 3 dimensional vector:
\begin{equation}
\vec{w} = \begin{bmatrix}
a \\
b \\
c
\end{bmatrix}.
\end{equation}
We too can take their transposes, and we end up with a \emph{row vector!}

There are some important things to remember about transposes. First of all, it's an involution. What does this mean? Well, if you do it twice, you get back the original matrix. That is
\begin{equation}
(A^{\textrm{T}})^{\textrm{T}} = A.
\end{equation}
Go ahead and try to prove this one for yourself.

Secondly, it's \textbf{linear!} That is, given two matrices $A$ and $B$, we have
\begin{equation}
(A + B)^{\textrm{T}} = A^{\textrm{T}} + B^{\textrm{T}}
\end{equation}
which is trivial but an example will be given:
\begin{equation}
\left(
\begin{bmatrix}
a & b \\
c & d
\end{bmatrix} + 
\begin{bmatrix}
w & x\\
y & z
\end{bmatrix}
\right)^{\textrm{T}} = 
\left(
\begin{bmatrix}
a+w & b+x \\
c+y & d+z
\end{bmatrix}\right)^{\textrm{T}} = \begin{bmatrix}
a+w & c+y \\
b+x & d+z
\end{bmatrix}
\end{equation}
and 
\begin{equation}
\begin{bmatrix}
a & b \\
c & d
\end{bmatrix}^{\textrm{T}} + 
\begin{bmatrix}
w & x\\
y & z
\end{bmatrix}^{\textrm{T}} = 
\begin{bmatrix}
a & c \\
b & d
\end{bmatrix} + 
\begin{bmatrix}
w & y\\
x & z
\end{bmatrix} =
\begin{bmatrix}
a+w & c+y\\
x+b & d+z
\end{bmatrix}
\end{equation}
which is precisely what we just got for $(A+B)^\textrm{T}$! So we find it holds true for $2\times 2$ matrices.

Now, there is a rather counter-intuitive property that
\begin{equation}
(AB)^\textrm{T} = B^\textrm{T}A^\textrm{T}.
\end{equation}
This is precisely the property that we want to see!

\bibliographystyle{utcaps}
\bibliography{liebib}
\end{document}
