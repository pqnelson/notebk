\subsection{Example}

We will consider the process $A+A\to B+B$, which is represented by the following
Feynman diagram (note that the x axis is the spatial dimension, the y axis is
the time dimension):

\strut

\begin{center}
\begin{fmffile}{exOneImg1}
  \begin{fmfgraph*}(50,25)  \fmfpen{0.1mm}
%     \\fmfset{arrow_len}{4mm}\\fmfset{arrow_ang}{10}
    \fmfleft{i1,o1} % change i2->o1 
    \fmfright{i2,o2} % change o1->i2
    \fmflabel{$A,p_{1}$}{i1}
    \fmflabel{$B,p_{3}$}{o1} %
    \fmflabel{$A,p_{2}$}{i2} %
    \fmflabel{$B,p_{3}$}{o2}
    \fmf{plain}{i1,v1,o1} %
    \fmf{plain}{i2,v2,o2} %
    \fmf{plain,label=$C,,q$}{v1,v2}
  \end{fmfgraph*}
\end{fmffile}
\end{center}
\strut
\textbf{Step One:} We drew it careful about notation (note the internal momentum line $q$ and the external lines $p_j$). 



\textbf{Step Two:} We have to worry about the vertices, at each one we have to
award a term of 
\begin{equation*}
-ig
\end{equation*}
So here is the Feynman diagram with the vertices enlarged in red:

\strut

\begin{center}
\begin{fmffile}{exOneImg2}
  \begin{fmfgraph*}(50,25) \fmfpen{0.1mm}
%     \\fmfset{arrow_len}{4mm}\\fmfset{arrow_ang}{10}
    \fmfleft{i1,o1} % change i2->o1 
    \fmfright{i2,o2} % change o1->i2
    \fmflabel{$A,p_{1}$}{i1}
    \fmflabel{$B,p_{3}$}{o1} %
    \fmflabel{$A,p_{2}$}{i2} %
    \fmflabel{$B,p_{3}$}{o2}
    \fmf{plain}{i1,v1,o1} %
    \fmf{plain}{i2,v2,o2} %
    \fmf{plain,label=$C,,q$}{v1,v2}    
    \fmfv{decor.shape=circle,decor.filled=full,decor.size=2thick,fore=red}{v1,v2}
  \end{fmfgraph*}
\end{fmffile}\\
\end{center}

\strut


We see that there are two vertices, one where $A$ emits
$C$ and becomes $B$:

\strut


\begin{center}
\strut
\begin{fmffile}{exOneImg3}
  \begin{fmfgraph*}(50,25) \fmfpen{0.1mm}
%     \\fmfset{arrow_len}{4mm}\\fmfset{arrow_ang}{10}
    \fmfleft{i1,o1} % change i2->o1 
    \fmfright{i2,o2} % change o1->i2
    \fmflabel{$A,p_{1}$}{i1}
    \fmflabel{$B,p_{3}$}{o1} %
    \fmflabel{$A,p_{2}$}{i2} %
    \fmflabel{$B,p_{3}$}{o2}
    \fmf{plain}{i1,v1,o1} %
    \fmf{plain}{i2,v2,o2} %
    \fmf{plain,label=$C,,q$}{v1,v2}    
    \fmfv{decor.shape=circle,decor.filled=full,decor.size=2thick,fore=red}{v1}
  \end{fmfgraph*}
\end{fmffile}
\strut
\end{center}
\strut
\\
 and the other where $A$ receives $C$ and turns into $B$:
\strut
\\
\begin{center}
\strut
\begin{fmffile}{exOneImg4}
  \begin{fmfgraph*}(50,25) \fmfpen{0.1mm}
%     \\fmfset{arrow_len}{4mm}\\fmfset{arrow_ang}{10}
    \fmfleft{i1,o1} % change i2->o1 
    \fmfright{i2,o2} % change o1->i2
    \fmflabel{$A,p_{1}$}{i1}
    \fmflabel{$B,p_{3}$}{o1} %
    \fmflabel{$A,p_{2}$}{i2} %
    \fmflabel{$B,p_{3}$}{o2}
    \fmf{plain}{i1,v1,o1} %
    \fmf{plain}{i2,v2,o2} %
    \fmf{plain,label=$C,,q$}{v1,v2}    
    \fmfv{decor.shape=circle,decor.filled=full,decor.size=2thick,fore=red}{v2}
  \end{fmfgraph*}
\end{fmffile}
\end{center}
\strut


By our rules, this means we get two factors of 
\begin{equation*}
-ig.
\end{equation*}
That is to say, our integrand is thus
\begin{equation}
(-ig)^2
\end{equation}
and we will add even more to it!

\textbf{Step Three:} (Let $m_{C}$ be the mass of a $C$ particle.) We also need a propagator for the internal line; we have below the Feynman diagram with the 
internal line in red:
\strut

\begin{center}
\begin{fmffile}{exOneImg5}
  \begin{fmfgraph*}(50,25) \fmfpen{0.1mm}
%     \\fmfset{arrow_len}{4mm}\\fmfset{arrow_ang}{10}
    \fmfleft{i1,o1} % change i2->o1 
    \fmfright{i2,o2} % change o1->i2
    \fmflabel{$A,p_{1}$}{i1}
    \fmflabel{$B,p_{3}$}{o1} %
    \fmflabel{$A,p_{2}$}{i2} %
    \fmflabel{$B,p_{3}$}{o2}
    \fmf{plain}{i1,v1,o1} %
    \fmf{plain}{i2,v2,o2} %
    \fmf{plain,label=$C,,q$,fore=red}{v1,v2}    

  \end{fmfgraph*}
\end{fmffile}
\end{center}
\strut

This means we have a factor of
\begin{equation*}
\frac{i}{q^2 - m^{2}_{C}c^2}.
\end{equation*}
We multiply this into our integrand which becomes
\begin{equation}
(-ig)^2\frac{i}{q^2 - m^{2}_{C}c^2}.
\end{equation}


\textbf{Step Four:} Now conservation of momentum demands two delta functions;
we see that the momentum has to be conserved at the vertices, so we have two
diagrams in color this time. At one vertex, we have the input momentum (the red line be)
equal to the sum of the output momentum (blue lines):

\strut

\begin{center}
\begin{fmffile}{exOneImg6}
  \begin{fmfgraph*}(50,25) \fmfpen{0.1mm}
%     \\fmfset{arrow_len}{4mm}\\fmfset{arrow_ang}{10}
    \fmfleft{i1,o1} % change i2->o1 
    \fmfright{i2,o2} % change o1->i2
    \fmflabel{$A,p_{1}$}{i1}
    \fmflabel{$B,p_{3}$}{o1} %
    \fmflabel{$A,p_{2}$}{i2} %
    \fmflabel{$B,p_{3}$}{o2}
    \fmf{plain,fore=red}{i1,v1} %
    \fmf{plain,fore=blue}{v1,o1}
    \fmf{plain}{i2,v2,o2} %
    \fmf{plain,label=$C,,q$,fore=blue}{v1,v2}    

  \end{fmfgraph*}
\end{fmffile}
\end{center}
\strut

This means we have the conservation of momentum:
\begin{equation}
p_1 = p_3 + q.
\end{equation}
This corresponds to the delta function of
\begin{equation*}
(2\pi)^4\delta^{(4)}(p_1 - p_3 - q)
\end{equation*}
(i.e. the left vertex has momentum conserved). We multiply this into the integrand,
which becomes
\begin{equation}
(2\pi)^4(-ig)^2\frac{i}{q^2 - m^{2}_{C}c^2}\delta^{(4)}(p_1 - p_3 - q).
\end{equation}
We have another vertex too,
which we have the ``input momenta lines'' in red summed to have the same momentum 
as the ``output momenta lines'' in blue:

\strut

\begin{center}
\begin{fmffile}{exOneImg7}
  \begin{fmfgraph*}(50,25) \fmfpen{0.1mm}
%     \\fmfset{arrow_len}{4mm}\\fmfset{arrow_ang}{10}
    \fmfleft{i1,o1} % change i2->o1 
    \fmfright{i2,o2} % change o1->i2
    \fmflabel{$A,p_{1}$}{i1}
    \fmflabel{$B,p_{3}$}{o1} %
    \fmflabel{$A,p_{2}$}{i2} %
    \fmflabel{$B,p_{3}$}{o2}
    \fmf{plain,fore=red}{i2,v2} %
    \fmf{plain,fore=blue}{v2,o2}
    \fmf{plain}{i1,v1,o1} %
    \fmf{plain,label=$C,,q$,fore=red}{v1,v2}    

  \end{fmfgraph*}
\end{fmffile}
\end{center}
\strut

This corresponds to the conservation of momentum
\begin{equation}
p_2 + q = p_4\Rightarrow p_2 + q - p_4 = 0
\end{equation}
and this corresponds to a delta function of
\begin{equation*}
(2\pi)^4\delta^{(4)}(p_{2}+q-p_{4})
\end{equation*}
(i.e. the right vertex has momentum conserved). The integrand becomes
\begin{equation}
(2\pi)^8(-ig)^2\frac{i}{q^2 - m^{2}_{C}c^2}\delta^{(4)}(p_1 - p_3 - q)\delta^{(4)}(p_{2}+q-p_{4}).
\end{equation}

\textbf{Step Five:} We integrate over the internal lines, luckily we only have one! We have one integration thus one term
\begin{equation*}
\frac{1}{(2\pi)^4}d^{4}q.
\end{equation*}
Combining rules 1 through 5 gives us the final expression
\begin{equation}
-i(2\pi)^4g^2\int\frac{1}{q^2-m_{C}^2c^2}\delta^{(4)}(p_1 - p_3 - q)\delta^{(4)}(p_{2}+q-p_{4}).
\end{equation}
The second delta function serves to pick out the value of everything else at the
point $q=p_4-p_2$, so we have
\begin{equation}
-ig^2\frac{1}{(p_4-p_2)^2-m_{C}^2c^2}(2\pi)^4\delta^{(4)}(p_1+p_2-p_3-p_4).
\end{equation}
And we have one last delta function which tells us that we conserved the
overall energy and momentum. We erase it by rule 6 and we get the amplitude for
this particular process to be
\begin{equation}
\mathcal{M} = \frac{g^2}{(p_4-p_2)^2 - m_{C}^2c^2}.
\end{equation}
This particular process is called a ``\textbf{tree diagram}'' because we
do not have any internal loops. Lets consider such an example next.
