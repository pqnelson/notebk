\subsection{The Rules to the Game}

So we want to calculate out the probability amplitude $\mathcal{M}$ associated
with a particular Feynman diagram, we proceed as follows:
\begin{enumerate}
\item{(Notation)} We must be more careful here! We label the incoming and 
outgoing four-momenta $p_1$, $p_2$, \ldots, $p_n$ and the corresponding spins
$s_1$, $s_2$, \ldots, $s_n$. We label the inernal four-momenta $q_1$, $q_2$, \ldots.
Assign arrows to the lines as follows: the arrows on \emph{external} fermion lines 
indicates whether it is an electron or positron (if the arrow points forward in 
time, it is an electron; backwards in time it is a positron); arrows on
\emph{internal} fermion lines are assigned so that the ``direction of the flow''
through the diagram is preserved (i.e. every vertex must have at least one arrow
entering and one arrow leaving). The arrows on photon lines (which is optional,
since arrows are used to indicate whether the particle is an antiparticle or not;
bosons are their own antipartners) point ``forward'' in time.

\item{(External Lines)} External lines contribute factors as follows:
\begin{equation*}
\mbox{Electrons: } \begin{cases} \mbox{Incoming: } u\\
\mbox{Outgoing: }\bar{u} \end{cases}
\end{equation*}
\begin{equation*}
\mbox{Positrons: } \begin{cases} \mbox{Incoming: } \bar{v}\\
\mbox{Outgoing: }v\end{cases}
\end{equation*}
\begin{equation*}
\mbox{Photons: } \begin{cases} \mbox{Incoming: }\epsilon^\mu \\
\mbox{Outgoing: }(\epsilon^\mu)^* \end{cases}
\end{equation*}

\item{(Vertex Factors)} Each vertex contributes a factor
\begin{equation}
ig_{e}\gamma^\mu
\end{equation}
The dimensionless coupling constant $g_e$ is related to the charge of the
positron $g_e$ = $e\sqrt{4\pi/\hbar c}$ = $\sqrt{4\pi\alpha_E}$\footnote{Here
$\alpha_E$ is the coupling constant of the electromagnetic force. In \emph{general},
the QED coupling is $-q\sqrt{4\pi/\hbar c}$ where $q$ is the charge of the \emph{particle}
(as opposed to antiparticle). For electrons $q=-e$, for an up quark $q=(2/3)e$.}
\item{(Propagators)} Each internal line contributes a factor as follows
\begin{equation}
\mbox{Electrons and Positrons: }\frac{i\gamma^\mu q_\mu + mc}{q^2 - m^2c^2}
\end{equation}
\begin{equation}
\mbox{Photons: }\frac{-ig_{\mu\nu}}{q^2}
\end{equation}
\item{(Conservation of Energy and Momentum)} For each vertex, write a delta 
function of the form
\begin{equation}
(2\pi)^4\delta^{(4)}(k_1 + k_2 + k_3)
\end{equation}
This enforces the conservation of momentum at the vertex.
\item{(Integrate Over Internal Momenta)} For each internal momentum $q$, write
a factor
\begin{equation}
\frac{d^4 q}{(2\pi)^4}
\end{equation}
and integrate.
\item{(Cancel the Delta Function)} The result will include a factor
\begin{equation}
(2\pi)^4\delta^{(4)}(p_1 + p_2 + \cdots - p_n)
\end{equation}
which corresponds to the overall energy-momentum conservation. Cancel this factor,
and we get $-i\mathcal{M}$.
\item{(Antisymmetrization)} Include a minus sign between diagrams that differ
only in the interchange of two incoming (or outgoing) electrons (or positrons),
or of an incoming electron with an outgoing positron (or vice versa).
\end{enumerate}
