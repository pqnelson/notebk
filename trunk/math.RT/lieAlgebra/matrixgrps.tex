\section{Matrix Groups}

If the reader is unfamiliar with either group theory or matrices, it is advised 
to refer to the appropriate appendix first. Whenever we discuss matrices, it is
assumed to be a square matrix.

Allow us to first introduce the \emph{general linear group}. If we have two 
matrices which are invertible $A$ and $B$, then their product is invertible.
We know this because if $\det(A)\neq 0$ then $A$ is invertible, and
\begin{equation}
\det(AB)=\det(A)\det(B).
\end{equation}
Thus the only way that $AB$ is not invertible would be if
\begin{equation}\label{necessaryConditionInvertibility}
\det(AB) = 0
\end{equation}
but
\begin{equation}
\det(A)\neq 0 \quad\textrm{and}\quad\det(B)\neq 0
\end{equation}
which means that Eq (\ref{necessaryConditionInvertibility}) is no longer true.
Thus we have a contradiction.

We therefore conclude the product of two invertible matrices is invertible. We
have a group! The identity element is the identity matrix, and the binary
operation is matrix multiplication. This group is called the \textbf{general
linear group} of $n\times n$ matrices over the field $\mathbb{F}$, or more
succinctly just $GL_{n}(\mathbb{F})$. We usually have $\mathbb{F}=\mathbb{R}$ or
$\mathbb{C}$.

\marginpar{Meaning of ``special''}We can then normalize the matrices by a simple
trick. If we have an $n\times n$ matrix $A$ with a determinant not equal to one,
we can simply map it to
\begin{equation}
A\mapsto\frac{1}{\sqrt[n]{\det(A)}}A
\end{equation}
which preserves the property of unit determinant. The subgroup of $GL_{n}(\mathbb{F})$
with this extra condition is called the \textbf{special linear group} and denoted
$SL_{n}(\mathbb{F})$.

The next property of matrices that we can investigate would be orthogonality. 
That is, when we diagonalize a matrix $X$, we end up multiplying it by $AXA^{-1}$.
The property of orthogonal matrices that we are interested in simply is
\begin{equation}
A^\textrm{T} = A^{-1}.
\end{equation}
The subgroup of $GL_{n}(\mathbb{F})$ that satisfies such a property is the
\textbf{orthogonal group} denoted as $O_{n}(\mathbb{F})$. And similarly, the
subgroup of $SL_{n}(\mathbb{F})$ that satisfies the orthogonality property is
the \textbf{special orthogonal group} denoted $SO_{n}(\mathbb{F})$.

The favorite property of quantum physicists, unitarity (``self-adjointness''), 
also forms a group. The basic property is this, for an $n\times n$ matrix $A$
with complex entries, we have
\begin{equation}
\bar{A}^\textrm{T} = A^{-1}
\end{equation}
the complex conjugate of the transpose is the inverse of $A$. This forms the
\textbf{unitary group} denoted as $U_{n}(\mathbb{C})$. We can form the 
\textbf{special unitary group} by the normalization routine outlined above, and
we denote this group as $SU_{n}(\mathbb{C})$. 

The most bizarre condition on a $2n\times 2n$ matrix $M$ (note even dimensions!) is that of being symplectic. What does it mean anyways? It means that $M$ satisfies
\begin{equation}
M^\textrm{T} \Omega M = \Omega
\end{equation}
where (typically)
\begin{equation}\label{usualSuspect}
\Omega = \begin{bmatrix} 0 & I_n \\ -I_n & 0 \\ \end{bmatrix}.
\end{equation} 
What is the consequenec of this? Well, it's a bit stronger than orthogonality because
\begin{equation}
M^{-1} = \Omega^{-1}M^\textrm{T}\Omega
\end{equation}
holds. We can form $M$ into block matrix form where
\begin{equation}
M = \begin{bmatrix}A & B \\ C & D\end{bmatrix}
\end{equation}
where $A,B,C,D$ are all $n\times n$ matrices satisfying the following properties:
\begin{subequations}
\begin{align}
A^\textrm{T}D - C^\textrm{T}B &= I \\
A^\textrm{T}C &= C^\textrm{T}A \\
D^\textrm{T}B &= B^\textrm{T}D.
\end{align}
\end{subequations}
The symplectic property of matrices is not as well known as the orthogonality condition or self-adjointness.

We have this neat little table of matrix groups:
\begin{center}
  \begin{tabular}{ | l | l | }
\hline
$GL_{n}(\mathbb{R})$ & all invertible matrices with real entries \\ \hline
$SL_{n}(\mathbb{R})$ & all matrices in $GL_{n}(\mathbb{R})$ with determinant 1 \\ \hline
$O_{n}(\mathbb{R})$ & all matrices with their inverse equal to their transpose \\ \hline
$SO_{n}(\mathbb{R})$ & all matrices in $O_{n}(\mathbb{R})$ with determinant 1 \\ \hline
$U_{n}$ & all unitary $n\times n$ matrices \\ \hline
$SU_{n}$ & all matrices in $SL_{n}(\mathbb{C})$ and in $U_{n}$. \\ \hline
$SP_{2n}(\mathbb{R})$ & $P\in GL_{2n}(\mathbb{R})$ such that $P^\textrm{T}JP = J$\\
 & for a given antisymmetric matrix $J$ (usually the one in Eq (\ref{usualSuspect})) \\ \hline
  \end{tabular}
\end{center} 
