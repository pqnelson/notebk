%%
%% internalization.tex
%% 
%% Made by alex
%% Login   <alex@tomato>
%% 
%% Started on  Sun Feb 19 17:12:21 2012 alex
%% Last update Sun Feb 19 17:12:21 2012 alex
%%
\section{Internalization}

\begin{quote}
When you are collecting mushrooms, you only see the mushroom
itself. But if you are a mycologist, you know that the real
mushroom is in the Earth. There's an enormous thing down there,
and you just see the fruit, the body that you eat. In
mathematics, the upper part of the mushroom corresponds to the
theorems that you see, but you don't see the things that are
below, that is: \emph{problems, conjectures, mistakes, ideas},
etc. --- V.~I.~Arnold~\cite{arnol2004hilbert}
\end{quote}

We will try several attempts to internalize the notion of a
pseudogroup of transformations. It may take several attempts,
leading to wrong turns, wrong guesses, etc. 

Recall, given a topological space $X$ (or more generally any set
$X$), we can construct a category $\mathcal{O}(X)$ whose objects
are open subsets of $X$ and morphisms are restrictions and
inclusions. We will need to use this to construct the notion of a
pseudogroup of transformations. Weinstein notes~\cite{weinstein1996groupoids}
that germs of elements of a pseudogroup form a groupoid.

\begin{defn}
A \define{Pseudogroup} on a topological space $X$ consists of a
groupoid $G$ whose
\vspace{-.75pc}
\begin{description*}
\item[Objects:] Open subsets of $X$
\item[Morphisms:] Homeomorphisms between those subsets; 
\end{description*}
\vspace{-.75pc}
satisfying
\vspace{-.75pc}
\begin{description*}
\item[Covering Property:] The objects cover $X$.
\item[Restriction Property:] If $g\colon V\to W$ is a morphism in
  $G$, and $U\subset V$, then the restriction $g|_{U}\colon U\to
  g(U)$ belongs to $G$. 
\item[Sheaf Property:] Let $g\colon U\to V$ be a homeomorphism, and if
  there is a covering $\{U_{\alpha}\}$ of $U$ such that the
  restrictions $g|_{U_{\alpha}}\colon U_{\alpha}\to
  g(U_{\alpha})$ are morphisms in $G$, then $g$ is also a
  morphism of $G$.
\end{description*}
\end{defn}

How do we use a pseudogroup? Well, we pick some ``nice space''
$X$ (usually $X=\RR^n$), and we consider its pseudogroup
$G$. Then maybe we want to study a space $M$ which ``locally
looks'' like $X$, so for an open subset $U\subset M$ we have an
embedding
\begin{equation}
\phi\colon U\to X.
\end{equation}
This pair $(\phi,U)$ gives us a \define{Chart}. We say two charts 
\begin{equation}
\phi\colon U\to X,\quad\mbox{and}\quad\psi\colon V\to X
\end{equation}
are \define{Compatible} (a property!) if and only if the
transition function
\begin{equation}
\psi\circ\phi^{-1}\colon \phi(U\cap V)\to\psi(U\cap V)
\end{equation}
belongs to $G$. So pseudogroups are useful for determining
compatibility of charts (usually for manifolds), and we consider
pseudogroups on the ``model space''.

\begin{xca}\label{xca:L202pi}
Consider $L^{2}(0,2\pi)$. What is its pseudogroup?
\end{xca}
\begin{xca}
Consider $\CC[x]$ equipped with the Zariski topology. What is its pseudogroup?
\end{xca}
\begin{xca}
Working in the category $\Top$ of topological spaces, consider
the function space $\hom(X,Y)$. What is its pseudogroup? If we
consider $\hom(S^{1},Y)$, does its pseudogroup relate to the one
investigated in exercise \ref{xca:L202pi}?
\end{xca}

Observe that the approach taken is to consider what the space
``locally looks like''. The pseudogroup is some extra stuff
assigned to the model space. On the other hand, we could go in
the opposite way: gluing together a bunch of patches to assemble
our space. The modern approach to algebraic geometry does this
using schemes, which are glorified ringed spaces.

\begin{rmk}[On Ringed Spaces and Schemes]
A different-yet-related idea is also captured in the notion of a \emph{Ringed Space},
where we have a topological space $X$ as a category (induced from
its poset structure), and a sheaf $\mathcal{O}_{X}$ which assigns
to each open subset of $X$ a ring. The intuition is that $X$ is
the ``model space'' and the ring assigned is the continuous real
functions (or smooth real functions, or holomorphic complex functions, or
\dots). We patch together a bunch of ringed spaces into a bigger
space, just as we patch together charts into a manifold. This
``composite space'' is called a \emph{Scheme}. Vakil~\cite{vakil}
reviews this in some modest detail. 
\end{rmk}
\begin{rmk}[Schemes are Good!]
The notion of a scheme is a straight-forward generalization of a
manifold. In fact, any manifold can be turned into a scheme (but
not every scheme is a manifold!).
\end{rmk}

\begin{rmk}[Internalization]
Now, we should remember that we \emph{can} internalize a groupoid
in various nice categories. This would suggest the notion of a
pseudogroup can likewise be internalized. However, after much
reflection, it seems altogether the wrong approach towards
generalizing geometry for a variety of reasons (most of them
emotional). 
\end{rmk}
