%%
%% lecture05.tex
%% 
%% Made by alex
%% Login   <alex@tomato>
%% 
%% Started on  Thu Jan 19 09:27:40 2012 alex
%% Last update Thu Jan 19 09:27:40 2012 alex
%%
We introduced the notion of cellular homology and considered a
single example. Lets consider more examples.
%\begin{ex}[Sphere]

\medbreak
\phantomsection\refstepcounter{thm}
\noindent\textbf{Example \thethm\enspace}(Sphere)\textbf{.}\quad\ignorespaces %
Lets consider the $n$-dimensional sphere $S^n$. Observe, if we
delete one point, we have
\begin{equation}
S^n-\{x\}\iso\sigma^{n}
\end{equation}
``is'' Euclidean space, or an $n$-dimensional ball. But
\begin{equation}
\{x\}=\sigma^0
\end{equation}
So we have two cells:
\begin{equation}
S^{n}=\sigma^0\cup\sigma^n.
\end{equation}
So
\begin{equation}
\bdry\sigma^0=0
\end{equation}
and
\begin{equation}
\bdry\sigma^n=0.
\end{equation}
%\end{ex}

\begin{wrapfigure}{r}{4pc}
  \centering
  \includegraphics{B.img/lecture05.0}
\end{wrapfigure}
\noindent\ignorespaces %
We can picture this for a circle; so everything is a cycle,
nothing is a boundary. So $C_n=Z_n=G$ is the group of
coefficients. This immediately implies $H_0(S^n,G)\iso H_n(S^n,G)\iso G$.
If we consider the reduced homology we find $\widetilde{H}_{0}(S^n,G)=0$
and $\widetilde{H}_{n}(S^{n},G)\iso G$.
\medbreak

\begin{wrapfigure}{r}{14pc}
  \vspace{-12pt}
  \centering
  \includegraphics{B.img/lecture05.1}
  \vspace{-12pt}
\end{wrapfigure}

\medbreak
\phantomsection\refstepcounter{thm}\noindent\textbf{Example \thethm}
(Sphere with different cell complex)\textbf{.}\quad\ignorespaces %
To prove that the choice of cellular decomposition doesn't
matter, lets consider a different decomposition for $S^n$.
Lets work with $S^{2}$. Take a sphere, divide it into
hemispheres, and take two points. We have two cells in every
dimension! (This is doodled on the right.) We orient it in such 
a way that the boundary of $\sigma^{2}_{+}$ is an oriented circle.
We have
\begin{subequations}
\begin{equation}
\bdry\sigma^{2}_{\pm}=\pm(\sigma^{1}_{+}-\sigma^{1}_{-})
\end{equation}
and
\begin{equation}
\bdry\sigma^{1}_{\pm}=\pm(\sigma^{0}_{+}-\sigma^{0}_{-}).
\end{equation}
\end{subequations}
It follows immediately we obtain
\begin{equation}
\begin{split}
\bdry(\bdry\sigma^2_{\pm})
&=\pm\bdry(\sigma^{1}_{+}-\sigma^{1}_{-})\\
&=0
\end{split}
\end{equation}
as desired.
Lets look at the homology. Everything is very, very
simple. Consider
\begin{equation}
Z_{2} = \{g_{+}+g_{-}=0\mid g_{\pm}\in G\}
\end{equation}
and 
\begin{equation}
Z_{1} =
\{\widetilde{g}_{+}+\widetilde{g}_{-}=0\mid\widetilde{g}_{\pm}\in G\}.
\end{equation}
We have
\begin{equation}
H_{2}=Z_{2}\iso G
\end{equation}
since there are no boundaries. But for $H_{1}$, we \emph{do} have
boundaries. Here we have
\begin{equation}
Z_{1}=B_{1}
\end{equation}
so
\begin{equation}
H_{1}=0.
\end{equation}
This is because
\begin{equation}
\im(\bdry_2)=\{g_{+}+g_{-}\}\iso G
\end{equation}
and $Z_1\iso G$. So we see that the choice of cell complex
doesn't matter, we get the same answer for the sphere.

\begin{ex}
Consider projective space $\RP^n$. It has a very simple cell
decomposition. Remember we may define it in many ways, e.g., take
$(n+1)$ coordinates in $\RR^{n+1}$ and we identify
$(x_0,\dots,x_n)$ with $(\lambda x_0,\dots,\lambda x_n)$ where
$\lambda\in\RR$ is some (nonzero) scaling constant. We can scaled
to make $x_0=1$ provided $x_0\not=0$. Then multiplication by
$\lambda$ ``disappears''---i.e., we fixed
$\lambda=x_{0}^{-1}$. So we now have $n$ coordinates
$(x_1,\dots,x_n)$ and they are arbitrary. This describes a cell
$\sigma^n$. If we take $x_0=0$ we may forget about $x_0$,
what remains is $\RP^{n-1}$. So
\begin{equation}
\RP^n=\RP^{n-1}\sqcup\sigma^{n}
\end{equation}
disjoint. We may repeat it, and we have the decomposition
\begin{equation}
\RP^n=\sigma^0\sqcup\dots\sqcup\sigma^n
\end{equation}
(disjoint unions). All these considerations work for \emph{any}
projective space. So for 
\begin{equation}
\CP^n=\sigma^{2n}\sqcup\CP^{n-1}
\end{equation}
since $\CC\iso\RR^2$. A decomposition for $\CP^n$ would be
\begin{equation}
\CP^n=\sigma^0\sqcup\dots\sqcup\sigma^{2n}.
\end{equation}
For complex projective space, all boundaries vanish and
homologies are extremely simple: everything is a cycle!

It is not so simple for the case of real projective spaces. Lets
consider $\RP^2$. We considered it already, so we know the
answer. We have this picture of $S^2$ for the cell complex, and
it is invariant with respect to the transformation
$x\mapsto-x$. We know
\begin{equation}
\RP^{n}=S^{n}/(x\sim-x)
\end{equation}
So we use this identification procedure with the cell complex we
identified for $S^2$, and we can use the relations we
discovered. The answer is 
\begin{equation}
\bdry\sigma^k=0
\end{equation}
for $k$ odd, and
\begin{equation}
\bdry\sigma^{2k}=2\sigma^{2k-1}.
\end{equation}
With orientations, what do we know? Well, for linear
transformations $A\vec{x}=\vec{b}$, if $\det(A)>0$ then the
orientation is preserved. For nonlinear transformations, we look
at the determinant of the Jacobian matrix.

Now what are the homology groups? Again, we computed this stuff,
and our current calculations are precisely the same. When we
don't have any cells, we don't have any homology. Then we have
two cases.
%
If $n$ is odd for $\RP^n$ then $\bdry\sigma^n=0$ and $H_n\iso G$.
%
On the other hand, if $n$ is even, then
$\bdry\sigma^n=2\sigma^{n-1}$ and we see that
\begin{equation}
H_{n}(\RP^{n},G)=\ker\left(G\xrightarrow{x\mapsto2x}G\right)
\end{equation}
So in particular
\begin{equation}
H_{n}(\RP^n,\ZZ)=0
\end{equation}
What about the intermediate dimensions? We have the following
picture for $2k<n$
\begin{equation}
H_{2k}(\RP^n,G)=Z_{2k}/B_{2k}=\ker\left(G\xrightarrow{x\mapsto2x}G\right)
\end{equation}
and for $2k+1<n$
\begin{equation}
H_{2k+1}(\RP^n,G)=G/2G.
\end{equation}
That concludes the calculations of homology for $\RP^n$.
\end{ex}


\begin{wrapfigure}{r}{5pc}
  \vspace{-12pt}
  \centering
  \includegraphics{B.img/lecture05.2}
\end{wrapfigure}
\medbreak
\phantomsection\refstepcounter{thm}
\noindent\textbf{Example \thethm\enspace}(M\"obius Band)\textbf{.}\quad\ignorespaces %
Consider the cell complex as doodled on the right. There is one
2-cell, and so on. This is the cell complex describing the
M\"obius band. From this picture we see that
$\bdry\sigma^2=0$. If we had


