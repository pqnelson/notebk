%%
%% lecture04.tex
%% 
%% Made by alex
%% Login   <alex@tomato>
%% 
%% Started on  Thu Jan 19 08:11:14 2012 alex
%% Last update Thu Jan 19 08:11:14 2012 alex
%%
We introduced the ntoion of homology groups $H_{i}(X)$ which are
based on singular cubes. We considered maps $f\colon X\to Y$
which generates a morphism of homology
\begin{equation}
f_{*}\colon H_{k}(X)\to H_{k}(Y).
\end{equation}
If we have
\begin{subequations}
\begin{equation}
f,g\colon X\to Y,\quad\mbox{and}\quad f\homotopic g
\end{equation}
homotopic, and specifically if
\begin{equation}
g\circ f\homotopic\id{X}
\end{equation}
homotopic, then
\begin{equation}
g_{*}\circ f_{*}=\id{H_{k}(X)}
\end{equation}
and moreover this implies
\begin{equation}
f_{*}\circ g_{*}=\id{H_{k}(Y)}.
\end{equation}
\end{subequations}
This means homology groups are homotopic invariant.\index{Homology!is-a Homotopic Invariant}

We computed various homologies. Recall for
\begin{equation}
H_{0}(X) = \underbrace{\ZZ\oplus\dots\oplus\ZZ}_{\mathclap{\text{number of components}}}
\end{equation}
and
\begin{equation}
H_{1}(X)=\pi_{1}(X)/\bigl[\pi_{1}(X),\pi_{1}(X)\bigr]
\end{equation}
is the ``Abelianization'' of the Fundamental group.

We will compute the homology of any convex set. It is easy, every
convex set is \emph{contractible!} So every convex set $X$ is
then
\begin{equation}
X\homotopic\{\mathrm{pt}\}
\end{equation}
homotopic to a point. We should compute the homology of a
\emph{point!} But that's not too difficult, we see
\begin{subequations}
\begin{equation}
H_{0}(X)=\ZZ
\end{equation}
and
\begin{equation}
H_{k}(X)=0
\end{equation}
\end{subequations}
for $k>0$. The chain groups $C_{k}(X)$ are also vanishing for
$k>0$. Thus any convex set (a ball, a cube, a Euclidean space)
are cyclic: \emph{NO HOMOLOGY!} This isn't very convenient. What
should we do? We define what is called the \define{Reduced Homology}%
\index{Homology!Reduced|textbf}\index{Reduced Homology|textbf}
$\widetilde{H}_{0}(X)$. We have in general $H_{0}(X)=\ZZ^n$ where
$n$ is the number of components; thus we define
\begin{equation}
\widetilde{H}_{0}(X)=\{(z_1,\dots,z_n)\in\ZZ^n\mid z_1+\dots+z_n=0\}.
\end{equation}
So this implies
\begin{subequations}
\begin{equation}
H_{0}(X)=\ZZ^n\implies\widetilde{H}_{0}(X)\iso\ZZ^{n-1}
\end{equation}
and
\begin{equation}
H_{k}(X)\iso\widetilde{H}_{k}(X)
\end{equation}
\end{subequations}
for $k>0$. (We may repeat this definition for
$\widetilde{H}_{k}(X,G)$ using a different group of coefficients
$G$.)

\subsection{Cellular Homology}\index{Homology!Cellular}
The notion of singular homology is very nice but has the
disadvantage of being hard to calculate. Remember cell
complexes. We may define a notion of homology for cell complexes
which is much simpler. This notion coincides with singular
homology, we'll prove this later.

Lets recall the notion of a cell complex. It is the union of c
ells. So
\begin{equation}
X=\bigcup_{i,k}{\sigma_{i}}^{k}
\end{equation}
where $k$ refers to the dimension, and $i$ refers to teh cell. We
construct it by induction
\begin{equation}
X=\bigcup_{k}X_{k}
\end{equation}
where
\begin{equation}
X_{k}=\bigcup_{i=0}^{k-1}X_{i}
\end{equation}
is the $k$-skeleton. We take $D^{k+1}\propersupset S^k$, now we
take a map
\begin{equation}
f\colon S^k\to X_k
\end{equation}
and we paste $D^{k+1}$ to $X_{k}$ via this map. We take
\begin{equation}
X_{k}\sqcup D^{k+1}/f(\bdry D^{k+1})\simeq \bdry D^{k+1}
\end{equation}
that is, we identify $x\simeq f(x)$ for any $x\in\bdry
D^{k+1}$. We may attach any number of cells by means of this
construction. 

If we consider $X_{k}$ and factorize $X_{k}/X_{k-1}$, this will
be the wedge sum
\begin{equation}
X_{k}/X_{k-1}=S^{k}_{(1)}\wedgeSum\dots\wedgeSum S^{k}_{(\alpha_{k})}
\end{equation}
where $\alpha_{k}$ is the number of $k$-cells. What we do is we
contract $X_{k-1}$ to a point, so
\begin{equation}
X_{k}/X_{k-1}=\left.\left(\bigsqcup\mbox{open balls}\right)\right/\begin{pmatrix}
\mbox{common}\\\mbox{point}
\end{pmatrix}.
\end{equation}
We have this map
\begin{equation}
S^{k}\xrightarrow{f}X_{k}\to X_{k}/X_{k-1}=S^{k}_{(1)}\wedgeSum\dots\wedgeSum S^{k}_{(\alpha_{k})}.
\end{equation}
This map $X_{k}\to X_{k}/X_{k-1}$ is called the
\define{Identification Map}\index{Identification Map}\index{Map!Identification ---}.
We consider
\begin{equation}
S^{k}\xrightarrow{f}X_{k}\to X_{k}/X_{k-1}\xrightarrow{\lambda_{i}}S^{k}
\end{equation}
where $\lambda_{i}$ keeps $S^{k}_{(i)}$ ``in tact'' while it
contracts all others to the common point. But observe: we obtain
a map
\begin{equation}
S^{k}\to S^{k}
\end{equation}
by composition! We know that $\homotopyClass(S^k,S^k)$ are
classified by $\ZZ$ called the
\define{Degree}\index{Degree}. Lets denote this number
$m_{i}\in\ZZ$, where $0\leq i\leq\alpha_{k}$ runs over the
different $k$-cells. But this also depends on
\begin{equation}
f\colon S^{k}\to X_{k}
\end{equation}
which is really the $(k+1)$-cells. So ${m_{i}}^{j}\in\ZZ$ where
$i=0,\dots,\alpha_{k}$ and $j=0,\dots,\alpha_{k+1}$.
What is the geometric meaning of this number? It describes the
number of times a given $k$-cell appears in the boundary of a
given $(k+1)$-cell.

Now, we should note we have been a little bit sloppy here. The
degree of the map depends on the \emph{orientation} of the
sphere. There is a bit of pathology here.

We will define the notion of a  cellular
chain\index{Chain!Cellular ---|textbf}\index{Cell Complex!Chain Group of ---}
as
\begin{equation}
C^{\text{cell}}_{k}(X) = \{\sum_{j}c_{j}{\sigma_{j}}^{k}\mid c_{j}\in\ZZ\}
\end{equation}
where ${\sigma_{j}}^{k}$ are oriented $k$-cells. We may interpret
very cellular cell as a singular cell, because for every cell we
have a mapping
\begin{equation}
D^{k}\to\overline{\sigma^{k}},
\end{equation}
i.e., the closure of every cell is a singular cube. For every
cellular chain, we may construct a singular chain (``by linearity'').

Now, we may construct for $k$-chains a map called the
\define{Boundary}\index{Boundary!of $k$-chain}\index{Chain!Cellular!Boundary}
\begin{equation}
\bdry\colon C^{\text{cell}}_{k+1}(X)\to C^{\text{cell}}_{k}(X).
\end{equation}
How to do this? We consider $\bdry\sigma^{k+1}$, extend
$\bdry$ by linearity. Here this will be given by
\begin{equation}
\bdry\sigma^{k+1}=\sum m_{i}{\sigma_{i}}^{k}.
\end{equation}
It's a triviality.

Until now, everything was rigorous. Well, almost everything! Now
we do not want to prove $\bdry^2=0$ but it is both true and obvious.
It will follow from another definition of the boundary. But if it
is true, then we have a notion of homology given by
\begin{equation}
H^{\text{cell}}_{k}(X) = \ker(\bdry)/\im(\bdry)
\end{equation}
which is called the \define{Cellular Homology}\index{Homology!Cellular|textbf}
\index{Cellular Homology|see{Homology}} which allows us to write
\begin{equation}
H^{\text{cell}}_{k}(X)=Z^{\text{cell}}_{k}(X)/B^{\text{cell}}_{k}(X)
\end{equation}
as usual.

Now we should calculate something.\marginpar{Example: cellular homology of $\RP^2$} The calculations are very
simple. The simplest is to calculate the homology of a
surface. We can calculate the homology of the projective 

\begin{wrapfigure}{r}{6pc}
  \vspace{-12pt}
  \centering
  \includegraphics{B.img/lecture04.0}
\end{wrapfigure}
\noindent\ignorespaces %
plane,
which is given by a disc and identifying the two semicircles as
explained. 
This is doodled on the right.
The cell complex is obvious: we have two vertices but they are
identified as the same. We have two edges, but again they are
identified. We have a single 2-cell. So we have $\sigma^0$,
$\sigma^1$, and $\sigma^2$. We have
\begin{equation}
\bdry\sigma^0=0
\end{equation}
trivially, and
\begin{equation}
\begin{split}
\bdry\sigma^1
&=\sigma^0-\sigma^0\\
&=0
\end{split}
\end{equation}
and lastly
\begin{equation}
\begin{split}
\bdry\sigma^2
&=\sigma^1+\sigma^1\\
&=2\sigma^1.
\end{split}
\end{equation}
We can compute the homologies quickly
\begin{equation}
H_{0}(\RP^2)=\ZZ
\end{equation}
since there is a single connected component. Similarly, we have
\begin{equation}
H_{1}(\RP^{2})=\ZZ_{2}
\end{equation}
since all 1-chains are cycles, i.e.,
\begin{equation}
Z_{1}(\RP^2)=\ZZ\quad\mbox{but}\quad B_{1}(\RP^2)=2\ZZ,
\end{equation}
which implies
\begin{equation}
H_{1}(\RP^2)=\ZZ/2\ZZ=\ZZ_{2}.
\end{equation}
Last, we deduce
\begin{equation}
H_{2}(\RP^2)=0.
\end{equation}
More generally, if $G$ is any Abelian group, then
\begin{equation}
H_{0}(\RP^2,G)\iso G
\end{equation}
We also find
\begin{equation}
Z_{1}(\RP^{2},G)\iso G,\quad\mbox{and}\quad
B_{1}(\RP^{2},G)\iso 2G.
\end{equation}
Thus
\begin{equation}
H_{1}(\RP^2,G)\iso G/2G
\end{equation}
and what about $H_{2}(\RP^2,G)$? Well, we see
\begin{equation}
H_{2}(\RP^2,G)=\ker\left(G\xrightarrow{x\mapsto 2x}G\right)
\end{equation}
which is not necessarily trivial (hint: think of $G=\ZZ_{2}$).
