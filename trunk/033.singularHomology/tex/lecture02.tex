%%
%% lecture02.tex
%% 
%% Made by alex
%% Login   <alex@tomato>
%% 
%% Started on  Wed Jan  4 11:54:16 2012 alex
%% Last update Wed Jan  4 11:54:16 2012 alex
%%
We will always denote the unit cube with $I^k$. Let $X$ be a
topological space, a \define{Singular ($k$-Dimensional) Cube}\index{Singular Cube}\index{Cube!Singular} in $X$ consists of a
mapping
\begin{equation}
\varphi\colon I^k\to X.
\end{equation}
A \define{$k$-Dimensional (Singular) Chain}\index{Singular Chain}\index{Chain!Singular} is 
a linear combination of singular $k$-cubes
\begin{equation}
\gamma=\sum_{n}c_{n}\varphi_{n}
\end{equation}
which intuitively corresponds to a surface\index{Surface!as-a Chain} in $X$.
We can generalize this a little, by letting $c_n\in G$ where $G$
is an arbitrary Abelian group. This is sometimes convenient, the
surfaces come with an orientation\index{Orientation!and Chain Coefficients}. 
For a nonorientable surfaces\index{Surface!Nonorientable}, we
take $G=\ZZ_2$.

There is on more definition, namely, the \define{Degenerate Cube}\index{Cube!Degenerate}\index{Degenerate Cube}.
Now, a cube is a function of $k$ variables, but sometimes it
doesn't depend on certain variables. For example, if 
\begin{equation}
f\colon I^2\to X
\end{equation}
is such that $f(x,-)$ is independent of the second slot. We get a
curve. We should ignore chains with degenerate cubes.
We define $C_{n}(X,G)$\marginpar{$C_{n}(X,G)$ is the group of nondegenerate $n$-chains},
and let $C_{n}(X)$\index{$C_{n}(X)$} be the case when $G=\ZZ$. We take the group of
chains (pop quiz: do they form a group?), and we factorize all
chains with respect to degenerate cubes. We get
\begin{equation}
C_{n}(X,G) = \mbox{chains}/\mbox{(Degenerate Chains)}.
\end{equation}
The interesting question to ask is: what is the
boundary\index{Boundary!of a Chain}\index{Chain!Boundary of a ---} of a chain?

\index{Boundary!of a Cube|(}\index{Cube!Boundary of a ---|(}
Lets look at the boundary for a cube $\partial I^n$\index{$\partial I^n$}. Lets look at simpler examples
first. What is $\partial I$? It consists of two points, it is a
zero-dimensional chain because each point is a zero-dimensional
cube. We have
\begin{equation}
\partial I^1=I^{0}_{+}-I^{0}_{-}
\end{equation}
be our chain. But what about the singular cube case? We have
\begin{equation}
\partial\varphi=\left.\varphi\right|_{I^{0}_{+}}
-\left.\varphi\right|_{I^{0}_{-}}
\end{equation}
and (for example) if we have a closed path, there is no doubt
that the boundary is zero.

\begin{wrapfigure}{r}{8pc}
  \vspace{-12pt}
  \centering
  \includegraphics{B.img/lecture02.0}
  \vspace{-12pt}
\end{wrapfigure}
\noindent\ignorespaces %
Let us take $I^2$. It is a square, what is its boundary? It
consists of four lines, but again we have some freedom, because
we can choose orientations. We have two degrees of freedom (well,
two choices for orientations): clockwise or anti-clockwise. There
is another choice which is more algebraic in nature. 

Consider, for example, the $k$-dimensional cube. Its boundary
$\partial I^k$ has $2k$ faces, which are $(k-1)$-dimensional
cubes. We have $I^{k-1}_{i,+}$ and $I^{k-1}_{i,-}$ which are
described by
\begin{equation*}
(t_1,\dots,t_{i-1},t_{i+1},\dots,t_{k})
\end{equation*}
and the ordering depends on the orientation. Now, we should write
what is the boundary. Let us return to the case when $k=2$. It is
clear that now what we are doing amounts to orienting the edges,
the boundary, in a different way. For example:
\begin{center}
  \includegraphics{B.img/lecture02.1}
\end{center}
We cannot merely write the boundary as
\begin{equation}
\sum_{i}I^{k-1}_{i,+}-I^{k-1}_{i,-}\stackrel{??}{=}\partial I^{k},
\end{equation}
because we also need to take into account orientations. So it is
not the end of the story, we now have
\begin{equation}
\partial I^{k} =
\sum_{j}(-1)^{j}\bigl(I^{k-1}_{j,+}-I^{k-1}_{j,-}\bigr)
\end{equation}
Observe the boundary of the cube \emph{now} becomes a chain! 

\index{Singular Cube!Boundary of a ---|(}
But what about \emph{singular} cubes? It's the same stuff! For a
singular cube $\varphi\colon I^k\to X$, we have
\begin{equation}
\partial\varphi =
\sum_{j}(-1)^{j}\left(\left.\varphi\right|_{I^{k-1}_{j,+}}-\left.\varphi\right|_{I^{k-1}_{j,-}}\right).
\end{equation}
This is the definition of the boundary for a singular cube.
\index{Singular Cube!Boundary of a ---|)}
\index{Cube!Boundary of a ---|)}\index{Boundary!of a Cube|)}

We would like to consider the boundary of a chain. How to do
this? By linearity! We have
\begin{equation}
\bdry\left(\sum c_{n}\varphi_{n}\right)=\sum
c_{n}\bdry\varphi_{n};
\end{equation}
we will often extend notions from a cube to a chain by
linearity. At any rate, we get a homomorphism
\begin{equation}\index{$\bdry$}
\bdry\colon C_{k}(X)\to C_{k-1}(X)
\end{equation}
which is the boundary morphism\index{Boundary!Morphism}.
The main property of this morphism is that
\begin{equation}
\bdry^2=0.
\end{equation}
We will be a little more precise, and write
\begin{equation}
\bdry_k\colon C_{k}(X)\to C_{k-1}(X)
\end{equation}
then conclude that
\begin{equation}\index{$\bdry^2=0$}
\bdry_{k-1}\circ\bdry_{k}=0.
\end{equation}
We can do the following thing: construct the group of all chains
\begin{equation}\index{$C(X)$}
C(X)=\bigoplus C_{n}(X),
\end{equation}
and by additivity define the boundary there.
This is a \emph{graded} group. This operator
\begin{equation}
\bdry\colon C(X)\to C(X)
\end{equation}
which affects the grading by decreasing it by 1. So how to prove
$\bdry^2=0$? It's nothing difficult, so we won't give the
proof. After all, it is obvious!

We were cheating a little bit, because $C_{k}(X)$ does not
consist of chains. We obtain by means of factorization, so our
operator is really defined on this quotient. We should check that
it maps degenerate chains into degenerate chains (it does!).

Now the notion of homology. We have this operator
\begin{equation}
\bdry\colon C(X)\to C(X)
\end{equation}
and we define the \define{Homology}\index{Homology|textbf} as
\begin{equation}
H_{\bullet}=\ker(\bdry)/\im(\bdry).
\end{equation}
In\marginpar{$k$-Cycles $Z_{k}(X)=\ker(\bdry_{k})$} other words, we factorize cycles modulo boundaries. We can
give another definition of the $k$-dimensional homology.
We take
\begin{equation}
H_{k}(X)=\ker(\bdry_k)/\im(\bdry_{k+1})
\end{equation}
Remember that
\begin{equation}
\bdry_{k+1}\bigl(C_{k+1}(X)\bigr)\propersubset C_{k}(X)
\end{equation}
We usually write this as
\begin{equation}
H_{k}(X)=Z_{k}(X)/B_{k+1}(X).
\end{equation}
It is trivial to check we have
\begin{equation}
H_{\bullet}(X)=\bigoplus H_{k}(X).
\end{equation}
It is therefore pretty obvious that
\begin{equation}
(\mbox{Cycles}) = Z(X)=\bigoplus Z_{k}(X)
\end{equation}
and
\begin{equation}
(\mbox{Boundaries}) = B(X) = \bigoplus B_{k}(X).
\end{equation}
Note these are just notation.

Again a definition. Let us calculate something. We would like to
calculate $H_{0}(X)$. First of all, what is a zero-dimensional
chain? It is a linear combination of points. What are
zero-dimensional cycles? They are zero-dimensional chains! So
$Z_{0}=C_{0}$, but now what are the zero-dimensional boundaries?
They are boundaries of 1-dimensional chains. They consist of
several 1-dimensional cubes, i.e., a path.

Two paths are homologous if they are contained in the same
connected component. We may connect the chains in the components,
and the answer is as follows:
\begin{equation}
H_{0}(X)=\underbrace{\ZZ\oplus\dots\oplus\ZZ}_{\text{\# of components}}
\end{equation}
We may say, if $X$ is connected, then definitely
\begin{equation}
H_{0}(X)=\ZZ
\end{equation}
We may say every zero-dimensional cycle is an 
\begin{equation*}
\gamma = (\mbox{integer})(\mbox{point}).
\end{equation*}
But if $X$ is disconnected, if $X$ is the disjoint union of
several guys, then we have a general statement:
\begin{equation}
H_{k}(X) = \bigoplus H_{k}(\mbox{component}).
\end{equation}
This is true for \emph{any} $k\in\NN$. The disconnected
components do not speak to each other! Thus everything is reduced
to components.
