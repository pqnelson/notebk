%%
%% lecture08.tex
%% 
%% Made by alex
%% Login   <alex@tomato>
%% 
%% Started on  Thu Jan  5 08:09:15 2012 alex
%% Last update Thu Jan  5 08:09:15 2012 alex
%%









\exercises
\begin{xca}
Calculate relative homology $H_k(\RP^5, \RP^3)$ using cell
decomposition of $\RP^5$. (The projective space $\RP^3$ is
embedded into $\RP^5$ in standard way.) Describe the exact
homology sequence of the pair $(\RP^5 , \RP^3)$ and check that it
is exact.
\end{xca}
\begin{xca}
Let $X$ be a two-dimensional sphere $S^2$. Calculate relative homology $H_k (X, A)$ where $A$ is a finite subset of $X$.
\end{xca}
\begin{xca}
Let $X$ be a two-dimensional torus $S^1\times S^1$. Calculate relative homology $H_k(X, A)$ where $A$ is a finite subset of $X$.
\end{xca}
\begin{xca}
\begin{enumerate}
\item Calculate relative homology $H_k(X, A)$ where $X$ is a
  handle and $A$ stands for the boundary circle.
\item Calculate relative homology $H_k (X, A)$ where $X$ is a
  sphere with two handles and $A$ is the boundary circle of one
  of handles. 
\end{enumerate}
\textsc{Hint:} You can solve these problems using appropriate
cell decomposition or using exact homology sequence. 
\end{xca}
