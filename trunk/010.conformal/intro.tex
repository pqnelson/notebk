Recall, in General Relativity, a heuristic derivation usually begins
with the Newtonian gravity's Poisson equation
\begin{equation}\label{newtonPoisson}
\nabla^{2}\Phi_{N} = 4\pi G_{N}\rho
\end{equation}
where $\Phi_{N}$ is the Newtonian potential, $G_{N}$ is the
gravitation constant, and $\rho$ is the mass-density. It has a
solution of
\begin{equation}\label{generalSolutionSecondOrder}
\Phi_{N}(r) = \frac{c_{0}}{r}
\end{equation}
where $c_{0}$ is constant.
From here, one typically identifies the right hand side of Eq
(\ref{newtonPoisson}) as the time-time component of stress-energy
tensor, and the left hand side is identified as the time-time
component of the Ricci tensor. This is how most approaches to gravity begin.

In conformal gravity, we begin with a different Poisson
equation. Instead of a second order one, we begin with a fourth order
one~\cite{Mannheim:1994ph}
\begin{equation}\label{fourthOrderPoisson}
\nabla^{4}B(r) = f(r)
\end{equation}
which has the general solution for a spherical source
\begin{equation}\label{generalSolutionFourthOrder}
B(r) = \frac{-r}{2}\int^{R}_{0}dxf(x)x^{2} -
\frac{1}{6r}\int^{R}_{0}dxf(x)x^{4} = \widetilde{c}_{0}r + \frac{\widetilde{c}_{1}}{r}.
\end{equation}
Observe that when $r\ll 1$, Eq (\ref{generalSolutionFourthOrder}) has
$\widetilde{c}_{1}/r$ be the dominant term and
$\widetilde{c}_{0}r\to 0$. Thus for small $r$, we can recover the
Newtonian Poisson equation (\ref{newtonPoisson}).

At first, this may be startling to see Eq
(\ref{generalSolutionFourthOrder}) as being proposed for the
gravitational potential. It is counter-intuitive to propose adding an
$\mathcal{O}(r)$ term, as we don't observe it at ``small'' scales
(dropping an apples behaves as being in a $\mathcal{O}(1/r)$
potential!). However, at such scales, the potential for
a fourth order Poisson equation behaves as the potential for a second
order one. Additionally, there is observational problems with gravity
that departs from a second order Poisson equation at \emph{large}
distances. So the fourth order approach modifies only what is expected
at \emph{large} distances, and agrees with what we expect at
\emph{small} distances.

Here, we must intervene and confess that there is a terribly strong
no-go theorem: Ostrogradski's theorem (for a beautiful introduction,
see section 2 of Woodard~\cite{Woodard:2006nt}). With
Lagrangians involving terms of second order (or higher) time
derivatives of the position term is unstable. Woodard notes that in
Lagrangians of the form $L(q_{i},\dot{q}_{j},\ddot{q}_{k})$ ``there is
not even any barrier to decay''. Adding insult to injury, the
situation does not improve if we add in higher order derivatives!

However, Adler~\cite{adler1982} has proposed recovering
Einstein's General Relativity from Spontaneous Symmetry
Breaking\footnote{In all fairness, Mannheim credits Adler's work
  with recovering Einstein's general relativity, but it appears
  that Adler's paper is more related to the role of spontaneous
  symmetry breaking in the context of avoiding some problems in
  induced gravity like the logarithmic and quadratic divergence
  of the effective Gravitation constant.}
and Zee~\cite{1983AnPhy.151..431Z} has proposed using
spontaneously breaking conformal invariance to give rise to
masses in Weyl gravity. This is very much analogous to how
spontaneous symmetry breaking in the weak force generates the
Fermi constant. The aim of Adler and Zee is that in breaking
symmetry, there is a sort of ``macroscopic/low energy'' limit in
which we can recover Newtonian gravity (or some generalization of
it). 

If instead we consider conformal gravity as a theory in and of
itself, as Mannheim suggests, we find that -- from the solution for
a static, spherically symmetric body -- scale invariance is
spontaneously broken to give rise to a nonzero cosmological
constant term. Further, if we consider working with a stress
energy tensor involving a spontaneous symmetry breaking boson
(which is to be expected if the standard model of particle
physics is correct), we get a sort of ``induced conformal
cosmology'' model which avoids a Big Bang singularity. So the plan is
to first examine the case of the static, spherically symmetric
gravitating body which resembles (up to some negligibly small terms at
the solar-system level) the Schwarzschild solution. Then we will
proceed to consider symmetry breaking at the cosmological
scale and review the consequences in conformal cosmology. Included is
an appendix which constitutes an extremely brief (``five-minute'')
introduction to spontaneous symmetry breaking. 
