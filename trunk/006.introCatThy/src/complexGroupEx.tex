%%
%% complexGroupEx.tex
%% 
%% Made by Alex Nelson
%% Login   <alex@tomato>
%% 
%% Started on  Tue Jun 30 23:13:30 2009 Alex Nelson
%% Last update Tue Jun 30 23:13:30 2009 Alex Nelson
%%
Consider the group generated by $\{1,i=\sqrt{-1}\}$ equipped with
multiplication. Explicitly, we have the multiplication table
\begin{equation}%\label{eq:}
\begin{array}{c|cccc}
\times & 1  & -1 & i  & -i\\\hline
1      & 1  & -1 & i  & -i \\
-1     & -1 & 1  & -i & i\\
-i     & -i & i  & 1  & -1\\
i      & i  & -i & -1 & 1 
\end{array}
\end{equation}
We have this categorified in the following diagram
\begin{equation}
\vcenter{
\xymatrix{            &                              & \\
\txt{*} \ar@/^2pc/[uurrdd]^{1}
\ar[d]_{i} \ar[dr]_{-1} \ar[r]^{-i} & \txt{*}               \ar[r]^{i} & \txt{*}  \\
\txt{*} \ar[r]_{i}            & \txt{*} \ar[r]_{i} \ar[u]_{i} \ar[ur]_{-1}& \txt{*} \ar[u]_{i}      }}
\end{equation}
%% \begin{equation}%\label{eq:}
%% \entrymodifiers={++[o][F-]}
%% \SelectTips{cm}{}
%% \xymatrix @-1pc {
%%  *\txt{start} \ar[r]
%%  & 0 \ar@(r,u)[]^b \ar[r]_a
%%  & 1 \ar[r]^b \ar@(r,d)[]_a
%%  & 2 \ar[r]^b
%%    \ar ‘dr_l[l] ‘_ur[l] _a [l]
%%  &*++[o][F=]{3}
%%    \ar ‘ur^l[lll]‘^dr[lll]^b [lll]
%%    \ar ‘dr_l[ll] ‘_ur[ll]    [ll] }

%% %% \xymatrix{

%% %% }
%% %% \begindc{0}[5]
%% %% \obj(0,10){$*$}
%% %% \obj(0,0){$*$}
%% %% \obj(10,0){$*$}
%% %% \obj(10,10){$*$}
%% %% \obj(20,0){$*$}
%% %% \obj(20,10){$*$}
%% %% \mor(0,10)(0,0){\scriptsize{$i$}}
%% %% \mor(0,0)(10,0){\scriptsize{$i$}}
%% %% \mor(10,0)(10,10){\scriptsize{$i$}}
%% %% \mor(0,10)(10,0){\scriptsize{$-1$}}
%% %% \mor(0,10)(10,10){\scriptsize{$-i$}}
%% %% \mor(10,10)(20,10){\scriptsize{$i$}}
%% %% \mor(10,0)(20,10){\scriptsize{$-1$}}
%% %% \mor(10,0)(20,0){\scriptsize{$i$}}
%% %% \mor(20,0)(20,10){\scriptsize{$i$}}
%% %% %\cmor((0,12)(1,13)(2,15)(6,16)(18,15)(19,13)(20,12))
%% %% \cmor((0,12)(3,15)(10,16)(17,15)(20,12))
%% %%   \pdown(10,17){\scriptsize{$1$}}
%% %% %  \pdown(10,17){\scriptsize{$id_{*}$}}
%% %% \enddc
%% \end{equation}
This encodes all the information about the group generated by
$\{1,i\}$. We can do the diagram chasing to see that the
multiplication is really encoded in our category theoretic
doodling.

%% \begin{equation}%\label{eq:}
%% \begindc{0}[50]
%% \obj(0,1){$*$}
%% \obj(0,0){$*$}
%% \obj(1,0){$*$}
%% \obj(1,1){$*$}
%% \obj(2,0){$*$}
%% \obj(2,1){$*$}
%% \mor(0,1)(0,0){\scriptsize{$i$}}
%% \mor(0,0)(1,0){\scriptsize{$i$}}
%% \mor(1,0)(1,1){\scriptsize{$i$}}
%% \mor(0,1)(1,0){\scriptsize{$-1$}}
%% \mor(0,1)(1,1){\scriptsize{$-i$}}
%% \mor(1,1)(2,1){\scriptsize{$i$}}
%% \mor(1,0)(2,1){\scriptsize{$-1$}}
%% \mor(1,0)(2,0){\scriptsize{$i$}}
%% \mor(2,0)(2,1){\scriptsize{$i$}}
%% \cmor((0,1)(0,2)(2,2)(2,1))
%% \pdown(1,2){\scriptsize{$id_{*}$}}
%% \enddc
%% \end{equation}
