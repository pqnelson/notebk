The motivating problem is that we want to look at mappings 
between mathematical objects, but it may not always make sense.
So we want to have two things: (1) mappings between objects, and 
(2) a setting where it make sense.

The heart of mathematics is definitions, so we will begin by
bluntly defining a category:
\begin{framed}
\begin{defn}\label{defn:category}
A \define{Category} $\ms{C}$ consists of
\begin{enumerate}
\item a collection $\ob{\ms{C}}$ of \define{Objects};
\item for any pair of objects $x,y\in\ob{\ms{C}}$, we have a set
  $\hom(x,y)$ of \define{morphisms} from $x$ to $y$ (so if
  $f\in\hom(x,y)$, then $f:x\to y$)
\end{enumerate}
equipped with
\begin{enumerate}
\item for any object $x\in\ob{\ms{C}}$, an \define{identity morphism}
  $\id{x}:x\to x$;
\item for any pair of morphisms $f:x\to y$ and $g:y\to z$, a
  morphism $g\circ f:x\to z$ called the \define{composition} of
  $f$ and $g$ (note it's written and read from right to left,
  like Chinese, to confuse undergraduates)
\end{enumerate}
such that
\begin{enumerate}
\item for any morphism $f:x\to y$, the \define{left and right
  unit laws} hold: $f\circ\id{x}=f=\id{y}\circ f$;
\item for any triple of morphisms $f:w\to x$, $g:x\to y$, $h:y\to
  z$, the \define{associative law} holds: $(h\circ g)\circ f=h\circ(g\circ f)$.
\end{enumerate}
\end{defn}
\end{framed}
This is not all too enlightening, at least immediately. It'd be a
pity to end here, state ``We will not insult the intelligence of
the reader with examples or theorems, since everything follows
immediately'', and pronounce the reader a \emph{bona fide} expert
category theorist. We will gently introduce various examples before
moving on to start thinking categorically.

\begin{rmk}
The composition of morphisms $g\circ f$ should be read from right
to left (like Chinese or bra-ket notation in quantum mechanics)
and it should be intuitively regarded as
\begin{equation}%\label{eq:}
\text{first do } f \text{ then do }g.
\end{equation}
So do not confuse it with everyday multiplication, \emph{order matters here!}
\end{rmk}

\begin{rmk}
With the arrows, or ``morphisms", we have
\begin{equation}%\label{eq:}
\operatorname{Source}(f)\xrightarrow{\;\;f\;\;}\operatorname{Target}(f)
\end{equation}
or (more categorically, with \hyperref[dualityPrinciple]{duality} in mind) 
\begin{equation}%\label{eq:}
\operatorname{Domain}(f)\xrightarrow{\;\;f\;\;}\operatorname{Codomain}(f)
\end{equation}
where $f$ is a morphism, and the domain/source of $f$ is an
object, the target/codomain of $f$ is also an object.
\end{rmk}

% If this is the first time one has been introduced to
% categories, or if one feels insecure with the nature of
% commutative diagrams, it is suggested to skip ahead to the
% appendix on the calculus of commutative diagrams.

Wait, we just gave a definition! We can show, as with our conjecture
that everything ``is-a'' mathematical object, that a category and
a morphism both are mathematical objects! We will only show that a
category is an object for now (something to bear in mind: what's a 
morphism \emph{between} categories?).

We want to have a taxonomy of morphisms, so we begin very simply
by supposing what happens when we have the domain be the
codomain.
\begin{defn}%\label{defn:}
Let $\ms{C}$ be a category, and $x\in\ob{\ms{C}}$. We define an
\define{endomorphism} to be a morphism $f\in\hom(x,x)$. That is
\begin{equation}%\label{eq:}
f:x\to x
\end{equation}
so its domain is its codomain.
\end{defn}

Now, intuitively the notion of composition of morphisms should be
likened to multiplication. We have a notion of an inverse operation for
multiplication, we call it division. What about the notion of an
inverse for a morphism? That is, given some category
$\ms{C}$, some objects $x,y\in\ms{C}$, and a morphism
$f\in\hom(x,y)$, can we find a $g:y\to x$ such that
\begin{equation}\label{eq:rightInverse}
f\circ g = \id{y}
\end{equation}
and if so, is it true that
\begin{equation}\label{eq:leftInverse}
g\circ f = \id{x}?
\end{equation}
Here the analogy to commutative multiplication should be discarded. The
morphisms represent processes, and the order of processes
matter. Consider for example $h$ being the morphism of mixing the
ingredients of a cake, and $i$ being the process of pouring the
contents of the mixing bowl into a pan and placing the pan in an
oven preheated to 350 degrees Fehrenheit. Well, $i\circ h$ is
baking the normal way, $h\circ i$ is putting the pan in the oven,
and then mixing the ingredients together.

Consequently we don't expect Eq \eqref{eq:rightInverse} to necessarily imply
Eq \eqref{eq:leftInverse} (nor do we expect Eq
\eqref{eq:leftInverse} to necessarily imply Eq \eqref{eq:rightInverse}). We
have two new definitions:
\begin{defn}%\label{defn:}
Let $\ms{C}$ be a category, $x,y\in\ob{\ms{C}}$,
$f\in\hom(x,y)$. Then we define:
\begin{enumerate}
\item $f$ be a \define{monomorphism} (or \define{monic}) if for
  morphisms $g_{1},g_{2}:x\to \omega$ we have $f\circ g_1=f\circ
  g_2$ imply $g_1=g_2$;
\item a \define{epimorphism} (or \define{epic}) if for morphisms
  $g_{1},g_{2}:z\to x$ we have $g_1\circ f=g_2\circ f$ imply $g_1=g_2$;
\item the \define{left inverse} or \define{retraction of $f$} to be a morphism
  $g\in\hom(y,x)$ such that $g\circ f=\id{x}$; 
\item the \define{right inverse} or \define{section of $f$} to be a morphism
  $h\in\hom(y,x)$ such that $f\circ h=\id{y}$.
\end{enumerate}
\end{defn}
\begin{rmk}
Epic is more general than having a right inverse, and monic is
more general than having a left inverse. Consider the following
situation: we have a category with two objects $A$, $B$, three
morphisms (the identity morphisms $\id{A}$, $\id{B}$, and $f:A\to
B$). The morphism $f$ is epic since it can be composed on the
left by exactly one morphism --- $\id{B}$. Similarly, when it can
be composed on the right by exactly one morphism $\id{A}$. This
is more or less a trivial situation that is true when we can only
compose by the identity on the left or on the right. Whenever
this is not true, and a morphism is both epic and monic, it has
left and right inverses.
\end{rmk}

There is one last important type of morphism to consider, it's a
way to state that two objects are ``the same'' in some sense. The
idea is if there is an invertible morphism between two objects,
it's necessarily ``equivalent'' in some sense. Why should this be
so? Well, an invertible mapping requires having the left and
right inverses (a) exist and (b) be the same. 

\begin{defn}%\label{defn:}
Let $\ms{A}$ be a category, $X,Y\in\ob{\ms{A}}$, and $f:X\to
Y$. We say $f$ is an \define{isomorphism} if there is a morphism
$g:Y\to X$ such that $g\circ f=\id{X}$ and $f\circ g=\id{Y}$.
\end{defn}

We also have an invertible endomorphism being given a special name:

\begin{defn}%\label{defn:}
Let $\ms{A}$ be a category, $X\in\ob{\ms{A}}$, and $f:X\to
X$ be an isomorphism. Then $f$ is called an \define{automorphism}.
\end{defn}

To summarize this section, we have this handy table of morphism
properties: 

\begin{table}[H]
\begin{tabular}{| p{3.5cm} | p{6cm} |}
\hline
\textbf{Morphism Property} & \textbf{What it means}\\\hline
Automorphism & Invertible endomorphism. \\\hline
Endomorphism & The domain and codomain are the same.\\ \hline
Epimorphism & Generalization of having a right inverse.\\ \hline
Isomorphism & Generalization of equivalence of domain with codomain.\\\hline
Monomorphism & Generalization of having a left inverse.\\ \hline
Retraction of $f$ & Left Inverse of $f$\\ \hline
Section of $f$ & Right Inverse of $f$\\\hline
\end{tabular}
\caption{A Table of Morphism Properties}\label{tab:morphisms}
\end{table}

\subsection{Categories Are Also Mathematical Objects} 

It seems trivial to demonstrate that categories are mathematical
objects, but for the sake of rigor we will demonstrate it. 
\begin{description}
\item[Stuff] What is the stuff? Well, there is the collection of
  objects. Additionally, for any pair of objects, we have a set
  of morphisms.
\item[Structure] The structure of a category is simply this: 1) for
  any object, we have an identity morphism; 2) for any pair of
  morphisms, we have a composition of morphisms.
\item[Properties] Each component of our structure has a property
  to satisfy. The identity morphism satisfies the left and right
  unit laws. The composition of morphisms satisfies the
  associativity law. We demand nothing more, nothing less.
\end{description}
\noindent This is precisely a mathematical object description of
a category.

\subsection{A Grocery List of Examples}

\begin{ex}We have a few examples to start with that are all pretty similar.

$\ms{0}$ the category with zero objects and zero morphisms. Well,
it's trivially a category, albeit a really boring one.

$\ms{1}$ the category with a single object, and a single morphism
(the identity morphism), is a category.

$\ms{2}$ the category with two objects $a$ and $b$, and just one 
morphism $a\to{}b$ that \emph{is not} an identity.

$\ms{3}$ the category with three objects $a$, $b$, and $c$ whose 
morphisms are $a\to{}b$, $b\to{}c$ and $c\to{}a$ (none are identities).

$\downdownarrows$ the category with two objects $a,b$ and 
two morphisms $a\rightrightarrows b$ that aren't the identity.
\end{ex}

A few properties of categories are worth noting here and now.

\begin{defn}\label{defn:thin}
A category is said to be \define{Thin} if there is at most one 
morphism from an object to another.
\end{defn}
\begin{defn}\label{defn:discrete}
A category is said to be \define{Discrete} if every morphism is
an identity.
\end{defn}
Discrete categories are really just sets since each object has
exactly one morphism, and aside from that there are no additional
morphisms.

We can continue introducing examples, one large class of examples
that the reader may be inclined to notice is that we have a lot of
mathematical objects that are ``sets equipped with structure'', so
why not have a category of such ``sets equipped with the same 
structure''? Like the category of all groups, or of all rings, or
of all sets? Such categories are called ``constructs'' or ``concrete''
depending on the author.

\begin{defn}\label{defn:concreteCategory}
A category $\ms{C}$ is called \define{Concrete} iff there is a functor
$U:\ms{C}\to\ms{Set}$ to the underlying sets of $\ms{C}$ called the
\define{Forgetful Functor} (it is faithful, we'll come back to this 
later).
\end{defn}

\begin{ex}\label{ex:Mon}
Consider \cat{Mon}, the category whose objects are monoids, and whose
morphisms are monoidal homomorphisms. It is concrete.
\end{ex}
\begin{ex}\label{ex:Grp}
Consider \cat{Grp}, the category whose objects are groups and whose
morphisms are group homomorphisms. It is concrete.
\end{ex}
\begin{ex}\label{ex:Set}
Consider \cat{Set}, the category whose objects are sets and whose
morphisms are functions. It is concrete trivially.
\end{ex}

We can also describe a group using category theory. That is given
some group, we can turn it into a category defined by a single
object and the elements of the group become morphisms in the
category. We'll give a few examples of these.
\subsection{Example: Complex Group}
%%
%% complexGroupEx.tex
%% 
%% Made by Alex Nelson
%% Login   <alex@tomato>
%% 
%% Started on  Tue Jun 30 23:13:30 2009 Alex Nelson
%% Last update Tue Jun 30 23:13:30 2009 Alex Nelson
%%
Consider the group generated by $\{1,i=\sqrt{-1}\}$ equipped with
multiplication. Explicitly, we have the multiplication table
\begin{equation}%\label{eq:}
\begin{array}{c|cccc}
\times & 1  & -1 & i  & -i\\\hline
1      & 1  & -1 & i  & -i \\
-1     & -1 & 1  & -i & i\\
-i     & -i & i  & 1  & -1\\
i      & i  & -i & -1 & 1 
\end{array}
\end{equation}
We have this categorified in the following diagram
\begin{equation}
\vcenter{
\xymatrix{            &                              & \\
\txt{*} \ar@/^2pc/[uurrdd]^{1}
\ar[d]_{i} \ar[dr]_{-1} \ar[r]^{-i} & \txt{*}               \ar[r]^{i} & \txt{*}  \\
\txt{*} \ar[r]_{i}            & \txt{*} \ar[r]_{i} \ar[u]_{i} \ar[ur]_{-1}& \txt{*} \ar[u]_{i}      }}
\end{equation}
%% \begin{equation}%\label{eq:}
%% \entrymodifiers={++[o][F-]}
%% \SelectTips{cm}{}
%% \xymatrix @-1pc {
%%  *\txt{start} \ar[r]
%%  & 0 \ar@(r,u)[]^b \ar[r]_a
%%  & 1 \ar[r]^b \ar@(r,d)[]_a
%%  & 2 \ar[r]^b
%%    \ar ‘dr_l[l] ‘_ur[l] _a [l]
%%  &*++[o][F=]{3}
%%    \ar ‘ur^l[lll]‘^dr[lll]^b [lll]
%%    \ar ‘dr_l[ll] ‘_ur[ll]    [ll] }

%% %% \xymatrix{

%% %% }
%% %% \begindc{0}[5]
%% %% \obj(0,10){$*$}
%% %% \obj(0,0){$*$}
%% %% \obj(10,0){$*$}
%% %% \obj(10,10){$*$}
%% %% \obj(20,0){$*$}
%% %% \obj(20,10){$*$}
%% %% \mor(0,10)(0,0){\scriptsize{$i$}}
%% %% \mor(0,0)(10,0){\scriptsize{$i$}}
%% %% \mor(10,0)(10,10){\scriptsize{$i$}}
%% %% \mor(0,10)(10,0){\scriptsize{$-1$}}
%% %% \mor(0,10)(10,10){\scriptsize{$-i$}}
%% %% \mor(10,10)(20,10){\scriptsize{$i$}}
%% %% \mor(10,0)(20,10){\scriptsize{$-1$}}
%% %% \mor(10,0)(20,0){\scriptsize{$i$}}
%% %% \mor(20,0)(20,10){\scriptsize{$i$}}
%% %% %\cmor((0,12)(1,13)(2,15)(6,16)(18,15)(19,13)(20,12))
%% %% \cmor((0,12)(3,15)(10,16)(17,15)(20,12))
%% %%   \pdown(10,17){\scriptsize{$1$}}
%% %% %  \pdown(10,17){\scriptsize{$id_{*}$}}
%% %% \enddc
%% \end{equation}
This encodes all the information about the group generated by
$\{1,i\}$. We can do the diagram chasing to see that the
multiplication is really encoded in our category theoretic
doodling.

%% \begin{equation}%\label{eq:}
%% \begindc{0}[50]
%% \obj(0,1){$*$}
%% \obj(0,0){$*$}
%% \obj(1,0){$*$}
%% \obj(1,1){$*$}
%% \obj(2,0){$*$}
%% \obj(2,1){$*$}
%% \mor(0,1)(0,0){\scriptsize{$i$}}
%% \mor(0,0)(1,0){\scriptsize{$i$}}
%% \mor(1,0)(1,1){\scriptsize{$i$}}
%% \mor(0,1)(1,0){\scriptsize{$-1$}}
%% \mor(0,1)(1,1){\scriptsize{$-i$}}
%% \mor(1,1)(2,1){\scriptsize{$i$}}
%% \mor(1,0)(2,1){\scriptsize{$-1$}}
%% \mor(1,0)(2,0){\scriptsize{$i$}}
%% \mor(2,0)(2,1){\scriptsize{$i$}}
%% \cmor((0,1)(0,2)(2,2)(2,1))
%% \pdown(1,2){\scriptsize{$id_{*}$}}
%% \enddc
%% \end{equation}

\subsection{Example: Quaternion Group}
%%
%% quaternionGroupEx.tex
%% 
%% Made by Alex Nelson
%% Login   <alex@tomato>
%% 
%% Started on  Tue Jun 30 23:51:16 2009 Alex Nelson
%% Last update Tue Jun 30 23:51:16 2009 Alex Nelson
%%
Consider the quaternions, which were characterized by the
equation
\begin{equation}%\label{eq:}
i^2 = j^2 = k^2 = ijk = -1
\end{equation}
We have the multiplication table
\begin{equation}%\label{eq:}
\begin{array}{c|cccccccc}
\times &  1  & -1  &  i  & -i  &  j  & -j  &  k  &  -k\\\hline
1      &  1  & -1  &  i  & -i  &  j  & -j  &  k  &  -k\\
-1     & -1  &  1  & -i  &  i  & -j  &  j  & -k  &   k\\
i      &  i  & -i  & -1  &  1  &  k  & -k  & -j  &   j\\
-i     & -i  &  i  &  1  & -1  & -k  &  k  &  j  &  -j\\
j      &  j  & -j  & -k  &  k  & -1  &  1  &  i  &  -i\\
-j     & -j  &  j  &  k  & -k  &  1  & -1  & -i  &   i\\
k      &  k  & -k  &  j  & -j  & -i  &  i  & -1  &   1\\
-k     & -k  &  k  & -j  &  j  &  i  & -i  &  1  &  -1
\end{array}
\end{equation}
We have various ways to encode the information in the quaternions
into diagrams, e.g.
%% \begin{equation}%\label{eq:}
%% jk=i\quad\iff\quad
%% \begindc{0}[5]
%% \obj(0,10){$*$}
%% \obj(0,0){$*$}
%% \obj(10,0){$*$}
%% \mor(0,10)(0,0){\scriptsize{$j$}}
%% \mor(0,0)(10,0){\scriptsize{$k$}}
%% \mor(0,10)(10,0){\scriptsize{$i$}}
%% \enddc
%% \end{equation}
\begin{equation}%\label{eq:}
jk=i\quad\iff\quad
\vcenter{
\xymatrix{
\txt{*}\ar[d]_{j}\ar[dr]^{i} & \\
\txt{*}\ar[r]_{k} & \txt{*}
}}
\end{equation}
We can cycle through the rest of the identities, or we can skip
ahead to the final diagram
\begin{equation}
\vcenter{
\xymatrix{            &                              & \\
\txt{*} \ar@/^2pc/[uurrdd]^{-1}
\ar[d]_{j} 
\ar[dr]_{k} 
\ar[r]^{i}       & \txt{*} \ar[r]^{i} \ar[d]_{j} & \txt{*}  \\
\txt{*}\ar@/_3pc/[dddrruuuu]_{j}             & \txt{*}\ar[l]^{i}   \ar[ur]_{k}          &
  }}
\end{equation}
%% \begin{equation}%\label{eq:}
%% \begindc{0}[5]
%% \obj(0,20){$*$}
%% \obj(10,20){$*$}
%% \obj(20,20){$*$}
%% \obj(10,10){$*$}
%% \obj(0,10){$*$}
%% \mor(0,20)(10,20){\scriptsize{$i$}}
%% \mor(10,20)(20,20){\scriptsize{$i$}}
%% \mor(0,20)(0,10){\scriptsize{$j$}}
%% \mor(0,20)(10,10){\scriptsize{$k$}}
%% \mor(10,10)(0,10){\scriptsize{$i$}}
%% \mor(10,20)(10,10){\scriptsize{$j$}}
%% \mor(10,10)(20,20){\scriptsize{$k$}}
%% \cmor((0,22)(3,25)(10,26)(17,25)(20,22))
%%   \pdown(10,27){\scriptsize{$-1$}}
%% \cmor((0,9)(3,5)(10,5)(16,9)(20,18))
%%   \pup(6,3){\scriptsize{$j$}}
%% \enddc
%% \end{equation}
With the understanding that $-1\circ-1=1\equiv\id{*}$.
%% \begin{equation}%\label{eq:}
%% \begindc{0}[5]
%% \obj(0,20){$*$}
%% \obj(10,20){$*$}
%% \obj(0,10){$*$}
%% \obj(10,10){$*$}
%% \obj(0,0){$*$}
%% \mor(0,20)(0,10){\scriptsize{$i$}}
%% \mor(0,20)(10,20){\scriptsize{$j$}}
%% \mor(0,20)(10,10){\scriptsize{$k$}}
%% \mor(10,10)(10,20){\scriptsize{$i$}}
%% \mor(0,10)(0,0){\scriptsize{$i$}}
%% \mor(10,10)(0,0){\scriptsize{$k$}}
%% \mor(0,10)(10,10){\scriptsize{$j$}}
%% \enddc
%% \end{equation}

