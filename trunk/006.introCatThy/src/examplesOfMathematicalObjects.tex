%%
%% examples.tex
%% 
%% Made by Alex Nelson
%% Login   <alex@tomato>
%% 
%% Started on  Wed Jun 17 17:18:17 2009 Alex Nelson
%% Last update Wed Jun 17 18:02:44 2009 Alex Nelson
%%

Now that we have introduced the notion of a ``mathematical
object'', perhaps we should start examining
mathematical objects. Well, it turns out that it's all of math, so perhaps we
should consider certain examples. It's only really necessary to
look at a few examples, it turns out that in categories the
objects play second fiddle to the morphisms.

\subsection{A Topology on a Set}\label{ex:topologyAsObject}

Topologies are interesting, they specify the open subsets of a
given set $X$. Recall that a topology $\mathcal{T}$ on $X$ is a
collection of subsets of $X$ having the following properties:
\begin{enumerate}
\item $\emptyset$ and $X$ are in $\mathcal{T}$
\item The union of the elements of any subcollection of
  $\mathcal{T}$ are in $\mathcal{T}$
\item The intersection of the elements of any finite
  subcollection of $\mathcal{T}$ is in $\mathcal{T}$.
\end{enumerate}
This is the definition of a topology on $X$.

Now we want to demonstrate that it is a mathematical object.
\begin{description}
\item[Stuff] The underlying set $X$ is the stuff topologies are
  made of...
\item[Structure] We have some extra structure by considering a
  collection of subsets of $X$.
\item[Properties] We just listed what the properties of this
  structure should be! It's closed under arbitrary union of
  elements of $\mathcal{T}$, finite closure of elements of $\mathcal{T}$, and the collection
  contains both the empty set and $X$ itself.
\end{description}
\noindent So we see that a topology is a mathematical object.

\subsection{Binary Operator on a Set}

Let $S$ be a set, a binary operation on $S$ is a binary relation
that maps elements of the Cartesian product $S\times S$ to $S$:
\begin{equation}%\label{eq:}
f:S\times S\to S.
\end{equation}
The underlying argument is a generalization of the argument that
a function is a mathematical object. In fact, the argument is
exactly the same replacing $X$ with $S\times S$, and $Y$ with
$S$. But since a function was demonstrated as the first example
of a mathematical object in this paper, it follows that a binary
operator on a set is a mathematical object.

\subsection{Groups}

Let $(G,\cdot)$ be a group with elements $G$ and binary operator
represented as multiplication. A group is closed under its binary
operator and necessarily has inverses exist. We will show that
this is clearly a mathematical object:

\begin{description}
\item[Stuff] The underlying set $G$ is the stuff.
\item[Structure] The binary operator acting on $G$ is our
  structure.
\item[Properties] We demand certain properties hold. First for
  any $x\in G$, there is a unique $e\in G$ such that
  $xe=ex=x$. Second, for each $x\in G$, there is a corresponding
  $y\in G$ such that $xy=yx=e$. Third, for each $x,y\in G$, we
  demand that $xy\in G$.
\end{description}
\noindent The pattern is kind of clear how to get from a
definition to our grocery list of ``stuff-structure-properties''.
This is how most people do abstract math, with such grocery
lists.
