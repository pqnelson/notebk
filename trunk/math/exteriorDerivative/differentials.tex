\appendix
\section{A Note on Differentials}

We have seen that
\begin{equation}
d = \sum_{i}\varepsilon^{i}\frac{\partial}{\partial x^i}
\end{equation}
but how exactly do we end up with $dx$, $dy$, $dz$, etc.?

Well, we take in this case
\begin{equation}
\varepsilon^i = dx^i
\end{equation}
and we can rewrite the exterior derivative as
\begin{equation}
d = \sum_{i}dx^{i}\frac{\partial}{\partial x^i}.
\end{equation}
A $k$-form is then
\begin{equation}
\omega = \partial_{i_1}\ldots\partial_{i_k}f\varepsilon^{i_1}\ldots\varepsilon^{i_k}
\end{equation}
and integration of a one form is simply
\begin{equation}
\int \omega d^nx d^n\varepsilon = \int \partial_{i_1}\ldots\partial_{i_n}fd^nx.
\end{equation}
Furthermore change of coordinates is absolutely trivial, if
\begin{equation}
x^i = \psi^i(\tilde{x})
\end{equation}
then
\begin{equation}
\int f(x)dx^1\ldots dx^n = \int f(\psi(\tilde{x}))\underbrace{\prod(\frac{\partial x^i}{\partial\tilde{x}^j}d\tilde{x}^j)}_\textrm{Super-Jacobian} = \int f(\psi(\tilde{x}))d^{n}\tilde{x}.
\end{equation}
Note that the super Jacobian is 1.
