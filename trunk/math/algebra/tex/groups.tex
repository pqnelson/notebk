%%
%% groups.tex
%% 
%% Made by Alex Nelson
%% Login   <alex@tomato>
%% 
%% Started on  Wed Dec 24 14:35:51 2008 Alex Nelson
%% Last update Wed Dec 24 14:35:51 2008 Alex Nelson
%%

\begin{figure}[ht!]
\begin{center}
\includegraphics{img/img.1}
\end{center}
\caption[Relation between Monoids and Groups]{A Venn Diagram
  to illustrate the relation of the monoids and the
  groups. Observe all groups are monoids, but not all
  monoids are groups. Similarly, all trout are fish, but not
  all fish are trouts.}
\end{figure}
A \textbf{group}\index{Group} $G$ is a monoid with some
extra structure. Namely, for each element $x\in G$ there is
an element $y\in G$ such that
\begin{equation}
xy = yx = e.
\end{equation}
This element $y$ is called an
\textbf{inverse}\index{Inverse} for $x$. Such an inverse is
unique, a simple proof to illustrate this: suppose that
there are two inverses $y,y'$ of $x$, then
\begin{equation}
y' = y'e = y'(xy) = (y'x)y = ey = y.
\end{equation}
For multiplication, we denote the inverse of $x$ by
$x^{-1}$, and for addition the inverse of $x$ is denoted by
$-x$. The \textbf{order}\index{Group!Order} of a group is
the number of elements in the group, the size of the set so
to speak.

It is probably worth going on without saying that $e$ is its
own inverse. But if we have (an identity $e$) an element $x$ and its inverse
$x^{-1}$ and a given law of composition, then we can create
a group consisting of the elements
\begin{equation}
G = \{ (x^{-1})^{n}:n\in\mathbb{N}\}\cup\{(x)^{n}:n\in\mathbb{N}\}\cup\{e\}
\end{equation}
which is trivial. 

\begin{ex}
Consider the matrix
\begin{equation}
z = \begin{bmatrix}1 & 0 \\0 & -1\end{bmatrix}
\end{equation}
observe that $z^2=I$. That tells us that $\{I,z\}$ is a
matrix group under matrix multiplication. \qef
\end{ex}

Notice we are being a bit ambiguous here in specifying
``inverses'' and ``unit elements''. E.g. with matrices, it makes
a difference if we multiply by the right or by the left
(e.g. $X$ multiplied on the right by $Y$ is $XY$ but
multiplied on the left is $YX$, and in general $XY-YX\neq0$
-- if it is zero, then $X$, $Y$ are diagonal matrices). But
we have been sufficiently general so left inverses and left
identity elements are also inverses and identity
elements. We can be precise in specification and prove it
too:
\begin{prop}
Let $G$ be a set with an associative law of composition, let
$e$ be a left unit for that law, and assume that every
element has a left inverse. Then $e$ is a unit and each left
inverse is also an inverse. In particular, $G$ is a group.
\end{prop}
\begin{proof}
Let $b\in G$ and let $a\in G$ be such that $ba=e$. Then
\begin{equation}
bab = eb = b.
\end{equation} 
This much should be trivial, it's just substitution. We can
see that
\begin{equation}
abab = a(ba)b = aeb = ab
\end{equation}
which is equivalent to
\begin{equation}
(ab)^2 = ab.
\end{equation}
We can multiply both sides by $(ab)^{-1}$ on the left to see
that
\begin{equation}
(ab) = e
\end{equation}
which implies that $a$ is the left inverse for $b$, or
equivalently that $b$ is the right inverse for $a$. We then
see that
\begin{equation}
aba = a(e) = (e)a
\end{equation}
which implies the left identity $e$ is a ``bi''-identity
(i.e. both a left and right identity).
\end{proof}
\begin{ex}
Let $G$ be a group, and $S$ be some nonempty set. The set of
maps from $S$ to $G$, denoted $M(S,G)$, is itself a
group. That is, for two maps $f,g:S\to G$, we define $fg$ to
be the map such that
\begin{equation}\label{eq:groups:exLawOfComposition}
(fg)(x) = f(x)g(x)
\end{equation}
The inverse $f^{-1}$ multiplied by $f$ is equal to the
identity element in the group $\mathbb{1}$. That means,
$f^{-1}(x)=f(x)^{-1}$. If $f(x)^{-1}$ is well defined,
meaning that $f(x)$ -- an element of the group $G$ -- has an
inverse, which is necessarily true because that's the
definition of a group, then each element of $M(S,G)$ has an
inverse under the law of composition defined by Eq
\eqref{eq:groups:exLawOfComposition}. We also have for
arbitrary $f\in M(S,G)$ the product of it well defined too
since $f(x)$ is an element in the group, so $f(x)^n$ is an
element in the group raised to the $n^{th}$ power -- which
is well defined because it's a monoid! So it is a group.\qef
\end{ex}
\begin{ex}
Let $S$ be a non-empty set. Let $G$ be the set of bijections
from $S$ to $S$. Then we see that, making the composition of
mappings the law of composition, $G$ is a group. (Remember a
bijection is invertible, so each map has an inverse and we
have the identity defined in the obvious way as the identity
map of $S$.) The elements of $G$ are called
\textbf{permutations}\index{Permutations} of $S$. We denote
$G$ by $\operatorname{Perm}(S)$. \qef
\end{ex}
\begin{ex}
Consider the group $\mathbb{Z}_{2}$, which consists of two
elements $0$ and $1$. One can imagine this as a ``bit'' or a
binary digit. It has the operation of addition
\begin{equation}
0+0=1+1=0,\quad 0+1=1+0=1
\end{equation}
which is commutative. It is a cyclic group.\index{Group!Cyclic}\index{Cyclic Group} We can further
generalize this to $\mathbb{Z}_{3}$ with 3 elements: 0, 1,
and 2. It has the laws of addition
\begin{equation}
0+0=1+2=2+1=0,\quad 0+1=1+0=2+2=1,\quad 0+2=2+0=1+1=2
\end{equation}
which is also commutative. We have inverses well defined,
etc. etc. etc. We can generalize this to $\mathbb{Z}_{n}$
where $n\in\mathbb{N}$. This is a family of finite groups. \qef
\end{ex}
\begin{rmk}
Typically we ``represent'' a number $p$ in a cyclic group
$\mathbb{Z}_{n}$ (where $0\leq p<n$ is some integer) by
$\exp[i2\pi (p/n)]$. In this ``representation'', we have
instead of addition simple multiplication. Note that the
``generator'' of the group (the only element we really need
that we can subject to the law of composition of the group
to) is the primitive $n^{th}$ root of unity
$\exp[i2\pi/n]$. So $p$``=''$(\exp[i2\pi/n])^p$.
\end{rmk}
\begin{ex}
Let $G_1$, $G_2$ be groups. Let $G_1\times G_2$ be the
\textbf{direct product}\index{Direct Product!Groups}\index{Group!Direct Product}
which is similar to the direct product of sets, so
$G_1\times G_2$ is the set of all pairs $(x_1, x_2)$ with
$x_1\in G_1$ and $x_2\in G_2$. We define the product
componentwise by
\begin{equation}
(x_1,x_2)(y_1,y_2) = (x_1y_1,x_2y_2).
\end{equation}
We see that $G_1\times G_2$ is a group, with unit element
$(e_1, e_2)$ (where $e_1$ is the unit of $G_1$, and $e_2$ is
the unit of $G_2$). We can do this for $n$ groups, with
componentwise multiplication. \qef
\end{ex}

Let $G$ be a group. A \textbf{subgroup}\index{Group!Subgroup}\index{Subgroup} $H$
of $G$ is a subset of $G$ containing theunit element, and
such that $H$ is closed under the law of composition and
inverse (or equivalently, it is a submonoid such that if
$x\in H$ then $x^{-1}\in H$). A subgroup is
\textbf{trivial}\index{Subgroup!Trivial} if it consists of
the unit element alone. And trivially, the intersection of
an arbitrary non-empty family of subgroups is a subgroup.

Let $G$ be a group and $S\subset G$ be a subset. We say that
$S$ \textbf{generates} $G$ or that $S$ is a set of
\textbf{generators} for $G$ if every element of $G$ can be
expressed as a product of elements of $S$ or inverses of
elements of $S$, i.e. as a product $x_1\cdots x_n$ where
each $x_i$ or $x_{i}^{-1}$ is in $S$. It is clear that the
set of al such products is a subgroup of $G$ (remember,
$x^0=e$) and is the smallest subgroup of $G$ containing
$S$. What the hell does this mean? Well, $S$ generates $G$
if and only if the smallest subgroup of $G$ containing $S$
is $G$ itself. 

Let's stop and reiterate what we have just introduced. We
have a subgroup $S$ of a group $G$ is a submonoid that is a
group. So the submonoid contains inverses for each element.
We have some finite subset $S'$ of a group $G$ which is such that
any element of $G$ can be written as a product of any finite
number of elements of $S'$. This is a bit vague, so perhaps
an example is needed. 

\begin{ex}
Consider the quaternions, which have elements
\begin{equation}
i^2=j^2=k^2=ijk=-1
\end{equation}
We see that
\begin{align*}
i(ijk) &= i(-1)\\
\Rightarrow jk &= i\\
\Rightarrow (jk)k &= ik\\
\Rightarrow -j &= ik\\
\Rightarrow ij &= k
\end{align*}
So from $i$ and $j$, we can get $1=(-1)^2=(i)^4=(j)^4$ as
well as $k=ij$. All we need are two elements to have the
rest of the quaternions. \qef
\end{ex}

Note typically, if $G$ is a group, and $S$ is a set of
generators, then we write $G=\<S\>$. \marginpar{cyclic group has one generator}By 
definition, a cyclic group is a group which has one
generator.

\begin{ex} 
There are two non-abelian groups of order 8. One is the
quaternions which we already say, the other is the
\textbf{symmetries of the square},\index{Symmetries of Square}\index{Symmetry!Square}\index{Square!Symmetry} generated
by two elements $\sigma$, $\tau$ such that
\begin{equation}
\sigma^4 = \tau^2 = e,\quad\text{and }\tau\sigma\tau^{-1} =
\sigma^{3}
\end{equation}
We see that
\begin{align*}
(\tau\sigma\tau^{-1})\tau &=
\sigma^{3}\tau\\
&= \tau\sigma
\end{align*}
but also that
\begin{equation}
\sigma^{-1} = \sigma^{3}\Rightarrow \sigma^{-1}\sigma = e =
\sigma^4
\end{equation}
and
\begin{equation}
\tau^{-1} = \tau\Rightarrow \tau\sigma\tau=\sigma^3.
\end{equation}
So that means that
\begin{equation}
\tau\sigma\tau = \sigma^{-1}
\end{equation}
and
\begin{equation}
\tau\sigma^{-1}\tau = \sigma.
\end{equation}
But doesn't this imply
\begin{equation}
\tau = \sigma\tau\sigma
\end{equation}
by multiplying by the right by $\tau\sigma$. So we can write
the inverses in terms of $\sigma$ and $\tau$ alone. \qef
\end{ex}
\begin{ex}
The quaternion group is more generally the group generated
by two elements $i$, $j$ and setting $ij=k$ and $m=i^2$, we
have
\begin{equation}
i^4 = j^4 = k^4 = e,\quad\text{and }i^2=j^2=k^2=m,\; ij=mji
\end{equation}
but observe that as we introduced it before, it works
perfectly fine setting $m=-1$ and $e=1$. \qef
\end{ex}

Let $G,G'$ be monoids. A \textbf{monoid-homomorphism} (or
simply \textbf{homomorphism}\index{Homomorphism}) of $G$
into $G'$ is a mapping $f:G\to G'$ such that
$f(xy)=f(x)f(y)$ for all $x,y\in G$ and mapping the unit
element of $G$ into that of $G'$. If additionally $G,G'$ are
groups, the mapping is given a special name called a
\textbf{group-homomorphism}. 

\begin{ex}
Consider the monoid $\mathbb{N}$, a map
\begin{equation}
f:\mathbb{N}\to\mathbb{Z}_2
\end{equation}
is a homomorphism defined such that
\begin{equation}
f(n) = \begin{cases}0 &\text{$n$ is even}\\
1 &\text{otherwise}\end{cases}
\end{equation}
Then $f$ is a homomorphism. It maps the 0 element to the 0
element, and it maps 
\begin{align*}
f(m+n) &= f(m)+f(n)
\end{align*} 
trivially (odd+odd=even, even+even=even, even+odd=odd, and
1+1=0, 0+0=0, 0+1=1 respectively). \qef
\end{ex}
\begin{ex}
Let $V$, $W$ be arbitrary vector spaces. Let
\begin{equation}
f:V\to W
\end{equation}
be a linear transformation. Trivially it is a homomorphism,
as
\begin{equation}
f(u+v)=f(u)+f(v)
\end{equation} 
and by the definition of a linear transformation, we have
\begin{equation}
f(0)=0.
\end{equation}
Why? Well, it's trivial, observe
\begin{equation}
f(0+0)=f(0)+f(0)=f(0)\Rightarrow f(0)=0.
\end{equation}
So it preserves vector addition, and it maps the identity of
vector addition to the identity of vector addition. \qef
\end{ex}

Observe that for a group homomorphism $f:G\to G'$, we have
\begin{align*}
f(xx^{-1}) &= f(e)\text{ since }xx^{-1}=e\\
&= f(x)f(x^-1)\text{ since $f$ is a homomorphism}\\
&= e'\text{ since $f$ is a homomorphism}\\
\Rightarrow \left(f(x)\right)^{-1}f(e) &= f(x^{-1})\\
&= f(x)^{-1}.
\end{align*}
It's trivial.

\begin{ex}
Let $G$ be a commutative group. Then for $x\in G$,
\begin{equation}
x\mapsto x^n
\end{equation}
for some fixed integer $n$, is a homomorphism called the
\textbf{$n$-th power map}\index{Homomorphism!$n$-th Power Map}. \qef
\end{ex}
\begin{ex}
Let $G_i$ be some collection of groups, and $i\in I$ be an
element of some indexing set. Let
\begin{equation}
G = \prod G_{i}
\end{equation}
be direct product of all $G_i$, for all $i\in I$. So an
element of $G$ would be a tuple consisting of components
from each of the $G_i$. Let
\begin{equation}
p_i:G\to G_i
\end{equation}
be the projection of the $i^{th}$ factor. It selects the
component of the tuples in $G$ which corresponds to the
contribution from the group $G_i$. Then $p_i$ is a
homomorphism. \qef
\end{ex}\begin{quote}\begin{thm}
Let $G$, $G'$ be groups, and $S$  be a set of generators of
$G$. Let
\begin{equation}
f:S\to G'
\end{equation}
be a map. If there exists a homomorphism $\widetilde{f}:G\to
G'$ such that when we restrict $G$ to be $S$,
$\widetilde{f}=f$, then there exists only one
$\widetilde{f}$. 
\end{thm}
\end{quote}
In other words, $f$ has at most one extension to a
homomorphism of $G$ into $G'$. 

Let $f:G\to G'$ and $g:G'\to G''$ be two
group-homomorphisms. Then the composition $g\circ f$ is also
a group homomorphism. If $f,g$ are isomorphisms then so is
$g\circ f$. Further, $f^{-1}:G'\to G$ is also an
isomorphism. In particular, the set of all automorphisms of
$G$ is itself a group, denoted as $\aut(G)$.
\begin{defn}
Let $f:G\to G'$ be a group homomorphism. Let $e,e'$ be the
respective unit elements of $G,G'$. We can then define the
\textbf{kernel}\index{Kernel!Group Homomorphism} of $f$ to
be the subset of $G$ consisting of all elements $x$ such
that $f(x)=e'$.
\end{defn}
We immediately see that the kernel forms a subgroup of $G$,
since for any two $x,y\in\ker(f)$, we have
\begin{align*}
f(x+y) &= f(x)+f(y)\\
&= e'+e'\\
&= e'.
\end{align*}
Furthermore, since $f$ is a homomorphism, we have $f(e)=e'$
and
\begin{equation}
f(e) = f(xx^{-1}) = f(x)f(x)^{-1} = e'f(x^{-1})=e'
\end{equation}
which implies $f(x^{-1})=e'$. So $x^{-1}$ is in $\ker(f)$.
\begin{prop}
Let $f:G\to G'$ be a group homomorphism. Let $H'$ be the
\textbf{image}\index{Homomorphism!Image} of $f$. Then $H'$
is a subgroup of $G'$.
\end{prop}
\begin{proof}
Observe that for $x,y\in G$, that
\begin{equation}
f(xy)=f(x)f(y)\in H'.
\end{equation}
Further, since $f(e)=e'\in H'$ (so $H'$ has an identity
element), we see
\begin{equation}
f(xx^{-1})=f(x)f(x^{-1})=e'\Rightarrow f(x^{-1})\in H'.
\end{equation}
So the inverse of an arbitrary element is in $H'$, as is the
identity element, which is sufficient for $H'$ to be a
subgroup. 
\end{proof}
\begin{rmk}
The kernel and image of $f$ are denoted by $\ker(f)$ and
$\imag(f)$ respectively.
\end{rmk}
