%%
%% algebra.tex
%% 
%% Made by Alex Nelson
%% Login   <alex@tomato>
%% 
%% Started on  Mon Dec 22 11:28:40 2008 Alex Nelson
%% Last update Mon Dec 22 11:28:40 2008 Alex Nelson
%%
\documentclass[10pt]{article}
% for usuage of conditionals see http://zyliu2005.blogspot.com/2007/06/latex-how-to-use-conditional.html
\usepackage{ifthen}
\newboolean{useIndex}
\setboolean{useIndex}{false}
%\setboolean{useIndex}{true}
\usepackage{fly}
\ifthenelse{\boolean{useIndex}}
{
\usepackage{makeidx} % for the index e.g. \index{key}
\makeindex
}
{\renewcommand{\index}[1]{ }}
\theoremstyle{definition}
\newtheorem*{sketch}{Sketch Of Proof}
\def\qef{$\blacksquare$}
\def\aut{\operatorname{Aut}}
\def\ker{\operatorname{Ker}}
\def\imag{\operatorname{Im}}
\title{Notes on Abstract Algebra}
\date{December 22, 2008}
\begin{document}
\maketitle
\tableofcontents
\listoffigures
\section*{Introduction}
\begin{rmk}
Just a prefatory note, to indicate an example has ended I
will use either `QEF' or \qef ~and for the end of a proof, I
will use an empty square; to show that a sketch of a proof
has ended I will use a spade $\spadesuit$.
\end{rmk}
%%
%% algebraIndex.tex
%% 
%% Made by Alex Nelson
%% Login   <alex@tomato>
%% 
%% Started on  Mon Dec 22 11:31:48 2008 Alex Nelson
%% Last update Fri Jan 30 12:35:29 2009 alex
%%
\graphicspath{img}
\section{Groups}
\subsection{Monoids}
%%
%% monoids.tex
%% 
%% Made by Alex Nelson
%% Login   <alex@tomato>
%% 
%% Started on  Mon Dec 22 11:22:31 2008 Alex Nelson
%% Last update Mon Dec 22 11:22:31 2008 Alex Nelson
%%

Let $S$ be some set. A mapping
\begin{equation}
S\times S\to S
\end{equation}
is sometimes called a \textbf{law of composition}\index{Law Of Composition} (of $S$
into itself). If $x,y$ are elements of $S$, the image of the
pair $(x,y)$ under this mapping is also called their
\textbf{product}\index{Product} under the law of composition. It will be
denoted as $xy$.
\begin{ex}
In $\mathbb{R}^3$, our favorite vector space from calculus
21 C, an example of a noncommutative law of composition is
the \textbf{cross product}\index{Cross Product}
\begin{subequations}
\begin{align}
\vec{u}\times\vec{v} &= \begin{vmatrix}\hat{x} & \hat{y} & \hat{z}\\
u_{x} & u_{y} & u_{z}\\
v_{x} & v_{y} & v_{z}
\end{vmatrix}\\
&= (u_{y}v_{z}-u_{z}v_{y})\hat{x} +
(u_{z}v_{x}-u_{x}v_{z})\hat{y} + (u_{x}v_{y}-u_{y}v_{x})\hat{z}
\end{align}
\end{subequations}
Observe that the dot product does not really satisfy the
criteria for the composition, because the dot product
``takes in'' two vectors and it ``spits out'' a scalar. But
for a composition, it needs to ``take in'' two of the same
object, and it ``spits out'' the same object. The cross
product does this, it ``takes in'' two 3-vectors and it
``spits out'' a 3-vector. \qef
\end{ex}
\begin{ex}
Consider multiplication of complex numbers
\begin{equation}
(t + iu)(x + iy) = (xt - yu) + i(ux + ty)
\end{equation}
It takes in two complex numbers, and it returns a complex
number. It is commutative. \qef
\end{ex}
\begin{ex}
Consider an arbitrary vector space $V$ over some field
$\mathbb{F}$. Given any two vectors $\vec{u}$, $\vec{v}\in
V$, we have the commutative composition operation of vector
addition
\begin{equation}
(\vec{u}+\vec{v})\in V
\end{equation}
which means that we have a law of composition for vector
spaces. \qef
\end{ex}
\begin{ex}
Let $W$ be a vector space over the field
$\mathbb{F}$. Consider two linear operators: 
\begin{equation}
T,U:W\to W
\end{equation}
then we may compose the linear operators into a new linear
operator
\begin{equation}
V = T\circ U
\end{equation}
so when acting on some element $\vec{w}\in W$ we have
\begin{equation}
V(\vec{w}) = T\Big(U(\vec{w})\Big)
\end{equation}
which is also a linear operator. \qef
\end{ex}
Similarly, for more general (not necessarily commutative, but possibly
commutative) operations, we use the symbol `$\cdot$' and
write $x\cdot y$ (or more generally just $xy$).

Now in general, if the composition is commutative (so
$xy=yx$) then we usually use the symbol `+' to indicate this
and generically call it ``\textbf{addition}''\index{Addition}. So when
$x+y=y+x$ (i.e. when the operation is commutative) we use
the generic term ``addition'' and the symbol `+' to indicate
this. 

But to be fully general, a generic law of composition is
dubbed ``multiplication'', and if it is commutative we call
it ``addition''. We use the corresponding terminology
(e.g. ``the sum of $x$ and $y$ is $x+y$'', ``given $u$ and
$v$, their product is $uv$'').

Now let us consider a set $S$ with a law of composition. If
$x,y,z$ are elements of $S$, we can write their product in
two ways: $(xy)z$ and $x(yz)$. \marginpar{Screw with the
  parenthese, i.e. order of operating doesn't matter,
  operation is associative}If $(xy)z=x(yz)$ for all
$x,y,z\in S$, then we say that their law of composition is ``\textbf{associative}''\index{Associativity}.

An element $e$ of $S$ such that
\begin{equation}
ex = xe = x
\end{equation}
for all $x\in S$ is called a \textbf{unit element}\index{Unit Element}.
(When the law of composition is written additively, the unit
element is denoted by 0, and is \textbf{zero element}\index{Zero Element}.) 
A unit element is unique, suppose that there is another unit
element $e'$ distinct from $e$, then
\begin{equation}
e = ee' = e'
\end{equation}
by assumption. In most cases, the unit element is written
simply as 1 (instead of $e$). It is a generalization of the
notion of the identity element with respect to some given
``Law of Composition''. For most of this section, we'll use
$e$ for clarity when specifying the basic properties.

Now, a \textbf{monoid}\index{Monoid} is a set $G$ with a law
\marginpar{Monoid: a set of elements equipped with some law of composition, that is closed under said law of composition}of composition which is associative, and
having a unit element (which implies that $G$ is never
empty). Note that there is nothing about being finite or
infinite, nor is there any specification about the law of
composition possessing an inverse.

\begin{ex}
Consider the natural numbers with 0\footnote{The author, not
  being French, doesn't believe 0 is either natural or a number.},
that is
$\mathbb{N}\cup\{0\}$. We have the law of composition
defined by addition in the usual way (so $1+0=1$, $2+4=6$,
etc.) and we have the additive identity $0$. If we didn't
have 0, we'd be a bit out of luck as there is no additive
identity in $\mathbb{N}$. So all by itself, $\mathbb{N}$ is
not a monoid, but the union $\mathbb{N}\cup\{0\}$ is a
monoid as it has the additive identity. \qef
\end{ex}
\begin{ex}
Let $V$ be a vector space over the field $\mathbb{F}$. Let
$\mathcal{L}(V)$ be the set of linear operators on
(``endomorphisms of'') $V$. Then $\mathcal{L}(V)$ is a
monoid in two different ways: one is with the law of
composition being matrix addition with the unit element
being the zero matrix; the other is with the law of
composition being matrix multiplication with the unit
element being the identity matrix. \qef
\end{ex}

Let $G$ be a monoid, and $x_1,\ldots,x_n$ be elements of $G$
\marginpar{Remember, we have the law of composition generically referred to as ``multiplication''}(where $n>1$ is some integer). We can define their product
inductively:
\begin{equation}
\prod^{n}_{\nu=1}x_{\nu}=x_1\cdots x_n = (x_1\cdots
x_{n-1})x_n.
\end{equation}
Now, don't jump to conclusions! It is very tempting to think
of this as just the usual product series as we learned from
analysis, but this is more general due to this just being a
way to iterate the law of composition on a sequence of
elements in $G$. Note that we did define it to be careful
about the order of operating. 

\marginpar{Extended associativty for an arbitary number of elements}We can also observe that we have the following rule
\begin{equation}
\prod^{m}_{\mu=1}x_{\mu}\prod^{n}_{\nu=1}x_{m+\nu}=\prod^{m+n}_{\nu=1}x_{\nu}
\end{equation}
which essentially asserts \emph{that we can insert
  parentheses in any manner in our product without changing
  its value}. This is actually relatively trivial since
monoids have their law of composition be associative.
\begin{sketch}
We can do this proof by induction on $m$. So for an
arbitrary positive integer, we have
\textbf{Base Case:} $m=2$ Observe that
\begin{subequations}
\begin{align}
\prod^{m}_{\mu=1}x_{\mu}\cdot\prod^{n}_{\nu=1}x_{m+\nu}
&= (x_1\cdot x_2)\cdot(x_3\cdot(\cdots)\cdot x_{2+n})\\
&= x_1\cdot x_2 \cdot x_3 \cdot (\cdots)\cdot x_{2+n}\text{ (by Associativity)}\\
&= \prod^{2+n}_{\nu=1}x_{\nu}
\end{align}
\end{subequations}
where we justify the last step by just grouping terms by
associativity of the law of composition.

\textbf{Inductive Hypthotesis:} Assume this works for
arbitrary $m$.

\textbf{Inductive Case:} For $m+1$ we have
\begin{subequations}
\begin{align}
\prod^{m+1}_{\mu=1}x_{\mu}\cdot\prod^{n}_{\nu=1}x_{\nu+m+1} &= 
\left(\prod^{m}_{\mu=1}x_{\mu}\right)\cdot x_{m+1}\cdot\left(\prod^{n}_{\mu=1}x_{\mu+m+1}\right)\\
&= \left(\prod^{m}_{\mu=1}x_{\mu}\right)\cdot\left(\prod^{n}_{\mu=1}x_{\mu+m}\right)
\end{align}
\end{subequations}
where we justify that last step by, again, the associativity
of the law of composition, since we're using a monoid. Then
we see that this is precisely the case for $m$ which we
assumed worked! So that concludes our proof by induction. $\spadesuit$
\end{sketch}

Also note just a few standards\index{Product!Conventions} we have with the product. For
instance, we have
\begin{equation}
\prod^{m+n}_{m+1}x_{\nu}\text{  instead of  }
\prod^{n}_{\nu}x_{m+\nu}
\end{equation}
and we \emph{define}
\begin{equation}
\prod^{0}_{\nu=1}x_\nu = e.
\end{equation}

It should be possible to define more general laws of
composition, that is to say maps $S_{1}\times S_{2}\to
S_{3}$ where $S_1,S_2,S_3$ are arbitrary sets. One can then
express associativity and commutativity in any setting where
they make sense to express them (it will make more sense
that than sentence just written). For instance, to have
commutativity we need the law of composition take the form
of 
\begin{equation}
f: S\times S\to T
\end{equation}
where the two sets that are ``eaten'' by $f$ are necessarily
the same. \textbf{Commutativity}\index{Commutativity} then
means that $f(x,y)=f(y,x)$ (or omitting $f$, $xy=yx$). 

For associativity, when would it make sense? There are 8 possible
cases one can imagine
\begin{equation*}
\begin{array}{ccc}
S\times S\to S & S\times T\to S & T\times S\to S\\
T\times T\to T & S\times T\to T & T\times S\to T\\
S\times S\to T & T\times T\to S & 
\end{array}
\end{equation*}
To have associativity, we need to have 
\begin{equation}
f(f(a,b),c) = f(f(a,c),b)\text{ or } f(a,f(b,c))=f(b,f(a,c))
\end{equation}
so that means that $b,c$ are both elements of the same
set. That automatically rules out two possibilities
\begin{equation*}
S\times S\to T\quad\text{and}\quad T\times T\to S.
\end{equation*}
By symmetry, we can see that if $S\times S\to S$ is
associative, then $T\times T\to T$ is also
associative. Similarly, if $S\times T\to S$ is associative,
then $T\times S\to T$ is also associative. Commuting the
position of the arguments doesn't change anything either, so
if $S\times T\to S$ is associative then $T\times S\to S$ is
also associative. So the only cases that can be associative
\marginpar{When associativity could make sense}are
\begin{equation}
\begin{array}{ccc}
S\times S\to S & S\times T\to S & T\times S\to S\\
T\times T\to T & S\times T\to T & T\times S\to T
\end{array}
\end{equation}
and all others cannot be associative.

Now, if the law of composition of $G$ is commutative, we say
that $G$ is \textbf{commutative} (or more often \textbf{Abelian}\index{Abelian!Module}\index{Module!Abelian}).

\begin{prop}
Let $G$ be a commutative monoid, and $x_1$, $\ldots$, $x_n$
be elements of $G$. Let $\psi$ be a bijection of the set of
integers $(1,\ldots,n)$ onto itself (in other words, it's a
permutation of $(1,\ldots,n)$). Then
\begin{equation}
\prod^{n}_{\nu=1}x_{\psi(\nu)} = \prod^{n}_{\nu=1}x_{\nu}
\end{equation}
\end{prop}
\begin{proof}
First let us show by induction that we can reorder a product
of a sequence of elements in arbitrary order.
\textbf{Base Case:} $n=2$ So
\begin{equation}
xy = yx
\end{equation}
which is trivially true from the definition of a commutative
monoid.

\textbf{Inductive Hypothesis:} Assume this works for
arbitrary $n$.

\textbf{Inductive Step:} We can show that since the law of
composition maps $G\times G\to G$ that
\begin{equation}
\prod^{n}_{\nu=1}x_{\psi(\nu)} = y
\end{equation}
and by the inductive hypothesis we have
\begin{equation}
\prod^{n}_{\nu=1}x_{\nu} = y
\end{equation}
thus
\begin{equation}
yx_{n+1} = x_{n+1}y
\end{equation}
by commuting $x_{n+1}$ through all of the elements in the
product. Thus we cover arbitrary permutations of
$(1,\ldots,n+1)$ onto itself.
\end{proof}

Let $G$ be a commutative monoid, let $I$ be a set, and let
$f:I\to G$ be a mapping such that $f(i)=e$ for almost all
$i\in I$. (Here and thereafter, \textbf{almost all}\index{Almost All} 
means \emph{all but a finite number}.) Let $I_{0}$ be the
subset of $I$ consisting of those $i$ such that $f(i)\neq
e$. By
\begin{equation}
\prod_{i\in I}f(i)
\end{equation}
we shall mean the product
\begin{equation}
\prod_{i\in I_{0}}f(i)
\end{equation}
taken in any order. 

(When $G$ is written additively, then instead of a product
sign, we write the sum $\sum$ instead of the product $\prod$.)

We can continue rattling off a grocery list of properties
that the product has. But it would be a pain in the rear to
do, and not all that educational. \marginpar{note notation}
We will, out of laziness, write
\begin{equation}
\prod f(i)
\end{equation}
omitting the $i\in I$ bit, since we have just seen the order
doesn't matter. Another matter of convention, we use the
exponent to indicate the number of times the operation of a
set is used on a given element. So
\begin{equation*}
x^n = \prod^{n}_{i=1}x
\end{equation*}
This allows us to have $x^0=e$ and $x^1=x$. We are content
with this convention, since it allows us to use familiar
notation making anything to the zeroeth power be the
identity element.

\begin{ex}
Consider the set of all invertibe matrices of a given vector
space $V$ on a field $\mathbb{F}$. This is typically denoted
as $GL(V)$, and for an arbitrary element $X\in GL(V)$ we
have
\begin{equation}
X^0 = I
\end{equation}
where $I$ is the identity matrix. \qef
\end{ex}

Let $S$, and $S'$ be two subsets of a monoid $G$, then we
define $SS'$ to be the subset consisting of all elements
$xy$ with $x\in S$ and $y\in S'$. Inductively, we can define
the product of a finite number of subsets, and we have
associativity. Why? How can we see this assertion? Well,
consider three subsets $S,S',S''$ of $G$. Then we can write
$(SS')S''=S(S'S'')$. \marginpar{Remember law of compositions map $G\times G\to G$}Observe that we can say $GG=G$ since
$G$ has a unit element. If $x\in G$, then we can define $xS$
to be $\{x\}S$ where $\{x\}$ is the set with a single
element $x$ (this is sort of like a coset). Thus $xS$
consists of all elements $xy$ with $y\in S$.

\begin{ex}
Consider all the square matrices on the vector space
$V=\mathbb{C}^2$ over the scalar field $\mathbb{C}$. Let 
\begin{equation}
z = \begin{bmatrix}1&0\\0&-1\end{bmatrix}
\end{equation}
and let $S$ be the set of all traceless operators acting on
$V$. Then
\begin{equation}
zS = \{ zy:y\in S\} = \left\{\begin{bmatrix}0&\beta\\-\alpha&0\end{bmatrix}:\alpha,\beta\in\mathbb{C} \right\}
\end{equation}
is the, uh, ``comonoid''.\qef
\end{ex}

We can specify a monoid which is a subset of a given monoid.
We call it a \textbf{submonoid}\index{Monoid!Submonoid}\index{Submonoid}
of $G$. To be fully clear, a submonoid of $G$ is a subset
$H$ of $G$ containing the unit element $e$, and such that --
if $x,y\in H$ then $xy\in H$ -- or in other words, $H$ is
\textbf{closed} under the law of composition. Well, by the
definition of a monoid, we see that $H$ is also a monoid.

We can see a trivial example of a submonoid given a monoid
is just the powers of an arbitrary element. What does this
mean? Well, choose some element $x$. What can we do with
this? Well, we have the law of composition, so we can
multiply it to itself to get $x^2$. We can then multiply
this by $x$ again to get $x^3$. We can keep going and going
for arbitrary $n\in\mathbb{N}$ have the set of $x^n$. But
this alone is not a monoid. Why? There is no identity
element, which isn't nice. So we copy the French, and have
$n\in\{0\}\cup\mathbb{N}$, then $\{x^n:
n\in\{0\}\cup\mathbb{N}\}$ is a monoid and its closed under
the law of composition. We see that it is commutative and
associative. So it's a really nice monoid! 

\subsection{Groups}
\section{A Few Remarks On Groups}

We will follow Dummit and Foote~\cite{dummitFoote} here when introducing groups
to be as general as possible. 

\begin{defn}
A \textbf{binary operation} $\star$ on a set $G$ is a function
\begin{equation}
\star: G\times G\to G.
\end{equation}
For any $a,b\in G$ we shall write $a\star b$ instead of $\star(a,b)$.
\end{defn}

\begin{ex}
The operator of addition defined on integers is a binary operation.
\end{ex}

\begin{defn}
A binary operation $\star$ on $G$ is \textbf{associative} if for all $a,b,c\in G$
we have 
\begin{equation}
a\star(b\star c)=(a\star b)\star c.
\end{equation}
\end{defn}
\begin{rmk}
Associativity means we can screw around with the order of evaluation, or more to
the point we can screw around with parentheses as we please. Most things (the
vast majority of things) are associative in math.
\end{rmk}

\begin{defn}
If $\star$ is a binary operation on a set $G$ we say elements $a,b$ of $G$
\textbf{commute} if
\begin{equation}
a\star b = b\star a.
\end{equation}
We say $\star$ (or $G$) is \textbf{commutative} if this holds for all $a,b\in G$.
\end{defn}

\begin{rmk}
Let $\star$ be a binary operation defined on $G$ and $H\subset G$ be a subset.
If for all $a,b\in H$ we have $a\star b\in H$, then $H$ is said to be \textbf{closed}
under $\star$. Observe further that if $\star$ is associative on $G$, then
necessarily it is associative on $H$.
\end{rmk}

\begin{defn}
A \textbf{group} is an ordered pair $(G,\star)$ where $G$ is a set and $\star$ 
is a binary operation on $G$ satisfying:
\begin{enumerate}
\item $(a\star b)\star c = a\star(b\star c)$ for all $a,b,c\in G$, i.e. $\star$ is
associative;
\item There exists an element $e\in G$ called the \textbf{identity} of $G$ such
that for all $a\in G$ we have $a\star e = e\star a = a$;
\item For each $a\in G$ there is an element $a^{-1}\in G$ called the \textbf{inverse} of $a$ such that $a\star a^{-1} = a^{-1}\star a = e$.
\end{enumerate}
\end{defn}

\begin{defn}
Let $(G,\star)$ be a group. If 
\begin{equation}
a\star b = b\star a
\end{equation}
holds for all $a,b\in G$, then the group is called \textbf{abelian}.
\end{defn}

\begin{rmk}
Abelian merely means the group has a commutative binary operation, or equivalently
the group is commutative.
\end{rmk}

\begin{thm}
Let $G$ be a group with the operation $\star$, then
\begin{enumerate}
\item The identity of $G$ is unique;
\item For each $a\in G$, $a^{-1}$ is unique;
\item $(a^{-1})^{-1}=a$ for all $a\in G$;
\item $(a\star b)^{-1}=(b^{-1})\star(a^{-1})$
\item For any $a_1,\ldots,a_n\in G$, the expression $a_1\star\ldots\star a_n$ is
independent of how the expression is bracketed (``generalized associativity law'').
\end{enumerate}
\end{thm}

\marginpar{Change of Notation}Note that the notation $a\star b$ gets tedious to write, so we instead just
write $a\cdot b$ or $ab$. 

\begin{thm}
Let $G$ be a group and $a,b\in G$. The equations $ax = b$ and $ya = b$ have
unique solutions for $x,y\in G$. In particular, the left and right cancellation
laws hold in $G$, i.e.
\begin{enumerate}
\item if $au = av$, then $u=v$, and
\item if $ub = vb$, then $u=v$.
\end{enumerate}
\end{thm}

\begin{defn}
For $G$ a group and $x\in G$, the \textbf{order} of $x$ is defined to be the
smallest positive integer $n$ such that 
\begin{equation}
x^n = 1
\end{equation}
and denote this integer by $|x|$. In this case, $x$ is said to be of order $n$.
If no positive power of $x$ is the identity, the order of $x$ is defined to be 
infinity and $x$ is said to be of infinite order.
\end{defn}

This last definition is useful only when dealing with finite groups, which we
may possibly deal with.

\begin{defn}
Let $G = \{g_1,\ldots,g_n\}$ be a finite group with $g_1=e$. The \textbf{multiplication table} or \textbf{group table} of $G$ is the $n\times n$ matrix
whose $i,j$ entry is the group element $g_ig_j$.
\end{defn}

\begin{ex}
Let our group be $\mathbb{Z}_2 = \{0,1\}$ with the usual multiplication. Then
the multiplication table is
\begin{equation}
\begin{array}{c|cc}
\times & 0 & 1 \\ \hline
0 & 0 & 0 \\
1 & 0 & 1 
\end{array}
\end{equation}
\end{ex}

\begin{defn}
Let $G$ be a group. If $H\subset G$ is closed under the binary operator $\star$,
and $H$ satisfies all the axioms of a group, then $H$ is called a \textbf{subgroup}.
\end{defn}

\begin{defn}
The binary operation on $G$ defined by the map $G\times G\to G$ defined by
\begin{equation}
(g,x)\mapsto gxg^{-1}
\end{equation}
is called \textbf{conjugation}.
\end{defn}

\begin{defn}
A subgroup $N$ of $G$ is called a \textbf{normal subgroup} if it has the
following property: For every $a\in N$ and every $b\in G$, the conjugate
$bab^{-1}$ is in $N$.
\end{defn}

\subsection{Homomorphisms and Isomorphisms}

\begin{defn}
Let $(G,\star)$ and $(H,\Diamond)$ be groups. A map $\varphi:G\to H$ such that
\begin{equation}
\varphi(x\star y) = \varphi(x)\Diamond\varphi(y)
\end{equation}
(for all $x,y\in G$) is called a \textbf{homomorphism}.
\end{defn}

\begin{ex}
The map
\begin{equation}
\{x^i,p_j\}=\delta^{i}_{j} \mapsto [\hat{x}^i,\hat{p}_j] = i\hbar\delta^{i}_{j}
\end{equation}
used in quantization of classical systems is an example of a homomorphism.
\end{ex}

\begin{defn}
A Homomorphism $\varphi:G\to H$ is a \textbf{isomorphism} if and only if $\varphi$ is bijective.
We say $G$ and $H$ are \textbf{isomorphic} and denote it as $G\cong H$.
\end{defn}

\begin{rmk}
An isomorphism is a homorphism that can be undone. We can think of it as a sort
of equivalence relation. That is, if $G\cong H$ then we can intuitively think
of $G$ and $H$ as being ``equivalent''.
\end{rmk}

\begin{defn}
The \textbf{kernel of a linear operator} $T$ is the set of all arguments which
are mapped to zero, i.e. the additive identity.
\end{defn}

\begin{defn}
The \textbf{kernel of a homomorphism} $\varphi: G\to G'$ is the set of elements
of $G$ mapped to the identity in $G'$:
\begin{equation}
\ker(\varphi) = \{ a\in G : \varphi(a) = 1 \}
\end{equation}
which can be described as $\varphi^{-1}(1)$ -- the inverse image of the identity.
The kernel is a subgroup because (for $a,b\in G$)
\begin{equation}
\varphi(ab)=\varphi(a)\varphi(b)=1\cdot 1 = 1
\end{equation}
thus $ab\in\ker(\varphi)$.
\end{defn}

\subsection{Centralizers, Normalizers, Stabilizers, etc.}

\begin{defn}
Let $H$ be a subgroup of $G$. A \textbf{left coset} is a subset of the form
\begin{equation}
aH = \{ ah : h\in H \}.
\end{equation}
Observe that the subgroup $H$ itself is a coset, i.e. $H=1H$.
\end{defn}

\begin{ex}
If we have a collection of vectors over the field $\mathbb{F}_2$ such that they
all satisfy some system of equations (they are eigenvectors of a given operator),
then we may construct a coset by adding some constant vector $\vec{v}$ to all
of these vectors.
\end{ex}

\begin{cor}
The left cosets of a subgroup partition the group.
\end{cor}



It is customary to denote the set of cosets of a normal subgroup $N$ of $G$ by
the notation
\begin{equation}
G/N = \textrm{set of cosets of }N\textrm{ in }G.
\end{equation}
We will also use the symbols $\bar{G} = G/N$.

We can now study a set $S$ on which the group $G$ operates, then decompose
$S$ into ``orbits'' which we will now define.

\begin{defn}
The \textbf{orbit} of $s\in S$ is the set
\begin{equation}
O_{s} = \{ gs : g\in G\}.
\end{equation}
It is a subset of $S$.
\end{defn}

\begin{defn}
The \textbf{stabilizer} of an element $s\in S$ is the subgroup $G_{s}$ of $G$ of
elements leaving $s$ fixed
\begin{equation}
G_{s} = \{ g\in G : gs = s \}.
\end{equation}
\end{defn}





\begin{defn}
The stabilizer of an element $x\in G$ for the operation of conjugation is called
the \textbf{centralizer} of $x$ and is denoted $Z(x)$:
\begin{equation}
Z(x) = \{ g\in G: gxg^{-1} = x\} = \{ g\in G: gx = xg\}.
\end{equation}
The centralizer is the set of group elements which commute with $x$.
\end{defn}

\section{Examples of Character Tables}
%%
%% characterTableEx.tex
%% 
%% Made by alex
%% Login   <alex@ubuntu>
%% 
%% Started on  Fri Jan 30 12:21:55 2009 alex
%% Last update Fri Jan 30 15:07:59 2009 alex
%%

\textbf{NOTE!} We will be working with irreps over a finite
dimensional vector space $V$ over the field $\mathbb{C}$ of
complex numbers.

%% \subsection{The Character Table for $S_{3}$}

%% Consider the symmetric group $S_3$. We have 
%% \begin{equation}

%% \end{equation}

\subsection{The Character Table for $D_3$}

Consider the group $D_3$ the symmetries of a regular
triangle. It has 6 elements, and the presentation
\begin{equation}
\<x,y|x^3=y^2=1,yx=x^{-1}y\>.
\end{equation}
It has, as mentioned, 6 elements
\begin{equation}
D_{3} = \{1,x,x^2,y,yx,yx^2\}.
\end{equation}
It has the multiplication table
\begin{equation}
\begin{array}{c|cccccc}
\times & 1   & x  & x^2 & y & yx & yx^2\\
\hline
1      & 1   & x  & x^2 & y & yx & yx^2\\
x^2    & x^2  & 1 & x   & yx & yx^2 & y\\ 
x      & x   & x^2 & 1 & yx^2 & y & yx\\
y      & y   & yx & yx^2 & 1 & x & x^2\\
yx     & yx  & yx^2 & y & x^2 & 1 & x\\
yx^2   & yx^2 & y & yx & x & x^2 & 1
\end{array}
\end{equation}
We find the cosets of the group
\begin{subequations}
\begin{align}
C(x) &= \{x^{k}xx^{-k}=x,
yxy=x^{-1},yx^{k}xyx^{k}=x^{-1}\}\\
&=\{x,x^2\}\\
C(y) &= \{x^kyx^{-k}=yx^{-2k}, yyy=y,
yx^{k}yyx^{k}=x^{-k}yx^{k}=yx^{2k}\}\\
&=\{y,yx,yx^2\}
\end{align}
\end{subequations}
Observe that $(x^2)^2=x^4=x$ in our group. Refer to the
multiplication table if in doubt. We can set up the
character table. We expect there to be some number of
irreducible representations $k$ such that the sum of the
squares of the dimensions of the irrep $\rho_k$ is the size
of the group. That is to say, 
\begin{equation}
d_{1}^{2} + d_{2}^{2} + \cdots + d_{k}^{2} = 6
\end{equation}
How many ways are there to do this? Well, observe $1^2=1<6$,
so $d_i$ can be at least 1. Further, $2^2=4<6$, so 2 is a
possible dimension. But $3^2=9>6$, so we can only have 1 or
2 dimensional irreps. How many different combinations are
there to write this? Trivially, two different ways
\begin{equation}
1^2+1^2+1^2+1^2+1^2+1^2=6,\quad\text{or }1^2+1^2+2^2=6.
\end{equation}
Those are the only two ways to write it! Further, we expect
the character table to be a square matrix, we have three
cosets: $C(1)$ (or as I denote it $C(e)$), $C(x)$, and
$C(y)$. That means we expect there to be three terms in our
sum of squares, and the only formula with three terms is
\begin{equation}
1^2+1^2+2^2=1+1+4=6.
\end{equation} 
So we have a good idea of how many irreps there are for
$D_3$. Let us now construct the character table:
\begin{equation}
\begin{array}{c|cccc}
\text{size of coset} & (1) & (2) & (3) \\
\text{coset rep} & 1 & x & y\\ \hline
\chi_1           &   &   &  \\ 
\chi_2           &   &   &  \\
\chi_3           &   &   & 
\end{array}
\end{equation}
Observe that we have the number of elements, which is
\emph{not standard} in character tables. I just like to add
them in as a reminder for quick reference. Now the first
column is the character of the irrep $\rho_i$. The first
irrep $\rho_1$ is always the trivial representation, it maps
everything to the identity in 1 dimension.
\begin{equation}
\begin{array}{c|cccc}
\text{size of coset} & (1) & (2) & (3) \\
\text{coset rep} & 1 & x & y\\ \hline
\chi_1           & 1 & 1 & 1\\ 
\chi_2           &   &   &  \\
\chi_3           &   &   & 
\end{array}
\end{equation}
Now we should remember that the character of the identity is
always the number of dimensions of the irrep. So we can fill
in the second column.
\begin{equation}
\begin{array}{c|cccc}
\text{size of coset} & (1) & (2) & (3) \\
\text{coset rep} & 1 & x & y\\ \hline
\chi_1           & 1 & 1 & 1\\ 
\chi_2           & 1 &   &  \\
\chi_3           & 2 &   & 
\end{array}
\end{equation}
We also have orthogonality of the rows (in a weighted inner
product), and orthonormality of the columns. So we can
\textbf{guess} the second row will be something of the form
\begin{equation}
\begin{array}{c|cccc}
\text{size of coset} & (1) & (2) & (3) \\
\text{coset rep} & 1 & x & y\\ \hline
\chi_1           & 1 & 1 & 1\\ 
\chi_2           & 1 & a & b\\
\chi_3           & 2 &   & 
\end{array}
\end{equation}
where $a,b\in\mathbb{C}$ are to be determined. We know that
the inner product between two characters is
\begin{equation}
\<\chi_i,\chi_j\> =
\frac{1}{n}\sum_{g}\overline{\chi_{j}(g)}\chi_{i}(g) = \delta_{ij}
\end{equation}
where $g$ is a coset representative and $\delta_{ij}$ is the
Kronecker delta, $\chi_i$ is the character for the irrep $\rho_i$. So we have
\begin{subequations}
\begin{align}
\<\chi_1,\chi_2\> = 1(1\cdot1) + 2(1\cdot\bar{a})+3(1\cdot\bar{b}) &= 0\\
\<\chi_2,\chi_1\> = 1(1)+2(a)+3(b) &= 0\\
\<\chi_2,\chi_2\> = 1(1\cdot1) + 2(a\cdot\bar{a})+3(b\cdot\bar{b}) &= 6.
\end{align}
\end{subequations}
So from the first two of these equations, we find $a,b$ are
real. We know that characters of irreps of finite groups are
integer combinations of roots of unity, so that means that
$a,b\in\mathbb{Z}$. From the condition that
\begin{equation}
1+2a+3b=0
\end{equation}
we find that
\begin{equation}
a=1,\;\; b=-1.
\end{equation}
So, we plug these into our character table
\begin{equation}
\begin{array}{c|cccc}
\text{size of coset} & (1) & (2) & (3) \\
\text{coset rep} & 1 & x & y\\ \hline
\chi_1           & 1 & 1 & 1\\ 
\chi_2           & 1 & 1 & -1\\
\chi_3           & 2 &   & 
\end{array}
\end{equation}
The orthonormality of the columns then demand that the rest
of the character table is trivially
\begin{equation}
\begin{array}{c|cccc}
\text{size of coset} & (1) & (2) & (3) \\
\text{coset rep} & 1 & x & y\\ \hline
\chi_1           & 1 & 1 & 1\\ 
\chi_2           & 1 & 1 & -1\\
\chi_3           & 2 & -1  & 0
\end{array}
\end{equation}
How do we see this? Take the first two columns and take
their inner product
\begin{equation}
1\cdot1+1\cdot1+2\cdot(-1)=0
\end{equation}
and similarly for the first and third column
\begin{equation}
1\cdot1-1\cdot1+0\cdot2=1-1=0.
\end{equation}
The second and third column are trivially orthogonal.

This is then the full character table for $D_3$. 

%% Let us try to construct the two dimensional irrep for
%% $D_3$. It is sufficient to do so for the coset
%% representatives I think. So we have $\rho_{3}(e)=I_{2}$ be
%% the two by two identity matrix, and for $\rho_{3}(x)$,
%% $\rho_{3}(y)$ -- since we do not know what they are! -- we
%% set up the matrices with unknown quantities we are solving
%% for. We know that $\chi_{3}(x)=-1$ and $\chi_{3}(y)=0$, so
%% that tells us that $\rho_{3}(x)$ has a trace of -1 and
%% $\rho_{3}(y)$ has a trace of 0 (it's traceless!). We set up
%% the matrices 
%% \begin{equation}
%% x = \begin{bmatrix}a & b\\ c & -(1+a)\end{bmatrix},\quad
%% y = \begin{bmatrix}\alpha & \beta\\ \gamma & -\alpha\end{bmatrix}.
%% \end{equation}
%% We can calculate the other elements of the group in our
%% representation by matrix multiplication
%% \begin{equation}
%% x^2 = \begin{bmatrix}a^2+bc & -b\\ -c & bc+(1+a)^2\end{bmatrix}
%% \end{equation}
%% so we find
%% \begin{equation}
%% yx = \begin{bmatrix} a\alpha+c\beta & b\alpha - (1+a)\beta\\
%% a\gamma-c\alpha & b\gamma+(1+a)\alpha\end{bmatrix}
%% \end{equation}
%% and
%% \begin{equation}
%% yx^2=xy=\begin{bmatrix}a\alpha+b\gamma & a\beta-b\alpha\\
%% c\alpha-(1+a)\gamma & c\beta+(1+a)\gamma\end{bmatrix}.
%% \end{equation}
%% From our character table, we demand that $yx$ and $yx^2$ are
%% traceless, and $x^2$ has a trace of -1. We have one more
%% condition worth mentioning, namely that $\rho_{3}(y)^2=I$
%% and $\rho_{3}(x)^3=I$, or in equation form
%% \begin{equation}
%% y^2 = \begin{bmatrix}\alpha^2+\beta\gamma & 0\\0 &
%%   \alpha^2+\beta\gamma\end{bmatrix} = I
%% \end{equation}
%% Similarly, an equation is derived from $x^3=I$ but we will
%% simplify our expression for $x$ first.

%% So we have a system of equations we need to solve
%% \begin{subequations}
%% \begin{align}
%% \tr(x^2) = 1+2a+2a^2+2bc &= -1\\
%% \tr(yx) = (1+2a)\alpha+(\beta+\gamma)c &= 0\\
%% \tr(yx^2) = (1+2a)\alpha+b\gamma+c\beta &= 0\\
%% (y^2=I) \;\; \alpha^2+\beta\gamma &= 1
%% \end{align}
%% \end{subequations} 
%% We can subtract the second from the third to find
%% \begin{equation}
%% b\gamma-c\gamma=0\Rightarrow b=c
%% \end{equation}
%% Plug this into the first equation we find
%% \begin{equation}
%% 1+a+a^2+b^2=0\Rightarrow b^2=a-(1+a)^2
%% \end{equation}
%% Things are getting slightly better, but now we can write the
%% matrix  representation of $x$ in a more convenient form
%% \begin{equation}
%% x = \begin{bmatrix}a & b\\b & -(1+a)\end{bmatrix}
%% \end{equation}
%% so
%% \begin{equation}
%% x^3 = \begin{bmatrix} a(a^2+b^2)-b^3 & -ab+b^3+b(1+a)^2\\
%% b(a^2+b^2)+(1+a)b & -b^2-(1+a)^3-b^{2}(1+a)\end{bmatrix} = I.
%% \end{equation}
%% This gives us an additional 4 equations.



\ifthenelse{\boolean{useIndex}}{
\printindex}{}
\nocite{*}
\bibliographystyle{utcaps}
\bibliography{algebra}
\end{document}
