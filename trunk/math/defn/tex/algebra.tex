%%
%% algebra.tex
%% 
%% Made by Alex Nelson
%% Login   <alex@tomato>
%% 
%% Started on  Fri Aug  7 12:32:43 2009 Alex Nelson
%% Last update Fri Aug  7 12:32:43 2009 Alex Nelson
%%
One of the basic algebraic tools:
\begin{defn}%\label{defn:}
Let $S$ be some set. A \define{Law of Composition}
$f:S\times{S}\to{S}$ consists of
\begin{itemize}
\item a pair of sets $\operatorname{dom}(f)=S\times{S}$, and $\operatorname{cod}((f)=S$
\end{itemize}
equipped with
\begin{itemize}
\item a set $f\subseteq\operatorname{dom}(f)\times\operatorname{cod}(f)$
\end{itemize}
such that
\begin{itemize}
\item for each $x\times{y}\in\operatorname{dom}(f)$, there is a corresponding
  $z\in{\operatorname{cod}(f)}$ such that $(x\times{y},z)\in{f}$.
\end{itemize}
\end{defn}
We can use it to introduce various mathematical objects, e.g.
\begin{defn}%\label{defn:}
A \define{Monoid} consists of
\begin{itemize}
\item a set $M$
\end{itemize}
equipped with
\begin{itemize}
\item a law of composition $*:M\times{M}\to{}M$
\item an identity element $e\in{M}$
\end{itemize}
such that
\begin{itemize}
\item the law of composition is associative,
  i.e. $(x*y)*z=x*(y*z)$ for all $x,y,z\in{M}$;
\item closure under the law of composition, i.e. $(x*y)\in{M}$
  for all $x,y\in{M}$;
\item the identity satisfies $e*x=x*e=x$ for all $x\in{M}$.
\end{itemize}
\end{defn}

\begin{defn}%\label{defn:}
A \define{Group} $G$ consists of
\begin{itemize}
\item a monoid $G$
\end{itemize}
equipped with
\begin{itemize}
\item an inversion operator $(\cdot{})^{-1}:G\to{G}$
\end{itemize}
such that
\begin{itemize}
\item $x^{-1}*x=x*x^{-1}=e$ for all $x\in{G}$.
\end{itemize}
\end{defn}
