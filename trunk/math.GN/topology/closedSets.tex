
\begin{prob}
We defined what it means for a set to be ``open'' by just demanding it
be in a topology on a space. So how do we define a set to be
``closed''? Intuitively, it's an ``open set that contains its
boundary'', but what is a ``boundary'' topologically?
\end{prob}

\begin{defn}
Let $X$ be a topological space, $A\subset X$ be some subset. We say
that $A$ is ``\define{Closed}'' if $X-A$ is open.
\end{defn}
\begin{rmk}
This is kind of weaseling out of the problem, just define a set to be
closed if its compliment is open. This doesn't seem intuitive or
immediately clear why one would want to define it this way, but we
will see why later on.
\end{rmk}

\begin{defn}
Let $X$ be a topological space, $Y\subset X$.
\begin{enumerate}
\item The \define{Closure} of $Y$ (denoted $\overline{Y}$ or
  $\operatorname{closure}(Y)$) is the intersection of all closed sets
  containing $Y$.
\item The \define{Interior} of $Y$ (denoted $Y^0$ or
  $\operatorname{int}(Y)$) is the union of all open sets contained in
  $Y$.
\end{enumerate}
\end{defn}
\begin{rmk}
Observe that the closure of a set is closed, the interior of a set is open.
\end{rmk}
\begin{rmk}
The closure of a set is the ``smallest'' closed set containing it, but
we conveniently avoided the problem of ``What do we mean by
`smallest'?'' by using intersections of closed sets. Similarly, the
interior of a set is the ``largest'' open set contained in it.
\end{rmk}
\begin{rmk}
A set $Y$ is closed iff $Y=\overline{Y}$ and it is open iff $Y=Y^0$.
\end{rmk}
\begin{defn}
Let $X$ be a topological space, $A\subset X$ be some subset. We define
the ``\textbf{Boundary}'' of $A$ (denoted $\partial A$) to be
\begin{equation}
\overline{A}\cap\overline{(X-A)}
\end{equation}
the intersection of the closure of $A$ with the closure of the
compliment of $A$.
\end{defn}
\begin{ex}
Consider the set $X=\{a,b,c\}$. Consider the topology
\begin{equation}
\mathcal{T} = \left\{\emptyset,\;\{a\},\; \{b\},\; \{a,b\},\; X\right\}
\end{equation}
We see that $\{a,c\}$ is closed since its compliment $\{b\}$ is open,
and we also see that $\{b,c\}$ is closed since its compliment $\{a\}$
is open. The intersection of these two closed sets $\{c\}$ is closed,
since its compliment is open and the intersection of closed sets is
closed.

Observe the closure of $\{a\}$ is given by the intersection of
all closed sets containing $a$:
\begin{equation}
\{a,c\}\cap\{a,b,c\}=\{a,c\}
\end{equation}
and the closure of $\{b,c\}$ (that is, the compliment of $\{a\}$) is
itself, i.e. $\{b,c\}$ is a closed set so its closure is itself. The
boundary of $\{a\}$ is then the intersection of these two sets
\begin{equation}
\partial \{a\} = \{b,c\}\cap\{a,c\} = \{c\}
\end{equation}
which we could not have found if we didn't define the boundary in a
topological way!
\end{ex}
\begin{rmk}
Let $X$ be a topological space, $A\subset X$ be a subset. The definition
of the boundary of $A$ is the same as the closure of $A$ minus its
interior
\begin{equation}
\partial A = \overline{A}-A^0.
\end{equation}
How can we see this? Well, the interior of $A$ is the union of all
open sets contained in $A$. Its compliment would be the closure of
$X-A$. We see that the intersection of $\overline{(X-A)}$ with
$\overline{A}$ is just the closure of $A$ intersected with the
compliment of its interior, i.e.
\begin{equation}
\overline{A}\cap\overline{(X-A)}=\overline{A}\cap\left(A^0\right)^C =
\overline{A}-A^0.
\end{equation}
This is by virtue of the property of compliments
\begin{equation}
(A-B)^C = A\cap B^C
\end{equation}
where $A$ and $B$ are subsets of some set $U$.
\end{rmk}

\begin{defn}
Let $X$ be a topological space, $Y\subset X$. A point $x\in X$ is 
a \define{Limit Point} of $Y$ if every neighborhood of $x$ intersects
$Y-\{x\}$.
\end{defn}
\begin{rmk}
This definition makes more sense given the notion of what a
``continuous function'' is, since we defined continuity in the
real analysis situation as the limit as $f(x)$ approaches
$f(x_0)$. We take this notion, and use the topological notion of
continuity to concoct a \emph{topological} notion of a limit.
\end{rmk}
