%%
%% main.tex
%% 
%% Made by Alex Nelson
%% Login   <alex@tomato>
%% 
%% Started on  Wed Jun  3 16:29:04 2009 Alex Nelson
%% Last update Wed Jun  3 16:29:04 2009 Alex Nelson
%%
\documentclass{amsart}
\usepackage{url}
\usepackage{manfnt}
\usepackage{amsthm}
\usepackage{amsmath}
\usepackage{amsthm}
\usepackage{amssymb}
\usepackage{amsfonts}
\usepackage{amscd}
\usepackage{graphicx}
\usepackage{mathrsfs}

\numberwithin{equation}{section}

\theoremstyle{definition}
\newtheorem{defn}{Definition}
\newtheorem{thm}{Theorem}
\newtheorem{rmk}{Remark}
\newtheorem{lem}{Lemma}
\newtheorem{cor}{Corollary}
\newtheorem{ex}{Example}
\newtheorem{prop}{Proposition}
\newtheorem{sch}{Scholium}
\newtheorem{axm}{Axiom}
\newtheorem*{prob}{Problem}

\def\re{\operatorname{Re}}
\def\tr{\operatorname{Tr}}
\def\<{\langle}
\def\>{\rangle}

%%
% This macro header is what controls the ``dangerous bend''
% paragraph
%%
\def\rd{\noindent\begingroup\hangindent=3pc\hangafter=-2\def\par{\endgraf\endgroup}\hbox
  to0pt{\hskip-\hangindent\dbend\hfill}\ignorespaces}
%%
% This command allows you to write stuff in small font size and
% use the
% bourbaki ``dangerous bend'' so it's great when you want to
% ramble on 
% about some extra stuff!
%%
\newcommand{\danger}[1] {\rd{\small {#1}}}

%%
% This macro header is what controls the ``dangerous bend''
% paragraph
%%
\def\ddbend{\dbend\kern1pt\dbend}

\def\rdd{\noindent\begingroup\hangindent=4pc\hangafter=-2\def\par{\endgraf\endgroup}\hbox
  to0pt{\hskip-\hangindent\ddbend\hfill}\ignorespaces}

\newcommand{\ddanger}[1] {\rdd{\small {#1}}}

\newcommand{\define}[1] {\textbf{#1}\index{#1}}
\title{Notes on the Quantum Harmonic Oscillator}
\date{June 03, 2009}
\email{pqnelson@gmail.com}
\author{Alex Nelson}
\begin{document}
\begin{abstract}
We briefly introduce the notion of creation and annihilation
operators through a change of coordinates. We introduce the
number operator, its relation to the Hamiltonian operator, and
find the vacuum state.
\end{abstract}
\maketitle
\section{Introduction}
%%
%% intro.tex
%% 
%% Made by Alex Nelson
%% Login   <alex@tomato>
%% 
%% Started on  Wed Jun  3 16:30:09 2009 Alex Nelson
%% Last update Wed Jun  3 16:30:09 2009 Alex Nelson
%%
Recall for classical mechanics, the Harmonic oscillator potential
is
\begin{equation}%\label{eq:}
V(x) = \frac{1}{2}kx^2 = \frac{1}{2}m\omega^{2}x^{2}
\end{equation}
where $\omega$ is the angular velocity. We plug this into
Schrodinger's equation
\begin{equation}%\label{eq:}
\left[\frac{-\hbar^2}{2m}\frac{\partial^{2}}{\partial x^{2}}+\frac{1}{2}kx^{2}\right]|\psi\>=E|\psi\>.
\end{equation}
How to solve this? Well, there are two ways: the smart way and
the stupid way. We'll do it the smart way.

We introduce a change of variable\footnote{Note that this is done
  \emph{classically}, that is \emph{before} quantization. After
  quantization changing coordinates is always a fuzzy
  subject. These two steps are done tacitly in most derivations,
  but it should be known in the back of one's mind what's going on.}:
\begin{equation}%\label{eq:}
Q = \sqrt{\frac{m\omega}{2\hbar}},\qquad
\widehat{P}=\frac{\partial}{\partial Q}=\sqrt{\frac{-1}{2m\omega\hbar}}\widehat{p}.
\end{equation}
Note these are dimensionless and simplify computations
significantly. We can factor the Hamiltonian, since in these new
variables we have
\begin{equation}%\label{eq:}
\hbar\omega\left[Q^{2}-\frac{\partial^2}{\partial Q^2}\right] = \widehat{H}.
\end{equation}
We want to use the coordinates
\begin{equation}%\label{eq:}
a=\left[Q+\frac{\partial}{\partial Q}\right],\qquad a^{\dag}=\left[Q-\frac{\partial}{\partial Q}\right]
\end{equation}
which we call ``\define{annihilation and creation operators}''
respectively.

Observe that
\begin{subequations}
\begin{align}
2a^{\dag}a &= \left[Q+\frac{\partial}{\partial
    Q}\right]\left[Q-\frac{\partial}{\partial Q}\right]\\
&= Q^2 + \frac{\partial}{\partial Q} Q -
Q\frac{\partial}{\partial Q} - \frac{\partial^{2}}{\partial
  Q^{2}}\\
&= Q^{2}-\frac{\partial^2}{\partial Q^2} + [\widehat{P},Q].
\end{align}
\end{subequations}
All computation is by definition and substitution. Nothing too
fancy so far. We can now write the Hamiltonian operator as
\begin{equation}%\label{eq:}
\widehat{H} = \hbar\omega\left(a^{\dag}a+\frac{1}{2}[Q,\widehat{P}]\right)
\end{equation}
Observe that
\begin{equation}%\label{eq:}
[Q,\widehat{P}] = 1
\end{equation}
since $Q$ and $\widehat{P}$ are the dimensionless counterparts to
$x$ and $\widehat{p}$ which implies we set $\hbar\to1$. We
end up with the form of the Hamiltonian operator
\begin{equation}%\label{eq:}
\widehat{H} = \hbar\omega\left(a^{\dag}a+\frac{1}{2}\right).
\end{equation}
It would then be logical to investigate how these creation and
annihilation operators behave.

\section{Ladder Operators}
%%
%% ladder.tex
%% 
%% Made by Alex Nelson
%% Login   <alex@tomato>
%% 
%% Started on  Mon Mar 30 10:40:35 2009 Alex Nelson
%% Last update Mon Mar 30 10:40:35 2009 Alex Nelson
%%
\documentclass{amsart}
\numberwithin{equation}{section}
\title{Notes on Ladder Operators}
\author{Alex Nelson}
\date{March 30, 2009}
\begin{document}
\maketitle
\section{Angular Momentum in Quantum Mechanics}
%%
%% angular.tex
%% 
%% Made by Alex Nelson
%% Login   <alex@tomato>
%% 
%% Started on  Mon Mar 30 11:07:25 2009 Alex Nelson
%% Last update Mon Mar 30 11:07:25 2009 Alex Nelson
%%

In classical mechanics, we described the angular momentum of a
body by 
\begin{equation}
\vec{x}\times\vec{p}=\vec{L}.
\end{equation}
In quantum mechanics, we do the same thing more or less, but we
need to make the replacement
\begin{equation}
p\to\widehat{p}=\frac{\hbar}{i}\frac{\partial}{\partial x}
\end{equation}
and so on. We will recklessly switch notations at random between
using hats to denote operators and not using hats. It is
understood, unless otherwise specified, we will work henceforth
with operators.

Here it is convenient to note that we can write the cross product
in a slick way using matrix multiplication. Observe
\begin{equation}
\vec{a}\times\vec{b} = \begin{bmatrix}0 & a_3 & -a_2\\
-a_3 & 0 & a_1\\
a_2 & -a_1 &
0\end{bmatrix}\begin{bmatrix}b_1\\b_2\\b_3\end{bmatrix}.
\end{equation}
Also observe that
\begin{equation}
\begin{bmatrix}0 & a_3 & -a_2\\
-a_3 & 0 & a_1\\
a_2 & -a_1 &
0\end{bmatrix}^2 = - \begin{bmatrix} a_{2}^{2}+a_{3}^{2} & -a_1a_2 & -a_1a_3\\
-a_1a_2 & a_{1}^{2}+a_{3}^{2} & -a_{2}a_{3}\\
-a_{1}a_{3} & -a_{2}a_{3} & a_{1}^{2}+a_{2}^{2}
0\end{bmatrix}.
\end{equation}
If we perform that computations, we find that (using, apologies
to the physicists struggling with it, abstract index notation)
\begin{subequations}
\begin{align}
L^{2} &= \vec{L}\cdot\vec{L}\\
&=
\sum_{i,j,k,m,n}\epsilon^{ijk}\epsilon^{imn}x^{j}p_{k}x^{m}p_{n}\\
&= \sum_{i,j,k,m,n}(\delta^{jm}\delta^{kn}-\delta^{jn}\delta^{km})x^{j}p_{k}x^{m}p_{n}\\
&= \sum_{i,j,k,m,n}x^{j}p_{k}x^{j}p_{k} - x^{j}p_{k}x^{k}p_{j}.
\end{align}
\end{subequations}
We can using the commutation relations
\begin{equation}
[\widehat{x}^{i},\widehat{p}_{j}]=i{\delta^{i}}_{j}
\end{equation}
to reorder terms, for example
\begin{subequations}
\begin{align}
p_{k}x_{j} &= x_{j}p_{k}-i\delta_{kj}\\
x_{j}p_{k} &= p_{k}x_{j}+i\delta_{kj}.
\end{align}
\end{subequations}
We can rewrite the angular momentum squared as
\begin{subequations}
\begin{align}
L^2 &= x^{j}(x{j}p_{k} -
i{\delta^{j}}_{k})p_{k}-(p_{k}x^{j}+i{\delta^{j}}_{k})x_{k}p_{j}\\
&=
x^{j}x_{j}p^{k}p_{k}-ix^{j}p_{j}-p_{k}x^{k}x^{j}p_{j}-ix^{j}p_{j}\\\
&=x^{j}x_{j}p^{k}p_{k}-2ix^{j}p_{j}-(x^{k}p_{k}-i{\delta^{k}}_{k})x^{j}p_{j}.
\end{align}
\end{subequations}

\section{Rotations in Two and Three Dimensions}
%%
%% rotations.tex
%% 
%% Made by Alex Nelson
%% Login   <alex@tomato>
%% 
%% Started on  Mon Mar 30 10:41:53 2009 Alex Nelson
%% Last update Mon Mar 30 10:41:53 2009 Alex Nelson
%%

Recall in 2 dimensions, we can use complex analysis to simplify
rotations. That is, due to Euler's formula
\begin{equation}
e^{i\theta}=\cos(\theta)+i\sin(\theta)
\end{equation}
where $i^2=-1$, we can write any ordered pair $(x,y)$ as
\begin{equation}
x+iy=re^{i\theta_0}.
\end{equation}
If we want to rotate by some angle $\theta$ anticlockwise, we
simply multiply by $\exp(i\theta)$ to find
\begin{equation}
(x+iy)e^{i\theta} = (x\cos(\theta)-y\sin(\theta))+i(x\sin(\theta)+y\cos(\theta)).
\end{equation}
We can write this in matrix form as
\begin{equation}
\begin{bmatrix}
x'\\y'
\end{bmatrix}
=\begin{bmatrix} \cos(\theta) & -\sin(\theta)\\
\sin(\theta) & \cos(\theta)\end{bmatrix}\begin{bmatrix}x\\y\end{bmatrix}.
\end{equation}
We denote the rotated coordinates with primes.

For three dimensions, we can rotate about the $x$, $y$, or $z$
axis. These have similar forms as their two dimensional
counterparts. Namely a rotation about the $x$ axis demands
$x'=x$, so
\begin{equation}
R_{x}(\alpha) = \begin{bmatrix}1 & 0 & 0\\
0 & \cos(\alpha) & -\sin(\alpha)\\
0 & \sin(\alpha) & \cos(\alpha)\end{bmatrix}\end{equation}
which is intuitively clear since we treat the $y-z$ plane as a
two dimensional plane and ``rotate in it''. Similarly, for the
rotation about the $z$ axis we have for anticlockwise rotations
by an angle $\gamma$
\begin{equation}
R_{z}(\gamma) = \begin{bmatrix}\cos(\gamma) & -\sin(\gamma) & 0\\
\sin(\gamma) & \cos(\gamma) & 0\\
0 & 0 & 1\end{bmatrix}.
\end{equation}
For rotations about the $y$ axis, it is a bit more tricky if we
wish to maintain sign. That is, the sign of the sine changes,
otherwise we'd do a clockwise rotation. (Read twice, and prove
this important fact to yourself. It will become evident later
on.) For an anticlockwise rotation about the $y$ axis by an angle
$\beta$ we have
\begin{equation}
R_{y}(\beta) = \begin{bmatrix}\cos(\beta) & 0 & \sin(\beta)\\
0 & 1 & 0\\
-\sin(\beta) & 0 & \cos(\beta)\end{bmatrix}
\end{equation}
Observe the change of signs on sine. Remember too that
$\sin(-x)=-\sin(x)$ for a huge hint why.

An additional property of rotations that are worthy of note is
that we may compose them. That is
\begin{equation}
R(\alpha)R(\beta)=R(\alpha+\beta)
\end{equation}
for some rotation $R(\cdot)$. This means we can write a rotation
by an angle $\theta$ as
\begin{equation}
R(\theta) = \left[R\left(\frac{\theta}{N}\right)\right]^{N}
\end{equation}
for some $N\in\mathbb{N}$. For large $N$ we make the
approximations
\begin{subequations}
\begin{align}
\cos(\theta/N)&\approx 1\\
\sin(\theta/N)&\approx \frac{\theta}{N}.
\end{align}
\end{subequations}
So we can rewrite our rotation matrices as the sum of two
matrices
\begin{subequations}
\begin{align}
R_{x}\left(\frac{\alpha}{N}\right) &= \left[I +
\left(\frac{\alpha}{N}\right)T_{x}\right]^{N}\\
R_{y}\left(\frac{\beta}{N}\right) &= \left[I +
\left(\frac{\beta}{N}\right)T_{y}\right]^{N}\\
R_{z}\left(\frac{\gamma}{N}\right) &= \left[I +
\left(\frac{\gamma}{N}\right)T_{z}\right]^{N}
\end{align}
\end{subequations}
where we have silently introduced the matrices
\begin{subequations}
\begin{align}
T_{x} &= \begin{bmatrix}0 & 0 & 0\\
0 & 0 & -1\\
0 & 1 & 0\end{bmatrix}\\
T_{y} &= \begin{bmatrix}0 & 0 & 1\\
0 & 0 & 0\\
-1 & 0 & 0\end{bmatrix}\\
T_{z} &= \begin{bmatrix}0 & -1 & 0\\
1 & 0 & 0\\
0 & 0 & 0\end{bmatrix}.
\end{align}
\end{subequations}
Observe that by formally taking the limit $N\to\infty$, we have
for some rotation operator $R$
\begin{equation}
R(\theta) = \lim_{N\to\infty}\left[I+\frac{\theta}{N}T\right]^{N}=\exp(\theta
T)
\end{equation}
where $\exp(\cdot)$ here is matrix exponentiation. We simply plug
in the matrix into the Taylor series of $e^x$, using matrix
multiplication and matrix addition. 



\end{document}

\nocite{sakurai}
\bibliographystyle{plain}
\bibliography{main}
\end{document}
