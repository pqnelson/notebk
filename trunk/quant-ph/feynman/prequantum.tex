\section{Klein-Gordon Review}

Recall that the Schrodinger equation for the free nonrelativistic particle is
\begin{equation}
\frac{-\hbar^2}{2m}\nabla^2 \ket{\psi} = -i\hbar\partial_{t}\ket{\psi}
\end{equation}
which corresponds to a sort of quantized Newton's second law
\begin{equation}
\frac{p^2}{2m} \approx E.
\end{equation}
However, in special relativity we have the mass shell constraint
\begin{equation}\label{massShell}
p^\mu p_\mu = E^2 - \bold{p}\cdot\bold{p} = m^2
\end{equation}
(when $c=1$) using Einstein summation convention. If we naively quantize this, we end up with
\begin{equation}\label{KGeqn}
\partial_{\mu}\partial^{\mu} - m^2 \ket{\psi} = 0
\end{equation}
by moving the mass term onto the left hand side. This is the Klein-Gordon equation, it is plagued by problems such as negative probabilities, etc.


%%%%%%%%%%%%%%%%%%%%%%%%%%%%%%%%%%%%%%%%%%%%%%%%%%%%%%%%%%%%%%%%%%%%%%%%%%

\section{Dirac takes it up a notch...bam!}

Naively, we want something simpler than this. We can rewrite Eq (\ref{massShell}) to be
\begin{equation}
E^2 = \bold{p}\cdot\bold{p} + m^2\Rightarrow E = \sqrt{\bold{p}\cdot\bold{p} + m^2}
\end{equation}
then quantize it to be
\begin{equation}
i\hbar\frac{\partial}{\partial t}\ket{\psi} = \sqrt{-\hbar^2\nabla^2 + m^2}\ket{\psi}.
\end{equation}
We end up being forced to use pseudo-differential operators, unfortunately, and it turns out that this results in nonlocality\footnote{In general, whenever there is a squareroot quantity, there is nonlocality.}. For further details see Laemmerzahl~\cite{pseudodifferentialKG}.

The approach Dirac takes is basically taking the squareroot of the operator, but he does it with class. He uses a clifford algebra with generators $\gamma^\mu$\footnote{If the reader is unfamiliar with the Gamma Matrices, see the appendix A and/or CORE~\cite{Borodulin:1995xd}.} such that the squareroot of the Klein Gordon equation breaks into two equations:
\begin{equation}\label{diracEqn}
(i\hbar\gamma^\mu\partial_\mu - m)\psi(x) = 0
\end{equation}
where $\psi(x)$ is a spinor wave function with 4 components. The adjoint field $\bar{\psi}(x)$ is defined by 
\begin{equation}
\bar{\psi}(x) = \psi^{\dag}(x)\gamma^0
\end{equation}
and satisfies the \emph{adjoint} Dirac equation
\begin{equation}\label{adjointEqn}
\bar{\psi}(x)(i\hbar\gamma^\mu\partial_\mu + m) = 0.
\end{equation}
It is to be understood here the differential operator $\partial^\mu$ acts on the left. Observe that when we multiply the two operators together we get
\begin{equation}
(i\hbar\gamma^\mu\partial_\mu + m)(i\hbar\gamma^\mu\partial_\mu - m) = -\hbar^{2}\left(\gamma^\mu\right)^2\partial_{\mu}\partial^{\mu} - m^2 = \hat{p}^\mu\hat{p}_\mu - m^2 
\end{equation}
which is \emph{precisely} the Klein-Gordon operator (\ref{KGeqn})! We should be content now with the connection back to what we already know. \notetoself{And an added advantage is that the Dirac equation is a first order partial differential equation, whereas the Klein-Gordon equation is a second order one!}

%%%%%%%%%%%%%%%%%%%%%%%%%%%%%%%%%%%%%%%%%%%%%%%%%%%%%%%%%%%%%%%%%%%%%%%%%%

\subsection{A Somewhat Rigorous Derivation of the Dirac Equation}

We want the squareroot of the wave operator thus
\begin{equation}
\nabla^2 - \frac{1}{c^2}\frac{\partial^2}{\partial t^2} = (A \partial_x + B \partial_y + C \partial_z + \frac{i}{c}D \partial_t)(A \partial_x + B \partial_y + C \partial_z + \frac{i}{c}D \partial_t)
\end{equation}
We see multiplying out the right hand side all the cross-terms must vanish. To have this we want
\begin{equation}
AB + BA = 0, 
\end{equation}
and so on for all cross-term coefficients, with the property that
\begin{equation}
A^2 = B^2 = C^2 = D^2 = 1.
\end{equation}
Dirac had previously worked out rigorous results with Heisenberg's matrix mechanics, and concluded that these conditions could be met if $A, B,\ldots$ were \emph{matrices} which has the implication that the wave function has \emph{multiple components}. 

In the mean time, Pauli had been working on quantum mechanics as well. Pauli had a model with two-component wave functions that was involved in a phenomenological theory of spin. At this point in time, spin was not well understood. 

Given the factorization of these matrices, one can now write down immediately an equation
\begin{equation}
(A\partial_x + B\partial_y + C\partial_z + \frac{i}{c}D\partial_t)\psi = \kappa\psi
\end{equation}
with $\kappa$ to be determined. Applying the same operation on either side yields
\begin{equation}
(\nabla^2 - \frac{1}{c^2}\partial_{t}^2)\psi = \kappa^2\psi.
\end{equation}
If one take $\kappa = mc/\hbar$ we find that all the components of the wave function \emph{individually} satisfy the mass-shell relation (\ref{massShell}). Thus we have a first order differential equation in both space and time described by
\begin{equation}
(A\partial_x + B\partial_y + C\partial_z + \frac{i}{c}D\partial_t - \frac{mc}{\hbar})\psi = 0
\end{equation}
where $(A,B,C)=i\beta\alpha_k$ and $D=\beta$, which is precisely the Dirac equation for a spin-1/2 particle of rest mass $m$.


%%%%%%%%%%%%%%%%%%%%%%%%%%%%%%%%%%%%%%%%%%%%%%%%%%%%%%%%%%%%%%%%%%%%%%%%%%

\subsection{A Comparison to the Pauli Theory}

The necessity of introducing half-integer spin goes back experimentally to the results of the Stern-Gerlach experiment. \notetoself{A beam of atoms is run through a strong inhomogeneous magnetic field, which then splits into N parts depending on the intrinsic angular momentum of the atoms. It was found that for silver atoms, the beam was split in two - the ground state therefore could not be integral, because even if the intrinsic angular momentum of the atoms were as small as possible, 1, the beam would be split into 3 parts, corresponding to atoms with $L_z = -1, 0, +1$. The conclusion is that silver atoms have net intrinsic angular momentum of 1/2.} Pauli set up a model which explained the splitting by introducing a two-component wave-function and a corresponding correction term in the Hamiltonian(representing a semiclassical coupling of this wave function to an applied magnetic field) as
\begin{equation}
H = \frac{1}{2m}(\sigma^{I}_{i}(p^{i} - \frac{e}{c}A^{i})\sigma_{Ij}(p^{j} - \frac{e}{c}A^{j})) + e A^0 \qquad (i,j,I=1,2,3).
\end{equation}
We have here $A^\mu$ is the magnetic potential, and the van Warden symbols $\sigma^{J}_{j}$ which translates a vector into the Pauli matrix basis (if one is unfamiliar with Pauli matrices, see \S\ref{Representations of the Gamma Matrices}), $e$ is the electric charge of the particle (here $e=-e_0$ for the electron), and $m$ is the mass of the particle. Now we have just described the Hamiltonian of our system by a 2 by 2 matrix. The Schrodinger equation based on it
\begin{equation}
H \phi = i\hbar \frac{\partial\phi}{\partial t}
\end{equation}
must use a two-component wave function. (If you're like me you're too lazy to flip to the appendix, so I'll reproduce some of it here)\marginpar{SU(2) is the set of all 2 by 2 matrices that is self-adjoint and has a determinant of 1} Pauli used the SU(2) matrices
\begin{equation}\label{Pauli}
\sigma_k = \begin{bmatrix} 0 & 1 \\ 1 & 0 \end{bmatrix},\begin{bmatrix} 0 & -i \\ i & 0 \end{bmatrix},\begin{bmatrix} 1 & 0 \\ 0 & -1 \end{bmatrix}
\end{equation}
due to \emph{phenomenological reasons} (explaining the Gerlach experiment). Dirac now has a \emph{theoretical argument} that implies spin is a \emph{consequence} of introducing special relativity into quantum theory.

The Pauli matrices share the same properties as the Dirac matrices -- they are all self-adjoint, when squared are equal to the identity, and they anticommute. We can now use the Pauli matrices Eq (\ref{Pauli}) to describe a representation of the Dirac matrices:
\begin{equation}
\alpha_k = \begin{bmatrix} 0 & \sigma_k \\ \sigma_k & 0 \end{bmatrix}\qquad \beta = \begin{bmatrix} 1_2 & 0 \\ 0 & -1_2 \end{bmatrix}.
\end{equation}
We now may write the Dirac equation as an equation coupling two-component spinors:
\begin{equation}
\begin{bmatrix} mc^2 & c\sigma\cdot p \\ c\sigma\cdot p & -mc^2 \end{bmatrix} \begin{bmatrix} \phi_+ \\ \phi_- \end{bmatrix} = i\hbar\frac{\partial}{\partial t}\begin{bmatrix} \phi_+ \\ \phi_- \end{bmatrix}.
\end{equation}
Observe that we have on the diagonal the rest mass. If we bring the particle to rest, we have
\begin{equation}
i\hbar\frac{\partial}{\partial t}\begin{pmatrix} \phi_+ \\ \phi_- \end{pmatrix} = \begin{pmatrix} mc^2 & 0 \\ 0 & -mc^2 \end{pmatrix} \begin{pmatrix} \phi_+ \\ \phi_- \end{pmatrix}.
\end{equation}
The equations for the individual two-spinors are now decoupled, and we see that the ``spin-up'' and ``spin-down'' (or ``right-handed'' and ``left-handed'', ``positive frequency'' and ``negative frequency'' respectively) are individual eigenfunctions \snote{Eigenspinors?} of the energy with eigenvalues equal to $\pm$ the rest energy. The appearence of \emph{negative} energy should not be alarming, it is completely consistent with relativity.

\textbf{Note} that this seperation is in the rest frame and \textbf{is not an invariant statement} -- the bottom component does not generally represent antimatter. The \emph{entire} four-component spinor represents an \emph{irreducible whole} -- in general states will have an admixture of positive \emph{and} negative energy components.

\subsection{Covariant Form and Relativistic Invariance}

The explicity covariant form of the Dirac Equation is (using Einstein summation convention)
\begin{equation}
i\hbar\gamma^\mu\partial_\mu\psi - mc\psi = 0,
\end{equation}
where $\gamma^\mu$ are the Dirac gamma matrices. We have
\begin{equation}
\gamma^0 = \beta \qquad \gamma^k = \gamma^0\alpha_k.
\end{equation}
See the appendix for more details on this representation.

The Dirac equation may be interpreted as an eigenvalue expression, where the rest mass is proportional to an eigenvalue of the 4-momentum operator, the proportion being $c$:
\begin{equation}
\hat{P}\psi = mc\psi.
\end{equation}
In practice we often work in units where we set $\hbar$ and $c$ to be 1. The equation is multiplied by $-i$ and takes the form
\begin{equation}
\left(\gamma^\mu\partial_\mu + im \right)\psi = 0.
\end{equation}
We may employ the Feynman slash notation to simplify this to
\begin{equation}
(\slashed{\partial} + im)\psi = 0. 
\end{equation}
For any two representations of the Dirac Gamma matrices, they are related by a unitary transformation. Likewise, the solutions in the two representations are related by the same way.

%%%%%%%%%%%%%%%%%%%%%%%%%%%%%%%%%%%%%%%%%%%%%%%%%%%%%%%%%%%%%%%%%%%%%%%%%

\subsection{Conservation Laws and Canonical Structure}

Recall the Dirac equation and its adjoint version, Eqns (\ref{diracEqn}) and (\ref{adjointEqn}). We notice from the definition of the adjoint
\begin{equation*}
\bar{\psi} = \psi^\dag\gamma^0
\end{equation*}
that
\begin{equation}
\left(\gamma^\mu\right)^\dag\gamma^0 = \gamma^0\gamma^\mu
\end{equation}
we can obtain the Hermitian conjugate of the Dirac equation and multiplying from the right by $\gamma^0$  we get its adjoint version:
\begin{equation*}
\bar{\psi}(\gamma^\mu\overleftarrow{\partial}_\mu - im) = 0
\end{equation*} 
where $\overleftarrow{\partial}_\mu$ acts on the left. When we multiply the Dirac equation by $\bar{\psi}$ from the left
\begin{equation}
\bar{\psi}(\gamma^\mu\overrightarrow{\partial}_\mu + im)\psi = 0
\end{equation}
(where $\overrightarrow{\partial}_\mu$ acts on the right) and multiply the adjoint equation by $\psi$ on the right
\begin{equation}
\bar{\psi}(\gamma^\mu\overleftarrow{\partial}_\mu - im)\psi = 0
\end{equation}
then add the two together we get\marginpar{Conservation Law of Dirac Current}
\begin{equation}
\bar{\psi}(\gamma^\mu\overrightarrow{\partial}_\mu + im)\psi + \bar{\psi}(\gamma^\mu\overleftarrow{\partial}_\mu - im)\psi = \partial(\bar{\psi}\gamma^\mu\psi) = \partial_\mu J^\mu 0
\end{equation}
(where $J^\mu$ is the Dirac Current) which is the law of conservation of the Dirac current in covariant form. We see the huge advantage this has over the Klein-Gordon equation: this has conserved probability current desnity as required by relativistic invariance...only now its temporal component is \emph{positive definite}:
\begin{equation}
J^0 = \bar{\psi}\gamma^0\psi = \bar{\psi}\psi.
\end{equation}
From this we can find a conserved charge
\begin{equation}
Q = q\int \psi^\dag(\bold{x})\psi(\bold{x})d^3x
\end{equation}
where $q$ is to be thought of as ``charge''.

We can now see that the Dirac equation (and its adjoint) are the Euler-Lagrange equations of motion of the four dimensional invariant action
\begin{equation}
S = \int \mathcal{L}d^4x
\end{equation}
where the Dirac Lagrangian density $\mathcal{L}$ is given by
\begin{equation}
\mathcal{L} = c\bar{\psi}(x)\Big[ i\hbar\gamma^\mu\partial_\mu - mc \Big]\psi(x)
\end{equation}
and for purposes of variation, $\psi$ and $\bar{\psi}$ are considered to be independent fields. Relativistic invariance follows from the variational principle.

\marginpar{Canonical Structure, Hamiltonian and Momentum operators}We can find the canonically conjugate momenta to the fields $\psi$ and $\bar{\psi}$:
\begin{equation}
\pi(x) = \frac{\partial\mathcal{L}}{\partial\dot{\psi}} = i\hbar\psi^\dag\qquad\bar{\pi}(x) = \frac{\partial\mathcal{L}}{\partial\dot{\bar{\psi}}} = 0.
\end{equation}
We can find the Hamiltonian of the Dirac field
\begin{equation}\label{Hamiltonian}
H = \int d^{3}x( \pi_{\alpha}(x)\dot{\psi}^{\alpha}(x) - \mathcal{L} ) = \int d^{3}x \bar{\psi}(x)[-i\hbar c\gamma^{j}\partial_{j} + mc]\psi(x).
\end{equation}
Similarly, the momentum of some field $\phi$ to be given by
\begin{equation*}
cP^\alpha \equiv \int d^3x\mathcal{T}^{0\alpha} = \int d^{3}x\left[c\pi_r(x)\frac{\partial\phi_r(x)}{\partial x_{\alpha}} - \mathcal{L}\eta^{0\alpha}\right]
\end{equation*}
where $\mathcal{T}^{\alpha\beta}$ is the stress-energy density tensor of the field $\phi$. Recall that we define the stress-energy density tensor by the equation
\begin{equation}
\mathcal{T}^{\alpha\beta}\equiv \frac{\partial\mathcal{L}}{\partial\phi_{r,\alpha}}\frac{\partial\phi_r}{\partial x_\beta} - \mathcal{L}\eta^{\alpha\beta}.
\end{equation}
Using this, we can find the momentum of the Dirac Field to be
\begin{equation}
\bold{P} = -i\hbar\int d^3x\psi^\dag(x)\nabla\psi(x).
\end{equation}
Of course, the Hamiltonian given by (\ref{Hamiltonian}) could have been discovered by finding the Hamiltonian density applied to the current case.

We\marginpar{Angular Momentum} can similarly find the angular momentum of the Dirac Field by simply following the scheme of finding the angular momentum for a general field. That is, an infinitesmal transformation of the coordinates
\begin{equation}
x_\alpha\to x'_\alpha\equiv x_\alpha + \delta x_\alpha = x_\alpha + \varepsilon_{\alpha\beta}x^\beta + \delta_\alpha
\end{equation}
(where $\delta_\alpha$ is an infinitesmal displacement and $\varepsilon_{\alpha\beta}$ is an infinitesmal antisymmetric tensor to ensure invariance of $x_\alpha x^\alpha$ under homogeneous Lorentz transformations, i.e. ones with $\delta_\alpha=0$) induces an infinitesmal transformation of the field $\phi$:
\begin{equation}
\phi_r(x)\to\phi'_r(x') = \phi_r(x) + \frac{1}{2}\varepsilon_{\alpha\beta}S^{\alpha\beta}_{rs}\phi_{s}(x).
\end{equation}
Here the coefficients $S^{\alpha\beta}_{rs}$ are antisymmetric in $\alpha$ and $\beta$, like $\varepsilon_{\alpha\beta}$, and are determined by the transformation properties of the fields.

For a rotation (i.e. $\delta_\alpha=0$) we have the continuity equation
\begin{equation}
\frac{\partial\mathcal{M}^{\alpha\beta\gamma}}{\partial x^{\alpha}} = 0
\end{equation}
where
\begin{equation}
\mathcal{M}^{\alpha\beta\gamma}\equiv\frac{\partial\mathcal{L}}{\partial\phi_{r,\alpha}}S^{\beta\gamma}_{rs}\phi_{s}(x) + [x^{\beta}\mathcal{T}^{\alpha\gamma} - x^{\gamma}\mathcal{T}^{\alpha\beta}],
\end{equation}
(note that $\mathcal{M}^{\alpha\beta\gamma}=-\mathcal{M}^{\alpha\gamma\beta}$) and the six conserved quantities are\marginpar{We interpret $M^{\alpha\beta}$ as angular momentum}
\begin{eqnarray}
cM^{\alpha\beta} &=& \int d^3x \mathcal{M}^{0\alpha\beta} \nonumber\\
&=& \int d^3x \Big( [x^{\beta}\mathcal{T}^{0\alpha} - x^{\beta}\mathcal{T}^{0\alpha}] + c\pi_r(x)S^{\alpha\beta}_{rs}\phi_{s}(x) \Big).\label{angMom}
\end{eqnarray}
We have stated that $\mathcal{T}^{0i}/c$ is the momentum density of the field, so we interpret the square brackets of Eq (\ref{angMom}) as the orbital momentum, and the last term as the intrinsic spin angular momentum.

We can apply similar technqiues to the Dirac field. The transformation of the Dirac field under an infinitesmal Lorentz transformation is given by
\begin{equation}
\psi_{\alpha}\to\psi'_{\alpha}(x') = \psi_{\alpha}(x) - \frac{i}{4}\epsilon_{\mu\nu}\sigma^{\mu\nu}_{\alpha\beta}\psi_{\beta}(x),
\end{equation}
where summation over $\mu,\nu=0,\ldots,3$ and $\beta=1,\ldots,4$ is implied, and where $\sigma^{\mu\nu}_{\alpha\beta}$ is the $(\alpha,\beta)$ matrix element of the $4\times 4$ matrix
\begin{equation}
\sigma^{\mu\nu}\equiv\frac{i}{2}[\gamma^{\mu},\gamma^{\nu}].
\end{equation}
 We can now ``plug and chug'' to find the angular momentum of the Dirac field
\begin{equation}\label{diracAngMom}
\bold{M} = \int d^{3}x \psi^{\dag}(x)[\bold{x}\wedge(-i\hbar\nabla)]\psi(x) + \int d^{3}x\psi^{\dag}\left(\frac{\hbar}{2}\bold{\sigma}\right)\psi(x)
\end{equation}
where
\begin{equation}
\bold{\sigma} = (\sigma^{23},\sigma^{31},\sigma^{12})
\end{equation}
are 4$\times$4 matrices generalizing Pauli matrices. We also observe that Eq (\ref{diracAngMom}) represent the orbital and spin angular momentum of particles of spin 1/2.

\subsection{Solutions to the Dirac Equation}

The easiest approach to find solutions to the Dirac equation is to insist that
the solution is independent of spatial position:
\begin{equation}
\frac{\partial\psi}{\partial x} = \frac{\partial\psi}{\partial y} = \frac{\partial\psi}{\partial z} = 0.
\end{equation}
This really describes a particle with zero momentum, since the momentum operator
is $i\hbar\partial_\mu$ and all the spatial eigenvalues vanish. The Dirac 
equation simplifies to
\begin{equation}
\frac{i\hbar}{c}\gamma^0\frac{\partial\psi}{\partial t} - mc\psi = 0
\end{equation}
or equivalently
\begin{equation}
\begin{bmatrix}
1 & 0 \\
0 & -1
\end{bmatrix}
\begin{bmatrix}
\partial\psi_A/\partial t\\
\partial\psi_B/\partial t
\end{bmatrix}
= -i\frac{mc^2}{\hbar}
\begin{bmatrix}
\psi_A \\
\psi_B
\end{bmatrix}
\end{equation}
where
\begin{equation}
\psi_A = \begin{bmatrix}
\psi_1\\
\psi_2
\end{bmatrix}
\end{equation}
carries the upper two components and 
\begin{equation}
\psi_B = \begin{bmatrix}
\psi_3\\
\psi_4
\end{bmatrix}
\end{equation}
carries the lower two components. Thus
\begin{equation}
\frac{\partial\psi_A}{\partial t} = -i\left(\frac{mc^2}{\hbar}\right)\psi_A,\quad -\frac{\partial\psi_B}{\partial t} = -i\left(\frac{mc^2}{\hbar}\right)\psi_B
\end{equation}
and the solutions are
\begin{equation}
\psi_A(t) = \exp[-i(mc^2/\hbar)t]\psi_A(0),\quad\psi_B(t)=\exp[i(mc^2/\hbar)t]\psi_B(0).
\end{equation}
We should know that in Quatum mechanics, the term
\begin{equation}
\exp(-iEt/\hbar)
\end{equation}
is the characteristic for time dependence of a quantum state with energy $E$. It
follows that at rest with $\bold{p}=0$, the energy of the particle is $E=mc^2$.
So $\psi_A$ is what we expect.

What about $\psi_B$? It has negative energy! What the heck?! This is a famous
disaster, and Dirac's response was like the Hindenberg of physics. He suggested
something called the Hole theory, we will not discuss it here. 

We interpret these ``negative'' energy particles as \emph{antiparticles} with
\emph{positive} energy. So for us in our Dirac equation, $\psi_B$ describes
positrons (or antielectrons if one prefers to be outlandish) and $\psi_A$ 
describes electrons. Each of them is a 2 component spinor (a 2 column vector).
This is ideal as such a mathematical object describes a spin 1/2 particle. So,
to sum up, we have 2 particles that are each 2 solutions for a grand total of
4 independent solutions with momentum $\bold{p}=0$:
\begin{equation}
\psi^{(1)} = \exp(i(mc^2/\hbar)t)\begin{bmatrix}
1\\
0\\
0\\
0
\end{bmatrix}\quad\psi^{(2)} = \exp(i(mc^2/\hbar)t)\begin{bmatrix}
0\\
1\\
0\\
0
\end{bmatrix}
\end{equation}
\begin{equation}
\psi^{(3)} = \exp(-i(mc^2/\hbar)t)\begin{bmatrix}
0\\
0\\
1\\
0
\end{bmatrix}\quad\psi^{(4)} = \exp(-i(mc^2/\hbar)t)\begin{bmatrix}
0\\
0\\
0\\
1
\end{bmatrix}
\end{equation}
describing (respectively) an electron with spin up, an electron with spin
down, a positron with spin up and an electron with spin down.

So to look at this from the perspective of solving differential equations, we
have a solution to the homogeneous equation and we will use the method of variation
of parameters to get solutions to the Dirac equation. What does this mean? Well,
it means we are looking for ``plane wave solutions'' that look like
\begin{equation}
\psi(\bold{r},t) = ae^{-i(Et-\bold{p}\cdot\bold{r})/\hbar}u(E,\bold{p})
\end{equation}
where $a$ is a normalization constant (so probabilities add up to 1). We want to
solve for $u(E,\bold{p})=u(p)$ (we will use $p=(E/c,\bold{p})$ which is a 4 vector, and
similarly $x=(ct,\bold{x}$), which is a mathematical object called a ``bispinor''.
We don't want any old bispinor, we want one that will solve Dirac's equation!
We have $x$ dependence only in the exponent, so we find
\begin{equation}
\partial_\mu\psi = \frac{-i}{\hbar}p_\mu a e^{-(i/hbar)x^\mu p_\mu}u
\end{equation}
By plugging this into Dirac's equation, we get
\begin{equation}
\gamma^\mu p_\mu a e^{-(i/\hbar)x\cdot p}u - mcae^{-(i/\hbar)x\cdot p}u = 0
\end{equation}
or if we want a neater and cleaner way to present it
\begin{equation}\label{momentumSpaceDiracEquation}
(\gamma^\mu p_\mu - mc)u = 0.
\end{equation}
This is the ``momentum space Dirac equation'' (which we get by taking the
Fourier Transform of the Dirac equation we all know and love). Notice this is
purely algebraic, no derivatives! That's the beauty of Fourier transforms in
solving differential equations! If $u$ satisfies (\ref{momentumSpaceDiracEquation})
then $\psi$ satisfies the Dirac equation.

Now to \emph{prove} this (because an assertion is always meaningless without a
rigorous proof -- take note of this social ``scientists'') we need to use a lot
of gamma matrix manipulations. Remember all representations are ``equivalent''
in the sense that they are related by unitary transformations. First we have
\begin{equation}
\gamma^\mu p_\mu = \gamma^0 p^0 - \bold{\gamma}\cdot\bold{p} = \frac{E}{c}\begin{bmatrix}
1 & 0 \\
0 & -1
\end{bmatrix}
- \bold{p}\cdot\begin{bmatrix}
0 & \sigma \\
-\sigma & 0 
\end{bmatrix}
=
\begin{bmatrix}
E/c & -\bold{p}\cdot\bold{\sigma} \\
\bold{p}\cdot\bold{\sigma} & -E/c
\end{bmatrix}
\end{equation}
SO it follows that
\begin{eqnarray*}
(\gamma^\mu p_\mu - mc)u &=& \begin{bmatrix}
\left(\frac{E}{c}-mc\right) & -\bold{p}\cdot\sigma \\
\bold{p}\cdot\sigma & \left(\frac{-E}{c}-mc\right)
\end{bmatrix}
\begin{bmatrix}
u_A \\
u_B
\end{bmatrix} \\
&=& \begin{bmatrix}
\left(\frac{E}{c}-mc\right)u_A & -\bold{p}\cdot\sigma u_B\\
\bold{p}\cdot\sigma u_A & \left(\frac{-E}{c}-mc\right) u_B
\end{bmatrix}
\end{eqnarray*}
where the subscript $A$ is for the upper two components and the $B$ stands for
the lower two. In order to satisfy the momentum space Dirac equation, we
must have
\begin{equation}\label{stepTowardsSolutions}
u_A = \frac{c}{E - mc^2}(\bold{p}\cdot\sigma)u_B,\quad u_B = \frac{c}{E + mc^2}(\bold{p}\cdot\sigma) u_A
\end{equation}
We substitute the second into the first to give us
\begin{equation}
u_A =\frac{c^2}{E^2 - m^2c^4}(\bold{p}\cdot\sigma)^2 u_A
\end{equation}
Observe
\begin{eqnarray*}
\bold{p}\cdot\sigma &=& p_x\begin{bmatrix}
0 & 1\\
1 & 0 
\end{bmatrix}
+ p_{y}\begin{bmatrix}
0 & -i\\
i & 0
\end{bmatrix}
+ p_{z}\begin{bmatrix}
1 & 0 \\
0 & -1
\end{bmatrix} \\
&=& \begin{bmatrix}
p_{z} & (p_{x} - ip_{y}) \\
(p_{x} + ip_{y}) & -p_{z}
\end{bmatrix}
\end{eqnarray*}
We find then by matrix multiplication (we will not calculate this out with every
detail, but we will show the result):
\begin{equation}
(\bold{p}\cdot\sigma)^2 = \begin{bmatrix} p_{z}^2 + (p_x - ip_{y})(p_{x} + ip_{y}) & p_{z}(p_{x} - ip_{y}) - p_{z}(p_{x} - ip_{y}) \\
p_{z}(p_{x} + ip_{y}) - p_{z}(p_{x} + ip_{y}) & (p_{x} + ip_{y})(p_{x} - ip_{y}) + p_{z}^2
\end{bmatrix} = \bold{p}^2 I
\end{equation}
where $I$ is the 2 by 2 identity matrix. We see then that by plugging this into our equation for $u_A$
\begin{equation}
u_A = \frac{\bold{p}^2c^2}{E^2 - m^2c^4}u_A
\end{equation}
which can be rearranged to be
\begin{eqnarray*}
(E^2 - m^2c^4)u_A &=& \bold{p}^2c^2 u_A \\
\Rightarrow (E^2 - \bold{p}^2c^2)u_A &=& m^2c^4 u_A
\end{eqnarray*}
and thus
\begin{equation}
E^2 - \bold{p}^2c^2 = m^2c^4
\end{equation}
which is the famous Einstein equation we all know and love. This tells us that
in order to satisfy the Dirac equation, we have to obey the mass shell constraint.
This admits two solutions for $E$:
\begin{equation}
E = \pm\sqrt{m^2 c^4 + \bold{p}^2c^2}
\end{equation}
where the positive root is associated with particle states, and the negative
root with antiparticle states.

Using Eq (\ref{stepTowardsSolutions}), it is straightforward to calculate out
the solutions to the Dirac equation (ignoring normalization constants):
\begin{equation*}
\mbox{Pick } u_A = \begin{bmatrix} 1\\0\end{bmatrix}\quad\mbox{then } 
u_B = \frac{c}{E + mc^2}(\bold{p}\cdot\sigma)\begin{bmatrix}1\\0\end{bmatrix} 
= \frac{c}{E + mc^2}\begin{bmatrix} p_{z}\\ p_{x}+ip_{y}\end{bmatrix}
\end{equation*}
\begin{equation*}
\mbox{Pick } u_A = \begin{bmatrix} 0\\1\end{bmatrix}\quad\mbox{then } 
u_B = \frac{c}{E + mc^2}(\bold{p}\cdot\sigma)\begin{bmatrix}0\\1\end{bmatrix} 
= \frac{c}{E + mc^2}\begin{bmatrix} p_{x}-ip_{y}\\ -p_{z}\end{bmatrix}
\end{equation*}
\begin{equation*}
\mbox{Pick } u_B = \begin{bmatrix} 1\\0\end{bmatrix}\quad\mbox{then } 
u_A = \frac{c}{E - mc^2}(\bold{p}\cdot\sigma)\begin{bmatrix}1\\0\end{bmatrix} 
= \frac{c}{E - mc^2}\begin{bmatrix} p_{z}\\ p_{x}+ip_{y}\end{bmatrix}
\end{equation*}
\begin{equation*}
\mbox{Pick } u_B = \begin{bmatrix} 0\\1\end{bmatrix}\quad\mbox{then } 
u_A = \frac{c}{E - mc^2}(\bold{p}\cdot\sigma)\begin{bmatrix}0\\1\end{bmatrix} 
= \frac{c}{E - mc^2}\begin{bmatrix} p_{x}-ip_{y}\\-p_{z}\end{bmatrix}
\end{equation*}
For the first two of these, we must use the positive energy otherwise we have
division by zero, and if you divide by zero you go to hell. For the same reason,
the energy in the latter two are negative. It is convenient to ``normalize'' these
spinors in such a way that
\begin{equation}
u^\dag u = 2|E|/c
\end{equation}
where the dagger indicates the transpose conjugate (``Hermitian conjugate'') is
used:
\begin{equation*}
u = \begin{bmatrix}a\\b\\c\\d\end{bmatrix}\Rightarrow\quad u^\dag = (a^*, b^*, c^*, d^*)
\end{equation*}
so that
\begin{equation}
u^\dag u = |a|^2 + |b|^2 + |c|^2 + |d|^2.
\end{equation}
So we find that the four solutions are:
\begin{equation}
u^{(1)} = N\begin{bmatrix}1\\0\\\frac{\displaystyle cp_{z}}{\displaystyle E+mc^2}\\ \frac{\displaystyle c(p_{x}+ip_{y})}{\displaystyle E+mc^2}\end{bmatrix}
\end{equation}
\begin{equation}
u^{(2)} = N\begin{bmatrix}0\\1\\ \frac{\displaystyle c(p_{x}-ip_{y})}{\displaystyle E+mc^2}\\\frac{\displaystyle -cp_{z}}{\displaystyle E+mc^2}\end{bmatrix}
\end{equation}
with $E = +\sqrt{m^2c^4 + \bold{p}^2c^2}$
\begin{equation}
u^{(3)} = N\begin{bmatrix}\frac{\displaystyle cp_{z}}{\displaystyle E-mc^2}\\ \frac{\displaystyle c(p_{x}+ip_{y})}{\displaystyle E-mc^2}\\1\\0\end{bmatrix}
\end{equation}
\begin{equation}
u^{(4)} = N\begin{bmatrix} \frac{\displaystyle c(p_{x}-ip_{y})}{\displaystyle E-mc^2}\\\frac{\displaystyle c(-p_{z})}{\displaystyle E-mc^2}\\ 0\\1\end{bmatrix}
\end{equation}
with $E = -\sqrt{m^2c^4 + \bold{p}^2c^2}$, and the normalization constant is
\begin{equation}
N = \sqrt{(|E|+mc^2)/c}.
\end{equation}
Now we are really tempted to say that $u^{(1)}$ is an electron with spin up,
and $u^{(2)}$ is an electron with spin down, and so on, but this is not quite so.
For Dirac Particles, the spin matrices are
\begin{equation}
S = \frac{\hbar}{2}\Sigma\quad\mbox{with }\Sigma\equiv\begin{bmatrix}\sigma & 0\\0 & \sigma\end{bmatrix}
\end{equation}
and it's easy to check that $u^{(1)}$ is \emph{not} an eigenstate of $\Sigma$.
However, if we orient the $z$ axis so it points along the direction of motion
(in which case $p_x = p_y = 0$) then $u^{(1)}$, $u^{(2)}$, $u^{(3)}$, and $u^{(4)}$
are eigenspinors of $S_z$; $u^{(1)}$ and $u^{(3)}$ are spin up, and
$u^{(2)}$ and $u^{(4)}$ are spin down\footnote{It is actually mathematically
impossible to construct spinors that satisfies the momentum Dirac equation and
are simultaneously eigenspinors of $S_z$ (except for the special case $\bold{p}$ = $p_z\hat{z}$).
The reason is that $S$ by itself is \emph{not a conserved quantity.} Only the
\emph{total} angular momentum $L+S$ is conserved. It is possible to construct
eigenspinors of \emph{helicity}, $\Sigma\cdot\hat{p}$ (there's no \emph{orbital}
angular momentum about the direction of motion), but these are cumbersome and in
practice we like to work with the spinors we have constructed, even though it is
difficult to have a physical intuition to what they mean. In the end, all that
really matters is that we have a complete set of solutions.}

Now we have to discuss the importance of $E$ and $\bold{p}$, which are mathematical
parameters which correspond physically to energy and momentum. At least, this is
true for the electron states $u^{(1)}$ and $u^{(2)}$; but in $u^{(3)}$ and 
$u^{(4)}$ the $E<0$...so it \emph{cannot} represent positron energy. All free
particles -- electrons and positrons alike -- carry \emph{positive} energy.
The ``negative-energy'' solutions must be reinterpreted as \emph{positive}
energy \emph{antiparticle} states. To express these solutions in terms of 
the \emph{physical} energy and momentum of the positron, we flip the signs of
$E$ and $\bold{p}$:
\begin{equation}
\psi(\bold{r},t) = ae^{i/\hbar(Et - \bold{p}\cdot\bold{r})}u(-E,-\bold{p})
\end{equation}
for solutions (3) and (4) of course. These are the same solutions, we just have
changed the signs of two parameters so it is physically appealing. It is 
customary to use $v$ for positron states, expressed in terms of the physical
energy and momentum:
\begin{equation}
v^{(1)}(E,\bold{p}) = u^{(4)}(-E,-\bold{p}) = N\begin{bmatrix} 
\frac{\displaystyle c(p_x - ip_y)}{\displaystyle E + mc^2}\\
\frac{\displaystyle c(-p_z)}{\displaystyle E + mc^2}\\
0\\
1\end{bmatrix}
\end{equation}
\begin{equation}
v^{(2)}(E,\bold{p}) = u^{(4)}(-E,-\bold{p}) = N\begin{bmatrix} 
\frac{\displaystyle c(p_z)}{\displaystyle E + mc^2}\\
\frac{\displaystyle c(p_x + ip_y)}{\displaystyle E + mc^2}\\
1\\
0\end{bmatrix}
\end{equation}
(with $E=\sqrt{m^2c^4 + \bold{p}^2c^2}$).

So we will no longer be working with $u^{(3)}$ and $u^{(4)}$; instead, the
set of solutions we will be working with are $u^{(1)}$, $u^{(2)}$ (representing
the two spin states of an electron with energy $E$ and momentum $\bold{p}$),
and $v^{(1)}$, $v^{(2)}$ (representing the two spin states of a positron with
energy $E$ and momentum $\bold{p}$). Notice that whereas the $u$'s satisfy the
momentum space Dirac equation in the form
\begin{equation}
(\gamma^\mu p_\mu - mc)u = 0
\end{equation}
the $v$'s obey the equation with the sign of $p_\mu$ reversed:
\begin{equation}
(\gamma^\mu p_\mu + mc)v = 0.
\end{equation}
Sure this is interesting, but it's only the special case of plane waves. Why
bother? Well, they are of interest because they describe particles with 
specified energies and momenta, and in a typical experiment that's what we
control and measure.

