\section{Klein-Gordon Review}

Recall that the Schrodinger equation for the free nonrelativistic particle is
\begin{equation}
\frac{-\hbar^2}{2m}\nabla^2 \ket{\psi} = -i\hbar\partial_{t}\ket{\psi}
\end{equation}
which corresponds to a sort of quantized Newton's second law
\begin{equation}
\frac{p^2}{2m} \approx E.
\end{equation}
However, in special relativity we have the mass shell constraint
\begin{equation}\label{massShell}
p^\mu p_\mu = E^2 - \bold{p}\cdot\bold{p} = m^2
\end{equation}
(when $c=1$) using Einstein summation convention. If we naively quantize this, we end up with
\begin{equation}\label{KGeqn}
\partial_{\mu}\partial^{\mu} - m^2 \ket{\psi} = 0
\end{equation}
by moving the mass term onto the left hand side. This is the Klein-Gordon equation, it is plagued by problems such as negative probabilities, etc.


%%%%%%%%%%%%%%%%%%%%%%%%%%%%%%%%%%%%%%%%%%%%%%%%%%%%%%%%%%%%%%%%%%%%%%%%%%

\section{Dirac takes it up a notch...bam!}

Naively, we want something simpler than this. We can rewrite Eq (\ref{massShell}) to be
\begin{equation}
E^2 = \bold{p}\cdot\bold{p} + m^2\Rightarrow E = \sqrt{\bold{p}\cdot\bold{p} + m^2}
\end{equation}
then quantize it to be
\begin{equation}
i\hbar\frac{\partial}{\partial t}\ket{\psi} = \sqrt{-\hbar^2\nabla^2 + m^2}\ket{\psi}.
\end{equation}
We end up being forced to use pseudo-differential operators, unfortunately, and it turns out that this results in nonlocality\footnote{In general, whenever there is a squareroot quantity, there is nonlocality.}. For further details see Laemmerzahl~\cite{pseudodifferentialKG}.

The approach Dirac takes is basically taking the squareroot of the operator, but he does it with class. He uses a clifford algebra with generators $\gamma^\mu$\footnote{If the reader is unfamiliar with the Gamma Matrices, see the appendix A and/or CORE~\cite{Borodulin:1995xd}.} such that the squareroot of the Klein Gordon equation breaks into two equations:
\begin{equation}\label{diracEqn}
(i\hbar\gamma^\mu\partial_\mu - m)\psi(x) = 0
\end{equation}
where $\psi(x)$ is a spinor wave function with 4 components. The adjoint field $\bar{\psi}(x)$ is defined by 
\begin{equation}
\bar{\psi}(x) = \psi^{\dag}(x)\gamma^0
\end{equation}
and satisfies the \emph{adjoint} Dirac equation
\begin{equation}\label{adjointEqn}
\bar{\psi}(x)(i\hbar\gamma^\mu\partial_\mu + m) = 0.
\end{equation}
It is to be understood here the differential operator $\partial^\mu$ acts on the left. Observe that when we multiply the two operators together we get
\begin{equation}
(i\hbar\gamma^\mu\partial_\mu + m)(i\hbar\gamma^\mu\partial_\mu - m) = -\hbar^{2}\left(\gamma^\mu\right)^2\partial_{\mu}\partial^{\mu} - m^2 = \hat{p}^\mu\hat{p}_\mu - m^2 
\end{equation}
which is \emph{precisely} the Klein-Gordon operator (\ref{KGeqn})! We should be content now with the connection back to what we already know. \notetoself{And an added advantage is that the Dirac equation is a first order partial differential equation, whereas the Klein-Gordon equation is a second order one!}

%%%%%%%%%%%%%%%%%%%%%%%%%%%%%%%%%%%%%%%%%%%%%%%%%%%%%%%%%%%%%%%%%%%%%%%%%%

\subsection{A Somewhat Rigorous Derivation of the Dirac Equation}

We want the squareroot of the wave operator thus
\begin{equation}
\nabla^2 - \frac{1}{c^2}\frac{\partial^2}{\partial t^2} = (A \partial_x + B \partial_y + C \partial_z + \frac{i}{c}D \partial_t)(A \partial_x + B \partial_y + C \partial_z + \frac{i}{c}D \partial_t)
\end{equation}
We see multiplying out the right hand side all the cross-terms must vanish. To have this we want
\begin{equation}
AB + BA = 0, 
\end{equation}
and so on for all cross-term coefficients, with the property that
\begin{equation}
A^2 = B^2 = C^2 = D^2 = 1.
\end{equation}
Dirac had previously worked out rigorous results with Heisenberg's matrix mechanics, and concluded that these conditions could be met if $A, B,\ldots$ were \emph{matrices} which has the implication that the wave function has \emph{multiple components}. 

In the mean time, Pauli had been working on quantum mechanics as well. Pauli had a model with two-component wave functions that was involved in a phenomenological theory of spin. At this point in time, spin was not well understood. 

Given the factorization of these matrices, one can now write down immediately an equation
\begin{equation}
(A\partial_x + B\partial_y + C\partial_z + \frac{i}{c}D\partial_t)\psi = \kappa\psi
\end{equation}
with $\kappa$ to be determined. Applying the same operation on either side yields
\begin{equation}
(\nabla^2 - \frac{1}{c^2}\partial_{t}^2)\psi = \kappa^2\psi.
\end{equation}
If one take $\kappa = mc/\hbar$ we find that all the components of the wave function \emph{individually} satisfy the mass-shell relation (\ref{massShell}). Thus we have a first order differential equation in both space and time described by
\begin{equation}
(A\partial_x + B\partial_y + C\partial_z + \frac{i}{c}D\partial_t - \frac{mc}{\hbar})\psi = 0
\end{equation}
where $(A,B,C)=i\beta\alpha_k$ and $D=\beta$, which is precisely the Dirac equation for a spin-1/2 particle of rest mass $m$.


%%%%%%%%%%%%%%%%%%%%%%%%%%%%%%%%%%%%%%%%%%%%%%%%%%%%%%%%%%%%%%%%%%%%%%%%%%

\subsection{A Comparison to the Pauli Theory}

The necessity of introducing half-integer spin goes back experimentally to the results of the Stern-Gerlach experiment. \notetoself{A beam of atoms is run through a strong inhomogeneous magnetic field, which then splits into N parts depending on the intrinsic angular momentum of the atoms. It was found that for silver atoms, the beam was split in two - the ground state therefore could not be integral, because even if the intrinsic angular momentum of the atoms were as small as possible, 1, the beam would be split into 3 parts, corresponding to atoms with $L_z = -1, 0, +1$. The conclusion is that silver atoms have net intrinsic angular momentum of 1/2.} Pauli set up a model which explained the splitting by introducing a two-component wave-function and a corresponding correction term in the Hamiltonian(representing a semiclassical coupling of this wave function to an applied magnetic field) as
\begin{equation}
H = \frac{1}{2m}(\sigma^{I}_{i}(p^{i} - \frac{e}{c}A^{i})\sigma_{Ij}(p^{j} - \frac{e}{c}A^{j})) + e A^0 \qquad (i,j,I=1,2,3).
\end{equation}
We have here $A^\mu$ is the magnetic potential, and the van Warden symbols $\sigma^{J}_{j}$ which translates a vector into the Pauli matrix basis (if one is unfamiliar with Pauli matrices, see \S\ref{Representations of the Gamma Matrices}), $e$ is the electric charge of the particle (here $e=-e_0$ for the electron), and $m$ is the mass of the particle. Now we have just described the Hamiltonian of our system by a 2 by 2 matrix. The Schrodinger equation based on it
\begin{equation}
H \phi = i\hbar \frac{\partial\phi}{\partial t}
\end{equation}
must use a two-component wave function. (If you're like me you're too lazy to flip to the appendix, so I'll reproduce some of it here)\marginpar{SU(2) is the set of all 2 by 2 matrices that is self-adjoint and has a determinant of 1} Pauli used the SU(2) matrices
\begin{equation}\label{Pauli}
\sigma_k = \begin{bmatrix} 0 & 1 \\ 1 & 0 \end{bmatrix},\begin{bmatrix} 0 & -i \\ i & 0 \end{bmatrix},\begin{bmatrix} 1 & 0 \\ 0 & -1 \end{bmatrix}
\end{equation}
due to \emph{phenomenological reasons} (explaining the Gerlach experiment). Dirac now has a \emph{theoretical argument} that implies spin is a \emph{consequence} of introducing special relativity into quantum theory.

The Pauli matrices share the same properties as the Dirac matrices -- they are all self-adjoint, when squared are equal to the identity, and they anticommute. We can now use the Pauli matrices Eq (\ref{Pauli}) to describe a representation of the Dirac matrices:
\begin{equation}
\alpha_k = \begin{bmatrix} 0 & \sigma_k \\ \sigma_k & 0 \end{bmatrix}\qquad \beta = \begin{bmatrix} 1_2 & 0 \\ 0 & -1_2 \end{bmatrix}.
\end{equation}
We now may write the Dirac equation as an equation coupling two-component spinors:
\begin{equation}
\begin{bmatrix} mc^2 & c\sigma\cdot p \\ c\sigma\cdot p & -mc^2 \end{bmatrix} \begin{bmatrix} \phi_+ \\ \phi_- \end{bmatrix} = i\hbar\frac{\partial}{\partial t}\begin{bmatrix} \phi_+ \\ \phi_- \end{bmatrix}.
\end{equation}
Observe that we have on the diagonal the rest mass. If we bring the particle to rest, we have
\begin{equation}
i\hbar\frac{\partial}{\partial t}\begin{pmatrix} \phi_+ \\ \phi_- \end{pmatrix} = \begin{pmatrix} mc^2 & 0 \\ 0 & -mc^2 \end{pmatrix} \begin{pmatrix} \phi_+ \\ \phi_- \end{pmatrix}.
\end{equation}
The equations for the individual two-spinors are now decoupled, and we see that the ``spin-up'' and ``spin-down'' (or ``right-handed'' and ``left-handed'', ``positive frequency'' and ``negative frequency'' respectively) are individual eigenfunctions \snote{Eigenspinors?} of the energy with eigenvalues equal to $\pm$ the rest energy. The appearence of \emph{negative} energy should not be alarming, it is completely consistent with relativity.

\textbf{Note} that this seperation is in the rest frame and \textbf{is not an invariant statement} -- the bottom component does not generally represent antimatter. The \emph{entire} four-component spinor represents an \emph{irreducible whole} -- in general states will have an admixture of positive \emph{and} negative energy components.

\subsection{Covariant Form and Relativistic Invariance}

The explicity covariant form of the Dirac Equation is (using Einstein summation convention)
\begin{equation}
i\hbar\gamma^\mu\partial_\mu\psi - mc\psi = 0,
\end{equation}
where $\gamma^\mu$ are the Dirac gamma matrices. We have
\begin{equation}
\gamma^0 = \beta \qquad \gamma^k = \gamma^0\alpha_k.
\end{equation}
See the appendix for more details on this representation.

The Dirac equation may be interpreted as an eigenvalue expression, where the rest mass is proportional to an eigenvalue of the 4-momentum operator, the proportion being $c$:
\begin{equation}
\hat{P}\psi = mc\psi.
\end{equation}
In practice we often work in units where we set $\hbar$ and $c$ to be 1. The equation is multiplied by $-i$ and takes the form
\begin{equation}
\left(\gamma^\mu\partial_\mu + im \right)\psi = 0.
\end{equation}
We may employ the Feynman slash notation to simplify this to
\begin{equation}
(\slashed{\partial} + im)\psi = 0. 
\end{equation}
For any two representations of the Dirac Gamma matrices, they are related by a unitary transformation. Likewise, the solutions in the two representations are related by the same way.

%%%%%%%%%%%%%%%%%%%%%%%%%%%%%%%%%%%%%%%%%%%%%%%%%%%%%%%%%%%%%%%%%%%%%%%%%

\subsection{Conservation Laws and Canonical Structure}

Recall the Dirac equation and its adjoint version, Eqns (\ref{diracEqn}) and (\ref{adjointEqn}). We notice from the definition of the adjoint
\begin{equation*}
\bar{\psi} = \psi^\dag\gamma^0
\end{equation*}
that
\begin{equation}
\left(\gamma^\mu\right)^\dag\gamma^0 = \gamma^0\gamma^\mu
\end{equation}
we can obtain the Hermitian conjugate of the Dirac equation and multiplying from the right by $\gamma^0$  we get its adjoint version:
\begin{equation*}
\bar{\psi}(\gamma^\mu\overleftarrow{\partial}_\mu - im) = 0
\end{equation*} 
where $\overleftarrow{\partial}_\mu$ acts on the left. When we multiply the Dirac equation by $\bar{\psi}$ from the left
\begin{equation}
\bar{\psi}(\gamma^\mu\overrightarrow{\partial}_\mu + im)\psi = 0
\end{equation}
(where $\overrightarrow{\partial}_\mu$ acts on the right) and multiply the adjoint equation by $\psi$ on the right
\begin{equation}
\bar{\psi}(\gamma^\mu\overleftarrow{\partial}_\mu - im)\psi = 0
\end{equation}
then add the two together we get\marginpar{Conservation Law of Dirac Current}
\begin{equation}
\bar{\psi}(\gamma^\mu\overrightarrow{\partial}_\mu + im)\psi + \bar{\psi}(\gamma^\mu\overleftarrow{\partial}_\mu - im)\psi = \partial(\bar{\psi}\gamma^\mu\psi) = \partial_\mu J^\mu 0
\end{equation}
(where $J^\mu$ is the Dirac Current) which is the law of conservation of the Dirac current in covariant form. We see the huge advantage this has over the Klein-Gordon equation: this has conserved probability current desnity as required by relativistic invariance...only now its temporal component is \emph{positive definite}:
\begin{equation}
J^0 = \bar{\psi}\gamma^0\psi = \bar{\psi}\psi.
\end{equation}
From this we can find a conserved charge
\begin{equation}
Q = q\int \psi^\dag(\bold{x})\psi(\bold{x})d^3x
\end{equation}
where $q$ is to be thought of as ``charge''.

We can now see that the Dirac equation (and its adjoint) are the Euler-Lagrange equations of motion of the four dimensional invariant action
\begin{equation}
S = \int \mathcal{L}d^4x
\end{equation}
where the Dirac Lagrangian density $\mathcal{L}$ is given by
\begin{equation}
\mathcal{L} = c\bar{\psi}(x)\Big[ i\hbar\gamma^\mu\partial_\mu - mc \Big]\psi(x)
\end{equation}
and for purposes of variation, $\psi$ and $\bar{\psi}$ are considered to be independent fields. Relativistic invariance follows from the variational principle.

\marginpar{Canonical Structure, Hamiltonian and Momentum operators}We can find the canonically conjugate momenta to the fields $\psi$ and $\bar{\psi}$:
\begin{equation}
\pi(x) = \frac{\partial\mathcal{L}}{\partial\dot{\psi}} = i\hbar\psi^\dag\qquad\bar{\pi}(x) = \frac{\partial\mathcal{L}}{\partial\dot{\bar{\psi}}} = 0.
\end{equation}
We can find the Hamiltonian of the Dirac field
\begin{equation}\label{Hamiltonian}
H = \int d^{3}x( \pi_{\alpha}(x)\dot{\psi}^{\alpha}(x) - \mathcal{L} ) = \int d^{3}x \bar{\psi}(x)[-i\hbar c\gamma^{j}\partial_{j} + mc]\psi(x).
\end{equation}
Similarly, the momentum of some field $\phi$ to be given by
\begin{equation*}
cP^\alpha \equiv \int d^3x\mathcal{T}^{0\alpha} = \int d^{3}x\left[c\pi_r(x)\frac{\partial\phi_r(x)}{\partial x_{\alpha}} - \mathcal{L}\eta^{0\alpha}\right]
\end{equation*}
where $\mathcal{T}^{\alpha\beta}$ is the stress-energy density tensor of the field $\phi$. Recall that we define the stress-energy density tensor by the equation
\begin{equation}
\mathcal{T}^{\alpha\beta}\equiv \frac{\partial\mathcal{L}}{\partial\phi_{r,\alpha}}\frac{\partial\phi_r}{\partial x_\beta} - \mathcal{L}\eta^{\alpha\beta}.
\end{equation}
Using this, we can find the momentum of the Dirac Field to be
\begin{equation}
\bold{P} = -i\hbar\int d^3x\psi^\dag(x)\nabla\psi(x).
\end{equation}
Of course, the Hamiltonian given by (\ref{Hamiltonian}) could have been discovered by finding the Hamiltonian density applied to the current case.

We\marginpar{Angular Momentum} can similarly find the angular momentum of the Dirac Field by simply following the scheme of finding the angular momentum for a general field. That is, an infinitesmal transformation of the coordinates
\begin{equation}
x_\alpha\to x'_\alpha\equiv x_\alpha + \delta x_\alpha = x_\alpha + \varepsilon_{\alpha\beta}x^\beta + \delta_\alpha
\end{equation}
(where $\delta_\alpha$ is an infinitesmal displacement and $\varepsilon_{\alpha\beta}$ is an infinitesmal antisymmetric tensor to ensure invariance of $x_\alpha x^\alpha$ under homogeneous Lorentz transformations, i.e. ones with $\delta_\alpha=0$) induces an infinitesmal transformation of the field $\phi$:
\begin{equation}
\phi_r(x)\to\phi'_r(x') = \phi_r(x) + \frac{1}{2}\varepsilon_{\alpha\beta}S^{\alpha\beta}_{rs}\phi_{s}(x).
\end{equation}
Here the coefficients $S^{\alpha\beta}_{rs}$ are antisymmetric in $\alpha$ and $\beta$, like $\varepsilon_{\alpha\beta}$, and are determined by the transformation properties of the fields.

For a rotation (i.e. $\delta_\alpha=0$) we have the continuity equation
\begin{equation}
\frac{\partial\mathcal{M}^{\alpha\beta\gamma}}{\partial x^{\alpha}} = 0
\end{equation}
where
\begin{equation}
\mathcal{M}^{\alpha\beta\gamma}\equiv\frac{\partial\mathcal{L}}{\partial\phi_{r,\alpha}}S^{\beta\gamma}_{rs}\phi_{s}(x) + [x^{\beta}\mathcal{T}^{\alpha\gamma} - x^{\gamma}\mathcal{T}^{\alpha\beta}],
\end{equation}
(note that $\mathcal{M}^{\alpha\beta\gamma}=-\mathcal{M}^{\alpha\gamma\beta}$) and the six conserved quantities are\marginpar{We interpret $M^{\alpha\beta}$ as angular momentum}
\begin{eqnarray}
cM^{\alpha\beta} &=& \int d^3x \mathcal{M}^{0\alpha\beta} \nonumber\\
&=& \int d^3x \Big( [x^{\beta}\mathcal{T}^{0\alpha} - x^{\beta}\mathcal{T}^{0\alpha}] + c\pi_r(x)S^{\alpha\beta}_{rs}\phi_{s}(x) \Big).\label{angMom}
\end{eqnarray}
We have stated that $\mathcal{T}^{0i}/c$ is the momentum density of the field, so we interpret the square brackets of Eq (\ref{angMom}) as the orbital momentum, and the last term as the intrinsic spin angular momentum.

We can apply similar technqiues to the Dirac field. The transformation of the Dirac field under an infinitesmal Lorentz transformation is given by
\begin{equation}
\psi_{\alpha}\to\psi'_{\alpha}(x') = \psi_{\alpha}(x) - \frac{i}{4}\epsilon_{\mu\nu}\sigma^{\mu\nu}_{\alpha\beta}\psi_{\beta}(x),
\end{equation}
where summation over $\mu,\nu=0,\ldots,3$ and $\beta=1,\ldots,4$ is implied, and where $\sigma^{\mu\nu}_{\alpha\beta}$ is the $(\alpha,\beta)$ matrix element of the $4\times 4$ matrix
\begin{equation}
\sigma^{\mu\nu}\equiv\frac{i}{2}[\gamma^{\mu},\gamma^{\nu}].
\end{equation}
 We can now ``plug and chug'' to find the angular momentum of the Dirac field
\begin{equation}\label{diracAngMom}
\bold{M} = \int d^{3}x \psi^{\dag}(x)[\bold{x}\wedge(-i\hbar\nabla)]\psi(x) + \int d^{3}x\psi^{\dag}\left(\frac{\hbar}{2}\bold{\sigma}\right)\psi(x)
\end{equation}
where
\begin{equation}
\bold{\sigma} = (\sigma^{23},\sigma^{31},\sigma^{12})
\end{equation}
are 4$\times$4 matrices generalizing Pauli matrices. We also observe that Eq (\ref{diracAngMom}) represent the orbital and spin angular momentum of particles of spin 1/2.

%%
%% solutions.tex
%% 
%% Made by Alex Nelson
%% Login   <alex@tomato3>
%% 
%% Started on  Sun Dec 12 11:26:23 2010 Alex Nelson
%% Last update Sun Dec 12 11:27:36 2010 Alex Nelson
%%
\renewcommand{\leftmark}{Solutions to Exercises}
\section{Solutions to Exercises}

\subsection{Problem Set 1}
\begin{exercise}\label{hw1:ex:one}
Check:
\begin{enumerate}
\item that the vector space $\Bbb{R}^{3}$ is a Lie algebra with
respect to the cross product of vectors;
\item this Lie algebra is simple (i.e. does not have any
  non-trivial ideals);
\item all derivations of this Lie algebra are inner derivations.
\end{enumerate}
\end{exercise}


\answer For the matter of
$\RR^{3}$ being a Lie algebra, we have the following proof:
\begin{proof}
We have a vector space $\RR^{3}$ over $\RR$. We need to
show that when we equip it with the cross product operation, we
obtain a Lie algebra. That is, we induce a Lie Bracket
\begin{equation}
[\vec{v},\vec{w}]:=\vec{v}\times\vec{w}
\end{equation}
where $\vec{v}$, $\vec{w}\in\RR^{3}$. We need to check that
it obeys the properties of the Lie bracket, and that the property
of distributivity holds. 

For the properties of the bracket, we see that antisymmetry
holds:
\begin{subequations}
\begin{align}
[\vec{v},\vec{w}]&=\vec{v}\times\vec{w}\\
&=-\vec{w}\times\vec{v}\\
&=-[\vec{w},\vec{v}].
\end{align}
\end{subequations}
We see that it is linear in the second slot (and by antisymmetry,
the first slot too):
\begin{subequations}
\begin{align}
[\vec{u},\vec{v}+\vec{w}]&=\vec{u}\times(\vec{v}+\vec{w})\\
&=\vec{u}\times\vec{v}+\vec{u}\times\vec{w}\\
&=[\vec{u},\vec{v}]+[\vec{u},\vec{w}].
\end{align}
\end{subequations}
Lastly we see that the Jacobi identity holds. We first observe, by
Lagrange's identity
\begin{subequations}\label{ex1:eq:lagrangeID}
\begin{align}
\vec{u}\times(\vec{v}\times\vec{w}) &= \vec{v}(\vec{u}\cdot\vec{w})-\vec{w}(\vec{u}\cdot\vec{v})\\
\vec{v}\times(\vec{w}\times\vec{u}) &= \vec{w}(\vec{v}\cdot\vec{u})-\vec{u}(\vec{v}\cdot\vec{w})\\
\vec{w}\times(\vec{u}\times\vec{v}) &= \vec{u}(\vec{w}\cdot\vec{v})-\vec{v}(\vec{w}\cdot\vec{u})
\end{align}
\end{subequations}
Then we consider the Jacobi identity by plugging in our results
from Eq \eqref{ex1:eq:lagrangeID}:
\begin{subequations}
\begin{align}
\left[\vec{u},[\vec{v},\vec{w}]\right] +
\left[\vec{v},[\vec{w},\vec{u}]\right] +
\left[\vec{w},[\vec{u},\vec{v}]\right] &= \vec{u}\times(\vec{v}\times\vec{w})+\vec{v}\times(\vec{w}\times\vec{u})+\vec{w}\times(\vec{u}\times\vec{v})\\
&=\vec{v}(\vec{u}\cdot\vec{w})-\vec{w}(\vec{u}\cdot\vec{v})+\vec{w}(\vec{v}\cdot\vec{u})-\vec{u}(\vec{v}\cdot\vec{w})\nonumber\\
&\phantom{=+\vec{v}}+\vec{u}(\vec{w}\cdot\vec{v})-\vec{v}(\vec{w}\cdot\vec{u})\\
&=\left[\vec{v}(\vec{u}\cdot\vec{w})-\vec{v}(\vec{w}\cdot\vec{u})\right]+\left[\vec{w}(\vec{v}\cdot\vec{u})-\vec{w}(\vec{u}\cdot\vec{v})\right]\nonumber\\
&\phantom{=\left[\vec{w}\right]}+\left[\vec{u}(\vec{w}\cdot\vec{v})-\vec{u}(\vec{v}\cdot\vec{w})\right]\\
&=[0]+[0]+[0]=0
\end{align}
\end{subequations}
which holds. Thus the cross product satisfies the properties of
the Lie bracket, implying $\Bbb{R}^{3}$ equipped with the cross
product is a Lia algebra.
\end{proof}

\medbreak\noindent\textbf{Answer 1.2:\enspace}
For the matter of this Lie algebra being simple, we have another
proof.

\begin{proof}
Assume for contradiction there is an ideal
$I\subset\Bbb{R}^{3}$ which is nontrivial. Then there is a
nontrivial center for the Lie algebra. That is, we have some
$\vec{x}\in I$ such that
\begin{equation}
[\vec{x},\vec{y}]=0
\end{equation}
for all $\vec{y}\in\Bbb{R}$. However, this happens if and only if
\begin{equation}
\vec{x}=\lambda\vec{y}\quad\hbox{for some nonzero $\lambda$, or}\quad\vec{x}=0.
\end{equation}
The second case is trivial, the first case implies
$I=\Bbb{R}^{3}$. In either case, $\Bbb{R}^{3}$ does not have a
nontrivial center, so it doesn't have a nontrivial ideal.
\end{proof}

\medbreak\noindent\textbf{Answer 1.3:\enspace}
Last part of the first exercise, we need to show that all
derivations of this Lie algebra are inner derivations. We thus
produce the following proof.

\begin{proof}
We find that a derivation $\alpha\colon\Bbb{R}^{3}\to\Bbb{R}^{3}$
would be of the form
\begin{equation}
\alpha([\vec{u},\vec{v}])=[\alpha(\vec{u}),\vec{v}]+[\vec{u},\alpha(\vec{v})].
\end{equation}
which occurs whenever $\alpha=[\vec{w},-]$ (for some
$\vec{w}\in\Bbb{R}^{3}$) by the Jacobi identity. We want to show
that there are no other derivations. We see that $\alpha$ is
represented by an antisymmetric matrix $X+X^{T}=0$. But we also
recall any matrix $B$ can be written as
\begin{equation}
B = A + S
\end{equation}
where $A$ is antisymmetric, and $S$ is symmetric. Then if $B$
were a derivation we see that
\begin{equation}
B[\vec{u},\vec{v}]=[B\vec{u},\vec{v}]+[\vec{u},B\vec{v}]
\end{equation}
but this would have
\begin{equation}
S[\vec{u},\vec{v}]=[S\vec{u},\vec{v}]+[\vec{u},S\vec{v}]
\end{equation}
which is not true. This means that a derivation is of the form of
an antisymmetric matrix, which is the same as being of the form $\alpha[\vec{w},-]$.
\end{proof}

\begin{exercise}
Check the Lie algebra in problem \ref{hw1:ex:one} is:
\begin{enumerate}
\item isomorphic to the Lie algebra $\frak{so}(3)$ of real
  antisymmetric $3\times 3$ matrices; and
\item isomorphic to the Lie algebra $\frak{su}(2)$ of complex
  anti-Hermitian traceless $2\times 2$ matrices.
\end{enumerate}
\end{exercise}

\medbreak\noindent\textsc{\textbf{Answer 2.1:\enspace}}
For the first matter of $\mathfrak{so}(3)$ we have the following
proof:
\begin{proof}
The linear map $\varphi$ basically maps bijectively
\begin{subequations}
\begin{align}
\vec{e}_{1}=\begin{bmatrix}1\\0\\0\end{bmatrix}&\mapsto\varphi(\vec{e}_{1})=
\begin{bmatrix}0 & 0 &
  0\\0&0&-1\\0&1&0\end{bmatrix}\\
\vec{e}_{2}=\begin{bmatrix}0\\1\\0\end{bmatrix}&\mapsto\varphi(\vec{e}_{2})=
\begin{bmatrix}0 & 0 &
  1\\0&0&0\\-1&0&0\end{bmatrix}\\
\vec{e}_{3}=\begin{bmatrix}0\\0\\1\end{bmatrix}&\mapsto\varphi(\vec{e}_{3})=
\begin{bmatrix}0 & -1 &
  0\\1&0&0\\0&0&0\end{bmatrix}\\
\vec{x}=\begin{bmatrix}x_{1}\\x_{2}\\x_{3}\end{bmatrix}
&\mapsto \varphi(\vec{x})=\begin{bmatrix}0 & -x_{3} &
  x_{2}\\x_{3}&0&-x_{1}\\-x_{2}&x_{1}&0\end{bmatrix}
\end{align}
\end{subequations}
which we will prove is an isomorphism.
By direct computation, we find the commutator
$[\varphi(\vec{x}),\varphi(\vec{y})]$ is:
\begin{subequations}
\begin{align}
&\begin{bmatrix}0 & -x_{3} &
  x_{2}\\x_{3}&0&-x_{1}\\-x_{2}&x_{1}&0\end{bmatrix}
\begin{bmatrix}0 & -y_{3} &
  y_{2}\\y_{3}&0&-y_{1}\\-y_{2}&y_{1}&0\end{bmatrix}
-
\begin{bmatrix}0 & -y_{3} &
  y_{2}\\y_{3}&0&-y_{1}\\-y_{2}&y_{1}&0\end{bmatrix}
\begin{bmatrix}0 & -x_{3} &
  x_{2}\\x_{3}&0&-x_{1}\\-x_{2}&x_{1}&0\end{bmatrix}\nonumber\\
&=
\begin{bmatrix}
-x_2 y_2-x_3 y_3 & x_2 y_1 & x_3 y_1 \\
x_1 y_2 & -x_1 y_1-x_3 y_3 & x_3 y_2 \\
x_1 y_3 & x_2 y_3 & -x_1 y_1-x_2 y_2\end{bmatrix}
-\begin{bmatrix}
-x_2 y_2-x_3 y_3 & x_1 y_2 & x_1 y_3 \\
x_2 y_1 & -x_1 y_1-x_3 y_3 & x_2 y_3 \\
x_3 y_1 & x_3 y_2 & -x_1 y_1-x_2 y_2
\end{bmatrix}\\
&= \begin{bmatrix}
0 & x_2 y_1-x_1 y_2 & x_3 y_1-x_1 y_3 \\
x_1 y_2-x_2 y_1 & 0 & x_3 y_2-x_2 y_3 \\
x_1 y_3-x_3 y_1 & x_2 y_3-x_3 y_2 & 0
\end{bmatrix}=\varphi(\vec{x}\times\vec{y})
\end{align}
\end{subequations}
so it preserves the Lie bracket. We see by inspection
\begin{equation}
\varphi^{-1}\left([\varphi(\vec{x}),\varphi(\vec{y})]\right)=\vec{x}\times\vec{y}
\end{equation}
the inverse map also preserves the Lie bracket. This implies that
this linear mapping is a Lie algebra isomorphism.
\end{proof}

\medbreak\noindent\textsc{\textbf{Answer 2.2:\enspace}}
The isomorphism with $\mathfrak{su}(2)$ is contained in the proof:
\begin{proof}
We have another mapping $\psi$ which is an isomorphism of vector
spaces which behave on basis vectors and an arbitrary vector by:
\begin{subequations}
\begin{align}
\vec{e}_{1}=\begin{bmatrix}1\\0\\0\end{bmatrix}&\mapsto\psi(\vec{e}_{1})=
i\begin{bmatrix}0 & 1\\1 & 0\end{bmatrix}\\
\vec{e}_{2}=\begin{bmatrix}0\\1\\0\end{bmatrix}&\mapsto\psi(\vec{e}_{2})=
i\begin{bmatrix}0&i\\-i&0\end{bmatrix}\\
\vec{e}_{3}=\begin{bmatrix}0\\0\\1\end{bmatrix}&\mapsto\psi(\vec{e}_{3})=
i\begin{bmatrix}1&0\\0&-1\end{bmatrix}\\
\vec{x}=\begin{bmatrix}x_{1}\\x_{2}\\x_{3}\end{bmatrix}
&\mapsto \psi(\vec{x})=i\begin{bmatrix}
x_{3} & x_{1}-ix_{2}\\
x_{1}+ix_{2}&-x_{3}
\end{bmatrix}.
\end{align}
\end{subequations}
To show that $\psi$ is an isomorphism of Lie algebras, we need to
show that the Lie bracket is preserved. We see that the
commutator of basis elements of $\mathfrak{su}(2)$ are
\begin{subequations}
\begin{align}
[\psi(\vec{e}_{3}),\psi(\vec{e}_{1})]&=-\begin{bmatrix}1&0\\0&-1\end{bmatrix}\begin{bmatrix}0 & 1\\1 & 0\end{bmatrix}+\begin{bmatrix}0 & 1\\1 & 0\end{bmatrix}\begin{bmatrix}1&0\\0&-1\end{bmatrix}\\
&=-\begin{bmatrix}0&1\\-1&0\end{bmatrix}
+\begin{bmatrix}0&-1\\1&0\end{bmatrix}\\
&=2\psi(\vec{e}_{2})\\
[\psi(\vec{e}_{1}),\psi(\vec{e}_{2})]&=i
\begin{bmatrix}0 & 1\\1 & 0\end{bmatrix}\begin{bmatrix}0&-1\\1&0\end{bmatrix}
-i\begin{bmatrix}0&-1\\1&0\end{bmatrix}
\begin{bmatrix}0 & 1\\1 & 0\end{bmatrix}\\
&=i\begin{bmatrix}1&0\\0&-1\end{bmatrix}
-i\begin{bmatrix}-1&0\\0&1\end{bmatrix}\\
&=2\psi(\vec{e}_{3})\\
[\psi(\vec{e}_{2}),\psi(\vec{e}_{3})]&=i\begin{bmatrix}0&-1\\1&0\end{bmatrix}\begin{bmatrix}1&0\\0&-1\end{bmatrix}
-i\begin{bmatrix}1&0\\0&-1\end{bmatrix}\begin{bmatrix}0&-1\\1&0\end{bmatrix}\\
&=\begin{bmatrix}0&i\\i&0\end{bmatrix}-\begin{bmatrix}0&-i\\-i&0\end{bmatrix}\\
&=2\psi(\vec{e}_{1})
\end{align}
\end{subequations}
We see that this is isomorphic to the Lie algebra on
$\Bbb{R}^{3}$ equipped with the Lie bracket incuced by
\begin{equation}
[\vec{u},\vec{v}]=\vec{u}\times\vec{v}-\vec{v}\times\vec{u}
\end{equation}
the commutator of the cross products. 
\end{proof}

\subsection{Problem Set 2}
\begin{exercise}\label{exercise1HW2}
Find the Lie algebras for the following matrix groups:
\begin{enumerate}
\item The group of real upper triangular matrices.
\item The group of real upper triangular matrices with diagonal
  entries equal to 1.
\item The group $T_{k}$ of real $n\times n$ matrices obeying
  $a_{ii}=1$, $a_{ij}=0$ if $j-i<k$ and $j\not=i$.
\end{enumerate}
\end{exercise}

\noindent{\textbf{Answer \ref{exercise1HW2}.1:\enspace}}
We see that a curve $\gamma\colon[0,1]\to G$ in the group of real
upper triangular matrices such that $\gamma(0)=I$ has components
of the form
\begin{equation}
\gamma(t)=I+c(t)
\end{equation}
where $c(t)$ has zero lower triangular components. This implies
that the Lie algebra consists of matrices $c'(0)$ which are upper
triangular. 

\medbreak
\noindent{\textbf{Answer \ref{exercise1HW2}.2:\enspace}}
We see that curves $\gamma\colon[0,1]\to G$ in the group of real
upper triangular matrices with the diagonal components equal to 1
(such that $\gamma(0)=I$) is of the form
\begin{equation}
\gamma(t)=I+\begin{bmatrix} 
0 & c_{12}(t) & \cdots & c_{1n}(t)\\
0 & 0        & \cdots & c_{2n}(t)\\
\vdots & \vdots & \ddots & \vdots\\
%0 & 0        & \cdots & c_{n-1,n}(t)\\
0 & 0        & \cdots & 0\end{bmatrix}
\end{equation}
The Lie algebra is then consisting of matrices of the form
\begin{equation}
\gamma'(0)=\begin{bmatrix} 
0 & c_{12}'(0) & \cdots & c_{1n}'(0)\\
0 & 0        & \cdots & c_{2n}'(0)\\
\vdots & \vdots & \ddots & \vdots\\
%0 & 0        & \cdots & c_{n-1,n}'(0)\\
0 & 0        & \cdots & 0\end{bmatrix}
\end{equation}
where $c_{ij}'(0)\in\Bbb{R}$ for $0<i<j\leq n$. These are
strictly upper triangular matrices with real entries.

\medbreak
\noindent{\textbf{Answer \ref{exercise1HW2}.3:\enspace}}
We have a matrix $T_{k}$ with components $a_{ij}=0$ if both $j<i+k$
\textsc{\textbf{and}} $j\not=i$. For example, consider $3\times3$
matrices. We see that
\begin{equation}
T_{1}=\begin{bmatrix}
1 & a_{12} & a_{13}\\
0 & 1 & a_{23}\\
0 & 0 & 1
\end{bmatrix}
\end{equation}
where $a_{12}$, $a_{13}$, and $a_{23}$ are real numbers not
necessarily zero (we use the well known fact that $2\not<1+1$,
$3\not<1+1$, and $3\not<2+1$ respectively). Similarly, we see
that
\begin{equation}
T_{2}=\begin{bmatrix}
1 & 0 & b_{13}\\
0 & 1 & 0\\
0 & 0 & 1
\end{bmatrix}
\end{equation}
where $b_{13}\in\Bbb{R}$, since here $b_{12}=b_{23}=0$. Lastly we
see that
\begin{equation}
T_{3}=I
\end{equation}
is the identity element. We see then that the product of two
matrices $T_{a}T_{b}=\widetilde{T}_{\min\{a,b\}}$ produce another
matrix in the group.

We see, however, that this group's elements of the form $T_{1}$
form a subgroup which is isomorphic to the Lie group
described in \hyperref[exercise1HW2]{Exercise \ref{exercise1HW2}.2.}
We also see that $\{T_{i+1}\}\subset\{T_{i}\}$ are subgroups, for
all $i\in\Bbb{N}$. The Lie algebra for the group described in \hyperref[exercise1HW2]{Answer \ref{exercise1HW2}.2} is thus completely
isomorphic to the Lie algebra we are interested in. 

\begin{exercise}\label{ex2HW2}
Check the groups of \hyperref[exercise1HW2]{Exercise \ref{exercise1HW2}}
 and corresponding Lie algebras are solvable.
\end{exercise}
\medbreak
\noindent\textsc{\textbf{Answer \ref{ex2HW2}:\enspace}}
Recall that if $\mathfrak{g}$ is the Lie Algebra for the group
$G$, then we use the notation from Knapp's \emph{Lie Groups: Beyond An Introduction}
 (Second Ed.) that
\begin{equation}
\mathfrak{g}^{0}=
\mathfrak{g},\qquad
\mathfrak{g}^{1}=[
\mathfrak{g},
\mathfrak{g}],\qquad
\mathfrak{g}^{j+1}=[
\mathfrak{g}^{j},
\mathfrak{g}^{j}].
\end{equation}
We say that $\mathfrak{g}$ is \define{Solvable} if
$\mathfrak{g}^{j}=0$ for some $j\in\Bbb{Z}$. Here we are using
notation that
\begin{equation}
[\mathfrak{a},\frak{b}]=\mathop{\rm span}\nolimits\{[X,Y]\mid X\in\frak{a},Y\in\frak{b}\}
\end{equation}
wbere $\frak{a}$, $\frak{b}$ are subsets of a Lie algebra
$\frak{g}$ and we take the span over a field $\Bbb{F}$ (in our
case $\Bbb{R}$). A group $G$ is solvable iff it is connected and
its Lie algebra is solvable. So we need to check for each group
that: 1) it is connected, and 2) its Lie algebra is solvable.

\medbreak
\noindent\textsc{\textbf{Answer \ref{ex2HW2}.1:\enspace}}
We consider $\frak{g}$ the Lie algebra for the group of $n\times
n$ upper triangular matrices with real value entries. Let
$X,Y\in\frak{g}$, write
\begin{equation}
X=D_{X}+\widetilde{X},\qquad Y=D_{Y}+\widetilde{Y}
\end{equation}
where $D_{X}$, $D_{Y}$ are diagonal matrices, and
$\widetilde{X}$, $\widetilde{Y}$ are off-diagonal matrices. We
see that the commutator of these two elements are then
\begin{subequations}
\begin{align}
[X,Y]&=[D_{X}+\widetilde{X},D_{Y}+\widetilde{Y}]\\
&=[D_{X},D_{Y}]+[D_{X},\widetilde{Y}]+[\widetilde{X},D_{Y}]+[\widetilde{X},\widetilde{Y}]\\
&=0+0+0+[\widetilde{X},\widetilde{Y}]
\end{align}
\end{subequations}
and we can write in block form that
\begin{equation}
[\widetilde{X},\widetilde{Y}] = a_{ij} = \begin{cases}\not=0&\hbox{if
  }i+1\leq j\\
0 & \hbox{otherwise}.
\end{cases}
\end{equation}
By induction, we see that elements of $\frak{g}^{m}$ are matrices
of the form
\begin{equation}
a_{ij} = \begin{cases}\not=0&\hbox{if
  }i+m\leq j\\
0 & \hbox{otherwise}
\end{cases}
\end{equation}
which is zero for all $m\geq n$.

We need to prove that the group is connected in order to conclude
that it is solvable. We can see that given any two matrices $X$,
$Y\in G$ there is a path $\gamma\colon[0,1]\to G$ connecting them
defined by
\begin{equation}
\gamma(t)=tX+(1-t)Y
\end{equation}
which is always in the group $\forall t\in[0,1]$. Thus the group
is path-connected. This implies that $G$ is connected.

\medbreak
\noindent\textsc{\textbf{Answer \ref{ex2HW2}.2:\enspace}}
We see that the Lie algebra we are working with is actually
isomorphic to $\frak{g}^{1}$ from Answer 2.1, which we saw is
solvable. We need to show that the group is connected to prove
that it is solvable. Let $X$, $Y\in G$ be arbitrary group
elements. We construct a path $\gamma\colon[0,1]\to G$ from $X$
to $Y$ defined by
\begin{equation}
\gamma(t)=tY+(1-t)X.
\end{equation}
We see that $\gamma(t)\in G$ for all $t\in[0,1]$, which means
that the group is path-connected. This implies that the group is
connected and, moreover, solvable.


\medbreak
\noindent\textsc{\textbf{Answer \ref{ex2HW2}.3:\enspace}}
We see that  the Lie algebra we are working with is, again,
isomorphic to $\frak{g}^{1}$ from Answer 2.1, which we saw is
solvable. We need to show that the group is connected, which we
will do by proving it is path-connected (a stronger notion!). For
$T_{i}$, $T_{j}$ be arbitrary matrices in our group, we construct
a path $\gamma\colon[0,1]\to G$ by
\begin{equation}
\gamma(t)=tT_{j}+(1-t)T_{i}.
\end{equation}
We see that this path $\gamma(t)\in G$ for all $t\in[0,1]$. This
implies path-connectedness and, more importantly, solvability of
the group.

\begin{defn}
A group is \define{Solvable} if
\begin{equation}
G=G_{0}\supset G_{1}\supset\cdots\supset G_{n}=\{e\}
\end{equation}
is a tower of groups such that $G_{n-1}\supset G_{n}$ and
$G_{n-1}/G_{n}$ is Abelian.
\end{defn}
\begin{defn}
A Lie Algebra $\frak{g}$ is \define{Solvable} iff we can find
ideals $\frak{g}_{0}=\frak{g}$, $\frak{g}_{1}$, ...,
$\frak{g}_{n}=0$ such that $\frak{g}_{i}\supset\frak{g}_{i+1}$
and $\frak{g}_{i}/\frak{g}_{i+1}$ is Abelian.
\end{defn}
\noindent{\bf N.B.:\enspace} %% Sometimes $\frak{g}_{i+1}$ is given
%% the extra condition of being an ideal in $\frak{g}_{i}$ (see
%% e.g. Knapp's {\it Lie Groups: Beyond an Introduction} Second Ed., Chapter I.5). 
Recall that an ideal $\frak{h}$ for a Lie algebra $\frak{g}$
satisfies the property that
\begin{equation}
[\frak{h},\frak{g}]=\mathop{\rm span}\nolimits\{[X,Y]\mid X\in\frak{h},Y\in\frak{g}\}\subseteq\frak{h}.
\end{equation}

\begin{prop}\label{prop:ab}
Let $\frak{g}$ be a Lie algebra,
$\frak{g}^{1}=[\frak{g},\frak{g}]$, and inductively
$\frak{g}^{i+1}=[\frak{g}^{i},\frak{g}^{i}]$. Then
$\frak{g}^{i}/\frak{g}^{i+1}$ is Abelian.
\end{prop}
\begin{proof}
It is obvious. We mod out all commutation relations to vanish,
which is the necessary and sufficient conditions for a Lie
algebra to be Abelian.
\end{proof}
\begin{prop}
Let $\frak{g}$ be a Lie algebra, $\frak{h}\subset\frak{g}$ be an
ideal. Then $[\frak{h},\frak{h}]\subset\frak{h}$ is an ideal of $\frak{h}$.
\end{prop}
\begin{proof}
It is obvious.
\end{proof}
\begin{prop}
Let $N\subset G$ be a normal Lie subgroup. Then
$\frak{n}\subset\frak{g}$ is an ideal of the Lie algebra.
\end{prop}
\begin{proof}
We see that since $N$ is normal, for any $g\in G$ that
$gNg^{-1}\subset N$. Consider a curve $\gamma\colon[0,1]\to N$
such that $\gamma(0)=I$ is the identity element.
\end{proof}
\begin{thm}
Let $G$ be a Lie group and $\frak{g}$ its Lie algebra. If
$\frak{g}$ is solvable, then $G$ is solvable.
\end{thm}
\begin{proof}
We recall that $\exp\colon\frak{g}\to G$ recovers the Lie
group. We see that if we have a tower of ideals
\begin{equation}
\frak{g}_{0}\supset\frak{g}_{1}\supset\cdots\supset\frak{g}_{n}=\{0\},
\end{equation}
then by exponentiation we get a tower of normal Lie subgroups
\begin{equation}
G\supset G_{1}\supset\cdots\supset G_{n}=\{\exp(0)\}.
\end{equation}
We also see that proposition \ref{prop:ab} gives us a method to
construct a tower of ideals to demonstrate solvability for a Lie
algebra. Additionally, if $\frak{g}^{i}/\frak{g}^{i+1}$ is
Abelian, by exponentiation $G^{i}/G^{i+1}$ is Abelian. This is
sufficient to stating that if $\frak{g}$ is solvable, then $G$ is
solvable too.
\end{proof}
\begin{rmk}
The Lie algebra $\frak{sl}(n)$ (also denoted by the symbol
$A_{n-1}$) consists of traceless $n\times n$ complex
matrices. The symbol $E_{i,j}$ denotes a matrix with only one
nonzero entry that is equal to 1 and located in the $i$-th row
and $j$-th column.
\end{rmk}

\subsection{Exercises}
\begin{exercise}
\begin{enumerate}
\item Check that the matrices $E_{i,j}$ (for $i\not=j$) and the matrices $h_{i}=E_{i,i}-E_{i+1,i+1}$ form a basis of $\frak{sl}(n)$.
\item Find the structure constants in this basis.
\end{enumerate}
\end{exercise}

\answer{1}
We need to check that any matrix $X\in\frak{sl}(n)$ can be
written as a linear combination of $E_{i,j}$ and $h_{i}$. We see
that if we work in the canonical basis of $\Bbb{C}^{n}$, then we
can write out $X$ with components
\begin{equation}
X=D+\widetilde{X}=\delta_{ij}\lambda_{i}+[\widetilde{x}_{ij}]
\end{equation}
where $D$ is diagonal, $\widetilde{X}$ has the diagonal
components identically zero. We see then that we can trivially
write out
\begin{equation}
\widetilde{X}=\sum_{i,j}\widetilde{x}_{ij}E_{i,j}.
\end{equation}
So we need to show that we can write out the diagonal part $D$ in
terms of $h_{i}$.

We see that we can write
\begin{equation}
D = \lambda_{1}h_{1}+(\lambda_{1}+\lambda_{2})h_{2}+\cdots+(\sum^{k}_{j=1}\lambda_{j})h_{k}+\cdots+(\sum^{n}_{j=1}\lambda_{j})h_{n}
\end{equation}
which permits us to verify that $D$ is indeed traceless, and a
linear combination of the basis vectors $h_{i}$.

\answer{2}
We see first of all that
\begin{equation}
[h_{i},h_{j}]=0
\end{equation}
for all $i,j=1,...,n$. It is obvious, since $h$ is diagonal.

We observe that
\begin{equation}
E_{i,j}E_{j,k}=E_{i,k}\quad\Rightarrow\quad E_{i,j}E_{k,l}=\delta_{j,k}E_{i,\,l}
\end{equation}
where $\delta_{j,k}$ is the Kronecker delta, which implies
\begin{equation}
[E_{i,j},E_{k,l}]=\delta_{j,k}E_{i,l}-\delta_{i,l}E_{j,k}.
\end{equation}
This permits us to observe that
\begin{subequations}
\begin{align}
[E_{i,j},h_{k}] &= [E_{i,j},E_{k,k}-E_{k+1,k+1}]\\
&=[E_{i,j},E_{k,k}]-[E_{i,j},E_{k+1,k+1}]\\
&=(\delta_{j,k}E_{i,k}-\delta_{i,k}E_{j,k})
-(\delta_{j,k+1}E_{i,k+1}-\delta_{i,k+1}E_{j,k+1})
\end{align}
\end{subequations}
Thus we have the commutation relations for the generators of
$\frak{sl}(n)$, which permits us to deduce the structure
constants. Observe if we omit the commas in the indices, we can write
\begin{subequations}
\begin{align}
[E_{ij},E_{kl}] &= {f_{ijkl}}^{ab}E_{ab}\\
&=(\delta^{a}_{i}\delta^{b}_{l}\delta_{jk}-\delta^{a}_{j}\delta^{b}_{k}\delta_{il})E_{ab}
\end{align}
\end{subequations}
which permits us to deduce some of the structure constants. We
also have
\begin{equation}
[E_{ij},h_{k}]={f_{ijlm}}^{ab}(\delta^{l}_{k}\delta^{m}_{k}-\delta^{l}_{k+1}\delta^{m}_{k+1})E_{ab}
\end{equation}
and
\begin{equation}
[h_{i},h_{j}]=0.
\end{equation}

\begin{exercise}
\begin{enumerate}
\item Check that subalgebra $\frak{h}$ of all diagonal matrices is a maximal
commutative subalgebra.
\item Prove that there exists a basis of $\frak{sl}(n)$
  consisting of eigenvectors for elements of $\frak{h}$. (This means
  that $\frak{h}$ is a Cartan subalgebra of $\frak{sl}(n)$.)
\end{enumerate}
\end{exercise}

\answer{1} 
Let $\frak{h}$ consist of diagonal matrices in $\frak{sl}(n)$. We
need to show (a) it is Abelian, (b) it is maximal. We see that
\begin{subequations}
\begin{align}
[\frak{h},\frak{h}] &= \mathop{\rm Span}\nolimits\{[X,Y]\mid X,Y\in\frak{h}\}\\
&= \mathop{\rm Span}\nolimits\{0\}\\
&=0.
\end{align}
\end{subequations}
Thus it is Abelian and moreover a nilpotent Lie algebra.

We see that
\begin{subequations}
\begin{align}
N_{\frak{sl}(n)}(\frak{h}) &= \{X\in\frak{sl}(n)\mid
[X,Y]\in\frak{h}\quad\forall Y\in\frak{h}\}\\
&=\{ E_{ij}\in\frak{sl}(n)\mid
[E_{ij},Y]\in\frak{h}\quad\forall Y\in\frak{h}\}\cup\{ h_{k}\in\frak{sl}(n)\mid
[h_{k},Y]\in\frak{h}\quad\forall Y\in\frak{h}\}\\
&=\emptyset\cup\{ h_{k}\in\frak{sl}(n)\mid
[h_{k},Y]\in\frak{h}\quad\forall Y\in\frak{h}\}\\
&=\emptyset\cup\frak{h}=\frak{h}.
\end{align}
\end{subequations}
That is to say that $\frak{h}$ is self-normalising, so $\frak{h}$
is a Cartan subalgebra.

Since $\frak{h}$ is self-normalising, if $\frak{h}'$ is another
Abelian subalgebra, then by definition for each $x\in\frak{h}'$
\begin{equation}
[x,y]\in\frak{h}
\end{equation}
for every $y\in\frak{h}$. This implies that $x\in\frak{h}$ and
more importantly $\frak{h'}\subset\frak{h}$.

\answer{2} 
We observe that
\begin{equation}
[a^{i}E_{i,i},E_{j,k}]=(a^{j}-a^{k})E_{j,k}.
\end{equation}
It implies that
\begin{equation}
[a^{i}h_{i},E_{j,k}]=[\widetilde{a}^{i}E_{i,i},E_{j,k}]=(\widetilde{a}^{j}-\widetilde{a}^{k})E_{j,k}
\end{equation}
or that $E_{j,k}$ is an eigenvector for $\ad(a^{i}h_{i})$.

\begin{exercise}
Check that $e_{i} = E_{i,i+1}$ and $f_{i} = E_{i+1,i}$ form a system of multiplicative generators of
$\frak{sl}(n)$. Prove relations
\begin{subequations}
\begin{align}
  [e_{i} , f_{j} ] = \delta_{ij} h_{i},\qquad [h_{i} , h_{j} ] = 0,\\
  [h_{i} , e_{j} ] = a_{ij} e_{j} ,\qquad [h_{i} , f_{j} ] = -a_{ij} f_{j},\\
  (\ad{e_{i}} )^{-a_{ij} +1} e_{j} = 0,\qquad (\ad{f_{i}} )^{-a_{ij} +1} f_{j} = 0
\end{align}
\end{subequations}
for some choice of matrix $a_{ij}$.

 We use here the notation $\ad(x)$ for the operator transforming $y$ into $[x, y]$.
\end{exercise}

\answer{} Observe that
\begin{equation}
E_{j,k} = \prod^{k}_{p=j} e_{p}
\end{equation}
assuming that $k>j$ and
\begin{equation}
E_{j,k} = \prod^{j}_{p=k}f_{p}
\end{equation}
otherwise. So $e_{i}$ and $f_{j}$ generate the algebra.

With regards to the commutation relations, we see first that 
\begin{equation}
[h_{i},h_{j}]=0
\end{equation}
for all $i,j$ since $h_{i}$ is diagonal and thus commutes with
other diagonal matrices. 

We also see that
\begin{subequations}
\begin{align}
e_{i}f_{j} &= E_{i,i+1}E_{j+1,j}\\
&= \delta_{i,j}E_{i,i}
\end{align}
\end{subequations}
which implies
\begin{equation}
[e_{i},f_{j}]=\delta_{i,j}E_{i,i}-\delta_{i,j}E_{i+1,i+1}=\delta_{i,j}h_{i}.
\end{equation}
It follows then from the last homework problem that
\begin{equation}
[h_{i},e_{j}] = 2\delta_{ij}e_{j}-\delta_{i,j+1}e_{j}.
\end{equation}
Similarly
\begin{equation}
[h_{i},f_{j}] = -2\delta_{ij}f_{j}+\delta_{i,j+1}f_{j}.
\end{equation}
If we write
\begin{equation}
[e_{i},e_{j}]={c_{ij}}^{k}e_{k},\qquad[f_{i},f_{j}]={{\widetilde{c}}_{ij}}{}^{k}f_{k}
\end{equation}
then we see that if $|j-i|>1$ then
\begin{equation}
\ad(e_{i})(e_{j})=0,\qquad \ad(f_{i})(f_{j})=0.
\end{equation}
If $i=j$, then
\begin{equation}
\ad(e_{j})e_{j}=0,\qquad \ad(f_{j})f_{j}=0
\end{equation}
and for $i=j+1$ we have
\begin{equation}
\ad(e_{j+1})e_{j}=0,\qquad \ad(f_{j+1})f_{j}=0.
\end{equation}
This is precisely as desired.

\subsection{Exercises}
\subsubsection{Algebra \texorpdfstring{$\ClassicalGroup{D}_{n}$}{Dn}}

The Lie algebra $\ClassicalGroup{D}_{n}$ consists of $2n\times2n$ complex matrices
$L$ obeying
\begin{equation}
(FL)^{T}+FL=0
\end{equation}
where, in block form,
\begin{equation}
F=\begin{bmatrix}
0&1\\
1&0
\end{bmatrix}.
\end{equation}

\begin{exercise}
Check that $\ClassicalGroup{D}_{n}$ is isomorphic to the complexification of the
Lie algebra of the orthogonal group $O(2n)$.
\end{exercise}

\answer
We see first that we can diagonalize $F$. Observe that
\begin{equation}
\begin{bmatrix}
1&1\\
1&-1
\end{bmatrix}
\begin{bmatrix}
0&1\\
1&0
\end{bmatrix}
\begin{bmatrix}
1&1\\
1&-1
\end{bmatrix}=\begin{bmatrix}
1&1\\
-1&1
\end{bmatrix}\begin{bmatrix}
1&1\\
1&-1
\end{bmatrix}=\begin{bmatrix}
2&0\\
0&-2
\end{bmatrix}
\end{equation}
So we have
\begin{equation}
F=\left(\frac{1}{\sqrt{2}}\begin{bmatrix}
1&1\\
1&-1
\end{bmatrix}\right)\begin{bmatrix}1&0\\0&-1\end{bmatrix}
\left(\frac{1}{\sqrt{2}}\begin{bmatrix}
1&1\\
1&-1
\end{bmatrix}\right)
\end{equation}
We observe that since we are working with complex matrices, we
can write
\begin{equation}
\begin{bmatrix}1&0\\
0&-1
\end{bmatrix}=\begin{bmatrix}1&0\\
0&i\end{bmatrix}I\begin{bmatrix}1&0\\
0&i\end{bmatrix}
\end{equation}
where $I$ is the $2n\times2n$ identity matrix. Thus we have a
mapping
\begin{equation}
\varphi(X)=\frac{1}{2}\begin{bmatrix}1&1\\1&-1\end{bmatrix}\begin{bmatrix}1&0\\
0&i\end{bmatrix}X\begin{bmatrix}1&0\\
0&i\end{bmatrix}\begin{bmatrix}1&1\\1&-1\end{bmatrix}
\end{equation}
which maps the condition
\begin{equation}
\varphi\left((FX)^{T}+FX\right)=\varphi(X)^{T}+\varphi(X)
\end{equation}
to the condition for $\varphi(X)\in\CC\frak{o}(2n)$. We see
that this mapping is invertible trivially.

\begin{exercise}
Check that the matrices
\begin{equation}
e_{ij}:=\begin{bmatrix}E_{ij}&0\\
0&-E_{ji}
\end{bmatrix}
\end{equation}
together with the matrices
\begin{equation}
f_{pq}:=\begin{bmatrix}0&E_{pq}-E_{qp}\\
0&0
\end{bmatrix},\qquad
g_{pq}:=\begin{bmatrix}0&0\\
E_{pq}-E_{qp}&0
\end{bmatrix}
\end{equation}
form a basis of $\ClassicalGroup{D}_{n}$. 

Here $i,j=1,\ldots,n$, $1\leq p<q\leq n$, and $E_{i,j}$ has only
one nonzero entry that is equal to unity located in the $i^{th}$
row and $j^{th}$ column.
\end{exercise}

\answer We see that, when written in block form, the condition
for $\ClassicalGroup{D}_{n}$ implies that
\begin{subequations}
\begin{align}
F\begin{bmatrix}A&B\\C&D
\end{bmatrix}&=\begin{bmatrix}C&D\\A&B\end{bmatrix}\\
\begin{bmatrix}A^{T}&C^{T}\\B^{T}&D^{T}
\end{bmatrix}F&=\begin{bmatrix}C^{T}&A^{T}\\
D^{T}&B^{T}
\end{bmatrix}\\
\Rightarrow\quad\begin{bmatrix}C&D\\A&B\end{bmatrix}&=-\begin{bmatrix}C^{T}&A^{T}\\
D^{T}&B^{T}
\end{bmatrix}
\end{align}
\end{subequations}
Thus we see that if $L\in \ClassicalGroup{D}_{n}$, it can be written in block form
as
\begin{equation}
L=\begin{bmatrix}A&B\\
C&-A^{T}
\end{bmatrix}
\end{equation}
where $A$ is any $n\times n$ matrix, $B,C$ are antisymmetric
$n\times n$ matrices. We see we can write this as
\begin{equation}
L=a^{ij}e_{ij}+b^{pq}f_{pq}+c^{pq}g_{pq}
\end{equation}
where we sum over $i$, $j$, $p$, $q$.

\begin{exercise}
Check that the subalgebra $\frak{h}$ of all matrices of the form
\begin{equation}
\begin{bmatrix}
A&0\\
0&-A
\end{bmatrix}
\end{equation}
(where $A$ is a diagonal matrix) is a maximal commutative
subalgebra, and prove that there exists a basis of $\ClassicalGroup{D}_{n}$
consisting of eigenvectors for elements of $\frak{h}$ acting on
$\ClassicalGroup{D}_{n}$ by means of adjoint representation. (This means that
$\frak{h}$ is a Cartan subalgebra of $\ClassicalGroup{D}_{n}$.)
\end{exercise}

\answer\marginpar{$\frak{h}$ is Cartan Subalgebra} Well, we see
that $h_{i}=e_{i,i}$ forms a basis of $\frak{h}$. We know from
homework 1 that if we include nonzero off-diagonal components,
the algebra is non-Abelian. So every Abelian subalgebra must be
generated by some subset of $h_{i}$. This means that
\begin{equation}
\frak{h}={\rm span}\{a^{i}h_{i}\colon a^{i}\in\Bbb{C}\}
\end{equation}
is a Maximal Abelian subalgebra.

\marginpar{Eigenbasis}We see that
\begin{subequations}
\begin{align}
[h_{i},e_{jk}]&=h_{i}e_{jk}-e_{jk}h_{i}\\
&=\begin{bmatrix}
E_{i,i}&0\\
0     &-E_{i,i}
\end{bmatrix}
\begin{bmatrix}E_{jk}&0\\
0&-E_{kj}
\end{bmatrix}
-\begin{bmatrix}E_{jk}&0\\
0&-E_{kj}
\end{bmatrix}
\begin{bmatrix}
E_{i,i}&0\\
0     &-E_{i,i}
\end{bmatrix}\\
&=\begin{bmatrix}
E_{i,i}E_{j,k} - E_{j,k}E_{i,i} & 0\\
0 & E_{i,i}E_{k,j}-E_{k,j}E_{i,i}
\end{bmatrix}\\
&=\delta_{i,j}e_{i,k}-\delta_{k,i}e_{j,i}\\
&=(\delta_{i,j}-\delta_{k,i})e_{j,k}.
\end{align}
\end{subequations}
Similarly
\begin{subequations}
\begin{align}
[h_{i},f_{pq}]&=
\begin{bmatrix}E_{i,i}&0\\0&-E_{i,i}\end{bmatrix}
\begin{bmatrix}
0&E_{pq}-E_{qp}\\
 0&0
 \end{bmatrix}
-\begin{bmatrix}
0&E_{pq}-E_{qp}\\
 0&0
 \end{bmatrix}
\begin{bmatrix}E_{i,i}&0\\0&-E_{i,i}\end{bmatrix}\\
&=\begin{bmatrix}
0&E_{i,i}E_{p,q}-E_{i,i}E_{q,p}\\
0&0
\end{bmatrix}
-\begin{bmatrix}0&-(E_{p,q}E_{i,i}-E_{q,p}E_{i,i})\\
0&0\end{bmatrix}\\
&=\begin{bmatrix}
0&E_{i,i}E_{p,q}+E_{p,q}E_{i,i}-E_{i,i}E_{q,p}-E_{q,p}E_{i,i}\\
0&0
\end{bmatrix}\\
&=(\delta_{i,p}+\delta_{i,q})f_{pq}.
\end{align}
\end{subequations}
Lastly
\begin{subequations}
\begin{align}
[h_{i},g_{pq}]&=
\begin{bmatrix}E_{i,i}&0\\0&-E_{i,i}\end{bmatrix}
\begin{bmatrix}
0&0\\
E_{pq}-E_{qp} &0
 \end{bmatrix}
-\begin{bmatrix}
0&0\\
E_{pq}-E_{qp}&0
 \end{bmatrix}
\begin{bmatrix}E_{i,i}&0\\0&-E_{i,i}\end{bmatrix}\\
&=\begin{bmatrix}0&0\\
-E_{i,i}E_{p,q}+E_{i,i}E_{q,p}&0
\end{bmatrix}
-\begin{bmatrix}
0&0\\
E_{p,q}E_{i,i}-E_{q,p}E_{i,i}&0
\end{bmatrix}\\
&=-(\delta_{i,p}+\delta_{i,q})g_{p,q}.
\end{align}
\end{subequations}
Thus $e_{jk}$, $f_{pq}$, and $g_{pq}$ form a basis for the Lie
algebra, and are weight vectors.


\begin{exercise}
Check that $e_{i}=e_{i,i+1}$ for $i=1,...,n-1$ and
$e_{n}=f_{n-1,n}$; $f_{i}=e_{i+1,i}$ for $i=1,...,n-1$ and
$f_{n}=g_{n-1,n}$ form a system of multiplicative generators of
$D_{n}$. Prove the relations
\begin{subequations}
\begin{align}
[e_{i},f_{j}]&=\delta_{ij}h_{i}\\
[h_{i},h_{j}]&=0\\
[h_{i},e_{j}]&=a_{ij}e_{j}\\
[h_{i},f_{j}]&=-a_{ij}f_{j}\\
({\rm ad}e_{i})^{1-a_{ij}}e_{j}&=0\qquad\hbox{when $i\not=j$}\\
({\rm ad}f_{i})^{1-a_{ij}}f_{j}&=0\qquad\hbox{when $i\not=j$}
\end{align}
\end{subequations}
\end{exercise}

\answer It follows immediately from the calculations performed in
the answer to 3.

\subsubsection{Algebra \texorpdfstring{$\ClassicalGroup{C}_{n}$}{Cn}}

Consider the Lie algebra $\ClassicalGroup{C}_{n}$ consisting of $2n\times2n$
complex matrices obeying
\begin{equation}
(FL)^{T}+FL=0
\end{equation}
where
\begin{equation}
F=\begin{bmatrix}0&1\\
-1&0
\end{bmatrix}.
\end{equation}

\begin{exercise}
Check that $\ClassicalGroup{C}_{n}$ is isomorphic to the complexification of the
Lie algebra of the compact group $\Sp{2n}\cap \U{2n}$ where
$\Sp{2n}$ stands for the group of linear transformations of
$\CC^{2n}$ preserving non-degerate anti-symmetric bilinear
form and $\U{2n}$ denotes unitary group.
\end{exercise}

\answer We see first of all that we may diagonalize $F$ when we
write it as
\begin{equation}
\frac{1}{2}\begin{bmatrix}-i&1\\i&1
\end{bmatrix}
\begin{bmatrix}
i&0\\0&i
\end{bmatrix}
\begin{bmatrix}-i&i\\
1&1
\end{bmatrix}=F.
\end{equation}
We see that we can construct a morphism
\begin{equation}
\varphi(X)=\frac{1}{2}\begin{bmatrix}-i&i\\
1&1
\end{bmatrix}X\begin{bmatrix}-i&1\\i&1
\end{bmatrix}
\end{equation}
which is invertible, whose domain is $\ClassicalGroup{C}_{n}$ and whose codomain
is precisely the complexified Lie algebra for $\U{2n}\cap\Sp{2n}$.

\begin{exercise}\label{hw4:ex6}
Check that the matrices
\begin{subequations}
\begin{align}
e_{ij} &= \begin{bmatrix}E_{ij}&0\\0&-E_{ji}
\end{bmatrix}\\
f_{pq}&= \begin{bmatrix}0&E_{pq}+E_{qp}\\0&0
\end{bmatrix}\\
g_{pq}&=\begin{bmatrix}0&0\\E_{pq}+E_{qp}&0
\end{bmatrix}
\end{align}
\end{subequations}
form a basis of $\ClassicalGroup{C}_{n}$, where $i,j=1,...,n$ and $1\leq p\leq
q\leq n$.
\end{exercise}

\answer If we write an element of our algebra in block form as
\begin{equation}
L = \begin{bmatrix}
A&B\\C&D\end{bmatrix}
\end{equation}
then by our conditions, we deduce
\begin{subequations}
\begin{align}
FL &= \begin{bmatrix}C&D\\-A&-B\end{bmatrix}\\
L^{T}F&=\begin{bmatrix}C^{T}&-A^{T}\\D^{T}&-B^{T}\end{bmatrix}
\end{align}
\end{subequations}
which implies that $B$ and $C$ are symmetric, and $-A^{T}=D$. So
in other words we can write
\begin{equation}
L = \begin{bmatrix} A & \frac{1}{2}(B+B^{T})\\
\frac{1}{2}(C+C^{T}) & -A^{T}
\end{bmatrix}.
\end{equation}
However, this permits us to write
\begin{subequations}
\begin{align}
L &= \begin{bmatrix}A&0\\0&-A^{T}
\end{bmatrix}+\begin{bmatrix}0&\frac{1}{2}(B+B^{T})\\
0&0
\end{bmatrix}+\begin{bmatrix}0&0\\
\frac{1}{2}(C+C^{T})&0
\end{bmatrix}\\
&=a^{ij}e_{ij}+b^{pq}f_{pq}+c^{pq}g_{pq}
\end{align}
\end{subequations}
which is precisely what we desired to demonstrate.

\begin{exercise}
Check that the subalgebra $\frak{h}$ of all matrices of the form
\begin{equation}
\begin{bmatrix}
A&0\\
0&-A
\end{bmatrix}
\end{equation}
where $A$ is a diagonal matrix, is a maximal commutative
subalgebra. Prove there exists a basis of $\ClassicalGroup{C}_{n}$ consisting of
eigenvectors for elements of $\frak{h}$ acting on $\ClassicalGroup{C}_{n}$ by
means of adjoint representation.
\end{exercise}

\answer The reasoning for the maximal commutative subalgebra is
the same as for the $\ClassicalGroup{D}_{n}$ case.

\begin{exercise}
Check that $e_{i}=e_{i,i+1}$ ($i=1,...,n-1$) and $e_{n}=f_{n,n}$;
$f_{i}=e_{i+1,i}$ ($i=1,...,n-1$) and $f_{n}=g_{n,n}$ form a
system of generators of $\ClassicalGroup{C}_{n}$. Prove
\begin{subequations}
\begin{align}
[e_{i},f_{j}]&=\delta_{ij}h_{i},\\
[h_{i},h_{j}]&=0\\
[h_{i},e_{j}]&=a_{ij}e_{j},\\
[h_{i},f_{j}]&=-a_{ij}f_{j},\\
({\rm ad}e_{i})^{1-a_{ij}}e_{j}&=0,\qquad i\not=j \\
({\rm ad}f_{i})^{1-a_{ij}}f_{j}&=0,\qquad i\not=j
\end{align}
\end{subequations}
\end{exercise}
\answer We see that it follows from the calculations performed in
the previous answer to exercise \ref{hw4:ex6}.
\subsubsection{Algebra \texorpdfstring{$\ClassicalGroup{B}_{n}$}{Bn}}

The algebra $\ClassicalGroup{B}_{n}$ consists of $(2n+1)\times(2n+1)$ complex
matrices obeying
\begin{equation}
L^{T}F+FL=0
\end{equation}
where
\begin{equation}
F=\begin{bmatrix}1&0&0\\
0&0&1\\
0&1&0
\end{bmatrix}
\end{equation}
is written in block form.

\begin{exercise}
Show that $\ClassicalGroup{B}_{n}$ is isomorphic to the complexified Lie algebra of $\ORTH{2n+1}$.
\end{exercise}

\answer We construct an isomorphism by diagonalizing $F$:
\begin{equation}
F = \begin{bmatrix} 1&0&0\\
0&\frac{1}{\sqrt{2}}&-\frac{1}{\sqrt{2}}\\
0&\frac{1}{\sqrt{2}}&\frac{1}{\sqrt{2}}
\end{bmatrix}
\begin{bmatrix}
1&0&0\\
0&-1&0\\
0&0&1
\end{bmatrix}
\begin{bmatrix} 1&0&0\\
0&\frac{1}{\sqrt{2}}&\frac{1}{\sqrt{2}}\\
0&-\frac{1}{\sqrt{2}}&\frac{1}{\sqrt{2}}
\end{bmatrix}
\end{equation}
which allows us to write
\begin{equation}
\varphi(X)=\begin{bmatrix} 1&0&0\\
0&\frac{1}{\sqrt{2}}&\frac{1}{\sqrt{2}}\\
0&-\frac{1}{\sqrt{2}}&\frac{1}{\sqrt{2}}
\end{bmatrix}
\begin{bmatrix}1&0&0\\
0&i&0\\
0&0&1\end{bmatrix}X
\begin{bmatrix}1&0&0\\
0&i&0\\
0&0&1\end{bmatrix}\begin{bmatrix} 1&0&0\\
0&\frac{1}{\sqrt{2}}&-\frac{1}{\sqrt{2}}\\
0&\frac{1}{\sqrt{2}}&\frac{1}{\sqrt{2}}
\end{bmatrix}
\end{equation}
which clearly is an isomorphism as desired.

\begin{exercise}
Check that the subalgebra $\frak{h}$ of all matrices of the form
\begin{equation}
\begin{bmatrix}0&0&0\\
0&A&0\\
0&0&-A
\end{bmatrix}
\end{equation}
(where $A$ is a diagonal matrix) is a maximal Abelian subalgebra,
and prove there is a basis of $\ClassicalGroup{B}_{n}$ consisting of eigenvectors
for elements of $\frak{h}$ acting on $\ClassicalGroup{B}_{n}$ by the adjoint
representation.
\end{exercise}
\answer The proof for $\frak{h}$ being a Cartan subalgebra is
\emph{ALMOST the same} as for the $\ClassicalGroup{D}_{n}$ case. The basic
reasoning is the same, we have our beloved isomorphism
\begin{equation}
\varphi\colon \ClassicalGroup{B}_{n}\to\Bbb{C}\lie\big(\ORTH{2n+1}\big)
\end{equation}
which is defined by
\begin{equation}
\varphi(X)=PXP^{T}
\end{equation}
where $P$ is the orthogonal matrix
\begin{equation}
P = \begin{bmatrix}1&0&0\\
0&\frac{I}{\sqrt{2}}&\frac{-I}{\sqrt{2}}\\
0&\frac{I}{\sqrt{2}}&\frac{I}{\sqrt{2}}
\end{bmatrix}
\end{equation}
where $I$ here is the $n\times n$ identity matrix.
The inverse to this morphism would be
\begin{equation}
\varphi^{-1}(Y)=P^{T}YP
\end{equation}
and we know the Cartan subalgebra is spanned by
$h_{i}=E_{i,i+1}-E_{i+1,i}$, by applying $\varphi^{-1}$ to it we
deduce that
\begin{equation}
\varphi^{-1}(h_{i})=E_{1+i,1+i}-E_{1+n+i,1+n+i}
\end{equation}
up to some change of coordinates by multiplication by $i$.


\begin{exercise}
Find a system $e_{i}$, $f_{j}$ of multiplicative generators of
$\ClassicalGroup{B}_{n}$ obeying
\begin{subequations}
\begin{align}
[e_{i},f_{j}]&=\delta_{ij}h_{i}\\
[h_{i},h_{j}]&=0\\
[h_{i},e_{j}]&=a_{ij}e_{j}\\
[h_{i},f_{j}]&=-a_{ij}f_{j}\\
({\rm ad}e_{i})^{1-a_{ij}}e_{j}&=0,\qquad i\not=j\\
({\rm ad}f_{i})^{1-a_{ij}}f_{j}&=0,\qquad i\not=j
\end{align}
\end{subequations}
for ``some'' matrix $a_{ij}$.
\end{exercise}
\answer We know that $\CC\lie\big(\ORTH{2n+1}\big)\cong \ClassicalGroup{B}_{n}$, and that
the basis for $\CC\lie\big(\ORTH{2n+1}\big)$ consists of $n(2n+1)$
antisymmetric matrices. We observe
\begin{equation}
\varphi^{-1}\left(
\begin{bmatrix}
0 & -\vec{u}^{T} & -\vec{v}^{T}\\
\vec{u} & 0 & 0\\
\vec{v} & 0 & 0
\end{bmatrix}
\right)
=
\frac{1}{\sqrt{2}}\begin{bmatrix}
0                   & -i(\vec{v}-\vec{u})^{T} & -(\vec{v}+\vec{u})^{T}\\
i(\vec{v}-\vec{u})  & 0                      &0\\
(\vec{v}+\vec{u})   & 0                      &0
\end{bmatrix}
\end{equation}
which permits us to deduce how these particular basis vectors
transform. The others are remarkably similar to $\ClassicalGroup{D}_{n}$, which
calculations have been performed or given.

If we let
\begin{equation}
H = \begin{bmatrix} 0&0&0\\
0&D&0\\
0&0&-D
\end{bmatrix}\in\frak{h}
\end{equation}
(for some diagonal $n\times n$ matrix $D$) and
\begin{equation}
L=\begin{bmatrix}0&-\vec{u}^{T}&-\vec{v}^{T}\\
\vec{u}&A&\frac{1}{2}(B-B^{T})\\
\vec{v}&\frac{1}{2}(C-C^{T})&-A^{T}
\end{bmatrix}\in B_{n}
\end{equation}
(for arbitrary $n\times n$ matrices $A$, $B$, $C$, and
$n$-vectors $\vec{u}$, $\vec{v}$) be arbitrary, then we find
\begin{equation}
[H,L]=\begin{bmatrix} 0 & -\vec{u}^{T}D & \vec{v}^{T}D\\
D\vec{u}   & [D,A]                     & \frac{1}{2}\{B-B^{T},D\}\\
-D\vec{v}  & -\frac{1}{2}\{C-C^{T},D\} & [D,A^{T}]
\end{bmatrix}
\end{equation}
where $\{a,b\}=ab+ba$ is the anticommutator. We see that in
addition to the basis root vectors given by $\ClassicalGroup{D}_{n}$, we have the
additional root vectors
\begin{equation}
\widetilde{e}_{i}=\begin{bmatrix} 0 & -\vec{u}^{T}_{i} & 0\\
\vec{u}_{i}   & 0 & 0\\
0  & 0 & 0
\end{bmatrix}
\end{equation}
 and
\begin{equation}
\widetilde{f}_{i}=\begin{bmatrix} 0 & 0 & -\vec{u}^{T}_{i}\\
0   & 0 & 0\\
\vec{u}_{i}  & 0 & 0
\end{bmatrix}
\end{equation}
 where $\{\vec{u}_{i}\}$ is the canonical basis
for $\RR^{n}$. We see that, if
\begin{equation}
h_{i}=\begin{bmatrix}
0 & 0     & 0\\
0 & E_{i,i}  & 0\\
0 & 0     & -E_{i,i}
\end{bmatrix}\in\frak{h}
\end{equation}
then
\begin{equation}
[h_{i},\widetilde{e}_{j}]=\delta_{ij}\widetilde{e}_{j}
\end{equation}
and
\begin{equation}
[h_{i},\widetilde{f}_{j}]=-\delta_{ij}\widetilde{f}_{j}
\end{equation}
which tell us the roots corresponding to $\widetilde{e}_{j}$ and $\widetilde{f}_{j}$.


%%  We first make a quick observation in general about weight
%% vectors.
%% 
%% \begin{prop}
%% When the Cartan subalgebra is presented as the span of a finite
%% set of diagonal matrices, the weight vectors in the fundamental
%% representation correspond to canonical basis vectors.
%% \end{prop}
%% \begin{proof}
%% It is obvious, since the weight vectors are eigenvectors of the
%% representation of the Cartan subalgebra elements. Since the
%% Cartan subalgebra elements are diagonal matrices, the
%% eigenvectors are naturally canonical basis vectors.
%% \end{proof}
%% 
%% This permits us to ``cheat'' and find representations for the
%% elements $e_{j}$ and $f_{k}$ which raise and lower the weight
%% vectors. 

\begin{exercise}
Describe the roots and root vectors of $\ClassicalGroup{A}_{n}$, $\ClassicalGroup{B}_{n}$, $\ClassicalGroup{C}_{n}$, $\ClassicalGroup{D}_{n}$.
\end{exercise}
\answer We find the roots and root vectors described by the previous exercises
for algebras $\ClassicalGroup{D}_{n}$, $\ClassicalGroup{C}_{n}$, and $\ClassicalGroup{B}_{n}$
respectively. We also examined the root system for $\ClassicalGroup{A}_{n}$ in a
previous homework.

We can describe the roots by examining the Cartan matrices of
these algebras. These are obtained by the commutation relations,
the coefficients $a_{ij}$ are the components of the Cartan
matrix. Additionally we have by definition $a_{ij}\leq 0$ for
non-diagonal components. With these conditions in mind, we can
write the Cartan matrices merely from the results we have already
computed. For $A_{n}$ we have
\begin{equation}
a_{ij} = \begin{bmatrix}
2  & -1 & 0  & \cdots & 0\\
-1 & 2  & -1 & \cdots & 0\\
0  & -1 & 2  & \cdots & 0\\
\vdots & \vdots & \vdots & \ddots & \vdots\\
0 & 0 & 0 & \cdots & 2
\end{bmatrix}.
\end{equation}
For $B_{n}$ we have
\begin{equation}
a_{ij} = \begin{bmatrix}
2   & -1 & 0  & 0 & \cdots & 0 & 0\\
-1  & 2  & -1 & 0 & \cdots & 0 & 0\\
0   & -1 &  2 &-1 & \cdots & 0 & 0\\
0   & 0  & -1 & 2 &  \cdots & 0 & 0\\
\vdots & \vdots & \vdots & \vdots & \ddots & \vdots & \vdots\\
0   & 0  & 0  & 0 &  \cdots & 2 & -2\\
0   & 0  & 0  & 0 &  \cdots & -1 & 2
\end{bmatrix}.
\end{equation}
For $C_{n}$ 
\begin{equation}
a_{ij} = \begin{bmatrix}
2   & -1 & 0  & 0 & \cdots & 0 & 0\\
-1  & 2  & -1 & 0 & \cdots & 0 & 0\\
0   & -1 &  2 &-1 & \cdots & 0 & 0\\
0   & 0  & -1 & 2 &  \cdots & 0 & 0\\
\vdots & \vdots & \vdots & \vdots & \ddots & \vdots & \vdots\\
0   & 0  & 0  & 0 &  \cdots & 2 & -1\\
0   & 0  & 0  & 0 &  \cdots & -2 & 2
\end{bmatrix}.
\end{equation}
For $D_{n}$
\begin{equation}
a_{ij} =
\begin{bmatrix}
2      & -1    & \cdots & 0      &  0     &   0    & 0\\
-1     &  2    & \cdots & 0      &  0     &   0    & 0\\
\vdots & \vdots& \ddots & \vdots & \vdots & \vdots & \vdots\\
0      & 0     & \cdots & 2      & -1     &   0    & 0\\
0      & 0     & \cdots & -1     &  2     &  -1    &-1\\  
0      &  0    & \cdots & 0      &  -1    &   2    & 0\\
0      &  0    & \cdots & 0      &  -1    &   0    & 2
\end{bmatrix}.
\end{equation}
\subsection{Additional Exercises}
\begin{exercise}
Let $V$ be the fundamental representation of $\frak{sl}(n)$. Let
$V^{*}$ be the representation dual to $V$. Find  the
decomposition of $V\otimes V$ and $V\otimes V^{*}$ into the
direct sum of irreducible representations.
\end{exercise}

\answer Well, we first see that if
\begin{equation}
\rho\colon\frak{sl}(n)\to \frak{gl}(V)
\end{equation}
is our representation morphism and
\begin{equation}
\rho^{*}\colon\frak{sl}(n)\to \frak{gl}(V^{*})
\end{equation}
is the dual representation, then we have
\begin{equation}
\rho\otimes\rho\colon\frak{sl}(n)\to \frak{gl}(V)\otimes \frak{gl}(V)
\end{equation}
and
\begin{equation}
\rho\otimes\rho^{*}\colon\frak{sl}(n)\to \frak{gl}(V)\otimes \frak{gl}(V^{*}).
\end{equation}
We know that we can decompose
\begin{equation}\label{eq:decomposition}
V\otimes V\iso {\rm Sym}^{2}(V)\oplus\Lambda^{2}(V)
\end{equation}
which intuitively corresponds to writing an arbitrary matrix as
the sum of an antisymmetric matrix and a symmetric matrix. We
would like to show that this decomposition corresponds to a
direct sum of irreps.

We first of all see that if
\begin{equation}
V={\rm span}\{u_{1},u_{2},...,u_{n}\}
\end{equation}
where $u_{1}$ is the highest weight vector, $e_{1}u_{2}=u_{1}$,
and so on, then we can write
\begin{equation}
V\otimes V={\rm Span}\{u_{i}\otimes u_{j}\mid i,j=1,...,n\}.
\end{equation}
However we note that the decomposition in eq \eqref{eq:decomposition}
amounts to
\begin{equation*}
V\otimes V=\underbracket[0.5pt]{{\rm Span}\left\{\frac{1}{2}\left(u_{i}\otimes
u_{j}+u_{j}\otimes u_{i}\right)\mid i,j=1,...,n\right\}}_{={\rm Sym}^{2}(V)}\oplus\underbracket[0.5pt]{{\rm
  Span}\left\{\frac{1}{2}\left(u_{i}\otimes u_{j}-u_{j}\otimes
u_{i}\right)\mid i\not=j,\quad i,j=1,...,n\right\}}_{=\Lambda^{2}(V)}.
\end{equation*}
To show that the representation of the group acting on $V\otimes
V$ is the direct sum of two irreducible representations, we will
show that the representation has exactly one highest weight
vector in ${\rm Sym}^{2}(V)$ and exactly one highest weight
vector in $\Lambda^{2}(V)$. We will also show that the group acts
on all basis vectors, which is sufficient to demand that the
representations is irreducible (thus implying $\rho\otimes\rho$ is
the direct sum of two irreps).
\vskip 4pt
\noindent\textbf{C\uppercase{{\footnotesize laim}} 1:}~ There is exactly one highest weight
vector in ${\rm Sym}^{2}(V)$.

\begin{proof}
We see that 
\begin{equation}
\rho\otimes\rho(h_{i})(u_{i}\otimes
u_{j})=(\lambda_{i}+\lambda_{j})(u_{i}\otimes u_{j})
\end{equation}
and thus
\begin{equation}
\rho\otimes\rho(h_{i})(u_{1}\otimes u_{1})=2\lambda_{1}(u_{1}\otimes u_{1}).
\end{equation}
We see that
\begin{equation}
\rho\otimes\rho(e_{i})(u_{1}\otimes u_{1})=0
\end{equation}
for all $i$, implying that $u_{1}\otimes u_{1}$ is a highest
weight vector. There are no others since
\begin{equation}
\rho\otimes\rho(e_{i})(u_{j}\otimes
u_{k})=\delta_{i,j}u_{j-1}\otimes u_{k}+\delta_{i,k}u_{j}\otimes
u_{k-1}
\end{equation}
up to some constant, which implies there are no other
possibilities for a highest weight vector.
\end{proof}


\vskip 4pt
\noindent\textbf{C\uppercase{{\footnotesize laim}} 2:}~ There is exactly one highest weight
vector in $\Lambda^{2}(V)$.

\begin{proof}
We see that similar reasoning holds. More explicitly
\begin{equation}
\rho\otimes\rho(e_{i})(u_{j}\otimes u_{k}-u_{k}\otimes u_{j})=
\left(\delta_{i,j}u_{j-1}\otimes u_{k}+\delta_{i,k}u_{j}\otimes
u_{k-1}\right)-
\left(\delta_{i,j}u_{k}\otimes u_{j-1}+\delta_{i,k}u_{k-1}\otimes
u_{j}\right)
\end{equation}
but only
\begin{equation}
\rho\otimes\rho(e_{i})(u_{1}\otimes u_{2}-u_{2}\otimes u_{1})=0
\end{equation}
for all $i$ identically. This is precisely uniqueness of a vector
that vanishes when acted upon by $e_{i}$ for any $i$, which is
the condition for the highest weight vector to be unique.
\end{proof}

\begin{lem}
If $\rho\colon G\to GL(V)$ is a representation of a
Lie group, its induced dual representation for the Lie Algebra is 
\begin{equation}
\rho^{*}(x):=-\rho(x)^{T}.
\end{equation}
\end{lem}
\begin{proof}
We observe that, from Lecture 15, the dual representation of a
Lie group is defined to be
\begin{equation}
\rho^{*}(g)=\rho(g^{-1})^{T}
\end{equation}
for any $g\in G$. If we take $g=1+\varepsilon X$ for an
``infinitesimal'' $\varepsilon$, then we can write
\begin{equation}
(1+\varepsilon X)^{-1}=1-\varepsilon X
\end{equation}
and
\begin{equation}
\rho^{*}(1+\varepsilon X)=\rho(1-\varepsilon X)^{T}.
\end{equation}
By Taylor expanding about 1, we deduce
\begin{equation}
\rho^{*}(1+\varepsilon X)=\rho(1)^{T}+\varepsilon\left[-\rho(X)^{T}\right]
\end{equation}
which permits us to induce a ``dual'' representation for the Lie
Algebra defined by
\begin{equation}
\rho^{*}(X)=-\rho(X)^{T}
\end{equation}
precisely as desired.
\end{proof}
\vskip 4pt

Since we are working with the fundamental representation $\rho$
of $\frak{sl}(n)$ we have
\begin{equation}
\rho({e_{i}})^{T}=\rho(f_{i}).
\end{equation}
Additionally, the Cartan subalgebra elements are such that
\begin{equation}
\rho(h_{i})^{T}=\rho(h_{i})
\end{equation}
since they are represented by diagonal matrices. So we have
\begin{equation}
(\rho\otimes\rho^{*})(e_{i})(u_{j}\otimes
  u^{k})=(\rho(e_{i})u_{j})\otimes u^{k}-u_{j}\otimes(\rho(f_{i})u^{k})
\end{equation}
where superscript indices here indicate that it is a covector, a
linear functional, an element of the dual vector space. We observe
\begin{equation}
\rho(f_{i})u_{i}=u_{i+1},\qquad\hbox{and}\quad\rho(e_{i})u_{i}=u_{i-1}
\end{equation}
since we noted in Lecture 16 the weight vectors for the
fundamental representation is nothing more than the canonical
basis. This permits us to deduce
\begin{equation}
(\rho\otimes\rho^{*})(e_{i})(u_{1}\otimes u^{n})=0
\end{equation}
for all $i$, which means $u_{1}\otimes u^{n}$ is a weight vector
for this representation. We also see that
\begin{equation}
(\rho\otimes\rho^{*})(e_{i})(u_{j}\otimes u^{j})=0
\end{equation}
identically, since
\begin{equation}
(\rho\otimes\rho^{*})(e_{i})(u_{j}\otimes u^{k})=(\lambda_{j}-\lambda_{k})(u_{j}\otimes u^{k}).
\end{equation}
Setting $j=k$ we find the right hand side vanishes \emph{identically}.

\begin{exercise}
Prove the fundamental represenation of $\frak{so}(n)$ is
equivalent to dual representation. Find the decomposition of the
tensor square of this representation into direct sum of
irreducible representations.
\end{exercise}

\noindent\textbf{C\uppercase{{\footnotesize laim}} 1:}~
The fundamental representation of $\frak{so}(n)$ is equivalent to
its dual representation.

\begin{proof}
Well, we know the dual representation for $\rho\colon G\to GL(V)$
is
\begin{equation}
\rho^{*}(g)=\rho(g^{-1})^{T}.
\end{equation}
By making $g=1+\varepsilon X$ where $\varepsilon$ is
``infinitesimal,'' we get by Taylor expanding about 1 
\begin{equation}
\rho(1-\varepsilon X)^{T}=\rho(1)+\varepsilon\left(-\rho(X)^{T}\right)
\end{equation}
which permits us to deduce for a Lie algebra, the dual
representation for an element $X$ is
\begin{equation}
\rho^{*}(X)=-\rho(X)^{T}.
\end{equation}
We know for the fundamental representation of $\frak{so}(n)$ we
are working with $n\times n$ antisymmetric matrices. However,
this means that for the fundamental representation $\rho$ of $\frak{so}(n)$
\begin{equation}
\rho(X)^{T}=-\rho(X)
\end{equation}
for all $X\in\frak{so}(n)$, which implies
\begin{equation}
\rho(X)=-\rho(X)^{T}=\rho^{*}(X)
\end{equation}
precisely as desired.
\end{proof}

\begin{prop}
The Cartan subalgebra is
\begin{equation}\label{eq:sonCartan}
\frak{h}={\rm span}\{\sum_{j=1}^{n}x_{j}(E_{2j,2j-1}-E_{2j-1,2j})\}
\end{equation}
and the weight vectors are, for the fundamental representation
\begin{equation}
\vec{u}_{j}=\vec{e}_{2j}+i\vec{e}_{2j-1}
\end{equation}
for $j=1,...,n$.
\end{prop}
\begin{proof}
It follows from the definition of the maximal torus that eq \eqref{eq:sonCartan}
 holds. The eigenvalues for the $2\times2$ matrix
\begin{equation}
\begin{bmatrix}0&-1\\1&0
\end{bmatrix}
\end{equation}
is $\pm1$ with eigenvectors $(i,1)$ and $(-i,1)$ for $-i$, $i$
respectively. Thus we deduce the weight vectors to be as desired.
\end{proof}

\begin{prop}
We deduce that 
\begin{equation}
[h_{j},\lambda^{k}(E_{k,2j}-E_{2j,k})+\mu^{\ell}(E_{\ell,2j-1}-E_{2j-1,\ell})]=-\lambda^{k}(E_{k,2j-1}-E_{2j-1,k})+\mu^{\ell}(E_{\ell,2j}-E_{2j,\ell}).
\end{equation}
\end{prop}
\begin{proof}
We find this by direct calculation.
\end{proof}

\vskip 1pc
We can further compute how these terms act on the weight
vectors. We observe
\begin{subequations}
\begin{align}
\lambda^{k}(E_{k,2j}-E_{2j,k})\vec{u}_{j}&=\lambda^{k}\vec{e}_{k}-i\lambda^{k}\delta_{k,2j-1}\vec{e}_{2j}\\
\mu^{\ell}(E_{\ell,2j-1}-E_{2j-1,\ell})\vec{u}_{j}&=i\mu^{\ell}\vec{e}_{\ell}-\mu^{\ell}\delta_{\ell,2j}\vec{e}_{\ell}.
\end{align}
\end{subequations}
Thus we observe that when $\ell=2j+1$ and $k=2j+2$, we get
\begin{equation}
\lambda^{k}(E_{k,2j}-E_{2j,k})\vec{u}_{j}+\mu^{\ell}(E_{\ell,2j-1}-E_{2j-1,\ell})\vec{u}_{j}=\vec{u}_{j+1}.
\end{equation}
We also observe that when $\ell=2j-3$ and $k=2j-2$, we get
\begin{equation}
\lambda^{k}(E_{k,2j}-E_{2j,k})\vec{u}_{j}+\mu^{\ell}(E_{\ell,2j-1}-E_{2j-1,\ell})\vec{u}_{j}=\vec{u}_{j-1}.
\end{equation}
Thus we have found our raising and lowering operators $e_{j}$ and
$f_{j}$ respectively. 

But notice since the representation is ``self-dual'', we have the
decomposition of
\begin{equation}
V\otimes V={\rm Sym}^{2}(V)\oplus\Lambda^{2}(V).
\end{equation}
Thus we have two irreps, one acting on ${\rm Sym}^{2}(V)$ and the
other acting on $\Lambda^{2}(V)$. We can observe the following
thing: a symmetric matrix $X$ is such that
\begin{equation}
\tr(PXP^{T})=\tr(X)
\end{equation}
for $P\in \ORTH{n}$. So, we can consider traceless symmetric matrices
$X$ which forms an invariant subspace of ${\rm
  Sym}^{2}(V)$. Additionally, matrices of the form $cI$ where
$c\in\Bbb{R}$ is some constant, also obey
\begin{equation}
P(cI)P^{T}=c(PIP^{T})=c(PP^{T})=cI
\end{equation}
for $P\in O(n)$. Note these are group elements and the group
representation acting on $V\otimes V$, which permit us to find
invariant subspaces and thus subrepresentations for the group
which correspond to subrepresentations of the Lie algebra.

So the invariant subspaces are three:
\begin{enumerate}
\item $\Lambda^{2}V$ the antisymmetric part;
\item ${\rm Span}\{I\}$ the scalar matrices; and
\item $\{X\in{\rm Sym}^{2}V|\tr(X)=0\}$ traceless matrices.
\end{enumerate}
Each of these are irreducible, since there is precisely one
highest weight vector for each of them.
\begin{exercise}
  (Schur's lemma) Let us consider a complex irreducible
  representation $\varphi$ of Lie algebra $\mathscr{G}$. Let us
  assume that the operator $A$ in the representation space
  commutes with all operators of the form $\varphi(x)$ where
  $x\in\mathscr{G}$. Prove that $A = \lambda\cdot 1$ where
  $\lambda\in\Bbb{C}$ and $1$ stands for the identity operator.


  Hint: Consider $\ker(A -\mu\cdot1)$.
\end{exercise}

\answer 
We will first make a small lemma.

\begin{lem}\label{lem:invertibility}
Let $\varphi\colon \mathscr{G}\to V$ be an irreducible
representation of the Lie algebra $\mathscr{G}$, and $L\colon
V\to V$ be a linear mapping. Then either $L$ is an isomorphism,
or it is zero.
\end{lem}
\begin{proof}
Consider $\ker(L)$. It would be an invariant subspace
of $V$, but since $\varphi$ is irreducible this subspace is
either $V$ or $0$. For nonzero $L$, we have $\ker(L)=0$. This
implies that $L$ is injective. But since we have an injective
endomorphism, thus it is surjective and moreover bijective. We have
$L$ be an isomorphism or, alternatively, 0.
\end{proof}

Consider $A$. It has at least 1 nonzero eigenvalue $\lambda$, or
it is the zero mapping necessarily (in which case, $A=0\cdot
I$). We see that $A-\lambda\cdot1$ is not an isomorphism, by our
lemma since its kernel is all of $V$ it follows that
$A=\lambda\cdot1$ for some $\lambda\in\Bbb{C}$.

\begin{exercise}
Let us consider a Lie algebra $\mathscr{G}$ with basis $e_1$,
$...$, $e_n$ and structure constants ${c^{k}}_{ij}$:
\begin{equation}
  [e_i , e_j ] = {c^{k}}_{ij} e_k.
\end{equation}
We define universal enveloping algebra $U(\mathscr{G})$ as a
unital associative algebra with generators $e_i$ and relations
\begin{equation}    
e_i e_j - e_j e_i = {c^{k}}_{ij} e_k.
\end{equation}
Let us assume that there exists an invariant inner product on $\mathscr{G}$
and that the basis $e_1$, $...$, $e_n$ is orthonormal with
respect to this product. Prove that that the element 
\begin{equation}
\omega=\sum_{i}  e_i e_i
\end{equation}
(Casimir element) belongs to the center of enveloping algebra.

Hint. It is sufficient to check that Casimir element commutes
with all generators. Write the condition for $e_k \omega = \omega
e_k$ in terms of structure constants and check that the same
condition guarantees invariance of inner product. 
\end{exercise}

\answer
Let $B$ be a nondegenerate symmetric invariant bilinear form on
$\mathscr{G}$. Let $x_{1}$, ..., $x_{n}$ be the basis for
$\mathscr{G}$, $x^{1}$, ..., $x^{n}$ be the dual basis so 
\begin{equation}
B(x_{i},x^{j})={\delta_{i}}^{j}.
\end{equation}
Let $z\in\mathscr{G}$, we have
\begin{equation}
[z,x_{j}]=\sum_{i}{a^{i}}_{j}x_{i}={a^{i}}_{j}x_{i}
\end{equation}
and
\begin{equation}
[z,x^{j}]=\sum_{i}{b_{i}}^{j}x^{i}={b_{i}}^{j}x^{i}
\end{equation}
where ${a_{i}}^{j}$ and ${b_{i}}^{j}$ are ``some constants.'' Since $B$ is
invariant we have
\begin{equation}
0=B([z,x_{i}],x^{j})+B(x_{i},[z,x^{j}])={a_{i}}^{j}+{b_{i}}^{j}.
\end{equation}
We see
\begin{subequations}
\begin{align}
z(x_{i}x^{i})&=[z,x_{i}]x^{i}+x_{i}zx^{i}\\
&=\sum_{j}{a_{i}}^{j}x_jx^{i}+x_{i}zx^{i}
\end{align}
\end{subequations}
and
\begin{equation}
(x_{i}x^{i})z=(x_{i}[x^{i},z])+x_{i}zx^{i}=-\sum_{j}{b_{i}}^{j}x_{i}x^{j}+x_{i}zx^{i}
\end{equation}
Thus we find
\begin{equation}
[z,x_{i}x^{i}]=\sum_{j}{a_{i}}^{j}x_{j}x^{i}+{b^{i}}_{j}x_{i}x^{j}=\sum_{j}({a_{j}}^{i}+{b^{i}}_{j})x_{i}x^{j}
\end{equation}
but due to the invariance of $B$, we see that the parenthetic
term must be zero (we sum over dummy indices, which we may
rewrite as we like). This implies
\begin{equation}
[z,x_{i}x^{i}]=0
\end{equation}
for any $z\in\mathscr{G}$.

\begin{exercise}
A representation $\varphi\colon\mathscr{G}\to\mathfrak{gl}(n)$
can be extended to a homomorphism of universal enveloping algebra
to the algebra of $n\times n$ matrices. If the representation is
irreducible the image of the Casimir element has the form
$\lambda\cdot1$. Show that this follows from Schur's lemma.
\end{exercise}

\answer
We have that the Casimir element be denoted by $\omega$. In the
universal enveloping algebra, it is an $n\times n$ matrix. It has
at least one nonzero eigenvalue $\lambda\in\CC$ or it is the
zero matrix. By Lemma \ref{lem:invertibility} we have for
$\omega-\lambda\cdot1$ its kernel is either 0 or all of
$\CC^{n}$. If 
\begin{equation}
\ker(\omega-\lambda\cdot1)=\CC^{n}
\end{equation}
then 
\begin{equation}
\omega=\lambda\cdot1
\end{equation}
by Schur's lemma. Otherwise it is the zero matrix.

\begin{exercise}
Calculate the image of Casimir element for an irreducible
representation of the Lie algebra $\frak{sl}(2)$. 
\end{exercise}
\answer

Let
\begin{equation}\label{eq:sl2irrep}
\rho\colon \frak{sl}(2)\to\frak{gl}(V)
\end{equation}
be an irrep.
We have the dual basis for $e$, $f$, $h$ in $\frak{sl}(2)$ be
$2f$, $2e$ and $h$ respectively. By our previous calculations, it
follows that
\begin{equation}
\omega=H^{2}+2EF+2FE
\end{equation}
where $F=\rho(f)$, $E=\rho(e)$, $H=\rho(h)$. 

Observe if $\vec{v}$ is the highest weight vector for the irrep
in eq \eqref{eq:sl2irrep}, and $\lambda$ is the corresponding
highest weight, we have
\begin{subequations}
\begin{align}
\rho(h^{2})\vec{v} &=\rho(h)^{2}\vec{v}\\
&=\lambda(h)^{2}\vec{v}.
\end{align}
\end{subequations}
Similarly, we see
\begin{subequations}
\begin{align}
\rho(ef+fe)\vec{v}&=(\rho(e)\rho(f)+\rho(f)\rho(e))\vec{v}\\
&=\rho(e)\rho(f)\vec{v}+\rho(f)\left(\rho(e)\vec{v}\right)\\
&=\rho(e)\rho(f)\vec{v}+0\\
&=\rho(e)\rho(f)\vec{v}-0\\
&=\rho(e)\rho(f)\vec{v}-\rho(f)\left(\rho(e)\vec{v}\right)\\
&=(\rho(e)\rho(f)-\rho(f)\rho(e))\vec{v}\\
&=[\rho(e),\rho(f)]\vec{v}\\
&=\rho(h)\vec{v}.
\end{align}
\end{subequations}
Thus the Casimir acting on the highest weight vector is
\begin{equation}
\omega\vec{v}=(\rho(h)^{2}+\rho(h))\vec{v}=\lambda(h)\left(1+\lambda(h)\right)\vec{v}.
\end{equation}

