\subsection*{Please Take Note!}

We will be covering everything relevant here, and when the time comes we will
be performing in excrutiating detail every Feynman diagram of significance in
QED.

\section{Maxwell's Equations in a Nutshell}

Recall in classical electromagnetism we have it summed in Maxwell's equations~\cite{jackson}. In the presence of a charge density $\rho(\vec{x},t)$ and a 
current density $\vec{j}(\vec{x},t)$, the electric and magnetic fields $\vec{E}$
and $\vec{B}$ satisfy the equations
\begin{subequations}\label{maxwellsEquations}
\begin{align}
\nabla\cdot\vec{E}&=\rho\label{gaussLaw}\\
\nabla\times\vec{B}&=\frac{1}{c}\vec{j} + \frac{1}{c}\frac{\partial\vec{E}}{\partial t}\label{AmpereLaw}\\
\nabla\cdot\vec{B} &= 0\label{GaussLawMagnet}\\
\nabla\times\vec{E} &= -\frac{1}{c}\frac{\partial\vec{B}}{\partial t} \label{FaradayLaw}
\end{align}
\end{subequations}
where cgs units are used.

In the second pair of equations (Eqs \ref{GaussLawMagnet} and \ref{FaradayLaw})
follows the existence of scalar and vector potentials $\phi(\vec{x},t)$ and
$\vec{A}(\vec{x},t)$ defined by
\begin{equation}
\vec{B} = \nabla\times\vec{A},\quad\vec{E}=-\nabla\phi - \frac{1}{c}\frac{\partial\vec{A}}{\partial t}.
\end{equation}
However, this does not determine the system uniquely, since for an \emph{arbitrary}
function $f(\vec{x},t)$ the transformation
\begin{equation}\label{emGaugeTransformation}
\phi\to\phi'=\phi + \frac{1}{c}\frac{\partial f}{\partial t},\quad\vec{A}\to\vec{A}' = \vec{A} - \nabla f
\end{equation}
leaves the fields $\vec{E}$ and $\vec{B}$ unaltered. The transformation (\ref{emGaugeTransformation})
is known as a gauge transformation of the second kind\footnote{I.e. it is described
mathematically in differential geometry as a connection form.}. Since all 
observable quantities can be expressed in terms of 
$\vec{E}$ and $\vec{B}$, it is a fundamental requirement of any theory
formulated in terms of potentials that is gauge; i.e. the predictions for the 
observable quantities are invariant under such gauge transformations.

When we express Maxwell's equations in terms of potentials, the second pair
are automatically satisfied. The first pair (\ref{gaussLaw} and \ref{AmpereLaw}) become
\begin{subequations}
\begin{align}
-\nabla^2\phi - \frac{1}{c}\frac{\partial}{\partial t}(\nabla\cdot\vec{A}) = \Box\phi - \frac{1}{c}\frac{\partial}{\partial t}\left(\frac{1}{c}\frac{\partial\phi}{\partial t} + \nabla\cdot\vec{A}\right) = \rho\\
\Box\vec{A} + \nabla\left(\frac{1}{c}\frac{\partial\phi}{\partial t} + \nabla\cdot\vec{A}\right) = \frac{1}{c}\vec{j}
\end{align}
\end{subequations}
where
\begin{equation}
\Box \equiv \frac{1}{c^2}\frac{\partial^2}{\partial t^2} - \nabla^2
\end{equation}
is called the ``D'Alembertian''.

We can now consider the so-called ``free field'' case. That is, we have no
charge or current so $\rho=0$ and $\vec{j}=0$. We can choose a gauge for the
system such that
\begin{equation}\label{radiationGauge}
\nabla\cdot\vec{A} = 0.
\end{equation}
The condition (\ref{radiationGauge}) defines the \textbf{Coulomb or radiation gauge}. A vector field with vanishing divergence (ie satisfying Eq (\ref{radiationGauge})) is called a ``transverse field'' since for a wave
\begin{equation}
\vec{A}(\vec{x},t) = \vec{A}_0 \exp(i(\vec{k}\cdot\vec{x}-\omega t))
\end{equation}
gives
\begin{equation}
\vec{k}\cdot\vec{A} = 0,
\end{equation}
or in other words $\vec{A}$ is perpendicular to the direction of propagation 
$\vec{k}$ of the wave. In the Coulomb gauge, the vector potential is a transverse
vector.


\section{A Comically Brief Review}
The following is a table summarizing the properties of the solutions of the 
Dirac equation.

\begin{tabular}{|p{3cm}|c|c|}
\hline
Property & Electrons & Positrons\\ \hline
Spinor components & $\psi(x)=au^{(s)}(p)\exp[-(i/\hbar)p\cdot x]$ & $\psi(x)=av^{(s)}(p)\exp[-(i/\hbar)p\cdot x]$\\ \hline
Momentum Space Dirac Equation & $(\gamma^\mu p_\mu - mc)u = 0$ & $(\gamma^\mu p_\mu + mc)v = 0$ \\ \hline
Adjoint Dirac Equation & $\bar{u}(\gamma^\mu p_\mu - mc) = 0$ & $\bar{v}(\gamma^\mu p_\mu + mc) = 0$ \\ \hline
Orthogonality & $\bar{u}^{(1)}u^{(2)} = 0$ & $\bar{v}^{(1)}v^{(2)} = 0$ \\ \hline
Normalization & $\bar{u}u = 2mc$ & $\bar{v}v = -2mc$ \\ \hline
Complete & $\sum_{s} u^{(s)}\bar{u}^{(s)} = (\gamma^\mu p_\mu + mc)$ & $\sum_{s} v^{(s)}\bar{v}^{(s)} = (\gamma^\mu p_\mu - mc)$ \\ \hline
\end{tabular}

A free photon, on the other hand, of momentum $p = (E/c, \bold{p})$ with
$E = |\bold{p}|c$ is represented by the wave function
\begin{equation}
A^{\mu}(x) = ae^{-(i/\hbar)p\cdot x}\epsilon^\mu_{(s)}
\end{equation}
where $\epsilon^\mu$ is a spin dependent vector, $s=1,2$ for the two polarizations
(``spin states'') of the photon. The polarization vectors $\epsilon^\mu_{(s)}$
satisfy the \emph{momentum space Lorentz condition:}
\begin{equation}
\epsilon^\mu p_\mu = 0.
\end{equation}
 They are orthogonal in the sense that 
\begin{equation}
\epsilon_{\mu(1)}^{*}\epsilon^{\mu}_{(2)} = 0.
\end{equation}
They are further normalized
\begin{equation}
\epsilon^{*}_{\mu}\epsilon^{\mu} = 1.
\end{equation}
In the Coulomb gauge
\begin{equation}
\epsilon^0=0,\quad \epsilon\cdot p = 0
\end{equation}
and the polarization three-vectors obey the completeness relation
\begin{equation}
\sum_{s=1,2} (\epsilon_{(s)})_i (\epsilon^{*}_{(s)})_j = \delta_{ij} - \hat{p}_{i}\hat{p}_{j}.
\end{equation}

\section{Quantum Electrodynamics}
\subsection{The Rules to the Game}

So we want to calculate out the probability amplitude $\mathcal{M}$ associated
with a particular Feynman diagram, we proceed as follows:
\begin{enumerate}
\item{(Notation)} We must be more careful here! We label the incoming and 
outgoing four-momenta $p_1$, $p_2$, \ldots, $p_n$ and the corresponding spins
$s_1$, $s_2$, \ldots, $s_n$. We label the inernal four-momenta $q_1$, $q_2$, \ldots.
Assign arrows to the lines as follows: the arrows on \emph{external} fermion lines 
indicates whether it is an electron or positron (if the arrow points forward in 
time, it is an electron; backwards in time it is a positron); arrows on
\emph{internal} fermion lines are assigned so that the ``direction of the flow''
through the diagram is preserved (i.e. every vertex must have at least one arrow
entering and one arrow leaving). The arrows on photon lines (which is optional,
since arrows are used to indicate whether the particle is an antiparticle or not;
bosons are their own antipartners) point ``forward'' in time.

\item{(External Lines)} External lines contribute factors as follows:
\begin{equation*}
\mbox{Electrons: } \begin{cases} \mbox{Incoming: } u\\
\mbox{Outgoing: }\bar{u} \end{cases}
\end{equation*}
\begin{equation*}
\mbox{Positrons: } \begin{cases} \mbox{Incoming: } \bar{v}\\
\mbox{Outgoing: }v\end{cases}
\end{equation*}
\begin{equation*}
\mbox{Photons: } \begin{cases} \mbox{Incoming: }\epsilon^\mu \\
\mbox{Outgoing: }(\epsilon^\mu)^* \end{cases}
\end{equation*}

\item{(Vertex Factors)} Each vertex contributes a factor
\begin{equation}
ig_{e}\gamma^\mu
\end{equation}
The dimensionless coupling constant $g_e$ is related to the charge of the
positron $g_e$ = $e\sqrt{4\pi/\hbar c}$ = $\sqrt{4\pi\alpha_E}$\footnote{Here
$\alpha_E$ is the coupling constant of the electromagnetic force. In \emph{general},
the QED coupling is $-q\sqrt{4\pi/\hbar c}$ where $q$ is the charge of the \emph{particle}
(as opposed to antiparticle). For electrons $q=-e$, for an up quark $q=(2/3)e$.}
\item{(Propagators)} Each internal line contributes a factor as follows
\begin{equation}
\mbox{Electrons and Positrons: }\frac{i\gamma^\mu q_\mu + mc}{q^2 - m^2c^2}
\end{equation}
\begin{equation}
\mbox{Photons: }\frac{-ig_{\mu\nu}}{q^2}
\end{equation}
\item{(Conservation of Energy and Momentum)} For each vertex, write a delta 
function of the form
\begin{equation}
(2\pi)^4\delta^{(4)}(k_1 + k_2 + k_3)
\end{equation}
This enforces the conservation of momentum at the vertex.
\item{(Integrate Over Internal Momenta)} For each internal momentum $q$, write
a factor
\begin{equation}
\frac{d^4 q}{(2\pi)^4}
\end{equation}
and integrate.
\item{(Cancel the Delta Function)} The result will include a factor
\begin{equation}
(2\pi)^4\delta^{(4)}(p_1 + p_2 + \cdots - p_n)
\end{equation}
which corresponds to the overall energy-momentum conservation. Cancel this factor,
and we get $-i\mathcal{M}$.
\item{(Antisymmetrization)} Include a minus sign between diagrams that differ
only in the interchange of two incoming (or outgoing) electrons (or positrons),
or of an incoming electron with an outgoing positron (or vice versa).
\end{enumerate}

\section{Elastic Processes}

An elastic (relativistic) process is one where kinetic energy, rest energy, and
mass are all conserved. We will explore such examples in QED.

\subsection{Electron-Muon Scattering}

We draw the diagram (note the use of $\mu$ and $\nu$ at the vertices, which are 
used to sum over in the integral):

\strut

\begin{center}
\begin{fmffile}{QEDexOneImg1}
  \begin{fmfgraph*}(50,25) \fmfpen{0.2mm}
    \fmfset{arrow_len}{3mm}\fmfset{arrow_ang}{10}
    \fmfleft{i1,o1} % change i2->o1 
    \fmfright{i2,o2} % change o1->i2
    \fmflabel{$p_{1},s_{1}$}{i1}
    \fmflabel{$p_{3},s_{3}$}{o1} %
    \fmflabel{$p_{2},s_{2}$}{i2} %
    \fmflabel{$p_{4},s_{4}$}{o2}
    \fmflabel{$\mu$}{v1}
    \fmflabel{$\nu$}{v2}
    \fmf{fermion}{i1,v1,o1} %
    \fmf{dbl_plain_arrow}{i2,v2,o2} %
    \fmf{boson,label=$q$}{v1,v2}
  \end{fmfgraph*}
\end{fmffile}
\end{center}
\strut

We will now evaluate it in a haphazard manner. Observe how it is
done when spinors are in the game.

\textbf{Step One:} We will evaluate the part emboldened in Red
first.


\strut
\begin{center}
\begin{fmffile}{QEDexOneImg2}
  \begin{fmfgraph*}(50,25)  \fmfpen{0.2mm}
    \fmfset{arrow_len}{3mm}\fmfset{arrow_ang}{10}
    \fmfleft{i1,o1} % change i2->o1 
    \fmfright{i2,o2} % change o1->i2
    \fmflabel{$p_{1},s_{1}$}{i1}
    \fmflabel{$p_{3},s_{3}$}{o1} %
    \fmflabel{$p_{2},s_{2}$}{i2} %
    \fmflabel{$p_{4},s_{4}$}{o2}
    \fmflabel{$\mu$}{v1}
    \fmflabel{$\nu$}{v2}
    \fmf{fermion,fore=red}{i1,v1,o1} %
    \fmf{dbl_plain_arrow}{i2,v2,o2} %
    \fmf{boson,label=$q$}{v1,v2}
  \end{fmfgraph*}
\end{fmffile}
\end{center}
\strut

We will now analyze it in careful detail so we will ``pull it
out'' and ``dissect'' it carefully.

We evaluate it in the following manner: since we write quantum mechanics like
we write chinese (from right to left), we begin with 
\begin{equation}
\Diagram{\vertexlabel^{p_3,s_3}\\
fd \\
& g\vertexlabel_{\mu,q}\\
\vertexlabel_{p_1,s_1} {\color{red}fu}\\
} = u(s_1,p_1),
\qquad
\Diagram{\vertexlabel^{p_3,s_3}\\
fd \\
& {\color{red}g}\vertexlabel_{\mu,q}\\
\vertexlabel_{p_1,s_1} fu\\
} = (ig_{e}\gamma^{\mu})u(s_1,p_1)
\end{equation}
We have one last step to do
\begin{equation}
\Diagram{\vertexlabel^{p_3,s_3}\\
{\color{red}fd} \\
  & g\vertexlabel_{\mu,q} \\
\vertexlabel_{p_1,s_1} fu\\
} = \bar{u}(s_3,p_3)(ig_{e}\gamma^{\mu})u(s_1,p_1)
\end{equation}
So this contributes
\begin{equation}\label{exOneElectronTerm}
(\bar{u}(s_3,p_3))(ig_{e}\gamma^\mu)(u(s_1, p_1))
\end{equation}
to the integrand. Our integrand is going to take the form
\begin{equation}
[(\bar{u}(s_3,p_3))(ig_{e}\gamma^\mu)(u(s_1, p_1))]\begin{pmatrix} $photon$\\$propagator$\end{pmatrix}\begin{pmatrix}$muon$\\$terms$\end{pmatrix} \begin{pmatrix}$conservation$\ $of$\\
$momentum$\ $delta$\\
$functions$\end{pmatrix}
\end{equation}
We will now move on to step two.

\textbf{Step Two:} We will consider the photon propagator, which corresponds to
the red line in the following diagram


\strut
\begin{center}
\begin{fmffile}{QEDexOneImg3}
  \begin{fmfgraph*}(50,25)  \fmfpen{0.2mm}
    \fmfset{arrow_len}{3mm}\fmfset{arrow_ang}{10}
    \fmfleft{i1,o1} % change i2->o1 
    \fmfright{i2,o2} % change o1->i2
    \fmflabel{$p_{1},s_{1}$}{i1}
    \fmflabel{$p_{3},s_{3}$}{o1} %
    \fmflabel{$p_{2},s_{2}$}{i2} %
    \fmflabel{$p_{4},s_{4}$}{o2}
    \fmflabel{$\mu$}{v1}
    \fmflabel{$\nu$}{v2}
    \fmf{fermion}{i1,v1,o1} %
    \fmf{dbl_plain_arrow}{i2,v2,o2} %
    \fmf{boson,label=$q$,fore=red}{v1,v2}
  \end{fmfgraph*}
\end{fmffile}
\end{center}
\strut


This corresponds to the term
\begin{equation}
\frac{-ig_{\mu\nu}}{q^2}
\end{equation}
giving our integrand to be
\begin{equation}
[(\bar{u}(s_3,p_3))(ig_{e}\gamma^\mu)(u(s_1, p_1))]\frac{-ig_{\mu\nu}}{q^2}\begin{pmatrix}$muon$\\$terms$\end{pmatrix} \begin{pmatrix}$conservation$\ $of$\\
$momentum$\ $delta$\\
$functions$\end{pmatrix}.
\end{equation}

\textbf{Step Three and Four:} Moving right along to the Muon terms, we have exactly a term
analagous to Eq (\ref{exOneElectronTerm}). Muons are fermions with spin 1/2,
with the same electric charge as an electron. So translating this into Feynman
diagram terms, it is translated in the exact same fashion we translated the
electron terms. So we have a contribution of
\begin{equation}
(\bar{u}(s_4,p_4))(ig_{e}\gamma^\nu)(u(s_2, p_2)).
\end{equation}
Our integrand now becomes
\begin{equation}
[(\bar{u}(s_3,p_3))(ig_{e}\gamma^\mu)(u(s_1, p_1))]\frac{-ig_{\mu\nu}}{q^2}[(\bar{u}(s_4,p_4))(ig_{e}\gamma^\nu)(u(s_2, p_2))] \begin{pmatrix}$conservation$\ $of$\\
$momentum$\ $delta$\\
$functions$\end{pmatrix}.
\end{equation}

\textbf{Step Five:} We kind of ``fudged up'' steps 1 through 4 because they are
so interconnected it is hard to seperate them out from each other. We are now
safely onto step 5 of the Feynman rules of QED: conservation of momentum! We
have two places to do this (at the $\mu$ and $\nu$ vertices). We have for $\mu$
(chosen randomly) the input momentum in red and output momentum in blue:


\strut
\begin{center}
\begin{fmffile}{QEDexOneImg4}
  \begin{fmfgraph*}(50,25)  \fmfpen{0.2mm}
    \fmfset{arrow_len}{3mm}\fmfset{arrow_ang}{10}
    \fmfleft{i1,o1} % change i2->o1 
    \fmfright{i2,o2} % change o1->i2
    \fmflabel{$p_{1},s_{1}$}{i1}
    \fmflabel{$p_{3},s_{3}$}{o1} %
    \fmflabel{$p_{2},s_{2}$}{i2} %
    \fmflabel{$p_{4},s_{4}$}{o2}
    \fmflabel{$\mu$}{v1}
    \fmflabel{$\nu$}{v2}
    \fmf{fermion,fore=red}{i1,v1} %
    \fmf{fermion,fore=blue}{v1,o1}
    \fmf{dbl_plain_arrow}{i2,v2,o2} %
    \fmf{boson,label=$q$,fore=blue}{v1,v2}
  \end{fmfgraph*}
\end{fmffile}
\end{center}
\strut


This corresponds to the conservation of momentum
\begin{equation}
p_1 = p_3 + q \quad\Rightarrow\quad p_1 - p_3 - q = 0
\end{equation}
which gives us the delta function
\begin{equation}\label{exOneTermToTakeAdvantageOf}
(2\pi)^{4}\delta^{(4)}(p_1 - p_3 - q).
\end{equation}
We have another conservation of momentum point, which is at the
vertex $\nu$:



\strut
\begin{center}
\begin{fmffile}{QEDexOneImg5}
  \begin{fmfgraph*}(50,25)  \fmfpen{0.2mm}
    \fmfset{arrow_len}{3mm}\fmfset{arrow_ang}{10}
    \fmfleft{i1,o1} % change i2->o1 
    \fmfright{i2,o2} % change o1->i2
    \fmflabel{$p_{1},s_{1}$}{i1}
    \fmflabel{$p_{3},s_{3}$}{o1} %
    \fmflabel{$p_{2},s_{2}$}{i2} %
    \fmflabel{$p_{4},s_{4}$}{o2}
    \fmflabel{$\mu$}{v1}
    \fmflabel{$\nu$}{v2}
    \fmf{fermion}{i1,v1} %
    \fmf{fermion}{v1,o1}
    \fmf{dbl_plain_arrow,fore=red}{i2,v2} %
    \fmf{dbl_plain_arrow,fore=blue}{v2,o2} %
    \fmf{boson,label=$q$,fore=red}{v1,v2}
  \end{fmfgraph*}
\end{fmffile}
\end{center}
\strut


Which corresponds to a conservation of momentum 
\begin{equation}
p_2 + q = p_4\quad\Rightarrow\quad p_2 + q - p_4 = 0
\end{equation}
and thus contributes the delta function term
\begin{equation}
(2\pi)^4 \delta^{(4)}(p_2 + q - p_4)
\end{equation}
rendering our integrand to be
\begin{eqnarray}
\quad&&[(\bar{u}(s_3,p_3))(ig_{e}\gamma^\mu)(u(s_1, p_1))]\frac{-ig_{\mu\nu}}{q^2}[(\bar{u}(s_4,p_4))(ig_{e}\gamma^\nu)(u(s_2, p_2))] (2\pi)^{8}\nonumber\\
& &\times \delta^{(4)}(p_1 - p_3 - q)\delta^{(4)}(p_2 + q - p_4) d^4q. \nonumber
\end{eqnarray}

\textbf{Step Six:} We integrate over the internal momenta (in our case the photon's momentum), so we have the integral expression:
\begin{eqnarray}
i\mathcal{M} &\textrm{``=''}& (2\pi)^{4} \int [(\bar{u}(s_3,p_3))(ig_{e}\gamma^\mu)(u(s_1, p_1))]\frac{-ig_{\mu\nu}}{q^2}[(\bar{u}(s_4,p_4))(ig_{e}\gamma^\nu)(u(s_2, p_2))] \nonumber\\
& &\times \delta^{(4)}(p_1 - p_3 - q)\delta^{(4)}(p_2 + q - p_4) d^4q. \nonumber
\end{eqnarray}
Observe this is harder than it looks because we are taking the trace of gamma
matrices! That is the whole point of having the metric tensor $g^{\mu\nu}$ here.
So it is a bit tricky to compute...

We will integrate over $q$ and take advantage of the delta function term (\ref{exOneTermToTakeAdvantageOf})
to make the switch
\begin{equation*}
q\to p_1 - p_3
\end{equation*}
giving us the result from the integral
\begin{equation}\label{exOneIntegralResult}
(2\pi)^{4} \frac{ig_{e}^2}{(p_1 - p_3)^2} [(\bar{u}(s_3,p_3))(ig_{e}\gamma^\mu)(u(s_1, p_1))][(\bar{u}(s_4,p_4))(ig_{e}\gamma_\mu)(u(s_2, p_2))]\delta^{(4)}(p_2 + p_1 - p_3 - p_4).
\end{equation}

\textbf{Step Seven:} We simply set Eq (\ref{exOneIntegralResult}) to be equal
to $-i\mathcal{M}\delta^{(4)}(p_2 + p_1 - p_3 - p_4)$, and we solve to find
\begin{equation}
\mathcal{M} = \frac{-g_{e}^2}{(p_1 - p_3)^2} [(\bar{u}(s_3,p_3))(ig_{e}\gamma^\mu)(u(s_1, p_1))][(\bar{u}(s_4,p_4))(ig_{e}\gamma_\mu)(u(s_2, p_2))]
\end{equation}
is the probability amplitude. In spite of this nightmarish appearence, with 
four spinors and eight $\gamma$ matrices, this is still just a number. We can
figure it out when the spins are specified.

\subsection{Moller Scattering}

Moller scattering is the scattering of electrons
\begin{equation}
e^{-} + e^- \to e^- + e^-.
\end{equation}
We have two diagrams to consider this time! In fact, from here on out, we will
always have two diagrams to consider (the exception being one third order example,
which is the most important third order example because it explains the 
anamolous magnetic moment of an electron -- we'll burn that bridge when we get
to it).

\textbf{Step One:} The first diagram to consider is the
following:



\strut
\begin{center}
\begin{fmffile}{mollerImg1}
  \begin{fmfgraph*}(50,25)  \fmfpen{0.2mm}
    \fmfset{arrow_len}{3mm}\fmfset{arrow_ang}{10}
    \fmfleft{i1,o1} % change i2->o1 
    \fmfright{i2,o2} % change o1->i2
    \fmflabel{$p_{1},s_{1}$}{i1}
    \fmflabel{$p_{3},s_{3}$}{o1} %
    \fmflabel{$p_{2},s_{2}$}{i2} %
    \fmflabel{$p_{4},s_{4}$}{o2}
    \fmflabel{$\mu$}{v1}
    \fmflabel{$\nu$}{v2}
    \fmf{fermion}{i1,v1} %
    \fmf{fermion}{v1,o1}
    \fmf{fermion}{i2,v2} %
    \fmf{fermion}{v2,o2} %
    \fmf{boson,label=$q$}{v1,v2}
  \end{fmfgraph*}
\end{fmffile}
\end{center}
\strut


This is precisely the electron-muon diagram with the exception that the muon
has been replaced by an electron. Thus we will simply use the \textbf{exact
same steps} we did in the first example; we will copy/paste the results here.

The integrand should take the form
\begin{eqnarray}
\quad&&[(\bar{u}(s_3,p_3))(ig_{e}\gamma^\mu)(u(s_1, p_1))]\frac{-ig_{\mu\nu}}{q^2}[(\bar{u}(s_4,p_4))(ig_{e}\gamma^\nu)(u(s_2, p_2))] (2\pi)^{8}\nonumber\\
& &\times \delta^{(4)}(p_1 - p_3 - q)\delta^{(4)}(p_2 + q - p_4) d^4q. \nonumber
\end{eqnarray}
This has the contribution to the total probability amplitude that this process
will happen of
\begin{equation}
\mathcal{M}_1 = \frac{-g_{e}^2}{(p_1 - p_3)^2} [(\bar{u}(s_3,p_3))(ig_{e}\gamma^\mu)(u(s_1, p_1))][(\bar{u}(s_4,p_4))(ig_{e}\gamma_\mu)(u(s_2, p_2))]
\end{equation}
We will add it to the probability amplitude from the other graph to get the total
probability amplitude of the process happening.

The second diagram is odd:

\strut
\begin{center}
\begin{fmffile}{mollerImg2}
  \begin{fmfgraph*}(25,50)  \fmfpen{0.2mm}
    \fmfset{arrow_len}{3mm}\fmfset{arrow_ang}{10}
    \fmfleft{i2,o2}
    \fmfright{i1,o1}
    \fmf{fermion}{i1,v1}
    \fmf{phantom}{v1,o1} % Invisible rubber band
    \fmf{fermion}{i2,v2}
    \fmf{phantom}{v2,o2} % also invisible rubber band
    \fmf{photon,label=$q$}{v1,v2}
    % These are visible, but have no tension.
    \fmf{fermion,tension=0}{v1,o2}
    \fmf{fermion,tension=0}{v2,o1}
    \fmfdot{v1,v2}
    \fmflabel{$p_2,s_2$}{i1}
    \fmflabel{$p_1,s_1$}{i2}
    \fmflabel{$p_3,s_3$}{o1}
    \fmflabel{$p_4,s_4$}{o2}
    \fmflabel{$\mu$}{v1}
    \fmflabel{$\nu$}{v2}
  \end{fmfgraph*}
\end{fmffile}
\end{center}
\strut


We make the switch of $(s_3,p_3)\iff (s_4,p_4)$ for this diagram, and low and
behold we have a rule that takes care of this!

\textbf{Step Eight:} (Yes we are hopping right along!) We have by the eighth rule
a change in signs. So the probability amplitude from this second diagram is
(when we make the switches of $p_3\mapsto p_4$, $p_4\mapsto p_3$, $s_3\mapsto s_4$, $s_4\mapsto s_3$)
\begin{equation}
\mathcal{M}_2 = \frac{g_{e}^2}{(p_1 - p_4)^2} [(\bar{u}(s_4,p_4))(ig_{e}\gamma^\mu)(u(s_1, p_1))][(\bar{u}(s_3,p_3))(ig_{e}\gamma_\mu)(u(s_2, p_2))]
\end{equation}
So the total probability amplitude is then
\begin{eqnarray*}
\mathcal{M} &=& \frac{-g_{e}^2}{(p_1 - p_3)^2} [(\bar{u}(s_3,p_3))(ig_{e}\gamma^\mu)(u(s_1, p_1))][(\bar{u}(s_4,p_4))(ig_{e}\gamma_\mu)(u(s_2, p_2))] \\
&&+ \frac{g_{e}^2}{(p_1 - p_4)^2} [(\bar{u}(s_4,p_4))(ig_{e}\gamma^\mu)(u(s_1, p_1))][(\bar{u}(s_3,p_3))(ig_{e}\gamma_\mu)(u(s_2, p_2))].
\end{eqnarray*}


%
% \section{Inelastic Processes}
% An inelastic (relativistic) process is one where kinetic energy, rest energy,
% or mass are not conserved. We will explore such examples in QED.

% % Anamolous magnetic moment for the electron
% \input{thirdOrderEx}
%
