%%
%% torus.tex
%% 
%% Made by Alex Nelson
%% Login   <alex@tomato>
%% 
%% Started on  Sat Aug 29 12:56:09 2009 Alex Nelson
%% Last update Sat Aug 29 12:56:09 2009 Alex Nelson
%%
\begin{figure}[ht]
\includegraphics{img/img.0}
\caption{Identifying the top of the cube with the bottom, we are
  then left to identify one pair of sides of the cube as the
  same, and the remaining pair of sides as the same.}\label{fig:img0}
\end{figure}

Consider the Torus. We will construct it by considering a cube,
which has 6 faces, 12 edges, and 8 vertices. We will identify
opposite pairs of faces as being ``the same'' (glued
together). We will consider what this looks like as far as the
vertices, edges, and faces are concerned.

If we start by examining the vertices, we will cheat and identify
the top face with the bottom face. We are left with 4 distinct
vertices, as seen in figure \ref{fig:img0}. We then identify one
pair of opposite faces as the same, then the other pair of
opposite faces as the same. This is precisely what we do in
figure \ref{fig:img0}. We see that there is thus only one vertex
for the 3-torus.

Now, with regard to the edges, this is a bit trickier. Lets begin
with something easier: faces. We identify opposite pairs of faces
as being ``the same''. There are 6 faces, thus 3 such pairs. We
have 3 faces in the 3-torus (if one is unsatisfied with this
quick construction, one can pull out a regular six sided die and
observe that the each opposite face sums to 7; there are 3
distinct ways to add positive integers up to be 7, precisely 1+6,
2+5, and 3+4).

We have 3 faces, 1 vertex, and now the hard part: edges. We see
in figure \ref{fig:img1} that the distinct faces are colorized,
with the duplicate faces removed. We only, therefore, have to
worry about the edges which are drawn in black. This means that
there are 3 distinct edges. The others are duplicates.

\begin{figure}[ht]
\includegraphics{img/img.1}
\caption{The distinct faces for the 3-torus are colorized, with the duplicate faces removed.}\label{fig:img1}
\end{figure}

We can now start considering the chain complex for the
3-torus. We see that there are 1 vertex, 3 edges, 3 faces, and 1
3-cell. This implies the chain complex should look like
\begin{equation}%\label{eq:}
\mathbb{Z}\xleftarrow{\partial_{0}}\mathbb{Z}^{3}\xleftarrow{\partial_{1}}\mathbb{Z}^{3}\xleftarrow{\partial_{2}}\mathbb{Z}.
\end{equation}
We would like to consider what exactly the boundary maps
$\partial$ are as matrices. We need more information, we need to
pick some orientations for the faces, edges, and the
3-cell. \comment{We would also like to calculate that
  $\partial^{2}=0$ to make certain we're on the right track!}
We will use the orientation indicated in figure \ref{fig:img1}. 

We will consider the operator
$\partial_{0}:\mathbb{Z}^{3}\to\mathbb{Z}$. Since there is only
one vertex, that means all edges have \emph{the same} source and
target. This means that the operator $\partial_{0}(e)=t(e)-s(e)$
would be zero, since $t(e)=s(e)$.

If we consider $\partial_{1}:\mathbb{Z}^{3}\to\mathbb{Z}^{3}$
which takes faces to edges, we have something more interesting.
If we let
\begin{equation}%\label{eq:}
af_{1}+bf_{2}+cf_{3} = \begin{bmatrix}a\\b\\c\end{bmatrix},\quad\text{and}\quad
ae_{1}+be_{2}+ce_{3} = \begin{bmatrix}a\\b\\c\end{bmatrix}
\end{equation}
(where $a,b,c\in\mathbb{Z}$), we can consider the $\partial_{1}$
boundary operator as a square matrix. Consider face $f_{1}$ as
drawn in figure \ref{fig:img2}. Since there's only one distinct
vertex, we see travelling along the path in red gets us back to
where we started. This means that
\begin{equation}%\label{eq:}
\partial_{1}\left(\begin{bmatrix}1\\0\\0\end{bmatrix}\right) = \begin{bmatrix}1\\1\\0\end{bmatrix}
\end{equation}
since both $e_{1}$, $e_{2}$ are moving ``along'' the orientation
chosen for $f_{1}$. Referring to figure \ref{fig:img1}, we see
that the other faces are similar. This allows us to explicitly compute
\begin{equation}%\label{eq:}
\partial_{1}\left(\begin{bmatrix}0\\1\\0\end{bmatrix}\right) = \begin{bmatrix}1\\0\\1\end{bmatrix}\quad\text{and}\quad
\partial_{1}\left(\begin{bmatrix}0\\0\\1\end{bmatrix}\right) = \begin{bmatrix}0\\1\\1\end{bmatrix}.
\end{equation}
But this \emph{completely determines} what $\partial_{1}$ must
be, explicitly as a linear operator it is ``merely''
\begin{equation}%\label{eq:}
\partial_{1} = \begin{bmatrix}1 & 1 & 0\\
1 & 0 & 1\\
0 & 1 & 1\end{bmatrix}
\end{equation}
which is a curious operator in its own right. It's also a good
hint that our value for $\partial_{0}$ is correct, since
$\det(\partial_{1})=-2\neq0$ implies $\partial_{1}$ has a trivial
kernel. The only way $\partial_{0}\circ\partial_{1}=0$ is iff
$\partial_{0}=[0,0,0]$.

\begin{figure}[t]
\includegraphics{img/img.2}
\caption{The face $f_{1}$ of the 3-torus, boundary edges $e_{1}$
  and $e_{2}$ are drawn in. The path we use for the chain complex
  is drawn in red.}\label{fig:img2}
\end{figure}

We now are stuck with our last operator $\partial_{2}$. We can
cheat, and use the same argument for $\partial_{0}=[0,0,0]$
to deduce that
\begin{equation}%\label{eq:}
\partial_{2} = \begin{bmatrix}0\\0\\0\end{bmatrix}.
\end{equation}
I have a feeling the preferred route would be to prove that it
must be this via picking a preferred orientation, etc.
