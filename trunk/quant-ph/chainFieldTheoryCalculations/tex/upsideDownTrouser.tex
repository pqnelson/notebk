%%
%% upsideDownTrouser.tex
%% 
%% Made by Alex Nelson
%% Login   <alex@black-cherry>
%% 
%% Started on  Mon Aug 24 10:37:32 2009 Alex Nelson
%% Last update Mon Aug 24 10:37:32 2009 Alex Nelson
%%

\begin{figure}[t]
\includegraphics{img/img.5}
\caption{The upside down trousers diagram.}
\label{fig:img5}
\end{figure}

This is nearly identical to the rightside up trousers. However,
since the initial state has two (disjoint) circles, and the final
state  has one circle, the chain diagram becomes
\begin{equation}%\label{eq:}
\begin{CD}
\mathbb{Z}^{2} @<<< \mathbb{Z}^{2} \\
@VVV                 @VVV\\
\mathcal{M}_{1} @<<< \mathcal{M}_{2} @<<< \mathcal{M}_{3} \\
@AAA                 @AAA\\
\mathbb{Z}      @<<< \mathbb{Z}    
\end{CD}
\end{equation}
In fact we can fill in the missing  pieces using the same
arguments as for the rightside up trousers:
\begin{equation}%\label{eq:}
\begin{CD}
\mathbb{Z}^{2} @<<< \mathbb{Z}^{2} \\
@VVV                 @VVV\\
\mathbb{Z}^{3} @<<< \mathbb{Z}^{5} @<<< \mathbb{Z} \\
@AAA                 @AAA\\
\mathbb{Z}      @<<< \mathbb{Z}    
\end{CD}
\end{equation}
since there are 3 vertices, 5 edges, and 1 3-cell.

We use the same reasoning to suggest that the time evolution
operator is described by
\begin{equation}%\label{eq:}
Z(m):Z(\mathbb{Z}^{2})\to Z(\mathbb{Z})
\end{equation}
where (by the same arguments as last time)
\begin{equation}%\label{eq:}
Z(\mathbb{Z})\cong L^{2}(U(1)),\quad\text{and}\quad
Z(\mathbb{Z}^{2})\cong L^{2}(U(1))^{\otimes 2}.
\end{equation}
We pick out some basis vectors in the initial and final states,
specifically
\begin{equation}%\label{eq:}
\psi = e^{ik_{1}A_{1}}e^{ik_{2}A_{2}}
\end{equation}
for the initial state, and
\begin{equation}%\label{eq:}
\phi = e^{ikA_{3}}
\end{equation}
for the final state (where we use shorthand $A_{i}=A(e_{i})$). We
calculate the transition probability to deduce what the operator
for time evolution is
\begin{subequations}
\begin{align}
\<\phi,Z(m)\psi\> &= \int_{\mathcal{A}(m)}\overline{\phi(A|_{S})}\psi(A|_{S})e^{-S(A)}\mathcal{D}A\\
&= \int_{\mathcal{A}(m)}
e^{-ikA_{3}}e^{ik_{1}A_{1}}e^{ik_{2}A_{2}}\sum_{n\in\mathbb{Z}}e^{\frac{-1}{2e^{2}V}(F+2n\pi)^{2}}\mathcal{D}A\\
&= \iiint^{2\pi}_{0}
e^{-ikA_{3}}e^{ik_{1}A_{1}}e^{ik_{2}A_{2}}\sum_{n\in\mathbb{Z}}e^{\frac{-e^{2}Vn^{2}}{2}}e^{iFn} \frac{dA_{1}}{2\pi}\frac{dA_{2}}{2\pi}\frac{dA_{3}}{2\pi}
\end{align}
\end{subequations}
where $F$ is the curvature (field strength tensor),
$\mathcal{A}(m)$ is the space of connections on $m$, and the last
line is (of course) up to some constant.

To compute the field strength we need to calculate the curvature
of the connection cochains. If we start at $v_1$, go against the
orientation of $e_1$ (add term of $-A(e_{1})$), then down $e_4$
(add term of $+A(e_{4})$), around $e_3$ in the direction towards
its orientation (add a term of $+A(e_{3})$), up $e_{3}$ (add a
factor of $-A(e_{5})$), around $e_{2}$ against its orientation
(add a term of $-A(e_{2})$), back down $e_{5}$ (add a term of
$+A(e_{5})$), then up $e_{4}$ (add a term of $-A(e_{4})$) to end our computation.
This results in the curvature being
\begin{equation}%\label{eq:}
F = A_{3}-A_{1}-A_{2}
\end{equation}
where $A_{i}=A(e_{i})$. We plug this into our calculations to find
\begin{subequations}%\label{eq:}
\begin{align}
\<\phi,Z(m)\psi\> &= \int_{\mathcal{A}(m)}\overline{\phi(A|_{S})}\psi(A|_{S})e^{-S(A)}\mathcal{D}A\nonumber\\
&=  \iiint^{2\pi}_{0}
e^{-ikA_{3}}e^{ik_{1}A_{1}}e^{ik_{2}A_{2}}\sum_{n\in\mathbb{Z}}e^{\frac{-e^{2}Vn^{2}}{2}}e^{i(A_{3}-A_{1}-A_{2})n} \frac{dA_{1}}{2\pi}\frac{dA_{2}}{2\pi}\frac{dA_{3}}{2\pi}\\
&= \int e^{-ikA_{3}}\sum_{n\in\mathbb{Z}} e^{inA_{3}}e^{-e^{2}Vn^{2}/2}\left(\int^{2\pi}_{0}e^{i(n-k_{1})A_{1}}\frac{dA_{1}}{2\pi}\right)\left(\int^{2\pi}_{0}e^{i(n-k_{2})A_{2}}\frac{dA_{2}}{2\pi}\right)\frac{dA_{3}}{2\pi}\\
&= \int e^{-ikA_{3}}\sum_{n\in\mathbb{Z}} e^{inA_{3}} e^{-e^{2}Vn^{2}/2}\delta_{k_{1},n}\delta_{k_{2},n}\frac{dA_{3}}{2\pi}
\end{align}
\end{subequations}
It's easy to tell by inspection that we expect
\begin{equation}%\label{eq:}
Z(m):e^{ik_{1}A_{1}}e^{ik_{2}A_{2}}\mapsto e^{inA_{3}} e^{-e^{2}Vn^{2}/2}\delta_{k_{1},n}\delta_{k_{2},n}
\end{equation}
to be how basis vectors in $L^{2}(U(1))^{\otimes 2}$ behave, just
as we desire.
