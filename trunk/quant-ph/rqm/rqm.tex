%%
%% rqm.tex
%% 
%% Made by Alex Nelson
%% Login   <alex@tomato>
%% 
%% Started on  Tue Jul 21 10:59:59 2009 Alex Nelson
%% Last update Tue Jul 21 10:59:59 2009 Alex Nelson
%%
\documentclass{amsart}
\usepackage{url}
\usepackage{manfnt}
\usepackage{amsthm,amsmath,amsthm,amssymb,amsfonts}
\usepackage{paralist}
\usepackage{graphicx}
\usepackage{mathrsfs}
\usepackage{hyperref}
\hypersetup{
    colorlinks,%
    citecolor=black,%
    filecolor=black,%
    linkcolor=black,%
    urlcolor=black
}
\usepackage{fancyvrb,comment}
\numberwithin{equation}{section}

\theoremstyle{definition}
\newtheorem{defn}{Definition}
\newtheorem{thm}{Theorem}
\newtheorem{rmk}{Remark}
\newtheorem{lem}{Lemma}
\newtheorem{cor}{Corollary}
\newtheorem{ex}{Example}
\newtheorem{nonex}[ex]{NON-Example}
\newtheorem{prop}{Proposition}
\newtheorem{sch}{Scholium}
\newtheorem{axm}{Axiom}
\newtheorem*{prob}{Problem}

\def\re{\operatorname{Re}}
\def\tr{\operatorname{Tr}}
\def\<{\langle}
\def\>{\rangle}

%%
% This macro header is what controls the ``dangerous bend''
% paragraph
%%
\def\rd{\noindent\hangindent=2pc\hangafter=-2\def\par{\endgraf}\hbox
  to0pt{\hskip-\hangindent\dbend\hfill}\ignorespaces}
%%
% This command allows you to write stuff in small font size and
% use the
% bourbaki ``dangerous bend'' so it's great when you want to
% ramble on 
% about some extra stuff!
%%
%\newenvironment{danger}{\par\begingroup\rd\small}{\par\endgroup}
\newenvironment{danger}{\rd\begin{footnotesize}\noindent}{\end{footnotesize}}
%%
% This macro header is what controls the ``dangerous bend''
% paragraph
%%
\def\ddbend{\dbend\kern1pt\dbend}

\def\rdd{\noindent\hangindent=2.5pc\hangafter=-2\def\par{\endgraf}\hbox
  to0pt{\hskip-\hangindent\ddbend\hfill}\ignorespaces}

\newenvironment{ddanger}{\rdd\begin{footnotesize}}{\end{footnotesize}\newline}
\newcommand{\define}[1] {\textbf{#1}\index{#1}}
% \stackrel{\text{def}}{=}

%%%%%%%%%%%%%%%%%%%%%%%%%%%%%%%%%%%%%%%%%%%%%%%%%%%%%%%%%%%%%%%%%%%%%%%%%%%%%%%
% BEGIN SIGNATURE CONVENTION SECTION
%%%%%%%%%%%%%%%%%%%%%%%%%%%%%%%%%%%%%%%%%%%%%%%%%%%%%%%%%%%%%%%%%%%%%%%%%%%%%%%
\newif\ifeastcoast
\newif\ifwestcoast

% east coast true
\eastcoasttrue

% magically set the west coast flag to be !(east coast)
\ifeastcoast
	\westcoastfalse
\else
	\westcoasttrue
\fi

\newcommand{\massShell}%
{%
\ifeastcoast %
p^{\mu}p_{\mu} = -\mu^{2}%
\else%
p^{\mu}p_{\mu} = \mu^{2}%
\fi%
}
\newcommand{\massShellOperators}%
{%
\ifeastcoast %
\widehat{p}^{\mu}\widehat{p}_{\mu} = -\mu^{2}%
\else%
\widehat{p}^{\mu}\widehat{p}_{\mu} = \mu^{2}%
\fi%
}
\newcommand{\massShellSpectrum}%
{%
\ifeastcoast %
\widehat{p}^{\mu}\widehat{p}_{\mu}|k\> = -\mu^{2}|k\>%
\else%
\widehat{p}^{\mu}\widehat{p}_{\mu}|k\> = \mu^{2}|k\>%
\fi%
}
%%%%%%%%%%%%%%%%%%%%%%%%%%%%%%%%%%%%%%%%%%%%%%%%%%%%%%%%%%%%%%%%%%%%%%%%%%%%%%%
% END SIGNATURE CONVENTION SECTION
%%%%%%%%%%%%%%%%%%%%%%%%%%%%%%%%%%%%%%%%%%%%%%%%%%%%%%%%%%%%%%%%%%%%%%%%%%%%%%%

\title{Notes on Relativistic Quantum Mechanics}
\date{July 21, 2009}
\email{pqnelson@gmail.com}
\author{Alex Nelson}
\begin{document}
\maketitle\footnote{Note we are using the West Coast convention, i.e. + - - - metric signature, and setting $c=1$ and $\hbar=1$.}
\tableofcontents

\section{One Particle Systems: Mathematical Formalism}
%%
%% oneParticleState.tex
%% 
%% Made by Alex Nelson
%% Login   <alex@tomato>
%% 
%% Started on  Tue Jul 21 12:54:12 2009 Alex Nelson
%% Last update Tue Jul 21 12:54:12 2009 Alex Nelson
%%

The simplest system to consider is a single particle. The
function space used to model quantum-mechanical states is a
Hilbert Space $\mathcal{H}$ of square integrable functions on the
physical space (denoted by $\mathcal{C}$):
\begin{equation}%\label{eq:}
L^{2}(\mathcal{C}) = \left\{f\,:\;\int_{\mathcal{C}}|f(\overline{x})|^{2}d^{3}\overline{x}<\infty\right\}
\end{equation}
Note that in all fairness, $\mathcal{H}$ can be written in either
position coordinates $\overline{x}$ or momentum coordinates
$\overline{p}$. The relationship between the position-space and
momentum-space is precisely the familiar Fourier transform:
\begin{equation}%\label{eq:}
\mathcal{F}(f)(\overline{p}) \stackrel{\text{def}}{=} \int e^{i\overline{x}\cdot\overline{p}}f(\overline{x})d^{3}\overline{x}
\end{equation}
Despite the change of variables, $\mathcal{F}$ sends
$\mathcal{H}$ to itself, so both $f$ and its Fourier transform
$\mathcal{F}(f)$ are in $\mathcal{H}$.
\begin{rmk}
It should be emphasized that if $f$ is square-integrable, then
$e^{i\overline{x}\cdot\overline{p}}f(\overline{x})$ is
square-integrable \emph{but not necessarily integrable!} That is,
we have no guarantee that
$e^{i\overline{x}\cdot\overline{p}}f(\overline{x})\in L^{1}(\mathcal{C})$.

To define the Fourier transform on $\mathcal{H}$, we should first
define it on some suitably nice subspace of $\mathcal{H}$
(e.g. the space of smooth functions with ``compact support'' ---
i.e. they are zero outside of a compact subset of their
domain). Then we observe that the Fourier transform is an
isometry (up to some scale factor) on our nice subspace, so we
extend this isometry from our nice subspace to all of $\mathcal{H}$.
\end{rmk}

We represent the observables by operators. More relevantly, the
position operators $\widehat{x}_{m}$ and momentum operators
$\widehat{p}_{m}$ are represented in position-space by
multiplication by the coordinate fuhnctions $x_{m}$ and the
partial derivative operators $-i\partial_{m}$
(respectively). Observe also that the Fourier transform converts
multiplication by $x_{m}$ on functions of $\overline{x}$ into the
differential operators $-i\partial_{m}$ on functions of
$\overline{p}$:
\begin{equation}%\label{eq:}
\mathcal{F}(x_{m}f)(\overline{p})=-i\partial_{m}\mathcal{F}(f)(\overline{p}).
\end{equation}

The natural question to ask is ``What are the eigenstates of
these operators?'' Well, in position space, we find the position
eigenstates are just delta functions
\begin{subequations}
\begin{align}
(\widehat{x}_{m}\delta_{\overline{q}})(\overline{x}) &=
  \widehat{x}_{m}\delta(\overline{x}-\overline{q})\\
&= q_{m}\delta(\overline{x}-\overline{q})\\
&= (q_{m}\delta_{\overline{q}})(\overline{x})
\end{align}
\end{subequations}
Similarly, for the eigenstates of the momentum operators
$\widehat{p}_{m}$, we see that the eigenstates in position-space
are $e_{\overline{p}}(\overline{x})$:
\begin{subequations}
\begin{align}
(\widehat{p}_{m}e_{\overline{p}})(\overline{x}) &= -i\partial_{m}\exp(i\overline{p}\cdot\overline{x})\\
&= p_{m}\exp(i\overline{p}\cdot\overline{x})\\
&= (p_{m}e_{\overline{p}})(\overline{x}).
\end{align}
\end{subequations}

But we have just two minor problems: \begin{inparaenum}
\item neither $\widehat{x}_{m}$ nor $\widehat{p}_{n}$ act on all
  of $\mathcal{H}$, and
\item $\mathcal{H}$ doesn't contain the eigenstates of either
  operators.
\end{inparaenum}
We can solve the first problem fairly easily --- we'll consider
the subspace $S\subset{\mathcal{H}}$ where the operators map $S$
to itself. Similarly, we resolve the second problem by defining
the kets as elements of $S^{*}$, the space of continuous
antilinear functionals on $S$. Since $\widehat{p}_{n}$ acts on
all functions of $S$, these functions must be infinitely
differentiable, and so $S^{*}$ will contain the
$\delta$-functions and all their derivatives. Similarly, by
taking the Fourier Transform, since $\widehat{x}_{m}$ acts on
$S$, it follows that $S^{*}$ will contain exponential functions
$\exp(i\overline{p}\cdot\overline{x})$. 

Instead of a single Hilbert space, we end up with a triple
\begin{equation}%\label{eq:}
S\subset{\mathcal{H}}\subset{S^{*}}
\end{equation}
The physical states live in $S$, and the operator eigenstates
live in $S^{*}$. With appropriate demands on the space $S$, 
this triple ends up being a \emph{Rigged Hilbert Space}~\cite{delamadrid}~\cite{Madrid:2004zy}. 
In this context ``Rigged'' \emph{is  not} in the sense of ``This
game is rigged'' but rather in the sense of ``equipped'' --- like
how a boat is ``rigged'' or ``equipped to sail''. 

\bigskip
\begin{ddanger}
In fact, the triple $S\subset\mathcal{H}\subset{S^{*}}$ is a
rigged Hilbert space if $S$ is a nuclear subspace of
$\mathcal{H}$. See Gelfand~\cite{gelfandgeneralized} or
Maurin~\cite{maurin} for rigorous details about the notion of
nuclear spaces. We'll discuss one such criteria for $S$ to be
nuclear. Specifically,
\begin{enumerate}
\item there exists a countable family $\|\cdot\|_{k}$ of norms on
$S$ with respect to which convergence is defined
by 
\begin{equation}
f_{n}\to{f}\quad\iff\quad\|f_{n}-f\|_{k}\to0\;\;\forall k\geq0;
\end{equation}
\item $S$ is complete with respect to this notion of
  convergence; and 
\item there exists a Hilbert-Schmidt operator on
  $S$ with a continuous inverse.
\end{enumerate}
We'll leave the interested reader to refer to the cited sources.
\end{ddanger}
\bigskip

In a rigged Hilbert Space we have eigenfunction expansions. More
precisely, consider a state $|f\>$ represented by the function
$f$, let $|\overline{x}\>$ be the position eigenstate represented
by the distribution $\delta_{\overline{x}}$. We assume the
relationship between the functions and the kets is such that
\begin{equation}%\label{eq:}
f(\overline{x}) = \<\overline{x}|f\>.
\end{equation}
We can then expand the state $|f\>$ in terms of the position
eigenstate $|\overline{x}\>$ which should be of the form
\begin{equation}%\label{eq:}
|f\> = \int |\overline{x}\>\,\<\overline{x}|f\>\,d^{3}\overline{x} = \int f(\overline{x})\,|\overline{x}\>\,d^{3}\overline{x}.
\end{equation}
The conditions on $S$ in a rigged Hilbert Space ensure that
$f(\overline{x})|\overline{x}\>$ is integrable for all $f\in{S}$.

\section{One Particle Systems: Physical Aspects}
%%
%% oneParticleSystemPhysics.tex
%% 
%% Made by Alex Nelson
%% Login   <alex@tomato>
%% 
%% Started on  Tue Jul 21 15:55:44 2009 Alex Nelson
%% Last update Tue Jul 21 15:55:44 2009 Alex Nelson
%%

We're interested in a toy model of relativistic quantum
mechanics, so we begin with a single particle. All we really
need, truth be told, is a state space plus a Hamiltonian
operator. We should remember, from Special Relativity, the
energy-momentum four-vector $\widehat{p}_{\mu}$ has as its time component the
Hamiltonian $\widehat{p}_{0}=\widehat{H}$. For convenience, we'll
work in the momentum space with the momentum operator eigenbasis
\begin{equation}%\label{eq:}
\widehat{p}_{m}|\overline{k}\>=k_{m}|\overline{k}\>
\end{equation}
We assume the states are normalized thus
\begin{equation}%\label{eq:}
\<\overline{k}|\overline{k}'\>=\delta^{(3)}(\overline{k}-\overline{k}').
\end{equation}
This means that the length of a ket is undefined. It is,
nonetheless, a normalization suitable for integration over
momentum. As an added bonus, we also get the resolution of the
identity
\begin{equation}%\label{eq:}
\mathbf{1}=\int\,|\overline{k}\>\,\<\overline{k}|\,d^{3}\overline{k}
\end{equation}

Since energy-momentum is a four-vector, we demand that
\begin{equation}%\label{eq:}
\widehat{p}^{\mu}\widehat{p}_{\mu} = \widehat{H}^{2}-|\widehat{p}_{m}\widehat{p}^{m}|
\end{equation}
needs to be constant on the orbits of the Poincar\'e
group. Further if $|\overline{k}\>$ and $|\overline{k}\,'\>$ are
two states of a single particle, then there exists a Lorentz
boost from one to the other (up to scale). Hence we assume the
existence of a scalar quantity $\mu$ (the particle mass) which
satisfies
\begin{equation}%\label{eq:}
(\widehat{H}^{2} - \widehat{p}_{m}\widehat{p}^{m})|\overline{k}\>
= \mu^{2}|\overline{k}\>
\end{equation}
This implies that the Hamiltonian operator $\widehat{H}$ is
diagonal in the momentum eigenbasis (i.e. the basis of
eigenstates of the momentum operator):
\begin{equation}%\label{eq:}
\widehat{H}|\overline{k}\> = \left(\|\overline{k}\|^{2}+\mu^{2}\right)^{1/2}|\overline{k}\>
\end{equation}
The eigenvalues of the Hamiltonian operator come up enough times
that we introduce the shorthand for it:
\begin{equation}%\label{eq:}
\omega(\overline{k}) \stackrel{\text{def}}{=} \left(\|\overline{k}\|^{2}+\mu^{2}\right)^{1/2}
\end{equation}
(This should be vaguely reminiscent of the de Broglie relations
$E=\hbar\omega$.)

\begin{rmk}
Observe that this entire scheme we've devised is equivalent to
taking the limit of the state space for a cube of side $L$ under
periodic boundary conditions, i.e. the particle in a box
situation. In such a cube, we should recall the spectrum of the
momentum operator is discrete and the normalization is given by
the Kronecker delta:
\begin{equation}%\label{eq:}
\overline{k}=\frac{2\pi}{L}(n_x,n_y,n_z),\quad\text{and}\quad\<\overline{k}|\overline{k}\,'\>=\delta_{\,\overline{k}\, ,\,\overline{k}\,'}
\end{equation}
This observation is taken advantage of when deriving the
differential transition probability per unit time for particle
scattering.
\end{rmk}

\section{Unitary Representation of Poincar\'e Group}
\subsection{Action of Translation on States}
%%
%% poincareRep.tex
%% 
%% Made by Alex Nelson
%% Login   <alex@tomato>
%% 
%% Started on  Wed Jul 22 12:09:17 2009 Alex Nelson
%% Last update Wed Jul 22 12:09:17 2009 Alex Nelson
%%

The Lorentz transformation is usually ``represented'' by a matrix
$\Lambda$ which, when written explicitly, is
\begin{equation}%\label{eq:}
(\Lambda x)^{\mu} = {\Lambda^{\mu}}_{\nu}x^{\nu}
\end{equation}
where Einstein convention is used (implicit sum over $\nu$
occurs). We have that the matrix ${\Lambda^{\mu}}_{\nu}$ must
satisfy
\begin{equation}%\label{eq:}
{\Lambda^{\lambda}}_{\mu}{\Lambda_{\lambda}}_{\nu} = \eta_{\mu\nu}
\end{equation}
where $\eta_{\mu\nu}$ is the Minkowski metric (metric for flat
spacetime).

Now, the Poincar\'e group is the set of Lorentz transformations
and space-time translations, so the element of the group would be
$(\Lambda,a)$ such that
\begin{equation}%\label{eq:}
x^{\mu}\to y^{\mu} = {\Lambda^{\mu}}_{\nu}x^{\nu}+a^{\mu}.
\end{equation}
The group multiplication law is then just
\begin{equation}%\label{eq:}
(\Lambda_{2},a_{2})(\Lambda_{1},a_{1}) = (\Lambda_{2}\Lambda_{1},\Lambda_{2}a_{1}+a_{2}).
\end{equation}
We are interested in irreducible unitary representations $U(\cdot)$ of our
group, all we need to worry about are the generators.

The translations, rotations, and boosts of the Poincar\'e group
must act on the space of states. A Poincar\'e group element $g$
acts as a unitaruy operator $U(g)$ on the state space. The action
must satisfy a multipication condition
\begin{equation}%\label{eq:}
U(gh)=U(g)U(h)
\end{equation}
for all $g,h$ in the Poincar\'e group.

Translation of spacetime by a four-vector $a^{\mu}$ is defined by
\begin{equation}%\label{eq:}
\Delta_{a}(x) = x+a.
\end{equation}
Translation of a state $\psi$, on the other hand, should be
moving the graph by $a$. This means that
$\Delta_{a}\psi(x)=\psi(x-a)$. The unitary representation
$U(\Delta_{a})$ of $\Delta_{a}$ must thus be defined by
\begin{equation}%\label{eq:}
U(\Delta_{a})|\psi\>=|\Delta_{a}\psi\>.
\end{equation}
We'd like to find an expression for $U(\Delta_a)$ in terms of the
energy-momentum four-vector $\widehat{p}_{\mu}$.

Evolution in time is translation of the observer forward in time,
or (equivalently) translation of the system backwards in time:
\begin{equation}%\label{eq:}
\exp(-it\widehat{H})|\psi(x)\> = |\psi(x_{0}+t,\overline{x})\>.
\end{equation}
Let $\tau^{\mu}=(-t,\vec{0})$ be a four-vector, then we can
rewrite our translation in time as
\begin{equation}%\label{eq:}
\exp(i\tau^{\mu}\widehat{p}_{\mu})|\psi\> = |\Delta_{\tau}\psi\>.
\end{equation}
Lorentz invariance implies that this equation is true whenever
$\tau$ is timelike, and the additivity of translations then shows
this to be true for all four-vectors $\tau$. From this definition
of $U(\Delta_{a})$ we can therefore deduce that
\begin{equation}%\label{eq:}
U(\Delta_a) = \exp(ia^{\mu}\widehat{p}_{\mu}).
\end{equation}

Although this unitary representation is derived in the
position-space formulation of quantum mechanics, it works equally
well in the momentum-space formulation. We can deduce that the
unitary representation of translations on momentum eigenstates is
given by
\begin{equation}%\label{eq:}
U(\Delta_{a})|\overline{k}\> =
\exp(ia^{\mu}\widehat{p}_{\mu})|\overline{k}\> = \exp(ia^{\mu}k_{\mu})|\overline{k}\>
\end{equation}
where $k_{0} = \omega(\overline{k})$.

\begin{rmk}
Recall Taylor's theorem in real analysis can be formulated as
\begin{equation}%\label{eq:}
f(x+h) =
\left(\sum_{n=0}^{\infty}h^{n}\frac{d^{n}}{dx^{n}}\right)f(x) = \exp\left(h\frac{d}{dx}\right)f(x)
\end{equation}
which should look familiar: we just deduced the unitary
representation of spacetime translations should be
\begin{equation}%\label{eq:}
\exp(i\tau^{\mu}\widehat{p}_{\mu})|\psi\> = U(\Delta_{\tau})|\psi\>.
\end{equation}
If we don't distinguish $|\psi\>$ from $\psi(x)$, we see that
Taylor's theorem guarantees our representation to be of spacetime
translations.
\end{rmk}

\subsection{Action of the Lorentz Group}
%%
%% actionLorentzGroup.tex
%% 
%% Made by Alex Nelson
%% Login   <alex@tomato>
%% 
%% Started on  Wed Jul 22 13:01:36 2009 Alex Nelson
%% Last update Wed Jul 22 13:01:36 2009 Alex Nelson
%%

The space of particle states is three dimensional. The energy
$k_0$ of a particle with momentum $\overline{k}$ is constrained
by
\begin{equation}%\label{eq:}
k_{0}\geq0
\end{equation}
and
\begin{equation}%\label{eq:}
k^{2} = k_{\mu}k^{\mu} = \mu^{2}.
\end{equation}
Therefore the possible energy-momentum vectors lie on a
hyperbolic sheet in $k$-space, the mass hyperboloid. We need an
integration measure on this hyperboloid if we want to do Lorentz
invariant computations.

Let $\theta(t)$ be the Heaviside step function
\begin{equation}%\label{eq:}
\theta(t) = \begin{cases} 0 &\text{if }t<0\\
1 & \text{if }t>0
\end{cases}
\end{equation}
Define an integration $d\lambda(k)$ on the positive hyperboloid
as follows:
\begin{equation}%\label{eq:}
d\lambda(k) \stackrel{\text{def}}{=} d^{4}k \delta(k^{2}-\mu^{2})\theta(k_{0})
\end{equation}
The Lebesgue measure $d^{4}k$ is Lorentz invariant due to the
Lorentz transformation having unit determinant. Here since
$k^{2}-\mu^{2}$ is Lorentz invariant, the $\delta$ function is
Lorentz invariant. Similar reasoning holds for $\theta(k_{0})$
being Lorentz invariant.

We can take advantage of the identity
\begin{equation}%\label{eq:}
\delta(f(k)) = \sum_{\{k:f(k)=0\}}\frac{1}{\|f'(k)\|}\delta(k)
\end{equation}
and  the fact that
\begin{subequations}
\begin{align}
k^{2}-\mu^{2} &= (k_{0}^{2}-\|\overline{k}\|^{2})-\mu^{2}\\
&= k_{0}^{2} - (\|\overline{k}\|^{2} + \mu^{2}) \\
&= k_{0}^{2} - \omega(\overline{k})^{2} \\
&= (k_{0} - \omega(\overline{k}))(k_{0} + \omega(\overline{k}))
\end{align}
\end{subequations}
to deduce that
\begin{subequations}
\begin{align}
\delta(k^{2}-\mu^{2})\theta(k_{0}) &= \delta\left((k_{0} - \omega(\overline{k}))(k_{0} + \omega(\overline{k}))\right)\theta(k_0)\\
&=\frac{1}{2\omega(\overline{k})}(\delta(k_0-\omega(\overline{k}))\theta(k_0)+\delta(k_0+\omega(\overline{k}))\theta(k_0))\\
&=\frac{1}{2\omega(\overline{k})}\delta(k_0-\omega(\overline{k}))\theta(k_0)
\end{align}
\end{subequations}
since $\delta(k_0+\omega(\overline{k}))$ requires $k_0<0$ which
then demands that $\theta(k_0)=0$, so that term drops out completely.
Observe that this means we can effectively eliminate $k_0$ from
any integral with respect to $\omega(\overline{k})$ as follows:
\begin{subequations}
\begin{align}
\int f(k)d\lambda(k) &= \int f(k)\left(\frac{\delta(k_{0}-\omega(\overline{k}))}{2\omega(\overline{k})}\theta(k_{0})d^{3}\overline{k}dk_{0}\right)\\
&= \int f\left(\omega(\overline{k}),\overline{k}\right)\frac{d^{3}\overline{k}}{2\omega(\overline{k})}
\end{align}
\end{subequations}
This integral and the arbitrary function $f$ are commonly
eliminated from this result, leaving an equality of measures
\begin{equation}%\label{eq:}
d\lambda(k) = \frac{d^{3}\overline{k}}{2\omega(\overline{k})}
\end{equation}
and
\begin{equation}%\label{eq:}
k_{0} = \omega(\overline{k}).
\end{equation}

\begin{comment}
\begin{ddanger}We can now ask if the measure
%\begin{equation}%\label{eq:}
$d^{3}\overline{k}/2\omega(\overline{k})$
%\end{equation}
is Lorentz invariant or not. We expect it to be so, but lets try
to demonstrate it explicitly by computing the Jacobian of a
Lorentz boost $\Lambda$. Without loss of generality, we can
assume that we are working with Cartesian coordinates. Note we
can factorize an Lorentz boost as
\begin{equation}%\label{eq:}
{\Lambda^{\mu}}_{\nu} = {L^{\mu}}_{\alpha}{R^{\alpha}}_{\nu}
\end{equation}
where $L$ is a rotation in the $t-x$ plane, and $R$ is an
arbitrary spatial rotation. We know from Euler's theorem that
\begin{equation}%\label{eq:}
\det(R)=1
\end{equation}
so that means that
\begin{equation}%\label{eq:}
\det(\Lambda)=\det(L).
\end{equation}
But we can write in block form
\begin{equation}%\label{eq:}
{L^{\mu}}_{\alpha} = \begin{bmatrix} L & 0\\ 0 & I_{2} \end{bmatrix}
\end{equation}
which means that 
\begin{equation}%\label{eq:}
\det({L^{\mu}}_{\alpha}) = {L^{0}}_{0}{L^{1}}_{1} - {L^{1}}_{0}{L^{0}}_{1}.
\end{equation}
This means, without loss of generality, we can write
${\Lambda^{\mu}}_{\nu}={L^{\mu}}_{\nu}$. We make the switch
$k^{0}=\omega(\overline{k})$, so our transformation
yields the coordinates
\begin{subequations}
\begin{align}
l^{0} &= {L^{0}}_{0}\omega(\overline{k}) - {L^{0}}_{1}k^{1}\\
l^{1} &= {L^{1}}_{0}\omega(\overline{k}) - {L^{1}}_{1}k^{1}\\
l^{2} &= k^{2}\\
l^{3} &= k^{3}
\end{align}
\end{subequations}
One may be at first alarmed by the switch to
$\omega(\overline{k})$ but it is invariant under spatial
rotations, and we've seen that $k^0=\omega(\overline{k})$ so it
is kosher. By Lorentz invariance, we demand further that
\begin{equation}%\label{eq:}
l^{\mu}l_{\mu} = k^{\mu}k_{\mu} = \mu^{2}
\end{equation}
which in turn implies that
\begin{equation}%\label{eq:}
l^{0}l_{0} = \|\overline{l}\|^{2} + \mu^{2}\;\Rightarrow\; l^{0}
= \omega(\overline{l})
\end{equation}
all by Lorentz invariance.
\end{ddanger}
\end{comment}

If we define Lorentz-normalized kets $|k\>$ by
\begin{equation}%\label{eq:}
|k\> = \left(2\omega(\overline{k})\right)^{1/2}(2\pi)^{3/2}|\overline{k}\>
\end{equation}
with $k_{0}=\omega(\overline{k})$, then the new normalization
conditions is
\begin{equation}%\label{eq:}
\<k|k'\> = 2\omega(\overline{k})(2\pi)^{3}\delta^{(3)}(\overline{k}-\overline{k}')
\end{equation}
and the resolution of the identity is based on the Lorentz
invariant measure:
\begin{equation}%\label{eq:}
\mathbf{1} = \int|k\>\<k|\frac{d^{3}\overline{k}}{(2\pi)^{3}2\omega(\overline{k})}.
\end{equation}
With these Lorentz-normalized states, we can define the unitary
representation of the Lorentz group simply:
\begin{thm}%\label{thm:}
If we define $U(\Lambda)$ by $U(\Lambda)|k\>=|\Lambda k\>$, then
$U$ is a unitary representation of the Lorentz group.
\end{thm}
\begin{proof}
The multiplications property
$U(\Lambda\Lambda')=U(\Lambda)U(\Lambda')$ follows immediately
from definition. To show that the representation is unitary, we
use the resolution of the identity
\begin{subequations}
\begin{align}
U(\Lambda)U(\Lambda)^{\dag} &= \int U(\Lambda)|k\>\<k|U(\Lambda)^{\dag}\frac{d^{3}\overline{k}}{(2\pi)^{3}2\omega(\overline{k})}\\
&= \int |\Lambda k\>\<\Lambda k|\frac{d^{3}\overline{k}}{(2\pi)^{3}2\omega(\overline{k})}\\
&= \mathbf{1}
\end{align}
\end{subequations}
since the measure is Lorentz-invariant.
\end{proof}
It is mildly surprising that $U(\Lambda)$ defined in our theorem
is a unitary operator due to $|k\>$ and $|\Lambda k\>$ appear to
have different lengths when $\Lambda$ is a boost. \emph{However,}
$\delta^{(3)}(0)$ is undefined, so the normalization of the kets
does not determine a length. We regard the uniformly unlocalized
state described by $|k\>$ as \emph{unphysical}. The physical
states have the form 
\begin{equation}%\label{eq:}
|\psi\>\stackrel{\text{def}}{=} \int \psi(\overline{k})|k\>\frac{d^{3}\overline{k}}{(2\pi)^{3}2\omega(\overline{k})}
\end{equation}
where the measure is structured so
$\<k|\psi\>=\psi(\overline{k})$. We can check that the length of
$|\psi\>$ is well defined whenever $\psi(\overline{k})$ is
square-integrable and that our definition of $U(\Lambda)$ makes
the representation unitary on the space of physical states.

\subsection{Representing the Poincar\'e Group}
%%
%% actionPoincareGroup.tex
%% 
%% Made by Alex Nelson
%% Login   <alex@tomato>
%% 
%% Started on  Sat Jul 25 14:01:05 2009 Alex Nelson
%% Last update Sat Jul 25 14:01:05 2009 Alex Nelson
%%

We really want to find a unitary representation of the Poincar\'e
group, which is the Lorentz group plus spacetime translations
(i.e. rotations, Lorentz boosts, and space-time translations). We
have the representation condition $U(gh)=U(g)U(h)$ must hold for
all $g,h$ in the Poincar\'e group. We've seen what happens when
both $g,h$ are in the Lorentz group, and when both $g,h$ are
space-time translations. We now need to ask: what happens when
one is a translation and the other is a boost?

We can uniquely factor any element $g$ of the Poincar\'e group as
the product
\begin{equation}%\label{eq:}
g = \Delta_{a}\Lambda
\end{equation}
where $\Lambda$ is in the Lorentz group, and $\Delta_a$ is a
translation. Multiplication in the Poincar\'e group depends on
multiplication in the Lorentz group and addition of translations
through an interchange in the order of the two facts:
\begin{subequations}
\begin{align}
gh &= \Delta_{a}\Lambda\Delta_{b}M\\
&= \Delta_{a}(\Lambda\Delta_{b}\Lambda^{-1})\Lambda M\\
&= \Delta_{a}\Delta_{\Lambda b}\Lambda M
\end{align}
\end{subequations}
where we have used the identity
\begin{equation}%\label{eq:}
\Lambda\Delta_{b}\Lambda^{-1} = \Delta_{\Lambda b}
\end{equation}
a relation trivially verified when we act on a 4-vector $x$.

Our definition of $U$ so far covers translations and Lorentz
group elements only; when we extend to the Poincar\'e group, we
do so through the definition
\begin{equation}%\label{eq:}
U(\Delta_{a}\Lambda) \stackrel{\text{def}}{=} U(\Delta_{a})U(\Lambda)
\end{equation}
We can now see that $U$ is a representation of the Poincar\'e
group if and only if $U$ preserves the action
$\Lambda\Delta_{b}\Lambda^{-1}=\Delta_{\Lambda b}$ of Lorentz
group elements on translations:
\begin{subequations}
\begin{align}
 & U(\Delta_{a}\Lambda)U(\Delta_{b}M) =
  U(\Delta_{a}\Delta_{\Lambda b}\Lambda M) \\
\iff & U(\Delta_{a})U(\Lambda)U(\Delta_{b})U(M) =
U(\Delta_{a})U(\Delta_{\Lambda b})U(\Lambda)U(M)\\
\iff & U(\Lambda)U(\Delta_{b}) = U(\Delta_{\Lambda b})U(\Lambda)\\
\iff & U(\Lambda)U(\Delta_{b})U(\Lambda)^{\dag} = U(\Delta_{\Lambda b})
\end{align}
\end{subequations}
We verify the final condition by evaluating both sides on some
test state $|k\>$. From the right hand side, we have
\begin{equation}%\label{eq:}
U(\Delta_{\Lambda b})|k\> = \exp(i\Lambda b^{\mu}k_{\mu})|k\>
\end{equation}
and from the left hand side
\begin{subequations}
\begin{align}
U(\Lambda)U(\Delta_{b})U(\Lambda)^{\dag}|k\> &=
U(\Lambda)U(\Delta_{b})|\Lambda^{-1}k\>\\
&=
U(\Lambda)\exp(ib^{\mu}{\Lambda_{\mu}}^{\nu}k_{\nu})|\Lambda^{-1} k\>\\
&=\exp(ib^{\mu}{\Lambda_{\mu}}^{\nu}k_{\nu})|k\>.
\end{align}
\end{subequations}
The equality of the two sides follows from the Lorentz-invariance
of the inner product.

We can now summarize our results of $U$ in the following theorem:
\begin{thm}%\label{thm:}
The map $U$ from the Poincar\'e group to operators on the state
space defined by
\begin{subequations}
\begin{align}
U(\Delta_{a})|k\> &= e^{ia^{\mu}k_{\mu}}|k\>\\
U(\Lambda)|k\> &= |\Lambda k\>\\
U(\Delta_{a}\Lambda) &= U(\Delta_{a})U(\Lambda) 
\end{align}
\end{subequations}
is a unitary representation of the Poincar\'e group.
\end{thm}

The unitary representation $U$ is often boasted to successfully
combines the principle (as represented by the Poincar\'e group)
with the principles of quantum mechanics (as represented by
unitary operators and state-space formalisms). This combined
structure of a one-particle state space provides the foundation
for the many-particle state space used in all quantum field theories.


\section{Notes on a Position Operator}
%%
%% positionOperator.tex
%% 
%% Made by Alex Nelson
%% Login   <alex@tomato>
%% 
%% Started on  Sat Jul 25 14:17:02 2009 Alex Nelson
%% Last update Sat Jul 25 14:17:02 2009 Alex Nelson
%%

The astute reader would probably have realized by now we
``implemented'' relativity in the momentum space. The question
that naturally presents itself is ``Why not try to implement
relativity in position-space, as we usually do when introducing
relativity classically?'' In this section, we'll answer that
question. 

The short answer is that it turns out to be inconsistent. We can
sketch out the general scheme and its problem in this paragraph
too. Consider putting a particle (of mass $m$) into a box whose
sides are small compared to the Compton wavelength $\lambda$,
then the uncertainty in position satisfies
\begin{equation}%\label{eq:}
\Delta x\lll\lambda
\end{equation}
and the uncertainty in momentum satisfies
\begin{equation}%\label{eq:}
\Delta p\ggg m.
\end{equation}
But this makes the range of energies so large that pair
production becomes possible. Hence, from first principles, the
position of a one-particle system is not so well defined. We'll
show (slightly more rigorously) that the notion of Lorentz
causality is violated by measuring the position operator.

We first set up the axioms for (properties satisfied by) the
position operator $\widehat{x}^{m}$. We want:
\begin{description}
\item[Axiom 1] $\widehat{x}=\widehat{x}^{\dag}$ (i.e. it's
  self-adjoint, so it has real eigenvalues);
\item[Axiom 2] If $\Delta_{a}$ is a spatial translation, then
  $U(\Delta_{a})^{\dag}\widehat{x}^{m}U(\Delta_{a}) = \widehat{x}^{m}+a^{m}$
\item[Axiom 3] If $R$ is a spatial rotation, then
  $U(R)^{\dag}\widehat{x}^{m}U(R) = {R^{m'}}_{m}\widehat{x}^{m}$.
\end{description}
From axiom 2 and $U(\Delta_{a}) = \exp(ia^{m}\widehat{P}_{m})$,
we deduce
\begin{equation}%\label{eq:}
e^{ia^{m}\widehat{p}_{m}}\widehat{x}^{n}e^{-ia^{m}\widehat{p}_{m}}=\widehat{x}^{n}+a^{n}.
\end{equation}
(Note that the sign in the exponent reflects the relationship
between the Lorentz dot product and the Euclidean dot product of
3-vectors.) Differentiating both sides with respect to the
component $a^{n}$ of $a$ then setting $a^{m}=\vec{0}$, we recover
the usual commutation relations:
\begin{equation}\label{eq:recoveryCanonicalCommutationRelations}
[i\widehat{p}_{n},\widehat{x}^{m}] = {\delta^{m}}_{n}.
\end{equation}

\begin{comment}
Although $\widehat{p}_{n}$ is unbdefined on $|x^{m}\>$, the
axioms for the position operator imply that the exponential
$\exp(-ia^{m}\widehat{p}_{m})$ must be defined on these states:
\begin{equation}%\label{eq:}
e^{-ia^{m}\widehat{p}_{m}}|x^{n}\> = |x^{n}+a^{n}\>.
\end{equation}
Also, if $\<\overline{k}|\overline{x}=\overline{0}\>=1$, then
$\<\overline{k}|\overline{a}\>=\exp(-ia^{m}k_{m})$. 
\end{comment}

\begin{rmk}
The position operator is ``essentially'' unique. That is to say,
it's unique up to unitarity. Suppose we have two operators
$\widehat{y}^{m'}$, $\widehat{x}^{m}$ that satisfy our
axioms. We'll demonstrate that there exists a unitary operator
$U$ such that $\widehat{y}^{m}=U^{\dag}\widehat{x}^{m}U$.

Assume that $\widehat{y}^{m}$ is the position operator with
respect to the basis $|\overline{k}\>$. The canonical commutation
relations eq \eqref{eq:recoveryCanonicalCommutationRelations}
shows that $\widehat{p}_{n}$ commutes with
$\widehat{x}^{n}-\widehat{y}^{n}$. Therefore, supposing any operator can
be expressed using $\widehat{x}^{m}$ and $\widehat{p}_{n}$, we
have
\begin{equation}%\label{eq:}
\widehat{y}^{m} = \widehat{x}^{m}+f^{m}(\widehat{p}).
\end{equation}
Axiom 3 however implies that $f^{m}(\widehat{p})\sim
g(\|\widehat{p}\|^{2})\widehat{p}_{m}$. This vector-valued
function of a vector has zero curl and thus may be written as the
gradient of a scalar function. Lets denote this scalar function
as $\phi(\|\widehat{p}\|^{2})$ where 
\begin{equation}%\label{eq:}
\phi(\xi)\stackrel{\text{def}}{=}\int^{\xi}_{0}g(\eta)d\eta
\end{equation}
If we define our new kets using a unitary operator $U$ to change
phases
\begin{equation}%\label{eq:}
|\overline{k}\>_{\text{new}}\stackrel{\text{def}}{=}U|\overline{k}\>\stackrel{\text{def}}{=}\exp(-i\phi(\|\overline{k}\|^{2}))|\overline{k}\>,
\end{equation}
then since
\begin{equation}%\label{eq:}
\<\psi'|\widehat{y}^{m}|\psi\> = {}_\text{new}\<\psi'|U\widehat{y}^{m}U^{\dag}|\psi\>_\text{new}
\end{equation}
the new operators are $U\widehat{y}^{m}U^{\dag}$. Writing
$U^{\dag}=e^{A}$ we find
\begin{subequations}
\begin{align}
U\widehat{y}^{m}U^{\dag} &= U\left(U^{\dag}\widehat{y}^{m}+[\widehat{y}^{m},U^{\dag}]\right)\\
&= \widehat{y}^{m} +
U[\widehat{y}^{m},1+A+\frac{1}{2}A^{2}+\cdots]\\
&= \widehat{y}^{m} + U(1+A+\frac{1}{2}A^{2}+\cdots)[\widehat{y}^{m},A]\\
&= \widehat{y}^{m} +
[\widehat{y}^{m},i\phi(\|\widehat{p}\|^{2})]\\
&= \widehat{y}^{m} - g(\|\widehat{p}\|^{2})\widehat{p}_{m}\\
&= \widehat{x}^{m}.
\end{align}
\end{subequations}
We therefore conclude that any two sets of position operators
$\widehat{x}^{m}$, $\widehat{y}^{n}$ are related by a change of
basis. We also note since $U$ is a function of the momentum
operators, the new momentum operators $U\widehat{p}_{m}U^{\dag}$
are precisely the old ones $\widehat{p}_{m}$. This shows that the
axioms determining the position operator uniquely up to a choice
of phase in the momentum eigenstates, and this concludes our remark.
\end{rmk}

The simplest inconsistency emerges when we consider a state
initially localized at the origin and see whether it can be
detected outside the forward lightcone of the origin.

Suppose we have a position operator $\widehat{x}^{m}$. Let
$|\overline{x}\>$ be a basis of position eigenstates. Then, form
our knowledge of nonrelativistic quantum mechanics, we can choose
the normalization of these kets to be such that
\begin{equation}%\label{eq:}
\<\overline{x}|\overline{k}\> = \exp(i\overline{x}\cdot\overline{k}).
\end{equation}
Now consider the evolution $|\psi\>$ of a state $|\psi_{0}\>$
initially localized at the origin:
\begin{equation}%\label{eq:}
\psi_{0}(\overline{x})\stackrel{\text{def}}{=}(2\pi)^{3}\delta^{(3)}(\overline{x})\,\Rightarrow\,\widehat{\psi}_{0}(\overline{k})=1\,\Rightarrow\,|\psi_{0}\>=\int|k\>d^{3}\overline{k},
\end{equation}
where $\widehat{\psi}_{0}$ is the Fourioer transform of
$\psi_{0}$. The evolution of this state is given by:
\begin{subequations}
\begin{align}
\psi(t,\overline{x}) &= \<\overline{x}|e^{-iHt}|\psi_{0}\>\\
&= \int \<\overline{x}|e^{-iHt}|\overline{k}\>d^{3}\overline{k}\\
&= \int
\<\overline{x}|e^{-i\omega(\overline{k})t}|\overline{k}\>d^{3}\overline{k}\\
&= \int e^{-i\omega(\overline{k})t}e^{i\overline{x}\cdot\overline{k}}d^{3}\overline{k}.
\end{align}
\end{subequations}
If the theorey is relativistic, then a state initially localized
at the origin should have zero amplitude outside the lightcone
(otherwise, there is a positive probability that something could
travel faster than light). We therefore proceed to estimate
$\psi(t,\overline{x})$ outside the light cone. Using spherical
coordinates, letting $k=\|\overline{k}\|$, $r=\|\overline{x}\|$,
we find that
\begin{subequations}
\begin{align}
\psi(t,\overline{x}) &=
\int^{1}_{-1}d(\cos\theta)\int^{2\pi}_{0}d\phi\int^{\infty}_{0}k^{2}e^{-it\sqrt{k^{2}+\mu^{2}}}e^{ikr\cos\theta}dk\\
&=\frac{2\pi}{ir}\int^{\infty}_{0}ke^{-it\sqrt{k^{2}+\mu^{2}}}(e^{ikr}-e^{-ikr})dk\\
&=\frac{2\pi}{ir}\int^{\infty}_{-\infty}ke^{-it\sqrt{k^{2}+\mu^{2}}}e^{ikr}dk.
\end{align}
\end{subequations}
We can use complex analysis to evaluate this integral when $r>t$,
we deform the contour of integration from $\mathbb{R}$ to the
first principal branch cut from $i\mu$ to $i\infty$. Substituting
$k=iz$, we find
\begin{equation}%\label{eq:}
\psi(t,\overline{x}) = \frac{4\pi i}{r}\int^{\infty}_{\mu}z\sinh(t\sqrt{z^{2}-\mu^{2}})e^{-zr}dz
\end{equation}
which is clearly nonzero.

\begin{rmk}
The integral we've been manipulating is actually divergent. This
is a consequence of the extreme nature of the initial state
$|\psi_{0}\>$. If we had started with a physical state instaead
of a position eigenstate, there would be no convergence
problem. The moral of the story is to treat integrals which arise
in such situations as defining distributions.
\end{rmk}

The outcome is that a position operator is inconsistent with
relativity. This compels us to find another way of modeling
localization of events. In field theory, we do this by making
observable operators dependent on position in spacetime.


\section{Conclusion}
We introduced a different action which is based off of Weyl's attempt
to unify gravity and electromagnetism. Instead of attempting such a
unified field theory, we observed that it has interesting
gravitational properties. 

The vacuum satisfies the Schwarzschild solution for general relativity
with a nonzero cosmological constant, plus some nonzero term and a
term linear in $r$ negligibly small at the ``local'' scale. Due to
these extra terms, the scale invariance was spontaneously broken. This
was purely accidental.

We also observed that when we solve the fourth order field equations
for the isotropic and homogeneous case, we end up breaking symmetry
again. But in doing so, we recover the standard cosmological model,
and we explained why gravity is accelerating within the framework of
the Conformal gravity model. Further, we have an effective
gravitational constant that is scale dependent which allows gravity to
be repulsive globally but (due to inhomogeneities in the scalar field)
is locally attractive. This is consistent with the first investigation
of spontaneous symmetry breaking in solving the static, spherically
symmetric body's gravitational field as locally (``for small enough
$r$'') resembling Schwarzschild's solution.

Observe that this is really nothing surprising, since this is just
another version of the Brans-Dicke theory. The Brans-Dicke action is
\begin{equation}
I = \frac{1}{16\pi}\int d^{4}x\sqrt{-g}\left(\phi R
- \omega \frac{\partial_{\mu}\phi\partial^{\mu}\phi}{\phi} + L_{matter}\right)
\end{equation}
one can rearrange it by introducing $\Phi^2=\phi$ to look like
\begin{equation}
I = \frac{1}{16\pi}\int d^{4}x\sqrt{-g}\left(\Phi^2R -
4\omega\partial_\mu\Phi\partial^{\mu}\Phi + L_{matter}\right)
\end{equation}
which resembles the action in Eq \eqref{symmetryBreakingAction}. What
the Brans-Dicke theory effectively does is replace $k=16\pi G/c^4$
with a scalar field $\phi$. We did something similar, except our
scalar field spontaneously broke the scale invariance (so,
analogously, we had a bare minimum value for $k$) which gave rise
to a cosmological constant in addition to recovering the standard
cosmological model. Further, we used covariant derivatives instead of
partial derivatives, so we would need to include in the $L_{matter}$
the extra terms, the $\Phi^4$ term, and the coupling to
matter. Nonetheless, the cosmological constant naturally emerges when
we break symmetry. 


\nocite{*}
\bibliographystyle{utcaps}
\bibliography{rqm}
\end{document}
