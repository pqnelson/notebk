%%
%% positionOperator.tex
%% 
%% Made by Alex Nelson
%% Login   <alex@tomato>
%% 
%% Started on  Sat Jul 25 14:17:02 2009 Alex Nelson
%% Last update Sat Jul 25 14:17:02 2009 Alex Nelson
%%

The astute reader would probably have realized by now we
``implemented'' relativity in the momentum space. The question
that naturally presents itself is ``Why not try to implement
relativity in position-space, as we usually do when introducing
relativity classically?'' In this section, we'll answer that
question. 

The short answer is that it turns out to be inconsistent. We can
sketch out the general scheme and its problem in this paragraph
too. Consider putting a particle (of mass $m$) into a box whose
sides are small compared to the Compton wavelength $\lambda$,
then the uncertainty in position satisfies
\begin{equation}%\label{eq:}
\Delta x<<<\lambda
\end{equation}
and the uncertainty in momentum satisfies
\begin{equation}%\label{eq:}
\Delta p>>> m.
\end{equation}
But this makes the range of energies so large that pair
production becomes possible. Hence, from first principles, the
position of a one-particle system is not so well defined. We'll
show (slightly more rigorously) that the notion of Lorentz
causality is violated by measuring the position operator.

We first set up the axioms for (properties satisfied by) the
position operator $\widehat{x}^{m}$. We want:
\begin{description}
\item[Axiom 1] $\widehat{x}=\widehat{x}^{\dag}$ (i.e. it's
  self-adjoint, so it has real eigenvalues);
\item[Axiom 2] If $\Delta_{a}$ is a spatial translation, then
  $U(\Delta_{a})^{\dag}\widehat{x}^{m}U(\Delta_{a}) = \widehat{x}^{m}+a^{m}$
\item[Axiom 3] If $R$ is a spatial rotation, then
  $U(R)^{\dag}\widehat{x}^{m}U(R) = {R^{m'}}_{m}\widehat{x}^{m}$.
\end{description}
From axiom 2 and $U(\Delta_{a}) = \exp(ia^{m}\widehat{P}_{m})$,
we deduce
\begin{equation}%\label{eq:}
e^{ia^{m}\widehat{p}_{m}}\widehat{x}^{n}e^{-ia^{m}\widehat{p}_{m}}=\widehat{x}^{n}+a^{n}.
\end{equation}
(Note that the sign in the exponent reflects the relationship
between the Lorentz dot product and the Euclidean dot product of
3-vectors.) Differentiating both sides with respect to the
component $a^{n}$ of $a$ then setting $a^{m}=\vec{0}$, we recover
the usual commutation relations:
\begin{equation}\label{eq:recoveryCanonicalCommutationRelations}
[i\widehat{p}_{n},\widehat{x}^{m}] = {\delta^{m}}_{n}.
\end{equation}

\begin{comment}
Although $\widehat{p}_{n}$ is unbdefined on $|x^{m}\>$, the
axioms for the position operator imply that the exponential
$\exp(-ia^{m}\widehat{p}_{m})$ must be defined on these states:
\begin{equation}%\label{eq:}
e^{-ia^{m}\widehat{p}_{m}}|x^{n}\> = |x^{n}+a^{n}\>.
\end{equation}
Also, if $\<\overline{k}|\overline{x}=\overline{0}\>=1$, then
$\<\overline{k}|\overline{a}\>=\exp(-ia^{m}k_{m})$. 
\end{comment}

\begin{rmk}
The position operator is ``essentially'' unique. That is to say,
it's unique up to unitarity. Suppose we have two operators
$\widehat{y}^{m'}$, $\widehat{x}^{m}$ that satisfy our
axioms. We'll demonstrate that there exists a unitary operator
$U$ such that $\widehat{y}^{m}=U^{\dag}\widehat{x}^{m}U$.

Assume that $\widehat{y}^{m}$ is the position operator with
respect to the basis $|\overline{k}\>$. The canonical commutation
relations eq \eqref{eq:recoveryCanonicalCommutationRelations}
shows that $\widehat{p}_{n}$ commutes with
$\widehat{x}^{n}-\widehat{y}^{n}$. Therefore, supposing any operator can
be expressed using $\widehat{x}^{m}$ and $\widehat{p}_{n}$, we
have
\begin{equation}%\label{eq:}
\widehat{y}^{m} = \widehat{x}^{m}+f^{m}(\widehat{p}).
\end{equation}
Axiom 3 however implies that $f^{m}(\widehat{p})\sim
g(\|\widehat{p}\|^{2})\widehat{p}_{m}$. This vector-valued
function of a vector has zero curl and thus may be written as the
gradient of a scalar function. Lets denote this scalar function
as $\phi(\|\widehat{p}\|^{2})$ where 
\begin{equation}%\label{eq:}
\phi(\xi)\stackrel{\text{def}}{=}\int^{\xi}_{0}g(\eta)d\eta
\end{equation}
If we define our new kets using a unitary operator $U$ to change
phases
\begin{equation}%\label{eq:}
|\overline{k}\>_{\text{new}}\stackrel{\text{def}}{=}U|\overline{k}\>\stackrel{\text{def}}{=}\exp(-i\phi(\|\overline{k}\|^{2}))|\overline{k}\>,
\end{equation}
then since
\begin{equation}%\label{eq:}
\<\psi'|\widehat{y}^{m}|\psi\> = {}_\text{new}\<\psi'|U\widehat{y}^{m}U^{\dag}|\psi\>_\text{new}
\end{equation}
the new operators are $U\widehat{y}^{m}U^{\dag}$. Writing
$U^{\dag}=e^{A}$ we find
\begin{subequations}
\begin{align}
U\widehat{y}^{m}U^{\dag} &= U\left(U^{\dag}\widehat{y}^{m}+[\widehat{y}^{m},U^{\dag}]\right)\\
&= \widehat{y}^{m} +
U[\widehat{y}^{m},1+A+\frac{1}{2}A^{2}+\cdots]\\
&= \widehat{y}^{m} + U(1+A+\frac{1}{2}A^{2}+\cdots)[\widehat{y}^{m},A]\\
&= \widehat{y}^{m} +
[\widehat{y}^{m},i\phi(\|\widehat{p}\|^{2})]\\
&= \widehat{y}^{m} - g(\|\widehat{p}\|^{2})\widehat{p}_{m}\\
&= \widehat{x}^{m}.
\end{align}
\end{subequations}
We therefore conclude that any two sets of position operators
$\widehat{x}^{m}$, $\widehat{y}^{n}$ are related by a change of
basis. We also note since $U$ is a function of the momentum
operators, the new momentum operators $U\widehat{p}_{m}U^{\dag}$
are precisely the old ones $\widehat{p}_{m}$. This shows that the
axioms determining the position operator uniquely up to a choice
of phase in the momentum eigenstates, and this concludes our remark.
\end{rmk}

The simplest inconsistency emerges when we consider a state
initially localized at the origin and see whether it can be
detected outside the forward lightcone of the origin.

Suppose we have a position operator $\widehat{x}^{m}$. Let
$|\overline{x}\>$ be a basis of position eigenstates. Then, form
our knowledge of nonrelativistic quantum mechanics, we can choose
the normalization of these kets to be such that
\begin{equation}%\label{eq:}
\<\overline{x}|\overline{k}\> = \exp(i\overline{x}\cdot\overline{k}).
\end{equation}
Now consider the evolution $|\psi\>$ of a state $|\psi_{0}\>$
initially localized at the origin:
\begin{equation}%\label{eq:}
\psi_{0}(\overline{x})\stackrel{\text{def}}{=}(2\pi)^{3}\delta^{(3)}(\overline{x})\,\Rightarrow\,\widehat{\psi}_{0}(\overline{k})=1\,\Rightarrow\,|\psi_{0}\>=\int|k\>d^{3}\overline{k},
\end{equation}
where $\widehat{\psi}_{0}$ is the Fourioer transform of
$\psi_{0}$. The evolution of this state is given by:
\begin{subequations}
\begin{align}
\psi(t,\overline{x}) &= \<\overline{x}|e^{-iHt}|\psi_{0}\>\\
&= \int \<\overline{x}|e^{-iHt}|\overline{k}\>d^{3}\overline{k}\\
&= \int
\<\overline{x}|e^{-i\omega(\overline{k})t}|\overline{k}\>d^{3}\overline{k}\\
&= \int e^{-i\omega(\overline{k})t}e^{i\overline{x}\cdot\overline{k}}d^{3}\overline{k}.
\end{align}
\end{subequations}
If the theorey is relativistic, then a state initially localized
at the origin should have zero amplitude outside the lightcone
(otherwise, there is a positive probability that something could
travel faster than light). We therefore proceed to estimate
$\psi(t,\overline{x})$ outside the light cone. Using spherical
coordinates, letting $k=\|\overline{k}\|$, $r=\|\overline{x}\|$,
we find that
\begin{subequations}
\begin{align}
\psi(t,\overline{x}) &=
\int^{1}_{-1}d(\cos\theta)\int^{2\pi}_{0}d\phi\int^{\infty}_{0}k^{2}e^{-it\sqrt{k^{2}+\mu^{2}}}e^{ikr\cos\theta}dk\\
&=\frac{2\pi}{ir}\int^{\infty}_{0}ke^{-it\sqrt{k^{2}+\mu^{2}}}(e^{ikr}-e^{-ikr})dk\\
&=\frac{2\pi}{ir}\int^{\infty}_{-\infty}ke^{-it\sqrt{k^{2}+\mu^{2}}}e^{ikr}dk.
\end{align}
\end{subequations}
We can use complex analysis to evaluate this integral when $r>t$,
we deform the contour of integration from $\mathbb{R}$ to the
first principal branch cut from $i\mu$ to $i\infty$. Substituting
$k=iz$, we find
\begin{equation}%\label{eq:}
\psi(t,\overline{x}) = \frac{4\pi i}{r}\int^{\infty}_{\mu}z\sinh(t\sqrt{z^{2}-\mu^{2}})e^{-zr}dz
\end{equation}
which is clearly nonzero.

\begin{rmk}
The integral we've been manipulating is actually divergent. This
is a consequence of the extreme nature of the initial state
$|\psi_{0}\>$. If we had started with a physical state instaead
of a position eigenstate, there would be no convergence
problem. The moral of the story is to treat integrals which arise
in such situations as defining distributions.
\end{rmk}

The outcome is that a position operator is inconsistent with
relativity. This compels us to find another way of modeling
localization of events. In field theory, we do this by making
observable operators dependent on position in spacetime.
