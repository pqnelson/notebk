%%
%% lecture17.tex
%% 
%% Made by alex
%% Login   <alex@tomato>
%% 
%% Started on  Wed Oct  5 09:11:26 2011 alex
%% Last update Wed Oct  5 09:11:26 2011 alex
%%
Last time, we wrote the formula
\begin{equation}
f(x)=\int^{\infty}_{0}\E^{xh(\zeta)}\D\zeta
\end{equation}
where $h(\zeta_{0})$ is a maximum and $h''(\zeta_{0})<0$. We know
from last time that
\begin{equation}
f(x)\asymptote\E^{xh(\zeta_{0})}\sqrt{\frac{\pi}{x}}\left(a_{1}+\frac{1\cdot3}{x}a_{3}+\frac{1\cdot3\cdot5}{x^{2}}a_{5}+\dots\right)
\end{equation}
We wrote
\begin{equation}
h(\zeta)=h(\zeta_{0})-\omega(\zeta)^{2}
\end{equation}
so
\begin{equation}
\omega(\zeta)=\sqrt{h(\zeta_{0})-h(\zeta)}>0
\end{equation}
since $h(\zeta_{0})$ is the maximum for $h$. We see that
$\omega(\zeta)$ is smooth, so we can express
$\zeta(\omega)=\omega^{-1}(\zeta)$, thus
\begin{equation}
\zeta(\omega)=\zeta_{0}+a_{1}\omega+a_{2}\omega^{2}+\dots.
\end{equation}
Certainly this could be applied to the $\Gamma$ function, which
is very famous.

\subsection{The Laplace Transform}

The Laplace transform is used to solve differential equations, it
is a version of the Fourier transform. It takes sums to sums and
products to convolutions.

Let
\begin{equation}
S=\{z\in\CC\lst\|z\|=1\}
\end{equation}
we have characters of a group $G$ which are maps $G\to S$, they
form a group denoted
\begin{equation}
G^{*}=\mathrm{Char}(G).
\end{equation}
If $G=S$, then characters are all raising to integral powers. So
we have $G^{*}=\ZZ$. Let $f$ be a function on $G$, $\widehat{f}$
be a function on $G^{*}$, we write
\begin{equation}
\widehat{f}=\int_{G}\chi(g)f(g)\D g
\end{equation}
What is $\D g$? It is a Haar measure, there exists some ability
to integrate functions on groups (if the group is topological,
the measure is continuous). We need to consider functions that
are integrable, if the group is noncompact (e.g. $\RR$) we could
get divergent integrals. We have $\widehat{f}(\chi)$ be the
Fourier coefficients $c_{n}$.

\begin{rmk}
What we are doing is considering the dual group $\widehat{G}$ in
the sense of Pontryagin duality.
\end{rmk}

We change now to let 
\begin{subequations}
\begin{align}
G=\RR,\\
f\colon\RR\to\CC
\end{align}
and we write
\begin{equation}
\widehat{f}(t)=\frac{1}{\sqrt{2\pi}}\int^{\infty}_{-\infty}\E^{\I tx}f(x)\D x
\end{equation}
\end{subequations}
It is the Fourier transform. If we have two functions
\begin{equation}
f_{1},f_{2}\colon\RR\to\CC
\end{equation}
their convolution is
\begin{equation}
(f_{1}*f_{2})(h)=\int_{G}f_{1}(g)f_{2}(g^{-1}h)\D g
\end{equation}
in a reasonable setting it is a sort of involution. Let us return
to discuss the Laplace transform.

Consider $f\colon(0,\infty)\to\CC$ (we can generalize any half
line), the Laplace transform is defined as
\begin{equation}
\widetilde{f}(t)=\int^{\infty}_{0}\E^{-tx}f(x)\D x
\end{equation}
The onyl condition we have on $f$ is that
\begin{equation}
\|f(x)\|<A\E^{Bx}
\end{equation}
where $A,B\in\RR$ are ``Big Constants''.

So, after all, we are doing asymptotics, if $f$ has two properties:
\begin{enumerate}
\item $f\in C^{\infty}(0,\infty)$, and
\item $\displaystyle\|f^{(n)}(x)\|<A_{n}\exp(B_{n}x)$,
\end{enumerate}
then we have the following asymptotics:
\begin{equation}
\widetilde{f}(x)\asymptote\frac{f(0)}{x}+\frac{f'(0)}{x^{2}}+\frac{f^{(2)}(0)}{x^{3}}+\dots+\frac{f^{(n-1)}(0)}{x^{n}}+\dots
\end{equation}
Observe that we do not demand its convergence since we do not
weight by factorials.
