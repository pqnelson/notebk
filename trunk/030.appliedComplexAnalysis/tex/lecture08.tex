%%
%% lecture08.tex
%% 
%% Made by alex
%% Login   <alex@tomato>
%% 
%% Started on  Sun Oct  2 16:12:40 2011 alex
%% Last update Sun Oct  2 16:12:40 2011 alex
%%

A confocal family of conics --- every point lies in a hyperbola
and an ellipse which are perpendicular to each other.

Now, let us continue considerations of the reflection
principle. If $f\colon\mathcal{U}\to\CC$ is analytic, and just to
explicitly reiterate the notion of reflection we have
\begin{equation}
\overline{\mathcal{U}}=\{\overline{z}\lst z\in\mathcal{U}\}
\end{equation}
be a reflection of $\mathcal{U}$, $f$ be real on $I$, we can
analytically continue $f$ on $\overline{\mathcal{U}}$. 

\begin{wrapfigure}[14]{r}{3.3in}
\vspace{-25pt}
\begin{center}
\includegraphics{img/lecture08.0}
\end{center}
\end{wrapfigure}
We can similarly do this on the unit circle, how? Well, we can
use one of our beloved conformal mappings $\psi$ which maps the
region $\mathcal{U}$ to a domain $\psi(\mathcal{U})$ in the unit
circle with part of its boundary $\psi(I)$ on the boundary of the
unit circle. Then $f$ can be inverted. Let the domain of the
inversion be $\mathcal{V}$. Let $i$ inverse be in the circle, $j$
inverse be in the image. So
\begin{equation}
f(z)=jf(i(z)).
\end{equation}
We may construct a fractional linear transformation $\varphi$.

We simply replace $f$ by $\psi^{-1}\circ f\circ\varphi$. We
generalize the reflection principle to circles.


\subsection*{And Now, For Something Completely Different}

\begin{wrapfigure}[5]{r}{1.7in}
\vspace{-20pt}
\begin{center}
\includegraphics{img/lecture08.1}
\end{center}
\end{wrapfigure}
The Riemann mapping theorem may be generalized to non-simply
connected regions. We can extract it from the circle reflection
principle. How to do this witchcraft? Well, suppose we have 2
discs, and inside 2 subdiscs of radii $r$ and $R$. Then we have a
conformal mapping from one to the other if $r=R$. The boundary
circles are mapped to the boundary circles; and moreover $f$ is a
rotation.

\begin{wrapfigure}[7]{r}{1.05in}
\vspace{-35pt}
\begin{center}
\includegraphics{img/lecture08.2}
\end{center}
\end{wrapfigure}
We can see from inversion as doodled to the right takes the
annulus of radially shaded domain to the angularly shaded
domain. We can iterate over and over again. We end up with
$D\setminus\{0\}$ --- the unit disc missing the origin.
Well, to be more precise, we do not invert the annulus: we invert
the inner disc of radius $r$ or $R$, we get the result doodled on
the right. We have a holomorphic map
\begin{equation}
f\colon D\setminus\{0\}\to D\setminus\{0\}
\end{equation}
which is bijective. We interpret this as mapping the center to
the center, so we have a conformal map. We notice that the only
undetermined property is how the mapping treats orientation, but
since it is conformal then all orientation is preserved. The only
such map is a rotation! Quite ingenious to say $f\colon D\to D$
such that $0\mapsto 0$.

What happens if we have a conformal map that is not concentric
(i.e., the domain is not a concentric annulus)? It becomes more
of a challenge to prove that the domains are conformal. Some
examples of these would look like:
\begin{center}
\includegraphics{img/lecture08.3}
\end{center}

\subsection{Argument Principle}
We will consider several theorems whose name change but results
remain the same. (I believe Led Zeppelin had a song with this
title!) Suppose we have some domain, and we have some function
that is not zero on the boundary of the domain and there are no
poles on the boundary. So all poles are inside the domain. We
have the function be meromorphic. We compute:
\begin{equation}
\begin{split}
Z &= \mbox{number of zeroes}\\
P &= \mbox{number of poles}
\end{split}
\end{equation}
If we travel $f$ along the boundary, when we return to our
departure point $\|f\|$ is the same but the argument differs by
$2\pi k$ ($k\in\ZZ$).

\begin{wrapfigure}[5]{r}{0.75in}
\vspace{-30pt}
\begin{center}
\includegraphics{img/lecture08.4}
\end{center}
\end{wrapfigure}
We taake the unit disc, the function $f(z)=z$. We start at 1. We
travel along the boundary and when we get back to the point of
departure we find
\begin{equation}
\Delta\arg(f(z))=2\pi\cdot1.
\end{equation}
In general we let $\gamma$ be the boundary of the domain. We
have, then, in general
\begin{equation}\label{eq:lecture08:moveAntiClockwise}
\Delta_{\gamma}\arg(f(z))=2\pi(Z-P)
\end{equation}
If we know the behavior of the function on the boundary, we have
a good clue to the number of zeroes. Also note, if we were to move
\emph{clockwise} then our formula changes to
\begin{equation}\label{eq:lecture08:moveClockwise}
\Delta_{\gamma}\arg(f(z))=2\pi(P-Z)
\end{equation}
instead. Note the difference between eq
\eqref{eq:lecture08:moveAntiClockwise} and
\eqref{eq:lecture08:moveClockwise}. 

\begin{thm}
Suppose we have a meromorphic function
$f\colon\mathcal{U}\to\CC$, and $\mathcal{U}$ is our domain, and
also suppose $f$ has zeroes and poles. Consider a closed curve $\gamma$ in
this domain not passing through zeroes or poles. Then
\begin{equation}
\Delta_{\gamma}\arg(f(z))=2\pi(Z-P)
\end{equation}
holds for our oriented closed curve $\gamma$.
\end{thm}
\begin{thm}
We have
\begin{equation}
\int_{\gamma}\frac{f'(z)}{f(z)}\D z=2\pi\I\left(
\sum_{{\text{zeroes }a}}I(\gamma,a)-\sum_{\mathclap{\text{poles }b}}I(\gamma,b)
\right)
\end{equation}
Observe that this integral is the same as
\begin{equation}
\int_{\gamma}\frac{f'(z)}{f(z)}\D z=\int_{\gamma}\D\left(\log\left(f(z)\right)\right).
\end{equation}
\end{thm}
\begin{rmk}
This second theorem implies the first. The second theorem follows
directly from the residue theorem.
\end{rmk}
