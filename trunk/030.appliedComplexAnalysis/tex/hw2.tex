%%
%% hw2.tex
%% 
%% Made by alex
%% Login   <alex@tomato>
%% 
%% Started on  Mon Oct  3 09:39:20 2011 alex
%% Last update Mon Oct  3 09:39:20 2011 alex
%%

\section*{Homework 2}\renewcommand{\leftmark}{Homework 2}\phantomsection\addcontentsline{toc}{section}{Homework 2}
\begin{exercise}
Let ${\cal U}$ be the domain obtained from $\CC$ by deleting all
real $x$ with $|x|\geq1$. Describe (explicitly) a conformal map
of ${\cal U}$ onto the upper half-plane. 
\end{exercise}
\begin{exercise}
Prove the following result of Karl Weierstrass. Let 
\begin{equation*}
f(z)=\sum^{\infty}_{n=0}z^{n!}
\end{equation*}
Then $f$ cannot be analytically continued to \emph{any} open set
properly containing
\begin{equation*}
A=\{z\in\CC\lst\|z\|<1\}.
\end{equation*}
\emph{Hint}: First consider $z=r\exp(\I 2\pi p/q)$ where $p$ and
$q$ are integers.
\end{exercise}
\begin{exercise}
Describe, in terms of cutting and pasting, the Riemann surface of the function
$w = w(z)$ given by the implicit formula $w^{3} - w + z = 0$ (you
can use the algebraic fact that a complex number a is a multiple
root of an algebraic equation $p(x) = 0$ if and only if $p(a) =
p'(a) = 0$). Try to generalize to $w^{n} - w + z = 0$. 
\end{exercise}
\begin{exercise}
Let f be a conformal map of the half-disk $D^{+} = \{z\in\CC \lst
\|z\| < 1, \im(z) > 0\}$ 
onto itself which can be extended to a continuous map of $D^{+}\cup (−1, 1)$ onto itself. Prove that
\begin{equation*}
f (z) = \frac{z-a}{1-az}
\end{equation*}
for some $a\in (-1, 1)$. (Hint: Schwarz's Reflection Principle.)
(By the way, is the extendability condition really needed? The
answer can be deduced immediately from the Riemann mapping
theorem.) 
\end{exercise}
\begin{exercise}
Let $d_{1}$, $d_{2}$ be two closed discs contained in the open
unit disk $D$. Prove that any conformal map
$D\setminus\{d_{1}\}\to D\setminus\{d_{2}\}$ can be extended to a
fractional linear map. Try to deduce a condition on $d_{1}$,
$d_{2}$ under which such $f$ can exist. (Hint: the circular
Schwarz Reflection Principle, Page 370.) 
\end{exercise}
