%%
%% lecture12.tex
%% 
%% Made by alex
%% Login   <alex@tomato>
%% 
%% Started on  Mon Oct  3 12:19:17 2011 alex
%% Last update Mon Oct  3 12:19:17 2011 alex
%%
So recall our theorem
\begin{thm}
If $a_{1}$, \dots, is a sequence where $a_{i}\not=0$ for every
$i$, and $\sum\|a_{n}\|^{-2}$ converges, then 
\begin{equation}
f(z) = \prod_{n}\left(1-\frac{z}{a_{n}}\right)\E^{z/a_{n}}
\end{equation}
converges and is entire with zeroes at $a_{1}$, \dots.
\end{thm}
We know
\begin{equation}
\sin(z)=z\prod^{\infty}_{\substack{n=-\infty\\n\not=0}}\left(1-\frac{z}{\pi n}\right)
\E^{z/(\pi n)}
\end{equation}
We collect terms
\begin{subequations}
\begin{align}
\sin(z)&=z\prod^{\infty}_{n=1}\left(1-\frac{z}{\pi n}\right)\left(1+\frac{z}{\pi n}\right)
\E^{z/(\pi n)}\E^{z/(-\pi n)}\\
&=z\prod^{\infty}_{n=1}\left(1-\frac{z^{2}}{\pi^{2}n^{2}}\right)
\end{align}
\end{subequations}
If we wrote
\begin{equation}
f(z) = \prod_{n\not=0}\left(1-\frac{z}{\pi n}\right)
\end{equation}
instead, then $f(z)$ diverges entirely. This problem is similar
to how
\begin{equation}
\sum(-1)^{n}=\left(\sum 1\right)+\left(\sum -1\right)
\end{equation}
diverges but
\begin{equation}
\sum(-1)^{n}=\sum (1-1)=0
\end{equation}
converges. So if we disregard the contribution of $\exp[z/(\pi
  n)]$, the series diverges very much as the harmonic series diverges.
Consider
\begin{equation}
\sum_{n\not=0}\frac{1}{z+n}
\end{equation}
which diverges, but
\begin{equation}
\sum_{n\not=0}\left(\frac{1}{z+n}-\frac{1}{n}\right)
\end{equation}
converges! But
\begin{equation}
\sum_{n\not=0}\frac{1}{n}=0
\end{equation}
which changes nothing.

\subsection{Gamma Function}
We will begin with examining the $\Gamma$ function. What do we
know about it? Well,
\begin{equation}
\Gamma(n+1)=n!
\end{equation}
and $\Gamma(-n)$ are singular (simple poles really). We also know
\begin{equation}
\Gamma(\mu+1)=\mu\Gamma(\mu)
\end{equation}
There is a lot of information on the $\Gamma$ function in
Marsden~\cite{marsden}. 

Consider its definition: the $\Gamma$ function is defined by
\begin{equation}\label{eq:lecture12:defnOfGammaFn}
\Gamma(\mu)=\int^{\infty}_{0}x^{\mu-1}\E^{-x}\D x
\end{equation}
Consider some $x\in\RR$, $\mu\in\CC$. But are these arbitrary?
When will the integral be defined? Well, lets consider
\begin{subequations}
\begin{align}
x^{a+\I b} &= x^{a}x^{\I b}\\
&=x^{a}\E^{\I b\log(x)}\\
&=x^{a}\E^{\I\log(x^{b})}
\end{align}
\end{subequations}
Note that $\|x^{\I b}\|=1$ for $a,b\in\RR$. We are not interested
in how our integral behaves near $x\to\infty$, since $e^{-x}$
wins out (i.e., vanishes faster than $x^{\mu-1}$ explodes). So we
are interested in the behavior of the integrand near zero. The
condition is that $\re(\mu)>0$, the integral is defined.

\begin{wrapfigure}{r}{2.75in}
\vspace{-30pt}
\begin{center}
\includegraphics{img/lecture12.0}
\end{center}
\vspace{-20pt}
\end{wrapfigure}
We need to employ our favorite phrase: analytic continuation. We
see that when $\mu=0$, the integral diverges. Similarly for
$\mu=-1$, $-2$, \dots, the integral diverges too.

There are poles but no zeroes, and for this reason it is not
entire; but its inverse $1/\Gamma(z)$ is entire.

We will consider the inverse and bring in the infinite product
for the sine. Why? Well, we have zeroes for $G(z)=1/\Gamma(z)$,
which are $\NN$. They are all multiplicity zero, so we'll use
$\NN$ to indicate the zeroes. This is a little bit sloppy, but
meh, that's life. Now we know how to construct the product for
$G(z)$ since we know its zeroes. We construct it explicitly as
\begin{equation}
G(z)=\prod^{\infty}_{n=1}\left(1+\frac{z}{n}\right)\E^{-z/n}
\end{equation}
observe $G(0)=1$. We know the relation between $G(z)$ and
$\sin(z)$ by the product series. Indeed, observe
\begin{equation}
\sin(\pi z)=\pi zG(z)G(-z).
\end{equation}
Lets see to what extent $\Gamma(\mu+1)\mu\Gamma(\mu)$ is
preserved in $G(z)$, i.e. $G(z-1)=H(z)$. We see by inspecting the
zeroes of $H(z)$ that
\begin{equation}
H(z)=\E^{g(z)}z G(z)
\end{equation}
we have some unknown factor $g(z)$.
\begin{thm}
The factor $g(z)$ is a constant denoted as $\gamma$, shorthand
for
\begin{equation}
\lim_{N\to\infty}\left(\sum^{N}_{n=1}\frac{1}{n}-\log(N)\right)
\end{equation}
This is certainly one definition of $\gamma$. (This is drawn above
to the right.)
\end{thm}
Note that numerically,
\begin{equation}
\gamma\approx0.57721\; 56649\; 01532\; 86060
\end{equation}
We will now define the $\Gamma$ function as
\begin{equation}
\Gamma(z)=\left[z\E^{\gamma z}G(z)\right]^{-1}
\end{equation}
We need to prove that $g(z)=\gamma$ is constant though. We do the
following:
\begin{equation}
\frac{\D}{\D z}\log(H(z))=\frac{\D}{\D z}\log(G(z-1))
\end{equation}
We observe
\begin{subequations}
\begin{align}
\log(H(z)) &= g(z) +\log(z) + \log(G(z))\\
&= g(z) + \log(z) +
\sum^{\infty}_{n=1}\log\left(1+\frac{z}{n}\right)-\frac{z}{n}
\end{align}
\end{subequations}
We see
\begin{subequations}
\begin{align}
\frac{\D}{\D z}\log(H(z)) &=\frac{\D}{\D z}\left(g(z) + \log(z) +
\sum^{\infty}_{n=1}\log\left(1+\frac{z}{n}\right)-\frac{z}{n}\right)\\
&=g'(z)+\frac{1}{z}+\sum^{\infty}_{n=1}\left(\frac{1}{z+n}-\frac{1}{n}\right).
\end{align}
\end{subequations}
Now we consider calculations involving $G(z)$ thus
\begin{subequations}
\begin{align}
\log(G(z-1)) &=
\log(\prod^{\infty}_{n=1}\left(1+\frac{z-1}{n}\right)\E^{(1-z)/n})\\
&=\sum^{\infty}_{n=1}\left(\log\left(1+\frac{z-1}{n}\right)-\frac{z-1}{n}\right)\\
\implies\frac{\D}{\D z}\log(G(z-1)) &=
\sum^{\infty}_{n=1}\left(\frac{1}{z+n-1}-\frac{1}{n}\right)\\
&=
\frac{1}{z}+\sum^{\infty}_{n=1}\left(\frac{1}{z+n}-\frac{1}{n}\right)
\end{align}
\end{subequations}
This implies that
\begin{equation}
g'(z)=0\implies g(z)=\mbox{(constant)}
\end{equation}
which concludes the proof of the theorem.

How to find the value of $g(z)$? Take $H(z)$ and set $z=1$, so
\begin{equation}
H(1)=e^{g(1)}G(1)
\end{equation}
which gives us a new problem: what is $G(1)$? Well,
\begin{equation}
G(1)=\lim_{N\to\infty}\prod^{N}_{n=1}\left(1+\frac{1}{n}\right)\E^{-1/n}
\end{equation}
Observe that
\begin{equation}
\prod^{N}_{n=1}\left(1+\frac{1}{n}\right)=N+1
\end{equation}
which means
\begin{equation}
G(1)=\lim_{N\to\infty}(N+1)\E^{-\sum^{N}_{n=1}(1/n)}
\end{equation}
So
\begin{subequations}
\begin{align}
\log(G(1)) &=
\lim_{N\to\infty}\log(N+1)-\sum^{N}_{n=1}\frac{1}{n}\\
&=-\gamma.
\end{align}
\end{subequations}
This concludes this lecture.
