%%
%% lecture19.tex
%% 
%% Made by alex
%% Login   <alex@tomato>
%% 
%% Started on  Wed Oct  5 11:02:21 2011 alex
%% Last update Wed Oct  5 11:02:21 2011 alex
%%

\begin{wrapfigure}{l}{0.9in}
\vspace{-24pt}
\begin{center}
\includegraphics{img/lecture19.0}
\end{center}
\end{wrapfigure}
\noindent{}Lets continue on the Laplace transform. Consider
\begin{equation}
f\colon(0,\infty)\to\RR.
\end{equation}
The expression 
\begin{equation}
\widetilde{f}(z)=\int^{\infty}_{0}\E^{-tz}f(t)\D t
\end{equation}
it converges for mild conditions on $f$, namely it should
satisfy
\begin{equation}
\|f(t)\|<A\exp(Bt)
\end{equation}
where $A,B\in\RR\setminus\{0\}$. We demand
\begin{equation}
\re(z)>B
\end{equation}
and the imaginary part of $z$ won't affect convergence. Then we
are working with the half of the complex plane doodled to the left.

Now last time we considered for $f\in C^{\infty}(0,\infty)$ and
\begin{equation}
\|f^{(n)}(t)\|\leq A_{n}\exp(B_{n}t)
\end{equation}
the asymptotics for its Laplace transform is
\begin{equation}
\widetilde{f}(x)\asymptote\frac{1}{x}f(0)+\frac{1}{x^{2}}f'(0)+\dots+\frac{1}{x^{n+1}}f^{(n)}(0)+\frac{1}{x^{n+1}}\int^{\infty}_{0}\E^{-xt}f^{(n+1)}(t)\D{}t.
\end{equation}
If $g=f'$, and we know $\widetilde{f}$, how can we express
$\widetilde{g}$? Well, we go back to the definition of the
Laplace transform
\begin{subequations}
\begin{align}
\widetilde{g}(z) &= \int^{\infty}_{0}\E^{-tz}g(t)\D t\\
&=\int^{\infty}_{0}\E^{-tz}f'(t)\D t\\
\intertext{integrate by parts}
&=\E^{-tz}f(z)|^{\infty}_{0}-\int^{\infty}_{0}f(t)\left(\frac{\D}{\D{}z}\E^{-tz}\right)\D{}t\\
&=0-f(0)+z\int^{\infty}_{0}f(t)\E^{-tz}\D t\\
&=-f(0)+z\widetilde{f}(z)
\end{align}
\end{subequations}
This is the relation between $\widetilde{f}(z)$ and $\widetilde{g}(z)$.

Consider
\begin{equation}
g(t)=f(t)\E^{-at}
\end{equation}
for some $a\in\CC$. This is Laplace transformed into a shift
\begin{subequations}
\begin{align}
\widetilde{g}(z) &=\widetilde{f}(z+a)\\
&=\int^{\infty}_{0}\E^{-tz}\E^{-at}f(t)\D t\\
&=\int^{\infty}_{0}\E^{-t(z+a)}f(t)\D t\\
\intertext{let $\widetilde{z}=z+a$}
&=\int^{\infty}_{0}\E^{-t\widetilde{z}}f(t)\D t\\
&=\widetilde{f}(\widetilde{z})=\widetilde{f}(z+a)
\end{align}
\end{subequations}
Consider
\begin{equation}
g(t)=\begin{cases}0 & \mbox{if }t\leq-a\\
f(t+a)&\mbox{if }t\geq-a
\end{cases}
\end{equation}
we see then that
\begin{equation}
\widetilde{g}(z)=\widetilde{f}(z)\E^{-az}
\end{equation}
If we have $h(t)$ defined such that
\begin{equation}
\widetilde{h}(z)=\widetilde{f}(z)\widetilde{g}(z)
\end{equation}
What $h(t)$ could do this? A convolution of $f$ and $g$, that is
\begin{equation}
h(t)=(f*g)(t)=\int^{\infty}_{0}f(t-\tau)g(\tau)\D\tau
\end{equation}
Suppose that $g$ is bounded and 
\begin{equation}
f(x)=\frac{1}{\varepsilon}
\end{equation}
when $x\in(-\varepsilon,\varepsilon)$ for ``small'' $0<\varepsilon\lll1$.
Then we have
\begin{subequations}
\begin{align}
\int^{\infty}_{0}f(t-\tau)g(\tau)\D\tau
&=\int^{t+\varepsilon}_{t-\varepsilon}f(t-\tau)g(\tau)\D\tau\\
&\approx\frac{1}{2\varepsilon}(g(t+\varepsilon)+g(t-\varepsilon))\\
&\approx g(t)\mbox{ as }\varepsilon\to0.
\end{align}
\end{subequations}
That concludes today's lecture.
