%%
%% Lagrangian.tex
%% 
%% Made by alex
%% Login   <alex@tomato>
%% 
%% Started on  Sat Nov 22 13:34:10 2008 alex
%% Last update Sat Nov 29 14:55:45 2008 Alex Nelson
%%
\section{Starting Point}

When generalizing gravity, the usual departing point is to start adding
extra terms to general relativity which vanish in the conditions of
the solar system (so they ``piggy-back'' off the predictions of
general relativity). Conformal gravity begins with a novel approach.
The starting point for the conformal gravity approach is to use a
completely different, higher order Lagrangian~\cite{Mannheim:1994ph}\cite{Mannheim:2007ki}
\begin{equation}
I_{W} = -\alpha_g\int
d^4x\sqrt{-g}C_{\lambda\mu\nu\kappa}C^{\lambda\mu\nu\kappa}
\end{equation}
where $C_{\lambda\mu\nu\kappa}$ is the Weyl tensor, and $\alpha_g$ is
a dimensionless coupling constant. Historically, this was first
proposed by Hermann Weyl in an attempt to unify electromagnetism with
gravity. Weyl wanted to use $\alpha_{g}(x)$ for both conformal
transformations and the electromagnetic gauge transformations. The
problem with such a proposal is that conformal invariance implied all
particles had to be massless. Mannheim~\cite{Mannheim:1994ph} has
proposed using the mechanism of spontaneous symmetry breaking (for a
quick review, see the appendix) to give masses to the particles.

Observe that the so-called
``Kretschmann scalar''
\begin{equation}
K = R_{\mu\nu\rho\sigma}R^{\mu\nu\rho\sigma}
\end{equation}
allows us to see 
\begin{equation}
C_{\mu\nu\rho\sigma}C^{\mu\nu\rho\sigma} 
= K-
2R_{\mu\nu}R^{\mu\nu}+
\frac{1}{3}R^{2}
\end{equation}
in four dimensions.

The field equation becomes
\begin{equation}\label{conformFieldEqn}
4\alpha_{g}W^{\mu\nu} = T^{\mu\nu}
\end{equation}
where $W^{\mu\nu}$ is given by
\begin{eqnarray}
W^{\mu \nu}&= &
\frac{1}{2}g^{\mu\nu}(R^{\alpha}_{\phantom{\alpha}\alpha})   
^{;\beta}_{\phantom{;\beta};\beta}+
R^{\mu\nu;\beta}_{\phantom{\mu\nu;\beta};\beta}                     
 -R^{\mu\beta;\nu}_{\phantom{\mu\beta;\nu};\beta}                        
-R^{\nu \beta;\mu}_{\phantom{\nu \beta;\mu};\beta}                          
 - 2R^{\mu\beta}R^{\nu}_{\phantom{\nu}\beta}                                    
+\frac{1}{2}g^{\mu\nu}R_{\alpha\beta}R^{\alpha\beta}
\nonumber \\
&&-\frac{2}{3}g^{\mu\nu}(R^{\alpha}_{\phantom{\alpha}\alpha})          
^{;\beta}_{\phantom{;\beta};\beta}                                              
+\frac{2}{3}(R^{\alpha}_{\phantom{\alpha}\alpha})^{;\mu;\nu}                           
+\frac{2}{3} R^{\alpha}_{\phantom{\alpha}\alpha}
R^{\mu\nu}                              
-\frac{1}{6}g^{\mu\nu}(R^{\alpha}_{\phantom{\alpha}\alpha})^2.
\label{3}
\end{eqnarray}    
Observe that this is a function of $R={R^{\alpha}}_{\alpha}$ and
$R_{\alpha\beta}$. There is an important no-go theorem --
Ostrogradski's theorem -- (see \S 2 of
Woodard~\cite{Woodard:2006nt}) that says (in a nutshell): If a given system
is described by a Lagrangian that has terms involving time derivatives of
second order or higher, then it is also unstable. Observe that our
Lagrangian is fourth order and depends on \emph{up to fourth order
  time derivatives of the metric tensor!} 

One can show, however, that $R_{\mu\nu} = 0$ is an exact exterior
solution to Eq (\ref{conformFieldEqn}). So we have the Schwarzschild
solution to this new field equation too.
