%%
%% nonrelativisticSymmetryBreaking.tex
%% 
%% Made by Alex Nelson
%% Login   <alex@tomato>
%% 
%% Started on  Sun Dec  7 14:49:33 2008 Alex Nelson
%% Last update Sun Dec  7 16:38:38 2008 Alex Nelson
%%

Mannheim notes~\cite{Mannheim:1993rs} that from the fourth order
Poisson equation
\begin{equation}
\nabla^{4}B(r) = f(r)
\end{equation} 
where for a spherical source its exact exterior solution is
\begin{equation}
B(r>R) = \frac{-r}{2}\int^{R}_{0}d\rho f(\rho)\rho^2 -
\frac{1}{6r}\int^{R}_{0}d\rho f(\rho)\rho^4
\end{equation}
we can use the non-relativistic potential
\begin{equation}
V(r) = \frac{-\beta}{r} + \frac{\gamma}{2}r.
\end{equation}
(Numerically the $\beta\gamma$ term is negligible.) It turns
out~\cite{Mannheim:1994ph} that we can write the spherically
symmetric source function as
\begin{equation}
f(r) = \frac{3({T^{0}}_{0} - {T^{r}}_{r})}{4\alpha B(r)}
\end{equation}
which allows us to see, since the radial piece of
$\nabla^{4}B(r)$ is $(rB)''''/r$, that the exterior metric
\eqref{exteriorMetric} emerges as the most general solution to
the fourth order Laplace equation $\nabla^4 B(r)=0$ More
importantly, this lets us write
\begin{equation}
\gamma = \frac{-1}{2}\int^{R}_{0}d\rho f(\rho) \rho^2
\end{equation}
and
\begin{equation}
\beta(2-3\beta\gamma) = \frac{1}{6}\int^{R}_{0}d\rho
f(\rho)\rho^{4}
\end{equation}
which means that $\gamma$ is some negative constant and $\beta$
is some positive constant for a body with positive $f(r)$ (it
also has to be positive so the potential contains the correct
Newtonian limit).

We see then that
\begin{equation}
\frac{d}{dr}V(r) = \frac{\beta}{r^2} + \frac{\gamma}{2}
\end{equation}
has extrema at
\begin{equation}
r_{0} = \pm\sqrt{\frac{-2\beta}{\gamma}}
\end{equation}
which is nonzero for massive bodies. (Also, don't be fooled by
the negative in the squareroot, $\gamma$ is negative so the two
become positive thus $r_0\in\mathbb{R}$.) This is an indicator
that symmetry is spontaneously broken even nonrelativistically!
