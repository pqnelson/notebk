%%
%% lecture05.tex
%% 
%% Made by alex
%% Login   <alex@tomato>
%% 
%% Started on  Sat Oct  1 16:47:46 2011 alex
%% Last update Sat Oct  1 16:47:46 2011 alex
%%

\begin{wrapfigure}{r}{1.75in}
\vspace{-40pt}
\begin{center}
\includegraphics{img/lecture05.0}
\end{center}
\vspace{-20pt}
\end{wrapfigure}
Applications of conformal maps to partial differential equations
will not be covered. Instead we will skip ahead to ANALYTIC
CONTINUATION! We have a domain $\mathcal{U}$ and a domain
$\mathcal{V}$. We have a function $f\colon\mathcal{U}\to\CC$. 

We wish to extend $f$ to $g\colon\mathcal{V}\to\CC$. That is,
when we restrict $g$ to $\mathcal{U}$ we recover $f$. The natural
questions that arise are: does such an extension exist, and if so
is it unique?

The famous example is the\marginpar{$\zeta$ function;\\
  Riemann conjecture} Riemann zeta conjecture: the real part
of the nontrivial zeroes of the zeta function is $1/2$, where the
zeta function is 
\begin{equation}
\zeta(z)=\sum_{n=1}n^{-z}. 
\end{equation}
Note it converges for $\re(z)>1$. Also observe for $z=1$ we have
the divergent Harmonic series.

Consider the\marginpar{Gamma function} Gamma function
\begin{equation}
\Gamma(\mu)=\int^{\infty}_{0}x^{\mu-1}\E^{-x}\D x,
\end{equation}
if $\mu<0$ we are in trouble. We will consider
$\re(\mu)>0$. Using this, we can express the zeta function as
\begin{equation}
\zeta(\mu)=\frac{1}{\Gamma(\mu)}\int^{\infty}_{0}\frac{x^{\mu-1}}{\E^{x}-1}\D
x
\end{equation}
Nifty!

\begin{lem}
Let $f,g\colon\mathcal{U}\to\CC$ (where $\mathcal{U}$ is
connected) be analytic. Let
$\mathcal{V}\propersubset\mathcal{U}$,
$f|_{\mathcal{V}}=g|_{\mathcal{V}}$, and $\mathcal{V}$ be open
and nonempty. Then $f=g$.
\end{lem}
\begin{lem}[``Sublemma'']
For the same $f,g,\mathcal{U}$. Suppose that we have a sequence
$z_{i}\in\mathcal{U}$, $z_{i}\not=z_{j}$ if $i\not=j$, suppose 
\begin{equation}
\lim_{i\to\infty}z_{i}=z_{0}\in\mathcal{U}.
\end{equation}
If $f(z_{i})=g(z_{i})$ for all $i\in\NN$, then $f=g$.
\end{lem}
\begin{proof}[Proof of Sublemma.]
Let $f-g=: h$. Then $h$ is also analytic. We take
\begin{equation}
h(z)=a(z-z_{0})^{k}+\dots
\end{equation}
then it follows
\begin{equation}
|a-\varepsilon|\cdot\|z-z_{0}\|^{k}<\|h(z)\|<|a+\varepsilon|\cdot\|z-z_{0}\|^{k}
\end{equation}
in a small neighborhood of $z_{0}$. So $\|h(z)\|>0$, a
contradiction, $h(z)\not=0$ in the neighborhood with the point
$z_{0}$ removed.
\end{proof}
\begin{proof}[Proof of Lemma.]
Let 
\begin{equation}
\mathcal{V}=\{z_{0}\in\mathcal{U}\lst f(z)=g(z)\mbox{ in some
  neighborhood of }z_{0}\}.
\end{equation}
Well then, $\mathcal{V}$ is open (it's obvious), and it follows
from the sublemma that $\mathcal{V}$ is closed: it contains its
boundary points. So either $\mathcal{V}$ is $\mathcal{U}$ or
$\emptyset$, but by hypothesis $\mathcal{V}\not=\emptyset$.
\end{proof}

\begin{wrapfigure}{r}{2in}
\vspace{-30pt}
\begin{center}
\includegraphics{img/lecture05.1}
\end{center}
\vspace{-20pt}
\end{wrapfigure}
Suppose we have two functions $f\colon\mathcal{U}\to\CC$ and
$g\colon\mathcal{V}\to\CC$. We demand that
$\mathcal{U}\cap\mathcal{V}\not=\emptyset$, we do not demand
simply connectedness. Suppose also that the functions agree on
the overlap, i.e.\ $f|_{\mathcal{U}\cap\mathcal{V}}=g|_{\mathcal{U}\cap\mathcal{V}}$.
Now if we ``combine'' these two functions, we get an analytic
function in $\mathcal{U}\cap\mathcal{V}$.

Let
\begin{equation}
h(z)=\begin{cases} f(z) & \mbox{if }z\in\mathcal{U}\\
g(z) & z\in\mathcal{V}
\end{cases}
\end{equation}
Our lemma implies if $\mathcal{U}$, $\mathcal{V}$ are connected,
then $f$ determines $g$.

\begin{wrapfigure}{l}{2in}
\vspace{-20pt}
\begin{center}
\includegraphics{img/lecture05.2}
\end{center}
\vspace{-20pt}
\end{wrapfigure}
Suppose again we have some function $f\colon\mathcal{U}\to\CC$
and let $z_{0}\in\partial\mathcal{U}$. We wish to define a function on a
small neighborhood $B_{\varepsilon}(z_{0})$ of $z_{0}$ such that
$g\colon B_{\varepsilon}(z_{0})\to\CC$ such that it agrees with
$f$ on the overlap:
$f|_{B\cap\mathcal{U}}=g|_{B\cap\mathcal{U}}$. If no such $g$
exists, it is impossible to extend $f$ to any
$\mathcal{V}=\mathcal{U}\cup B_{\varepsilon}(z_{0})$. On the
other hand, if $g$ exists, we can extend $f$ to this new domain. 

If $f$ is so badly behaved that $f$ is not defined on the
boundary of $\mathcal{U}$, it cannot be extended at all.

\begin{ex}
Consider the function
\begin{equation}
f(z)=\sum^{\infty}_{n=0}z^{n!}
\end{equation}
it is defined on the unit disc in $\CC$, it is analytic, but it
cannot be extended and it is ill defined on the boundary.
\end{ex}
\begin{rmk}
For several complex variables, we cannot abuse any knowledge of a
single complex variable concept.
\end{rmk}

Of course any domain of interest is conformally equivalent to
the unit disc, and this interesting open domain will (by the
Riemann mapping theorem) always have a nonextendable function on
the boundary. 

\begin{wrapfigure}{l}{2in}
\vspace{-20pt}
\begin{center}
\includegraphics{img/lecture05.3}
\end{center}
\vspace{-30pt}
\end{wrapfigure}
\noindent{}Consider the function $f(z)=\sqrt{z}$, then
$f(z)^{2}=z$. If we extend it on some patchwork, we go around
this patch work and end up in the shaded region on our original
domain. This may be seen as doodled on the left. However, the
resulting extension \emph{is not our beloved $f$!} We do not end
up with a function with a domain in the plane, no! We see this is
a Riemann surface!

\medskip
\noindent\textit{Aside.\enspace} Riemann surfaces are fairly
interesting as a subject. Note that for the most part, we will be
working with discs (which are ``the same'' as the upper half
complex plane). They are homotopically equivalent to spheres with
finitely many punctures. So, in short, all we care about is how
do we glue together the discs from analytically continuing a
given function? Provided, of course, that it does not ``close''
under continuation.
