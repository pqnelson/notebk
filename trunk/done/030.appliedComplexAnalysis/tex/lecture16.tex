%%
%% lecture16.tex
%% 
%% Made by alex
%% Login   <alex@tomato>
%% 
%% Started on  Tue Oct  4 14:25:21 2011 alex
%% Last update Tue Oct  4 14:25:21 2011 alex
%%
\begin{wrapfigure}[14]{l}{1.85in}
\vspace{-24pt}
\begin{center}
\includegraphics{img/lecture16.0}
\end{center}
\end{wrapfigure}
\noindent{}We consider the function
\begin{equation}
f(x)=\int^{\infty}_{0}\E^{x h(\zeta)}\D\zeta
\end{equation}
and we want to consider it for big values of $x$. We see that for
any $\zeta\not=\zeta_{0}$ that
\begin{equation}
\|h(\zeta_{0})-h(\zeta)\|=\alpha
\end{equation}
and since $h(\zeta_{0})$ is the location of the maxima of $h$ we
see
\begin{equation}
h(\zeta_{0})-h(\zeta)=-\alpha.
\end{equation}
Thus we have
\begin{equation}
\E^{xh(\zeta)}\approx\E^{-x\alpha}\approx0
\end{equation}
for big $x$. We see that
\begin{equation}
f(x)=\int^{\infty}_{0}\E^{xh(\zeta)}\D\zeta\expequiv\int^{\zeta_{0}+\varepsilon}_{\zeta_{0}-\varepsilon}\E^{xh(\zeta)}\D\zeta
\end{equation}
which is an exponential equivalence, \emph{not} an
``asymptotically behaves as''. The \define{Exponential
  Equivalence} of $f$ and $g$ means that they have the same
asymptotics, it is considered to be ``very strong''.

\begin{notation}
We will indicate $f$ is exponentially equivalent to $g$ by the
notation
\begin{equation}
f\expequiv g.
\end{equation}
It is odd notation, but it is \emph{our} odd notation!
\end{notation}

Now, we have assumed that $h$ has an extrema at $\zeta_{0}$, so
let us Taylor expand about $\zeta_{0}$
\begin{equation}
h(\zeta)=h(\zeta_{0})+\frac{1}{2}h''(\zeta_{0})\cdot(\zeta-\zeta_{0})^{2}+\dots,
\end{equation}
from just our simple Taylor expansion. We then have
\begin{subequations}
\begin{equation}
f\expequiv
\int^{\zeta_{0}+\varepsilon}_{\zeta_{0}-\varepsilon}\E^{xh(\zeta)}\D\zeta
\end{equation}
as we have discussed, so we change variables $\zeta=\zeta_{0}+z$
\begin{equation}
\int^{\zeta_{0}+\varepsilon}_{\zeta_{0}-\varepsilon}\E^{xh(\zeta)}\D\zeta
\asymptote
\int^{\varepsilon}_{\varepsilon}\E^{xh(\zeta_{0}+z)}\D z
\end{equation}
and then plug in the Taylor expansion of $h$ about $\zeta_{0}$
(up to second order --- additional orders would be negligible):
\begin{equation}
\int^{\varepsilon}_{\varepsilon}\E^{xh(\zeta_{0}+z)}\D z
\asymptote\int^{\varepsilon}_{\varepsilon}\E^{x[h(\zeta_{0})+h''(\zeta_{0})z^{2}/2]}\D z
\end{equation}
and we observe that
\begin{align}
\int^{\varepsilon}_{\varepsilon}\E^{x[h(\zeta_{0})+h''(\zeta_{0})z^{2}/2]}\D z
&= \E^{xh(\zeta_{0})}\int^{\varepsilon}_{\varepsilon}\E^{xh''(\zeta_{0})z^{2}/2}\D z\\
&= \exp[xh(\zeta_{0})]\sqrt{\pi}/\sqrt{-xh''(\zeta_{0})/2}
\end{align}
\end{subequations}
Thus we conclude
\begin{equation}
f(x)\asymptote\frac{\E^{xh(\zeta_{0})}\sqrt{2\pi}}{\sqrt{x}\sqrt{-h''(\zeta_{0})}}
\end{equation}
as the asymptotic behavior for $f(x)$.

\begin{wrapfigure}{r}{1in}
\vspace{-36pt}
\begin{center}
\includegraphics{img/lecture16.1}
\end{center}
\end{wrapfigure}
This steepest descent method holds for $f(z)$ provided that $z$
lies in the sector of $\CC$ where $x>0$ and $y$ is on the right
hand side. See the shaded part of the diagram to the right for
details. 

Now we should remember
\begin{equation}
h(\zeta)=h(\zeta_{0})+\underbracket[0.5pt]{\frac{1}{2}h''(\zeta_{0})(\zeta-\zeta_{0})^{2}+\dots}_{=-\omega(z)^{2}}
\end{equation}
In a reasonable neighborhood of 0, $\omega(z)$ is invertible.
We have from our computations
\begin{subequations}
\begin{align}
f(x)
&\expequiv\int^{\varepsilon}_{-\varepsilon}\E^{xh(\zeta_{0}+z)}\D
z\asymptote\int^{\infty}_{-\infty}\E^{xh(\zeta)}\D\zeta\\
&=\E^{xh(\zeta_{0})}\int^{\varepsilon}_{-\varepsilon}\E^{-\omega^{2}x}\D z
\end{align}
where we have used the fact that $\omega(z)$ is invertible in a
neighborhood of 0, so we get $z=z(\omega)$ and $\D
z=z'(\omega)\D\omega$, thus
\begin{align}
f(x)&\asymptote \E^{xh(\zeta_{0})}\int^{\varepsilon}_{-\varepsilon}\E^{-\omega^{2}x}\D z\nonumber\\
&=\E^{xh(\zeta_{0})}\int^{\omega(\varepsilon)}_{\omega(-\varepsilon)}\E^{-\omega^{2}x}z'(\omega)\D\omega
\end{align}
Now we suppose that we can power-series expand $z(\omega)=\sum
a_{n}\omega^{n}$, and thus we obtain
\begin{align}
f(x)&\asymptote\E^{xh(\zeta_{0})}\int^{\omega(\varepsilon)}_{\omega(-\varepsilon)}\E^{-\omega^{2}x}z'(\omega)\D\omega\nonumber\\
&=\E^{xh(\zeta_{0})}\sum^{\infty}_{n=1}\int^{\omega(\varepsilon)}_{\omega(-\varepsilon)}a_{n}n\omega^{n-1}\E^{-\omega^{2}x}\D\omega\\
&\asymptote\E^{xh(\zeta_{0})}\sum^{\infty}_{n=1}a_{n}n\int^{\infty}_{-\infty}\omega^{n-1}\E^{-\omega^{2}x}\D\omega
\intertext{and only the even integrands survive, so we get}
&=\E^{xh(\zeta_{0})}\sum^{\infty}_{n=1}a_{2n-1}(2n-1)\int^{\infty}_{-\infty}\omega^{(2n-1)-1}\E^{-\omega^{2}x}\D\omega
\end{align}
\end{subequations}
We recall that
\begin{equation}
\int^{\infty}_{-\infty}t^{n}\E^{-t^{2}}\D t=\underbracket[0.5pt]{\left.\left(\frac{t^{n+1}}{n+1}\right)\E^{-t^{2}}\right|^{+\infty}_{-\infty}}_{=0}+\frac{2}{n+1}\int^{\infty}_{-\infty}\E^{-t^{2}}t^{n+2}\D t
\end{equation}
and thus inductively we find
\begin{equation}
\int^{\infty}_{-\infty}t^{2m}\E^{-t^{2}}\D t=\frac{(2m)!\sqrt{\pi}}{m!2^{m-1}}.
\end{equation}
When we plug this result into the integral, we find precisely
\begin{equation}
f(x)\asymptote\E^{xh(\zeta_{0})}\sum^{\infty}_{m=0}a_{2m+1}(2m+1)\frac{(2m)!\sqrt{\pi}}{m!2^{m+1}}
\end{equation}
Note that this method is useful when considering the classical
limit in path integral quantization.
