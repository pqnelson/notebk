%%
%% lecture07.tex
%% 
%% Made by alex
%% Login   <alex@tomato>
%% 
%% Started on  Sat Oct  1 19:36:14 2011 alex
%% Last update Sat Oct  1 19:36:14 2011 alex
%%
We discussed the Riemann surface of
\begin{equation}
h(z)=\sqrt{(z-1)z(z+1)}.
\end{equation}
This is doodled thus:
\begin{center}
\includegraphics{img/lecture07.0}
\end{center}
Consider
\begin{equation}
\omega^{3}-\omega+z=0,
\end{equation}
we see that
\begin{equation}
z=\omega-\omega^{3},
\end{equation}
so we may write $z=z(\omega)$. We can invert this to find
$\omega=\omega(z)$. More generally we can suppose that
$p(\omega)=z$ and its inverse is $q(z)=\omega$. We can now
consider the Riemann surface of this function. Consider the graph
of this function (roughly doodled below to the left).

\begin{wrapfigure}{l}{3in}
\begin{center}
\includegraphics{img/lecture07.1}
\end{center}
\end{wrapfigure}
\noindent{}The inverse function has several values, so we get in this
complex analogue a Riemann surface. From the projection, which is
multivalued, we have the Riemann surface. In this case for
$h(z)$, the Riemann surface is homeomorphic to a torus.

The reader should make a mental note on the importance of branch
cuts in this method of constructing Riemann surfaces. Also note
that we are projecting onto \emph{Riemann spheres} which are
distinguished from the notion of \emph{Riemann surfaces}. The
Riemann sphere is $\CC$ as a sphere, obtained from Stereographic
projection. 

\begin{prop}[Fact from geometry]
An orientable, closed, compact surface is homeomorphic to a torus
(or more generally a sphere with $p$ handles).
\end{prop}

How do we find the genus of a Riemann surface? (I.e., how do we
find the value of $p$?) We have $n$ roots, we count how many
times we glue points together. We subtract the number of boundary
points of the cuts. So we have
\begin{equation}
\chi=2n-\#(\mbox{boundary points of the cuts})
\end{equation}
\marginpar{Euler Characteristic of Riemann Surface}which is precisely the Euler characteristic. The genus is
\begin{equation}
\frac{2-\chi}{2}=\mbox{genus}.
\end{equation}
This is for polynomials, however.

Riemann surfaces are defined for algebraic functions. Consider
the famous example of the logarithm function, it covers the
complex plane infinitely many times. When we consider the Riemann
surface, it's like an infinite Helix. This is the logarithmic
staircase. See Penrose's \emph{Road to Reality} for a good
doodle.

\subsection{Reflection Principle}

\begin{wrapfigure}[10]{r}{1.5in}
\vspace{-50pt}
\begin{center}
\includegraphics{img/lecture07.2}
\end{center}
\vspace{-40pt}
\end{wrapfigure}
We are nonetheless interested in extending functions. We have the
Reflection principle. We have some domain which contains on the
boundary part of the $x$ (real) axis.
We consider some function $f\colon\mathcal{U}\to\CC$, we extend
$f$ to another function $\widetilde{f}$ on $\mathcal{U}\cup I$
where $I$ is the real part of $\partial\mathcal{U}$, i.e.,
$I=\RR\cap\partial\mathcal{U}$. We demand that $\widetilde{f}$ be
continuous, and demand that $\widetilde{f}|_{I}$ be real. We
consider the complex conjugation of $\mathcal{U}$, doodled to the
right, which is $\bar{\mathcal{U}}=\{\bar{z}\lst
z\in\mathcal{U}\}$. We introduce a function $\widehat{f}$ such
that
\[
\widehat{f}(z) = \begin{cases}
f(z) & \mbox{if }z\in\mathcal{U}\\
\overline{f(\bar{z})} & \mbox{if }z\in\bar{\mathcal{U}}\\
\widetilde{f}(z) & \mbox{if }z\in I.
\end{cases}
\]
We have then $\mathcal{V}=\mathcal{U}\cup
I\cup\bar{\mathcal{U}}$, and we see that $\widehat{f}$ is
continuous on $\mathcal{V}$. We see that if $\widehat{f}$ is
analytic on $\mathcal{U}$, then $\widehat{f}$ is analytic on
$\bar{\mathcal{U}}$. 

Let $z=x+\I y\in\mathcal{U}$,
\begin{equation}
\widehat{f}(x+\I y)=u(x,y)+\I v(x,y)
\end{equation}
and for $\overline{z}\in\bar{\mathcal{U}}$ we see we have
\begin{equation}
\overline{f(\overline{z})}=u(x,-y)-\I v(x,-y)
\end{equation}
By the chain rule we see that $\overline{f(\overline{z})}$ satisfies
the Cauchy-Riemann equations.

\begin{wrapfigure}[5]{r}{1.25in}
\vspace{-30pt}
\begin{center}
\includegraphics{img/lecture07.3}
\end{center}
\vspace{-20pt}
\end{wrapfigure}
\medbreak\noindent\textbf{A General Statement.\enspace}
We will use the diagrams doodled on the right for reference.
Consider $\varphi\colon W\to\CC$ such that $\varphi$ is
continuous on $W$. If $\varphi|_{W_{1}}$ and
$\bar{\varphi}|_{W_{2}}$ are analytic, then $\varphi$ is analytic
on $W$.

\begin{lem}
Suppose we have in some domain $\mathcal{U}$ a continuous
function $$f\colon\mathcal{U}\to\CC.$$ Let $\gamma\colon
I\to\mathcal{U}$ be a continuous, closed curve, and for any such $\gamma$
that is simply connected, i.e., $$\int_{\gamma}f(z)\D z=0.$$ Then the
function is analytic.
\end{lem}

We see that the integral over a closed, simply connected path is
zero if its contained entirely in $W_{1}$ or $W_{2}$. We just
need to check for a path that crosses $\gamma$, we just treat it
by breaking it up into pieces.
