%%
%% hw1.tex
%% 
%% Made by alex
%% Login   <alex@tomato>
%% 
%% Started on  Mon Oct  3 09:24:26 2011 alex
%% Last update Mon Oct  3 09:24:26 2011 alex
%%

\section*{Homework 1}\renewcommand{\leftmark}{Homework 1}\phantomsection\addcontentsline{toc}{section}{Homework 1}

\begin{exercise}
Prove that if a conformal map of a domain $\mathcal{U}$ in
$\RR^{2}$ takes straight lines parallel to the $x$ axis into
parallel lines, then it takes any straight lines into straight
lines and circles into circles. 
\end{exercise}
\begin{exercise}
Prove that if a conformal map of a domain $\mathcal{U}$ in
$\RR^{2}$ takes straight lines into straight lines, then, at
least locally, $f$ coincides with a similarity map (a combination
of a rotation and a dilation). (Hint: triangles.) 
\end{exercise}
\begin{exercise}
Prove that if a smooth map $f\colon{\cal U}\to{\cal V}$ between
domains in the plane preserves perpendicularity (that is, if two
curve, $\gamma$, $\gamma'$ in ${\cal U}$ intersect each other
under a right angle, then so do their images), then it is
conformal. 
\end{exercise}
\begin{exercise}
Let $u$, $v$, $w$ be three arbitrary different points in
$\CC\cup\{\infty\}$. Prove that for any three different points
$u$, $v$, $w\in\CC\cup\{\infty\}$ there exists a unique
fractional linear transformation $f$ such that $f (u) = u$, $f
(v) = v$, $f (w) = w$. [Hint One: it is sufficient to consider 
$u = 0$, $v = 1$, $w = \infty$. Hint Two: cross-ratio (Page 331 of the
book); Hint Two is helpful, but not necessary.]  
\end{exercise}
\begin{exercise}
Prove that (a) a fractional linear transformation must have at least one fixed point;

\noindent(b) if a fractional linear transformation has no finite (different from $\infty$) fixed points, then
it is a translation, $f (z) = z + b$.
\end{exercise}
\begin{exercise}
Let $f (z) = z + \frac{1}{z}$. Prove that $f$ takes concentric
circles $\|z\| = R$, $R \geq 1$ and straight rays $z = t\alpha$,
$\|\alpha\| = 1$, $t\geq 1$ into ellipses and hyperbolas with the
foci $(−2, 0)$ and $(2, 0)$. See the picture. (Comment. This is
an ``honest'' picture, the figure on the right is obtained by the
given transformation from the figure on the left.) 
\end{exercise}
