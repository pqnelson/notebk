%%
%% lecture23.tex
%% 
%% Made by alex
%% Login   <alex@tomato>
%% 
%% Started on  Wed Oct  5 13:49:39 2011 alex
%% Last update Wed Oct  5 13:49:39 2011 alex
%%

We are referred to:
\begin{itemize}
\item E.\ Titchmarsh,\newblock
\emph{The Zeta-Function of Riemann}.
\end{itemize}
for further reading on the Riemann zeta function. We will
continue analytically continuing $\zeta(s)$ to
$\CC\setminus\{1\}$. Now what happens to Eq
\eqref{eq:reflectionPrincipleForZeta} when we take
$\zeta(1-s)=\dots$?
It becomes
\begin{equation}
\zeta(1-s)=2(2\pi)^{-s}\cos\left(\frac{s\pi}{2}\right)\Gamma(s)\zeta(s).
\end{equation}
Notice that for $s=2n+1$, for any $n\in\ZZ$, we have
\begin{equation}
\zeta(1-s)=0.
\end{equation}
But due to the $\Gamma$ function, we have $s>0$, otherwise
$\Gamma$ is undefined. But they're killed off by $\cos(\pi s/2)$,
so it's all good.
There are the trivial zeroes $\zeta(-2)=\zeta(-4)=\dots=0$.

Now, the most famous unsolved problem in mathematics: the Riemann
hypothesis. The only zeroes (in addition to the trivial zeroes)
are at 
\begin{equation}
s=\frac{1}{2}+\I\mbox{(something)}.
\end{equation}
Note $\zeta(s)=\zeta(\overline{s})$.

\begin{prop}
There are infinitely many ``nontrivial zeroes'' for $\zeta(s)$.
\end{prop}
\begin{prop}
If $\gamma_{n}$ is the $n^{\rm th}$ nontrivial zero,
$\lim_{n\to\infty}(\gamma_{n}-\gamma_{n-1})=0$.
\end{prop}
\begin{con}
If $N(T)$ is the number of $\gamma$ with $\im(\gamma_{n})<T$,
then $N(T)\asymptote T\ln(T)$ is the asymptotic behavior.
\end{con}
The first nontrivial zero is at 
\begin{equation}
\gamma_{1}\approx\frac{1}{2}+\I14.1347\;2514\;17346.
\end{equation}
So what are the distribution of these nontrivial zeroes?
We see that the number $k\in\NN$ with $\im(\gamma_{k})<n$ is
described on the following table:

\medskip
\begin{center}
\begin{tabular}{cc}
\toprule
$n$ & $\max\{k\in\NN\lst\im(\gamma_{k})<n\}$  \\
\midrule
100       & 29\\
1000      & 649\\
100,000   & 10142\\
1,000,000 & 1,747,146\\
\bottomrule
\end{tabular}
\end{center}


Let
\begin{equation}
\pi(N):=\mbox{(number of primes$<N$)}
\end{equation}
We have
\begin{equation}
\pi(N)\asymptote \Li(N)
\end{equation}
where
\begin{equation}
\Li(x)=\int^{x}_{2}\frac{\D u}{\log(u)}
\end{equation}
is the Logarithmic integral function. The Riemann hypothesis then
states there exists constants $c,C\in\RR$ such that $c>0$
and $C>0$ which obey
\begin{equation}
c\sqrt{N}\ln(N)<|\pi(N)-\Li(N)|<C\sqrt{N}\ln(N)
\end{equation}
How beautiful.

% Some notes on Poincar\'e's half-plane model omitted
