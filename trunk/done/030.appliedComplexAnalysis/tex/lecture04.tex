%%
%% lecture04.tex
%% 
%% Made by alex
%% Login   <alex@tomato>
%% 
%% Started on  Sat Oct  1 14:11:55 2011 alex
%% Last update Sat Oct  1 14:11:55 2011 alex
%%
It is sometimes important to give conformal equivalence between
two domains. There are some books on the subject of equivalence
of conformal domains.

\begin{ex}
One of the important functions that is a conformal mapping is the
exponential mapping
\begin{equation}
f(z)=\E^{z}.
\end{equation}
It maps $\RR\times[0,2\pi)\propersubset\CC$ to
$\CC\setminus\{0\}$. We shade the domain in gray, and consider
how horizontal lines behave under this mapping:
\begin{center}
\includegraphics{img/lecture04.0}
\end{center}
Note that horizontal lines are mapped to lines, and vertical
lines are mapped to circles.

The line $y\I$ (that is, $z(t)=t\I$ which is purely imaginary) is
mapped to the unit circle, the lines with $x<0$ are mapped to
concentric circles. The vertical lines (to the right) are mapped
to circles with radius $\exp(x)$.
So the region 
$$\{(x,y)\in\CC\lst x\geq0,\quad 0\leq y\leq2\pi\}$$
is mapped to the region \emph{outside} the unit circle, since
$\exp(x)\geq\exp(0)=1$. 
\end{ex}
\begin{ex}
Consider
\begin{equation}
f(z)=z+\frac{1}{z}
\end{equation}
How does this behave? Lets consider a nice subdomain of the upper
half plane:
\begin{center}
\includegraphics{img/lecture04.1}
\end{center}
Wee see that $f(1)=2$ and $f(-1)=-2$ by direct computation; when
$\|z\|=1$, we see that $1/z=\exp(-i\theta)$ and we have more
generally $1/z=\bar{z}$. Thus $f(z)=z+\bar{z}=2\re(z)$. So we see
that this boundary is mapped to the real line. We see that this
maps the domain to the upper half-plane. But to make things
interesting, we cut up the domain in the manner we have
doodled. The circular arcs are mapped to elliptical arcs. The
lines are mapped to hyperbolas. We have a family of ellipses
given by the relation
\begin{equation}
\frac{x^{2}}{a+t}+\frac{y^{2}}{b+t}=1
\end{equation}
where $t\in\RR$ is ``some parameter''. There are places where it
is badly behaved, but that's okay. We also have imaginary
ellipses when $t<a$ or $t<b$.
\end{ex}
There are many beautiful properties of confocal family (conformal
foci family) which we will not pursue here. I believe my good
friend, Dmitry B.\ Fuchs, has beautifully examined it in
\emph{Mathematical Omnibus: Thirty Lectures on Classic Mathematics}. 
Additionally, Serge Tabachnikov's \emph{Geometry and Billiards}
(American Mathematical Society, 2005) is a good resource.


\begin{wrapfigure}{r}{1in}
\vspace{-20pt}
\begin{center}
\includegraphics{img/lecture04.2}
\end{center}
\vspace{-20pt}
\end{wrapfigure}
There is one more transformation to be considered which is fairly
beautiful, so we will consider it. Suppose we have a convex
$n$-gon ($n$ sided polygon), there is a conformal map from the
upper half plane to this $n$-gon.

Let $p$ be a convex $n$-gon with exterior angle $\alpha_{1}\pi$,
\dots, $\alpha_{n}\pi$ as doodled to the right. We see that
$\alpha_{1}+\dots+\alpha_{n}=2$, and $0<\alpha_{j}<1$. Consider
the following: fix $n-1$ points (on the real line) $x_{1}$,
\dots, $x_{n-1}$ ordered from left to right with distances left
unspecified. 

Consider
\begin{equation}
f(z)=b+a\int^{z}_{z_{0}}(\xi-x_{1})^{-\alpha_{1}}(\dots)(\xi-x_{n-1})^{-\alpha_{n-1}}\D\xi
\end{equation}
where $a,b\in\CC$. Why consider only $(n-1)$ such $\alpha$'s?
Well, the $n^{\rm th}$ is determined completely by our relations
above. 

When we consider $f(z)$ on the real line, what happens when $\xi$
approaches, e.g., $x_{1}$? Does this converge or diverge as an
improper integral? Well, since we have
\begin{equation}
0<\alpha_{j}<1
\end{equation}
we see the integral converges since
\begin{equation}\label{eq:lecture4:integralsomethinglike}
\int^{1}_{0}\frac{\D x}{\sqrt{x}}=2
\end{equation}
is finite. We can change variables from $\xi$ to $\xi-x_{j}$, and
we get something like our integral in Eq \eqref{eq:lecture4:integralsomethinglike}.
Now, it really looks like
\begin{equation}
f(z)\propto\int^{z}_{z_{0}}\xi^{-2+\alpha_{n}}\D\xi
\end{equation}
and this converges. So this function $f(z)$ is defined at the
dangerous points $x_{1}$, \dots, $x_{n}$ and $\pm\infty$. Lets
set $b=0$ and $a=1$ for now.

The fundamental theorem fo calculus says
\begin{equation}
f'(z)=(\xi-x_{1})^{-\alpha_{1}}(\dots)(\xi-x_{n-1})^{-\alpha_{n-1}}
\end{equation}
Let $x_{i}<z<x_{i+1}$. We are interested in
\begin{equation}
\arg\big(f'(z)\big)=?
\end{equation}
We see for real numbers $\arg(x)=0$. By direct computation we
find
\begin{equation}
\arg\left(f'(z)\right)=(-\alpha_{j+1}-\dots-\alpha_{n-1})\pi
\end{equation}
This means that the direction is constant.

\begin{wrapfigure}{r}{2.5in}
\vspace{-30pt}
\begin{center}
\includegraphics{img/lecture04.3}
\end{center}
\vspace{-20pt}
\end{wrapfigure}
We see that as we move along the real axis, the argument changes
in ``discrete chunks''. If $z<x_{1}$, then
what may we say about $\arg(f'(z))$? Well well well, we see that
\begin{equation}
\arg(f'(z))=(-\alpha_{1}-\dots-\alpha_{n-1})\pi=(2-\alpha_{n})\pi=-\alpha_{n}\pi
\end{equation}
The last step is because $2\pi\equiv0$ since we mod out by $2\pi$.

\begin{wrapfigure}{l}{2.5in}
\vspace{-20pt}
\begin{center}
\includegraphics{img/lecture04.4}
\end{center}
\vspace{-20pt}
\end{wrapfigure}
\noindent{}We have no $x_{n}$ on the real axis specified because
we have $f(\infty)$ ``='' $f(x_{n})$, so we have in effect what
is doodled on the left. We have the identification
$f(\infty)=f(-\infty)$. 

What did we just do? We verified that the boundary is mapped to
the boundary, but what about the \emph{interior}?

Consider the rectangle. As a holomorphic function (obtained by
our construction) we get the elliptic integral.

\begin{rmk}[Further Reading]
For more on this, see Stein~\cite{stein}, chapter 8 \S4
``Conformal mappings onto polygons.''
\end{rmk}
