%%
%% lecture01.tex
%% 
%% Made by alex
%% Login   <alex@tomato>
%% 
%% Started on  Thu Sep 29 07:39:00 2011 alex
%% Last update Thu Sep 29 07:39:00 2011 alex
%%

The goals today are to explain: what is a topology, and why is it
useful to know topology?

What is topology? The first notion is topological equivalence, or
equivalence of two spaces, i.e., a homeomorphism. If we have $X$,
$Y$ be topological spaces, then a topological equivalence is a
map
\begin{equation}
f\colon X\to Y
\end{equation}
that is a one-to-one correspondence, $f$ and $f^{-1}$ are both
continuous. But what kind of spaces are these? The most general
situation is when $X$ is a topological space, where we have a
notion of an open set.

One of the main questions of topology is: given two spaces, are
they equivalent? If we can construct a homeomorphism, we got
it. What about if they're nonequivalent? We need to use
properties called \define{Topological Invariants}\index{Topological Property}\index{Topological Invariant}\index{Invariant!Topological}.\marginpar{Topological Invariant = Property Invariant under Homeomorphisms} If two spaces
are topologically equivalent, then the topological invariants of
the two spaces are the same.

The basic topological invariant for a space is
connectedness\index{Connectedness}\index{Topological Invariant!Connectedness}\index{Topological Property!Connectedness}. We
will use the notion of pathwise connected. A space is pathwise
connected if and only if any two points are connected by a
path. What is a path? Well, it is a parametrized curve
\begin{equation}
\gamma\colon[0,1]\to X
\end{equation}
such that $\gamma(0)=x$ and $\gamma(1)=y$, and $\gamma$ is
continuous.

We can consider the connected component of $X$. We can consider
$x\sim y$ iff there is a path connecting the points. We see that
the letter ``A'' is connected whereas ``i'' is disconnected (it
has two components), so they are not topologically equivalent.

There is a simple idea regarding ho to construct a topological
invariant given some topological invariant. We consider a
functor $F$ which is such that
\begin{equation}
X\sim Y\quad\implies\quad F(X)\sim F(Y)
\end{equation}
We can consider, for example, the construction
\begin{equation}
F(-)=\hom(A,-)
\end{equation}
for some fixed topological space $A$. We may construct
topological invariants this way. For example the number of
connected components of $\hom(A,X)$ gives an invariant of $X$. 

We may consider topological spaces with a marked point\index{Marked Points} $(X,x_{0})$
where $x_{0}\in X$. We may consider
\begin{equation}
f\colon (X,x_{0})\to (Y,y_{0})
\end{equation}
such that $f(x_{0})=y_{0}$ preserves the marked point. The most
interesting object of this kind is obtained in the following way:
take a sphere with a marked point, take maps of this sphere to
some other pointed space
\begin{equation}
\hom\left((S^{n},s_{0}),(X,x_{0})\right)
\end{equation}
which is a topological space with marked point $(X,x_{0})$.
We consider the components of this space, this set has a group
structure which we will call $\pi_{n}(X)$\index{$\pi_{n}(X)$} called the
\define{Homotopy Group}\index{Homotopy Group}\index{Group!Homotopy}. For $n=1$ we get the fundamental group\index{Fundamental Group}.

\index{Euler Characteristic|(}
Some other invariants we will consider later are the Euler
characteristic. We may decompose $X$ into the disjoint union of
open balls
\begin{equation}
X=\bigsqcup\mbox{(open balls)}
\end{equation}
For example the sphere is equivalent to a point and the remainder
is homologically equivalent to an open disc. So the Euler
characteristic is then
\begin{equation}
\chi(X)=\sum(-1)^{n}\alpha_{n}
\end{equation}
where $\alpha_{n}$ is the number of open $n$-balls. So
\begin{equation}
\chi(X\sqcup Y)=\chi(X)+\chi(Y)
\end{equation}
This sum should not depend on how we decompose the space. To give
a proper definition of the Euler characteristic, we need to use
homology.
\index{Euler Characteristic|)}

Topology may be applied to more-or-less everywhere and
everything. That doesn't mean it answers every question. But we
should ask ourselves ``What is topology saying about this?''

Historically the first application was to study integrals. For\index{Stoke's Theorem}
example, Stoke's formula, Green's formula\index{Green's Formula}, etc., are of the form
\begin{equation}
\int_{\partial S}\omega=\int_{S}\D\omega,
\end{equation}
so the notion of an integral\index{Integral} is closely related to the notion of
the boundary\index{Boundary!and Integration} of some surface.

The homology\index{Homology} is closed surfaces mod boundaries, meh we are sloppy
here.

At any rate, integers are relevant to topology, viz.\ in
$\CC$. Another thing we'd like to mention is the application
of topology to the study of $\vec{f}(\vec{x})=\vec{0}$. The first
question is how many solutions do we have? Topology cannot say
the number of solutions, but it can tell us the \emph{algebraic
  number} of solutions. For example, when considering $y=f(x)$
where
\begin{subequations}
\begin{equation}
\lim_{x\to-\infty}f(x)<0
\end{equation}
and
\begin{equation}
\lim_{x\to\infty}f(x)>0
\end{equation}
\end{subequations}
then there are an odd number of points $x_{1}$, \dots, $x_{2n+1}$
such that 
\begin{equation}
f(x_{i})=0
\end{equation}
This is topological, and looks like:
\begin{center}
\includegraphics{img/lecture1.0}\quad\includegraphics{img/lecture1.1}\quad
etc.
\end{center}

The last thing to mention is the calculation of the index of $A$,
a Fredholm operator\index{Fredholm Operator}. We consider $\ker(A)$ and assume
$\dim(\ker(A))$ is finite. We can consider
\begin{equation}
A\colon E\to E
\end{equation}
then
\begin{equation}
\coker(A)=\im(A)/\ker(A)
\end{equation}
and $\dim(\coker(A))$ is finite. The difference between these
finite numbers is precisely the \define{Index}\index{Operator Index}\index{Index!of Operator} of the operator.

Topology has very important applications in physics. Namely, one
of the ways is when we work with fields. We can consider the
space of all fields (possibly with some restrictions, e.g. with
finite energy). It is possibly disconnected. Foe example, in
classical mechanics
\begin{equation}
V(x)=x^{4}
\end{equation}
the space of solutions is disconnected. But to show this, we need
the homotopic group of the space. There are other applications of
topology in physics, e.g., TQFT.

There are other topological applications, e.g., in the calculus
of variation we may compute the number of critical points
topologically. 
