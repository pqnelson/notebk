%%
%% lecture13.tex
%% 
%% Made by alex
%% Login   <alex@tomato>
%% 
%% Started on  Mon Dec 26 21:23:33 2011 alex
%% Last update Mon Dec 26 21:23:33 2011 alex
%%

\subsection{Projective Spaces}
\index{Space!Projective|(}\index{Projective Space|(}
Recall the notion of a projective space $\PP^n$. Consider the
$(n+1)$-dimensional vector space $\FF^{n+1}$ over $\FF$. Consider
all lines in $\FF^{n+1}$ that contain the origin. We need to know
only one point---then we know the line. (Why?) If $x\in\FF^{n+1}$
and $x\not=0$, then $\lambda x$ describes the lines, for
arbitrary $\lambda\in\FF$. The set of all such lines is the
projective space over $\FF$, denoted $\FP^{n}$. We can take
\begin{equation}
(\FF^{n+1}-0)/(x\sim \alpha x)\eqdef\FP^{n}
\end{equation}
This is another description of it.

A point in projective space may be described
by \define{Homogeneous Coordinates}\index{Projective Space!Homogeneous Coordinates}\index{Homogeneous Coordinates}
denoted
\begin{equation}
(x_0:x_1:\cdots:x_n)\sim(\alpha x_0:\alpha x_1:\cdots:\alpha x_n).
\end{equation}
(In physics, we have a similar picture where wave functions are
defined up to a constant factor.) So this projective space
contains an $n$-dimensional vector space
\begin{equation}
\FF^n\propersubset\FP^n.
\end{equation}
How? Consider $x_0\not=0$. Every point with this condition is
equivalent to
\begin{equation}
(x_0:x_1:\cdots:x_n) = \left(1:\frac{x_1}{x_0}:\cdots:\frac{x_n}{x_0}\right).
\end{equation}
We take
\begin{equation}
y_i\eqdef x_i/x_0,
\end{equation}
then
\begin{equation}
(x_0:x_1:\cdots:x_n) = \left(1:y_1:\cdots:y_n\right).
\end{equation}So $\FF^n$ is sitting in projective space. The next
thing we can do is take our projective space and delete this
$\FF^n$. What do we get? Well, it's quite simple: we get all the
points where $x_0=0$. We thus get the picture that
\begin{equation}
\FP^n-\FF^n=\FP^{n-1}.
\end{equation}
So if you like, we may say that
\begin{equation}
\FP^{n}=\FF^n\sqcup\FP^{n-1}
\end{equation}
disjoint union of topological spaces.

Now we go to topology and consider two cases: $\FF=\CC$ and
$\FF=\RR$. Lets look at the simplest situations. What is
$\RP^1$?\index{$\RP^1$} It is very easy to see
\begin{equation}
\RP^1=S^1
\end{equation}
We see
\begin{equation}
\RP^1=\RR^1\sqcup\RP^0
\end{equation}
but $\RP^0$ consists of just a single point. What is
$\CP^1$?\index{$\CP^1$} Of course, we see
\begin{equation}
\begin{split}
\CP^1 &= \CC\cup\CP^0\\
&=\CC\cup\{\mbox{point}\}\\
&=S^2
\end{split}
\end{equation}
as desired.

Let us look a little bit at 
\begin{equation}
\RP^{n} = (R^{n+1}-0)/(x\sim\lambda x)
\end{equation}
But we may do something different. Namely every point on
$\RR^{n+1}-0\homotopic S^n$, why? We may divide $x$ by $\|x\|$ so
every point on $\RP^n$ may be represented by a point on a
sphere. But still we should identify $x\sim\lambda x$ where both
are on the sphere\dots but this happens when
$|\lambda|=1$. Therefore the only thing we should do is
\begin{equation}
\RP^n=S^n/(x\sim-x)
\end{equation}
This may be represented by 
\begin{equation}
\RP^n=\RR^n\sqcup\RP^{n-1},
\end{equation}
which is how we get a cellular decomposition (where we have a single
$k$-cell in every dimension $k\leq n$).

For the complex case, we see
\begin{equation}
\begin{split}
\CP^n &= (\CC^{n+1}-0)/(x\sim\lambda x)\\
&= S^{2n+1}/(x\sim\lambda x)
\end{split}
\end{equation}
where $|\lambda|=1$. Is this true? Let first note
\begin{equation}
\CC^{n+1}=\RR^{2(n+1)}
\end{equation}
but we demand $\|x\|=1$ which eliminates a dimension, giving us
\begin{equation}
\RR^{2(n+1)}/\sim = S^{2n+1}
\end{equation}
This implies $|\lambda|=1$. We can consider this set $S^{1}=\{\lambda :
|\lambda|=1\}$ as a group. This group acts on $S^{2n+1}$ simply
by
\begin{equation}
x\mapsto\lambda x
\end{equation}
\index{Hopf Fibration!Derived from $\CP^1$|(}%
One more definition of $\CP^n$. One more definition of
$\CP^n$. This is
\begin{equation}
\CP^n=S^{2n+1}/S^{1}
\end{equation}
where we mod out by this action of $S^1$; we can write similarly
\begin{equation}
\RP^n=S^n/\ZZ_2
\end{equation}
since $\lambda\in\RR$ and $|\lambda|=1$ implies
$\lambda=\pm1$. Take $n-1$ we get
\begin{equation}
\CP^1=S^3/S^1=S^2
\end{equation}
In a different way, we may say this as follows: there exists a
mapping
\begin{equation}
h\colon S^3\to S^2
\end{equation}
such that the preimage of a point $h^{-1}(\mbox{point})=S^1$. We
call $h$ the \define{Hopf Map}\index{Hopf Map}, later we will see
$h$ is not homotopic to 0, so it's very non-trivial.
\index{Hopf Fibration!Derived from $\CP^1$|)}%
\index{Space!Projective|)}\index{Projective Space|)}

\subsection{Knot Rejoinder}
How do we use $h$ to analyze the structure of knots? We did say
\begin{equation}
h^{-1}(x)=S^1
\end{equation}
for any $x\in S^2$. We can see that\index{Torus!Solid}
\begin{equation}
h^{-1}(\mbox{disc})=\begin{pmatrix}\mbox{solid}\\\mbox{torus}
\end{pmatrix}.
\end{equation}
How? Well, we could see the preimage $h^{-1}(S^{1})$ is a torus,
and $S^1$ is a boundary of a disc. So we fill in the disc
``continuously'' and we fill in the torus. 

Lets be clear here. We are looking at $S^{2}$ by constructing it
from two 2-discs 
\begin{equation}
S^{2}=(\bar{D}^{2}_{0}\sqcup\bar{D}^{2}_{1})/(\partial\bar{D}^{2}_{0}\sim\partial\bar{D}^{2}_{1})
\end{equation}
We let
\begin{subequations}
\begin{equation}
A=h^{-1}(\bar{D}^{2}_{0})
\end{equation}
and
\begin{equation}
B=h^{-1}(\bar{D}^{2}_{1})
\end{equation}
\end{subequations}
and we will use van Kampen's theorem.

So
\begin{equation}
h^{-1}(S^{2})=A\cup B
\end{equation}
where $A$, $B$ are solid tori, and
\begin{equation}
A\cap B=T^{2}
\end{equation}
is a (non-solid) torus. We obtain this since
\begin{equation}
S^{3}=\bar{D}^{2}\cup\bar{D}^{2}.
\end{equation}
We obtain this from the Hopf map.

Consider the solid torus in $\RR^3$, it is bounded by the torus
in $\RR^3$. Consider this stuff in
\begin{equation}
S^{3}=\RR^{3}\cup\infty.
\end{equation}
We consider
\begin{equation}
S^{3}-(\mbox{open solid torus}).
\end{equation}
What do we get? A solid torus! We can see this result in our
picture also. Of course, this is something with the solid torus
as a boundary. The analog of the solid torus in higher dimension
--- we take a body with handle, it's called a \define{Handle Body}.\index{Handle Body}
It's important to consider representations of them in ``Heegaard
Diagrams''\index{Heegaard Diagrams!we won't speak of ---}\dots but
we won't speak of it here.

We can take as a knot invariant $\pi_{1}(S^3-K)$. Let us take for
$K$ a \emph{trivial} knot, i.e., $K=S^1$. Then it's very simple,
because we can take a small neighborhood of $S^1$, which is a
solid torus, and 
\begin{equation}
S^{3}-K\homotopic\mbox{solid torus}\homotopic\mbox{circle}
\end{equation}
are homotopy equivalences. Therefore
\begin{equation}
\pi_{1}(S^{3}-K)\iso\pi_{1}(S^1)\iso\ZZ
\end{equation}
But we would like to distinguish two unlinked circles from two
linked circles: \includegraphics{img/lecture12.6} vs.\ \includegraphics{img/lecture12.7}.
We may take the first circle as $K$,
\begin{equation}
S^{3}-K=\begin{pmatrix}\mbox{solid}\\ \mbox{torus}
\end{pmatrix}
\end{equation}
we then consider the second circle as an elemetn of
$\pi_1(S^3-K)$. If it's trivial, the knots are unlinked.

A knot is a topological circle in $\RR^3$ or $S^3$. Now it is
possible the circle lies on a torus. Then it's called
a \define{Torus Knot}\index{Knot!Torus|textbf}\index{Torus!Knot}.
The next problem is to find
\begin{equation*}
\pi_1\bigl(S^3-(\mbox{solid torus})\bigr).
\end{equation*}
This will give us an invariant of the knot, we can classify them.

%%%%%%%%%%%%%%%%%%%%%%%%%%%%%%%%%%%%%%%%%%%%%%%%%%%%%%%%%%%%%%%%%%%%%%%%%%%
\exercises
\begin{xca}
Let $\FF_{p}$ be the finite field with $p$ elements, where $p$ is
prime. What is the cardinality of the set $\FP^{n}_{p}$?
\end{xca}
\begin{xca}
Projective space $\RP^n$ can be obtained from the sphere $S^n$ by
means of identification of antipodes $( x \sim -x)$. Describe the
cell decomposition of $\RP^n$ and use it to calculate its
fundamental group. 
\end{xca}
\begin{xca}\label{xca:lec13:prob2}
Let us consider an $n$-dimensional manifold $X$ and its subspace
$X = X \setminus D^n$ (the space $X$ with deleted open ball
$D^n$). Express the fundamental group of $X$ in terms of the
fundamental group of $X$. 
\end{xca}
\begin{xca}\label{xca:lec13:prob3}
The connected sum of two $n$-dimensional manifolds $X$ and $Y$ is
defined by means of deleting of open balls from $X$ and $Y$ and
identification of boundaries of deleted balls. (In notations of
Problem \ref{xca:lec13:prob2} we identify the boundary spheres in
$X$ and $Y$). Calculate the fundamental group of connected sum. 
\end{xca}
\begin{rmk}
In Problems \ref{xca:lec13:prob2} and \ref{xca:lec13:prob3} we
assume that $n>1$. In the case $n=2$ you can use the fact that
every two-dimensional compact manifold is a sphere with attached
handles and Moebius bands;\index{Mobius Band!and Surfaces} if the manifold is not compact one
should consider a sphere with holes (with deleted closed disks)
instead of sphere.
\end{rmk}
