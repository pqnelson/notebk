%%
%% lecture14.tex
%% 
%% Made by alex
%% Login   <alex@tomato>
%% 
%% Started on  Wed Dec 28 20:08:28 2011 alex
%% Last update Wed Dec 28 20:08:28 2011 alex
%%
We would like to talk about toric knots. Recall last time we did
the following: we considered
\begin{equation}
S^3=\begin{pmatrix}\mbox{solid}\\ \mbox{torus}
\end{pmatrix}\cup\begin{pmatrix}\mbox{solid}\\ \mbox{torus}\end{pmatrix}
\end{equation}
Remember that
\begin{equation}
\RR^3\cup\{\infty\}=S^3,
\end{equation}
so
\begin{equation}
\RR^3-\begin{pmatrix}\mbox{solid}\\ \mbox{torus}\end{pmatrix}
=\begin{pmatrix}\mbox{solid}\\ \mbox{torus}\end{pmatrix}\cup\{
\infty\}
\end{equation}
If we take a knot on the solid torus, we get a knot.
Specifically when it is a non-self-intersecting curve on the
torus.\marginpar{\centering\includegraphics{img/lecture14.0}} Recall a torus is identified from a square as doodled on
the right, with the
relation
\begin{equation}
aba^{-1}b^{-1}=1\iff ab=ba,
\end{equation}
so we get the free Abelian group with 2 generators. We may
consider instead $\RR^2/\ZZ^2$ which is equivalent to what we
have drawn.

\begin{wrapfigure}{l}{5pc}
  \centering
  \includegraphics{img/lecture14.1}
\end{wrapfigure}
\noindent But a vertex $(a,b)\sim(a+1,b)\sim(a,b+1)$. If we'd like to draw
a closed curve on a torus we draw a line. Suppose we draw it on
$\RR^2$ with slope $n/m$ for $m,n\in\ZZ$. Well, we can see if
\begin{equation}
\gcd(m,n)=1
\end{equation}
then the curve on the torus (obtained by transporting the line
using the quotient $\RR^2/\ZZ^2$) is not
self-intersecting. However, for 
\begin{equation}
\gcd(m,n)\not=1
\end{equation}
this curve covers the torus several times. There is another way
to look at this picture. We may say when we factorize
$\RR^2/\ZZ^2$, we may choose the basis in whatever manner we want.
The canonical choice is $(0,1)$ and $(1,0)$ \dots but we may
choose instead $(a_1,b_1)$ and $(a_2,b_2)$ but we require
\begin{equation}\label{eq:lec14:detCond}
\det\begin{pmatrix}
a_1 & b_1\\
a_2 & b_2
\end{pmatrix}=\pm1,
\end{equation}
so we have
\begin{equation}
\begin{pmatrix}m\\n
\end{pmatrix}=x\begin{pmatrix}a_1\\b_1
\end{pmatrix}+y\begin{pmatrix}a_2\\b_2
\end{pmatrix}
\end{equation}
for any integers $m$ and $n$. This is a linear equation, but we
need integer solutions. This demands the determinant conditions
in Eq \eqref{eq:lec14:detCond}.

If we have $\gcd(a_1,b_1)=1$, then we may always find a
$(a_2,b_2)$ such that
\begin{equation}
\det\begin{pmatrix}
a_1 & b_1\\
a_2 & b_2
\end{pmatrix}=1,
\end{equation}
but that's trivial. To create a non-self-intersecting closed
line, we can choose $\gcd(m,n)$.

What can we say about the complement of the curve? If we delete a
single edge of the cell complex, we end up with a cylinder. How
can we see this? Well, recall how we constructed the torus from
the rectangle through first constructing a cylinder and then
gluing the cylinder's top and bottom together. We simply undo
this last step. For any line described by a relatively prime pair
$m$ and $n$, we can change the coordinates to get the same
picture. What is the fundamental group generated by this guy?
Well, if
\begin{subequations}
\begin{equation}
u=(1,0)
\end{equation}
and
\begin{equation}
v=(0,1)
\end{equation}
then
\begin{equation}
u^{m}v^{n}=1.
\end{equation}
\end{subequations}
Every solid torus has its fundamental group be
$\ZZ$.\index{Fundamental Group!Solid Torus}\index{Solid Torus!Fundamental Group of}
But
\begin{equation}
(\mbox{solid torus})\homotopic S^1
\end{equation}
homotopic, so we see why!

We take $m,n\in\ZZ$ such that $\gcd(m,n)=1$, in the torus this
gives us some information in the fundamental group, namely
\begin{equation}
\pi_{1}(\mbox{torus})=u^{m}v^{n}.
\end{equation}
We have two morphisms to fundamental groups of solid torus:
\begin{equation}
\begin{diagram}[small]
\pi_{1}(\mbox{torus})&=u^{m}v^{n}&\rTo\pi_{1}\begin{pmatrix}\mbox{solid-torus} \end{pmatrix}=v^{n}\\
\dTo                          &&\\
\pi_{1}\begin{pmatrix}\mbox{solid-torus} \end{pmatrix}&=u^{m}
&
\end{diagram}
\end{equation}
We wil use van Kampen's theorem, and use the fact that
\begin{equation}
S^{3}-K=\begin{pmatrix}\mbox{open}\\\mbox{solid}\\\mbox{torus}\end{pmatrix}
\cup\begin{pmatrix}\mbox{open}\\\mbox{solid}\\\mbox{torus}\end{pmatrix}\cup(T^{2}-K)
\end{equation}
for some knot $K$, and $T^{2}=S^1\times S^1$ is the torus. We
would like to apply van Kampen's theorem, and then we have a
problem: the conditions of the theorem are not satisfied. Oru
space 
\begin{equation}
S^{3}-K=A\cup B
\end{equation}
should be represented as the union of open sets $A$, $B$. We
considered morphisms
\begin{equation}
\pi_{1}(B)\gets \pi_{1}(A\cap B) \to \pi_{1}(A)
\end{equation}
but here we do not have this situation. The union of open sets
\emph{are not the whole space}. We also have this piece
$(T^{2}-K)$. If we consider the closures of the open tori, then
$(T^{2}-K)$ is contained in the other two. But this implies
\begin{equation}
\overline{(\mbox{solid torus})}
\cap 
\overline{(\mbox{solid torus})}
=(T^{2}-K),
\end{equation}
but \emph{it is not open!} Well from the homotopy viewpoint,
nothing happens.

First, this intersection has fundamental group
\begin{equation}
\pi_{1}(A\cap B)=\ZZ
\end{equation}
But this generator may be written as $u^mv^n$, which is mapped to
$u^m$ for one of the solid torus' fundamental group, and $v^n$
for the other's. The relation is simple:
\begin{equation}
u^m=v^n.
\end{equation}
We computed the fundamental group of the toric know --- but we
have a question: can we say that toric knots are ``the same''? It
depends on $m,n\in\ZZ$. If we have the generators and relations,
there is no algorithm to determine the group\footnote{This is a
  ``well known'' result in group theory.}. This is not an
expression of our ignorance, but our knowledge: we \emph{know}
there exists no such algorithm.

The first tool is Abelianization\index{Abelianization}\index{Abelianization!of Fundamental Group}. We may
factorize with respect to our commutator. But if
\begin{equation}
\gcd(m,n)=1,
\end{equation}
then
\begin{equation}
\pi_{1}(S^{3}-K)/[-,-]\iso\ZZ,
\end{equation}
but this doesn't work in any case for knots --- Abelianization of
$\pi_{1}(S^{3}-K)$ always results in $\ZZ$. We may take the
fundamental group and factorize with respect to its center
$\bigl(\pi_1/Z(\pi_1)\bigr)$ this is also an invariant of the knot.
So for $\<u,v\mid u^m=v^n\>$, what is the center?

First we see that $u^m$, $v^n$ are both in the center, since the
both commute with $u$ and $v$. They're really generators for the
center. If we consider $\pi_1/(u^m=v^n)$, we get a group
$\<u,v\mid u^m=v^n=1\>$. What we get is the free product
\begin{equation}
\pi_1/(u^m=v^n) = \ZZ_m*\ZZ_n.
\end{equation}
But what can we do? Again, we may take the Abelianization of this
group
\begin{equation}
(\ZZ_m*\ZZ_n)/[-,-]=\ZZ_m\oplus\ZZ_n,
\end{equation}
which becomes the direct sum. We have the same relation plus
commutativity:
\begin{equation*}
\<u,v\mid u^m=v^n=1,uv=vu\>.
\end{equation*}
Since $\gcd(m,n)=1$ we obtain
\begin{equation}
(\ZZ_m*\ZZ_n)/[-,-]=\ZZ_{mn},
\end{equation}
so we get $mn$ is an invariant of the knot. But we still don't
know if $m$, $n$ \emph{separately} are invariants of the knot.

We return to $\ZZ_m*\ZZ_n$, and consider torsion elements of this
group. We may consider the maximal order of the torsion elements,
and one can provide it is $\max(m,n)$. So we have complete
information, assuming $m>0$ and $n>0$ (which we can always
assume). It is sufficient to say both $m$ and $n$ are
invariants. The lesson is \emph{we have some tools to answer
these questions.} If the group is finite, look at the order of
elements, etc.
