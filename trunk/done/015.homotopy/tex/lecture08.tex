%%
%% lecture08.tex
%% 
%% Made by alex
%% Login   <alex@tomato>
%% 
%% Started on  Mon Dec 26 21:17:57 2011 alex
%% Last update Mon Dec 26 21:17:57 2011 alex
%%

We classified homotopy classes $\homotopyClass(S^1,S^1)$ using
angular coordinates. The requirement of continuity is that if
$f\colon S^1\to S^1$, then 
\begin{equation}
f(2\pi)-f(0)=2\pi k
\end{equation}
where $k\in\ZZ$ is the degree of the map. It is an invariant of
the homotopy class, i.e., a ``homotopy invariant''\index{Invariant!Homotopy}.
Later we will see analogously
\begin{equation}
\deg\colon\homotopyClass(S^n,S^n)\iso\ZZ
\end{equation}
for $n\geq1$, where this is a one-to-one correspondence.
But more generically, what is the meaning of the ``degree of a
map''?

It tells us how many times the codomain is covered by the
domain. Let us take for definiteness $f(0)=0$. The simplest case
is
\begin{equation}
f(\alpha)=k\alpha
\end{equation}
What happens? Look, we see that the image raps around th circle
$k$ times, so $f^{-1}$ is a one-to-$k$ function. More precisely,
\begin{equation}
f^{-1}(0)=\left\{0,\frac{2\pi}{k},2\frac{2\pi}{k},\dots,(k-1)\frac{2\pi}{k}\right\}.
\end{equation}
But this is a very simple case. We should solve
\begin{equation}
f(\alpha)=0\bmod2\pi,
\end{equation}
we know
\begin{equation}
f'(\alpha)<0\quad\mbox{or}\quad f'(\alpha)>0
\end{equation}
implies the curve grows with a positive/negative slope between
the roots. We could say that the degree is the algebraic number
of solutions to the equation
\begin{equation}
f(\alpha)\equiv\alpha_0\bmod2\pi
\end{equation}
where $f'(\alpha_0)\not=0$. We can only calculate the
\emph{algebraic} number of solutions for an equation.

\index{Fundamental Theorem of Algebra!proof|(}
We can prove the main (fundamental) theorem of algebra. One way
to prove it is very simple (our proof will not be rigorous!). A
trivial proof of the fact that every algebraic equation has a
solution. What is an algebraic equation? We have a polynomial
\begin{equation}
p(x)=x^{n}+a_{1}x^{n-1}+\dots+a_{n},
\end{equation}
which is a map. It is a map
\begin{equation}\label{eq:lec8:pMapsS}
p\colon S^2\to S^2
\end{equation}
where we recall
\begin{equation}
S^2=\RR^2\cup\{\infty\}
\end{equation}
by Stereographic Projection\index{Stereographic Projection}.
But
\begin{equation}
p(\infty)=\infty
\end{equation}
so our claim in Eq \eqref{eq:lec8:pMapsS} is correct. We can
compute the degree of this map, and that is very easy since the
degree is an invariant of the homotopy class. And look, we will
write
\begin{equation}
p_{t}(x)=x^{n}+t\bigl(a_{1}x^{n-1}+\dots+a_{n}\bigr)
\end{equation}
what do we see? We see
\begin{subequations}
\begin{equation}
p_{1}(x)=p(x)
\end{equation}
is our original polynomial, and
\begin{equation}
p_{0}(x)=x^{n}
\end{equation}
\end{subequations}
is the same degree as our original polynomial. But the degree of
the map $x\mapsto x^n$ is very simple, since
\begin{equation}
p(x)\equiv\alpha\pmod2\pi
\end{equation}
has $n>0$ solutions (well, $n\not=0$ solutions). Thus $p(x)$ is a
map of degree $n$, or the algebraic number of solutions is
$n$. We have no right in replacing $t$ to be
\begin{equation}
\widetilde{p}_{t}(x)=a_{k-1}x^{k}+t\bigl(x^{n}+a_{1}x^{n-1}+\dots\bigr)
\end{equation}
otherwise we violate continuity.

There is a simple trick to prove $p(x)$ has a solution. We want
to prove $p(x)$ is zero somewhere. Let us assume it is
\emph{nowhere zero}. We take
\begin{equation}
|x|=b
\end{equation}
and consider
\begin{equation}
\varphi(x)=\frac{p(x)}{|p(x)|}
\end{equation}
we can do this as we assumed $p(x)\not=0$ for any $x$. It is a
continuous map from a circle to a circle. Let us calculate the
degree of this map. We take $b=\varepsilon>0$ to be small, then
$\varphi\homotopic0$ homotopic. We can take $b\gg0$, but then
it is easy to see only the first term $x^n$ dominates. We've seen
that the degree of $p$ is $n$ in this case, so the degree of
$\varphi$ would be nonzero. A contradiction! Caused by what? By
dividing by $p(x)$ since we cannot divide by 0.
\index{Fundamental Theorem of Algebra!proof|)}
