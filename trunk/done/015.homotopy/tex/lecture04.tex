%%
%% lecture04.tex
%% 
%% Made by alex
%% Login   <alex@tomato>
%% 
%% Started on  Mon Dec 26 17:57:21 2011 alex
%% Last update Mon Dec 26 17:57:21 2011 alex
%%


\begin{wrapfigure}{l}{0.868in}
  \vspace{-20pt}
  \includegraphics{img/lecture4.0}
  \vspace{-20pt}
\end{wrapfigure}
The last thing we did was consider something called a
``handle''. It is a torus, but we cut out a hole. We take our
hole anywhere we'd like. So lets have it be touching a vertex, as
doodled on the left.

Now what we claim is that we have an equality of cell complexes:
\begin{equation}
\includegraphics{img/lecture4.0}\quad
\begin{array}{c}
= \\
\quad \\
\quad \\
\quad\\
\end{array}\quad
\includegraphics{img/lecture4.1}
\vspace{-24pt}
\end{equation}
We obtain the right hand side through a cut. But is a cut
allowed? Usually not, but since all the vertices are the same so
it's okay here. We identify the base points to be the same. We
also consider the Euler characteristic. Observe
\begin{equation}
\begin{split}
\chi(\mbox{handle}) &= (\mbox{1 vertex})-(\mbox{3 edges})
+(\mbox{1 face})\\
&= -1.
\end{split}
\end{equation}
This is just the first step in our considerations.

\begin{wrapfigure}{r}{1.5in}
  \vspace{-20pt}
  \centering
  \includegraphics{img/lecture4.2}
  \vspace{-20pt}
\end{wrapfigure}
The next thing we may consider is a sphere with several holes,
and we paste on each hole a handle, as doodled on the right. So
this is a sphere with $g$-holes and each hole we glue a handle to
it. We should draw its cell complex as a polygon with
$4g$-edges. What is the Euler characteristic for this surface? We
can calculate it quickly as:
\begin{subequations}
\begin{align}
\chi(\mbox{$g$-handled sphere})
&=g\chi(\mbox{handles})+\chi(\mbox{sphere with $g$ holes})\\
&=g\chi(\mbox{handles})+\chi\bigl((S^{2}-\mbox{$g$ holes}-(\mbox{$g$ holes}\bigr)\\
&=g(-1)+(2-g),
\end{align}
\end{subequations}
where we quickly compute
\begin{equation}
\chi(S^{2})=\chi\bigl((S^{2}-\mbox{$g$ discs})\cup(\mbox{$g$ discs}\bigr)=2
\end{equation}
for the sphere.

\begin{wrapfigure}{r}{4.25pc}
  \vspace{-24pt}
  \includegraphics{img/lecture4.3}
  \vspace{-24pt}
\end{wrapfigure}
One more interesting thing, take a rectangle and identify
opposite edges and reverse orientation. We doodle this on the
right. We will cut this to get 

\begin{wrapfigure}{l}{4.25pc}
  \vspace{-16pt}
  \includegraphics{img/lecture4.4}
  \vspace{-16pt}
\end{wrapfigure}
\noindent two triangles along the dashed
line. We erase the vertical line distinguishing the two
triangles and we get the cell complex doodled to the left. This
is the cell structure for the Mobius band\index{Mobius Band!Cell Structure|textbf}.
The boundary of the Mobius band is a circle. We may take a sphere
with several punctures and paste Mobius bands instead of
handles. We do not want to go into the theory of surfaces, so we
leave it to the reader's imagination how this is done.

We will work with compact 2-dimensional manifolds.

\begin{thm}
All compact 2-dimensional manifolds are spheres with handles or
Mobius bands.
\end{thm}

\subsection{Homotopy}
Now we would like to go to the definition of homotopy. Before
going to this topic, we'd like to discuss operations on
topological spaces. Let us take two sets $A$, $B$ and we may
construct their (direct) product
\begin{equation}
A\times B=\{(a,b)\mid a\in A, b\in B\}.
\end{equation}
Now everything is very simple, if we have
\begin{equation}
f\colon X\times Y\to Z,
\end{equation}
then this map is a function of two variables $f(x,y)$. That's
obvious. Now there is a standard procedure. We may consider one
of the variables as a \emph{parameter}. Fix $x$, we get a map
\begin{equation}
f_{x}\colon Y\to Z
\end{equation}
We say\index{Currying Functions}
\begin{equation}
\hom(X\times Y,Z)=\hom\bigl(X,\hom(Y,Z)\bigr).
\end{equation}
This is a completely trivial formula.

\index{Product Topology|(}
We now say that $A$, $B$ are topological spaces. Then $A\times B$
may also be considered as a topological space. If $a\in A$ has  a
neighborhood $a\in U\propersubset A$, and similarly let $b\in
V\propersubset B$ be a neighborhood, then
\begin{equation}
(a,b)\in U\times V\propersubset A\times B
\end{equation}
is a neighborhood.  And all other open sets of $A\times B$ are
obtained by arbitrary unions and finite intersections of these
guys.
\index{Product Topology|)}

Now, we have topological spaces $X$, $Y$. We may speak of
continuous maps
\begin{equation}
f\colon X\times Y\to Z
\end{equation}
and consider this construction $f(x,y)=f_{x}(y)$, obtaining a map
\begin{equation}\label{eq:lec04:belovedFormula}
f_{x}\colon Y\to Z
\end{equation}
fixing $x\in X$. What about our beloved formula \eqref{eq:lec04:belovedFormula}? Is it correct
when the maps we take are continuous maps? It's a meaningless
question, we don't know the topology of continuous maps $Y\to
Z$. There is a meaningful question, namely what is the topology
of the set of continuous maps $\hom(Y,Z)$? This is not entirely
honest, but not dishonest either! We define the topology on
$\hom(Y,Z)$ to satisfy
\begin{equation}
\hom(X\times Y,Z)=\hom\bigl(X,\hom(Y,Z)\bigr).
\end{equation}
We would like to construct topological invariants (for this space
of continuous maps). 

One is the number of connected components. Consider $\hom(X,Y)$,
if we fix $X$ it's a topological invariant for $Y$; and if we fix
$Y$, the number of connected components of $\hom(X,Y)$ is a
topological invariant for $X$. 
But still we may consider the number of components for
$\hom(X,Y)$, and we indicate this by\index{$\homotopyClass(X,Y)$|textbf}
\begin{equation}
\hom(X,Y)=\homotopyClass(X,Y)
\end{equation}
denoted with the brackets. Lets rephrase this in a more
pedestrian way. Remember a component is connected if for any pair
of points, there is a path connecting them. We
say\index{$f\homotopic g$!Homotopic Maps}\index{Homotopic!Maps|textbf}
\begin{equation}
f_{0}\homotopic f_{1}\mbox{ homotopic}
\end{equation}
if and only if there is a path in the space of maps connecting
these guys. What does it mean? Well, we have a path $f_{t}$ where
as $t\in[0,1]$ varies and $f_{t}\in\hom(X,Y)$ continuously
varies. So really, it's a path in $\hom(X,Y)$. But it is
specifically such that
\begin{equation}
f_{t}|_{t=0}=f_{0}\quad\mbox{and}\quad
f_{t}|_{t=1}=f_{1}
\end{equation}
This may be seen as a deformation of the path from $f_0$ to
$f_1$. We avoid difficulties by writing
\begin{equation}
f_{t}(x)=f(x,t)
\end{equation}
as a continuous family of paths. We can now give another
definition; let
\begin{equation}
f_{0},f_{1}\colon X\to Y
\end{equation}
be continuous, we say they are
\begin{equation}
f_{0}\homotopic f_{1}\mbox{ homotopic}
\end{equation}
if there exists a map
\begin{equation}
F\colon X\times I\to Y
\end{equation}
such that
\begin{equation}
F(x,0)=f_{0}(x)\quad\mbox{and}\quad F(x,1)=f_{1}(x).
\end{equation}
We did nothing new, we just gave a different definition of what
we had\footnote{Alright, you got me, it's not even a different
  definition: it's just slightly different notation!}.

\begin{wrapfigure}{r}{1in}
  \vspace{-20pt}
  \centering
  \includegraphics{img/lecture4.5}
\end{wrapfigure}

The first example is $S^{1}\to\RR^2$. It's a closed curve, as
doodled on the right. We see every such map is homotopic to the
zero map. How do we see this? Well, the light gray lines indicate
the $t$ value, with $t=1$ being the outer most curve and $t=0$
being the centered dot. This means that $S^{1}\propersubset\RR^2$  
is contractible to a point. We may contract $\RR^2$ to a point
since
\begin{equation}
\id{\RR^2}\homotopic\mbox{trivial map}
\end{equation}
homotopic, which implies $\RR^{2}$ is contractible. We can write 
\begin{equation}
f_{t}(x)=tx
\end{equation}
but that's a triviality.

Lets consider the simplest nontrivial case. Lets take
\begin{equation}
S^{1}\to\RR^{2}-0
\end{equation}

\begin{wrapfigure}{r}{2pc}
  \vspace{-20pt}
  \centering\includegraphics{img/lecture4.6}
  \vspace{-20pt}
\end{wrapfigure}
Here we have the following picture: if the point deleted is
inside the circle, we cannot do anything. We cannot contract it
to a point. We could, on the other hand, 

\begin{wrapfigure}{l}{4pc}
  \vspace{-16pt}
  \centering\includegraphics{img/lecture4.7}
  \vspace{-20pt}
\end{wrapfigure}
\noindent consider maps that go
around the deleted point twice. This map is not homotopic to
either zero or the map which goes around the deleted point
once. Convince yourself this is the only topological invariant.

\index{Homotopic!Spaces}We have a space $X$ and a space $Y$, we have maps
\begin{equation}
f\colon X\to Y\quad\mbox{and}\quad g\colon Y\to X
\end{equation}
We say $X\homotopic Y$ homotopic if and only if
\begin{equation}
f\circ g\homotopic\id{Y}\quad\mbox{and}\quad g\circ
f\homotopic\id{X}.
\end{equation}
\begin{thm}
If we have homotopically equivalent spaces $X$ and $Y$, then
$\homotopyClass(A,X)=\homotopyClass(A,Y)$ for any space $A$, and
$\homotopyClass(X,B)=\homotopyClass(Y,B)$ for any space $B$.
\end{thm}
We call $\homotopyClass(X,Y)$ the \define{Homotopic Classification of Maps}.
\index{Homotopy!Classification of Maps}
We are going to prove that this relation $X\homotopic Y$
homotopic, sometimes called ``homotopic equivalence'', is really
an equivalence relation.
\exercises
\begin{xca}[Reflexivity]
Prove or find a counter-example: for any $X$, $X\homotopic X$
homotopic. 
\end{xca}
\begin{xca}[Symmetry]
Prove or find a counter-example: for any $X$, $Y$ topological
spaces, $X\homotopic Y$ if and only if $Y\homotopic X$.
\end{xca}
\begin{xca}[Transitivity]
Prove or find a counter-example: let $X$, $Y$, $Z$ be topological
spaces such that $X\homotopic Y$ and $Y\homotopic Z$ homotopic,
both imply  $X\homotopic Z$ homotopic.
\end{xca}
