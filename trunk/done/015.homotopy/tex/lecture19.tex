%%
%% lecture19.tex
%% 
%% Made by alex
%% Login   <alex@tomato>
%% 
%% Started on  Tue Dec 27 20:31:28 2011 alex
%% Last update Tue Dec 27 20:31:28 2011 alex
%%

%\marginpar
{\textbf{TODO:}\quad{}Write up the notess on homotopy
groups from Schwarz's book~\cite[Ch.\ 8]{schwarz}.}

\index{Homotopy Group!from Loop Space|(}
\marginpar{Iterative construction of homotopy groups}We will start with a reminder of the fundamental group. We took a
space $X$ with a marked point $*\in X$. We take $\Omega(X,*)$ to
be the space which consists of loops in $X$ that have base point
$*$. We considered the connected components of the space
$\Omega$, i.e., $\pi_{0}(\Omega,*)$. We then just
\begin{equation}
\pi_{1}(X,*)=\pi_{0}(\Omega,*).
\end{equation}
We use topological invariant of $\Omega$ to get a topological
invariant of $X$. But we can use anything we want! We can keep
iterating, and take
\begin{equation}
\pi_{1}(\Omega,*)=\pi_{2}(X,*)
\end{equation}
is a topological invariant of $X$, we have functoriality. If
$f\colon X\to Y$, then $\Omega_{X}\to\Omega_{Y}$.
We can iterate this construction $n$-times to get $\pi_{n}$, but
we will focus on $\pi_{2}$.
\index{Homotopy Group!from Loop Space|)}

\index{Homotopy Group!from Spheroids|(}
\marginpar{Different Construction}%
Look, what are the elements of $\Omega$? They are maps of an
itnerval to $X$:
\begin{equation}
\alpha\colon I\to X
\end{equation}
which are closed paths, i.e.,
\begin{equation}
\alpha(0)=\alpha(1)=*.
\end{equation}
We want to consider the fundamental group of $\Omega$. A path of
$\Omega$ is then of the form $\alpha_{\tau}(t)$ where $\tau$ is
the parameter of the path in $\Omega$. Perhaps it is better to
write
\begin{equation}
\alpha_{\tau}(t)=\alpha(\tau,t).
\end{equation}
We want to have a closed path that starts and ends at
\begin{equation}
\alpha_{\tau}(0)=\alpha_{\tau}(1)=*,\quad\mbox{and}\quad
\alpha_{0}(t)=\alpha_{1}(t)=*.
\end{equation}
We can look at $(\tau,t)\in I^2$ as a square. So in other words,
our path $\alpha$ is a mapping
\begin{equation}
\alpha\colon I^{2}\to X
\end{equation}
such that
\begin{equation}
\alpha(\partial I^{2})=*
\end{equation}
it takes the boundary to the marked point. This mapping is called
a \define{Spheroid}\index{Spheroid}. Why is this name a good one?
Well, recall $I^{2}/\partial I^{2}\iso(S^{2},*)$ is a
homeomorphism. So equivalently, we have it be
\begin{equation}
\alpha\colon(S^2,*)\to(X,*)
\end{equation}
a mapping from the marked 2-sphere to the marked space
$(X,*)$. Using this language, $\pi_{2}$\index{$\pi_{2}$} is the
collection of homotopy classes of spheroids.

\begin{rmk}[``Spheroid'' and Nationality]
IT appears that only the Russian topologists use the word
``spheroid'' in their writing. A cursory google search would
reveal that it is a relatively antiquated or esoteric word, and
indeed only Russians employ it.
\end{rmk}
\begin{ddanger}
We will abuse language and notation to refer to the mapping
and/or the cube as the ``spheroid.''
\end{ddanger}

\begin{wrapfigure}{r}{1.75in}
  \vspace{-20pt}
  \centering
  \includegraphics{img/lecture19.0}
  \end{wrapfigure}
Suppose we have two spheroids $\alpha$ and $\beta$. 
We can construct a third spheroid using
\begin{equation}
\gamma(\tau,t)=\begin{cases}\alpha(2\tau,t) & 0\leq\tau\leq\frac{1}{2}\\ 
\beta(2\tau-1,t) & \frac{1}{2}\leq\tau\leq1
\end{cases}
\end{equation}
This is doodled to the above right, taking the two spheroids
labeled $\alpha$ and $\beta$ and combines them into the third as
we specified. This is precfisely a reformulation of the
definition of $\pi_2$ in terms of $\pi_1$.

\begin{thm}
The group $\pi_{2}(X,*)$ is commutative.
\end{thm}
There are two proofs of this fact. One is long and boring. Ours
is a sequence of pictures:

\includegraphics{img/lecture19.1}

\bigskip
Now, let us introduce the ntoion of a topological
group\index{Topological!Group}\index{Topological Group}\index{Group!Topological}
$G$ which is a group object internal to $\Top$. In other words,
it consists of a topological space $G$ equipped with ``group
structure'': continuous maps $\mu\colon G\times G\to G$ and
$\iota\colon G\to G$. We will prove that
\begin{equation}
\pi_{1}\begin{pmatrix}\mbox{Topological}\\
\mbox{Group}
\end{pmatrix}\in\Ab
\end{equation}
We can define the multipication of paths as one of two ways: (1)
concatenation $\alpha*\beta$; (2) $\gamma(t)=\alpha(t)\beta(t)$
pointwise using the group's multiplication.

These are the same up to homotopy. And that is obvious. Why? Lets
write down
\begin{equation}
\alpha'=\alpha * e
\end{equation}
where the constant path $e(t)=e$ is the multiplicative identity. Obviously
$\alpha'\homotopic\alpha$ homotopic.
Also take
\begin{equation}
\beta'=e*\beta
\end{equation}
then we see
\begin{equation}
\alpha'*\beta' = \alpha'\beta'
\end{equation}
pointwise. 

We can consider the opposite group $G^{\op}$ where multiplication
is defined as
\begin{equation}
x\circ^{\op}y=x\circ y
\end{equation}
But
\begin{equation}
\alpha\beta\homotopic\beta\alpha
\end{equation}
Thus we deduce commutativity up to homotopy. We want to prove
$\pi_{2}$ is commutative. What to do? We apply this statement to
$\Omega$. But it is not a group. It has multiplication, and a
unit. But this is up to homotopy! Nevertheless, everything goes
through for $\Omega$, so $\pi_{1}(\Omega)$, which is an $H$-space
(``almost topological group''), is commutative. Thus $\pi_{2}(X)$
is commutative.
\index{Homotopy Group!from Spheroids|)}
