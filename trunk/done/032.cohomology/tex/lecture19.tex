%%
%% lecture19.tex
%% 
%% Made by alex
%% Login   <alex@tomato>
%% 
%% Started on  Sun Dec 25 11:01:26 2011 alex
%% Last update Sun Dec 25 11:01:26 2011 alex
%%
The first problem will concern the relation between
$K(B)$\index{$K(B)$!relation to $\widetilde{K}(B)$} and
$\widetilde{K}(B)$.\index{$\widetilde{K}(B)$!relation to $K(B)$} We know that 
\begin{equation}
K(B)\iso\widetilde{K}(B)\oplus\ZZ.
\end{equation}
but make it more detailed, namely from this decomposition there
are maps
\begin{equation}
\begin{diagram}[height=1pc]
    &       & \widetilde{K}(B) \\
    & \ruTo & \\
K(B)&       & \\
    & \rdTo & \\
    &       & \ZZ
\end{diagram}
\end{equation}
and
\begin{equation}
\widetilde{K}(B)\to K(B),\quad\mbox{and}\quad\ZZ\to K(B).
\end{equation}
But still describe all of these morphisms.

The second problem is more of a problem: namely, calculate
$K(\CP^n)$ and moreover $K^i(\CP^n)$. We defined
$K^{-i}(B)=K(S^{i}B)$. There is also the Bott periodicity which
states $K^{i}=K^{i+2}$. We do this by induction, noting we have a
pair $\CP^n\propersupset\CP^{n-1}$. Then everything follows.

Now we will go ahead and note that we have a ring with respect to
$K$. If we have two vector bundles $(E,B,F,p)$ and
$(E',B,F',p')$ then we may take the product of these vector
bundles which amounts to the tensor product. At each point $b\in
B$ we take $F_{b}\otimes F_{b}'$; this is a definition that's
correct but not quite precise. For trivial vector bundles,
$U\times F\to U$ and $U\times F'\to U$, we may create
$U\times(F\otimes F')\to U$. We would like to consider two sets
$U,U_{1}\propersubset B$ where the fibration is trivial, then on
$U\cap U_{1}$ we have the clutching function. We have a map
\begin{equation}
\varphi_{\alpha\beta}\colon U_{\alpha}\cap U_{\beta}\to\GL{F}
\end{equation}
which is the clutching function\index{Clutching Function}. We have on $U\cap U_{1}$ two
fibres from two representations of the fibre, we just paste them
together---pasting is done by an element of $\GL{F}$. But for the
other bundle, we have
\begin{equation}
\varphi'_{\alpha\beta}\colon U_{\alpha}\cap U_{\beta}\to\GL{F'};
\end{equation}
is it possible to create a \index{Transition Function}transition
function\marginpar{Transition Function, $\phi_{\alpha\beta}$}
\begin{equation}
\phi_{\alpha\beta}\colon U_{\alpha}\cap U_{\beta}\to\GL{F\otimes F'}?
\end{equation}
Yes, it is easy to do, because there is a map of products of
groups
\begin{equation}
\GL{F}\times\GL{F'}\to\GL{F\otimes F'}.
\end{equation}
This can be done for any linear operators
\begin{equation}
A\colon F\to F,\quad\mbox{and}\quad B\colon F'\to F'
\end{equation}
to get
\begin{equation}
A\otimes B\colon F\otimes F'\to F\otimes F'.
\end{equation}
In any definition of the tensor product, we have
functoriality. So 
\begin{equation}
\phi_{\alpha\beta}=\varphi_{\alpha\beta}\otimes\varphi'_{\alpha\beta}.
\end{equation}
We would like to consider a special case, namely if $F=\CC$ and
$F'=\CC$ are one-dimensional, then the tensor product is
one-dimensional
\begin{equation}
F\otimes F'=\CC,
\end{equation}
and the tensor product is just the usual product, not much more
than that.

Now we go to $K$-theory. There is nothing to say here. What do we
need? We need the definition of elements of the $K$ group. So if
we want to multiply
\begin{equation}
[E]\cdot[E']=[E\otimes E']
\end{equation}
so $E$, $E'$ are vector bundles over $B$. Some complications
arise but they are trivial complications. We have virtual
bundles, so we should define multiplication
\begin{equation*}
([E_{0}]-[E_{1}])(\dots)
\end{equation*}
by merely invoking distributivity. We should prove this is
correct. In particular, we know if we have\dots well we should
recall the tensor product is distributive. We also know it is
associative and commutative, so we're done.

Our goal is to consider $K(S^{2})$ and multiplication therein. We
know that
\begin{subequations}
\begin{equation}
K(S^{2})=\ZZ\oplus\widetilde{K}(S^{2})
\end{equation}
and
\begin{equation}
\widetilde{K}(S^{2})=\pi_{1}\bigl(\U{\infty}\bigr)=\ZZ.
\end{equation}
\end{subequations}
We can describe elements of $\widetilde{K}(S^{2})$. Without loss
of generality, we may consider line bundles. The higher
dimensionality here does not contribute since we consider
\begin{equation}
\pi_{1}\bigl(\U{\infty}\bigr)=\pi_{1}\bigl(\U{1}\bigr)=\dots
\end{equation}
So we use the Hopf bundle
\begin{equation}
S^{3}\xrightarrow{\;S^{1}}S^{2}
\end{equation}
where we let $H=$ the Hopf vector bundle\index{Hopf Vector Bundle}\index{Hopf Bundle}\index{Bundle!Hopf}.

Now we may take the trivial bundle of dimension 1. It is very easy to
understand the role played by this bundle in the $K$-ring. We
know
\begin{equation}
\CC\otimes F=F.
\end{equation}
So this is the unit of tensor multiplication. We have
\begin{equation}
\varepsilon\cdot E=E
\end{equation}
Perhaps we should write $\varepsilon=1$? No, we will instead
write
\begin{equation}
\varepsilon=\mathbf{1}.
\end{equation}
We have
\begin{equation}
\mathbf{1}\cdot H=H,
\end{equation}
but what is $H^{2}=$? Look, $H$ is the Hopf vector bundle that
may have transition function
\begin{equation}
\phi = z
\end{equation}
since $S^{2}=\CC\cup\infty$. We have the transition function. How
to prove this? The simpler way is to consider
\begin{equation}
\pi_{1}\bigl(\U{1}\bigr)=\pi_{1}\bigl(\GL{1,\CC}\bigr).
\end{equation}
The generating elment of $\pi_{1}$, what is it? We have a circle
\begin{equation}
S^{1}=\U{1}\homotopic\GL{1,\CC},
\end{equation}
and the transition function is thus the identity $z\mapsto z$.

But what is $H^{2}$? It is clear the transition function is
$z^{2}$, because it's a line bundle so it amounts to
multiplication in the usual way. Now what about addition? What is
$H\oplus H$? We get 2-dimensional vector bundle, and the
transition functions are matrices
\begin{equation}
\phi = \begin{bmatrix}z & 0\\0&z
\end{bmatrix}.
\end{equation}
Every two-dimensional bundle is the sum of two one-dimensional
bundles. We should map
\begin{equation}
K(S^{2})\xrightarrow{\dim}\ZZ.
\end{equation}
This is, of course,
\begin{equation}
H\oplus H\to\ZZ
\end{equation}
which is the inclusion mapping.

We also have a map
\begin{equation}
K(S^{2})\to\pi_{1}\bigl(\U{\infty}\bigr)
\end{equation}
which is easy to calculate. We have one path but it is diagonal;
we see here that
\begin{equation}
\begin{bmatrix}z&0\\0&z
\end{bmatrix}\mapsto 2.
\end{equation}
Now we may write down what is $H^{2}$. The formula is that
\begin{equation}
H^{2}=2H-\mathbf{1}.
\end{equation}
This is precisely what we wanted to get! It's correct in
$\widetilde{K}$, we get $\dim(H^{2})=1$ and $\dim(2H)=2$ while
$\dim(-\mathbf{1})=-1$.


