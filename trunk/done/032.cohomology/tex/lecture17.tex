%%
%% lecture17.tex
%% 
%% Made by alex
%% Login   <alex@tomato>
%% 
%% Started on  Sat Dec 17 13:39:36 2011 alex
%% Last update Sat Dec 17 13:39:36 2011 alex
%%
Remember that $B$ is a connected cell complex, we defined a
notion of stable equivalence of vector bundles, using the
notation
\begin{equation}
\varepsilon^n=B\times\CC^n
\end{equation}
for the trivial vector bundles over $B$, $B\times\CC^n\to B$. We
consider vector bundles over $B$. We are saying there exists a
stable equivalence
\begin{equation}
E\stabequiv E'\iff E\oplus\varepsilon^m\iso E'\oplus\varepsilon^n.
\end{equation}
We defined
\begin{equation}
\widetilde{K}(B)=\{\mbox{stably equivalence classes}\}
\end{equation}
as the stable-equivalence classes of vector bundles over $B$. So
we have
\begin{equation}
\widetilde{K}(B)=\{\mbox{vector bundles over }B\}/\stabequiv.
\end{equation}
We have addition, which turns $\widetilde{K}(B)$ into an Abelian group.
Is this really an Abelian group?

Lets verify this claim, $\widetilde{K}(B)$ equipped with
addition, is an Abelian group. Is there an additive inverse?
Well, we should note that we may always embed a vector bundle $E$
into a trivial bundle
\begin{equation}
E\subset\varepsilon^N
\end{equation}
so
\begin{equation}
E\oplus E^{\bot}=\varepsilon^N.
\end{equation}
But $\varepsilon^N\stabequiv0$ is the identity element, which
implies that $E^\bot$ is the additive inverse of $E$.

The last thing we did was calculate $\widetilde{K}(S^{r})$; we
know vector bundles over $S^{r}$ are in one-to-one correspondence
with $\pi_{r-1}\bigl(\U{k}\bigr)$ where $\U{k}$ is the group
acting on the $k$-dimensional vector bundle. We may add to $E$ a
trivial guy $E+\varepsilon^N$. We have
$\widetilde{K}(S^{r})=\pi_{r-1}\bigl(\U{\infty}\bigr)$. 

\begin{prob}
What is $\widetilde{K}(S^{r})$?
\end{prob}

By the Bott periodicity theorem\index{Bott Periodicity}, we have
\begin{equation}
\pi_{r+2}\bigl(\U{\infty}\bigr)=
\pi_{r}\bigl(\U{\infty}\bigr).
\end{equation}
It is clear that the group is connected implying
$\pi_{0}\bigl(\U{\infty}\bigr)=0$. We also have
$\pi_{1}\bigl(\U{\infty}\bigr)=\ZZ$ which is true for every
$\U{n}$, $n\in\NN$, that $\pi_{1}\bigl(\U{n}\bigr)=\ZZ$. We also
see that
\begin{subequations}
\begin{equation}
\SU{n}\propersubset\U{n}
\end{equation}
\begin{equation}
\U{n}/\SU{n}=S^{1}=\U{1}
\end{equation}
and
\begin{equation}
\pi_{1}\bigl(\SU{n}\bigr)=0.
\end{equation}
\end{subequations}
We knw
\begin{equation}
\pi_{2k}\bigl(\U{\infty}\bigr)=0,
\end{equation}
but
\begin{equation}
\pi_{2}(G)=0
\end{equation}
for any Lie group $G$. We also know that
$\pi_{2k+1}\bigl(\U{\infty}\bigr)=\ZZ$. The answer is that
\begin{equation}
\widetilde{K}(S^{r})=\begin{cases} 0 & \mbox{if $r$ is odd}\\
\ZZ & \mbox{if $r$ is even}.
\end{cases}
\end{equation}

We may repeat all this for real vector bundles but this requires
changing $\U{n}\to\ORTH{n}$. Then
\begin{equation}
\widetilde{K}_{\text{real}}(S^{r})=\pi_{r-1}\bigl(\ORTH{\infty}\bigr).
\end{equation}
The Bott periodicity for the orthogonal group has period 8, i.e., $\pi_{r+8}\bigl(\ORTH{\infty}\bigr)=\pi_{r}\bigl(\ORTH{\infty}\bigr)$.
We write down our table
\begin{equation*}
\begin{array}{c|c|c|c|c|c|c|c}
\pi_{0}\bigl(\ORTH{\infty}\bigr) &
\pi_{1}\bigl(\ORTH{\infty}\bigr) &
\pi_{2}\bigl(\ORTH{\infty}\bigr) &
\pi_{3}\bigl(\ORTH{\infty}\bigr) &
\pi_{4}\bigl(\ORTH{\infty}\bigr) &
\pi_{5}\bigl(\ORTH{\infty}\bigr) &
\pi_{6}\bigl(\ORTH{\infty}\bigr) &
\pi_{7}\bigl(\ORTH{\infty}\bigr) \\\hline
\ZZ_2 & \ZZ_2 & 0 & \ZZ & 0 & 0 & 0 & \ZZ
\end{array}
\end{equation*}
We have
$\pi_{r}\bigl(\Sp{\infty}\bigr)=\pi_{r\pm4}\bigl(\ORTH{\infty}\bigr)$
to get
$\pi_{4}\bigl(\ORTH{\infty}\bigr)=\pi_{0}\bigl(\Sp{\infty}\bigr)$,
and everything is completely determined.

\bigskip
\marginpar{A second approach to $K$-theory}Lets focus on a different explanation of the $K$ group. Consider
vector bundles $E$, $E'$. If we know  $E\oplus E'$, we want to
consider the subtraction $E-E'$. In second grade, we did
subtraction $3-5$ by introducing a new number called
``$-2$''. See \hyperref[box:reviewAlgebraicConstructions]{Box \ref*{box:reviewAlgebraicConstructions}}.
Lets \emph{formally} introduce $E-E'$ by introducing $-E'$ as a
\define{Virtual Vector Bundle}\index{Virtual Vector Bundle}\index{Vector Bundle!Virtual}.
We should ask ourselves what is the meaning of
\begin{equation}
E-E'\stackrel{?}{=}E_{1}-E'_{1}?
\end{equation}
This may be rewritten as
\begin{equation}
E+E_{1}'=E_{1}+E'
\end{equation}
which gives us our definition.  If we consider these differences
we get a group, this is the \define{Grothendieck Construction}\index{Grothendieck Construction}
and it may be applied to any commutative semigroup. We get $K(B)$
as a group. We may say that $K(B)\onto\widetilde{K}(B)$ 
(are the two equal? Not really\dots). Why can we map it? It's
clear that every element of $K(B)$ is a vector bundle, but an
element of $\widetilde{K}(B)$ is an equivalence class of vector
bundles (under stable-equivalence relation).

\begin{Boxed}{Review of Algebraic Constructions}\label{box:reviewAlgebraicConstructions}
The basic approach taken with $K$-groups from vector bundles
makes sense if we remember the construction of the integers from
the natural numbers.

Remember, when we constructed the integers from the natural
numbers $\NN_0$ we did the following: consider the collection
$\NN_0\times\NN_0$. Impose an equivalence relation
\begin{equation}
(a,b)\sim(x,y)\iff a+y=x+b
\end{equation}
and this equivalence class intuitively corresponds to $a-b$. 

\begin{prob}
Consider the image of the diagonal map
$\Delta\colon\NN_0\to\NN_0\times\NN_0$. Does this correspond to
an equivalence class? If so, what equivalence class?
\end{prob}

If we pick out the representatives where at least one component is
zero, then we have our intuitive notion of the integers. We just
make $(0,1)=-1$ and $(1,0)=1$. So we have produced from an
Abelian monoid an Abelian group.
\end{Boxed}

So what's going on? We have two pictures, 
lets consider both of these guys up to stable equivalence
$E_{s}\oplus(-E)_{s}=0$. Note:
\begin{subequations}
\begin{equation}
\mbox{in $\widetilde{K}(B)$}\qquad E_{s}\oplus
E_{s}^{\bot}=\varepsilon^{N}
\end{equation}
\begin{equation}
\mbox{in $K(B)$}\qquad E\oplus(-E)=0.
\end{equation}
\end{subequations}
So this map is a surjective group morphism
$K(B)\onto\widetilde{K}(B)$ and it has a kernel. Under this
morphism, $\varepsilon^N\mapsto0$. It is easy to check that
nothing else goes to zero. So
\begin{equation}
K(B)/\ZZ=\widetilde{K}(B)
\end{equation}
where $\ZZ$ refers to the dimension of the trivial vector bundle.

We may say that
\begin{equation}
K(B)\iso \ZZ\oplus\widetilde{K}(B).
\end{equation}
From the viewpoint of doing calculations, $K(B)$ and
$\widetilde{K}(B)$ are equivalent. We don't worry about dimension
with $\widetilde{K}(B)$---it's not even defined!---but the notion
of dimension remains in $K(B)$.

Also $K(B)$ is a ring. Why? Because there exists the tensor
product operation for two vector bundles and one can check this
operation satisfies the axioms for rings. In particular,
one-dimensional trivial vector bundle
$\varepsilon^{1}=\mathbf{1}$ is the unit element. \textsc{Hint:}
instead of embedding $\widetilde{K}$ into $K$, we may consider a
morphism $\dim\colon K(B)\to\ZZ$.

\index{K-Theory@$K$-Theory!and Supersymmetry|(}\marginpar{A third, super, approach}Another approach to
$K$-theory: work with $\ZZ_{2}$-graded vector bundles. We have
\begin{equation}
E=E_{\mathbf{0}}\oplus E_{\mathbf{1}}
\end{equation}
where one is ``even'' $E_{\mathbf{0}}$ and the other $E_{\mathbf{1}}$ is
``odd''. If 
\begin{equation}
E=\sum_{n\in\ZZ}E_{n}
\end{equation}
then
\begin{subequations}
\begin{equation}
E_{\mathbf{0}}=\sum_{n\in\ZZ}E_{2n}
\end{equation}
and
\begin{equation}
E_{\mathbf{1}}=\sum_{n\in\ZZ}E_{2n+1}.
\end{equation}
\end{subequations}
We have the notion of parity inversion $\Pi\colon E\mapsto\Pi{E}$
where
\begin{equation}
(\Pi E)_{\mathbf{0}}=E_{\mathbf{1}}\quad\mbox{and}\quad
(\Pi E)_{\mathbf{1}}=E_{\mathbf{0}}.
\end{equation}
If every fibre is $\ZZ_2$-graded, this is a $\ZZ_2$-graded vector
bundle. We may attempt to classify $\ZZ_2$-graded vector bundles
if we assume
\begin{equation}
E\oplus\Pi E\sim 0.
\end{equation}
Then $K$-theory is a theory of $\ZZ_2$-graded vector bundles with
\begin{equation}
K(B)=(\mbox{$\ZZ_2$-graded vector bundles over $B$})/(E\oplus\Pi{E}\sim0)
\end{equation}
and we definitely get an Abelian group. It is easy to see this is
equivalent to the old definition.

\begin{rmk}
Does multiplication respect parity? Yes, but when you multiply,
you do it carefully\dots
\end{rmk}
\index{K-Theory@$K$-Theory!and Supersymmetry|)}
