%%
%% lecture03.tex
%% 
%% Made by Alex Nelson
%% Login   <alex@tomato3>
%% 
%% Started on  Sat May 21 13:00:19 2011 Alex Nelson
%% Last update Sat May 21 13:35:00 2011 Alex Nelson
%%
Today we will discuss the classification of principal fibre
bundles, principal fibrations. Recall if a space $E$ has a right
action of a (topological) group $G$, and if the action is free,
then there is a one-to-one correspondence between $G$ and its
orbits. So we have a fibration
\begin{equation}
p\colon E\to E/G
\end{equation}
A map of principal fibrations 
\begin{equation}
f\colon E\to E'
\end{equation}
is a $G$-map
\begin{equation}
f(xg)=f(x)g
\end{equation}
but
\begin{equation}
F_{x}=\{xg\lst g\in G\}\to F_{f(x)}'=\{f(x)g\lst g\in G\}
\end{equation}
describes the behavior on the fibres.

There is a reduced map
\begin{equation}
\varphi\colon E/G\to E'/G
\end{equation}
which is a map
\begin{equation}
B\to B'
\end{equation}
But this is a special map, we still have this property of being a
$G$-map. The equivalence of fibrations is given by a $G$-map
that's a homeomorphism.

\begin{ex}
Consider
\begin{equation}
B=\{b\}
\end{equation}
a singleton. Let 
\begin{equation}
x\in E
\end{equation}
so
\begin{equation}
E=\{xg\lst g\in G\}.
\end{equation}
We have one orbit.

It is clear, if we have another space $E'$, that a map
\begin{equation}
f\colon E\to E'
\end{equation}
and if we know the image of this one point, we know everything!
We have
\begin{equation}
x\mapsto f(x)
\end{equation}
and
\begin{equation}
xg\mapsto f(x)g
\end{equation}
The fibre in $E$ is mapped to a fibre in $E'$. We may consider
the map
\begin{equation}
\varphi\colon E/G\to E'/G
\end{equation}
described by
\begin{equation}
\varphi([x])=(p'\circ f)(x)=p'\left(f(x)\right),
\end{equation}
where
\begin{equation}
p'\colon E'\to E'/G.
\end{equation}
So do we know what happens with $x$? We have
\begin{equation}
\begin{diagram}[small]
  x          & \rMapsto & f(x) \\
\dMapsto>{p} &          & \dMapsto>{p'}\\
 b           & \rMapsto & b'
\end{diagram}
\end{equation}
We see that 
\begin{equation}
\underbracket[0.5pt]{F_{b}}_{=p^{-1}(\{b\})}\to \underbracket[0.5pt]{F_{b'}'}_{=(p')^{-1}(\{b'\})}
\end{equation}
is notr arbitrary. What can we say? Every point of the fibre may
be identified with $G$, so can we write
\begin{equation}
F_{b}=G?
\end{equation}
Yes and no. We need to identify a base point $x$ so we have 
\begin{equation}
xg\sim g
\end{equation}
but this is \emph{not unique}. This identification does not
preserve the group structure. So $G$ may be thought of as a
$G$-space. This identification respects right action only, not
left action.
In the same way
\begin{equation}
F_{b'}'=G
\end{equation}
may be identified, namely
\begin{equation}
x'g\sim g
\end{equation}
satisfies the identification. Now when we map the fibre ot the
fibre, we get
\begin{equation}
f(xg)=f(x)g
\end{equation}
\textbf{BUT} $f(x)$ is not necessarily $x'$. This map, it turns
out, is left action.
\end{ex}
Consider the map 
\begin{equation}
\alpha\colon G\to G
\end{equation}
which commutes with right action, how to describe it?
Well,
\begin{equation}
\alpha(g)=hg
\end{equation}
Why? Well, this definitely commutes with right action because
multiplication on the left commutes with multiplication on the
right by associativity
\begin{equation}
\alpha(gk)=\alpha(g)k.
\end{equation}
Therefore we may say really this map\dots

We may explain this in a little different way to describe these
maps when $B$ is a singleton. We simply identify the fibre,
identify them both with $G$, then it's merely multiplication on
the left.

But what about the case when $B$ is not a singleton. Easy, $B$ is
not a point but consists of points, so
\begin{equation}
\begin{diagram}[small]
F_{b} & \rDashto^{hg} & F_{b'}' = (p')^{-1}(b')\\
\dTo &                & \dTo \\
\{b\}& \rMapsto       & \{b'\}
\end{diagram}
\end{equation}
and we can identify this map by multiplication on the left.
Well, we take
\begin{equation}
B=B'
\end{equation}
and assume that $\varphi$ is the identity map. Then, what should
we do? Simply consider a map for every $b$ a map
\begin{equation}
F_{b}\to F_{b}'
\end{equation}
since 
\begin{equation}
b=b'.
\end{equation}
These maps are labeled by $h\in G$, so everything is very
simple. Well, everything is fine if everything is continuous.

Lets write
\begin{equation}
h\in G_{b}
\end{equation}
where $G$ depends on $b\in B$. So for every $b\in B$ we have the
set $G_{b}$ which is equivalent to $G$ but there is no natural
identification.

We reduced our problem to finding a continuous section of some
fibration. But we may prove a lot of things by triviality.
\begin{prop}
If $B$ is contractible, then the fibration is trivial $E=B\times F$.
\end{prop}
If $B$ is contractible, then there exists a continuous
section. For principal bundles, we have the additional property
that the existence of sections implies the fibration is trivial.

Suppose we have a continuous family of principal fibrations. If
we have
\begin{equation}
E\to B=E/G,
\end{equation}
then we may include it in
\begin{equation}
\widetilde{E}\to B\times I
\end{equation}
where $I=[0,1]$ is the closed interval. Then we may consider two
fibrations: one over the left side of the interval, the other
over the right side. That is
\begin{equation}
\begin{diagram}[small]
\widetilde{E}  & \quad & \widetilde{E} & \quad & \widetilde{E}\\
\dTo           & \quad & \dTo          & \quad & \dTo \\
B\times\{0\}   & \quad & B\times(0,1)  & \quad & B\times\{1\}
\end{diagram}
\end{equation}
we are saying these fibrations are homotopic. We have a
continuous deformation from left to right.
\begin{prop}
If two principal fibrations are homotopic, they are equivalent.
\end{prop}
How to prove this? In reality, we proved it already. We reduced
the problem of finding maps of fibrations to finding a section of
some auxiliary fibration.
\begin{lem}
If we have a fibration with a base $B\times I$ and the fibration
has a section over $B$, then it has a section over the whole base
(i.e., $B\times I$).
\end{lem}
This is a complete triviality.

We take the case when $G$ is connected and
\begin{equation}
B=S^{n}.
\end{equation}
Then we will prove that classes of principal fibrations over
$S^{n}$ are in one-to-one correspondence with homotopy classes
$\homotopyClass(S^{n-1},G)$. 

Let us take a sphere, it may be considered to be pasted together
by a pair of $n$-dimensional discs, i.e., we have two
hemispheres. Over each hemisphere, the fibration is trivial. Why?
The hemisphere is a disc, and the disc is contractible. Now we
shall paste the fibrations together. This requires a transition
function. We see that
\begin{equation}
(B_{1}\times G)\propersupset (S^{n-1}\times G)\propersubset
(B_{2}\times G)
\end{equation}
where $S^{n-1}\times G$ is on the equator, and we have an
equivalence
\begin{equation}
S^{n-1}\times G\to S^{n-1}\times G.
\end{equation}
This gives a map
\begin{equation}
S^{n-1}\to G
\end{equation}
The homotopy class of this map does not depend on any of the
choices we've made.
