%%
%% lecture15.tex
%% 
%% Made by alex
%% Login   <alex@tomato>
%% 
%% Started on  Sat Dec 17 11:11:56 2011 alex
%% Last update Sat Dec 17 11:11:56 2011 alex
%%
We gave several definitions and we would like to repeat them and
give a couple of new ones, all of them are useful. We may prove
their equivalence later.

\marginpar{Definition 1}A \define{Chern Class}\index{Chern Class} are obstructions for complex manifolds
with structure group $\GL{n}$ or $\U{n}\propersubset\GL{n}$. We are
working with complex vector bundles. We had taken the space
$V_{n,k}$ (the Stiefel manifold\index{Stiefel Manifold}) and $\U{n}$ or $\GL{n,\CC}$
acts there. Well, they act on $V^{orth}_{n,k}$ and $V_{n,k}$
respectively. The Chern class was introduced as the first
obstruction to the corresponding bundle. So $c_{l}$ comes in
every even $k$ since $\pi_{l+1}(V_{n,l})\not=0$ for us, and we
get $H^{l+1}\bigl(B,\pi_{l}(F)\bigr)$. The $l$ can be expressed
in terms of $k$, namely $l=2k\pm1$ (the $\pm1$\dots well, look up
the sign). So $c_{k}\in H^{2k}(B,\ZZ)$.

There are other definitions.\marginpar{Definition 2} We can define the characteristic
class as $c_{k}\in H^{2k}(B_{\U{n}},\ZZ)$ for the classifying
space $B_{\U{n}}$. We use the classifying map 
\begin{equation}
\varphi\colon B\to B_{\U{n}}
\end{equation}
then use
\begin{equation}
\varphi^{*}\colon H^{2k}(B_{\U{n}},\ZZ)\to H^{2k}(B,\ZZ)
\end{equation}
to find $\varphi^{*}(c_{k})$. We know
$B_{\U{n}}=\Gr_{\infty,n}$. \hyperref[defn:maximalTorus]{\textsc{Recall}}
(page \pageref{defn:maximalTorus}) the notion of a maximal torus
$T\propersubset\U{n}$ which consists of diagonal matrices. We
have an embedding
\begin{equation}
B_{T}\to B_{\U{n}}
\end{equation}
there is a map
\begin{equation}
E/T\to E/\U{n}.
\end{equation}
This gives a morphism
\begin{equation}
H^{\bullet}(B_{\U{n}})\to H^{\bullet}(B_{T},\ZZ)
\end{equation}
but we know that if $T$ is one-dimensional then
\begin{equation}
B_{T}=\CP^{\infty},
\end{equation}
and if $T$ is $n$-dimensional, then
\begin{equation}
B_{T}=\bigl(\CP^{\infty}\bigr)^{n}.
\end{equation}
We know the cohomology ring
\begin{equation}
H^{\bullet}(B_{T})=\ZZ[\zeta_{1},\dots,\zeta_{n}]
\end{equation}
with $\dim(\zeta_{i})=2$. We may asily describe the image of
\begin{equation}
H^{\bullet}(B_{\U{n}},\ZZ)\to H^{\bullet}(B_{T},\ZZ),
\end{equation}
the image consists of symmetric polynomials. The group of
permutations act trivially on $H^{\bullet}(B_{\U{n}},\ZZ)$ but
nontrivially on $H^{\bullet}(B_{T},\ZZ)$. This is not new.

\index{Chern Class!as Symmetric Polynomial|(}
\marginpar{Chern classes as Symmetric Polynomials}But now we would like to say that this $c_{k}$ is an elementary
symmetric polynomial. This is really nice as a definition, since
$c_{k}$ are the only characteristic class; because every
symmetric polynomial may be expressed in terms of elementary
symmetric polynomials, so all characteristic classes may be
expressed in terms of $c_{k}$. We have
\begin{subequations}
\begin{align}
c_{1} &= \zeta_{1}+\dots+\zeta_{n},\\
c_{2} &= \sum_{i<j}\zeta_{i}\zeta_{j}
\end{align}
\end{subequations}
and so on. \marginpar{Usefulness of definition}This is the most useful definition if we wish to prove
any theorem about Chern classes---use this definition!
\index{Chern Class!as Symmetric Polynomial|)}

\index{Chern Class!Axiomatic Definition|(}
\marginpar{Definition 3: Axiomatic}The last definition is axiomatic in character. The first is that
\begin{enumerate}
\item $c_{k}$ is a characteristic class;
\item if we have the direct sum of bundles, then
\begin{equation}
c_{n}(E\oplus F)=\sum_{k+l=n}c_{k}(E)c_{l}(F)
\end{equation}
and these two axioms requires a normalization condition, i.e.,
another axiom:
\item $c_{n}(E)=0$ if $n>\dim(E)$
\end{enumerate}
C.f.\ Milnor and Stasheff's axiomatic framework for the
Stiefel--Whitney classes~\cite[Ch.\ 4]{milnor}.

If we have a line bundle, then we have only one Chern
class. Consider a line bundle $E$ over $\CP^{n}$. This is then
related to $B_{\U{1}}=\CP$ and we have $c_{1}$ be the generator
of $H^{2}(\CP^n,\ZZ)$.
\index{Chern Class!Axiomatic Definition|)}

\marginpar{\emph{\textbf{TODO:}} prove equivalence of definitions}So we have these three definitions of the Chern class. We should
prove the existence and uniqueness of the axiomatic version, then
prove the equivalence of all three definitions.

We will very briefly discuss the case of $\GL{n,\RR}$ (the same
with $\ORTH{n,\RR}$). We should consider $V_{n,k}(\RR)$. We have
the notion of Chern class work, but it is $w_{k}\in
H^{k}(B,\ZZ_{2})$ --- the analog of the first definition. We have
the third axiomatic definition merely change $c_{k}\mapsto
w_{k}$, nothing but notation changes. For the second definition,
don't consider the torus, but
$(\ZZ_{2})^{n}\propersubset\ORTH{n}$ the diagonal matrices have
components be $\pm1$.

Now this is not the end of the story; Stiefel--Whitney classes
are classes over $\ZZ_2$. We also have $\ZZ$-classes for the
group $\SO{n}$---the Euler characteristic is one example. But we
have more: namely, we have the \define{Pontryagin Classes}\index{Pontryagin Classes}\index{Characteristic Class!Pontryagin Classes}.
The picture is very simple, look we have an embedding
$\GL{n,\RR}\propersubset\GL{n,\CC}$. What does it mean? It means
\begin{equation}
B_{\GL{n,\RR}}\to B_{\GL{n,\CC}}
\end{equation}
or we may work with homotopy equivalent stuff
$\ORTH{n}\propersubset\U{n}$. Therefore we have the map
\begin{equation}
B_{\ORTH{n}}\to B_{\U{n}}.
\end{equation}
We may also go the other way, a $n$-dimensional complex bundle
may be considered a  $2n$-dimensional real bundle. So we have a
map
\begin{equation}
H^{\bullet}(B_{\U{n}},\ZZ)\to H^{\bullet}(B_{\ORTH{n}},\ZZ)
\end{equation}
which maps $c_{i}\mapsto\textbf{??}$. But the problem is that it is not
quite clear we have a nonzero characteristic class. We need to
take into account the Weyl group to see 
\begin{subequations}
\begin{equation}
c_{2k+1}\mapsto 0
\end{equation}
and
\begin{equation}
c_{2k}\mapsto\pm P_{k}.
\end{equation}
\end{subequations}
The $P_{k}$ are called ``\emph{Pontryagin Classes\/}'' which have
$\dim(P_{k})=4k$; Chern classes had $\dim(c_{k})=2k$. So there is
nothing magical here.

\subsection*{EXERCISES}
\begin{xca}
Calculate the first obstruction to the construction of non-vanishing vector field on a sphere with $h$ handles.
\end{xca}
\begin{xca}\label{xca:lec15:prob2}
Let us consider a principal bundle with total space $S^{2n+1}$, group $S^1$ and base $\CP^n$. Calculate the first obstruction to the construction of section of this bundle.
\end{xca}
\begin{xca}
Consider a bundle with a fiber $S^{2k-1}$ associated with the principal bundle of Problem \ref{xca:lec15:prob2}. (We assume that the sphere $S^{2k-1}$ is realized as a unit sphere in complex space $\CC^k$ and $z\in S^1$, where $z$ is a complex number having absolute value 1, transforms a point $(z_1, \dots, z_k)\in S^{2k-1}$ into $(zz_1, \dots, zz_k)$.) Calculate the first obstruction to the construction of section of this bundle.
\end{xca}
\subsection{K-Theory}
Right now we will give some main definitions and the relationship
to characteristic classes. Complex vector bundles are classified
by maps $B\to B_{\U{n}}$, so homotopy classification
$\homotopyClass(B,B_{\U{n}})$ is hard. But we may discuss stable
homotopy groups! Indeed, $K$-theory\index{K-Theory@$K$-Theory!as stable theory of vector bundles} is the stable theory of
vector bundles, if you like; it's much simpler.

Let $B$ be a connected, compact space (we will assume its
cellular decomposition is a finite polyhedron). We have trivial
bundles\index{$\varepsilon^{r}$|see{Trivial Bundle}}
\begin{equation}
E_{trivial} = \varepsilon^{r} = B\times\CC^{r}.
\end{equation}
Now what to do? We will say if we have  a vector bundle $E$ and
we take a direct sum
\begin{equation}
E\oplus\varepsilon^{r}\stabequiv E
\end{equation}
are \define{Stably Equivalent}\index{Stably Equivalent}\index{Equivalence!Stably}.
This should be transitive, so if 
\begin{equation}
E\oplus\varepsilon^r\iso E'\oplus\varepsilon^s\quad\mbox{then}\quad
E\stabequiv E'.
\end{equation}
Two guys are stably equivalent if and only if they're equal when
adding trivial bundles.

We'd lke to stress that characteristic
classes\index{Characteristic Class!and Stable-Equivalent Bundles} are equal for
stable-equivalent bundles. Why? The  trivial bundle has a
characteristic class equal to 0, and from axiom (2) it follows
immediately. 
