%%
%% lecture06.tex
%% 
%% Made by Alex Nelson
%% Login   <alex@tomato3>
%% 
%% Started on  Sun Aug 14 14:59:43 2011 Alex Nelson
%% Last update Sun Aug 14 15:32:10 2011 Alex Nelson
%%
We described the construction involving $\GL{n,\RR}$,
$\GL{n,\CC}$ that are its classifying spaces. We are only
interested in homotopical properties. Here we may say 
\begin{equation}
\GL{n,\RR}\homotopic\ORTH{n}
\end{equation}
is homotopic, and
\begin{equation}
\GL{n,\CC}\homotopic\U{n}
\end{equation}
is also homotopic.

What is $\GL{n}$ as a space? Well, pick a basis for 
\begin{equation}
V\iso\FF^{n}
\end{equation}
then change the basis. The ``Jacobian'' is then a member of
$\GL{n}$. So $\GL{n,\RR}$ is the space of frames in $\RR^{n}$. We
can always orthonormalize a given basis by the Grahm-Schmidt
procedure. This gives us a map
\begin{equation}
\GL{n}\to\ORTH{n}
\end{equation}
and of course
\begin{equation}
\ORTH{n}\Into\GL{n}
\end{equation}
is an embedding. We merely have to prove that $\GL{n}\to\ORTH{n}$
can be done ``gradually''. We then have
\begin{equation}
\GL{n,\RR}\homotopic\ORTH{n},
\end{equation}
and similar reasoning suggests that $\GL{n,\CC}$ is homotopic to
$\ORTH{n,\CC}$. But $\ORTH{n,\CC}$ is in one-to-one
correspondence with $\U{n}$, so $\GL{n,\CC}\homotopic\U{n}$ is
homotopic.

If 
\begin{equation}
\mathrm{GL}_{+}(n,\RR)=\{X\in\GL{n,\RR}\lst\det(X)>0\}
\end{equation}
then $\mathrm{GL}_{+}(n,\RR)\homotopic\SO{n}$ is homotopic.

We come to the notion of a Stiefel manifold\index{Stiefel Manifold}, which we covered
Fall quarter. We have several different definitions of
$V_{n,k}(\RR)$ which are homotopy equivalent definitions, so we
do not distinguish between them. We can say that $V_{n,k}(\RR)$
is the space of $k$-framed in $\RR^{n}$. Observe that
\begin{equation}
V_{n,n}\homotopic\GL{n,\RR}
\end{equation}
is homotopic. We may say $V_{n,k}(\RR)$ is the space of $k$
orthonormal vectors in $\RR^{n}$. These two are homotopy
equivalent spaces. We may extend to $V_{n,k}(\CC)$.

There are various different groups that act on Stiefel
manifolds. We see that $\GL{n}$ acts on $\RR^{n}$, and thus acts
on $V_{n,k}(\RR)$ by transitivity. So $V_{n,k}$ is an orbit, thus
we may say 
\begin{equation}
V_{n,k}=\GL{n}\Big/(\text{some stable subgroup}).
\end{equation}
But it is another way to describe Stiefel manifolds.

Let $V_{n,k}^{\mathrm{orth}}(\RR)$\index{$V_{n,k}^{\mathrm{orth}}(\RR)$} be the Stiefel manifold
describing orthonormal $k$-frames. We may say $\ORTH{n}$ acts
transitively on $V_{n,k}^{\mathrm{orth}}(\RR)$, rotating one
frame into another. But in this situation, we may describe the
stabilizer in a simple way.

We look at orthogonal transformations that keep this frame in
tact. We may write
\begin{equation}
\RR^{n}=\RR^{k}\oplus\RR^{n-k}
\end{equation}
the transformations preserves $\RR^{k}$. So therefore the
stabilizer rotates $\RR^{n-k}$ into itself, and
\begin{equation}
V^{\mathrm{orth}}_{n,k}(\RR)=\ORTH{n}/\ORTH{n-k}.
\end{equation}
Precisely the same consideration goes in the complex case. The
only difference is 
\begin{equation}
V^{\mathrm{orth}}_{n,k}(\CC)=\U{n}/\U{n-k}.
\end{equation}
i.e., we work with unitary transformations.

We will look at fibrations involving Stiefel manifolds. Suppose
we had $V_{n,k}$. Consider \emph{only} $V_{n,k}^{\mathrm{orth}}$,
we will thus without loss of generality remove the
superscript. We observe
\begin{equation}
p\colon V_{n,k}\to V_{n,k-1}
\end{equation}
by forgetting the $k^{\mathrm{th}}$ vector. This map is a locally
trivial fibration. We see that
\begin{equation}
p^{-1}(e_{1},\dots,e_{k-1})=\{(e_{1},\dots,e_{k})\lst
e_{k}\in\FF\}
\end{equation}
where $\FF$ is the field we're working with. 


If
\begin{equation}
V=\Span(e_{1},\dots,e_{k-1})
\end{equation}
then $e_{k}\in V^{\bot}$ and
\begin{equation}
\dim(V^{\bot})=n-(k-1)
\end{equation}
but $e_{k}$ should be normalized and thus $e_{k}\in S^{n-k}$. We
obtain a fibration with base $V_{n,k-1}$ and fibre $S^{n-k}$ and
the total space is $V_{n,k}$.

When we work over $\CC$, we see that
\begin{equation}
\dim(V^{\bot})=2\left(n-(k-1)\right)
\end{equation}
and thus the fibre has dimension $2n-2k+1$, so
$F=S^{2n-2k+1}$. 

We would like to compute homotopy groups of the fibrations of the
Stiefel manifolds. The exact sequence of homotopy groups of the
fibration is
\begin{equation}
\pi_{i+1}(S^{\bullet})\to\pi_{i}(V_{n,k})\to\pi_{i}(V_{n,k-1})\to\pi_{i}(S^{\bullet})
\end{equation}
and we see for ``small dimensional spheres,'' we have
$\pi_{i}(S^{\bullet})=0$ which implies
\begin{equation}
\pi_{i}(V_{n,k})\iso\pi_{i}(V_{n,k-1}).
\end{equation}
We see that
\begin{equation}
V_{n,1}(\RR)\iso S^{n-1}
\end{equation}
the orthonormal (real) vector lives in the sphere, and similarly
\begin{equation}
V_{n,1}(\CC)\iso S^{2n-1}.
\end{equation}
We may inductively compute other homotopy groups. We see that
\begin{equation}
\pi_{i}(V_{\infty,k})=0
\end{equation}
which is unsurprising since $\pi_{i}(S^{\infty})=0$.

Observe that on $V_{n,k}(\RR)$ we have the action or $\ORTH{k}$
or $\SO{k}$ [and for $V_{n,k}(\CC)$ we have the action of $\U{k}$
or $\SU{k}$]. We consider an action of the form
\begin{equation}
(e_{1},\dots,e_{k})\mapsto(e_{1}',\dots,e_{k}')
\end{equation}
where
\begin{equation}
e_{i}'=\sum_{j}a_{ij}e_{j}
\end{equation}
where $(a_{ij})$ is ``invertible.'' This is a \emph{free}
action. Now we have a question: we can factorize this stuff
\begin{equation}
V_{n,k}(\CC)/\U{k}=\Gr_{n,k}(\CC)
\end{equation}
is precisely the Grassmann manifold. It is obvious. It is clear
that if we have a frame $(e_{1},\dots,e_{k})$ we can map it to 
$\Span(e_{1},\dots,e_{k})$; it is clear that this is a fibration
$V_{n,k}(\CC)\to \Gr_{n,k}(\CC)$, and the fibre is $k$-frames
living in $k$-dimensional space.

Now lets take $n=\infty$ and we find
\begin{equation}
V_{\infty,k}(\CC)/\U{k}=\Gr_{\infty,k}(\CC).
\end{equation}
We see that we have a universal bundle\index{Universal Bundle!for $\U{k}$} 
since $V_{\infty,k}$\index{$V_{\infty,k}$} is
contractible, $\U{k}$ acts freely, so $\Gr_{\infty,k}(\CC)$ is the
classifying space\index{Classifying Space!for $\U{k}$}.

Lets consider real Stiefel manifolds, we may take the
factorization
\begin{equation}
V_{n,k}(\RR)/\ORTH{k}=\Gr_{n,k}(\RR)
\end{equation}
and repeat the same arguments to find
\begin{equation}
V_{\infty,k}(\RR)/\ORTH{k}=\Gr_{\infty,k}(\RR).
\end{equation}
We may take the quotient with $\SO{k}$ which produces the space
of \emph{oriented} $k$-dimensional planes. 

In reality we did more than it seems, because we can now
construct the classifying space for \emph{any} matrix group $G$,
a closed subgroup of $\GL{k,\CC}$. We have a free action of
$\GL{k,\CC}$ on $V_{\infty,k}$. Thus we can induce a free action
of $G$ on $V_{\infty,k}$. So the classifying space is simply
$V_{\infty,k}/G$. 

\subsection*{EXERCISES}
\begin{xca}\label{xca:lec06:prob1}
Show that Moebius band can be represented as a space of fibration
having the circle $S^1$ as a base and the interval $I$ as a
fiber. Prove that this fibration is not trivial. Does it have a
section? Show that this fibration can be obtained pasting
together two trivial fibrations; find a transition function
(clutching function) taking values in the group $\ZZ_2$.
\end{xca}
\begin{xca}
The group $\ZZ_2$ acts on the circle $S^1$ (the non-trivial
transformation acts as reflection $x\to x$, $y\to-y$). Therefore
using the transition function of the
Problem \ref{xca:lec06:prob1} we can construct a fibration with
the fiber $S^1$ and the base $S^1$. Prove that the total space of
this fibration is a Klein bottle. Does this fibration have a
section?
\end{xca}
\begin{xca}\label{xca:lec06:prob3}
Let us consider principal fibrations with the group $\U{1}$,
total space $E$ and base $S^2$. Show that classes of these
fibrations are in one-to-one correspondence with integers. Let us
consider fibrations corresponding to integer numbers $n = 0, 1,
2$. Prove that we have $E = S^2\times S^1$ for $n = 0$, $E = S^3$
for $n = 1$, $E = \SO(3) = \RP^3$ for $n = 2$.

Hint. Consider Hopf fibration and tangent fibre bundle to
$S^2$. (To define Hopf fibration we consider the group $\U{1}$
acting on the $S^3$ defined by the equation $|z_{1}|^2 +
|z_{2}|^2 = 1$ in $\CC^2$; element $z\in\U{1}$ transforms $(z_1,
z_2)$ into $(zz_1, zz_2)$.) Calculate the transition function
explicitly or use exact homotopy sequence of fibration.
\end{xca}
\begin{xca}
Let us consider a principal fibration of Problem \ref{xca:lec06:prob3} corresponding to arbitrary integer
$n$. Calculate homotopy groups of total space of the fibration (for dimensions $\leq3$ you
should give an explicit answer, in dimensions $> 3$ you should express the homotopy group
of total space in terms of homotopy groups of $S^2$ ).
\end{xca}
