%%
%% lecture22.tex
%% 
%% Made by alex
%% Login   <alex@tomato>
%% 
%% Started on  Mon Dec 26 10:35:29 2011 alex
%% Last update Mon Dec 26 10:35:29 2011 alex
%%
The $\widetilde{K}\to K$ morphism, consider
$\widetilde{\alpha}\in[\alpha]$, and map it to 
\begin{equation}
\widetilde{\alpha}\mapsto\widetilde{\alpha}-\varepsilon^{\dim(\widetilde{\alpha})}.
\end{equation}
There are many morphisms due to the decomposition
$K=\widetilde{K}\oplus\ZZ$, but some are more natural.

\begin{thm}
Consider a cell complex with even dimensional cells, then
$K^{0}(X)=\ZZ^{n}$ where $n$ is the number of cells, and
$K^{1}(X)=0$.
\end{thm}
\begin{proof}
The proof is done by induction on the number of cells. Consider
$A\propersubset X$ a subcomplex such that 
\begin{equation}
X/A = S^{2n}.
\end{equation}
Then we write the exact sequence
\begin{equation}
\underbracket[0.5pt]{K^{1}(X/A)}_{=0}\gets
\underbracket[0.5pt]{K^{0}(A)}_{=\ZZ^{n-1}}\gets
K^{0}(X)\gets
\underbracket[0.5pt]{K^{0}(X/A)}_{=\ZZ}\gets
\underbracket[0.5pt]{K^{-1}(A)}_{=0},
\end{equation}
we know
\begin{equation}
K^{1}(X/A)=K^{1}(S^{2n})=0.
\end{equation}
So by exactness, we obtain
\begin{equation}
K^{0}(X)=\ZZ^{n}.
\end{equation}
This is half of the story.

We should have a simultaneous induction for $K^{1}(X)$, namely
\begin{equation}
K^{2}(X/A)\gets
\underbracket[0.5pt]{K^{1}(A)=0\;\;}_{\text{by induction}}\!\!\!\gets
K^{1}(X)\gets
\underbracket[0.5pt]{K^{1}(X/A)}_{=0}\gets
K^{0}(A).
\end{equation}
Thus it follows that
\begin{equation}
K^{1}(X)=0.
\end{equation}
We have this exact sequence for $\widetilde{K}$ but not for
$K$. So $\widetilde{K}$ is the analog of relative cohomology, we
may speak about relative $K$-theory too.
\end{proof}

We constructed a homomorphism 
\begin{equation*}
\chernChar\colon K^{0}\to H^{\text{even}}
\end{equation*}
the ``Chern character''. So what about $K^{1}$? Well, look,
\begin{equation}
\widetilde{K}^{1}(X)=\widetilde{K}^{-1}(X)=\widetilde{K}^{0}(SX)\xrightarrow{\chernChar}H^{\text{even}}(SX)=H^{\text{odd}}(X).
\end{equation}
This is easy to understand. Remember, when \dots\ suppose $X$ is a
cell complex.

\begin{wrapfigure}{l}{0.69in}
  \vspace{-12pt}
  \includegraphics{img/lecture22.0}
\end{wrapfigure}
\noindent{}How do we create the suspension? We create the
following picture, and it is clear everything comes from $X\times
I$, we just consider cells in $X\ni\sigma$ and multiply by open
intervals $\sigma\times(0,1)$. We have a one-to-one
correspondence, etc.\ etc.\ etc.
This basic construction is doodled on the left, well the
\emph{result} from the construction is doodled.

We may construct something denoted
\begin{equation}
\widetilde{K}^{\bullet}=\widetilde{K}^{0}+\widetilde{K}^{-1} 
\end{equation}
to
get the full $K$ group. We also take
\begin{equation}
\begin{split}
K^{\bullet} &= \ZZ\oplus\widetilde{K}^{\bullet} \\
&= \ZZ+\widetilde{K}^{0}+\widetilde{K}^{-1} 
\end{split}
\end{equation}
The only difference is in $K^{0}$ and $\widetilde{K}^{0}$, since
$K^{1}=\widetilde{K}^{1}$. So
\begin{equation}
K^{\bullet}=K^{0}+\widetilde{K}^{1}.
\end{equation}
But $K^{\bullet}$ is not only a group but also a
ring!\index{$K^{\bullet}$!is a Ring}
We have a definition of ring structure in $K$, which goes from
the external direct product
\begin{equation}
K(X)\otimes K(Y)\to K(X\times Y).
\end{equation}
Really how may we construct a product? We note the embedding
\begin{equation}
\Delta\colon X\into X\times X
\end{equation}
where $x\mapsto(x,x)$. This gives us a map
\begin{equation}
K(X\times X)\to K(X),
\end{equation}
and we have a map
\begin{equation}
K(X)\otimes K(X)\to K(X\times X)
\end{equation}
which induces by composition a map
\begin{equation}
K(X)\otimes K(X)\to K(X).
\end{equation}
But for the reduced $\widetilde{K}$ we should consider the
\define{Smash Product}\index{Smash Product!for $\widetilde{K}$}
\begin{equation}
\widetilde{K}(X)\otimes\widetilde{K}(Y)\to\widetilde{K}(X\smashProd Y),
\end{equation}
by considering
\begin{equation}
X\times Y/X\wedge Y\eqdef X\smashProd Y.
\end{equation}
We have
\begin{equation}
\widetilde{K}(X)\otimes\widetilde{K}(X)\to\widetilde{K}(X\smashProd X)\to\widetilde{K}(X).
\end{equation}
We are askng ourselves: what is
\begin{equation}
\begin{split}
\widetilde{K}^{i}(X) &= \widetilde{K}(S^{i}X)\\
&=\widetilde{K}(S^{i}\smashProd X)
\end{split}
\end{equation}
But we may define a product of
\begin{equation}
\begin{split}
\widetilde{K}(S^{i}\smashProd
X)\otimes\widetilde{K}(S^{j}\smashProd X)\to
\widetilde{K}\bigl((S^{i}\smashProd X)\times(S^{j}\smashProd X)\bigr)
&=\widetilde{K}(S^{i}\smashProd S^{j}\smashProd X\smashProd X)\\
&=\widetilde{K}(S^{i+j}\smashProd X\smashProd X)\\
&=\widetilde{K}^{i+j}(X\smashProd X)\to\widetilde{K}^{i+j}(X).
\end{split}
\end{equation}
So we have thus constructed a mapping
\begin{equation}
\widetilde{K}^{i}(X)\otimes\widetilde{K}^{j}(X)\to\widetilde{K}^{i+j}(X)
\end{equation}
as desired.
