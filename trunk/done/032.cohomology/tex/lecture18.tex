%%
%% lecture18.tex
%% 
%% Made by alex
%% Login   <alex@tomato>
%% 
%% Started on  Sun Dec 25 09:45:58 2011 alex
%% Last update Sun Dec 25 09:45:58 2011 alex
%%
Let $B$ be a topological space. We want to consider the
collection of vector bundles over $B$. Denote this collection of
vector bundles by $A=\Vect(B)$. Observe that $A$ is a semigroup.
But we may promote it to a group by
\begin{equation}
A\mapsto\widetilde{A}=K(B).
\end{equation}
Two elements $E,E'\in\Vect(B)$ define the same element of $K(B)$
if and only if for some $n$ we have
\begin{equation}
E\oplus\varepsilon^n=E'\oplus\varepsilon^n.
\end{equation}
We have stable equivalence
\begin{equation}
E\stabequiv E'
\end{equation}
defined by
\begin{equation}
E\stabequiv E' \iff E\oplus\varepsilon^m=E\oplus\varepsilon^n.
\end{equation}
Please note the ``exponents'' differ here. Thus we get a map
\begin{equation}
K(B)\to\widetilde{K}(B)
\end{equation}
We have a morphism
\begin{equation}
\dim\colon K(B)\to\ZZ
\end{equation}
It is clear we have a notion of dimension in $K(B)$, but it's
not an invariant in $\widetilde{K}(B)$. But we may consider
\begin{equation}
K(B)\xrightarrow{\sim}\ZZ\oplus\widetilde{K}(B)
\end{equation}
which is an isomorphism.

How do we calculate these $K$ groups? We know how to calculate
soem of them. We have seen how $\widetilde{K}(S^{n})$, the
calculations boiled down to considering
\begin{equation}
\pi_{n-1}\bigl(\U{\infty}\bigr)=\widetilde{K}(S^{n}).
\end{equation}
We know, for example,
$\widetilde{K}(S^{2})=\pi_{1}\bigl(\U{\infty}\bigr)=\ZZ$.

Consider a ``good'' subspace $A\propersubset B$ (if $B$ is a cell
complex, $A$ is a subcomplex and thus always ``good''). The
relative cohomology
\begin{equation}
H^{k}(B\bmod A)=\widetilde{H}^{k}(B/A).
\end{equation}
We have $A\propersubset B$, which implies we have a map in the
opposite direction
\begin{equation}
\dots\gets H^{k}(A)\gets H^{k}(B)\gets H^{k}(B,A)\gets\dots
\end{equation}
so we have an exact sequence. We would like to prove that for
$K$-theory we have a similar exact sequence.

This is an extraordinary sequence\index{Sequence!Extraordinary}\index{Extraordinary Sequence}
\begin{equation}
\widetilde{K}(A)\gets\widetilde{K}(B)\gets\widetilde{K}(B/A),
\end{equation}
we have the map $\widetilde{K}(A)\gets\widetilde{K}(B)$ because
we pull-back the bundle to $A$ and likewise $B/A\to B$ pulled
backed.  We have a sequence, but we should prove it is exact.

\begin{wrapfigure}{r}{7pc}
  \includegraphics{img/lecture18.0}
\end{wrapfigure}
Lets first recall some basic topological constructions\index{Obstruction!Topological}. If we
have a space $A$, then we can construct the cone over
$A$\index{Cone!Over a Space} ---
denoted $CA$ --- as doodled on the right. But from this, we can
form the suspension over $A$, which is denoted by $S(A)$ and
turns out completely equivalent to $CA/A$. Again, the
suspension\index{Suspension!Over a Space}
is doodled on the right, and we can obtain it from a continuous
mapping $CA\to S(A)$. It amounts to taking our original space $A$
(which is shaded) and contracting it to a single point.

Really, we have
\begin{equation}
\widetilde{K}(A)\gets
\widetilde{K}(B)\gets
\widetilde{K}(\underbracket[0.5pt]{B\cup CA}_{B/A})\gets
\widetilde{K}(\underbracket[0.5pt]{B\cup CA\cup CB}_{S(A)})\gets
\dots
\end{equation}

\begin{wrapfigure}{l}{7pc}
  \vspace{-30pt}
  \includegraphics{img/lecture18.1}
\end{wrapfigure}
\noindent{}Lets consider what this looks like topologically. We
have $A$ be a subcomplex of $B$, so a nice picture might be a
smaller disc contained in a larger disc. We consider the cone
$CA$ over $A$. We should observe the critically important
property that $B\cup CA\cup CB$ is the same as the suspension of
$A$, denoted by $SA$. This is the same stuff as $CA/A$, as we
have already discussed. We get another term in our exact sequence
\begin{equation}
\widetilde{K}(A)\gets
\widetilde{K}(B)\gets
\widetilde{K}(B/A)\gets
\widetilde{K}(SA)\gets
\widetilde{K}(SB)\gets
\dots
\end{equation}
So now we'd like a cohomological analog of what we have done
here. Instead of $\widetilde{K}(A)$ we will write
$\widetilde{K}^{0}(A)$. This is the same stuff, just different
notation, so $\widetilde{K}=\widetilde{K}^{0}$. Now our exact
sequence becomes
\begin{equation}
\widetilde{K}^{0}(A)\gets
\widetilde{K}^{0}(B)\gets
\widetilde{K}^{0}(B/A)\gets
\widetilde{K}^{-1}(A)\gets
\widetilde{K}^{-1}(B)\gets
\dots
\end{equation}
We use the notation
\begin{equation}
\widetilde{K}(SA)=\widetilde{K}^{-1}(A).
\end{equation}
So if we introduce this notation, we see immediately that it is a
cohomological theory---it has the same exact sequence as
cohomology. For real bundles, we use the notation $KO$. For
complex bundles we have Bott periodicity\index{Bott Periodicity}:
\begin{equation}
\widetilde{K}(S^{2}A)=\widetilde{K}(A).
\end{equation}
This is a slightly more general statement than Bott
periodicity. We had Bott periodicity describe
\begin{equation}
\widetilde{K}(S^{n+2})=\widetilde{K}(S^{n}).
\end{equation}
Our notation reads
\begin{equation}
\widetilde{K}(S^{n}A)=\widetilde{K}^{-n}(A),
\end{equation}
so $\widetilde{K}^{-n-2}(A)=\widetilde{K}^{n}(A)$.
