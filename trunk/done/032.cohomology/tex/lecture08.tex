%%
%% lecture08.tex
%% 
%% Made by alex
%% Login   <alex@tomato>
%% 
%% Started on  Wed Sep 28 11:51:30 2011 alex
%% Last update Wed Sep 28 11:51:30 2011 alex
%%
Lets discuss the homework. It is clear on the Mobius band we have
a local trivialization; it may be better to think of $\ZZ_{2}$ as
$\ORTH{1}$. We may do this with the cell diagram transforming the
Mobius band into the Klein bottle\index{Klein Bottle} by identifying the top and
bottom:
\begin{center}
\includegraphics{img/lecture8.0}
\end{center}
which implies the fibre is $S^{1}$.

The ``second'' two problems are about principal fibre bundles
with base $B=S^{2}$ and group $\U{1}$. We have for
\begin{equation}
B=S^{n}
\end{equation}
(in general) $\homotopyClass(S^{n-1},G)$ when $G$ is the fibre,
then this gives us information about the fibration. It is wrong
to say we computed the homotopy groups and they correspond to
$S^{3}$, $\RP^{3}$, etc., at least to homotopic
equivalence. Knowing the homotopy groups does not tell us about
equivalent spaces. But we know that $\SO{3}$ is a principal
fibration.

\bigskip

We will return to characteristic classes, we consider principal
bundles with the group $G$ over the base $B$. We always assume
that $B$ is ``good enough'' (for us, it's a polyhedron). We would
like to classify the principal bundles by the maps
$\homotopyClass(B,B_{G})$ where
\begin{equation}
B_{G}=E_{G}/G
\end{equation}
and $E_{G}$ is contractible. There is an obvious way to construct
invariants of a principal bundle: take any $c\in
H^{\bullet}(B_{G})$, if $\varphi\colon B\to B_{G}$, then
$\varphi^{*}(c)$ is an invariant. We considered the case when
$G=\U{n}$, and formulated the answer for any connected, compact
matrix group. We take its maximal torus
\begin{equation}
T\propersubset G
\end{equation}
consider the Weyl group\index{Weyl Group!Acting on $H^{\bullet}(T)$} $W$. Then it follows that $W$ acts on
$H^{\bullet}(T)$. Then we may say 
\begin{equation}
H^{\bullet}(B_{G})=H^{\bullet}(B_{T})^{W}
\end{equation}
is $W$-invariant.

We will return to this later, but now we'd like to apply
characteristic classes to something interesting.

\begin{rmk}
We may give another definition of characteristic classes that do
not formally involve the classifying space. To specify a
characteristic class of the group $G$ is to give every principal
bundle with group $G$ an element 
\begin{equation}
c_{B}\in H^{\bullet}(B)
\end{equation}
such that for every map of principal fibrations
\begin{equation}
(E,B,G,p)\to(E',B',G,p')
\end{equation}
we have a map of bases
\begin{equation}
\psi\colon B\to B'
\end{equation}
and thus we have
\begin{equation}
c_{B}=\psi^{*}(c_{B'})
\end{equation}
This is precisely the same as the old definition. We had
$\varphi\colon B\to B_{G}$, but $B_{G}$ is a good guy\dots so we
have something special, namely $c_{B_{G}}\in H^{\bullet}(B_{G})$
so by the second definition we should have
\begin{equation}
c_{B}=\varphi^{*}(c_{B_{G}}). 
\end{equation}
So the second definition implies the first one; to prove the
first implies the second we just use the fact that 
\begin{equation}
B\xrightarrow{\;\psi\;}B'\xrightarrow{\;\varphi\;}B_{G}
\end{equation}
gives us the stuff.
\end{rmk}

We will focus on something else. Every locally trivial
fibration\index{Fibration!Locally Trivial}\index{Locally Trivial Fibration}
is affiliated with a principal fibration\index{Fibration!Principal}\index{Fibration!Principal!from Locally Trivial Fibration}. It is locally trivial,
we may paste the local trivializations together by transition
functions\index{Transition Function}. If we have $\{U_{i}\}$ be a covering of the base,
locally we'd have a map
\begin{equation}
(U_{i}\cap U_{j})\times F\to (U_{i}\cap U_{j})\times F
\end{equation}
This is a homeomorphism of fibres. We have for each $x\in
U_{i}\cap U_{j}$ a map 
\begin{equation}
\varphi_{x}\colon F\to F
\end{equation}
so we may say that we really have a mapping
\begin{equation}
U_{i}\cap U_{j}\to\aut(F),
\end{equation}
which describe the transition functions.
If we have a subgroup $G\propersubset\hom(F,F)$, then the
transition functions live as
\begin{equation}
U_{i}\cap U_{j}\to G\propersubset\hom(F,F).
\end{equation}
Thus we obtain a $G$-bundle\index{Bundle!$G$!from Locally Trivial Fibration}.

When the fibre $F$ is a vector space $V=F$, then it turns out
that
\begin{equation}
G=\GL{V}.
\end{equation}
We have really a vector bundle. We should observe that $G$ acts
on itself by left multiplication. 

Or we can go in the opposite. Let us take a principal bundle
\begin{equation}
G\into E\xrightarrow{p}B,
\end{equation}
then we have transition functions. So we have local
trivializations, and transition functions for a covering
$\{U_{i}\}$ of $B$, and the transition function $\varphi_{ij}$
for $U_{i}\cap U_{j}$ is $\varphi_{ij}(x)\in G$. This is not
quite unique, we may take a section, and another covering,
etc.\ etc.\ etc. Just verify consistency on overlaps.

We may try to construct a bundle with a fibre $F$ invariant. So
what do we have? We have $F$ and this stuff $(E_{F}, B, F,
p_{F})$. We can construct this stuff. How? Well, lets first
consider
\begin{equation}
E_{G}=E\times F
\end{equation}
which may be fibred with the base $B$, its very simple, the
projection map is
\begin{equation}
(e,f)\mapsto p(e).
\end{equation}
But the fibre here is too big, it's $G\times F$. Let me kill
$G$. To kill $G$ we do the following identification in $E\times
F$:
\begin{equation}
(e,g)\sim (eg,g^{-1}f).
\end{equation}
So we really have
\begin{equation}
E_{G}=(E\times F)/G
\end{equation}
Then everything is absolutely clear, we have a fibre bundle with
a fibre $F$.
