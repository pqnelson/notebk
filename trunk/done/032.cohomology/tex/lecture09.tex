%%
%% lecture09.tex
%% 
%% Made by alex
%% Login   <alex@tomato>
%% 
%% Started on  Wed Sep 28 12:21:48 2011 alex
%% Last update Wed Sep 28 12:21:48 2011 alex
%%
We talked about $G$-bundles\index{Bundle!$G$!has Vector Fibration} having a vector fibration where
transition functions\index{Transition Function} belong to the group $G$. We should assume
the fibre $F$ is a $G$-space, i.e., $G$ acts on $F$ from the left
(it could be the right, it doesn't matter; our conventions is
left). Then we had a notion of a principal fibration, so $G$ acts
without fixed points on $E$. The fibre is in one-to-one
correspondence with $G$, but the fibre doesn't have the group
structure. NB: we never made rigorous the notion of equivalent fibrations.

Now if $F$ is a $G$-space, then considering a principal
$G$-bundle
\begin{equation}
F\into E \xonto{p} B
\end{equation}
we consider $E\times F$ and a map
\begin{equation}
\widetilde{p}(e,f)=p(e).
\end{equation}
We factorize $E\times F/\sim$ by the equivalence relation
\begin{equation}
(e,f)\sim (eg,g^{-1}f)
\end{equation}
But then $E\times F/\sim\to B$ is a fibration with the fibre $F$
(although really it's $G\times F$ but $G$ is factorized
out). This may be considered the definition, if you'd like.

We would like to consider a particular case: a vector
bundle. This is a particular case when
\begin{equation}
G=\GL{n}
\end{equation}
which acts on some vector space $V$ as a group of its linear
transformations. Her ewe may consider either $\GL{n,\CC}$ or
$\GL{n,\RR}$, i.e. complex or real vector bundles and they're not
the same stuff. Again, this is not the best definition, but it is
convenient.

We will give a little different definition (of vector bundle).\index{Bundle!Vector} We
consider a fibre bundle
\begin{equation}
V\into E\xonto{p}B
\end{equation}
where every fibre has the structure of a vector space. We
consider $B\times V$ as a vector space, then $(b,v)$ forms a
vector. 

Then we have vector bundle morphisms
\begin{equation}
(E,B,V,p)\to(E',B',V',p')
\end{equation}
We have a map
\begin{equation}
\varphi\colon E\to E'
\end{equation}
which sends a fibre to a fibre. This induces a map of the bases,
\begin{equation}
\begin{diagram}[small]
E    & \rTo^{\varphi} & E'\\
\dTo &               & \dTo \\
B    & \rTo          & B'
\end{diagram}
\end{equation}
But if these guys are vector bundles, then this map 
\begin{equation}
\mbox{(fibre)}\to\mbox{(fibre)}
\end{equation}
must be a linear map. We can now speak of isomorphic vector
bundles, we know what a locally trivial vector bundle is, so we
now know what a locally trivial vector bundle is.

Every vector bundle comes from principal fibration. How to prove
this? Let $\widetilde{E}$ be the collection of all bases in all
fibres; we can make it into a principal fibration. We define an
action of $\GL{n}$ on $\widetilde{E}$, by rotating an element of
a basis. The frames in a given fibre form a fibre of frames, so
we have a map
\begin{equation}
\widetilde{E}\to B
\end{equation}
to the base space by selecting the base point. This is a
principal fibre. So we can go in the opposite direction, from
this principal fibration, we get a vector bundle.

We have $\GL{n}$-fibrations yielding vector bundles, so suppose
we begin with a $\U{n}$-fibration. Let $V$ have a $\U{n}$
action. So $\U{n}$ fibration corresponds to complex vector
bundles with Hermitian inner product (in fibres). If we suppose
that every fibre is a Hilbert space, then consider the fibration
with fibres of orthonormal basis. Then it is determined up to
some $\U{n}$ action.

We may likewise consider $\ORTH{n}$-fibration, but it is with a
real Hilbert space equipped with the usual inner product on the
fibres. 

\begin{ex}
Consider a submanifold of $\RR^{n}$, i.e., we consider 
\begin{subequations}
\begin{align}
f_{1}(\vec{x}) & = 0\\
\dots & \dots \\
f_{k}(\vec{x}) &= 0
\end{align}
\end{subequations}
defining the submanifold $M\propersubset\RR^{n}$. Note that by the\marginpar{Whitney embedding theorem: any smooth $k$-manifold $M$ may be embedded in $\RR^{2k}$}  
Whitney embedding theorem, $n=2k$ for finite-dimensional real manifolds.

The simplest case is (really) linear equations
\begin{equation}
f_{i}(\vec{x})=\vec{a}_{i}^{\mathrm{T}}\vec{x}
\end{equation}
where $i=1,\dots,k$. We may imagine we have an $n\times k$ matrix
whose rows are given by $\vec{a}_{i}^{\mathrm{T}}$. The dimension
of the subspace is determined by the rank of the matrix;
generically it is $k$. Really each equation kills 1 dimension, so
we get an $(n-k)$-dimensional submanifold. If we have $f_{1}$,
\dots, $f_{k}$ be smooth, then we may approximate by a linear
function at each point. We have the Jacobian matrix
\begin{equation}
\frac{\partial f_{i}}{\partial x^{j}}=J_{ij}
\end{equation}
If $\rank(J)=k$, we may use the implicit function theorem. 

Now we may consider a curve to this submanifold $x^{i}(t)$, then
we obtain tangent vectors
\begin{equation}
\left.\frac{\D}{\D t}x^{i}(t)\right|_{t=\tau}.
\end{equation}
We obtain
\begin{equation}
\sum_{j}\frac{\partial f_{i}}{\partial x^{j}}\frac{\D
  x^{j}(t)}{\D t}=0
\end{equation}
when we have a tangent vector of the submanifold. We define
$T_{x}M$ as the space of tangent vectors at $x\in M$.

But if we take
\begin{equation}
E=\bigsqcup_{x\in M}T_{x}M
\end{equation}
then we have a map
\begin{equation}
E\to M
\end{equation}
which gives a vector bundle called the \define{Tangent Vector
Bundle}\index{Bundle!Tangent Vector}\index{Tangent Vector Bundle}. By the way, this is even an $\ORTH{n}$-bundle since we
are working in $\RR^{n}$.

If we consider $N_{x}M=(T_{x}M)^{\bot}$, then 
\begin{equation}
NM=\bigsqcup_{x\in M}N_{x}M
\end{equation}
is another vector bundle called the \define{Normal Bundle}. In general, we
have (using the Whitney embedding result):
\begin{equation}
T_{x}M\iso\RR^{k},\quad\mbox{and}\quad
N_{x}M\iso\RR^{n-k}.
\end{equation}
For
$S^{2}$, we have $T_{x}S^{2}\iso\RR^{2}$, and $N_{x}S^{2}\iso\RR$.
\end{ex}

Consider any locally trivial fibration $(E,B,F,p)$ and consider
the homology of the fibres. For each $b\in B$,
$H_{k}(F_{b},\RR)$, then consider\marginpar{Remember $k=\dim(B)$}
\begin{equation}
\bigsqcup_{b\in B}H_{k}(F_{b},\RR)=\mbox{\define{Total Space of Vector Bundle}}
\end{equation}
If $U\propersubset B$, then by local triviality
\begin{equation}
p^{-1}(U)=U\times F.
\end{equation}

We would like to stress that it is tempting to say it is almost
always a locally trivial fibration\index{Fibration!Locally Trivial}, but we cannot say this. We
have a natural identification\dots we have a path in $B$ be
lifted to a homotopy of fibres, but this may be done in many
different ways. But now look at the homology! And the homology
goes in an absolutely definite way since
\begin{equation}
F_{b}\rightrightarrows F_{b'}
\end{equation}
are homotopic maps, it follows $H_{k}(F_{b})\to H_{k}(F_{b'})$ is
then unambiguous.

We have a \define{Connection}\index{Connection}\index{Vector Bundle!Connection} in a vector bundle (well, not
necessarily a vector bundle) if for every\footnote{\textsc{Note:\quad}\ignorespaces this notion
depends on the paths on $B$, and they may be classified into homotopy equivalent
classes $\pi_{1}(B)$. This is important, since considerations of flat connections enables us to use singular cohomology later on.} path $\gamma\colon[0,1]\to B$ we have a
definite lifting $F_{\gamma(0)}\to F_{\gamma(1)}$. 
If $U$ is contractible, then
\begin{equation}
H_{k}(p^{-1}(U))\iso H_{k}(F)
\end{equation}
since
\begin{equation}
U\homotopic\mathrm{point}
\end{equation}
is homotopic, then we lift this contraction of $p^{-1}(U)$ to a
fibre. That gives us an isomorphism with the homology. Since we
can do this for a small interval, as long as we go back and forth
in a contractible neighborhood, we have such an
identification. This is not always the case, e.g., with the Klein
bottle:
\begin{center}
\includegraphics{img/lecture9.0}
\end{center}
which runs into problems when we consider too large of an open
neighborhood in $B$.

\subsection*{EXERCISES}
\begin{xca}
Find first non-trivial homotopy group, first non-trivial homology group and first non-trivial cohomology group of complex Stiefel manifold $V_{n,k}$.
\end{xca}
\begin{xca}
Using the identification of the group $\SU{2}$ with the group of quaternions of norm 1 prove that infinite-dimensional quaternionic projective space is a classifying space of $\SU{2}$. Calculate the cohomology of this space.
\end{xca}
\begin{xca}
Calculate Chern class of the principal fibration with total space $\SO{3} = \RR^3$, group
$S^1 = \U{1}$ and base $S^2$.
\end{xca}
