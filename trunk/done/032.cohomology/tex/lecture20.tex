%%
%% lecture20.tex
%% 
%% Made by alex
%% Login   <alex@tomato>
%% 
%% Started on  Sun Dec 25 12:19:59 2011 alex
%% Last update Sun Dec 25 12:19:59 2011 alex
%%
We should recall we constructed a product in the $K$-group $K(B)$
as
\begin{equation}
K(B)\otimes K(B)\to K(B).
\end{equation}
Recall we also have the product in the cohomology group
\begin{equation}
H^{\bullet}(B)\otimes H^{\bullet}(B)\to H^{\bullet}(B)
\end{equation}
where
\begin{equation}
H^{\bullet}(B)=\bigoplus_{k}H^{k}(B)
\end{equation}
is the full cohomology group. It could be considered
\begin{equation}
H^{k}(B)\otimes H^{\ell}(B)\to H^{k+\ell}(B)
\end{equation}
But with cohomology we have what is sometimes called the
\define{External Product}\index{External Product}\index{Cohomology!External Product}
\begin{equation}
H^{\bullet}(B)\otimes H^{\bullet}(B')\to H^{\bullet}(B\times B')
\end{equation}
The same stuff holds for $K$-theory:
\begin{equation}
K(B)\otimes K(B')\to K(B\times B').
\end{equation}
How to construct it?

The construction is trivial. If $b\in B$, $b'\in B'$, and we have
vector bundles $(E,B,F,p)$ and $(E',B',F',p')$, and suppose we
are given $F_{b}$ over $b$ and $F'_{b'}$ over $b'$, then we want to
construct a fibre bundle over $B\times B'$. How? Well, merely by
taking the tensor product $F_{b}\otimes F'_{b'}=F$ as the
fibre. Of course, this is not so simple: we want a vector bundle!
If $U\propersubset B$, $U'\propersubset B'$ and the given bundles
are locally trivial, then we have locally $U\times U'\times
F_{b}\times F'_{b'}$ in a natural way. It is something that may
be called the \define{Tensor Product for Vector Bundles}\index{Vector Bundle!Tensor Product}\index{Tensor Product!for Vector Bundle}
$E\otimes E'$ over $B\otimes B'$.

What we should prove for this product $K(B)\otimes K(B')\to
K(B\times B')$ is distributivity, how it behaves on virtual
bundles, etc.

But if $H(B)$, $H(B')$ are Torsion free, then
\begin{equation}
H(B)\otimes H(B')\iso H(B\times B').
\end{equation}
This is by Kuenneth's theorem\index{Kuenneth's Theorem}. If they're not torsion free, we
have a short exact sequence
\begin{equation}
0\to H(B)\otimes H(B')\to H(B\times B')\to \Tor(\dots)\to 0.
\end{equation}
What happens in $K$-theory? If at least one of the groups $K(B)$
or $K(B')$ is torsion-free, then
\begin{equation}
K(B\times B')\iso K(B)\otimes K(B').
\end{equation}
Really?! No. We need to recall
\begin{equation}
K^{-i}(B)=K(S^{i}B),
\end{equation}
so the analog should be present (c.f., in Cohomology). In the
particular case when $B'=S^2$, we have
\begin{equation}
K(B)\otimes K(S^{2})=K(B\times S^{2}).
\end{equation}
Really what we did is called ``cheating.'' The full proof of this
formula may be found in Hatcher's
book~\cite[\S2.1]{hatcher2009vector}. (The difference between $K$ and
$\widetilde{K}$ is analogous to the difference between $H$ and $\widetilde{H}$.)

We would like to do the following. We notice that an exact
sequence may be written for any pair. If we have a pair $X$, $Y$
we may embed $X\vee Y\propersubset X\times Y$. Recall we mark a
point $x_0\in X$ and $y_0\in Y$, and the wedge sum is $(X\times
y_0\sqcup x_0\times Y)/(x_0\sim y_0)$. Then we may consider the quotient
\begin{equation}
(X\times Y)/(X\vee Y)=X\smashProd Y
\end{equation}
is the \define{Smash Product}\index{Smash Product}. (This was covered first quarter)

\begin{wrapfigure}{r}{1.74in}
  \vspace{-30pt}
  \includegraphics{img/lecture20.0}
\end{wrapfigure}
\noindent{}We'd like to consider
\begin{equation}
S^{1}\smashProd Y=\Sigma Y
\end{equation}
the \define{Weak Suspension}\index{Suspension!Weak}\index{Weak Suspension} of $Y$, which is homotopic to $SY$. 
For example, we may consider the case of $S^{1}\times Y/S^{1}\vee
Y$ as doodled on the right.

\begin{ddanger}
Note that in the literature, sometimes people use the term
\define{Reduced Suspension}\index{Suspension!Reduced}\index{Reduced Suspension} instead of ``weak suspension''. Also
note that the reduced suspension is defined as $\Sigma X =
(X\times I)/(X\times\{0\}\cup X\times\{1\}\cup \{x_0\}\times I)$
where $I$ is the interval. This is \emph{homeomorphic} to the the
definition we produced.
\end{ddanger}

Consider $\widetilde{K}$ calculations.
We have
\begin{equation}
\widetilde{K}(X\vee S^2)\gets
\widetilde{K}(X\times S^2)\gets
\widetilde{K}(X\smashProd S^{2})
\end{equation}
as the beginning of a long exact sequence. We could continue and
write more terms, but for now we note that we have
\begin{equation}
\widetilde{K}(X\vee S^{2})=\widetilde{K}(X)\oplus\widetilde{K}(S^{2}).
\end{equation}
We also know
\begin{equation}
K(X\times S^2)=K(X)\otimes K(S^2)
\end{equation}
but we also recall
\begin{equation}
K(X)=\widetilde{K}(X)\oplus\ZZ.
\end{equation}
Thus
\begin{equation}
K(S^{2})=\widetilde{K}(S^{2})\oplus\ZZ,
\end{equation}
so we can write
\begin{equation}
\widetilde{K}(X\times S^2)=\bigl(\widetilde{K}(X)\otimes\widetilde{K}(S^2)\bigr)\oplus\widetilde{K}(X)\oplus\widetilde{K}(S^2).
\end{equation}
Now we would like to use the fact that we have $S^{2}$ here.
Then we have the exact sequence be
\begin{equation}
\dots\gets
\widetilde{K}(X\smashProd S^2)\gets
0.
\end{equation}
We use the fact that
\begin{equation}
S^1\smashProd S^2=S^3
\end{equation}
to get what we wanted. We get a formula for the smash product, we
see that
\begin{equation}
0\gets
\widetilde{K}(X)\oplus\widetilde{K}(S^2)\gets
\widetilde{K}(X)\otimes\widetilde{K}(S^2)\gets
\widetilde{K}(X)\oplus\widetilde{K}(S^2)\gets
\dots
\end{equation}
So we see
\begin{equation}
\widetilde{K}(S^2\smashProd X)=\widetilde{K}(X)
\end{equation}
and thus\marginpar{Bott Periodicity}
\begin{equation}
\widetilde{K}(S^2X)=\widetilde{K}(X)
\end{equation}
which is Bott periodicity\index{Bott Periodicity!and $K$-Theory}. Note we did \emph{not} prove/deduce
Bott periodicity, we merely related it to Kuenneth's theorem at
an intuitive level.

\index{K-Theory@$K$-Theory!and Cohomology}\index{Cohomology!and $K$-Theory}There is a strong similarity between $K$-theory and
cohomology. There is not just a similarity, there is a morphism!
We will construct it.We really have $K^0(X)$ and $K^1(X)$. We
have many cohomology groups $H^{r}(X)$. The statement is as
follows: we may consider
\begin{equation}
H^{\text{even}}(X)=\sum H^{2k}(X),\quad\mbox{and}\quad
H^{\text{odd}}(X)=\sum H^{2k+1}(X).
\end{equation}
What is our goal? To construct morphisms
\begin{equation}
K^{0}(X)\to H^{\text{even}}(X),\quad\mbox{and}\quad
K^{1}(X)\to H^{\text{odd}}(X),
\end{equation}
and we may consider
\begin{equation}
K^0(X)+K^1(X)\to H^{\bullet}(X)
\end{equation}
where we must note the full cohomology group $H^{\bullet}(X)$ is
not just a group: it's a ring! So we infer $K^0(X)+K^1(X)$ is a
ring, moreover the statement is that this is a ring morphism
called the \marginpar{Chern Character $\chernChar$}\define{Chern Character}\index{Chern Character|textbf} denoted by
\begin{equation}
\chernChar\colon K^{0}(X)+K^{1}(X)\to H^{\bullet}(X).
\end{equation}
Note we never considered the group of coefficients for the
cohomology, we use the integers $H^{\bullet}(X,\ZZ)$. Now what is
this relationship? Is $\chernChar$ an isomorphism? No!

But it is close to an isomorphism in the following sense: both
the $K$ group and $H$ have torsion. The part without torsion are
the same, but the torsion part is different. We may conclude
$K^{0}(X)\otimes\QQ$ or $K^{0}(X)\otimes\RR$ to kill the torsion
part. If we write 
\begin{equation}
K^{0}(X)=\sum\ZZ\oplus\sum\ZZ_{m}, 
\end{equation}
then we see $K^{0}(X)\otimes\QQ$ has two parts:
$(\sum\ZZ)\otimes\QQ=\sum(\ZZ\otimes\QQ)=\sum\QQ$ and
$(\sum\ZZ_{m})\otimes\QQ=\sum(\ZZ_{m}\otimes\QQ)=0$. And we have
\begin{equation}
K^{0}(X)\otimes\QQ\to H^{\text{even}}(X,\QQ)=H^{\text{even}}(X,\ZZ)\otimes\QQ.
\end{equation}
We definitely have this map, just multiply the Chern
character\index{Chern Character!relating $K$-theory to Cohomology} by $\QQ$. But the statement is \emph{this is now an isomorphism}.
