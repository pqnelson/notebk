%%
%% lecture04.tex
%% 
%% Made by Alex Nelson
%% Login   <alex@tomato3>
%% 
%% Started on  Sun Apr 10 12:03:16 2011 Alex Nelson
%% Last update Sat May 21 13:59:02 2011 Alex Nelson
%%
We are continuing to classify principal bundles when $B=S^{n}$
and $G$ is connected. We see classes of principal fibrations over
$B$ are in one-to-one correspondence with $\homotopyClass(S^{n-1},G)$.

Let $S^{n}=D^{n}_{1}\cup D^{n}_{2}$, we assume
\begin{equation}
D^{n}_{1}\cap D^{n}_{2}=S^{n-1}
\end{equation}
which generalizes the notion of a sphere having two hemispheres
that meets at an equator. We know that on the ball the fibration
is trivial. Thus we may take
the \define{trivialization}\index{Fibration!Trivialization|textbf}\index{Trivialization!of Fibration}
of the fibration, i.e.,
\begin{equation}
p^{-1}(D^{n}_{1})=D^{n}_{1}\times G
\end{equation}
is identified as the trivial principal bundle; and trivialization
over the other hemisphere means
\begin{equation}
p^{-1}(D^{n}_{2})=D^{n}_{2}\times G
\end{equation}
But trivialization is not unique, we need to choose it in some
way. But we choose it in some way: we paste them together over
the sphere. The pasting is described by a \define{Transition function}\index{Transition Function}
a map
\begin{equation}
\varphi\colon S^{n-1}\to G
\end{equation}
The only problem is this is \emph{any} continuous map. This
establishes the one-to-one correspondence.

Really?! No, not really. We have some freedom in the choice of
trivialization. This means we have the same stuff, the preimage
of the first hemisphere may be represented in two different ways,
and we have a morphism between these two different ways (i.e., we
have the following commutative diagram)
\begin{equation}
\begin{diagram}[small]
p^{-1}(\bar{D}^{n}_{1}) & \rEq & \bar{D}^{n}_{1}\times G\\
\dEq                 & \ruTo>{\varphi} & \\
\bar{D}^{n}_{1}\times G &           & 
\end{diagram}
\end{equation}
that is, we have a morphism between these two ways
\begin{equation}
\begin{split}
\varphi\colon\bar{D}^{n}_{1}\times G\to \bar{D}^{n}_{1}\times G\\
\varphi(x,g)=(x,\alpha(x)g)
\end{split}
\end{equation}
where
\begin{equation}
\alpha\colon\bar{D}^{n}_{1}\to G.
\end{equation}
So what happens on the sphere? Well, $\alpha$ is an arbitrary
map, but on the boundary sphere $S^{n-1}$ we see that $\alpha$ is
not arbitrary because it can be extended to the ball. That means
\begin{equation}
\alpha\big|_{S^{n-1}}\homotopic\mbox{trivial map}
\end{equation}
that is, a map sending everything to a point. But we assumed $G$
is connected, so this notion of a trivial map is unambiguous.

The transition function is not uniquely defined, but only up to
homotopy. This means we have $\alpha\psi$. In any case, when we
multiply by $\alpha$ (or its analog from $\bar{D}^{n}_{2}$) we do
not change the homotopy class. This concludes the proof. But this
is only for the case when $B=S^{n}$. What happens in the
arbitrary case?

Well, we have some notions we need to first define. So suppose
that we have a principal bundle $(E_{G},B_{G},G)$ having the
property that $E_{G}$ is contractible. Then this bundle is called
a \define{Universal Bundle}\index{Universal Bundle|textbf}\index{Bundle!Universal} and $B_{G}$ is called the
\define{Classifying Space}\index{Classifying Space|textbf}. Can we construct any such bundles?
Why is this name appropriate? We will consider interesting cases
later.

\begin{thm}
Classes of principal bundles with base $B$ and group $G$ are in
one-to-one correspondence with $\homotopyClass(B,B_{G})$.
\end{thm}

Observe if
\begin{equation}
B\iso S^{n}
\end{equation}
then it is easy to prove (at least if $G$ is connected) that
\begin{equation}
\pi_{n}(B_{G})=\{S^{n},B_{G}\}
\end{equation}
since $E_{G}$ is contractible we may consider the exact homotopy
sequence that identifies
\begin{equation}
\pi_{n}(B_{G})\iso\pi_{n-1}(G)
\end{equation}
and this is the same as $\homotopyClass(S^{n-1},G)$. So this new theorem
contains the old one.

\begin{rmk}
Perhaps one should bear in mind the exact homotopy sequence of
the bundle. Don't forget we can lift $S^{n-1}$ to $\bar{D}^{n}$.
\end{rmk}

Now to prove this theorem, we need to explain some
constructions. The first thing is let us suppose we have two
principal bundles $(E,B,G)$ and $(E',B',G)$ and we have a map
\begin{equation}
f\colon (E,B,G)\to(E',B',G)
\end{equation}
which agrees with the structure
\begin{equation}
f(xg)=f(x)g.
\end{equation}
This map \emph{induces} a map of the base space
\begin{equation}
\varphi\colon B\to B'.
\end{equation}
Now there is an interesting remark: knowing $\varphi$ and the
target, we may restore the map $f$. How to do this? It's easy.

First we should explain how to recover $E$ if we know $E'$ and
$B$. Let
\begin{subequations}
\begin{align}
p\colon E\to B\\
p'\colon E'\to B'\\
f\colon E\to E'.
\end{align}
\end{subequations}
We can combine these maps, what do we get? We get a map
\begin{equation}
F\colon E\to E'\times B.
\end{equation}
How? Very simply:
\begin{equation}
F(e)=\left(f(e),p(e)\right)
\end{equation}
We may consider the image $F(E)$. We have the commutative diagram
\begin{equation}
\begin{diagram}[small]
E & \rTo^{f} & E'\\
\dTo<{p} &  & \dTo>{p'}\\
B & \rTo^{\varphi} & B'
\end{diagram}
\end{equation}
which implies
\begin{equation}
p'\circ f=\varphi\circ p.
\end{equation}
But observe that we have exactly that!

We have a space
\begin{equation}
\widetilde{E}\subset E'\times B
\end{equation}
consisting of pairs
\begin{equation}
(e',b)\in \widetilde{E}
\end{equation}
satisfying the relation
\begin{equation}
p'(e')=\varphi(b).
\end{equation}
Now what can we say? We can say the following: we have a map
\begin{equation}
F\colon E\to\widetilde{E}
\end{equation}
and it is surjective since $\widetilde{E}$ is the image, and of
course it is injective and continuous. Therefore we may say that
there is a continuous bijection
\begin{equation}
E\to\widetilde{E}.
\end{equation}
In all good cases, this is a homeomorphism.
\begin{rmk}
If a space is compact, then every continuous bijection is a
homeomorphism. 
\end{rmk}
Thus we may conclude we may restore $E$ if we know: $E'$, $f$,
$\varphi$. 

We will introduce the notion of
a \define{Pullback}\index{Pullback!of Principal Fibrations}\index{Pullback|textbf}. Consider a
space, more precisely a principal fibration
\begin{equation}
(E',B',G',p'\colon E'\to B').
\end{equation}
We consider a map
\begin{equation}
\varphi\colon B\to B'
\end{equation}
There is a way to transfer the fibration from $B'$ to $B$. We
construct 
\begin{equation}
\widetilde{E}\subset E\times B
\end{equation}
consisting of pairs $(e',b)$ that satisfy the condition
\begin{equation}
p'(e')=\varphi(b).
\end{equation}
We have a map
\begin{equation}
p\colon\widetilde{E}\to B
\end{equation}
satisfying
\begin{equation}
p(e',b)=b.
\end{equation}
This is a principal fibration. To be more general, we may
use \emph{any} fibration in this construction.

Now, suppose we have the universal bundle ($E_{G}$, $B_{G}$, $G$,
$p_{G}\colon E_{G}\to B_{G}$) with classifying space $B_{G}$. If
we have 
\begin{equation}
\varphi\colon B\to B_{G}
\end{equation}
then we may construct the pullback. The theorem that we prove
next time is if 
\begin{equation}
\varphi\homotopic\varphi',
\end{equation}
then their pullbacks are homotopic and in one-to-one
correspondence.

