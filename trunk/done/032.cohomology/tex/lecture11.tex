%%
%% lecture11.tex
%% 
%% Made by alex
%% Login   <alex@tomato>
%% 
%% Started on  Wed Sep 28 13:48:51 2011 alex
%% Last update Wed Sep 28 13:48:51 2011 alex
%%
(This is just a review of homework problems.)

The first problem is to compute the homotopy groups of $V_{n,k}$;
we should have done mathematical induction over $k$:
\begin{equation}
\begin{split}
V_{n,k} & \to V_{n,k-1}\\
(e_{1},\dots,e_{k}) & \mapsto (e_{1},\dots,e_{k-1})
\end{split}
\end{equation}
We get a fibration. The orthogonal complement to $V_{n,k-1}$ is a
$(n-k+1)$-dimensional space. We have the fibration
\begin{equation}
S^{2n-2k+1}\into V_{n,k}\onto V_{n,k-1}
\end{equation}
which gives an exact homotopy sequence
\begin{equation}
\dots\to\pi_{i}(S^{2n-2k+1})\to\pi_{i}(V_{n,k})\to\pi_{i}(V_{n,k-1})\to\pi_{i-1}(S^{2n-2k+1})\to\dots
\end{equation}
which induces an isomorphism 
\begin{equation}
\pi_{i}(V_{n,k})\iso\pi_{i}(V_{n,k-1})\quad\mbox{for }i<2n-2k+1.
\end{equation}
We see
\begin{equation}
\pi_{i}(V_{n,1})\iso\pi_{i}(S^{2n-1})
\end{equation}
which is $\ZZ$ at $i=2n-1$ and $0$ for all others before. Don't
forget the universal coefficient theorem yields the cohomology
given homology. 

We really do not need the notion of cohomology with local
coefficients. We should know only one thing, namely, if our space
is simply connected, then teh homology with local coefficients is
\emph{precisely} the conventional homology.

We itnroduced the notion of a local coefficient system, we had
some space $B$. We considered the group bundle, where fore each
$b\in B$ we have a group $G_{b}$ corresponding to this
point. This bundle is assumed to be locally trivial. So if
$b,b'\in U\propersubset B$, then we may identify the fibres
$G_{b}$ and $G_{b'}$; moreover this identification is unique. If
we have a vector bundle $(E,F,B,p)$, then the homology of the
fibres $H_{k}(F_{b})$ can be used as the local coefficients. 

Now we would like to say if we have two points that may be far
away on $B$, but are connected by a path, then there is an
identification of the fibres on the endpoints $G_{b}$, $G_{b'}$
but the identification depends on the path. In all of our
considerations, we never considered the topology of $G$---it
played no role. We would consider the discrete topology. Then we
obtain what is called the \define{Covering Space}; recall that we
defined the covering as a fibration with a discrete fibre. This
will give us another ``construction'' of $G_{b}\to G_{b'}$
identification. Really, this map doesn't depend on the path but
the \emph{homotopy class} of the path.

If we have paths $\alpha\colon I\to B$ and $\beta\colon I\to B$
such that $\alpha(1)=\beta(0)$, then the concatenation of paths
would correspond to $G_{\alpha(1)}\to G_{\beta(0)}$
identifications.

It follows that local coefficient systems on simply connected
spaces are trivial. So $G_{b}$ are canonically isomorphic to
$G_{b'}$ for all $b,b'\in B$. But this is the usual homology we
all know and love.

If we have a local coefficient system on a connected space, then
we may take a closed path. The homotopy class of these paths
$\pi_{1}(B,b)$. The local coefficient system generates a morphism
of $\pi_{1}(B,b)\to\aut(G)$. But a morphism
$\pi_{1}(B,b)\to\aut(G)$ generates a local coefficient system.
