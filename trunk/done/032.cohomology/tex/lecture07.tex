%%
%% lecture07.tex
%% 
%% Made by Alex Nelson
%% Login   <alex@tomato3>
%% 
%% Started on  Sun Aug 14 15:32:31 2011 Alex Nelson
%% Last update Sun Aug 14 15:51:28 2011 Alex Nelson
%%
We would like to apply our classification theorem to invariants;
this will lead to characteristic classes. We introduced the
notion of a universal bundle $(E_{G},B_{G},G)$ with topological
group $G$. We may replace the contractibility of $E_{G}$ with 
\begin{equation}
\pi_{k}(E_{G})=0
\end{equation}
for all $k$. We have the possibility of weakening this condition
to be trivial up to a constant; then we get universality up to a
constant. Then we have a one-to-one correspondence
\begin{equation}
\begin{pmatrix}
\text{classes of principal}\\
\text{fibrations with base $B$}
\end{pmatrix}
\iso\homotopyClass(B,B_{G}).
\end{equation}
This may be understood as saying there is a map of principal
bundles up to homotopy.

Then we constructed the universal bundle for simple cases,
e.g. for $\ZZ_{2}$ we get $\RP^{n}$, for $S^{1}$ we got
$\CP^{n}$. For the groups $\SO{n}$ and $\U{n}$ we got Grassmann
manifolds as the classifying space.

Remember if we have a map of two spaces\marginpar{Characteristic Class of the Principal Fibration}
\begin{equation}
\varphi\colon B\to B_{G}
\end{equation}
then this map induces a map of cohomology
\begin{equation}
\varphi^{*}\colon H^{\bullet}(B_{G})\to H^{\bullet}(B),
\end{equation}
and this homomorphism does not depend on the choice of element in
homotopy class. To know $\varphi^{*}$, we only need to know the
homotopy class of $\varphi$. Take a cohomology class $c\in
H^{\bullet}(B_{G})$, then $\varphi^{*}c$ is an invariant of the
principal fibration. This is called the \define{Characteristic
  Class of the Principal Fibration}\index{Characteristic Class!of Principal Fibration}.\marginpar{Characteristic
  classes associates to each fibre bundle a cohomology class
  measuring the ``twistedness''} The first question is what
coefficient group to use? For us, it doesn't matter.

Lets consider this for $S^{1}$. Recall 
\begin{equation}
\CP^{\infty}=S^{\infty}/S^{1}
\end{equation}
in reality we do not need this! We only need
\begin{equation}
\CP^{n}=S^{2n+1}/S^{1}.
\end{equation}
But we know the cohomology of $\CP^{n}$, we have the cell
complex constructed by the embedding
\begin{equation}
\CP^{n-1}\Into\CP^{n},
\end{equation}
and
\begin{equation}
\CP^{n}\setminus\CP^{n-1}=\sigma^{2n}.
\end{equation}
Thus we have the cell decomposition
\begin{equation}
\CP^{n}=\sigma^{0}\cup\sigma^{2}\cup\dots\cup\sigma^{2n},
\end{equation}
and all boundaries vanish. Thus we compute
\begin{equation}
H^{2k}(\CP^{n},G)=G
\end{equation}
and we should remember we had multiplication in cohomology; it is
related to intersection in homology.

If we fix a generator $c\in H^{2}(\CP^{n},\ZZ)$, then the
generator of $H^{2k}(\CP^{n},\ZZ)$ is $c^{k}$. By Poincar\'e
  duality, 
\begin{equation}
\poincareDual{c}\in H_{2n-2}(\CP^{n},\ZZ)
\end{equation}
Then $\poincareDual{c}$ is a homotopy class of
$\CP^{n-1}\propersubset\CP^{n}$. When we take the intersection of
two different copies of $\CP^{n-1}$ we get a projective
subspace. But for $\CP^{\infty}$ we have 
\begin{equation}
H^{\bullet}(\CP,\ZZ)=\ZZ[c]
\end{equation}
with $\dim(c)=2$. For $\CP^{n}$ the polynomials are truncated to
be of degree $\leq n$. What does it mean for characteristic
classes?

Well, $\varphi^{*}$ is a morphism of cohomology rings, so if we
take $\varphi^{*}(c^{n})=\left(\varphi^{*}(c)\right)^{n}$. If
$G=S^{1}$, basically we have one characteristic class
$\varphi^{*}(c)$ where $c\in H^{2}(\CP^{n})$. In this case,
$\varphi^{*}(c)$ is denoted as $c_{1}$ and it is called the
\define{Chern Class}\index{Chern Class}.

Now we consider the ``commutative torus''
\begin{equation}
T=(S^{1})^{n}
\end{equation}
If we wish to construct the corresponding classifying space
\begin{equation}
E_{T}=E_{S^{1}}\times\dots\times E_{S^{1}}
\end{equation}
then we take $n$ copies of the classifying space for $S^{1}$. We
get thus 
\begin{equation}
\begin{split}
B_{T}&=B_{S^{1}}\times\dots\times B_{S^{1}}\\
&=\CP^{\infty}\times\dots\times\CP^{\infty}
\end{split}
\end{equation}
and everything goes completely independently. Using Kuenneth's
theorem, or the product of cell complexes, or\dots anyway, the
answer is obvious
\begin{equation}
H^{\bullet}(B_{T})=\ZZ[\xi_{1},\dots,\xi_{n}]
\end{equation}
where
$\dim(\xi_{i})=2$ for each $i$. We may say our characteristic
classes are $\varphi^{*}(\xi_{1})$, \dots,
$\varphi^{*}(\xi_{n})$. This is a kind of triviality. The most
important thing in mathematics is to understand all trivialities.

\phantomsection\label{defn:maximalTorus}
Lets look at $\U{n}$ and describe characteristic classes.
We will consider a \define{Maximal Torus}\index{Maximal Torus}, ti is the main hero in
the story of Lie groups, $T\propersubset\U{n}$. The maximal torus
for $\U{n}$ is very simple, consider only diagonal matrices and
the diagonal components take values in $\U{1}$. If we consider a
permutation matrix $\omega\in P_{n}$, and consider any $\alpha\in
T$, then we see that $\omega\alpha\omega^{-1}=\omega(\alpha)$ is
again diagonal. This group $P_{n}$ is usually denoted by $W$ and
is referred to as the \define{Weyl group}\index{Weyl group} of $\U{n}$.

Let us take the universal bundle for $\U{n}$, denoted
$E_{\U{n}}$. It is a space where $\U{n}$ acts freely. But
$T\propersubset\U{n}$, so $T$ also acts freely. We may identify
\begin{equation}
E_{\U{n}}=E_{T}
\end{equation}
but we may say that
\begin{equation}
E_{T}/T\to E_{\U{n}}/\U{n}.
\end{equation}
Why? Well, the numerators are the same, and $T$ is smaller than
$\U{n}$. SO we may say that we have a corresponding map 
\begin{equation}
B_{T}\to B_{\U{n}}
\end{equation}
and thus a map of cohomologies
\begin{equation}
H^{\bullet}(B_{\U{n}})\xrightarrow{\text{inj}}H^{\bullet}(B_{T})
\end{equation}
We will describe its image. Really, look, $W$ acts
on $T$ and therefore on $B_{T}$, and thus on $H^{\bullet}(B_{T})$.
The description is as follows: the image of
\begin{equation}
H^{\bullet}(B_{\U{n}})\xrightarrow{\text{inj}}H^{\bullet}(B_{T})
\end{equation}
consists of $W$-invariant elements of $H^{\bullet}(B_{T})$. We
see that $W$ acts on $\U{n}$ and on $B_{\U{n}}$, and on
$H^{\bullet}(B_{\U{n}})$ it acts trivially. Why? By the simple
reason that it acts by means of conjugation, but $\omega
g\omega^{-1}\homotopic g$ is homotopic for all $g\in\U{n}$. That
is to say, we can continuously map $\omega\mapsto\id{}$ so the
action is trivial on $\U{n}$, and thus concludes the lecture.
