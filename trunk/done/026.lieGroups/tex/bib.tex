%%
%% bib.tex
%% 
%% Made by Alex Nelson
%% Login   <alex@tomato3>
%% 
%% Started on  Sat Dec 11 10:05:46 2010 Alex Nelson
%% Last update Sun Dec 12 11:54:21 2010 Alex Nelson
%%

\appendix
\renewcommand{\leftmark}{Bibliography and Further References}
\section{Bibliography and Further References}

The required text for the course was
\begin{itemize}
\item V. S. Varadarajan, \emph{Lie Groups, Lie Algebras, and
  Their Representation}. Graduate Texts in Mathematics v. 102,
  Springer-Verlag (1984).
\end{itemize}
There was a supplementary text that was recommended informally
beforehand by another professor:
\begin{itemize}
\item William Fulton, Joe Harris, \emph{Representation Theory: A
  First Course}. Graduate Texts in Mathematics, Springer-Verlag
  (1991).
\item Brian C. Hall, \emph{Lie Groups, Lie Algebras, and Representations: An Elementary Introduction}.
  Springer (2003).
\item Daniel Bump, \emph{Lie Groups}. Graduate Texts in Mathematics,
  Springer (2004).
\item Anthony W. Knapp, \emph{Lie Groups: Beyond an Introduction}.
  Birkh\"{a}user, 2nd edition (2002).
\item Robert Gilmore, \emph{Lie Groups, Lie Algebras, and Some of Their Applications}.
  Dover Publications (2006).
\end{itemize}

With regards to further studies of infinite-dimensional Lie
Algebras, there are a variety of different texts to read. Kac
notes there are roughly 4 different meanings of
``infinite-dimensional Lie Algebras'':
\begin{enumerate}
\item The Lie algebras of vector fields and the corresponding
groups of diffeomorphisms of a manifold.
\item Lie groups (resp. Lie algebras) of smooth mappings of a
given manifold into a finite-dimensional Lie group (resp. Lie
algebra). So this is a group (Lie algebra) of matrices over some
function algebra but viewed over the base field. (Physicists
refer to certain central extensions of these Lie algebras as
current algebras.) The main subject of study in this case has
been certain special families of representations.
\item Classical Lie groups and algebras of operators in a
Hilbert or Banach space. There is apparantely a rather large
number of scattered results in this area, which study the
structure of these Lie groups and algebras and their
representations. A representation which plays an important role
in quantum field theory is the Segal-Shale-Weil (or metaplectic)
representation of an infinite-dimensional symplectic group. %%  {\bf
%% Remark:} it seems that Neretin focuses on this in his book.
\item Kac-Moody algebras, which Kac investigates in his book.
\end{enumerate}
The references to investigate this subject are:
\begin{enumerate}
\item Minorus Wakimoto, \emph{Infinite-Dimensional Lie
  Algebras}. Translations of Mathematical Monographs, vol
  195. American Mathematical Society (2000). 
\item Victor Kac, \emph{Infinite Dimensional Lie
  Algebras}. Cambrdige University Press (1990). 
\item Yu. A. Neretin, \emph{Categories of Symmetries and
  Infinite-Dimensional Groups}. London Mathematical society
  Monographs, New Series vol 16. Oxford University Press (1996).
\end{enumerate}
\noindent{}Wakimoto describes his book as an ``hors d'oeuvres''
to Kac's book and the ``great feast'' of infinite-dimensional Lie
algebras. 
