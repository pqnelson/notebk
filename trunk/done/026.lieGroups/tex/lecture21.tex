%%
%% lecture21.tex
%% 
%% Made by Alex Nelson
%% Login   <alex@tomato3>
%% 
%% Started on  Sun Dec 12 10:45:41 2010 Alex Nelson
%% Last update Sun Dec 12 11:31:58 2010 Alex Nelson
%%
Let us discuss some notions important in themselves, namely, the
notion of a \define{Universal Enveloping Algebra}. Lets start
with a Lie Algebra $\mathscr{G}$, and the commutation relations
\begin{equation}
[e_{\alpha}, e_{\beta}] = c^{\gamma}_{\alpha\beta}e_{\gamma}.
\end{equation}
We want to embed this into an associative unital algebra
\begin{equation}
\mathscr{G}\subset\mathcal{U}(\mathscr{G})
\end{equation}
in such a way that
\begin{equation}
[a,b] \mapsto ab-ba.
\end{equation}
We take $\mathcal{U}(\mathscr{G})$ to be unital and associative
with generators $e_{1}$, \dots, $e_{n}$, and with relations
\begin{equation}
e_{\alpha}e_{\beta}-e_{\beta}e_{\alpha}=c^{\gamma}_{\alpha\beta}e_{\gamma}.
\end{equation}
It is clear in $\mathcal{U}(\mathscr{G})$ we have part which
consists of linear guys $c^{\alpha}e_{\alpha}$ which is the Lie
algebra with respect to the commutator, but this is only a small
part of the Universal Enveloping Algebra. It has, first of all,
an arbitrary guy written as
\begin{equation}
c\cdot\1 + c^{\alpha}e_{\alpha} +
c^{\alpha\beta}e_{\alpha}e_{\beta} + \cdots
\end{equation}
with quadratic, cubic, and higher order terms. We can consider
commutators of these guys, but note that
\begin{equation}
e_{\alpha}e_{\beta} = e_{\beta}e_{\alpha}+c^{\gamma}_{\alpha\beta}e_{\gamma}
\end{equation}
so not all of these guys are unique. We may assume that always
$\alpha\leq\beta\leq\gamma\leq\cdots$. This sum is finite. The
alternative to doing them ordered, we may do them symmetrically.

\begin{rmk}
The universal enveloping algebra is an infinite dimensional
algebra, even for the simplest case! Consider
$c^{\gamma}_{\alpha\beta}=0$ for all $\alpha$, $\beta$,
$\gamma$. So $[e_{\alpha},e_{\beta}]=0$, then
$\mathcal{U}(\mathscr{G})$ is a commutative polynomial algebra of
$n$-variables. 
\end{rmk}

We would like to say
\begin{equation}
\hom\big(\mathscr{G},\mathfrak{gl}(n)\big)=\hom\big(\mathcal{U}(\mathscr{G}),\mat_{n}\big).
\end{equation}
We represent the generator $e_{\alpha}\in\mathscr{G}$ by
$E_{\alpha}$ such that
\begin{equation}
E_{\alpha}E_{\beta}-E_{\beta}E_{\alpha}=c^{\gamma}_{\alpha\beta}E_{\gamma}
\end{equation}
which specifies (by $c^{\gamma}_{\alpha\beta}$) the
reprepresentation of the Lie Algebra. That's pretty important in
physics, we considered such a thing even if not in this
vocabulary. Consider $M_{x}$, $M_{y}$, $M_{z}$ which are all the
angular momentum operators, which are precisely a representation
$\mathfrak{so}(3)$. The commutator of these guys are precisely
the commutation relations from the Lie Algebra. We have the total
angular momentum
\begin{equation}
M^{2}=M_{x}^{2}+M_{y}^{2}+M_{z}^{2}
\end{equation}
but this isn't in $\mathscr{G}$, it's in
$\mathcal{U}(\mathscr{G})$.

The universal enveloping algebra is useful in many
relations. What is important is to consider the center of
$\mathcal{U}(\mathscr{G})$. When considering stuff that commutes
with everything; it is sufficient to consider the stuff that
commutes with the generators.

\begin{schur}
If we have an operator that commutes with all operators in the
irreducible representations of a Lie Algebra, then it is a scalar
times the identity.
\end{schur}

As a corollary the central element is a scalar, if not it's an
irreducible representation. We may consider the eigensubspaces,
which form a decomposition into irreducible representations.

Now we will talk about the highest weight representation. We have
generators $e_{\alpha}$, $f_{\alpha}$, $h_{\alpha}$ (in principle
these are either multiplicative or linear basis elements); we may
consider the subalgebras
\begin{enumerate}
\item $\mathscr{G}_{+}$ generated by $e_{\alpha}$ (it's basis as
  a vector space is all positive roots, although in principle we
  can use simple roots since we can obtain all other roots this way);
\item $\mathscr{G}_{-}$ generated by $f_{\alpha}$;
\item $\mathscr{H}$ generated by $h_{\alpha}$.
\end{enumerate}
Whe ngenerating the universal enveloping algebra, we use all
roots. We use wither the simple roots or all the roots for the
Verma module. We may consider as vector spaces
\begin{equation}
\mathscr{G}=\mathscr{G}_{+}\oplus\mathscr{G}_{-}\oplus\mathscr{H}
\end{equation}
How to construct the highest weight representation? We take the
highest weight vector $\vec{v}$, and we may apply the operators
$e_{\alpha}$ and we get zero
\begin{equation}
e_{\alpha}\vec{v}=0
\end{equation}
for all $\alpha$. We can apply $h\in\mathscr{H}$, we get
\begin{equation}
h\vec{v}=\lambda(h)\vec{v}
\end{equation}
Now we apply $f_{\alpha_{1}}$, $\dots$,
$f_{\alpha_{n}}\in\mathscr{G}_{-}$ and
\begin{equation}
\Span\{f_{\alpha_{1}}\cdots f_{\alpha_{n}}\vec{v}\}
\end{equation}
forms an invariant subspace. Further it forms a
representation. Why is this an invariant subspace? For the simple
reason that
\begin{equation}
[e_{\alpha},f_{\beta}]=\delta_{\alpha\beta}h_{\beta},
\end{equation}
in the representation it is
\begin{equation}
[E_{\alpha},F_{\beta}]=\delta_{\alpha\beta}H_{\beta}
\end{equation}
and so we may interchange $e$ and $f$ at some price. We can push
all the $e$'s to the right, etc.

So we have some subspace. What if we don't have it? Then we will
\emph{construct} it. We will take $\mathcal{U}(\mathscr{G}_{-})$,
or more precisely $\mathcal{U}(\mathscr{G}_{-})\vec{v}$
 which are the combinations
\begin{equation}
\sum_{n}c^{\alpha_{1}\cdots\alpha_{n}}f_{\alpha_{1}}\cdots f_{\alpha_{n}}\vec{v}
\end{equation}
where $\alpha_{1}\leq\cdots\leq\alpha_{n}$. We can define the
action of $\mathscr{G}$ on this set, and this is called a
\define{Verma Module}. There is another construction which is
more or less immediate.

So for every $\lambda$ we may construct an infinite dimensional
module, an infinite dimensional representation with highest
weight $\lambda$\dots and $\lambda$ is \emph{completely arbitrary.}
It is infinite dimensional. Why? Well, consider
$\mathfrak{sl}(2)$ where we have $e$, $f$, $h$. We consider all
elements of the form $f^{n}\vec{v}$, then
\begin{equation}
h(f^{n}\vec{v})=(\lambda-n)f^{n}\vec{v}
\end{equation}
and in principle it's an infinite dimensiona lrepresentation
since we are completely arbitrary here. But is this
representation irreducible? We know for $\mathfrak{sl}(2)$, the
representation is reducible when $\lambda\geq0$ and
$\lambda\in\ZZ$. At some moment we would get
\begin{equation}
h(f^{n}\vec{v})=0
\end{equation}
and we would get more importantly
\begin{equation}
e(f^{n}\vec{v})=0
\end{equation}
we have 2 highest weight vector in the same representation. Which
means we can do the following: we can factorize our
representation with respect to this subrepresentation and get a
finite dimensional and irreducible representation.

So for every $\lambda$ we have an infinite dimensional
representation with highest weight $\lambda$ called the Verma
Module. What can we do? This is not necessarily reducible. If
$\lambda(h_{i})\in\ZZ$ and $\lambda(h_{i})\geq0$, then the Verma
module is reducible. Let us take the largest subrepresentations
(well largest nontrivial representations), then the quotient
\begin{equation}
(\mbox{Verma Module})/(\mbox{Largest Nontrivial Subrepresentation})
\end{equation}
is a finite dimensional irreducible representation.

The fact it is an irreducible representation is trivial, since we
factorized by the largest nontrivial subrepresentation; if we
have a subrepresentation in the quotient, then there's a larger
nontrivial subrepresentation in the Verma module, which is
impossible. The nontrivial part is the finite dimensionality of
the irreducible representation.
