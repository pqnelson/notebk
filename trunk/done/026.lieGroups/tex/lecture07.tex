%%
%% lecture07.tex
%% 
%% Made by Alex Nelson
%% Login   <alex@tomato3>
%% 
%% Started on  Thu Jan 21 12:56:40 2010 Alex Nelson
%% Last update Thu Jan 21 13:49:02 2010 Alex Nelson
%%
We took a matrix group $G$ and considered the tangent space at
$e\in G$, $T_{e}G=\lie(G)$ and showed it was the Lie algebra for
$G$. If we took a curve $x\colon[0,1]\to G$ with $x(0)=e$, then
we took $x'(0)\in T_{e}G$. This tangent space is closed under the
commutator, and addition, so it's a Lie algebra.
\medbreak
\begin{quest}
Given a Lie Algebra, could we ``restore'' all elements of the group?
\end{quest}
\medbreak
Yes, lets take $x\colon[0,1]\to G$ without the condition
$x(0)=e$. If $x(\tau_{0})=e$, then 
\begin{equation}
\left.\frac{d}{d\tau}x(\tau)\right|_{\tau=\tau_{0}}\in\Lie(G).
\end{equation}
The value of the parameter is completely irrelevant. What if we
take $x(\tau_0)\not=e$? Then nothing dangerous! Because look, we
can take another curve
\begin{equation}
y(\tau)=x(\tau_{0})^{-1}x(\tau)
\end{equation}
which is still a curve in the group. Now
$y(\tau_{0})=e$. Therefore we can say that
$y'(\tau_{0})\in\lie(G)$. Or by plugging in the definition, we
see $y'(\tau_{0})=x(\tau_{0})^{-1}x(\tau_{0})\in\lie(G)$. This is
due to the product rule and $x(\tau_{0})$ being a constant. So
\emph{for any curve} in the group $x(\tau)\in G$ we have
$x(\tau)^{-1}x'(\tau)\in\Lie(G)$. Let
\begin{equation}
\xi(\tau)=x(\tau)^{-1}x'(\tau).
\end{equation}
So a curve in $G$ will generate a curve in the Lie Algebra of $G$
by means of this simple way. But now we would like to have a
curve in $\lie(G)$ give rise to a curve in $G$. We know
\begin{equation}
\frac{dx(\tau)}{d\tau}=x(\tau)\xi(\tau)
\end{equation}
gives a system of differential equations. To solve it, we need
initial conditions. If we take $x(0)=e$, then a solution exists.

How to restore the Lie group? Take all the curves in the Lie
algebra, then we get all the curves in the Lie group starting at
1, then...we get the whole matrix group? Not really, only the
\define{Connected Part}!!! We use the notion of path connected,
that any two points of the group have a continuous path
connecting them. 

\begin{ex}
$\ORTH{3}$ is not connected, since $\det(X)=\pm1$ for all
  $X\in\ORTH{3}$. We cannot connect two matrices $X$, $Y\in\ORTH{3}$ if
  $\det(XY)=-1$. But $\SO{3}$ is the connected component of
  $\ORTH{3}$ since $\det(X)=1$ for all $X\in\SO{3}$.
\end{ex}

\medbreak
\noindent\textbf{Moral:} We still have a way to get back the
connected part of the group.
\medbreak

We get $x(\tau)^{-1}x'(\tau)=\xi(\tau)\in\lie(G)$. We can take
very simple curves in this algebra, namely constants! So
$\xi(\tau)$ are constant matrices. Then we get $\xi(\tau)=A$
implies $x(\tau)^{-1}x'(\tau)=A$ if and only if
\begin{equation}
\frac{dx(\tau)}{d\tau}=x(\tau)A,
\end{equation}
with $x(0)=e$. Then $x(\tau)=\exp(A\tau)$. One way to define
$\exp(A\tau)$ is as a solution to this differential equation. The
other way is as a series
\begin{equation}
x(\tau)=\sum^{\infty}_{n=0}\frac{1}{n!}(A\tau)^{n}
\end{equation}

\begin{cor}
If $A\in\Lie(G)$, then $\exp(A)\in G$.
\end{cor}
\medbreak

In reality this is almost sufficient. We have the
\define{Exponential Map}
\begin{equation}
\exp\colon\Lie(G)\to G,\qquad A\mapsto\exp(A).
\end{equation}
It maps a neighborhood of $O\in\Lie(G)$ onto a neighborhood of
$1\in G$. Why is this true? Because look, this is definitely true
if the matrix group is a Lie group. Why? Because look, what we
have is a map of the tangent spaces to the manifold. It is
obvious this matriux is nondegenerate. It is really easy to check
for classical groups.

If $B=\exp(A)$, then $A=\log(B)$. So we have
\begin{equation}
\exp\colon\Lie(G)\to G
\end{equation}
and 
\begin{equation}
\log\colon G\to\Lie(G)
\end{equation}
which exists in the neighborhood of 1. In reality matrix groups
are the only thing that are interesting. However we should also
consider other groups. This requires a general definition of Lie
algebras. 

Take any Lie group $G$ (so take $1=e\in G$, we can introduce a
coordinate system in the neighborhood of 1, $(x^1,...,x^n)$,
so it is topologically equivalent to the unit ball, and moreover
permits functions to be differentiable). Take the tangent space
at the identity $T_{{\bf 1}}G$ and introduce an operation in such
a way that makees it a Lie algebra. We consider curves in $G$,
$x(\tau)$, such that $x(0)=1$ and
\begin{equation}
\left.\frac{dx}{d\tau}\right|_{\tau=0}\in\Lie(G).
\end{equation}
Although it is coordinate dependent, we know how to change
coordinates $x\to y(x)$, $y^i=y^i(x)$, and
\begin{equation}
\frac{dy^i}{d\tau}=\frac{\partial y^i}{\partial x^j}\frac{d x^j}{d\tau},
\end{equation}
so we could define curves in any coordinate system and obtain the
curve in any coordinate system.

We still need to define the Lie Bracket $[\xi,\eta]$ for
$\xi,\eta\in T_{{\bf 1}}G$. We would like it to be compatiible
with the commutator of matrices. So what to do?

We will draw
\begin{equation}
\xi=\left.\frac{dx}{d\tau}\right|_{\tau=0},\qquad\hbox{and}\qquad\eta=\left.\frac{d\widetilde{x}}{d\tau}\right|_{\tau=0}
\end{equation}
where $x$, $\widetilde{x}$ are curves in the group. We introduce
the commutator
\begin{equation}
x(\sqrt{\tau})\widetilde{x}(\sqrt{\tau})x(\sqrt{\tau})^{-1}\widetilde{x}(\sqrt{\tau})^{-1},
\end{equation}
we would like
\begin{equation}
[\xi,\eta]=\left.\frac{d}{d\tau}x(\sqrt{\tau})\widetilde{x}(\sqrt{\tau})x(\sqrt{\tau})^{-1}\widetilde{x}(\sqrt{\tau})^{-1}\right|_{\tau=0}.
\end{equation}
We have already proven when we work with matrix groups, this is
the exact same thing as the commutator for matrices.

This is not quite the end of the story. But from this definition,
we can derive a lot of important stuff. Namely, if $G$, $G'$ are
two Lie groups and $\Phi\colon G\to G'$ is a Lie group morphism,
then $\Lie(\Phi)\colon\Lie(G)\to\Lie(G')$ is a Lie Algebra
morphism.
\medbreak
\begin{prop}
$\Lie$ is a functor.
\end{prop}
\medbreak
Note that the Lie group morphism maps $\Phi(e)=e'$,
$\Phi_{*}(T_{{\bf 1}}G)=T_{{\bf 1}'}G'$ which is then a Lie
algebra morphism.

\medbreak\noindent\textbf{N.B.} If the group is simply connected,
Lie algebra morphisms induce Lie group morphisms.

\subsection{Exercises}

Lie algebra $\mathfrak{sl}(n)$ (denoted also by the symbol $\ClassicalGroup{A}_{n-1}$ ) consists of traceless $n\times n$ complex matrices. The symbol $E_{i,j}$ denotes a matrix with only one non-zero entry that is equal to 1 and located in $i$-th row and $j$-th column.
\begin{exercise}
Check that the matrices $E_{i,j}$ for $i = j$ and the matrices $h_{i} = E_{i,i} - E_{i+1,i+1}$ form a
basis of $\mathfrak{sl}(n)$. Find the structure constants in this basis.
\end{exercise}
\begin{exercise}
Check that subalgebra $\frak{h}$ of all diagonal matrices is a maximal
commutative subalgebra. Prove that there exists a basis of $\frak{sl}(n)$
consisting of eigenvectors for elements of $\frak{h}$. (This means
that $\frak{h}$ is a Cartan subalgebra of $\frak{sl}(n)$.)
\end{exercise}
\begin{exercise}
Check that $e_{i} = E_{i,i+1}$ and $f_{i} = E_{i+1,i}$ form a system of multiplicative generators of
$\frak{sl}(n)$. Prove relations
\begin{subequations}
\begin{align}
  [e_{i} , f_{j} ] = \delta_{ij} h_{i},\qquad [h_{i} , h_{j} ] = 0,\\
  [h_{i} , e_{j} ] = a_{ij} e_{j} ,\qquad [h_{i} , f_{j} ] = -a_{ij} f_{j},\\
  (\ad{e_{i}} )^{-a_{ij} +1} e_{j} = 0,\qquad (\ad{f_{i}} )^{-a_{ij} +1} f_{j} = 0
\end{align}
\end{subequations}
for some choice of matrix $a_{ij}$.

 We use here the notation $\ad(x)$ for the operator transforming $y$ into $[x, y]$.
\end{exercise}
