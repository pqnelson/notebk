%%
%% lecture03.tex
%% 
%% Made by Alex Nelson
%% Login   <alex@tomato3>
%% 
%% Started on  Thu Jan 21 10:33:51 2010 Alex Nelson
%% Last update Thu Jan 21 13:10:56 2010 Alex Nelson
%%

Let us consider $n$-dimensional space $\Bbb{R}^{n}$, with
coordinates $(x^1,\ldots,x^n)$, and we can consider either
functions or polynomials of these coordinates
$\Bbb{C}[x^1,\ldots,x^n]$ and we will consider the differential
operators on $\Bbb{C}[x^1,...,x^n]$. It is an associative
algebra, but also a Lie algebra (when the Lie bracket is the
commutator). We can consider Lie subalgebras, e.g. first order
differential operators\footnote{Note that we are using Einstein
  summation convention; when one index is upstairs and another is
  downstairs, we sum over it as a dummy index.}$\widehat{A}=A^{i}\partial_{i}$. But this
is \emph{NOT} a subalgebra of derivations, the product
$\widehat{A}\widehat{B}$ is a second order differential operator;
however note that
\begin{subequations}
\begin{align}
[\widehat{A},\widehat{B}](f)&=A^{i}\partial_{i}(B^{j}\partial_{j}f)-B^{j}\partial_{j}(A^{i}\partial_{i}f)\\
&=A^{i}(\partial_{i}B^{j})\partial_{j}f+A^{i}B^{j}\partial_{i}\partial_{j}f-B^{j}(\partial_{j}A^{i})\partial_{i}f-A^{i}B^{j}\partial_{i}\partial_{j}f\\
&=A^{i}(\partial_{i}B^{j})\partial_{j}f-B^{j}(\partial_{j}A^{i})\partial_{i}f
\end{align}
\end{subequations}
So we write
\begin{equation}
\widehat{C}=C^{k}\partial_{k}=A^{i}(\partial_{i}B^{j})\partial_{j}-B^{j}(\partial_{j}A^{i})\partial_{i}
\end{equation}
This $\widehat{C}$ is a derivation on $\Bbb{C}[x^1,...,x^n]$. We
can write
$C^k=A^{j}(\partial_{j}B^{k})-B^{j}(\partial_{j}A^{k})$. The
commutator of first order differential operators is again a first
order differential operator.

We would like to express this operator in two different
ways. First what are the coefficients $A^i$? They are teh
components of a vector field. So this is really the algebra of
vector fields, the commutator of vector fields yield a Lie
Algebra. Second, we want to introduce the notion of derivation of
algebra. It is something satisfying the Leibniz rule. Suppose we
have an $\scr{A}$-algebra, and a linear map
\begin{equation}
\alpha\colon\scr{A}\to\scr{A}
\end{equation}
such that
\begin{equation}
\alpha(ab)=\alpha(a)b+a\alpha(b).
\end{equation}

First, these first order differential operators are derivations,
and moreover all derivations are first order differential
operators.

This is a bit ambiguous, the algebra considered are left
unspecified (smooth functions or polynomials!). We willl prove it
for polynomials, but not for smooth functions. If we know how a
derivation behaves on the generators of the polynomial, then we
know everything. Let $\widehat{A}(x^i)=A^i(x)$ where $A^i(x)$ is
a polynomial.

\begin{rmk}
All this stuff works on smooth manifolds despite never specifying
what a ``smooth manifold'' is!
\end{rmk}

\begin{thm}
Given an algebra $\scr{A}$, then we may consider $\der(\scr{A})$
of derivations of $\scr{A}$ which form a Lie algebra.
\end{thm}
\begin{proof}
We should prove it is a vector space, but it is obvious; we
should prove the commutator of derivations
$\alpha,\beta\in\der(\scr{A})$ is a derivation
$[\alpha,\beta]\in\der(\scr{A})$. We consider
\begin{subequations}
\begin{align}
(\alpha\circ\beta)(ab)&=\alpha(\beta(ab))\\
&=\alpha(\beta(a)\cdot b+a\cdot\beta(b))\\
&=\alpha(\beta(a)\cdot b)+\alpha(a\cdot \beta(b))\qquad\hbox{by linearity}\\
&=(\alpha\circ\beta)(a)\cdot b+\beta(a)\alpha(b)+\alpha(a)\beta(b)+a\cdot(\alpha\circ\beta)(b)
\end{align}
\end{subequations}
Now we can consider the commutator expression of $\alpha$ with
$\beta$, which amounts to
\begin{subequations}
\begin{align}
[\alpha,\beta](ab) &= \bigg((\alpha\circ\beta)(a)\cdot b+\beta(a)\alpha(b)+\alpha(a)\beta(b)+a\cdot(\alpha\circ\beta)(b)\bigg)\nonumber\\
&\quad-\bigg((\beta\circ\alpha)(a)\cdot b+\beta(a)\alpha(b)+\alpha(a)\beta(b)+a\cdot(\beta\circ\alpha)(b)\bigg)\\
&=(\alpha\circ\beta)(a)b+a\cdot(\alpha\circ\beta)(b)-(\beta\circ\alpha)(a)b-a\cdot(\beta\circ\alpha)(b)\\
&=[\alpha,\beta](a)\cdot b+a\cdot[\alpha,\beta](b).
\end{align}
\end{subequations}
This concludes our proof.
\end{proof}

One last example of derivations. Consider an algebra $\scr{A}$
(either associative or Lie), take $a,x\in\scr{A}$ where $a$ is
fixed. Consider the derivation
\begin{equation}
\alpha_{a}(x)=[a,x].
\end{equation}
For Lie algebras it is absolutely trivial:
\begin{subequations}
\begin{align}
\alpha_{a}([x,y]) &= [\alpha_{a}(x),y]+[x,\alpha_{a}(y)]\\
\iff[a,[x,y]] &=[[a,x],y]+[x,[a,y]]\\
&=-[x,[y,a]]-[y,[a,x]]\qquad\hbox{Jacobi Identity!}
\end{align}
\end{subequations}
\begin{rmk}
Such derivations are called \define{Inner Derivations}.
\end{rmk}
Lets compute the commutator of two inner derivations, the answer
is the
\begin{equation}
[\alpha_a,\alpha_b]=\alpha_{[a,b]}
\end{equation}
the result is an  inner derivation. We have a homomorphism, so we
have a $\scr{G}$-Lie algebra so we get a map
$\scr{G}\to\der({\scr{G}})$ which, for all $a\in\scr{G}$, is mapped
to
\begin{equation}
\alpha_a=[a,-].
\end{equation}

\medbreak
\noindent\textbf{N.B.:} Henceforth and throughout, I will use the
term ``morphism'' and ``homomorphism'' interchangeably. 

We can consider the morphism $\scr{G}\to L(\scr{G})$ where
$L(\scr{G})$ is the linear operators on $\scr{G}$. We have a
representation of our Lie algebra $\scr{G}$, called the
\define{Adjoint Representation} where
$a\mapsto\alpha_a=[a,-]$. We write $\ad_{a}=\alpha_a$. This is
one of the simplest and most important examples of the
representation of Lie algebras.

Consider $\ker(\ad)$. Then $\alpha_a=0$. What does it mean that
$[a,x]=0$ for all $x\in\scr{G}$? This is precisely the
\define{Center of $\scr{G}$} denoted by $\ker(\alpha)=Z$.
\begin{thm}
If a finite dimensional Lie Algebra has no center (or a trivial one), then it is
isomorphic to a matrix Algebra.
\end{thm}
\begin{proof}
We see that $\im(\ad)$ consists of a subalgebra of a Lie algebra
of matrices since the Lie algebra \emph{IS} a vector space and it
is finite dimensional. Thus $\ad$ is a matrix algebra.
\end{proof}
