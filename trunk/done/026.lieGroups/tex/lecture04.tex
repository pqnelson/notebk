%%
%% lecture04.tex
%% 
%% Made by Alex Nelson
%% Login   <alex@tomato3>
%% 
%% Started on  Thu Jan 21 11:03:01 2010 Alex Nelson
%% Last update Fri Dec 10 11:46:22 2010 Alex Nelson
%%

We started discussing the adjoint representation last time. Given
a Lie algebra $\scr{G}$ and $a\in\scr{G}$, we can construct
$\alpha_a=[a,-]$, $\alpha_a\colon\scr{G}\to\scr{G}$ and obeys
\begin{equation}
\alpha_{[a,b]}=[\alpha_a,\alpha_b].
\end{equation}
The adjoint representation for $\Bbb{R}^{3}$ equipped with the
cross-product form $\Lie({\rm SO}(3))$ --- the Lie algebra for
$\SO{3}$. The derivations are defined completely if we know the
derivations of the generators; moreover, for e.g. $\widehat{\imath}$,
$\widehat{\jmath}$, $\widehat{k}$ we only need to think about
$\widehat{\imath}$, $\widehat{\jmath}$ since
$\widehat{\imath}\times\widehat{\jmath}=\widehat{k}$. 

\begin{defn}
The \define{Structure Constants} for a Lie algebra $\scr{G}$
(with generators $e_1$, ..., $e_n$ which form a basis of the
vector space) are specified by
$[e_\alpha,e_\beta]={f_{\alpha\beta}}^{\gamma}e_{\gamma}$, where
${f_{\alpha\beta}}^{\gamma}$ are the structure constants.
\end{defn}
\begin{rmk}
If we have a real Lie algebra, the structure constants are real;
on the other hand, for complex Lie algebra, the structure
constants are complex.
\end{rmk}
\begin{defn}
Let $\scr{G}$ be a real Lie algebra. Its
\define{Complexification} consists of a Lie algebra denoted by
$\Bbb{C}\scr{G}$ constructed by making the structure constants complex.
\end{defn}
So complexification is some ``mapping''
\begin{align*}
(\hbox{Real Lie Algebras})&\to(\hbox{Complex Lie Algebras})\\
x^{\alpha}e_{\alpha}&\mapsto z^{\alpha}e_{\alpha}
\end{align*}
where $z^{\alpha}=x^{\alpha}+iy^{\alpha}\in\Bbb{C}$,
$x^\alpha\in\Bbb{R}$. So we can write
$\Bbb{C}\scr{G}=\scr{G}\oplus i\scr{G}$ where we have this direct
sum be the direct sum of vector spaces. This induces a notion of
multiplication.

\begin{ex}
Consider the Lie algebra $\frak{so}(n)$ which is a real Lie
algebra consisting of antisymmetric $n$-by-$n$ matrices:
\begin{equation}\label{eq:lec4:son}
A^{T}+A=0,
\end{equation}
where $A\in\mat_{n}(\Bbb{R})$. We want to complexify it, which is
very easy We consider the same condition but take matrices with
complex entries $A\in\mat_{n}(\Bbb{C})$ which obey Eq \eqref{eq:lec4:son}.
So we write $\frak{so}(n,\Bbb{C})=\Bbb{C}\frak{so}(n,\Bbb{R})$.
\end{ex}
\begin{ex}
The algebra $\frak{u}(n)$ of anti-Hermitian matrices $A+A^{\dagger}=0$
where $A^{\dagger}$ is the Hermitian conjugate (i.e. conjugate
transpose of $A$). We have $A\in\mat_{n}(\Bbb{C})$ but what is
its complexification? Observe
\begin{equation}
\Bbb{C}\frak{u}(n)=\frak{gl}(n,\Bbb{C}),
\end{equation}
why? Well, for $X\in\mat_{n}(\Bbb{C})$ we have
\begin{equation}
X = \frac{1}{2}\underbracket[0.5pt]{(X-X^{\dagger})}_{\text{anti-Hermitian}}+\frac{1}{2}\underbracket[0.5pt]{(X+X^{\dagger})}_{\text{Hermitian}}
\end{equation}
but observe if $A$ is anti-Hermitian, we have $A+A^{\dagger}=0$,
then $iA$ is Hermtian since
\begin{equation}
(\I A)^{\dagger}+\I A=\I (-A^{\dagger}+A)=0.
\end{equation}
So $\scr{G}$ is anti-Hermitian, $\I \scr{G}$ is Hermitian, and
$\Bbb{C}\scr{G}$ is everything.
\end{ex}

We like to work in $\Bbb{C}$ since it is simpler than working in $\Bbb{R}$.

\begin{thm}
Complex representations of real Lie algebras are in one-to-one
correspondence with the complex representations of its complexification.
\end{thm}
A representation maps basis vectors to linear operators,
requiring us to solve
\begin{equation}
[\widehat{e}_{\alpha},\widehat{e}_{\beta}]={f_{\alpha\beta}}^{\gamma}\widehat{e}_{\gamma},
\end{equation}
where $\varphi(e_{\alpha})=\widehat{e}_{\alpha}$. For the
complexified Lie algebra, we do \emph{precisely the same thing!}

We will define matrix groups, then matrix Lie algebras. It will
not be a general definition.
\begin{defn}
A \define{Matrix Group} is a closed subgroup of $\GL{n,\RR}$
or $\GL{n,\CC}$.
\end{defn}
\begin{thm}
The tangent space to the matrix group at the point $I=e=1$ the
identity is a Lie algebra called the Lie algebra of the matrix group.
\end{thm}

Suppose we have a curve $x(t)\in\Bbb{R}^{m}$ or in any
topological space. Well, since $x$ is a curve, it's a mapping
\begin{equation}
x\colon[\alpha,\beta]\to\RR^{m}
\end{equation}
from an interval of the real line $[\alpha,\beta]$ to the space,
the tangent vector is
\begin{equation}
\left.\frac{\D x(t)}{\D t}\right|_{t=\alpha}=x'(\alpha).
\end{equation}
The tangent space at $x_{0}\in M$ for a manifold $M$ is the
vector space of all tangent vectors at $x_{0}$. If the surface is
given by
\begin{equation}
f(x^1,\dots,x^n)=0,
\end{equation}
we promote $x\mapsto x^{i}(t)$ to be components of a curve,
implying
\begin{equation}
f(x^{1}(t),\dots,x^{n}(t))=0.
\end{equation}
We thus have by the chain rule
\begin{equation}
\left.\frac{\D }{\D t}f(x^{1}(t),\dots,x^{n}(t))\right|_{t=\alpha}=\left.\frac{\partial
  f}{\partial x^i}\frac{\D x^i}{\D t}\right|_{t=\alpha}=0.
\end{equation}
If we use the implicit function theorem, we can consider $A^{i}$
vectors such that
\begin{equation}
\left.A^{i}\frac{\partial f}{\partial x^{i}}\right|_{t=\alpha}=0.
\end{equation}

\begin{ex}
Consider $O(n)=\{A\mid A^{T}A=I\}$ the group of $n$-by-$n$
orthogonal matrices. We take $A(t)$ to be a curve in $O(n)$ such
that $A(0)=I$ is the identity. So $A(t)=I+a(t)$. Consider then
\begin{equation}
(I+a(t))(I+a(t)^{T})=I+a(t)+a(t)^{T}+\mathcal{O}(t^{2}),
\end{equation}
then we can deduce the structure of the Lie algebra for
``infinitesimal $a(t)$'' to be precisely the matrices $X$ such
that
\begin{equation}
X+X^{T}=0.
\end{equation}
That is, all antisymmetric matrices.
\end{ex}


If one has forgotten the implicit function theorem, here it is reproduced:
\begin{implicitFunctionThm}
Let $f\colon\Bbb{R}^{n+m}\to\Bbb{R}^{m}$ be a continuously
differentiable function. Fix
$(\boldsymbol{x},\boldsymbol{y})\in\Bbb{R}^{n+m}$,
$\boldsymbol{x}\in\Bbb{R}^{n}$,
$\boldsymbol{y}\in\Bbb{R}^{m}$ such that
$f(\boldsymbol{x},\boldsymbol{y})=\boldsymbol{c}$ where $\boldsymbol{c}\in\Bbb{R}^{m}$. If the matrix
\begin{equation}
{J_{i}}^{j}=\frac{\partial f_{i}(\boldsymbol{x},\boldsymbol{y})}{\partial y^{j}}
\end{equation}
is invertible, then there exists an open set $U$ containing
$\boldsymbol{x}$, an open set $V$ containing $\boldsymbol{b}$,
and a unique continuously differentiable function $g\colon U\to
V$ such that
\begin{equation}
\{ (\boldsymbol{x}, g(\boldsymbol{x})) \} = \{ (\boldsymbol{x}, \boldsymbol{y}) | f(\boldsymbol{x}, \boldsymbol{y}) = \boldsymbol{c} \} \cap (U \times V).
\end{equation}
\end{implicitFunctionThm}
