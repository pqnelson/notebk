%%
%% lecture14.tex
%% 
%% Made by Alex Nelson
%% Login   <alex@tomato3>
%% 
%% Started on  Sun Jun 20 12:44:46 2010 Alex Nelson
%% Last update Sun Jun 20 13:21:21 2010 Alex Nelson
%%
We have defined a reductive Lie algebra as the complexification
of the Lie algebra for a compact group. The Cartan subalgebra
(usually denoted $\mathfrak{h}$) corresponds to the maximal torus
in the group $T\subset G$ is the maximal, Abelian, connected
subgroup. Diagrammatically 
\begin{equation}
\begin{diagram}[small]
\mathfrak{g} & \rTo^{\exp} & G\\
\uTo         &             & \uTo\\
\mathfrak{h} & \rTo^{\exp} & T
\end{diagram}
\end{equation}
It is obvious that $T=\U{1}\times\cdots\times\U{1}$.

We can see $\Bbb{R}$ is the Lie algebra for
$\Bbb{R}^{\times}_{>0}$ positive reals equipped with
multiplication. Note that
\begin{equation}
\Bbb{R}^{\times}_{>0}\iso\Bbb{R}^{+}
\end{equation}
is an isomorphism for $\Bbb{R}^{+}$ the reals equipped with
addition. This is not a unique Lie algebra, we should factorize
with respect to a discrete subgroup. To get a compact group,
$\U{1}^{n}$ is the only choice.

Let $G$ be a compact group. Let $\mathscr{G}=\Lie(G)$, since
$T\subset G$, we see $\mathbb{G}\Lie(T)=\mathscr{H}$ is the
Cartan subalgebra. This is its definition.

It is tempting to define the Cartan subalgebra as the maximal
Abelian subalgebra of $\mathscr{G}$, but this is wrong. It is
clear the Cartan subalgebra is really important, since the
character restricted to the maximal Torus permits us to
reconstruct all information of the representation. We will
consider a representation of the Lie algebra
\begin{equation}
\varphi\colon\mathscr{G}\to\mathfrak{gl}(V),
\end{equation}
and we will assume the representation comes from a compact Lie
group
\begin{equation}
\Phi\colon G\to\GL{V}.
\end{equation}
This is true for compact groups, and the corresponding
representation of the corresponding Lie algebra. Consider a
representation for an Abelian Lie algebra, it is not necessarily
semisimple; not every representation of Abelian Lie algebras
comes from $\U{1}$, it comes from $\Bbb{R}^{+}$. We have for any
$z$, a representation of $\U{1}$ is $z\mapsto z^{n}$,
$n\in\Bbb{Z}$. We can restrict the representation $\Phi$ to the
maximal torus, and it is completely reducible. The same may be
said about
\begin{equation}
\varphi\colon\mathscr{H}\to\mathfrak{gl}(V)
\end{equation}
the representation of the Cartan subalgebra, it is completely
reducible. What are the irreps of Ablian Lie algebras? It is very
easy to see irrep[s of Abelian Lie Algebras are one-dimensional;
it may be proven in precisely one million different ways.

Now lets introduce the notion of a \define{Weight Vector} of a
representation $\varphi$, it is a vector $x\in V$ such that
\begin{equation}
\varphi(h)x=\lambda(h)x
\end{equation}
it is an eigenvector for the Cartan subalgebra, we call
$\lambda(h)$ the weight (it is a linear functional for the vector
space underlying $\mathscr{H}$). That is to say
$\lambda\in\mathscr{H}^{*}$. 

\begin{prop}
There exists a basis of $V$ consisting of weight vectors.
\end{prop}

If we know all the weights, we may compute the character of the
representation. Suppose we write
\begin{equation}
\mathscr{H}=\left\{\sum_{k}\xi^{k}h_{k}\right\},
\end{equation}
then any linear fuction $h_{k}\mapsto\alpha_{k}$ acts as
\begin{equation}
\sum \xi^{k}h_{k}\mapsto \sum\xi^{k}\alpha_{k}.
\end{equation}
There is an exponential map
\begin{equation}
A\mapsto\exp(A)
\end{equation}
that maps $A\in\mathscr{G}$ to $\exp(A)\in G$. If we have a
one-parameter family in the Lie algebra $tA$, then we have a
one-parameter family in the subgroup $\exp(tA)$.

Suppose the coordinates of the Lie group is
$\{z^{1},\ldots,z^{k}\}$, then the basis for the Lie algebra
would be
\begin{equation}
\xi^{j}=\log(z^{j}),
\end{equation}
to recover the group you should exponentiate. We get the maximal
torus, and then we can compute the trace.

We have the adjoint representation
\begin{equation}
\alpha_{i}(h)E_{i} = [h,E_{i}]
\end{equation}
where $E_{i}$ is the weight vector for $\alpha_{i}(h)$. Nonzero
weights for the adjoint representation are called \define{Roots}\index{Root}
and weight vectors for the adjoint representation are
called \define{Root Vectors}\index{Root Vector}.We can give a
different definition for the Cartan subalgebra:
\begin{defn}\index{Cartan Subalgebra}\index{Lie Algebra!Cartan Subalgebra}
The \define{Cartan Subalgebra} is a maximal Abelian subalgebra
that is completely reducible in (all representations, but in
particular) the Adjoint representation.
\end{defn}

Consider $\U{n}$, and
$\mathfrak{gl}(n)=\Bbb{C}\Lie\big(\U{n}\big)$, the maximal torus
here consists of all diagonal matrix
\begin{equation}
H = \begin{bmatrix}
\exp(i\varphi_{1}) &        &\\
                   & \ddots & \\
                   &        & \exp(i\varphi_{n})
\end{bmatrix}
\end{equation}
is the maximal Torus. Here the basis of the weight vectors may be
taken to be the standard basis, the representation is
\begin{equation}
\rho\colon\U{n}\to\aut(\Bbb{C}^{n}).
\end{equation}
So we have $n$ weights which are functionals on $\mathscr{H}$.
