%%
%% 16May2008.tex
%% 
%% Made by Alex Nelson
%% Login   <alex@tomato>
%% 
%% Started on  Fri Dec 26 18:15:38 2008 Alex Nelson
%% Last update Fri Dec 26 18:15:38 2008 Alex Nelson
%%
%% Fourier inversion formula, applications

\begin{lem}
If $f\in L^{1}$, then $\widehat{f}$ is continuous.
\end{lem}
\begin{proof}
Observe by direct computation
\begin{align*}
|\widehat{f}(\xi)-\widehat{f}(\eta)| &=
\left|\int\left(e^{-i\xi x}-e^{-i\eta
  x}\right)f(x)dx\right|\\
&\leq 2\int|f(x)|dx\quad\forall\eta,\xi\\
&\leq \int\left|e^{-i\xi x}-e^{-i\eta x}\right||f(x)|dx
\end{align*}
But as $\xi\to\eta$, $e^{-i\xi x}-e^{-i\eta x}\to 1-1=0$, so
the quantity
\begin{equation}
\int\left|e^{-i\xi x}-e^{-i\eta x}\right||f(x)|dx\to 0
\end{equation}
which concludes our proof.
\end{proof}
The Fourier transform so far has been shown to be one way,
mapping functions of $x$ to functions of $\xi$. Can we go
the other way? That is to say, is the Fourier transform
invertible\marginpar{Inverse Fourier Transform}?

\begin{invFourier}\index{Fourier Inversion Theorem}\index{Fourier Transform!Inverse}\index{Inverse Fourier Transform}
Suppose $f\in L^{1}$ and $f\in PC(\mathbb{R})$, then
\begin{equation}
\lim_{\varepsilon\to0}\frac{1}{2\pi}\int e^{i\xi x}e^{-\varepsilon^2\xi^2/2}\widehat{f}(\xi)d\xi=\frac{1}{2}\left[f(x^+)+f(x^-)\right]
\end{equation}
for all $x\in\mathbb{R}$. Moreover if both $f,\widehat{f}\in
L^{1}$, then $f$ is continuous, and
\begin{equation}
f(x) = \frac{1}{2\pi}\int e^{i\xi x}\widehat{f}(\xi)d\xi
\end{equation}
for all $x\in\mathbb{R}$.
\end{invFourier}
\begin{proof}
We have
\begin{equation}
\frac{1}{2\pi}\int e^{i\xi x}e^{-\varepsilon^2\xi^2/2}\widehat{f}(\xi)d\xi = \frac{1}{2\pi}\int e^{i\xi x}e^{-\varepsilon^{2}\xi^2/2}\int e^{-i\xi y}f(y)dyd\xi
\end{equation}
We can interchange the variables of integration, thus
\begin{align*}
\frac{1}{2\pi}\int e^{i\xi x}e^{-\varepsilon^{2}\xi^2/2}\int e^{-i\xi y}f(y)dyd\xi
&= \frac{1}{2\pi}\int\int e^{-i\xi(x-y)}e^{-\varepsilon^2\xi^2/2}f(y)d\xi dy\\
&= \frac{1}{2\pi}\int f(y)\left[\int e^{-i\xi(y-x)}e^{-\varepsilon^{2}\xi^2/2} \right]dy\\
&= \frac{1}{2\pi}\int f(y)\left[\sqrt{\frac{2\pi}{\varepsilon^2}}e^{-(y-x)^2/2\varepsilon^2}\right]dy\\
&= \frac{1}{\varepsilon\sqrt{2\pi}}\int f(y)e^{-(y-x)^{2}/2\varepsilon^2}dy\\
&=(f*\phi_{\varepsilon})(x)
\end{align*}
It turns out that $\phi(x) = \exp(-x^2/2)/\sqrt{2\pi}$ is
the Gaussian. This function is even, and integral is
evaluated to one. So it's $C^\infty$. By theorem \eqref{thm:12May2008:thmUsedInInverseFourierTransform},
$f*\phi_{\varepsilon}(x)\to\frac{1}{2}[f(x^-)+f(x^+)]$.

Suppose $f\in L^{1}$, then we need to show $f$ is
continuous. Since
\begin{equation}
|e^{i\xi x}e^{-\varepsilon^2\xi^2/2}\widehat{f}(\xi)|\leq|\widehat{f}(\xi)|
\end{equation}
for all $\xi$. We see that $|\exp(-\varepsilon^2\xi^2/2)|<1$
and $|\exp(i\xi x)|=1$. So
\begin{equation}
\int e^{i\xi
  x}e^{-\varepsilon^2\xi^2/2}\widehat{f}(\xi)d\xi\leq
\int|\widehat{f}(\xi)|d\xi = \|\widehat{f}\|_{L^1}
\end{equation}
for all $\varepsilon>0$. So
\begin{align*}
\mathcal{F}\{\widehat{f}\}(-x) &= \frac{1}{2\pi}\int e^{i\xi x}\widehat{f}(\xi)d\xi\\
&=\frac{1}{2\pi}\int e^{i\xi x}\left(\lim_{\varepsilon\to0}e^{-\varepsilon^2\xi^2/2}\right)\widehat{f}(\xi)d\xi\\
&=\lim_{\varepsilon\to0}\frac{1}{2\pi}\int e^{i\xi x}e^{-\varepsilon^2\xi^2/2}\widehat{f}(\xi)d\xi\\
&=\frac{1}{2}[f(x^+)+f(x^+)]\\
&= f(x)
\end{align*}
where we justify the last step since $f$ is continuous. So
$\mathcal{F}\{\widehat{f}\}(-x)$ is continuous since
$\widehat{f}\in L^1$, therefore $f$ is continuous so we're done.
\end{proof}

\begin{rmk}
If $f\in L^{1}$ and $f\in PC$, then $\widehat{f}$ may not be
in $L^1$. So we only have an approximation, which is the
first part of the theorem, then
\begin{equation*}
\underbracket[0.5pt]{e^{-\varepsilon^2\xi^2/2}\widehat{f}(\xi)}_{\text{approx
    of }\widehat{f}}=\underbracket[0.5pt]{\mathcal{F}\left\{f*\left(\frac{1}{\sqrt{2\pi}\varepsilon}e^{-(x/\varepsilon)^2/2}\right)\right\}}_{\text{approaches $f$ as $\varepsilon\to0$}}
\end{equation*}
\end{rmk}

\subsection{Consequences of the Inversion Theorem}

\begin{enumerate}
\item If $\widehat{f}=\widehat{g}$, then
  $\widehat{F}\{f-g\}=\widehat{f}-\widehat{g}=0$ which
  implies $f=g$. Thus the Fourier transform is
  unique\index{Fourier Transform!Uniqueness}. Further
  $\mathcal{F}^{-1}$ is well defined exactly by the
  inversion formula.
\item We have $\phi(x) = \exp(-x^2/2)/\sqrt{2\pi}$ in the
  inversion formula can be replaced by any (normalized)
  $g\in L^1$ with $\widehat{g}\in L^1$.
\item If $f\in L^1$, $\widehat{f}\in L^1$, then $f$ is
  continuous and $f\in L^2$. So
\begin{equation}
\int |f(x)|^2dx = \int|f(x)|\left|\int e^{i\xi x}\widehat{f}(\xi)d\xi\right|dx
\end{equation}
but
\begin{align*}
\left|\int e^{i\xi x}\widehat{f}(\xi)d\xi\right| &\leq
\int\left|e^{i\xi x}\widehat{f}(\xi)\right|d\xi\\
&\leq\int|\widehat{f}(\xi)|d\xi
\end{align*}
thus
\begin{align*}
\int|f(x)|^2dx &\leq \int|f(x)||\widehat{f}(\xi)|d\xi dx\\
&\leq \int|\widehat{f}(\xi)|d\xi\int|f(x)|dx.
\end{align*}
\end{enumerate}

\subsection{Fourier Transform on $L^2$}

Let $f\in L^2$, then $\widehat{f}(\xi)=\int\exp(-i\xi
x)f(x)dx$ converges in $L^2$ norm. This exists for almost
every $\xi$. Similarly
\begin{equation}
f(x) = \frac{1}{2\pi}\int e^{i\xi x}\widehat{f}(\xi)d\xi
\end{equation}
for almost every $x$.

\begin{plancherel}\index{Plancherel's Theorem}\index{Fourier Transform!In $L^2$}\index{Fourier Transform!Plancherel's Theorem}
The Fourier Transform is an operator
\begin{equation}
\mathcal{F}:L^2\to L^2
\end{equation}
and $\<\widehat{f},\widehat{g}\>=2\pi\<f,g\>$. If this is
the case, this implies 
\begin{equation}
\|\widehat{f}\|^{2} = 2\pi\|f\|^2.
\end{equation}
This is our Parseval identity in the continuous
case.\index{Parseval Equality!Fourier Transform}
\end{plancherel}
\begin{proof}
We see by direct computation
\begin{align*}
2\pi\<f,g\> &= 2\pi\int f(x)\overline{g(x)}dx\\
&=\frac{2\pi}{2\pi}\int
f(x)\overline{\int\widehat{g}(\xi)e^{i\xi x}d\xi}dx\\
&=\int f(x)\int\overline{\widehat{g}(\xi)}e^{-i\xi x}d\xi dx\\
&= \int \widehat{f}(\xi)\overline{\widehat{g}(\xi)}d\xi\\
&=\<\widehat{f},\widehat{g}\>
\end{align*}
\end{proof}
