\subsection{A Slightly More Complicated Example}

This example will teach you that any idiot can complicate a simple scheme, consider
the following diagram:

\strut

\begin{center}
\begin{fmffile}{exTwoImg1}
  \begin{fmfgraph*}(70,25)  \fmfpen{0.1mm}

    \fmfleft{i1,o1} 
    \fmfright{i2,o2} 
    \fmflabel{$A,p_{1}$}{i1}
    \fmflabel{$B,p_{3}$}{o1} %
    \fmflabel{$A,p_{2}$}{i2} %
    \fmflabel{$B,p_{3}$}{o2}
    \fmf{plain}{i1,v1,o1} %
    \fmf{plain}{i2,v4,o2} %
    \fmf{plain,label=$C,,q_1$}{v1,v2}
    \fmf{plain,left,label=$B,,q_3$,tension=0.5}{v2,v3}
    \fmf{plain,left,label=$A,,q_2$,tension=0.5}{v3,v2}
    \fmf{plain,label=$C,,q_4$}{v3,v4}
  \end{fmfgraph*}
\end{fmffile}
\end{center}
\strut

\textbf{Step One:} We drew it carefully and made special care of the notation used.

\textbf{Step Two:} We need to take note of how many vertices we have, so we have
them enlarged in red to see how many:

\strut

\begin{center}
\begin{fmffile}{exTwoImg2}
  \begin{fmfgraph*}(70,25)  \fmfpen{0.1mm}

    \fmfleft{i1,o1} 
    \fmfright{i2,o2} 
    \fmflabel{$A,p_{1}$}{i1}
    \fmflabel{$B,p_{3}$}{o1} %
    \fmflabel{$A,p_{2}$}{i2} %
    \fmflabel{$B,p_{3}$}{o2}
    \fmf{plain}{i1,v1,o1} %
    \fmf{plain}{i2,v4,o2} %
    \fmf{plain,label=$C,,q_1$}{v1,v2}
    \fmf{plain,left,label=$B,,q_3$,tension=0.5}{v2,v3}
    \fmf{plain,left,label=$A,,q_2$,tension=0.5}{v3,v2}
    \fmf{plain,label=$C,,q_4$}{v3,v4}
    \fmfv{decor.shape=circle,decor.filled=full,decor.size=2thick,fore=red}{v1,v2,v3,v4}
  \end{fmfgraph*}
\end{fmffile}
\end{center}
\strut


we have 4 vertex terms, that is we have
\begin{equation*}
(-ig)^4
\end{equation*}
in the integrand so far.

\textbf{Step Three:} We need to take care of the internal lines now, so we will
see which lines those are exactly:


\strut

\begin{center}
\begin{fmffile}{exTwoImg3}
  \begin{fmfgraph*}(70,25)  \fmfpen{0.1mm}

    \fmfleft{i1,o1} 
    \fmfright{i2,o2} 
    \fmflabel{$A,p_{1}$}{i1}
    \fmflabel{$B,p_{3}$}{o1} %
    \fmflabel{$A,p_{2}$}{i2} %
    \fmflabel{$B,p_{3}$}{o2}
    \fmf{plain}{i1,v1,o1} %
    \fmf{plain}{i2,v4,o2} %
    \fmf{plain,label=$C,,q_1$,fore=red}{v1,v2}
    \fmf{plain,left,label=$B,,q_3$,tension=0.5,fore=red}{v2,v3}
    \fmf{plain,left,label=$A,,q_2$,tension=0.5,fore=red}{v3,v2}
    \fmf{plain,label=$C,,q_4$,fore=red}{v3,v4}
  \end{fmfgraph*}
\end{fmffile}
\end{center}
\strut

Since we are being pedagogical, we will go through one by one and indicate which
line we are dealing with and what we evaluate it to be. We will begin in any
old arbitrary manner we please with this particular example, \textbf{it won't
be that way in general!} We will first consider:

\strut
\begin{center}
\begin{fmffile}{exTwoImg4}
  \begin{fmfgraph*}(70,25)  \fmfpen{0.1mm}

    \fmfleft{i1,o1} 
    \fmfright{i2,o2} 
    \fmflabel{$A,p_{1}$}{i1}
    \fmflabel{$B,p_{3}$}{o1} %
    \fmflabel{$A,p_{2}$}{i2} %
    \fmflabel{$B,p_{3}$}{o2}
    \fmf{plain}{i1,v1,o1} %
    \fmf{plain}{i2,v4,o2} %
    \fmf{plain,label=$C,,q_1$,fore=red}{v1,v2}
    \fmf{plain,left,label=$B,,q_3$,tension=0.5}{v2,v3}
    \fmf{plain,left,label=$A,,q_2$,tension=0.5}{v3,v2}
    \fmf{plain,label=$C,,q_4$}{v3,v4}
  \end{fmfgraph*}
\end{fmffile}
\end{center}
\strut


We evaluate this to be the propagator described by
\begin{equation*}
\frac{i}{q_{1}^2 - m_{C}^2c^2}
\end{equation*}
so we multiply it into the integrand. The integrand is then
\begin{equation}
(-ig)^4 \frac{i}{q_{1}^2 - m_{C}^2c^2}.
\end{equation}
We continue on and we see the term given by the internal line in
red


\strut
\begin{center}
\begin{fmffile}{exTwoImg5}
  \begin{fmfgraph*}(70,25)  \fmfpen{0.1mm}

    \fmfleft{i1,o1} 
    \fmfright{i2,o2} 
    \fmflabel{$A,p_{1}$}{i1}
    \fmflabel{$B,p_{3}$}{o1} %
    \fmflabel{$A,p_{2}$}{i2} %
    \fmflabel{$B,p_{3}$}{o2}
    \fmf{plain}{i1,v1,o1} %
    \fmf{plain}{i2,v4,o2} %
    \fmf{plain,label=$C,,q_1$}{v1,v2}
    \fmf{plain,left,label=$B,,q_3$,tension=0.5}{v2,v3}
    \fmf{plain,left,label=$A,,q_2$,tension=0.5,fore=red}{v3,v2}
    \fmf{plain,label=$C,,q_4$}{v3,v4}
  \end{fmfgraph*}
\end{fmffile}
\end{center}
\strut

This corresponds to the propagator described by
\begin{equation*}
\frac{i}{q_{2}^2 - m_{A}^2c^2}.
\end{equation*}
We multiply this into our integrand now and we get
\begin{equation}
(-ig)^4 \frac{i}{q_{1}^2 - m_{C}^2c^2}\frac{i}{q_{2}^2 - m_{A}^2c^2} = -g^4\frac{1}{q_{1}^2 - m_{C}^2c^2}\frac{1}{q_{2}^2 - m_{A}^2c^2}.
\end{equation}
Similarly we can do likewise for the other part of the loop in
red:

\strut
\begin{center}
\begin{fmffile}{exTwoImg6}
  \begin{fmfgraph*}(70,25)  \fmfpen{0.1mm}

    \fmfleft{i1,o1} 
    \fmfright{i2,o2} 
    \fmflabel{$A,p_{1}$}{i1}
    \fmflabel{$B,p_{3}$}{o1} %
    \fmflabel{$A,p_{2}$}{i2} %
    \fmflabel{$B,p_{3}$}{o2}
    \fmf{plain}{i1,v1,o1} %
    \fmf{plain}{i2,v4,o2} %
    \fmf{plain,label=$C,,q_1$}{v1,v2}
    \fmf{plain,left,label=$B,,q_3$,tension=0.5,fore=red}{v2,v3}
    \fmf{plain,left,label=$A,,q_2$,tension=0.5}{v3,v2}
    \fmf{plain,label=$C,,q_4$}{v3,v4}
  \end{fmfgraph*}
\end{fmffile}
\end{center}
\strut

This corresponds to the propagator
\begin{equation*}
\frac{i}{q_{3}^2 - m_{B}^2c^2}
\end{equation*}
and we just multiply it into the integrand, which becomes
\begin{equation}
-g^4\frac{1}{q_{1}^2 - m_{C}^2c^2}\frac{1}{q_{2}^2 - m_{A}^2c^2}\frac{i}{q_{3}^2 - m_{B}^2c^2}.
\end{equation}
We have one last internal line left! We highlight it in red:


\strut
\begin{center}
\begin{fmffile}{exTwoImg7}
  \begin{fmfgraph*}(70,25)  \fmfpen{0.1mm}

    \fmfleft{i1,o1} 
    \fmfright{i2,o2} 
    \fmflabel{$A,p_{1}$}{i1}
    \fmflabel{$B,p_{3}$}{o1} %
    \fmflabel{$A,p_{2}$}{i2} %
    \fmflabel{$B,p_{3}$}{o2}
    \fmf{plain}{i1,v1,o1} %
    \fmf{plain}{i2,v4,o2} %
    \fmf{plain,label=$C,,q_1$}{v1,v2}
    \fmf{plain,left,label=$B,,q_3$,tension=0.5}{v2,v3}
    \fmf{plain,left,label=$A,,q_2$,tension=0.5}{v3,v2}
    \fmf{plain,label=$C,,q_4$,fore=red}{v3,v4}
  \end{fmfgraph*}
\end{fmffile}
\end{center}
\strut

This corresponds to the propagator
\begin{equation*}
\frac{i}{q_{4}^2-m_{C}^2c^2}
\end{equation*}
and we multiply it into the integrand, which becomes
\begin{equation}
-g^4\frac{1}{q_{1}^2 - m_{C}^2c^2}\frac{1}{q_{2}^2 - m_{A}^2c^2}\frac{i}{q_{3}^2 - m_{B}^2c^2}\frac{i}{q_{4}^2-m_{C}^2c^2} = g^4\frac{1}{q_{1}^2 - m_{C}^2c^2}\frac{1}{q_{2}^2 - m_{A}^2c^2}\frac{1}{q_{3}^2 - m_{B}^2c^2}\frac{1}{q_{4}^2-m_{C}^2c^2}.
\end{equation}

\textbf{Step Four:} We need to enfore the conservation of momentum, so what do
we do? We simply go back to our graph and go one vertex at a time and enforce
conservation of momentum. At the first vertex, the input momentum is in red
and the output momentum is in blue:


\strut
\begin{center}
\begin{fmffile}{exTwoImg8}
  \begin{fmfgraph*}(70,25)  \fmfpen{0.1mm}

    \fmfleft{i1,o1} 
    \fmfright{i2,o2} 
    \fmflabel{$A,p_{1}$}{i1}
    \fmflabel{$B,p_{3}$}{o1} %
    \fmflabel{$A,p_{2}$}{i2} %
    \fmflabel{$B,p_{3}$}{o2}
    \fmf{plain,fore=red}{i1,v1} %
    \fmf{plain,fore=blue}{v1,o1}
    \fmf{plain}{i2,v4,o2} %
    \fmf{plain,label=$C,,q_1$,fore=blue}{v1,v2}
    \fmf{plain,left,label=$B,,q_3$,tension=0.5}{v2,v3}
    \fmf{plain,left,label=$A,,q_2$,tension=0.5}{v3,v2}
    \fmf{plain,label=$C,,q_4$}{v3,v4}
  \end{fmfgraph*}
\end{fmffile}
\end{center}
\strut

So we want to have momentum here conserved, i.e.
\begin{equation}
p_1 = p_3 + q_1
\end{equation}
so we multiply the integrand by the term
\begin{equation*}
(2\pi)^{4}\delta^{(4)}(p_{1}-p_{3}-q_{1}).
\end{equation*}
Our integrand, which is ever expanding, is then
\begin{equation}
g^4\frac{1}{q_{1}^2 - m_{C}^2c^2}\frac{1}{q_{2}^2 - m_{A}^2c^2}\frac{1}{q_{3}^2 - m_{B}^2c^2}\frac{1}{q_{4}^2-m_{C}^2c^2}(2\pi)^{4}\delta^{(4)}(p_{1}-p_{3}-q_{1}).
\end{equation}
There are three other vertices which we must impose conservation laws on, so we
will move right along to the next vertex; again the input momentum is in red,
and the output momentum is in blue:


\strut
\begin{center}
\begin{fmffile}{exTwoImg9}
  \begin{fmfgraph*}(70,25)  \fmfpen{0.1mm}

    \fmfleft{i1,o1} 
    \fmfright{i2,o2} 
    \fmflabel{$A,p_{1}$}{i1}
    \fmflabel{$B,p_{3}$}{o1} %
    \fmflabel{$A,p_{2}$}{i2} %
    \fmflabel{$B,p_{3}$}{o2}
    \fmf{plain}{i1,v1} %
    \fmf{plain}{v1,o1}
    \fmf{plain}{i2,v4,o2} %
    \fmf{plain,label=$C,,q_1$,fore=red}{v1,v2}
    \fmf{plain,left,label=$B,,q_3$,tension=0.5,fore=blue}{v2,v3}
    \fmf{plain,left,label=$A,,q_2$,tension=0.5,fore=blue}{v3,v2}
    \fmf{plain,label=$C,,q_4$}{v3,v4}
  \end{fmfgraph*}
\end{fmffile}
\end{center}
\strut

This corresponds to a conservation of momentum of 
\begin{equation}
q_1 \approx q_2 + q_3
\end{equation}
which corresponds to the dirac delta function term of
\begin{equation*}
(2\pi)^4\delta^{(4)}(q_1 - q_2 - q_3).
\end{equation*}
Multiplying this into our integrand, we get
\begin{equation}
(2\pi)^8g^4\frac{1}{q_{1}^2 - m_{C}^2c^2}\frac{1}{q_{2}^2 - m_{A}^2c^2}\frac{1}{q_{3}^2 - m_{B}^2c^2}\frac{1}{q_{4}^2-m_{C}^2c^2}\delta^{(4)}(p_{1}-p_{3}-q_{1})\delta^{(4)}(q_1 - q_2 - q_3).
\end{equation}
Two vertices down, two to go! We simply move right along to find the next
conservation of momentum to be at the next vertex. The input momentums are in
red, and the output momentum is in blue:


\strut
\begin{center}
\begin{fmffile}{exTwoImg10}
  \begin{fmfgraph*}(70,25)  \fmfpen{0.1mm}

    \fmfleft{i1,o1} 
    \fmfright{i2,o2} 
    \fmflabel{$A,p_{1}$}{i1}
    \fmflabel{$B,p_{3}$}{o1} %
    \fmflabel{$A,p_{2}$}{i2} %
    \fmflabel{$B,p_{3}$}{o2}
    \fmf{plain}{i1,v1} %
    \fmf{plain}{v1,o1}
    \fmf{plain}{i2,v4,o2} %
    \fmf{plain,label=$C,,q_1$}{v1,v2}
    \fmf{plain,left,label=$B,,q_3$,tension=0.5,fore=red}{v2,v3}
    \fmf{plain,left,label=$A,,q_2$,tension=0.5,fore=red}{v3,v2}
    \fmf{plain,label=$C,,q_4$,fore=blue}{v3,v4}
  \end{fmfgraph*}
\end{fmffile}
\end{center}
\strut

This corresponds to the conservation
\begin{equation}
q_2 + q_3 \approx q_4
\end{equation}
which means we have a delta function of the form
\begin{equation*}
(2\pi)^4\delta^{(4)}(q_2 + q_3 - q_4).
\end{equation*}
Our integrand becomes
\begin{equation}
(2\pi)^{12}g^4\frac{1}{q_{1}^2 - m_{C}^2c^2}\frac{1}{q_{2}^2 - m_{A}^2c^2}\frac{1}{q_{3}^2 - m_{B}^2c^2}\frac{1}{q_{4}^2-m_{C}^2c^2}\delta^{(4)}(p_{1}-p_{3}-q_{1})\delta^{(4)}(q_1 - q_2 - q_3)\delta^{(4)}(q_2 + q_3 - q_4).
\end{equation}
\snote{As we can see, this is getting really really messy! God help us when
we try to feebly evaluate this beast!} Thank god only one vertex left! This
is the last term to add prior to integration. The input momentums are in
red and the output momentum is in blue:


\strut
\begin{center}
\begin{fmffile}{exTwoImg10}
  \begin{fmfgraph*}(70,25)  \fmfpen{0.1mm}

    \fmfleft{i1,o1} 
    \fmfright{i2,o2} 
    \fmflabel{$A,p_{1}$}{i1}
    \fmflabel{$B,p_{3}$}{o1} %
    \fmflabel{$A,p_{2}$}{i2} %
    \fmflabel{$B,p_{3}$}{o2}
    \fmf{plain}{i1,v1} %
    \fmf{plain}{v1,o1}
    \fmf{plain,fore=red}{i2,v4} %
    \fmf{plain,fore=blue}{v4,o2}
    \fmf{plain,label=$C,,q_1$}{v1,v2}
    \fmf{plain,left,label=$B,,q_3$,tension=0.5}{v2,v3}
    \fmf{plain,left,label=$A,,q_2$,tension=0.5}{v3,v2}
    \fmf{plain,label=$C,,q_4$,fore=red}{v3,v4}
  \end{fmfgraph*}
\end{fmffile}
\end{center}
\strut


This has the conservation of
\begin{equation}
q_4 + p_2\approx p_4
\end{equation}
which takes the delta form of
\begin{equation}
(2\pi)^4\delta^{(4)}(q_4 + p_2 - p_4)
\end{equation}
and our integrand finally becomes
\begin{equation}
(2\pi)^{16}g^4\frac{\delta^{(4)}(p_{1}-p_{3}-q_{1})}{q_{1}^2 - m_{C}^2c^2}\frac{\delta^{(4)}(q_1 - q_2 - q_3)}{q_{2}^2 - m_{A}^2c^2}\frac{\delta^{(4)}(q_2 + q_3 - q_4)}{q_{3}^2 - m_{B}^2c^2}\frac{\delta^{(4)}(q_4 + p_2 - p_4)}{q_{4}^2-m_{C}^2c^2} 
\end{equation}
At this point, if I were a professor, I would say ``This is trivial...bye!'' But
I am no professor!

\textbf{Step Five:} We then integrate over the internal lines. This is the fun 
part, like when the dentist says he needs to give you a root canal and he's
all outta Novocaine! We integrate over $q_1$, $q_2$, $q_3$, $q_4$. We notice
that the factors of $2\pi$ completely cancel out which is nice, so all we have
is
\begin{equation}
g^4\int\frac{\delta^{(4)}(p_{1}-p_{3}-q_{1})}{q_{1}^2 - m_{C}^2c^2}\frac{\delta^{(4)}(q_1 - q_2 - q_3)}{q_{2}^2 - m_{A}^2c^2}\frac{\delta^{(4)}(q_2 + q_3 - q_4)}{q_{3}^2 - m_{B}^2c^2}\frac{\delta^{(4)}(q_4 + p_2 - p_4)}{q_{4}^2-m_{C}^2c^2}d^{4}q_{1}d^{4}q_{2}d^{4}q_{3}d^{4}q_{4}.
\end{equation}
We will take this slow and step by step, we see that in the first delta function
we have the replacement of $q_1$ by $p_1-p_3$ which is nice! We make this move:
\begin{equation}
\frac{g^4}{(p_1-p_3)^2 - m_{C}^2c^2}\int\frac{\delta^{(4)}(p_1-p_3 - q_2 - q_3)}{q_{2}^2 - m_{A}^2c^2}\frac{\delta^{(4)}(q_2 + q_3 - q_4)}{q_{3}^2 - m_{B}^2c^2}\frac{\delta^{(4)}(q_4 + p_2 - p_4)}{q_{4}^2-m_{C}^2c^2}d^{4}q_{2}d^{4}q_{3}d^{4}q_{4}.
\end{equation}
Similarly, we find the our last delta function allows us to make the switcheroo
of $q_4$ for $p_2 - p_4$, so we make it so:
\begin{equation}
\frac{g^4}{(p_1-p_3)^2 - m_{C}^2c^2}\frac{1}{(p_2 - p_4)^2-m_{C}^2c^2}\int\frac{\delta^{(4)}(p_1-p_3 - q_2 - q_3)}{q_{2}^2 - m_{A}^2c^2}\frac{\delta^{(4)}(q_2 + q_3 - p_2 - p_4)}{q_{3}^2 - m_{B}^2c^2}d^{4}q_{2}d^{4}q_{3}.
\end{equation}
The term $\delta^{(4)}(p_1-p_3 - q_2 - q_3)$ tells us $q_2$ is replaced by
$p_1 - p_3 - q_3$ so our last delta function (after monkeying around with
integration) becomes
\begin{equation*}
\delta^{(4)}(p_1 + p_2 - p_3 - p_4).
\end{equation*}
We are left with
\begin{eqnarray}
\frac{g^4}{(p_1-p_3)^2 - m_{C}^2c^2}\frac{1}{(p_2 - p_4)^2-m_{C}^2c^2} \nonumber \\
\quad\times\int
\frac{1}{(p_1 - p_3 - q_3)^2 - m_{A}^2c^2}
\frac{\delta^{(4)}(p_1 - p_3 + p_2 - p_4)}{q_{3}^2 - m_{B}^2c^2}d^{4}q_{3}.
\end{eqnarray}
So we can skip ahead to rule 6 and assert: our contribution to the probability
amplitude from this diagram is
\begin{eqnarray}\label{divergentSonOfAGun}
\mathcal{M} &=& i\left(\frac{g}{2\pi}\right)^4\frac{1}{[(p_1-p_3)^2 - m_{C}^2c^2][(p_2 - p_4)^2-m_{C}^2c^2]} \nonumber\\
&&\times \int\frac{1}{[(p_1 - p_3 - q_3)^2 - m_{A}^2c^2][q_{3}^2 - m_{B}^2c^2]}d^{4}q_{3}.
\end{eqnarray}
