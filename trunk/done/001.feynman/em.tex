\section{Maxwell's Equations in a Nutshell}

Recall in classical electromagnetism we have it summed in Maxwell's equations~\cite{jackson}. In the presence of a charge density $\rho(\vec{x},t)$ and a 
current density $\vec{j}(\vec{x},t)$, the electric and magnetic fields $\vec{E}$
and $\vec{B}$ satisfy the equations
\begin{subequations}\label{maxwellsEquations}
\begin{align}
\nabla\cdot\vec{E}&=\rho\label{gaussLaw}\\
\nabla\times\vec{B}&=\frac{1}{c}\vec{j} + \frac{1}{c}\frac{\partial\vec{E}}{\partial t}\label{AmpereLaw}\\
\nabla\cdot\vec{B} &= 0\label{GaussLawMagnet}\\
\nabla\times\vec{E} &= -\frac{1}{c}\frac{\partial\vec{B}}{\partial t} \label{FaradayLaw}
\end{align}
\end{subequations}
where cgs units are used.

In the second pair of equations (Eqs \ref{GaussLawMagnet} and \ref{FaradayLaw})
follows the existence of scalar and vector potentials $\phi(\vec{x},t)$ and
$\vec{A}(\vec{x},t)$ defined by
\begin{equation}
\vec{B} = \nabla\times\vec{A},\quad\vec{E}=-\nabla\phi - \frac{1}{c}\frac{\partial\vec{A}}{\partial t}.
\end{equation}
However, this does not determine the system uniquely, since for an \emph{arbitrary}
function $f(\vec{x},t)$ the transformation
\begin{equation}\label{emGaugeTransformation}
\phi\to\phi'=\phi + \frac{1}{c}\frac{\partial f}{\partial t},\quad\vec{A}\to\vec{A}' = \vec{A} - \nabla f
\end{equation}
leaves the fields $\vec{E}$ and $\vec{B}$ unaltered. The transformation (\ref{emGaugeTransformation})
is known as a gauge transformation of the second kind\footnote{I.e. it is described
mathematically in differential geometry as a connection form.}. Since all 
observable quantities can be expressed in terms of 
$\vec{E}$ and $\vec{B}$, it is a fundamental requirement of any theory
formulated in terms of potentials that is gauge; i.e. the predictions for the 
observable quantities are invariant under such gauge transformations.

When we express Maxwell's equations in terms of potentials, the second pair
are automatically satisfied. The first pair (\ref{gaussLaw} and \ref{AmpereLaw}) become
\begin{subequations}
\begin{align}
-\nabla^2\phi - \frac{1}{c}\frac{\partial}{\partial t}(\nabla\cdot\vec{A}) = \Box\phi - \frac{1}{c}\frac{\partial}{\partial t}\left(\frac{1}{c}\frac{\partial\phi}{\partial t} + \nabla\cdot\vec{A}\right) = \rho\\
\Box\vec{A} + \nabla\left(\frac{1}{c}\frac{\partial\phi}{\partial t} + \nabla\cdot\vec{A}\right) = \frac{1}{c}\vec{j}
\end{align}
\end{subequations}
where
\begin{equation}
\Box \equiv \frac{1}{c^2}\frac{\partial^2}{\partial t^2} - \nabla^2
\end{equation}
is called the ``D'Alembertian''.

We can now consider the so-called ``free field'' case. That is, we have no
charge or current so $\rho=0$ and $\vec{j}=0$. We can choose a gauge for the
system such that
\begin{equation}\label{radiationGauge}
\nabla\cdot\vec{A} = 0.
\end{equation}
The condition (\ref{radiationGauge}) defines the \textbf{Coulomb or radiation gauge}. A vector field with vanishing divergence (ie satisfying Eq (\ref{radiationGauge})) is called a ``transverse field'' since for a wave
\begin{equation}
\vec{A}(\vec{x},t) = \vec{A}_0 \exp(i(\vec{k}\cdot\vec{x}-\omega t))
\end{equation}
gives
\begin{equation}
\vec{k}\cdot\vec{A} = 0,
\end{equation}
or in other words $\vec{A}$ is perpendicular to the direction of propagation 
$\vec{k}$ of the wave. In the Coulomb gauge, the vector potential is a transverse
vector.
