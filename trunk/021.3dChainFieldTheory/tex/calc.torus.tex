%%
%% calc.torus.tex
%% 
%% Made by Alex Nelson
%% Login   <alex@tomato>
%% 
%% Started on  Thu Sep  3 13:26:37 2009 Alex Nelson
%% Last update Thu Sep  3 13:26:37 2009 Alex Nelson
%%

We will consider the time evolution operator for the 3-torus. We
see that the complex for spacetime is
\begin{equation}%\label{eq:}
\begin{CD}
\{0\}         @<<<     \{0\}     @<<<     \{0\}\\
@VVV                   @VVV                @VVV\\
\mathbb{Z}    @<0<<     \mathbb{Z}^{3}     @<0<<     \mathbb{Z}^{3} @<0<< \mathbb{Z}\\
@VVV                   @VVV                @VVV\\
\{0\}         @<<<     \{0\}     @<<<     \{0\}
\end{CD}
\end{equation}
This means the corresponding initial and final states are
described by the Hilbert space $\mathbb{C}$ and $\mathbb{C}$
(respectively). The time evolution operator for the torus is thus
\begin{equation}%\label{eq:}
Z(T):\mathbb{C}\to\mathbb{C}.
\end{equation}
We pick out the basis state vectors $1\in\mathbb{C}$ for the
initial and final state and evaluate the expression
\begin{equation}%\label{eq:}
\<1,Z(T)\cdot1\> = \int_{\mathcal{A}(T)}e^{-S(A)}\mathcal{D}A
\end{equation}
where $\mathcal{A}(T)$ is the space of connections on the torus.

We can calculate the action expression to be
\begin{equation}%\label{eq:}
e^{-S(A)} = \sum_{n\in\mathbb{Z}}\exp\left(\frac{-1}{e^{2}V}[2n\pi]^{2}\right).
\end{equation}
We find that our integral expression simplifies when we see that
the curvature (field strength quantity) $F=0$
\begin{equation}%\label{eq:}
e^{-S(A)} = \sum_{n\in\mathbb{Z}}e^{-n^{2}e^{2}V/2}e^{in(0)}
\end{equation}
thus
\begin{equation}%\label{eq:}
\<1,Z(T)\cdot 1\> = \iiint
\sum_{n\in\mathbb{Z}}e^{-n^{2}e^{2}V/2}\mathcal{D}A = \sum_{n\in\mathbb{Z}}e^{-n^{2}e^{2}V/2}
\end{equation}
is our expression. We deduce that
\begin{equation}%\label{eq:}
Z(T):1\to \sum_{n\in\mathbb{Z}}e^{-n^{2}e^{2}V/2}
\end{equation}
is precisely the time evolution operator.
