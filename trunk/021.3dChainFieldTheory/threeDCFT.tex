%%
%% threeDCFT.tex
%% 
%% Made by Alex Nelson
%% Login   <alex@tomato>
%% 
%% Started on  Sat Aug 29 12:38:56 2009 Alex Nelson
%% Last update Sat Aug 29 12:38:56 2009 Alex Nelson
%%
\documentclass[10pt,oneside]{article}
\usepackage{fly}
\usepackage{brackets}
\usepackage{danger}
\usepackage{float}
%\def\comment#1{}
\title{Calculations in Three Dimensional Chain Field Theory}
\date{August 29, 2009}
%\author{Alex Nelson\\{\tt Email: \href{mailto:pqnelson@gmail.com}{pqnelson@gmail.com}}}
\begin{document}
\maketitle

\section{Notational Warning!}

We will use slightly odd notation for oriented $p+1$ cells
$X_{p+1}$. We will denote it and its orientation in one fell
swoop by
\begin{equation}%\label{eq:}
X_{p+1} = [X_{p} \quad X^{\prime}_{p}]
\end{equation}
as being oriented \emph{from} $X_{p}$ directed \emph{towards} $X^{\prime}_{p}$.
Also we'll use the notation
\begin{equation}%\label{eq:}
e^{-1}_{n}=-e_{n}
\end{equation}
i.e. inverses as such for edges, vertices, faces, $p$-cells, are
actually additive inverses. We can easily deduce how the boundary
operator behaves, since the orientation and boundaries are clearly noted.

\section{The 3-Torus}
%%
%% torus.tex
%% 
%% Made by Alex Nelson
%% Login   <alex@tomato>
%% 
%% Started on  Sat Aug 29 12:56:09 2009 Alex Nelson
%% Last update Sat Aug 29 12:56:09 2009 Alex Nelson
%%
\begin{figure}[ht]
\begin{center}
\includegraphics{img/img.0}
\end{center}
\caption{Identifying the top of the cube with the bottom, we are
  then left to identify one pair of sides of the cube as the
  same, and the remaining pair of sides as the same.}\label{fig:img0}
\end{figure}

Consider the Torus. We will construct it by considering a cube,
which has 6 faces, 12 edges, and 8 vertices. We will identify
opposite pairs of faces as being ``the same'' (glued
together). We will consider what this looks like as far as the
vertices, edges, and faces are concerned.

If we start by examining the vertices, we will cheat and identify
the top face with the bottom face. We are left with 4 distinct
vertices, as seen in figure \ref{fig:img0}. We then identify one
pair of opposite faces as the same, then the other pair of
opposite faces as the same. This is precisely what we do in
figure \ref{fig:img0}. We see that there is thus only one vertex
for the 3-torus.

Now, with regard to the edges, this is a bit trickier. Lets begin
with something easier: faces. We identify opposite pairs of faces
as being ``the same''. There are 6 faces, thus 3 such pairs. We
have 3 faces in the 3-torus (if one is unsatisfied with this
quick construction, one can pull out a regular six sided die and
observe that the each opposite face sums to 7; there are 3
distinct ways to add positive integers up to be 7, precisely 1+6,
2+5, and 3+4).

We have 3 faces, 1 vertex, and now the hard part: edges. We see
in figure \ref{fig:img1} that the distinct faces are colorized,
with the duplicate faces removed. We only, therefore, have to
worry about the edges which are drawn in black. This means that
there are 3 distinct edges. The others are duplicates.

\begin{figure}[ht]
\begin{center}
\includegraphics{img/img.1}
\end{center}
\caption{The distinct faces for the 3-torus are colorized, with the duplicate faces removed.}\label{fig:img1}
\end{figure}

We can now start considering the chain complex for the
3-torus. We see that there are 1 vertex, 3 edges, 3 faces, and 1
3-cell. This implies the chain complex should look like
\begin{equation}%\label{eq:}
\mathbb{Z}\xleftarrow{\partial_{0}}\mathbb{Z}^{3}\xleftarrow{\partial_{1}}\mathbb{Z}^{3}\xleftarrow{\partial_{2}}\mathbb{Z}.
\end{equation}
We would like to consider what exactly the boundary maps
$\partial$ are as matrices. We need more information, we need to
pick some orientations for the faces, edges, and the
3-cell. %We would also like to calculate that
% $\partial^{2}=0$ to make certain we're on the right track!
We will use the orientation indicated in figure \ref{fig:img1}. 

We will consider the operator
$\partial_{0}:\mathbb{Z}^{3}\to\mathbb{Z}$. Since there is only
one vertex, that means all edges have \emph{the same} source and
target. This means that the operator $\partial_{0}(e)=t(e)-s(e)$
would be zero, since $t(e)=s(e)$.

If we consider $\partial_{1}:\mathbb{Z}^{3}\to\mathbb{Z}^{3}$
which takes faces to edges, we have something more interesting.
If we let
\begin{equation}%\label{eq:}
af_{1}+bf_{2}+cf_{3} = \begin{bmatrix}a\\b\\c\end{bmatrix},\quad\text{and}\quad
ae_{1}+be_{2}+ce_{3} = \begin{bmatrix}a\\b\\c\end{bmatrix}
\end{equation}
(where $a,b,c\in\mathbb{Z}$), we can consider the $\partial_{1}$
boundary operator as a square matrix. Consider face $f_{1}$ as
drawn in figure \ref{fig:img2}. Since there's only one distinct
vertex, we see travelling along the path in red gets us back to
where we started. This means that
\begin{equation}%\label{eq:}
\partial_{1}\left(\begin{bmatrix}1\\0\\0\end{bmatrix}\right) = \begin{bmatrix}1\\1\\0\end{bmatrix}
\end{equation}
since both $e_{1}$, $e_{2}$ are moving ``along'' the orientation
chosen for $f_{1}$. Referring to figure \ref{fig:img1}, we see
that the other faces are similar. This allows us to explicitly compute
\begin{equation}%\label{eq:}
\partial_{1}\left(\begin{bmatrix}0\\1\\0\end{bmatrix}\right) = \begin{bmatrix}1\\0\\1\end{bmatrix}\quad\text{and}\quad
\partial_{1}\left(\begin{bmatrix}0\\0\\1\end{bmatrix}\right) = \begin{bmatrix}0\\1\\1\end{bmatrix}.
\end{equation}
But this \emph{completely determines} what $\partial_{1}$ must
be, explicitly as a linear operator it is ``merely''
\begin{equation}%\label{eq:}
\partial_{1} = \begin{bmatrix}1 & 1 & 0\\
1 & 0 & 1\\
0 & 1 & 1\end{bmatrix}
\end{equation}
which is a curious operator in its own right. It's also a good
hint that our value for $\partial_{0}$ is correct, since
$\det(\partial_{1})=-2\neq0$ implies $\partial_{1}$ has a trivial
kernel. The only way $\partial_{0}\circ\partial_{1}=0$ is iff
$\partial_{0}=[0,0,0]$.

\begin{figure}[t]
\begin{center}
\includegraphics{img/img.2}
\end{center}
\caption{The face $f_{1}$ of the 3-torus, boundary edges $e_{1}$
  and $e_{2}$ are drawn in. The path we use for the chain complex
  is drawn in red.}\label{fig:img2}
\end{figure}

We now are stuck with our last operator $\partial_{2}$. We can
cheat, and use the same argument for $\partial_{0}=[0,0,0]$
to deduce that
\begin{equation}%\label{eq:}
\partial_{2} = \begin{bmatrix}0\\0\\0\end{bmatrix}.
\end{equation}
I have a feeling the preferred route would be to prove that it
must be this via picking a preferred orientation, etc.

\section{The 3-Sphere}
%%
%% sphere.tex
%% 
%% Made by Alex Nelson
%% Login   <alex@tomato>
%% 
%% Started on  Sat Aug 29 16:03:43 2009 Alex Nelson
%% Last update Sat Aug 29 16:03:43 2009 Alex Nelson
%%

Observe the inductive procedure to constructing an $n$-sphere. We
start with $B^{0}$, the 0-ball, which is just a humble point. We
have $S^{0}$, the 0-sphere, which consists of 2 points. The
1-ball is a line segment between 2 points. Now here is the first
inductive case study: the 1-sphere is a circle. It consists of 2
$B^{1}$ with their boundaries glued together: 
\begin{figure}[h]
\begin{center}
\includegraphics{img/img.3}
\end{center}
\end{figure}
The region whose boundary is $S^{1}$ is precisely $B^{2}$. We
should consider the chain complex describing $S^{1}$ before
moving on. We have already drawn in the orientation of the edges
(and labeled them too!), so we need to pick an orientation of the
face itself. We see first of all that there are 2 edges and 2
vertices. This allows us to set up the chain complex as
\begin{equation}%\label{eq:}
\mathbb{Z}^{2}\xleftarrow{\partial_{0}}\mathbb{Z}^{2}
\end{equation}
We have picked an orientation for the edges. We see that
\begin{equation}%\label{eq:}
ae_{1}+be_{2}=\begin{bmatrix}a\\b\end{bmatrix}\quad\text{and}\quad 
av_{1}+bv_{2}=\begin{bmatrix}a\\b\end{bmatrix}
\end{equation}
We can describe the boundary operator by
\begin{equation}%\label{eq:}
\partial_{0}\left(\begin{bmatrix}1\\0\end{bmatrix}\right)=\begin{bmatrix}-1\\1\end{bmatrix}\quad\text{and}\quad
\partial_{0}\left(\begin{bmatrix}0\\1\end{bmatrix}\right)=\begin{bmatrix}1\\-1\end{bmatrix}
\end{equation}
which means that
\begin{equation}%\label{eq:}
\partial_{0}:\begin{bmatrix}a\\b\end{bmatrix}\mapsto\begin{bmatrix}b-a\\a-b\end{bmatrix}
\end{equation}
describes the behavior of the boundary operator.

We can now similarly construct $S^{2}$. We have 2 $B^{2}$ and glue their
boundaries together, or diagrammaticaly:
\begin{figure}[H]
\includegraphics{img/img.4}
\end{figure}
\noindent We should consider now the chain complex that describes
$S^{3}$. Observe that the two vertices on each of the boundaries
of the $B^{2}$ are identified to be the same. We are thus left
with two vertices for $S^{3}$ (as doodled on the right hand side
of our figure). The two edges for the boundary of $B^{2}$ are
likewise identified to be the same. We have, however, two
\emph{distinct faces} for $S^{3}$ which corresponds to the two
distinct $B^{2}$. This produces the following chain
\begin{equation}%\label{eq:}
\mathbb{Z}^{2}\xleftarrow{\partial_{0}}\mathbb{Z}^{2}\xleftarrow{\partial_{1}}\mathbb{Z}^{2}
\end{equation}
now we are left to deduce the boundary operators. This demands a
choice of orientation of edges, and faces. We have picked an
orientation for the edges as drawn in our figure, let $e_{1}$
begin at $v_{1}$ and end at $v_{2}$; $e_{2}$ begin at $v_{2}$ and
end at $v_{1}$. We then let
\begin{equation}%\label{eq:}
ae_{1}+be_{2}=\begin{bmatrix}a\\b\end{bmatrix}\quad\text{and}\quad 
av_{1}+bv_{2}=\begin{bmatrix}a\\b\end{bmatrix}
\end{equation}
and then deduce that
\begin{equation}%\label{eq:}
\partial_{0}\left(\begin{bmatrix}1\\0\end{bmatrix}\right)=\begin{bmatrix}-1\\1\end{bmatrix}\quad\text{and}\quad
\partial_{0}\left(\begin{bmatrix}0\\1\end{bmatrix}\right)=\begin{bmatrix}-1\\1\end{bmatrix}
\end{equation}
which implies that
\begin{equation}%\label{eq:}
\partial_{0}:\begin{bmatrix}a\\b\end{bmatrix}\mapsto\begin{bmatrix}-a-b\\a+b\end{bmatrix}
\end{equation}
determines one of the boundary operators completely. To determine
the other, we can cheat and figure out what square matrix $A$ is such
that
\begin{equation}%\label{eq:}
\begin{bmatrix} -1 & -1\\1 & 1\end{bmatrix}A = 0
\end{equation}
so the desired relation $\partial^{2}=0$ holds. This would
determine $A$ up to an arbitrary sign (which is determined by the
orientation of the faces). We see that
\begin{equation}%\label{eq:}
A = \begin{bmatrix}a & b\\
-a & -b\end{bmatrix}
\end{equation}
is the structure of $\partial_{1}$. This in fact implies
$(\partial_{0})^{2}=0$. We are forced to pick orientations for
the actual open balls $B^{2}$, but open doing so we see that it's
simply one $B^{2}$ minus the other, implying that (up to a sign
determined by orientation!) $a=b=\pm1$.

The inductive ritual should be clear: we take two copies of
$B^{n}$ and glue the boundaries, the $S^{n-1}$, together to
create a $S^{n}$. More formally, we present it as a theorem.

\begin{thm}%\label{thm:}
Consider the sphere $S^{n}$. Then the chain complex describing it
is
\begin{equation}%\label{eq:}
\mathbb{Z}^{2} \xleftarrow{\partial_{0}}\mathbb{Z}^{2}\xleftarrow{\partial_{1}}\cdots\xleftarrow{\partial_{n-1}}\mathbb{Z}^{2}
\end{equation}
where $\partial_{i}$ are the boundary operators.
\end{thm}
\begin{proof}
\noindent\textbf{Base Case (n=1):} We have already done this, two $B^{1}$
(line segments) glued together at their boundary $S^{0}$ (two
points) results in the chain complex
\begin{equation}%\label{eq:}
\mathbb{Z}^{2}\xleftarrow{\partial_{0}}\mathbb{Z}^{2}
\end{equation}
where
\begin{equation}%\label{eq:}
\partial_{0} = \begin{bmatrix}-1 & -1\\1 & 1\end{bmatrix}
\end{equation}
up to a sign determined by orientation.

\noindent\textbf{Inductive Hypothesis:} assume this works for arbitrary
$n$. That is, we have
\begin{equation}%\label{eq:}
\mathbb{Z}^{2}\xleftarrow{\partial_{0}}\cdots\xleftarrow{\partial_{n-1}}\mathbb{Z}^{2}
\end{equation}
for our chain complex.

\noindent\textbf{Inductive Case (n+1):} We have the boundaries of the
$B^{n+1}$, which takes care of the first $n-1$ boundary operators
and the first $n$ $\mathbb{Z}^{2}$ components of the chain. We
are still left with two copies of the \emph{interior} of
$B^{n+1}$. This constitutes an additional $\mathbb{Z}^{2}$
component in the chain diagram, and implies that the chain
complex looks like
\begin{equation}%\label{eq:}
\underbracket[0.25pt]{\left(\mathbb{Z}^{2}\xleftarrow{\partial_{0}}\cdots\xleftarrow{\partial_{n-1}}\mathbb{Z}^{2}\right)}_{\text{inductive hypothesis}}\xleftarrow{\partial_{n}}\mathbb{Z}^{2}
\end{equation}
where $\partial$ is the boundary operator. This concludes the
inductive case, and the proof.
\end{proof}
For our intentions however, we need to consider
two $B^{3}$, and glue their boundaries together!


\section{Chain Field Theoretic Calculations}
%%
%% calc.tex
%% 
%% Made by Alex Nelson
%% Login   <alex@black-cherry>
%% 
%% Started on  Wed Jul  4 12:55:27 2012 Alex Nelson
%% Last update Wed Jul  4 12:55:43 2012 Alex Nelson
%%

\section{Alternating Series Tests}
%%
%% altSeriesTest.tex
%% 
%% Made by Alex Nelson
%% Login   <alex@black-cherry>
%% 
%% Started on  Sat Jun 16 17:09:03 2012 Alex Nelson
%% Last update Sun Jun 17 14:45:03 2012 Alex Nelson
%%
%\documentclass{article}
%\usepackage{blog}
%\begin{document}
\N{Proposition} Let
\begin{equation}
\sum^{\infty}_{n=1}(-1)^{n+1}a_{n} =
a_{1}-a_{2}+a_{3}-a_{4}+\dots
\end{equation}
be a given series where $a_{n}>0$. Then the series converges if
\begin{enumerate}
\item $a_{n}\geq a_{n+1}$ for each $n$;
\item $\displaystyle\lim_{n\to\infty}a_{n}=0$.
\end{enumerate}
\begin{proof}
We have in the sequence of partial sums
\begin{equation}
S_{2n}=\underbrace{(a_{1}-a_{2})}_{\geq0}
+\underbrace{(a_{3}-a_{4})}_{\geq0}+\dots+\underbrace{(a_{2n-1}-a_{2n})}_{\geq0}\geq0
\end{equation}
so
\begin{equation}
0\leq S_{2}\leq S_{4}\leq \dots\leq S_{2n}\leq\dots
\end{equation}
But we also have
\begin{equation}
\begin{aligned}
S_{2n} &= a_{1} - (a_{2}-a_{3})-(a_{4}-a_{5})-(\dots)-a_{2n}\\
&\leq a_{1}.
\end{aligned}
\end{equation}
So
\begin{equation}
\lim_{n\to\infty}S_{2n}=L\leq a_{1}
\end{equation}
and
\begin{equation}
\begin{aligned}
\lim_{n\to\infty}S_{2n+1} &=\lim_{n\to\infty}(S_{2n}+a_{2n+1})\\
&=\left(\lim_{n\to\infty}S_{2n}\right)+\left(\lim_{n\to\infty}a_{2n+1}\right)\\
&=L+0=L.
\end{aligned}
\end{equation}
Conclusion: $\displaystyle\sum^{\infty}_{n=1}(-1)^{n+1}a_{n}=L$.
\end{proof}
\begin{example}
Consider the series
\begin{equation}
\sum^{\infty}_{n=1}\frac{(-1)^{n+1}}{n}
\end{equation}
We see that $1/n\geq1/(n+1)$ for each $n$, and 
\begin{equation}
\lim_{n\to\infty}\frac{1}{n}=0.
\end{equation}
Thus the Alternating Series Test implies the alternating Harmonic
series converges.
\end{example}
\begin{example}
Consider the series
\begin{equation}
\sum^{\infty}_{n=1}\frac{(-1)^{n+1}n}{(n+1)(n+2)}
\end{equation}
We see
\begin{equation}
\frac{n}{(n+1)(n+2)}\geq\frac{n+1}{(n+2)(n+3)};
\end{equation}
why? Well, multiply both sides by $(n+2)$ and we get
\begin{equation}
\frac{n}{n+1}\geq \frac{n+1}{n+3}
\end{equation}
Cross multiplication gives us
\begin{equation}
n(n+3)\geq (n+1)^{2} \iff n^{2}+3n \geq n^{2}+2n+1.
\end{equation}
Subtracting $n^{2}+2n$ from both sides gives us
\begin{equation}
n\geq1.
\end{equation}
So our series satisfies the first condition for the alternating
series test.

We also see that
\begin{equation}
\frac{n}{(n+1)(n+2)}\approx\frac{1}{n}
\end{equation}
for ``large $n$''. So we see 
\begin{equation}
\lim_{n\to\infty}\frac{n}{(n+1)(n+2)}=0.
\end{equation}
Thus our series satisfies the criteria for the alternating series
test, which implies convergence.
\end{example}

\N{Definitions} A series $\sum a_{n}$ is \textbf{``Absolutely Convergent''}
if $\sum|a_{n}|$ converges.

On the other hand, if $\sum|a_{n}|$ is divergent, then we call
the series $\sum a_{n}$ \textbf{``Conditionally Convergent''}.

\begin{example}
We see that $\sum(-1)^{n}/n$ is conditionally convergent, since
$\sum 1/n$ diverges.
\end{example}
\begin{example}
The series $\sum (-1)^{n}n^{-3/2}$ is absolutely convergent since
the integral test tells us $\sum n^{-3/2}$ converges.
\end{example}

\medbreak\noindent\emph{Question:} let $\sum a_{n}$ be a
convergent series, and $a_{n}\geq0$ for each $n$. Does the series
$\sum (-1)^{n}a_{n}$ converge? 

Stop and think before continuing!

\N{Absolute Convergence Test} 
If $\sum |a_{n}|$ conveges, then $\sum a_{n}$ converges.

\begin{proof}
We see first that 
\begin{equation}
-|a_{n}|\leq a_{n}\leq|a_{n}|\quad\mbox{for each }n
\end{equation}
So what? Well, we see that
\begin{equation}
0\leq a_{n}+|a_{n}|\leq2|a_{n}|\quad\mbox{for each }n
\end{equation}
Since $\sum|a_{n}|$ converges, we see $\sum2|a_{n}|$ converges
too. But by the comparison test, we see
\begin{equation}
\sum^{\infty}_{n=1}a_{n}+|a_{n}|
\end{equation}
converges. So what? We haven't proven $\sum a_{n}$ converges,
have we? Consider the following trick
\begin{equation}
\sum a_{n} = \underbrace{\sum
  a_{n}+|a_{n}|}_{\text{converges}}-\underbrace{\sum
  |a_{n}|}_{\text{converges}}.
\end{equation}
Therefore the series $\sum a_{n}$ converges.
\end{proof}
\begin{example}
Consider the series
\begin{equation}\label{eq:altSeriesTest:ex4:eq1}
\sum^{\infty}_{n=1}\frac{\cos(n)}{n^{3/2}}.
\end{equation}
What to do? We know
\begin{equation}
\left|\frac{\cos(n)}{n^{3/2}}\right|\leq\frac{1}{n^{3/2}}.
\end{equation}
So what? We know $\sum n^{-3/2}$ converges by the integral
test. Then
\begin{equation}
\sum^{\infty}_{n=1}\left|\frac{\cos(n)}{n^{3/2}}\right|
\end{equation}
converges by comparison. By the absolute convergence test, we know
our series in Equation \eqref{eq:altSeriesTest:ex4:eq1} converges.
\end{example}


%\end{document}



\section{Other Alternating Series Tests}
%%
%% altRatioTest.tex
%% 
%% Made by Alex Nelson
%% Login   <alex@black-cherry>
%% 
%% Started on  Sun Jun 17 14:33:38 2012 Alex Nelson
%% Last update Sun Jun 17 15:36:44 2012 Alex Nelson
%%
%% \documentclass{article}
%% \pdfinfo{ /CreationDate (20120617143338)}
%% \usepackage{blog}
%% \begin{document}
\M
We have another couple of methods testing if an alternating
series converges. It's worth knowing as many different ways as
possible, because sometimes one doesn't work well (or at all).

We will consider a couple tests. For each test we provide a
proof that it works, and a couple examples.

\N{Alternating Ratio Test}
Consider the series 
\begin{equation}
\sum^{\infty}_{n=0}(-1)^{n}a_{n}
\end{equation}
Let
\begin{equation}
L = \lim_{n\to\infty}\frac{a_{n+1}}{a_{n}}.
\end{equation}
\begin{enumerate}
\item If $L<1$, then the series $\sum (-1)^{n}a_{n}$ conveges
  absolutely.
\item If $L>1$ (or $L=\infty$), then the series
  $\sum (-1)^{n}a_{n}$ diverges.
\end{enumerate}

\begin{proof}[Proof of Convergence]
We will prove if $L<1$, then the series
\begin{equation}
\sum^{\infty}_{n=0}a_{n}
\end{equation}
converges. The absolute convergence test implies the alternating
series will converge. What to do?

We first consider some $\varepsilon$ satisfying
$L<\varepsilon<1$. (Can we do this? Sure, pick $(L+1)/2$, and
we're good!) Since we suppose $L<1$, then there exists an $N$
such that
\begin{equation}
\left|\frac{a_{n+1}}{a_{n}}\right|<r\quad\mbox{for any }n\geq N.
\end{equation}
We have
\begin{equation}
\begin{aligned}
a_{N+1}&<r a_{N}\\
a_{N+2}&<r a_{N+1}<r^{2}a_{N}\\
a_{N+3}&<r^{3}a_{N}\\
a_{N+k}&<r^{k}a_{N}
\end{aligned}
\end{equation}
So we form a geometric series
\begin{equation}
\sum^{\infty}_{k=0}a_{N}r^{k} = \frac{a_{N}}{1-r}
\end{equation}
which bounds the ``most'' of our series
\begin{equation}
0\leq\sum^{\infty}_{k=0}a_{N+k}\leq\sum^{\infty}_{k=0}a_{N}r^{k}
\end{equation}
The comparison test tells us that ``most'' of our series
converges. But what about our whole series? We write it as
\begin{equation}
\sum^{\infty}_{n=0}a_{n} = \underbrace{\sum^{N-1}_{n=0}a_{n}}_{\text{finite}}+\underbrace{\sum^{\infty}_{k=0}a_{N+k}}_{\text{converges}}
\end{equation}
so it converges.
\end{proof}

\begin{proof}[Proof of Divergence]
We assume that $L>1$. The ratio $a_{n+1}/a_{n}$ will eventually
be greater than 1, too. So there exists an $N$ such that
\begin{equation}
\left|\frac{a_{n+1}}{a_{n}}\right|>1\quad\mbox{for any }n\geq N.
\end{equation}
So we see $|a_{n+1}|>|a_{n}|$ whenever $n\geq N$, thus
\begin{equation}
\lim_{n\to\infty}a_{n}\not=0.
\end{equation}
Thus it's impossible for the series $\sum a_{n}$ to converge!
\end{proof}


\begin{remark}
We only really proved that the series doesn't converge
\emph{absolutely}. If we pick some $r$ between $1<r<L$, then
there exists an $N$ such that
\begin{equation}
\left|\frac{a_{n+1}}{a_{n}}\right|>r\quad\mbox{for any }n\geq N.
\end{equation}
So we have
\begin{equation}
a_{N+k} > r^{k}a_{N}.
\end{equation}
We have our series
\begin{equation}
\begin{aligned}
\sum^{\infty}_{k=1}(-r)^{k}a_{N}
&= a_{N} \sum^{\infty}_{k=1}(r^{2k}-r^{2k-1})\\
&= a_{N} (r-1)\sum^{\infty}_{k=1}r^{2k}.
\end{aligned}
\end{equation}
We see since $r>1$ that the series
\begin{equation}
\sum^{\infty}_{k=1}r^{2k}\quad\mbox{diverges}
\end{equation}
Thus
\begin{equation}
\sum^{\infty}_{k=1}(-r)^{k}a_{N}\quad\mbox{diverges}
\end{equation}
We see then that the series
\begin{equation}
\sum^{\infty}_{k=1}a_{N+k}\geq\sum^{\infty}_{k=1}(-r)^{k}a_{N}
\end{equation}
diverges by the comparison test.
\end{remark}


\N{The Root Test}
Consider the series 
\begin{equation}
\sum^{\infty}_{n=1}(-1)^{n}a_{n}
\end{equation}
Let
\begin{equation}
L = \lim_{n\to\infty}\sqrt[n]{a_{n}}
\end{equation}
\begin{enumerate}
\item If $L<1$, then the series converges absolutely;
\item If $L>1$ (or $L=\infty$), then the series diverges.
\end{enumerate}

\begin{proof}
It's similar to the ratio test, for the convergent case we pick
some $r$ satisfying $L<r<1$. Then we have some $N$ satisfying
\begin{equation}
\sqrt[n]{a_{n}}<r\quad\mbox{for any }n\geq N.
\end{equation}
which implies
\begin{equation}
a_{n}<r^{n}.
\end{equation}
Thus we have
\begin{equation}
\sum^{\infty}_{k=1}a_{N+k}<\sum^{\infty}_{k=1}r^{N+k}<\sum^{\infty}_{k=0}r^{k}=\frac{1}{1-r}
\end{equation}
which by the  comparison test implies convergence. The proof for
divergence is similar.
\end{proof}
\begin{remark}
When $L=1$, the ratio test doesn't say anything about convergence
or divergence for the series.
\end{remark}

%\end{document}

\section{Power Series}
%%
%% powerSeries.tex
%% 
%% Made by Alex Nelson
%% Login   <alex@black-cherry>
%% 
%% Started on  Sat Jun 16 17:37:15 2012 Alex Nelson
%% Last update Sun Jun 17 15:45:52 2012 Alex Nelson
%%
%\documentclass{article}
%\usepackage{blog}
%\begin{document}

\N{Definitions}
Let $x$ be a variable. a series of the form
\begin{equation}
\sum^{\infty}_{n=0}a_{n}x^{n}=a_{0}+a_{1}x+a_{2}x^{2}+\dots+a_{n}x^{n}+\dots
\end{equation}
is called a \textbf{``Power Series about $x=0$''}.

A series of the form
\begin{equation}
\sum^{\infty}_{n=0}b_{n}(x-a)^{n}=b_{0}+b_{1}(x-a)+b_{2}(x-a)^{2}+\dots+b_{n}(x-a)^{n}+\dots
\end{equation}
is called a \textbf{``Power Series about $x=a$''}.

\begin{example}
Consider the series
\begin{equation}
f(x)=\sum^{\infty}_{n=0}\frac{n+1}{2^{n}}x^{n}.
\end{equation}
For what values of $x$ does this series converge? When will the
series diverge?\more{}

\emph{Solution.} Using the absolute ratio test, we see
\begin{equation}
\begin{aligned}
\lim_{n\to\infty}\left|\frac{\displaystyle\frac{n+2}{2^{n+1}}x^{n+1}}{\displaystyle\frac{n+1}{2^{n}}x^{n}}\right|
&=\lim_{n\to\infty}\frac{n+2}{n+1}\cdot\frac{2^{n}}{2^{n+1}}\cdot|x|\\
&=(1)\cdot\left(\frac{1}{2}\right)\cdot|x|\\
&=\frac{1}{2}|x|
\end{aligned}
\end{equation}
For convergence, we need
\begin{equation}
\frac{1}{2}|x|<1\quad\Longrightarrow\quad|x|<2.
\end{equation}
So divergence would be when
\begin{equation}
|x|>2.
\end{equation}
We need to check the $|x|=2$ case. Observe, for $x=2$, we have
\begin{equation}
f(2)=\sum^{\infty}_{n=0}\frac{n+1}{2^{n}}2^{n}=\sum^{\infty}_{n=0}n+1
\end{equation}
which diverges. And at $x=-2$ we have
\begin{equation}
f(-2)=\sum^{\infty}_{n=0}\frac{n+1}{2^{n}}(-2)^{n}=\sum^{\infty}_{n=0}(-1)^{n}(n+1)
\end{equation}
which still diverges!
\end{example}

\N{Theorem}
Let $c\not=0$ and 
\begin{equation}
f(x)=\sum^{\infty}_{n=0}a_{n}x^{n}
\end{equation}
converge at $x=c$. Then $f(x)$ converges absolutely for
$|x|<|c|$.

\begin{proof}
Let $x$ be any value such that $|x|<|c|$. Since $f(c)$ converges,
there exists an $M$ such that
\begin{equation}
|a_{n}c^{n}|\leq M\quad\mbox{for any }n.
\end{equation}
Well, we see
\begin{equation}
|x/c|\leq 1
\end{equation}
so
\begin{equation}
a_{n}x^{n}=a_{n}c^{n}\left(\frac{x}{c}\right)^{n}
\end{equation}
and moreover
\begin{equation}
|a_{n}x^{n}| = |a_{n}c^{n}|\cdot\left|\frac{x}{c}\right|^{n}\leq M\left|\frac{x}{c}\right|^{n}
\end{equation}
But observe the series
\begin{equation}
g(x)=\sum^{\infty}_{n=0}M\cdot\left|\frac{x}{c}\right|^{n}
\end{equation}
is a geometric series which converges since $|x/c|\leq1$.

Therefre the series $\sum|a_{n}x^{n}|$ converges by comparison,
and the absolute convergence test tells us $\sum a_{n}x^{n}$
converges. Therefore $\sum a_{n}x^{n}$ is absolutely convergent.
\end{proof}


\begin{example}
Find the values of $x$ for which
\begin{equation}
\sum^{\infty}_{n=1}\frac{(x+7)^{n}}{\sqrt{n}}
\end{equation}
converge.

\emph{Solution}: Using the absolute ratio test, we find
\begin{equation}
\begin{aligned}
\lim_{n\to\infty}\left|\frac{\left(\displaystyle\frac{(x+7)^{n+1}}{\sqrt{n+1}}\right)}{\left(\displaystyle\frac{(x+7)^{n}}{\sqrt{n}}\right)}\right|
&=\lim_{n\to\infty}\frac{\sqrt{n}}{\sqrt{n+1}}|x+7|\\
&=|x+7|.
\end{aligned}
\end{equation}
We get convergence for $|x+7|<1$. So
\begin{equation}
-1<x+7<1\quad\Longrightarrow\quad-8<x<-6.
\end{equation}
We have to check the boundary cases.

When $x=-6$, we have
\begin{equation}
\sum^{\infty}\frac{(7-6)^{n}}{\sqrt{n}}=\sum^{\infty}_{n=1}\frac{1}{\sqrt{n}}
\end{equation}
which diverges by the integral test (or the comparison test with
the Harmonic series). For $x=-8$ we have our series become
\begin{equation}
\sum^{\infty}_{n=1}\frac{(-1)^{n}}{\sqrt{n}}
\end{equation}
which converges by the alternating series test.
\end{example}


%\end{document}

\section{Taylor Series}
%%
%% taylorSeries.tex
%% 
%% Made by Alex Nelson
%% Login   <alex@black-cherry>
%% 
%% Started on  Sun Jun 17 15:47:04 2012 Alex Nelson
%% Last update Mon Jun 18 10:34:00 2012 Alex Nelson
%%
\M
Consider the function
\begin{equation}
f(x)  = \sum^{\infty}_{n=0}b_{n}(x-a)^{n} = b_{0} + b_{1}(x-a) + \dots.
\end{equation}
We suppose it is smooth (i.e., has infinitely many
derivatives). We see
\begin{equation}
\begin{aligned}
f'(x) &= \sum^{\infty}_{n=1}b_{n}n(x-a)^{n-1}\\
f'(a) &= b_{1}
\end{aligned}
\end{equation}
and
\begin{equation}
\begin{aligned}
f''(x) &= \sum^{\infty}_{n=2}b_{n}n(n-1)\cdot(x-a)^{n-2}\\
f''(a) &= 2b_{2}.
\end{aligned}
\end{equation}
We also have
\begin{equation}
\begin{aligned}
f'''(x) &= \sum^{\infty}_{n=3}b_{n}n(n-1)(n-2)\cdot(x-a)^{n-3}\\
f'''(a) &= 3\cdot2\cdot1 b_{3} = 6b_{3}.
\end{aligned}
\end{equation}
The general case appears to be
\begin{equation}
\left.\frac{\D^{n}}{\D x^{n}}f(x)\right|_{x=a}=n! b_{n}
\end{equation}
Thus the coefficients are
\begin{equation}
b_{n} = \left.\frac{1}{n!}\frac{\D^{n}}{\D x^{n}}f(x)\right|_{x=a}
\end{equation}
and we can reconstruct the function $f(x)$.

\N{Definition}
Let $f(x)$ be any function. The \textbf{``Taylor Series about $x=a$''}
is a series
\begin{equation}
\sum^{\infty}_{n=0}f^{(n)}(a)\cdot(x-a)^{n}
\end{equation}
When $a=0$, it's called a \emph{MacLaurin Series}.

\begin{remark}
If we have the MacLaurin series for a function, and if the series
converges for any value of $x$, then we can use the MacLaurin
series as a synonym for the original function. That's the
usefulness of MacLaurin series.
\end{remark}
\begin{remark}
The Taylor series helps us compute $f(x+h)$ when $h\ll x$ and
when $f(x)$ is known. For example, $\sqrt{1.001}$ can be computed
using the Taylor series of $\sqrt{1+x}$ about $x=1$.
\end{remark}

\begin{example}
Consider the function $\exp(x)$. We see that
\begin{equation}
\frac{\D\exp(x)}{\D x}=\exp(x).
\end{equation}
Thus the MacLaurin series for the exponential function is
\begin{equation}
\exp(x)=\sum^{\infty}_{n=0}\frac{x^{n}}{n!}
\end{equation}
Where does this series converge?

Using the absolute ratio test, we find
\begin{equation}
\lim_{n\to\infty}\frac{x^{n+1}/(n+1)!}{x^{n}/n!} =
\lim_{n\to\infty}\frac{x}{n+1}=0.
\end{equation}
So the series converges \emph{for any value of $x$}.
\end{example}
\begin{example}
Consider the sine function $\sin(x)$. What is its MacLaurin
series? Writing
\begin{equation}
\sum^{\infty}_{n=0}c_{n}x^{n}=\sin(x)
\end{equation}
we have
\begin{equation}
\begin{aligned}
c_{0}&=\sin(0)=0\\
c_{1}&=\cos(0)=1\\
c_{2}&=\frac{-\sin(0)}{2!}=0\\
c_{3}&=\frac{-\cos(0)}{3!}=-1/3!
\end{aligned}
\end{equation}
The coefficients are sort of ``periodic'' in the sense that only
the odd ones remain, and their sign alternates. We have
\begin{equation}
c_{n} = c_{2k-1} = \frac{(-1)^{n}}{n!} = \frac{(-1)^{k+1}}{(2k-1)!}
\end{equation}
thus the MacLaurin series is
\begin{equation}
\sin(x)=\sum^{\infty}_{n=1}\frac{(-1)^{n+1}x^{n}}{(2n-1)!}.
\end{equation}
Where does this converge? 

Using the ratio test, we have
\begin{equation}
\lim_{n\to\infty}\left|\frac{\left(\displaystyle\frac{x^{2n+3}}{(2n+3)!}\right)}{\left(\displaystyle\frac{x^{2n+1}}{(2n+1)!}\right)}\right|
=\lim_{n\to\infty}\frac{x^{2}}{(2n+1)(2n+2)} =  0.
\end{equation}
Thus it converges for any value of $x$.
\end{example}


\begin{example}

Lets find the MacLaurin series for $\cos(x)$. We see that
\begin{equation}
\cos(x)=\sum^{\infty}_{n=0}b_{n}x^{n},
\end{equation}
we need to find the $b_{n}$. We see that
\begin{equation}
\begin{aligned}
b_{0} &= \cos(0) = 1\\
b_{1} &= -\sin(0) = 0\\
b_{2} &= \frac{-1}{2!}\cos(0) = -1/2\\
b_{3} &= \frac{1}{3!}\sin(0) = 0\\
b_{2k} &= \frac{(-1)^{k}}{(2k)!}
\end{aligned}
\end{equation}
So we have the MacLaurin series be
\begin{equation}
\cos(x) = \sum^{\infty}_{n=0}\frac{(-1)^{n}}{(2n)!}x^{2n}.
\end{equation}
Where will it converge? Using the absolute ratio test, we have
\begin{equation}
\begin{aligned}
\lim_{n\to\infty}\left|\frac{\left(\displaystyle\frac{(-1)^{n+1}x^{2n+2}}{(2n+2)!}\right)}{\left(\displaystyle\frac{(-1)^{n}x^{2n}}{(2n)!}\right)}\right|
&=\lim_{n\to\infty}\frac{x^{2}}{(2n+1)(2n+2)}\\
&=0.
\end{aligned}
\end{equation}
Thus the MacLaurin series for $\cos(x)$ converges for any value
of $x$.
\end{example}



\N{Euler's Formula}
Recall Euler's formula states
\begin{equation}
\exp(\I\theta)=\cos(\theta)+\I\sin(\theta)
\end{equation}
and we have the MacLaurin  series for $\exp(x)$. Setting
$x=\I\theta$, we find
\begin{equation}
\begin{aligned}
\exp(\I\theta) &= \sum^{\infty}_{n=0}\frac{(\I\theta)^{n}}{n!}\\
&=1+\I\theta-\frac{\theta^{2}}{2!}-\frac{\I\theta^{3}}{3!}+\frac{\theta^{4}}{4!}+\dots
\end{aligned}
\end{equation}
Gathering the real and imaginary parts together we get
\begin{equation}\label{eq:taylorSeries:expITheta:comparison}
\exp(\I\theta) =
\left(\sum^{\infty}_{n=0}\frac{(-1)^{n}x^{2n}}{(2n)!}\right)
+\I\left(\sum^{\infty}_{n=0}\frac{(-1)^{n}x^{2n+1}}{(2n+1)!}\right)
\end{equation}
But look: the imaginary part is precisely the MacLaurin series
for $\sin(x)$! And the real part is the MacLaurin series for
$\cos(x)$, too! So what did we do? We just derived Euler's
formula. 

\N{Taylor Series Makes Approximations}
Consider the function
\begin{equation}
f(x)=\sqrt{x}.
\end{equation}
Its Taylor series about $x=1$ is \emph{the same} as the MacLaurin
series for the function
\begin{equation}
g(h)=\sqrt{1+h}.
\end{equation}
Just write $x=1+h$. For small $0<h\ll1$, we can use the first
couple of terms in the Taylor series as an approximate value. So
what's the first 6 nonzero terms in the Taylor series? We see
\begin{subequations}
\begin{equation}
f(x)=x^{1/2}\quad\implies\quad f(1)=1
\end{equation}
describes the constant term. The linear term has coefficient
\begin{equation}
f'(x)=\frac{x^{-1/2}}{2}\quad\implies\quad f'(1)=\frac{1}{2}.
\end{equation}
The quadratic term
\begin{equation}
f''(x)=\frac{-x^{-3/2}}{2^{2}}\quad\implies\quad f''(1)=-1/2^{2}.
\end{equation}
Observe how the exponent behaves when differentiating:
$(1/2)-n$. The numerator is always odd, the denominator doesn't
change. So the $n^{th}$ derivative would be
\begin{equation}
f^{(n)}(x)=\frac{1\cdot3\cdot5\cdot(\dots)\cdot(2n-1)}{2^{n}}(-1)^{n-1}x^{(1-2n)/2}
\end{equation}
This gives us our coefficients! We then have
\begin{equation}
c_{n} = \frac{(-1)^{n-1}(2n)!}{(n!2^{n})^{2}}
\end{equation}
for $n>0$. Observe the numerator appears odd, but we can justify
it thus:
\begin{equation}
\begin{aligned}
\frac{(2n)!}{n!2^{n}}
&= \frac{1\cdot2\cdot3\cdot(\dots)\cdot(2n)}{(1\cdot2\cdot(\dots)\cdot n)2^{n}}\\
&= \frac{\bigl(1\cdot3\cdot(\dots)\cdot(2n-1)\bigr)\bigl(2\cdot4\cdot(\dots)\cdot 2n\bigr)}{(2\cdot4\cdot(\dots)\cdot2n)}\\
&= 1\cdot3\cdot(\dots)\cdot(2n-1)\frac{\bigl(2\cdot4\cdot(\dots)\cdot 2n\bigr)}{(2\cdot4\cdot(\dots)\cdot2n)}\\
&=
1\cdot3\cdot(\dots)\cdot(2n-1)\left(\vphantom{\frac{a}{a}}1\right)\\
&= 1\cdot3\cdot(\dots)\cdot(2n-1)
\end{aligned}
\end{equation}
\end{subequations}
At any rate, this gives us a polynomial expression for $f(1+h)$
as
\begin{equation}
f(1+h) = \left(1+\sum^{6}_{n=1}\frac{(-1)^{n-1}(2n)!}{(n!2^{n})^{2}}h^{n}\right)
+ R_{6}(h)
\end{equation}
where $R_{6}(h)$ is the ``error'' term. What does the error term
tell us? Precisely how good or bad our polynomial approximates
$f(1+h)$. That is to say, how $(1+c_{1}h+\dots+c_{6}h^{6})$
approximates $f(1+h)$.

\emph{Question}: how do we find the error term?

\subsection{Taylor Polynomials}
%%
%% taylorErrorPt1.tex
%% 
%% Made by Alex Nelson
%% Login   <alex@black-cherry>
%% 
%% Started on  Mon Jun 18 10:34:28 2012 Alex Nelson
%% Last update Mon Jun 18 10:51:51 2012 Alex Nelson
%%

\M
So last time we constructed a polynomial using the first $n$
terms in the Taylor series. This ``Taylor polynomial''
approximated our function, and we want to know \emph{how well does it approximate?}

We will derive the Taylor series differently. Our derivation will
give us a natural error term. \more{}

\N{Set Up} Lets consider $f(x)$ which is sufficiently nice around
$x=0$ (i.e., it has enough derivatives at $x=0$). The fundamental
theorem of calculus tells us that
\begin{equation}\label{eq:taylorErr1:line1}
f(x) = f(0) + \int^{x}_{0}f'(t)\,\D t.
\end{equation}
Wonderful.

\N{Linear Approximation}
We now use integration by parts for the integral in Equation \eqref{eq:taylorErr1:line1}:
\begin{equation}
\begin{aligned}
\int^{x}_{0}f'(t)\,\D t &=
\left.tf'(t)\right|^{x}_{0}-\int^{x}_{0}tf''(t)\,\D t\\
&= xf'(x)-0f'(0) - \int^{x}_{0}tf''(t)\,\D t\\
&= xf'(x) - \int^{x}_{0}tf''(t)\,\D t.
\end{aligned}
\end{equation}
Now we can plug in Equation \eqref{eq:taylorErr1:line1} for the
first term in the right hand side:
\begin{equation}
\begin{aligned}
xf'(x) - \int^{x}_{0}tf''(t)\,\D t &=
x\left(f'(0)+\int^{x}_{0}f'(t)\,\D t\right) - \int^{x}_{0}tf''(t)\,\D t\\
&= xf'(0) - \int^{x}_{0}(t-x)f''(t)\,\D t.
\end{aligned}
\end{equation}
We plug this back into Equation \eqref{eq:taylorErr1:line1}
\begin{equation}
\begin{aligned}
f(x) &= f(0) + \int^{x}_{0}f'(t)\,\D t\\
&= f(0) +\left[ xf'(0) - \int^{x}_{0}(t-x)f''(t)\,\D t\right].
\end{aligned}
\end{equation}
Observe the first two terms are precisely the linear
approximation to $f(x)$. What's the third term? \emph{The error term!}
But we're not done yet!

\N{Inductive Procedure}
We can iterate a procedure to get a better and better
approximation. Performing integration by parts on
\begin{equation}
\int^{x}_{0}(t-x)f''(t)\,\D t
\end{equation}
will give us the next term in our approximation plus an
integral. The integral gives us the error for using a quadratic
approximation. 

Lets do it! We use integration by parts 
\begin{equation}
\begin{aligned}
\int^{x}_{0}(t-x)f''(t)\,\D t &=
\left.\frac{(t-x)^{2}}{2!}f''(t)\right|^{x}_{0}-\int^{x}_{0}\frac{(t-x)^{2}}{2!}f'''(t)\,\D t\\
&=\frac{-(-x)^{2}}{2!}f''(0)-\int^{x}_{0}\frac{(t-x)^{2}}{2!}f'''(t)\,\D t
\end{aligned}
\end{equation}
So our approximation becomes
\begin{equation}
\begin{aligned}
f(x) &= f(0) +  xf'(0) - \left[\int^{x}_{0}(t-x)f''(t)\,\D  t\right]\\
&=f(0) + xf'(0) - \left[\frac{-(-x)^{2}}{2!}f''(0)-\int^{x}_{0}\frac{(t-x)^{2}}{2!}f'''(t)\,\D t\right]\\
&=f(0) + xf'(0) + \frac{x^{2}}{2!}f''(0)+\left[\int^{x}_{0}\frac{(t-x)^{2}}{2!}f'''(t)\,\D t\right]
\end{aligned}
\end{equation}
The brackted term in the final line is the error term for using
the quadratic approximation. But look: the quadratic
approximation corresponds to the first 3 terms of the Taylor series!
When we do this iterative procedure on the error term,
integrating by parts will produce a higher order
approximation. And for free, we get the error term telling us how
good our approximation is!

\section{Geometry of Space}
%%
%% threeD.tex
%% 
%% Made by Alex Nelson
%% Login   <alex@black-cherry>
%% 
%% Started on  Wed Jun 20 13:38:30 2012 Alex Nelson
%% Last update Thu Jun 21 17:40:32 2012 Alex Nelson
%%

\N{The 3 Dimensional Coordinate System}
The $x$, $y$, $z$ axes are perpendicular to each other. We will
doodle three dimensions as follows:
\begin{center}
\includegraphics{img/threeD.0}
\end{center}
The formula for distance between two points $(x_{0}, y_{0},
z_{0})$ and $(x_{1},y_{1},z_{1})$ would be
\begin{equation}
s = \sqrt{(x_{1}-x_{0})^{2}+(y_{1}-y_{0})^{2}+(z_{1}-z_{0})^{2}}.
\end{equation}
A sphere with its center at $(a,b,c)$ would be the points which
are a distance $r$ away from the center. So we would have
\begin{equation}
S^{2} = \{\,(x,y,z) : \sqrt{(x-a)^{2}+(y-b)^{2}+(z-c)^{2}}=r\,\}
\end{equation}
Usually the formula is given as
\begin{equation}
(x-a)^{2}+(y-b)^{2}+(z-c)^{2}=r^{2}
\end{equation}
describes a sphere.

\N{Vectors}
A \textbf{``Vector''} is a directed line segment.

We know a directed line segment has both length (magnitude) and
direction, so any two directed line segments with the same length
\emph{and} direction represent the same vector.

Vectors are ``transportable'' in the sense that we may translate
their base point. We will represent the length of a vector
$\vec{u}$ as $\|\vec{u}\|$ or $|\vec{u}|$.

The notation for a vector would be $\langle x,y\rangle$ (in two
dimensions) or $\langle x,y,z\rangle$ (in three dimensions). The
vector from $(0,0,0)$ to $(x,y,z)$ is given as $\langle x,y,z\rangle$.
For $P=(-1,4,7)$ and $Q=(2,5,3)$, then the vector from $P$ to $Q$
is denoted $\overrightarrow{PQ}$.

We can add vectors graphically:
\begin{center}
\includegraphics{img/threeD.1}
\end{center}
Subtraction would amount to $\vec{A}-\vec{B}=\vec{A}+(-\vec{B})$,
and graphically this is:
\begin{center}
\includegraphics{img/threeD.2}
\end{center}
Algebraically, if
\begin{equation}
\vec{A}=\langle x_{a},y_{a},z_{a}\rangle,\quad
\vec{B}=\langle x_{b},y_{b},z_{b}\rangle
\end{equation}
then
\begin{equation}
\begin{aligned}
\vec{A}+\vec{B} &= \langle x_{a}+x_{b}, y_{a}+y_{b},
z_{a}+z_{b}\rangle\\
\vec{A}-\vec{B} &= \langle x_{a}-x_{b}, y_{a}-y_{b},
z_{a}-z_{b}\rangle
\end{aligned}
\end{equation}
This describes vector addition and subtraction on the components.

Note in two-dimensional space, the vectors
\begin{equation}
\widehat{\textbf{\i}}=\langle 1,0\rangle
\quad\mbox{and}\quad
\widehat{\textbf{\j}}=\langle 0,1\rangle
\end{equation}
are unit vectors (i.e., vectors whose length is $1$). They are
also called \emph{basis vectors} since any other vector $\vec{v}$
in two-dimensions can be written as
\begin{equation}
\begin{aligned}
\vec{v}&=\langle v_{x}, v_{y}\rangle\\
&=\langle v_{x},0\rangle + \langle0,v_{y}\rangle\\
&=v_{x}\langle1,0\rangle + v_{y}\langle0,1\rangle\\
&=v_{x}\widehat{\textbf{\i}}+v_{y}\widehat{\textbf{\j}}
\end{aligned}
\end{equation}
where $\langle v_{x}$ and $v_{y}$ are called the vector's 
\emph{components}. Note that the components of the vector depends on a
choice of coordinates (i.e., a choice of basis vectors).

The last notion we will discuss: given any vector $\vec{v}$ which
is nonzero, then we can construct the unit vector
\begin{equation}
\widehat{v} = \frac{\vec{v}}{\|\vec{v}\|}
\end{equation}
which has magnitude 1. We use hats to indicate unit vectors, and
arrows for arbitrary vectors.

\N{Caution:} Everything stated about vectors is a
half-truth. Really, these are ``tangent vectors'' which has a
base point and a vector part (i.e., where we stick the line
segment, and the directed line segment itself). We can only
add/subtract two tangent vectors if they have the same base point. 
But since we work in Euclidean space (which is flat), we can
transport vectors without a problem. This is a very special
situation! 

Since this is never mentioned, often students become confused
when they finish vector calculus and begin studying linear
algebra. Linear algebra fixes a base point, and considers the
collection of all vectors sharing the same base point. This is
the honest definition of a vector. 

\begin{remark}
We will also use the phrase ``three-space'' instead of
``three-dimensional space'', and ``two-space'' replacing
``two-dimensional space''. In general $n$-space is
$n$-dimensional Euclidean space.
\end{remark}

\subsection{Dot Products}
\N{Definition}
Given two vectors
\begin{equation}
\vec{u}=u_{1}\widehat{\textbf{\i}}+u_{2}\widehat{\textbf{\j}}+u_{3}\widehat{\textbf{k}},\quad\mbox{and}\quad
\vec{v}=v_{1}\widehat{\textbf{\i}}+v_{2}\widehat{\textbf{\j}}+v_{3}\widehat{\textbf{k}}
\end{equation}
their \textbf{``Dot Product''} is the number
\begin{equation}
\begin{aligned}
\vec{u}\cdot\vec{v} &= \sum_{i} u_{i}v_{i}\\
&= u_{1}v_{1}+u_{2}v_{2}+u_{3}v_{3}
\end{aligned}
\end{equation}
Let $\vec{u}$ and $\vec{v}$ be given vectors in three-space. 

\N{Angles}
How do we find the angle $\theta$ between the two vectors?
\begin{center}
\includegraphics{img/threeD.3}
\end{center}
We find that
\begin{equation}
\theta = \arccos\left(\frac{\|\vec{v}\|}{\|\vec{u}\|}\right)
\end{equation}
How is this? Well, we should recall the law of cosines
\begin{equation}
\|\vec{w}\|^{2}=\|\vec{u}\|^{2}+\|\vec{v}\|^{2}-2\|\vec{u}\|\cdot\|\vec{v}\|\cos(\theta)
\end{equation}
which can be written as
\begin{equation}
\begin{aligned}
\vec{w}\cdot\vec{w} &=
(u_{1}-v_{1})^{2}+(u_{2}-v_{2})^{2}+(u_{3}-v_{3})^{2}\\
&= \|\vec{u}\|^{2}-2(\vec{u}\cdot\vec{v})+\|\vec{v}\|^{2}
\end{aligned}
\end{equation}
Setting equals to equals gives us
\begin{equation}
-2(\vec{u}\cdot\vec{v})=-2\|\vec{u}\|\cdot\|\vec{v}\|\cos(\theta)
\end{equation}
and thus
\begin{equation}
\frac{\vec{u}\cdot\vec{v}}{\|\vec{u}\|\cdot\|\vec{v}\|}=\cos(\theta)
\end{equation}
Taking the arc cosine of both sides yields
\begin{equation}
\arccos\left(\frac{\vec{u}\cdot\vec{v}}{\|\vec{u}\|\cdot\|\vec{v}\|}\right)=\theta.
\end{equation}
A useful formula worth remembering 
\begin{equation}
(\vec{u}\cdot\vec{v})=\|\vec{u}\|\cdot\|\vec{v}\|\cos(\theta)
\end{equation}
\begin{example}
Let $\vec{u}=\langle3,-1,4\rangle$ and
$\vec{v}=\langle1,5,-2\rangle$. What's the angle between them?

\emph{Solution}: We first find
\begin{equation}
\begin{aligned}
\vec{u}\cdot\vec{v} &= (3\cdot1)+(-1\cdot5)+(4\cdot-2)\\
&=3-5-8=-10.
\end{aligned}
\end{equation}
We then compute
\begin{equation}
\|\vec{u}\|=\sqrt{9+1+16}=\sqrt{26}
\end{equation}
and
\begin{equation}
\|\vec{v}\|=\sqrt{1+25+4}=\sqrt{30}.
\end{equation}
Thus the angle between $\vec{u}$ and $\vec{v}$ is
\begin{equation}
\theta=\arccos\left(\frac{-10}{\sqrt{26}\sqrt{30}}\right)\approx1.937
\end{equation}
(radians).
\end{example}

\subsection{Orthogonality}

\M
Vectors are perpendicular or \textbf{``Orthogonal''} if
$\vec{u}\cdot\vec{v}=0$. Sometimes this is denoted $\vec{u}\bot\vec{v}$.

\begin{example}
Consider
\begin{equation}
\vec{u}=\langle6,-3,8\rangle,\quad\mbox{and}\quad
\vec{v}=\langle-2,4,3\rangle.
\end{equation}
We see
\begin{equation}
\vec{u}\cdot\vec{v}=-12-12+24=0
\end{equation}
which implies $\vec{u}$ and $\vec{v}$ are orthogonal.
\end{example}

\M
Consider the following diagram
\begin{center}
\includegraphics{img/threeD.4}
\end{center}
We're given vectors $\vec{u}=\overrightarrow{PQ}$ and
$\vec{v}=\overrightarrow{PS}$ in 3-space. Notice that if the
angle between the vectors $\theta$ is acute, as doodled, then
$\overrightarrow{PR}$ is the projection of $\vec{u}$ onto
$\vec{v}$. 

However, if $\theta$ is obtuse, we doodle the situation thus:
\begin{center}
\includegraphics{img/threeD.5}
\end{center}
Observe the projection of $\vec{u}$ onto $\vec{v}$ will not fall
on $\vec{v}$. The projection of $\vec{u}$ onto $\vec{v}$ is
syntactically 
\begin{equation}
\proj_{\vec{v}}\vec{u}
\end{equation}
The natural question: \emph{what is the formula for projecting
$\vec{u}$ onto $\vec{v}$?} We have
\begin{equation}
\|\overrightarrow{PR}\|=\begin{cases}\|\vec{u}\|\cos(\theta)
&\mbox{for $\theta$ acute}\\
-\|\vec{u}\|\cos(\theta)&\mbox{for $\theta$ obtuse}
\end{cases}
\end{equation}
The direction of $\overrightarrow{PR}$ depends on whether
$\theta$ is acute or obtuse; we have its unit vector be
\begin{equation}
\widehat{PR}=\begin{cases}\widehat{v}&\mbox{for $\theta$ acute}\\
-\widehat{v}&\mbox{for $\theta$ obtuse}
\end{cases}
\end{equation}
But now look, for both obtuse and acute $\theta$ we have
\begin{equation}
\begin{aligned}
\overrightarrow{PR}
&=\|\vec{u}\|\cos(\theta)\frac{\vec{v}}{\|\vec{v}\|}\\
&=\frac{\|\vec{u}\|\|\vec{v}\|\cos(\theta)}{\|\vec{v}\|}\frac{\vec{v}}{\|\vec{v}\|}\\
&=\frac{\vec{u}\cdot\vec{v}}{\|\vec{v}\|}\frac{\vec{v}}{\|\vec{v}\|}
\end{aligned}
\end{equation}
This describes the projection of $\vec{u}$ onto $\vec{v}$
for \emph{any} $\theta$:
\begin{equation}
\proj_{\vec{v}}\vec{u}=\frac{\vec{u}\cdot\vec{v}}{\|\vec{v}\|}\frac{\vec{v}}{\|\vec{v}\|}
\end{equation}
Notice that its magnitude is $\vec{u}\cdot\vec{v}/\|\vec{v}\|$. 

\M
Consider the same situation again. We have a vector
$\vec{w}=\overrightarrow{RQ}$ as doodled
\begin{center}
\includegraphics{img/threeD.6}
\end{center}
This vector $\vec{w}$ is orthogonal to the projection of
$\vec{u}$ onto $\vec{v}$. For this reason, we write
\begin{equation}
\vec{w} = \mathop{\mathrm{orth}}\nolimits_{\vec{v}}\vec{u}
\end{equation}
What is it? Well, using basic vector arithmetic, we find
\begin{equation}
\vec{w} = \vec{u} - \proj_{\vec{v}}\vec{u}.
\end{equation}
We will conclude our discussion of vectors here, but continue
next time.

\subsection{More Vector Fun}
%%
%% moreVectors.tex
%% 
%% Made by Alex Nelson
%% Login   <alex@black-cherry>
%% 
%% Started on  Thu Jun 21 15:16:44 2012 Alex Nelson
%% Last update Thu Jun 21 17:28:46 2012 Alex Nelson
%%

\M
Last time we ended with discussing how to project a vector onto
another. So if we consider projecting $\vec{u}$ onto $\vec{v}$,
we can write this as
\begin{equation}
\proj_{\vec{v}}\vec{u} = \vec{u}_{\|}
\end{equation}
Observe, we have another vector constructed
\begin{equation}
\vec{u}_{\bot} = \vec{u}-\vec{u}_{\|}
\end{equation}
which is orthogonal to $\vec{v}$. We have a closed form
expression for projection, namely
\begin{equation}
\proj_{\vec{v}}\vec{u} = (\vec{u}\cdot\widehat{v})\widehat{v}
\end{equation}
where $\widehat{v}=\vec{v}/\|\vec{v}\|$ is a unit vector. But do
we have a closed form expression for $\vec{u}_{\bot}$?\more

\N{Cross-Product}
The cross product of $\vec{u}$ and $\vec{v}$ is 
\begin{equation}
\vec{u}\times\vec{v} = \begin{vmatrix}
\widehat{\textbf{\i}} & \widehat{\textbf{\j}} & \widehat{\mathbf{k}}\\
       u_{1}          &            u_{2}      & u_{3}\\
       v_{1}          &            v_{2}      & v_{3}
\end{vmatrix} =
\bigl(\|\vec{u}\|\|\vec{v}\|\sin(\theta)\bigr)\widehat{u}
\end{equation}
Note we are using notation from linear algebra writing, recursively,
\begin{equation}
\begin{aligned}
\det(A) &= \begin{vmatrix} a_{11} & a_{12} & a_{13}\\
a_{21} & a_{22} & a_{23}\\
a_{31} & a_{32} & a_{33}
\end{vmatrix} \\
&= a_{11}\begin{vmatrix} a_{22} & a_{23} \\ a_{32} &
  a_{33}
\end{vmatrix}
- a_{12} \begin{vmatrix} a_{21} & a_{23}\\ a_{31} & a_{33}
\end{vmatrix}
+a_{13}\begin{vmatrix} a_{21} & a_{22}\\ a_{31} & a{32}
\end{vmatrix}
\end{aligned}
\end{equation}
where
\begin{equation}
\begin{vmatrix} a & b\\ c & d
\end{vmatrix} = ad-bc.
\end{equation}
\begin{remark}
Observe this implies
$\widehat{\textbf{\i}}\times\widehat{\textbf{\j}}=\widehat{\textbf{k}}$, 
$\widehat{\textbf{\j}}\times\widehat{\textbf{k}}=\widehat{\textbf{\i}}$,
and 
$\widehat{\textbf{k}}\times\widehat{\textbf{\i}}=\widehat{\textbf{\j}}$.
\end{remark}
\begin{remark}
The cross-product takes two vectors, and \emph{produces a third vector}.
It \emph{does not} produce a scalar (a number, unlike the dot product).
\end{remark}
\emph{Pop quiz}: let $\vec{u}$ and $\vec{v}$ be vectors. Is $\vec{u}\times\vec{v}=\vec{v}\times\vec{u}$?

\begin{example}
Consider $\vec{u}=\langle2,1,-3\rangle$ and
$\vec{v}=\langle1,-2,1\rangle$. What is $\vec{u}\times\vec{v}$?

\emph{Solution}: we find
\begin{equation}
\begin{aligned}
\vec{u}\times\vec{v} &= \begin{vmatrix}
\widehat{\textbf{\i}} & \widehat{\textbf{\j}} &\widehat{\textbf{k}}\\
2 & 1 & -3\\
1 & -2 & 1
\end{vmatrix}\\
&= (1\cdot1-(-3)\cdot(-2))\widehat{\textbf{\i}} 
- (2\cdot1-(-3)\cdot1)\widehat{\textbf{\j}} 
+(2\cdot(-2)-1\cdot1)\widehat{\textbf{k}}\\
&=(1-6)\widehat{\textbf{\i}} 
- (2+3)\widehat{\textbf{\j}} 
+(-4-1)\widehat{\textbf{k}}\\
&=\langle-5,5,-5\rangle
\end{aligned}
\end{equation}
Another approach would have been to write
\begin{equation}
\vec{u}\times\vec{v} 
= (2\widehat{\textbf{\i}} + \widehat{\textbf{\j}} -3\widehat{\textbf{k}})
\times(\widehat{\textbf{\i}} -2 \widehat{\textbf{\j}} +\widehat{\textbf{k}})
\end{equation}
and used the cross-product's anticommutativity to do the calculations.
\end{example}

\N{Parallelogram Area}
Consider three distinct points $P$, $Q$, and $R$. We can
construct a parallelograph, as in the following diagram:
\begin{center}
\includegraphics{img/moreVectors.0}
\end{center}
We see that
$\overrightarrow{PQ}\times\overrightarrow{PR}=\vec{N}$, then the
area of the parallelogram is $\|\vec{N}\|$.

\N{Parallepiped Volume}
If we work in 3-space, and we have a six-sided region whose sides
are each parallelograms, we call this region a
parallepiped. Observe that we only need 3 vectors to specify the
vertices: $\vec{u}$, $\vec{v}$, and $\vec{w}$. Then we consider
$\vec{u}+\vec{v}$, $\vec{u}+\vec{w}$, $\vec{v}+\vec{w}$, and
$\vec{u}+\vec{v}+\vec{w}$ for the remaining vertices. What is the
volume of this region?

Lets draw a diagram:
\begin{center}
\includegraphics{img/moreVectors.1}
\end{center}
Lets first consider the face described by $\overrightarrow{PQ}=\vec{u}$ and
$\overrightarrow{PR}=\vec{v}$. We see the parallepiped may be considered as a
``stack'' of such faces, whose height is given by the third
vector $\vec{w}$. Then we see the area of the face is shaded in
the diagram, and algebraically it's given by
$\vec{u}\times\vec{v}$, and this produces a vector whose
magnitude is the area of the face. When we ``dot'' this with
$\proj_{\vec{u}\times\vec{v}}\vec{w}$, it's intuitively taking the product of the ``area of
aa face'' ($\vec{u}\times\vec{v}$) and the ``height of the
parallepiped'' ($\proj_{\vec{u}\times\vec{v}}\vec{w}$) producing the volume 
\begin{equation}
\mbox{volume } = (\vec{u}\times\vec{v})\cdot\vec{w}.
\end{equation}
Note this can be negative, and this just tells us information
regarding the parallepiped's \emph{orientation}.


\subsection{Line Constructions}
%%
%% lines.tex
%% 
%% Made by Alex Nelson
%% Login   <alex@black-cherry>
%% 
%% Started on  Thu Jun 21 17:27:08 2012 Alex Nelson
%% Last update Thu Jun 21 17:28:45 2012 Alex Nelson
%%
\N{Constructing Lines}
Suppose we have two points
\begin{equation}
A = (4,2,-1)\quad\mbox{and}\quad
B = (3,5,7).
\end{equation}
We want to find a line $\ell$ passing through these points. What to do?

First we form the vector
\begin{equation}
\begin{aligned}
\vec{v}=\overrightarrow{AB}
&= (3-4)\widehat{\textbf{\i}}+(5-2)\widehat{\textbf{\j}}+(7+1)\widehat{\textbf{k}}\\
&=-\widehat{\textbf{\i}}+3\widehat{\textbf{\j}}+8\widehat{\textbf{k}}
\end{aligned}
\end{equation}
This vector is parallel to $\ell$; the numbers given by this
vector's components (i.e., -1, 3, 8) are called
the \textbf{``Direction Numbers''} of $\ell$.  

In general, we have two distinct points $P_{0}=(x_0,y_0,z_0)$ and
$P=(x,y,z)$ on the line $\ell$, then we construct the vector
\begin{equation}
\overrightarrow{P_{0}P} = (x-x_{0})\widehat{\textbf{\i}}+
(y-y_{0})\widehat{\textbf{\j}}+(z-z_{0})\widehat{\textbf{k}}
\end{equation}
and this is equal to some scalar multiple of $\vec{v}$ (i.e.,
it's a dilation of the vector).  We write
\begin{equation}
(x-x_{0})\widehat{\textbf{\i}}+
(y-y_{0})\widehat{\textbf{\j}}+(z-z_{0})\widehat{\textbf{k}}
=t\vec{v}
\end{equation}
which lets us write
\begin{equation}
\left.\begin{array}{rl}
x-x_{0} &=tv_{1}\\
y-y_{0} &=tv_{2}\\
z-z_{0} &=tv_{3}
\end{array}
\right\}\implies
\left\{\begin{array}{rl}
x &=x_{0}+tv_{1}\\
y &=y_{0}+tv_{2}\\
z &=z_{0}+tv_{3}
\end{array}\right.
\end{equation}
This is the parametric equations of $\ell$. So returning to our
example, we have
\begin{equation}
\begin{aligned}
x &= 4 - t\\
y &= 2 + 3t\\
z &= -1 + 8t
\end{aligned}
\end{equation}
where the constant terms are precisely the values of the
components of $A$, and the coefficients of $t$ are the components
of the vector $\overrightarrow{AB}$.

\N{Distance From a Point to a Line}
What's the distance from any point $S$ in 3-space to a given line
$\ell$? 

We pick a point $P$ on $\ell$ and form a vector
$\overrightarrow{PS}$. The distance from $S$ to $\ell$ can be
given as $d$, as in the following diagram:
\begin{center}
\includegraphics{img/moreVectors.2}
\end{center}
We see $d=\|\overrightarrow{PS}\|\sin(\theta)$. If $\vec{v}$ is a
vector parallel to $\ell$, then we have
\begin{equation}
\|\overrightarrow{PS}\times\widehat{v}\|=\|\overrightarrow{PS}\|\sin(\theta)
\end{equation}
This is all abstract, lets consider an example.

\begin{example}
Find the distance from $S=(2,1,3)$ to the line given by
\begin{equation}
\begin{aligned}
x &= -1+t\\
y &= 2+t\\
z &= 1+2t
\end{aligned}
\end{equation}
First we pick the point when $t=0$, we call it
\begin{equation}
P = (-1,2,1).
\end{equation}
Observe
\begin{equation}
\overrightarrow{PS} =
3\widehat{\textbf{\i}}-\widehat{\textbf{\j}}+2\widehat{\textbf{k}}. 
\end{equation}
Now we need to find a vector parallel to the line. What to do?
Construct a vector by considering the point when $t=1$, which
would be $P_{1}=(0,3,3)$. Thus
\begin{equation}
\begin{aligned}
\vec{v}
&=\overrightarrow{PP_{1}}\\
&=(-1-0)\widehat{\textbf{\i}}+(2-3)\widehat{\textbf{\j}}+(1-3)\widehat{\textbf{k}}\\
&=-\widehat{\textbf{\i}}-\widehat{\textbf{\j}}-2\widehat{\textbf{k}}
\end{aligned}
\end{equation}
Its unit vector is
\begin{equation}
\begin{aligned}
\widehat{v} &= \vec{v}/\|\vec{v}\|\\
&= \vec{v}/\sqrt{1+1+4}\\ 
&= \vec{v}/\sqrt{6}
\end{aligned}
\end{equation}
We have
\begin{equation}
\begin{aligned}
\overrightarrow{PS}\times\widehat{v}
&= \frac{1}{\sqrt{6}}\begin{vmatrix}
\widehat{\textbf{\i}} & \widehat{\textbf{\j}} & \widehat{\textbf{k}}\\
3 & -1 & 2\\
-1 & -1 & -2
\end{vmatrix}\\
&= \frac{4\widehat{\textbf{\i}}+4\widehat{\textbf{\j}}-4\widehat{\textbf{k}}}{\sqrt{6}}
\end{aligned}
\end{equation}
This has its magnitude be
\begin{equation}
\|\overrightarrow{PS}\times\widehat{v}\|
= \frac{4\sqrt{3}}{\sqrt{6}}=2\sqrt{2},
\end{equation}
which describes the distance between our point $S$ and the given line.
\end{example}


\subsection{Constructing Planes}
%%
%% planes.tex
%% 
%% Made by Alex Nelson
%% Login   <alex@black-cherry>
%% 
%% Started on  Thu Jun 21 17:24:56 2012 Alex Nelson
%% Last update Thu Jun 21 17:25:10 2012 Alex Nelson
%%

\N{Determining a Plane From a Point and Normal Vector}
Given a vector
\begin{equation}
\vec{N} = A\widehat{\textbf{\i}}+b\widehat{\textbf{\j}}+C\widehat{\textbf{k}}
\end{equation}
and a point $P_{0}=(x_{0},y_{0},z_{0})$, there exists a unique
plane which is perpindicular to $\vec{N}$ and contains $P_{0}$. 

How? Well, let $P$ be an arbitrary point on the plane. Then the
vector $\overrightarrow{P_{0}P}$ would be parallel to the
plane. Being parallel to the plane implies its orthogonal to
$\vec{N}$:
\begin{equation}
\overrightarrow{P_{0}P}\cdot\vec{N}=0.
\end{equation}
This gives us an equation
\begin{equation}
\begin{aligned}
\overrightarrow{P_{0}P}\cdot\vec{N}&=
\bigl(
(x-x_{0})\widehat{\textbf{\i}}+(y-y_{0})\widehat{\textbf{\j}}+(z-z_{0})\widehat{\textbf{k}}
\bigr)\cdot(A\widehat{\textbf{\i}}+b\widehat{\textbf{\j}}-C\widehat{\textbf{k}})
\\
&=A(x-x_{0})+B(y-y_{0})+C(z-z_{0})=0.
\end{aligned}
\end{equation}
So any point $(x,y,z)$ lies on the plane if it satisfies this
equation. 

\N{Determine a Plane from Three Points}
Given three points $A$, $B$, $C$, find a plane containing these
points. 

We construct the vectors $\overrightarrow{AB}$ and
$\overrightarrow{AC}$. Take the cross product, which produces the
normal vector
\begin{equation}
\vec{N} = \overrightarrow{AB}\times\overrightarrow{AC}.
\end{equation}
If we write $A=(x_{0},y_{0},z_{0})$ and $\vec{N}=\langle
N_{1},N_{2},N_{3}\rangle$, then
\begin{equation}
N_{1}(x-x_{0})+N_{2}(y-y_{0})+N_{3}(z-z_{0}) = 0
\end{equation}
describes the plane. (It follows from our last construction of
the plane.)

\section{Curves}
\subsection{Curves, Classical Kinematics}
%%
%% curves.tex
%% 
%% Made by Alex Nelson
%% Login   <alex@black-cherry>
%% 
%% Started on  Fri Jun 29 11:50:28 2012 Alex Nelson
%% Last update Sat Jun 30 20:32:51 2012 Alex Nelson
%%
\N{Curves}
We are interested in describing the motion of my car. Well,
\emph{everyone} is interested in the motion of my car. How can we
describe it mathematically? 

First we approximate the car as a point. The point-like car moves
in time, so the value of its components are functions of
time. More precisely, the position of my car is 
\begin{equation}
\vec{r}(t) = \langle f(t),g(t),h(t)\rangle =
f(t)\widehat{\textbf{\i}} +
g(t)\widehat{\textbf{\j}} +
h(t)\widehat{\textbf{k}}
\end{equation}
where the functions $f(t)$, $g(t)$, and $h(t)$ are sometimes
called \emph{component functions}. Another way to think about
this is writing
\begin{equation}
\vec{r}\colon[0,1]\to\RR^{3}
\end{equation}
where $0\leq t\leq1$. 

Classical mechanics studies such curves under various
circumstances. We will discuss some notions of kinematics, and
study what it means to differentiate curves.

\begin{example}
Consider a point traveling in circular motion in the
$xy$-plane. What does this look like? 

Well, it's a paramteric curve, using trigonometric functions we write
\begin{equation}
\vec{r}(t) = \cos(t)\widehat{\textbf{\i}}+\sin(t)\widehat{\textbf{\j}}.
\end{equation}
This descrivbes an anti-clockwise circular motion with radius 1,
lying in the $xy$-plane.
\end{example}

\begin{example}
Suppose a particle travels along a parabolic curve, what does the
curve look like? We can write it explicitly as
\begin{equation}
\vec{r}(t) =
t\,\widehat{\textbf{\i}}+(t^{2}-1)\widehat{\textbf{\j}}
\end{equation}
This is precisely aa parabola.
\end{example}

\N{Calculus with Vector-Valued Functions}
We should recall the construction of the tangent line to a curve
$y=f(x)$ at a point $(x_{0},f(x_{0})=y_{0})$ had us write
\begin{equation}
t(h) = y_{0} + f'(x_{0}) \cdot h.
\end{equation}
When we consider the situation when we work with $\vec{r}(t)$
instead of a function $f(x)$. We have
$\vec{r}_{0}=\vec{r}(t_{0})$ be the base point for the tangent to
the curve, then we have
\begin{equation}
\vec{T}(h) = \vec{r}_{0} + \vec{r}'(t_{0})\cdot h
\end{equation}
The problem: what exactly is $\vec{r}'(t_{0})$?

\M
We can let $\vec{r}\colon(0,1)\to\RR^{3}$ (or more generally the
codomain can be $\RR^{n}$ for any positive integer $n\in\NN$). We
have
\begin{equation}
\frac{\D\vec{r}(t)}{\D t}=\vec{v}(t) = \lim_{\Delta t\to0}\frac{\vec{r}(t+\Delta t)-\vec{r}(t)}{\Delta t}
\end{equation}
describe the rate of change of the position vector $\vec{r}(t)$
with respect to time. What does this look like? Well, writiing out
\begin{equation}
\vec{r}(t) = \langle f(t),g(t),h(t)\rangle =
f(t)\widehat{\textbf{\i}} +
g(t)\widehat{\textbf{\j}} +
h(t)\widehat{\textbf{k}}
\end{equation}
we have
\begin{equation}
\frac{\D\vec{r}(t)}{\D t} = \langle f'(t), g'(t), h'(t)\rangle =
f'(t)\widehat{\textbf{\i}} +
g'(t)\widehat{\textbf{\j}} +
h'(t)\widehat{\textbf{k}}
\end{equation}
where primes denote differentiation with respect to time.

\begin{remark}
We can keep iterating this procedure to obtain higher order
derivatives of a curve.
\end{remark}

\N{Kinematics}
We have $\vec{r}(t)$ describe the position of a particle. The
velocity of the particle is a vector-valued function
\begin{equation}
\begin{aligned}
\vec{v}(t)
&=\lim_{\Delta t\to 0}\frac{\vec{r}(t+\Delta
  t)-\vec{r}(t)}{\Delta t}\\
&=\frac{\D\vec{r}(t)}{\D t}
\end{aligned}
\end{equation}
However, we also can consider the \emph{speed} or the magnitude
of the velocity
\begin{equation}
\|\vec{v}(t)\|=\frac{\D s}{\D t} = \begin{pmatrix}
\mbox{rate of change of distance}\\
\mbox{with respect to time}
\end{pmatrix}
\end{equation}
Observe the speed is a scalar quantity: it's just some function
of time. The velocity is a vector-valued function of time. 

We have one last kinematical quantity to consider: the
acceleration. This is just the rate of change of velocity with
respect to time:
\begin{equation}
\vec{a}(t) = \frac{\D\vec{v}(t)}{\D t} =
\frac{\D^{2}\vec{r}(t)}{\D t^{2}}
\end{equation}
Observe we can reconstruct the position from the velocity by
considering
\begin{equation}
\vec{r}(t) = \vec{r}(t_{0}) +
\int^{t}_{t_{0}}\frac{\D\vec{r}(\tau)}{\D\tau}\,\D\tau
\end{equation}
which when we consider $\vec{r}(t)=\langle x(t),y(t),z(t)\rangle$
we have the integral evaluated ``component-wise'':
\begin{equation}
\vec{r}(t) = \vec{r}(t_{0}) +
\left\langle \int^{t}_{t_{0}}\frac{\D x(\tau)}{\D\tau}\,\D\tau,
\int^{t}_{t_{0}}\frac{\D y(\tau)}{\D\tau}\,\D\tau,
\int^{t}_{t_{0}}\frac{\D z(\tau)}{\D\tau}\,\D\tau\right\rangle.
\end{equation}
We can similarly reconstruct velocity from acceleration.

\begin{example}
Consider the curve describing circular motion
\begin{equation}
\vec{r}(t) = \cos(t)\widehat{\textbf{\i}}
+\sin(t)\widehat{\textbf{\j}}
\end{equation}
What is its velocity vector, acceleration vector, and speed?

\emph{Solution}: We find its velocity
\begin{equation}
\begin{aligned}
\vec{v}(t) &= \frac{\D\vec{r}(t)}{\D t}\\
&=-\sin(t)(t)\widehat{\textbf{\i}}
+\cos(t)\widehat{\textbf{\j}}
\end{aligned}
\end{equation}
From this we can compute its speed as
\begin{equation}
\begin{aligned}
\|\vec{v}(t)\| &= \sqrt{\vec{v}(t)\cdot\vec{v}(t)}\\
&=\sqrt{\sin^{2}(t)+\cos^{2}(t)} = 1.
\end{aligned}
\end{equation}
The acceleration is precisely the derivative of the velocity
vector
\begin{equation}
\frac{\D\vec{v}(t)}{\D t}
= -\cos(t)\widehat{\textbf{\i}}-\sin(t)\widehat{\textbf{\j}}
\end{equation}
That concludes our example.
\end{example}

\begin{exercise}
Find the velocity vector, speed, and acceleration of the
parabolic curve $\vec{r}(t) =
t\,\widehat{\textbf{\i}}+(t^{2}-1)\widehat{\textbf{\j}}$
\end{exercise}

\begin{exercise}
Let $\vec{u}(t)$ and $\vec{v}(t)$ be differentiable vector-valued
functions of time. Prove or find a counter-example that
\begin{equation}
\frac{\D}{\D t}\bigl[\vec{u}(t)\times\vec{v}(t)\bigr]=
\frac{\D\vec{u}(t)}{\D t}\times\vec{v}(t) +
\vec{u}(t)\times\frac{\D\vec{v}(t)}{\D t}.
\end{equation}
\end{exercise}
\begin{exercise}
Calculate $\displaystyle\frac{\D}{\D t}\bigl[\vec{a}(t)\cdot\bigl(\vec{b}(t)\times\vec{c}(t)\bigr)\bigr]$
\end{exercise}


\section{Surfaces}
\section{Surfaces}

\M What is a surface? Intuitively, it's an ``$\RR^{2}$'' subset of
$\RR^{3}$. What does that even mean?

A set should be two-dimensional if it can be built out of pieces that
look like open sets of $\RR^{2}$, i.e., with two-dimensional patches of
``fabric'' we may ``sew'' together. Conceptually captured by this picture:
\begin{center}
  \includegraphics{img/surfaces.0}
\end{center}
The map like the one above gives ``coordinates'' to each point on a
surface. So a map like this is called a \emph{coordinate
patch}.\footnote{This will be made precise shortly, but caution should
be given: the literature is inconsistent on which way the arrow
goes. Some authors prefer taking the green patch of the surface, and
mapping it to some subset of $\RR^{2}$. It is a matter of convention,
and either choice is perfectly acceptable.}

\N{Possible Problems With Our Definition} We should pause a moment and
ponder if our notion of a surface is really well-defined, or if there
are some problems with it. The main issues we should think about:
\begin{enumerate}
\item The coordinates could be ``degenerate'', meaning that different
  values of $(u,v)$ correspond to the same point on the surface.

  \textsc{Solution:} We demand the coordinate patch be
  injective\footnote{Recall, a function $f\colon X\to Y$ is injective
  means for every $x_{1}$, $x_{2}\in X$ we have $f(x_{1})=f(x_{2})$ implies $x_{1}=x_{2}$.} to
  avoid this problem.
\item Even if $\chart{x}$ is injective, it could behave badly in other
  ways and not define a smooth surface. For example, the following
  ``surfaces'' are too ``pointy'' to be smooth:
  \begin{center}
    \includegraphics{img/surfaces.1} \includegraphics{img/surfaces.2}
  \end{center}

  \textsc{Solution:} Require that $\chart{x}$ be \emph{regular}.
\end{enumerate}

\begin{definition}\label{defn:euclidean-space:tangent-map}
Given a map $F\colon \RR^{m}\to\RR^{n}$ (suppose $m\leq n$), its
\define{Tangent Map} at $\vec{p}\in\RR^{m}$
\begin{equation*}
F_{*\vec{p}}\colon\T_{\vec{p}}\RR^{m}\to\T_{F(\vec{p})}\RR^{n}
\end{equation*}
is defined as follows: given any $\vec{v}_{\vec{p}}\in\T_{\vec{p}}\RR^{m}$,
pick some curve $\alpha\colon I\to\RR^{m}$ such that $\alpha(0)=\vec{p}$
and $\alpha'(0) = \vec{v}_{\vec{p}}$. Then define
\begin{equation}
F_{*\vec{p}}(\vec{v}_{\vec{p}}) = \left.\frac{\D}{\D t}F\bigl(\alpha(t)\bigr)\right|_{t=0}.
\end{equation}
\begin{center}
  \includegraphics{img/surfaces.3}
\end{center}
\end{definition}

\begin{remark}
We should intuitively think of $F_{*\vec{p}}$ as ``the best linear
approximation to $F$ at $\vec{p}$''.
\end{remark}

\begin{remark}
This definition does not depend on choice of the curve $\alpha$.
\end{remark}

\begin{definition}
A map $F\colon\RR^{m}\to\RR^{n}$ is \define{Regular} if for every
$\vec{p}\in\RR^{m}$ we have $F_{*\vec{p}}$ be injective.
\end{definition}

\begin{remark}
%The intuition is that regular maps act like infinitesimals on tangent vectors.
This is a good definition, because if $\alpha$ is a regular curve, and
$F$ is a regular map, then the composition $F\circ\alpha$ is a regular
curve. Composing regular stuff together gives us something regular.
\end{remark}

\begin{definition}
A \define{(Coordinate) Chart} in $\RR^{3}$ is an injective regular map
$\chart{x}\colon D\to\RR^{3}$ where $D\subset\RR^{2}$ is some open
subset called the \define{Patch}.

Further, we call a chart \define{Proper} if $\chart{x}^{-1}\colon\chart{x}(D)\to\RR^{2}$
is continuous.

We may abuse language, and refer to the $D$ as the patch, and
$\chart{x}$ as the chart or parametrization. Technically, the local
coordinates refer to the components of the vector-valued function
$\chart{x}^{-1}\colon\chart{x}(D)\to D$ mapping a patch of our surface
to Euclidean space (the ``space of parameters'').
\end{definition}

\begin{remark}[Abuse of language]
Again, just to reiterate, people mix up what they're referring to when
using the terms ``chart'' and ``patch''. Undoubtedly \emph{we} will
too. 
\end{remark}

\begin{remark}
We must stress the importance of a patch $\chart{x}\colon D\to\RR^{3}$
being regular, which means for any $(u,v)\in D$, the map
\begin{equation}
\chart{x}_{*}\colon\T_{(u,v)}\RR^{2}\to\T_{\chart{x}(u,v)}\RR^{3}
\end{equation}
is injective.
\end{remark}

\begin{remark}
``Proper'' patches convey topological information. 
\end{remark}

\begin{remark}
The image of any coordinate patch gives an example of a surface.
\end{remark}

\M
Most surfaces cannot be covered by one coordinate patch. The famous
example: any coordinates on a sphere is degenerate around the
poles. Consequently, we need to use a set of patches to define a
surface.

\begin{definition}
Given a subset $M\subset\RR^{3}$ and a point $\vec{p}\in M$, a
\define{Neighborhood} of $\vec{p}$ is a set consisting of all points in
$M$ whose Euclidean distance to $\vec{p}$ is less than $\varepsilon$,
for some $\varepsilon>0$ [fixed for the neighborhood].
\end{definition}

\begin{definition}
A \define{Surface} in $\RR^{3}$ is a subset $M\subset\RR^{3}$ such that
for each point $\vec{p}\in M$ there exists a neighborhood $N$
of $\vec{p}$ in $M$ and a proper patch $\chart{x}\colon D\to\RR^{3}$
such that $N\subset\chart{x}(D)\subset M$.

\begin{center}
  \includegraphics{img/surfaces.4}
\end{center}
\end{definition}

\medbreak
\begin{remark}
We don't want self-intersecting surfaces, we want to avoid the following
doodle:
\begin{center}
\includegraphics{img/surfaces.5}
\end{center}
This is because there's no way to have a neighborhood ``near the
intersection''. It would locally look like:
\begin{center}
\includegraphics{img/surfaces.6}
\end{center}
Why is this a problem?\footnote{It's not Hausdorff, that's the problem.} This is a neighborhood of some point on the
intersection, say $N(\vec{p})$. We would like to find a chart
$\chart{x}\colon D\to\RR^{3}$ such that $N(\vec{p})\subset\chart{x}(D)$.
But this is impossible, because $\chart{x}(D)$ couldn't contain an
intersection (thanks to topology).
\end{remark}

\N{Determining if a Patch is Regular}
How do we even determine if a patch is regular, anyways?
Well, if $F\colon\RR^{m}\to\RR^{n}$ were regular at $\vec{p}\in\RR^{m}$,
then $F_{*\vec{p}}$ is injective. We know from linear algebra this means
the dimension of the image equals the dimension of the domain, i.e.,
\begin{equation}
\dim(\T_{F(\vec{p})}\RR^{n})=\dim(\T_{\vec{p}}\RR^{m}).
\end{equation}
This is equivalent to saying that the rank of $F_{*\vec{p}}$ is of
maximal rank for every $\vec{p}\in\RR^{m}$.
In other words, we know a patch $\chart{x}\colon D\to\RR^{3}$ is regular
if for each $\vec{p}\in D$ we have $\chart{x}_{*\vec{p}}$ be of maximal
rank. Our strategy for checking this will be to find some frame field
$\vec{e}_{1}$, $\vec{e}_{2}$ defined on $D$ and some frame field
$E_{1}$, $E_{2}$, $E_{3}$ on $\chart{x}(D)\subset\RR^{3}$. Then we will
express $\chart{x}_{*}$ as a $2\times3$ matrix, and we could use row
reduction to find the rank.

What we do is we consider the following diagram:
\begin{center}
  \includegraphics{img/surfaces.7}
\end{center}
Consider a curve $\alpha\colon I\to\RR^{2}$ such that
$\alpha(0)=\vec{p}$ and $\alpha'(0)=\vec{e}_{1}$ --- i.e., the curve
points in the $u$-direction. We compute
$\chart{x}_{*}(\vec{e}_{1})$ to get the components in the first column of
the matrix representing $\chart{x}_{*}$, and we find another curve
pointing in the $\vec{e}_{2}$ (i.e., in the $v$) direction to find the
second column of the matrix of $\chart{x}_{*}$.

We find,
\begin{subequations}
  \begin{align}
    \chart{x}_{*}(\vec{e}_{1}) &= \left.\frac{\D}{\D t}\chart{x}\bigl(\alpha(t)\bigr)\right|_{t=0}\\
&= \left.\left(\frac{\partial x^{1}}{\partial\alpha^{1}}\frac{\D\alpha^{1}}{\D t} +
\frac{\partial x^{1}}{\partial\alpha^{2}}\frac{\D\alpha^{2}}{\D t},
\frac{\partial x^{2}}{\partial\alpha^{1}}\frac{\D\alpha^{1}}{\D t} +
\frac{\partial x^{2}}{\partial\alpha^{2}}\frac{\D\alpha^{2}}{\D t},
\frac{\partial x^{3}}{\partial\alpha^{1}}\frac{\D\alpha^{1}}{\D t} +
\frac{\partial x^{3}}{\partial\alpha^{2}}\frac{\D\alpha^{2}}{\D t}\right)
\right|_{t=0}\\
&=\left(\frac{\partial x^{1}}{\partial u},
\frac{\partial x^{2}}{\partial u},
\frac{\partial x^{3}}{\partial u}\right)=\sum^{3}_{j=1}\frac{\partial x^{j}}{\partial u}E_{j}=:\chart{x}_{u}.
  \end{align}
\end{subequations}
Similarly, we find
\begin{equation}
\chart{x}_{*}(\vec{e}_{2})=\left(\frac{\partial x^{1}}{\partial v},
\frac{\partial x^{2}}{\partial v},
\frac{\partial x^{3}}{\partial v}\right)=\sum^{3}_{j=1}\frac{\partial x^{j}}{\partial v}E_{j}=: \chart{x}_{v}.
\end{equation}
We call the quantities $x_{u}$ and $x_{v}$ \define{Partial Velocities}.
Hence, if
$\vec{w}_{\vec{p}}=(w^{1},w^{2})_{\vec{p}}\in\T_{\vec{p}}\RR^{2}$, then
\begin{equation}
\chart{x}_{*}\begin{pmatrix}w^{1}\\w^{2}
\end{pmatrix}=\begin{pmatrix}
\displaystyle\frac{\partial x^{1}}{\partial u} &\displaystyle\frac{\partial x^{1}}{\partial v}\\
\displaystyle\frac{\partial x^{2}}{\partial u} &\displaystyle\frac{\partial x^{2}}{\partial v}\\
\displaystyle\frac{\partial x^{3}}{\partial u} &\displaystyle\frac{\partial x^{3}}{\partial v}
\end{pmatrix}\begin{pmatrix}w^{1}\\w^{2}
\end{pmatrix}.
\end{equation}
Now that we have expressed $\vec{x}_{*}$ as a matrix, we just need to
check there are at least 2 linearly independent rows, which can be done
by row reduction. Enough humourless logic, let us look at some examples.

\begin{example}
Consider the unit sphere $S^{2}=\{(x,y,z)\in\RR^{3}\mid x^{2}+y^{2}+z^{2}=1\}$
in three-dimensions. We have a path formed by ``bending'' the 
open unit disc $D^{2}=\{(x,y)\in\RR^{2}\mid x^{2}+y^{2}<1\}$.
More explicitly,
\begin{subequations}
\begin{equation}
\chart{x}\colon D^{2}\to\RR^{3}
\end{equation}
defined by
\begin{equation}
\chart{x}(u,v) = (u, v, \sqrt{1 - u^{2} - v^{2}}).
\end{equation}
\end{subequations}
This is just one possible patch, we could consider another by taking the
third component to be $-\sqrt{1-u^{2}-v^{2}}$, and we can consider
others by swapping the third component with either the first or second
components.

Now, our patch is clearly injective. If you do not believe it, then just
examine $\chart{x}(u_{1},v_{1})=\chart{x}(u_{2},v_{2})$; the first two
components reads $u_{1}=u_{2}$ and $v_{1}=v_{2}$. It follows that
$(u_{1},v_{1})=(u_{2},v_{2})$ and moreover $\chart{x}$ is injective.

But is our patch \emph{regular}? We can find the matrix of
$\chart{x}_{*}$ relative to the canonical frame fields, which reads
\begin{equation}
\chart{x}_{*} = \begin{bmatrix}1 & 0\\0 & 1\\\mbox{stuff}_{1} & \mbox{stuff}_{2}
\end{bmatrix}.
\end{equation}
Since the top $2\times 2$ submatrix is the identity matrix, it follows
that $\chart{x}_{*}$ has rank 2. Hence our patch is regular.
\end{example}

\begin{remark}
This example is a special case of a more general fact: if we have a
smooth function $f\colon D\to\RR$, and we consider its graph
$\Gamma(f)=\{(x,y,f(x,y))\in\RR^{3}\mid(x,y)\in D\}$ (or more generally,
for any $D\subset\RR^{n}$, we have
$\Gamma(f)=\{(\vec{x},f(\vec{x}))\in\RR^{n+1}\mid\vec{x}\in D\}$), then
this graph is a patch of a surface.

Algebraic geometry generalizes this further, by studying the zero sets
of functions $\{\vec{x}\in\RR^{n}\mid f(\vec{x})=0\}$. These generalize
the notion of surfaces. A lot of differential geometry is generalized in
this manner, it's very deep and profound.
\end{remark}

\begin{example}[Surface of revolution]
%% Suppose we have a regular curve $\alpha\colon I\to\RR^{2}$.
%% We can embed it in three-dimensions by mapping it into the $xy$-plane:
%% \begin{equation}
%% t\mapsto(\alpha^{1}(t),\alpha^{2}(t),0).
%% \end{equation}
Undergraduates are taught in integral calculus of a single variable
about a surface of revolution by taking a curve $y=f(x)$, then sweeping
it out around the $x$-axis, in the sense that
\begin{equation}
y^{2}+z^{2}=f(x)^{2}.
\end{equation}
This yields a parametrization in terms of $x$ and $\theta$. Our patch
would be
\begin{equation}
\chart{x}(x,\theta) = \left(x, f(x)\cos(\theta),f(x)\sin(\theta)\right).
\end{equation}
If $f$ is a regular curve, then we have a regular surface.
\end{example}

\N{What Patches Give Us}
The basic idea for patches is that they let us transfer data on the
surface $M$ to data (of various kinds) on the domain $D\subset\RR^{2}$
where we know how to do calculus. What kinds of things do patches give
us?
\begin{enumerate}
\item \textsc{``Local coordinates'' on $M$.} Grid lines in $D$ are paths like
  $\alpha(t)=(u_{0},v_{0}+t)$ which pass through the point
  $(u_{0},v_{0})$. These induce grid lines on $M$ by
  $\chart{x}\circ\alpha(t)=\chart(u_{0},v_{0}+t)$. (Although these
  describe grid lines of constant $u_{0}$, we can form grid lines of
  constant $v_{0}$ by examining $\chart(u_{0}+t,v_{0})$ for example.)
  \begin{center}
    \includegraphics{img/surfaces.8}
  \end{center}
\item \textsc{Convenient ways to get tangent vectors on $M$.} We have
  regularity map basis vectors (frame fields) to basis vectors (frame
  fields) by $\chart{x}_{*\vec{p}}\colon\T_{\vec{p}}\RR^{2}\to\T_{\chart{x}(\vec{p})}M$.
  Regularity guarantees we get a whole tangent plane, not just a line.
  \begin{center}
    \includegraphics{img/surfaces.9}
  \end{center}
\item \textsc{``Local'' frame fields on $M$.} This is given to us by the
  partial velocities $\chart{x}_{u}(u,v)$ and $\chart{x}_{v}(u,v)$.
  \begin{center}
    \includegraphics{img/surfaces.10}
  \end{center}
\item \textsc{Convenient ways to compute ``the normal vector'' to a surface.}
  Since we have found $\chart{x}_{u}(u,v)$ and $\chart{x}_{v}(u,v)$ are
  frame fields for the tangent vectors on the surface, we can consider
  their cross product
  $\vec{n}=\chart{x}_{u}(u,v)\times\chart{x}_{v}(u,v)$ which is normal
  to the surface.
  We can do this globally only for ``orientable'' surfaces (e.g., not
  for the M\"{o}bius strip).
  \begin{center}
    \includegraphics{img/surfaces.11}
  \end{center}
\end{enumerate}


\phantomsection{}
\subsection*{Exercises}
\addcontentsline{toc}{subsection}{Exercises}

%% This roughly is before \S2.2 in Arcade's transcription of the notes.

%% At this point, homework 5 would be due [May 7, 2008].
%% It consisted of the following
%% exercises from O'Neill's \emph{Elementary Differential Geometry}
%% (Revised second ed.),
%% \begin{itemize}
%% \item 2.7 \# 1, 4, 5
%% \item 2.8 \# 2, 4
%% \end{itemize}

Here are some review questions, to make sure you don't forget too
quickly what we learned from section 2. (This is an experiment, let me
know if you hate this technique. It probably won't happen again in these
notes, though.)

\begin{enumerate}
  % 2.7 #4 
\item Recall Parabolic coordinates --- we have $0\leq u<\infty$,
$0\leq v<\infty$, and $0\leq\varphi<2\pi$, and the Cartesian coordinates
  are parametrized as
  \begin{subequations}
    \begin{align}
      x &= uv\cos(\varphi)\\
      y &= uv\sin(\varphi)\\
      z &= \frac{1}{2}(u^{2}-v^{2}).
    \end{align}
  \end{subequations}
  \begin{enumerate}
  \item Compute the Parabolic frame field $E_{1}$, $E_{2}$, $E_{3}$
  \item Compute the connection forms for the parabolic frame field.
  \end{enumerate}
% 2.7 # 5
\item Let $E_{1}$, $E_{2}$, $E_{3}$ constitute a frame field and
  $W=\sum_{j}f_{j}E_{j}$. Let $V$ be an arbitrary vector field.
  Prove or find a counter-example: the covariant derivative satisfies,
\begin{equation}
\nabla_{V}W = \sum_{j}\left(V[f_{j}] + \sum_{i}f_{i}\omega_{ij}[V]\right)E_{j}.
\end{equation}
\item Check the structure equations for the parabolic frame field.
% 2.8 # 4, frame fields on \RR^2. Prove there is an angle \theta such that
  % E_{1} = \cos(\theta)U_{1} + \sin(\theta)U_{2}
  % E_{2} = -\sin(\theta)U_{1} + \cos(\theta)U_{2}
  % (a) Express the connection form and dual 1-forms in terms of
  % $\theta$ and the natural coordinates $x$ and $y$
  % (b) What are the structural equations in this case? Check the
  % results of (a) satisfy them.
\end{enumerate}
\vfill\eject

% calculus on a surface
\subsection{Calculus on a Surface}

OK, we're on the home stretch now. We've generalized calculus in
$\RR^{n}$ using the machinery of tangent vectors and differential forms,
talked about curves and surfaces. Now our goal is to figure out how to
do calculus on surfaces. Ready? Let's go!

\M
Our goal is to completely generalize what we know about calculus on
$\RR^{2}$ to any surface.

This means we need to define: tangent vectors, vector fields, frame
fields, one-forms, differential forms, smooth functions, covariant
derivatives, etc., \emph{on a surface}.\footnote{The generalization to arbitrary
manifolds will be simple.} We'll use this to study various surfaces and
properties they have.

\emph{What's most fundamental in mathematics is making the correct definitions.}

\begin{figure}[h]
\centering
  \includegraphics{img/surfaces.12}
\caption{Intuition of a function $f\colon M\to\RR^{n}$ being smooth}\label{fig:surfaces:smooth-function-from-surface}
\end{figure}

\begin{definition}\label{defn:surfaces:smooth-function-from-surface}
Let $M\subset\RR^{3}$ be a surface, $f\colon M\to\RR^{n}$ be some
function on the surface.
We say that $f$ is \define{Smooth} if, for every patch $\chart{x}\colon D\to M$
(where $D\subset\RR^{2}$ is open),
\begin{equation}
f\circ\chart{x}\colon D\to\RR^{n}
\end{equation}
is smooth in the usual sense, i.e., $f\circ\chart{x}\in C^{\infty}(D)$.
This is schematically doodled in Figure~\ref{fig:surfaces:smooth-function-from-surface}.
\end{definition}

\begin{figure}[h]
\centering
  \includegraphics{img/surfaces.13}
\caption{Intuition of a function $f\colon \RR^{n}\to M$ being smooth}\label{fig:surfaces:smooth-function-to-surface}
\end{figure}

\begin{definition}\label{defn:surfaces:smooth-function-to-surface}
Let $M\subset\RR^{3}$ be a surface, $f\colon\RR^{n}\to M$ be a function
to the surface. We call $f$ \define{Differentiable} if, for every patch
$\chart{x}\colon D\to M$, the map
\begin{equation}
\chart{x}^{-1}\circ f\colon\mathcal{E}\to D,
\end{equation}
where $\mathcal{E}=f^{-1}(\chart{x}(D))\subset\RR^{n}$ is the preimage
of the patch under $f$, is a smooth ($C^{\infty}$) function. Note: we \emph{do
not} require $f(\RR^{n})=M$.

The intuition of this definition is doodled in Figure~\ref{fig:surfaces:smooth-function-to-surface}.
\end{definition}

\begin{remark}
This requires a bit of explanation. Consider the preimage of
$\chart{x}(D)$ under $f$, i.e., the set of points $\vec{x}\in\RR^{n}$
which $f$ maps into the image of the chart $\mathbf{x}(D)$; call this
set $\mathcal{E} = \{\vec{x}\in\RR^{n}\mid f(\vec{x})\in\chart(D)\}$.
We want this to be an open set for topological reasons (this makes $f$
continuous, a necessary prerequisite for derivatives). Now we could
consider $f|_{\mathcal{E}}\colon\mathcal{E}\to M$ by restricting $f$.
We know its image will be within the image of the chart, so we
then take the preimage of $f(\mathcal{E})\subset M$ under the chart
$\chart{x}$ to produce the mapping
$\chart{x}^{-1}\colon f|_{\mathcal{E}}(\mathcal{E})\to D$.
But this is the same as considering the composition
$\chart(x)^{-1}\circ f|_{\mathcal{E}}\colon\mathcal{E}\to D$.
The restriction of $f$ to $\mathcal{E}$ has been purely a crutch, the
preimage of $\chart{x}$ will restrict the composite function for us. So
we arrive at our definition.
\end{remark}

\begin{remark}
As a quick check, we could consider $M=\RR^{3}$ with $\chart{x}=\id$
being the identity function. Then $f\colon\RR^{n}\to\RR^{3}$ being
differentiable is the same as $f\in C^{\infty}(\RR^{n},\RR^{3})$. This
is good! Our definition of differentiable functions to surfaces
coincides with our pre-existing definition of differentiable
multivariate vector-valued functions.
\end{remark}

\begin{example}
Let $M\subset\RR^{3}$ be a surface, and consider a [smooth] path
$\alpha\colon I\to\RR^{3}$
such that the curve lies on the surface $\alpha(I)\subset M$.
For any patch $\chart{x}\colon D\to M$, we could consider 
the interval $J=\alpha^{-1}(\chart{x}(D))$ given by the preimage of the
curve which lies in $\chart{x}(D)$. The situation is as doodled below:
\begin{center}
  \includegraphics{img/surfaces.14}
\end{center}
Proving $\alpha$ is smooth on $M$ amounts to proving, for every patch
$\chart{x}\colon D\to M$ such that $\chart{x}(D)$ contains some part of
the path, the restriction $\alpha|_{J}$ is smooth in the preimage of the
chart in the familiar way. 
\end{example}

\N{Smooth Functions Between Surfaces}\label{defn:smooth-functions-between-surfaces}
Suppose now we have two surfaces $M_{1}$ and $M_{2}$. We can construct a
notion of a smooth function $f\colon M_{1}\to M_{2}$ between these
surfaces. The solution is to cheat.

Given arbitrary patches $\chart{x}\colon D\to M_{1}$ on $M_{1}$
and $\chart{y}\colon E\to M_{2}$ on $M_{2}$, we have the situation as
doodled below:
\begin{center}
  \includegraphics{img/surfaces.15}
\end{center}
Now we have to make sense of $f$. We first take the preimage of
$\chart{y}(E)$ under $f$, which may or may not intersect $\chart{x}(D)$
on $M_{1}$. If it doesn't, then we're in the trivial situation, and
everything works out fine. So let's examine the exciting case where
$f^{-1}\left(\chart{y}(E)\right)\cap\chart{x}(D)\neq\emptyset$. This
gives us the etched region doodled below:
\begin{center}
  \includegraphics{img/surfaces.16}
\end{center}
We can pull back $f^{-1}\left(\chart{y}(E)\right)$ to the patch $D$
using the preimage of $\chart{x}$, which produces the following
situation (with the hatched region indicating the $\chart{x}^{-1}\circ f^{-1}$
preimage):
\begin{center}
  \includegraphics{img/surfaces.17}
\end{center}
It looks like we're making this more complicated, doesn't it? There is
one thing we have not yet exploited: we can move \emph{forward} as well
as backward. If we start with $\chart{x}|_{f^{-1}(E)}^{-1}(D)$ the
portion of the patch which, when charted onto $M_{1}$ will be mapped by
$f$ to part of $\chart{y}(E)$, then move forward along these lines, we
end up with a subset
\begin{equation}
(f\circ\chart{x})\left(\chart{x}|_{f^{-1}(E)}^{-1}(D)\right)\subset\chart{y}(E).
\end{equation}
We can take its preimage under $\chart{y}$ to get a subset in
$E\subset\RR^{2}$. This gives us a mapping, however, from $D$ to $E$:
\begin{equation}
\chart{y}^{-1}\circ f\circ\chart{x}\colon D\to E,
\end{equation}
which is a function where we can sensibly discuss smoothness and
derivatives. In pictures, we get the situation as follows:
\begin{center}
  \includegraphics{img/surfaces.18}
\end{center}
The induced function is drawn with a dashed arrow, and it is the one
\emph{we know} how to determine if it's smooth or not (because it's a
function of an open subset in $\RR^{2}$ to an open subset in $\RR^{2}$).
And if we do this for every possible pair of patches on $M_{1}$ and
$M_{2}$, we end up verifying $f$ is smooth and differentiable.

More precisely, we have $f\circ\chart{x}$ be smooth function to $M_{2}$
in the sense of Definition~\ref{defn:surfaces:smooth-function-to-surface}
We also have $\chart{y}^{-1}\circ f$ be a smooth function on $M_{1}$, in
the sense of Definition~\ref{defn:surfaces:smooth-function-from-surface}.
Since this is done for every possible charts on $M_{1}$ and $M_{2}$, we
conclude that $f\colon M_{1}\to M_{2}$ is smooth.
\vfill\eject

\phantomsection{}
\subsection*{Exercises}
\addcontentsline{toc}{subsection}{Exercises}

%% This roughly is before \S2.2 in Arcade's transcription of the notes.

%% At this point, homework 6 would be due [May 26(?), 2008].
%% It consisted of the following
%% exercises from O'Neill's \emph{Elementary Differential Geometry}
%% (Revised second ed.),
%% \begin{itemize}
%% \item 4.2 \# 2
%% \item 4.3 \# 2, 4
%% \end{itemize}

\begin{enumerate}
\item Partial velocities $\chart{x}_{u}$, $\chart{x}_{v}$ are defined for
  an arbitrary mapping $\chart{x}\colon D\subset\RR^{2}\to\RR^{3}$, so
  we can consider the [real-valued] functions
  \begin{equation}
E=\chart{x}_{u}\cdot\chart{x}_{u},\quad
F=\chart{x}_{u}\cdot\chart{x}_{v},\quad
G=\chart{x}_{v}\cdot\chart{x}_{v}
  \end{equation}
  on $D$.
\begin{enumerate}
\item Prove $\|\chart{x}_{u}\times\chart{x}_{v}\|^{2} = EG-F^{2}$.
\item Prove $\chart{x}$ is regular if and only if $EG-F^{2}$ is never zero.
\end{enumerate}
% 4.3 # 2, 4  
\end{enumerate}

\begin{framed}
\begin{quotation}
  \begin{center}
    {\large\bfseries Homework: Stereographic Projection}\medbreak

    \textbf{Mathematics 116 --- Differential Geometry}

    Spring 2008

    Derek Wise
  \end{center}
  
Stereographic projection gives a nice coordinate patch on the unit
sphere $x^{2}+y^{2}+z^{2}=1$. It is defined by
\begin{equation*}
\vec{x}\colon\RR^{2}\to\RR^{3}
\end{equation*}
where $\vec{x}(u,v)$ is defined to be the unique point in $\RR^{3}$ that
lies both on the unit sphere and the ray from $(0,0,1)$ through $(u,v,0)$.

\begin{enumerate}
\item Derive an explicit formula for $\vec{x}(u,v)$. [Hint: use a
  parameterization of the line, and solve for the time $t$ when it
  passes through the unit sphere.]
\item Find the matrix of the tangent map $\vec{x}_{*}$, relative to the
  natural frame fields on $\RR^{2}$ and $\RR^{3}$.
\item Prove that $\vec{x}$ is a patch.
\item Show that $\vec{x}$ is \emph{conformal}, meaning that it preserves
  angles. That is, given a pair of tangent vectors $w_{p}$, $z_{p}$ at
  the same point in $\RR^{2}$, show that the angle between them (defined
  by the dot product in $\RR^{2}$) is the same as the angle between
  $\vec{x}_{*}(w_{p})$ and $\vec{x}_{*}(z_{p})$ (defined by the dot
  product in $\RR^{3}$).
\end{enumerate}
\end{quotation}
\end{framed}

\vfill\eject

\subsection{Vectors on Surfaces}

\M We require $\chart{x}\colon D\to\RR^{3}$ be smooth ($C^{\infty}$) and
$\chart{x}^{-1}\colon D\gets\chart{x}(D)$ be continuous. We require an
additional property for the inverse to be differentiable.

\begin{definition}
Let $M$ be a surface, let $\vec{p}\in M$ be some point.
We define the \define{Tangent Space} to $\vec{p}$ in $M$, denoted
$\T_{\vec{p}}M$, is the set of all vectors
$\vec{v}_{\vec{p}}\in\T_{\vec{p}}\RR^{3}$ such that
$\vec{v}_{\vec{p}}=\alpha'(0)$ for some curve on the surface $\alpha\colon I\to M$.
\end{definition}

\N{Base points are important}
In $\RR^{2}$, we often just ignored the point of tangency and pretended
that $\T_{\vec{p}}\RR^{2}$ and $\T_{\vec{q}}\RR^{2}$ are the same just
by sliding $\vec{p}$ to $\vec{q}$. But for general surfaces (of which
$\RR^{2}$ is just the most boring example), we cannot do this. There is
no way to slide $\T_{\vec{p}}M$ to $\T_{\vec{q}}M$ without embedding $M$
into $\RR^{2}$.
\begin{center}
  \includegraphics{img/surfaces.19}
\end{center}

\begin{definition}
A \define{Vector Field} $V$ on a surface $M$ is an assignment to each
point $\vec{p}\in M$ a vector $V(\vec{p})\in\T_{\vec{p}}M$.
\end{definition}

\M
Originally, we defined the derivative of $f$ in the direction of some tangent
vector $\vec{v}_{\vec{p}}$ as
\begin{equation}
\vec{v}_{\vec{p}}[f] = \left.\frac{\D}{\D t}f(\vec{p}+t\vec{v})\right|_{t=0}.
\end{equation}
This captures the information of how much $f$ changes in the direction
of $\vec{v}$ (at base-point $\vec{p}$). There's a problem generalizing
this to a surface: it doesn't work if $f$ is defined only on $M$. Why
not? Well, the line $\vec{p}+t\vec{v}$ will leave the surface, and $f$
is undefined off the surface, so we're out of luck.

But later we proved, if $\alpha\colon I\to\RR^{n}$ passes through
$\vec{p}=\alpha(0)$ and it has velocity $\vec{v}_{\vec{p}}=\alpha'(0)$
there, then we could define the directional derivative as:
\begin{equation}
\vec{v}_{\vec{p}}[f] = \left.\frac{\D}{\D t}f(\alpha(t))\right|_{t=0}.
\end{equation}
So if $\alpha\colon I\to M$ has initial position $\alpha(0)=\vec{p}$ and
initial velocity $\alpha'(0)=\vec{v}_{\vec{p}}\in\T_{\vec{p}}M$, and
if $f\colon M\to\RR$ is smooth, then we can define the directional
derivative of a function on our surface by
\begin{equation}
\vec{v}_{\vec{p}}[f] = \left.\frac{\D}{\D t}f(\alpha(t))\right|_{t=0}.
\end{equation}
This is independent of the choice of such $\alpha$.

If we do this at every point, we can differentiate functions with
respect to vector fields using
\begin{equation}
V[f](\vec{p}) = V(\vec{p})[f].
\end{equation}
This works out perfectly.

\begin{remark}[Boring]
This should be boring, because we defined things in a clever way.
Generalizations follow easily once we have the right definitions.
So if you find this boring, good: it means you have a grasp of the
concepts underlying the definitions.
\end{remark}
\vfill\eject

\subsection{Differential Forms on Surfaces}

\M A one-form $\phi$ on $M$ assigns to each point $\vec{p}\in M$ a
covector on $\T_{\vec{p}}M$, i.e., a linear function
\begin{equation}
\phi_{\vec{p}}\colon \T_{\vec{p}}M\to\RR.
\end{equation}
The most important examples: let $f\colon M\to\RR$ be a smooth function,
then $\D f$ is a one-form given by
\begin{equation}
\D f[\vec{v}_{\vec{p}}] = \vec{v}_{\vec{p}}[f],
\end{equation}
for every $\vec{v}_{\vec{p}}\in\T_{\vec{p}}M$.
But let us see what the zoo of differential forms becomes on a surface.

\N{Zero-Forms}The 0-forms are just smooth functions on $M$, i.e.,
functions like $\phi\colon M\to\RR$ such thatfor every patch
$\chart{x}\colon D\to M$, the function $\phi\colon\chart{x}\colon D\to\RR$
is smooth.

\N{One-Forms}
The 1-forms are defined just like in Euclidean space. A
\define{One-Form} $\phi$ is a linear map at each point $\vec{p}\in M$
taking tangent vectors to real numbers
\begin{equation}
\phi_{\vec{p}}\colon\T_{\vec{p}}M\to\RR
\end{equation}
in a linear way, and taking vector fields to functions
\begin{equation}
\phi\colon\Vect(M)\to C^{\infty}(M).
\end{equation}
Given a zero-form $f$, the differential $\D f$ is the one-form given by
\begin{equation}
\D f[V] = V[f],
\end{equation}
for any vector field $V\in\Vect(M)$.

\N{Two-Forms} Now we have something slightly different. But it tells us
what 2-forms \emph{do}. A 2-form $\eta$ on $M$ is a map at each point
$\vec{p}\in M$ that takes an \emph{ordered pair} of tangent vectors and
gives a number, that is to say,
\begin{equation}
\eta_{\vec{p}}\colon\T_{\vec{p}}M\times\T_{\vec{p}}M\to\RR
\end{equation}
such that
\begin{enumerate}
\item Antisymmetry: $\eta(\vec{v},\vec{w}) = -\eta(\vec{w},\vec{v})$
\item Linearity in first slow: for any $a,b\in\RR$,
  $\eta(a\vec{u}+b\vec{v},\vec{w}) = a\eta(\vec{u},\vec{w})+b\eta(\vec{v},\vec{w})$.
\end{enumerate}
It's easy to prove from these two properties that a 2-form is also
linear in the second slot; that is to say, it's \emph{bilinear}.

If we use $\eta$ at every point, we get a mapping
\begin{equation}
\eta\colon\Vect(M)\times\Vect(M)\to C^{\infty}(M).
\end{equation}

\N{But\dots the wedge product?}
Earlier we defined 2-forms in terms of the formal wedge product. Let us now
endeavour to produce a definition of the wedge product for differential
forms on a surface which is consistent with how we defined 2-forms.

Let $\phi$, $\psi$ be two 1-forms on $M$. We want to make a 2-form out
of them, and call it $\phi\wedge\psi$ (and make it a mapping
$\Vect(M)\times\Vect(M)\to C^{\infty}(M)$). The most obvious thing we
could try is,
\begin{equation}
(\phi\wedge\psi)(V,W) = \phi(V)\psi(W).
\end{equation}
Does it work? No, not by a long shot, since
\begin{equation}
(\phi\wedge\psi)(V,W)=\phi(V)\psi(W)\neq-\phi(W)\psi(V)\mbox{ in general}.
\end{equation}
Let us try
\begin{equation}
(\phi\wedge\psi)(V,W)\stackrel{???}{=}\phi(V)\psi(W)-\phi(W)\psi(V).
\end{equation}
Does it work?

We can see it is antisymmetric, since
\begin{subequations}
  \begin{align}
    (\phi\wedge\psi)(V,W)=\phi(V)\psi(W)-\phi(W)\psi(V)\\
    &=-\phi(W)\psi(V)-(-\phi(V)\psi(W))\\
    &=-(\phi\wedge\psi)(W,V),
  \end{align}
\end{subequations}
which is a relief. So this is possibly a good definition.

Now the real moment of truth: is it linear in the
first slot? We have something stronger than \emph{mere} linearity, it's
linear with respect to arbitrary smooth function $f,g\in C^{\infty}(M)$,
we have
\begin{equation}
(\phi\wedge\psi)(fU + gV,W) = f(\phi\wedge\psi)(U,W) + g(\phi\wedge\psi)(V,W).
\end{equation}
This is awesome!

And what's really cute: we had an axiom
(\S\ref{sec:introduction:axioms-of-wedge-product}) that the formal wedge
product is anticommutative on 1-forms. We see that
\begin{subequations}
  \begin{align}
    (\phi\wedge\psi)(V,W)
    &=\phi(V)\psi(W) -\phi(W)\psi(V)\\
    &=-(-\phi(V)\psi(W)+\phi(W)\psi(V))\\
    &=-(-\psi(W)\phi(V)+\psi(V)\phi(W))\\
    &=-(\psi\wedge\phi)(V,W).
  \end{align}
\end{subequations}
In fact we have
\begin{equation}
\begin{array}{ccc}
(\phi\wedge\psi)(V,W) &=& -(\phi\wedge\psi)(W,V)\\
\rotatebox[origin=c]{-90}{$=$} & & \rotatebox[origin=c]{-90}{$=$}\\
-(\psi\wedge\phi)(V,W) &=& (\psi\wedge\phi)(W,V).
\end{array}
\end{equation}
It's consistent!

\begin{remark}
We see the 3-form a 2-dimensional surface $M\subset\RR^{3}$ is zero.
\end{remark}

\N{Corollary: Nilpotence}
The reader can verify that, for any one-form $\phi$, we have
$\phi\wedge\phi=0$. We proved this formally, as a consequence of
antisymmetry, but the reader may verify this is true for our concrete
realization of the wedge product.

\N{Computing 2-form using basis vectors}
Let us consider a 2-form $\eta$ and suppose $\vec{e}_{1}$,
$\vec{e}_{2}\in\T_{\vec{p}}M$ is a basis. We will do some multilinear
algebra: once we know how $\eta$ acts on all possible linear
combinations of our basis vectors $\vec{e}_{1}$ and $\vec{e}_{2}$,
then we will know how it acts on any arbitrary vector in $\T_{\vec{p}}M$.

Consider
\begin{subequations}
\begin{align}
\eta(\alpha\vec{e}_{1}+\beta\vec{e}_{2},
\gamma\vec{e}_{1}+\delta\vec{e}_{2})
&=\alpha\eta(\vec{e}_{1}, \gamma\vec{e}_{1}+\delta\vec{e}_{2})
+\beta\eta(\vec{e}_{2}, \gamma\vec{e}_{1}+\delta\vec{e}_{2})\\
&=\alpha(\gamma\eta(\vec{e}_{1}, \vec{e}_{1}) +\delta\eta(\vec{e}_{1}, \vec{e}_{2}))
+\beta(\gamma\eta(\vec{e}_{2}, \vec{e}_{1}) +\delta\eta(\vec{e}_{2}, \vec{e}_{2}))\\
&=\alpha\gamma\eta(\vec{e}_{1}, \vec{e}_{1})
+\alpha\delta\eta(\vec{e}_{1}, \vec{e}_{2})
+\beta\gamma\eta(\vec{e}_{2}, \vec{e}_{1})
+\beta\delta\eta(\vec{e}_{2}, \vec{e}_{2})\\
&=0 +\alpha\delta\eta(\vec{e}_{1}, \vec{e}_{2})
-\beta\gamma\eta(\vec{e}_{1}, \vec{e}_{2})
+ 0\\
&= (\alpha\delta-\beta\gamma)\eta(\vec{e}_{1}, \vec{e}_{2})\\
&=\det\begin{pmatrix}\alpha & \beta\\
\gamma & \delta
\end{pmatrix}\eta(\vec{e}_{1}, \vec{e}_{2}).
\end{align}
\end{subequations}
We only need to compute $\eta(\vec{e}_{1}, \vec{e}_{2})$ once, and then
computing $\eta(\vec{v},\vec{w})$ amounts to computing a determinant.


\N{Exterior Derivative}
Let us talk about one forms on $\RR^{2}$, call such a 1-form $\phi$
and let $\vec{e}_{1}$, $\vec{e}_{2}$ be the natural frame field in the
$u^{1}$, $u^{2}$ coordinate directions. By the above formula, to figure
out $\D\phi$, we just need to know what $\D\phi(\vec{e}_{1}, \vec{e}_{2})$
is. In $\RR^{2}$, we know any one-form can be written as
\begin{equation}\label{eq:surfaces:one-form-as-linear-combo-of-coframe}
\phi = f_{1}\,\D u^{1} + f_{2}\,\D u^{2}.
\end{equation}
We have
\begin{subequations}
\begin{align}
\D\phi(\vec{e}_{1}, \vec{e}_{2})
&=\D(f_{1}\,\D u^{1} + f_{2}\,\D u^{2})(\vec{e}_{1}, \vec{e}_{2})
=(\D f_{1}\wedge\D u^{1} + \D f_{2}\wedge\D u^{2})(\vec{e}_{1}, \vec{e}_{2})\\
&=\left(\left(\frac{\partial f_{1}}{\partial u^{1}}\D u^{1} + \frac{\partial f_{1}}{\partial u^{2}}\D u^{2}\right)\wedge\D u^{1} + \left(\frac{\partial f_{2}}{\partial u^{1}}\D u^{1} + \frac{\partial f_{2}}{\partial u^{2}}\D u^{2}\right)\wedge\D u^{2}\right)(\vec{e}_{1}, \vec{e}_{2})\\
&=\left(\frac{\partial f_{1}}{\partial u^{2}}\D u^{2}\wedge\D u^{1}
+\frac{\partial f_{2}}{\partial u^{1}}\D u^{1}\wedge\D u^{2}\right)(\vec{e}_{1}, \vec{e}_{2})\\
&=\left[\left(\frac{\partial f_{2}}{\partial u^{1}} - \frac{\partial f_{1}}{\partial u^{2}}\right)\D u^{1}\wedge\D u^{2}\right](\vec{e}_{1}, \vec{e}_{2})\\
&=\left(\frac{\partial f_{2}}{\partial u^{1}} - \frac{\partial f_{1}}{\partial u^{2}}\right)\left(\D u^{1}(\vec{e}_{1})\D u^{2}(\vec{e}_{2})-\D u^{1}(\vec{e}_{2})\D u^{2}(\vec{e}_{1})\right)\\
&=\left(\frac{\partial f_{2}}{\partial u^{1}} - \frac{\partial f_{1}}{\partial u^{2}}\right)(1-0)=\left(\frac{\partial f_{2}}{\partial u^{1}} - \frac{\partial f_{1}}{\partial u^{2}}\right).
\end{align}
\end{subequations}
We stipulated in
Eq~\eqref{eq:surfaces:one-form-as-linear-combo-of-coframe} that we could
write a one-form $\phi$ using $f_{1}$, $f_{2}$. This let us find $\D\phi(\vec{e}_{1}, \vec{e}_{2})$.
Could we perform our calculation without this stipulation? That is to
say, can we write $\D\phi(\vec{e}_{1}, \vec{e}_{2})$ in terms of
derivatives of $\phi$? Yes! We can, thus:
\begin{equation}
\D\phi(\vec{e}_{1}, \vec{e}_{2}) = \frac{\partial}{\partial u^{2}}\phi(u^{1})-
\frac{\partial}{\partial u^{1}}\phi(u^{2}).
\end{equation}
This motivates the following definition of exterior derivatives of
one-forms on surfaces, recalling the partial velocities form a frame
field on the surface.

\begin{definition}
On a surface $M\subset\RR^{3}$, in a given patch, we have coordinate
vector fields $\chart{x}_{u}$, $\chart{x}_{v}$. We define for any 1-form
$\phi$ on $M$ the 2-form $\D\phi$ on $M$ given by
\begin{equation}
\D\phi(\chart{x}_{u}, \chart{x}_{v}) = \frac{\partial}{\partial u}\phi(\chart{x}_{v})-
\frac{\partial}{\partial v}\phi(\chart{x}_{u}),
\end{equation}
for any patch.
\end{definition}


\N{Consistency of Definitions}
One potential problem is that our definition might not make sense. We
demand consistency when the patches overlap, like the situation doodled
below:
\begin{center}
  \includegraphics{img/surfaces.20}
\end{center}
Whenever we define \emph{anything} in differential geometry, we
\emph{must} worry about consistency on the overlap of patches.

\vfill\eject
\N{Maps of Surfaces}
Recall our discussion of smooth maps between surfaces
(\S\ref{defn:smooth-functions-between-surfaces}).
If we have smooth surfaces $M$ and $N$, is there a smooth function
$F\colon M\to N$? In order for $F$ to be smooth, we need for any patch
$D\subset\RR^{2}$ with chart $\chart{x}\colon D\to M$ and for any patch
$E\subset\RR^{2}$ with chart $\chart{y}\colon E\to N$, there exists a
map $f\colon D\to E$ (defined by $f=\chart{y}^{-1}\circ
F\circ\chart{x}$) which is smooth in the usual sense.
We have the situation similar to what we have doodled below:
\begin{center}
  \includegraphics{img/surfaces.21}
\end{center}
Now, we have the tangent map in such an approach, defined for $f\colon D\to E$
using Definition~\ref{defn:euclidean-space:tangent-map}. We patch them
together to induce a tangent map associated for $F$.

\begin{definition}
Let $M$, $N$ be surfaces, $F\colon M\to N$ be a smooth map.
The \define{Tangent Map} is defined, for each $\vec{p}\in M$, as
$F_{*}\colon\T_{\vec{p}}M\to\T_{F(\vec{p})}N$.
\end{definition}

\M
The really slick way to approach this is if we have some path
$\alpha(t)$ that goes through the given point $\vec{p}=\alpha(0)$. Then
we have
\begin{equation}
F_{*}(\alpha'(0)) = \left.\frac{\D}{\D t}F(\alpha(t))\right|_{t=0}.
\end{equation}
We call $F_{*}(\vec{v}_{\vec{p}})$ the \define{Pushforward}
of $\vec{v}_{\vec{p}}\in\T_{\vec{p}}M$ along $F$.

\M
Recall that one-forms are maps from tangent vector spaces to
$\RR$. Given a one-form $\phi$ on $N$, and a smooth map $F\colon M\to N$,
we can obtain a one-form on $M$ called the \define{Pullback} of $\phi$
along $F$, denoted $F^{*}(\phi)$. This is defined by, for any $\vec{v}_{\vec{p}}\in\T_{\vec{p}}M$,
\begin{equation}
(F^{*}\phi)[\vec{v}_{\vec{p}}] = \phi[F_{*}(\vec{v}_{\vec{p}})].
\end{equation}
In short, any smooth map $F\colon M\to N$ on surfaces gives me two
induced maps, the pushforward
\begin{equation*}
F_{*}\colon\{\mbox{tangent vectors on $M$}\}\to\{\mbox{tangent vectors on $N$}\}
\end{equation*}
and, going in the opposite direction, the pullback which maps one-forms
to one-form,
\begin{equation*}
F^{*}\colon\{\mbox{1-forms on $M$}\}\gets\{\mbox{1-forms on $N$}\}.
\end{equation*}

\subsection{Shape Operators}

\N{Gauss Map}
The most important map for differential geometry is the Gauss map. Let
us try to describe it.

Given any oriented surface $M\subset\RR^{3}$ (meaning we can choose a
specific unit normal vector field, i.e., $M$ is equipped with a chosen
unit normal vector field $U$). There is a canonical map of surfaces
(``canonical'' meaning we have no need to make arbitrary choices):
\begin{equation}
\Gauss\colon M\to S^{2},
\end{equation}
where $S^{2}$ is the unit sphere in $\RR^{3}$. This is defined by taking
$U$ at each point $\vec{p}\in M$ and translating it to the origin of
$\RR^{3}$. More explicitly, if we have
$\vec{n}_{\vec{p}}=(\vec{n})_{\vec{p}}=U(\vec{p})$ be the unit normal
vector at $\vec{p}$ with vector part $\vec{n}\in\RR^{3}$, then
$G(\vec{p})=\vec{n}$ is the vector part of $\vec{n}_{\vec{p}}$ \emph{as
a point on the unit sphere, \textbf{not} a tangent vector on the unit sphere}.
This tells us how the unit normal $U$ rotates as we move
$\vec{p}$ infinitesimally. The intuition is sketched in Figure~\ref{fig:gauss-map-intuition}.
\begin{figure}[t]
\centering
  \includegraphics{img/surfaces.22}
\caption{Gauss map for a generic surface. The value of
  $\Gauss(\vec{p})$ is identical to the vector part of the unit normal vector $U(\vec{p})$.}\label{fig:gauss-map-intuition}
\end{figure}

\begin{example}
If $M$ is a plane, then $\Gauss(\vec{p})$ is a constant, since the plane
has a constant normal vector (so the Gauss map sends everything in the
surface to a single point on the unit sphere).
\end{example}

\begin{example}
If $M$ is a sphere of radius $r>0$, then the Gauss map is just a
``rescaling'' of the sphere together with a translation to the origin.
\end{example}

\N{Differential of Gauss Map}
For \emph{any} surface $M\subset\RR^{3}$, at any point $\vec{p}\in M$,
the tanget space $\T_{\vec{p}}M$ and its pushforward along the Gauss map
to $\T_{\Gauss(\vec{p})}S^{2}$, they are parallel. What's
more: they are \emph{canonically isomorphic}, which is nicer than
``just'' isomorphic, because we just have to change basis vectors.

This is a bold claim, so let us prove it more explicitly. Let
$\alpha\colon I\to M$ be a map such that $\alpha(0)=\vec{p}$ and
$\alpha'(0)=\vec{v}_{\vec{p}}$. We find, by definition of the pushforward,
\begin{equation}
\Gauss_{*}(\vec{v}_{\vec{p}}) = \left.\frac{\D}{\D t} \Gauss\left(\alpha(t)\right)\right|_{t=0}.
\end{equation}
But $\Gauss\left(\alpha(t)\right)$ ``is'' $U\left(\alpha(t)\right)$, in
the sense that the vector part of $U\left(\alpha(t)\right)$ equals the
coordinates of the point $\Gauss\left(\alpha(t)\right)\in S^{2}$. So we
have
\begin{subequations}
\begin{align}
\Gauss_{*}(\vec{v}_{\vec{p}}) &= \left.\frac{\D}{\D t}U\left(\alpha(t)\right)\right|_{t=0}\\
&= \left.\frac{\D}{\D t}\left(\sum_{i}u^{i}\left(\alpha(t)\right)U_{i}\right)\right|_{t=0}\\
&= \sum_{i}\left(\left.\frac{\D}{\D t}u^{i}\left(\alpha(t)
\right)\right|_{t=0}U_{i}\right)\\
&= \sum_{i}\vec{v}_{\vec{p}}[u^{i}]U_{i}\\
&= \nabla_{\vec{v}_{\vec{p}}}U.
\end{align}
\end{subequations}

\begin{definition}
Let $M\subset\RR^{3}$ be a surface, $\vec{p}\in M$ be an arbitrary point.
We define a \define{Shape Operator} $S_{\vec{p}}$ to be a function that
maps tangent vectors from $\T_{\vec{p}}M$ and gives new tangent vectors
(i.e., $S_{\vec{p}}\colon\T_{\vec{p}}M\to\T_{\vec{p}}M$)
by the following formula:
\begin{equation}
S_{\vec{p}}(\vec{v}_{\vec{p}}) = -\nabla_{\vec{v}_{\vec{p}}}U = -\Gauss_{*}(\vec{v}_{\vec{p}})+\mbox{translation},
\end{equation}
where $U$ is the unit normal vector field on $M$ and
$\vec{v}_{\vec{p}}\in\T_{\vec{p}}M$. 
\end{definition}

\begin{remark}
We should worry that $U$ is defined only on the surface, so the
covariant derivative operation may not be defined everywhere. However,
if we demand the vector $\vec{v}_{\vec{p}}$ be on the surface, then
we're golden.
\end{remark}

\begin{remark}
We see the shape operator is \emph{linear} since the covariant
derivative is linear. Also (and this is not obvious) it's symmetric with
respect to the dot product [i.e., it's self-adjoint]:
$S_{\vec{p}}(\vec{w})\cdot\vec{v}=S_{\vec{p}}(\vec{v})\cdot\vec{w}$. 
\end{remark}

\begin{example}
Let $M$ be a plane, then $S_{\vec{p}}(\vec{v}) = -\nabla_{\vec{v}}U = 0$
since the normal vector is constant.
\end{example}

\begin{example}
Let $M$ be a sphere of radius $r$ centered at $(x_{0},y_{0},z_{0})$. We
find
\begin{subequations}
\begin{align}
S_{\vec{p}}(\vec{v})
&= -\nabla_{\vec{v}}U\\
&= -\nabla_{\vec{v}}\left(\sum_{i}\frac{x^{i}-x^{i}_{0}}{r}U_{i}\right)\\ 
&= -\sum_{i}\vec{v}\left[\frac{x^{i}-x^{i}_{0}}{r}\right]U_{i}\\
&= -\sum_{i}\sum_{j}v^{j}U_{j}\left[\frac{x^{i}-x^{i}_{0}}{r}\right]U_{i}\\
&= -\sum_{i}\sum_{j}\frac{v^{j}}{r}{\delta^{i}}_{j}U_{i}\\
&= \frac{-1}{r}\sum_{i}v^{i}U_{i}\\
&= \frac{-\vec{v}}{r}.
\end{align}
\end{subequations}
Hence $S_{\vec{p}}(\vec{v}) = -\vec{v}/r$ for any tangent vector $\vec{v}\in\T_{\vec{p}}M$.
\end{example}

\begin{example}\label{ex:shape-operator:cylinder}
  We see on the cylinder, pick some frame field as sketched thus:
  \begin{center}
    \includegraphics{img/surfaces.23}
  \end{center}
  In the $v$-direction, it's geometrically ``a line''; whereas in the
  $w$-direction, it's geometrically ``a circle''. The reader can verify
  therefore:
  \begin{subequations}
    \begin{align}
      S_{\vec{p}}(\vec{v}_{\vec{p}}) &= 0\\
      S_{\vec{p}}(\vec{w}_{\vec{p}}) &= \frac{-1}{r}\vec{w}_{\vec{p}}.
    \end{align}
  \end{subequations}
\end{example}

\N{Shape Operator as Matrix}
Let $M$ be a surface in $\RR^{3}$, and suppose we have a frame field
$E_{1}$, $E_{2}$ for $M$.
We can find a matrix representation of the shape operator relative to
the frame field in the usual way,
\begin{equation}
  \begin{split}
S_{\vec{p}}(E_{1}) &= s^{1,1}E_{1} + s^{1,2}E_{2},\\
S_{\vec{p}}(E_{2}) &= s^{2,1}E_{1} + s^{2,2}E_{2}.
  \end{split}
\end{equation}
Recall the eigenvalues $\lambda_{j}$ for a square matrix $M$ contain
nearly all possible information about the matrix and further
\begin{subequations}
\begin{align}
\det(M) = \prod_{j}\lambda_{j}\\
\tr(M) = \sum_{j}\lambda_{j}.
\end{align}
\end{subequations}
For the shape operator, we find the trace and determinant contain
crucial geometric information. Specifically, these quantities encode
different aspects of the curvature of the surface.

\subsection{Curvature of a Surface}

\begin{proposition}
If $\alpha\colon I\to M$ is any curve on the surface $M$ with
$\alpha(0)=\vec{p}$ and $\alpha'(0)=\vec{v}_{\vec{p}}$, then the normal
components of its acceleration is given by $\vec{v}_{\vec{p}}\cdot S_{\vec{p}}(\vec{v}_{\vec{p}})$.
\end{proposition}

\begin{proof}
Let $M$ be a surface with unit normal vector field $U$, let $\alpha$ be
an arbitrary curve given by hypothesis. We can restrict $U$ to the curve
$\alpha$, denote it $U_{\alpha}(t) := U\left(\alpha(t)\right)$.
Now, the velocity vector of $\alpha$ at time $t\in I$ is tangent to $M$,
\emph{not normal to $N$}. This means $\alpha'(t)$ is orthogonal to the
normal vector at $\alpha(t)$. Consequently, we find
\begin{equation}
\alpha'(t)\cdot U_{\alpha}(t)=0.
\end{equation}
We can take the time derivative to find
\begin{equation}
\alpha''(t)\cdot U_{\alpha}(t) + \alpha'(t)\cdot U_{\alpha}'(t) = 0,
\end{equation}
where
\begin{equation}
U_{\alpha}'(t) = \nabla_{\alpha'}U = -S_{\alpha'}(\alpha').
\end{equation}
In particular, we find
\begin{equation}
\alpha'(t)\cdot S_{\alpha(t)}(\alpha'(t)) = \alpha''(t)\cdot U_{\alpha}(t).
\end{equation}
What this tells us is $\alpha''(t)\cdot U_{\alpha}(t)$ is the component
of the acceleration in the direction to the normal of the surface.
\end{proof}

In fact, this motivates the following definition:

\begin{definition}
Let $M\subset\RR^{3}$ be a surface and
$\vec{v}_{\vec{p}}\in\T_{\vec{p}}M$ be a unit vector. We define the
\define{Normal Curvature} of $M$ in the direction spanned by
$\vec{v}_{\vec{p}}$ is
\begin{equation}
k(\vec{v}_{\vec{p}}) = \vec{v}_{\vec{p}}\cdot S_{\vec{p}}(\vec{v}_{\vec{p}}).
\end{equation}
\end{definition}


\begin{example}
Recall Example~\ref{ex:shape-operator:cylinder}
where we worked out the shape operator for a cylinder of radius $r$. We
have two unit vectors $\vec{v}_{\vec{p}}$ and $\vec{w}_{\vec{p}}$ sketched:
  \begin{center}
    \includegraphics{img/surfaces.23}
  \end{center}
  We find
\begin{subequations}
\begin{align}
k(\vec{v}_{\vec{p}}) &= \vec{v}_{\vec{p}}\cdot S_{\vec{p}}(\vec{v}_{\vec{p}})=0\\
k(\vec{w}_{\vec{p}}) &= \vec{w}_{\vec{p}}\cdot S_{\vec{p}}(\vec{w}_{\vec{p}})=\frac{-1}{r}.
\end{align}
\end{subequations}
    \end{example}

\N{On the Normal ``Curvature''}
If we have our surface $M\subset\RR^{3}$ with unit normal vector field
$U$ and, at some point $\vec{p}\in M$, a [unit] vector
$\vec{v}_{\vec{p}}\in\T_{\vec{p}}M$, then we can construct a plane
$P\subset\RR^{3}$ spanned by $U(\vec{p})$ and $\vec{v}_{\vec{p}}$ (so $P=\{c_{1}U(\vec{p})+c_{2}\vec{v}_{\vec{p}}\in\RR^{3}\mid c_{1},c_{2}\in\RR\}$).
We obtain a curve given by the intersection of the surface with this plane,
$\gamma = P\cap M$. Observe $\gamma$ passes through $\vec{p}$.
We can further observe the curvature of $\gamma$ at $\vec{p}$ is exactly
equal to the normal curvature
$k(\vec{v}_{\vec{p}})$.

\begin{definition}
Given a point $\vec{p}\in M$, we define the \define{Principal Curvature}
at $\vec{p}$ to be:
\begin{subequations}
\begin{align}
k_{1} &= \max\{k(\vec{v}_{\vec{p}})\mid \vec{v}_{\vec{p}}\in\T_{\vec{p}}M,\|\vec{v}_{\vec{p}}\|=1\},\\
%\intertext{and}
k_{2} &= \min\{k(\vec{v}_{\vec{p}})\mid \vec{v}_{\vec{p}}\in\T_{\vec{p}}M,\|\vec{v}_{\vec{p}}\|=1\}.
\end{align}
\end{subequations}
\end{definition}

\begin{remark}
If $k_{1}\neq k_{2}$ at a point $\vec{p}\in M$, then there is a unique
vector pointing in the $k_{1}$-direction and a unique vector pointing in
the $k_{2}$-direction. Furthermore, these vectors are orthogonal.
\end{remark}

\begin{proposition}
Let $M$ be a surface.
If $k_{1}\neq k_{2}$ at $\vec{p}\in M$,
and $\vec{e}_{1},\vec{e}_{2}\in\T_{\vec{p}}M$ are unit vectors such that
$k(\vec{e}_{1})=k_{1}$ and $k(\vec{e}_{2})=k_{2}$,
then $\vec{e}_{1}$ and $\vec{e}_{2}$ are unique up to sign, and
$\vec{e}_{1}\cdot\vec{e}_{2}=0$ and $S_{\vec{p}}(\vec{e}_{1})=k_{1}\vec{e}_{1}$
and $S_{\vec{p}}(\vec{e}_{2})=k_{2}\vec{e}_{2}$.
We call the directions given by $\vec{e}_{1}$, $\vec{e}_{2}$ the
\define{Principal Directions} tangent to $\vec{p}$.
\end{proposition}

\M
Why is this so great? We can rewrite the shape operator. The recurring
theme of the course is that we may choose a frame that is particularly
nice for the situation. We have a nice basis of eigenvectors. This is
great because we have the operator (which these are eigenvectors of) be
diagonal,
\begin{equation}
S_{\vec{p}}(\alpha\vec{e}_{1} + \beta\vec{e}_{2}) = 
\alpha k_{1}\vec{e}_{1} + \beta k_{2}\vec{e}_{2}.
\end{equation}
So with respect to this basis $\vec{e}_{1}$, $\vec{e}_{2}$ (which are
the principal directions) the shape operator acts by
\begin{equation}
\begin{bmatrix}\alpha\\ \beta
\end{bmatrix}\mapsto
\begin{bmatrix}k_{1}\alpha\\ k_{2}\beta
\end{bmatrix}=\begin{bmatrix}k_{1} & 0\\0 & k_{2}
\end{bmatrix}
\begin{bmatrix}\alpha\\ \beta
\end{bmatrix}.
\end{equation}
This leads to a much more beautiful notion of curvature. We'd like to
get ``just a number'' at each point (or something like that) for
describing curvature of a surface. We have:

\begin{definition}
Let $M$ be a surface. We define the \define{Gaussian Curvature}
$K=\det(S)$ as the determinant of the shape operator, and the
\define{Mean Curvature} $H=\tr(S)/2$ as the trace divided by the number
of dimensions of $M$.
\end{definition}

\begin{remark}
The Gaussian curvature is an intrinsic geometric quantity. Often finding
the principal curvatures is hard, but computing the Gaussian curvature
can be easier.
\end{remark}

\begin{remark}
The mean curvature is an extrinsic geometric quantity.
\end{remark}

\N{Examples}
Here are a few examples of the Gaussian and mean curvatures:
\begin{enumerate}
\item Cylinder: $K = (0)\cdot(-1/r) = 0$, $H=-1/(2r)$
\item Sphere: $K = 1/r^{2}$, $H=-1/r$
\item Plane: $K=0$, $H=0$.
\end{enumerate}

\begin{definition}
  Let $M$ be a surface.
\begin{enumerate}
\item We call $M$ \define{Flat} if its Gaussian curvature is zero, $K=0$.
\item We call $M$ \define{Minimal} (or \emph{minimum}) if its mean
  curvature is zero, $H=0$.
\end{enumerate}
\end{definition}





\phantomsection{}
\subsection*{Exercises}
\addcontentsline{toc}{subsection}{Exercises}

%% June 4, 2008 due date.
%% At this point, homework 7 would be due [June 4, 2008].
%% It consisted of the following
%% exercises from O'Neill's \emph{Elementary Differential Geometry}
%% (Revised second ed.),
%% \begin{itemize}
%% \item 4.4 \# 3
%% \item 5.1 \# 1, 4, 5, 6
%% \item 5.2 \# 4 a,c
%% \item 5.3 \# 4a
%% \end{itemize}

\begin{enumerate}
\item Compute the Gaussian and mean curvature for:
  \begin{enumerate}
  \item the saddle $z=x^{2}-y^{2}$,
  \item the monkey saddle: $z=x^{3}-3xy$.
  \end{enumerate}
\item Let $f\colon\Sigma\to\RR$ be a real-valued function on the surface
  $\Sigma$, let $g\colon\RR\to\RR$ be some smooth function.
  \begin{enumerate}
  \item Prove or find a counter-example: for any
    $\vec{v}_{\vec{p}}\in\T_{\vec{p}}\Sigma$, we have $\vec{v}_{\vec{p}}[g\circ f] = g'(f(\vec{p}))\vec{v}_{\vec{p}}[f]$
  \item Deduce $\D(g\circ f) = (g'\circ f)\,\D f$.
  \end{enumerate}
% 5.1 # 4: describe the Gauss map for a variety of surfaces (the plane,
% sphere, cylinder, cone)
\item Let $T^{2}$ be the familiar torus.
  \begin{enumerate}
  \item What are the image curves under
    the Gauss map for meridians and parallels of $T^{2}$?
  \item Are there any points $\vec{q}\in G(T^{2})$ for which exactly two
    distinct points on the torus $\vec{p}_{1}\in T^{2}$ and
    $\vec{p}_{2}\in T^{2}$ are mapped to $G(\vec{p}_{1})=G(\vec{p}_{2})=\vec{q}$?
  \end{enumerate}
\item Consider the surface $\Sigma$ defined by $z=xy$.
  \begin{enumerate}
  \item What are the image curves under
    the Gauss map for $x=\mbox{constant}$ on $\Sigma$?
  \item Are there any points $\vec{q}\in G(\Sigma)$ for which exactly two
    distinct points on the surface $\vec{p}_{1}\in\Sigma$ and
    $\vec{p}_{2}\in\Sigma$ (so $\vec{p}_{1}\neq\vec{p}_{2}$) which are mapped to $G(\vec{p}_{1})=G(\vec{p}_{2})=\vec{q}$?
  \end{enumerate}
\item For each of the following surfaces, find the quadratic
  approximation near the origin:
  \begin{enumerate}
  \item $z = x^{2}+y^{2}$
  \item $z = x^{2}-y^{2}$
  \item $x^{2}+y^{2}-z^{2}=0$.
  \item $z = (2x + 3y)^{5}$.
  \end{enumerate}
\item Recall from linear algebra, if $A$ is any $n\times n$ matrix, the
  \define{Characteristic Polynomial} of $A$ is the polynomial in
  $\lambda$ defined by
  \begin{equation*}
p(\lambda) = \det(A - \lambda\cdot I_{n}),
  \end{equation*}
  where $I_{n}$ is the $n\times n$ identity matrix. \textbf{Compute} the
  characteristic polynomial for the shape operator.
\end{enumerate}
\subsection{Limits for Multivariable Functions, Partial Derivatives}
%%
%% multivarContinuity.tex
%% 
%% Made by Alex Nelson
%% Login   <alex@black-cherry>
%% 
%% Started on  Wed Jun 27 11:16:54 2012 Alex Nelson
%% Last update Thu Jun 28 12:47:00 2012 Alex Nelson
%%
\M
We considered differentiating and integrating functions of a
single-variable. How? We began with the notion of a limit, and
then considered the derivative. If we have a, e.g., polynomial
\begin{equation}
p(x,y) = x^{3}+x^{2}y+xy^{2}+y^{3}
\end{equation}
we see
\begin{equation}
p(x+\Delta x,y) = x^{3}+x^{2}y+xy^{2}+y^{3}+\bigl(3x^{2}+2xy+y\bigr)\Delta x+
\mathcal{O}(\Delta x^{2})
\end{equation}
Again we stop and reflect: this treats $y$ as if it were
constant. So the derivative formed by
\begin{equation}
\lim_{\Delta x\to0}\frac{p(x+\Delta x,y)-p(x,y)}{\Delta x}=3x^{2}+2xy+y
\end{equation}
are ``incomplete'' or \textbf{partial}. There is some subtlety
here due to using multiple variables, and we have to discuss the
problems of limits first.\more

\N{Definition}
The function $z=f(x,y)$ is \textbf{``Continuous''} at
$(x_{0},y_{0})$ if

(i) $f(x_{0},y_{0})$ is defined and finite;

(ii) $\displaystyle\lim_{(x,y)\to(x_{0},y_{0})}f(x,y)=f(x_{0},y_{0})$ is
defined;

(iii) $\displaystyle\lim_{(x,y)\to(x_{0},y_{0})}f(x,y)$ is defined (and finite).

\noindent{Note:} this can be determined by picking any curve
$\gamma\colon[0,1]\to\RR^{2}$ which satisfies
\begin{equation}
\gamma(t_{0}) = (x_{0},y_{0})
\end{equation}
for some $0\leq t_{0}\leq 1$, then taking 
\begin{equation}
\lim_{t\to t_{0}}f\bigl(\gamma(t)\bigr) = \lim_{(x,y)\to(x_{0},y_{0})}f(x,y).
\end{equation}
The subtletly here lies with $\gamma$ being \emph{arbitrary}. If
two different curves produce two different results, the limit
\emph{does not exist}. Lets consider some examples and non-examples.

\begin{example}[Limit Exists]
Find
\begin{equation}
\lim_{(x,y)\to(2,4)}\frac{y+4}{x^{2}y-xy+4x^{2}-4x}.
\end{equation}

\emph{Solution}: for this, we can simply plug in the values
\begin{equation}
\begin{aligned}
\lim_{(x,y)\to(2,4)}\frac{y+4}{x^{2}y-xy+4x^{2}-4x}
&=\frac{(4)+4}{(2)^{2}(4)-(2)(4)+4(2^{2})-4(2)}\\
&=\frac{8}{16-8+16-8}=\frac{1}{2}.
\end{aligned}
\end{equation}
This is because the function is sufficiently nice.
\end{example}

\begin{example}[Limit Doesn't Exist]
What is
\begin{equation}
\lim_{(x,y)\to(0,0)}\frac{x^{4}}{x^{4}+y^{2}}=?
\end{equation}

\emph{Solution}: Lets first approach it along the $x$-axis,
i.e. first setting $y=0$. We find
\begin{equation}
\lim_{(x,y)\to(0,0)}\frac{x^{4}}{x^{4}+y^{2}}=\lim_{x\to0}\frac{x^{4}}{x^{4}}=1.
\end{equation}
Now lets approach it on the $y$-axis, i.e. first setting
$x=0$. We see
\begin{equation}
\lim_{(x,y)\to(0,0)}\frac{x^{4}}{x^{4}+y^{2}}=\lim_{y\to0}\frac{0}{0+y^{2}}=0.
\end{equation}
Still, approaching along the curve $y=x^{2}$ we see
\begin{equation}
\lim_{(x,y)\to(0,0)}\frac{x^{4}}{x^{4}+y^{2}}=\lim_{x\to0}\frac{x^{4}}{x^{4}+x^{4}}=\frac{1}{2}.
\end{equation}
But we have a problem: this implies $0=1/2=1$. This cannot be! So
the limit \emph{cannot exist!} Very sad.
\end{example}


\N{Definition} Let $z=f(x,y)$ be defined on a region $R$ in the
$xy$-plane, and let $(x_{0},y_{0})$ be an inerior point of $R$,
we just don't want a boundary point!

If
\begin{equation}
\lim_{\Delta x\to0}\frac{f(x_{0}+\Delta
  x,y_{0})-f(x_{0},y_{0})}{\Delta x}
\end{equation}
exists, then it is called the \textbf{``Partial Derivative''} of
$z=f(x,y)$ at $(x_{0},y_{0})$ with respect to $x$. It is denoted
\begin{equation}
\frac{\partial}{\partial x}f = \frac{\partial}{\partial x}z
=f_{x} = \partial_{x}f = \partial_{x}z
\end{equation}
evaluated at $(x_{0},y_{0})$. NB: the subscripts in the
$\partial_{x}$ indicate what variable we are taking the partial
derivative of, i.e., it's shorthand for
$\partial_{x}=\partial/\partial x$.

Under similar conditions,
\begin{equation}
\lim_{\Delta y\to0}\frac{f(x_{0},y_{0}+\Delta y)-f(x_{0},y_{0})}{\Delta y}
\end{equation}
is the partial derivative of $z=f(x,y)$ with respect to $y$ at
$(x_{0},y_{0})$. We denote this by
\begin{equation}
\frac{\partial f}{\partial y}=\frac{\partial z}{\partial
  y}=\partial_{y}f = \partial_{y}z
\end{equation}
among a myriad of different conventions.

\M Higher order partial derivatives are done by taking it one at
a time. So if 
\begin{equation}
z=\E^{xy}
\end{equation}
for example, we have
\begin{equation}
\partial_{y}z=x\E^{xy}
\end{equation}
and taking its derivative again yields
\begin{equation}
\begin{aligned}
\partial_{y}^{2}z &= \partial_{y}\left(x\E^{xy}\right)\\
&=x\partial_{y}(\E^{xy})
\end{aligned}
\end{equation}
Note we factor $x$ out in front of the partial derivative with
respect to $y$ because $x$ is constant with respect to $y$. So we
then obtain
\begin{equation}
\partial_{y}^{2}z = x^{2}\E^{xy}.
\end{equation}
We take partial derivatives one at a time, from right to left:
\begin{equation}
\partial_{x}\partial_{y}z
= \partial_{x}\bigl(\partial_{y}z\bigr).
\end{equation}
\emph{Question}: do partial derivatives commute? I.e., is
$\partial_{x}\partial_{y}=\partial_{y}\partial_{x}$ always?
Lets first consider an example calculation before considering an answer.

\begin{example}
Consider the function $u=x^{2}-y^{2}$. Find
$\partial_{x}^{2}u+\partial_{y}^{2}u$.

\emph{Solution}: We find that
\begin{equation}
\partial_{x}u = 2x\implies \partial_{x}^{2}u = 2.
\end{equation}
Similarly, we find
\begin{equation}
\partial_{y}u=-2y\implies \partial_{y}^{2}y=-2.
\end{equation}
Thus we conclude
\begin{equation}
\partial_{x}^{2}u+\partial_{y}^{2}u=2-2=0.
\end{equation}
\end{example}

\N{Do Partial Derivatives Commute?}
Answer: not always. The conditions are fairly weak: if
$\partial_{x}z$, $\partial_{y}z$, $\partial_{x}\partial_{y}z$ and
$\partial_{y}\partial_{x}z$ are continuous throughout their
respective domains, then 
\begin{equation}
\partial_{x}\partial_{y}z = \partial_{y}\partial_{x}z.
\end{equation}

\begin{exercise}
Let $u(x,t) = f(x+vt) + g(x-vt)$ where $v\not=0$ is some
constant. Prove
\begin{equation}
\partial_{t}^{2}u(x,t)=v^{2}\partial_{x}^{2}u(x,t).
\end{equation}
\end{exercise}
\begin{exercise}
Let $f(x,y)=\ln(x^{2}+y^{2})$. What is
$\partial_{x}^{2}f(x,y)+\partial_{y}^{2}f(x,y)$? 
\end{exercise}
\begin{exercise}
Consider $g(x,y)=1/\sqrt{x^{2}+y^{2}}$. What is
$\partial_{x}^{2}g(x,y)$? What is $\partial_{y}^{2}g(x,y)$? Is
$\partial_{x}\partial_{y}g(x,y)=\partial_{y}\partial_{x}g(x,y)$? 
\end{exercise}
\begin{exercise}
Let $f(x,y)=3x^{2}+4y^{3}++x^{2}y^{3}+\sin(xy)$. What is
$\partial_{x}f$? What is $\partial_{y}f$?
\end{exercise}
\begin{exercise}
Let $z=\arctan(x^{2}\E^{2y})$. What is $\partial_{x}z$? What is
$\partial_{y}z$? 
\end{exercise}

\section{Partial Derivatives}
\subsection{Chain Rule for Partial Derivatives}
%%
%% pdChain.tex
%% 
%% Made by Alex Nelson
%% Login   <alex@black-cherry>
%% 
%% Started on  Thu Jun 28 15:42:27 2012 Alex Nelson
%% Last update Tue Jul  3 10:54:12 2012 Alex Nelson
%%

\N{Problem}
Consider a function $f(x,y)$ where we parametrize 
\begin{equation}
x=x(t,u),\quad\mbox{and}\quad y=y(t,u).
\end{equation}
If $t\to t+\Delta t$, how does $f\to f+\Delta f$ change?

\M
We first note
\begin{equation}
f(x+\Delta x,y) = f(x,y)+\Delta x\partial_{x}f(x,y)+\bigO(\Delta
x^{2}).
\end{equation}
Similarly 
\begin{equation}
f(x,y+\Delta y)= f(x,y)+\Delta y\partial_{y}f(x,y)+\bigO(\Delta
y^{2}).
\end{equation}
Thus we find
\begin{equation}
\begin{aligned}
f(x+\Delta x,y+\Delta y)
&=f(x,y+\Delta y)+\Delta x\partial_{x}f(x,y + \Delta y)+\bigO(\Delta
x^{2})\\
&=\bigl(f(x,y) + \Delta y\partial_{y}f(x,y) + \bigO(\Delta
y^{2})\bigr)\\
&\quad+ \Delta x\partial_{x}\bigl(f(x,y) + \Delta y\partial_{y}f(x,y) + \bigO(\Delta
y^{2})\bigr)\\
&\quad+\bigO(\Delta x^{2})\\
&= f(x,y) + \Delta y\partial_{y}f(x,y) + \Delta
x\partial_{x}f(x,y)\\
&\quad + \bigO(\Delta x\Delta y)%\partial_{x}\partial_{y}f(x,y)
+\bigO(\Delta x^{2})+\bigO(\Delta y^{2}).
\end{aligned}
\end{equation}
But specifically, we are interested in
\begin{equation}
\Delta x = \Delta t\partial_{t}x(t,u)+\bigO(\Delta t^{2})
\end{equation}
and
\begin{equation}
\Delta y = \Delta t\partial_{t}y(t,u)+\bigO(\Delta t^{2})
\end{equation}
Plugging this in allows us to write
\begin{equation}
\Delta f=\Delta y\partial_{y}f(x,y) + \Delta
x\partial_{x}f(x,y) + \bigO(\Delta x\Delta y)%\partial_{x}\partial_{y}f(x,y)
+\bigO(\Delta x^{2})+\bigO(\Delta y^{2})
\end{equation}
as
\begin{equation}
\Delta f = \bigl(\Delta t\partial_{t}y\bigr)\bigl(\partial_{y}f\bigr) +
\bigl(\Delta t\partial_{t}x\bigr)\bigl(\partial_{x}f\bigr) +
\bigO(\Delta t^{2})
\end{equation}
Observe under our substitution, we have the $\bigO(\Delta
x\Delta y)$ and other big O terms be gathered into the
$\bigO(\Delta t^{2})$ term.

So what? Observe
\begin{equation}
\begin{aligned}
\frac{\partial f}{\partial t} &= \lim_{\Delta t\to 0}\frac{f\bigl(x(t+\Delta t,u),y(t+\Delta t)\bigr)-f\bigl(x(t,u),y(t,u)\bigr)}{\Delta t} \\
&=\frac{\partial f}{\partial x}\frac{\partial x}{\partial t}+
\frac{\partial f}{\partial y}\frac{\partial y}{\partial t}.
\end{aligned}
\end{equation}
This is precisely the chain rule. Similarly, we find
\begin{equation}
\frac{\partial f}{\partial u}
=\frac{\partial f}{\partial x}\frac{\partial x}{\partial u}+
\frac{\partial f}{\partial y}\frac{\partial y}{\partial u}.
\end{equation}

\N{Implicit Differentiation Revisited}
Recall implicit differentiation required us to find $\D y/\D x$
from some complicated expression like
\begin{equation}
\E^{xy}+4y^{2}+\tan(x+y)=0
\end{equation}
What to do? First we write
\begin{equation}
z = F(x,y) = \E^{xy}+4y^{2}+\tan(x+y)=0.
\end{equation}
Next we say $y=y(x)$. So we find
\begin{equation}
\begin{aligned}
\frac{\D z}{\D x}
&= \frac{\partial F(x,y)}{\partial x}\frac{\D x}{\D x}+\frac{\partial F(x,y)}{\partial y}\frac{\D y}{\D x}\\
&=0
\end{aligned}
\end{equation}
where we set the derivative of $F$ to be zero since it's equal to the
derivative of zero. We can then write (taking $\D x/\D x=1$)
\begin{equation}
-\frac{\partial F(x,y)}{\partial x}=\frac{\partial
  F(x,y)}{\partial y}\frac{\D y}{\D x}
\end{equation}
and divide both sides by $\partial_{y}F$ to get
\begin{equation}
\frac{-\partial_{x}F}{\partial_{y}F} = \frac{\D y}{\D x}.
\end{equation}
But this is precisely what implicit differentiation gives us!

\N{Warning for Physicists} 
Physicists often use partial derivative notation slightly
differently. If $q(t)$ is the position of a particle, and $p(t)$
is its momentum, physicists consider arbitrary functions of the
form
\begin{equation}
f=f(q,p,t)
\end{equation}
and write
\begin{equation}
\frac{\partial f}{\partial t} = \lim_{\Delta t\to0}
\frac{f\bigl(q(t),p(t),t+\Delta t\bigr)-f\bigl(q(t),p(t),t\bigr)}{\Delta t}.
\end{equation}
This is strictly speaking not quite true. The error committed
lies in treating $q$ and $p$ as functions of time: really they
are variables whom we are trying to express as functions of
time. 

\begin{exercise}
Let $w=\sqrt{x}+y^{2}/z$ where $x=\exp(2t)$, $y=t^{3}+4t$, and
$z=t^{2}-t$. Find $\D w/\D t$.
\end{exercise}
\begin{exercise}
Let $z=\cos(xy)+y\sin(x)$ where $x=v^{2}+u$ and $y=u-v$. Find
$\partial_{u} z$ and $\partial_{v}z$.
\end{exercise}

\subsection{Directional Derivatives}
%%
%% directionalDerivatives.tex
%% 
%% Made by Alex Nelson
%% Login   <alex@black-cherry>
%% 
%% Started on  Fri Jun 29 12:30:18 2012 Alex Nelson
%% Last update Tue Jul  3 11:02:55 2012 Alex Nelson
%%
\M
Suppose we have a scalar function of several variables
\begin{equation}
f\colon\RR^{3}\to\RR
\end{equation}
Let $\widehat{u}$ be some unit vector. How does $f$ change in the
$\widehat{u}$ direction?

We can consider this quantity as a function
\begin{equation}
g(\vec{x}) =
\lim_{h\to0}\frac{f(\vec{x}+h\widehat{u})-f(\vec{x})}{h}
\end{equation}
What does this look like?

\M Lets restrict our attention to the
smallest non-boring case: $f\colon\RR^2\to\RR$. Then we write
$\widehat{u} = \langle p,q\rangle$. We have
\begin{equation}
f(\vec{x}+h\widehat{u}) = f(x+hp,y+hq).
\end{equation}
Expanding this to first order in $h$ lets us write
\begin{equation}
\begin{aligned}
f(x+hp,y+hq) &= f(x,y+hq) +
hp\partial_{x}f(x,y+hq)+\bigO(h^{2})\\
&= \bigl(f(x,y)+hq\partial_{y}f(x,y)+\bigO(h^{2})\bigr)\\
&\quad+hp\partial_{x}\bigl(f(x,y)+hq\partial_{y}f(x,y)+\bigO(h^{2})\bigr)\\
&\quad+\bigO(h^{2})\\
&= f(x,y) + h\bigl(q\partial_{y}f(x,y) +
p\partial_{x}f(x,y)\bigr) +\bigO(h^{2}).
\end{aligned}
\end{equation}
But what does this look like? It's simply
\begin{equation}
f(\vec{x}+h\widehat{u}) = f(\vec{x}) +
h\widehat{u}\cdot\langle\partial_{x},\partial_{y}\rangle
f(\vec{x}) + \bigO(h^{2}).
\end{equation}

\begin{example}
What is the derivative of $f(x,y)=x/y$ in the direction of
$\vec{v}=\langle 1,3\rangle$ at $(5,3)$?

\emph{Solution}: We find the directional derivative is
\begin{equation}
v_{1}\partial_{x}f + v_{2}\partial_{y}f
\end{equation}
We compute
\begin{equation}
\partial_{x}f = 1/y
\end{equation}
and
\begin{equation}
\partial_{y}f = -x/y^{2}
\end{equation}
Thus we have
\begin{equation}
v_{1}\partial_{x}f + v_{2}\partial_{y}f = v_{1}(1/y) + v_{2}(-x/y^{2}).
\end{equation}
We plug in $v_{1}=1$, $v_{2}=3$
\begin{equation}
v_{1}\partial_{x}f + v_{2}\partial_{y}f = (1/y) + 3(-x/y^{2}).
\end{equation}
Then we evaluate $(x,y)=(5,3)$ to get
\begin{equation}
v_{1}\partial_{x}f + v_{2}\partial_{y}f = (1/3) + 3(-5/3^{2})
=-4/3.
\end{equation}
This gives us the directional derivative of $f$.
\end{example}


\M We denote
\begin{equation}
\vec{\nabla} = \langle\partial_{1},\dots,\partial_{n}\rangle
\end{equation}
and call it the \textbf{``Gradient''}. Note we will write
$\nabla$ interchangeably with the vector arrow $\vec{\nabla}$,
and they mean the same thing. The vector arrow doesn't add
anything semantically, it's just different syntax.

The directional derivative is then
\begin{equation}
\widehat{u}\cdot\vec{\nabla}f(\vec{x}) = \widehat{u}\cdot\langle
\partial_{1}f,\dots,\partial_{n}f\rangle.
\end{equation}
Note that the gradient acting on a scalar function produces a
vector-valued function of several variables, but we can also take
the dot product of the gradient with such a monstrosity.

\N{Question:} What is a vector-valued function of several variables?

For us, in practice, we think of this as a \emph{Vector Field}: a
``function'' which assigns to each point a vector. Each
vector-component is a function, usually smooth (i.e., infinitely
differentiable). 

(Again, just as we warned the reader with vectors, this too is a
lie. A vector field is a bit more than just a function
$\RR^{n}\to\RR^{n}$, it's a more complicated beast which is
studied further in differential geometry.)

\N{Meaning of Gradient}
Consider a family of level curves $f(\vec{x})=c$. The gradient
points towards the direction of increasing $c$. How can we see
this? Well, consider the function
\begin{equation}
f(x,y)=x^{2}-y^{2}.
\end{equation}
We see its gradient is
\begin{equation}
\vec{\nabla}f(x,y) = \langle 2x, -2y\rangle.
\end{equation}
Lets draw a few level-curves and see what the vectors point to:
\begin{center}
\includegraphics{img/gradient.0}
\end{center}
We see the vectors point towards $(x,y)\to(\pm\infty,0)$. 

\begin{exercise}
Consider the function $f\colon\RR^{2}\to\RR$ defined by
$f(x,y)=\ln(x^{2}+y^{2})$. What is its gradient?
\end{exercise}
\begin{exercise}
Let $g\colon\RR^{3}\to\RR$ be defined by $g(x,y,z) =
(x^{2}+y^{2}+z^{2})^{-1/2}$. Find its gradient.
\end{exercise}
\begin{exercise}
Let $f(x,y,z)=x/(y+z)$. Find its derivative in the direction
$\vec{u}=\langle 1,1,1\rangle$ at the point $(1,6,2)$. 
\end{exercise}
\begin{exercise}
Let $f(x,y)=x\exp(-y) + 3y$. Find its gradient.
\end{exercise}

\subsection{Extrema}
%%
%% extrema.tex
%% 
%% Made by Alex Nelson
%% Login   <alex@black-cherry>
%% 
%% Started on  Fri Jun 29 13:09:57 2012 Alex Nelson
%% Last update Fri Jun 29 13:43:18 2012 Alex Nelson
%%

\M
Remember for a curve $y=f(x)$, we have maxima and minima occur
whenever
\begin{equation}
f'(x_{0}) = 0
\end{equation}
What's the multivariable analog to this notion? \more

\N{Definition}
If $\vec{\nabla}f(\vec{x}_{0})=\vec{0}$, we say $\vec{x}_{0}$ is
a \textbf{``Critical Point''} of $f$.

\begin{example}
Consider $f(x,y) = x^{2}+y^{2} - 2x - 8y$. What are its critical
points?

\emph{Solution}: We find its gradient first
\begin{equation}
\vec{\nabla}f = \langle 2x - 2, 2y - 8\rangle
\end{equation}
Next we need to set each component to vanish. This implies
\begin{equation}
\vec{x}_{0} = \langle 1, 4\rangle
\end{equation}
is the only critical point.
\end{example}

\N{Problem:} How do we determine if a critical point describes a
maxima or minima?

Lets consider the critical point $\vec{x}_{0}$ for $f$. We Taylor
expand $f$ to second order about $\vec{x}_{0}$ writing
\begin{equation}
f(\vec{x}_{0}+\vec{h}) \approx f(\vec{x}_{0}) +
\vec{h}\cdot\underbrace{\vec{\nabla}f(\vec{x}_{0})}_{=0} + \frac{1}{2} \vec{h}\cdot\mathrm{Hess}(f)\vec{h}
\end{equation}
where we use the matrix
\begin{equation}
\mathrm{Hess}(f) = \begin{bmatrix}
\partial_{1}^{2} f & \dots & \partial_{1}\partial_{n}f \\
 \vdots  & \ddots & \vdots \\
\partial_{n}\partial_{1} f & \dots & \partial_{n}^{2}f
\end{bmatrix}
\end{equation}
Each row is precisely $\vec{\nabla}\partial_{j}f$, and each
column likewise is $\partial_{i}\vec{\nabla}f$; the intuition is
$\mathrm{Hess}(f) \approx \vec{\nabla}^{2}f$. 

Now, since we are Taylor expanding about a critical point, our
approximation becomes
\begin{equation}
f(\vec{x}_{0}+\vec{h}) \approx f(\vec{x}_{0}) +
\frac{1}{2} \vec{h}\cdot\mathrm{Hess}(f)\vec{h}.
\end{equation}
We can consider the behaviour of $f$ near $\vec{x}_{0}$ by
studying the properties of $\mathrm{Hess}(f)$. Specifically, the
signs of the eigenvalues tells us whether the critical point is a
local maxima (all eigenvalues are positive) or a minima (all are
negative) or some saddle point (mixture having both positive and
negative eigenvalues). If there exists at least one eigenvalue
that vanishes, this test is inconclusive.

\N{Parting Thoughts:} What if we want to optimize a function
constrained to live on a surface? 

\subsection{Lagrange Multipliers}
%%
%% lagrangeMultiplier.tex
%% 
%% Made by Alex Nelson
%% Login   <alex@black-cherry>
%% 
%% Started on  Fri Jun 29 13:44:22 2012 Alex Nelson
%% Last update Tue Jul  3 10:48:45 2012 Alex Nelson
%%

\M
So, we considered finding extrema for some function
$f\colon\RR^{n}\to\RR$, but what if we constrain our focus to
some surface $g\colon\RR^{n}\to\RR$? For example, the unit circle
would have
\begin{equation}
g(x,y) = x^{2}+y^{2} - 1=0
\end{equation}
How do we find extrema for $f(x,y)=xy$ on the unit circle?\more

\M What can we do? First we can consider the level curves
$f(x,y)=c$. These are precisely the curves $y=c/x$. The gradient
vector for $f$ is precisely
\begin{equation}
\nabla f=\langle y,x\rangle.
\end{equation}
This points in the direction of increasing values of $f$. 

\M
We want to consider the situation when $\nabla f=\lambda\nabla
g$, i.e., when the gradient of $f$ is precisely a scaled tangent
of $g$. Why? Because we want the gradient of $f$ to point in the
direction of a tangent to our surface. We draw the circle, the
gradient vector $\nabla f$ in red, and $\nabla g$ in
blue. Remember, the red vectors point in the direction of
increasing values of $f$, and we restrict our movement along the circle:
\begin{center}
  \includegraphics{img/lagrangeMultiplier.0}
\end{center}
Observe when the red and blue vectors are perpendicular,
$f=0$. But when they overlap as a purple vector or point in
completely opposite direction, what happens?

This happens when
\begin{equation}
\langle y,x\rangle = \lambda\langle 2x,2y\rangle
\end{equation}
or equivalently
\begin{equation}
y=2\lambda x,\quad\mbox{and}\quad
x=2\lambda y.
\end{equation}
Solving for $2\lambda$, we find
\begin{equation}
2\lambda = \frac{y}{x} = \frac{x}{y}
\end{equation}
which implies
\begin{equation}
x^{2}=y^{2}.
\end{equation}
But we're not quite done!

\M
We must remain on the circle, so we also must demand that
$x^{2}+y^{2}=1$. This equivalently implies
\begin{equation}
x^{2}=\frac{1}{2}\implies x=\pm\frac{\sqrt{2}}{2}.
\end{equation}

\M
Is this optimal? Lets try approaching the problem differently. We
are working on the circle, which is the parametric curve
\begin{equation}
x(t) = \cos(t),\quad\mbox{and}\quad
y(t) = \sin(t).
\end{equation}
Thus the function we are optimizing becomes
\begin{equation}
\begin{aligned}
f(t) &= f\bigl(x(t),y(t)\bigr)\\
&=\cos(t)\sin(t)
\end{aligned}
\end{equation}
We see
\begin{equation}
f'(t) = -\sin^{2}(t)+\cos^{2}(t) = 0.
\end{equation}
We need to solve for $t$, to do so we rearrange terms
\begin{equation}
\sin^{2}(t)=\cos^{2}(t)
\end{equation}
and divide through by $\cos^{2}(t)$, getting
\begin{equation}
\tan^{2}(t) = 1.
\end{equation}
But this implies $t=(2n+1)\pi/4$ where $n=0$, $1$, $2$, or
$3$. Look: that's precisely describing
$(x,y)=(\pm1/\sqrt{2},\pm1/\sqrt{2})$. 


\N{Lagrange Multipliers, Constraints}
One way to consider this situation ``Optimize $f$ subject to the
constraint $g=0$'' is to say \emph{Okay, so suppose $g=0$, then
wouldn't we have}
\begin{equation}
f + \lambda g\approx f?
\end{equation}
The $g$ term vanishes anyways, so intuitively it seems ``equal-ish''.

\begin{example}[Minimizing Surface Area]
Find the dimensions of the cylinder with smallest surface area
whose volume is fixed at $16\pi$.

\emph{Solution}: The outline takes several steps, namely, (1)
construct the function, (2) take the derivatives, (3) solve.

\emph{Step One: Construct the Functions}. We first write
\begin{equation}
A(r,h) = \pi r^{2} + \pi r^{2} + 2\pi rh
\end{equation}
for the surface area, and
\begin{equation}
V(r,h) = \pi r^{2}h
\end{equation}
describes the area. The constraint is
\begin{equation}
C(r,h) = V(r,h) - 16\pi.
\end{equation}
Thus we construct the function
\begin{equation}
F(r,h) = A(r,h) - \lambda C(r,h).
\end{equation}
This concludes the first step.

\emph{Step Two: Take the Derivatives}. We find
\begin{equation}
\nabla F(r,h) = \langle \partial_{r}A(r,h)
- \lambda\partial_{r}C(r,h), \partial_{h}A(r,h)-\lambda\partial_{h}C(r,h)\rangle
  = 0.
\end{equation}
Observe
\begin{equation}
\partial_{r}A(r,h)=4\pi r+2\pi h,\quad\mbox{and}\quad
\partial_{r}C(r,h)=2\pi rh.
\end{equation}
We also have
\begin{equation}
\partial_{h}A(r,h)=2r\pi,\quad\mbox{and}\quad
\partial_{h}C(r,h)=\pi r^{2}.
\end{equation}
The derivative with respect to the Lagrange multiplier gives us
\begin{equation}
\partial_{\lambda}F(r,h) = C(r,h) = 0.
\end{equation}
So we can set up our equations as
\begin{equation}
\begin{aligned}
(\partial_{r}F&=0)&\quad &\quad &2\pi r+2\pi h&=\lambda 2\pi rh\\
(\partial_{h}F&=0)&\quad &\quad &2\pi r &=\lambda\pi r^{2}\\
(\partial_{\lambda}F&=0)&\quad &\quad &\pi r^{2}h-16\pi&=0.
\end{aligned}
\end{equation}
That concludes our second step.

\emph{Step Three: Solve}. We see immediately from the
$\partial_{h}F$ equation that
\begin{equation}
2\pi r = \lambda \pi r^{2}\implies \lambda=\frac{2}{r}.
\end{equation}
We plug this into the $\partial_{r}F$ equation, we get
\begin{equation}
\begin{aligned}
2\pi r+2\pi h &= \lambda2\pi rh\\
&=\left(\frac{2}{r}\right)2\pi rh\\
&=4\pi h
\end{aligned}
\end{equation}
and subtracting $2\pi h$ from both sides yields
\begin{equation}
2\pi r=2\pi h\implies r=h.
\end{equation}
Now we use the constraint
\begin{equation}
\pi r^{2}h = 16\pi
\end{equation}
substituting $r=h$ we get
\begin{equation}
\pi h^{3} = 16\pi\implies h = \sqrt[3]{16}.
\end{equation}
Thus when we take $r=h=\sqrt[3]{16}$, we minimize the surface area.
\end{example}


\begin{exercise}
Find the extrema of $f(x,y,z)=xy+yz+zx$ subject to the constraint
$g(x,y,z)=x^{2}+y^{2}+z^{2}-1$.
\end{exercise}
\begin{exercise}
Find the extrema of $f(x,y)=\exp(-xy)$ subject to the constraint
$g(x,y)=x^{2}+y^{2}-1$.
\end{exercise}



\subsection{Total Differential}
%%
%% totalDifferential.tex
%% 
%% Made by Alex Nelson
%% Login   <alex@black-cherry>
%% 
%% Started on  Fri Jun 29 14:21:33 2012 Alex Nelson
%% Last update Fri Jun 29 14:21:58 2012 Alex Nelson
%%
\N{Total Differential}
Lets consider some function
\begin{equation}
f\colon\RR^{2}\to\RR.
\end{equation}
What is $\D f$? Well, we can write it as
\begin{equation}
\D\vec{x}\cdot\vec{\nabla}f = (\D x\,\partial_{x}+\D
y\,\partial_{y})f(x,y)
\end{equation}
in Cartesian coordinates. We can write a ``linear approximation''
to $f(x,y)$ as
\begin{equation}
f(x+\Delta x, y+\Delta y)\approx f(x,y) + \langle\Delta x,\Delta
y\rangle\cdot\vec{\nabla}f(x,y).
\end{equation}
Again, this should remind us of the linear approximation when we
used the tangent line as an approximation to a curve.

\M
Another way to consider to total differential $\D{f}$ is by using
the ``chain rule'' (wink wink). We write
\begin{equation}
\D f = \frac{\partial f(x,y)}{\partial x}\D x
+\frac{\partial f(x,y)}{\partial y}\D y.
\end{equation}
This can be quite useful!

\begin{example}
Lets consider a cylinder whose radius is $r=10$ and height is
$h=100$. Its volume is $V(r,h)=\pi r^{2}h$. The measurement is
correct to $0.1$ precision, what's the error in our measurement?

Well, we approximate it as
\begin{equation}
\begin{aligned}
\D V(r,h) &\approx 0.1\partial_{r}V(r,h) + 0.1\partial_{h}
V(r,h)\\
&= 0.1(2\pi rh) + 0.1(\pi r^{2})
\end{aligned}
\end{equation}
The error is thus
\begin{equation}
\D V(10,100) \approx 0.1 (2\pi\cdot10\cdot100) + 0.1(\pi\cdot 10^{2})
\approx 210\pi.
\end{equation}
The error is approximately $659.7339$, and our estimated volume
is $31415$. The error is about 2\%, which is quite good.
\end{example}



\nocite{*}
\bibliographystyle{utcaps}
\bibliography{chain}
\end{document}
