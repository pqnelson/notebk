%%
%% differentiableStructure.tex
%% 
%% Made by Alex Nelson
%% Login   <alex@tomato3>
%% 
%% Started on  Thu Jan 14 10:54:21 2010 Alex Nelson
%% Last update Thu Jan 14 10:54:46 2010 Alex Nelson
%%
\section{Differentiable Structures}

We want to consider a so-called ``differentiable structure'' that
we can equip a topological space with. The intuition already
should be that this is a mapping of some sort, where for our
topological space $M$ we have a mapping
\begin{equation}
U\mapsto \diff(U)
\end{equation}
where $U\subseteq M$ is open. As we are working with a
topological space, we demand consistancy on overlaps, so if
$U\cap V\not=\emptyset$ and $U,V\subseteq M$ are open, then we
would want
\begin{equation}
\diff(U\cap V)=\diff(\left.U\right|_{U\cap V})=\diff(\left.V\right|_{U\cap V})
\end{equation}
that is to say, if we restrict our attention of $\diff(U)$ on the overlap
$\diff(\left.U\right|_{U\cap V})$, it should be equal to the
differentiable structure on the overlap. 

But what exactly is this ``differentiable structure'' we are
assigning to each open subset of $M$? Already it sounds like a
presheaf or sheaf, so intuitively we can think of it as a
generalization of a vector space. Instead of assigning a vector
at each point of $M$, it assigns something ``smooth'' to each
point in such a way that varies ``nicely'' as we vary the
base-point. We will use something that we know very well: an
algebra of smooth functions.

So basically, given some topological space $M$, we have
\begin{equation}
\diff:U\mapsto C^{\infty}(U).
\end{equation}
That is, for each open subset $U$ of $M$, we assign an algebra of
$C^{\infty}$ functions (i.e. infinitely differentiable functions)
which includes the unit function $1(x)=1$. \marginpar{Notation
  Change! $\diff(U)$ changed to $C^{\infty}(U)$}\textbf{N.B:} we will
\emph{change notation} to use instead of $\diff(U)$ the
mathematically clearer $C^{\infty}(U)$.\def\diff{C^{\infty}}

Is it enough to assign ``any old'' algebra of smooth functions to
open subsets $U\subseteq M$? Well, we should make some
restrictions. Namely, we want it to be ``consistent on
overlaps''. So let $U,V\subseteq M$ are open, $U\cap
V\not=\emptyset$, and $f\in\diff(U)$. We will denote the
restriction mapping as 
\begin{equation}
r_{U\cap V,U}:U\to U\cap V
\end{equation}
we then \emph{demand}
\begin{equation}
f\circ r_{U\cap V,U}\in\diff(U\cap V)
\end{equation}
that $f$ restricted to the overlap lie in the differentiable
structure assigned to the overlap. This is the naive way to
demand consistency on overlaps.

So we have a ``differentiable structure'' be an assignment of
``sufficiently nice'' unital algebras of smooth functions to open
subsets of a topological space, but is that all? Well, for
starters, how do we define a derivative? It requires a choice of
some coordinates! This is a bit of a problem, so we need to equip
each open subset $U\subseteq M$ some extra information which
permits some notion of coordinates. Locally each open subset
$U\subseteq M$ is ``the same'' as $\Bbb{R}^{n}$ (for some fixed
$n\in\Bbb{N}$ called the \define{Dimension of
  $M$}\index{Dimension!Of Manifold}\marginpar{Notion of dimension}).  What does this rigorously translate to? How can
we rigorously translate the ``sameness'' of two topological
spaces? Well, we use a mapping. From the category theoretic
perspective, it should be an isomorphism. But for topological
spaces that is a \emph{homeomorphism} (a continuous map with a
continuous inverse --- so we can translate open subsets of the
domain into open subsets of the codomain and vice-versa). So we
equip each open subset $U\subseteq M$ with a homeomorphism
\begin{equation}
\varphi\colon U\to\Bbb{R}^{n}
\end{equation}
specified by components
\begin{equation}
\varphi\colon y\mapsto (x_{1}(y), x_{2}(y), \ldots, x_{n}(y))
\end{equation}
where $y\in U$, $x_{1},\ldots,x_{n}\in\diff(U)$. These functions
$x_{i}\in\diff(U)$ are called the \define{Coordinate Functions}.

So we just introduced some extra structure, some special
functions defined for each open subset $U\subseteq M$ called the
coordinate functions on $U$, which allows us to solve the problem
of specifying differentiability! Why? Well, observe that the
coordinate function is invertible, so
\begin{equation}
\varphi^{-1}\colon\Bbb{R}^{n}\to U
\end{equation}
can be composed with a function with domain $U$. Why is this
important? Well, if 
\begin{equation}
f\colon U\to\Bbb{R},
\end{equation}
then the composition
\begin{equation}
f\circ\varphi^{-1}\colon\Bbb{R}^{n}\to\Bbb{R}
\end{equation}
can be differentiable. That is, \emph{we can use the familiar
 notion of differentiability from ordinary calculus!} We just
work ``locally'' in $\Bbb{R}^{n}$ when differentiating.

So to reiterate our specification of a differentiable structure,
it is a mapping from a topological space $M$ to a unital associative
algebra of smooth $\diff$ functions on $M$. This mapping is
``sufficiently nice'' on overlaps of open subsets of $M$. Further
it has preferred functions which are the ``coordinate functions,'' 
whose \emph{raison d'\^{e}tre} is to permit differentiation in
the obvious way. This is precisely sufficient information to have
some notion of ``smoothness'' in $M$.

\begin{rmk}
The pair $(M,{\rm Diff})$ is often referred to as a 
\define{Smooth Manifold}. However, we will abuse language and
simply refer to $M$ as the smooth manifold with the understanding
that it really is equipped with a smooth structure.
\end{rmk}
