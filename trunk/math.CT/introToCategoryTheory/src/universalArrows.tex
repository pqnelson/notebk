%%
%% universalArrows.tex
%% 
%% Made by Alex Nelson
%% Login   <alex@tomato>
%% 
%% Started on  Sat Jul 18 14:43:46 2009 Alex Nelson
%% Last update Sat Jul 18 14:43:46 2009 Alex Nelson
%%

As previously mentioned, universal arrows arise whenever phrases
like ``\ldots\emph{there exists} a \emph{unique} function such
that\ldots'' occur. We're concerned about (1) existence, and (2)
uniqueness. We'll start by defining a universal arrow.

\begin{defn}%\label{defn:}
If $F:\ms{D}\to\ms{C}$ is a functor and $C\in\ms{C}$ is an
object, a \define{Universal Arrow from $C$ to $F$} is a pair
$(D,u)$ consisting of
\begin{itemize}
\item an object $D\in\ms{D}$;
\item an arrow $u:C\to F(D)$ of $\ms{C}$;
\end{itemize}
such that
\begin{itemize}
\item to every pair $(D',f)$ with $D'\in\ms{D}$ and $f:C\to
  F(D')$ an arrow of $\ms{C}$, there is a unique arrow $f':D\to
  D'$ with $F(f')\circ{}u=f$.
\end{itemize}
Or in other words, every arrow $f$ to $F$ factors uniquely
through the universal arrow $u$ as in the commutative diagram
\begin{equation}%\label{eq:}
\vcenter{\xymatrix{
C\ar@{=}[d]\ar[r]^{u}&F(D)\ar@{-->}[d]^{F(f')}&D\ar@{-->}[d]^{f'}\\
C\ar[r]^{f}&F(D')&D'
}}
\end{equation}
\end{defn}
As previously mentioned, we can come up with an equivalent
definition using initial objects instead. We say that $u:C\to
F(D)$ is universal from $C$ to $F$ when the pair $(D,u)$ is
an initial object in the comma category $(C\downarrow{}F)$ whose
objects are arrows $C\to{} F(D')$. As with any initial object, it
turns out that $(D,u)$ is unique up to isomorphism in
$(C\downarrow{}F)$. This is the typical use of the comma
categories.

Like most of category theory, this definition is best illuminated
with many novel examples.

\begin{ex}[Tensor Algebra]\index{Functor!Example!Tensor Algebra}
Let $V$ be a vector space over $\mathbb{K}$, and $A$ be an
algebra over $\mathbb{K}$. To construct the tensor algebra over
$V$ we typically do a mathematical procedure, let:
\begin{equation}%\label{eq:}
T^kV = V^{\otimes k} = V\otimes V \otimes \cdots \otimes V. 
\end{equation}
then we can define the tensor algebra over $V$ as
\begin{equation}%\label{eq:}
T(V)= \bigoplus_{k=0}^\infty T^kV = K\oplus V \oplus (V\otimes V) \oplus (V\otimes V\otimes V) \oplus \cdots.
\end{equation}
Our intuition about category theoretic descriptions of
mathematical procedures should be alerted: we have $T(V)$ as a
mathematical procedure! It's a \emph{functor!} 

Now, the question that should come to mind next is \emph{Suppose the Tensor Algebra is a functor, what categories are its domain and codomain?}
Excellent question! We should suspect since it acts on $V$, a
finite dimensional vector space over $\mathbb{K}$, that its
domain category should be $\ms{Vect}_{\mathbb{K}}$ and its
codomain category is yet to be known. It wouldn't be too much of
a stretch to suppose its codomain category would be the category
of graded algebras over $\mathbb{K}$. Why
graded? Well, observe that our direct sum is taken
over $\mathbb{N}$, so thinking of it as a graded algebra is
natural. So we can state that
\begin{equation}
T:\ms{Vect}_{\mathbb{K}}\to\ms{Alg}_{\mathbb{K}}
\end{equation}
is a functor encoding our procedure.

Lets stop and think for a second: we have a functor $T$ and an
object $V\in\ms{Vect}_{\mathbb{K}}$. How can we construct a
universal arrow with this information? Well, we can rephrase the
question thus: what is an initial object in the comma category $(V\downarrow{}T)$?

Any linear transformation $f:V\to{}A$ can be uniquely extended to
an algebra homomorphism $\widetilde{f}$ from $T(V)$ to $A$, diagramatically
depicted as
\begin{equation}\label{eq:universalPropertyForTensorAlgebra}
\vcenter{\xymatrix{
V\ar[dr]_{f}\ar[r]^{i} & T(V)\ar@{-->}[d]^{\widetilde{f}}\\
& A
}}
\end{equation}
where $i$ is the canonical inclusion of $V$ into $T(V)$. Or to
phrase it in the more familiar manner: for each linear
transformation $f:V\to{}A$ \emph{there exists} an \emph{unique} algebra
homomorphism $\widetilde{f}:T(V)\to{}A$ \emph{such that} the
diagram in eq \eqref{eq:universalPropertyForTensorAlgebra}
commutes. Note this uniqueness is upto isomorphism. This
concludes our lengthy example.
\end{ex}

