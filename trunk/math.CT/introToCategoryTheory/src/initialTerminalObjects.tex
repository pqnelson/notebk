%%
%% initialTerminalObjects.tex
%% 
%% Made by Alex Nelson
%% Login   <alex@tomato>
%% 
%% Started on  Sat Jul 18 12:03:44 2009 Alex Nelson
%% Last update Sat Jul 18 12:03:44 2009 Alex Nelson
%%

The notions of initial and terminal objects are very useful later
on when thinking about universal arrows --- that is, whenever we
have phrases like ``\ldots\emph{there exists} a \emph{unique}
function such that\ldots'' should ring an alarm that we're
working with a universal arrow, which is an initial object in
some category.

\begin{defn}%\label{defn:}
An object $0\in\ms{C}$ is called an \define{Initial Object} if, for every
object $A\in\ms{C}$, there is exactly one arrow $0\xrightarrow{\;\;!\;\;}A$.
\end{defn}

This is fairly simple as a definition: an initial object is
mapped to every object in the category, including itself (by the
identity morphism).

\begin{defn}%\label{defn:}
An object $1\in\ms{C}$ is called a \define{Terminal Object} if, for every
object $A\in\ms{C}$, there is exactly one arrow from $A\xrightarrow{\;\;!\;\;}1$.
\end{defn}

Note we denote morphisms to initial (respectively, from terminal)
objects by $!$.


\begin{ex}
In $\ms{Set}$, the empty set $\emptyset=\{\}$ is the only initial
object. For every set $S$, the empty function is the unique
function from $\emptyset\to S$.

In $\ms{Set}$, each one-element set is a terminal object. Why?
Well, for each set $S\in\ms{Set}$ there is a function from $S$ to
a one element set $\{x\}$ mapping every element of $S$ to $x$
(the constant function).
\end{ex}

\begin{ex}
In $\ms{Grp}$, the trivial group $G=\{e\}$ is the initial object,
since there is exactly one group homomorphism from $G$ to every
group $G'\in\ms{Grp}$. 
\end{ex}

\begin{ex}
In $\ms{Cat}$, the category $\ms{0}$ is the initial object and
$\ms{1}$ is the terminal object for the exact same reasoning that
$\emptyset$ and $\{x\}\in\ms{Set}$ are initial (terminal) objects
(respectively).  
\end{ex}

\begin{ex}
If we think of a topological space $(X,T)$ (where $T$ is a
topology on $X$) as a category whose objects are open sets
$U\in{}T$, we have the morphisms be inclusions --- i.e. $U\to{}V$
iff $U\subseteq{}V$. Then $\emptyset$ is an initial object, and
$X$ is a terminal object.
\end{ex}

\begin{defn}%\label{defn:}
If $C\in\ms{C}$ is both an initial and a terminal object, then it
is called a \define{Null Object}.
\end{defn}
