\documentclass{amsart}
%\documentclass{amsbook}
\usepackage{ifthen}
\newboolean{isBook}
%\setboolean{isBook}{true} % set true if using amsbook
\setboolean{isBook}{false} % set true if using amsbook
% TODO:
%       introduce reviews of each movement,
%       monads and algebras of monads,
%       monoidal categories,
%       groupoids?,
%       categorification (which I think would be a natural way to
%       introduce n-categories), 
%       introduce more examples and applications (I've got an
%       appendix on the calculus of commutative diagrams that
%       needs to be typed up too),
\usepackage{manfnt}
\usepackage{hyperref}
\hypersetup{
    colorlinks,%
    citecolor=black,%
    filecolor=black,%
    linkcolor=black,%
    urlcolor=black
}
\usepackage{amsthm,amsmath,amssymb,amsfonts}
\usepackage{float,graphicx}
\usepackage{mathrsfs} % for mathscr used in defining a basis
\usepackage{verbatim}
\usepackage{paralist}
\usepackage{framed}
\usepackage[all,cmtip]{xy}




\theoremstyle{plain}
\newtheorem*{dual}{Dual Principle for Categories}
\theoremstyle{definition}
\newtheorem{defn}{Definition}[section]
\newtheorem{thm}[defn]{Theorem}
\newtheorem{rmk}[defn]{Remark}
\newtheorem{lem}[defn]{Lemma}
\newtheorem{cor}[defn]{Corollary}
\newtheorem{ex}[defn]{Example}
\newtheorem{nonex}[defn]{Non-Example}
\newtheorem{prop}[defn]{Proposition}
\newtheorem{sch}[defn]{Scholium}
\newtheorem{con}[defn]{Conjecture}
\newtheorem*{prob}{Problem}
\newtheorem*{quest}{Question}
\newtheorem*{defn*}{Definition}
\newtheorem*{yoneda}{Yoneda Lemma}

\def\re{\operatorname{Re}}
\def\tr{\operatorname{Tr}}
\def\<{\langle}
\def\>{\rangle}

%%
% This macro header is what controls the ``dangerous bend''
% paragraph
%%
\def\rd{\noindent\begingroup\hangindent=3pc\hangafter=-2\def\par{\endgraf\endgroup}\hbox
  to0pt{\hskip-\hangindent\dbend\hfill}\ignorespaces}
%%
% This command allows you to write stuff in small font size and
% use the
% bourbaki ``dangerous bend'' so it's great when you want to
% ramble on 
% about some extra stuff!
%%
\newcommand{\danger}[1] {\rd{\small {#1}}}

%%
% This macro header is what controls the ``dangerous bend''
% paragraph
%%
\def\ddbend{\dbend\kern1pt\dbend}

\def\rdd{\noindent\begingroup\hangindent=4pc\hangafter=-2\def\par{\endgraf\endgroup}\hbox
  to0pt{\hskip-\hangindent\ddbend\hfill}\ignorespaces}
\newcommand{\ddanger}[1] {\rdd{\small {#1}}}

% if using amsmidx
%\let\oldindex\index
%\renewcommand{\index}{\oldindex{Index}}

\newcommand{\define}[1]{``\textbf{#1}''\index{#1}}
\newcommand{\id}[1]{\operatorname{id}_{#1}}
\newcommand{\ob}[1]{\operatorname{Ob}(#1)}
\newcommand{\cat}[1]{\textbf{#1}\index{Category!#1}}
\newcommand{\nat}[1]{\operatorname{Nat}(#1)}
\let\ms\mathbf


\setlength{\marginparwidth}{1.2in}
\let\oldmarginpar\marginpar
\renewcommand\marginpar[1]{\-\oldmarginpar[\raggedleft\footnotesize #1]%
{\raggedright\footnotesize #1}}
\let\noparskip\relax 

\setlength{\parskip}{0cm}

\ifthenelse{\boolean{isBook}}
{\numberwithin{equation}{chapter}}
{\numberwithin{equation}{section}}

\ifthenelse{\boolean{isBook}}{}
{\let\chapter\part}

\renewcommand{\hom}{\operatorname{Hom}}
\let\Hom\hom
\let\ss\scriptsize
%\let\oldxymatrix\xymatrix
%\renewcommand{\xymatrix}[1]{\vcenter{\oldxymatrix{#1}}}
\title{Notes on Category Theory}
\date{June 12, 2009} % patched together from various notes from
                     % the end of school, so it started June 12
                     % in my book...
\email{pqnelson@gmail.com}
\author{Alex Nelson}
\begin{document}
\maketitle\footnotetext{Last Updated: \today}
\tableofcontents
\listoftables
%%%%%%%%%%%%%%%%%%%%%%%%%%%%%%%%%%%%%%%%%%%%%%%%%%%%%%
%% INTRO
%%%%%%%%%%%%%%%%%%%%%%%%%%%%%%%%%%%%%%%%%%%%%%%%%%%%%%
\section{Introduction}
%%
%% intro.tex
%% 
%% Made by alex
%% Login   <alex@tomato>
%% 
%% Started on  Mon May 21 15:09:58 2012 alex
%% Last update Wed May 30 10:11:39 2012 Alex Nelson
%%
\chapter*{\phantomsection\addcontentsline{toc}{chapter}{Introduction}Introduction}

\textbf{Warning:\quad\ignorespaces} This introduction may be
skipped for most readers. The intended audience are logicians
interested in the foundations used throughout, and people
attempting to write ``formal mathematics''.

We will be working without much ``formal'' framework. That is, we
are largely working symbolically and not rigorously. The logical
foundation made could be described as ``High School Algebra''. We
use variables (uncontroversial for logicians). Our operations
are:
\begin{enumerate}
\item Addition. We have $a+b$ be commutative, so $a+b=b+a$ and
  associatve $(a+b)+c=a+(b+c)$. There is an identity element $0$
  such that $0+a=a$ for any $a$. Negation produces the additive
  inverse $-(a)=-a$. 
\item Subtraction. This is just adding by the additive inverse:
  $a-b = a+(-(b))$. 
\item Multiplication. Written $a\cdot{b}$, it's commutative
  $a\cdot{b}=b\cdot{a}$ and associative
  $a\cdot(b\cdot{c})=(a\cdot{b})\cdot{c}$. Its identity element
  is denoted $1\cdot{a}=a$ for any $a$. Multiplicative inverse is
  denoted $1/a$ for any nonzero $a$ (i.e., $a\not=0$).
\item Division. This, like subtraction, multiplies by the
  multiplicative inverse.
\item Exponentiation. We write $a^b$. It is the first
  noncommutative operation $a^b\not=b^a$ and it is not
  associative $a^{(b^{c})}\not=(a^{b})^{c}$. It has an identity
  element in the sense that $a^{1}=a$ for any $a$.
\end{enumerate}
Tacitly, we have an ordering of numbers:
\begin{equation}
a<b\iff b-a\mbox{ is positive}
\end{equation}
We have an ordered field. 

The astute student would realize that exponentiation could have
two inverses: the logarithm and the $n^{\rm th}$-root. We say
\begin{equation}
\log_{b}(x)=y\iff b^{y}=x.
\end{equation}
The root-approach specifies for any number $a$ another number is
produced $a^{1/n}$ satisfying
\begin{equation}
(a^{1/n})^{n}=a.
\end{equation}
What about $\sqrt{-1}$? We run into difficulties: the root
approach gives us unreal (or \emph{imaginary}) numbers. Really,
we work with real numbers ``embedded'' in an ambient complex
number system.



\ifthenelse{\boolean{isBook}}{\part{Introducing Categories for Mathematicians}}{}

%%%%%%%%%%%%%%%%%%%%%%%%%%%%%%%%%%%%%%%%%%%%%%%%%%%%%%
%% A MATHEMATICIAN'S APPROACH (OVERTURE)
%%%%%%%%%%%%%%%%%%%%%%%%%%%%%%%%%%%%%%%%%%%%%%%%%%%%%%
\chapter{A Mathematician's Approach (\emph{Overture})}
\section{Mathematical Objects: Stuff, Structure, and Properties}
%%
%% object.tex
%% 
%% Made by Alex Nelson
%% Login   <alex@tomato>
%% 
%% Started on  Wed Jun 17 15:28:51 2009 Alex Nelson
%% Last update Wed Jun 17 17:43:46 2009 Alex Nelson
%%
\begin{prob}
Suppose we wanted to describe all of math by some common
``meta-structure''. How would we do it? It has to be general enough to
encompass all of the diverse aspects of every mathematical field,
but the danger is becoming \emph{too general}. This scares the
physicists. 

An alternate way to think of the problem is perhaps
this programming problem: we want to program an object oriented
computer algebra system with a common base class, with
functions/homomorphisms/operators/etc as instances of a 
{\tt function} class which extends our base class. Is there an
elegant way to solve this problem?
\end{prob}

So, to answer the initial problem, how can we describe some
generic mathematical object? We will follow the examples of
Baez~\cite{Baez:2004pa}, Baez and Wise~\cite{BaezWise:2004ln},
Baez and Shulman~\cite{baez-2006}, and Baez and Dolan~\cite{baez-2001-1} in
the answer to this question. Well, in linear algebra when we
introduce the notion of a vector space, it is defined as the set
of vectors. Similarly, in abstract algebra, a group is defined as
a set with a binary operator with various properties. The
recurring theme appears to be that there is some set underlying
the mathematical object, but lets not be so strict. Lets instead
give the following proposition:
\begin{prop}%\label{prop:}
A \define{Mathematical Object} is defined by at least specifying
some underlying ``\emph{stuff}'' (e.g. a set, several sets, etc.).
\end{prop}
Well, returning to our example of a vector space, what else makes
this set a vector space? There is some ``extra structure'' that
makes it so. Specifically we can do two things: 1) we can add two
vectors together to get a third vector, 2) we can multiply a
vector by a scalar to get another vector. These are binary
operators, which really are just functions
\begin{equation}%\label{eq:}
f:X\times Y\to Z
\end{equation}
where $X$, $Y$, and $Z$ are the underlying sets, $f$ is the
binary operator in question. It's just that in practice, it looks
kinda funny writing $+(\vec{x},\vec{y})$ for vector addition.
This suggestion of binary operators as functions is -- at first
glance -- foreign. So, being general (again), we suggest that
there are functions, relations, some specified elements, etc. all
fit in this ``extra structure'' description. In e.g. a topology
or a sigma algebra, we are worried about collections of subsets,
which should be taken into account as ``extra structure'' as well
since we are working with distinguished subsets. So to be fully
general we propose the following revised proposition
\begin{prop}%\label{prop:}
A \define{Mathematical Object} is defined by at least specifying
\begin{enumerate}
\item some underlying ``\emph{stuff}'' (e.g. a set, several sets, etc.)
\item equipping this ``stuff'' with some ``\emph{structure}''
  (e.g. functions, binary operators, collections of subsets,
  distinguished elements, etc.).
\end{enumerate}
\end{prop}
This still is not enough. We can't just have any old
``structure'', we need to specify conditions our desired
structure satisfies. If we are considering a binary operator, is
our ``stuff'' closed under this binary operator? Is there some
identity element $e$ so when we consider the binary operator
applied to $e$ and some $x$ (i.e. $f(e,x)$) we end up with our
$x$? If so, does every $x$ in our stuff have an inverse? And so
on, the list is endless.

These conditions are really just demanding certain equations (or
in some cases inequalities or inclusions) holds. These are just
algebraically described as ``\emph{properties}'' of our
structure. This is enough to give a fully general account of a
mathematical object:
\begin{framed}
\begin{defn}\label{defn:object}
\addcontentsline{toc}{section}{*\quad\;Definition: Mathematical Object}
A \define{Mathematical Object} is defined by:
\begin{enumerate}
\item specifying some underlying ``\emph{stuff}'' (e.g. a set, several sets, etc.)
\item equipping this ``stuff'' with some ``\emph{structure}''
  (e.g. functions, binary operators, collections of subsets,
  distinguished elements, etc.).
\item with this ``structure'' satisfying certain
  ``\emph{properties}'' (e.g. equations, inequalities, etc.).
\end{enumerate}
\end{defn}
\end{framed}
This is a sufficient generalized notion of a mathematical
object. We can use this in the definition of a category.

\begin{rmk}
If one really pushed, I doubt that any of these conditions can be
rigorously defined (what do you mean by ``equations''?
``functions''? etc.). The point is, as with every definition, to
give some intuition behind the concept as well as some defining
characteristics of the object. But if one really pushed, I
suspect this is where Godel's incompleteness comes into play.
\end{rmk}

\begin{rmk}
As Baez and Wise point out~\cite{BaezWise:2004ln}, we can always
\emph{check} the properties -- they are either true or false. We
can also \emph{choose} structures from a set of
possibilities. And we can \emph{choose} stuff from a category of
possibilities. But each step depends on the following
ones. Structure depends on stuff, and properties depends on
structure. This should be somewhat intuitive, we can't demand
conditions on structure we don't have, nor can we have structure
depend on stuff we don't have.
\end{rmk}

\begin{prop}
Every definition in math is defining one of the following: a
mathematical object, some ``stuff'', some ``structure'', or some
``property''.
\end{prop}
\begin{con} In relating definitions to mathematical objects, we
  conjecture:
\begin{enumerate}
\item The definition of some ``stuff'' or ``structure'' can be
  reformulated as a definition of a mathematical object.
\item The definition of some ``property'' can be reformulated as
  a definition of some ``structure''; moreover, it can be
  reformulated as a definition of a mathematical object.
\end{enumerate}
Or every definition defines a mathematical
object --- defining a mathematical object is sometimes more
circuitous and not as interesting or useful. 
\end{con}
This conjecture is summarized in the single phrase
``\emph{everything defined in math is-a mathematical object}''.

\section{Examples of Mathematical Objects}
%%
%% examples.tex
%% 
%% Made by Alex Nelson
%% Login   <alex@tomato>
%% 
%% Started on  Wed Jun 17 17:18:17 2009 Alex Nelson
%% Last update Wed Jun 17 18:02:44 2009 Alex Nelson
%%

Now that we have introduced the notion of a ``mathematical
object'', perhaps we should start examining
mathematical objects. Well, it turns out that it's all of math, so perhaps we
should consider certain examples. It's only really necessary to
look at a few examples, it turns out that in categories the
objects play second fiddle to the morphisms.

\subsection{A Topology on a Set}\label{ex:topologyAsObject}

Topologies are interesting, they specify the open subsets of a
given set $X$. Recall that a topology $\mathcal{T}$ on $X$ is a
collection of subsets of $X$ having the following properties:
\begin{enumerate}
\item $\emptyset$ and $X$ are in $\mathcal{T}$
\item The union of the elements of any subcollection of
  $\mathcal{T}$ are in $\mathcal{T}$
\item The intersection of the elements of any finite
  subcollection of $\mathcal{T}$ is in $\mathcal{T}$.
\end{enumerate}
This is the definition of a topology on $X$.

Now we want to demonstrate that it is a mathematical object.
\begin{description}
\item[Stuff] The underlying set $X$ is the stuff topologies are
  made of...
\item[Structure] We have some extra structure by considering a
  collection of subsets of $X$.
\item[Properties] We just listed what the properties of this
  structure should be! It's closed under arbitrary union of
  elements of $\mathcal{T}$, finite closure of elements of $\mathcal{T}$, and the collection
  contains both the empty set and $X$ itself.
\end{description}
\noindent So we see that a topology is a mathematical object.

\subsection{Binary Operator on a Set}

Let $S$ be a set, a binary operation on $S$ is a binary relation
that maps elements of the Cartesian product $S\times S$ to $S$:
\begin{equation}%\label{eq:}
f:S\times S\to S.
\end{equation}
The underlying argument is a generalization of the argument that
a function is a mathematical object. In fact, the argument is
exactly the same replacing $X$ with $S\times S$, and $Y$ with
$S$. But since a function was demonstrated as the first example
of a mathematical object in this paper, it follows that a binary
operator on a set is a mathematical object.

\subsection{Groups}

Let $(G,\cdot)$ be a group with elements $G$ and binary operator
represented as multiplication. A group is closed under its binary
operator and necessarily has inverses exist. We will show that
this is clearly a mathematical object:

\begin{description}
\item[Stuff] The underlying set $G$ is the stuff.
\item[Structure] The binary operator acting on $G$ is our
  structure.
\item[Properties] We demand certain properties hold. First for
  any $x\in G$, there is a unique $e\in G$ such that
  $xe=ex=x$. Second, for each $x\in G$, there is a corresponding
  $y\in G$ such that $xy=yx=e$. Third, for each $x,y\in G$, we
  demand that $xy\in G$.
\end{description}
\noindent The pattern is kind of clear how to get from a
definition to our grocery list of ``stuff-structure-properties''.
This is how most people do abstract math, with such grocery
lists.

\section{Quick and Painful Introduction to Categories}
The motivating problem is that we want to look at mappings 
between mathematical objects, but it may not always make sense.
So we want to have two things: (1) mappings between objects, and 
(2) a setting where it make sense.

The heart of mathematics is definitions, so we will begin by
bluntly defining a category:
\begin{framed}
\begin{defn}\label{defn:category}
A \define{Category} $\ms{C}$ consists of
\begin{enumerate}
\item a collection $\ob{\ms{C}}$ of \define{Objects};
\item for any pair of objects $x,y\in\ob{\ms{C}}$, we have a set
  $\hom(x,y)$ of \define{morphisms} from $x$ to $y$ (so if
  $f\in\hom(x,y)$, then $f:x\to y$)
\end{enumerate}
equipped with
\begin{enumerate}
\item for any object $x\in\ob{\ms{C}}$, an \define{identity morphism}
  $\id{x}:x\to x$;
\item for any pair of morphisms $f:x\to y$ and $g:y\to z$, a
  morphism $g\circ f:x\to z$ called the \define{composition} of
  $f$ and $g$ (note it's written and read from right to left,
  like Chinese, to confuse undergraduates)
\end{enumerate}
such that
\begin{enumerate}
\item for any morphism $f:x\to y$, the \define{left and right
  unit laws} hold: $f\circ\id{x}=f=\id{y}\circ f$;
\item for any triple of morphisms $f:w\to x$, $g:x\to y$, $h:y\to
  z$, the \define{associative law} holds: $(h\circ g)\circ f=h\circ(g\circ f)$.
\end{enumerate}
\end{defn}
\end{framed}
This is not all too enlightening, at least immediately. It'd be a
pity to end here, state ``We will not insult the intelligence of
the reader with examples or theorems, since everything follows
immediately'', and pronounce the reader a \emph{bona fide} expert
category theorist. We will gently introduce various examples before
moving on to start thinking categorically.

\begin{rmk}
The composition of morphisms $g\circ f$ should be read from right
to left (like Chinese or bra-ket notation in quantum mechanics)
and it should be intuitively regarded as
\begin{equation}%\label{eq:}
\text{first do } f \text{ then do }g.
\end{equation}
So do not confuse it with everyday multiplication, \emph{order matters here!}
\end{rmk}

\begin{rmk}
With the arrows, or ``morphisms", we have
\begin{equation}%\label{eq:}
\operatorname{Source}(f)\xrightarrow{\;\;f\;\;}\operatorname{Target}(f)
\end{equation}
or (more categorically, with \hyperref[dualityPrinciple]{duality} in mind) 
\begin{equation}%\label{eq:}
\operatorname{Domain}(f)\xrightarrow{\;\;f\;\;}\operatorname{Codomain}(f)
\end{equation}
where $f$ is a morphism, and the domain/source of $f$ is an
object, the target/codomain of $f$ is also an object.
\end{rmk}

% If this is the first time one has been introduced to
% categories, or if one feels insecure with the nature of
% commutative diagrams, it is suggested to skip ahead to the
% appendix on the calculus of commutative diagrams.

Wait, we just gave a definition! We can show, as with our conjecture
that everything ``is-a'' mathematical object, that a category and
a morphism both are mathematical objects! We will only show that a
category is an object for now (something to bear in mind: what's a 
morphism \emph{between} categories?).

We want to have a taxonomy of morphisms, so we begin very simply
by supposing what happens when we have the domain be the
codomain.
\begin{defn}%\label{defn:}
Let $\ms{C}$ be a category, and $x\in\ob{\ms{C}}$. We define an
\define{endomorphism} to be a morphism $f\in\hom(x,x)$. That is
\begin{equation}%\label{eq:}
f:x\to x
\end{equation}
so its domain is its codomain.
\end{defn}

Now, intuitively the notion of composition of morphisms should be
likened to multiplication. We have a notion of an inverse operation for
multiplication, we call it division. What about the notion of an
inverse for a morphism? That is, given some category
$\ms{C}$, some objects $x,y\in\ms{C}$, and a morphism
$f\in\hom(x,y)$, can we find a $g:y\to x$ such that
\begin{equation}\label{eq:rightInverse}
f\circ g = \id{y}
\end{equation}
and if so, is it true that
\begin{equation}\label{eq:leftInverse}
g\circ f = \id{x}?
\end{equation}
Here the analogy to commutative multiplication should be discarded. The
morphisms represent processes, and the order of processes
matter. Consider for example $h$ being the morphism of mixing the
ingredients of a cake, and $i$ being the process of pouring the
contents of the mixing bowl into a pan and placing the pan in an
oven preheated to 350 degrees Fehrenheit. Well, $i\circ h$ is
baking the normal way, $h\circ i$ is putting the pan in the oven,
and then mixing the ingredients together.

Consequently we don't expect Eq \eqref{eq:rightInverse} to necessarily imply
Eq \eqref{eq:leftInverse} (nor do we expect Eq
\eqref{eq:leftInverse} to necessarily imply Eq \eqref{eq:rightInverse}). We
have two new definitions:
\begin{defn}%\label{defn:}
Let $\ms{C}$ be a category, $x,y\in\ob{\ms{C}}$,
$f\in\hom(x,y)$. Then we define:
\begin{enumerate}
\item $f$ be a \define{monomorphism} (or \define{monic}) if for
  morphisms $g_{1},g_{2}:x\to \omega$ we have $f\circ g_1=f\circ
  g_2$ imply $g_1=g_2$;
\item a \define{epimorphism} (or \define{epic}) if for morphisms
  $g_{1},g_{2}:z\to x$ we have $g_1\circ f=g_2\circ f$ imply $g_1=g_2$;
\item the \define{left inverse} or \define{retraction of $f$} to be a morphism
  $g\in\hom(y,x)$ such that $g\circ f=\id{x}$; 
\item the \define{right inverse} or \define{section of $f$} to be a morphism
  $h\in\hom(y,x)$ such that $f\circ h=\id{y}$.
\end{enumerate}
\end{defn}
\begin{rmk}
Epic is more general than having a right inverse, and monic is
more general than having a left inverse. Consider the following
situation: we have a category with two objects $A$, $B$, three
morphisms (the identity morphisms $\id{A}$, $\id{B}$, and $f:A\to
B$). The morphism $f$ is epic since it can be composed on the
left by exactly one morphism --- $\id{B}$. Similarly, when it can
be composed on the right by exactly one morphism $\id{A}$. This
is more or less a trivial situation that is true when we can only
compose by the identity on the left or on the right. Whenever
this is not true, and a morphism is both epic and monic, it has
left and right inverses.
\end{rmk}

There is one last important type of morphism to consider, it's a
way to state that two objects are ``the same'' in some sense. The
idea is if there is an invertible morphism between two objects,
it's necessarily ``equivalent'' in some sense. Why should this be
so? Well, an invertible mapping requires having the left and
right inverses (a) exist and (b) be the same. 

\begin{defn}%\label{defn:}
Let $\ms{A}$ be a category, $X,Y\in\ob{\ms{A}}$, and $f:X\to
Y$. We say $f$ is an \define{isomorphism} if there is a morphism
$g:Y\to X$ such that $g\circ f=\id{X}$ and $f\circ g=\id{Y}$.
\end{defn}

We also have an invertible endomorphism being given a special name:

\begin{defn}%\label{defn:}
Let $\ms{A}$ be a category, $X\in\ob{\ms{A}}$, and $f:X\to
X$ be an isomorphism. Then $f$ is called an \define{automorphism}.
\end{defn}

To summarize this section, we have this handy table of morphism
properties: 

\begin{table}[H]
\begin{tabular}{| p{3.5cm} | p{6cm} |}
\hline
\textbf{Morphism Property} & \textbf{What it means}\\\hline
Automorphism & Invertible endomorphism. \\\hline
Endomorphism & The domain and codomain are the same.\\ \hline
Epimorphism & Generalization of having a right inverse.\\ \hline
Isomorphism & Generalization of equivalence of domain with codomain.\\\hline
Monomorphism & Generalization of having a left inverse.\\ \hline
Retraction of $f$ & Left Inverse of $f$\\ \hline
Section of $f$ & Right Inverse of $f$\\\hline
\end{tabular}
\caption{A Table of Morphism Properties}\label{tab:morphisms}
\end{table}

\subsection{Categories Are Also Mathematical Objects} 

It seems trivial to demonstrate that categories are mathematical
objects, but for the sake of rigor we will demonstrate it. 
\begin{description}
\item[Stuff] What is the stuff? Well, there is the collection of
  objects. Additionally, for any pair of objects, we have a set
  of morphisms.
\item[Structure] The structure of a category is simply this: 1) for
  any object, we have an identity morphism; 2) for any pair of
  morphisms, we have a composition of morphisms.
\item[Properties] Each component of our structure has a property
  to satisfy. The identity morphism satisfies the left and right
  unit laws. The composition of morphisms satisfies the
  associativity law. We demand nothing more, nothing less.
\end{description}
\noindent This is precisely a mathematical object description of
a category.

\subsection{A Grocery List of Examples}

\begin{ex}We have a few examples to start with that are all pretty similar.

$\ms{0}$ the category with zero objects and zero morphisms. Well,
it's trivially a category, albeit a really boring one.

$\ms{1}$ the category with a single object, and a single morphism
(the identity morphism), is a category.

$\ms{2}$ the category with two objects $a$ and $b$, and just one 
morphism $a\to{}b$ that \emph{is not} an identity.

$\ms{3}$ the category with three objects $a$, $b$, and $c$ whose 
morphisms are $a\to{}b$, $b\to{}c$ and $c\to{}a$ (none are identities).

$\downdownarrows$ the category with two objects $a,b$ and 
two morphisms $a\rightrightarrows b$ that aren't the identity.
\end{ex}

A few properties of categories are worth noting here and now.

\begin{defn}\label{defn:thin}
A category is said to be \define{Thin} if there is at most one 
morphism from an object to another.
\end{defn}
\begin{defn}\label{defn:discrete}
A category is said to be \define{Discrete} if every morphism is
an identity.
\end{defn}
Discrete categories are really just sets since each object has
exactly one morphism, and aside from that there are no additional
morphisms.

We can continue introducing examples, one large class of examples
that the reader may be inclined to notice is that we have a lot of
mathematical objects that are ``sets equipped with structure'', so
why not have a category of such ``sets equipped with the same 
structure''? Like the category of all groups, or of all rings, or
of all sets? Such categories are called ``constructs'' or ``concrete''
depending on the author.

\begin{defn}\label{defn:concreteCategory}
A category $\ms{C}$ is called \define{Concrete} iff there is a functor
$U:\ms{C}\to\ms{Set}$ to the underlying sets of $\ms{C}$ called the
\define{Forgetful Functor} (it is faithful, we'll come back to this 
later).
\end{defn}

\begin{ex}\label{ex:Mon}
Consider $\cat{Mon}$, the category whose objects are monoids, and whose
morphisms are monoidal homomorphisms. It is concrete.
\end{ex}
\begin{ex}\label{ex:Grp}
Consider $\cat{Grp}$, the category whose objects are groups and whose
morphisms are group homomorphisms. It is concrete.
\end{ex}
\begin{ex}\label{ex:Set}
Consider $\cat{Set}$, the category whose objects are sets and whose
morphisms are functions. It is concrete trivially.
\end{ex}

We can also describe a group using category theory. That is given
some group, we can turn it into a category defined by a single
object and the elements of the group become morphisms in the
category. We'll give a few examples of these.
\subsection{Example: Complex Group}
%%
%% complexGroupEx.tex
%% 
%% Made by Alex Nelson
%% Login   <alex@tomato>
%% 
%% Started on  Tue Jun 30 23:13:30 2009 Alex Nelson
%% Last update Tue Jun 30 23:13:30 2009 Alex Nelson
%%
Consider the group generated by $\{1,i=\sqrt{-1}\}$ equipped with
multiplication. Explicitly, we have the multiplication table
\begin{equation}%\label{eq:}
\begin{array}{c|cccc}
\times & 1  & -1 & i  & -i\\\hline
1      & 1  & -1 & i  & -i \\
-1     & -1 & 1  & -i & i\\
-i     & -i & i  & 1  & -1\\
i      & i  & -i & -1 & 1 
\end{array}
\end{equation}
We have this categorified in the following diagram
\begin{equation}
\vcenter{
\xymatrix{            &                              & \\
\txt{*} \ar@/^2pc/[uurrdd]^{1}
\ar[d]_{i} \ar[dr]_{-1} \ar[r]^{-i} & \txt{*}               \ar[r]^{i} & \txt{*}  \\
\txt{*} \ar[r]_{i}            & \txt{*} \ar[r]_{i} \ar[u]_{i} \ar[ur]_{-1}& \txt{*} \ar[u]_{i}      }}
\end{equation}
%% \begin{equation}%\label{eq:}
%% \entrymodifiers={++[o][F-]}
%% \SelectTips{cm}{}
%% \xymatrix @-1pc {
%%  *\txt{start} \ar[r]
%%  & 0 \ar@(r,u)[]^b \ar[r]_a
%%  & 1 \ar[r]^b \ar@(r,d)[]_a
%%  & 2 \ar[r]^b
%%    \ar ‘dr_l[l] ‘_ur[l] _a [l]
%%  &*++[o][F=]{3}
%%    \ar ‘ur^l[lll]‘^dr[lll]^b [lll]
%%    \ar ‘dr_l[ll] ‘_ur[ll]    [ll] }

%% %% \xymatrix{

%% %% }
%% %% \begindc{0}[5]
%% %% \obj(0,10){$*$}
%% %% \obj(0,0){$*$}
%% %% \obj(10,0){$*$}
%% %% \obj(10,10){$*$}
%% %% \obj(20,0){$*$}
%% %% \obj(20,10){$*$}
%% %% \mor(0,10)(0,0){\scriptsize{$i$}}
%% %% \mor(0,0)(10,0){\scriptsize{$i$}}
%% %% \mor(10,0)(10,10){\scriptsize{$i$}}
%% %% \mor(0,10)(10,0){\scriptsize{$-1$}}
%% %% \mor(0,10)(10,10){\scriptsize{$-i$}}
%% %% \mor(10,10)(20,10){\scriptsize{$i$}}
%% %% \mor(10,0)(20,10){\scriptsize{$-1$}}
%% %% \mor(10,0)(20,0){\scriptsize{$i$}}
%% %% \mor(20,0)(20,10){\scriptsize{$i$}}
%% %% %\cmor((0,12)(1,13)(2,15)(6,16)(18,15)(19,13)(20,12))
%% %% \cmor((0,12)(3,15)(10,16)(17,15)(20,12))
%% %%   \pdown(10,17){\scriptsize{$1$}}
%% %% %  \pdown(10,17){\scriptsize{$id_{*}$}}
%% %% \enddc
%% \end{equation}
This encodes all the information about the group generated by
$\{1,i\}$. We can do the diagram chasing to see that the
multiplication is really encoded in our category theoretic
doodling.

%% \begin{equation}%\label{eq:}
%% \begindc{0}[50]
%% \obj(0,1){$*$}
%% \obj(0,0){$*$}
%% \obj(1,0){$*$}
%% \obj(1,1){$*$}
%% \obj(2,0){$*$}
%% \obj(2,1){$*$}
%% \mor(0,1)(0,0){\scriptsize{$i$}}
%% \mor(0,0)(1,0){\scriptsize{$i$}}
%% \mor(1,0)(1,1){\scriptsize{$i$}}
%% \mor(0,1)(1,0){\scriptsize{$-1$}}
%% \mor(0,1)(1,1){\scriptsize{$-i$}}
%% \mor(1,1)(2,1){\scriptsize{$i$}}
%% \mor(1,0)(2,1){\scriptsize{$-1$}}
%% \mor(1,0)(2,0){\scriptsize{$i$}}
%% \mor(2,0)(2,1){\scriptsize{$i$}}
%% \cmor((0,1)(0,2)(2,2)(2,1))
%% \pdown(1,2){\scriptsize{$id_{*}$}}
%% \enddc
%% \end{equation}

\subsection{Example: Quaternion Group}
%%
%% quaternionGroupEx.tex
%% 
%% Made by Alex Nelson
%% Login   <alex@tomato>
%% 
%% Started on  Tue Jun 30 23:51:16 2009 Alex Nelson
%% Last update Tue Jun 30 23:51:16 2009 Alex Nelson
%%
Consider the quaternions, which were characterized by the
equation
\begin{equation}%\label{eq:}
i^2 = j^2 = k^2 = ijk = -1
\end{equation}
We have the multiplication table
\begin{equation}%\label{eq:}
\begin{array}{c|cccccccc}
\times &  1  & -1  &  i  & -i  &  j  & -j  &  k  &  -k\\\hline
1      &  1  & -1  &  i  & -i  &  j  & -j  &  k  &  -k\\
-1     & -1  &  1  & -i  &  i  & -j  &  j  & -k  &   k\\
i      &  i  & -i  & -1  &  1  &  k  & -k  & -j  &   j\\
-i     & -i  &  i  &  1  & -1  & -k  &  k  &  j  &  -j\\
j      &  j  & -j  & -k  &  k  & -1  &  1  &  i  &  -i\\
-j     & -j  &  j  &  k  & -k  &  1  & -1  & -i  &   i\\
k      &  k  & -k  &  j  & -j  & -i  &  i  & -1  &   1\\
-k     & -k  &  k  & -j  &  j  &  i  & -i  &  1  &  -1
\end{array}
\end{equation}
We have various ways to encode the information in the quaternions
into diagrams, e.g.
%% \begin{equation}%\label{eq:}
%% jk=i\quad\iff\quad
%% \begindc{0}[5]
%% \obj(0,10){$*$}
%% \obj(0,0){$*$}
%% \obj(10,0){$*$}
%% \mor(0,10)(0,0){\scriptsize{$j$}}
%% \mor(0,0)(10,0){\scriptsize{$k$}}
%% \mor(0,10)(10,0){\scriptsize{$i$}}
%% \enddc
%% \end{equation}
\begin{equation}%\label{eq:}
jk=i\quad\iff\quad
\vcenter{
\xymatrix{
\txt{*}\ar[d]_{j}\ar[dr]^{i} & \\
\txt{*}\ar[r]_{k} & \txt{*}
}}
\end{equation}
We can cycle through the rest of the identities, or we can skip
ahead to the final diagram
\begin{equation}
\vcenter{
\xymatrix{            &                              & \\
\txt{*} \ar@/^2pc/[uurrdd]^{-1}
\ar[d]_{j} 
\ar[dr]_{k} 
\ar[r]^{i}       & \txt{*} \ar[r]^{i} \ar[d]_{j} & \txt{*}  \\
\txt{*}\ar@/_3pc/[dddrruuuu]_{j}             & \txt{*}\ar[l]^{i}   \ar[ur]_{k}          &
  }}
\end{equation}
%% \begin{equation}%\label{eq:}
%% \begindc{0}[5]
%% \obj(0,20){$*$}
%% \obj(10,20){$*$}
%% \obj(20,20){$*$}
%% \obj(10,10){$*$}
%% \obj(0,10){$*$}
%% \mor(0,20)(10,20){\scriptsize{$i$}}
%% \mor(10,20)(20,20){\scriptsize{$i$}}
%% \mor(0,20)(0,10){\scriptsize{$j$}}
%% \mor(0,20)(10,10){\scriptsize{$k$}}
%% \mor(10,10)(0,10){\scriptsize{$i$}}
%% \mor(10,20)(10,10){\scriptsize{$j$}}
%% \mor(10,10)(20,20){\scriptsize{$k$}}
%% \cmor((0,22)(3,25)(10,26)(17,25)(20,22))
%%   \pdown(10,27){\scriptsize{$-1$}}
%% \cmor((0,9)(3,5)(10,5)(16,9)(20,18))
%%   \pup(6,3){\scriptsize{$j$}}
%% \enddc
%% \end{equation}
With the understanding that $-1\circ-1=1\equiv\id{*}$.
%% \begin{equation}%\label{eq:}
%% \begindc{0}[5]
%% \obj(0,20){$*$}
%% \obj(10,20){$*$}
%% \obj(0,10){$*$}
%% \obj(10,10){$*$}
%% \obj(0,0){$*$}
%% \mor(0,20)(0,10){\scriptsize{$i$}}
%% \mor(0,20)(10,20){\scriptsize{$j$}}
%% \mor(0,20)(10,10){\scriptsize{$k$}}
%% \mor(10,10)(10,20){\scriptsize{$i$}}
%% \mor(0,10)(0,0){\scriptsize{$i$}}
%% \mor(10,10)(0,0){\scriptsize{$k$}}
%% \mor(0,10)(10,10){\scriptsize{$j$}}
%% \enddc
%% \end{equation}


\section{Functors}
Consider this: we have morphisms be generalizations of mappings
from one object to another, and we have categories be a
mathematical object. What is a mapping from one category to
another? We call such things \emph{functors} and they encode
mathematical procedures, or assigning information to objects and
morphisms.
%%%%%%%%%%%%%%%%%%%%%%%%%%%%%%%%%%%%%%%%%%%%%%%%%%%%%%%%%%%%%%%%%%%%%%%%%%%%%%%%
% DEFINITION OF FUNCTOR, COMPOSITION OF FUNCTORS
%%%%%%%%%%%%%%%%%%%%%%%%%%%%%%%%%%%%%%%%%%%%%%%%%%%%%%%%%%%%%%%%%%%%%%%%%%%%%%%%

\begin{defn}\label{defn:functor}
Given categories $\ms{C},\ms{D}$ a \define{functor}
$F:\ms{C}\to\ms{D}$ consists of
\begin{itemize}
\item a function $F:\ob{\ms{C}}\to\ob{\ms{D}}$;
\item for any pair of objects $X,Y\in\ob{\ms{C}}$, a function
  $F:\hom(X,Y)\to\hom(F(X),F(Y))$;
\end{itemize}
such that 
\begin{description}
\item[$F$ preserves identities] for any $X\in\ms{C}$,
  $F(1_{X})=1_{F(X)}$;
\item[$F$ preserves composition] for any pair of morphisms
  $f:X\to Y$ and $g:Y\to{}Z$ in $\ms{C}$, $F(fg)=F(f)F(g)$.
\end{description}
\end{defn}
\begin{rmk}
It is not uncommon to see expressions like
\begin{equation}
F\left(X\xrightarrow{\;\;f\;\;}Y\right) = F(X)\xrightarrow{\;\;F(f)\;\;}F(Y)
\end{equation}
in practice, which tells us how the functor behaves.

We can compose two functors in the obvious way, if
$F:\ms{B}\to\ms{C}$ and $G:\ms{C}\to\ms{D}$, then
$G\circ{}F:\ms{B}\to\ms{D}$ is defined by
\begin{equation}
G\circ{}F\left(X\xrightarrow{\;\;f\;\;}Y\right) = (G\circ{}F)(X)\xrightarrow{\;\;(G\circ{}F)(f)\;\;}(G\circ{}F)(Y)
\end{equation}
componentwise composition of the functor on objects and the
functor on morphisms.

Also, we have the intuition that a functor is a ``morphism
between categories''. So we'll often use the adjectives and
definitions for morphisms for functors, e.g. an endofunctor has
its codomain and domain be the same category, an isomorphism
between categories is an invertible functor, etc.
\end{rmk}

%%%%%%%%%%%%%%%%%%%%%%%%%%%%%%%%%%%%%%%%%%%%%%%%%%%%%%%%%%%%%%%%%%%%%%%%%%%%%%%%
% FUNCTORS PRESERVE ISOMORPHISMS
%%%%%%%%%%%%%%%%%%%%%%%%%%%%%%%%%%%%%%%%%%%%%%%%%%%%%%%%%%%%%%%%%%%%%%%%%%%%%%%%
\begin{thm}
Let $F:\ms{C}\to\ms{D}$ be a functor, and $f:A\to{}B$ be an
isomorphism in $\ms{C}$. Then $F(f)$ is an isomorphism in $\ms{D}$
\end{thm}
\begin{proof}
The proof is straightforward, we see that
\begin{subequations}
\begin{align}
F(f)\circ{}F(f^{-1}) &= F(f\circ{}f^{-1})\\
&= F(\id{B}) = \id{F(B)}
\end{align}
\end{subequations}
and similarly $F(f^{-1})\circ{}F(f)=\id{F(A)}$ which implies that
$F(f^{-1})$ is the two-sided inverse of $F(f)$ --- i.e. $F(f)$ is
an isomorphism in $\ms{D}$.
\end{proof}

This is a powerful tool, since we often will take the approach in
disproving something is a functor by showing that $f$ is an
isomorphism but $F(f)$ is not.

NOTE: just because $F(k)$ may be an isomorphism, it \emph{in no way implies ANYTHING} about $k$ being an isomorphism. It may
possibly be an isomorphism, it may not. This property ($F(k)$ is
an iso $\Rightarrow$ $k$ is an iso) is called ``reflection'' of
isomorphisms, it is not necessarily true for functors. 

%%%%%%%%%%%%%%%%%%%%%%%%%%%%%%%%%%%%%%%%%%%%%%%%%%%%%%%%%%%%%%%%%%%%%%%%%%%%%%%%
% IDENTITY FUNCTORS
%%%%%%%%%%%%%%%%%%%%%%%%%%%%%%%%%%%%%%%%%%%%%%%%%%%%%%%%%%%%%%%%%%%%%%%%%%%%%%%%
We have a number of examples and properties of functors. The most
important example to consider is the identity functor.
\begin{ex}\label{ex:identityFunctor}
Let $\ms{C}$ be some category, with $X,Y\in\ms{C}$ and
$f:X\to{}Y$. Let $F=\id{\ms{C}}:\ms{C}\to\ms{C}$ be the identity
functor. Then
\begin{equation}
F\left(X\xrightarrow{\;\;f\;\;}Y\right) = X\xrightarrow{\;\;f\;\;}Y
\end{equation}
just as one expects.
\end{ex}
Why is this an important example? Well, we can introduce the
notion of inverse functors now. That is, if $F:\ms{C}\to\ms{D}$
we can ask if there is a $G:\ms{D}\to\ms{C}$ such that
$G\circ{}F=\id{\ms{C}}$ and $F\circ{}G=\id{\ms{D}}$? If we can
find such a $G$ then $F$ is invertible or more precisely an
``\emph{isomorphism}''. How do we know when we are dealing with
isomorphisms? Well, the intuition to have is that they are
analogies. That is, we can set up tables of objects corresponding
uniquely to objects, and morphisms corresponding uniquely to
morphisms. 

%%%%%%%%%%%%%%%%%%%%%%%%%%%%%%%%%%%%%%%%%%%%%%%%%%%%%%%%%%%%%%%%%%%%%%%%%%%%%%%%
% ISOMORPHISM FUNCTOR
%%%%%%%%%%%%%%%%%%%%%%%%%%%%%%%%%%%%%%%%%%%%%%%%%%%%%%%%%%%%%%%%%%%%%%%%%%%%%%%%
\begin{defn}\label{defn:isomorphismFunctor}
Let $F:\ms{C}\to\ms{D}$ be a functor such that there is a functor
$F^{-1}:\ms{D}\to\ms{C}$ satisfying (1)
$F^{-1}\circ{}F=\id{\ms{C}}$, (2)
$F\circ{}F^{-1}=\id{\ms{D}}$. Then we define $F$ to be an
\define{Isomorphism} and say that $\ms{C}$ is \define{isomorphic}
to $\ms{D}$.
\end{defn}

Note that isomorphic categories are ``essentially the same''. We
have analogous objects and analogous morphisms between them. The
intuition one should have is that an isomorphism functor sets up
an analogy between two categories.

\begin{ex}\label{ex:pointAsFunctor}
We have a functor $X:\ms{1}\to\ms{C}$ be a \define{point}, that
is it picks out a single object $X(*)\in\ms{C}$. Most of the
times, we use the notation that $X(*)=X$ the point's name is the
object it points out. We then have an interesting point of view
that \emph{an object is a functor}.
\end{ex}
\begin{ex}\label{ex:morphismAsFunctor}
We have $F:\ms{2}\to\ms{C}$ be a functor which is then just
defined by
\begin{equation}
F\left(a\xrightarrow{\;\;f\;\;}b\right)=F(a)\xrightarrow{\;\;F(f)\;\;}F(b)
\end{equation}
which just specifies a morphism in $\ms{C}$. So the morphisms
from $\ms{2}$ to $\ms{C}$ picks out morphisms.
\end{ex}
\begin{ex}\label{ex:morphismAsFunctor}
We have $F:\ms{3}\to\ms{C}$ be a functor. What does it do? Well,
it takes in the category with 3 objects and 3 distinct,
non-identity morphisms, and it spits out at most 3 objects and 3
morphisms in $\ms{C}$. The functor has to obey the composition of
morphisms, so this specifies one composite morphism in
$\ms{C}$. It is then just defined by
\begin{equation}
F\left(a\xrightarrow{\;\;f\;\;}b\xrightarrow{\;\;g\;\;}c\right)=F(a)\xrightarrow{\;\;F(f)\;\;}F(b)\xrightarrow{\;\;F(g)\;\;}F(c)
\end{equation}
which just specifies a morphism in $\ms{C}$. This defines a
composite morphism $F(g)\circ{}F(f):F(a)\to{}F(c)$. So the morphisms
from $\ms{3}$ to $\ms{C}$ picks out composite morphisms and its components.
\end{ex}
\begin{ex}\label{ex:homFunctor}
Consider some arbitrary category $\ms{C}$. We can construct a
functor from $\ms{C}\to\ms{Set}$ by choosing some object
$C\in\ms{C}$ and then considering
\begin{equation}
\hom_{\ms{C}}(C,-)\left(A\xrightarrow{\;\;f\;\;}B\right)=\hom(C,A)\xrightarrow{\;\;\hom(C,f)\;\;}\hom(C,B)
\end{equation}
where $\hom(C,A)$ and $\hom(C,B)$ are sets of morphisms from $C$
to $A$ (resp. $B$) and $\hom(C,f)$ is a function given by
$g\mapsto f\circ{}g$ for each $g\in\hom(C,A)$. The functor
$\hom(C,-):\ms{C}\to\ms{Set}$ is called the \define{Hom-functor}.
\end{ex}

We have a few useful definitions we should cover before getting
to examples. Our taxonomy of functors are:
\begin{defn}%\label{defn:}
Let $F:\ms{A}\to\ms{B}$ be a functor. $F$ is called an
\define{embedding} provided that $F$ is injective on morphisms.
\end{defn}
\begin{rmk}
This is for the obvious reason. If we think of a directed
subgraph embedded in a directed graph, there is an injective
relation between the edges of the subgraph to the graph.
\end{rmk}
\begin{defn}%\label{defn:}
Let $F:\ms{A}\to\ms{B}$ be a functor. $F$ is called
\define{faithful} provided that all the hom-set restrictions
\begin{equation}%\label{eq:}
F:\hom_{\ms{A}}(A,A')\to\hom_{\ms{B}}(F(A),F(A'))
\end{equation}
are injective.
\end{defn}
\begin{defn}%\label{defn:}
Let $F:\ms{A}\to\ms{B}$ be a functor. $F$ is called \define{full}
provided all the hom-set restrictions are surjective.
\end{defn}
\begin{defn}%\label{defn:}
Let $F:\ms{A}\to\ms{B}$ be a functor. $F$ is called
\define{amnestic} provided that an $\ms{A}$-isomorphism $f$ is an
identity whenever $F(f)$ is an identity.
\end{defn}
\begin{prop}%\label{prop:}
Observe that a functor is:
\begin{enumerate}
\item an embedding if and only if it is injective on
objects and it is faithful;
\item an isomorphism if and only if it is bijective on objects,
  full, and faithful. 
\end{enumerate}
\end{prop}
\begin{cor}
A embedding $F$ is an isomorphism iff it is surjective on objects
and full.
\end{cor}

We can also specify functors as covariant or contravariant. Most
of the times it's covariant, it's as we defined it. When it's
contravariant, the domain is the dual category. That is
$F:\ms{C}\to\ms{D}$ is covariant, then $F':\ms{C}^{op}\to\ms{D}$
is contravariant. In other words, we can define it thus:
\begin{defn}\label{defn:contravariantFunctor}
A \define{Contravariant Functor}\index{Functor!Contravariant} is
a functor $F:\ms{C}^{op}\to\ms{D}$.
\end{defn}
So to each morphism $f:Y\to X$  in $\ms{C}$, we have
$F(f):F(X)\to F(Y)$ in $\ms{D}$. We also have the compositions go
the other way, that is for $f:X\to Y$, $g:Y\to Z$, we have
$F(g\circ f)=F(f)\circ F(g)$.

We summarize our taxonomy of functors in the following table:
\begin{table}[h]
\begin{tabular}{| p{3.5cm} | p{6cm} |}
\hline
\textbf{Functor Property} & \textbf{What it means}\\\hline
Amnestic & Reflects identities.\\ \hline
Equivalence & Full, Faithful, and essentially surjective.\\\hline
Embedding & Injective on morphisms.\\ \hline
Essentially Surjective & For each object $B$ in the codomain,
there is an $A$ in the domain s.t. $F(A)$ is
isomorphic to $B$.\\\hline
Faithful & Hom-set restrictions are injective\\ \hline
Full & Hom-set restrictions are surjective\\ \hline
Isomorphism & Bijective on morphisms and objects.\\\hline
& (Alternatively) It's invertible. \\ \hline
& (Alternatively) It's an equivalence relation on the
conglomerate of all categories (i.e. the domain and codomain are
``essentially the same'').\\ \hline
\end{tabular}
\caption{A Table of Functor Properties.}
\end{table}

%%%%%%%%%%%%%%%%%%%%%%%%%%%%%%%%%%%%%%%%%%%%%%%%%%%%%%%%%%%%%%%%%%%%%%%%%%%%%%%%
% FUNCTORS ARE MATHEMATICAL PROCEDURES
%%%%%%%%%%%%%%%%%%%%%%%%%%%%%%%%%%%%%%%%%%%%%%%%%%%%%%%%%%%%%%%%%%%%%%%%%%%%%%%%
\subsection{Functors are Mathematical Procedures}
To give an example of how a functor encodes mathematical
procedures, consider the following:
\begin{ex}\label{ex:powersetFunctor}
Consider $\mathcal{P}:\ms{Set}\to\ms{Set}$ be the powerset
functor. That is, it takes objects (sets) to sets of all possible
subsets of it, and functions $f:A\to{}B$ to functions between
the powersets $\mathcal{P}(A)$ and $\mathcal{P}(B)$. It's a
functor since (1) it maps the identity of $A$ to the identity
$\mathcal{P}(A)$, and (2) it maps the composition to the
composition of morphisms in the obvious way.
\end{ex}

\begin{ex}\label{ex:freeMonoidFunctor}
Consider $F:\ms{Set}\to\ms{Mon}$ the functor which associates to
each set $S\in\ms{Set}$ the monoid freely generated by it
$F(S)\in\ms{Mon}$. What happens is if we have a set $\{a,b\}$,
the monoid freely generated by it consists of strings whose
letters are $a$ or $b$, the operator concatenates two strings
together. That is it takes one string $(aababbbaababa)$ and another
$(bbabbaabbaa)$ and the operator then
\begin{equation}
(aababbbaababa)*(bbabbaabbaa)=(aababbbaabababbabbaabbaa)
\end{equation}
just glues the two strings together in the obvious way. The
identity element is just the empty string $()$ since
\begin{equation}
(ababaabbaaa)*() = (ababaabbaaa)
\end{equation}
attaches nothing to the end of the string, and it attaches
nothing to the end of every string, so it ``does nothing''.
\end{ex}

%%%%%%%%%%%%%%%%%%%%%%%%%%%%%%%%%%%%%%%%%%%%%%%%%%%%%%%%%%%%%%%%%%%%%%%%%%%%%%%%
% FUNCTORS ASSIGN INFORMATION TO OBJECTS AND
% INFORMATION-PRESERVING MORPHISMS TO MORPHISMS
%%%%%%%%%%%%%%%%%%%%%%%%%%%%%%%%%%%%%%%%%%%%%%%%%%%%%%%%%%%%%%%%%%%%%%%%%%%%%%%%
\subsection{Functors as Assigning Information: Sheaves as Example}

This is one use of functors, to embody mathematical
procedures. The other use is to assign information to each object
in a category. This could be viewed as one particular
mathematical procedure, the intuition should be the same as that
of a vector field assigning a vector to each point in a
space. The most famous generalization of this is known as
``sheaves'' and ``presheaves''.

Recall in topology~\cite{alexTopology}, we defined a topology (in
an example in subsection~\ref{ex:topologyAsObject}) $\mathcal{T}$
over a set $X$ to be a sufficiently nice collection of subsets of
a given set. We defined these subsets to be ``open subsets'' and
elements of $X$ to be ``points'' of $X$.

Our real aim is to generalize the notion of a vector field (a
mathematical object that assigns to each point of
$\mathbb{R}^{n}$ a vector). We need to use the aforementioned
notions from topology to generalize ``where'' we assign
mathematical objects. That is, we proceed in our generalization
by assigning to each open subset $U$ of our space $X$ some
``local data'' i.e. mathematical object defined on $U$.

As this is a collection of notes on category theory, we will
approach the problem categorically. First we realize that open
subsets $V\subseteq U\subseteq X$ we have an inclusion mapping
\begin{equation}%\label{eq:}
i_{V,U}:V\to U
\end{equation}
which is continuous.

Now, what would assign a mathematical object to an open subset?
Well, we conveniently used insight that an open subset is an
object in a category $\ms{O}(X)$ which is our topological space as a
category. Why not claim that a functor $F:\ms{O}(X)\to\ms{V}$
assigns to each open subset $U\in\ms{O}(X)$ some information $F(U)$?
We have to be careful about consistency problems. It would be bad
if two local regions that overlap end up having inconsistent data
on the overlap\footnote{It'd be unimaginable, for example, that
  upon hearing the temperature throughout California ranges from
  80 to 100 degrees that Davis, California is 110 degrees. We
  don't have consistency on the overlap!}. We then specify that
for each inclusion of open sets $V\subseteq U$, we need a
restriction morphism
\begin{equation}%\label{eq:}
\rho_{V,U}:F(U)\to F(V)
\end{equation}
in the category $\ms{V}$. This morphism has two properties of
importance:
\begin{enumerate}
\item for each open $U\subseteq X$, the restriction morphism
  $\rho_{U,U}:F(U)\to F(U)$ is the identity morphism on $F(U)$; 
\item if we have $W\subseteq V\subseteq U$ open, then
  $\rho_{W,V}\circ\rho_{V,U} = \rho_{W,U}$.
\end{enumerate}
This is a step towards ensuring consistency on overlaps.

\emph{BUT This means that our first attempt at a definition is WRONG!} 
Observe the arrows \emph{go the wrong way in the codomain!} This
means that our functor is really a \emph{contravariant} functor
$F:\ms{O}(X)^{op}\to\ms{V}$, but we did this rather quickly and
subtle details are swept under the rug so lets explain why this
is a good definition. First observe a contravariant functor
behaves as follows:
\begin{equation}%\label{eq:}
F\left(U\xrightarrow{\;\;f\;\;}V\right) = F(V)\xrightarrow{\;\;F(f)\;\;}F(U)
\end{equation}
which is not good, \emph{UNLESS} we reverse the arrow again! That
is, we have in our category $\ms{O}(X)$ a term
\begin{equation*}%\label{eq:}
U\xrightarrow{\;\;f\;\;}V
\end{equation*}
so if we first reverse this arrow by using the category dual to
it $\ms{O}(X)^{op}$ we have our arrows be
\begin{equation*}%\label{eq:}
V\xrightarrow{\;\;f\;\;}U
\end{equation*}
which means a contravariant functor would behave thusly:
\begin{equation}%\label{eq:}
F\left(V\xrightarrow{\;\;f\;\;}U\right) = F(U)\xrightarrow{\;\;F(f)\;\;}F(V)
\end{equation}
which is precisely what is desired! The cost, however, is to have
the functor take in the dual category as its domain. But we are
rich enough to pay.

We can now define a presheaf on some general category thus:
\begin{defn}\label{defn:presheaf}
Let $\ms{V}$ be some category, then a $\ms{V}$-valued
\define{Presheaf} $F$ on a category $\ms{C}$ is a functor
\begin{equation}%\label{eq:}
F:\ms{C}^{op}\to\ms{V}.
\end{equation}
If we restrict our attention to the category $\ms{O}(X)$ of open
subsets of $X$, we recover the intuitive generalization of vector
fields we introduced in this section.
\end{defn}
\begin{comment}
\begin{rmk}
A presheaf with consistency on overlaps is precisely a sheaf. We
are uninterested in such things for several reasons. First, it
would be an insult to dedicated a few meager paragraphs to the
subject when it deserves a book in its own right. Second, we can
do a number of nifty things with presheaves in category theory,
which is all we care about at the moment.
\end{rmk}
\end{comment}

If $F$ is a $\ms{V}$-valued presheaf on $X$, and $U$ is an open
subset of $X$, then we call $F(U)$ the \define{sections of $F$
  over $U$}. If $\ms{V}$ is a concrete category, then each
element of $F(U)$ is called a \define{section}. A section over
$X$ is called a \define{global section}. We adopt the notion that
the restriction of a section
\begin{equation}%\label{eq:}
\rho_{V,U}(s)=s|_{V}
\end{equation}
This is somewhat similar to sections with fiber bundles.

We will now quickly define a sheaf in the most intuitively appealing
way:
\begin{defn}\label{defn:sheaf}
A \define{sheaf with value in a concrete category $\ms{V}$} is a
presheaf with values in $\ms{V}$ such that
\begin{enumerate}
\item{(Normalization)} $F(\emptyset)$ is the terminal object of
  $\ms{V}$;
\item{(Local Identity)} If ($U_i$) is an open covering of an open
set $U$, and if $s,t\in F(U)$ are such that when restricted on
each $U_i$ of the open covering $s|_{U_i}=t|_{U_i}$, then $s=t$;
and
\item{(Gluing)} If $(U_i)$ is an open covering of an open set
  $U$, and if for each $i$ there is a section $s_i$ of $F$ over
  $U_i$ such that for each pair $U_i$, $U_j$ of the covering
  sets, the restrictions $s_i$ and $s_j$ agree on the overlaps
  $s_{i}|_{U_{i}\cap U_{j}}=s_{j}|_{U_{i}\cap U_{j}}$, then there
  is a section $s\in F(U)$ such that $s|_{U_i}=s_i$ for each $i$.
\end{enumerate}
\end{defn}
This definition is really not as general as it could be, since
the normalization property assumes the objects are open subsets
in a topology.

At any rate, the section $s$ whose existence is guaranteed by the
third property is usually called the ``gluing'',
``concatenation'', or ``collation'' of the section $s_{i}$. We
see that it is unique (by our local identity property). Sections
$s_{i}$ satisfying the condition of gluing property are often called
\define{compatible}; so when we look at the gluing and local
identity properties, we can summarise their meaning thus:
\emph{compatible sections can be uniquely glued together.} This
guarantees consistency on overlap.

\subsection{Quantization (Elaborate Non-Example of a Functor)}

(For finite dimensional symplectic vector spaces over finite
fields, there actually is a quantization
functor~\cite{gurevich-2007}. But for the general case in
physics, there is no such functor -- there is a no go theorem due
to Groenewald and van Hove. The interested reader is referred to~\cite{ali-2005-17,gotay-1996-6}.)

Consider the canonical procedure to quantizing classical
systems. Mathematically, we describe classical systems by
symplectic manifolds, and quantum systems by Hilbert spaces. So,
quantization should naively be a functor
\begin{equation}%\label{eq:}
\mathcal{Q}:\mathbf{Symp}\to\mathbf{Hilb}
\end{equation}
where $\mathbf{Symp}$ is the category of symplectic manifolds whose
morphisms are symplectic maps (or in the jargon of Hamiltonian
mechanics, ``canonical transformations''); $\mathbf{Hilb}$ is the
category of Hilbert spaces, whose morphisms are unitary mappings.

This should seem somewhat unnatural for a host of reasons. First,
classical mechanics should be \emph{obtained from quantum mechanics} 
in some appropriate limit. Why would would expect to obtain a
unique quantum theory from a given classical description of a
system? 

Ignoring this, we have a problem that requires some
explanation. Suppose we had such a functor. Let $u^{i}=(q,p)$ be
a choice of variables for an object in $\mathbf{Symp}$ with
symplectic structure $\omega_{ij}$. Recall the symplectic
structure is such that
\begin{equation}%\label{eq:}
\{u_i,u_j\}=\omega_{ij}\quad
\Rightarrow\quad\{f,g\}=\frac{\partial f}{\partial
  u_i}\omega_{ij}\frac{\partial g}{\partial u_{j}}
\end{equation}
where $\{-,-\}$ is the Poisson bracket. Then we have some
properties for $\mathcal{Q}$:
\begin{enumerate}
\item{(Linearity)} $\mathcal{Q}(c_{1}f+c_{2}g)=c_{1}\mathcal{Q}(f)+c_{2}\mathcal{Q}(g)$;
\item{(Identity Preserved)} $\mathcal{Q}(1)=I$;
\item{(Poisson Bracket)} $\displaystyle\mathcal{Q}\left(\{f,g\}\right)=(-i/\hbar)[\mathcal{Q}(f),\mathcal{Q}(g)]$;
\item{(Irreducible)} the operators
  $\mathcal{Q}(\mathbf{x})$ and $\mathcal{Q}(\mathbf{p})$ are
  irreducibly represented.
\end{enumerate}
The domain of such a mapping is called the ``\emph{space of quantizable observables}''.
The irreducibility condition is somewhat enigmatic. It can be
reworded as:
\begin{quote}
Let $\mathcal{H}$ be the Hilbert space $\mathcal{Q}(\mathbf{x})$
and $\mathcal{Q}(\mathbf{p})$ act on. Then there are no subspaces
$\mathcal{H}_{0}\subset\mathcal{H}$ (other than $\{0\}$ and
$\mathcal{H}$ itself) that are stable under the action of all the
operators $\mathcal{Q}(\mathbf{x})$ and $\mathcal{Q}(\mathbf{p})$.
\end{quote}

Now, here comes the problems: lets work in 1 dimension, and let
\begin{equation}%\label{eq:}
f(p,q) = p^{2}q^{2} = (pq)^2
\end{equation}
We can write
\begin{equation}%\label{eq:}
(p+q)^{2} = 2pq + p^{2}+q^{2}\quad\Rightarrow\quad pq = \frac{(p+q)^{2}-p^{2}-q^{2}}{2}.
\end{equation}
When we quantize $\mathcal{Q}(pq)$ we can quantize this: 
\begin{subequations}
\begin{align}
\mathcal{Q}(pq)
&=\frac{1}{2}\left((\mathcal{Q}(p)+\mathcal{Q}(q))(\mathcal{Q}(p)+\mathcal{Q}(q))-\mathcal{Q}(q)^{2}-\mathcal{Q}(p)^{2}\right)\\
&= \frac{\mathcal{Q}(p)\mathcal{Q}(q)+\mathcal{Q}(q)\mathcal{Q}(p)}{2}.
\end{align}
\end{subequations}
Similarly we can write
\begin{equation}%\label{eq:}
p^{2}q^{2} = \frac{(p^{2}+q^{2})^{2}-p^{4}-q^{4}}{2}\quad\Rightarrow\quad p^{2}q^{2} = \frac{(p^{2}+q^{2})^{2}-p^{4}-q^{4}}{2}.
\end{equation}
and by the same reasoning as before, when quantized we get
\begin{equation}%\label{eq:}
\mathcal{Q}(p^{2}q^{2}) = \frac{\mathcal{Q}(p^{2})\mathcal{Q}(q^{2})+\mathcal{Q}(q^{2})\mathcal{Q}(p^{2})}{2}
\end{equation}
Now, our argument is summarized in the following diagram
\begin{equation*}
\vcenter{\xymatrix{
p^{2}q^{2} \ar[dd]_{\mathcal{Q}}\ar[rr]^{id}&& (pq)^{2}\ar[dd]^{\mathcal{Q}}\\
&&\\
\displaystyle\left(\frac{(\mathcal{Q}(p)^{2}+\mathcal{Q}(q)^{2})^{2}-\mathcal{Q}(p)^{4}-\mathcal{Q}(q)^{4}}{2}\right)&&\displaystyle\left(\frac{\mathcal{Q}(p)\mathcal{Q}(q)+\mathcal{Q}(q)\mathcal{Q}(p)}{2}\right)^{2}
}}
\end{equation*}
\noparskip\noindent where $id$ is just the identity
morphism. Equivalently, we could write
\begin{multline}
\mathcal{Q}\left(
\frac{(p^{2}+q^{2})^{2}-p^{4}-q^{4}}{2}
\xrightarrow{\;\;id\;\;}
\left[\frac{pq+qp}{2}\right]^{2}
\right)\\
=
\left(\frac{(\mathcal{Q}(p)^{2}+\mathcal{Q}(q)^{2})^{2}-\mathcal{Q}(p)^{4}-\mathcal{Q}(q)^{4}}{2}\right)
\xrightarrow{\;\;???\;\;}
\left(\frac{\mathcal{Q}(p)\mathcal{Q}(q)+\mathcal{Q}(q)\mathcal{Q}(p)}{2}\right)^{2}
\end{multline}
What is the morphism on the right hands side of this equation?

By our specification of quantization, we should expect an
identity morphism to make the diagram commute. But clearly
\begin{equation}%\label{eq:}
 \frac{\mathcal{Q}(p^{2})\mathcal{Q}(q^{2})+\mathcal{Q}(q^{2})\mathcal{Q}(p^{2})}{2}
 \neq \left(\frac{\mathcal{Q}(p)\mathcal{Q}(q)+\mathcal{Q}(q)\mathcal{Q}(p)}{2}\right)^{2}
\end{equation}
In other words, our diagram doesn't commute! So the property
\begin{equation}%\label{eq:}
\mathcal{Q}(id)=id
\end{equation}
doesn't always hold, which is a critical property of a
functor. We then conclude that quantization \emph{is not a functor.} 

\section{Natural Transformations}
%%
%% naturalTransformation.tex
%% 
%% Made by Alex Nelson
%% Login   <alex@tomato>
%% 
%% Started on  Wed Jun 24 14:40:20 2009 Alex Nelson
%% Last update Wed Jun 24 14:40:20 2009 Alex Nelson
%%

We might start thinking: we had objects and morphisms between
them. We had categories and morphisms (called ``functors'')
between them. Now we have functors, do we have morphisms between
them? It turns out we can, this is the importance of natural
transformations. Before getting to them, it should be noted that
Category Theory was invented to investigate natural
transformations. It is more useful than functors, but it is
trickier to start thinking in terms of them.

\begin{defn}%\label{defn:}
Given two functors $S,T:\ms{A}\to\ms{B}$, a \define{Natural Transformation} 
is a function $\tau:S\Rightarrow T$ which assigns to each object
$X\in\ob{\ms{A}}$ an arrow $\tau_{X}=\tau(X):S(X)\to T(X)$ of
$\ms{B}$ in such a way that every arrow $f:X\to Y$ in $\ms{A}$
yields a diagram
\begin{equation}\label{eq:naturalityCondition}
\vcenter{\xymatrix{
S(X)\ar[d]^{\tau_{X}} \ar[r]^{S(f)} & S(Y)\ar[d]^{\tau_{Y}}\\
T(X)\ar[r]^{T(f)} & T(Y)
}}
\end{equation}
which is commutative. This condition that the diagram described
by eq \eqref{eq:naturalityCondition} is commutative is called the
\define{naturality condition}; when this holds, we also say that
$\tau_{X}:S(X)\to T(X)$ is \define{natural} in $X$. We call
$\tau_{X}$, $\tau_{Y}$ the \define{components} of  the natural
transformation. 
\end{defn}

\begin{rmk}[Notation]
We have a natural transformation $\tau:F\Rightarrow G$ usually
denoted with the $F\Rightarrow G$ instead of $F\to G$.
\end{rmk}

%%%%%%%%%%%%%%%%%%%%%%%%%%%%%%%%%%%%%%%%%%%%%%%%%%%%%%%%%%%%%%%%%%%%%%%%%%%%%%%%
% NATURAL TRANSFORMATION AS ANALOGOUS TO HOMOTOPY DEFORMATION OF FUNCTORS
%%%%%%%%%%%%%%%%%%%%%%%%%%%%%%%%%%%%%%%%%%%%%%%%%%%%%%%%%%%%%%%%%%%%%%%%%%%%%%%%
There are probably more than two ways to picture a natural
transformation. We'll focus on the two obvious ones: as a
deformation of one functor into another, or --- if we view functors
as assigning information to objects of a category --- as a
natural way to transform information assigned to objects of a
category.

One way is as a ``deformation'' (in the homotopy-theoretic sense
of the word) of one functor into another. But recall with a
homotopy, we had specified one path
$\gamma_{1}:[0,1]\to\mathbb{C}$ is homotopic to another
$\gamma_{2}:[0,1]\to\mathbb{C}$ if we have a continuous function
\begin{equation}
H(s,t)=(1-s)\gamma_{1}(t)+s\gamma_{2}(t)
\end{equation}
such that $H(0,t)=\gamma_{1}(t)$ and $H(1,t)=\gamma_{2}(t)$. To
construct an analogous deformation, we need some category
analogous to the $s\in[0,1]$ term. This is precisely $\ms{2}$
category. But a natural transformation from $F:\ms{C}\to\ms{D}$
to $G:\ms{C}\to\ms{D}$ becomes a functor
\begin{equation}
\alpha:\ms{C}\times\ms{2}\to\ms{D}
\end{equation}
where $\ms{2}$ is the categorical analog to the
``interval''. 

An aside on product categories, if the reader is unfamiliar with
them, one should envision the objects of $\ms{C}\times\ms{2}$
being ordered pairs $(C,0)$ and $(C,1)$ for all
$C\in\ms{C}$. That is, we end up with two copies of the category
$\ms{C}$. The morphisms are also ordered pairs $(g,f)$ where
$f:0\to{}1$ and $g:C\to{}D$. We can ``break them up'' in the
sense that the diagram
\begin{equation}
\vcenter{\xymatrix{
(C,0)\ar[d]_{(\id{C},f)}\ar[drr]_{(g,f)}\ar[rr]^{(g,\id{0})}&&(D,0)\ar[d]^{(\id{D},f)}\\
(C,1)\ar[rr]_{(g,\id{1})}&&(D,1)
}}
\end{equation}
commutes. So $(id,f)$ transaltes from one copy to the other, and
$(g,\id)$ acts on one copy.

Observe that if we let $0,1\in\ms{2}$ and
$f:0\to{}1$ so for some fixed object denoted by $(\cdot)$ we have 
$\alpha(\cdot,0)=F(\cdot)$, $\alpha(\cdot,1)=G(\cdot)$, and
\begin{equation}
\alpha(\cdot,0\xrightarrow{\;\;f\;\;}1)=F(\cdot)\xrightarrow{\;\;\alpha(\cdot,f)\;\;}G(\cdot)
\end{equation}
which should begin looking vaguely familiar. We shouldn't be too
surprised since the category really looks like
\begin{equation}%\label{eq:}
\vcenter{\xymatrix{
\ms{C}\times{}0 & \cdot\ar[d]_{f}\ar[r] & \cdot\ar[d]^{f} \\
\ms{C}\times{}1 & \cdot\ar[r] &\cdot
}}
\end{equation}
which is two copies of $\ms{C}$. We have specifically, since it's
a functor, we see how it acts on the diagram
\begin{equation}%\label{eq:}
\vcenter{\xymatrix{
\alpha(\ms{C}\times{}0) & F(X)\ar[d]_{\alpha(X,f)}\ar[r]^{F(g)} & F(Y)\ar[d]^{\alpha(Y,f)} \\
\alpha(\ms{C}\times{}1) & G(X)\ar[r]^{G(g)} &G(Y)
}}
\end{equation}
which is precisely a naturality condition! It's not too far a
stretch to state that $\alpha(X,f)$ and $\alpha(Y,f)$ are
components of the natural transformation.
\begin{comment}
%\begin{equation}
%\alpha(X\xrightarrow{\;\;g\;\;}Y,0\xrightarrow{\;\;f\;\;}1)=F(X\xrightarrow{\;\;g\;\;}Y)\xrightarrow{\;\;\alpha(X\xrightarrow{\;\;g\;\;}Y,f)\;\;}G(X\xrightarrow{\;\;g\;\;}Y)
%\end{equation}
which should behave like a natural transformation. We just don't
know how $\alpha(X\xrightarrow{\;\;g\;\;}Y,f)$ behaves
exactly. We expect it to ``break up'' into three parts:
$\alpha(X,f)$, $\alpha(g,f)$, and $\alpha(Y,f)$. We suspect that
$\alpha(X,f)$ and $\alpha(Y,f)$ correspond to the components
$\alpha_X$ and $\alpha_Y$ of the natural transformation, but the
remaining bit $\alpha(g,f)$ remains a mystery. Intuitively, it
should either map $\alpha_X\to\alpha_Y$ or map $F(g)\to G(g)$. We
just don't have a good intuition of ``morphisms of morphisms''! 
\end{comment}

%%%%%%%%%%%%%%%%%%%%%%%%%%%%%%%%%%%%%%%%%%%%%%%%%%%%%%%%%%%%%%%%%%%%%%%%%%%%%%%%
% NATURAL TRANSFORMATIONS AS MORPHISMS OF SHEAVES
%%%%%%%%%%%%%%%%%%%%%%%%%%%%%%%%%%%%%%%%%%%%%%%%%%%%%%%%%%%%%%%%%%%%%%%%%%%%%%%%
The other intuition to have is to think of a functor as assigning
some mathematical object to each object of its domain. We have
some intuition of how to transform objects into other objects via
morphisms and functors. A natural transformation, then, is
nothing more than changing the information assigned to each
object in the domain. But this is done in some ``natural''
way. It's specifically natural if it satisfies the naturality
condition.

Think for a moment about the importance of the diagram
commuting. What this means is that
\begin{equation}%\label{eq:}
T(f)\circ\tau_{X} = \tau_{Y}\circ S(f)
\end{equation}
or intuitively ``translate how to assign information, then
translate information = translate information, then translate how
to assign information.'' This means the result of a mathematical
process from one category translated into another category is the
same as translating the ``ingredients'' from one category then
applying the other process. Or, in terms of our problem, both
recipes yield cake.

Now, we can think of categories as consisting of diagrams. The
above definition gives us instructions how to ``translate'' from
a morphism in the category described by $S(\ms{A})$ to the
corresponding morphism in the category described by
$T(\ms{A})$. This is precisely what the intuition behind the
definition of a natural transformation as a functor from
$S(\ms{A})\times\ms{2}\to{}T(\ms{A})$ is!

We have diagrams ``built'' from morphisms, and we know how to
``translate'' each morphism individually, so it's not too hard of
a stretch to figure out how to ``translate'' a diagram. We can
also have the intuition that a natural transformation is a
``\emph{morphism of functors}''.

%%%%%%%%%%%%%%%%%%%%%%%%%%%%%%%%%%%%%%%%%%%%%%%%%%%%%%%%%%%%%%%%%%%%%%%%%%%%%%%%
% EXAMPLE OF NATURAL TRANSFORMATION AS EVALUATION OF A FUNCTION
%%%%%%%%%%%%%%%%%%%%%%%%%%%%%%%%%%%%%%%%%%%%%%%%%%%%%%%%%%%%%%%%%%%%%%%%%%%%%%%%
\begin{ex}\label{ex:naturalTransformation}
Let $S$ be a fixed set, $X^{S}$ be the set of all functions
$h:S\to X$. We want to show
\begin{enumerate}
\item $X\mapsto X^{S}$ is the object function of a functor
  $\ms{Set}\to\ms{Set}$, and that
\item evaluation $e_{X}:X^{S}\times S\to X$ (defined by
  $e(h,s)=h(s)$) is a natural transformation.
\end{enumerate}
The overall scheme of things is we wish to assign some
information on each set in $\ms{Set}$. We wish to assign on
$X\in\ms{Set}$ the information $\hom(S,X)\times\id{S}$ on the one
hand, and $X$ itself on the other. We wish to go from the first
to the second in the ``natural'' way of evaluating functions. It
seems that this should be a natural transformation, but we should
show it in two steps.

(1) Consider the category of sets $\cat{Set}$. The Covariant-Hom
functor (recall example \ref{ex:homFunctor}) is:
\begin{equation}%\label{eq:}
\hom(S,-):\ms{Set}\to\ms{Set}
\end{equation}
defined by
\begin{equation}%\label{eq:}
\hom(S,-)\left(B\xrightarrow{\;\;f\;\;}C\right)=\hom(S,B)\xrightarrow{\;\;\hom(S,f)\;\;}\hom(S,C)
\end{equation}
maps $X\mapsto X^{S}$ where $X^{S}=\hom(S,X)$.

\noindent(2) We wish to show that evaluation of a function in the
obvious way is a natural transformation. The naive way to set up
the commutative diagram describing our naturality condition would
be thus:
\begin{equation}
\vcenter{
\xymatrix{
S(X)\ar[d]^{\tau_{X}} \ar[r]^{S(f)} & S(Y)\ar[d]^{\tau_{Y}}\\
T(X)\ar[r]^{T(f)} & T(Y)
}}
\end{equation}
\noindent This is assuming, of course, that the functor in
question is $\hom(S,-)\times\id{S}$. What would be the codomain
of our natural transformation? Well, we would have to describe
evaluation as
\begin{equation}%\label{eq:}
e:\hom(S,-)\times\id{S}\to\id{\ms{Set}}
\end{equation}
but this is precisely a natural transformation. This concludes
our example
\end{ex}

\begin{rmk}
Observe that here our natural transformation was really just
transforming information on each set $X\in\ms{Set}$ assigned via
the functors $\hom(S,X)\times\id{S}$ and $\id{\ms{Set}}$. The
first functor assigns all functions mapping $S$ to $X$ and the
set $S$, the second is just the identity. What's the natural
thing to do? Simply take a function and ``feed in'' $S$. This
transforms the first functor into the second.
\end{rmk}

%%%%%%%%%%%%%%%%%%%%%%%%%%%%%%%%%%%%%%%%%%%%%%%%%%%%%%%%%%%%%%%%%%%%%%%%%%%%%%%%
% NATURAL EQUIVALENCE, WEAK INVERSE DEFINED
%%%%%%%%%%%%%%%%%%%%%%%%%%%%%%%%%%%%%%%%%%%%%%%%%%%%%%%%%%%%%%%%%%%%%%%%%%%%%%%%
\begin{defn}%\label{defn:}
Let $F,G:\ms{A}\to\ms{B}$ be functors, a natural transformation
$\tau:F\Rightarrow G$ with every component $\tau(X)$ invertible
in $\ms{B}$ is called a \define{Natural Equivalence} or (better)
a \define{Natural Isomorphism}. We denote this with the special
symbol $\tau: F\cong G$. In such a case, $(\tau(Y))^{-1}$ in
$\ms{B}$ are the components of a natural transformation
$\tau^{-1}:G\Rightarrow F$.
\end{defn}

\begin{defn}%\label{defn:}
We wish to give a natural transformation definition for an
\define{equivalence}. That is, a functor $F:\ms{A}\to\ms{B}$ is
an ``equivalence'' if it has a \define{weak inverse}\index{Inverse!Weak}, 
i.e. a functor $G:\ms{B}\to\ms{A}$ such that there exists natural
isomorphisms $\alpha:G\circ F\cong 1_\ms{A}$, $\beta:F\circ G\cong 1_\ms{B}$.
\end{defn}

%%%%%%%%%%%%%%%%%%%%%%%%%%%%%%%%%%%%%%%%%%%%%%%%%%%%%%%%%%%%%%%%%%%%%%%%%%%%%%%%
% CONSTRUCTION OF NATURAL TRANSFORMATION AS ``COMPOSED'' WITH A FUNCTOR
%%%%%%%%%%%%%%%%%%%%%%%%%%%%%%%%%%%%%%%%%%%%%%%%%%%%%%%%%%%%%%%%%%%%%%%%%%%%%%%%
We can also introduce several ways to construct more natural
transformations:
\begin{prop}\label{prop:composeFunctorsAndNaturalTransformations}
Given functors $F,G:\ms{C}\to\ms{D},$ $H:\ms{D}\to\ms{E},$
$K:\ms{B}\to\ms{C}$, and natural transformation
$\eta:F\Rightarrow G$, we can construct:
\begin{itemize}
\item a natural transformation $H\eta:HF\Rightarrow{}HG$ by
  defining $(H\eta)_{X}=H_{\eta(X)}$;
\item a natural transformation $\eta K:FK\Rightarrow{}GK$ by
  defining $(\eta K)_{X}=\eta_{K(X)}$.
\end{itemize} 
\end{prop}
This turns out to be used a number of times later on in category
theory. The intuition is that we can ``compose'' a natural
transformation with a functor ``componentwise'' (i.e. with the
domain and codomain functors), resulting with a natural transformation.

We can also compose natural transformations with natural
transformations ``in the obvious way'' (componentwise). More
precisely, we can sketch it out in a proposition:
\begin{prop}%\label{prop:}
Let $F,G,H:\ms{C}\to\ms{D}$ be functors,
$\varepsilon:F\Rightarrow{}G$ and $\eta:G\Rightarrow{}H$ be
natural transformations. Then we can compose the natural
transformations to be $\eta\circ\varepsilon:F\Rightarrow{}H$
whose components are
$(\eta\circ\varepsilon)_{X}=\eta_{X}\circ\varepsilon_{X}$
componentwise composed.
\end{prop}


%%%%%%%%%%%%%%%%%%%%%%%%%%%%%%%%%%%%%%%%%%%%%%%%%%%%%%
%% A MATHEMATICIAN'S APPROACH PART ONE
%%%%%%%%%%%%%%%%%%%%%%%%%%%%%%%%%%%%%%%%%%%%%%%%%%%%%%
\chapter{A Mathematician's Approach (Movement One \emph{Allegretto})}
\section{Adjunctions and Adjoints}
%%
%% adjointFunctors.tex
%% 
%% Made by Alex Nelson
%% Login   <alex@tomato>
%% 
%% Started on  Sun Jul 19 13:55:59 2009 Alex Nelson
%% Last update Sun Jul 19 13:55:59 2009 Alex Nelson
%%

\subsection{Definition via Hom-Set Adjunction}

Consider two functors $F:\ms{D}\to\ms{C}$ and
$G:\ms{C}\to\ms{D}$, and the natural isomorphism
\begin{equation}%\label{eq:}
\Phi:\hom_{\ms{C}}(F-,-)\cong{}\hom_{\ms{D}}(-,G-).
\end{equation}
This specifies a family of bijections for each pair of objects
$X\in\ms{C}$ and $Y\in\ms{D}$ 
\begin{equation}%\label{eq:}
\Phi_{X,Y}:\hom_{\ms{C}}(F(Y),X)\cong{}\hom_{\ms{D}}(Y,G(X)).
\end{equation}
In this situation, we say that $F$ is \define{Left Adjoint} to
$G$ and $G$ is \define{Right Adjoint} to $F$.

\subsection{Definition via Unit and Counit}

We'll introduce the notion of an adjunction as a weaker form of
equivalence. That is, we have two categories $\ms{C},\ms{D}$. We
have a small hierarchy so far of the notion of $\ms{C}$ being
``the same'' as $\ms{D}$. We have the notion they are isomorphic
if there is an isomorphism $F:\ms{C}\to\ms{D}$, which happens if
it is invertible i.e. there is a $G:\ms{D}\to\ms{C}$ such that
$F\circ{G}=\id{\ms{D}}$ and $G\circ{F}=\id{\ms{C}}$.

We can weaken this to the notion of $\ms{C}$ and $\ms{D}$ are
``equivalent'' if there are two functors $F:\ms{C}\to\ms{D}$ and
$G:\ms{D}\to\ms{C}$ such that we have two natural isomorphisms
\begin{equation}%\label{eq:}
F\circ{G}\cong{\id{\ms{C}}}
\end{equation}
and
\begin{equation}%\label{eq:}
\id{\ms{D}}\cong{G\circ{F}}.
\end{equation}
Now, why did we choose to write it this way?

Well, we can weaken the notion of an equivalence of two
categories even further. Instead of demanding that we have a
pair of natural isomorphisms, we can demand we have ``some''
arbitrary pair of natural transformations
\begin{equation}%\label{eq:}
\varepsilon:F\circ{G}\Rightarrow{\id{\ms{C}}}
\end{equation}
and
\begin{equation}%\label{eq:}
\eta:\id{\ms{D}}\Rightarrow{G\circ{F}}.
\end{equation}
We call them the counit and unit (respectively). We demand they
satisfy the demand that
\begin{equation}\label{eq:triangleIdentitiesOne}
F\xrightarrow{\;F\eta\;}F\circ{G\circ{F}}\xrightarrow{\;\varepsilon F\,}F
\end{equation}
and
\begin{equation}\label{eq:triangleIdentitiesTwo}
G\xrightarrow{\;\eta G\;}G\circ{F\circ{G}}\xrightarrow{\;G \varepsilon\,}G
\end{equation}
when composed are the identity natural transformations on $F$ and
$G$ (respectively). We call eqs \eqref{eq:triangleIdentitiesOne}
\eqref{eq:triangleIdentitiesTwo} the \define{Triangle Identities}.
When this happens we say that $F$ is \define{Left Adjoint} to
$G$, and $G$ is \define{Right Adjoint} to $F$.

\subsection{Equivalence of Two Definitions}
This derivation is really inspired from
Baez~\cite{BaezWeek79}. We have
the definition of an ``adjunction'' as two functors
\begin{equation}
L:\ms{C}\to\ms{D}\quad\text{and}\quad{}R:\ms{D}\to\ms{C}
\end{equation}
and a natural isomorphism between $\hom_{\ms{D}}(Lc,d)$ and
$\hom_{\ms{C}}(c,Rd)$. So it's just a pair of functors $L,R$ and
a natural isomorphism.

What happens if we take $c=Rd$? This could only affect one of
three things (either one of the functors or the natural
isomorphism). We see that our natural isomorphism becomes
\begin{equation}
\hom(LRd,d)\cong\hom(Rd,Rd)
\end{equation}
which is interesting. We know there is a special object in
$\hom(Rd,Rd)$, namely the identity morphism
$\id{Rd}:Rd\to{}Rd$. This implies that there is a special object
in $\hom(LRd,d)$ since the two objects are
isomorphic. Specifically we'll denote
\begin{equation}
e_{d}:LRd\to d
\end{equation}
denote this special object.

What is this ``special object''? What does it do, what intuition
should we have when we see it? Well, take $L:\ms{Set}\to\ms{Mon}$
be the functor which associates to each set $S\in\ms{Set}$ the
free monoid generated by it. It behaves in the obvious way, the
elements of $L(S)$ are lists of elements in $S$, and the binary
operator of $L(S)$ simply concatenates two lists together. We'll
let $R:\ms{Mon}\to\ms{Set}$ be the forgetful functor.  It simply
forgets the binary operator, and we have -- to no great surprise
-- the set of elements of the monoid. Now our ``special
morphism'' maps $LRd$ to $d$. Step by step we see that $Rd$ is
the set underlying $d$ and $L(Rd)$ to be the free monoid
generated by $Rd$. So the morphism maps $LRd$ to $d$, what can do
this? Well, our binary operator is usually written tacitly as
just multiplication, so if we look at these lists as strings of
elements of $d$, what could map strings of elements of $d$ to
$d$? The simplest answer would be to evaluate the string of
elements of $d$, that is carry out the multiplication. This
necessarily yields an element in $d$. That is what our ``special
morphism'' does.

In fact, this is not just the ``simplest'' choice of morphisms,
it's fairly (dare I say) ``\emph{natural}'' to choose such a
morphism. It's not too much of a stretch to say that the morphism
$e_{d}$ defines a natural transformation
\begin{equation}
e:LR\Rightarrow{}\id{\ms{D}}
\end{equation}
where $\id{\ms{D}}$ is the identity functor on $\ms{D}$.

We can similarly ask what happens if we take $d=Lc$? Then we have
a natural transformation between $\hom(c,RLc)$ and
$\hom(Lc,Lc)$. As before, we have a ``special'' morphism in
$\hom(Lc,Lc)$ which is $\id{Lc}$. This gives a special object in
$\hom(c,RLc)$. We'll denote this by
\begin{equation}
i_{c}:c\to{}RLc.
\end{equation}
Again, we ask ``What does it do?''

Using notation from the previous example, we have $i_{c}$ taking
a set of ``stuff'' $c$ to the set underlying the free monoid
generated by this ``stuff''.

As before, this yields a natural transformation
\begin{equation}
i:\id{\ms{C}}\Rightarrow{}RL.
\end{equation}
In other words, we end up with a pair of natural transformations
from our definition of adjoint functors. When one sees this
notion of a pair of natural transformations, one should be
reminded of an equivalence of categories. With adjunctions, we
have a sort of weaker form of equivalence. We no longer demand
that the natural transformations are invertible. (Just as a group
is a monoid, so too is an equivalence an adjunction?)

Recall, an equivalence of categories $\ms{C}$ and $\ms{D}$ is
defined as a pair of functors $F:\ms{C}\to\ms{D}$ and
$G:\ms{D}\to\ms{C}$ equipped with a pair of natural
transformations $e:FG\Rightarrow{}\id{\ms{D}}$ and
$i:\id{\ms{C}}\Rightarrow{}GF$ such that these natural
transformations are invertible.

\begin{rmk}
Some people define adoint functors in this manner as a sort of
distant cousin to equivalences, others prefer using the ``pair of
functors equipped with a natural isomorphism'' definition. We see
that we have derived the former from the latter. 
\end{rmk}

\section{Product Categories}
%%
%% products.tex
%% 
%% Made by Alex Nelson
%% Login   <alex@tomato>
%% 
%% Started on  Mon Jul 20 12:44:22 2009 Alex Nelson
%% Last update Mon Jul 20 12:44:22 2009 Alex Nelson
%%

\begin{prob}
We wish to introduce some notion of a ``product'' of two
categories $\ms{A}$ and $\ms{B}$, which has the intuition that
the objects are ``ordered pairs'' of objects from $\ms{A}$ and
$\ms{B}$ respectively, and whose morphisms are ``ordered pairs''
of morphisms from $\ms{A}$ and $\ms{B}$
respectively. Composition, and everything in general, is done
componentwise --- at least, intuitively. We would like to abandon
the use of set theory, and exclusively use category theory instead.
\end{prob}

We'll begin by leading by misexample and use set theoretic terms
to make our intuitions more precise.

Lets first begin by introducing a set theoretic definition of
product categories:
\begin{defn}%\label{defn:}
Given two categories $\ms{A}$ and $\ms{B}$, we defined the
\define{Product Category} $\ms{A}\times\ms{B}$ consisting of
\begin{enumerate}
\item all pairs of objects $(X_{A},Y_{B})$ where
  $X_{A}\in\ob{\ms{A}}$, $Y_{B}\in\ob{\ms{B}}$, and
\item all pairs of objects $(X_{A},Y_{B})$, $(X_{A}',Y_{B}')$ a
  set of morphisms $(f,g)$ where $f:X_{A}\to X_{A}'$, $g:Y_{B}\to Y_{B}'$.
\end{enumerate}
equipped with
\begin{enumerate}
\item for any object $(X_{A},Y_{B})$, an identity morphism
  $(\id{X_{A}},\id{Y_{B}})$;
\item for any pair of morphisms 
$$(T,U)\xrightarrow{\;\;(f,g)\;\;}(T',U')\xrightarrow{\;\;(f',g')\;\;}(T'',U'')$$
we have a morphism
$$(T,U)\xrightarrow{\;\;(f',g')\circ(f,g)\;\;}(T'',U'')$$
which is calculated componentwise $(f',g')\circ(f,g)=(f'\circ
f,g'\circ g)$.
\end{enumerate}
\end{defn}

Now that we have introduced the notion of taking the ``product''
of two categories, we can introduce ``projection functors'' to
recover the components of the product.

\begin{defn}%\label{defn:}
Let $\ms{A},$ $\ms{B}$ be two categories. We define the
\define{Projection Functors}\index{Functor!Projection} $P,Q$ to
be given by
\begin{equation}%\label{eq:}
\ms{A}\xleftarrow{\;\;P\;\;}\ms{A}\times\ms{B}\xrightarrow{\;\;Q\;\;}\ms{B}
\end{equation}
which is specified by
\begin{equation*}%\label{eq:}
P\left((X,Y)\xrightarrow{\;\;(f,g)\;\;}(X',Y')\right) = X\xrightarrow{\;\;f\;\;}X'
\end{equation*}
and
\begin{equation*}%\label{eq:}
Q\left((X,Y)\xrightarrow{\;\;(f,g)\;\;}(X',Y')\right) = Y\xrightarrow{\;\;g\;\;}Y'.
\end{equation*}
\end{defn}

Now we want to generalize this notion of a product to a more
general setting, and in a more category theoretic manner. We have
the product category for $\ms{Set}\times\ms{Set}$ defined as
above, the Cartesian product of elements and morphisms with
everything done componentwise. This is too set theoretic. We
would like to relax the notion of a product a wee bit.

\begin{defn}%\label{defn:}
Let $\ms{C}$ be a category, $X_{1},X_{2}\in\ms{C}$, we define the
\define{product} of $X_1$ and $X_2$ to be an object $P\in\ms{C}$
with two morphisms $\pi_{1}:P\to X_{1}$ and $\pi_{2}:P\to X_{2}$
such that for every object $C\in\ms{C}$ equipped with the
morphisms
\begin{equation}%\label{eq:}
f_{1}:C\to X_{1},\quad\text{and}\quad f_{2}:C\to X_{2}
\end{equation}
we have that there is a unique function $f:C\to P$ such that the
following diagram commute:
\begin{equation}%\label{eq:}
\vcenter{\xymatrix{
&\ar[dl]_{f_{2}}C\ar@{-->}[d]_{f}\ar[dr]^{f_{1}}&\\
X_{2}&\ar[l]^{\pi_{2}}P\ar[r]_{\pi_{1}}&X_{1}
}}
\end{equation}
\noindent Sometimes we denote $P=X_{1}\times X_{2}$ and $f=\<f_{1},f_{2}\>$.
\end{defn}
This may possibly seem like an arbitrary condition to let such a
diagram hold, but it secretly contains two lifting problems
whose solution is $f$.

%\section{Quotient Categories}
%\input{src/quotients}

%%%%%%%%%%%%%%%%%%%%%%%%%%%%%%%%%%%%%%%%%%%%%%%%%%%%%%
%% A MATHEMATICIAN'S APPROACH PART TWO
%%%%%%%%%%%%%%%%%%%%%%%%%%%%%%%%%%%%%%%%%%%%%%%%%%%%%%
\chapter{A Mathematician's Approach (Movement Two \emph{Andante})}
\section{Introduction to Movement Two: Constructions in Categories}
%%
%% introMovtTwo.tex
%% 
%% Made by Alex Nelson
%% Login   <alex@tomato>
%% 
%% Started on  Sat Jul 18 12:12:48 2009 Alex Nelson
%% Last update Sat Jul 18 12:12:48 2009 Alex Nelson
%%

We introduced the basic notions in category theory (objects,
morphisms, functors, and natural transformations), and we
introduced the notion of objects as ``stuff'' equipped with
``structure'' such that ``properties'' hold. We went out of our
way to show that categories are objects. Now we are interested in
some structure we equip on categories and properties we demand
categories to obey.

The first major concept we want to tackle is the notion of
universal arrows. But this is too deep a concept to be tackled
``head on'', we need to cover a few preliminary notions first.

We introduce the notion of duality, and specifically how to find
dual properties in category theory. Intuitively, we find the
``dual'' to something by ``reversing the direction of the
arrows''. 

Following this principle, we'll move on to initial and terminal
objects. Put simply, initial objects have exactly one arrow to
every object in the category. Terminal objects are ``dual'' to
this (every object has exactly one arrow to the terminal object).


We then proceed to introduce the notion of comma categories. That
is, a category has objects and morphisms, and morphisms are
themselves (in a sense) objects. The logical question is: can we
have a category whose objects are morphisms? This is precisely
what comma categories formalize.

These notions provide sufficient structure and properties to
introduce the notion of universal arrows. That is, in math we
come across the recurring phrase ``\ldots\emph{there exists} a
\emph{unique} function such that\ldots'' which is concerned with:
(1) existence, i.e. define entities; and (2) uniqueness,
i.e. prove properties. It turns out that a universal arrow is an
initial object in a comma category. 

We'll cover a few examples of universal arrows and its
usefulness, but that concludes this movement of the
mathematician's approach.

\section{Duality Principle}
%%
%% duality.tex
%% 
%% Made by Alex Nelson
%% Login   <alex@tomato>
%% 
%% Started on  Sat Jul 18 12:15:08 2009 Alex Nelson
%% Last update Sat Jul 18 12:15:08 2009 Alex Nelson
%%
Duality is a very useful notion in category theory. We use it to
get ``two objects for the price of one definition''. 

\begin{defn}%\label{defn:}
Given a category $\ms{A}$, we can define a \define{dual category of $\ms{A}$}
consists of
\begin{enumerate}
\item a collection of objects $\ob{\ms{A}}$, and
\item for each pair of objects $X,Y\in\ob{\ms{A}}$ the set
  $\hom_{\ms{A}^{op}}(A,B)=\hom_{\ms{A}}(B,A)$.
\end{enumerate}
All the structure and properties of a dual category are inherited
from a category in the obvious way (loosely, ``by reversing the arrows'').
\end{defn}

\noindent Usually, we use the prefix \emph{co-} for dual objects, hence why
we use codomain --- when we reverse the direction of $f$, we get
\begin{equation}%\label{eq:}
f^{\text{op}}:y\to x
\end{equation}
where $f^{\text{op}}$ is the dual to $f$, and $y$ is the domain
of the dual to $f$ \emph{or the ``codomain''}.


\begin{rmk}
Observe that the dual some property $\mathcal{P}^{\text{op}}$ is
$\mathcal{P}$ (reversing the direction of reversed arrows returns
the original direction, kind of like squaring -1 yields 1).
\end{rmk}

We will give the general procedure for finding the dual of a
property about objects $X$ in $\ms{A}$:

\begin{ex}
Consider the property of objects $X$ in $\ms{A}$:
\begin{align*}
\mathcal{P}_{\ms{A}}(X) \equiv&\text{\emph{ For any }}\ms{A}\text{\emph{-object }}
A\text{\emph{ there exists is exactly one}}\\
 &\ms{A}\text{\emph{-morphism }}f:A\to X
\end{align*}
Step 1: In $\mathcal{P}_{\ms{A}}(X)$, replace all occurrences of
$\ms{A}$ by $\ms{A}^{op}$, thus yielding the property
\begin{align*}
\mathcal{P}_{\ms{A}^{op}}(X) \equiv&\text{\emph{ For any }}\ms{A}^{op}\text{\emph{-object }}
A\text{\emph{ there exists is exactly one}}\\
 &\ms{A}^{op}\text{\emph{-morphism }}f:A\to X
\end{align*}
Step 2: ``Translate it into the logically equivalent statment.''
That is translate it into the equivalent statement, translating
the dual category into the original category, dual morphisms into
the original ones, etc.:
\begin{align*}
\mathcal{P}^{op}_{\ms{A}}(X) \equiv&\text{\emph{ For any }}\ms{A}\text{\emph{-object }}
A\text{\emph{ there exists is exactly one}}\\
 &\ms{A}\text{\emph{-morphism }}f:X\to A.
\end{align*}
\end{ex}

Similarly, there is a procedure for finding the dual for a
property about morphisms:
\begin{ex}
Consider the property for a morphism $X\xrightarrow{\;\;f\;\;}Y$ in $\ms{A}$
\begin{align*}
\mathcal{Q}_{\ms{A}}(f) \equiv&\text{\emph{There exists an
}}\ms{A}\text{\emph{-morphism
}}Y\xrightarrow{\;\;g\;\;}X\\
&\text{\emph{ with }}X\xrightarrow{\;\;f\;\;}Y\xrightarrow{\;\;g\;\;}X=X\xrightarrow{\;\;id\;\;}X
\end{align*}
Step 1: In $\mathcal{P}_{\ms{A}}(X)$, replace all occurrences of
$\ms{A}$ by $\ms{A}^{op}$, thus yielding the property
\begin{align*}
\mathcal{Q}_{\ms{A}^{op}}(f) \equiv&\text{\emph{There exists an
}}\ms{A}^{op}\text{\emph{-morphism
}}Y\xrightarrow{\;\;g\;\;}X\\
&\text{\emph{ with }}X\xrightarrow{\;\;f\;\;}Y\xrightarrow{\;\;g\;\;}X=X\xrightarrow{\;\;id\;\;}X
\end{align*}
Step 2: Translate $\mathcal{Q}_{\ms{A}^{op}}(f)$ into the
logically equivalent statement:
\begin{align*}
\mathcal{ Q}_{\ms{A}}^{op}(f) \equiv&\text{\emph{There exists an
}}\ms{A}\text{\emph{-morphism
}}X\xrightarrow{\;\;g\;\;}Y\\
&\text{\emph{ with }}Y\xrightarrow{\;\;g\;\;}X\xrightarrow{\;\;f\;\;}Y=Y\xrightarrow{\;\;id\;\;}Y
\end{align*}
\end{ex}

We now introduce one of the most foundational concepts in
category theory:

\begin{framed}
\begin{dual}\addcontentsline{toc}{section}{*** Important Concept: Duality Principle}
\label{dualityPrinciple}
Whenever a property $\mathcal{P}$ holds for all categories, then
the property $\mathcal{P}^{op}$ holds for all categories.
\end{dual}
\end{framed}

We can observe several properties,
\begin{enumerate}
\item $(\ms{A}^{op})^{op}=\ms{A}$, and
\item $\mathcal{P}^{op}(\ms{A})$ holds iff $\mathcal{P}(\ms{A}^{op})$ holds.
\end{enumerate}
Further, we say a property $\mathcal{P}$ is \define{Self-Dual} if
$\mathcal{P}^{op}=\mathcal{P}$.
%% \begin{framed}
%% \begin{dual}\addcontentsline{toc}{mysubsection}{*** Important Concept: Duality Principle}
%% \label{dualityPrinciple}
%% Given a category $\ms{C}$ with some property $\mathcal{P}$, 
%% we can find its dual (denoted by $\ms{C}^{\text{op}}$) by
%% simply reversing the direction of its morphisms. The dual
%% category has the dual property $\mathcal{P}^{\text{op}}$.
%% \end{dual}
%% \end{framed}

\section{Initial and Terminal Objects}
%%
%% initialTerminalObjects.tex
%% 
%% Made by Alex Nelson
%% Login   <alex@tomato>
%% 
%% Started on  Sat Jul 18 12:03:44 2009 Alex Nelson
%% Last update Sat Jul 18 12:03:44 2009 Alex Nelson
%%

The notions of initial and terminal objects are very useful later
on when thinking about universal arrows --- that is, whenever we
have phrases like ``\ldots\emph{there exists} a \emph{unique}
function such that\ldots'' should ring an alarm that we're
working with a universal arrow, which is an initial object in
some category.

\begin{defn}%\label{defn:}
An object $0\in\ms{C}$ is called an \define{Initial Object} if, for every
object $A\in\ms{C}$, there is exactly one arrow $0\xrightarrow{\;\;!\;\;}A$.
\end{defn}

This is fairly simple as a definition: an initial object is
mapped to every object in the category, including itself (by the
identity morphism).

\begin{defn}%\label{defn:}
An object $1\in\ms{C}$ is called a \define{Terminal Object} if, for every
object $A\in\ms{C}$, there is exactly one arrow from $A\xrightarrow{\;\;!\;\;}1$.
\end{defn}

Note we denote morphisms to initial (respectively, from terminal)
objects by $!$.


\begin{ex}
In $\ms{Set}$, the empty set $\emptyset=\{\}$ is the only initial
object. For every set $S$, the empty function is the unique
function from $\emptyset\to S$.

In $\ms{Set}$, each one-element set is a terminal object. Why?
Well, for each set $S\in\ms{Set}$ there is a function from $S$ to
a one element set $\{x\}$ mapping every element of $S$ to $x$
(the constant function).
\end{ex}

\begin{ex}
In $\ms{Grp}$, the trivial group $G=\{e\}$ is the initial object,
since there is exactly one group homomorphism from $G$ to every
group $G'\in\ms{Grp}$. 
\end{ex}

\begin{ex}
In $\ms{Cat}$, the category $\ms{0}$ is the initial object and
$\ms{1}$ is the terminal object for the exact same reasoning that
$\emptyset$ and $\{x\}\in\ms{Set}$ are initial (terminal) objects
(respectively).  
\end{ex}

\begin{ex}
If we think of a topological space $(X,T)$ (where $T$ is a
topology on $X$) as a category whose objects are open sets
$U\in{}T$, we have the morphisms be inclusions --- i.e. $U\to{}V$
iff $U\subseteq{}V$. Then $\emptyset$ is an initial object, and
$X$ is a terminal object.
\end{ex}

\begin{defn}%\label{defn:}
If $C\in\ms{C}$ is both an initial and a terminal object, then it
is called a \define{Null Object}.
\end{defn}

\section{Comma Category}
%%
%% commaCategory.tex
%% 
%% Made by Alex Nelson
%% Login   <alex@tomato>
%% 
%% Started on  Sat Jul 18 13:57:26 2009 Alex Nelson
%% Last update Sat Jul 18 13:57:26 2009 Alex Nelson
%%

%%%%%%%%%%%%%%%%%%%%%%%%%%%%%%%%%%%%%%%%%%%%%%%%%%%%%%%%%%%%%%%%%%%%%%%%%%%%%%%%
% CATEGORY OVER-B AND UNDER-A
%%%%%%%%%%%%%%%%%%%%%%%%%%%%%%%%%%%%%%%%%%%%%%%%%%%%%%%%%%%%%%%%%%%%%%%%%%%%%%%%
\subsection{Category of objects over $B$ and under $A$}
If $B\in\ms{C}$ is an object, we can construct a \define{Category of Objects Under $B$}
is the category $(B\downarrow{}\ms{C})$ with:
\begin{itemize}
\item objects be ordered pairs $(f,C)$ where $f:B\to{}C$;
\item arrows $h:(f,C)\to(f',C')$ where $h:C\to{}C'$ is such that $h\circ{}f=f'$/
\end{itemize}
In other words, the objects are arrows from $B$ to $C\in\ms{C}$,
and arrows are commutative triangles with the top vertex be
$B$. Or diagramatically
\begin{equation}%\label{eq:}
\text{Objects }(f,C):\vcenter{\xymatrix{
B\ar[d]^{f}\\
C
}}\quad
\text{Arrows }(f,C)\xrightarrow{\;\;h\;\;}(f',C'):
\vcenter{\xymatrix{
& \ar[dl]_{f}B\ar[dr]^{f'}&\\
C\ar[rr]^{h}&&C'
}}
\end{equation}
The composition of arrows in $(B\downarrow\ms{C})$ is just the
composition in $\ms{C}$ of the base arrows $h$ of these triangles.

\begin{ex}
Consider any one-point set denoted by $*$, let
$X\in\ms{Set}$. Well, each function $*\to{}X$ is an element of
$X$; hence the category of objects under $*$,
$(*\downarrow\ms{Set})$, is the category of pointed sets.
\end{ex}

We can similarly (letting $A\in\ms{C}$ be an object in a category
$\ms{C}$) define a \define{Category of Objects Over $A$} denoted
by $(\ms{C}\downarrow{}A)$ as sort of dual to
$(A\downarrow\ms{C})$. We diagramatically note that it has
\begin{equation}%\label{eq:}
\text{Objects }(f,C):\vcenter{\xymatrix{
C\ar[d]^{f}\\
A
}}\quad
\text{Arrows }(f,C)\xrightarrow{\;\;h\;\;}(f',C'):
\vcenter{\xymatrix{
%& \ar[dl]_{f}B\ar[dr]^{f'}&\\
C\ar[dr]_{f}\ar[rr]^{h}&&\ar[dl]^{f'}C'\\
&A&
}}
\end{equation}
This is ``dual'' in the sense that it has its objects be arrows
with \emph{codomain} $A$ as opposed to \emph{domain} $B$.

\begin{ex}
In $\ms{Set}$, one-point sets $*$ are terminal, so there is
exactly one arrow from each object $S\in\ms{Set}$ to $*$. That
is, $S\to{}*$ is unique. So $(\ms{Set}\downarrow{}*)$ is
isomorphic to $\ms{Set}$.
\end{ex}

%%%%%%%%%%%%%%%%%%%%%%%%%%%%%%%%%%%%%%%%%%%%%%%%%%%%%%%%%%%%%%%%%%%%%%%%%%%%%%%%
% CATEGORY F-UNDER B AND G-OVER A
%%%%%%%%%%%%%%%%%%%%%%%%%%%%%%%%%%%%%%%%%%%%%%%%%%%%%%%%%%%%%%%%%%%%%%%%%%%%%%%%
\subsection{Category of Objects $F$-Under $B$ and $G$-Over $A$}
If $B\in\ms{C}$ is an object of the category $\ms{C}$, and
$F:\ms{D}\to\ms{C}$ is a functor, we can introduce the notion of
a \define{Category of Objects $F$-Under $B$} which has as
\begin{itemize}
\item objects all ordered pairs $(f,D)$ where $D\in\ms{D}$ and $f:B\to{}F(D)$;
\item arrows $h:(f,D)\to{}(f',D')$ all arrows $h:D\to{}D'$ in
  $\ms{D}$ for which $f'=S(h)\circ{}f$.
\end{itemize}
In pretty pictures, we can write them as
\begin{equation}%\label{eq:}
\text{Objects: }\vcenter{\xymatrix{
B\ar[d]^{f}\\
F(D)
}}\quad
\text{Arrows: }
\vcenter{\xymatrix{
& \ar[dl]_{f}B\ar[dr]^{f'}&\\
F(D)\ar[rr]^{F(h)}&&F(D')
}}
\end{equation}
As before, arrow composition takes place in $\ms{D}$.

If $A\in\ms{C}$ is an object of the category $\ms{C}$ and
$G:\ms{D}\to\ms{C}$ is a functor, we can analogously construct
the \define{Category of Objects $G$-Over $A$}. We leave it as an
exercise to the reader.

\subsection{Comma Categories}

We'll describe the basic construction. Given functors and
categories
\begin{equation}%\label{eq:}
\ms{E}\xrightarrow{\;\;T\;\;}\ms{C}\xleftarrow{\;\;S\;\;}\ms{D}
\end{equation}
the \define{Comma Category} $(T\downarrow{}S)$ --- also written
sometimes as $(T,S)$ --- has as
\begin{itemize}
\item objects all triples $(E,D,f)$ with $E\in\ms{E}$, $D\in\ms{D}$ and $f:T(E)\to{}S(D)$
\item arrows $(E,D,f)\to(E',D',f')$ all pairs $(k,h)$ of arrows
  $k:E\to E'$ in $\ms{E}$ and $h:D\to D'$ in $\ms{D}$, such that $f'\circ{}T(k)=S(h)\circ{}f$.
\end{itemize}
In prettier pictures
\begin{equation}
\text{Objects }(D,E,f):\vcenter{\xymatrix{ T(E)\ar[d]^{f}\\ S(D)}}\quad
\text{Arrows }(k,h):\vcenter{\xymatrix{
T(E)\ar[d]^{f}\ar[r]^{T(k)}&T(E')\ar[d]^{f'}\\
S(D)\ar[r]_{S(h)}& S(D')
}}
\end{equation}
with the square commutative. The composition of arrows is done
componentwise $(k',h')\circ(k,h)=(k'\circ{}k,h'\circ{}h)$, when
defined.

Note that this notion of a comma category is more general than
the notions previously introduced, and actually embodies them
quite naturally. Any object $C\in\ms{C}$ can be considered as a
functor from $C:\ms{1}\to\ms{C}$. So we can construct, easily,
the category of objects over $C$, or under $C$, or $S$-under $C$
or $T$-over $C$. The comma category embodies it all quite
naturally.

\section{Universal Arrows}
%%
%% universalArrows.tex
%% 
%% Made by Alex Nelson
%% Login   <alex@tomato>
%% 
%% Started on  Sat Jul 18 14:43:46 2009 Alex Nelson
%% Last update Sat Jul 18 14:43:46 2009 Alex Nelson
%%

As previously mentioned, universal arrows arise whenever phrases
like ``\ldots\emph{there exists} a \emph{unique} function such
that\ldots'' occur. We're concerned about (1) existence, and (2)
uniqueness. We'll start by defining a universal arrow.

\begin{defn}%\label{defn:}
If $F:\ms{D}\to\ms{C}$ is a functor and $C\in\ms{C}$ is an
object, a \define{Universal Arrow from $C$ to $F$} is a pair
$(D,u)$ consisting of
\begin{itemize}
\item an object $D\in\ms{D}$;
\item an arrow $u:C\to F(D)$ of $\ms{C}$;
\end{itemize}
such that
\begin{itemize}
\item to every pair $(D',f)$ with $D'\in\ms{D}$ and $f:C\to
  F(D')$ an arrow of $\ms{C}$, there is a unique arrow $f':D\to
  D'$ with $F(f')\circ{}u=f$.
\end{itemize}
Or in other words, every arrow $f$ to $F$ factors uniquely
through the universal arrow $u$ as in the commutative diagram
\begin{equation}%\label{eq:}
\vcenter{\xymatrix{
C\ar@{=}[d]\ar[r]^{u}&F(D)\ar@{-->}[d]^{F(f')}&D\ar@{-->}[d]^{f'}\\
C\ar[r]^{f}&F(D')&D'
}}
\end{equation}
\end{defn}
As previously mentioned, we can come up with an equivalent
definition using initial objects instead. We say that $u:C\to
F(D)$ is universal from $C$ to $F$ when the pair $(D,u)$ is
an initial object in the comma category $(C\downarrow{}F)$ whose
objects are arrows $C\to{} F(D')$. As with any initial object, it
turns out that $(D,u)$ is unique up to isomorphism in
$(C\downarrow{}F)$. This is the typical use of the comma
categories.

Like most of category theory, this definition is best illuminated
with many novel examples.

\begin{ex}[Tensor Algebra]\index{Functor!Example!Tensor Algebra}
Let $V$ be a vector space over $\mathbb{K}$, and $A$ be an
algebra over $\mathbb{K}$. To construct the tensor algebra over
$V$ we typically do a mathematical procedure, let:
\begin{equation}%\label{eq:}
T^kV = V^{\otimes k} = V\otimes V \otimes \cdots \otimes V. 
\end{equation}
then we can define the tensor algebra over $V$ as
\begin{equation}%\label{eq:}
T(V)= \bigoplus_{k=0}^\infty T^kV = K\oplus V \oplus (V\otimes V) \oplus (V\otimes V\otimes V) \oplus \cdots.
\end{equation}
Our intuition about category theoretic descriptions of
mathematical procedures should be alerted: we have $T(V)$ as a
mathematical procedure! It's a \emph{functor!} 

Now, the question that should come to mind next is \emph{Suppose the Tensor Algebra is a functor, what categories are its domain and codomain?}
Excellent question! We should suspect since it acts on $V$, a
finite dimensional vector space over $\mathbb{K}$, that its
domain category should be $\ms{Vect}_{\mathbb{K}}$ and its
codomain category is yet to be known. It wouldn't be too much of
a stretch to suppose its codomain category would be the category
of graded algebras over $\mathbb{K}$. Why
graded? Well, observe that our direct sum is taken
over $\mathbb{N}$, so thinking of it as a graded algebra is
natural. So we can state that
\begin{equation}
T:\ms{Vect}_{\mathbb{K}}\to\ms{Alg}_{\mathbb{K}}
\end{equation}
is a functor encoding our procedure.

Lets stop and think for a second: we have a functor $T$ and an
object $V\in\ms{Vect}_{\mathbb{K}}$. How can we construct a
universal arrow with this information? Well, we can rephrase the
question thus: what is an initial object in the comma category $(V\downarrow{}T)$?

Any linear transformation $f:V\to{}A$ can be uniquely extended to
an algebra homomorphism $\widetilde{f}$ from $T(V)$ to $A$, diagramatically
depicted as
\begin{equation}\label{eq:universalPropertyForTensorAlgebra}
\vcenter{\xymatrix{
V\ar[dr]_{f}\ar[r]^{i} & T(V)\ar@{-->}[d]^{\widetilde{f}}\\
& A
}}
\end{equation}
where $i$ is the canonical inclusion of $V$ into $T(V)$. Or to
phrase it in the more familiar manner: for each linear
transformation $f:V\to{}A$ \emph{there exists} an \emph{unique} algebra
homomorphism $\widetilde{f}:T(V)\to{}A$ \emph{such that} the
diagram in eq \eqref{eq:universalPropertyForTensorAlgebra}
commutes. Note this uniqueness is upto isomorphism. This
concludes our lengthy example.
\end{ex}


\section{Yoneda Lemma}
%%
%% yonedaLemma.tex
%% 
%% Made by Alex Nelson
%% Login   <alex@tomato>
%% 
%% Started on  Sun Jul 19 12:20:25 2009 Alex Nelson
%% Last update Sun Jul 19 12:20:25 2009 Alex Nelson
%%

We can rephrase the notion of universality with hom-sets, which
we summarize in the following proposition:
\begin{prop}\label{prop:firstPropositionInYonedaLemma}
For a functor $F:\ms{D}\to\ms{C}$ a pair $(D,u:C\to{}F(D))$ is
universal from $C$ to $F$ iff the function sending each
$f':D\to{}D'$ into $S(f')\circ{}u:C\to{}F(D)$ is a bijection of
hom-sets
\begin{equation}\label{eq:alternateConditionUniversalityWithHomSets}
\hom_{\ms{D}}(D,D')\cong\hom_{\ms{C}}(C,F(D')).
\end{equation}
This bijection is natural in $D'$. Conversely, given $D$ and $C$,
any natural isomorphism \eqref{eq:alternateConditionUniversalityWithHomSets} is determined in this way by a
unique arrow $u:C\to{}F(D)$ such that $(D,u)$ is universal from
$C$ to $F$.
\end{prop}
\begin{proof}
The statement that $(D,u)$ is universal is basically the same as
$f'\mapsto{}F(f')\circ{}u=f$ is bijective. Why can we state this?
Well, for each $f$ there is a corresponding $f'$ such that
$F(f')\circ{}u=f$. This is a one-to-one correspondence, implying
there is a bijection. This is also natural in $D$, what does this
mean? Well, if we had a $g:D'\to{}D''$, then we would have
$F(g\circ{}f')\circ{}u=F(g)\circ{}(F(f')\circ{}u)$. 

Conversely, a natural isomorphism
\eqref{eq:alternateConditionUniversalityWithHomSets} gives for
each $D'\in\ms{D}$ a bijection
$\varphi_{D'}:\hom_{\ms{D}}(D,D')\cong{}\hom_{\ms{C}}(C,F(D'))$. In
particular, choosing $D'=D$, we end up with
$\varphi_{D}:\hom_{\ms{D}}(D,D)\cong{}\hom_{\ms{C}}(C,F(D))$ and
we know there is a special element of $\hom_{\ms{D}}(D,D)$ ---
the identity $\id{D}\in\hom_{\ms{D}}(D,D)$! This means there is a
corresponding special element in $\hom_{\ms{C}}(C,F(D))$, by our
natural isomorphism! Lets consider what happens in our naturality
condition, for any $f':D\to{}D''$ the diagram
\begin{equation}%\label{eq:}
\vcenter{\xymatrix{
\hom_{\ms{D}}(D,D)\ar[d]_{\hom_{\ms{D}}(D,f')}\ar[rr]^{\varphi_{D}}
&& \hom_{\ms{C}}(C,F(D))\ar[d]^{\hom_{\ms{C}}(C,F(f'))}\\
\hom_{\ms{D}}(D,D'')\ar[rr]_{\varphi_{D''}} && \hom_{\ms{C}}(C,F(D''))
}}
\end{equation}
commutes by the naturality of $\varphi$. But by the top right of
the diagram, $\id{D}$ is mapped to $F(f')\circ{}u$, and to the
bottom left of the diagram it's mapped to
$\varphi_{D''}(f')$. Since $\varphi$ is a bijection, this states
that each $f:C\to{}F(D)$ has the form $f=F(f')\circ{}u$ for some
corresponding (unique) $f'$. This is precisely stating $(D,u)$ is
universal, so this concludes our proof.
\end{proof}

Note that this technique, when we have a natural isomorphism from
$\hom_{\ms{C}}(X,X')$ to $\hom_{\ms{D}}(Y,F(X'))$, to pick $X=X'$
and have the insight to deduce that there is a special element in
$\hom_{\ms{D}}(Y,F(X))$ corresponding to
$\id{X}\in\hom_{\ms{C}}(X,X)$ is one recurring pattern in
category theoretic proofs.

Observe if $\ms{C}$, $\ms{D}$ have small hom-sets, then our
proposition is precisely stating that the functor
$\hom_{\ms{C}}(C,F(-)):\ms{D}\to\ms{Set}$ is naturally isomorphic
to a covariant hom-functor
$\hom(D,-):\ms{D}\to{}\ms{Set}$. Before we can really say ``Woah,
awesome!'' we should really introduce a new notion:

\begin{defn}%\label{defn:}
Let $\ms{D}$ have small hom-sets. A \define{Representation of a Functor} 
$K:\ms{D}\to\ms{Set}$ is a pair $(D,\psi)$ with $D\in\ms{D}$ and
\begin{equation}%\label{eq:}
\psi:\hom_{\ms{D}}(D,-)\cong{}K
\end{equation}
a natural isomorphism. The object $D$ is called the
\define{Representing Object}. The functor $K$ is said to be
\define{Representable} when such a representation exists.
\end{defn}

Up to isomorphism, a representable functor is just a covariant
hom-functor $\hom_{\ms{D}}(D,-)$. Very interesting, we began with
universal arrows, and ended up with a notion of representations
of functors. Perhaps we can make the connection more explicit?

\begin{prop}%\label{prop:}
Let $*$ be any one-point set, $\ms{D}$ have small hom-sets. If
$(D,u:*\to{}K(D))$ is a universal arrow from $*$ to
$K:\ms{D}\to{}\ms{Set}$, then
\begin{enumerate}
\item the function $\psi$ which (for each $D'\in\ms{D}$) sends
  the arrow $f':D\to{}D'$ to $K(f')(u*)\in{}K(D')$ is a
  representation of $K$;
\item every representation of $K$ is obtained this way from
  exactly one such universal arrow.
\end{enumerate}
\end{prop}
\begin{proof}
For any set $X$, the function $f:*\to{}X$ is determined by the
element $f(*)\in{}X$. The correspondence $f\mapsto{}f(*)$ is a
bijection $\hom_{\ms{Set}}(*,X)\to{}X$, natural in
$X\in\ms{Set}$. How can we see this claim? Well,
$\hom_{\ms{Set}}(*,X)$ is the set of all function
$f:*\to{}X$. These are in one-to-one correspondence to each
element of $X$. This means there is such a bijection. 

Now, we can compose with $K$ to obtain a natural isomorphism
\begin{equation}%\label{eq:}
\hom_{\ms{Set}}(*,K(-))\cong{}K.
\end{equation}
This should be fairly straightforward to see, by proposition \ref{prop:composeFunctorsAndNaturalTransformations}.

This together with the representation $\psi$ gives (by definition
of a representation of a functor):
\begin{equation}%\label{eq:}
\hom_{\ms{Set}}(*,K(-))\cong{}K\cong{}\hom_{\ms{D}}(D,-).
\end{equation}
A representation of $K$ amounts to a natural isomorphism
$\hom_{\ms{Set}}(*,K(-))\cong{}\hom_{\ms{D}}(D,-)$. The rest of
the proposition holds from proposition \ref{prop:firstPropositionInYonedaLemma}.
\end{proof}

These propositions allows us to consider one of the most
important lemmas in category theory: the Yoneda Lemma. It
basically states that, when studying a small category $\ms{D}$,
we should study the category of all functors from $\ms{D}$ to
$\ms{Set}$. That is, the objects are functors
$F,F':\ms{D}\to\ms{Set}$ and the morphisms are natural
transformations $\alpha:F\Rightarrow{F'}$. This notion (that the
functors from $\ms{D}\Rightarrow\ms{Set}$ determines everything
of interest about $\ms{D}$) is similar to the remark we made
earlier, about having expressions like
$\hom_{\ms{C}}(C,C')\cong{\hom_{\ms{D}}(D,F(C'))}$ be determined
completely by choosing $C'=C$ and figuring out what happens to $\id{C}$.

\begin{framed}
\begin{yoneda}\index{Yoneda Lemma}\addcontentsline{toc}{section}{*** Important Concept: Yoneda Lemma}
If $K:\ms{D}\to\ms{Set}$ is a functor from $\ms{D}$, and
$D\in\ms{D}$ is an object in a category $\ms{D}$ with small
hom-sets, then there is a bijection
\begin{equation}\label{eq:yonedaMap}
y:\nat{\hom_{\ms{D}}(D,-),K}\cong{}K(D)
\end{equation}
which sends each natural transformation
$\alpha:\hom_{\ms{D}}(D,-)\Rightarrow{}K$ to $\alpha_{D}\id{D}$,
the image of the identity $D\to D$.
\end{yoneda}
\end{framed}
\begin{proof}
Trivial, it follows from the commutative diagram
\begin{equation}%\label{eq:}
\vcenter{\xymatrix{
\hom_{\ms{D}}(D,D)\ar[rr]^{\alpha_{D}}\ar[d]_{f_{*}=\hom_{\ms{D}}(D,f)}
&& K(D)\ar[d]^{K(f)}  & D\ar[d]_{f}\\
\hom_{\ms{D}}(D,D')\ar[rr]^{\alpha_{D'}}
&& K(D'), & D'.
}}
\end{equation}
\end{proof}

\begin{cor}%\label{cor:}
For objects $A,B\in\ms{D}$, each natural transformation
$$\hom_{\ms{D}}(A,-)\Rightarrow{\hom_{\ms{D}}(B,-)}$$ has the form
$\hom_{\ms{D}}(h,-)$ for a unique arrow $h:B\to{A}$.
\end{cor}

First remark to make about the Yoneda Lemma: note the Yoneda
mapping $y$ basically states the natural
transformations $$\hom_{\ms{D}}(D,-)\Rightarrow{}K$$ is ``the same as'' 
(naturally isomorphic to) $K(D)$. It may be hard to read ``on the
fly'', so one should digest it here.

Now, the Yoneda map $y$ is ``natural'' in $K$ and $D$. To be more
precise, consider $K\in\ms{Set}^{\ms{D}}$ as an object in the
functor category. We basically want to show that ``evaluation''
as a functor $E:\ms{Set}^{\ms{D}}\times{\ms{D}}\to\ms{Set}$ and
natural transformations as a functor
$N:\ms{Set}^{\ms{D}}\times{\ms{D}}\to\ms{Set}$ (it maps each
object $(K,D)$ in its domain to the set of natural
transformations $\nat{\hom_{\ms{D}}(D,-),K}$) are naturally isomorphic.

Consider further both the domain and codomain of the map $y$ as
functors of the pair $(K,D)$, and consider this pair as an object
in the category $\ms{Set}^{\ms{D}}\times{\ms{D}}$. The codomain
of $y$ is then merely the evaluation functor $E$ which maps
$(K,D)$ to the value $K(D)$ of $K$ at $D$. The domain is the
functor $N$ which maps the object $(K,D)$ to the set
$\nat{\hom_{\ms{D}}(D,-),K}$ of all natural transformations from
$\hom_{\ms{D}}(D,-)$ to $K$ and which maps arrows $F:K\to{K'}$,
$f:D\to{D'}$, to $\nat{\hom_{\ms{D}}(f,-),F}$. With these
observations, we may prove an addendum to the Yoneda Lemma:
\begin{lem}%\label{lem:}
The bijection in eq \eqref{eq:yonedaMap} is a natural isomorphism
$y:N\cong{E}$ between the functors $E,N:\ms{Set}^{\ms{D}}\times{\ms{D}}\to\ms{Set}$.
\end{lem}


%%%%%%%%%%%%%%%%%%%%%%%%%%%%%%%%%%%%%%%%%%%%%%%%%%%%%%
%% A MATHEMATICIAN'S APPROACH PART THREE
%%%%%%%%%%%%%%%%%%%%%%%%%%%%%%%%%%%%%%%%%%%%%%%%%%%%%%
\chapter{A Mathematician's Approach (Movement Three \emph{Minuet})}
\section{Introduction: Constructions with Cones and Limits}
%%
%% introMovtThree.tex
%% 
%% Made by Alex Nelson
%% Login   <alex@tomato>
%% 
%% Started on  Sat Jul 18 14:59:53 2009 Alex Nelson
%% Last update Sat Jul 18 14:59:53 2009 Alex Nelson
%%

We continue this movement by introducing some more structure on a
category. We'll do this in a sort of unique way: first we'll
introduce the notions of cones and limits. Then we'll introduce a
number of interesting structure as \emph{examples} of cones and
limits! 

In particular, the examples which we'll cover are products of
categories, pullbacks, and equalizers (as well as terminal
objects). The manner which we'll cover these exotic structures is
(appropriately enough) in the manner of a minuet --- we'll dance
back and forth over the definition, giving examples of each
structure, and (hopefully) some examples reasoning with such things.

\section{Diagrams, Cones, and Limits}
%%
%% cones.tex
%% 
%% Made by Alex Nelson
%% Login   <alex@tomato>
%% 
%% Started on  Mon Jul 20 13:51:20 2009 Alex Nelson
%% Last update Mon Jul 20 13:51:20 2009 Alex Nelson
%%

\subsection{Diagrams}

We have an intuition in our heart of hearts of what a diagram in
category theory is, but is there a rigorous way to describe it?
The answer is ``Yes, there is a rigorous way to describe it.''
What we do is we typically have a category which describes the
shape of the diagram. It is typically finite (or at least small),
we shall denote the shape category by $\mathbb{I}$.

Now, the diagram is actually a functor which embeds the shape
into our category $\ms{C}$. That is, a diagram
$$ D:\mathbb{I}\to\ms{C} $$
is a functor that identifies the diagram in $\ms{C}$.

\subsection{Cones}

We need to first introduce the constant functor before leaping to
cones. Consider a functor $\Delta_{U}:\mathbb{I}\to\ms{C}$ be
such that
\begin{equation}%\label{eq:}
\Delta_{U}\left(A\xrightarrow{\;\;f\;\;}B\right) = U\xrightarrow{\;\;\id{U}\;\;}U
\end{equation}
or in other words, every object is mapped to $U$ and every
morphism is mapped to $\id{U}$. This shouldn't be too surprising
behavior for a functor dubbed ``the constant functor''!

Now, we can define a cone:
\begin{defn}%\label{defn:}
Let $D:\mathbb{I}\to\ms{C}$ be a diagram in $\ms{C}$ and
$\Delta_{U}:\mathbb{I}\to\ms{C}$ be the constant functor.
A \define{Cone over $D$ with Vertex $U$} is a natural
transformation $\Delta_{U}\Rightarrow{D}$ with components (for
all $I\in\mathbb{I}$)
\begin{equation}\label{eq:coneNaturalityCondition}
\vcenter{\xymatrix{
U\ar[d]_{\id{U}}\ar[r]^{P_{I}}&D(I)\ar[d]^{D(f)}\\
U\ar[r]_{P_{I'}}&D(I')
}}
\end{equation}
Observe that this is secretly a triangle with vertex $U$! The
naturality condition encodes all the information about the cone.
\end{defn}

(We can also have the notion of a \emph{cocone} which is dual to
the cone; we just have the \emph{target} of the arrows be $U$
instead of having the \emph{source} of the arrows be $U$.)

Now, the reason for calling it a cone is somewhat clear, it
secretly is a triangle with vertex $U$ after all. But the reason
why it's important is not so clear at the moment. What happens is
that we have a diagram and an object $U$. We basically have
morphisms from $U$ to each object in the diagram, and demand the
resulting larger diagram commutes. But this larger diagram can be
broken up into many smaller diagrams similar to eq
\eqref{eq:coneNaturalityCondition}. Why is this a good thing? Why
should we care about such a gadget that has morphisms from $U$ to
each object in a diagram? It turns out that we can express many
notions (structures we can equip a category with) in category
theory as a cone (or more precisely, a ``limit''). We will now
turn our focus to limits.

\subsection{Limits}

Now in analysis, we typically have limits of sequences of numbers
\begin{equation}%\label{eq:}
\lim_{n\to\infty}x_{n}=x
\end{equation}
or we have a limit of a function
\begin{equation}%\label{eq:}
\lim_{x\to{x_{0}}}f(x)=L.
\end{equation}
We (should) have a good intuition about these procedures. But in
category theory, what can we take the limit of? How about taking
the limit \emph{of diagrams?}

It turns out that a limit for a diagram in $\ms{C}$ is a
universal cone. What do we mean by a ``universal cone''? Well,
universal objects usually means that every other object ``factors
uniquely'' through it. So what would this mean for our precious
cone? It means that if we have a universal cone with vertex $U$,
and another cone with vertex $V$, that $V$ factors through $U$,
i.e. there is a unique morphism $h:V\to{U}$ such that
\begin{equation}%\label{eq:}
\vcenter{\xymatrix{
V\ar[dr]_{f}\ar@{-->}[rr]^{h}&&\ar[dl]^{g}U\\
&X&
}}
\end{equation}
commutes for all $X$. This might seem innocent enough, but it
turns out that many many important notions in category theory can
be expressed as limits or colimits over some vertex and some
simple diagram. We will turn our attention to the rest of the
chapter to examples.


\section{Pullbacks and Pushouts}
%%
%% pullbackPushout.tex
%% 
%% Made by Alex Nelson
%% Login   <alex@tomato>
%% 
%% Started on  Mon Jul 20 14:21:40 2009 Alex Nelson
%% Last update Mon Jul 20 14:21:40 2009 Alex Nelson
%%

We can describe pullbacks (and its dual, pushouts) using limits
(respectively colimits). We'll unfortunately have to dance around
the intuitive notion of a pullback, its rigorous definition, and
various examples (hence why it's in the Minuet chapter!). 

\subsection{Limit Point of View}

Consider the situation when we have a pair of morphisms
\begin{equation}%\label{eq:}
\vcenter{\xymatrix{
            & B\ar[d]^{g}\\
A\ar[r]_{f} & C
}}
\end{equation}
A \define{Pullback} is a universal cone with vertex $U$ over this
diagram. That means we have a pair of projection maps $p:U\to{A}$
and $q:U\to{B}$ such that
\begin{equation}%\label{eq:}
\vcenter{\xymatrix{
U\ar[d]_{p}\ar[r]^{q}   & B \ar[d]^{g}\\
A \ar[r]_{f}           & C
}}
\end{equation}
commutes. This is sort of like a product, at least intuitively,
but if we have another object $V$ and pair of projection
morphisms $t:V\to{B}$ and $s:V\to{A}$, then we see we have to demand
\begin{equation}%\label{eq:}
\vcenter{\xymatrix@C+2em@R+2em{
   V \ar@/_10pt/[ddr]_{s} \ar@/^10pt/[drr]^{t} \ar@{-->}[dr]^{\scriptstyle !} & & \\
   & U \ar[d]_(0.4){p} \ar[r]^(0.4){q} & B \ar[d]^{g}\\
   & A \ar[r]_{f} & C
}}
\end{equation}
commutes. We denote this pullback by $A\times_{C}B$. It's
sometimes known as a fibred product or a Cartesian square.




%\ifthenelse{\boolean{isBook}}{\part{Using Categories in Mathematics}}{}
%
%

%%%%%%%%%%%%%%%%%%%%%%%%%%%%%%%%%%%%%%%%%%%%%%%%%%%%%%
%% APPENDIX
%%%%%%%%%%%%%%%%%%%%%%%%%%%%%%%%%%%%%%%%%%%%%%%%%%%%%%
\part*{Appendix}
\begin{appendix}
\section{Grocery List of Categories}
%%
%% groceryListOfCats.tex
%% 
%% Made by Alex Nelson
%% Login   <alex@tomato>
%% 
%% Started on  Fri Jun 19 12:24:01 2009 Alex Nelson
%% Last update Fri Jun 19 12:24:01 2009 Alex Nelson
%%

We have categorical description of the various different
mathematical objects, here's some selected examples:
\begin{enumerate}
\item[(Concrete)] If the category's objects are ``sets with
  structure'' (e.g. $\ms{Mon}$ has monoids, $\ms{Top}$ has
  topological spaces, etc.), then the category is ``concrete''. 
\item[(Discrete Categories)] The only morphisms are identity
  morphisms.
\item[(Groups)] If the category is a monoid with the extra
  condition the morphisms are invertible.
\item[(Monoids)] The category has just one object, so it's completely
  determined by its morphisms.
\item[(Thin)] If there is at most one morphism from an object to another.
\end{enumerate}

Here we simply give a grocery list of categories that common come
up. The following are commonly used categories, given by their
shorthand notation commonly used in the literature.
\begin{description}
\item[0] is the empty category (no objects, no arrows).
\item[1] is the singleton category, with one object and only one
  arrow -- the identity morphism.
\item[2] is the category with two objects $a,b$ and exactly one
  morphism $a\to b$ which is not the identity.
\item[3] is the category with three objects $a,b,c$ and exactly
  three morphisms $a\to{}b$, $b\to{}c$, $c\to{}a$, none identities.
\item[$\downdownarrows$] is the category with two objects $a,b$ and
  two morphisms $a\rightrightarrows{}b$ none identities.
\item[Mat$_\text{K}$] Matrices over a commutative ring $K$, we
  have objects be positive integers $m,n,\ldots$ and each
  $m\times n$ matrix $A$ eats in a vector in $K^n$ and spits out
  a vector in $K^m$, so it's a morphism $A:n\to m$.
\item[Set] Objects are all small sets, and morphisms are
  functions between them.
\item[Cat] Objects are all small categories, and morphisms are
  functors.
\item[Mon] Objects are small monoids, morphisms are morphisms of monoids.
\item[Grp] Objects are small groups, morphisms are morphisms of groups.
\item[Ab] Objects are small Abelian groups, morphisms are
  morphisms between them.
\item[Rng] Objects are small rings, morphisms are morphisms of rings.
\item[CRng] Objects are small commutative rigns, morphisms are
  morphisms between them.
\item[$R$-Mod] All small left modules over the ring $R$, with
  linear maps.
\item[Mod-$R$] Small right $R$-modules.
\item[Top] Small topological spaces and continuous maps.
\end{description}

\nocite{*}
\bibliographystyle{utcaps}
\bibliography{main}
\end{appendix}

\end{document}
