%%
%% 4April2008.tex
%% 
%% Made by Alex Nelson
%% Login   <alex@tomato>
%% 
%% Started on  Fri Dec 19 19:00:28 2008 Alex Nelson
%% Last update Fri Dec 19 19:00:28 2008 Alex Nelson
%%
%
%
% Bessel Inequality
%

\begin{bessel}\index{Bessel Inequality}
If $f$ is $2\pi$-periodic and Riemann integrable from $-\pi$
to $\pi$, and $c_n$ are the Fourier coefficients of $f$,
then 
\begin{equation}
\sum^{\infty}_{n=-\infty}|c_{n}|^2\leq\frac{1}{2\pi}\int^{\pi}_{-\pi}|f(\psi)|^2d\psi
< +\infty
\end{equation}
\end{bessel}

This is important because it allows us to create a bound for
the Fourier series that guarantees convergence. It should be
noted that equivalently it can be shown
\begin{equation}
\frac{1}{4}|a_0|^2 + \frac{1}{2}\sum^{\infty}_{n=1}|a_{n}|^2
+ |b_{n}|^{2} = \sum^{\infty}_{n=-\infty}|c_{n}|^{2} \leq \frac{1}{2\pi}\int^{\pi}_{-\pi}|f(\psi)|^2d\psi< +\infty.
\end{equation}
Why? Well, recall
\begin{subequations}
\begin{align}
a_{0} &= 2c_{0} \\
a_{n} &= c_{n} + c_{-n} \\ 
b_{n} &= i(c_{n}-c_{-n})
\end{align}
\end{subequations}
and also recall in complex analysis, $|z|^2 = z\bar{z}$ for
$z\in\mathbb{C}$. We would like
\begin{equation}
|a_{n}|^{2}\to 0\qquad\text{as } |n|\to\infty
\end{equation}
or
\begin{equation}\label{eq:4April2008:F1}
c_{n}\to 0\quad\text{as }|n|\to\infty.
\end{equation}
\marginpar{Riemann integrable and periodic?}The question
that arises is ``What kind of functions are Riemann
integrable and $2\pi$-periodic?''

Let us first review a number of definitions which will
simplify our life quite a bit.

\begin{defn} 
Let $f:\mathbb{R}\to\mathbb{R}$ be a function. Then the
\textbf{left limit}\index{Limit!Left Limit} of $f$ at $x$ is
\begin{equation}
f(x^{-}) = \lim_{y\to x^{-}}f(y) = \lim_{y\uparrow x}f(y) 
\end{equation}
($y$ is increasing to $x$). The \textbf{right
  limit}\index{Limit!Right Limit} of $f$ at $x$ is
\begin{equation}
f(x^{+}) = \lim_{y\to x^{+}}f(y) = \lim_{y\downarrow x}f(y)
\end{equation}
($y$ decreases to $x$).
\end{defn}

\begin{defn}
Let $f:\mathbb{R}\to\mathbb{R}$ be a function. Then $f$ is
\textbf{continuous}\index{Continuity!Continuous}\index{Continuous|see{Continuity}}
at $x$ if
\begin{equation}
\lim_{y\to x^{+}}f(y) = \lim_{y\to x^{-}}f(y) < \infty.
\end{equation}
More rigorously, for each $\varepsilon>0$ there is a
$\delta>0$ such that
\begin{equation}
|y-x|<\delta\Rightarrow|f(y)-f(x)|<\varepsilon.
\end{equation}
\end{defn}

Let us examine an example of the $\varepsilon-\delta$
definition of the limit.

\begin{ex}
Let $f(x)=x^{3}$ and $x_{0}\in\mathbb{R}$. We have
\begin{equation}
|x-x_{0}|<\delta
\end{equation}
and in our function we have
\begin{equation}
|x^{3} - x_{0}^{3}| < \varepsilon.
\end{equation}
We need to determine $\delta$ in terms of $\varepsilon$. So
we have
\begin{equation}
(x^2 + x_{0}x + x_{0}^{2})(x - x_{0}) = x^{3} - x_{0}^{3}
\end{equation}
so
\begin{subequations}
\begin{align}
|x^{3} - x_{0}^{3}|&\leq |x^2 + x_{0}x + x_{0}^{2}||x-x_{0}|\\
&\leq |x^2 + x_{0}x + x_{0}^{2}|\delta \\
&< \varepsilon.
\end{align}
\end{subequations}
Thus we set
\begin{equation}
\delta < \frac{\varepsilon}{|x^2 + x_{0}x + x_{0}^{2}|} <
\frac{\varepsilon}{x_{0}^{2}}
\end{equation}
where the right hand side can be justified since we want the
denominator to be as small as possible, so we have $x=x_{0}$
and the denominator becomes $3x_{0}^{2}$ but this is less
than $x_{0}^{2}$ so we have a good bound. Thus we set
\begin{equation}\label{eq:4April2008:exAnswerOne}
\delta = \frac{\varepsilon}{x_{0}^{2}}
\end{equation}
if $x_{0}\neq 0$. If $x_{0}=0$, then we have
\begin{equation}\label{eq:4April2008:exAnswerTwo}
\delta = \frac{\varepsilon}{x^2}
\end{equation}
which is a \emph{better} solution since it's \emph{more
  general} than the first solution. That is, Eq
\eqref{eq:4April2008:exAnswerTwo} contains the specific case
of Eq \eqref{eq:4April2008:exAnswerOne}. But we have found a
$\delta>0$ in terms of $\varepsilon>0$ that demonstrates
continuity for arbitrary $x_0$.
\end{ex}

\begin{defn}
Let $f:\mathbb{R}\to\mathbb{R}$ be a function. Then $f$ has
a \textbf{jump discontinuity}\index{Discontinuity!Jump Discontinuity} at $x_{0}$ if $f$ is
discontinuous at $x_{0}$ but $f(x^+)$ and $f(x^-)$ both
exist and are finite.
\end{defn}

\begin{defn}
Let $f:\mathbb{R}\to\mathbb{R}$ be a function. Then we say
$f$ is \textbf{piecewise continuous}\index{Continuity!Piecewise continuous} on $\mathbb{R}$ if on
any bounded interval $[a,b]$, where $-\infty<a<b<+\infty$,
$f$ has only jump discontinuities at finitely many
points.\index{$PC(\mathbb{R})$|see{Piecewise Continuous}}
\index{Piecewise Continuous|see{Continuity}}
\end{defn}

\begin{defn}
Let $f:\mathbb{R}\to\mathbb{R}$ be a function. We say that
$f$ is \textbf{piecewise smooth}\index{Smooth!Piecewise Smooth} 
on $\mathbb{R}$ (denoted $PS(\mathbb{R}$\index{$PS(\mathbb{R})$|see{Piecewise Smooth}}) if both
$f$ and $f'$ are piecewise continuous on $\mathbb{R}$
(i.e. on $[a,b]$ both $f$ and $f'$ have only finitely many
jump discontinuities).\index{Piecewise Smooth|see{Smooth}}
\end{defn}

These definitions are nothing really of interest, they are
weaker than just being continuous, or just being
smooth. Consequently, we will not review them in great
detail.

\subsection{Dirichlet Kernel}\index{Dirichlet Kernel}

We want to consider partial sums of the Fourier series of a
given function. To do this, we will introduce a mathematical
object called the $N^{th}$ \textbf{Dirichlet Kernel}
\begin{equation}
D_{N}(\phi) = \frac{1}{2\pi}\sum^{N}_{n=-N}e^{in\phi}
\end{equation}
Observe that
\begin{subequations}
\begin{align}
\int^{0}_{-\pi}D_{N}(\phi)d\phi &=
\int^{0}_{-\pi}\left(\frac{1}{2\pi}\sum^{N}_{n=-N}e^{in\phi}\right)d\phi\\
&= \frac{1}{2\pi}\phi +
\frac{1}{\pi}\sum^{N}_{n=1}\frac{\sin(n\phi)}{n}\Big|^{0}_{-\pi}\\
&=\frac{1}{2}+\frac{1}{\pi}\sum^{N}_{n=1}\frac{\sin(n\pi)}{n}\\
&=\frac{1}{2}+0
\end{align}
\end{subequations}
Similarly, we have
\begin{equation}
\int^{\pi}_{0}D_{N}(\phi)d\phi = \frac{1}{2}
\end{equation}
But let us now see something quite miraculous, we can
evaluate the partial sums of the Fourier series using the
Dirichlet kernel. Observe
\begin{subequations}
\begin{align}
\frac{1}{2\pi}\int^{\pi}_{-\pi}f(\theta)D_{N}(\theta-\phi)d\theta
&=
\frac{1}{2\pi}\sum^{N}_{n=-N}e^{-in\phi}\int^{\pi}_{-\pi}f(\theta)e^{in\theta}d\theta\\
&=\sum^{N}_{n=-N}e^{-in\phi}\left(\frac{1}{2\pi}\int^{\pi}_{-\pi}f(\theta)e^{in\theta}d\theta\right)\\
&=\sum^{N}_{n=-N}e^{-in\phi}c_{-n}\\
&=\sum^{N}_{n=-N}e^{in\phi}c_{n}
\end{align}
\end{subequations}
which is precisely the $N^{th}$ partial sum of the Fourier series.
