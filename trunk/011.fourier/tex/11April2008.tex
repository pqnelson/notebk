%%
%% 11April2008.tex
%% 
%% Made by Alex Nelson
%% Login   <alex@tomato>
%% 
%% Started on  Fri Dec 19 22:14:44 2008 Alex Nelson
%% Last update Fri Dec 19 22:14:44 2008 Alex Nelson
%%

Here is a deep concept that we will re-examine in Lie Groups
(perhaps if we get there): any periodic function can be
considered a function on a circle. (In a hand-wavy way, we
can change the period to be from $p$ to $2\pi$ by a change
of coordinates, then we've got from the Fourier Series a
linear combination of functions on the circle $\exp(inx)$
which are all linearly independent.)
\begin{defn}
Let $k\in\mathbb{N}$. A function $f:\mathbb{R}\to\mathbb{R}$
is of class $C^k$\index{$C^k$}, $f\in C^k$, if $f$ has derivatives up to
order $k$ and (in addition to the derivatives existing) they
are continuous. When we discuss \textbf{smoothness}\index{Smooth}, we are
discussing the existence of derivatives.
\end{defn}
At first it may seem useless to introduce such a definition,
but the strength in having such a system is that all
$C^{k+1}$ functions are also $C^{k}$ functions. It provides
a clear hierarchy of ``nice'' functions.\marginpar{$C^0$ are continuous functions}But also note, if a function is
continuous, then it is necessarily $C^0$. 

But let us reiterate what we have done before we
continue. Let $f$ be a $2\pi$ periodic, piecewise smooth and
continuous. This means $f\in C^0$ and $f'\in
PC(\mathbb{R})$. This means that $f(\theta)=\sum
c_{n}e^{in\theta}$ $\forall \theta$. This series \emph{converges
absolutely and uniformly}.\index{Convergence!Fourier Series} Also, the derivative of $f$ is
equal to the termwise derivatives of its Fourier series
\begin{equation}
f'(\theta) = \sum^{\infty}_{n=-\infty}inc_{n}e^{in\theta} =
\sum n(-a_{n}\sin(n\theta)+b_{n}\cos(n\theta))
\end{equation}
at $\theta$ where $f'$ is continuous. If we apply the Bessel
Inequality, we have the following inequality on $f'$:
\begin{equation}
\sum
|nc_{n}|^{2}\leq\frac{1}{2\pi}\int^{\pi}_{-\pi}|f'(\theta)|^{2}d\theta
< \infty
\end{equation}
So it follows that
\begin{equation}
nc_{n}\to 0\quad\text{as }n\to\infty
\end{equation}
Similarly we can argue that $\sum|na_{n}|^{2}$ and $\sum
|nb_{n}|^{2}$ are finite. So $na_{n}\to 0$ and $nb_{n}\to 0$
as $n\to\infty$.

So this implies $a_n, b_n\sim 1/n^k$ where $k>1$. If $f\in
C^1$ and $f''\in PC^{1}(\mathbb{R})$, then we can apply the
same result again
\begin{equation}
\sum |nc_{n}|^{2}\leq
\frac{1}{2\pi}\int^{\pi}_{-\pi}|f''(\theta)|^{2}d\theta <
\infty
\end{equation}
So $n^2c_n$, $n^2a_n$, $n^2b_n\to 0$ as $n\to\infty$. There
is a general pattern which is beginning to emerge here.

\begin{thm}
Let $f$ be $2\pi$-periodic, if $f\in C^{k-1}$ and
$f^{(k)}\in PC(\mathbb{R})$, then
\begin{equation}
\sum|c_{n}n^k|^2\leq\frac{1}{2\pi}\int^{\pi}_{-\pi}|f^{(k)}(\theta)|^{2}d\theta<\infty
\end{equation}
which implies $n^kc_n\to 0$ as $n\to\infty$. Thus
\begin{equation}
|c_n|\leq\frac{\text{(constant)}}{n^k +
  \varepsilon_{n}},\qquad\text{where }\varepsilon_{n}>0.
\end{equation}
\end{thm}
On the other hand, if we compute the Fourier coefficients of
a function and $|c_n|\leq c/|n|^{k+\alpha}$ where $c>0$,
$\alpha>0$, we may say that $f$ is $C^{k-1}$.

\subsection{Fourier Series on Arbitrary Interval}

Fourier series on an interval $[a,b]$ where $a<b$. How is
this done? Well, usually, as a rule of thumb, we don't work
all of $\mathbb{R}$ (so $-\infty<a<b<+\infty$). For
instance, the heat distribution
\begin{eqnarray*}
f(x) &= \text{temperature distribution along a rod of length
  $L$} \\
&f:[0,L]\to\mathbb{R}
\end{eqnarray*}
More generally, let $s(t)$ be a signal generated over a time
interval $[0,T]$.

The strategy is to first change variables to make the domain
from $[0,\pi]$; then we extend our (odd or even) Fourier
coefficients to make $f$ defined on $[-\pi,\pi]$. Then we
can just make it periodic to extend it to all of
$\mathbb{R}$. If one examines any Fourier expansion table,
the functions are usually defined on the interval
$[-\pi,\pi]$ (or $[0,2\pi]$). They are periodic extensions
of functions defined on $[-\pi,\pi]$ to $\mathbb{R}$. 

\begin{defn}
Let $f:[-\pi,\pi)\to\mathbb{C}$, then
  $f_{ext}:\mathbb{R}\to\mathbb{C}$ is defined by
  $f_{ext}(\theta+2\pi)=f(\theta)$ for all
  $\theta\in[-\pi,\pi]$ is the \textbf{periodic extension of
    $f$ to $\mathbb{R}$}\index{Periodic Extension}.
\end{defn}

\textbf{CAUTION:} $f_{ext}$ may be discontinuous and
nondifferentiable at $\theta=(2n-1)\pi$ where
$n\in\mathbb{Z}$. So $f_{ext}$ is continuous at
$\theta=(2n-1)\pi$ if and only if
$f(-\pi)=f(\pi^{-})$. Similarly, $f_{ext}^{(k)}((2n-1)\pi)$
exists if and only if $f^{(j)}(\pi^{-})=f^{(j)}(-\pi^{+})$
where $j=0,1,2,...,k$.

\begin{ex}
Consider $f(\theta)=\theta^2$ where
$\theta\in[-\pi,\pi)$. We have $f:[a,b]\to\mathbb{C}$, then
  we change variable
\begin{equation}
\theta = \pi\left(\frac{x-a}{b-a}\right),\qquad
0\leq\theta\leq\pi
\end{equation}
Observe what is going on, we rearrange these terms given
arbitrary $a$ and $b$ to find some change of variables from
the interval $[a,b]$ to a new interval. It is seen to be
\begin{equation}
g(\theta) = f(x) = f\left(\frac{b-a}{\pi}\theta +
a\right),\quad g:[0,\pi]\to\mathbb{C}
\end{equation}
So we can see a fairly clean and orderly scheme of how to
change the interval.
\end{ex}

\begin{defn}
Let $f:[0,\pi]\to\mathbb{C}$, then the \textbf{odd
  extension}\index{Periodic Extension!Odd Extension} of $f$ to $[-\pi,\pi]$ is
\begin{subequations}
\begin{align}
f_{odd}(-\theta) &= -f(\theta), \quad 0<\theta<\pi\\
f_{odd}(0) &= 0
\end{align}
\end{subequations}
and the \textbf{even extension}\index{Periodic Extension!Even Extension} of $f$ to $[-\pi,\pi]$ is 
\begin{equation}
f_{even}(-\theta) = f(\theta),\qquad 0\leq\theta\leq\pi
\end{equation}
(note the possible values for $\theta$ differ for even and
odd extensions).
\end{defn}

So what about the Fourier Series of Periodically Extended
functions?\index{Periodic Extension!Fourier Series} It's
fairly straightforward to find for odd extensions:
\begin{subequations}
\begin{align}
a_{n} &=
\frac{1}{\pi}\int^{\pi}_{0}f_{odd}(\theta)\cos(n\theta)d\theta
= 0\\
b_{n} &=
\frac{1}{\pi}\int^{\pi}_{0}f_{odd}(\theta)\sin(n\theta)d\theta\\
\Rightarrow \sum^{\infty}_{n=1}b_{n}\sin(n\theta) &=
\text{``Fourier Sine Series''}
\end{align}
\end{subequations}
and for even extensions:
\begin{subequations}
\begin{align}
a_{n} &=
\frac{1}{\pi}\int^{\pi}_{0}f_{even}(\theta)\cos(n\theta)d\theta\\
b_{n} &=
\frac{1}{\pi}\int^{\pi}_{0}f_{even}(\theta)\sin(n\theta)d\theta=0\\
\Rightarrow \sum^{\infty}_{n=1}a_{n}\cos(n\theta) &=
\text{``Fourier Cosine Series''}
\end{align}
\end{subequations}
