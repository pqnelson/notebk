%%
%% box1.tex
%% 
%% Made by Alex Nelson
%% Login   <alex@tomato3>
%% 
%% Started on  Sun Feb 20 10:55:50 2011 Alex Nelson
%% Last update Sun Feb 20 17:09:04 2011 Alex Nelson
%%
\begin{framed}
\noindent{\sectionfont Box\enspace \thesection.1 Dynkin Diagrams}
\bigskip
\noindent{}The problem we are facing is really two-fold: (a)
given a Dynkin diagram obtain the Cartan matrix, and (b) given
the Cartan matrix obtain the Dynkin diagram. This box is really
based off of \S4.7 of Kac's book \emph{Infinite Dimensional Lie Algebras}.
No secrets among friends: Kac provides the method of, given a
Cartan matrix, producing the Dynkin diagram. We review this, and
provide the algorithm going in the opposite direction. We also
consider examples. Throughout $A=[a_{ij}]$ is the Cartan matrix.

\medskip
\noindent\textbf{Given Cartan Matrix Obtain Dynkin Diagram.}\enspace
The basic idea is that we will have an $n\times n$ matrix
$A$. The Dynkin diagram is a graph that will have $n$
vertices, which are labeled by integers $i=1,\dots,n$. If
\begin{equation}
a_{ij}a_{ji}\leq4\quad\mbox{and}\quad |a_{ij}|\geq|a_{ji}|
\end{equation}
then vertices $i$ and $j$ are connected by $|a_{ij}|$ lines;
moreover if $|a_{ij}|>1$, then these lines are equipped with an
arrow pointing towards vertex $j$. 

Why do we need an arrow? The idea is that the Cartan matrix is
not symmetric, but has a weaker condition that $a_{ij}\not=0$
implies $a_{ji}\not=0$. Since we know the product by the number
of lines, we know the values by considering which direction the
arrow points.

\medskip
\noindent\textbf{Given a Dynkin Diagram Obtain Cartan Matrix.}\enspace
This occurs more often in practice (at least, for
physicists). What can we know immediately from the properties of
a Cartan matrix? Well, we know
\begin{equation}
a_{ii} = 2
\end{equation}
for all $i$. We know that the number of vertices $n$ gives
information about the number of rows, and the number of columns,
of the Cartan matrix --- i.e. $A$ is an $n\times n$ matrix. We
also know if $i\not=j$ that
\begin{equation}
a_{ij}\leq0\quad\mbox{and}\quad a_{ij}\in\Bbb{Z}.
\end{equation}
The rest we need to find from the diagram.

If vertex $i$ and $j$ are connected by $k$ lines, then $a_{ij}<0$. 
What values can this component be? Well, if $k=1$, then
\begin{equation}
a_{ij}=a_{ji}=-1
\end{equation}
since there is no arrow, it must be $-1$. If there are multiple
lines, we have an arrow to indicate which entry 

\begin{rmk}
Note that in these examples, the vertices are labeled by
\emph{indices} to keep track of which we are discussing. Usually,
the labels of a vertex are the relative (squared)
lengths of the fundamental roots as Gilmore describes
it [see Robert Gilmore, \emph{Lie Groups, Lie Algebras, and Some
    of Their Applications} Dover Publications (2002) Ch 8 \S III.2 pp 306 \emph{et seq.}].

{\bf N.B.} the method we have described are used to deduce a
\emph{generalized} Cartan matrix from a Dynkin diagram. So if we
restrict focus to Dynkin diagrams corresponding to ``strict''
Cartan matrices, we recover precisely the same information. But
we can do more! We can consider ``closed loops'' in our approach!
The only requirement we have for our considerations is that there
are less than 4 edges connecting any pair of vertices.
\end{rmk}
\begin{ex}
Consider the Dynkin diagram given by
\begin{center}
  \includegraphics{img/LieImg.11}
\end{center}
We see that there are 4 vertices, so immediately we know that the
Cartan matrix is $4\times4$ and we can write:
\begin{equation}
A = \begin{bmatrix}
2 &   &   &   \\
  & 2 &   &   \\
  &   & 2 &   \\
  &   &   & 2
\end{bmatrix}.
\end{equation}
We also see that there is one line connecting vertex 1 to vertex
2, so that means we can write
\begin{equation}
A = \begin{bmatrix}
2 &-1 &   &   \\
-1& 2 &   &   \\
  &   & 2 &   \\
  &   &   & 2
\end{bmatrix}.
\end{equation}
We then observe that there are no other edges connected to 1, so
\begin{equation}
A = \begin{bmatrix}
2 &-1 & 0 & 0 \\
-1& 2 &   &   \\
0 &   & 2 &   \\
0 &   &   & 2
\end{bmatrix}.
\end{equation}
Similar reasoning holds for vertex 4, it's connected by a single
edge to vertex 3
\begin{equation}
A = \begin{bmatrix}
2 &-1 & 0 & 0 \\
-1& 2 &   &   \\
0 &   & 2 &-1 \\
0 &   &-1 & 2
\end{bmatrix}.
\end{equation}
There are no other edges that connect vertex 4 to any other
vertex, so 
\begin{equation}
A = \begin{bmatrix}
2 &-1 & 0 & 0 \\
-1& 2 &   & 0 \\
0 &   & 2 &-1 \\
0 & 0 &-1 & 2
\end{bmatrix}.
\end{equation}
We see that there are \emph{two lines} connecting vertex 2 to
vertex 3 and there is an arrow. The arrow means that
\begin{equation}
a_{23}\not=a_{32}.
\end{equation}
The arrow points towards 3, so
\begin{equation}
|a_{32}|<|a_{23}|.
\end{equation}
Then we use the fact that there are two edges means that
\begin{equation}
|a_{23}|=2
\end{equation}
This is sufficient information to conclude
\begin{equation}
A = \begin{bmatrix}
2 &-1 & 0 & 0 \\
-1& 2 &-2 & 0 \\
0 &-1 & 2 &-1 \\
0 & 0 &-1 & 2
\end{bmatrix}.
\end{equation}
Thus we conclude our example.
\end{ex}
\begin{ex}
Consider the Dynkin diagram given by
\begin{center}
  \includegraphics{img/LieImg.12}
\end{center}
We see that there are 2 vertices, so immediately we know that the
Cartan matrix is $2\times2$ and we can write:
\begin{equation}
A = \begin{bmatrix}
 2 &   \\
   & 2 
\end{bmatrix}.
\end{equation}
The two vertices are connected by 3 edges. There is an arrow
pointing from vertex 1 to vertex 2. This implies that
\begin{equation}
A = \begin{bmatrix}
 2 &-3 \\
-1 & 2 
\end{bmatrix}.
\end{equation}
Observe that if the arrow pointed the other way, we would merely
have the transpose of our matrix.
\end{ex}
\end{framed}
