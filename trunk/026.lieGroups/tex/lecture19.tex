%%
%% lecture19.tex
%% 
%% Made by Alex Nelson
%% Login   <alex@tomato3>
%% 
%% Started on  Sat Dec 11 11:22:33 2010 Alex Nelson
%% Last update Sat Dec 11 12:05:59 2010 Alex Nelson
%%
Consider $\mathfrak{gl}(n)$ where the Cartan subalgebra is simply
the diagonal matrices, which we'll denote by $\diag(\lambda_{1},\dots,\lambda_{n})$
or simply $(\lambda_{1},\dots,\lambda_{n})$. By definition the
weight vectors are merely the canonical basis, denoted by
$\vec{u}_{1}$, \dots, $\vec{u}_{n}$ with the corresponding
weights $\lambda_{1}$, \dots, $\lambda_{n}$. This is the
fundamental representation. Also $-\lambda_{1}$, \dots,
$-\lambda_{n}$ are the weights of the dual fundamental
representation, or covector representation.

The adjoint representation consists of matrices (i.e. a tensor
with 1 upper and 1 lower index):
\begin{equation}
a^{i}_{j}\in V\otimes V^{*}.
\end{equation}
Consider the action of
\begin{equation}
(V\otimes V^{*})\times V\to V
\end{equation}
defined in the obvious way using the map
\begin{equation}
V^{*}\times V\to\Bbb{F}.
\end{equation}
At any rate the adjoint representation is precisely this tensor
product. So we see that $(\lambda_{i}-\lambda_{j})$ are the
weights of the adjoint representation. If it is nonzero we call
them \define{Roots} with the condition that $i\not=j$. Now
$\mathfrak{gl}(n)$ is not simple, so we should talk about
$\mathfrak{sl}(n)$. We have in $\mathfrak{sl}(n)$ the Cartan
subalgebra satisfy
\begin{equation}
\lambda_{1}+\dots+\lambda_{n}=0.
\end{equation}
The roots are (again) the same.

We would like to find simple roots (recall we can decompose roots
into positive and negative, the minimal set of positive roots are
called \define{Simple Roots}). We consider the situation when
$i>j$ and $i<j$, one should be called ``positive'', the other
``negative''. We choose $i>j$ to be negative roots, $i<j$ to be
positive roots. Thus $\lambda_{1}-\lambda_{2}$,
$\lambda_{2}-\lambda_{3}$, \dots, $\lambda_{n-1}-\lambda_{n}$ are
the simple roots of $\mathfrak{sl}(n)=\ClassicalGroup{A}_{n-1}$,
there are $n-1$ simple roots.

Now, for $\SO{2n}$ we have the fundamental representation be
$2n$-dimensional. For the Lie algebra business
$\mathfrak{so}(2n)$ the weights are $\lambda_{1}$,
$-\lambda_{1}$, \dots, $\lambda_{n}$, $-\lambda_{n}$ (weights of
the fundamental vector representation, to be precise). What are
the weights of the adjoint representation? We know the adjoint
representation for $\mathfrak{so}(2n)$ is
\begin{equation}
V\otimes V^{*}=V\otimes V
\end{equation}
since
\begin{equation}
V^{*}=V
\end{equation}
as far as the representations are concerned. The antisymmetric
part preserves the inner product. The weights of the adjoint
representation, we consider $\vec{u}_{i}$ ($i=1,\dots,2n$). When
we take the tensor product we should get
$\vec{u}_{i}\otimes\vec{u}_{j}$, and for $\Antisymmetric^{2}V$
consider
\begin{equation}
\vec{u}_{i}\otimes\vec{u}_{j}-\vec{u}_{j}\otimes\vec{u}_{i}
\quad\text{for }i\not=j.
\end{equation}
The weights should be
$\pm\lambda_{\alpha}\pm\lambda_{\beta}$. The nonzero guys are the
\define{Roots}. We say that $\lambda_{\alpha}+\lambda_{\beta}$
are the positive roots, $-\lambda_{\alpha}-\lambda_{\beta}$ are
the negative roots, and $\lambda_{\alpha}-\lambda_{\beta}$ be
positive iff $\alpha<\beta$. We have $\lambda_{1}-\lambda_{2}$,
\dots, $\lambda_{n-1}-\lambda_{n}$ be the simple roots\dots but
that is not sufficient, we have $n-1$ roots and
$\dim(\mathscr{H})=n$. So we should add $\lambda_{1}+\lambda_{n}$
to our list of simple roots. We can obtain everything by
considering these guys. The Dynkin diagram looks like:
\begin{center}
  \includegraphics{img/LieImg.4}
\end{center}
As previously noted in Lecture 16, the Cartan matrix here is
symmetric. 

The next Lie algebra we will consider is
$\ClassicalGroup{B}_{n}=\mathfrak{so}(2n+1)$. The Cartan
subalgebra is
\begin{equation}
\mathscr{H}=\left\{\;\begin{bmatrix}
1 &                    &        &  \\
  & [\mathfrak{so}(2)] &        &  \\
  &                    & \ddots &  \\
  &                    &        & [\mathfrak{so}(2)]
\end{bmatrix}\;\right\}
\end{equation}
What are the weights of the fundamental vector representation?
They are $0$, $\lambda_{1}$, $-\lambda_{1}$, \dots,
$\lambda_{n}$, $-\lambda_{n}$. The adjoint representation is
again $\Antisymmetric^{2}(V)$ and again we are doing the same
thing. We are adding the weights for $i\not=j$:
$\lambda_{i}+\lambda_{j}$, $\lambda_{i}-\lambda_{j}$,
$-\lambda_{i}-\lambda_{j}$ but we have more, we can take
$0+\lambda_{i}$ and $0-\lambda_{i}$. What are the positive
weights? They are: $\lambda_{i}$, $\lambda_{i}+\lambda_{j}$, and
$\lambda_{i}-\lambda_{j}$ for $i<j$. What are the simple roots?
We take $\lambda_{1}-\lambda_{2}$, \dots,
$\lambda_{n-1}-\lambda_{n}$, and $\lambda_{n}$. These are the $n$
simple roots. But in this situation, the Dynkin diagram is
\begin{center}
  \includegraphics{img/LieImg.6}
\end{center}
Remember the number of lines connecting the nodes $ij$ are
\begin{equation}
n_{ij}=a_{ij}a_{ji}.
\end{equation}
We will get
\begin{equation}
a_{ij}\not=a_{ji}
\end{equation}
the diagonal matrix $\diag(2,2,\dots,2,1)$ symmetrizes the Cartan
matrix. The root vectors $e_{\alpha}$ have positive roots, and
$f_{\alpha}$ has negative roots. We obtain
\begin{equation}
[e_{\alpha},f_{\alpha}]=h_{\alpha}\in\mathscr{H}\quad\mbox{and}\quad [h_{\alpha},e_{\beta}]=a_{\alpha\beta}e_{\beta}
\end{equation}
and so on.

For $\ClassicalGroup{C}_{n}=\mathfrak{sp}(n)$, the weights are
$\lambda_{\alpha}$, $-\lambda_{\alpha}$ and the adjoint
representation has $\lambda_{\alpha}+\lambda_{\beta}$,
$\lambda_{\alpha}-\lambda_{\beta}$,
$-(\lambda_{\alpha}+\lambda_{\beta})$ for all $\alpha$,
$\beta$. The descriptions of the simple roots begins with the
same stuff. The simple roots are $\lambda_{1}-\lambda_{2}$,
\dots, $\lambda_{n-1}-\lambda_{n}$, $2\lambda_{n}$. We are
allowed to add 2 guys, which is how we get $2\lambda_{n}$. The
Dynkin diagram is:
\begin{center}
  \includegraphics{img/LieImg.5}
\end{center}
Observe that this resembles $\ClassicalGroup{B}_{n}$'s Dynkin
diagram, but it is different. The diagonalization here requires
the matrix $\diag(1,\dots,1,2)$.
