%%
%% hamiltonian.tex
%% 
%% Made by Alex Nelson
%% Login   <alex@black-cherry>
%% 
%% Started on  Wed Aug 26 12:36:25 2009 Alex Nelson
%% Last update Wed Aug 26 12:36:25 2009 Alex Nelson
%%

If we want to begin the canonical treatment of fields, we need to
start by finding the canonically conjugate momenta. We need to
first ``foliate'' spacetime into space and time. Why do we do
this? Well, we are concerned about the time derivatives of the
field (how it changes in time), and we need to foliate spacetime
into constant-time surfaces to have this be meaningful in the
obvious way. 

Let $\phi$ be a (column) vector of scalar fields. So
to find the canonically conjugate momenta to  $\phi^{a}$, we
follow our nose to find
\begin{equation}%\label{eq:}
\Pi_{a} = \frac{\partial\mathcal{L}}{\partial(\partial_{0}\phi^{a})}.
\end{equation}
We will write $\Pi$ for the row vector
$\Pi=(\Pi_{1},\Pi_{2},\ldots,\Pi_{n})$. If the mapping
\begin{equation}%\label{eq:}
(\phi,\partial_{0}\phi,\partial_{1}\phi,\partial_{2}\phi,\partial_{3}\phi)\mapsto(\Pi,\phi,\partial_{1}\phi,\partial_{2}\phi,\partial_{3}\phi)
\end{equation}
is invertible (bijective, but when wouldn't it be?\footnote{The
  condition of this being invertible is equivalent to Henneaux
  and Teiteilboim's criteria of ``regularity'' it seems...}),
then we define the Hamiltonian density  function by
\begin{equation}%\label{eq:}
\mathcal{H}(\Pi,\phi,\partial_{1}\phi,\partial_{2}\phi,\partial_{3}\phi)
= \Pi\cdot\partial_{0}\phi-\mathcal{L}
\end{equation}
which is a generalization of the Legendre transformation.

By direct computation we can find Hamilton's equations to be
\begin{equation}%\label{eq:}
\dot{\phi}^{a} =
\frac{\partial\mathcal{H}}{\partial\Pi_{a}},\quad\text{and}\quad\dot{\Pi}_{a}
= -\frac{\partial\mathcal{H}}{\partial\phi^{a}}+\frac{d}{dx^{m}}\frac{\partial\mathcal{H}}{\partial(\partial_{m}\phi^{m})}
\end{equation}
where the equation on the right hand side has an implicit sum
over $m=1,2,3$. These equations naturally folow from Hamilton's
principle applied to
\begin{equation}%\label{eq:}
S = \int(\Pi\cdot\partial_{0}\phi-\mathcal{H})d^{4}x
\end{equation}
the action in canonical form.

\begin{ex}\begin{footnotesize}
(Working in the East coast convention $-+++$) Consider the Lagrangian density
\begin{equation}%\label{eq:}
\mathcal{L} = \frac{1}{2}\partial^{\mu}\phi\partial_{\mu}\phi + \frac{\mu^{2}}{2}\phi^{2}
\end{equation}
where $\mu$ is some ``mass'' scalar. We find the equations of
motion being
\begin{equation}%\label{eq:}
\frac{\delta S}{\delta \phi}=0
\;\Rightarrow\;
\frac{\partial\mathcal{L}}{\partial\phi}
-\partial_{\mu}\frac{\partial\mathcal{L}}{\partial(\partial_{\mu}\phi)}
= 0.
\end{equation}
We find by direct computation
\begin{subequations}
\begin{align}
\frac{\partial\mathcal{L}}{\partial\phi} &= +\mu^{2}\phi\\
\partial_{\mu}\frac{\partial\mathcal{L}}{\partial(\partial_{\mu}\phi)}
&= \partial_{\mu}\left(\partial^{\mu}\phi\right)=\partial^{2}\phi 
\end{align}
\end{subequations}
which yields the equations of motion to be
\begin{equation}%\label{eq:}
\left(\partial^{\mu}\partial_{\mu}+\mu^{2}\right)\phi = 0.
\end{equation}
This is precisely the Klein-Gordon equation!

We find that the canonically conjugate momenta is
\begin{equation}%\label{eq:}
\Pi = \frac{\partial\mathcal{L}}{\partial(\partial_{0}\phi)} =
\partial^{0}\phi = -\partial_{0}\phi.
\end{equation}
Thus we can find the Hamiltonian density to be
\begin{subequations}
\begin{align}
\mathcal{H} &= \Pi\cdot\partial_{0}\phi-\mathcal{L}\\
&= -\partial^{0}\phi\partial_{0}\phi -
\left(\frac{1}{2}\partial^{\mu}\phi\partial_{\mu}\phi+\frac{\mu^{2}}{2}\phi^{2}\right) \\
&=\frac{-1}{2}\partial^{0}\phi\partial_{0}\phi-\frac{1}{2}\vec{\nabla}\phi\cdot\vec{\nabla}\phi-\frac{\mu^{2}}{2}\phi^{2}\\
&=-\frac{1}{2}\left(\Pi\cdot\Pi+\vec{\nabla}\phi\cdot\vec{\nabla}\phi+\mu^{2}\phi^{2}\right)
\end{align}
\end{subequations}
Note our convention makes the sign of the $\mu^{2}$ term
different than what is conventionally used in most particle
physics texts. Because of this choice of signature convention,
our energy is \emph{negative}.

We can find Hamilton's equations for the Klein Gordon scalar
field quite easily. One we already found accidentally, it is
\begin{equation}%\label{eq:}
\Pi = \frac{\partial\mathcal{L}}{\partial(\partial_{0}\phi)} =
\partial^{0}\phi = -\partial_{0}\phi.
\end{equation}
The other is more interesting, it is
\begin{subequations}
\begin{align}
\dot{\Pi} &= -\frac{\partial\mathcal{H}}{\partial\phi}+\frac{d}{dx^{m}}\frac{\partial\mathcal{H}}{\partial(\partial_{m}\phi)}\\
&= -\mu^{2}\phi - \partial_{m}(\partial^{m}\phi)\\
&= -(\mu^{2}+\nabla^{2})\phi
\end{align}
\end{subequations}
Note that the sign of our results differs from what we derived
due to our choice of signature convention!
\end{footnotesize}\end{ex}

Now the Hamiltonian approach to fields, as previously mentioned,
is slightly diffferent because we work with a one-parameter
family of spatial hypersurfaces instead of spacetime. To consider
the scalar field $\phi(x)$ in this setting, we let it become
$\phi(\bar{x};t)$ where we use the semicolon to remind ourselves
that we are working with ``some parameter'' $t$. We obtain the
Hamiltonian simply by integrating the density over all space
\begin{equation}%\label{eq:}
H(\Pi,\phi) = \int_{\text{all space}}\mathcal{H}(\Pi,\phi,\partial_{1}\phi,\partial_{2}\phi,\partial_{3}\phi)d^{3}\bar{x}.
\end{equation}
We need to consider how the functional derivative behaves under
this change from spacetime to ``space plus time''. The functional
derivative with respect to $\phi(\bar{x};t)$ is
\begin{equation}%\label{eq:}
\frac{\delta A[\phi(\bar{y};t)]}{\delta\phi(\bar{x};t)} = \lim_{\varepsilon\to0}\frac{1}{\varepsilon}\left(A[\phi_{t}+\delta^{(3)}_{\bar{x}}]-A[\phi_{t}]\right)
\end{equation}
where $\phi_{t}(\bar{y})=\phi(\bar{y};t)=\phi_{t}$, and $\delta^{(3)}_{\bar{x}}=\delta^{(3)}(\bar{x}-\bar{y})$.

We obtain the equations of motion from the principle of
stationary action applied to the canonical form of the
action. The Euler-Lagrange equations encoded in the canonical
formalism are
\begin{equation}%\label{eq:}
\dot{\phi}^{a} =
\frac{\partial\mathcal{H}}{\partial\Pi_{a}},\quad\text{and}\quad\dot{\Pi}_{a}
= -\frac{\partial\mathcal{H}}{\partial\phi^{a}}+\frac{d}{dx^{m}}\frac{\partial\mathcal{H}}{\partial(\partial_{m}\phi^{m})}
\end{equation}
as we previously mentioned.

We can directly compute that
\begin{subequations}
\begin{align}
\frac{\delta H}{\delta\phi(\bar{x};t)} &= \int \frac{\delta\mathcal{H}}{\delta\phi(\bar{x};t)}d^{4}y\\
&= \lim_{\varepsilon\to0}\frac{1}{\varepsilon}\int\left[\mathcal{H}(\Pi,\phi_{t}+\varepsilon\delta^{(3)}_{\bar{x}},\partial_{k}\phi_{t}+\varepsilon\partial_{k}\delta^{(3)}_{\bar{x}})-\mathcal{H}(\Pi,\phi_{t},\partial_{m}\phi_{t})\right]d^{3}\bar{y}\\
&=\lim_{\varepsilon\to0}\frac{1}{\varepsilon}\int\left[\mathcal{H}(\Pi,\phi_{t},\partial_{m}\phi_{t})+\varepsilon\frac{\partial\mathcal{H}}{\partial\phi}\delta_{\bar{x}}^{(3)}+\varepsilon\frac{\partial\mathcal{H}}{\partial(\partial_{m}\phi)}\partial_{m}\delta_{\bar{x}}^{(3)}\right.\\\nonumber
&\left.\phantom{\lim_{\varepsilon\to0}\frac{1}{\varepsilon}\int\qquad}
+\mathcal{O}(\varepsilon^{2})-\mathcal{H}(\Pi,\phi_{t},\partial_{m}\phi_{t})\right]d^{3}\bar{y}\\
&=\int\left[\frac{\partial\mathcal{H}}{\partial\phi}\delta^{(3)}_{\bar{x}}+\frac{\partial\mathcal{H}}{\partial(\partial_{m}\phi)}\partial_{m}\delta^{(3)}_{\bar{x}}\right]d^{3}\bar{y}\\
&=\int\left[\frac{\partial\mathcal{H}}{\partial\phi}-\partial_{m}\frac{\partial\mathcal{H}}{\partial(\partial_{m}\phi)}\right]\delta^{(3)}_{\bar{x}}d^{3}\bar{y}\\
&=-\int\left[-\frac{\partial\mathcal{H}}{\partial\phi}+\partial_{m}\frac{\partial\mathcal{H}}{\partial(\partial_{m}\phi)}\right]\delta^{(3)}_{\bar{x}}d^{3}\bar{y}\\
&=-\int\dot{\Pi}\delta^{(3)}_{\bar{x}}d^{3}\bar{y}\\
&=-\dot{\Pi}(x).
\end{align}
\end{subequations}
This is precisely one of Hamilton's equations, and by similar
reasoning we find
\begin{equation}%\label{eq:}
\dot{\phi} = \frac{\delta
  H}{\delta\Pi_{t}(\bar{x})},\quad\text{and}\quad\dot{\Pi}_{t}(\bar{x})=-\frac{\delta H}{\delta\phi_{t}(\bar{x})}
\end{equation}
are both generalizations of Hamilton's equations to field
theoretic setting. This motivates us to define an analogous
Poisson bracket:
\begin{equation}%\label{eq:}
\{A,B\}\eqdef\int\left(\frac{\delta A}{\delta\phi_{t}(\bar{x})}\frac{\delta B}{\delta\Pi_{t}(\bar{x})}-\frac{\delta B}{\delta\phi_{t}(\bar{x})}\frac{\delta A}{\delta\Pi_{t}(\bar{x})}\right)d^{3}\bar{x}.
\end{equation}
This allows us to recover the equations of motion in a familiar
way
\begin{equation}%\label{eq:}
\{A,H\} = \int\left(\frac{\delta A}{\delta\phi_{t}(\bar{x})}\dot{\phi}_{t}(\bar{x})+\dot{\Pi}_{t}(\bar{x})\frac{\delta A}{\delta\Pi_{t}(\bar{x})}\right)d^{3}\bar{x}.
\end{equation}
Similarly, the canonical commutation relations hold classically
as
\begin{equation}%\label{eq:}
\{\phi^{a}(\bar{x};t),\Pi_{b}(\bar{y};t)\} = {\delta^{a}}_{b}\delta^{(3)}(\bar{x}-\bar{y}).
\end{equation}
Note that this only makes sense if the two variables are
considered on the same time slice. If we were to consider two
different timeslices, we'd necessarily have the commutation
relations vanish due to locality constraints.
