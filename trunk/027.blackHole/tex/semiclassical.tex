%%
%% semiclassical.tex
%% 
%% Made by Alex Nelson
%% Login   <alex@tomato3>
%% 
%% Started on  Sat Feb 20 11:42:25 2010 Alex Nelson
%% Last update Tue Mar 30 17:33:01 2010 Alex Nelson
%%

We have a box of gas with linear size $L$, mass $m$ and
temperature $T$. The black hole has mass $M$ and a horizon radius
$R=2GM$, thus it has a horizon area $A=16\pi G^{2}M^{2}$. The box
of gas will merge with the black hole when its proper distance
$\rho$ from the horizon is of order $L$, at which point the
disappearance of the gas will lead to a loss of entropy
\equation\label{eq:approximationOfEntropyLoss}
\Delta S\sim -m/T.
\endequation
Presumably this is from the relation
\equation
dE=\delta Q-\delta W
\endequation
where $dE$ is the infitesimal change in internal energy, $\delta
Q$ is the inexact differential describing the infinitesimal
change in heat, and $\delta W$ is the inexact differential in
work. For a {\it reversible process} we have
\equation\label{eq:secondLawThermodynamicsReversibleProcess}
dS=\frac{\delta Q}{T}=\frac{dE+\delta W}{T}\approx\frac{c^{2}dm}{T}
\endequation
where $dS$ is the differential change in entropy.

\problem
Is dropping a box of gas into a black hole a reversible process?
Because that's what is used in \ref{eq:approximationOfEntropyLoss}
derived via \ref{eq:secondLawThermodynamicsReversibleProcess}. But
if the process is not reversible, which intuitively it isn't,
there's not reason why \ref{eq:secondLawThermodynamicsReversibleProcess}
should be used.
\endproblem
\solution
It turns out that we can use \ref{eq:secondLawThermodynamicsReversibleProcess}
as a lower bound for non-reversible processes,
i.e. non-reversible processes have some {\it greater} change in entropy.
\endsolution


For a Schwarzschild Black Hole, the proper distance is $\rho=\int
\sqrt{g_{rr}}dr$. Why? Well, we can change coordinates to where
$\theta$ is constant, $\varphi$ is constant too, and work with a
constant time-slice. Thus the only surviving part of the
differential proper distance is
\equation
ds^{2}=g_{\mu\nu}dx^{\mu}dx^{\nu}=g_{rr}dr^{2}.
\endequation
Thus the proper distance from the horizon is
\equation
\rho=\int^{R+\delta r}_{R}\frac{dr}{\sqrt{1-2GM/r}}
=\left.r \sqrt{1-\frac{2GM}{r}}+ 2GM \ln\left(2 r
    \left(\sqrt{1-\frac{2GM}{r}}+1\right)-2GM\right)\right|^{r=R+\delta r}_{r=R}
\endequation
By imposing the conditions we have
\equation
\rho=\sqrt{\delta r (\delta r+2GM)}+2GM\left(\ln\left(\sqrt{\frac{\delta r}{\delta r+2GM}}+1\right)+\tanh^{-1}\left(\frac{\delta r}{\delta r+4GM}\right)\right)
\endequation
We can work with the highest order term, and for ``small $\delta
r$'' we get to first order in $\delta r$
\equation
\rho\sim\sqrt{2GM\delta r}.
\endequation
Well, already I am hestitant on several levels.

\problem
It seems that we are using classical thermodynamics. However, in
semiclassical gravity, energy is not conserved. That is, the
first Law no longer holds. Can we really rely on the second Law
holding in {\sc\ref{eq:approximationOfEntropyLoss}} or is there a way
to have it be rigorous?
\endproblem

At any rate, we say that $\rho\sim L$ when $\delta r\sim
L^{2}/GM$. This bothers me, it's unrigorous, so let's try to find
it more exactly. We find that really, the exact solution is
\equation
\delta r = \frac{1}{2}GM\left(\cosh \left(\frac{L}{GM}\right)-\sinh \left(\frac{L}{GM}\right)\right) \left(\sinh\left(\frac{L}{GM}\right)+\cosh \left(\frac{L}{GM}\right)-1\right)^2
\endequation
which is to first order
\equation
\delta r \approx \frac{L^{2}}{2GM}+\hbox{higher order terms}
\endequation
precisely as specified. The gas initially has mass $m$ but (as
seen by an observer from infinity) it is redshifted as the box
falls to the black hole. When $r=2GM+\delta r$, the black hole
gains a mass
\equation
\Delta M\approx m\sqrt{1-\frac{2GM}{2GM+\delta r}}=m\sqrt{\tanh(L^{2}/R^{2})}.
\endequation
Recall that the Thermal de Broglie relations give the wavelength 
\equation\label{eq:thermalDeBroglieRelation}
\Lambda = \frac{\hbar\sqrt{2\pi}}{\sqrt{mkT}}
\endequation
where $k$ is Boltzmann's constant, and $\hbar$ is Planck's
constant over $2\pi$. If we write
\equation
\Lambda=L\quad\iff\quad L\sim\frac{\hbar}{\sqrt{mT}}
\endequation
then we don't obtain the same results that Carlip gets...which is
quite bad. However, if we use $T$ for $E$ in the usual de Broglie
relations, we do get the same results. 

\problem\label{problem:ambiguityFromThermalDeBroglieRelations}
Can we really avoid the implications from \ref{eq:thermalDeBroglieRelation}
or does it doom the calculation?
\endproblem

Unfortunately due to our rigorous adherence to precision, we
cannot easily continue calculations to verify their
correctness. We are condemned to make an approximation. We can
Taylor expand
\equation
\sqrt{\tanh(x^{2})}=x-\frac{x^{5}}{6}+\frac{19 x^{9}}{360}-\frac{55
    x^{13}}{3024}+{\cal O}(x^{17}).
\endequation
Being physicists, we take the first term and plug it in for
$\Delta M$
\equation
\Delta M \approx \frac{mL}{2GM}\quad\Rightarrow\quad 2GM\Delta
M\approx mL.
\endequation
Thus we deduce from our temperature relations
\equation
L\sim\frac{\hbar}{\sqrt{mT}}\quad\Rightarrow\quad T=\frac{\hbar^{2}}{L^{2}m}
\endequation
and this is plugged into the entropy relation we began with
\equation
\Delta S \sim \frac{-m}{T} = \frac{-m^{2}L^{2}}{\hbar^{2}}.
\endequation
Thus we obtain
\equation
\Delta S \sim \frac{-(2GM\Delta M)^{2}}{\hbar^{2}}.
\endequation
Here is where things go horribly awry! We approximate
\equation
\Delta A\approx G^{2}M\Delta M
\endequation
so setting $G=1$, we get more or less
\equation
\Delta S\sim\frac{-(\Delta A)^{2}}{\hbar^{2}}
\endequation
which is horribly wrong!
So to answer \ref{problem:ambiguityFromThermalDeBroglieRelations}
we can definitively say {\bf NO, IT DOOMS THE CALCULATION!}  

Now, what if we try something different? Work with a single
quantum particle in a box, then the temperature {\it is precisely}
the kinetic energy. We know for the non-relativistic quantum
particle in a box the energy is
\equation
E_n = \frac{n^2\hbar^2 \pi^2}{2mL^2}
\endequation
where $n\in\Bbb{N}$, $m$ is the mass of the particle, and $L$ is
the length of the box. The momentum would be
\equation
p_{n}=\hbar\frac{n\pi}{L}
\endequation
which permits us to deduce the value of 
\equation
\lambda=n/2L.
\endequation
If we plug this value in, well, it still doesn't work.

\solution
We cannot have $\lambda\propto1/\sqrt{m}$, otherwise we end up
with $\Delta S\propto (\Delta A)^{2}$.
\endsolution
