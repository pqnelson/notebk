%%
%% continuity.tex
%% 
%% Made by Alex Nelson
%% Login   <alex@tomato>
%% 
%% Started on  Thu May 28 20:16:21 2009 Alex Nelson
%% Last update Thu May 28 20:16:21 2009 Alex Nelson
%%


\begin{defn} 
Let $f:D\to\mathbb{R}$ be a real valued function whose domain is a subset of
$D\subset\mathbb{R}$. Then $f$ is said to be
\textbf{continuous at}\index{Continuous!function of a single variable} 
$x_{0}\in D$ iff for each $\varepsilon>0$ there is a
corresponding $\delta>0$ such that
\begin{equation}
|x-x_0|<\delta\;\text{ implies }\;|f(x)-f(x_0)|<\varepsilon.
\end{equation}
\end{defn}
\begin{rmk}
The intuition behind this definition is if we ``wiggle'' around
in the range by some amount less than $\varepsilon$ about $f(x_0)$, then
we should ``wiggle'' about in the domain by some amount less than
$\delta$. If we end up have $\delta$ be too large as
$\varepsilon$ gets too small, there is a discontinuity (a jump
permitting a lot of wiggling). We can see a diagram illustrating
this point in fig \eqref{img1}. Observe that in the diagram, as
we make $|f(x)-f(x_{0})|$ decrease, $|x-x_{0}|$ decreases
faster. In general, we just want a relationship expressing some positive
$\delta$ in terms of a given $\varepsilon>0$. This relationship
should be injective(?) and such that as $\varepsilon$ decreases,
$\delta$ decreases too. So we can have an arbitrarily small
$\varepsilon$, and $\delta$ goes to zero as well. This is
fundamentally all a limit really is.
\end{rmk}
\begin{figure}
\includegraphics{img.1}
\caption{An example of the notion of continuity. The yellow
  region is the $\delta$ neighborhood of $x_{0}$, the blue region
  is the $\varepsilon$ neighborhood of $f(x_{0})$.}\label{img1}
\end{figure}

There is an alternate definition using sequences which is less
intuitive.

\begin{defn}
Let $f$ be a real function, we say $f$ is \text{continuous at
  $x_{0}$} if for every sequence ($x_{n}$) that converges to
$x_{0}$, then
\begin{equation}
\lim_{n\to\infty}f(x_n)=f(x_{0}).
\end{equation}
(We assume that the sequence all belongs to the domain of $f$.)
\end{defn}
\begin{rmk}
This can be worded slightly differently as ``\emph{Continuity
  preserves convergence of sequences.}''
\end{rmk}
\begin{ex} We will show that $f(x)=2x^{2}+1$ is continuous. So, we
  want to write $\delta$ in terms of $\varepsilon$. This will allow
  us to state ``For each $\varepsilon$, there is a $\delta$ such
  that...'' which implies continuity. So, let $x_0$ be some
  arbitrary real number, we have
\begin{equation}
|f(x)-f(x_{0})| = |\left[2x^2+1\right]-\left[2x_{0}^{2}+1\right]|
= 2|x^2-x_{0}^{2}|<\varepsilon
\end{equation}
We can rewrite this as
\begin{equation}
2|x^{2}-x_{0}^{2}|\leq 2|x-x_{0}||x+x_{0}|<2\delta|x+x_{0}|
\end{equation}
We also have a bound that $|x|<|x_{0}|+\delta$. We plug this in,
we set the result equal to $\varepsilon$ so
\begin{equation}
2\delta|x+x_{0}|\leq2\delta(|x|+|x_0|)<2\delta(2|x_{0}|+\delta)=\varepsilon.
\end{equation}
We rearrange terms to find
\begin{equation}
\delta^{2}+2|x_{0}|\delta - \frac{1}{2}\varepsilon = 0.
\end{equation}
As we learned from primary school, this is a quadratic equation
which has precisely two solutions, we choose the positive one:
\begin{equation}
\delta = \sqrt{x_{0}^{2}+\frac{1}{2}\varepsilon}-|x_{0}|.
\end{equation}
This is always greater than zero provided $\varepsilon>0$,
$x_{0}\in\mathbb{R}$. Thus $f$ is continuous. QEF.
\end{ex}
\begin{ex}
Consider 
\begin{equation}
g(x) = \begin{cases} x^{2}\sin\left(\frac{1}{x}\right), & x\neq0\\
0, & x=0.\end{cases}
\end{equation}
We can ``see'' that when $x\neq0$, $g$ is continuous. At $x=0$,
we need to prove it's continuous. So, we see that for each
$\varepsilon>0$ there is a corresponding $\delta>0$ such that
\begin{equation}
|x-0|<\delta\;\Rightarrow\;|g(x)-0|<\varepsilon.
\end{equation}
We can rewrite the relations to be (since $\sin(x)\leq1$)
\begin{equation}
|g(x)|\leq x^2 = (x)^2 < \delta^2 = \varepsilon.
\end{equation}
So $\delta=\sqrt{\varepsilon}$ is always positive and
nonzero. Thus $g(x)$ is continuous at $x=0$. QEF.
\end{ex}
\begin{thm}
Let 
\begin{equation}
f:D\to\mathbb{R}
\end{equation}
where $D\subseteq\mathbb{R}$. If $f$ is continuous at $x_{0}\in
D$, then $|f|$ and $kf$ are continuous at $x_{0}$ (where
$k\in\mathbb{R}$ is arbitrary).
\end{thm}
\begin{proof}
Since $f$ is continuous, then for each $\varepsilon>0$ there is a
corresponding $\delta>0$ such that
\begin{equation}
|x-x_{0}|<\delta\;\Rightarrow\;|f(x)-f(x_{0})|<\varepsilon.
\end{equation}
We see that
\begin{equation}
|kf(x)-kf(x_{0})|=|k||f(x)-f(x_{0})|<|k|\varepsilon
\end{equation}
so we choose $\varepsilon'=|k|\varepsilon$ for nonzero $k$. We
see then that for each $\varepsilon'>0$ there is a corresponding
$\delta>0$ such that
\begin{equation}
|x-x_{0}|<\delta\;\Rightarrow\;|kf(x)-kf(x_{0})|<\varepsilon'
\end{equation}
which is necessarily true if $f$ is continuous.

Observe that, for some $a,b\in\mathbb{R}$,
\begin{equation}
|b|\leq|a|+|b-a|.
\end{equation}
Let $b=f(x)$, $a=f(x_{0})$, then
\begin{equation}
|f(x)|<|f(x_{0})|+\varepsilon
\end{equation}
Similarly
\begin{equation}
|a|\leq|b|+|a-b|\Rightarrow|f(x_{0})|<|f(x)|+\varepsilon
\end{equation}
We have
\begin{equation}
-\varepsilon<|f(x)|-|f(x_{0}|<\varepsilon
\end{equation}
which implies
\begin{equation}
||f(x)|-|f(x_{0})||<\varepsilon.
\end{equation}
This implies continuity.
\end{proof}
\begin{thm}
Let $f,g$ be real functions continuous at $x_{0}$. Then $f+g$,
$fg$ and (if $g(x_{0})\neq0$) $f/g$ are continuous at $x_{0}$.
\end{thm}
\begin{proof}
Let $\varepsilon_{1},\delta_{1}$ be for $f$,
$\varepsilon_{2},\delta_{2}$ be for $g$. So for each
$\varepsilon_{1}>0$ and $\varepsilon_{2}>0$ there is a
corresponding $\delta_{1}>0$ and $\delta_{2}>0$ such that
\begin{equation}
|x-x_{0}|<\delta_{1}\Rightarrow|f(x)-f(x_{0})|<\varepsilon_{1}
\end{equation}
and
\begin{equation}
|x-x_{0}|<\delta_{2}\Rightarrow|g(x)-g(x_{0})|<\varepsilon_{2}.
\end{equation}
Since $\varepsilon_{i}$ ($i=1,2$) is arbitrary, we will impose the condition that $\varepsilon_{i}<1$.
We see that by the triangle inequality
\begin{equation}
|(f(x)+g(x))-(f(x_{0})+g(x_{0}))|\leq|f(x)-f(x_{0})|+|g(x)-g(x_{0})|<\varepsilon_{1}+\varepsilon_{2}
\end{equation}
We can also see that
\begin{equation}
|f(x)g(x)-f(x_{0})g(x_{0})|<|f(x_{0})|\varepsilon_{2} + |g(x)|
\end{equation}
\end{proof}
