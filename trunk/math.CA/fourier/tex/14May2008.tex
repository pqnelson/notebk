%%
%% 14May2008.tex
%% 
%% Made by Alex Nelson
%% Login   <alex@tomato>
%% 
%% Started on  Fri Dec 26 17:02:27 2008 Alex Nelson
%% Last update Fri Dec 26 17:02:27 2008 Alex Nelson
%%
%% Fourier transforms

Let us consider a few examples computing the Fourier
transform of a function. Remember the Fourier transform is
\begin{equation}
\widehat{f}(\xi) = \int f(x)e^{-i\xi x}dx.
\end{equation}

\begin{ex}
Let
\begin{equation}
\chi_a(x) = \begin{cases} 1 & \text{if }|x|<a\\
0 & \text{otherwise}\end{cases}
\end{equation}
be a characteristic function. Then
\begin{subequations}
\begin{align}
\widehat{\chi}_{a}(\xi) &= \int^{a}_{-a}\xi_{a}(x)e^{-i\xi
  x}dx\\
&= \int^{a}_{-a}e^{-i\xi x}dx\\
&= \frac{i}{\xi}(e^{ia\xi}-e^{-ia\xi})\\
&= -2\frac{\sin(a\xi)}{\xi}.
\end{align}
\end{subequations}
\end{ex}
\begin{ex}
Consider
\begin{equation}
\mathcal{F}\left[e^{-ax^{2}/2}\right] = \sqrt{\frac{2\pi}{a}}e^{-\xi^{2}/2a}
\end{equation}
where $\mathcal{F}[\cdot]$ is the fourier transform. We can
see that
\begin{subequations}
\begin{align}
\mathcal{F}\left[e^{-ax^{2}/2}\right] &= \int
e^{-ax^{2}/2-i\xi x}dx\\
&=\int e^{-y^2+(\xi^{2}/a)}\left(\frac{2}{\sqrt{a}}\right)dy \quad\text{where } y=x\sqrt{a/2}+i\xi/\sqrt{2a}\\
&=\sqrt{\frac{2}{a}}e^{-\xi^{2}/2a}\int e^{-y^2}dy\\
&=\sqrt{\frac{2\pi}{a}}e^{-\xi^{2}/2a}.
\end{align}
\end{subequations}
Since we know that 
\begin{equation}
\int e^{-x^2}dx = \sqrt{\pi}.
\end{equation}
\end{ex}

Now let us consider some of the basic properties of the
Fourier transform.
\begin{enumerate}
\item{(Shifting the Fourier Transform\index{Fourier Transform!Shift})} For $a\in\mathbb{R}$, 
\begin{align*}
\mathcal{F}\left[f(x-a)\right] &=
e^{-ia\xi}\mathcal{F}[f(x)]\\
&= e^{-a\xi}\widehat{f}(\xi)
\end{align*}
\item{(Dilation\index{Fourier Transform!Dilation})} For $\delta>0$, $f_{\delta}(x) = f(x/\delta)/\delta$,
  then $\mathcal{F}(f_{\delta}) =
  \widehat{f}(\delta\xi)$. If $\delta>1$, we're shrinking
  the width of $f$, but expanding the width of
  $\widehat{f}$. We also have
\begin{equation}
\mathcal{F}\left[f(\delta x)\right] =
\widehat{f}_{\delta}(\xi)
\end{equation}
\item\marginpar{Most important property of Fourier Transform!} If
$f$ is continuous, its first derivative $f'(x)\in
  PC(\mathbb{R})$ is piecewise continuous, and $f\in L^{1}$,
  then
\begin{equation}
\mathcal{F}\left[\frac{d}{dx}f(x)\right](\xi) = i\xi\widehat{f}(\xi).
\end{equation}
This should be familiar, recall that fo the Fourier series
we have $f$ has coefficients $c_n$ and $f'$ has coefficients
$inc_n$. On the other hand, if $xf(x)\in L^{1}$, then
\begin{equation}
\mathcal{F}\left[xf(x)\right](\xi) = i\frac{d}{d\xi}\widehat{f}(\xi).
\end{equation}
\item If both $f,g\in L^{1}$, then
\begin{equation}
\mathcal{F}[f*g](\xi) = \widehat{f}(\xi)\widehat{g}(\xi)
\end{equation}
\end{enumerate}
The last property is the most important as it lets us change
a given differential equation into an algebraic equation.
\begin{proof}
\begin{enumerate}
\item{(Shift)} We see that
\begin{align*}
\mathcal{F}[f(x-a)] &= \int f(x-a)e^{-ix\xi}dx \\
&= \int f(y)e^{-i\xi(y+a)}dy\quad\text{where }y=x-a\\
&= e^{-ia\xi}\int f(y)e^{-i\xi y}dy\\
&= e^{-a\xi}\widehat{f}(\xi).
\end{align*}
\item{(Dilation)} Again by direct computation we see that
\begin{align*}
\mathcal{F}[f_{\delta}](\xi) &= \int \frac{1}{\delta}f\left(\frac{x}{\delta}\right)e^{-ix\xi}dx\\
&=\int f(y)e^{-i\delta y\xi}dy,\quad\text{with }y=x/\delta\\
&=\int f(y)e^{-iy(\delta\xi)}dy\\
&=\mathcal{F}[f(x)](\delta\xi) = \widehat{f}(\delta\xi)
\end{align*}
\item{(Differentiation)} Observe, once more by direct
  computation
\begin{align*}
\mathcal{F}\left[\frac{d}{dx}f(x)\right] &= \int
f'(x)e^{-ix\xi}dx\\
&=f(x)e^{-ix\xi}|^{\infty}_{-\infty} - \int
f(x)\frac{d}{dx}e^{-i\xi x}dx 
\end{align*}
where we have just done integration by parts in the second
step. We see that since $f\in L^1$ that $f(x)\to 0$ as
$x\to\pm\infty$. So
\begin{align*}
f(x)e^{-ix\xi}|^{\infty}_{-\infty} - \int
f(x)\frac{d}{dx}e^{-i\xi x}dx  &= -\int
f(x)\frac{d}{dx}e^{-i\xi x}dx\\
&= \int f(x)i\xi e^{-ix\xi}dx\\
&= i\xi \int f(x)e^{-ix\xi}dx\\
&= i\xi \widehat{f}(\xi)
\end{align*}
Similarly, we see that 
\begin{equation}
xe^{-ix\xi} = i\frac{d}{d\xi}e^{-ix\xi}
\end{equation}
so
\begin{align*}
\mathcal{F}[xf(x)] &= \int xf(x)e^{-ix\xi}dx\\
&= \int \left(i\frac{d}{d\xi}e^{-i\xi x}\right)f(x)dx\\
&= i\frac{d}{d\xi}\int f(x)e^{-ix\xi}dx\\
&= i\frac{d}{d\xi}\widehat{f}(\xi)
\end{align*}
\item If 
\begin{align*}
\mathcal{F}[f*g](\xi) &= \int\left(\int f(x-y)g(y)dy\right)e^{-ix\xi}dx\\
&= \int\int f(x-y)g(y)e^{-ix\xi}dxdy
\end{align*}
Let $z=x-y$, then
\begin{align*}
\int\int f(x-y)g(y)e^{-ix\xi}dxdy&=\int\int f(z)e^{-i\xi(z+y)}dz g(y)dy\\
&=\int\left(\int f(z)e^{-i\xi(z+y)}dz\right)g(y)dy\\
&=\int\left(\int f(z)e^{-i\xi z}e^{-iy\xi}dz\right)g(y)dy\\
&=\int\left(\int f(z)e^{-iz\xi}dz\right)g(y)e^{-iy\xi}dy\\
&=\int\widehat{f}(\xi)e^{-iy\xi}g(y)dy\\
&=\widehat{f}(\xi)\int g(y)e^{-iy\xi}dy\\
&=\widehat{f}(\xi)\widehat{g}(\xi)
\end{align*}
which concludes our proof.
\end{enumerate}
\end{proof}

\begin{riemleb}
If $f\in L^{1}$, then $\mathcal{F}[f](\xi)\to0$ as $\xi\to\pm\infty$.
\end{riemleb}
\begin{sketch}
Intuitively we see
\begin{align*}
\widehat{f}(\xi) &= \int e^{-i\xi x}f(x)dx\\
&= \int e^{-iz}f(z)\frac{dz}{\xi}\\
|\widehat{f}(\xi)| &\leq
\int\left|f\left(\frac{z}{\xi}\right)\right|\frac{dz}{|\xi|}
= \frac{1}{|\xi|}\int\left|f\left(\frac{z}{\xi}\right)\right|dz\\
&\leq \frac{1}{|\xi|}k
\end{align*}
where $k$ is a constant, since $f\in L^{1}$. So
\begin{equation}
\lim_{\xi\to\pm\infty}f(\xi)\leq\lim_{\xi\to\pm\infty}\frac{1}{|\xi|}k
= 0.
\end{equation}
This concludes our sketch of the proof.
\end{sketch}
%%%%%%%%
%%%%%%%%
%%%% This is supposed to be part of 16 May 2008
%%%% 
%%%%%%%%
%%%%%%%%
What's the implication of the Riemann-Lebesgue lemma? Well,
in addition to the property (3) of the Fourier transform, it
implies the following:
\begin{quote}
If $f$ is smooth, $\widehat{f}$ decays quickly. If
$\widehat{f}$ decays fast, then $f$ is smooth.
\end{quote}
\begin{ex}
If $f\in C^{(k-1)}$, $f^{(k)}\in PC(\mathbb{R})$ and
$f^{(k)}\in L^{1}$, then we have
\begin{equation}
\mathcal{F}[f^{(k)}] = (i\xi)^{k}\widehat{f}(\xi).
\end{equation}
The Riemann-Lebesgue lemma says
$(i\xi)^{k}\widehat{f}(\xi)\to -$ as $|\xi|\to\infty$.
\end{ex}
