%%
%% 5May2008.tex
%% 
%% Made by Alex Nelson
%% Login   <alex@tomato>
%% 
%% Started on  Sun Dec 21 19:01:05 2008 Alex Nelson
%% Last update Sun Dec 21 19:01:05 2008 Alex Nelson
%%
%% We will review the function space $L^{2}$ and $L^{1}$,
%% and we will introduce the Fourier transform

Note that for short hand usage we will write
$L^{2}=L^{2}(-\infty,\infty)$. More generally, we have
\begin{equation}
L^{p} = \{f\text{ defined on }\mathbb{R}:
\int^{\infty}_{-\infty}|f(x)|^pdx\}
\end{equation}
where $p\in\mathbb{N}$. In a very special case, when $p=2$,
we have a Hilbert space with an inner product defined by
\begin{equation}
\<f,g\> = \int^{\infty}_{-\infty}f(x)\overline{g(x)}dx
\end{equation}
and an induced norm
\begin{equation}
\|f\| = \sqrt{\<f,f\>}
\end{equation}
All of the fundamental properties of the inner product and
the norm still hold for arbitrary end points, so they still
hold here.

The Cauchy-Schwarz inequality still holds
\begin{equation}
|\<f,g\>|\leq\|f\|\|g\|
\end{equation}
or equivalently
\begin{equation}
|\int f(x)\overline{g(x)}dx|\leq\left(\int|f(x)|^2dx\right)^{1/2}\left(\int|g(x)|^2dx\right)^{1/2}
\end{equation}
This is not as useful as
\begin{equation}
\<|f|,|g|\>=\int|f(x)g(x)|dx\leq\|f\|\|g\|
\end{equation}
which some (e.g~\cite{textbook}) refer to as the Cauchy-Schwarz.

Now, if we examine $L^{1}$,\index{$L^{1}$} it doesn't really have an inner
product, but it has a norm defined on it
\begin{equation}
\|f\|_{1} = \int|f(x)|dx.
\end{equation}

\begin{rmk}
Observe that $L^{2}(a,b)\subset L^{1}(a,b)$ for intervals,
but this is not true for $L^{2}$ and $L^{1}$. There are
$f\in L^2$ but $f\notin L^{1}$, and similarly there are
$g\in L^{1}$ but $g\notin L^{2}$.
\end{rmk}

\begin{ex}
Let
\begin{equation}
f(x) = \begin{cases}x^{-2/3}&\text{when }x>1\\
0&\text{otherwise}\end{cases}
\end{equation}
Now
\begin{subequations}
\begin{align}
\int^{\infty}_{1}|f(x)|dx &= \int^{\infty}_{1}x^{-2/3}dx\\
&= \lim_{b\to\infty}3x^{1/3}|^{b}_{1}\\
&= 3(\lim_{b\to\infty}b^{1/3}-1)=\infty
\end{align}
\end{subequations}
which means that $f\notin L^{1}$. But
\begin{subequations}
\begin{align}
\int^{\infty}_{1}|f(x)|^{2}dx &=
\int^{\infty}_{1}x^{-4/3}dx\\
&= \lim_{b\to\infty}-3x^{-1/3}|^{b}_{1}\\
&= -3(\lim_{b\to\infty}b^{-1/3}-1)\\
&= -3(-1) = 3
\end{align}
\end{subequations}
Thus $f\in L^2$. So $L^1$ functions go to zero but not as
fast as $L^2$ functions.
\end{ex}

Similarly we may show that
\begin{equation}
g(x) = \begin{cases} x^{-2/3} &\text{if }0< x<1\\
0 &\text{otherwise}\end{cases}
\end{equation}
Then we may show that $g\in L^1$ but $g\notin L^2$.

We will now give a list of useful facts without proofs.

\begin{enumerate}
\item If $f\in L^1$ and $|f(x)|\leq M$ for all
  $x\in\mathbb{R}$, then $f\in L^2$. So
\begin{equation}
\int |f(x)|^2dx\leq M\int|f(x)|dx<\infty
\end{equation}
thus $f\in L^2$. In other words, $L^{1}\cap L^{2}$ (the set
of functions in both $L^1$ and $L^2$) contains bounded
functions in both spaces.
\item If $f\in L^{2}$, and $f(x)=0$ for $x$ outside the
  interval $[a,b]$, then $f\in L^{1}$. Consider
\begin{equation}
\int|f(x)|dx = \int^{b}_{a}|f(x)|dx =
\int^{b}_{a}1\cdot|f(x)|dx
\end{equation}
then by the Cauchy-Schwarz inequality
\begin{subequations}
\begin{align}
\int^{b}_{a}1\cdot|f(x)|dx&\leq\left(\int^{b}_{a}|f(x)|^2dx\right)\left(\int^{b}_{a}1^{2}dx\right)^{1/2}\\
&\leq\sqrt{b-a}\|f(x)\|^{1/2}_{2}<\infty
\end{align}
\end{subequations}
\end{enumerate}

\begin{defn}\index{Fourier Transform}
Let $f$ be a function defined on the whole real line. Then
the \textbf{Fourier transform} is
\begin{subequations}
\begin{align}
\widehat{f}(\xi) &= \int^{\infty}_{\infty}f(x)e^{-ix\xi}dx\\
&=\<f,e^{i\xi x}\>
\end{align}
\end{subequations}
for all $\xi\in\mathbb{R}$. This is the continuous analog
for the Fourier series coefficients.
\end{defn}

The inversion formula is, if $f\in L^2$, 
\begin{equation}
f(x) = \int \widehat{f}(\xi)e^{ix\xi}d\xi
\end{equation}
for $x\in\mathbb{R}$.

\subsection{Derivation of Fourier Transform}
On $L^{2}(-\pi,\pi)$, we have the orthogonal basis
$\{\exp(inx)\}^{\infty}_{-\infty}$. If we have $f\in
L^{2}(-\pi,\pi)$, we may write it in the basis
\begin{equation}
f(x) = \sum c_{n}e^{inx}
\end{equation}
where
\begin{equation}
c_{n} = \frac{1}{2\pi}\int^{\pi}_{-\pi}f(x)e^{-inx}dx
\end{equation}
If $f$ is $2\pi$-periodic then the expansion is defined
almost everwhere (at least it's defined on the interval
$[-\pi,\pi]$).

We may scale this to the interval $f\in L^{2}(-\ell,\ell)$
where $\ell>0$, then the basis becomes
\begin{equation}
\exp(inx)\to\exp\left(\frac{in\pi}{\ell}\right).
\end{equation}
The Fourier expansion then becomes
\begin{equation}
f(x) = \sum
\widetilde{c}_{n}e^{ix(n\pi/\ell)},\quad\text{where
}\widetilde{c}_n=\frac{1}{2\ell}\int^{\ell}_{-\ell}f(y)e^{-iny\pi/\ell}dy
\end{equation}
For $f$ defined on $\mathbb{R}$, we want to expand $f$ as a
superposition of $\exp(i\xi x)$...\textbf{HOW TO DO IT?!?}

For $[-\ell,\ell]$ when $\ell>0$. The Fourier expansion on
$[-\ell,\ell]$ and take the limit as $\ell\to\infty$:
\begin{equation}
f(x) = \sum^{\infty}_{-\infty}\frac{c_n}{2\ell}e^{in\pi
  x/\ell}
\end{equation}
where
\begin{equation}
c_n = \<f,e^{inx\pi/\ell}\> =
\int^{\ell}_{-\ell}f(x)e^{-inx\pi/\ell}dx
\end{equation}
We'll now turn this into a Riemann sum\index{Fourier Transform!Derived Using Riemann Sums}, let
\begin{equation}
\Delta\xi = \pi/\ell
\end{equation}
then subdividing the interval from $-\pi$ to $\pi$ into
\marginpar{Note interpretation of $\xi_n$ as boundaries of subintervals}$2\ell$ intervals. Let
\begin{equation}
\xi_n = n\Delta \xi = \frac{n\pi}{\ell} = \text{ endpoints of subintervals of }[-\pi,\pi]\index{$\xi_n$!As Dual To $x$}
\end{equation}
Then
\begin{subequations}
\begin{align}
f(x) &= \sum\frac{c_n}{2\ell}e^{in\pi x/\ell}\\
&=\sum c_n\frac{1}{2\ell}\cdot1\cdot e^{in\pi x/\ell}\\
&= \sum c_n\left(\frac{1}{2\ell}\right)\left(\frac{\Delta \xi}{\pi/\ell}\right)e^{i\xi_nx}\\
&= \sum c_n\left(\frac{1}{2\ell}\frac{\Delta \xi}{\pi/\ell}\right)e^{i\xi_nx}\\
&= \sum \frac{c_n}{2\pi}e^{i\xi_nx}\Delta \xi
\end{align}
\end{subequations}
where 
\begin{equation}
c_n = \int^{\ell}_{-\ell}f(y)\exp(-i\xi_{n}y)dy
\end{equation}
which when we take $\ell\to\infty$ is approximately
\begin{equation}
c_n\approx\int^{\infty}_{-\infty}f(y)e^{-i\xi_{n}y}dy
\end{equation}
provided $f$ vanishes rapidly as $y\to\pm\infty$. But this
is the definition of $\widehat{f}(\xi_n)$. We then make this
change to find $\Delta \xi\to d\xi$:
\begin{equation}
f(x)\approx \int^{\infty}_{-\infty}\frac{\widehat{f}(\xi_n)}{2\pi}e^{i\xi_nx}d\xi_n.
\end{equation}
Thus we finish our derivation of the Fourier transform.
