We introduced a different action which is based off of Weyl's attempt
to unify gravity and electromagnetism. Instead of attempting such a
unified field theory, we observed that it has interesting
gravitational properties. 

The vacuum satisfies the Schwarzschild solution for general relativity
with a nonzero cosmological constant, plus some nonzero term and a
term linear in $r$ negligibly small at the ``local'' scale. Due to
these extra terms, the scale invariance was spontaneously broken. This
was purely accidental.

We also observed that when we solve the fourth order field equations
for the isotropic and homogeneous case, we end up breaking symmetry
again. But in doing so, we recover the standard cosmological model,
and we explained why gravity is accelerating within the framework of
the Conformal gravity model. Further, we have an effective
gravitational constant that is scale dependent which allows gravity to
be repulsive globally but (due to inhomogeneities in the scalar field)
is locally attractive. This is consistent with the first investigation
of spontaneous symmetry breaking in solving the static, spherically
symmetric body's gravitational field as locally (``for small enough
$r$'') resembling Schwarzschild's solution.

Observe that this is really nothing surprising, since this is just
another version of the Brans-Dicke theory. The Brans-Dicke action is
\begin{equation}
I = \frac{1}{16\pi}\int d^{4}x\sqrt{-g}\left(\phi R
- \omega \frac{\partial_{\mu}\phi\partial^{\mu}\phi}{\phi} + L_{matter}\right)
\end{equation}
one can rearrange it by introducing $\Phi^2=\phi$ to look like
\begin{equation}
I = \frac{1}{16\pi}\int d^{4}x\sqrt{-g}\left(\Phi^2R -
4\omega\partial_\mu\Phi\partial^{\mu}\Phi + L_{matter}\right)
\end{equation}
which resembles the action in Eq \eqref{symmetryBreakingAction}. What
the Brans-Dicke theory effectively does is replace $k=16\pi G/c^4$
with a scalar field $\phi$. We did something similar, except our
scalar field spontaneously broke the scale invariance (so,
analogously, we had a bare minimum value for $k$) which gave rise
to a cosmological constant in addition to recovering the standard
cosmological model. Further, we used covariant derivatives instead of
partial derivatives, so we would need to include in the $L_{matter}$
the extra terms, the $\Phi^4$ term, and the coupling to
matter. Nonetheless, the cosmological constant naturally emerges when
we break symmetry. 
