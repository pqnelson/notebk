%%
%% admMetric.tex
%% 
%% Made by Alex Nelson
%% Login   <alex@tomato>
%% 
%% Started on  Fri Jun  5 17:27:58 2009 Alex Nelson
%% Last update Fri Jun  5 17:27:58 2009 Alex Nelson
%%
So when turning to the canonical formalism, we need to split
spacetime into space and time. Although this ``goes against the
spirit of special relativity'', there is a theorem from Geroch~\cite{geroch1970dd}
proving it's completely kosher:
\begin{quotation}
\textit{Property 7: Let $S$ be a Cauchy surface for the
  space-time $M$. Then $M$ is topologically
  $S\times\mathbb{R}$. In particular, if $M$ is connected, so is
  $S$.}

[...]

\noindent\textit{Theorem 11: A spacetime $M$ is globally hyperbolic if and
  only if it has a Cauchy surface.}\footnote{Or equivalently\begin{quote}\textit{If the spacetime is globally hyperbolic, then it is
  necessarily topologically $M=\mathbb{R}\times S$, where $M$
  is our spacetime manifold, $\mathbb{R}$ is ``time'' and
  $S$ is our spatial hypersurface.}\end{quote}}
\end{quotation}
If it turns out that spacetime is indeed Lorentzian, it is
necessarily hyperbolic. It follows that such a decomposition is
always allowable, at least \emph{classically}. We need to
``lift'' this condition to the quantum case, so there are no
``rips'' or ``mending of holes'' in spatial hypersurfaces in the
quantum case. We will not worry about that too much for now.

Now that we know it's kosher to split spacetime up into space and
time, lets start by considering a foliation of spacetime. Let $M$
be the manifold we use for spacetime, $\Sigma_{t}:=X_{t}(S)$
be (for each $t\in\mathbb{R}$) an embedding (a globally injective
immersion) $X_{t}:S\to M$. So how this should appear intuitively,
it's ``layering'' spacetime with ``spatial surfaces''
(time slices) where time is constant. Each of these time
slices has local coordinates $x_i$ and an induced metric
$g_{ij}(x,t)$\footnote{We will denote the induced metric by latin
 indices or ${}^{(3)}g_{ij}$.}. We can reconstruct all four dimensions by
examining how one time slice described by $\Sigma_{t}$ to a
nearby time slice $\Sigma_{t+dt}$ fit together. Consider a point
$x^i$ on $\Sigma_t$ and displacing it in time by an infinitesimal
amount in the direction of the normal of the surface. We write
the resulting change in proper time as
\begin{equation}%\label{eq:}
d\tau = Ndt = \begin{pmatrix}$lapse$\ $of$\ $proper$\
$time$\\ $between$\ $lower$\
$and$\\ $upper$\ $hypersurfaces$
\end{pmatrix} 
\end{equation}
where $N(t,x^{i})$ is called the ``lapse function''. Such a
displacement is only a shift in the time coordinate, to be fully
general we should consider a generic shift in both spatial and
time coordinates. A general shift in spatial coordinates would be
described by
\begin{equation}%\label{eq:}
x^{i}(t+dt) = x^{i}(t)-N^{i}dt,
\end{equation}
where $N^{i}(t,x^i)$ is the ``shift vector''. By the Lorentzian
pythagorean theorem, we see graphically in figure~\ref{fig:lorentzPyth}, 
the interval between $(t,x^i)$ and $(t+dt,x^i+dx^i)$ is
\begin{eqnarray}%\label{eq:}
ds^{2} &=& -N^2dt^2 + {}^{(3)}g_{ij}(dx^i + N^idt)(dx^j + N^jdt)\\
&=&
-\begin{pmatrix} $proper$\ $time$\ $from$\\
$lower$\ $to$\ $upper$\\ $3-geometry$\end{pmatrix}^2
+\begin{pmatrix}$proper$\ $distance$\ $in$\\
$base-3$\ $geometry$\end{pmatrix}^2
\nonumber
\end{eqnarray}
This is the ADM form of the metric.
\begin{figure}[t]
\includegraphics{img.1}
\caption{ADM decomposition of spacetime.}\label{fig:lorentzPyth}
\end{figure}

We can observe from
\begin{equation}%\label{eq:}
ds^2 = g_{\mu\nu}dx^{\mu}dx^{\nu} 
\end{equation}
that we can write the metric tensor in block form
\begin{equation}%\label{eq:}
g_{\mu\nu} = \begin{bmatrix}
g_{00} & g_{0j}\\
g_{i0} & g_{ij}
\end{bmatrix} = \left[\begin{array}{c:c}
(N_{i}N^{i}-N^{2}) & N_{j}\\\hdashline
N_{i} & {}^{(3)}g_{ij}
\end{array}\right]
\end{equation}
where $q_{ij}$ is the induced metric on the time slice. How to
find the inverse of the four-metric using this decomposition? We
can observe that we can find the inverse by
\begin{equation}%\label{eq:}
g_{\mu\nu}g^{\nu\rho} = \begin{bmatrix}
(N_{i}N^{i}-N^{2}) & N_{j}\\
N_{i} & {}^{(3)}g_{ij}
\end{bmatrix}
\begin{bmatrix}
\alpha & \beta^{k}\\
\beta^{j} & \gamma^{jk}
\end{bmatrix} = {\delta^{\rho}}_{\mu}.
\end{equation}
We end up with, through matrix multiplication, four conditions
\begin{subequations}
\begin{align}
\alpha N_i + {}^{(3)}g_{ij}\beta^{j} &= 0\label{eq:conOne}\\
(N_iN^i - N^2)\beta^k + N_j \gamma^{jk} &= 0\label{eq:conTwo}\\
(N_iN^i - N^2)\alpha + N_{j}\beta^{j} &= 1\label{eq:conThree}\\
N_{i}\beta^{k} + {}^{(3)}g_{ij}\gamma^{jk} &= {\delta^{k}}_{i}\label{eq:conFour}
\end{align}
\end{subequations}
Let ${}^{(3)}g^{ik}$ be the inverse of the induced metric, i.e.
\begin{equation}%\label{eq:}
{}^{(3)}g^{ik}{}^{(3)}g_{kj}={}^{(3)}{\delta^{i}}_{j}
\end{equation}
where ${}^{(3)}{\delta^{i}}_{j}$ is the 3 by 3 Kronecker delta. We
find from eq \eqref{eq:conOne} that
\begin{equation}%\label{eq:}
\beta^{k} = -\alpha N_{i}{}^{(3)}g^{ik} = -\alpha N^{k}.
\end{equation}
We plug this into eq \eqref{eq:conThree} to find that
\begin{equation}%\label{eq:}
\alpha = \frac{-1}{N^2}.
\end{equation}
This implies 
\begin{equation}%\label{eq:}
\beta^{k} = \frac{N^{k}}{N^2}.
\end{equation}
We plug this into eq \eqref{eq:conFour} to find
\begin{equation}%\label{eq:}
{}^{(3)}g_{ij}\gamma^{jk}={\delta_{i}}^{k}-\frac{N_{i}N^{k}}{N^2}\Rightarrow \gamma^{kl}={}^{(3)}g^{kl}-\frac{N^lN^k}{N^2}
\end{equation}
by multiplying both sides by ${}^{(3)}g^{il}$. We can now recombine our
results to find the inverse of the metric tensor to be
\begin{equation}%\label{eq:}
g^{\mu\nu} =\frac{1}{N^2} \left[\begin{array}{c:c}
-1 & N^{k}\\\hdashline
N^{j} & {}^{(3)}g^{jk}N^2 - (N^jN^k)
\end{array}\right]
\end{equation}
Observe that there is a difference between the spatial component
of the 4-metric ${}^{(4)}g^{ij}={}^{(3)}g^{ij}+\alpha N^{i}N^{j}$ and the inverse
of the induced metric ${}^{(3)}g^{ij}$.

Lets consider some geometry in this ADM form of the metric. Let
$n_{\alpha}=-N\delta_{0\alpha}$ be the unit normal to the time
slice. We define the first fundamental form as
\begin{equation}%\label{eq:}
q_{\mu\nu}:=g_{\mu\nu}+n_{\mu}n_{\nu}.
\end{equation}
Intuitively ${q^{\alpha}}_{\beta}$ be thought of as a sort of ``projection'',
i.e. it projects any index into a ``purely spatial'' one (more
precisely, a ``purely spatial'' index is one which -- when
contracted with $n^\alpha$ or $n_\alpha$ -- vanishes). We then
define the \marginpar{Extrinsic Curvature Tensor $K_{\alpha\beta}$} extrinsic curvature tensor
\begin{equation}%\label{eq:}
K_{\mu\nu} = {q_{\mu}}^{\rho}{q_{\nu}}^{\sigma}\nabla_{\rho}n_{\sigma}.
\end{equation}
We assert that $K_{\alpha\beta}$ is symmetric in its indices. How
can we see this? From observing the Lie derivative of the metric
along a vector in section \ref{sstn:lieDerivativeOfMetricAlongVector}
 we see that
\begin{equation}%\label{eq:}
\mathscr{L}_{n}g_{\alpha\beta} = \nabla_{\alpha}n_{\beta}+\nabla_{\beta}n_{\alpha}.
\end{equation}
We see the left hand side is symmetric in its indices which
implies the right hand side is symmetric in its indices. We see
that
\begin{equation}%\label{eq:}
{q_{\mu}}^{\alpha}{q_{\nu}}^{\beta}\mathscr{L}_{n}g_{\alpha\beta} = K_{\mu\nu}+K_{\nu\mu}\neq0.
\end{equation}
It's nonvanishing since $n_{\alpha}$ is not a Killing vector. We
see that
\begin{equation}%\label{eq:}
{q_{\mu}}^{\alpha}{q_{\nu}}^{\beta}\mathscr{L}_{n}g_{\alpha\beta}
- K_{\nu\mu} = K_{\mu\nu}
\end{equation}
which implies
\begin{equation}%\label{eq:}
{q_{[\mu}}^{\alpha}{q_{\nu]}}^{\beta}\mathscr{L}_{n}g_{\alpha\beta}
- K_{[\nu\mu]} = K_{[\mu\nu]}.
\end{equation}
We also see that
\begin{equation}%\label{eq:}
{q_{\mu}}^{\alpha}{q_{\nu}}^{\beta}\mathscr{L}_{n}g_{\alpha\beta}
- 2K_{\nu\mu} = K_{\mu\nu}- K_{\nu\mu}
\end{equation}
Setting these two equations equal yields
\begin{equation}%\label{eq:}
{q_{\mu}}^{\alpha}{q_{\nu}}^{\beta}\mathscr{L}_{n}g_{\alpha\beta}
- 2K_{\nu\mu} = {q_{\mu}}^{\alpha}{q_{\nu}}^{\beta}\mathscr{L}_{n}g_{\alpha\beta}
- K_{\nu\mu} - {q_{\nu}}^{\alpha}{q_{\mu}}^{\beta}\mathscr{L}_{n}g_{\alpha\beta}
+ K_{\mu\nu}
\end{equation}
which
