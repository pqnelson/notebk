%%
%% canonicalGravity.tex
%% 
%% Made by Alex Nelson
%% Login   <alex@tomato>
%% 
%% Started on  Fri Jun  5 11:50:17 2009 Alex Nelson
%% Last update Sat Jan 16 11:15:15 2010 Alex Nelson
%%
\documentclass[10pt,draft]{article}
\usepackage{notebk}
\numberwithin{equation}{section}

\title{Notes on Canonical Gravity}
\date{June 5, 2009}
\begin{document}
\maketitle
\tableofcontents
\begin{abstract}
A review of the ADM Hamiltonian formalism for classical General
Relativity.
\end{abstract}
\bigbreak
Just a few remarks on notation. Latin indices $i,j,\ldots$ will
be for spatial components of four-tensors, Greek indices
$\alpha,\beta,\ldots$ will be for spacetime components of
four-tensors.

We will approach the subject of the Hamiltonian formulation of
general relativity (also known as canonical formulation of
gravity, canonical dynamics for general relatvity, canonical
gravity, among countless other names -- canonical here refers to
the use of Hamiltonian formalism as opposed to the Lagrangian
formalism) by the following process. %first reviewing the notion of hypersurfaces. 
%We will then apply this to the 
We will perform the decomposition of spacetime into space
plus time, or the ADM form of the metric. Given such a
decomposition, we look at how the Lagrangian gives way to the
Hamiltonian formalism. Then we review the constraints of General
Relativity, both the Hamiltonian and Momentum constraints.

\section{A Review of Hypersurfaces}
%%
%% hypersurface.tex
%% 
%% Made by Alex Nelson
%% Login   <alex@tomato>
%% 
%% Started on  Sat Jun  6 14:24:25 2009 Alex Nelson
%% Last update Sat Jun  6 14:24:25 2009 Alex Nelson
%%

Lets briefly turn our attention to the geometry of hypersurfaces
in four dimensions. In a four dimensional manifold, a
hypersurface is a three dimensional submanifold that can be
timelike, spacelike, or null. We may consider a particular
hypersurface $\Sigma$ by restricting the coordinates
\begin{equation}%\label{eq:}
C(x^{\alpha})=0
\end{equation}
or by parametrizing the coordinates
\begin{equation}%\label{eq:}
x^{\alpha} = x^{\alpha}(y^a)
\end{equation}
where $y^a$ ($a=1,$ 2, 3) are the coordinates intrinsic to the
hypersurface. Think of the sphere in three dimensions, we can
specify it by the restriction of the coordinates
\begin{equation}%\label{eq:}
C(x,y,z) = R^2-x^2-y^2-z^2 = 0
\end{equation}
or in parametric form
\begin{subequations}
\begin{align}
x(\theta,\phi) &= R\sin(\theta)\cos(\phi)\\
y(\theta,\phi) &= R\sin(\theta)\sin(\phi)\\
z(\theta,\phi) &= R\cos(\theta)
\end{align}
\end{subequations}
where $\theta$, $\phi$ are the coordinates intrinsic to the sphere.

\marginpar{Unit normal} Now, if we consider the vector $\partial_{\alpha}C$, it is normal
to the hypersurface. This is due to the observation the only
value of $C$ changes in a direction orthogonal to $\Sigma$. If
the surface is not null, we can introduce the unit normal
$n_{\alpha}$ defined such that
\begin{equation}%\label{eq:}
n^{\alpha}n_{\alpha} = s = \begin{cases}+1,&\text{if $\Sigma$ is timelike}\\
-1,&\text{if $\Sigma$ is spacelike.}\end{cases}
\end{equation}
We demand the condition that $n^\alpha$ point in the direction of
increasing $C$,
\begin{equation}%\label{eq:}
n^{\alpha}\partial_{\alpha} C > 0.
\end{equation}
We see that given this condition on the sign of the unit normal,
and the fact that $\partial_{\alpha}C$ points in the direction of
the unit normal, that
\begin{equation}%\label{eq:}
n_{\alpha} = \frac{s\partial_{\alpha}C}{|g^{\mu\nu}\partial_{\mu}C\partial_{\nu}C|^{1/2}}
\end{equation}
provided that the hypersurface is not null (i.e. it must be
either timelike or spacelike). If it were null, then the
denominator vanishes:
\begin{equation}%\label{eq:}
g^{\mu\nu}\partial_{\mu}C\partial_{\nu}C=0
\end{equation}
which is bad.

\marginpar{Induced Metric}The metric intrinsic to
the hypersurface is obtained by restricting the line element $ds^2$ to
displacements confined to the hypersurface. Remember we
parametrized the four dimensional coordinates of the surface by
the equations
\begin{equation}\label{eq:intrinsicRelations}
x^{\alpha}=x^{\alpha}(y^a).
\end{equation}
We see that the vectors
\begin{equation}%\label{eq:}
{e^{\alpha}}_{a} = \frac{\partial x^{\alpha}}{\partial y^a}
\end{equation}
are tangent to the curves contained in $\Sigma$. Why? Well, the
relations described by Eq \eqref{eq:intrinsicRelations} describe
curves contained entirely in $\Sigma$, parametrized in $y^a$, so
differentiating with respect to the parameters would yield the
tangent vectors.

Now, for displacements contained in $\Sigma$, we can write the
infinitesimal line element as
\begin{equation}%\label{eq:}
ds_{\Sigma}^{2} = g_{\alpha\beta}dx^{\alpha}dx^{\beta} =
g_{\alpha\beta}\left(\frac{\partial x^{\alpha}}{\partial y^{a}}dy^{a}\right)\left(\frac{\partial x^{\beta}}{\partial y^{b}}dy^{b}\right)
= h_{ab}dy^ady^b.
\end{equation}
where
\begin{equation}%\label{eq:}
h_{ab} = g_{\alpha\beta}\frac{\partial x^{\alpha}}{\partial
  y^{a}}\frac{\partial x^{\beta}}{\partial y^{b}} = g_{\alpha\beta}{e^{\alpha}}_{a}{e^{\beta}}_{b}
\end{equation}
is called the ``induced metric''. It behaves as a scalar when we
change coordinates $x^{\mu}\to x^{\mu'}$ but it behaves as a
tensor when we change coordinates $y^{m}\to y^{m'}$ intrinsic to
the surface. We'll call such things \define{three-tensors}. 

\subsection{Lie Derivative of the Metric Along a Vector}\label{sstn:lieDerivativeOfMetricAlongVector}
%%
%% lieDerivativeOfMetricAlongVector.tex
%% 
%% Made by Alex Nelson
%% Login   <alex@tomato>
%% 
%% Started on  Fri Jun  5 13:49:18 2009 Alex Nelson
%% Last update Fri Jun  5 13:49:18 2009 Alex Nelson
%%
The Lie derivative of the metric along a vector $\xi^{a}$ is
\begin{equation}\label{eq:lieDerivativeOfMetric}
\mathscr{L}_{\xi}g_{ab} =
g_{ac}\partial_{b}\xi^{c} + 
g_{bc}\partial_{a}\xi^{c} +
\xi^{c}\partial_{c}g_{ab}.
\end{equation}
Observe that
\begin{equation}\label{eq:covariantDerivativeContraction}
g_{bc}\nabla_{a}\xi^{c} = g_{bc}(\partial_a\xi^c + \Gamma^{c}_{ad}\xi^{d})
\end{equation}
where $\Gamma$ is the Christoffel symbol, $\nabla$ is the
covariant derivative. We specifically find
\begin{equation}\label{eq:firstManipulation}
g_{bc}\Gamma^{c}_{ad}\xi^{d} = \Gamma_{bad}\xi^{d}.
\end{equation}
For the affine connection, we have
\begin{equation}\label{eq:affineConnectionConditions}
\partial_{c}g_{ab} = \Gamma_{acb} + \Gamma_{bca} = 0.
\end{equation}
So we plug this into eq \eqref{eq:lieDerivativeOfMetric} to find
\begin{equation}\label{eq:lieDerivativeMutatisMutandi}
\mathscr{L}_{\xi}g_{ab} =
g_{ac}\partial_{b}\xi^{c} + 
g_{bc}\partial_{a}\xi^{c} +
\xi^{c}\left(\Gamma_{acb} + \Gamma_{bca}\right)
\end{equation}
By the properties of the Christoffel symbol, specifically
\begin{equation}%\label{eq:}
\Gamma_{cab} = \Gamma_{cba}
\end{equation}
we can rewrite eq \eqref{eq:covariantDerivativeContraction}
as
\begin{equation}%\label{eq:}
g_{bc}\nabla_{a}\xi^{c} = g_{bc}\partial_a\xi^c + \Gamma_{bad}\xi^{d}.
\end{equation}
Now observe the Lie derivative of the metric along our vector
$\xi^{a}$ can be grouped in terms
\begin{equation}%\label{eq:}
\mathscr{L}_{\xi}g_{ab} =
(g_{ac}\partial_{b}\xi^{c} + \Gamma_{abc}\xi^{c}) + 
(g_{bc}\partial_{a}\xi^{c} + \Gamma_{bac}\xi^{c})
\end{equation}
since the $c$ index is summed over, it's a dummy index. We can
rewrite this in more familiar terms
\begin{equation}%\label{eq:}
\mathscr{L}_{\xi}g_{ab} =
(g_{ac}\partial_{b}\xi^{c} + \Gamma_{abd}\xi^{d}) + 
(g_{bc}\partial_{a}\xi^{c} + \Gamma_{bad}\xi^{d})
\end{equation}
thus
\begin{equation}%\label{eq:}
\mathscr{L}_{\xi}g_{ab} = g_{ac}\nabla_{b}\xi^{c} + g_{bc}\nabla_{a}\xi^{c}.
\end{equation}
Since the connection is metric compatible, we can ``bring the
metric inside the derivative''
\begin{equation*}%\label{eq:}
g_{ab}\nabla_{c}(\cdots) \to \nabla_{c}(g_{ab}\cdots)
\end{equation*}
since $\nabla g_{ab} = 0$. Now we can rewrite our Lie derivative
as
\begin{equation}%\label{eq:}
\mathscr{L}_{\xi}g_{ab} = \nabla_{b}\xi_{a} + \nabla_{a}\xi_{b}.
\end{equation}
This is precisely the Killing equation.



\section{ADM Form of the Metric}
%%
%% admMetric.tex
%% 
%% Made by Alex Nelson
%% Login   <alex@tomato>
%% 
%% Started on  Fri Jun  5 17:27:58 2009 Alex Nelson
%% Last update Fri Jun  5 17:27:58 2009 Alex Nelson
%%
So when turning to the canonical formalism, we need to split
spacetime into space and time. Although this ``goes against the
spirit of special relativity'', there is a theorem from Geroch~\cite{geroch1970dd}
proving it's completely kosher:
\begin{quotation}
\textit{Property 7: Let $S$ be a Cauchy surface for the
  space-time $M$. Then $M$ is topologically
  $S\times\mathbb{R}$. In particular, if $M$ is connected, so is
  $S$.}

[...]

\noindent\textit{Theorem 11: A spacetime $M$ is globally hyperbolic if and
  only if it has a Cauchy surface.}\footnote{Or equivalently\begin{quote}\textit{If the spacetime is globally hyperbolic, then it is
  necessarily topologically $M=\mathbb{R}\times S$, where $M$
  is our spacetime manifold, $\mathbb{R}$ is ``time'' and
  $S$ is our spatial hypersurface.}\end{quote}}
\end{quotation}
If it turns out that spacetime is indeed Lorentzian, it is
necessarily hyperbolic. It follows that such a decomposition is
always allowable, at least \emph{classically}. We need to
``lift'' this condition to the quantum case, so there are no
``rips'' or ``mending of holes'' in spatial hypersurfaces in the
quantum case. We will not worry about that too much for now.

Now that we know it's kosher to split spacetime up into space and
time, lets start by considering a foliation of spacetime. Let $M$
be the manifold we use for spacetime, $\Sigma_{t}:=X_{t}(S)$
be (for each $t\in\mathbb{R}$) an embedding (a globally injective
immersion) $X_{t}:S\to M$. So how this should appear intuitively,
it's ``layering'' spacetime with ``spatial surfaces''
(time slices) where time is constant. Each of these time
slices has local coordinates $x_i$ and an induced metric
$g_{ij}(x,t)$\footnote{We will denote the induced metric by latin
 indices or ${}^{(3)}g_{ij}$.}. We can reconstruct all four dimensions by
examining how one time slice described by $\Sigma_{t}$ to a
nearby time slice $\Sigma_{t+dt}$ fit together. Consider a point
$x^i$ on $\Sigma_t$ and displacing it in time by an infinitesimal
amount in the direction of the normal of the surface. We write
the resulting change in proper time as
\begin{equation}%\label{eq:}
d\tau = Ndt = \begin{pmatrix}$lapse$\ $of$\ $proper$\
$time$\\ $between$\ $lower$\
$and$\\ $upper$\ $hypersurfaces$
\end{pmatrix} 
\end{equation}
where $N(t,x^{i})$ is called the ``lapse function''. Such a
displacement is only a shift in the time coordinate, to be fully
general we should consider a generic shift in both spatial and
time coordinates. A general shift in spatial coordinates would be
described by
\begin{equation}%\label{eq:}
x^{i}(t+dt) = x^{i}(t)-N^{i}dt,
\end{equation}
where $N^{i}(t,x^i)$ is the ``shift vector''. By the Lorentzian
pythagorean theorem, we see graphically in figure~\ref{fig:lorentzPyth}, 
the interval between $(t,x^i)$ and $(t+dt,x^i+dx^i)$ is
\begin{eqnarray}%\label{eq:}
ds^{2} &=& -N^2dt^2 + {}^{(3)}g_{ij}(dx^i + N^idt)(dx^j + N^jdt)\\
&=&
-\begin{pmatrix} $proper$\ $time$\ $from$\\
$lower$\ $to$\ $upper$\\ $3-geometry$\end{pmatrix}^2
+\begin{pmatrix}$proper$\ $distance$\ $in$\\
$base-3$\ $geometry$\end{pmatrix}^2
\nonumber
\end{eqnarray}
This is the ADM form of the metric.
\begin{figure}[t]
\includegraphics{img.1}
\caption{ADM decomposition of spacetime.}\label{fig:lorentzPyth}
\end{figure}

We can observe from
\begin{equation}%\label{eq:}
ds^2 = g_{\mu\nu}dx^{\mu}dx^{\nu} 
\end{equation}
that we can write the metric tensor in block form
\begin{equation}%\label{eq:}
g_{\mu\nu} = \begin{bmatrix}
g_{00} & g_{0j}\\
g_{i0} & g_{ij}
\end{bmatrix} = \left[\begin{array}{c:c}
(N_{i}N^{i}-N^{2}) & N_{j}\\\hdashline
N_{i} & {}^{(3)}g_{ij}
\end{array}\right]
\end{equation}
where $q_{ij}$ is the induced metric on the time slice. How to
find the inverse of the four-metric using this decomposition? We
can observe that we can find the inverse by
\begin{equation}%\label{eq:}
g_{\mu\nu}g^{\nu\rho} = \begin{bmatrix}
(N_{i}N^{i}-N^{2}) & N_{j}\\
N_{i} & {}^{(3)}g_{ij}
\end{bmatrix}
\begin{bmatrix}
\alpha & \beta^{k}\\
\beta^{j} & \gamma^{jk}
\end{bmatrix} = {\delta^{\rho}}_{\mu}.
\end{equation}
We end up with, through matrix multiplication, four conditions
\begin{subequations}
\begin{align}
\alpha N_i + {}^{(3)}g_{ij}\beta^{j} &= 0\label{eq:conOne}\\
(N_iN^i - N^2)\beta^k + N_j \gamma^{jk} &= 0\label{eq:conTwo}\\
(N_iN^i - N^2)\alpha + N_{j}\beta^{j} &= 1\label{eq:conThree}\\
N_{i}\beta^{k} + {}^{(3)}g_{ij}\gamma^{jk} &= {\delta^{k}}_{i}\label{eq:conFour}
\end{align}
\end{subequations}
Let ${}^{(3)}g^{ik}$ be the inverse of the induced metric, i.e.
\begin{equation}%\label{eq:}
{}^{(3)}g^{ik}{}^{(3)}g_{kj}={}^{(3)}{\delta^{i}}_{j}
\end{equation}
where ${}^{(3)}{\delta^{i}}_{j}$ is the 3 by 3 Kronecker delta. We
find from eq \eqref{eq:conOne} that
\begin{equation}%\label{eq:}
\beta^{k} = -\alpha N_{i}{}^{(3)}g^{ik} = -\alpha N^{k}.
\end{equation}
We plug this into eq \eqref{eq:conThree} to find that
\begin{equation}%\label{eq:}
\alpha = \frac{-1}{N^2}.
\end{equation}
This implies 
\begin{equation}%\label{eq:}
\beta^{k} = \frac{N^{k}}{N^2}.
\end{equation}
We plug this into eq \eqref{eq:conFour} to find
\begin{equation}%\label{eq:}
{}^{(3)}g_{ij}\gamma^{jk}={\delta_{i}}^{k}-\frac{N_{i}N^{k}}{N^2}\Rightarrow \gamma^{kl}={}^{(3)}g^{kl}-\frac{N^lN^k}{N^2}
\end{equation}
by multiplying both sides by ${}^{(3)}g^{il}$. We can now recombine our
results to find the inverse of the metric tensor to be
\begin{equation}%\label{eq:}
g^{\mu\nu} =\frac{1}{N^2} \left[\begin{array}{c:c}
-1 & N^{k}\\\hdashline
N^{j} & {}^{(3)}g^{jk}N^2 - (N^jN^k)
\end{array}\right]
\end{equation}
Observe that there is a difference between the spatial component
of the 4-metric ${}^{(4)}g^{ij}={}^{(3)}g^{ij}+\alpha N^{i}N^{j}$ and the inverse
of the induced metric ${}^{(3)}g^{ij}$.

Lets consider some geometry in this ADM form of the metric. Let
$n_{\alpha}=-N\delta_{0\alpha}$ be the unit normal to the time
slice. We define the first fundamental form as
\begin{equation}%\label{eq:}
q_{\mu\nu}:=g_{\mu\nu}+n_{\mu}n_{\nu}.
\end{equation}
Intuitively ${q^{\alpha}}_{\beta}$ be thought of as a sort of ``projection'',
i.e. it projects any index into a ``purely spatial'' one (more
precisely, a ``purely spatial'' index is one which -- when
contracted with $n^\alpha$ or $n_\alpha$ -- vanishes). We then
define the \marginpar{Extrinsic Curvature Tensor $K_{\alpha\beta}$} extrinsic curvature tensor
\begin{equation}%\label{eq:}
K_{\mu\nu} = {q_{\mu}}^{\rho}{q_{\nu}}^{\sigma}\nabla_{\rho}n_{\sigma}.
\end{equation}
We assert that $K_{\alpha\beta}$ is symmetric in its indices. How
can we see this? From observing the Lie derivative of the metric
along a vector in section \ref{sstn:lieDerivativeOfMetricAlongVector}
 we see that
\begin{equation}%\label{eq:}
\mathscr{L}_{n}g_{\alpha\beta} = \nabla_{\alpha}n_{\beta}+\nabla_{\beta}n_{\alpha}.
\end{equation}
We see the left hand side is symmetric in its indices which
implies the right hand side is symmetric in its indices. We see
that
\begin{equation}%\label{eq:}
{q_{\mu}}^{\alpha}{q_{\nu}}^{\beta}\mathscr{L}_{n}g_{\alpha\beta} = K_{\mu\nu}+K_{\nu\mu}\neq0.
\end{equation}
It's nonvanishing since $n_{\alpha}$ is not a Killing vector. We
see that
\begin{equation}%\label{eq:}
{q_{\mu}}^{\alpha}{q_{\nu}}^{\beta}\mathscr{L}_{n}g_{\alpha\beta}
- K_{\nu\mu} = K_{\mu\nu}
\end{equation}
which implies
\begin{equation}%\label{eq:}
{q_{[\mu}}^{\alpha}{q_{\nu]}}^{\beta}\mathscr{L}_{n}g_{\alpha\beta}
- K_{[\nu\mu]} = K_{[\mu\nu]}.
\end{equation}
We also see that
\begin{equation}%\label{eq:}
{q_{\mu}}^{\alpha}{q_{\nu}}^{\beta}\mathscr{L}_{n}g_{\alpha\beta}
- 2K_{\nu\mu} = K_{\mu\nu}- K_{\nu\mu}
\end{equation}
Setting these two equations equal yields
\begin{equation}%\label{eq:}
{q_{\mu}}^{\alpha}{q_{\nu}}^{\beta}\mathscr{L}_{n}g_{\alpha\beta}
- 2K_{\nu\mu} = {q_{\mu}}^{\alpha}{q_{\nu}}^{\beta}\mathscr{L}_{n}g_{\alpha\beta}
- K_{\nu\mu} - {q_{\nu}}^{\alpha}{q_{\mu}}^{\beta}\mathscr{L}_{n}g_{\alpha\beta}
+ K_{\mu\nu}
\end{equation}
which

\section{Einstein Hilbert Action to Canonical Variables}
%%
%% lagrangian.tex
%% 
%% Made by Alex Nelson
%% Login   <alex@black-cherry>
%% 
%% Started on  Wed Aug 26 12:02:16 2009 Alex Nelson
%% Last update Wed Aug 26 12:02:16 2009 Alex Nelson
%%

As with Hamiltonian mechanics, wherein one begins by taking the
Legendre transform of the Lagrangian, in Hamiltonian field theory
we ``transform'' the Lagrangian field treatment. So lets review
the calculations in Lagrangian field theory.

Consider the classical fields $\phi^{a}(t,\bar{x})$. We use the
index $a$ to indicate which field we are talking about. We should
think of $\bar{x}$ as another index, except it is \emph{continuous}.
We will use the confusing short hand notation $\phi$ for the
column vector $\phi^{1},\ldots,\phi^{n}$. Consider the Lagrangian
\begin{equation}%\label{eq:}
L(\phi) = \int_{\mathclap{\text{all space}}}\mathcal{L}(\phi,\partial_{\mu}\phi)d^{3}\bar{x}
\end{equation}
where $\mathcal{L}$ is the \emph{Lagrangian density}. Hamilton's
principle of stationary action is still used to determine the
equations of motion from the action
\begin{equation}%\label{eq:}
S[\phi] = \int L(\phi,\partial_{\mu}\phi)dt
\end{equation}
where we find the Euler-Lagrange equations of motion for the field
\begin{equation}%\label{eq:}
\frac{d}{dx^{\mu}}\frac{\partial\mathcal{L}}{\partial (\partial_{\mu}\phi^{a})}=\frac{\partial\mathcal{L}}{\partial\phi^{a}}
\end{equation}
where we note these are evil second order partial differential
equations. We also note that we are using Einstein summation
convention, so there is an implicit sum over $\mu$ but not over
$a$. So that means there are $n$ independent second order
partial differential equations we need to solve.

But how do we really know these are the correct equations? How do
we really know these are the Euler-Lagrange equations for
classical fields?  We can obtain it directly from the action $S$
by functional differentiation with respect to the field. Taking
$\phi$ to be a single scalar field (for simplicity's sake, it
doesn't change anything if we work with $n$ fields), functional
differentiation can be defined by
\begin{equation}%\label{eq:}
\frac{\delta S}{\delta\phi(x)}\eqdef
\lim_{\varepsilon\to0}\frac{1}{\varepsilon}\left(S\left[\phi(y)+\varepsilon\delta^{(4)}(x-y)\right]-S\left[\phi(y)\right]\right)
\end{equation}
where $\delta^{(4)}(y-x)$ is the 4-dimensional densitized Dirac
delta function. Note that we will often use the shorthand
notation $\delta^{(4)}_{x}=\delta^{(4)}(y-x)$. Applying this to the action yields
\begin{subequations}
\begin{align}
\frac{\delta S}{\delta\phi(x)} &= \lim_{\varepsilon\to0}\frac{1}{\varepsilon}\int\left[\mathcal{L}\left(\phi+\varepsilon\delta^{(4)}_{x},\partial_{\mu}\phi+\varepsilon\partial_{\mu}\delta^{(4)}_{x}\right)-\mathcal{L}(\phi,\partial_{\mu}\phi)\right]d^{4}y\\
&=\lim_{\varepsilon\to0}\frac{1}{\varepsilon}\int\left[\mathcal{L}(\phi,\partial_{\mu}\phi)+\frac{\partial\mathcal{L}}{\partial\phi}\delta^{(4)}_{x}\varepsilon+\frac{\partial\mathcal{L}}{\partial(\partial_{\mu}\phi)}\partial_{\mu}\delta^{(4)}_{x}\varepsilon+\mathcal{O}(\varepsilon^{2})-\mathcal{L}(\phi,\partial_{\mu}\phi)\right]d^{4}y\\
&= \lim_{\varepsilon\to0}\int\left[\frac{\partial\mathcal{L}}{\partial\phi}\delta^{(4)}_{x}+\frac{\partial\mathcal{L}}{\partial(\partial_{\mu}\phi)}\partial_{\mu}\delta^{(4)}_{x}+\mathcal{O}(\varepsilon)\right]d^{4}y\\
&=\int\left[\frac{\partial\mathcal{L}}{\partial\phi}\delta^{(4)}_{x}+\frac{\partial\mathcal{L}}{\partial(\partial_{\mu}\phi)}\partial_{\mu}\delta^{(4)}_{x}\right]d^{4}y\\
&=\int\left[\frac{\partial\mathcal{L}}{\partial\phi}-\partial_{\mu}\frac{\partial\mathcal{L}}{\partial(\partial_{\mu}\phi)}\right]\delta^{(4)}_{x}d^{4}y
\end{align}
\end{subequations}
Where we justify the second line by Taylor expanding to first
order, then in the third line we factor through by the
$(1/\varepsilon)$ factor, in the fourth line we take the limit,
and integrate by parts to yield the last line. Note also that we
factored out the delta function to make the last line
prettier. Now the last line is zero if and only if
\begin{equation}%\label{eq:}
\frac{\partial\mathcal{L}}{\partial\phi^{a}}-\frac{d}{dx^{\mu}}\frac{\partial\mathcal{L}}{\partial (\partial_{\mu}\phi^{a})}=0
\end{equation}
which is precisely the Euler-Lagrange equations of motion!

\begin{ddanger}
Why are we working with these delta functions? Well, we are
working  with something a little more than just time. We are
working with points in space. Locality means, mathematically, we
work with vectors sharing the same base point. Or in the jargon
of differential geometry, we are working in the tangent space
$T_{p}\mathcal{M}$ where $\mathcal{M}$ is our manifold, and
$p\in\mathcal{M}$ is our base point. If we work with multiple
base points at a time, not only is it mathematically not well
defined, but it is \emph{nonlocal} which results in \emph{a loss of causality!}
Needless to say this is bad, so we try to work with the evolution
of the field at a specified (but arbitrary) tangent space. If
time permits, we will revisit this notion of spatial coordinates
as an ``index'' in the appendix.

More precisely, we have the ``base space'' be the manifold
$\mathcal{M}$ representing spacetime. We have our fields assign
to each point $p\in\mathcal{M}$ some ``physical information''
$\phi(x)$. The question presents itself ``Where does this
`information' live?'' It lives in a generalization of the tangent
space, called the ``fiber''. In the Lagrangian setting, we work
with ordered pairs $(x^{\mu},\phi(x^{\mu}))$ which is actually
something called the ``section'' of the fiber bundle. That is,
the field is a mapping
\begin{equation}%\label{eq:}
\phi:\mathcal{M}\to\mathcal{M}\times{F}
\end{equation}
where $F$ is ``where'' the fields ``live'', i.e. it's the fiber.
\end{ddanger}


\section{Canonical Dynamics}
%%
%% canonicalAction.tex
%% 
%% Made by Alex Nelson
%% Login   <alex@tomato>
%% 
%% Started on  Fri Jun  5 12:08:09 2009 Alex Nelson
%% Last update Fri Jun  5 12:08:09 2009 Alex Nelson
%%
Now, we can write the action in terms of our new canonical
variables as
\begin{equation}%\label{eq:}
I = \int dt\; d^{3}x \left[\pi^{ij}\dot{q}_{ij} - N\mathscr{H} - N_{i}\mathscr{H}^{i}\right]
\end{equation}
where
\begin{equation}\label{eq:admHamiltonian}
\mathscr{H} = \frac{16\pi
  G_{N}}{\sqrt{q}}\left(\pi^{ij}\pi_{ij}-\pi^{2}\right) -
\frac{\sqrt{q}}{16\pi G_{N}}\sqrt{q}{}^{(3)}R
\end{equation}
and
\begin{equation}%\label{eq:}
\mathscr{H}^{i} = -2 D_{j}\pi^{ij}.
\end{equation}
Eq \eqref{eq:admHamiltonian} is a Hamiltonian for General
Relativity based off of a certain set of variables -- the metric
for a spatial hypersurface as the position variable, and its time
derivative as the canonically conjugate momenta.

We now find the dynamics in the usual way, by using the Poisson
bracket. We see that
\begin{equation}%\label{eq:}
\{ q_{ij}(x), \pi^{kl}(x') \} = \frac{1}{2}\left(\delta^{k}_{i}\delta^{l}_{j}+\delta^{k}_{j}\delta^{l}_{i}\right)\tilde{\delta}^{(3)}(x-x')
\end{equation}
where $\widetilde{\delta}^{(3)}$ is the densitized delta
function, i.e. the delta function such that
\begin{equation}%\label{eq:}
\int\tilde{\delta}^{(3)}(x)d^{3}x = 1
\end{equation}
so we won't need $\sqrt{q}$. Now, this is a completely
constrained system, with the momentum constraints generating
spatial change of coordinates. Consider the Poisson bracket of
the momentum constraints with the spatial metric:
\begin{subequations}
\begin{align}
\left\{\int\xi^{i}\mathscr{H}_{i}(x)d^{3}x,\; q_{kl}(x')\right\} &=  
\left\{-2\int\xi^{i}D^{j}\pi_{ij}(x)d^{3}x,\; q_{kl}(x')\right\}\\
&= \left\{\int(D_{i}\xi_{j}+D_{j}\xi_{i})\pi^{ij}(x)d^{3}x,\; q_{kl}(x')\right\}\\
&= -(D_{k}\xi_{l}+D_{l}\xi_{k})\\
&= -\mathscr{L}_{\xi}q_{kl}
\end{align}
\end{subequations}
where $\mathscr{L}_{\xi}$ is the Lie derivative. This means that
$\mathscr{H}_{i}$ are the generators of spatial coordinate
transformations. The Poisson bracket for the momentum constraints
and the $\pi^{ij}$ are a bit more complicated.

We are working on spatial hypersurfaces, so the question of \emph{what
``$\mathscr{H}$ generates time translations'' means} needs to be
investigated. Again, the easy Poisson bracket to consider is with
the spatial metric. Technically, these are ``spatial
deformations'' yielded by such a bracket.


\nocite{*}
\bibliographystyle{elements}\footnotesize
\bibliography{canonicalGravity}
\end{document}
