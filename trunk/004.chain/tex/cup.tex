%%
%% cup.tex
%% 
%% Made by Alex Nelson
%% Login   <alex@tomato>
%% 
%% Started on  Mon Aug 24 16:23:59 2009 Alex Nelson
%% Last update Mon Aug 24 16:23:59 2009 Alex Nelson
%%

\begin{wrapfigure}{l}{1.3in}
\includegraphics{img/img.7}
\end{wrapfigure}

For the next situation, we'll consider the doodle on the
left. That is, it's...well, a cup. Dual to the picture before,
where from the void there came about a cap, here from a cup
begath the void. We expect this to be remarkably similar to the
calculation we just performed. We should consider the initial
state to be described by the chain
\begin{equation*}%\label{eq:}
\mathbb{Z}^{n_{v}}\leftarrow \mathbb{Z}^{n_{e}}
\end{equation*}
where $n_v$, $n_e$ are the number of vertices and edges
(respectively). The final state is the vacuous one
\begin{equation*}%\label{eq:}
\mathbb{Z}^{\emptyset}\leftarrow\mathbb{Z}^{\emptyset}
\end{equation*}
since there are no edges or vertices, we have to work with the
empty set! Putting together everything we know, we get the chain
complex
\begin{equation}%\label{eq:}
\begin{CD}
\mathbb{Z}^{1} @<<< \mathbb{Z}^{1}\\
@VVV                 @VVV\\
\mathbb{Z}^{1} @<<< \mathbb{Z}^{1} @<<< \mathbb{Z}^{1}\\
@AAA                 @AAA\\
\mathbb{Z}^{\emptyset} @<<< \mathbb{Z}^{\emptyset}.
\end{CD}
\end{equation}
Where we have taken advantage of the obvious: there is one vertex
and one edge in the initial state.

We see that the Hilbert spaces describing the initial and the
final states are precisely those from the corresponding component
from the previous computation. That is, the initial state is
$L^{2}(U(1))$ and the final state is described by the Hilbert
space $\mathbb{C}$. We can similarly choose our favorite states
from each space ($e^{ikA}$ and 1 respectively). We describe the
time evolution operator as 
\begin{equation}%\label{eq:}
Z(\varepsilon):L^{2}(U(1))\to\mathbb{C}.
\end{equation}
To do so we need to consider the inner product
\begin{equation}%\label{eq:}
\<\phi,Z(\varepsilon)\psi\> = \int_{\mathcal{A}(\varepsilon)} \overline{\phi(A|_{S})}\psi(A|_{S'})e^{-S(A)}\mathcal{D}A
\end{equation}
where $\phi\in\mathbb{C}$ and $\psi\in L^{2}(U(1))$.

We can plug in our choices $\phi=1$ and $\psi=\exp(ikA)$:
\begin{equation}%\label{eq:}
\<1,Z(\varepsilon)e^{ikA}\> = \int_{\mathcal{A}(\varepsilon)}1\cdot e^{ikA}e^{-S(A)}\mathcal{D}A
\end{equation}
To compute the action, we need the curvature but the curvature is
simply $F=A$. We can plug this into the result of our proposition
2 to find
\begin{equation}%\label{eq:}
\<1,Z(\varepsilon)e^{ikA}\> = \int_{\mathcal{A}(\varepsilon)}1\cdot e^{ikA}\sum_{n\in\mathbb{Z}}e^{\frac{-e^{2}n^{2}V}{2}}e^{inA}\mathcal{D}A
\end{equation}
up to a constant. We see that
\begin{equation}%\label{eq:}
\mathcal{D}A = \frac{dA}{2\pi}
\end{equation}
and rearranging terms we can find
\begin{equation}%\label{eq:}
\<1,Z(\varepsilon)e^{ikA}\> = \sum_{n\in\mathbb{Z}}e^{\frac{-e^{2}n^{2}V}{2}}\underbracket[0.25pt]{\int^{2\pi}_{0}e^{i(n+k)A}\frac{dA}{2\pi}}_{\delta_{-k,n}}
\end{equation}
where the underbracketed factor turns out to be the Kronecker
delta, as noted in the equation. Rearranging terms we find that
\begin{equation}%\label{eq:}
Z(\varepsilon)e^{ikA} =
\sum_{n\in\mathbb{Z}}e^{-e^{2}n^{2}V/2}\delta_{-k,n} = \exp\left(\frac{-e^{2}k^{2}V}{2}\right)
\end{equation}
This is precisely what we desire to deduce.
