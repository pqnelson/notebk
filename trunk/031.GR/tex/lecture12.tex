%%
%% lecture12.tex
%% 
%% Made by alex
%% Login   <alex@tomato>
%% 
%% Started on  Thu Mar  1 10:59:21 2012 alex
%% Last update Thu Mar  1 10:59:21 2012 alex
%%

We will consider the slickest way to compute connections. First
recall the basic conditions for a connection are: 
\begin{enumerate}
\item Torsion free $\Gamma^{\rho}_{\mu\nu}=\Gamma^{\rho}_{\nu\mu}$
\item Metric Compatible $\nabla_{\mu}g_{ab}=0$.
\end{enumerate}
We start with an orthonormal basis for tangent vectors
\begin{equation}
e_{a}={e_{a}}^{\mu}\partial_{\mu}.
\end{equation}
This makes sense only if we already have a metric, but
orthonormality demands
\begin{equation}
g(e_{a},e_{b})=\eta_{ab}
\end{equation}
where $\eta_{ab}=\diag(1,-1,-1,-1)$.

The covariant derivative
\begin{equation}\label{eq:lec12:covariantDerivativeOfTetrad}
\nabla_{\mu}{e_{a}}^{\rho}={{\omega_{\mu}}^{c}}_{a}{e_{c}}^{\rho}
\end{equation}
where the index $a$ labels which vector we're discussing, $\rho$
labels the component of the vector we're discussing. This
$\omega$ is called the \define{Spin Connection}, the term
originated from trying to understand particle spin in general
relativity.

Lets examine the metric compatibility condition first
(torsion-free is trivial). First observe in components
\begin{equation}
g(e_{a},e_{b})=g_{\mu\nu}e^{\mu}_{a}e^{\nu}_{b}=\eta_{ab}
\end{equation}
so the metric compatibility condition becomes
\begin{subequations}
\begin{align}
\nabla_{\mu}g_{ab}
&=0\\
&=\partial_{\mu}g_{ab}-
({{\omega_{\mu}}^{c}}_{a}g_{cb}+{{\omega_{\mu}}^{c}}_{b}g_{ac})\\
&=0-\omega_{\mu ba}-\omega_{\mu ab}
\end{align}
\end{subequations}
Metric compatibility implies antisymmetry in spin connection's
orthonormal basis indices; this critically depends on
$\partial_{\mu}g_{ab}=0$. Thus metric compatibility implies
\begin{equation}
\omega_{\mu ab}=-\omega_{\mu ba}.
\end{equation}
Now let us consider the torsion free condition.

Remember the spin connection is a connection in an orthonormal
basis satisfies
\begin{equation}%\tag{\ref{eq:lec12:covariantDerivativeOfTetrad}}
\nabla_{\mu}{e^{a}}_{\rho}={\omega_{\mu}}^{ca}e_{c\rho}
\end{equation}
by the fact that the metric vanishes under covariant
differentiation. In a theory with nonmetricity, we would get an
extra term. This is antisymmetric, so
\begin{equation}
\nabla_{\mu}{e^{a}}_{\rho}=-{{\omega_{\mu}}^{a}}_{c}{e^{c}}_{\rho}
\end{equation}
We can write the frame as a one-form:
\begin{equation}\label{eq:lec12:tetradOneForm}
e^{a}={e^{a}}_{\rho}\,\D x^{\rho}
\end{equation}
Observe
\begin{equation}
\nabla_{\mu}{e^{a}}_{\rho}=\partial_{\mu}{e^{a}}_{\rho}
-{{\Gamma_{\mu}}^{\sigma}}_{\rho}{e^{a}}_{\sigma}
\end{equation}
where $a$ just labels which vector we're talking about. We can
rewrite this as
\begin{equation}
{{\Gamma_{\mu}}^{\sigma}}_{\rho}{e^{a}}_{\sigma}=\partial_{\mu}{e^{a}}_{\rho}
-\nabla_{\mu}{e^{a}}_{\rho}.
\end{equation}
This lets us translate from $\Gamma$ to $\omega$ given
$e$. Observe
\begin{equation}
(\Gamma^{\sigma}_{\mu\rho}-\Gamma^{\sigma}_{\rho\mu}){e^{a}}_{\sigma}=0
\end{equation}
is the torsion-free condition. This implies
\begin{equation}
\partial_{\mu}{e^{a}}_{\rho}-\partial_{\rho}{e^{a}}_{\mu}
+{{\omega_{\mu}}^{a}}_{c}{e^{c}}_{\rho}
-{{\omega_{\rho}}^{a}}_{c}{e^{c}}_{\mu}=0
\end{equation}
by substitution. If we use the tetrad one-form from Equation
\eqref{eq:lec12:tetradOneForm} we get something slick:
\begin{equation}
\boxed{
\D e^{a}+{\omega^{a}}_{c}\wedge e^{c}=0.
}
\end{equation}
Thus we have simply the same conditions as
\begin{enumerate}
\item Torsion-free\quad $\D e^{a}+{\omega^{a}}_{c}\wedge e^{c}=0$.
\item Metric Compatible\quad $\omega_{\mu ab}+\omega_{\mu ba}=0$.
\end{enumerate}
\begin{rmk}\index{Cartan Structure Equation!First ---}
The ``First Cartan Structure Equation'' is
\quad $\D e^{a}+{\omega^{a}}_{c}\wedge e^{c}=0$.
\end{rmk}
\marginpar{Applications}
\begin{ex}[2-Sphere]
Lets recall the usual sphere $S^{2}\subset\RR^{3}$, it has its
line element be
\begin{equation}
\D s^{2}=\D\theta^{2}+\sin^{2}(\theta)\,\D\varphi^{2}.
\end{equation}
We interpret the ``$\D$''s here as one-forms. So we obtain
\begin{equation}
\begin{split}
\D s^{2}
&=g_{\mu\nu}\D x^{\mu}\otimes\D x^{\nu}\\
&=g_{ab}({e^{a}}_{\mu}\,\D x^{\mu})\otimes({e^{b}}_{\nu}\,\D x^{\nu})
\end{split}
\end{equation}
We can immediately read off an orthonormal basis
\begin{equation}
e^{1}=\D\theta,\quad
e^{2}=\sin(\theta)\,\D\varphi.
\end{equation}
We can rotate and take linear combinations if we want a new
orthonormal basis. Finding an orthonormal basis is very much like
completing a square.

The spin connection has only one-component:
\begin{equation}
\omega_{12}=-\omega_{21}=\omega.
\end{equation}
The Cartan structure has two components
\begin{subequations}
\begin{align}
\D e^{1}+{\omega^{1}}_{2}\wedge e^{2}
&=0\\
&=\D(\D\theta)+\omega\wedge(\sin(\theta)\,\D\varphi)\\
&=0+(\omega\wedge\D\varphi)\sin(\theta)
\end{align}
\end{subequations}
which implies
\begin{equation}
\D\varphi\wedge\omega=0.
\end{equation}
We expect the general solution should look like
\begin{equation}
\omega=A\,\D\varphi
\end{equation}
The second structure equation yields
\begin{subequations}
\begin{align}
\D e^{2}+{\omega^{2}}_{1}\wedge e^{1}
&=0\\
&=\D(\sin(\theta)\,\D\varphi)+\omega\wedge\D\theta\\
&=\cos(\theta)\,\D\theta\wedge\D\varphi
+\sin(\theta)\,\D\varphi\wedge\D\varphi
-\omega\wedge\D\theta\\
&=\cos(\theta)\,\D\theta\wedge\D\varphi+0-\omega\wedge\D\theta
\end{align}
\end{subequations}
So, if we gather terms together we find
\begin{equation}
-\cos(\theta)\,\D\varphi\wedge\D\theta-\omega\wedge\D\theta=0
\end{equation}
and thus
\begin{equation}
-\bigl(
\cos(\theta)\,\D\varphi+\omega
\bigr)\wedge\D\theta=0.
\end{equation}
This tells us
\begin{equation}
\omega=-\cos(\theta)\,\D\varphi.
\end{equation}
Observe how easy it was to find these components, compared to the
approach using the metric components.
\end{ex}
\begin{ex}[Simple Cosmology]
Lets consider a simple metric
\begin{equation}
\D s^{2}=\D t^{2}-a(t)^{2}\bigl(\D x^{2}+\D y^{2}+\D z^{2}\bigr)
\end{equation}
We can choose a basis of one forms quite simply:
\begin{equation}
\begin{split}
e^{0} &= \D t\\
e^{i} &= a(t)\,\D x^{i},
\end{split}
\end{equation}
then apply Cartan's equation
\begin{equation}
\begin{split}
\D e^{0}+{\omega^{0}}_{i}\wedge e^{i}
&=0\\
&=a(t){\omega^{0}}_{i}\wedge e^{i}
\end{split}
\end{equation}
This tells us that ${\omega^{0}}_{i}$ does not have any $\D t$'s
in it, so
\begin{equation}
{\omega^{0}}_{i}=A_{ij}\,\D x^{j}.
\end{equation}
We then have
\begin{equation}
A_{ij}\,\D x^{i}\wedge\D x^{j}=0\implies
A_{ij}=A_{ji}.
\end{equation}
Now, we \textbf{GUESS} that
\begin{equation}
A_{ij}=A\delta_{ij}
\end{equation}
since this is the simplest symmetric tensor. Now we consider the
other part of Cartan's structure equation
\begin{equation}\label{eq:lec12:ex2:cartanStruct}
\D e^{i}+{\omega^{i}}_{j}\wedge e^{j}+{\omega^{i}}_{0}\wedge
e^{0}=0.
\end{equation}
We see
\begin{subequations}
\begin{align}
\D e^{i} &= \D a(t)\wedge \D x^{i}\\
&= \dot{a}(t) \D t\wedge\D x^{i}
\end{align}
\end{subequations}
and plug this back into Equation \eqref{eq:lec12:ex2:cartanStruct}
\begin{equation}
0=\dot{a}\,\D t\wedge\D x^{i}
+{\omega^{i}}_{j}\wedge(a\,\D x^{j})+(A\,\D x^{i}\wedge \D t)
\end{equation}
The simplest solution has ${\omega^{i}}_{j}=0$ and $A=\dot{a}$.
It turns out this is precisely the 
Friedmann--Lema\^\i{}tre--Robertson--Walker
metric describing a simple universe.
\end{ex}
