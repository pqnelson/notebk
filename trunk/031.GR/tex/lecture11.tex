%%
%% lecture11.tex
%% 
%% Made by alex
%% Login   <alex@tomato>
%% 
%% Started on  Wed Feb 29 11:42:55 2012 alex
%% Last update Wed Feb 29 11:42:55 2012 alex
%%
We ended up with an expression for the covariant derivative
\begin{equation}
\nabla_{\mu}v^{a}=\partial_{\mu}v^{a}+
{{\Gamma_{\mu}}^{a}}_{b}v^{b}
\end{equation}
where ${{\Gamma_{\mu}}^{a}}_{b}$ is the corrections to stay on
the manifold. The connection determines the geometry of the
manifold. 

Covariant differentiation commutes with contraction
\begin{equation}
\begin{split}
\nabla_{\mu}(v^{a}w_{a})
&=(\nabla_{\mu}v^{a})w_{a}+v^{a}(\nabla_{\mu}w_{a})\\
&=(\partial_{\mu}v^{a})w_{a}+v^{a}(\partial_{\mu}w_{a})
\end{split}
\end{equation}
to satisfy the Leibniz property for covariant derivatives. This
implies
\begin{equation}
\nabla_{\mu}w_{a}=\partial_{\mu}w_{a}-{{\Gamma_{\mu}}^{b}}_{a}w_{b}
\end{equation}
and we observe
\begin{equation}
\nabla_{\mu}{T^{ab}}_{c}
=
\partial_{\mu}{T^{ab}}_{c}+
{{\Gamma_{\mu}}^{a}}_{d}{T^{db}}_{c}+
{{\Gamma_{\mu}}^{b}}_{d}{T^{ad}}_{c}-
{{\Gamma_{\mu}}^{d}}_{c}{T^{ab}}_{d}.
\end{equation}
Lets consider the second covariant derivative of a function
\begin{equation}
\begin{split}
\nabla_{\mu}\nabla_{\nu}f
&= \nabla_{\mu}(\partial_{\nu}f)\\
&=
\partial_{\mu}\partial_{\nu}f-{{\Gamma_{\nu}}^{\rho}}_{\mu}\partial_{\rho}f. 
\end{split}
\end{equation}
The mixed partials cancel, yielding
\begin{equation}
(\nabla_{\mu}\nabla_{\nu}-\nabla_{\nu}\nabla_{\mu})f
=-({{\Gamma_{\mu}}^{\rho}}_{\nu}-{{\Gamma_{\nu}}^{\rho}}_{\mu})\partial_{\rho}f
\end{equation}
which is a tensor. We call
\begin{equation}
{{\Gamma_{\mu}}^{\rho}}_{\nu}-{{\Gamma_{\nu}}^{\rho}}_{\mu}=T^{\rho}_{\mu\nu}
\end{equation}
the \define{Torsion Tensor}. In Riemannian geometry, this is
equal to zero.

The next thing to look at is the covariant derivative of the
metric
\begin{equation}
\begin{split}
\nabla_{\rho}g_{\mu\nu}
&=\partial_{\rho}g_{\mu\nu}-{{\Gamma_{\rho}}^{\sigma}}_{\nu}g_{\mu\sigma}
-{{\Gamma_{\rho}}^{\sigma}}_{\mu}g_{\sigma\nu}\\
&=0
\end{split}
\end{equation}
where we obtain the second line through using metric
compatibility. Physically this means if we take the inner product
of two vectors, then move them along a geodesic, the inner
product should be invariant.

\index{Weyl!Unified Field attempt|(}
What if we drop this? We get\marginpar{Nonmetricity}
\begin{equation}
\nabla_{\rho}g_{\mu\nu}=K_{\rho\mu\nu}
\end{equation}
which is called \define{Nonmetricity}. There is one appealing
version of nonmetricity that Hermann Weyl introduced. Suppose we
require
\begin{equation}
\nabla_{\rho}g_{\mu\nu}=A_{\rho}g_{\mu\nu}
\end{equation}
where $A_{\rho}$ is the electromagnetic 4-potential. This yields
a tensor that looks like the field-strength tensor. But there is
a problem since the length of a tensor is history-dependent, but
this is observably untrue (e.g., the frequency of the photon from
a Hydrogen atom's electron changing orbitals).

Take two vectors $v^{\mu}$, $w^{\mu}$ and define a curve
$x^{\mu}$ with tangent vector
\begin{equation}
\frac{\D x^{\mu}}{\D s}=u^{\mu}.
\end{equation}
Observe, since we are moving along the curve
\begin{equation}
\frac{\D}{\D s}(g_{\mu\nu}v^{\mu}w^{\nu})=u^{\rho}\partial_{\rho}(g_{\mu\nu}v^{\mu}w^{\nu})
\end{equation}
and since this is the derivative of a scalar invariant, we have
\begin{equation}
u^{\rho}\partial_{\rho}(g_{\mu\nu}v^{\mu}w^{\nu})=
u^{\rho}\nabla_{\rho}(g_{\mu\nu}v^{\mu}w^{\nu}).
\end{equation}
Using the Leibniz rule yields
\begin{equation}
u^{\rho}\nabla_{\rho}(g_{\mu\nu}v^{\mu}w^{\nu})
=(u^{\rho}\nabla_{\rho}g_{\mu\nu})v^{\mu}w^{\nu}
+g_{\mu\nu}u^{\rho}\bigl(
(\nabla_{\rho}v^{\mu})w^{\nu}+
v^{\mu}(\nabla_{\rho}w^{\nu})
\bigr).
\end{equation}
If we move in such a way that\marginpar{Parallel Propagation}
\begin{equation}
u^{\rho}\nabla_{\rho}v^{\mu}=0,
\end{equation}
called ``parallel propagation,'' the change in the inner product
comes from the $\nabla g_{\mu\nu}$ term. The Weyl inner product
gives us
\begin{equation}
\frac{\D}{\D s}(g_{\mu\nu}v^{\mu}v^{\nu})=(u^{\rho}A_{\rho})(g_{\mu\nu}v^{\mu}v^{\nu})
\end{equation}
Observe this is how it changes ``infinitesimally'' along the
path. The total magnitude $\ell$ changes as
\begin{equation}
\ell^{2}\to\ell^{2}\exp\left(\int\! A_{\rho}\,\D x^{\rho}\right)
\end{equation}
It turns out, if we stick an $\I=\sqrt{-1}$ into Weyl's idea, we
recover some notions in quantum field theory (e.g., phase
shifting the Dirac field, etc.). For more historical details, see Straumann~\cite{Straumann:2005hj}.
\index{Weyl!Unified Field attempt|)}

Lets assume we have a Torsion-free, metric-compatible
connection. If we set the torsion to zero, we uniquely get
\begin{equation}
\Gamma^{\rho}_{\mu\nu}=\frac{1}{2}g^{\rho\sigma}
(\partial_{\mu}g_{\sigma\nu}+\partial_{\nu}g_{\mu\sigma}
-\partial_{\sigma}g_{\mu\nu})
\end{equation}
which is the \define{Christoffel Connection}. The geodesic
equation can be written as
\begin{equation}
\frac{\D^{2}x^{\rho}}{\D s^{2}}
+\Gamma^{\rho}_{\mu\nu}\frac{\D x^{\mu}}{\D s}\frac{\D x^{\nu}}{\D s}
=0
\end{equation}
If it turns out the nonmetricity is nonzero, we can write down
the equation with all the nonmetricity as well as all the metric
compatible components. This gives us two sets of paths that are different.

If we set
\begin{equation}
u^{\mu}=\frac{\D x^{\mu}}{\D s}
\end{equation}
we have the geodesic equation becoming\marginpar{Autoparallel transport}
\begin{equation}
u^{\rho}\nabla_{\rho}u^{\mu}=0
\end{equation}
which is in some sense the derivative of the tangent is zero (or,
in other words, ``it remains parallel to itself''). This is
precisely the condition of autoparallel transport. The two
definitions of autoparallel and shortest distance become
inequivalent for nonmetricity situations. 

\begin{prop}
Let $g=\det|g_{\mu\nu}|$, then ${\Gamma^{\rho}}_{\mu\rho}=(\sqrt{-g})^{-1}\partial_{\mu}\sqrt{-g}$.
\end{prop}
\begin{prop}
For any vector $v^{\mu}$ we have $\sqrt{-g}\nabla_{\mu}v^{\mu}=\partial_{\mu}(\sqrt{-g}v^{\mu})$.
\end{prop}
