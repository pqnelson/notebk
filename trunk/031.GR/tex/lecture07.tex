%%
%% lecture07.tex
%% 
%% Made by alex
%% Login   <alex@tomato>
%% 
%% Started on  Thu Feb 23 11:35:36 2012 alex
%% Last update Thu Feb 23 11:35:36 2012 alex
%%
So, some things worth knowing:
\begin{enumerate}
\item Vectors are defined independent of coordinates.
\item We can represent a vector as a directional derivative
  $v=v^{\mu}\partial_{\mu}$ which is also independent of
  coordinates.
\item The coordinate basis $\partial_{\mu}$, we can have an
  arbitrary basis $e_{a}={e_{a}}^{\mu}\partial_{\mu}$ which is a
  linear combination of basis vectors, and called a
  \define{Frame}\index{Frame} or
  \define{Vierbein}\index{Vierbein} in 4-dimensions (or tetrad).
\end{enumerate}

\begin{xca}
Prove in polar coordinates we have
$(\partial_{r},\partial_{\theta})$ is a coordinate basis, and
$(\partial_{r},r^{-1}\partial_{\theta})$ is a basis but not a
coordinates basis.
\end{xca}

A vector field is a map $\mathcal{M}\to\mathrm{T}\mathcal{M}$
such that at each point $p\in\mathcal{M}$ we assign to it a
tangent vector $v_{p}$ in a ``smooth way''. In other fields of
physics, we may work with a ``tangent spinor'' or something
similar. Assigning such a gadget to each point in spacetime is
really a \define{Section} of a fiber bundle (and doing it in such
a way that we have a ``tangent spinor'' requires something more,
something called a \define{Solder Form}).

\begin{ddanger}
Although it appears straightforward to generalize from tangent
vector to tangent spinor to an arbitrary tangent \emph{gadget},
there is dangerous subtlety here! There may be
\emph{obstructions} to such generalizations, we require tools
from algebraic topology to study such obstructions. See
Hatcher~\cite[\normalfont\S4.3]{hatcher1} and~\cite{hatcher2} for
topological aspects, and Sharpe~\cite{sharpe} for geometric
aspects. There are some very serious applications to physics
(e.g., involving Dirac operators), since Lorentzian manifolds
have particularly unique topology. 
\end{ddanger}

We should warn the reader, there are three competing notations
used in general relativity. We use Greek indices when referring
to components in a coordinate basis, and Latin indices give
components in an arbitrary basis, but later Latin indices in the
middle of the alphabet ($i$, $j$, $k$, \dots) gives spatial
components in a coordinate basis.

Now what is an example of a generalization we have discussed?
Well, quite simple: the notion of a \define{Covector} (a.k.a.,
covariant vector, dual vector, one-form, etc.). It is an element
in the dual space to $\mathrm{T}_{p}\mathcal{M}$, denoted
$\mathrm{T}^{*}_{p}\mathcal{M}$ and called the \define{Cotangent Space}.
We indicate the cotangent bundle as $\mathrm{T}^{*}\mathcal{M}$,
and it consists of all the covectors on $\mathcal{M}$.

\begin{ex}
Given a manifold $\mathcal{M}$ and a function
$f\colon\mathcal{M}\to\RR$, then the \define{Gradient} of $f$ is
$\D f\in \mathrm{T}^{*}\mathcal{M}$. The directional derivative
is
\begin{equation}
\D f[v]=v(f)
\end{equation}
where $v\in\mathrm{T}\mathcal{M}$. More generally, if
\begin{equation}
\D f=\omega_{\mu}\D x^{\mu},
\end{equation}
then
\begin{subequations}
\begin{align}
\<\D f|\partial_{\nu}\>
&=\partial_{\nu} f
\end{align}
but also
\begin{align}
\<\D f|\partial_{\nu}\>
&=\<\omega_{\mu}\D x^{\mu}|\partial_{\nu}\>\\
&=\omega_{\mu}\<\D x^{\mu}|\partial_{\nu}\>\\
&=\omega_{\mu}{\delta^{\mu}}_{\nu}\\
&=\omega_{\nu}
\end{align}
\end{subequations}
and setting equals to equals yields
\begin{equation}
\omega_{\nu}=\partial_{\nu} f.
\end{equation}
This implies
\begin{equation}
\D f=(\partial_{\mu}f)\D x^{\mu}
\end{equation}
as before.
\end{ex}
