%%
%% poincareRep.tex
%% 
%% Made by Alex Nelson
%% Login   <alex@tomato>
%% 
%% Started on  Wed Jul 22 12:09:17 2009 Alex Nelson
%% Last update Wed Jul 22 12:09:17 2009 Alex Nelson
%%

The Lorentz transformation is usually ``represented'' by a matrix
$\Lambda$ which, when written explicitly, is
\begin{equation}%\label{eq:}
(\Lambda x)^{\mu} = {\Lambda^{\mu}}_{\nu}x^{\nu}
\end{equation}
where Einstein convention is used (implicit sum over $\nu$
occurs). We have that the matrix ${\Lambda^{\mu}}_{\nu}$ must
satisfy
\begin{equation}%\label{eq:}
{\Lambda^{\lambda}}_{\mu}{\Lambda_{\lambda}}_{\nu} = \eta_{\mu\nu}
\end{equation}
where $\eta_{\mu\nu}$ is the Minkowski metric (metric for flat
spacetime).

Now, the Poincar\'e group is the set of Lorentz transformations
and space-time translations, so the element of the group would be
$(\Lambda,a)$ such that
\begin{equation}%\label{eq:}
x^{\mu}\to y^{\mu} = {\Lambda^{\mu}}_{\nu}x^{\nu}+a^{\mu}.
\end{equation}
The group multiplication law is then just
\begin{equation}%\label{eq:}
(\Lambda_{2},a_{2})(\Lambda_{1},a_{1}) = (\Lambda_{2}\Lambda_{1},\Lambda_{2}a_{1}+a_{2}).
\end{equation}
We are interested in irreducible unitary representations $U(\cdot)$ of our
group, all we need to worry about are the generators.

The translations, rotations, and boosts of the Poincar\'e group
must act on the space of states. A Poincar\'e group element $g$
acts as a unitaruy operator $U(g)$ on the state space. The action
must satisfy a multipication condition
\begin{equation}%\label{eq:}
U(gh)=U(g)U(h)
\end{equation}
for all $g,h$ in the Poincar\'e group.

Translation of spacetime by a four-vector $a^{\mu}$ is defined by
\begin{equation}%\label{eq:}
\Delta_{a}(x) = x+a.
\end{equation}
Translation of a state $\psi$, on the other hand, should be
moving the graph by $a$. This means that
$\Delta_{a}\psi(x)=\psi(x-a)$. The unitary representation
$U(\Delta_{a})$ of $\Delta_{a}$ must thus be defined by
\begin{equation}%\label{eq:}
U(\Delta_{a})|\psi\>=|\Delta_{a}\psi\>.
\end{equation}
We'd like to find an expression for $U(\Delta_a)$ in terms of the
energy-momentum four-vector $\widehat{p}_{\mu}$.

Evolution in time is translation of the observer forward in time,
or (equivalently) translation of the system backwards in time:
\begin{equation}%\label{eq:}
\exp(-it\widehat{H})|\psi(x)\> = |\psi(x_{0}+t,\overline{x})\>.
\end{equation}
Let $\tau^{\mu}=(-t,\vec{0})$ be a four-vector, then we can
rewrite our translation in time as
\begin{equation}%\label{eq:}
\exp(i\tau^{\mu}\widehat{p}_{\mu})|\psi\> = |\Delta_{\tau}\psi\>.
\end{equation}
Lorentz invariance implies that this equation is true whenever
$\tau$ is timelike, and the additivity of translations then shows
this to be true for all four-vectors $\tau$. From this definition
of $U(\Delta_{a})$ we can therefore deduce that
\begin{equation}%\label{eq:}
U(\Delta_a) = \exp(ia^{\mu}\widehat{p}_{\mu}).
\end{equation}

Although this unitary representation is derived in the
position-space formulation of quantum mechanics, it works equally
well in the momentum-space formulation. We can deduce that the
unitary representation of translations on momentum eigenstates is
given by
\begin{equation}%\label{eq:}
U(\Delta_{a})|\overline{k}\> =
\exp(ia^{\mu}\widehat{p}_{\mu})|\overline{k}\> = \exp(ia^{\mu}k_{\mu})|\overline{k}\>
\end{equation}
where $k_{0} = \omega(\overline{k})$.

\begin{rmk}
Recall Taylor's theorem in real analysis can be formulated as
\begin{equation}%\label{eq:}
f(x+h) =
\left(\sum_{n=0}^{\infty}h^{n}\frac{d^{n}}{dx^{n}}\right)f(x) = \exp\left(h\frac{d}{dx}\right)f(x)
\end{equation}
which should look familiar: we just deduced the unitary
representation of spacetime translations should be
\begin{equation}%\label{eq:}
\exp(i\tau^{\mu}\widehat{p}_{\mu})|\psi\> = U(\Delta_{\tau})|\psi\>.
\end{equation}
If we don't distinguish $|\psi\>$ from $\psi(x)$, we see that
Taylor's theorem guarantees our representation to be of spacetime
translations.
\end{rmk}
