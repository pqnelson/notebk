%%
%% gravityInfo.tex
%% 
%% Made by Alex Nelson
%% Login   <alex@tomato3>
%% 
%% Started on  Wed Jan 13 15:58:23 2010 Alex Nelson
%% Last update Wed Jan 13 16:55:30 2010 Alex Nelson
%%
\documentclass{article}
\usepackage{notebk}
\let\entropy=S
\title{Notes on {\tt arXiv:1001.0785v1 [hep-th]}}
\date{January 13, 2010}
\begin{document}
\maketitle
\tableofcontents

\section{Entropic Force}

\begin{defn}
An \define{Entropic Force} is an effective macroscopic force that
originates in a system with many degrees of freedom by the
statistical tendency to increase its entropy.
\end{defn}

We express the force equation in terms of entropy differences.

\begin{prop}
An entropic force is independent of the details of the
microscopic dynamics.
\end{prop}

\begin{prop}
No fundamental field is associated with an entropic force.
\end{prop}

\begin{ex}
The elasticity of a polymer molocule is an entropic force. A
single polymer molecule can be modeled as many monomers of fixed
length, where each monomer can rotate freely around the points of
attachment and direct itself in any spatial direction. Each of
these configurations has the same energy. When the polymer
molecule is immersed into a heat bath, it puts itself into a
randomly coiled configuration (as these are entropically
favored). There more configurations when the molecule is short,
compared to when it is stretched to an extended
configuration. The tendency to return to a maximal entropy states
translates into a macroscopic force, which is the elastic force.
\end{ex}

\begin{defn}
For a given system, the \define{Entropy} equals
\begin{equation}
S(E,x)=k_{B}\log\Omega(E,x)
\end{equation}
where $k_{B}$ is Boltzmann's constant and $\Omega(E,x)$ denotes
the volume of the configuration space for the entire system.
\end{defn}

\section{Emergence of the Laws of Newton}

Space is seen as ``literally just a storage space for
information'' (6). It turns out to be ``naturally'' associated
with matter. There is an interesting discussion that the maximal
allowed information is finite for each part of space, implying
that it is impossible to localize a particle with infinite
precision at a point of a continuum space; ``in fact'' points and
coordinates arise as derived concepts. ``One could assume that
information is stored in points of a discretized space'' which
has interesting applications with respect to causal sets.

Information is assumed to be stored on surfaces, called
``screens'' apparently. Screens seperate points, and ``thus'' are
``the natural place'' to store information about particles that
move from one side to the other.

\subsection{Force and Inertia}

In analogy to Bekenstein's derivation of black hole
entropy~\cite{Bekenstein:1973ur}, we will consider entropy
related to surface area in flat spacetime. We assume that the
change of entropy associated with the information on the boundary
equals
\begin{equation}
\Delta\entropy=2\pi k_{B}\qquad\hbox{when}\qquad\Delta x=\frac{\hbar}{mc}.
\end{equation}
The reason for the factor of $2\pi$ ``will become apparent
soon.'' Suppose that the change in entropy is linear in $\Delta
x$, that is to say
\begin{equation}
\Delta\entropy=2\pi k_{B}\frac{mc}{\hbar}\Delta x.
\end{equation}

\noindent$\langle$\emph{N.B. for an early proposal for a Goldstone graviton from
Lorentz-Invariance breaking, see
Ohanian~\cite{PhysRev.184.1305}. For more modern ideas, see \cite{Jenkins:2003hw}.}$\rangle$

Recall Unruh demonstrated that an observer in an accelerated
frame experiences a temperature
\begin{equation}
k_{B}T=\frac{1}{2\pi}\frac{\hbar a}{c}
\end{equation}
where $a$ is the acceleration. We see then that
\begin{subequations}
\begin{align}
T &= \frac{1}{2\pi}\frac{\hbar a}{ck_{B}}\\
T\Delta\entropy &= ma\Delta x\\
&=F\Delta x
\end{align}
\end{subequations}
Thus we get our expression
\begin{equation}
T\Delta\entropy = F\Delta x
\end{equation}
relating entropy to force, tacitly deriving Newton's second law.

\nocite{*}
\bibliographystyle{elements}
\bibliography{gravityInfo}
\end{document}
