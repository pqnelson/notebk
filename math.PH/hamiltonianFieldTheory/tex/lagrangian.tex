%%
%% lagrangian.tex
%% 
%% Made by Alex Nelson
%% Login   <alex@black-cherry>
%% 
%% Started on  Wed Aug 26 12:02:16 2009 Alex Nelson
%% Last update Wed Aug 26 12:02:16 2009 Alex Nelson
%%

As with Hamiltonian mechanics, wherein one begins by taking the
Legendre transform of the Lagrangian, in Hamiltonian field theory
we ``transform'' the Lagrangian field treatment. So lets review
the calculations in Lagrangian field theory.

Consider the classical fields $\phi^{a}(t,\bar{x})$. We use the
index $a$ to indicate which field we are talking about. We should
think of $\bar{x}$ as another index, except it is \emph{continuous}.
We will use the confusing short hand notation $\phi$ for the
column vector $\phi^{1},\ldots,\phi^{n}$. Consider the Lagrangian
\begin{equation}%\label{eq:}
L(\phi) = \int_{\mathclap{\text{all space}}}\mathcal{L}(\phi,\partial_{\mu}\phi)d^{3}\bar{x}
\end{equation}
where $\mathcal{L}$ is the \emph{Lagrangian density}. Hamilton's
principle of stationary action is still used to determine the
equations of motion from the action
\begin{equation}%\label{eq:}
S[\phi] = \int L(\phi,\partial_{\mu}\phi)dt
\end{equation}
where we find the Euler-Lagrange equations of motion for the field
\begin{equation}%\label{eq:}
\frac{d}{dx^{\mu}}\frac{\partial\mathcal{L}}{\partial (\partial_{\mu}\phi^{a})}=\frac{\partial\mathcal{L}}{\partial\phi^{a}}
\end{equation}
where we note these are evil second order partial differential
equations. We also note that we are using Einstein summation
convention, so there is an implicit sum over $\mu$ but not over
$a$. So that means there are $n$ independent second order
partial differential equations we need to solve.

But how do we really know these are the correct equations? How do
we really know these are the Euler-Lagrange equations for
classical fields?  We can obtain it directly from the action $S$
by functional differentiation with respect to the field. Taking
$\phi$ to be a single scalar field (for simplicity's sake, it
doesn't change anything if we work with $n$ fields), functional
differentiation can be defined by
\begin{equation}%\label{eq:}
\frac{\delta S}{\delta\phi(x)}\eqdef
\lim_{\varepsilon\to0}\frac{1}{\varepsilon}\left(S\left[\phi(y)+\varepsilon\delta^{(4)}(x-y)\right]-S\left[\phi(y)\right]\right)
\end{equation}
where $\delta^{(4)}(y-x)$ is the 4-dimensional densitized Dirac
delta function. Note that we will often use the shorthand
notation $\delta^{(4)}_{x}=\delta^{(4)}(y-x)$. Applying this to the action yields
\begin{subequations}
\begin{align}
\frac{\delta S}{\delta\phi(x)} &= \lim_{\varepsilon\to0}\frac{1}{\varepsilon}\int\left[\mathcal{L}\left(\phi+\varepsilon\delta^{(4)}_{x},\partial_{\mu}\phi+\varepsilon\partial_{\mu}\delta^{(4)}_{x}\right)-\mathcal{L}(\phi,\partial_{\mu}\phi)\right]d^{4}y\\
&=\lim_{\varepsilon\to0}\frac{1}{\varepsilon}\int\left[\mathcal{L}(\phi,\partial_{\mu}\phi)+\frac{\partial\mathcal{L}}{\partial\phi}\delta^{(4)}_{x}\varepsilon+\frac{\partial\mathcal{L}}{\partial(\partial_{\mu}\phi)}\partial_{\mu}\delta^{(4)}_{x}\varepsilon+\mathcal{O}(\varepsilon^{2})-\mathcal{L}(\phi,\partial_{\mu}\phi)\right]d^{4}y\\
&= \lim_{\varepsilon\to0}\int\left[\frac{\partial\mathcal{L}}{\partial\phi}\delta^{(4)}_{x}+\frac{\partial\mathcal{L}}{\partial(\partial_{\mu}\phi)}\partial_{\mu}\delta^{(4)}_{x}+\mathcal{O}(\varepsilon)\right]d^{4}y\\
&=\int\left[\frac{\partial\mathcal{L}}{\partial\phi}\delta^{(4)}_{x}+\frac{\partial\mathcal{L}}{\partial(\partial_{\mu}\phi)}\partial_{\mu}\delta^{(4)}_{x}\right]d^{4}y\\
&=\int\left[\frac{\partial\mathcal{L}}{\partial\phi}-\partial_{\mu}\frac{\partial\mathcal{L}}{\partial(\partial_{\mu}\phi)}\right]\delta^{(4)}_{x}d^{4}y
\end{align}
\end{subequations}
Where we justify the second line by Taylor expanding to first
order, then in the third line we factor through by the
$(1/\varepsilon)$ factor, in the fourth line we take the limit,
and integrate by parts to yield the last line. Note also that we
factored out the delta function to make the last line
prettier. Now the last line is zero if and only if
\begin{equation}%\label{eq:}
\frac{\partial\mathcal{L}}{\partial\phi^{a}}-\frac{d}{dx^{\mu}}\frac{\partial\mathcal{L}}{\partial (\partial_{\mu}\phi^{a})}=0
\end{equation}
which is precisely the Euler-Lagrange equations of motion!

\begin{ddanger}
Why are we working with these delta functions? Well, we are
working  with something a little more than just time. We are
working with points in space. Locality means, mathematically, we
work with vectors sharing the same base point. Or in the jargon
of differential geometry, we are working in the tangent space
$T_{p}\mathcal{M}$ where $\mathcal{M}$ is our manifold, and
$p\in\mathcal{M}$ is our base point. If we work with multiple
base points at a time, not only is it mathematically not well
defined, but it is \emph{nonlocal} which results in \emph{a loss of causality!}
Needless to say this is bad, so we try to work with the evolution
of the field at a specified (but arbitrary) tangent space. If
time permits, we will revisit this notion of spatial coordinates
as an ``index'' in the appendix.

More precisely, we have the ``base space'' be the manifold
$\mathcal{M}$ representing spacetime. We have our fields assign
to each point $p\in\mathcal{M}$ some ``physical information''
$\phi(x)$. The question presents itself ``Where does this
`information' live?'' It lives in a generalization of the tangent
space, called the ``fiber''. In the Lagrangian setting, we work
with ordered pairs $(x^{\mu},\phi(x^{\mu}))$ which is actually
something called the ``section'' of the fiber bundle. That is,
the field is a mapping
\begin{equation}%\label{eq:}
\phi:\mathcal{M}\to\mathcal{M}\times{F}
\end{equation}
where $F$ is ``where'' the fields ``live'', i.e. it's the fiber.
\end{ddanger}

