%%
%% 30April2008.tex
%% 
%% Made by Alex Nelson
%% Login   <alex@tomato>
%% 
%% Started on  Sun Dec 21 14:00:22 2008 Alex Nelson
%% Last update Sun Dec 21 14:00:22 2008 Alex Nelson
%%
%% We cover two things in this note, we 1) finish the
%% sturm-liouville problem; 2) apply the regular
%% sturm-liouville problem to solve the heat equation
%\subsection[Existence Theorem for Sturm-Liouville]{Existence of Eigenfunctions for the Regular Sturm-Liouville Problem}
\subsection{Existence of Eigenfunctions for the Regular Sturm-Liouville Problem}

\begin{thm}\label{thm:30April2008:thm3.10}
For the regular Sturm-Liouville problem,
\begin{enumerate}%[a)]
\item There is ano orthonormal basis $\{\phi_n\}^{\infty}_{1}$ of
$L^{2}_{\omega}(a,b)$ consisting of the eigenfunctions of
the regular Sturm-Liouville problem.
\item The eigenvalues satisfy  $\lim_{n\to\infty}\lambda_{n}=\infty$
\item If $f\in C^{2}(a,b)$ and $f$ satisfies the boundary
  conditions $B_1(f)=B_2(f)=0$, then
\begin{equation*}
\sum\<f,\phi_n\>_{\omega}\phi_{n}\to f\text{  uniformly}
\end{equation*}
\end{enumerate}
\end{thm}

\textbf{Implications of Theorems \eqref{thm:28April2008:thm3.9} and \eqref{thm:30April2008:thm3.10}:}
Regular Sturm-Liouville problem has a solution $\phi$ if and
only if $\lambda\in\mathbb{R}$ (it has real eigenvalues) and
for only countably many $\lambda_{n}$.

\begin{rmk}\index{Eigenvalues!Determined by Boundary Conditions}
The eigenvalues are determined by the boundary condition
\end{rmk}

\textbf{Applications:} Strategy for solving the heat
diffusion problem is as follows
\begin{align}
\omega(x)\partial_{t}u(x,t) =
\partial_{x}[r(x)\partial_{x}u(x,t)] + p(x)u(x,t) &+
\underbracket[0.5pt]{\text{ 0 }} \\
 & {\text{\small{no external force or source}}} \nonumber
\end{align}
with boundary conditions $B_{1}(u)=B_{2}(u)=0$. This is a
boundary value problem; the initial value problem has
$u(x,0)=f(x)$ which determines the initial distribution of
heat. With these conditions we have an initial value
problem.

\textbf{Step 1)} Apply seperation of variables
$u(x,t)=X(x)T(t)$
\begin{align*}
\Rightarrow T'(t)+\lambda T(t) = 0 &\Rightarrow T(t)=T_{0}e^{-\lambda t}\\
\Rightarrow L(X(x))+\lambda \omega(x)X(x)&=0\text{ (Sturm-Liouville Problem!)}
\end{align*}

\textbf{Step 2)} Find \textbf{all} eigenvalues and
normalized eigenfunctions for the Sturm-Liouville
problem. For each pair $(\lambda_n,\phi_n)$ we get one
solution to boundary value problem
\begin{equation}
u_{n}(x,t) = c_{n}e^{-\lambda_{n}t}\phi_{n}(x)
\end{equation}
so $u(x,t) =
\sum^{\infty}_{n=1}c_{n}\exp(-\lambda_{n}t)\phi_{n}(x)$ is
also a solution to the boundary value problem.

\textbf{Step 3)} Insert the initial value:
\begin{equation}
\sum^{\infty}_{n=1}c_{n}\phi_{n} = u(x,0) = f(x)
\end{equation}
So $c_n = \<f,\phi_n\>_{\omega}$, we conclude the solution
to the initial value problem is
\begin{equation}
u(x,t) =
\sum^{\infty}_{n=1}\<f,\phi_{n}\>_{\omega}e^{-\lambda_{n}t}\phi_{n}(x)
\end{equation}

(In practice, finding the eigenvalues $\lambda$ is always
the hardest part!)

\begin{ex}
Consider the one-dimensional heat flow
\begin{figure}[t]
  \begin{center}
    %%
%% 31March2008img1.tex
%% 
%% Made by Alex Nelson
%% Login   <alex@tomato>
%% 
%% Started on  Wed Dec 17 12:41:25 2008 Alex Nelson
%% Last update Wed Dec 17 12:41:25 2008 Alex Nelson
%% 2.1875in
\setlength{\unitlength}{0.0125in}
\begin{picture}(175,50)(0,-20)
\thinlines
\put(20,20){\circle{9}} % one end of the rod
\put(150,20){\arc{10}{5}{1}} % the other end of the rod
\drawline(0,0)(165,0) % the x axis
\drawline(21,24)(152,25) % the top of the rod
\drawline(21,16)(152,15) % the bottom of the rod
\dottedline{3}(20,16)(20,0) % the dashed bit to $x=0$
\dottedline{3}(152,16)(152,0) % dashed bit to $x=L$
\put(17,-10){\makebox(0,0)[lb]{\raisebox{0pt}[0pt][0pt]{\twltt $0$}}}
\put(147,-10){\makebox(0,0)[lb]{\raisebox{0pt}[0pt][0pt]{\twltt $L$}}}
\put(170,-3){\makebox(0,0)[lb]{\raisebox{0pt}[0pt][0pt]{\twltt $x$}}}
\end{picture}

  \end{center}
\caption{A thin insulated metal rod of length $L$}
\label{fig:30April2008:img1}
\end{figure}
The rod is insulated and of uniform material property. So
\begin{equation}
\partial_{t}u(x,t) = k\partial_{x}^{2}u(x,t)
\end{equation}
where $k$ is the conductance of the material. We have
\begin{equation}
\text{bdry conditions  }\bigg\{\begin{array}{rl}
\partial_{x}u(0,t) &= \alpha u(0,t),\,\,\alpha>0\\
\partial_{x}u(L,t) &= -\alpha u(L,t)
\end{array}
\end{equation}
and our initial value problem is
\begin{equation}
u(x,0) = f(x)\in L^{2}(0,1)
\end{equation}
What's the eigenvalues and eigenfunctions? Well, the
seperation of variables gives us
\begin{equation}
\frac{T'(t)}{T(t)} = \frac{kX''(x)}{X(x)} = \underbracket[0.5pt]{-k\nu^{2}}_{\text{constant}}
\end{equation}
We have two equations
\begin{subequations}
\begin{align}
T'(t)+k\nu^{2}T(t)&=0\\
kX''(x)+k\nu^{2}X(x)&=0
\end{align}
\end{subequations}
with the second equation be such that $X'(0)=\alpha X(0)$,
$X'(L)=-\alpha X(L)$. But the second equation is the regular
Sturm-Liouville problem with $\omega(x)=r(x)=1$,
$p(x)=0$. This is the usual $L^{2}(0,L)$ space.

So what are the eigenvalues $\lambda=\nu^{2}$ and what are
the eigenfunctions? Well, there are two cases:

\textbf{Case 1} $\nu^2=0$ which implies $X''(0)=0$ and
$\alpha X(0)=X'(0)$. So $X(x)=c_1 + c_2$ but $X'(0)=\alpha
X(0)$, thus $c_2 = \alpha c_1$. And $X'(L)=-\alpha X(L)$
implies $c_2 = \alpha(c_1 + c_2 L)$. Thus 
$$2c_2+\alpha c_2L=0=(2+\alpha L)c_2 $$
but $2+\alpha L>0$ so $c_2=0$, which implies trivially
$X(x)=0$.


\textbf{Case 2} We know what Theorem
\eqref{thm:28April2008:thm3.9} says about eigenvalues, so
$\lambda\in\mathbb{R}$. So either \begin{inparaenum}
\item $\nu^2>0$ which implies $\nu>0$%\comment{really? Wouldn't it just imply that $\nu\in\mathbb{R}$ and $\nu\neq0$?}
, \item $\nu^2<0$ which implies $\nu=i\mu$
  where $\mu\in\mathbb{R}$ and $\mu>0$.\end{inparaenum} For
\begin{equation} 
X''(x)+\nu^2X(x) = 0
\end{equation}
the characteristic equation is
\begin{equation}
r^2 + \nu = 0\Rightarrow \nu=\pm i\mu
\end{equation}
So the general solution is
\begin{equation}
X(x) = c_{1}\cos(\nu x) + c_{2}\sin(\nu x)
\end{equation}
By the boundary conditions
\begin{align*}
X'(0) &= \alpha X(0)\\
\Rightarrow c_{2}\nu &= c_{1}\alpha\Rightarrow c_{2} = \frac{\alpha}{\nu}c_{1}
\end{align*}
multiply the equation by $\nu/c_1$ we get
\begin{equation}
X(x) = \nu\cos(\nu x) + \alpha\sin(\nu x)
\end{equation}
which corresponds to eigenvalues of $\nu^2$. So all we need
to do is find $\nu$, so lets apply the second boundary
equation
\begin{subequations}
\begin{align}
X'(L) &= -\alpha X(L)\\
\Rightarrow -\nu^2\sin(\nu L)+\alpha\nu\cos(\nu L) &= -\alpha\left(\nu\cos(\nu L)+\alpha\nu\sin(\nu L)\right)\\
\Rightarrow 2\alpha\nu\cos(\nu L) &= (\nu^2-\alpha^2)\sin(\nu L)\\
\Rightarrow \tan(\nu L) &= \frac{2\alpha\nu}{\nu^2-\alpha^2}
\end{align}
\end{subequations}
\end{ex}

\begin{rmk}
Note that $\nu$ cannot possibly equal $i\mu$ because we can
plug it back in to what we just found to see that
\begin{equation}
\tan(\nu L) = \tan(i\mu L) =
\frac{i2\alpha\mu}{-\mu^2-\alpha^2}
\end{equation}
but from complex analysis we should remember that
\begin{equation}
\tan(i\theta) = i\tanh(\theta) = i\frac{e^{\theta}-e^{-\theta}}{e^{\theta}+e^{-\theta}}
\end{equation}
Thus
\begin{equation}
i\tanh(\mu L) = -i\frac{2\alpha\mu}{\mu^2+\alpha^2}
\end{equation}
but since $\mu>0$, $\tanh(\mu L)>0$ so we have a
contradiction. 
\end{rmk}
