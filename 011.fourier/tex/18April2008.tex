%%
%% 18April2008.tex
%% 
%% Made by Alex Nelson
%% Login   <alex@tomato>
%% 
%% Started on  Sat Dec 20 15:23:28 2008 Alex Nelson
%% Last update Sat Dec 20 15:23:28 2008 Alex Nelson
%%
%% The topics of this file are really just two thing: 1) the
%% normed vector space $L^{2}(a,b)$, its definition and some
%% prospects; 2) a look at $\{\exp(inx)/\sqrt{2\pi}\}$ is an
%% orthonormal basis in $\mathbb{R}^2\cong\mathbb{C}$.

Before beginning just a few definitions on the various types
of convergences.

\begin{defn}
Suppose $\{f_n\}$ is a sequence of functions in $L^2(a,b)$.
\begin{enumerate}
\item We say that $f_n$ converges to $f$ \textbf{pointwise}\index{Convergence!Pointwise}
if for each $x\in[a,b]$,
\begin{equation}
f_{n}(x)\to f(x)\text{ as }n\to\infty
\end{equation}
(.e. $|f_n(x)-f(x)|\to0$ as $n\to\infty$). This means for
any $\varepsilon>0$ and any $x\in[a,b]$ there is an integers
$N_{\varepsilon,x}>0$ which depends on $\varepsilon$ and $x$
such that
\begin{equation}
|f_{n}(x)-f(x)|<\varepsilon\quad\text{for all }n>N_{\varepsilon,x}
\end{equation}
\item We say that $f_n$ converges to $f$ \textbf{uniformly}\index{Convergence!Uniform}
if
\begin{equation}
\sup_{a\leq x\leq b}|f_{n}(x)-f(x)|\to0\text{ as }n\to\infty
\end{equation}
which means for each $\varepsilon>0$ there is an integer
$N>0$ such that
\begin{equation}
|f_{n}(x)-f(x)|<\varepsilon\quad\text{for all }n>N
\end{equation}
which is a stronger condition than pointwise convergence
since $N$ depends only on the choice of $\varepsilon$.
\item We say that $f_{n}$ converges to $f$ \textbf{in norm}\index{Convergence! In Norm}
if
\begin{equation}
\|f_n-f\|\to0\quad\text{as }n\to\infty
\end{equation}
where $\|\cdot\|$ is the norm on the vector space
$L^{2}(a,b)$. By the definition of the norm this means
\begin{equation}
\int^{b}_{a}|f_n(x)-f(x)|^2dx\to 0\quad\text{as }n\to\infty
\end{equation}
\end{enumerate}
\end{defn}

\textbf{Motivation:} We know that
\begin{align*}
\<\frac{1}{\sqrt{2\pi}}e^{imx},\frac{1}{\sqrt{2\pi}}e^{inx}\>
&= \frac{1}{2\pi}\int^{\pi}_{-\pi}e^{i(m-n)x}dx\\
&= \begin{cases}1&m=n\\
0&m\neq n\end{cases}
\end{align*}
So the set $\{\exp(inx)/\sqrt{2\pi}\}^{\infty}_{-\infty}$ is
an orthonormal set. We would like to make it an orthonormal
basis.

Recall in $\mathbb{C}^k$: any set $\{u_1,\ldots,u_k\}$ that
is orthonormal forms a basis for $\mathbb{C}^k$. It has two
properties of relevance
\begin{enumerate}
\item For any $v\in\mathbb{C}^k$, $v=\sum_j
  \<v,u_j\>u_k$. In other words, any vector can be written
  as a linear combination of the vectors $u_j$, $j=1,\ldots,k$.
\item the sum $\sum_{j}\alpha_{j}u_{j}$ (for all
  $\alpha_j\in\mathbb{C}$) forms some vector in $\mathbb{C}^k$.
\end{enumerate}

We want to mirror this for our $L^{2}(a,b)$ space; however,
the second property is not true for any orthonormal set
because $PC(a,b)$ is not ``complete''. 

So we move to a bigger space, $L^{2}$\index{$L^2$}, which is
defined as the set of all functions satisfying
\begin{equation}
L^{2}(a,b) = \{ \text{$f$ on $[a,b]$ s.t. }
\int^{b}_{a}|f(x)|^{2}dx<\infty \}
\end{equation}
where we use the Lebesgue integral\index{Lebesgue Integration}\index{Integration!Lebesgue Integral}. \marginpar{Don't Stress over the Lebesgue
  integral, it really is nothing special...just integration
  done in the way taught in math 21 B, but the only time we
  run into problems is with wacky functions and bizarre
  domains (e.g. domains with only a single point).}What the hell's the
Lebesgue integral, and why do we use it? The simplest
explanation is that it's the area between the $x$-axis and
the curve (one may ask ``But isn't that a Riemann
integral?'' And that's kind of true, but one typically has
vertical rectangles in the Riemann sum, whereas one would
have e.g. vertical rectangles in the Lebesgue integral).

If $f\in PC(a,b)$ then
$\int^{b}_{a}|f(x)|^{2}dx<\infty$...so $f\in L^{2}(a,b)$. So
this implies $PC(a,b)\subset L^{2}(a,b)$. Improper Riemann
integrable functions are also in $L^{2}(a,b)$. Lebesgue
integration is, to a physics undergrad, just ``really clever
integration'' (or integration done in the usual way being
careful about infinities).

\begin{ex}
Let $f(x)=x^{-1/3}$ on $[-1,1]$. This is not piecewise
continuous on $[-1,1]$ since it blows up around $0$. But
observe
\begin{align*}
\int^{1}_{-1}|f(x)|^{2}dx &= \int^{1}_{-1}x^{-2/3}dx\\
&=\lim_{\alpha\to 0^{-}}\int^{\alpha}_{-1}x^{-2/3}dx +
\lim_{\beta\to0^{+}}\int^{1}_{\beta}x^{-2/3}dx\\
&=\lim_{\alpha\to0^{-}}(3\alpha^{1/3}+3)+\lim_{\beta\to0^{+}}(3-3\beta^{1/3})\\
&= 6
\end{align*}
So $f\in L^{2}(-1,1)$.
\end{ex}

\begin{defn}
Let $E\subset\mathbb{R}$, $E$ is \textbf{(Lebesgue) measure
  zero}\index{Lebesgue Measure Zero} if for any
$\varepsilon>0$, there exists a set of intervals
$\{I_1,\ldots\}$ of length $\ell_1$, $\ldots$ such that
\begin{equation}
E\subset\cup_{j=1}I_{j}\quad\text{and
}\sum^{\infty}_{j=1}\ell_{j}<\infty
\end{equation}
Let $\Delta>0$, then $\int^{\Delta}_{0}1dx\neq 0$. For $E$
of measure zero, $\int_{E}1dx=0$.
\end{defn}

\begin{rmk}
Some functions with no Riemann integral have Lebesgue
integral.
\end{rmk}

\begin{ex}
Let $\{r_1,\ldots\}$ be enumeration of the rational numbers,
then it has Lebesgue measure zero. Let
\begin{equation}
f(x) = \begin{cases}1&\text{if $x$ is
    irrational}\\0&\text{otherwise}\end{cases}
\end{equation}
be defined on $[0,1]$. Then $f=0$ on a set of measure
zero. THe Riemann integral of $f$ does not exist, the upper
sum of $f$ is
\begin{equation}
\sum 1 \Delta x
\end{equation}
where $\Delta x$ is the width of the intervals, and the
lower sums are
\begin{equation}
\sum 0\Delta x = 0
\end{equation}
Thus
\begin{equation}
\lim_{\Delta x\to 0}\sum 1\Delta x = \sum 0\Delta
x\Rightarrow 1=0
\end{equation}
which is known to be wrong! However, for the Lebesgue
integral
\begin{equation}
\int^{1}_{0}|f(x)|^{2}dx = 1\Rightarrow f(x)\in L^{2}(0,1).
\end{equation}
\end{ex}

\textbf{Question:} \emph{Can we extend the definition of the
  inner product $\<\cdot,\cdot\>$ from $PC(a,b)$ to
  $L^{2}(a,b)$?}''

Well, we very generally defined the inner product to be
\begin{equation}
\<f,g\> = \int^{b}_{a}f(x)\overline{g(x)}dx
\end{equation}
If $\int f(x)\overline{g(x)}dx$ exists, for all $f,g\in
L^{2}(a,b)$, then $\<\cdot,\cdot\>$ is well defined.

Let $f,g$ be two functions in $L^{2}(a,b)$. Then
\begin{equation}
|\<f,g\>|\leq \int^{b}_{a}|f(x)\overline{g(x)}|dx
\end{equation}
So $|f(x)\overline{g(x)}|\leq|f(x)||g(x)|$ we know for
$s,t\in\mathbb{R}$
\begin{equation*}
0\leq (s-t)^{2} = s^{2} + t^{2} - 2st
\end{equation*}
if and only if
\begin{equation*}
\frac{1}{2}(s^2+t^2)\geq st \quad\forall s,t\in\mathbb{R}
\end{equation*}
Since $|f(x)|$ and $|g(x)|$ are both real (even if $f(x)$
and $g(x)$ are complex, their absolute value (aka their
modulus) returns the ``radial length'' which is a real
number), we see that
\begin{equation}
\frac{1}{2}\left(|f(x)|^2 + |g(x)|^2\right)\geq|f(x)||g(x)|
\end{equation}
so
\begin{equation}
|\<f,g\>|\leq\frac{1}{2}\int |f(x)|^2 + |g(x)|^2dx<\infty.
\end{equation}
Thus we may extend the inner product $\<\cdot,\cdot\>$ from
$PC(a,b)$ to $L^{2}(a,b)$. All properties of the inner
product and the norm still hold.

Three is one more property that also holds that we have been
trying to prove.

\begin{bessel}\index{Bessel Inequality}
If $\{\phi_{n}\}^{\infty}_{n=1}$ is an orthonormal set in
$L^{2}(a,b)$, and if $f\in L^{2}(a,b)$, then
\begin{equation}
\sum^{\infty}_{n=1}|\<f,\phi_{n}\>|^{2} \leq
\|f\|^2\qquad\text{in $L^{2}(a,b)$}
\end{equation}
\end{bessel}
\begin{proof}
Let $N$ be any positive integer,
\begin{align*}
0 &\leq \|f - \sum^{N}_{n=1}\<f,\phi_n\>\phi_n\|^2\\
&=\|f\|^2 - 2\re(\<f,\sum^{N}_{1}\<f,\phi_n\>\phi_n) +
\|\sum^{N}_{n=1}\<f,\phi_{n}\>\phi_{n}\|^{2}
\end{align*}
We can expand out
\begin{align*}
\<f,\sum^{N}_{n=1}\<f,\phi_n\>\phi_n\> &= \sum^{N}_{n=1}
\<f,\<f,\phi_n\>\phi_n\>\\
&= \sum^{N}_{n=1}\overline{\<f,\phi_n\>}\<f,\phi_n\>\\
&= \sum^{N}_{n=1}|\<f,\phi_n\>|^{2}
\end{align*}
We put all this together to get
\begin{align*}
0 &\leq \|f\| - \sum^{N}_{n=1}\<f,\phi_n\>\phi_n\|^2\\
&\leq \|f\|^2 - \sum^{N}_{n=1}|\<f,\phi_n\>|^2
\end{align*}
for all $N$.
\end{proof}
\marginpar{Some intuition of distance}Now, for something completely different. Recall back in one
real dimension, we measure distance by taking the absolute
value of the difference between two numbers. We generalize
this in multiple dimensions, from the Pythagorean theorem,
to be the norm of the difference between two vectors. We
generalize this still to be the norm of the difference
between two functions for having some intuition of distance
in $L^{2}$ (i.e. for $f,g\in L^{2}(a,b)$, $\|f-g\|$ measures
the ``distance'' between $f$ and $g$), we also say that
$f_n$ converges to $f\in L^2(a,b)$ \textbf{in norm}\index{Convergence!In Norm} if
\begin{equation}
\|f_n - f\|\to 0\quad\text{as }n\to\infty.
\end{equation}
