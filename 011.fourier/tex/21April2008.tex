%%
%% 21April2008.tex
%% 
%% Made by Alex Nelson
%% Login   <alex@tomato>
%% 
%% Started on  Sat Dec 20 20:51:48 2008 Alex Nelson
%% Last update Sat Dec 20 20:51:48 2008 Alex Nelson
%%
%% We cover only the Criteria for orthonormal basis in
%% $L^{2}(a,b)$ in this section

\begin{defn}
Let $f_n,f\in L^{2}(a,b)$, $f_n\to f$ \textbf{converges in
  norm}\index{Convergence!In Norm}  $\|f_n-f\|\to0$ as
$n\to\infty$ if and only if
$\int^{b}_{a}|f_{n}(x)-f(x)|^2dx\to0$ as $n\to\infty$.
\end{defn}

\begin{rmk} 
If one has the norm of two functions $\|f-g\|=0$, then $f=g$
in $L^{2}(a,b)$ if and only if $f(x)=g(x)$ for $x$ (outside
of sets of measure zero) in $[a,b]$.
\end{rmk}

\begin{defn}
We say $f=g$ \textbf{almost everywhere} if $f(x)=g(x)$ for
$x\in[a,b]-E$ where $E$ is measure 0.
\end{defn}

\begin{ex}
Let
\begin{equation}
f(x) = \begin{cases} 0 & \text{if } x\in[0,1]\text{ is
    irrational}\\
1 & \text{if }x\in\mathbb{Q}\cap[0,1]\end{cases}
\end{equation}
then $f(x)=1$ almost everywhere.
\end{ex}

\begin{rmk}
The pointwise convergence and convergence in norm do not
imply each other.
\end{rmk}

\begin{thm}
If $f_n\to f$ uniformly on an interval $[a,b]$, then $f_n\to
f$ in norm.
\end{thm}
\begin{proof}
Let $M_n = \sup_{x\in[a,b]}|f_{n}(x)-f(x)|\to 0$ as
$n\to\infty$. Now
$\|f_n-f\|^2=\int^{b}_{a}|f_{n}(x)-f(x)|^{2}dx\leq\int^{b}_{a}M_{n}^{2}dx=M_{n}^{2}(b-a)$. But
$M_{n}\to 0$ so $M_{n}^{2}(b-a)\to 0$ too.
\end{proof}

\begin{defn}
A sequence $\{a_{n}\}$ in a normed vector space $V$ (i.e. a
vector space with a norm $\|\cdot\|$) is called a
\textbf{Cauchy Sequence}\index{Cauchy Sequence}\index{Sequence!Cauchy} 
if
\begin{equation}
\|a_m-a_n\|\to 0\quad \text{as }m,n\to\infty
\end{equation}
$V$ is \textbf{complete}\index{Complete Vector Space}
if\index{Vector Space!Complete} every Cauchy sequence
converges to a vector in $V$.
\end{defn}
\begin{ex}
In $\mathbb{R}$, $a_n=1/n$, then
\begin{align*}
\|a_m - a_n\| &= \left|\frac{1}{m}-\frac{1}{n}\right| \\
&= \left|\frac{n-m}{mn}\right|\to 0
\end{align*}
as $m,n\to\infty$. So $a_n$ is Cauchy and goes to zero.
\end{ex}
\index{$PC(\mathbb{R})$!Not Complete}\marginpar{$PC(a,b)$ is not complete}One can observe that $PC(a,b)$ is not complete by this
counter-example to the claim. Consider $PC(0,1)$ and let
\begin{equation}
f_{n}(x) = \begin{cases} x^{-1/4}&\text{if }x>1/n\\
0&\text{otherwise}\end{cases}
\end{equation}
If $m>n$, $f_m(x)-f_n(x)$ equals $x^{-1/4}$ when
$m^{-1}<x\leq n^{-1}$ and equals 0 otherwise, so
\begin{equation}
\|f_m - f_n\|^2 = \int^{1/n}_{1/m}x^{-1/2}dx =
2x^{1/2}|^{1/n}_{1/m} = 2(n^{-1/2}-m^{-1/2}),
\end{equation}
which tends to zero as $m,n\to\infty$. Thus the sequence
$\{f_n\}$ is Cauchy; but clearly its limit (either pointwise
or in norm) is
\begin{equation}
f(x) =\begin{cases} x^{-1/4}&\text{if }x>0\\
0\text{if }x=0\end{cases}
\end{equation}
and this function does not belong to $PC(0,1)$ because it
becomes unbounded as $x\to 0$.

However, it is worthy of note that $L^{2}(a,b)$ is not only
complete but the completion of $PC(a,b)$.

Rudin's notion of completeness~\cite{rudinPrinciplesOfMathematicalAnalysis}
\begin{thm}{(Rudin 11.42)}\index{$L^2$!Complete}
If $\{f_{n}\}\in L^{2}(a,b)$ (n=1,2,...) is a Cauchy
sequence, then there exists some function $f\in L^{2}(a,b)$
such that $\{f_{n}\}$ converges to $f \in L^{2}(a,b)$.
\end{thm}
This says,in other words, that $L^{2}(a,b)$ is a
\emph{complete} metric space.

\begin{thm}{(Rudin's Theorem 11.38)}
For all $f\in L^{2}(a,b)$, and for each $\varepsilon>0$,
there is a $(b-a)$-periodic function and infinitely smooth
$\widetilde{f}\in C^{\infty}(a,b)$ such that
$\|f-\widetilde{f}\|<\varepsilon$.
\end{thm}

\begin{thm}
If $\{\phi_n\}$ is an orthonormal set in $L^{2}(a,b)$, $f\in
L^{2}(a,b)$ the Bessel inequality states\index{Bessel Inequality}: 
\begin{equation}
\sum^{\infty}_{n=1}|\<f,\phi_n\>|^2\leq\|f\|^2.
\end{equation} 
Thus
\begin{equation}
\sum^{M}_{N}|\<f,\phi_n\>|^2\to 0\quad\text{as }M,N\to\infty. 
\end{equation}
Further 
\begin{equation}
\|\sum^{N}_1 \<f,\phi_n\>\phi_n\|^2=\sum^{N}_{n=1}|\<f,\phi_n\>|^2
\end{equation}
by the Pythagorean theorem. It suffices to show that
$\<f,\phi_n\>\phi_n$ is a Cauchys series
\begin{equation}
\|\sum^{N}_{M}\<f,\phi_n\>\phi_n\|^2\to 0.
\end{equation}
\end{thm}

\begin{lem}
If $f\in L^{2}(a,b)$, $\{\phi_{n}\}$ is an orthonormal set
in $L^{2}(a,b)$, then
\begin{equation}
\sum^{\infty}_{n=1}\<f,\phi_n\>\phi_n
\end{equation}
converges in norm; moreover
\begin{equation}
\|\sum^{\infty}_{n=1}\<f,\phi_n\>\phi_n\|\leq\|f\|.
\end{equation}
\end{lem}

\subsection{Criteria for an Orthonormal Basis in $L^{2}(a,b)$}

\begin{thm}
Let $\{\phi_n\}^{\infty}_{n=1}$ be an orthonormal set in
$L^{2}(a,b)$, the following conditions are equivalent
\begin{enumerate}
\item if $\<f,\phi_n\>=0$ for all $n\in\mathbb{Z}$, then
  $f=0$
\item for all $f\in L^{2}(a,b)$,
  $f=\sum^{\infty}_{n=1}\<f,\phi_n\>\phi_n$ (convergence in
  norm)
\item for all $f\in L^{2}(a,b)$, we have the
  \textbf{Parseval equality}\index{Parseval Equality}
\begin{equation}
\|f\|^2=\sum^{\infty}_{n=1}|\<f,\phi_n\>|^2.
\end{equation}
\end{enumerate}
\end{thm}
\begin{proof}
We'll prove that (1) implies (2), (2) implies (3), and (3)
implies (1).

\textbf{(1)$\Rightarrow$(2)} Suppose $\<f,\phi_n\>=0$ for
all $n$. Let
\begin{align*}
g &= f-\sum^{\infty}_{n=1}\<f,\phi_n\>\phi_n\\
\<g,\phi_m\> &=
\<f,\phi_m\>-\sum^{\infty}_{n=1}\<f,\phi_n\>\<\phi_n,\phi_m\>\\
&=\<f,\phi_m\>-\sum^{\infty}_{n=1}\<f,\phi_n\>\delta_{nm}\\
&=\<f,\phi_m\> - \<f,\phi_m\>\\
&=0
\end{align*}
for any $m$. Therefore $g=0$.

\textbf{(2)$\Rightarrow$(3)} Suppose for any $f$, we have
\begin{equation}
f = \sum^{\infty}_{n=1}\<f,\phi_n\>\phi_n
\end{equation}
So
\begin{align*}
\|f\|^2 &= \<f,f\> =
\|\sum^{\infty}_{n=1}\<f,\phi_n\>\phi_n\|^2\\
&=\lim_{N\to\infty}\|\sum^{N}_{n=1}\<f,\phi_n\>\phi_n\|^2\\
&=\lim_{N\to\infty}\sum^{N}_{n=1}|\<f,\phi_n\>|^2\quad\text{by Pythagorean thm}\\
&=\sum^{\infty}_{n=1}|\<f,\phi_n\>|^2.
\end{align*}

\textbf{(3)$\Rightarrow$(1)} Suppose 
\begin{equation}
\|f\|^2 = \sum^{\infty}_{n=1}|\<f,\phi_n\>|^2
\end{equation}
Suppose $\<f,\phi_n\>=0$ for all $n$. Then $\|f\|^2=0$ which
implies $f=0$ in $L^{2}(a,b)$.
\end{proof}

\begin{rmk}
In $\mathbb{C}^k$ where $\vec{v}=(v_1,...,v_k)$, then
\begin{equation}
\|\vec{v}\|^2 = \sum^{k}_{j=1}|\<v,e_j\>|^2
\end{equation}
is the discrete version of Parseval equality.
\end{rmk}
