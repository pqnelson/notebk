%%
%% reviewVectorSpace.tex
%% 
%% Made by Alex Nelson
%% Login   <alex@tomato>
%% 
%% Started on  Wed Dec 17 14:54:08 2008 Alex Nelson
%% Last update Wed Dec 17 14:54:08 2008 Alex Nelson
%%
%%\section{A Review of Vector Spaces}\index{Vector Space}
Remember that the real numbers (denoted as $\mathbb{R}$) and
the complex numbers (denoted as $\mathbb{C}$) are usually
the scalars we work with in linear algebra (there are other
neurotic examples that mathematicians love, but since this
is for physicists, we won't examine them). We will generally
call this set of scalars, equipped with scalar
multiplication and scalar addition, a
``\textbf{field}''\index{Scalar Field}. We
will denote a generic field of scalars as $\mathbb{F}$.

Now, a \textbf{vector space} $V$ is always given as being ``over'' a
field of scalars $\mathbb{F}$. It is equipped with only two
operations:
\begin{enumerate}\index{Vector Space!Axioms}
\item Vector Addition, $+:V\times V\to V$ which is denoted
  as $\mathbf{v}+\mathbf{w}$, where
  $\mathbf{v},\mathbf{w}\in V$,
\item Scalar Multiplication, $*:\mathbb{F}\times V\to V$,
  denoted usually by $\alpha\mathbf{v}$ for all
  $\alpha\in\mathbb{F}$ and $\mathbf{v}\in V$
\end{enumerate}
It \emph{DOES NOT} have a cross product, or a dot product,
or any other sort of fun stuff. These \emph{ARE NOT}
properties of a vector space. The \emph{ONLY} properties are
scalar multiplication and vector addition. 

We need to specify several properties of vector addition and
scalar multiplication. There are four axioms of vector
addition
\begin{enumerate}\index{Vector Space!Vector Addition}
\item Vector addition is associative: for all
  $\mathbf{u},\mathbf{v},\mbf{w}\in V$, we have
  $\mbf{u}+(\mbf{v}+\mbf{w})=(\mbf{u}+\mbf{v})+\mbf{w}$;
\item Vector Addition is commutative: for all
  $\mbf{v},\mbf{w}\in V$, we have
  $\mbf{v}+\mbf{w}=\mbf{w}+\mbf{v}$;
\item Vector addition has an identity element: there exists
  an element $\mbf{0}\in V$ (dubbed the \textbf{zero
    vector}) such that $\mbf{v}+\mbf{0}=\mbf{v}$ for all
  $\mbf{v}\in V$;
\item Vector Addition has inverse elements: For all
  $\mbf{v}\in V$ there exists an element $\mbf{w}\in V$,
  called the \textbf{additive inverse} of $\mbf{v}$, such
  that $\mbf{v}+\mbf{w}=\mbf{0}$.
\end{enumerate}
Now, it is important to note that the zero vector is
unique. Observe a short proof: suppose we have two zero
vectors $\mbf{0}$ and $\widetilde{\mbf{0}}$, then for some
$\mbf{v}$ and its additive inverse $\mbf{w}$ we have
\begin{subequations}
\begin{align}
\mbf{v}+\mbf{w} &= \mbf{0} \\
&= \widetilde{\mbf{0}} \\
\Rightarrow \mbf{0} &= \widetilde{\mbf{0}}
\end{align}
\end{subequations}
Which implies the uniqueness of the zero vector.

There are similarly four axioms of scalar multiplication:
\begin{enumerate}\index{Vector Space!Scalar Multiplication}
\item Distributivity holds forscalar multiplication over
  vector addition:
  $\alpha(\mbf{v}+\mbf{w})=\alpha\mbf{v}+\alpha\mbf{w}$ for
  all $\alpha\in\mathbb{F}$, $\mbf{v},\mbf{w}\in V$;
\item Distributivity holds for scalar multiplication over
  field addition (ie. adding two scalars and multiplying by
  their sum):
  $(\alpha+\beta)\mbf{v}=\alpha\mbf{v}+\beta\mbf{w}$ for all
  $\alpha,\beta\in\mathbb{F}$ and $\mbf{v}\in V$.
\item Scalar multipication is compatible with multipocation
  in the sense of the field of scalars:
  $\alpha(\beta\mbf{v})=(\alpha\beta)\mbf{v}$ for all
  $\alpha,\beta\in\mathbb{F}$ and $\mbf{v}\in V$.
\item Scalar multiplication has an identity element:
  $\mbf{1}\mbf{v}=\mbf{v}$ for all $\mbf{v}\in V$ and the
  multiplicative identity of $\mathbb{F}$: $\mbf{1}$. 
\end{enumerate}
